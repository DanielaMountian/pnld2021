\chapterspecial{O x da questão}{}{}
 

X é um jovem dependente de crack e sua história é só mais uma de pó e
pedra.

X é uma relativa incógnita e também um lugar comum da miséria de outros
Xs da questão.

X foi trabalhador informal e camelô sem destino, certo dia expulso da
pífia benesse de vender bugigangas no centro da cidade.

X é rapaz conformado e melancólico, meio balofo por se entupir de
guloseimas baratas, filho único que vive com a mãe, e não conhece o pai.

X é paciente que não devolve o ânimo que lhe é dado numa conversa. Seu
discurso é lamento de uma vida da qual não brota o espírito da coisa, mas
brota a coisa sem espírito.

X expõe sua rotina atual de fazer bicos ao tomar conta de carros e
confessa que todo o dinheiro ganho vira pedra.

X procura ser ético ao obter a pedra e carece da disposição calhorda
para o furto oportunista. Seu abuso sofre limitação por restrição
orçamentária.

X revela que vem sendo tomado por um pensamento estranho lhe insinuando:
``se você não fumar pedra sua mãe vai morrer''.

X, sem achar que esse pensamento seja loucura e sim um comando ou alguma
coisa dotada de um irresistível poder de magia, luta contra isso, mas
acaba vencido.

X, algum tempo depois de começar a fumar pedra, sente-se na obrigação de
espreitar, cada noite, a mãe dormindo para verificar se ela ainda
respira.

X certo dia sentiu que precisava cumprir mais um ritual. Despiu-se até a
nudez adâmica, abriu a porta de casa e marchou feito um soldado
primitivo pelas ruas do bairro.

X mal percebeu transeuntes atônitos que, a princípio, nada fizeram até
que alguém chamou a polícia.

X revelou aos policiais que sua nudez era um gesto santo, e ainda
confessou, piedoso, que fizera aquilo para que sua mãezinha querida não
morresse.

X percebeu que os policiais acataram sua confissão como pagamento de
estranha promessa, ficaram penalizados e agiram com surpreendente
educação.

X sentiu-se à vontade quando um policial riu discretamente, um outro deu
de ombros conformado, um terceiro sacudiu a cabeça, e todos abriram
suavemente a porta da viatura e forraram o banco com cuidado.

X, sendo um rapaz lúcido, sabia que se tivesse contado outra versão
poderia ter levado umas porradas e ser jogado num terreno baldio.

X seguiu mansamente na viatura até um hospital público, e os policiais
providenciaram para ele um paninho íntimo do tamanho de uma folha de
parreira.

X, feito um bom ator, sentiu-se um reizinho sentado num troninho,
orgulhoso ao ser escoltado não como marginal e sim como santo carregado
num andor.

X sentiu-se triunfante ao pensar que os policiais teriam imaginado seu
gesto como o de São Francisco, que também se despiu até a nudez adâmica
e impressionou o mundo.

X foi bem tratado no hospital público e voltou para casa medicado e
vestidinho da silva.

X não mais andou nu pelas ruas, porém voltou a fumar pedra e era depois
compelido a verificar a cada noite se a sua mãe ainda respirava.

X não resiste ao vício da pedra, vai todo dia à biqueira, é tomado pelo
pensamento mágico, luta contra ele, mas o pensamento sempre o domina.

X, ao seguir caminho para casa, quando avista um cão, enxerga a nudez
natural do cão e pensa que o animal é destituído de qualquer sentimento
de culpa.

X sente-se muito solitário e lembra-se que a noite é longa e funda e que
as ruas parecem desertas.

X compenetra-se da solidão e prossegue caminho consciente de possuir
vestes lhe cobrindo a nudez adâmica.

X pensa numa vaga santidade e ao mesmo tempo luta contra uma vaga culpa.

X não é capaz de demonstrar, por A mais B, que é apenas uma anônima
incógnita nesta vida de mundo cão.

X não está seguro se, quando fuma pedra, solta o ar respirado como se
ele, feito um X qualquer da questão, quisesse expulsar tão somente um
resíduo maligno de seus pulmões.

X carrega um temor de estar sendo possuído de uma nebulosa vontade de
apagar para sempre a chama mortiça de uma vela ou de apagar, empunhando
seu cachimbo, a chama mortal de uma velha que dorme a seu lado.

\begin{center}\asterisc{}\end{center}
\begingroup\small

\emph{Este é um caso que conduz certamente a um diagnóstico, mas não
fecha a complexidade do assunto. Pelo contrário, abre para um terreno
pantanoso na vida deste X, que me chegou como um tipo meio apagado, como
um zé ninguém da vida.}

\emph{A~riqueza mórbida do caso veio na medida em que cutuquei uma
``alma'' perturbada e fechada em rituais. Porque, além da severidade da
adicção em crack, ele é portador de transtorno obsessivo compulsivo, ou
\versal{TOC}. O~que não resume, porém, um diagnóstico final a respeito dele.}

\emph{Não custa dizer que o instigante a respeito do \versal{TOC} é que não se
trata apenas de um mecanismo neuroquímico disparando mecanicamente e
conduzindo a um comportamento repetitivo do tipo ``mania''.}

\emph{Existe uma organização mental muito problemática por detrás da
repetição compulsiva do \versal{TOC}. Pode haver aí um pensamento delirante que
se deseja ``mágico'' e até ``onipotente''.}

\emph{X deixa isso claro ao admitir de maneira simplória que fuma pedras
para que a mãe não morra. Sua mente teme certos ``poderes'' e vem
construindo uma ilusão de neutralização desses poderes, o que fortalece
seu impulso adictivo pela pedra.}

\emph{Como em tantos outros casos de \versal{TOC} grave, pensamentos sombrios
precisam ser anulados ou ``limpados'', pois caso contrário a tensão
psíquica ficaria insuportável. Assim, o ritual de alívio pelo ato
obsessivo procura equilibrar de forma doentia um desnível crítico
subjacente ao mecanismo do \versal{TOC}.}

\emph{Por isso eu insisto em dizer que, quase sempre, por detrás do \versal{TOC}
existem outros problemas ou transtornos, ou existem desejos e pulsões
suspeitas vindas sabe-se lá de que instâncias psíquicas e de que fase da
vida.}

\emph{Mas esta história de X, quando eu a reconstruí para escrevê-la, me
pareceu levar a alguma ``racionalidade bizarra'' de um procedimento
obsessivo compulsivo, e me lembrou, curiosamente, uma paródia
matemática.}

\emph{Talvez por isso eu tenha dado o nome do texto x da questão. A~metáfora da incógnita serve para aquilo que fica aparentemente sem
explicação neste mundo cão.}

\emph{Existe, porém, alguma clareza na loucura adicta de X que, além de
ser vítima de seu pensamento mágico, parece ter algum núcleo psicótico
devido ao abuso cocaínico. Sem dizer que o ``comando interno'' de sua
mente e/ou do seu cérebro, adaptado para usar crack e garantir
estupidamente a vida da mãe, não é apenas desejo intenso de uso ou
fissura, e nem é apenas ``malandragem''.}

\emph{X é doente sincero. X quer falar através de sua doença. X é
enredado no seu x da questão patológico. X é preso na incógnita que é
ele mesmo. Eis aí também um caminho para um diagnóstico.}

\emph{Sem falar nas inúmeras implicações psicanalíticas e edípicas
óbvias e até ``clássicas''. Porque X é um rapaz regredido, de ego muito
fraco, atolado no matriarcado, talvez submetido à tirania de uma mãe
controladora.}

\emph{Eu cinicamente reconheço que não são raros os adictos praticamente
``casados'' com suas mães, muitos deles expressando, de maneira semiconsciente, uma relação viscosa dentro da qual procuram escancarar, aos
trancos e barrancos, um ``grude'' ambivalente de amor e ódio.}

\emph{O~abuso de droga pesada é um elixir maldito da falsa libertação
desse atoleiro edípico, no qual entra em jogo a ambivalência da adicção,
que é um querer e um não querer sair da busca infinita pela ``brisa''.}

\emph{Levando à seguinte conclusão possível: muitos dependentes químicos
ficam grudados de maneira instável à mãe ou a outra figura protetora a
quem espelham uma codependência parental.}

\emph{Mas a história de X, apesar da monotonia e apesar dos pesares, traz
alguma riqueza e, para mim, tem facetas literárias beirando o conto.}

\emph{A~história de X conduz a situações de um imenso cinismo e humor
negro, como, por exemplo, o encontro dele com a polícia, que se deu
assim mesmo como está no texto, creio eu.}

\emph{X --- sendo simulacro de cidadão ou herói torto --- me disse que
sentiu o chamado de uma missão antes de marchar nu em pelo nas ruas do
bairro. Ele parecia admitir que sua nudez adâmica e mais o descaramento
da imitação santa tiveram um escancarado conteúdo simbólico.}

\emph{Digo mais: isso tudo é praticamente uma cena cinematográfica. Juro
que ``vi'' uma sequência como se eu tivesse que roteirizar um curta
metragem.}

\emph{Porque mesmo nos redutos mais profundos da alma humana, como nos
romances de Dostoiévski, existem soluções estéticas para horrores que a
literatura pode resgatar. Sim, Dostoiévski e outros autores que o
digam!}

\emph{Tanto assim que eu imaginei um final para o conto-crônica
explorando os bastidores do desejo de X para com a mãe. Oportunamente,
resolvi me aventurar explorando uma contra-parte muito sombria de meu
quase personagem.}

\emph{Imaginei o que haveria por detrás do simulacro descarado de um São
Francisco nu e periférico, um São Francisco que é pura imagem, que é}
fake.

\emph{Então a contraparte desse São Francisco nu e periférico seria um
adicto feroz nada} fake \emph{e com pensamentos hediondos na calada da noite.}

\emph{Como está em uma belíssima passagem de Hamlet (}tis now the very
witching time of night when churchyards yawn and hell itself breathes
out contagion to this world; now could I drink hot blood and do such
bitter business as the bitter day would quake to look on \emph{--- é a hora mais
bruxesca da noite quando os cemitérios bocejam e o próprio inferno
despeja seu contágio sobre o mundo; agora eu poderia beber sangue quente
e realizar tarefas tão terríveis que o dia tremeria de olhar para elas).}

\emph{Assim, tão infernal em pensamento, X vira universal, vira algo
shakespeariano, e fica sendo também incógnita viva de todo ser humano
com sua sombra complexa. Mas X como solução ainda está longe de ser
alcançado.}

\emph{X ou não X ainda não é exatamente a questão.~}
\endgroup