\chapterspecial{A lágrima atrás do gorro ninja}{}{}
 

\emph{Alguns encontros são marcantes e jamais serão esquecidos.}

\emph{Transcrevo, ou então rescrevo, algumas semanas depois, o que
teria sido, aproximadamente, a confissão de \versal{WR}:}

\bigskip{} 

Fumei maconha quando tinha uns dez anos e até os quatorze era um moleque
quieto e bem comportado, o melhor aluno da minha sala. Usava óculos de
lente grossa e por isso os outros moleques me zoavam, especialmente um
moleque maior e mais alto que eu.

Teve um dia que ele me mandô trazê um pouco de cola, eu disse que não
dava pra trazê, ele me ameaçô, fez um gesto feio pra mim. Na saída me
cercô e me bateu, socô minha cabeça na parede, ficô tudo arranhado e
ferido e ainda o moleque quebrô meus óculos e uns estilhaços de vidro me
feriram.

Eu saí da escola desnorteado e fui até a oficina do meu tio, uma oficina
de desmanche. Meu tio, sabe como é, né, mexia com essas coisa e também
com droga, fazia tempo ele era do crime. Mas aí eu cheguei no meu tio
todo machucado e ele perguntô o que tinha acontecido. Quando expriquei
que foi briga com um moleque na escola, meu tio disse que não podia ficá
assim não, ele ia me ajudá a resolvê o caso. Aí eu me ofereci pra
trabalhá pro meu tio e disse que daquele dia em diante eu tava com ele
pro que ele precisasse.

Meu tio entendeu o acordo e sorriu. Depois pegô um treisoitão e me deu
de presente. Em seguida pegô uma caixa de bala. E~como eu dirigia desde
pequeno porque meu pai tinha me ensinado dirigi quando eu tinha uns dez
ano, meu tio me jogô as chaves de um Monza e disse que o carro era meu.

Daí eu virei um soldado do meu tio e comecei a trabalhá com garra.

Mas primeiro peguei o treisoitão e fui até a escola. Fiquei esperando o
moleque. Naquele dia ele não veio, mas depois eu sempre esperava por
ele. Um dia ele veio. Eu tinha o treisoitão na mochila e já fui dizendo
``qual é, mano, então você não se lembra quando você zoou comigo?''. Eu
encarei, ele encarô, eu saquei o treisoitão da mochila, o moleque ficô
paralisado, dei várias coronhada na cabeça dele. Eu batia, batia e ele
não fazia nada, só se encolhia com muito medo, e eu ainda segurei a
cabeça dele e soquei na parede, e depois ele saiu correndo e nunca mais
me zoou.

Depois daquele dia, mano, decidi ser bandido, quer dizê depois daquela
briga, quando comecei a trabalhá pro meu tio.

Logo apareceram uns serviços, feito um desses aí da hora em que no meio
teve --- vou confessá agora --- um homicídio. Que começô assim: uns
caras que trampavam pro meu tio eram sangue ruim, eram uns ratos, mano,
e como uns ratos eles ratearam a gente no bagulho de uma erva que na
época valia muito mais do que hoje, tá ligado? A gente precisava sabê
quem um deles era. Então andamo perguntando.

Tinha um cara lá perto da oficina do meu tio que sabia, mas era da mesma
turma, e a gente desconfiava desse cara também. Mas daí nós ficamo
sabendo por outros informantes quem era o rato.

Meu tio me comandô pra ação e eu tava a fim de arregaçá, tá ligado? Meu
tio então me deu arma pesada e nós saímo na caça. No começo foi difícil
achá, demo umas buscas e nada. Até que um dia eu avistei o cara perto da
caixa d'água, disse pro meu tio que era ele mesmo. Corremo pra lá quando
ele entrô num bar. Já fomo entrando arrepiando, mano, e eu já
descarreguei um balaço no peito dele e meu tio chegô também atirando, e
eu ainda mandei outro balaço no ouvido do cara, o sangue esguichô e ele
tombô. Eu me senti forte porque tava junto do meu tio.

Foi meu primeiro homicídio. Depois teve mais um quando nós fomo atrás de
um outro cara que também tinha rateado nossos bagulhos. Era um que tinha
trampado comigo na Feira do Rolo de São Mateus, e meu tio descobriu que
ele andava robando material nosso por lá.

Daquela vez foi mais difícil, mano, a gente precisô se disfarçá
colocando uns gorros ninja. Então invadimo a casa do cara e já fomo
arregaçando do mesmo jeito de quando fiz o outro serviço. Eu
descarreguei o primeiro tiro, meu tio em seguida, mas o cara era duro na
queda, difícil de caí, mano.

Então, lá de dentro da casa veio correndo um pivete de uns cinco anos, o
pivete segurô na minha perna e disse pra mim assim ``tio, não mata meu
pai''.

\asterisc{}

\versal{WR} ficou um pouco em silêncio, todo hesitante, o olhar parado no
infinito. Eu devolvi sua atitude com um correspondente silêncio
expressivo. Ele então superou um certo constrangimento, sentiu-se à
vontade e continuou:

\asterisc{}

Aí eu fiquei olhando pro pivete que segurava a minha perna e eu juro que
fiquei paralisado, mano, não sabia o que fazê, mas já tava tudo feito, e
aquilo foi duro, qué sabê, porque uma lágrima veio no meu olho atrás do
gorro ninja e ninguém felizmente viu, e eu me lembro que ainda teve uma
mulher, acho que a mãe do pivete, ela apareceu lá de dentro desesperada
e a gente em seguida vazô rápido.

O serviço já tava terminado, eu já tinha apagado o cara e a gente tinha
feito justiça com nossas mãos. Mas eu juro, mano, eu juro que aquele
pivetinho me segurando na perna foi o que me pegô, eu não consigo
esquecê, não consigo. Por isso mesmo, olha aqui truta, vou dizê agora,
aquele meu segundo homicídio foi o último. Depois, quando eu entrei
mesmo no crime de cabeça, quando peguei cadeia e tive oportunidade de
matá de novo, e quando me davam umas prensa pra apagá alguém --- sabe
como é essa vida bandida, né --- eu sempre dava um jeito de não fazê.

Olha aqui, eu fico com aquilo na cabeça até hoje, mesmo que, naquela
época eu tinha decidido segui na carreira do crime porque assim foi
minha vida desde a briga com o moleque e desde que meu tio me deu o
treisoitão, as balas e as chaves do Monza.

Mais tarde, quando eu ainda era de menor, consegui conversá com os
``irmãos'' do partido, assim de boa, e eles me concederam que eu podia
começá trabalhando por conta. Então virei traficante e montei minha
biqueira. Num dia tranquilo fui fazê propaganda do meu negócio debaixo
de um viaduto onde se junta uma pá de noia, e ofereci umas pedra de
graça pros noia, era um bagulho bom que eu tinha, porque na época eu
tava sossegado e de boa.

Qué dizê, eu tava é ganhando uma grana da hora, mano, e se eu mexia com
tráfico, de vez em quando aparecia, tá ligado, né, umas fitas de uns
assaltos também.

Então a situação ficô embaçada quando eu já tinha feito dezoito anos e
fui enquadrado no artigo 12 e no 157, fazer o quê, né, mano. Fui pará na
detenção porque o destino muitas vezes é esse, cruel.

Na cadeia até que me dei bem, e na primeira oportunidade virei fugitivo.
Me pegaram de novo, e acabei cumprindo cinco anos em regime fechado.

Olhe aqui, mano, eu já usava droga desde moleque, e não demorô eu tava
cheirando um pó da porra, mas nunca fumava pedra. Na cadeia eu tinha
sempre minha quota de maconha e farinha chegando pra mim toda semana, aí
eu cheirava e fumava mesmo. Mas com essa história de detenção minha
grana começô a sumi, eu também gastava muito com advogado, e quando saí
da detenção, já faz uns meses no começo desse ano, eu tava mal de grana
e com uma pá de dívida. Daí teve o caso de uma biqueira que eu entreguei
quase de graça, por cinco mil, mano.

Mas o pior tava por vir quando um dia eu conheci umas minas gostosa da
hora, e elas me levaram pra um lugar perto do viaduto dos noia, tá
ligado, e aí eu fiquei lá com as minas dando uns tirinhos, eu e um outro
mano chegado. A~gente ficava assim de boa, a mil, a gente ia despejando
farinha na bunda das minas, fazia umas carreirinha e dava os tirinho
assim com o nariz encostado na bunda delas, na maior brisa, mano, e
depois a gente fodia as minas, e cheirava, e fodia as minas e cheirava.

Mas foi quando uma das minas insistiu pra eu fumar uma pedra. Eu tava
muito loucão já de tanto pó. Resolvi experimentá aquele outro bagulho
pela primeira vez, pode crê, truta. Então foi de primeira, que quando eu
fumei percebi que o bagulho é loko, fumei nem sei quantas pedra com as
minas, e nóis tava ali vidrado, eu e mais o outro mano meu chegado, a
gente tava bem loucão mesmo. Que depois não deu outra, mano, eu só
pensava na pedra, na pedra, e pela pedra fui gastando o que tinha e o
que não tinha, até me desfiz de um monte de arma da pampa que eu
guardava, pode crer.

Depois comecei a robá tudo dentro de casa quando voltei a morá com meu
pai e minha mãe, porque minha mulhé não me aguentava mais e resolveu
ficá com meu filho pequeno lá na casa da família dela. Aí eu percebi que
tinha virado um noia, e tanto foi assim que eu não tava nem aí quando
fazia dívida em muitas biqueiras.

Óia só que coisa aconteceu em seguida. Eu, que tinha sido dono de
biqueira, fiquei jurado quando a dívida ficou alta, mano, mas o bagulho
é tão loko que a gente não liga pra dívida não, e eu vou contá uma coisa
de quando fiquei jurado: como os cara me conhecia de outras paradas do
crime, e eu ainda tinha um entendimento legal com os ``irmãos'' do
partido, eles ia adiando, me dando chances, tá ligado. Mas eu nem queria
sabê de nada, usava pedra direto, na maior falta de respeito.

Meu pai já tava desesperado, e o pior é que meu pai nunca me entendeu
direito, ainda mais agora ele não entendeu mesmo me vendo no crack, sem
falá que meu pai sempre me achô um folgado, um vagabundo. Pra dizê a
verdade a gente nunca se trombô bem, e pior agora, mas a minha mãe
sempre me entendeu melhor, só que a coitada tá lá com depressão direto,
nem consegue comê.

E eu, mano, felizmente tou aqui já faz cinco dias, foi uma benção, acho
que o senhor é um anjo que veio pra me salvá, e eu vou dizê uma coisa,
truta, quando vi o senhor de primeira, o senhor se lembra, né, olhei nos
olhos do senhor e pensei comigo, tive uma intuição de que ``ah, ele não
vai me caguetá não, só vai me dá uma mão''. Porque alguma coisa me dizia
isso e eu tou certo. Afinal de contas eu não tava certo? Tava sim.
Porque eu tinha me apegado com Deus e buscava força com Jesus, lutava
contra a piração da noia quando ouvia aquela voz de repente dizendo pra
mim saí daqui e se enfiá na favela e ficá lá usando pra sempre, mas
então eu punha Jesus no meio, me apegava com ele e tinha força pra dizer
não, que nem me aconteceu ontem no jogo de futebol, questão de dois
minutos, dois palitos mesmo, mano, quando fui tentado pelo maligno assim
do nada.

Mas tou aqui inteirão e da próxima vez, se a tentação do maligno vié, se
o maligno me tocá, eu vou me apegá na bíblia, e por isso quero dizê pro
senhor que eu abri a bíblia num salmo e foi uma surpresa, fiquei
emocionado quando percebi que eu me vi naquele salmo. Aquela passagem
era minha vida, dizia tudo de mim, o senhor pode crê e o senhor vai vê,
eu vou buscá a bíblia e faço questão do senhor lê pra mim essa passagem
porque eu sei que o senhor lê bem.

\asterisc{}

\versal{WR} foi buscar a bíblia. Abriu no salmo 88. Me pediu calmamente que eu
lesse. Eu li, pausadamente, com voz cheia, aquele salmo 88, escrito em
belas palavras. Salmo de um homem que, sentindo-se impotente e
mergulhado nas trevas ou, como se diria hoje, ``mergulhado no fundo do
poço'', implora a misericórdia de Deus.

Os olhos de \versal{WR} brilhavam de pura ternura. Eu não observei nele fanatismo
religioso, e ele não estava nem um pouco ``hipnotizado''. Pelo
contrário, estava bem consciente e lúcido.

Aquele era um momento religioso e bíblico. Não havia lavagem cerebral,
nem manipulação de seita. Pelo contrário: era um movimento próprio desse
rapaz do crime, que acabava de desabafar, com clareza e sinceridade, uma
história real e pungente de vida, dentro da qual as fraquezas eram
admitidas diretamente e com verdade e simplicidade.

Vinha da boca dele uma história comum que teve seu marco inicial num
episódio de \emph{bullying}, foi parar numa agressão sórdida, gerou o
ódio natural do ser humano, e gerou a correção do desagravo numa
vingança até que ``ética'' no mundo cão, sem a barbárie que costuma
ocorrer nesse mundo. Mas depois gerou uma aventura prolongada no crime,
uma aventura ``profissional'' cheia de ``adrenalina'' e conforme uma
busca de proteção paternal na idealização de um tio que hoje não possui
mais oficina de desmanche e, segundo \versal{WR}, está no meio do mato jurado por
alguns ``irmãos'' porque ``pisou na bola''. Sem falar que o ``livro da
vida'' reza que os ajustes de conta devem ser feitos para quem ``pisa na
bola''.

Bem a propósito do ``livro da vida'', eu disse a \versal{WR} que foi muito bom
ele desabafar.

E foi justamente naquele momento que enxerguei no brilho dos olhos dele
um alívio catártico em decorrência daquela confissão.

\versal{WR} insistiu várias vezes de novo que não saía da sua cabeça a imagem do
pivetinho segurando sua perna e falando aquelas palavras.

\versal{WR} admitiu que podia ter ``engolido sapo'' na briga em que apanhou do
moleque mais forte, porque se ele tivesse ``engolido sapo'' não teria
havido os dois homicídios, e ele ainda jurou, e eu acredito ser verdade,
que ele parou naqueles dois homicídios e nunca mais matou ninguém.

\versal{WR}, num momento de leveza, tocando com a mão na bíblia aberta, me disse
que vê de vez em quando o pivetinho, mas hoje ele é um rapaz. \versal{WR} passa
por ele, olha e se lembra daquela cena.

O rapaz nem sabe quem é \versal{WR} porque \versal{WR} usava um gorro ninja. O~gorro
máscara disfarçou, perante o tio bandido, que ele, \versal{WR}, fraquejou e não
conseguiu impedir a lágrima atrás da máscara.

Eu olhei nos olhos dele e falei, sem qualquer medo e sem receio, que
aquela dor que punge, e ainda está presente, é o melhor que ele tem de
todas as suas façanhas nas ``quebradas'' da vida bandida, dentro daquilo
que ele chamou de ``livro da vida''. Pois é essa dor pungente aquela dor
que não mente e vem de uma lembrança agindo como remédio, que pode dar
um jeito nas nossas mazelas, pode até cicatrizar uma ferida e nos
arrumar por dentro e no fundo da nossa alma.

\versal{WR} entendeu muito bem, ainda mais quando eu fiz menção à palavra-chave
``caráter'', e disse que aquela lágrima atrás do gorro ninja era uma
prova oculta e íntima do caráter que ele tinha. Aquela lágrima era um
bem precioso que nunca ninguém dele tiraria.

Eu ainda disse que o melhor que tinha acontecido em toda a vida dele
seria aquela lágrima muito especial.

\versal{WR} ficou bastante emocionado e me deu um abraço de uma intensidade
tamanha como eu tenho por vezes verificado nesses meninos do crime
quando eles têm lucidez, arrependimento e afeto verdadeiro. Abraço com
uma energia que eu diria dionisíaca, misturando devoção religiosa com
homoafetividade; misturando carência de pai a intensidade companheira de
afeto entre homens; de uma forma explosiva mas com uma energia vigorosa
que pode sair do mal e ir para o bem.

Sem qualquer constrangimento, de repente eu me liguei numa letra
bastante conhecida de uma música popular brasileira, e disse para mim
mesmo, repetindo em silêncio íntimo, como em Gente Humilde: ``eu que não
creio peço a Deus por minha gente, são gente humilde, e que vontade de
chorar''.

Então me virei para \versal{WR}, devolvi a bíblia e desabafei:

— Você sabe quem te salvou mesmo? Você sabe quem foi o anjo? O anjo não
sou eu. O~anjo foi aquele pivetinho que segurou tua perna. Ele foi a mão
de Deus.

\versal{WR} foi embora carregando a bíblia fechada, muito aliviado, o rosto
sereno e tranquilo.

\begin{center}\asterisc{}\end{center}
\begingroup\small

\emph{Foi uma experiência pessoal de grande intensidade. \versal{WR} apareceu no
meu serviço e pediu ajuda. Confiou na sua intuição de momento, foi
espontâneo: ``eu tenho certeza que o senhor não vai me dedurar''.}

\emph{De repente \versal{WR} foi tomado por uma confiança súbita e uma entrega.
Estava detonado pelo crack e por isso eu abri uma exceção no atropelo de
outros atendimentos.}

\emph{Logo conheci sua família --- um pai autoritário e uma mãe bastante
consciente e participativa. Os dois apareceram algumas vezes no meu
serviço, muito apreensivos.}

\emph{Tempos depois encontrei \versal{WR} internado numa comunidade terapêutica
onde eu prestava alguns serviços.}

\emph{Foi quando se deu aquela ``passagem bíblica'' sincera da leitura
do salmo bonito em um momento verdadeiramente religioso, sem fanatismo,
quando \versal{WR} escancarou sua história de vida a partir do episódio
de}~bullying\emph{. Quando ele não escondeu seus crimes capitais e expôs
francamente sua vida bandida, admitindo os dois homicídios na
adolescência.}

\emph{Acredito sinceramente que \versal{WR} não deva ter mentido. Ele não tinha
muito a perder. Além do mais, o que veio mesmo dele foi uma confissão
num momento de intensa verdade em que o paciente adicto grave não oculta
suas sombras profundas porque se sente momentaneamente livre, porque
está rendido na tragédia pessoal para a interlocução acolhedora, e
também porque mobiliza o fito sublime de expor sua luz irrompendo da
escuridão.}

\emph{Por esse motivo achei interessante analisar um pouco o caso do
ponto de vista psiquiátrico, uma vez que o comportamento de \versal{WR} pode ser
considerado um transtorno de conduta na adolescência, e sem que se
considere isso ainda transtorno de personalidade.}

\emph{O~transtorno de conduta é classificado dentro dos transtornos
disruptivos de infância, aos quais é atribuída uma base genética. Mas há
também uma participação decisiva de problemas de criação, ausência
paterna ou materna, agressões por parte da sociedade ou da família,
companhias perversas etc.}

\emph{O~transtorno de conduta pode conviver com a presença do caráter e
até de sentimentos elevados no adolescente, como é nitidamente o caso de
\versal{WR}. Pode conviver com a plena sensação da culpa e com o sentimento de
solidariedade humana.}

\emph{O~transtorno de personalidade é melhor percebido ou diagnosticado
apenas mais tarde, além da fase da adolescência, e quase sempre (como no
caso do transtorno antissocial) implica numa ausência de culpa e frieza
de intenções, e também num egocentrismo atroz e predador.}

\emph{A~história de \versal{WR} foge a esquematizações e ilustra as vicissitudes
horrendas da trajetória de vida de um dependente químico rebaixado na
vida do crime, que faliu, por assim dizer, ``comercialmente''.}

\emph{Se anteriormente \versal{WR} usava as drogas permitidas pelas facções,
como, por exemplo, a ``farinha branca'', com o crack a questão é outra.
O~crack é tido pelas facções como restolho de gente desclassificada, e
\versal{WR} perdeu tudo por causa do crack que lhe chegou por via da tentação das
``minas gostosas'' enquanto ele cheirava cocaína com seu parceiro.}

\emph{Eu fiquei sinceramente mobilizado pela confissão de fragilidade de
\versal{WR} perante as tentações do crack, o que acabou me revelando muita coisa
significativa sobre o mundo sombrio das drogas pesadas, como, por
exemplo, o motivo pelo qual as facções baniram o crack das cadeias.}

\emph{Para melhor entender isso, deve-se levar em conta o fato de que a
dependência de crack rapidamente desestrutura a vida do usuário, tanto
no caso dos que têm uma vida bandida quanto dos que têm uma vida de
trabalhador.}

\emph{\versal{WR} --- eu não hesito em confessar --- é um caso bem especial, e
faço questão de admitir que, dentre todas as crônicas que integram este
livro, esta ``lágrima atrás do gorro ninja'' é a lágrima que me pegou
pra valer. Quando releio o texto para fazer correções, sinto ao terminar
os olhos úmidos da emoção que extravasa.}

\emph{É por isso que nunca vou esquecer aquele primeiro encontro, quando
\versal{WR} --- notável figura marginal --- derramou-se diante de mim cheio de
confiança e entregou-se todo, movido pela intuição mais emotiva, e
formando um vínculo com sintonia e afeto.}

\emph{Faço questão de reconhecer ter visto com bons olhos o clima
religioso do encontro. Também me emociona a música Gente Humilde, da
qual me lembrei depois, e que coloquei no texto para fechar a crônica;
me emociona lembrar de novo do abraço tão forte, tão explosivo e tão
sincero que \versal{WR} me deu, num encerramento confessional da parte dele que
fez também, caro leitor, eu me colocar aqui de maneira intensa, pessoal
e confessional. ~}
\endgroup