\chapterspecial{Ivan, o terrível}{}{}
 

Ivan é um homem feito. Paulista, ele tem sotaque baiano que vem de sua
família de Bom Jesus da Lapa. Há tempos vive ao léu, sem emprego, mal
encostado nos familiares que o olham atravessado por causa dos
``vícios''.

Chegou a mim sujo e cheio de sarna, olhar parvo entre estuporado e
apavorado. Mas eu acredito que ele também possa ser razoavelmente
normal, se bem relate, de vez em quando, rompantes que me parecem
psicóticos.

Ivan não é apenas alcoolista. Quanto à cachaça antes diária, ele jura
não beber há seis meses. Sua rotina é feita de \emph{cannabis}, farinha
branca e pedra de crack.

\asterisc{}

Ivan mora sozinho no Pantanal entre São Miguel e o Itaim, próximo da
casa dos pais. É~separado e tem seis filhos, o mais velho com dezoito
anos.

Os filhos banzam desencaminhados por várias ``quebradas'', e meio que
moram com a mãe não mais casada porém ajuntada. O~padrasto dos meninos
tem apenas dezesseis anos, vejam só!

Ivan solta uma risada sem graça por causa da precocidade do padrasto
garoto e eu pergunto se o padrasto garoto se dá bem com os enteados
meninos mais velhos que ele. Ivan diz que sim, porque o padrasto e os
enteados usam droga juntos.

\asterisc{}

Ivan de vez em quando recebe visita dos filhos adolescentes, que o
procuram para usar droga. Eles desejam fazer da \emph{cannabis}, da
farinha branca e da pedra um estranho elixir de união para manter uma
frágil paternidade. Mas Ivan, tomado por lampejos moralistas de
consciência, hesita em compartilhar seus vícios com a filharada e os
espanta explosivamente de casa.

Eles vão aos cruzamentos e faróis fazer piruetas para ganhar tostões,
principalmente o de dezesseis anos e o de quatorze. O~de dezoito foi
expulso de um emprego, tenta arrumar uns ``bicos'' e não sabe o que quer
na vida.

\asterisc{}

\textls[-28]{Ivan é um homem que também não sabe o que quer na vida. Ele é uma
criatura de quem não se pode dizer que tenha perspectivas, afora o
desejo de sonhar eternamente nos embalos suaves da \emph{cannabis} ou
nos mais loucos embalos da pedra.}

\textls[-15]{O mesclado e a pedra pura lhe propiciam ``viagens'' que ele não sabe se
podem ser boas ou más quando tem alucinações e delírios
persecutórios à luz do dia e vê pessoas estranhas.}

Ele vê cavaleiros surgindo das trevas, em bandos, montados em garbosos
cavalos negros, com faces encapuzadas e empunhando espadas cintilantes.
Os cavaleiros sempre marcham em direção à casa dos pais de Ivan, lá no
Pantanal, e irão incendiar a casa a mando de uma Potência maligna.

Ivan confessa que passou o dia todo acossado pelo medo de um incêndio.
Viu os cavaleiros cruzando desembestados à sua frente. Então resolveu ir
até a casa dos pais para conferir se a casa ainda estava lá. Avistou
tudo do mesmo jeito e teve um momento de alívio.

\textls[-15]{Ivan também confessa que necessita dos cavaleiros negros. E~eu imagino
que ele, por meio desse tipo de alucinação, afunde"-se num sonho louco de
usar drogas para terrivelmente conquistar suas ``férias químicas'', como
escreveu Huxley.}

Pois ele certamente deve mergulhar em um íntimo mundo obscuro e
agonizante misturado a delícias esporádicas e a frequentes terrores
gratuitos produzidos por todas ou por qualquer uma das drogas que ele
pega na biqueira sem desembolsar nada.

Porque Ivan, apesar da miséria, é dono de seu barraco, a biqueira fica
nos fundos e o dono da biqueira é seu inquilino. Ivan recebe aluguel em
maconha, cocaína e crack.

Quanto ao arroz e feijão do dia, Ivan mendiga na casa dos pais, e sempre
volta para casa para fumar um baseado, dar um tirinho de farinha branca
ou fumar uma pedra.

De resto, Ivan se deixa abandonar feito um mendigo cheio de perebas,
fedendo, com o olhar parvo e infantilizado ou, por vezes, com o olhar
francamente psicótico, os olhos estranhamente vivos na busca do sonho,
na busca do pesadelo, ou na busca de ambos.

\asterisc{}

Quando Ivan se distancia do Pantanal, a horda de cavaleiros negros
encapuzados emerge de um fundo escuro. Aquilo é seu horror e é seu
contentamento. Aquilo é também seu cinema particular gratuito. E~a cada
vez que os cavaleiros negros surgem paramentados com espadas
cintilantes, Ivan teme por um incêndio que, no entanto, ainda não se
deu.

Mas Ivan, periodicamente aterrorizado, sempre retorna à casa dos pais e
confere a realidade do mundo. Porque ele acha que de repente tudo pode
ser real além de ser prefiguração de uma catástrofe.

Ivan, no entanto, sabe, por outro lado, que tem os pés no chão e é
habitante de uma favela chamada Pantanal.

Ivan sabe também que no outro dia parece que tudo vai pegar fogo outra
vez, que tudo vai acabar, que o mundo vai acabar. Mas, outra vez mais,
não acaba.

\asterisc{}

Eu percebo chegar para ele o momento crítico. Ivan súbito se inquieta.
De repente fala alto e responde como se acabasse de receber uma
mensagem:

— Preciso ir a São Miguel.

Sai apressado, quase correndo, a face apavorada, os olhos esbugalhados
de quem está vendo longe e vendo demais.

Um pobre diabo.

Mas como é terrível. Este Ivan.

\begin{center}\asterisc{}\end{center}
%\begingroup\small

\emph{Quando comecei a escrever este texto tive a impressão de que tudo
seria uma imensa banalidade no meio de uma extrema miséria. Tudo estaria
literalmente imerso no lodo de um pantanal nos cafundós da metrópole
paulistana.}

\emph{No entanto, não era verdade, e uma outra verdade eu captei quando
cutuquei fundo no depoimento deste Ivan e cheguei, talvez, próximo do
núcleo da coisa.}

\emph{Esta é uma história que se repete para quem lida com drogados na
periferia. Basta observar aí um mundo real degradado que é também algo
insano. E~se este Ivan, no meio desse mundo, é por acaso um tipo
diferenciado, ele é legião, ele vem do baú anônimo de tipos populares e
encarna mais uma vez o ``buscador de novidades'' num estado ensandecido,
desligado total das obrigações de uma vida de trabalhador, e à maneira
de um anarquista inútil e inconsciente.}

\emph{Em outras palavras: Ivan é malandro bizarro, é malandro
despirocado mal vivendo a busca estranha de um prazer também um tanto
estranho.}

\emph{Eu o acabei enxergando como um sujeito obsessivo protagonizando
uma vida torta, sobrevivendo no limiar da miséria, fazendo das tripas
coração e dando nó em pingo d'água.}

\emph{Conforme outra avaliação, trata"-se de um dependente de múltiplas
drogas a manifestar uma síndrome psicótica possivelmente induzida pelo
abuso de drogas.}

\emph{Parece haver nele também um transtorno obsessivo compulsivo
relacionado ao transtorno psicótico episódico. Isso eu notei quando Ivan
me confidenciou que se via compelido a fazer verificações, a ir
sucessivamente até São Miguel checar se a casa dos pais tinha sido
incendiada, de maneira a aliviar uma tensão psíquica. Isso me lembra o
personagem X (de O X da questão) a verificar a cada noite se a mãe
estava viva.}

\emph{Ambos os casos remetem a pensamentos infernais guardados nos
porões da mente a sete chaves. Então Ivan seria outro X sem solução à
vista, seria outro X nadando um pouco e se afogando um pouco no mar das
imagens universais de seu inconsciente.}

\emph{No entanto, Ivan é também um tipo anônimo, é um daqueles que
perpassam invisíveis feito incógnitas nos postos de atendimento, e de
maneira a engrossar as fileiras de um} Lumpenproletariat \emph{na sua banda
radical e esdrúxula.}

\emph{Eu acredito, porém, que ele seja um caso bem rico e delicado.
Tanto assim que, no transcurso do atendimento, foi preciso que eu
deixasse Ivan à vontade para que dele viesse a confissão daquelas
imagens repetitivas dos cavaleiros negros encapuzados empunhando espadas
cintilantes como se fosse o caso de uma batalha medieval.}

\emph{Lembrando, por sinal, cena de filme \versal{B} de aventuras que mistura
ficção com saga heroica de segunda classe. É~o mesmo esquema que aparece
em a história sem fim, cujo protagonista ``viaja'' toda noite no curso de
uma elaboração psicótica.}

\emph{Mas a história sem fim tem mais completude e sentido do que os
rompantes alucinatórios de Ivan. No entanto, eu tive a impressão de que
Ivan buscava manipular seus conteúdos íntimos e se apropriar de imagens
que lhe brotavam como alucinações de início espontâneas, mas que depois
eram exercitadas para novamente aparecerem durante a ``brisa''
impulsionada com o mesclado e também com a Cannabis e a pedra pura.}

\emph{Ivan, sendo tão despojadamente cínico e tão ``cara de pau'',
parecia regozijar"-se com as imagens móveis que também eram seu
contentamento, que também eram seu ``cinema''. Eis aí um diferencial
deste caso para com outros casos semelhantes de transtorno psicótico por
abuso de substância.}

\emph{Ivan se colocava como se fosse um contador de histórias ou de
façanhas pessoais a vangloriar"-se de estar ``lutando'' contra fantasmas.
Além disso tinha rompantes de endireitar capengamente sua família para
lá de torta por meio de um moralismo ridículo, ao mesmo tempo
enveredando por um descolamento anárquico do mundo real para viver num
mundo do deus-dará ou não dará.}

\emph{Ivan era todo ambivalente, e até com certo humor negro.}

\emph{De repente ele assumia um jeito de cidadão, dava pífias lições de
moral nos filhos que viviam com um padrasto de dezesseis anos, vejam só,
numa demonstração de que a família de Ivan era uma piada de humor negro,
ou era um núcleo de sociopatia, ou era um festival da droga.}

\emph{Caro leitor, eu tenho visto com frequência tipos populares
atolados na dobradinha da ``noia'' e da miséria, e admito que eles --- à
maneira da família de Ivan --- chegam a criar alternativas vivenciais de
discutível e complexa validade, e ainda mais num lugar como o Pantanal
(que não é o de Mato Grosso).}

\emph{Por isso me veio uma imagem, ou uma vaga ideia, de que Ivan teria
um pouco de suas ``férias químicas'', conforme a lembrança de um texto
de Aldous Huxley sobre o uso hedonista de substâncias psicoativas.}

\emph{E não me parece que Ivan seja um psicótico ou um esquizofrênico de
base. Do contrário, é mais sensato supor que ele integre uma grande
legião dos que se afundam na marginalidade da marginalidade.}

\emph{Ele lembra vários pacientes muito perturbados, cujo psiquismo
desvairado parece vir mais do fato de que são cobrados não bem por uma
instância superegoica, mas por rituais malignos que lhe foram
introjetados.}

\emph{Alguns deles conseguem ser cinicamente poéticos numa mistura de
loucura com um jeito de} enfant terrible\emph{, numa mistura de gozação com
traços de perversidade.}

\emph{Ivan é um representante dessa ``fina flor'' pantanosa problemática
e marginal dos que, lá no fundo, até desejam, desde o lodo, olhar
serenamente para o céu como na metáfora oriental e budista da flor de
lótus. Infelizmente, porém, eles não costumam atingir a serenidade, e
essa mesma serenidade longe está de ser real e de ser nirvânica.}

\emph{Eles chafurdam apenas no lodo e olham para o céu, apropriando"-se
mal e porcamente da própria loucura pelo uso contínuo de substâncias
tidas como drogas do prazer. Por isso, como dizem os manos, ``o bagulho
é loko''.}

\emph{Para mim resta dar ao pobre Ivan o epíteto de ``terrível'', numa
lembrança do distante czar da Rússia, também meio louco porém cruel. É~um personagem a servir de referência muito distante para este Ivan, que me
parece um manso perturbado, mas que dificilmente herdará o Reino do Céu.}
%\endgroup