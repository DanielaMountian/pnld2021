\chapter{Vida e obra}

\subsection{A obra}

O livro As crônicas do crack é uma composição de relatos colhidos
pelo médico psiquiatra Luis Marra, que trabalha com dependentes químicos
desde o ano de 2002. A forma de escrita destes relatos contidos dentro
do livro nos remete a algo bem literário, com uso de metáforas,
analogias e descrições. Ao olharmos para trechos do livro, vemos que ele
não é voltado para debater políticas públicas acerca do consumo de crack
na cidade de São Paulo, mas que o objetivo do autor é mirar no mundo do
dependente químico, nas suas ``viagens'', em suas relações sociais, no
ambiente em que ele está vivendo e até mesmo em sua própria loucura.

Dentro desta perspectiva, a escrita busca acabar com o estereótipo
construído de que a pessoa que consome crack é, única e exclusivamente,
um usuário. Assim, a obra denota que esses indivíduos têm sentimentos,
uma vida anterior ao vício, estando, muitas vezes, dentro de um contexto
desfavorável, além de indicar que o universo dos dependentes químicos
possui próprias relações de sociabilidade muito próprias.

\subsection{Um holofote a quem vive nas sombras}

Como já dito, o intuito e interesse do autor deste livro é mergulhar no
universo do dependente químico, sobretudo aquele que é usuário de crack.
Segundo suas próprias palavras em uma entrevista dada para a \textsc{tv} Estadão,
no ano de 2017: ``a pessoa viciada nesta droga contém dentro de si uma
criatividade tremenda''

Isso fica claro ao leitor no capítulo do livro ``Antônio Santiago: o
dito e o oculto'', onde o morador de rua Antônio Santiago se utiliza de
uma boa retórica e esbanja criatividade em um diálogo com o doutor para
conseguir uns trocados. Além do mais, ainda segundo o médico, o
dependente se aproxima muito daqueles autores e pensadores que
vivenciaram o movimento literário do romantismo no século \textsc{xix}, por
sempre estar buscando a viagem, a loucura extremada e o flerte com o
perigo e com a morte.

\subsection{Uma questão social ampla e complexa}

Outro ponto que vale ser debatido é de que as drogas transitam por todas
as classes sociais. A ideia de que o consumo era exclusivo das classes
mais baixas já foi, há muito, posto de lado. Entretanto, não se
individualizou o dependente. Deve ser superada também a ideia de um
único perfil de dependente químico, existem vários perfis. Isso porque
as pessoas são individuais, apesar de elementos em comum, cada indivíduo
tem suas particularidades.

Contudo, não se pode perder de vista que condições sociais podem
ocasionar maior ou menor peso para o consumo, vício, e superação deste.

Neste sentido, o norte"-americano Carl Hart, professor de psicologia pela
Universidade de Columbia e um dos especialistas no assunto nos diz que a
importância do ambiente em que o dependente químico está inserido é
crucial para o seu vício.

Dentro de um ambiente onde há diferentes alternativas que possam superar
o uso de drogas, a pessoa terá mais facilidade para sair da dependência.
Isto é evidenciado na obra em questão, já que o autor também nos mostra
as consequências dos ambientes e contextos com os quais os usuários
convivem.

Sendo assim, o autor ressalta que existe uma grande dificuldade com
relação a ajudar os dependentes químicos em condição de rua a superar a
dependência, posto que o ambiente em que estão inseridos é totalmente
desfavorável ao abandono dos psicoativos, sendo estimado que menos de
30\% dos usuários busquem tratamento ou qualquer tipo de ajuda.

\subsection{Uma discussão atual}

O livro é publicado em um momento em que o uso de psicoativos se tornou
endêmico em alguns locais, como por exemplo São Paulo, que abriga em seu
centro um dos maiores pontos de consumo de entorpecentes a céu aberto.

Ademais, dentro do mencionado contexto há de se lembrar também que os
últimos anos houve acalorados debates acerca de políticas públicas para
dependentes químicos, especialmente os que se encontram em situação de
rua.

Questões quanto a utilização do sistema público de saúde, a setores
urbanos que os dependentes ocupam e em que medida o problema
retroalimenta a violência são alguns pontos que podem ser levantados
acerca desta complexa questão.

Reflexões e provocações como essa se encontram ao longo das páginas da
obra de Luis Marra, uma vez que colhe relatos de regiões extremamente
marginalizadas, onde o poder do Estado não chega, e também traz
histórias do centro da cidade, escancarando a diferença de atuação do
Poder Público em cada um dos contextos

Ademais, vale frisar que, ao mesmo tempo que se pensa na questão dos
psicoativos como um problema social, que afeta a vida de milhões de
pessoas pelo mundo, existem estudos atuais que denotam que a boa
engenharia química de substâncias proscritas pode aprimorar a eficácia
de tratamentos medicinais, de sorte que o debate fomentado por essa obra
é, atual e riquíssimo e, sem dúvida, polêmico.

Portanto, Crônicas do crack, como já foi dito, não é um livro destinado
a defender esta o aquela política pública acerca da questão dos
dependentes químicos. Sua função, embora tangencie esses pontos é mais
humanitária. A obra está voltada para o descobrimento de seu universo,
dos atores que fazem parte deste ambiente, de suas sociabilidades e
pensamentos.

Ainda assim, não podemos tirar este escrito do contexto de ações mais
brutas em relação aos dependentes químicos que vivem nas ruas de São
Paulo, sendo impossível retirar a atuação do Estado do cotidiano desses
habitantes da capital paulistana.