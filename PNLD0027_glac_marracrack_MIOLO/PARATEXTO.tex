\chapter{Vida e obra}

\linespread{1.15}

\section{Sobre o autor}

Não são poucos os médicos (atuantes ou não na prática da medicina) que escrevem literatura --- célebres nomes como Anton Tchekhov, Guimarães Rosa, Moacyr Scliar e Arthur Conan Doyle figuram nessa lista. Luis Marra, nascido no ano de 1950, é também um desses clínicos-literatos --- médico psiquiatra, já publicou, além dessas crônicas, os livros de literatura \textit{O coletivo aleatório} e \textit{O diário perdido do Jardim Maia}.

Desde sua obra de estreia, \textit{O coletivo aleatório}, de 2001, Luis Marra já demonstra o olhar atento e delicado sobre as populações que vivem nas periferias de São Paulo.
Através de seus contos, percebemos como o narrador, que se distinguia e se alheava de seu objeto pela sua formação cultural, vai paulatinamente apagando esse traço divisor através de sua permanência continuada naquele ambiente.
A criação de suas personagens é muito rica, fruto de sua experiência na cidade grande e, principalmente, em suas zonas periféricas. Apesar desse ``coletivo aleatório'', experiência  típica da metrópole, ser a tônica do livro, o autor já demonstrava muita versatilidade navegando por diversos temas e formas de narrá"-los. No primeiro conto, por exemplo, temos a história de um rapaz à procura do pai, enquanto no último, que empresta seu título ao livro, acompanhamos a jornada dos passageiros de um coletivo que, quando muda de trajetória, fascina e supreende seus passageiros, extasiados pela simples mudança em um dia monótono. 

Em seu segundo livro, \textit{O diário perdido do Jardim Maia}, de 2009, a experiência médica de Luis Marra também é matéria"-prima para a sua criação literária. Dessa vez, porém, não estão registrados os anônimos que despontam de suas visitas e deambulações enquanto médico na periferia de São Paulo, e sim os anônimos que aparecem em seu consultório, configurando"-se quase como um registro do cotidiano no consultório.
São pessoas que, apesar de anônimas, ``chegam a ser pequenos heróis das luzes ou das sombras'', como aparece na introdução ao volune. Como registra Marra, os causos que escuta e as pessoas que observa surpreendem"-lhe pelo ``entusiasmo que costuma exceder o limite sensato; uma humilde nobreza que pode ser de um conformismo autoflagelante ou beirar a caricatura quixotesca; uma desobediência civil com ou sem a malícia de estar fora da lei; um franco desatino que pode chegar à insanidade ou ao crime''.

Por fim, em seu último livro, \textit{Crônicas do crack}, o leitor vai encontrar 
retratos que foram colhidas pelo autor ao longo de mais de quinze anos que vem trabalhando como médico na Zona Leste de São Paulo.
Seu primeiro contato com o universo dos dependentes químicos deu"-se após uma visita a uma cacrolândia na região de São Miguel Paulista, no início da década de 2000. Desde então, iniciou um projeto de teatro na comunidade e vem acompanhando de perto as trajetórias dos dependentes químicos e o poder da arte para operar"-lhes transformações. Além de São Miguel Paulista, Marra atua em diversas cracôlandias existentes na cidade de São Paulo.

Percebe"-se, através desse breve panorama, a importância da experiência enquanto médico para a composição literária de Luis Marra. Reafirmando, afinal, seu pertencimento à célebre lista de médicos"-escritores elencados acima.
Em entrevista do autor publicada em 16 de agosto de 2017 pelo jornal online \textit{Rotas Cult}, Marra entrelaça, de forma evidente, suas preocupações estéticas e literárias às suas preocupações enquanto médico atuante junto às populações marginalizadas das cracolândias:

\begin{quote}
Pretendo alcançar, a par do inegável desejo de reconhecimento da minha obra, algum impacto que ajude um pouco a descontruir tantos mitos e preconceitos. Gostaria muito que os leitores possam “viajar” nos meus relatos e nos textos adicionais de “making of”, de maneira a compreender que droga adicção não é sempre algo intrinsecamente mau ou perverso. Que há muita riqueza psíquica, existencial, ou até cultural, por detrás de muitas estórias até horripilantes de dependência de drogas. Não pretendo chegar a nenhuma conclusão final a respeito de um problema, e dele me resta a pretensão de apenas --- com perdão da redundância --- problematizar o problema. Trazendo à tona uma riqueza de detalhes para ir de encontro a essa costumeira massa de banalidades sobre um assunto tão importante nos dias de hoje. Sem falar da “pegada” literária que a maioria das crônicas tem, e sem quase nada recorrer à ficção. Em resumo: a realidade pode parecer mais ficcional do que a ficção.\footnote{Disponível em: \textit{https://bit.ly/3lsnlJW}.}
\end{quote}

É através desse caminho, portanto, entre o ético e o estético, a prática médica e a literária, que Luis Marra vem se somar aos grandes prosadores contemporâneos do Brasil, trazendo à sua prosa ágil as marcas de uma realidade ``mais ficcional do que a própria ficção''.

\pagebreak
\section{Sobre a obra}

O livro \textit{As crônicas do crack} é uma composição de vinte relatos colhidos
pelo médico psiquiatra Luis Marra, que trabalha com dependentes químicos
desde o ano de 2002. A forma de escrita destes relatos contidos dentro
do livro nos remete a algo bem literário, com uso de metáforas,
analogias e descrições. Ao olharmos para trechos do livro, vemos que ele
não é voltado para debater políticas públicas acerca do consumo de crack
na cidade de São Paulo, mas que o objetivo do autor é mirar no mundo do
dependente químico, nas suas ``viagens'', em suas relações sociais, no
ambiente em que ele está vivendo e até mesmo em sua própria loucura.

Dentro desta perspectiva, a escrita busca acabar com o estereótipo
construído de que a pessoa que consome crack é, única e exclusivamente,
um usuário. Assim, a obra denota que esses indivíduos têm sentimentos,
uma vida anterior ao vício, estando, muitas vezes, dentro de um contexto
desfavorável, além de indicar que o universo dos dependentes químicos
possui próprias relações de sociabilidade muito próprias.

\subsection{Um holofote a quem vive nas sombras}

Como já dito, o intuito e interesse do autor deste livro é mergulhar no
universo do dependente químico, sobretudo aquele que é usuário de crack.
Segundo suas próprias palavras em uma entrevista dada para a \textsc{tv} Estadão,
no ano de 2017: ``a pessoa viciada nesta droga contém dentro de si uma
criatividade tremenda''.

Isso fica claro ao leitor no capítulo do livro ``Antônio Santiago: o
dito e o oculto'', onde o morador de rua Antônio Santiago se utiliza de
uma boa retórica e esbanja criatividade em um diálogo com o doutor para
conseguir uns trocados. Além do mais, ainda segundo o médico, o
dependente se aproxima muito daqueles autores e pensadores que
vivenciaram o movimento literário do romantismo no século \textsc{xix}, por
sempre estar buscando a viagem, a loucura extremada e o flerte com o
perigo e com a morte.

\subsection{Uma questão social ampla e complexa}

Outro ponto que vale ser debatido é de que as drogas transitam por todas
as classes sociais. A ideia de que o consumo era exclusivo das classes
mais baixas já foi, há muito, posto de lado. Entretanto, não se
individualizou o dependente. Deve ser superada também a ideia de um
único perfil de dependente químico, existem vários perfis. Isso porque
as pessoas são individuais, apesar de elementos em comum, cada indivíduo
tem suas particularidades.

Contudo, não se pode perder de vista que condições sociais podem
ocasionar maior ou menor peso para o consumo, vício, e superação deste.

Neste sentido, o norte"-americano Carl Hart, professor de psicologia pela
Universidade de Columbia e um dos especialistas no assunto, nos diz que a
importância do ambiente em que o dependente químico está inserido é
crucial para o seu vício.

Dentro de um ambiente onde há diferentes alternativas que possam superar
o uso de drogas, a pessoa terá mais facilidade para sair da dependência.
Isto é evidenciado na obra em questão, já que o autor também nos mostra
as consequências dos ambientes e contextos com os quais os usuários
convivem.

Sendo assim, o autor ressalta que existe uma grande dificuldade com
relação a ajudar os dependentes químicos em condição de rua a superar a
dependência, posto que o ambiente em que estão inseridos é totalmente
desfavorável ao abandono dos psicoativos, sendo estimado que menos de
30\% dos usuários busquem tratamento ou qualquer tipo de ajuda.

\subsection{Uma discussão atual}

O livro é publicado em um momento em que o uso de psicoativos se tornou
endêmico em alguns locais, como por exemplo São Paulo, que abriga em seu
centro um dos maiores pontos de consumo de entorpecentes a céu aberto.

Ademais, dentro do mencionado contexto há de se lembrar também que nos
últimos anos houve acalorados debates acerca de políticas públicas para
dependentes químicos, especialmente os que se encontram em situação de
rua.

Questões quanto a utilização do sistema público de saúde, a setores
urbanos que os dependentes ocupam e em que medida o problema
retroalimenta a violência são alguns pontos que podem ser levantados
acerca desta complexa questão.

Reflexões e provocações como essa se encontram ao longo das páginas da
obra de Luis Marra, uma vez que colhe relatos de regiões extremamente
marginalizadas, onde o poder do Estado não chega, e também traz
histórias do centro da cidade, escancarando a diferença de atuação do
Poder Público em cada um dos contextos

Ademais, vale frisar que, ao mesmo tempo que se pensa na questão dos
psicoativos como um problema social, que afeta a vida de milhões de
pessoas pelo mundo, existem estudos atuais que denotam que a boa
engenharia química de substâncias proscritas pode aprimorar a eficácia
de tratamentos medicinais, de sorte que o debate fomentado por essa obra
é atual e riquíssimo e, sem dúvida, polêmico.

Portanto, \textit{Crônicas do crack}, como já foi dito, não é um livro destinado
a defender esta ou aquela política pública acerca da questão dos
dependentes químicos. Sua função, embora tangencie esses pontos, é mais
humanitária. A obra está voltada para o descobrimento de seu universo,
dos atores que fazem parte deste ambiente, de suas sociabilidades e
pensamentos.

Ainda assim, não podemos tirar este escrito do contexto de ações mais
brutas em relação aos dependentes químicos que vivem nas ruas de São
Paulo, sendo impossível retirar a atuação do Estado do cotidiano desses
habitantes da capital paulistana.

Ao final de cada crônica, seguem, em itálico, depoimentos pessoais,
observações, opiniões, dados complementares, e tudo isso de maneira a compor o
que se poderia livremente considerar como os “bastidores” de cada um dos
textos.

\pagebreak
\section{Sobre o gênero}

\textit{Crônicas do crack} reúnem relatos que caminham entre a crônica, o conto"-crônica ou o depoimento estilizado, abordando sempre --- eventualmente com certa objetividade porém mais
comumente de maneira vaga --- um (ou mais de um) transtorno psiquiátrico.
No entanto, e mesmo que todas as pessoas mencionadas no livro sejam dependentes
ou no mínimo abusem de drogas, por vezes não se identificam outros transtornos
além da própria adicção.

Na definição do crítico literário Antonio Candido, ``a crônica é filha do jornal e da era da máquina, onde tudo acaba tão depressa. Ela não foi feita originalmente para o livro, mas para essa publicação efêmera que se compra num dia e no seguinte é usada para embrulhar um par de sapatos ou forrar o chão de cozinha''.\footnote{\textsc{cândido}, Antonio. ``A vida ao rés do chão''. In: \textit{Para gostar de ler: crônicas}, volume 5. São Paulo: Ática.}

Como se observa, para Antonio Candido esse gênero literário se caracteriza por sua aparição no jornal, revelando uma de suas características que mais a afasta do conto, com o que é constantemente confundida: a crônica aborda os acontecimentos do dia a dia, fatos banais, cotidianos, pequenos eventos e conversas observadas pelo cronista e plasmada, muitas vezes de forma irônica e sarcástica, nos jornais e periódicos de circulação regular.

Apesar de seu estreito vínculo com os acontecimentos contemporâneos, lastreado pela atualidade e imediaticidade dos jornais, a crônica não se limita a registrar a realidade tal qual uma notícia de jornal, pois, no que tem de brevidade na forma, ganha muitas vezes na distenção em profundidade do que aborda. Em uma crônica como ``O mineirinho'', por exemplo, de Clarice Lispector, o texto não se prende ao acontecimento imediato. Na crônica, a escritora narra uma notícia de jornal da época, o assassinato do bandido Mineirinho, morto com treze tiros pela polícia.
A violência e a brutalidade da execução, no entanto, demovem a escritora do mero relato estilizado. Ela alça sua crônica a indagações morais, éticas e existenciais que, ao cabo, não dizem respeito apenas ao Mineirinho ou ao ato de roubar e matar, mas a todos enquanto humanidade.

Sob a ótica do cronista, qualquer fato banal ganha relevância e passa a ter uma existência autônoma em relação ao seu desdobramento fatídico.
Nas palavras de Massaud Moisés, a crônica

\begin{quote}
move-se entre ser no e para o jornal,
uma vez que se destina a ser lida na folha diária ou na revista. Difere, porém,
da maneira substancialmente jornalística naquilo em que, apesar de fazer do
seu cotidiano o seu húmus permanente, não visa à mera informação: o seu
objetivo, confesso ou não, reside em transcender o dia a dia pela universalização das suas virtualidades latentes, objetivo esse via de regra minimizado
pelo jornalista de ofício. O cronista pretende-se não o repórter, mas o poeta ou
o ficcionista do cotidiano.\footnote{\textsc{moisés}, Massaud. \textit{A criação literária II}. São Paulo: Cultrix, 2006, p.\,104.}
\end{quote}

A inscrição desse livro no gênero das crônicas envidencia"-se pois, por mais assustadoras que algumas histórias possam parecer ao leitor, elas são reais ou contêm muito pouca ficção, e foram construídas a partir de depoimentos colhidos em atendimentos médicos, porém não visando somente a intervenção médica.

Apenas ``o canto da noia'' surgiu de uma visita que o autor fez a uma
certa ``cracolândia'' da Zona Leste de São Paulo, visita que acabou sendo uma
consulta coletiva a céu aberto.
Todas as informações referentes ao ``Sobrevivente de Troia''
não chegaram ao autor através de atendimentos médicos e sim através de atividades com
seus grupos de teatro.

O lastro com o real também permite a aproximação do livro de Luis Marra com outro gênero literário: a biografia. Ao abordar a questão da dependência
química sob a perspectiva dos próprios dependentes, o médico-escritor chega praticamente a traçar uma pequena biografia dessas pessoas.

Convencionalmente, a biografia é o registro, em forma narrativa, de acontecimentos e fatos
particulares à vida de uma pessoa. Pode ser narrado em terceira pessoa,
como é o caso da obra \textit{Crônicas do crack}, ou em primeira pessoa, no caso de uma
autobiografia.

Em geral, prefere"-se a ordem cronológica dos fatos para a narrativa. Mas
isso nem sempre é o caso, em especial, na situação em que a pessoa
biografada é entrevistada e vai contando a sua história conforme vai se
lembrando dela.


