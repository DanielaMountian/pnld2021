\chapterspecial{O canto da noia}{}{}
 

\epigraph{Ma tu perché ritorni a tanta noia?\\ Perché non sali il dilettoso
monte\\ ch'è principio e cagion di tutta gioia?}{\versal{Dante Alighieri}, \textit{Divina Comédia}, canto I} 



\epigraph{This is the way the world ends\\ This is the way the world ends\\ This
is the way the world ends\\ Not with a bang but a whimper}{\versal{T.S.\,Eliot}, \textit{Os homens ocos}} 

 


Na Rua 4 em São Miguel Paulista, Zona Leste de São Paulo, existe um
casarão onde se ensina esperanto. O~casarão é uma lembrança de um
passado recente, e foi construído em outros tempos quando por aqui
existia um progresso industrial.

Na fachada do casarão, entre duas toscas colunas gregas, há frases em
esperanto. Lembrando um tímido pórtico.

O resto da vizinhança é apenas o resto.

\asterisc{}

O braço final da Rua 4 --- um beco sem saída --- é conhecido
popularmente como ``rua dos noias''. Há também outros becos próximos
para onde sempre retornam, ou para onde sempre ``recaem'', muitas
pessoas; e para onde algumas se vão de vez e outras se aventuram, tal
como nós --- visitantes sem Virgílio em vigília crepuscular atrás de
embalos químicos de outros.

Sem poema do beco, embora o que eu veja seja o beco.

Mas hoje, como desde muitos séculos, existe por perto algum
\emph{dilettoso monte}, a não muita distância, e que é reduto de
privilegiados.

\asterisc{}

Neste canto nada separa claramente este lugar do entorno da cidade. O~Aqueronte é um córrego imundo. Os Carontes são ``olheiros'', ``gansos'',
``malas'', ``vapores'' e ``aviões'' --- uns malditos porteiros da noite.

O único pórtico é o do casarão onde se ensina esperanto. As frases neste
pórtico poderiam encobrir a famosa inscrição que convida a deixar de
fora a ``última que morre'': \emph{lasciate ogni speranza, voi
ch`intrate.}

Todavia, nem sempre é assim por aqui, porque tudo é maleável nestes
caminhos que percorremos através de ilusões químicas alheias.

\asterisc{}

Se aqui estamos em terra de todo mundo e se aqui estamos em terra de
ninguém, vejam vocês todos, meus leitores, a paisagem que se estende
pela Rua 4.

Um pedaço de rua margeia uma linha de trem em proximidade a uma avenida
onde, do outro lado, fica o restante de uma gigantesca fábrica de
produtos químicos com galpões abandonados de vidros partidos, lembrando
um monstro pré"-histórico deitado, as chaminés eretas exibindo potência
industrial.

No canto noioso quase sem calçamento existem muitos tufos de mato, lixo
empilhado e pedras ordinárias. Onde transitam, neutros pela noia,
inocentes velhos aposentados e também tipos desempregados ou apenas
desocupados.

Uma escadaria caindo aos pedaços, subindo do beco até a avenida, é uma
das fronteiras da Rua 4. Ao longo da avenida, para os lados da estação
de trem de São~Miguel, há um ruidoso comércio popular: pequenos bares,
lojinhas de um e noventa e nove, hotéis nada familiares, barracas de
camelôs, e puteiros com trepada a doze reais sendo que a profissional do
sexo recebe seis. Seis paus, como dantes se falava.

Muita noia. Alguém duvida?

\asterisc{}

Um longo muro corre paralelo à linha do trem, dividindo um nada além de
uma noia aquém.

O entorno é no mínimo curioso: de um lado o comboio mecânico; do outro
um trem de gente em fila de duvidosa alegria.

Enquanto viajam velhos vagões, ajeita"-se uma gente nova e velha
acocorada em viagens de outra espécie, parados passageiros em químicos
embalos. Uns tomados --- fumando pedras ---, outros morgando sem nada a
fazer. Uns puramente loucos, outros loucos de pedra, e por pedra todos
ligados numa má educação pela pedra.

\asterisc{}

Pipocam moleques de rua, meninas bonitinhas, meninas recém-enfeiadas,~umas 
achadas e umas perdidas; pipocam putas novas e putas
velhas, umas carcomidas, outras, portadoras do vírus da Aids; circulam
malandros traficantes, míseros cafetões, noias que são ladrões, noias
que são apenas noias.

Eu sigo sob a sombra protetora de uma gente que dá assistência a essa
outra gente. Eles sempre retornam a tanta noia para fazerem a tal
``redução de danos''. Eles se infiltram, suaves e benfazejos, neste
canto como se tivessem passaporte exclusivo, azul, diplomático.

Nesta missão seguimos junto aos anjos até um pouco além do cair da
noite.

\asterisc{}

Pois olhem bem agora. Mirem a paisagem.

Surge aqui uma moça magérrima.\\ Ela veste mini-blusa preta que descobre
a barriga. Ela lembra uma vedete de show de variedades. Súbito se agita
e, sendo mal passista, arremeda um tango. Quando nos vê põe entusiasmo
na voz como se fosse mulher de boate de segunda anunciando um mísero
show de malícia e sedução.

Ela se transfigura, a voz fica áspera, ela enfia a mão pelos cabelos, se
torce toda, amaldiçoa os que entram neste canto. Vomita palavras:

— Vermes, vermes, vermes.

Inclui na plateia uma mulher madura à distância sentada numa calçada
partida e tomando cachaça, parecendo uma dona de casa ordinária.

A moça treme. Seus olhos faíscam. Ela escancara a boca e deixa entrever
dentes podres. Uma baba escorre da sua gengiva gasta. Súbito ela se
acalma, fica com um jeito comum de muitas mães, e --- feito mãe coruja
--- fala carinhosa de um filho de doze anos que é um homenzinho e já tem
namorada.

De repente ela se horroriza com medo da polícia. Afirma que juraram
matá"-la e que sempre o fazem. Confessa várias passagens atrás das
grades. Deseja internação em casas de recuperação. Apega"-se e
desapega"-se a quaisquer vozes sedutoras, liga"-se frouxamente a malandros
oportunistas conquistadores, a chefes do tráfico e a pastores de igrejas
pentecostais. Loucamente acrobática, faz de si própria um pequeno
carnaval pessoal entre risos e lágrimas.

Sim, tudo isso é muita noia. Alguém duvida?

\asterisc{}

Depois aparece um tipo cujo semblante transita entre o malicioso e o
pesado. De profissão ladrão --- diz ele sem delongas. E~revela suas
``fitas'' diárias.

Metódico, hospeda"-se em hotel ordinário para seguir no seu caminho
diário das pedras. Confessa que se fosse ladrão não viciado já teria
feito seu pé"-de"-meia. Aparenta uns trinta anos, tem tatuagens macabras,
tem marcas de feridas na cabeça e nos braços, olhos fundos brilhando em
perspicácia malandra.

Ele aprecia a retórica vulgar. Parece estar acostumado a porcos
discursos moralistas contra o vício no intuito de atiçar a piedade
alheia. Vem com uma conversa típica de bandido, com meias verdades e
meias mentiras, várias passagens por cadeias, o diabo feito lei e fora
da lei, o abandono da família, o futuro ausente, a miséria e o cacete. É~uma mesma história. Hoje como dantes.

\asterisc{}

Seguimos todos caminhando ao longo do muro.

Irrompe na cena um tipo louco de pedra a gritar. Parece um pregador
evangélico de subúrbio surtado que, de repente, some de vista.

Prosseguimos em cautelosos passos ritmados.

\asterisc{}

Nossos olhares se voltam para um garoto deitado entre tufos de mato e
papelão. Ele tem cheiro de solvente e não acorda com nossas vozes, nem
com toques no ombro. É~um menino loirinho amorenado tipo sarará. Tem
cicatrizes horríveis na face. É~chamado por aqui de ``noinha'' em
apelido adocicado, mascote deste canto.

Há pouco tempo jogaram gasolina no seu corpo e atearam fogo. Ele virou
ínfimo herói de si mesmo, sacudiu as chamas na poeira e deu a volta por
cima.

Adiante encontramos um rapaz simpático, olhos vermelhos inquietos numa
busca perdida. Percebe"-se que está possuído e olha para cima em gestos
caprichosamente confusos. Tem pouco mais de vinte anos, é infantil,
ingênuo na sua química fissura. Fala com pieguice de uma paixão
romântica que reside na pessoa ao lado.

Mas ele não olha diretamente para seu objeto do desejo ao apontar ---
com a mão desvinculada do olhar --- para uma morena bonita de cócoras
que, resignada, ri dos entusiasmos românticos do rapaz.

Ele é céu e ela é terra. Alucinado, ele é ridículo em seus arroubos, mas
sabe que neste território vale-tudo ao mesmo tempo que vale nada. Uma
vida humana custa aqui um real.

Ela, lúcida e sem químicos embalos, confessa ser mulher de bandido. Teme
que seu sombrio cara metade, em fuga já anunciada da penitenciária,
deite os dois amantes por terra rasgados de lâmina ou furados de bala.

Mas o rapaz tem a indiferença dos contempladores dos abismos passionais.
Ele se acomodou no fundo do seu poço e acredita que a distância para
cima seja infinita. Por isso proclama sua paixão acima dos perigos
mundanos. Fissurado, viaja rumo às estrelas de sua eleição como se o
caminho das pedras seguisse muito além da fumaça soprada rente ao céu da
boca.

Ah, isso lembra~\emph{stairway to heaven,}~lembra~\emph{knocking on
heaven's door}.~Anos sessenta, anos setenta!

E sempre cabe um pouco de Oriente nesses embalos nostálgicos! Os Beatles
e os Rolling Stones foram meditar na Índia, não foram?! Isso mesmo.
Tanta gente ouviu e viu \emph{lucy in the sky with diamonds!} Pois esta
``brisa'' --- profana ou sagrada --- faz parte da busca pela humanidade
de uma repetitiva sabedoria perene.

Não é à toa que um dos anjos da noite faz uma revelação: é budista.
Então ele nos conta a velha metáfora da flor de lótus que, do pântano,
mira o céu.

O rapaz simpático mal escuta a metáfora. Ele metaforiza a modo próprio,
fixado nas invisíveis estrelas por detrás do véu de imundície que cobre
a Pauliceia. A~morena bonita, de cócoras, torce a cabeça e sorri. Outros
viciados, também de cócoras, tamborilam e pitam seus humildes
cachimbinhos. Todos juntos parecem uma tribo, são mesmo uma tribo, e a
cidade tem várias.

 

Há um silêncio de falsa calmaria. Nós permanecemos olhando parados como
se o tempo ficasse suspenso.

 

Perpassa no ar uma ilusão geral como se um jogo de dados abolisse o
acaso.

 

O trem passa.~Dá impressão que ainda apita, que sai vapor da
maria"-fumaça. Mas é tudo brisa.

 

A terra treme um pouco, em transe, e a terra, neste canto, é gasta entre
a potência e o ato por onde cai a sombra.

 

Esta também é uma forma em que o mundo acaba.

 

No entanto, tudo retorna a um ponto de partida de modo a cumprir
um~retorno. E, se fumaça sobe, é a que vem dos cachimbinhos e junto a
uma voz solitária cantarolando um rap: ``o beque está queimando, fumaça
sobe… Fui atrás do paraíso e não encontrei nada não''. É~tudo pó e
pedra.

 

A missão dos anjos da noite está terminando. Organiza"-se o encerramento.
Na saída, como também na entrada, não há óbulos pois os anjos estão
isentos de taxas de fronteiras. Os Carontes estão amansados e suaves, e
com eles basta a linguagem maneira: --- é isso aí mano, tá limpo, tudo
beleza, tudo firmeza, falô irmão.

 

Cai noite funda.

Enquanto nos retiramos, a moça de blusa preta e barriga de fora continua
a vomitar as mesmas palavras contra o mundo: vermes, vermes, vermes. O
``noinha''~mascote ressona com a face colada na terra e poderá amanhecer
com a boca cheia de formigas. O~de profissão ladrão ginga ao léu
imaginando sua próxima ``fita'' pedra após pedra. O~rapaz simpático e
possuído segue olhando as estrelas invisíveis do céu paulistano, e
mirando, talvez, algum \emph{dilettoso monte}.

Ao longe, em elevações de terreno, reside uma pequena burguesia
suburbana: são comerciantes, profissionais liberais e similares. Por lá
transitam os relativamente felizes, os que têm mais e choram menos. E~todos os outros.

O inferno são os outros.

 

Mirem"-se um pouquinho na Rua 4.

Hoje, como antes.

De resto,~\emph{perché ritorni a tanta noia}?
