\chapterspecial{Vida e obra}{}{Bruno Lima}

\section{Sobre o autor}

\noindent{}Luiz Gonzaga Pinto da Gama nasceu em Salvador, Bahia, na manhã
de 21 de junho de 1830. Filho de uma mulher negra chamada Luiza Mahin,
africana da região da Costa da Mina, e de um homem de aparência branca,
herdeiro de uma família rica de ascendência portuguesa, cujo nome até
hoje ninguém sabe. A razão para não sabermos o nome de seu pai é
absolutamente impactante: quando Luiz Gama tinha dez anos de idade, seu
pai o vendeu como escravo. Já de sua mãe, Luiza Mahin, Luiz Gama conta
muita coisa. Além do seu nome, da sua cor e do lugar em que nasceu, Luiz
Gama fala que sua mãe era uma quitandeira muito trabalhadora, que era
muito bonita e, pelo que diz, parece que também era muito brava. Luiza
Mahin gostava muito de política e era bastante destemida. Ainda quando o
menino Luiz era uma criança pequena, Luiza Mahin lutava pela liberdade
dos escravizados negros. Lutando por justiça, Luiza Mahin foi presa mais
de uma vez. Quando Luiz tinha sete anos de idade, sua mãe foi ao Rio de
Janeiro e nunca mais voltou. Teria sido presa de novo? O pequeno Luiz
ficou sem saber. Muitos anos depois, ele foi algumas vezes ao Rio de
Janeiro procurar por sua mãe e soube, finalmente, que ela foi não só
presa, mas também, como tudo indicava, deportada de volta para a África.
Luiz Gama nunca mais voltou a ver sua mãe. Mesmo assim, ela continuou
sendo seu maior exemplo de vida, assim como sua maior inspiração na
carreira profissional a que se dedicaria anos mais tarde. Mas, como
dito, aos dez anos de idade, num ato estúpido de traição e covardia, seu
próprio pai o vendeu, provavelmente a pretexto de pagar dívidas de
bebidas e jogatinas. O violento trauma que sofreu acompanhou a vida toda
de Luiz Gama como uma sombra. Era uma ferida incurável no peito daquele
que viria a se tornar o maior abolicionista da história do Brasil.

De Salvador, o pequeno Luiz viajou no porão de um navio infestado de
ratos até o Rio de Janeiro. Ele ficou poucos dias no Rio de Janeiro.
Tempo suficiente, contudo, para ser comprado como escravo na cidade que
mais vendia escravos no mundo. De lá, seguiu nauseante viagem, também de
barco, para Santos, litoral de São Paulo. De Santos, o menino de dez
anos de idade subiu a pé a tão íngreme quanto perigosa Serra do Mar e
foi, junto a outros homens, mulheres e crianças, para a cidade de
Campinas, no interior de São Paulo. Forçado, sem dúvida, talvez até
mesmo acorrentado nas mãos e nos pés, o pequeno Luiz percorreu
aproximadamente 150 quilômetros a pé! Mais do que uma viagem penosa, a
jornada de Salvador até Campinas foi um verdadeiro tormento, que
deixaria marcas profundas na alma do menino de dez anos de idade. Vale
registrar que o terrível Antonio Cardozo, o homem que comprou e
escravizou o pequeno Luiz --- e muitas outras pessoas --- no Rio de Janeiro,
queria revendê"-los por preço mais alto em São Paulo. Assim, logo que
chegou em Santos quis vender os homens, mulheres e crianças que tinha
comprado na capital do Império. Porém, embora os compradores escolhessem
o pequeno Luiz, recusavam"-no imediatamente ao saber que ele era baiano.
Isso porque a Bahia era bastante conhecida como terra em que negros e
negras se organizavam para lutar por suas liberdades. Como era o caso de
sua mãe, Luiza Mahin! No final das contas, ninguém quis comprar o menino
negro da Bahia. Nem em Santos, nem em Jundiaí e nem em Campinas. Restou
ao criminoso contrabandista Antonio Cardozo escravizar Luiz em sua
própria casa, no centro da cidade de São Paulo, próximo da praça da Sé.
Durante oito anos, ou seja, dos dez aos dezoito anos de idade, Luiz Gama
viveu em cativeiro e trabalhou em muitos ofícios, principalmente como
sapateiro.

Mas a sua vida começou a mudar quando ele conheceu um amigo. Antonio
Rodrigues do Prado Júnior, o novo amigo de Luiz Gama, tinha dezessete
anos de idade e vinha a São Paulo estudar humanidades, visando, em
seguida, entrar na Faculdade de Direito do Largo de São Francisco.
Antonio deu aulas a Luiz durante o ano de 1847. No ano seguinte, em
1848, o menino negro vendido pelo pai branco já sabia ler e fazer contas
matemáticas. Alfabetizado, Luiz deu um passo além: conseguiu
secretamente provas incontestáveis de sua liberdade, provavelmente a
partir da leitura de documentos, e fugiu do cativeiro da casa do
contrabandista Antonio Cardozo. Era um caminho sem volta.

Para iniciar a nova vida, Gama assentou praça no quartel militar. Ao
longo de seis anos, exerceu diversas funções, chegando até a ser cabo de
esquadra. A vida militar, contudo, não era o que ele queria. Gama
desejava profundamente dedicar"-se ao mundo das letras e do direito.
Conseguiu um voto de confiança do delegado de polícia da capital e
começou a trabalhar com ele, primeiro como escrivão e depois como
amanuense, um cargo que gerenciava escrivães e outros funcionários. Na
polícia, Gama cuidava de muitos assuntos. Era um funcionário bastante
ativo, fazia investigações policiais, inquiria criminosos suspeitos,
testemunhas, bem como escrevia inúmeros tipos de documentos, cartas,
relatórios e inquéritos oficiais.

É verdade que Gama sempre achava tempo nas repartições de polícia para
ler poesias, sua paixão, e clássicos da literatura antiga e moderna.
Cada vez mais, Gama buscava a leitura como companheira, certamente
pensando longe, isto é, na busca pela liberdade de todos os seus
companheiros --- mais de um milhão de pessoas! --- que ainda estavam
acorrentados ao tenebroso cativeiro que era a escravidão comandada pelos
homens brancos.

\section{Sobre a obra}

Não deve ter demorado muito tempo para Luiz Gama descobrir"-se poeta.
Afinal, se contarmos do momento em que aprendeu a ler até a publicação
de seu primeiro livro de poesias, se passaram apenas doze anos. Assim, do
período que vai de 1847 a 1859, Gama deve ter estudado à beça, lendo
quase tudo que lhe caísse nas mãos. Porém, se é verdade que a escrita
vinha depois da leitura, Gama primeiro aprendeu a ler e depois a
escrever. Isso significa que, se considerarmos o aprendizado da escrita
como ponto de partida para contagem do tempo, Gama certamente levou
menos do que doze anos entre aprender a escrever e lançar um livro. A
rapidez com que aprendeu e logo dominou a arte da escrita era resultado,
sem dúvidas, de uma avassaladora fome de conhecimento.

Escrever um livro nunca foi fácil, mas, nas condições tão adversas que
enfrentava, Gama alcançou um feito heroico. Não se tratava apenas de
mais um livro de poemas. Os leitores brasileiros nunca tinham visto nada
igual. As \emph{Primeiras Trovas Burlescas} de um certo Getulino, o
pseudônimo adotado por Luiz Gama, foram verdadeiramente uma obra
inaugural. Nunca antes um homem negro tinha escrito um livro em toda a
história do Brasil. Gama, portanto, inicia um ciclo, que é sem"-fim, de
poetas negros na história da literatura de língua portuguesa.

Publicado em São Paulo no ano de 1859, o livro \emph{Primeiras Trovas
Burlescas (\textsc{ptb})} era composto de 22 poemas autorais. Dois anos depois,
em 1861, Gama dobrava a aposta e lançava uma segunda edição das
\textsc{ptb}, agora no Rio de Janeiro, contando o livro com um total de 39
poemas de sua lavra. Pelo tamanho do novo volume, além das correções que
o próprio Gama fez em diversas poesias, pode"-se dizer que a segunda
edição se tornou a edição definitiva. Nela, pode"-se ler a melhor versão
dos sentimentos do jovem poeta. Estava claro que ele vinha para ficar.
Gama demonstrava talento, erudição e coragem. Enquanto suas poesias
líricas primavam pela elegância e uma espécie de ternura sentimental, os
seus versos satíricos, por outra parte, provocavam a cólera de alguns
poucos e o riso alegre de muitos. Crítico dos costumes sociais de uma
sociedade opulenta, violenta e genuinamente escravocrata, Gama
ridicularizava os poderosos e as modas de ocasião. Da estética às
estruturas de poder, tudo era tema para o poeta criar uma rima feita
para a reflexão e o divertimento de seus leitores. As \emph{trovas
burlescas} eram basicamente isso: feitas sob medida para rir, é claro,
mas para que o leitor se sentisse provocado e então questionasse os
costumes de seu país e de sua época.

O poema \emph{Quem sou eu?} é o mais famoso de todos deles. A pergunta
existencial guardava uma reflexão singular. Qual o lugar de Luiz Gama no
mundo das letras? Qual o seu lugar naquela sociedade racista e
escravocrata?

\begin{verse}
Se negro sou, ou sou bode,\\
Pouco importa. O que isto pode?
\end{verse}

Nos versos de Getulino, o nome artístico de Gama, se lê as angústias do
jovem poeta negro com o mundo de vaidades da elite branca brasileira. 
Gama parecia tomado de espanto, ao perceber que a elite que se pretendia civilizada e culta
era a mesma que mandava torturar e escravizar homens, mulheres
e crianças negras. A sátira de Getulino, portanto, foi a válvula de
escape para que ele extravasasse todo seu espanto diante da cínica e
violenta sociedade escravista de São Paulo. Por outra parte, o poema
\emph{Minha Mãe}, escrito em delicado lirismo, enaltecia a figura
materna e o fazia pela beleza da negritude:

\begin{verse}
Era mui bela e formosa,\\
Era a mais linda pretinha,\\
Da adusta Líbia rainha,\\
E no Brasil pobre escrava!
\end{verse}

Gama é um poeta talentoso. Há verdade, beleza, sentimento e perspicácia
em cada verso e estrofe. Há, em síntese, uma voz poética original que
tem seu lugar definitivamente cravado na história da literatura
brasileira. Uma voz autêntica que, não se bastando com a métrica da
lírica e da sátira poética, achou seu espaço em outras formas de
discurso e conhecimento, como o jornalismo, a política e o direito.

\subsection{Luiz Gama e a imprensa}

Gama conhecia a imprensa por dentro. Numa época em que as tipografias
(onde se faziam os jornais) estavam se modernizando aceleradamente, com
novas máquinas e divisões de trabalho mais específicas, pode"-se dizer
que Gama jogou em todas as posições da imprensa, desde a entrega do
jornal na porta da casa do assinante até a criação de uma empresa de
comunicação. Por um lado, ele sabia operar máquinas trabalhosíssimas e
possuía larga experiência na composição tipográfica, isto é, em passar o
texto manuscrito, letra por letra, para a forma de impressão. Por outro
lado, Gama também dominava o trabalho intelectual de criação de um
<<<<<<< HEAD
jornal --- que era muito mais do que colocar ideias no papel --- e sabia
=======
jornal – que era muito mais do que colocar ideias no papel – e sabia
>>>>>>> 2a773d7f1f8d3d7371809c1b8c5e33c46afcf79c
como gerir financeiramente uma empresa com seus empregados.

Assim, combinando os conhecimentos do rigoroso e detalhista trabalho
manual com o inventivo e corajoso trabalho intelectual, Gama criou seu
espaço na imprensa de São Paulo. Por décadas, esteve ou nos bastidores
ou na linha de frente de um determinado jornal. Às vezes, é bom que se
diga, esteve simultaneamente tanto nos bastidores quanto na linha de
frente de um jornal!

Mas foi como linha de frente que Gama passou à história. Em 1864, ele
fundou o primeiro jornal ilustrado de São Paulo e um dos primeiros do
Brasil: o periódico satírico \emph{Diabo Coxo}. Como se vê pelo nome,
Gama queria chamar a atenção dos leitores e provocá"-los já na simples
leitura do nome do jornal. Para pensarmos no tamanho da provocação,
basta que lembremos que o Brasil era um país que constitucionalmente
tinha uma religião oficial, a saber, a religião católica. Portanto,
falar em diabo era arrumar encrenca na certa. Gama seguiu com o
\emph{Diabo Coxo} até 1865 e depois fundou outros jornais, entre eles,
<<<<<<< HEAD
\emph{O Cabrião} (1866--1867)\emph{, Democracia} (1867--1868) e
\emph{Radical Paulistano} (1869--1870).
=======
\emph{O Cabrião} (1866-1867), \emph{Democracia} (1867-1868) e
\emph{Radical Paulistano} (1869-1870).
>>>>>>> 2a773d7f1f8d3d7371809c1b8c5e33c46afcf79c

Se o tal do \emph{Cabrião} era tão satírico quanto o \emph{Diabo Coxo},
criticando os costumes sociais, estéticos e morais da sociedade
escravocrata, \emph{Democracia} e \emph{Radical Paulistano} eram bem
diferentes em forma e estilo. Não haveria mais lugar para os desenhos
que arrepiavam o público, nem para os textos que escandalizavam o
cinismo dos moralistas de plantão. Agora, com \emph{Democracia}, Gama
falaria principalmente de educação. O direito à alfabetização, ao livro
didático e a poder frequentar escolas, por exemplo, foi tratado por Gama
como obrigação do Estado e um direito humano básico. Na sequência da
\emph{Democracia}, Gama fundou o \emph{Radical Paulistano}, jornal
republicano e abolicionista que incomodou profundamente os interesses
dos possuidores de gente escravizada.

Esse foi um período, em especial o ano de 1869, muito agitado na vida
pessoal e profissional de Gama. Por sua prática abolicionista, sobretudo
na imprensa e em particular no \emph{Radical Paulistano}, Gama foi até
demitido de forma bastante violenta do cargo público que tinha na
polícia de São Paulo. Coincidência ou não, pouco mais de um mês depois
de sua demissão, o seu \emph{Radical Paulistano} fechou as portas.

De modo tão surreal quanto enigmático, Gama sairia da polícia por uma
porta e entraria para a advocacia por outra. Ele continuaria forte e
enérgico na imprensa. Durante todos os anos, até a sua morte, em 1882,
Gama foi presença garantida nas colunas da imprensa. Agora, mais do que
nunca, o direito seria a sua voz. O abolicionismo não poderia parar.

\subsection{Luiz Gama e o direito}

Antes dos livros e processos, Gama começou a ler o direito no mundo e
nas relações sociais. Por que alguém é escravizado? Por que alguém é
castigado? São questões que Gama certamente se fez. Isso remonta, sem
dúvidas, a um período incerto de sua infância na Bahia. Porém, foi em
São Paulo, quiçá na época em que encontrou provas de sua liberdade, que
passou a ler e aprender o conhecimento jurídico. E não foi, de jeito
nenhum, um aprendizado superficial. Gama foi às fontes do direito
romano, português e brasileiro, passando até mesmo por estudos
comparados com outras matrizes teóricas, para formar seu pensamento
jurídico.

Como funcionário público da polícia de São Paulo, Gama viu, na prática,
muita coisa do mundo do direito. Se, por exemplo, a polícia era chamada
para intervir em uma briga de vizinhos por limites de suas propriedades,
Gama via na sua frente uma demanda de direito civil. Se, por outro lado,
a polícia era comunicada de um homicídio, Gama via na cena do crime uma
página de direito criminal. Porque, mesmo no tempo em que era soldado ou
escrivão de polícia, estava estudando livros e mais livros de direito.

No final do ano de 1869, quando foi demitido da polícia, num processo
que ele próprio denunciou como ilegal, Gama não tinha muitas opções para
seguir em frente com algo que tocasse diretamente o mundo do direito. A
advocacia era um dos caminhos que ele certamente mais desejava. Mas,
como praticá"-la, se ele não tinha título de bacharel por uma das duas
faculdades de direito do Império? Era quase impossível. Havia, no
entanto, uma maneira perspicaz para conseguir licença como advogado sem
diploma de bacharel: uma norma permitia que juízes autorizassem pessoas
a advogar, ainda que elas não tivessem formação acadêmica e desde que
nos limites da jurisdição daquele mesmo juiz. Deveriam, contudo, possuir
notório conhecimento jurídico. Em rápida síntese, foi o que Gama fez.
Pediu para um juiz uma espécie de licença para advogar. E conseguiu! A
história é longa\ldots{} Mas, por hora, basta que lembremos que, já naquele
momento, Gama era reconhecido, inclusive por professores da faculdade de
direito, como um grande intérprete das leis e do direito.

Como dito antes, Gama saiu da polícia por uma porta e entrou para a
advocacia por outra. Desde então, Gama tornou"-se advogado de fato e de
direito --- e de papel passado! Tornou"-se um advogado de destaque,
travando as maiores lutas judiciais de sua época e cidade. Ao longo de
mais de doze anos, Gama advogou para milhares de pessoas. Por seu
escritório da rua da Imperatriz passaram centenas de africanas e
africanos de Angola, Benguela, Cassange, Congo e Moçambique, por
exemplo, assim como europeus da Alemanha, Espanha, França, Portugal e
Itália, sem contar, é claro, a multidão de brasileiros de diversas
regiões do país.

Entre os seus clientes estavam milhares de homens, mulheres e crianças
negras e pardas, ilegítima e ilegalmente escravizadas. Gama
comprometia"-se a usar de todas as armas legais do direito para tirá"-los
da escravidão. Assim, escreveu petições, contestações, réplicas,
requerimentos, libelos, bem como um sem número de outros documentos e
diligências para demandar a liberdade de alguém violentamente privado
dela. Foram centenas de causas de liberdade. Foi muitas vezes derrotado,
é verdade, mas muitas vezes mais saiu do tribunal vitorioso. Em qualquer
caso, na alegria ou na tristeza, seu escritório não era só um escritório
de advocacia. Era um escritório da liberdade. Era lá onde as angústias
do réu e os desejos por alforria encontravam escuta atenta e promessa de
luta aguerrida nos tribunais. Era de lá, afinal, que Gama escrevia a
página heroica do abolicionismo no Brasil.

\subsection{Luiz Gama e a abolição}

A luta contra a escravidão é uma luta que sempre existiu no Brasil.
Pode"-se até dizer que nunca houve escravidão sem que houvesse
resistência a ela. A história que Gama nos conta é um dos muitos
capítulos dessa luta que durou séculos. A propósito, é o seu capítulo
final, que tomou corpo com a organização do movimento social
abolicionista no final da década de 1860. Gama era incontestavelmente o
principal líder abolicionista em São Paulo. Desde o princípio do
movimento social organizado, lá estava Gama promovendo reuniões,
publicações, concertos de teatro, arrecadação de fundos para alforrias,
alfabetização de escravizados e libertos, além, é claro, de uma chuva de
ações judiciais em favor da liberdade. Gama agiu em muitas frentes
abolicionistas. Na verdade, articulou ações distintas e deu unidade a um
estilo de ação política, ou o que a historiadora Angela Alonso chamou de
``estilo Gama de ativismo''\footnote{Cf. Alonso, 2015, p. 103.}.

Nos textos dessa coleção, veremos Gama em muitas frentes de batalha.
Veremos como ele começou o seu ``estilo de ativismo'' abolicionista, no
início de 1869, e como ele desenvolveu um repertório de argumentos
jurídicos em favor da liberdade ao longo de mais de doze anos. Com
originalidade e coragem, Gama criou estratégias de liberdade
utilizando"-se com maestria o conhecimento normativo disponível nas
fontes do direito. Assim, veremos o momento inaugural dessa estratégia,
não à toa intitulado \emph{Questão de Liberdade}, que é seguido de
diversos textos em que ele apresenta dezenas de causas de liberdade e
quais seriam suas soluções jurídicas. Veremos, em síntese, o percurso
intelectual do maior jurista da história do país na criação de sua maior
obra literária: a escrita da abolição, como ele dizia, ``ao pé da letra'',
ou seja, na raiz das coisas.

A seção denominada \emph{O ás da Abolição} representa muito bem o
conjunto da obra. Nela, se lê um texto fantástico, aliás uma carta
fantástica, intitulada como \emph{Carta ao dr. Ferreira de Menezes}.
Essa carta é um dos mais importantes documentos da história da Abolição
da escravidão no Brasil. Pouquíssima gente conhece o texto na íntegra.
Assim, trazê"-lo de volta, cento e quarenta anos depois de publicado,
serve para renovarmos o tão necessário debate sobre o legado cruel da
escravidão e revisitarmos as experiências de liberdade, cidadania e
direitos no Brasil. Além de afigurar"-se como o mais longo texto em prosa
publicado individualmente por Luiz Gama, o que não é pouco, a
\emph{Carta ao dr. Ferreira de Menezes} revela acontecimentos e
histórias que oferecem uma nova perspectiva de análise para os leitores
em geral.

Escrita por um homem negro, para outro homem negro, na imprensa negra do
Rio de Janeiro, a \emph{Carta ao dr. Ferreira de Menezes} surge como retrato do mais radical
abolicionismo que teve lugar no Brasil. Dividida em onze diferentes
trechos, a \emph{Carta} é uma espécie de soco na boca do estômago,
tamanha a força das imagens e dos argumentos que Luiz Gama mobiliza para
apresentar dezenas de cenas de sangue e tortura ``neste país da sagrada
liberdade''. Na \emph{Carta}, Gama orienta o movimento abolicionista
para um novo estágio da luta política pela supressão da escravidão,
assim como documenta o passado da escravidão, a partir de um inventário
da crueldade dos possuidores de gente escravizada e da resistência
heroica dos escravizados em luta por liberdade.

Mas, mais do que isso, a \emph{Carta} é um documento, como confessa o
autor, destinado à ``posteridade'', às gerações futuras, que, por
tristes e múltiplas razões, só agora pode ler a mensagem, as denúncias,
os dramas, as histórias de vida e as memórias de um admirável
abolicionista ``digno das páginas da história''.

\section{Sobre o gênero}

A produção intelectual de Luiz Gama é tão vasta quanto diversa. Gama
começou a se dedicar ao mundo das letras a partir da escrita poética. No
ano de 1859, aos 29 anos de idade, Gama publicou seu primeiro livro,
composto de quase quarenta poemas líricos e satíricos. Desde então, Gama
nunca mais parou de escrever. Entrou para a redação de jornal e
desenvolveu especial talento para a crônica política, policial e
teatral. Gama levou muitos anos no jornalismo, integrando as equipes dos
melhores jornais da época e deixando, consequentemente, sua marca na
história da imprensa brasileira. Porém, quando se tornou advogado, sua
preocupação número um passou a ser o mundo do direito, que, como o leitor
pode imaginar, é uma parte do grande mundo das letras. Escrever petições
e requerimentos, além de outros documentos judiciários, exigia bastante
técnica. Mas não só. Para convencer juízes e jurados (além dos leitores
em geral, quando a causa se tornava pública), Gama tinha que aliar
técnica e retórica, aliás, diversas técnicas e sofisticados expedientes
retóricos. Desse modo, Gama desenvolve o que podemos chamar de
literatura normativo"-pragmática, que é, em síntese, uma linguagem que
usa fontes do direito para resolver casos concretos. Dominando metáforas
e métricas do cânone poético luso"-brasileiro, bem como, posteriormente,
a objetividade analítica do jornalismo político da época, Gama cria um
estilo singular para falar de direito, justiça e liberdade. Com tão
amplo repertório e técnicas literárias, Gama inaugura um novo tempo na
luta por direitos e liberdades no Brasil. O abolicionismo, pode"-se
dizer, começa e tem seu ponto de apoio na ação direta e na obra
literária de Luiz Gama.

\subsection{por que estudar luiz gama nos dias de hoje?}

A vida de Luiz Gama é um dos capítulos mais importantes da história do
Brasil. Sua obra literária é fantástica. A biografia do menino que
nasceu livre, mas que aos dez anos de idade foi escravizado pelo próprio
pai espanta, emociona e educa. Após oito anos de cativeiro, Gama
conseguiu provas de sua liberdade, entrou para a vida militar, publicou
livro e artigos na imprensa, tornou"-se funcionário público respeitado e,
ao fim da vida, chegou a ser considerado um dos maiores advogados do
país. A história de perseverança, superação e vitórias de Luiz Gama é um
exemplo imbatível, além de ser uma janela para os leitores saberem como
era o Brasil da época. A leitura, portanto, nos leva a ver de perto as
batalhas que o herói da Abolição enfrentou e também amplia nossa visão
sobre as lutas por direitos no Brasil.

Gama conta como ninguém a luta do abolicionismo. Líder político radical,
advogado, jurista e intelectual de primeira grandeza, além de educador e
poeta com assento definitivo na literatura brasileira, Gama foi um
cidadão exemplar que explicou muito bem as razões que colocariam fim à
monarquia e que levariam o Brasil até a Abolição e a República. É
verdade que ele não viveu para ver o seu sonho de um Brasil sem
escravizados se realizar. Apenas seis anos separam a morte de Luiz Gama,
em 1882, e a Abolição da escravidão, em 1888. Ainda assim, como
intérprete sofisticado que era da política e da história do Brasil, via
como inevitável o fim da sociedade escravocrata. Isso, aliás, servia de
combustível para Gama e seus companheiros lutarem mais e mais pela
liberdade imediata dos mais de um milhão de pessoas escravizadas que
habitavam o Brasil no início da década de 1880. Gama talvez seja o único
intérprete do Brasil que realmente compreendia a fundo o pensamento dos
escravizadores e dos escravizados. Ele conhecia a política da escravidão
e o racismo por dentro das estruturas de poder; conhecia em
profundidade, também, a sede de justiça dos humilhados e ofendidos de
sempre. Gama é um intelectual que leu o Brasil e escreveu para o futuro.
Resta a gente ler o passado e pensar o Brasil de hoje.


\begin{bibliohedra}
\tit{ALONSO}, Angela. \emph{Flores, votos e balas. O movimento abolicionista
brasileiro (1868--1888)}. São Paulo: Companhia das Letras, 2015.

\tit{AZEVEDO}, Elciene. \emph{Orfeu de carapinha: a trajetória de Luiz Gama na
imperial cidade de São Paulo.} Campinas, \textsc{sp}: Editora da \textsc{unicamp}, Centro
de Pesquisa em História Social da Cultura, 1999.

\tit{FERREIRA}, Ligia (org.). \emph{Com a palavra, Luiz Gama}. São Paulo:
Imprensa Oficial, 2011.

\titidem. ``Luiz Gama por Luiz Gama: carta
a Lúcio de Mendonça'', in: \emph{Teresa, Revista de Literatura
Brasileira,} 8/9, São Paulo, 2008, pp.\,300--321.

\tit{GAMA}, Luiz. \emph{Primeiras trovas burlescas de Getulino.} 1 ed. Rio de
Janeiro: Typ. Pinheiro \& Cia., 1861.

\tit{LIMA}, Bruno. \emph{The legal knowledge of Luiz Gama (1830--1882)}. Tese
de doutorado. Johann Wolfgang Goethe Universität Frankfurt am Main,
Alemanha, 2021 (em progresso).

\tit{MOORE}, Zelbert. \emph{Luiz Gama, Abolition and Republicanism in Sao
Paulo (1870--1888)}. Tese de doutorado. Temple University, Estados Unidos
da América, 1978.

\tit{SILVA}, João Romão da. \emph{Luiz Gama e suas poesias satíricas.} 2 ed.
Revista e aumentada. Rio de Janeiro: Livraria Editora Cátedra; Brasília,
Instituto Nacional do Livro, 1981.
\end{bibliohedra}
