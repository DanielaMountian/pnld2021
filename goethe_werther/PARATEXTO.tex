Johann Wolfgang von Goethe (Frankfurt am Main, 1749 -- Weimar, 1832)
licenciou-se em direito por insistência de seu pai, jurista renomado, e nunca
chegou a exercer plenamente a profissão.  Aos 25 anos, com a publicação
\textit{Os sofrimentos do jovem Werther}, Goethe tornou-se imediatamente
célebre. Um ano depois, a convite do duque Carlos Augusto, mudou-se para
Weimar, onde fixou residência até o fim da vida, com raras ausências, como a
viagem à Itália de 1786 a 1788. Os inúmeros poemas, dramas, romances,
traduções, ensaios sobre arte e natureza que produziu definiram o cenário
cultural da época, que passou a ser conhecida como \textit{Goethes Zeit}, tempo
de Goethe, incluindo movimentos como o classicismo, o romantismo e Tempestade e
Ímpeto (\textit{Sturm und Drang}). Fausto (\textit{Faust}), concluído em 1808,
é considerado seu texto mais relevante, mas, em vida, o escritor fora conhecido
principalmente como o autor de \textit{Werther}.  Ao lado de Schiller, Goethe
permanece o grande poeta da língua alemã.

Os sofrimentos do jovem Werther (\textit{Die Leiden des jungen Werthers}),
1774, é um romance em forma epistolar que narra a história
de um fracasso amoroso com final trágico. Werther retira-se ainda
jovem para uma pequena cidade, onde se apaixona por Carlota,
que está comprometida com Alberto. Goethe publicou
em 1787 uma segunda versão do texto, na qual algumas cartas
foram subtraídas e outras acrescentadas ou deslocadas.

A tradução anônima que reproduzimos tem como base a primeira
edição alemã de 1774 e foi originalmente publicada, no ano de
1821, em dois volumes pela Typographia Rollandiana de Lisboa,
Portugal, com licença expressa da comissão da censura.


