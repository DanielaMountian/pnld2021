\part{\textsc{para saber mais}}

\chapter[Sobre «O alienista», \emph{por Jean Pierre Chauvin}]{Sobre «O alienista»\subtitulo{De Maomé a Dom Pedro}}
\hedramarkboth{Sobre «O alienista»}{}

\begin{flushright}
\textsc{jean pierre chauvin}
\end{flushright}

\epigraph{Amava muito fazer justiça com direito. E, assim como quem faz
correição, andava pelo Reino; e, visitada uma parte, não lhe esquecia de
ir ver a outra; em guisa que poucas vezes acabava um mês em cada lugar
de estada}{\textsc{fernão lopes}.\footnotemark}\footnotetext{Lopes, Fernão. \emph{Crónica de D.\,Pedro}. Lisboa: Portugália Editora Lisboa, 1967, p. 44.}

\epigraph{\emph{Dios los remedie --- dije el cura ---, y estemos a la mira:
veremos em lo que para esta máquina de disparates de tal caballero y de
tal escudero, que parece que los forjaron a los dos em una turquesa y
que las locuras del señor sin las necedades del criado no valian un
ardite}}{\textsc{miguel de cervantes saavedra}.\footnotemark}\footnotetext{``-- Que Deus os ajude --- disse o padre ---, e fiquemos nós de atalaia; veremos onde vai parar essa máquina de disparates de tal cavaleiro e qual escudeiro,
pois parece que os dois foram forjados num mesmo molde e que as
loucuras do senhor sem as necedades do criado não valeriam mealha''
(Cervantes Saavedra, Miguel de. \emph{O engenhoso cavaleiro D.\,Quixote
de La Mancha} --- 2º livro. 3ª ed.
Tradução: Sérgio Molina São Paulo: Editora 34, 2012, p. 64).}

\epigraph{\emph{There is not anything which contributes more to the reputation
of particular persons, or to the honour of a nation in general, than
erecting and endowing proper edifices for the reception of those who
labour under different kinds of distress}}{\textsc{jonathan swift}.\footnotemark}\footnotetext{``Não
  há nada que contribua mais para a reputação das pessoas particulares,
  ou para a honra de uma nação em geral, que erigir e aprovisionar
  edifícios próprios para a recepção daqueles que lidam com diferentes
  espécies de sofrimento'' (Swift, Jonathan. \emph{A serious and useful
  scheme to make an hospital of incurables}. Londres: J. Roberts, 1733,
  p. \textsc{I}. [trad. minha]}

\noindent{}Durante bom tempo, \emph{O Alienista} foi aceito pacificamente como mais
um dentre os duzentos e poucos contos de Joaquim Maria Machado de Assis
(1839--1908) --- ao lado de narrativas muito mais breves, objetivas e mais
simples, como ilustram os demais textos da própria coletânea em que
apareceu. É curioso que isso ainda aconteça, considerando-se a extensão
que o texto ocupa em livro (noventa páginas, na primeira edição de
\emph{Papéis Avulsos}, de 1882, que somava trezentas e quinze). A
questão pode parecer de somenos importância, mas não seria impossível
supor que Machado tenha recorrido a uma narrativa com extensão maior que
a habitual para sugerir um gênero intermediário entre o conto e o
romance.

Quer dizer, \emph{O Alienista} poderia ser aproximado do gênero novela,
como se se tratasse de um experimento ficcional que transparecesse na
forma em que foi estruturado. Hipótese não destituída de toda razão, em
vista de um enredo cheio de peripécias, respaldado na controversa figura
do Dr.\,Simão Bacamarte: legítimo e ambíguo representante da coroa
portuguesa, na modesta vila. Na síntese de John Gledson:

\begin{quote}
\emph{O Alienista} não é um conto, e é difícil dizer a que gênero
pertence. Como o Cândido de Voltaire, que talvez seja seu parente mais
chegado, é um ataque às ingenuidades filosóficas do momento. Junto com o
humanitismo de \emph{Memórias Póstumas} e \emph{Quincas Borba}, é uma
investida devastadora contra o ``bando de ideias novas'', contra o
cientificismo.\footnote{Gledson, John. ``\emph{Papéis Avulsos}: um livro
  brasileiro?'' In: \textsc{machado de assis}. \emph{Papéis Avulsos}. São Paulo:
  Penguin Classics/Companhia das Letras, 2011, p. 18.}
\end{quote}

Hélcio Martins, José Aderaldo Castello e muitos outros constataram que a
ambivalência é um dos fundamentos do texto machadiano. O retrato que faz
das personagens é reforçado pelo modo desajeitado como elas se
comportam, o teor dicotômico do que pensam e o estilo indeciso de sua
fala: ``Desse repúdio às afirmações categóricas nasce o que o próprio
Machado de Assis chamou seu estilo ébrio, que vai guinando à direita e à
esquerda, andando e parando, ao passo que o leitor deseja a `narração
direita e nutrida, o estilo regular e fluente'\footnote{Afirmação do
  defunto/autor Brás Cubas, nas \emph{Memórias Póstumas}.}.''\footnote{Martins,
  Hélcio. ``A litotes em Machado de Assis''. In: \_\_\_\_\_. \emph{A
  rima na poesia de Carlos Drummond de Andrade} e \emph{outros ensaios}.
  Rio de Janeiro: Topbooks, 2005, p. 318.}

Outro aspecto a ser relembrado é que a novela circulou em folhetim,
antes de ser transportada como ``conto'' de abertura ao livro. Aliás,
``todos os contos de \emph{Papéis Avulsos} apareceram em jornais e
revistas antes da publicação em volume'', lembra John Gledson (2011, p.
33). No caso de \emph{O Alienista}, as partes saíram aos pedaços, na
revista \emph{A Estação}\footnote{``\emph{A Estação} foi fundada em 1872
  por Henri Gustave Lombaerts (1845--1897) com o nome de \emph{La Saison
  -- Jornal de Modas Parisienses}. Abaixo do título, na primeira página,
  vinha a informação de que era `Dedicado às Senhoras Brasileiras'.
  Publicava-se no dia 15 e no dia 30 de cada mês. Em seus primeiros oito
  anos, o jornal limitava-se a traduzir matérias alemãs e francesas. Em
  15 de janeiro de 1879, foi criada uma `Parte Literária', redigida no
  Brasil. A partir daí, o periódico passou a se chamar \emph{A Estação
  -- Jornal Ilustrado para a Família}. Circulou normalmente até 1904''
  (Teixeira, Ivan. \emph{O Altar \& o trono: dinâmica do poder em O
  Alienista}. Cotia: Ateliê; Campinas: Editora Unicamp, 2010, p. 47).},
entre outubro de 1881 e março de 1882.

Por que afirmo que se trata de uma novela, ao lado de \emph{Casa Velha}?
Porque, embora o fio condutor do enredo seja a prisão ou a liberdade dos
itaguaienses --- considerados mais ou menos loucos pelo extravagante
cientista ---, \emph{O Alienista} reserva espaço para narrativas
paralelas, dentre as quais a viagem (e fausta recepção) da comitiva
local, que viajara para o Rio de Janeiro; a disputa do poder entre os
barbeiros Porfírio Caetano das Neves e João Pina; as mudanças de posição
do vereador Sebastião Freitas --- figura que lembra em muito certos tipos
da política brasileira vigente, a oscilar ao (des) sabor dos recentes
golpes.

Dito isso, passemos a diferente matéria. Comecemos por uma breve
apreciação do enredo. No final do século \textsc{xviii}, um brasileiro erudito,
com largos conhecimentos em medicina, ciência, filosofia e religião,
recusa o magnânimo convite do rei Dom João \textsc{v}, para ocupar altos postos
na administração portuguesa. Em vez disso, mete-se na vila de Itaguaí e,
com o apoio da câmara de vereadores, manda construir um faraônico asilo
para loucos, a que dá o nome de Casa Verde.

Embora a novela tenha sido escrita entre 1881 e 1882, Machado a recuou
para um século antes, quando a cultura e os modos traziam embutidos o
acento político português. A operação não era tão simples. Ao ambientar
a narrativa no final do século \textsc{xviii}, o escritor reconstituiu protocolos
da época. Os capítulos parodiam as crônicas produzidas durante o Antigo
Regime. O narrador estiliza a maneira formal de relatar acontecimentos
(daí as constantes referências a supostas ``crônicas da vila de
Itaguaí''). Mesmo as andanças de Simão Bacamarte pela Europa, antes de
retornar ao Estado do Brasil, aludem a um procedimento comum ao
Setecentos luso-brasileiro, que respaldava a carreira dos profissionais
responsáveis pela saúde. De acordo com Daniela Buono Calainho:

\begin{quote}
O que marcou a figura do médico no Antigo Regime foi sua formação
erudita, acadêmica instrumentalizando-o na sua prática clínica e
terapêutica, distinguindo-se de outras categorias: os médicos usavam as
suas credenciais de cavalheiros eruditos para se separar e distinguir
dos que estavam mais abaixo na hierarquia social, como os cirurgiões,
barbeiros, boticários ou as parteiras.\footnote{Calainho, Daniela Buono.
  ``Curas e hierarquias sociais no mundo luso-brasileiro do século
  \textsc{xviii}''. In: \textsc{monteiro}, Rodrigo Bentes et al (orgs.). \emph{Raízes do
  privilégio: hierarquias sociais no mundo ibérico do Antigo Regime.}
  Rio de Janeiro: Civilização Brasileira, 2011, p. 486.}
\end{quote}

Diversos poderes entram em imprevista consonância, na novela: o poder
político, o poder do saber, o poder da psiquiatria, o poder marital, o
poder da psiquiatria, o poder da ciência, o poder personificado pelo
médico. Não se esqueça que Simão Bacamarte contou, desde sempre, com o
aval da Câmara de Vereadores da vila de Itaguaí. Ao chamar a atenção
para as alianças escusas entre medicina e política, Machado de Assis
resgatava um dos traços marcantes da colonização portuguesa no Estado do
Brasil, que vigorou entre os séculos \textsc{xvi} e \textsc{xviii}. Como descreve Joaquim
Romero Magalhães:

\begin{quote}
Entre as estruturas profundas da América de colonização portuguesa uma
há que suponho de excepcional relevância: a organização municipal. Ao
criar, em 1532, a vila de São Vicente segundo a legislação ordinária do
Reino fixada nas Ordenações, o rei de Portugal estendia ao Brasil o
regime judicial e administrativo em vigor no espaço português da Europa
e Ilhas do Atlântico. Aí assento e se foi alargando além-mar a
instituição concelhia que viera a tomar forma até finais do século \textsc{xiv} e
que só o liberalismo derrubaria.\footnote{Magalhães, Joaquim Romero.
  \emph{Concelhos e organização municipal na época moderna}. Coimbra:
  Imprensa da Universidade de Coimbra, 2011, p. 121.}
\end{quote}

Do poder político para o poder da palavra. Eugênio Gomes viu, em \emph{O
Alienista}, forte remissão a um texto satírico, em que Jonathan Swift
``sugeria a criação de um hospital para incuráveis, incuráveis morais:
os escrevinhadores, os vadios, os incréus, os mentirosos e, além de
tantos outros, os que fossem incuravelmente vaidosos, fátuos e
impertinentes''\footnote{Gomes, Eugênio. ``Swift''. In: \_\_\_\_\_.
  \emph{Machado de Assis: influências inglesas}. Rio de Janeiro: Pallas;
  Brasília: \textsc{inl}, 1976, p. 40.}. Publicada um século e meio após o ensaio
do escritor irlandês, a novela machadiana parte de premissa similar.
Desde o princípio, o projeto de construção da Casa Verde patenteava a
desproporção entre a reduzida quantidade dos dementes que circulavam na
vila de Itaguaí e o custo social e financeiro envolvidos em sua
manutenção.

A dicção empregada por Swift irmana-se ao tom zombeteiro do narrador de
\emph{O Alienista}, que, em linguagem elegante e castiça, parodia o
vocabulário técnico e o modo como médicos e filósofos se expressavam em
tratados escritos a sério. ``Frio como um diagnóstico'', o cientista
brasileiro tende a ampliar seu repertório (muito particular) para
(re)definir as manias com que se depara. Poder-se-ia aplicar à novela o
que Dirce Côrtes Riedel afirmou sobre o conto ``O empréstimo'': ``O
próprio \emph{kitsch} é usado pelo narrador metaforicamente, para
configurar um personagem que vive das aparências, do efeito fácil
produzido sobre os outros. {[}\ldots{}{]} Nem falta ao narrador o
\emph{gran finale} de mau gosto''\footnote{Riedel, Dirce Côrtes.
  \emph{Metáfora, o espelho de Machado de Assis}. Rio de Janeiro:
  Francisco Alves, 1974, pp. 35--36.}. Na leitura de outro conto, ``O
imortal'', João Adolfo Hansen nota que a persona do narrador
``{[}\ldots{}{]} estiliza elementos microtextuais, como o léxico antigo:
o termo `aleivosia' é divertidamente típico. E frases inteiras, que é
impossível ler sem sorrir de cumplicidade, pois o \emph{kitsch} não é de
Machado de Assis''\footnote{Cf. Hansen, João Adolfo. \emph{``O Imortal''
  e a verossimilhança}, nesta edição.}.

Em paralelo com a desconstrução da linguagem desmesurada e postiça, a
convenção social é outro componente observado pelas criaturas
machadianas. Aparentemente reto e virtuoso --- como convém aos sábios ---,
Simão Bacamarte cumpre todos os protocolos de um sujeito bem posicionado
na ramificada hierarquia reinol. No plano pessoal, o douto cientista é
casado com Dona Evarista --- uma mulher sem sal, frívola e subserviente,
que parece ter sido escolhida segundo critérios puramente racionais. Ele
também conta com um grande aliado, o boticário Crispim Soares --- que
vive a secundar as falas do ``ilustre médico'' com um discurso
francamente bajulatório. À medida que Bacamarte investiga os casos de
fúria, depressão e mania, a quantidade de pacientes da ``Casa de
Orates'' aumenta exponencialmente, a ponto de abrigar ``quatro quintos''
da população de Itaguaí e arredores.

Porém, alheio às pressões da câmara, à grita dos populares e às lágrimas
da consorte, Simão Bacamarte segue o seu roteiro vigorosamente,
cumprindo-o à risca. Se na primeira fase das investigações sobre
patologias mentais, ele supunha que a demência fosse uma exceção em meio
à normalidade, no segundo estágio o médico inverte o critério e passa a
encarcerar justamente aqueles que demonstrassem comportamento mais
coerente consigo mesmos e os outros.

Algumas dentre as páginas mais fascinantes está nos capítulos finais.
Bacamarte recorre a diversos expedientes para (des)curar a já reduzida
população de seu asilo. Aqui, a mistura de vozes --- de Machado, do
narrador e das personagens --- relembra uma das características de sua
ficção. José Leme Lopes percebeu que ``Ao descrever os delírios dos
recolhidos à Casa Verde, notamos que o escritor parece caricaturar com
finura as observações famosas dos tratados contemporâneos sobre as
doenças mentais''\footnote{Lopes, José Leme. ``A propósito de \emph{O
  Alienista}''. In: \_\_\_\_\_. \emph{A psiquiatria de Machado de
  Assis}. 2\textsuperscript{a} ed. Rio de Janeiro: Agir, 1981, p. 27.}.

Dois internados, acometidos de rematada modéstia, são submetidos à cura
pelo alarido narcisista. Em um caso, o médico recorre à divulgação do
nome do paciente --- como se se tratasse de notável talento poético ---,
através da matraca; em outro, obtém autorização régia para fazer
funcionar uma agremiação literária (que lembra em muito algumas
Academias de vida efêmera que existiram no país, enquanto colônia de
Portugal).

Embora haja constantes alusões ao vice-rei Luís de Vasconcelos e Sousa
(evocado pela viagem da comitiva, ao Rio de Janeiro, deslumbrada com o
Chafariz das Marrecas, construído durante o seu mandato), o narrador
interpõe dados de época posterior no enredo. Ora remete ao profeta
Maomé, que viveu no século \textsc{vii}; ora avança até o primeiro reinado, com
paródias aos gestos de Dom Pedro \textsc{i}.

Repare-se que o ``Dia do Fico'' foi recuado cronologicamente, como se
``antecipasse'' o episódio em que Simão Bacamarte resiste aos revoltosos
que esperneavam e se espremiam em frente a sua residência. Ao tomar a
câmara de assalto, o barbeiro Porfírio assina uma carta dirigida aos
itaguaienses, como ``Protetor da vila'' --- em alusão ao título concedido
ao Imperador Pedro I, como ``Defensor Perpétuo do Brasil''\footnote{Cf.
  Monteiro, Tobias. \emph{História do Império -- O Primeiro Reinado}. 2
  vol. Belo Horizonte: Itatiaia, 1982; Viana, Hélio. \emph{História do
  Brasil: período colonial, monarquia e república}. 15ª ed. São Paulo:
  Melhoramentos, 1994; Lustosa, Isabel. \emph{D.\,Pedro \textsc{i}: um herói sem
  nenhum caráter}. São Paulo: Companhia das Letras, 2006.}, outorgado no
ano de 1822 (ou seja, trinta anos após os episódios transcorridos em
Itaguaí). De um lado, a novela remete ao tempo em que éramos colônia de
Portugal, situada no final do século \textsc{xviii}; de outro, a postura do
barbeiro Porfírio alude ao turbulento processo de Independência, com
todos os seus reveses e sob o jugo da Inglaterra.

Outro fator histórico. Sabe-se que o ano de 1822 foi marcado pela
assinatura de cartas públicas e manifestos por Dom Pedro \textsc{i}, através da
imprensa oficial. A carta de Porfírio Caetano das Neves pode ser lida
como paródia do discurso empolado e cheio de vênias que o protocolo real
exigia, ao escrever, especialmente quando se dirigisse ao povo:

\begin{quote}
O Congresso de Lisboa arrogando-se o direito tirânico de impor ao Brasil
um artigo de nova crença, firmado em um juramento parcial, e
promissório, e que de nenhum modo poderia envolver a aprovação da
própria ruína {[}\ldots{}{]}. Brasileiros! Para vós não é preciso
recordar todos os males, a que estáveis sujeitos, e que vos impeliram à
Representação, que me fez a Câmara, e o Povo desta cidade no dia 23 de
Maio, o que motivou meu Real Decreto de 3 de Junho do corrente ano
{[}\ldots{}{]}. A Minha Felicidade (convencei-vos) existe na vossa
felicidade: é Minha Glória Reger um povo brioso, e livre. Dai-me o
exemplo das Vossas Virtudes e da Vossa União. Serei Digno de vós.
Palácio do Rio de Janeiro em o primeiro d'Agosto de 1822.\footnote{Pedro
  \textsc{i}, Dom {[}Príncipe Regente{]}. \emph{Manifesto de S.\,A.\,R. O Príncipe
  Regente Constitucional e Defensor Perpétuo do Reino do Brasil aos
  Povos deste Reino}. Rio de Janeiro: Imprensa Nacional, 1822, pp. 1--4.}
\end{quote}

O narrador, aderente às múltiplas tonalidades das figuras, reproduz até
mesmo a intercalação e a repetição de palavras, no discurso do barbeiro,
quando deixa a infrutífera entrevista com o médico:

\begin{quote}
--- \ldots{}porque \emph{eu velo}, podeis estar certos disso, \emph{eu
velo} pela execução das vontades do povo. Confiai em mim; e tudo se fará
pela melhor maneira. \emph{Só vos recomendo ordem}. \emph{A ordem}, meus
amigos, é a base do governo\ldots{}

--- Viva o ilustre Porfírio! Bradaram as trinta vozes, agitando os
chapéus {[}grifos meus{]}.
\end{quote}

Bacamarte sedimenta o saber desinteressado da ciência, a postura
dogmática e a síndrome do poder --- concedido pela administração
portuguesa. Com um quê de pesquisador, religioso e político, desde seu
retorno ao Brasil ele age em relativa consonância com a câmara e obtém
tudo de que necessita para implementar a morada dos loucos e tratá-los
em acordo com a sua lógica delirante. Kátia Muricy detectou:

\begin{quote}
{[}\ldots{}{]} no humor de \emph{O Alienista} uma crítica perspicaz às
intenções controladoras da nascente psiquiatria em relação à população,
bem como uma compreensão exata das alianças recíprocas entre ela e o
poder político. Mas é especialmente na ironia à positividade
experimental, aos altos ideais humanitários do saber psiquiátrico e à
sua suposta vinculação com os princípios universais da razão --- vínculo
que legitimava, no discurso médico, a intervenção da psiquiatria social
--- que a narrativa ganha sua inteligência mais requintada.\footnote{Muricy,
  Kátia. ``As desventuras da razão''. In: \_\_\_\_\_. \emph{A razão
  cética: Machado de Assis e as questões de seu tempo}. São Paulo:
  Companhia das Letras, 1988, p. 36.}
\end{quote}

Conhecedor dos meandros que ligam a extensa cadeia administrativa do
reino, age de modo mais cauteloso em relação ao padre Lopes e reserva
para sua esposa as migalhas que sobram de sua intensa dedicação à
ciência e à ambição de servir a humanidade. Altruísmo ou megalomania?
Para Alfredo Bosi:

\begin{quote}
Ele {[}Simão Bacamarte{]} \emph{pode} executar os projetos da ciência
que o obseda. Seu status de nobre e portador do valimento régio
transforma-o em ditador da pobre vila de Itaguaí. A população sofre os
efeitos de um terrorismo do prestígio de que as relações entre médico e
doente, psiquiatra e louco, são apenas casos particulares. O eixo da
novela será, portanto, o arbítrio do poder antes de ser o capricho de um
cientista de olho metálico.\footnote{Bosi, Alfredo. \emph{Machado de
  Assis -- o enigma do olhar}. São Paulo: Ática, 1999, p. 89 {[}grifo do
  autor{]}.}
\end{quote}

A novela é orientada por tópicos recorrentes: o saber do médico; o poder
descomunal de que ele está investido; a sua incomum relação matrimonial
com Dona Evarista\footnote{``{[}\ldots{}{]} o fundamento inicialmente
  reconhecido pelo romancista é o amor como aspiração selecionada da
  existência humana; a segunda é alimentada pelas ambições do poder. Mas
  pode haver uma terceira, determinada pela procura de fusão daquelas
  `duas forças principais da terra'*, o amor e o poder, ou da
  sobreposição da segunda à primeira, equilibrando-se ou gerando
  conflitos, no mundo machadiano. A conciliação, porém, é muito pouco
  frequente'' (Castello, José Aderaldo. \emph{Realidade e ilusão em
  Machado de Assis}. São Paulo: Companhia Editora Nacional; Edusp, 1969,
  p. 63). *Crônica publicada por Machado na coluna ``A Semana'' (jornal
  \emph{Gazeta de Notícias}), em 26 de junho de 1892.}; as assimetrias
sociais, cravadas no asilo, na câmara e na barbearia; as terapêuticas
pouco ortodoxas, aplicadas pelo cientista. Diante dessa profusão de
referências à saúde mental, religião, filosofia, história e política, o
leitor não fica imune aos métodos do cientista. Ele duvida dos bons (e
maus) propósitos do médico, tendo em vista os critérios --- nem sempre
objetivos --- a que ele recorre.

De fato, Bacamarte comporta-se de modo tão obstinado que a ciência
assume ares de doutrina, e a terapêutica beira a charlatanice. A maneira
como atesta o acerto de suas hipóteses beira o dogmatismo. Entretanto,
seria um equívoco supor que se trata de um tirano qualquer --- como
proclama a turba dos revoltosos, liderados pelo barbeiro Porfírio. Os
paralelos com a Revolução Francesa são tão evidentes que precisamos
examiná-los mais de perto:

\begin{quote}
Observe-se antes de mais nada que os títulos dos capítulos reproduzem as
grandes articulações da Revolução Francesa. O capítulo 5 é o Terror; o
capítulo 6 é a Rebelião; o capítulo 10 é a Restauração. Mas a cronologia
está deslocada. Em Paris, a sequência foi a Rebelião (a revolta popular
que culminou com a tomada da Bastilha); o Terror (o banho de sangue
decretado por Robespierre e Marat); e a Restauração (o retorno dos
Bourbon). {[}\ldots{}{]} A inversão temporal pode ser uma forma sutil de
aludir ao caráter reativo, reflexo, dos movimentos populares no Brasil,
em contraste com a Europa, onde o povo tem um protagonismo originário, e
não derivado.\footnote{Rouanet, Sérgio Paulo. ``Machado de Assis e o
  mundo às avessas''. In: \_\_\_\_\_ et al. \emph{Machado de Assis:
  cinco contos comentados}. Rio de Janeiro: Edições Casa de Rui Barbosa,
  2008, p. 83.}
\end{quote}

Para enfrentar essa questão, sugiro que procedamos como o douto
cientista faria. Em lugar de nos atermos meramente às disputas pelo
poder e/ou às arbitrariedades implementadas pelo médico, examinemos um
dos recantos responsáveis pelo êxito da narrativa. Em vez de nos
restringirmos aos sintomas de sabedoria/loucura, manifestados por
Bacamarte e pelos itaguaienses, desviemos nossa atenção para a voz que
narra. Uma das melhores sínteses sobre o estilo machadiano foi sugerida
por Alcides Maya, em 1912:

\begin{quote}
Um tristonho chancear é o processo permanente de Machado de Assis, que
se deleita em revelar o ridículo ora numa paráfrase mordaz, ora numa
redução folgazã da natureza e quase sempre na realidade individual,
falha e má, indiretamente estudada, com profundez ímpia, sob aparências
de sensatez, de virtude e dever. Daí o ser classificado com razão entre
os grandes humoristas.\footnote{Maya, Alcides. \emph{Machado de Assis --
  algumas notas sobre o humour}. 3ª ed. Porto Alegre: Movimento; Santa
  Maria: Editora \textsc{ufsm}, 2007, p. 47.}
\end{quote}

Em que pese a feição moralista do crítico, podemos aproveitar muito do
que ele diz. É ao narrador (talvez mais do que às personagens) que
podemos atribuir a sensação de que tudo foi colocado em questão. O seu
relato é impreciso, embora respalde-se nos cronistas de
Itaguaí.\footnote{``O tom da narrativa é dado pelo truque de imaginar
  que as peripécias do conto se fundamentam na verdade das crônicas: `As
  crônicas da vila de Itaguaí dizem que em tempos remotos\ldots{}'. E que,
  portanto, ao narrador cabia a função de trazer à luz um episódio
  histórico soterrado no pó dos manuscritos. É nessa atmosfera de
  aparente respeito à letra dos documentos que encontramos logo às
  primeiras linhas a chave para lhes interpretar o conteúdo''
  (Moisés, Massaud. ``\emph{O
  Alienista}: paródia do \emph{Dom Quixote}?''. \emph{Machado de Assis:
  ficção e utopia}. São Paulo: Cultrix, 2001, p. 128).} Os vereadores
são referidos como homens ``dignos'', embora a quase totalidade deles
defenda a manutenção de seus privilégios, dentre os quais a
exclusividade de serem imunes aos diagnósticos do médico --- amparados
pelo braço curto da lei. A semelhança com determinados procedimentos
adotados pelo Judiciário, nos últimos anos, dá maior consistência a essa
blindagem, no plano da ficção.

O poder supremo não reside exclusivamente na erudição do médico;
tampouco na proteção do reino, menos ainda na chancela provisória
concedida pela vereança. Ele passa pelo crivo do narrador, que nos leva
e traz, puxa e arremete, também com o intuito de desestabilizar a
leitura e alargar os seus efeitos para além do entretenimento. Sua
dicção, ora vaga, ora assertiva, confunde-se com a voz autoral. Autor e
narrador parecem nos aconselhar a não levar a literatura tão a sério,
embora dramatizem os abusos praticados pelas autoridades da vila.

\emph{O Alienista} dispõe de um desses enredos que, em meio ao tom
jocoso, deixa-nos adivinhar que os episódios transcorridos em Itaguaí
não diferem grandemente de outros eventos, situados para além da
historiografia (aproximada da ficção), fosse em Vila Rica --- enquanto
durou a violenta perseguição e castigo dos conjurados mineiros ---; fosse
em Paris, quando o terceiro estado pegou em armas contra a realeza e seu
sistema de privilégios, perpetuado havia séculos, na França.

Uma das consequências é que a angústia vivenciada pela população da vila
contagia o leitor. Nada mais coerente, em termos estilísticos, tendo-se
em vista que estavam diante de um mundo em miniatura não só invertido,
mas sujeito à constante reversão. Como disse, o exame sobre o cientista
pode situar-se na linguagem empregada pelo narrador e demais
personagens. Maria Nazaré Lins Soares disse-o com acurácia: ``Um abismo
parece mediar entre a magra realidade e a linguagem hiperbólica de que
fazem uso os personagens para exprimi-la''.\footnote{Soares, Maria
  Nazaré Lins. \emph{Machado de Assis e a análise da expressão}. Rio de
  Janeiro: \textsc{inl}, 1968, p. 13.}

Para além das paródias a eventos sangrentos da história, e as constantes
alusões a celebridades do vice-reinado e do primeiro império, o narrador
recorre à \emph{Bíblia} e ao \emph{Corão}, em diversas ocasiões, a
sugerir que Simão Bacamarte seria um daqueles sábios autênticos em
extinção, já que bebe, em quantidade, de todas as fontes. A esse
respeito, assinalem-se as várias palavras de origem árabe e o possível
paralelo entre a trajetória do médico e de Maomé --- o que inclui as
esposas com que se casaram\footnote{``A alusão à cultura árabe é fato
  merecedor de atenção. {[}\ldots{}{]} o profeta tinha 25 anos, ao se
  casar com uma viúva rica, 15 anos mais velha que ele. Já a personagem
  {[}Bacamarte{]} é financeiramente estabelecida, tem 40 anos, e sua
  esposa, Evarista, tem 25. {[}\ldots{}{]} {[}Maomé{]} foi acusado de
  demente, ao longo da vida, em razão da maneira pela qual, embora
  semianalfabeto, fizesse referências escritas a Alá {[}\ldots{}{]}
  Maomé enfrentou violeta oposição de grande parcela da população de
  Meca, em sua jornada'' (Chauvin, Jean Pierre. O Alienista: \emph{a
  teoria dos contrastes em Machado de Assis}. São Paulo: Reis Editorial,
  2005, pp. 88--9).}. A alusão a Averróis, outro nome fundamental da
filosofia, também pode ser vista como índice de que o pensamento árabe
se mantém, feito ideia fixa, no horizonte do médico. Isso, sem contar as
ações que parodiam certos atos de Dom Pedro \textsc{i}, considerado louco por uma
parcela de seus biógrafos, como refere Tobias Monteiro.

Detentor de múltiplos saberes, Bacamarte oscila da frieza à
impassibilidade. De um lado, pesquisa e aplica técnicas estritas do
campo médico; de outro, escuta e fala sem demonstrar aparente
perturbação, obedecendo ao rito recomendado pelos filósofos gregos da
Antiguidade. Simão Bacamarte resulta de um misto de cálculo e
\emph{ataraxia}. Como percebeu Benedito Nunes, ``Em Machado, essa
transposição parodística se faz como um artifício de teorização apoiado
numa \emph{imitatio} da exposição racional, argumentativa, ou do
comentário de realce erudito, com o aparato das citações de
autoridade''\footnote{Nunes, Benedito. ``Machado de Assis e a
  Filosofia''. In: \_\_\_\_\_. \emph{No tempo do niilismo e outros
  ensaios}. São Paulo: Ática, 1993, p. 136.}.

Essa aparente tranquilidade do cientista, filósofo e médico contribui
para que o leitor até simpatize com ele e, porventura, aceite a maior
parte dos argumentos que ele sustenta com tamanha segurança e altivez.
Temos que convir que ele não parece ser mais um oportunista, nos moldes
do que viria a ser o Cristiano Palha, de \emph{Quincas Borba}, nem um
bajulador de ofício --- como será José Dias, agregado da família
Santiago, em \emph{Dom Casmurro.} Para Maria Nazaré Lins Soares, a
linguagem utilizada pelas personagens guardava íntima relação com o seu
\emph{ethos}:

\begin{quote}
Para merecer alguma consideração não só aos olhos dos outros, mas aos
seus próprios, busca compensar sua situação social esdrúxula,
apresentando-se sempre impecável na sua roupa engomada e buscando com
superlativos dar pernas compridas a suas ideias chochas. José Dias é
pusilânime, interesseiro e falso como a linguagem que usa.\footnote{Soares,
  Maria Nazaré Lins. \emph{Machado de Assis e a análise da expressão},
  ed. cit., 1968, p. 7.}
\end{quote}

Não será preciso grande esforço para constatar que o boticário Crispim
Soares é um parente mais novo de José Dias. Ao agir de modo bajulatório,
quase emudecido pela verborrágica genialidade do médico, Crispim anula a
sua personalidade. A cada mudança na teoria formulada pelo médico, o
boticário ressente o dilema de se manter fiel ao amo, ainda que isso
implique aderir a uma e outra ideia, ao sabor dos humores do médico. A
caracterização de José Dias ajusta-se quase inteiramente ao perfil de
Crispim Soares:

\begin{quote}
E porque o boticário se admirasse de uma tal promiscuidade, o alienista
disse-lhe que era tudo a mesma coisa, e até acrescentou
sentenciosamente:

--- A ferocidade, Sr.\,Soares, é o grotesco a sério.

--- Gracioso, muito gracioso! --- exclamou Crispim Soares levantando as
mãos ao céu.
\end{quote}

Os contrastes do boticário com o médico são evidentes. A despeito de sua
impassibilidade pseudofilosófica, o que não falta a Simão Bacamarte é
uma personalidade enérgica que se coaduna com o seu proceder intrépido e
modos distintos. O narrador (sempre ele) sugere essa conduta em diversos
momentos, particularmente quando médico e boticário regressam, com
ânimos opostos, após a partida da comissão que fora para o Rio de
Janeiro. De um lado, o representante típico do ``vulgo'', a padecer de
saudade, como acontece aos medíocres; de outro, a vivacidade do médico,
imperturbável diante de qualquer contrariedade. O que num é paixão e
adversidade, no outro é ciência e planejamento.

Simão Bacamarte combate três rivais em potencial: a minoria da vereança
de Itaguaí, o padre Lopes e os barbeiros (Porfírio e João Pina),
apoiados por uma parte do povo local. Apesar da eventual disputa de
poder, em nenhuma ocasião a câmara representou ameaça aos desígnios
interpostos pelo cientista. Admirado pelo rei Dom João \textsc{v}, Bacamarte
recorreu à tribuna para fazer valer a sua força e autoridade. Por sua
vez, o padre Lopes ocupa lugar intermediário: habituado ao púlpito, fez
uso da palavra para se contrapor aos pensamentos e atitudes que ocorriam
ao ilustre pesquisador. Ao agir desse modo, ele reproduz a postura do
padre concebido por Cervantes em \emph{Dom Quixote}: ``-- Quanto a mim,
tornou o vigário, só se pode explicar pela confusão das línguas na torre
de Babel, segundo nos conta a Escritura; provavelmente, confundidas
antigamente as línguas, é fácil trocá-las agora, desde que a razão não
trabalhe\ldots{}''.

A maior demonstração de força não reside nos vereadores, nem no vigário;
ela parte dos populares. Aquilo que a política e a religião não venceram
transformou-se em alvo primário dos Canjicas --- denominação que satiriza
os apelidos dados aos revolucionários, no âmbito da história: dos
\emph{sans-culotte} a Tiradentes. Nessa confluência de saber, discurso e
força, pairam algumas dúvidas. Teria o alienista agido como agiu
exclusivamente em função do poder desmedido de que supunha estar
investido? Ou a sua postura mais firme devia-se à convicção inabalável
de que poderia prestar o melhor serviço ao povo local --- como se os
experimentos na Casa Verde fossem balões de ensaio para a humanidade?

Se aceitarmos a investigação psiquiátrica como um gesto de vaidade de
Bacamarte, a sua conduta lembra em muito a do herdeiro (e ocioso) Brás
Cubas, que padeceu da ideia fixa de ver seu nome irradiado através dos
tempos, no emplastro com que pretendia curar a ``melancólica
humanidade''. O que um e outro têm em comum? A concepção totalizante do
universo, a partir do seu umbigo.

O impacto de \emph{O Alienista} não repousa apenas no teor mirabolante
criado por Machado. Tão importante quanto o enredo inventado é o fato de
ser uma narrativa ágil, o que não impede ao narrador semear digressões
--- como se estivesse a ridicularizar o próprio relato das desventuras
sucedidas na vila. Nos lances mais risíveis, escutamos a sua voz a
referendar (ou a alimentar a imprecisão) (n)o que conta, com base nos
cronistas de Itaguaí. Ivan Teixeira percebeu, como poucos, que:

\begin{quote}
{[}\ldots{}{]} a voz narrativa parece muito clara, mas é de fato um
tanto sinuosa, visto produzir uma gradação que vai da opinião ao boato e
do boato ao boato duvidoso. {[}\ldots{}{]} o texto procura a clareza por
meio da insinuação, como que sugerindo que a gravidade da matéria
desaconselha a objetividade. Daí a condição algo paradoxal de sua
clareza relativa, mas que é prevista pela tradição dos procedimentos
retóricos. O texto apresenta também uma espécie de mistura entre o
disparate e o solene, que lembra certa visualidade amaneirada de gestos
e feições, próximos do desenho caricatural.\footnote{Teixeira, Ivan.
  \emph{O altar \& o trono: dinâmica do poder em} O Alienista, ed. cit.,
  2010, pp. 265--267.}
\end{quote}

Como reparou Augusto Meyer: ``\emph{O Alienista} é um espanto. O feitio
objetivo do entrecho, o tempo lento da narração, a contenção da ironia
sem malabarismos inúteis, a serenidade superior, a graça irresistível,
mas apagada e modesta --- tudo concorre para dar ao leitor, por
contraste, uma impressão de espantosa vertigem''\footnote{Meyer,
  Augusto. ``Na Casa Verde''. In: \_\_\_\_\_. \emph{Machado de Assis}.
  Rio de Janeiro: Simões, 1952, p. 60.}.

Embora a crônica seja considerada uma fonte historiográfica, o narrador
distribui incertezas em seu relato, a sugerir que ficção e
historiografia\footnote{``Há passagens no conto {[}\emph{O Alienista}{]}
  que fazem pensar numa clarividência histórica assombrosa em Machado. É
  o caso daquela atitude da Câmara Municipal de Itaguaí, aprovada por
  instigação do cientista, que autorizava o uso de um anel de prata no
  polegar da mão esquerda a todo habitante que declarasse ter sangue
  godo nas veias. Ora, O Alienista é contemporâneo da teorização racista
  de Chamberlain, Gobineau e certos antropólogos alemães. {[}\ldots{}{]}
  O espantoso, realmente, é a sensibilidade com que Machado percebeu
  aonde essa preocupação `científica' poderia levar'' (Schnaiderman,
  Boris. ``\emph{O Alienista}, um conto dostoievskiano?'' \emph{Revista
  Teresa}, n. 6/7, 2006, p. 272).} se equivalem, já que não passam de um
recorte ambivalente da realidade, disposto de maneira não a convencer o
leitor de uma pretensa veracidade, mas a relembrar que lhe cabe o papel
maior de se divertir, enquanto reflete sobre os episódios. Por sinal, o
tom grandiloquente expresso pelo narrador e pelas personagens parece
escorar-se no próprio gênero, estilizado pelo narrador: espécie de
cronista de segunda linha.

A novela parodia gêneros discursivos, retira a credibilidade do registro
historiográfico e disseca a expressão calculada e artificial das
personagens. O furor taxonômico do cientista, como apontado por José
Leme Lopes (1981) e Pierre Brunel\footnote{``Machado de Assis, assim
  como vários escritores do século \textsc{xix}, escolheu uma terceira via, a da
  narrativa. Parece-nos que ela permite manter distância, sem
  {[}possuir{]} a frieza do discurso científico'' (Brunel, Pierre.
  ``Itaguaï, ou le grand théâtre du monde''. In: Machado de Assis.
  \emph{L'Aliéniste}. Paris: Éditions Métailié, 1984, p. 9. [trad. minha]} (1984), não deveria enganar o leitor. Um dos maiores triunfos
da novela reside no modo como Simão Bacamarte e narrador medem suas
palavras, provocando a impressão de que elas fluem com máximo cuidado e
precisão, embora a narrativa desdiga a si mesma, a cada parágrafo.

Porventura não se trate de dissecarmos a alma humana --- configurada na
reduzida escala dos caricatos cidadãos itaguaienses ---, mas de
desvendarmos a nós mesmos. A novela demanda que desconfiemos dos outros
e riamos de nossos limites, mal equilibrados entre vícios e virtudes.
Como se vê, nem mesmo Fernão Lopes teria pintado retrato mais fiel e
gracioso do douto, louco e extraordinário Simão Bacamarte: ambivalente
``Hipócrates forrado de Catão''.

\chapter[Sobre «O imortal», \emph{por João Adolfo Hansen}]{Sobre «O imortal»\subtitulo{A verossimilhança do conto}} %``O Imortal'' e a verossimilhança
\hedramarkboth{Sobre «O imortal»}{}

\begin{flushright}
\textsc{joão adolfo hansen}
\end{flushright}

\noindent{}Entre 15 de julho e 15 de setembro de 1882, Machado de Assis publicou os
seis capítulos do conto ``O Imortal'' em \emph{A Estação}, uma revista
feminina do Rio de Janeiro. No final dele, o narrador afirma: ``Tal é o
caso extraordinário, que há anos, com outro nome, e por outras palavras,
contei a este bom povo, que provavelmente já os esqueceu a
ambos''\footnote{Cf. Assis, Joaquim Maria Machado de. ``O Imortal''. In:
  \_\_\_\_. \emph{Obra Completa}. Org. Aluizio Leite \emph{et al}. Rio
  de Janeiro: Editora Nova Aguilar, 2015, v. 3 (Conto, Poesia, Teatro,
  Miscelânea, Correspondência), p. 77.}. Segundo Jean-Michel Massa, o
conto é ``a repetição de 'Rui de Leão', assinado Max, publicado no
\emph{Jornal das Famílias}, janeiro-fevereiro-março 1872 e republicado
por Magalhães Júnior nos \emph{Contos Recolhidos}, pp. 89--117''\footnote{Cf.
  Massa, Jean-Michel. \emph{A Juventude de Machado de Assis 1839--1870}.
  Ensaio de biografia intelectual. Ed. ilustrada. Trad. Marco Aurélio de
  Moura Matos. Rio de Janeiro: Civilização Brasileira; Conselho Nacional
  de Cultura, 1971, p. 534}. ``O Imortal'' começa com a fala do Dr.\,Leão,
médico homeopata, que relata a história da vida de seu pai para o
Coronel Bertioga e o tabelião João Linhares: ``--- Meu pai nasceu em
1600\ldots{}''. É uma noite chuvosa de novembro de 1855, numa vila fluminense,
``suponhamos que Itaboraí ou Sapucaia'', diz o narrador. Bertioga é o
proprietário da casa onde estão; Linhares é um ``espírito forte'',
expressão francesa que no século \textsc{xvii} significava ``libertino'', e, no
\textsc{xix}, ``livre pensador''. ``--- Perdão, em 1800, naturalmente\ldots{} --- Não,
senhor, replicou o Dr.\,Leão, de um modo grave e triste, foi em 1600''.

É provável que a história que o médico começa a contar para seus
ouvintes também não coincida ``naturalmente'' com as opiniões do leitor. O
leitor considera falsa a opinião de que um homem possa viver 255 anos,
pois não conhece nenhuma evidência empírica que a comprove como fato
biológico natural, habitual e normal. O narrador põe em cena essa mesma
opinião, quando faz o homeopata antecipar-se às objeções dos ouvintes:
``{[}\ldots{}{]} na verdade a história de meu pai não é fácil de crer''.
Realmente, não é. Vejamos três ou quatro coisas que permitam discuti-lo.

Comecemos pelo gênero do conto. Machado de Assis o escreveu elegendo uma
tradição antiga para ele, a de Luciano de Samósata, um grego do segundo
século da era cristã, autor de obras satíricas e paródicas relacionadas
à chamada \textsc{ii} Sofística\footnote{Cf. Rego, Enylton José de Sá. \emph{O
  Calundu e a panaceia}. A sátira menipeia e a tradição lucianesca. Rio
  de Janeiro: Forense Universitária, 1981.}. Caso de \emph{História
Verdadeira}, uma paródia, informa Henrique Murachco, das narrativas de
Odisseu na corte do rei Alcino nos cantos \textsc{ix} e seguintes da
\emph{Odisseia}.\footnote{Luciano- \emph{Diálogo dos Mortos}: versão
  bilingüe grego/português. Trad., introdução e notas: Henrique G.
  Murachco. São Paulo: Palas Ahena: Edusp, 1996, p. 14.} Como outros
textos de Luciano, \emph{História Verdadeira} se caracteriza pela
improbabilidade das ações e dos eventos narrados, improbabilidade que
hoje chamamos ``fantástico''. O gênero, que foi usado por autores
conhecidos de Machado de Assis, como Swift, de \emph{Viagens de
Gulliver}, Campanella, de \emph{A Cidade do Sol,} ou Cyrano de Bergerac,
de \emph{Viagem à Lua,} tem regras específicas: é uma ficção falsa, ou
seja, ficção sobre coisas impossíveis e improváveis. Para especificá-la,
podemos repetir a pergunta de Espinosa: a narração de um evento que não
ocorreu em parte alguma é \emph{falsa} ou \emph{fictícia}? Há dois tipos
de critérios para responder, o de existência e o de essência.

Quando a narração se refere a algo que realmente \emph{existe} e o
relaciona com um evento que não ocorreu em parte alguma, tem-se a
``\emph{ficção primeira}''. Por exemplo, com a referência à existência de
uma pessoa conhecida, Machado de Assis, inventa-se a ficção de algo que
nunca ocorreu, como uma viagem à Inglaterra, onde Joaquim Maria faz
contatos com uma leitora de \emph{Otelo} chamada Capitolina. Tem-se a
``\emph{ficção segunda}'' quando a narração se refere somente à
\emph{essência} dos seres; com a referência à essência, é possível
inventar uma ficção verdadeira, como \emph{vera fictio}, e uma ficção
falsa, como \emph{falsa fictio}. Como exemplo desta, imaginemos uma
história absurda, onde um inseto infinito voa num espaço que,
teoricamente, deverá estar todo ocupado por seu corpo; ou uma personagem
que tem uma alma quadrada. Ou, ainda, um homem imortal.

A distinção permite conceber operacionalmente a ficção verdadeira como a
narração que relaciona a existência ou a essência verdadeira de algo com
eventos que não aconteceram. E também definir a ficção de algo falso,
que não é nem existe, como história que relaciona o não-ser com
acontecimentos que nunca ocorreram. A \emph{falsa fictio} inventa algo
impossível de ser e, assim, de ocorrer. Em ambos os casos, verdadeiro e
falso, o termo \emph{ficção} define uma operação da imaginação, uma
técnica, uma forma e um efeito aplicados ora ao conhecimento de
existência, ora ao conhecimento de essência.

As duas espécies de ficção podem ser relacionadas com a passagem da
\emph{Poética} em que Aristóteles prescreve que o gênero histórico trata
do que efetivamente ocorreu, como uma narrativa de existência que conta
eventos particulares e verdadeiros, diferentemente da poesia, que figura
o possível ou o universal, como ficção de essência sem necessidade de se
referir a eventos particulares. Como se sabe, Aristóteles considera a
história inferior à poesia, porque a história é \emph{mímesis} parcial
que trabalha com o conhecimento de existência do passado ou um
conhecimento particular fornecido por testemunhos. Por isso mesmo, em
suas versões clássicas, a história consegue estabelecer a variante
``verdadeira'', quando estabelece ``fatos'' que permitem eliminar outras
variantes concorrentes. Como dizia Jean-Pierre Faye, o incêndio do
Reichstag não pode ter sido produzido ao mesmo tempo pelos comunistas e
Van der Lubbe, versão nazista, pelos \textsc{sa} de Göring, versão da
Internacional comunista, e por Van der Lubbe sozinho, versão de Tobias.
As três versões se excluem logicamente e é impossível que sejam
verdadeiras ao mesmo tempo. Com isso, a história se opõe à fantasia
poética que, mesmo ao tratar do passado, como a novela histórica, não
refere o que efetivamente foi, mas o que poderia ter sido.

O Dr.\,Leão afirma que sua história não é fácil de crer. Realmente não é,
se o fantástico do seu gênero, como ficção falsa, for avaliado com as
opiniões positivistas e realistas que o leitor, assim como João
Linhares, considera verdadeiras quando pensa em ``realidade'', pressupondo
que a ficção é uma imitação direta da mesma. O gênero fantástico é
explicitamente incrível: a descrença é seu pressuposto, não seu efeito,
pois sua matéria é não-ser. Seu destinatário deve saber que lê uma arte
de representar o inacreditável do não-ser e do não-existente, aceitando,
contudo, a realidade da convenção e do artifício. Na história
fantástica, nada existe em que acreditar, a não ser o bom desempenho
técnico e artístico das convenções de um gênero que trata do falso. O
gênero prevê que seus personagens vivam aventuras e situações
improváveis. Por exemplo, ser um morto que escreve. Ou ser pernambucano
e tornar-se rei da Inglaterra. O último exemplo é tipicamente
fantástico, na medida mesma em que os possíveis de uma vida apenas
mortal são por definição restritos e restritivos. Afinal, lembremos
Sartre, cada um é o que faz com o que fizeram dele. Nunca é suficiente,
embora o pouco quase sempre seja demasiado. O que acontece ao personagem
de ``O Imortal'' é ser e fazer muito, acumulando e vivendo demasiadamente
na sua os vários possíveis das vidas de outros homens: pernambucano,
religioso franciscano, amante de índia, amante de lady escocesa, guarda
papal, rei de Inglaterra, traficante de escravos, soldado, espião etc. A
história incomum de sua vida é efetivamente uma história esticada como
somatória, por assim dizer, de existências, escolhas e ações de muitos
homens. Nos diversos momentos dos seus 255 anos, variam enormemente as
pessoas e as experiências; no entanto, em todas as situações que vive,
ano após ano, entre 1639 e 1855, sempre se lê a mesma história básica.
Como se, vivendo o impossível da imortalidade, a cada nova experiência
estivesse condenado a efetivamente viver as possibilidades restritas de
uma vida só mortal, repetindo na longa extensão da sua as mesmas poucas
experiências da vida breve de todos, o amor, a aventura e a intriga. Não
há moral da história, pois quer divertir; no entanto, se quiser, o
leitor cioso de moralidade poderá concluir que estar livre da morte, mas
sujeito às contingências da condição humana, é tristemente tedioso, uma
vez que Pangloss é um estúpido e este aqui não é, com toda evidência, o
melhor dos mundos possíveis. Não é sem alívio que Rui de Leão vai enfim
para o outro lado do mistério, onde está Brás Cubas, que lá continua
escrevendo.

Obviamente, Machado de Assis é um mestre insuperado na sátira e na
paródia que caracterizam a tradição luciânica, podendo-se supor que,
sendo o monstro de perversidade que é, também deseja que seu leitor
descreia. E não do tema, ``imortalidade'', nem da história do personagem
que vive 255 anos, porque as inventa como o fantástico que diverte a
fantasia do ``bom povo'' de 1872, como poderá, talvez, divertir o de 2004.
Ao republicar o conto em 1882, provavelmente para ganhar uns cobres, já
poderia supor que seus leitores fossem como aqueles cinco do prólogo de
\emph{Memórias Póstumas de Brás Cubas} (1880/1881). Não eram,
evidentemente, porque ``O Imortal'' foi publicado entre anquinhas atiradas
para cima, toucados de \emph{ondulations} \emph{et} \emph{chutes} e o
escocês e o xadrez vitorianos que então, pasmo!, influenciavam até a
moda de Paris, segundo \emph{A estação}. Provavelmente foi lido, se é
que foi, como um \emph{fait divers} a mais. A história republicada em
1882 já tinha sido contada ``por outras palavras'' em 1872, explica o
narrador. Como um avô de Pierre Menard, no entanto, em 1882 o mesmo
conto já não era o mesmo de 1872: \emph{Memórias Póstumas de Brás Cubas}
evidencia que era já impossível ler como discurso sério o romanesco
romântico com que recheia os lugares-comuns da vida fantástica de Rui de
Leão. O romantismo continuaria a divertir o ``bom povo'', como agora, com
o \emph{kitsch} da ideologia do ideal, complicação sentimental,
aventuras e intrigas; em 1882, contudo, a mesma história que diverte
também perverte a diversão, pois subordina os lugares-comuns e os
efeitos fantásticos a outros fins.

Quando publicou ``O Imortal'', Machado de Assis continuava interessado na
paródia e no pastiche como gêneros literários. A partir de 1880, tinha
começado a usar narradores que escrevem textos improváveis ou
inconfiáveis. Com eles, transformando a matéria social de seu tempo,
passou a relativizar e a destruir a representação fundamentada no
pressuposto da adequação entre os signos da linguagem, os conceitos da
mente e as estruturas da realidade objetiva. A partir de 1880,
tornaram-se mais e mais frequentes nos seus textos as imagens da morte,
do falso e do nada, como a falta de memória, a equivalência de razão e
loucura, a ação do diabo, o acaso das semelhanças, o arbitrário do
encadeamento da narrativa, o duplo, a improbabilidade e a
indeterminação. Basta lembrar que o narrador de \emph{Memórias Póstumas}
é um defunto. Que é a morte? angustia-se o leitor. Nada, certamente,
pois não é dizível ou escriptível e nela não há nenhum fazer. Nada se
pode afirmar sobre ela e qualquer ideia de fazer seu conceito é
autocontraditória. Evidentemente, não é natural, habitual ou normal que
um morto escreva. O uso de um como autor sugere que Machado de Assis faz
da falta de ser o princípio alegórico do sentido da sua arte como
negação da representação tradicional. A delegação da escrita do romance
para o morto, que incrivelmente recorda e impossivelmente escreve, é
fantástica e desloca a autoria para uma liberdade arbitrária e
artificiosa que não é mais definível por unidades de sentido das quais o
discurso fosse uma semelhança adequada. A escrita do morto esvazia as
representações unitárias da subjetividade, do mundo objetivo e da
linguagem, que são as da vida e sua ideologia, descartando com elas o
ilusionismo baseado em opiniões dadas como naturalmente verdadeiras e
evidentes, como as do livre-pensador João Linhares. Por várias razões,
Machado de Assis é um grande escritor e sua consciência da historicidade
das formas literárias é uma das principais. Em 1882, o mesmo ponto de
vista sério da complicação sentimental romântica é repetido, mas agora
se evidencia como convenção histórica, ou seja, particularidade apenas
mortal. As leitoras de \emph{A estação} decerto não pensavam em
literatura como coisa séria nem que pudesse ter algum sentido crítico.
Provavelmente, queriam a literatura como se deseja uma cama mental: a
história lá longe, na tempestade que passa lá fora e, aqui, dentro, o
calor do suplemento de alma, a maciez do romanesco, o sono morno da
razão nos lençóis do passatempo. Por isso mesmo, ignorando que seu modo
de ler já era ruína, elas eram lidas por \emph{Memórias Póstumas}
\emph{de Brás Cubas} e pelo conto, ainda quando não os liam, e passavam,
como seu tempo passava, entre anquinhas e outras coisas mais altas do
Império que ameaçavam desabar e já ruíam. Mas vamos ao conto.

A narração do Dr.\,Leão repete três procedimentos básicos: a exposição
linear, a complicação e a explicação. Ele conta ações do mais passado
para o presente, linearizando a história da vida do pai de 1600 a 1855;
simultaneamente, amplifica e complica cada uma das ações com acidentes
ordenados como repetição dos mesmos lugares-comuns de aventura, intriga
e amor. Como o que conta é fantástico ou improvável, dá explicações aos
ouvintes, fornecendo causas que tentam tornar plausível e mesmo verídico
o que lhes diz. Vejamos.

Rui de Leão nasce no Recife em 1600; seu pai era da nobreza de Espanha e
a mãe, de grande casa do Alentejo. Entra aos 25 anos para a ordem
franciscana em Igaraçu, fica no convento até 1639, quando é aprisionado
pelos holandeses; recebe um salvo-conduto, vai para o mato, chega a uma
aldeia de gentio, junta-se com Maracujá, filha do chefe Pirajuá. Antes
de morrer, o chefe lhe revela o local de uma igaçaba enterrada; contém
um boião com um líquido amarelo; preparado por um pajé, garante a
imortalidade a quem o bebe. O chefe morre, Rui de Leão adoece, vai ao
local do vaso, bebe a substância, volta à tribo, sara; outros índios
atacam a aldeia, morre Maracujá, Rui é ferido, sara, decide voltar para
o Recife. Quando os holandeses são expulsos, em 1654, vai para Portugal,
casa-se, tem um filho; em março de 1661, seu filho e sua mulher morrem e
ele parte para a França e a Holanda.

Até aqui, o Dr.\,Leão narra sessenta e um anos de vida do pai aplicando
tópicas ou lugares-comuns da aventura e do amor típicos do \emph{kitsch}
romântico usual. Depois deles, aplica lugares-comuns também românticos
de intriga. Por exemplo, na Holanda, ``por motivo de uns amores secretos,
ou por ódio de alguns judeus descendentes ou naturais de Portugal, com
quem entreteve relações comerciais na Haia, ou enfim por outros motivos
desconhecidos'', Rui de Leão é preso; levam-no para a Alemanha, donde
passa à Hungria, a cidades italianas, à França e à Inglaterra.

Machado de Assis compõe a complicação romanesca da vida do personagem
até 1654 como estilização da história colonial. A começar pelo nome do
personagem, Rui de Leão ``ou antes Rui Garcia de Meireles e Castro
Azevedo de Leão''. A somatória de nomes de família tradicionais era
séria, nos tempos coloniais e imperiais, pois distintiva da prosápia dos
``homens bons'', ``gente de representação'' ou ``melhores''; em 1882, é uma
afetação burguesa de arrivistas, barões Joões do Império, proprietários
de escravos e funcionários públicos aspirantes a ministérios que
hiperboliza, com o ``e'' e o ``de'', as alianças de parentesco, compadrio e
favor das oligarquias do tempo. A estilização reescreve textos de
cronistas e jesuítas sobre os contatos com as tribos indígenas; retoma
relatos sobre as guerras holandesas, a situação dos judeus portugueses
refugiados na Holanda, as negociações pela posse de Pernambuco etc. O
trecho citado no parágrafo anterior estiliza as intrigas políticas que
envolveram a ação diplomática de Vieira em Haia e Amsterdam. Machado
também estiliza a ficção indianista e histórica de José de Alencar e
mais românticos: Rui, Maracujá e Pirajuá refiguram Martim, Iracema e
Araquém; o tipo do religioso que abandona o hábito, vai para o mato e
amiga-se com índia pode ser rastreado em personagens ex-padres de
\emph{O} \emph{Guarani}, \emph{As Minas de Prata} e \emph{O Jesuíta.} O
poema escatológico de Bernardo Guimarães, ``O Elixir do Pajé'', é
estilizado na referência ao pajé que fez a beberagem. Nas aventuras de
Rui de Leão posteriores a 1661, também estiliza personagens como D.\,Juan, o atleta do amor, e os mitos eróticos da vida de poetas românticos
mais ou menos descabelados, como Byron, Lamartine, Musset. Estiliza
ainda os segredos, as traições, os desesperos, o patético e o
sentimental de narrativas de românticos ingleses, escoceses e franceses,
Walter Scott, Trilby, Lamartine, George Sand, Eugene Sue, Morand, Carco,
Musset, Alexandre Dumas e um grande etc. Também estiliza elementos
microtextuais, como o léxico antigo: o termo ``aleivosia'' é
divertidamente típico. E frases inteiras, que é impossível ler sem
sorrir de cumplicidade, pois o \emph{kitsch} não é de Machado de Assis.
Em todos os casos, a estilização mantém as características originais do
estilo romanesco da arte e da vida dos românticos, para subordiná-las a
outro fim, transformando o sério do ideal num pastiche irônico.

Vejamos um pouco mais do texto. Em Londres, Rui estuda inglês; sabe o
latim do convento, o hebraico aprendido em Haia com um amigo (nada menos
que um polidor de lentes, o filósofo Espinosa), e o francês, o italiano,
parte do alemão e do húngaro, tornando-se objeto de curiosidade e
veneração de plebeus e cortesãos. A história acumula mais
lugares-comuns. A enumeração dos múltiplos ofícios de Rui de Leão
condensa em poucos segundos de leitura o tempo de muitos anos vividos
por ele --- soldado, advogado, sacristão, mestre de dança, comerciante,
livreiro, espião na Áustria, guarda pontifício, armador de navios,
letrado, gamenho. Na aceleração narrativa, de novo se associa a esses
lugares o lugar-comum do amor. Mais que as \emph{mille e tre} mulheres
de D.\,Juan, informa o Dr.\,Leão, seu pai teve não menos de cinco mil.
Outros lugares-comuns se acrescentam ao exagero improvável que diverte a
inveja erótica do leitor: os da beleza feminina e sua psicologia vária,
com alfinetadas nas leitoras de figurinos. Por exemplo, a gentil
descortesia que é dizer elogiosamente que a estupidez das mulheres é
graciosa, usando para isso um preceito retórico do gênero cômico --- ``Há
casos em que uma mulher estúpida tem o seu lugar''. Os ``casos'' e o ``lugar'' para uma mulher estúpida são tópicas cômicas e elas sempre prescrevem
que a estupidez é só isso: estúpida. De novo, complicação sentimental e
aventuras amplificadas: na Haia, entre os novos amores, Rui de Leão
torna-se amante de lady Ema Sterling, ``senhora inglesa, ou antes
escocesa''. A caracterização de Ema estiliza heroínas como Corinne,
Norma, Graziella, Geneviève, Cosette, Delphine, Aurélia, Diva, Helena,
Iaiá Garcia e outras, cuja solidão moral assombra o imaginário dos
leitores românticos: ``formosa, resoluta e audaz --- tão audaz que chegou
a propor ao amante uma expedição a Pernambuco para conquistar a
capitania, e aclamarem-se reis do novo Estado''. Apaixonadamente dedicada
ou dedicadamente apaixonada, Ema deseja alçar Rui a grande posição: ``--- Tu
serás rei ou duque\ldots{} --- Ou cardeal, acrescentava ele rindo. --- Por que
não cardeal?''. Os dois enunciados elencam lugares-comuns de alta posição
social dos romances capa-e-espada. ``--- Por que não cardeal?''. As três
posições são aplicáveis à história romanesca de Rui de Leão, pois, sendo
imortal, tem tempo para viver todos os lugares. Assim, lady Ema o faz
entrar na conspiração que resulta na guerra civil inglesa. Segundo o
Dr.\,Leão, ela tem uma ideia espantosa: afirmar que Rui de Leão é o pai do
Duque de Monmouth, suposto filho natural de Carlos \textsc{ii} e principal
caudilho dos rebeldes. A tal ideia causa nova complicação narrativa que
obriga o Dr.\,Leão a justificar porque lady Ema pôde tê-la: ``A verdade é
que eram parecidos como duas gotas d'água. Outra verdade é que lady Ema,
por ocasião da guerra civil, tinha o plano secreto de fazer matar o
duque, se ele triunfasse, e substituí-lo pelo amante, que assim subiria
ao trono da Inglaterra. O pernambucano, escusado é dizê-lo, não soube de
semelhante aleivosia, nem lhe daria seu consentimento''.

O Dr.\,Leão usa o termo ``verdade'' duas vezes para justificar o que torna
plausível a ideia de lady Ema: a semelhança. A narração do conto
acontece em 1852; esse é um tempo romântico, o Dr.\,Leão é homeopata e a
semelhança ainda é tudo. Desde \emph{Memórias Póstumas}, porém, as
identidades e unidades metafísicas que a fundamentavam foram criticadas
e a semelhança já está arruinada como critério de validação da verdade
dos discursos transformados pela ficção. Machado de Assis ainda iria
escrever o texto decisivo, cujo núcleo é o equívoco da semelhança,
\emph{Dom Casmurro,} de 1899. Por ora, fiquemos com a história do Dr.\,Leão.

Nas idas e vindas da revolta, sempre envolvido em aventuras, intrigas e
no eterno amor de lady Ema, o pernambucano é aclamado rei de Inglaterra.
De novo, em poucos segundos de leitura, o leitor fica sabendo que Rui
governa o país, reprime sedições, baixa leis, é preso quando a fraude é
revelada, julgado, condenado à morte na Torre de Londres. Duas vezes o
machado do carrasco lhe atinge o pescoço, sem cortá-lo; solto, é
admirado, temido, amado, odiado, comparado a Cristo. O ano é 1686, Rui
de Leão tem oitenta e seis, não aparenta mais que quarenta.

No início do capítulo \textsc{v}, o Dr.\,Leão adverte Bertioga e Linhares: ``Já
veem, pelo que lhes contei, que não acabaria hoje nem em toda esta
semana, se quisesse referir miudamente a vida inteira de meu pai. Algum
dia o farei, mas por escrito, e cuido que a obra dará cinco volumes, sem
contar os documentos\ldots{}''. Aqui, afirma mais duas coisas relativas à
probabilidade da história que conta: a primeira é que poderia
amplificá-la ilimitadamente, acumulando detalhes. Por exemplo, se
parasse para contar miudamente a história de cada um dos ofícios
exercidos pelo pai; ou se tratasse de cada um dos seus cinco mil amores.
Para fazê-lo, bastaria aplicar novamente os lugares-comuns de aventura,
amor e intriga, que espichariam a história pelos cinco volumes com que
felizmente só ameaça o leitor. A segunda coisa é que afirma ter
documentos que comprovam a veracidade da história: ``títulos, cartas,
traslados de sentenças, de escrituras, cópias de estatísticas\ldots{}''.

Na historiografia, o leitor sabe, provas documentais atestam a
existência dos eventos narrados, distinguindo a narração histórica da
narração ficcional. Alegando as provas documentais que tornam o gênero
histórico provável, o médico homeopata propõe que o fantástico da sua
história tem a autenticidade e a autoridade de um discurso verdadeiro
sobre coisas e eventos reais- ``fatos'', como diziam os positivistas
também no tempo de Machado de Assis. A suposta realidade dos ``fatos''
assim constituídos pelos supostos documentos permite separar e excluir
como ``ficção'', irrealidade, o discurso que não pode apresentá-los. Se a
opinião de que um homem possa viver 255 anos é considerada falsa, a
história do Dr.\,Leão sobre a vida do pai é improvável; mas ela tende a
ser recebida não só como plausível, mas principalmente como verídica,
quando declara aos ouvintes que tem documentos que a comprovam. Voltemos
a Rui de Leão.

Sempre entre os quarenta e os cinquenta anos, vivendo oito, dez ou doze
numa cidade e noutra, perde a herança de lady Ema em um lugar de
complicação. Com os dez mil cruzados que lhe restam, tem a ideia de
meter-se no negócio de escravos. Aqui, mais lugares-comuns de aventura.
Obtém privilégio, arma navio negreiro, transporta escravos para o
Brasil. Mas, ainda lugar-comum sentimental que lhe enche as horas vagas
do negócio negreiro, sofre de ``vazio interior'', alargado pelas ``solidões
do mar''. Isso em 1694. Em 1695, mais lugares de aventura, combate o
quilombo de Palmares, perde um amigo e salva um jovem, Damião, em um
lugar-comum de heroísmo, no qual recebe no peito a flecha desferida
contra o rapaz por um quilombola. Outros lugares-comuns sentimentais,
gratidão, modéstia, amizade: ``A pobre mãe do oficial quis beijar-lhe as
mãos: -Basta-me um prêmio, disse ele; a sua amizade e a do seu filho''.
Mas a murmuração do povo de Pernambuco o aborrece e vai para a Bahia,
onde casa com D.\,Helena. Repete-se a situação narrativa da união
amorosa: D.\,Helena agora, antes \emph{lady} Ema, anteriormente a mulher
portuguesa, a índia Maracujá no início. Das outras mulheres o Dr.\,Leão
felizmente nada conta, deixando-as para os cinco volumes prometidos; mas
o leitor pode imaginar o que seria a história se ele narrasse todos os
casos de amor do pai com o mesmo lugar-comum do amor romântico:
dedicação, adoração, paixão, traição. Adiante, ele falará ainda sobre
duas espanholas e sua mãe e o leitor poderá ter seu trabalho de
imaginação reduzido, pois serão só quatro mil novecentas e noventa e
três as restantes que não entram na história.

Damião vai à Bahia, leva uma madeixa dos cabelos da mãe morta e um colar
que a moribunda ofereceu a D.\,Helena, lugares-comuns de gratidão. Três
meses depois, Damião e D.\,Helena aplicam em Rui o lugar-comum da
traição: ``Meu pai soube da aleivosia por um comensal da casa. Quis
matá-los; mas o mesmo que os denunciou avisou-os do perigo, e eles
puderam evitar a morte. Meu pai voltou o punhal contra si, e enterrou-o
no coração''. Três ou quatro lugares patéticos se atropelam no trecho: a
revelação da aleivosia, o ultraje da honra, o indizível do desespero, o
tresloucado do ato suicida. Como na Torre de Londres, repete-se a
experiência fantástica: Rui de Leão não pode morrer; foge, vai para o
Sul; no princípio do século \textsc{xviii}, nova aventura, está na descoberta das
minas: ``Era um modo de afogar o desespero, que era grande, pois amara
muito a mulher, como um louco\ldots{}'', Machado faz o Dr.\,Leão ir sendo
falado pelo melhor do \emph{kitsch} romântico. Em 1713, Rui de Leão está
no Rio de Janeiro, rico com as minas e com ideias de ser feito
governador. Repete-se situação narrativa também conhecida do leitor: o
pernambucano que já foi rei da Inglaterra agora deseja governar o Rio.
Aqui, complica-se a complicação sentimental: D.\,Helena retorna,
lugar-comum, mostra-lhe uma carta escrita pelo comensal, outro lugar;
nela, o denunciante pede perdão pela calúnia, mais um, declarando que
mentiu, outro, por ``criminosa paixão'', mais outro, comuníssimo. D.\,Helena volta para ele, trazendo a mãe e o tio; o tempo passa, Rui é
sempre o mesmo, eles envelhecem, morrem. Segundo o Coronel Bertioga,
``vieram ao cheiro dos cobres''. Linhares, também sempre positivo, afirma
que D.\,Helena ``não estava tão inocente como dizia''. Mas faz uma
ressalva, em que novamente aparece o termo ``verdade'' como fundamento de
uma explicação provável: ``É verdade que a carta do denunciante\ldots{}''. Mas
o Dr.\,Leão é peremptório e, explicando a perfídia da ação de D.\,Helena
--- ``O denunciante foi pago para escrever a carta'' ---, também explica por
que pode dar essa explicação: ``{[}\ldots{}{]}meu pai soube disso, depois
da morte da mulher ao passar pela Bahia''.

É meia-noite, o médico tem sono e quer dormir, mas os ouvintes insistem
em que termine a história: ``Mas, senhores\ldots{} Só se for muito por alto.
--- Seja por alto''. Outra vez, os mesmos lugares de aventura: seu pai
deixa o Brasil, passa por Lisboa, vai para a Índia, onde fica cinco anos
fazendo estudos, volta a Portugal, publica-os, é chamado pelas
autoridades, que o nomeiam governador de Goa. Os mesmos lugares-comuns
de intriga, inveja, maledicência e aleivosia são aplicados à situação
narrativa repetida pela terceira vez: antes rei de Inglaterra, depois
quase governador do Rio, agora talvez governador de Goa. Um candidato ao
cargo encomenda a um latinista a falsificação de um texto latino da obra
de Rui de Leão, atribuindo-o a um frade agostinho. A tacha de plagiário
o faz perder o governo de Goa; perde também a consideração pessoal e,
mais aventura, vai para Madri, onde mais lugares, amores com fidalgas
espanholas (romanticamente, as espanholas são morenas como as mouras,
misteriosas como a noite e ardentes como a lava, leitor), ``uma delas
viúva e bonita como o sol, a outra casada, menos bela, porém amorosa e
terna como uma pomba-rola'' etc., são aplicados para de novo engatar-se
neles o lugar-comum da honra: o marido ultrajado da aleivosia não duela
com Rui de Leão para lavar a honra em sangue, mas, lugar-comum de
falsidade vingativa e baixeza de caráter, manda assassiná-lo. Três
punhaladas, quinze dias de cama; um tiro e, como na Torre de Londres e
nas tentativas de suicídio, nada. Novamente, com um lugar-comum de
intriga, o marido o denuncia ao Santo Ofício da Inquisição. O Dr.\,Leão
explica por que o denunciante pôde fazer a denúncia: tinha visto coisas
religiosas da Índia com seu pai e elas lhe forneceram pretexto para
acusá-lo de ser dado a práticas supersticiosas.

Nesse ponto, o leitor bem pode concluir que Machado de Assis trabalha
com poucas situações narrativas básicas e três espécies também básicas
de lugares-comuns, na verdade os mesmos, só lhes variando o recheio.
Como diz o Dr.\,Leão, seu pai acha ``todas as caras novas; e essa troca de
caras {[}\ldots{}{]} dava-lhe a impressão de uma peça teatral, em que o
cenário não muda, e só mudam os atores''. Assim, mudam os atores da
história, mas não a própria história e seus actantes. Nas narrativas,
como o leitor sabe, sempre há um problema, para que os personagens
possam agir superando-o ou sendo vencidos por ele. Nessa, os problemas
vão como que desabando em cascata sobre o personagem, para que ele possa
viajar e meter-se em novas aventuras e problemas nos novos lugares. Os
problemas são diversos, diversas as viagens, diversas as aventuras,
diversos os locais para onde vai, mas sempre há muitos problemas, várias
viagens, inúmeras aventuras e, obviamente, os mesmíssimos
lugares-comuns. O personagem está sempre de tal modo ocupado por eles
que não tem tempo para viver a imortalidade na sua ação sempre exterior.
O mesmo acontece quando ama, tem pretensões políticas ou é vítima da
intriga de inimigos. Com que fim?

O leitor poderá pensar que Machado de Assis aplica os lugares-comuns
funcionalmente, para espichar a história, pois afinal ela é sobre a vida
fantástica de um personagem ``imortal''. E pensará bem, pois o gênero
pressupõe essas complicações. Mas pensará melhor se observar que o
espichamento é produzido redundantemente com os mesmos elementos típicos
do patetismo do romanesco romântico, aventura, intriga, amores. Por
serem exageros aplicados com redundância, tornam cada ponto e o todo do
conto também redundantes e exagerados. O patetismo dessa contínua
agitação exterior é uma deformação; como deformação, as paixões intensas
--- a solidão moral, a paixão amorosa, a honra ultrajada, o desespero
suicida etc., --- que passavam por sublimes, digamos que entre 1830 e
1870, são efetivamente cômicas em 1882. Na estilização, a seriedade
romântica evidencia-se como mera convenção de seriedade tornada
objetivamente ridícula pela marcha das coisas. ``Na verdade'', como diria
o Dr.\,Leão, Machado de Assis pouco se importa com que o leitor creia ou
não na história de ``O Imortal'', pois a crença é um efeito determinado
pelo gênero e será problema só do leitor se não sabe ler e confunde
gêneros literários com a empiria e pensa que personagens são pessoas. Na
verdade, Machado está interessado em parodiar um gênero e um estilo
romanescos em que a complicação sentimental da aventura exterior era
séria, passando em revista sua legibilidade. Por assim dizer, é o conto
que lê os leitores, propondo-lhes que não se trata de saber se a
imortalidade é possível, nem de duvidar da história narrada, mas de
evidenciar que é improvável propô-la diretamente como tema sério, porque
são justamente os modos de escrever sobre ela ou o que for com
lugares-comuns românticos de aventura, intriga e amor fundamentados na
semelhança que se tornaram improváveis, ou seja, inverossímeis. Decerto
seria possível propor a imortalidade indiretamente, como acontece com o
defunto Brás Cubas, que escreve \emph{Memórias Póstumas,} ou o
desmemoriado Bento Santiago, que lembra em \emph{Dom Casmurro.} A
imortalidade seria então um tema sério, como uma metáfora ou alegoria
para outras coisas importantes, como a crítica da vida. Isso porque,
digamos de novo, os lugares ficaram para trás com a mudança histórica
das coisas. Em 1882, as audazes Emas fidalgas dos textos românticos
transformaram-se para valer em pacatas burguesas leitoras de figurinos.
O amor, que antes só queria o absoluto de si mesmo, rendeu-se de vez ao
valor-de-troca. As intrigas e as aleivosias não são as da honra, mas
tornaram-se a estrutura mesma da imprensa, da política e do grande
negócio. E, como no de hoje, nada há de heroico na vida do ``bom povo'' de
então. O conto degrada gêneros e estilos românticos: o conto, a novela,
o poema narrativo, o romance, o folhetim; ironiza personagens típicos: o
herói aventuresco, a heroína apaixonada, o vilão intrigante; critica uma
espécie de ação: a aventura exterior, a complicação; desqualifica o
ideal heroico-erótico, honra, amor, devoção, que os negócios de 1882
tornam improvável.

Nos seis capítulos curtos, a repetição dos lugares incha e deforma o
texto no cômico que o transforma objetivamente em meio para outra coisa.
Como um cabide ou varal sempre esticáveis onde é pendurada a roupa
velha, os estereótipos se dependuram na imortalidade, uns após outros, e
são o que são: mortais. Em 1882, pelo viés de \emph{Memórias Póstumas}
já é impossível não rir de enunciados como: ``Disse-lhe que não me
esquecesse, dei-lhe uma trança de cabelos, pedi-lhe que perdoasse o
carrasco''; ``Era um modo de afogar o desespero, que era grande, pois
amara muito a mulher, como um louco\ldots{}''. Bem antes de \emph{Orlando}, em
que o personagem troca de sexo durante quatrocentos anos para que
Virginia Woolf componha o romance como um painel das mudanças históricas
da vida e da arte desde a Inglaterra elisabetana, no século \textsc{xvi}, até a
vitoriana, no fim do \textsc{xix}, ``O Imortal'' propõe sub-repticiamente ao leitor
do seu tempo a experiência da historicidade dos modos de escrever e
consumir ficção.

Correndo ao lado do tema da imortalidade, que provavelmente é o que mais
chama a atenção porque a história é fantástica, o tema na verdade
principal de ``O Imortal'' é subterrâneo e decisivo, porque corrosivo e
destruidor da representação: a impossibilidade moderna de contar
histórias em que a aventura, o amor, a intriga e a intervenção de causas
maravilhosas, como a beberagem do pajé, sejam temas sérios e também
causas ou motivos propostos como explicações naturais, habituais e
normais. O verdadeiro tema de ``O Imortal'' é a verossimilhança.

O texto pode ser relacionado diretamente com passagens dos \emph{Tópicos
I}, da \emph{Retórica} e da \emph{Poética,} onde Aristóteles escreve
sobre a atividade do historiador e do poeta, prescrevendo que devem
compor imitando as opiniões tidas por verdadeiras pelos sábios ou pela
maioria deles. As opiniões tidas por verdadeiras fornecem causas e
explicações que tornam o discurso \emph{verossímil} ou semelhante ao
verdadeiro da opinião. A história e a antropologia demonstram, como o
leitor sabe, que os critérios de ``verdadeiro'' são variáveis ao longo dos
tempos. ``A ciência de um século não sabia tudo'', diz o Dr.\,Leão. Em
Roma, por exemplo, o amor de escravo por patrícia não encontrava para
apoiá-lo nenhuma opinião estabelecida que o definisse como algo
``verdadeiro'', por isso era tido como improvável, sendo proposto como
assunto do ridículo, o pequeno riso da comédia. Definido como
não-natural, não-habitual e não-normal, era classificado como
inverossímil. Mas a mesma inverossimilhança era adequada no gênero
cômico, pois fazia rir com a desproporção. A partir do século \textsc{xviii}, na
Inglaterra, o tema do inferior que se apaixona pelo superior passou a
gozar de grande prestígio romântico, tornando-se natural, habitual e
normal um gênero de conto, novela, romance e poema que tratam dele de
modo não-cômico, mas sério, ingênuo-patético- sentimental.

De todo modo, é útil lembrar: \emph{a verossimilhança é uma relação de
semelhança entre discursos}. Ou seja: a verossimilhança decorre da
relação do texto de ficção não com a realidade empírica da sociedade do
autor, mas da sua relação com outros discursos da sua cultura, que
funcionam como explicações ou causas da história narrada, tornando-a
adequada àquilo que se considera natural, habitual e normal que aconteça
\emph{na} realidade e \emph{como} realidade. A ficção é verossímil
quando o leitor reconhece os códigos que julga verdadeiros e que são
aplicados pelo autor para motivar as ações da história. O verossímil
\emph{motiva} a ficção, ou seja, fornece motivos para as ações.
Aristotelicamente, cada gênero tem uma verossimilhança específica,
aplicando motivos particulares como explicação e causa das ações. O
discurso da história sempre começa pelo início da ação narrada, indo do
mais recuado no passado em direção ao presente em que é escrito,
seguindo uma ordem definida como natural. A poesia épica começa com a
ação pela metade, seguindo uma ordem artificial. O fantástico narra
ações impossíveis. Na tragédia, os personagens devem ser melhores que o
espectador, ao passo que a história não tem que melhorar a vida dos
homens. Na comédia, os personagens devem ser piores que o público etc.
Tradicionalmente, usar os motivos específicos que conferem
verossimilhança a um gênero para compor o verossímil de outro era
definido como inépcia artística e inverossimilhança. Por exemplo,
aplicar a baixeza do caráter dos personagens da comédia para escrever
uma tragédia. Tal uso só era admitido como a ``licença poética'' pela qual
as incongruências fingidas tinham por finalidade parodiar as convenções
do gênero imitado e causar riso. Assim, segundo o preceito aristotélico
do ``semelhante ao verdadeiro'', num primeiro momento ``O Imortal'' aparece
como inverossímil, pois não pode ser comparado com nenhuma opinião sobre
o assunto ``morte'' que possa ser tida como verdadeira. Segundo a mesma
concepção, no entanto, o que é inverossímil em um gênero torna-se
adequado ou verossímil ao gênero fantástico, que se ocupa justamente de
narrar coisas falsas e improváveis no registro da ``licença poética''. É o
caso de ``O Imortal'', em que três critérios históricos de verossimilhança
aparecem superpostos.

Um deles é a verossimilhança do gênero fantástico, apropriada por
Machado de Assis da longa tradição satírica de Luciano de Samósata. No
caso, o verossímil é construído por meio de ações e eventos falsos,
improváveis e inverossímeis, pois esse é o ``verdadeiro'' da opinião que
se tem sobre as convenções do gênero fantástico. Nesse sentido, logo no
início do conto, quando um dos ouvintes corrige a data do nascimento do
pai do Dr.\,Leão, tenta mudar o registro da narração, transformando o
fantástico que começa a ser contado em gênero realista baseado em
opiniões tidas por ``naturais''. Mas o Dr.\,Leão insiste e mantém a
história como gênero fantástico. Assim, quando diz que ela não é fácil
de crer, joga com a dupla perspectiva da recepção, que já apareceu na
correção feita por um dos ouvintes: a história não é fácil de crer, se
for lida por meio da verossimilhança positivista-realista; mas é
totalmente crível se for lida como gênero fantástico, que aplica
convenções críveis para narrar o incrível.

Outra espécie de verossimilhança que organiza o texto é a da ficção
romântica, que exigia causas, explicações e motivos ideais e idealistas
para as ações. Romanticamente, o amor, o heroísmo, a honra etc. são
opiniões verdadeiras como causas alegáveis para explicar qualquer ação.
No final do mesmo século \textsc{xix}, num momento em que todos os sistemas de
representação foram abalados pelo capital, novos critérios de definição
de ``verdadeiro'' passaram a reger a legibilidade da literatura e a
visibilidade das artes plásticas, tornando a verossimilhança romântica
improvável, inverossímil e, logo, cômica. Obviamente, para as leitoras
de \emph{A estação,} que talvez tenham lido ``O Imortal'', os motivos
românticos estilizados no conto pareciam naturais, porque a cultura é a
única natureza possível para os homens. O que costuma ocorrer é que o
leitor de literatura geralmente sofre de etnocentrismo ingênuo, pois
quase nunca pensa que sua cultura não é natural, como uma
particularidade entre outras, tendendo a generalizá-la como critério
universal de avaliação, como se fosse ``verdadeira'' para todos os tempos
e lugares. Pode-se supor que a leitora de \emph{A estação} lia ``O
Imortal'' desse modo: naturalmente, como você e eu, com uma atenção
delicadamente flutuante voltada para o enredo, ``o que ele quis dizer'',
não para a técnica ou para a crítica que o texto efetua. Só quando é
flagrantemente inverossímil o leitor percebe o artifício da ficção,
podendo pensar, quando pensa, as duas coisas ditas antes: o escritor é
incompetente, não conhece as regras da sua arte e escreve mal ou a
inverossimilhança é tão óbvia que deve ser proposital, tendo um sentido
que ainda deve ser achado.

Para especificar essa inverossimilhança produzida programaticamente pelo
escritor, é útil insistir em que a ficção não é a vida empírica,
confusão naturalista, pois esta não tem nenhum sentido predeterminado. A
ficção imita outra coisa, os discursos que regulam a vida, devendo ser
absolutamente lógica no modo como os imita para fazer sentido mesmo
quando seu efeito é a total falta de sentido. Quando conta uma estória,
o narrador constrói sequências somando palavras: ``Meu'', ``Meu pai'', ``Meu
pai nasceu'', ``Meu pai nasceu em\ldots{}''. Evidentemente, como diziam os
formalistas russos, diz alguma coisa antes para relacioná-la
funcionalmente com outra que vem depois, por isso inúmeras
circunstâncias poderiam ser usadas pelo Dr.\,Leão com a preposição ``em'':
``em Recife'', ``em um lugar distante'', ``em 1800'' etc\ldots{} No caso, Machado o
faz dizer ``em 1600''; a data exclui todas as outras circunstâncias e, ao
mesmo tempo, produz a necessidade de dar continuidade a uma sequência
que vai diretamente de encontro a uma expectativa realista fundamentada
em opiniões ``verdadeiras'', como as de João Linhares e Bertioga, que
julgam falsa, e com razão, a ideia de que um homem tenha nascido em 1600
e seja pai do personagem que lhes fala em 1855. Se o narrador afirma:
``Meu pai nasceu em'', eles esperam, segundo a opinião que fazem sobre ``o
verdadeiro'', que apareça algo provável e, por isso mesmo, previsível ---
``em 1800, naturalmente''. Ou seja: acreditam que ouvir ou ler uma
história significa \emph{reconhecer} algo já ouvido ou lido antes,
\emph{naturalmente}. Para eles, a semelhança é tudo. Como o personagem
insiste em afirmar ``em 1600'', também o leitor acha que isso não é
``natural''. O improvável do ``não-natural'' é imprevisível, por isso o
leitor fica imediatamente avisado de que ou o personagem mente, ou o
personagem é inepto ou o personagem lhe está propondo outro esquema
retórico, outro gênero literário e outra legibilidade.

A literatura moderna, como a de Machado de Assis a partir de
\emph{Memórias Póstumas de Brás Cubas}, fez desse arbitrário da direção
narrativa um dos seus eixos principais, produzindo a imprevisibilidade
que \emph{desnaturaliza} os modos habituais de ler. A desnaturalização
incide diretamente sobre a verossimilhança. Gerard Genette propôs que
há, basicamente, três graus da verossimilhança aplicáveis às
narrativas\footnote{Cf. Genette, Gérard. ``Verossímil e Motivação''. In
  Barthes, Roland \emph{et alii}. \emph{Literatura e Semiologia}.
  Seleção de ensaios da revista ``Communications''. Trad. Célia Neves
  Dourado. Petrópolis: Vozes, 1971.}. Nenhum deles é melhor ou pior e
todos podem ocorrer, mas a literatura moderna prefere um deles, como se
verá adiante. O primeiro caso é o de um ``grau zero'' de marcas do
verossímil. O discurso não apresenta quase nenhuma explicação ou causa
das ações dos personagens e a ausência de explicação corresponde
justamente à suposição, partilhada pelo narrador e leitor, de que o
narrado é totalmente natural, habitual e normal. É o caso do exemplo de
início idiota de narrativa dado por Valéry, ``A marquesa saiu às 5h'',
que é um enunciado tido como natural, habitual e normal, não
necessitando de nenhuma explicação, pois a existência de marquesas é um
fato, existe o hábito de sair e a hora, 5 da tarde, não parece
extraordinária ou inconveniente. Da mesma maneira, se o Dr.\,Leão
dissesse ``Meu pai nasceu em 1800'', nenhuma explicação seria necessária e
nenhum dos ouvintes interviria para tentar corrigi-lo, como ocorre na
segunda fala do conto.

Genette propõe como segundo grau de verossimilhança aquele em que
aparecem explicações motivando o que é narrado. As explicações
particularizam ou generalizam os motivos da ação. Por exemplo, quando o
Dr.\,Leão explica por que o nobre espanhol não quis duelar com seu pai,
dá uma explicação generalizante, por assim dizer ``sociológica'', o
código de honra da aristocracia ibérica, que impede, em geral, que nobre
suje as mãos com sangue plebeu em duelos. Uma explicação
particularizante ocorre quando o Dr.\,Leão diz que o marido traído pôde
denunciar seu pai à Inquisição porque tinha visto os objetos que tinha
trazido da Índia. Ou, ainda, quando desautoriza a interpretação que João
Linhares faz do comportamento de D.\,Helena, afirmando que seu pai ficou
sabendo que ela havia contratado o comensal para escrever a carta etc.
Tais explicações funcionam bem, pois correspondem às opiniões do leitor
sobre ``aristocracia'', ``vingança'', ``honra'', ``aleivosia'', ``cartas
anônimas'', ``maridos traídos'' etc. etc., podendo-se dizer que o leitor as
espera para que o artifício narrativo seja ``natural''.

O terceiro grau pode ser o mais interessante, no caso de ``O Imortal''.
Novamente, vejamos a fala inicial do Dr.\,Leão: ``Meu pai nasceu em
1600\ldots{}''. Se o leitor se lembra, desde que o médico afirma que sua
história não é fácil de crer, outros enunciados sem explicação vão sendo
justapostos a ``Meu pai nasceu em 1600\ldots{}'', como é o caso dos enunciados
referentes à poção que torna imortal. A não ser a explicação de Pirajuá,
que diz a Rui de Leão que foi preparada por um pajé de longe, o leitor
lê sobre o efeito da bebida e sobre coisas, ações e acontecimentos sem
explicação, passando para outros também sem esclarecimentos. Antes de
começar a história, o Dr.\,Leão adverte seus ouvintes de que não pode
``entrar em pormenores'' e, com isso, motiva ou explica o arbitrário da
falta de explicação ou motivação para muitas ações e para o encadeamento
delas. Os formalistas russos do início do século \textsc{xx} chamavam de
``procedimento a nu'' a técnica que representa para o leitor o próprio ato
que constrói o discurso, ou seja, as decisões do narrador, que conta sem
explicar. Esse procedimento que narra o insólito sem explicação é
nuclear na literatura moderna, que o aplica para criticar, negar e
destruir os sistemas causais de interpretação que o leitor julga
naturais, evidenciando a particularidade e a arbitrariedade deles num
mundo em que ``opiniões verdadeiras'' são ideologia. ``Certa manhã, Gregor
Samsa acordou\ldots{}'', escreve Kafka, sem nenhuma explicação, o que faz com
que o texto seja literal. É verdade, no entanto, que o Dr.\,Leão fala
várias vezes a expressão ``na verdade'', quando alega causas com que tenta
tornar plausível a narração, adequando a história às opiniões
``verdadeiras'' dos seus ouvintes.

Historicamente, a noção aristotélica de verossimilhança teve vigência
enquanto se acreditou que existia adequação substancial entre os signos
da linguagem, os conceitos e as estruturas da realidade empírica. No
final do século \textsc{xix}, como disse, os sistemas de representação
considerados suficientes até então para estabelecer essa adequação, como
a linguagem, entraram em crise e, com eles, a literatura, que deixou de
ser uma reprodução previsível de opiniões tidas como verdadeiras.
Deixando de ser semelhante às opiniões tidas por ``verdadeiras'', passou a
ser escrita como transformação dos próprios meios técnicos de produzir
literatura. Produzindo efeitos de sentido a partir de si mesma, ela
passou a chamar a atenção do leitor para o seu próprio artifício,
evidenciando-se como produto arbitrário, sem relação necessária com o
que se entendia por ``verdadeiro''.

A partir de \emph{Memórias Póstumas de Brás Cubas}, Machado de Assis
passou a escrever ficção como dispositivo que dissolve o princípio de
causalidade da verossimilhança. É por isso que, nesse livro, sua arte
aparece para o leitor como funcionalidade do procedimento a nu, que
torna indecidível o sentido da história narrada. Logo, na obra de uma
imaginação sempre racionalmente controlada como é a sua, a própria noção
de ``realidade'' torna-se crítica, pois são abaladas as opiniões tidas
como ``verdadeiras'' que o leitor tem da mesma. Devemos lembrar que a
falta de sentido da dor humana, a loucura, o desencontro e a desarmonia
do universo são temas obsessivamente tratados por ele, que escreveu num
tempo em que a ideologia evolucionista de ``nação'', ``ordem'' e ``progresso''
afirmava o darwinismo social como verdade científica que classificava e
excluía a gente explorada branca, negra, índia e mestiça como sub-raça.
Ao destruir as semelhanças previsíveis que pressupõem a naturalidade e a
normalidade das ``opiniões verdadeiras'' como fundamento da ação, as
histórias contadas pelos narradores machadianos são como palcos onde se
encena a inversão sistemática das convenções ``verdadeiras'' do leitor.
Sistematicamente, seus narradores opõem e invertem os termos de
\emph{realidade/aparência}, \emph{razão/loucura},
\emph{ideal/interesse}, \emph{verdade/falsidade},
\emph{verossimilhança/inverossimilhança}, que organizavam a
racionalidade das práticas de seu tempo. Esse modo de neutralizar as
significações familiares e previsíveis, que é observável no seu
compromisso exclusivamente artístico com a forma, talvez pretendesse a
autonomia de uma liberdade estética que recusa a instrumentalização da
arte, inclusive a ideologia naturalista da literatura como semelhança
refletora da realidade empírica, que com ele se torna indeterminada. A
ficção escrita como questionamento da possibilidade da existência da
ficção é um dos temas privilegiados da sua arte inventada como uma
singular teoria da enunciação. Como ``A Chinela Turca'', ``Singular
Ocorrência'', ``A Cartomante'' e outros contos, ``O Imortal'' joga com o
arbitrário de direção narrativa, dissolvendo a verossimilhança
tradicional por meio da estilização e paródia da mesma como gênero
cômico.

Mas o que ocorreu com Rui de Leão? preso pela Inquisição espanhola,
temeu inicialmente ficar detido para sempre; depois, acreditou que o
Santo Ofício o soltaria quando descobrisse que não morria; finalmente,
que seria um alívio ficar livre do ``espetáculo exterior'' do mundo etc.
Para encurtar, ele finalmente morreu em 1855. Como? \emph{Similia
similibus curantur}: \emph{os semelhantes são curados com os
semelhantes}. A caracterização do Dr.\,Leão como homeopata, no início do
conto, tem sua função revelada. Um dia, ouvindo o filho falar sobre a
homeopatia, Rui de Leão tem a ideia de beber novamente a poção e morre.
Como Ella, a feiticeira apaixonada pelo Calícrates da novela de Haggard,
que também morre, quando entra pela segunda vez no fogo sagrado que lhe
deu a imortalidade.

Falta talvez explicar a própria narrativa do Dr.\,Leão. Por que ele
contou tal história? O narrador afirma que a seriedade do médico era tão
profunda que Bertioga e Linhares ``creram no caso, e creram também
definitivamente na homeopatia''. Aqui, o narrador pluraliza as
explicações para reconduzir o leitor à questão da verossimilhança:

``Narrada a história a outras pessoas, não faltou quem supusesse que o
médico era louco; outros atribuíram-lhe o intuito de tirar ao coronel e
ao tabelião o desgosto manifestado por ambos de poderem viver
eternamente, mostrando-lhes que a morte é, enfim, um benefício. Mas a
suspeita de que ele apenas quis propagar a homeopatia entrou em alguns
cérebros, e não era inverossímil''.

Três opiniões tidas pelo leitor como fundamentos válidos de explicações
``verdadeiras'' são mobilizadas pelo narrador para explicar a causa da
ação do homeopata: a irracionalidade, a generosidade, o interesse. O
narrador não toma partido de nenhuma delas, deixando a solução em
aberto, talvez porque há uma quarta: o Dr.\,Leão conta a história da
imortalidade do pai não \emph{porque} é louco, generoso ou interesseiro,
mas \emph{para} Machado de Assis demonstrar ao leitor que a
historicidade do artifício da verossimilhança é mortal.


\chapter[Sobre «A cartomante», \emph{por Alcides Villaça}]{Sobre «A cartomante»\subtitulo{Machado de Assis, tradutor de si mesmo\footnote[*]{Originalmente publicado em: \emph{Revista Novos Estudos Cebrap.} São Paulo, 1998, n. 51, pp. 3--14.}}}
\hedramarkboth{Sobre «A cartomante»}{}

\begin{flushright}
\textsc{alcides villaça}
\end{flushright}

\section*{I}

\noindent{}Não nos cansaremos de encontrar nos textos machadianos o empréstimo de
ideias e de formas, as incontáveis alusões, as fontes veladas ou
explícitas, as citações e as glosas, os lapsos forjados ou verdadeiros
--- todo um arsenal, enfim, de dispositivos intertextuais que encorajam
estudos comparatistas. Cada detalhe revelado, cada minúcia
caprichosamente ampliada pode provocar o falso dilema: será o texto
machadiano muito mais rico ou muito menos original do que se pensava? A
intertextualidade não traz, por si mesma, o critério de valor que decida
a questão. É óbvio que o comparativismo consequente deve levar em conta
os processos e os contextos de composição, que sobredeterminam as
pontualidades comparadas. Como, no caso de Machado de Assis, as questões
que contam derivam da complexidade da narração processada e do modo como
esta se articula com a cultura nacional, corre-se o risco de querer
comparar \emph{a} com \emph{x}, e não com \emph{b}. Considere-se ainda o
fato de que dificilmente os artistas excepcionais são, por isso mesmo,
``comparáveis'' entre si; sempre foi mais útil compreendê-los pelo modo
como consideraram, para superá-las, expectativas firmadas na tradição.
As homenagens que Machado está permanentemente (e a seu modo) prestando
às incontáveis ``fontes'' de seu repertório não são reverências ao valor
intrínseco deste, mas pontes para um outro valor que ao mestre interessa
estabelecer. Podemos e devemos voltar por elas, pois não, mas justamente
para apreciar no caminho a qualidade desse olhar moderno, que
transfigurou a antiga paisagem. É provável que um príncipe florentino
não se sinta bem na companhia de Janjão (``Teoria do medalhão''), que
Hamlet distribua empurrões para não ver Rita e Camilo ladeando-o num
patético retrato (``A cartomante''), ou que o Império Romano se recuse a
caber no bolso de Custódio (``O empréstimo''); do nosso lado, Império e
República são letreiros de confeitaria (\emph{Esaú e Jacó}), e os
partidos nacionais em alternância no poder sempre podem inspirar
buliçosas polcas (``Um homem célebre''). O racionalismo ilustrado talvez
não pare em pé no trapézio de Brás Cubas (\emph{Memórias póstumas}), e
não é fácil aceitar que um copista de estudos de teologia venha a
emendar as Escrituras (``O enfermeiro''). A falibilidade de Deus,
revelada por Seu gesto de desalento resignado, coloca-o apenas um degrau
acima da ingenuidade de um Lúcifer humilhado (``A Igreja do Diabo'').
Para que não adotemos, enfim, o mesmo ponto de vista vertiginoso do
delírio que leva a Pandora, e para que não nos estarreça cada elemento
minúsculo de um particularíssimo processo de remissões, resta tentar um
pouco de luz nos subterrâneos desse narrador, com a esperança de vermos
alguma coisa em torno.

Interessa-me, aqui, analisar um dos processos de composição acionados
por Machado, uma certa ``tradução'', que buscarei determinar. Tal
processo está representado de forma privilegiada (e à maneira sempre
oblíqua do autor) no conto ``A cartomante'', de \emph{Várias histórias}.
São páginas das mais sugestivas que Machado se permitiu, no limiar de
uma autorrevelação a que altivamente se furtou.

\section*{ii}

Na perseguição do processo representado, o leitor escorregará primeiro
pela superfície da historieta melodramática que termina em morte ---
provavelmente o piso único para muitos dos leitores da \emph{Gazeta de
Notícias}, em 1884, quando o conto se publicou. Um pouquinho mais abaixo
do caso de adultério que transpira e é vingado situa-se o plano das
superstições, dos vaticínios e do destino caprichoso. Descendo ainda um
palmo, o leitor encontrará estímulos para discutir o papel da
causalidade ou do acaso e suas consequências no rumo dos acontecimentos.
Nessa verticalidade de planos, tanto a boa dama fluminense quanto um
Augusto Meyer saberão fixar o que lhes aproveite (este último deteve-se,
como se sabe, no do homem subterrâneo). Eu gostaria de explorar, na
horizontal, uma insistente sucessão de considerandos e analogias que o
narrador vai tecendo aos poucos, ao longo da história.

A alusão que abre a narrativa poderia ter pouco peso, não fosse ela o
mote mesmo de um processo que se irá reiterar:

\begin{quote}
Hamlet observa a Horácio que há mais cousas no céu e na terra do que
sonha a nossa filosofia. Era a mesma explicação que dava a bela Rita ao
moço Camilo, numa sexta-feira de novembro de 1869, quando este ria dela,
por ter ido na véspera consultar uma cartomante; a diferença é que o
fazia por outras palavras.
\end{quote}

Tomando Shakespeare, Machado se vale da batida frase de Hamlet a
Horácio; apesar de gasta, é sempre uma referência culta. No período
seguinte, as duas personagens da tragédia dão rapidamente lugar à ``bela
Rita'' e ao ``moço Camilo'', numa cena risonha e bem datada. Parece que
a ``filosofia'' de que falava o atormentado Hamlet encontra paralelismo
nos sortilégios da cartomante que a crédula Rita foi consultar. Mas a
ponte se armou foi no eixo da ``mesma explicação'' que tanto Shakespeare
quanto Machado colocaram na boca de suas personagens. Uma ponte, aliás,
meio torta, pois a ``\emph{mesma} explicação'' logo surge ressalvada:
``a diferença é que o fazia por \emph{outras} palavras'' (grifos meus).
Em suma: descontada a questão da forma, Hamlet e Rita podem estar muito
próximos, não sendo impossível a convergência de seus pensamentos.

O ``mesmo'' pode atuar dentro da ``diferença'': as palavras podem ser
``outras'', sem prejuízo aparente para as ``explicações''. Tal
dissociação entre forma e conteúdo talvez escandalize os princípios de
um bacharel em Letras, mas não soará razoável para um leitor comum?

Um pouco mais adiante, o narrador esclarecerá que Rita, ``sem saber que
traduzia Hamlet em vulgar, disse-lhe que havia muita cousa misteriosa e
verdadeira neste mundo''. Aqui, a forma do chavão aparece em sua
inteireza, na condição de uma vulgata que traduzisse, por pura
coincidência, o discurso hamletiano. O efeito de despropósito é
relativizado pelo narrador com a autoridade de quem, íntimo tanto de
Hamlet quanto de Rita, pode perfeitamente ajuizar quanto à procedência
de uma comparação, sobretudo quando as diferenças, tidas por mínimas,
acabam por se eliminar no fundo do ``mesmo''.

O conto faz ver que Rita, além de ``formosa'', é também ``tonta'' --- o
que parece vir desenhado na expressão meio aérea da ``boca fina e
interrogativa''. Por sua vez, Camilo ``era um ingênuo na vida moral e
prática'', não tendo ``nem experiência, nem intuição''. É muito natural,
pois, que venham a se atrair: ``não tardou que o sapato se acomodasse ao
pé'' --- a forma de acomodação indicando também o nível terrestre em que
se dá. Como o leitor não quer se identificar nem com tontos nem com
ingênuos, acompanhará os acontecimentos com curiosidade e distância.
Lembremos ainda que, na origem da sedução, Camilo aniversariante
recebera de Rita um ``cartão com um vulgar cumprimento a lápis'',
bilhetinho de ``palavras vulgares'', sim, mas --- propõe o narrador ---
``há vulgaridades sublimes, ou, pelo menos, deleitosas''. Não há dúvida
de que a ideia do vulgar está se expandindo no conto, e que na outra
ponta da escala o narrador vai buscar o sublime para montar --- e
relativizar --- o paradoxo. Esse processo de eliminação das diferenças
ganha expressiva ilustração na referência à ``velha caleça de praça''
que, para os namoradinhos que lá dentro se apertem, ``vale o carro de
Apolo''. Portanto: Rita ``vale'' Hamlet, um vulgar cumprimento ``vale''
uma mensagem sublime, a mesma explicação ``vale'' em diferentes
palavras. De valor a valor, de tradução em tradução, as vulgatas valem o
original, o prosaico vale o mitológico, a curiosidade vale a metafísica,
a cartomancia vale o conhecimento. Ao promover essas traduções
aparentemente disparatadas, o narrador cria um critério para sua
narração; acompanhemo-la.

Não faltam ao conto ingredientes de melodrama romântico: o nó da intriga
se aperta com a atuação acusatória e ameaçadora das cartas anônimas que
vão chegando a Camilo. Saberá delas o amigo Vilela? Que fará, se souber?
O recurso é novelesco, e foi duramente apontado pelo próprio Machado
como uma das debilidades de \emph{O primo Basílio}, de Eça de Queirós.
Também não falta ao conto certa pimenta naturalista que, a princípio
abrandada pela melodia italiana da expressão ``\emph{odor di femmina}'',
revela todo o ardor neste período de alusão bíblica: ``Rita, como uma
serpente, foi-se acercando dele, envolveu-o todo, fez-lhe estalar os
ossos num espasmo, e pingou-lhe o veneno na boca''. Expediente
melodramático e sensualismo rastejante vão alimentando o texto com sua
vulgaridade, sem que o leitor, no entanto, possa imputá-la ao estilo
propriamente dito, que se conserva elegante e precavido como quem calça
finas luvas para lidar na cozinha. Além do que, já nos prevenira o autor
quanto à hipótese das ``vulgaridades sublimes''.

A ameaça das cartas anônimas é seguida por um lacônico bilhete do amigo
Vilela a Camilo: ``Vem já, já, à nossa casa; preciso falar-te sem
demora''. Que valerá esta mensagem: uma ordem raivosa de quem se soube
traído? um chamado para negócios urgentes? Como faltam a Camilo aqueles
``óculos de cristal, que a natureza põe no berço de alguns para adiantar
os anos'', sobra-lhe o dilema: ir ou não ir à casa do outro? \emph{To go
or not to go?} O caso amoroso, tão banal quanto os que o vivem, é
rondado pelo trágico. Sem o concurso da experiência ou da intuição, o
ingênuo Camilo deve tomar uma decisão em que pode estar a diferença
entre viver ou morrer. Busca algum recado de Rita, ``que lhe explicasse
tudo'', mas ``não achou nada, nem ninguém''. Depois de imaginar cenas de
drama, de cogitar em ir armado, de crer que de fato já estava vendo o
que iria acontecer, Camilo parece abandonar-se àquele mesmo movimento
pelo qual o narrador o apresentara no início do conto: ``diante do
mistério, contentou-se em levantar os ombros, e foi andando''. É verdade
que esse impulso de sua natureza ingênua, incrédula e indiferente leva,
agora, um coração batendo muito forte, e se ainda é ingênuo e incrédulo,
não mais se dá ao luxo de ser indiferente, e ``dar de ombros''. Tão
desconfortável quanto o medo da morte, o dilema obrigado à decisão faz
Camilo entrar logo num tílburi, para ir até Vilela. ``-- Quanto antes,
melhor, pensou ele; não posso estar assim''. A inconsistência desse
``melhor'' (?) dá a medida da irracionalidade do impulso que decide pela
personagem.

A ironia machadiana apoia-se com muita frequência nas simetrias,
traduzindo uma situação por outra num eixo de equivalências --- processo
nada estranho, como se vê, ao que está sendo comentado por mim e pelo
próprio conto. Se uma coisa vale outra, se Camilo vale Rita, por que não
irá ele parar na mesma cartomante? Faltando-lhe iniciativa, o narrador
faz o tílburi deter-se junto a uma carroça atravessada na rua, bem em
frente à casa da adivinha. A simetria revela, mais literalmente do que
nunca, a ironia da sorte: ``Dir-se-ia a morada do indiferente Destino''.
Supersticioso por necessidade, e inconscientemente obedecendo às ordens
dos homens que procuram safar a carroça (``Anda! agora! empurra! vá!
vá!'' --- as duas últimas parecendo reforçar eufonicamente o ``Vem já,
já, à nossa casa'' de Vilela), sobe Camilo a escada, sugestivamente
arcaica, de ``degraus comidos dos pés'' e de ``corrimão pegajoso'',
conduzido pela cartomante. A cena da consulta contrapõe a ingenuidade do
consulente à sagacidade da italiana ``com grandes olhos sonsos e
agudos'' que, arrancando-lhe o que ele quer ouvir, vaticina o futuro
bonançoso e o despede: ``Vá, \emph{ragazzo innamorato}\ldots{}'', ``vá,
vá tranquilo'' (outra eufonia, outra correspondência). Levado por mais
esse empurrão, o amante de Rita segue confiante para a casa de Vilela,
olhando o mar, ``até onde a água e o céu dão um abraço infinito, e teve
assim uma sensação do futuro, longo, longo, interminável''. Vilela o
recebe sem palavras e o leva a uma saleta onde, antes de pegá-lo pela
gola e abatê-lo com dois tiros, dá-lhe o tempo de ver Rita morta e
ensanguentada num canapé.

\section*{iii}

Esta a história, notável também por tantos e tantos outros detalhes de
construção caprichosa, expressiva e provocadora, que o interesse de
outros leitores poderá analisar. Retomo o meu, que vinha se orientando
no plano das traduções, um dos expedientes-chave da poética machadiana.
Vamos a elas.

Numa acepção corrente, o termo \emph{paródia} indica a retomada de um
texto, de uma forma, de um estilo, para efeito de seu deslocamento a
novo eixo morfológico-expressivo, onde o sentido original se transvia,
quase sempre rebaixado, servindo pelo avesso a uma outra posição
crítica. Em ``A cartomante'', Machado cria uma estranha relação entre a
tragédia shakespeariana, o carro de Apolo e o sublime, de um lado, e as
vulgaridades todas da história de Rita e Camilo, de outro. O método é o
de ir emparelhando elementos de uma tradição alta e elementos prosaicos
de uma gozosa existência burguesa (``Adeus, escrúpulos!''). Mas a
impressão de paródia não passaria disso, pois o final é agudo e
trágico\ldots{} Trágico? Não, não é este o efeito estético das duas
mortes violentas, tão abruptas quanto inglórias, que parecem abrir o fim
do conto para uma picante manchete na página policial do dia seguinte; o
efeito estético é o do grande descompasso entre o fato e a fatura
literária, tão elegante e precisa esta, tão vulgar aquele. Afinal de
contas, os amantes morreram sobretudo pela má administração dos
colóquios, e sempre lhes teria faltado qualquer vocação para o heroico.
Pode-se dizer que morreram de vulgaridade, o ingênuo Camilo e a tonta
Rita, tendo no entanto encontrado o seu Shakespeare, que se não os fez
Hamlet e Ofélia, nem Romeu e Julieta, soube compreender o carro de Apolo
que parecia estar em sua caleça de praça. ``Há vulgaridades sublimes'':
nessa perspectiva, um prisma da modernidade já permite fundir os gêneros
e os planos artísticos, as virtudes e os vícios humanos, de modo que o
escritor se libere para contar histórias prosaicas que não desmentem a
grandeza de um Shakespeare, apenas a ``atualizam''. Nessa tradução
burguesa está por certo o novo leitor, ávido das emoções fortes que as
subnovelas românticas ou vagamente realistas lhe ofereciam nos jornais,
a \emph{Gazeta de Notícias} entre eles. É como se os componentes
clássicos do trágico, do heroico e do sublime estivessem agora à
disposição num eclético bazar da época, adaptados a um consumo
cotidiano, bem à mão dos consumidores. Caiu um pouco, por anacrônica, a
aura original? Não há por que lamentá-lo: ela se faz representar agora
por seu valor nominal, apeada do seu Pégaso, mas firme na caleça de
praça. A astúcia de Machado está em reafirmar que uma coisa vale a
outra; mas o leitor mais desconfiado não parará por aí. E aí começam as
interpretações.

\section*{iv}

É próprio do pensamento mítico que um arquétipo viva de sua atualização
e recriação, ainda que sob forma aparentemente mais prosaica (caso do
``Recado do morro'', de Guimarães Rosa) ou mesmo degradada (caso do
\emph{Ulisses}, de Joyce). Sob o melodrama, ``A cartomante'' parece
apontar para uma tradição de sibilas, deuses e tragédias, tomando em
bloco a presença do sublime e os ecos do estilo alto, e
providenciando-lhes ``traduções'' que caibam no espírito e no espaço de
um conto despretensioso. O narrador se acautela quanto a este disparate
por meio da fórmula ``há vulgaridades sublimes'', e apresenta o
estratagema do ``isto \emph{vale} aquilo'' como argumento para as
alusões. Tal expediente é típico de Machado. Na ``Teoria do medalhão''
(\emph{Papéis avulsos}), o conto terminava com a frase: ``Guardadas as
proporções, a conversa desta noite vale \emph{O Príncipe}, de
Machiavelli''; em ``O enfermeiro'' (\emph{Várias histórias}) o narrador
finalizava com a seguinte ressalva: ``Se achar que esses apontamentos
\emph{valem} alguma coisa {[}\ldots{}{]}'' --- e acabava fazendo uma
``emenda'' ao divino sermão da montanha (grifos meus); a história
contada por Jacobina, em ``O espelho'', vale por um ``Esboço de uma nova
teoria da alma humana'', assim como ``D.\,Benedita'' (ambos de Papéis
avulsos) é a própria personificação da Veleidade; etc. etc. O expediente
é, na verdade, uma autêntica profissão de fé que faz Machado de seu
processo de relativização, amplo, geral e aberto. Nesse processo, o
autor simula conformar-se em emprestar a altura e o fôlego mais
limitados do conto realista à representação de matérias que outra
altitude teriam alcançado na tradição. E de fato nada lamenta, pois ao
mesmo tempo que reconhece a diferença entre o sublime e o vulgar,
dissolve-a, digamos assim, em nome de evidências da prática. A operação
irônica não se apresenta, é claro, como irônica, mas como decantação
pura do cotidiano em que estamos todos: o leitor (ou leitora) que diga
se um momento de intimidade apaixonada não vale o Olimpo\ldots{} Se
assim é, cabe Hamlet entre Rita e Camilo, como cabe ao Destino fazer sua
morada na casa da cartomante sonsa.

Resta ao leitor relativizar, por sua vez, o que faz o escritor (operação
com que por certo contava Machado). A relativização nossa bate sobre
esse fingimento do narrador, que insiste em traduzir pelo mesmo o que
também sabe reconhecer como diferenças. Há uma caprichosa ``tabela'' de
traduções em ``A cartomante'', que se poderia assim organizar:

\begin{quote}
frase de Hamlet a Horácio / explicação de Rita a Camilo

mitologia, religiosidade / superstições ou incredulidade

carro de Apolo / caleça de praça

estilo alto / palavras vulgares, mal compostas

personagens trágicas / tonta Rita, ingênuo Camilo

a morada do Destino / a casa da cartomante

sibila / italiana sonsa

dilema do ``ser ou não ser'' / dilema do ``ir ou não ir''

nivelamento das personagens / nivelamento das personagens

pelo sublime pela vulgaridade
\end{quote}

Por esse sistema de traduções, seria lícito concluir que Machado vale
Shakespeare, ``guardadas as proporções''. Ora, o narrador age exatamente
como um operador do desproporcional, tirando todo o efeito da ironia de
não admitir isso. A partir desse ângulo privilegiado da lucidez que não
tem compromisso com qualquer valor senão consigo mesma, a História vale
um delírio, uma ópera, um papel avulso ou uma folha sem data, um
pretexto qualquer para se recolherem as ``páginas amigas'', que nem por
serem amenas deixam de concentrar uma espécie de ``suma da vida''.
Traduzindo desproporções como equivalências, o narrador atrai o leitor
para o seu sistema, do qual não é fácil sair. Para consegui-lo, teremos
que ter precisão quanto aos nossos valores e suas diferenças; teremos
que definir antagonismos reais, contradições verdadeiras, e ser
consequentes --- exatamente as tarefas mais problemáticas que enfrenta o
pensamento crítico, quando resiste às diluições da modernidade eufórica.
Parece-me ser esse o desafio que, politicamente, Machado armou para si e
para seu público, de ontem e de hoje.

Ainda retornando à fórmula das ``vulgaridades sublimes'', tentemos
aprofundar suas implicações. Sem sair da lógica dessa fusão de opostos,
poderíamos igualmente reconhecer o corolário das ``tragédias
vulgarizadas'', e com ela fundar uma nova perspectiva para o conto. Tal
inversão em nada contradiz o jogo das traduções levado a efeito pelo
narrador, que aliás está sempre a estimulá-lo. A determinação
estilística do perfeccionismo, da elegância culta e do requinte retórico
é o único traço que o narrador não pode ocultar, e talvez seja o único
que de fato o revele. Esse lugar do estilo não surge ``vulgarizado''; se
já não é o sublime, ou o épico, ou o trágico, é por certo ainda um lugar
privilegiado, de cuja altura retórica nos é lançado um olhar
condescendente. Que lugar é esse, onde nasce o princípio absoluto das
relativizações, dos ``caprichos'' (Augusto Meyer), da ``volubilidade''
(Roberto Schwarz), das ``simetrias'' (Alfredo Bosi)?

A pergunta supõe alguma estabilidade do ponto de vista em que a
consciência do narrador se detém para elaborar-se e para promover o
diálogo com o mundo, no desejo de sua representação. Sem essa
estabilidade, ainda que dissimulada, precária ou mínima, não há autor,
estilo e forma consequentes. Creio que em ``A cartomante'', como num
sem-número de outros lugares, o narrador machadiano instala-se nesse
ângulo tão peculiar de ``tradutor'': um tradutor das tradições que
constituem seu repertório de cultura, que vem da Bíblia e de Homero, da
antiguidade clássica e dos teólogos medievais, que passa por Dante,
Maquiavel, Montaigne, Cervantes, Shakespeare, Pascal, pelos
enciclopedistas, por Schopenhauer, pela literatura brasileira --- e acaba
caindo no colo da dama fluminense ou num chapéu elegante da rua do
Ouvidor. Essa ``queda'' --- na verdade o já reconhecido salto crítico do
Machado particularizante e universalmente nacional --- é a marca de fogo
de sua fase madura, quando a ironia se torna princípio e a ``tradução''
uma rica possibilidade de composição. Multiplicado nessa liberdade
vertiginosa, o narrador rastreia quaisquer horizontes para selecionar
com a aparência do arbítrio o que de fato se origina e resulta
determinado. Assentado que está em nível retórico-estilístico de altura
indiscutível, pode-se permitir a fusão da galhofa e da melancolia sem
perder a reverência básica de uma linguagem a ser cultuada pela mesma
dama fluminense ou pelo dono do chapéu. É desse lugar a um tempo
dialético e cristalizado que se podem ver Rita, Camilo, Hamlet e Horácio
numa sequência que, se de um lado promove uma dissolução de valores, de
outro ainda os distingue enquanto singularidades aptas à ``tradução''.
\emph{Ainda} os distingue: é dessa frágil reserva de tempo que parece
anunciar-se uma liquidação geral e efetiva, hipótese em razão da qual o
narrador já se deita ao modo de um defunto e toma posse de uma
(pós)última decisão da consciência teimosamente ativa. O vazio íntimo de
Rita e de Camilo é indicado segundo um parâmetro que poderia supor uma
tão secreta quanto reprimida nostalgia do absoluto, recoberta pelo senso
do realismo e pela análise do cotidiano burguês. Nessa perspectiva, o
sentimento do trágico já se banalizou, vulgarizando-se por consequência
toda uma custosa tradição de expressões do sublime, que perderam o lugar
próprio. Fulminados, Rita e Camilo atualizam com seu próprio estilo de
viver e de morrer uma tragédia burguesa que Machado escreveu para
traduzir o seu tempo e a si mesmo.

\section*{V}

É nesse específico lugar de ``tradutor'' que tantas e tantas vezes se
instalará o narrador livre mas sistemático da fase madura. O humor
principal que daí se destila vive de um paradoxo que busca se negar
enquanto tal: a liberdade ronda o caos, ameaça promover o absurdo e o
nonsense, como a reedição do Gênesis no capítulo ``O delírio'' ou a
conversa gestual entre Adão e Eva (\emph{Memórias póstumas}) --- mas o
leitor sente que o narrador nunca afrouxará o punho firme que segura a
pena racional e elegante, deslizando pelo estilo inconfundível. Tais
``traduções'' tornam-se tema e processo, com direito a um sem-número de
variações que sabem se adaptar à diversidade das situações narradas. Nos
contos, há alguns que traduzem outros (caso de ``Um homem célebre'' e
``Cantiga de esponsais'', por exemplo), variando detalhes, ênfases e
tonalidades, que reparticularizam tudo. O efeito inicial pode ser a
sensação do \emph{mesmo} nas \emph{diferenças} (quando se busca
reconhecer o modo de narrar ou alguma ``ideologia'' sistemática), mas
modula-se no efeito da percepção de diferenças que alcançam alguma
emancipação do \emph{mesmo} (quando se privilegia na análise o
particularismo da expressão artística). Sim, uma coisa não vem sem a
outra, e parece nascer da junção delas o efeito geral de paradoxo, assim
resumível: o narrador vive na multiplicidade das situações criadas sem
se deixar levar por essa mesma multiplicação, antes subordinando-as ao
seu sistema de ``tradutor''. A variedade dos tempos históricos, dos
valores, dos desejos humanos, das lutas pelo poder, dos gêneros e dos
estilos é considerada, sim, por um minuto, para no minuto seguinte
passar pelo funil estreito da perspectiva do narrador, onde a qualidade
original aparece ``traduzida''. Assim é que a metafísica da alma humana
pode exemplificar-se numa laranja (``O espelho'') e a patologia de um
Calígula atualizar-se em escala reduzida num certo Fortunato, que aliás
passa por benemérito (``A causa secreta''). Tais ``traduções'' em nada
escandalizariam um Schopenhauer que considera inacessível aos mortais o
rosto mesmo da Vontade; que dá como limite o estatuto das
\emph{representações} do mundo; e que nos lembra o fato de que uma
circunferência de diâmetro descomunal tem as mesmas propriedades
geométricas da de um diâmetro diminuto. Também para Machado parece certo
que ``muitas vezes uma só hora é a representação de uma vida inteira''
(``O empréstimo''), e nesse caso o contista (e o autor de tantos
capítulos de romances) teria o privilégio de poder ``traduzir'' em
poucas palavras o essencial de uma existência --- e por que não o da
História mesma? No curto espaço de \emph{O Príncipe}, Maquiavel oferece
a Lorenzo de Médicis ``tudo aquilo que, em tantos anos e à custa de
tantos incômodos e perigos, hei conhecido''. Se em estreito molde se
fundava a moderna ciência política (deslocando-se o sentido da
\emph{virtù} do moralismo medieval para a do pragmatismo do poder), por
que não caberia a específica astúcia de um medalhão caboclo numa teoria
exposta em sessenta minutos (``Teoria do medalhão'')? As ``proporções''
a que se refere a personagem desse conto, remetendo-nos a Maquiavel, têm
o sentido duplo das traduções machadianas: tanto implicam o reducionismo
implícito ou explícito do modelo quanto a manutenção do mesmo sentido
básico do ``original'', atualizado e ``traduzido''. Uma universalização
tão descarada parece, no entanto, aguda e verdadeira, quando ela parece
confirmar-se em cada caso particular, transitando da narrativa para a
vida, da ficção para a experiência, num retorno coerente e exemplar, que
irritou Augusto Meyer e o levou à expressão ``monstro cerebral''. O
lastro de realismo considerado pelo autor a cada página é pesado o
bastante para que não nos deliciemos impunemente com a amenidade que
Machado adiciona ao tom; nem aceita o bruxo o poder da imaginação
indiscriminada, pela qual viesse a afastar-se um só milímetro do eixo de
seu realismo básico.

Machado de Assis, que admitia tudo, menos ``ser empulhado'', não admitiu
para si mesmo a hipótese de ser apenas um dissolvente ``tradutor'' de
tradições, um mestre do \emph{divertissement} ou da ironia só engenhosa.
Quando, por exemplo, na ``Teoria do medalhão'', usa e abusa da retórica
decorativa, ela já se faz exemplo prático e funcional da linguagem
recomendada ao tipo; quando numa crônica se vale da filosofia ao tratar
de um recém-publicado manual de confeiteiro, o abuso é declarado. Na
crônica intitulada ``O autor de si mesmo'' (\emph{Gazeta de Notícias},
16/06/1895), Schopenhauer comparece em pessoa, como um Artur familiar,
ratificando seu pessimismo à luz do caso trágico, recente e
dolorosíssimo, que parece ter atingido Machado em cheio. São mostras de
como as ``traduções'', em níveis e em consequências tão diversos,
constituem um autêntico processo de criação e de crítica,
irreversivelmente aberto a partir das \emph{Memórias póstumas}.

Lembremos ainda dois contos: ``O empréstimo'' (\emph{Papéis avulsos}) e
``Um homem célebre'' (\emph{Várias histórias}). O primeiro se apresenta
como anedota verídica, mas não dispensa a entrada de Carlyle, Pitágoras,
Sêneca e Balzac, antes de se contar: é o caso de um pobre-diabo chamado
Custódio, que busca fazer de um tabelião um sócio seu numa fábrica de
agulhas de padrão inglês, sociedade que não custaria ao felizardo
parceiro mais do que \emph{cinco contos}. Diante das negativas do
tabelião, Custódio vai reduzindo o valor do empréstimo, reduzindo,
reduzindo, até que a empreitada inicial se transforma num ``jantar
certo'', patrocinado pelos cinco mil-réis que o outro lhe concede.
Decepção para Custódio? Absolutamente não: ele sai com a nota ``como se
viesse de conquistar a Ásia Menor''. Conclui o narrador: ``ele apertava
amorosamente os \emph{cinco mil-réis}, resíduo de uma grande ambição,
que ainda há pouco saíra contra o sol, num ímpeto de águia, e ora batia
modestamente as asas de frango rasteiro''. Está claro que no frango
rasteiro há um resíduo de águia, e que o pobre Custódio alegrou-se de
qualquer modo porque, afinal de contas, foi pedir um empréstimo e obteve
outro empréstimo, mantendo-se na astronômica \emph{diferença} entre as
quantias a salvaguarda do \emph{mesmo} princípio.

Quanto à obra-prima que é ``Um homem célebre'', não vale a pena insistir
na já decantada sombra autobiográfica que ronda o conto (a experiência
de vida de Machado projeta-se muito mais poderosamente do que parece, em
suas histórias; está por se fazer um estudo meticuloso dessas projeções,
em consequência das quais restaria bastante relativizado o semidogma do
\emph{distanciamento} do narrador --- no final das contas um estratagema
contra a ampla confissão). Interessa aqui sublinhar a distância que vai
das sonatas de Beethoven às polcas de Pestana, tão diferentes entre si,
mas afinal tão próximas para quem tira muito prazer destas, sem que
desmereça aquelas --- possibilidade que não se ofereceu ao romântico
Pestana, condenado pelo autor a sofrer em plena consagração mundana. A
condenação só concede um gesto de simpatia \emph{in extremis}: à morte,
Pestana mostra-se lúcido e revela súbito senso de humor, aproximando-se
enfim do próprio tom da narração e comungando, nesse único instante, da
verve do narrador. Fosse Pestana um Custódio, ouviria suas polcas como
se sonatas fossem, conquistando-as como se representassem a Ásia Menor,
valendo-se delas para ampliar até alguma Roma imperial a sensação que
vinha do sucesso fluminense. Já no plano coletivo da política, ao qual
Pestana servia a seu modo incauto, dedicando-lhe polcas, tudo também
estaria em ninguém se deixar impressionar com a queda ou ascensão dos
liberais ou dos conservadores, sabendo-se que nesse processo as
\emph{diferenças}, mais do que nunca, redundavam no \emph{mesmo}. Por
certo Machado preferiu tratar dessa questão política espelhando-a em seu
próprio processo de criação, processo político em sentido mais ativo e
problemático, resguardado na ampla ironia de tão provocadoras
``traduções''.

Arrisquemos, por fim, um paralelo. No \emph{Ulisses} de Joyce há, sim, o
paroxismo linguístico de quem crê em seu poder final de destruição e
criação, de quem é épico e pedestre na simultaneidade com que opera
tanto as linguagens faladas em Dublin como uma máquina poética
sofisticadíssima, em que os códigos se baralham e se reinventam a cada
momento, atingindo no coração o preceito da unidade estilística. Moderno
e modernista, Joyce encarna com esses pesos a função demiúrgica, própria
de quem tem deuses e demônios a reproduzir e a exorcizar com empenho
máximo. Machado de Assis, diferentemente, considera com impecável apuro
o triunfo da vulgaridade, e só a deixa fracassar quando ainda mais
decisiva do que ela é a falta de malícia ou do poder de adaptação do
sujeito em que ela se encarna. Esse foco objetivante, que de modo algum
quer se deixar confundir com seu objeto, precisa para isso de recursos
estáveis, independentes e autossuficientes, sem os quais o narrador se
arriscaria a escorregar entre valores, identificando-se com as fórmulas
já batidas do pessimismo, do cinismo, do niilismo, para nem falar das
posições afirmativas do moralismo, do cientificismo, do idealismo.
Diante desse estoque de tão aliciantes possibilidades, o narrador
machadiano faz com que elas se traduzam umas pelas outras, vivendo ele
próprio da estabilidade estilística desse lugar aparentemente imune a
qualquer contradição, que é o lugar do puro observador, ou, quando não
apenas isso, o do velho, o do morto, o do diplomata aposentado. Mas só
\emph{aparentemente} imune: quem quisesse suprimir as contradições,
traduzindo-as pelo que não seria mais do que a sempre mesma \emph{verità
effetuale delle cose} (Maquiavel), não abriria nunca o espaço político
da ironia e da análise lúcida, que se definem na diferença pela qual se
constitui, em definitivo, o sujeito que se recusa a ser traduzido pela
perspectiva das \emph{coisas-mesmas}.
