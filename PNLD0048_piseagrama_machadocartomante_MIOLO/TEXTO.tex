\chapter{Apresentação}

A coletânea \emph{O Alienista — O Imortal — A Cartomante} contém três
narrativas de Machado de Assis: as duas primeiras saíram na revista
\emph{A Estação}, periódico voltado para o público feminino que circulou
no Rio de Janeiro entre 1879 e 1904, e a terceira na \emph{Gazeta de
Notícias}. Nelas, reconfigura a matéria social do seu tempo,
desestabiliza a verossimilhança e desnaturaliza os modos habituais de
ler.

\emph{O Alienista}, conto"-novela de Machado de Assis publicado em
\emph{A Estação} entre outubro de 1881 e março de 1882 e incluído em
\emph{Papéis avulsos} em 1892, nos abala a pensar em múltiplas questões
atuais: a importância e os limites da ciência; o exercício tirânico do
poder, que se apropria segundo as conveniências da ciência; os conflitos
e sofrimentos decorrentes da nossa morada nos hábitos instituídos e da
recusa ao diferente; a prevalência de interesses particulares e do
capricho, acima de princípios éticos e coletivos.

``Itaguaí é o meu universo.'' Essa afirmação do alienista Simão
Bacamarte e as estratégias do narrador em terceira pessoa de evocar
crônicas itaguaienses para atestar a verossimilhança de sua história,
bem como de estabelecer um paralelo de episódios locais, como o Terror e
a Revolta dos Canjicas, com as fases da Revolução Francesa, compõem uma
forma peculiar machadiana, de \emph{humour} crítico: ele nos
possibilita, jocosa e melancolicamente, ver semelhanças e desproporções
entre realidades brasileiras e estrangeiras, criando uma compreensão
quanto às iniquidades conservadas desde a formação colonial brasileira,
e quanto aos motores egoístas das ações.

Essa forma específica de reflexão também se manifesta em ``O Imortal'',
que chama a atenção por aproximar"-se do gênero do conto fantástico.
Aqui, o leitor encontrará a história de Rui de Leão, homem que viveu
mais de duzentos anos depois de tomar um elixir indígena. Esse
protagonista participa da Invasão Holandesa e da brutal aniquilação do
Quilombo dos Palmares, no Brasil, além de correr mundo e contribuir para
eventos como a Rebelião de Monmouth, na Inglaterra, e daí de volta ao
Brasil, no tráfico de escravos, na chegada da corte portuguesa em 1808 e
na Independência, em 1822. Mais uma vez, Machado chama o leitor, de
forma galhofeira e soturna, a articular as contradições da realidade
brasileira a um contexto mais amplo.

Justamente ``A Cartomante'', conto presente também na antologia,
configura essa forma peculiar da ironia machadiana. Na aparência, o
enredo é bastante simples: Rita, casada com Vilela, tem um caso amoroso
com Camilo, amigo do marido. Os amantes se assustam com ameaças
recebidas por correspondência anônima, e, depois de muita tensão
crescente, o conto se conclui com uma tragédia final, só revelada no
último parágrafo. Em um dos ensaios contidos nesta edição, chamado
``Machado de Assis, tradutor de si mesmo'', o professor e crítico
Alcides Villaça afirma que o processo de composição desse conto, apoiado
frequentemente em simetrias, traduz ``desproporções como
equivalências'', distingue e dissolve valores, promovendo a
relativização de tudo, num processo que inclui ``o dialético e o
cristalizado'', de forma a deixar ao leitor a tarefa de perceber
``antagonismos reais'' e resistir às diluições. De maneira geral, no
conjunto da obra machadiana, o narrador, diante de múltiplas
possibilidades de ``traduções'', cria para si a ``estabilidade
estilística'' do lugar de observador, aparentemente imune a
contradições. Mas afinal ele abre ``o espaço político da ironia e da
análise lúcida'', por meio das quais se constitui como sujeito,
recusando"-se a ser traduzido pela ``perspectiva das coisas"-mesmas''.
Essas traduções, marcas em especial dos contos machadianos aqui
incluídos, constituem um processo crítico de criação concebido desde as
\emph{Memórias póstumas de Brás Cubas}.

Nos três textos de Machado de Assis que compõem esta antologia, o leitor
encontrará, portanto, exemplos marcantes da agudeza analítica do autor a
respeito do seu tempo, intrinsecamente associada à forma específica que
o consagrou, instrumento crítico e estético cujo alcance se faz notar
até hoje.


\chapter{Advertência}

Esta coletânea contém três narrativas de Machado de Assis, duas delas
publicadas na Revista \emph{A Estação} --- periódico voltado para o
público feminino que circulou no Rio de Janeiro entre 1879 e 1904 --- e
uma na \emph{Gazeta de Notícias}. Os textos foram rigorosamente
cotejados com as coletâneas em que apareceram após sua disseminação em
folhetim.

A novela ``O Alienista'' é introduzida por Jean Pierre Chauvin,
professor de \emph{Cultura e Literatura Brasileira} da Escola de
Comunicações e Artes e credenciado nos programas de pós-graduação
Estudos Comparados de Literaturas de Língua Portuguesa, na Faculdade de
Filosofia, Letras e Ciências Humanas (\textsc{usp}) e no programa de
pós-graduação em Letras da Escola de Filosofia, Letras e Ciências
Humanas (Unifesp). O conto ``O Imortal'' é apresentado por João Adolfo
Hansen\footnote{Agradeço a João Adolfo Hansen, que autorizou a
  reprodução de seu ensaio, publicado originalmente na revista
  \emph{Teresa} (Departamento de Letras Clássicas e Vernáculas,
  \textsc{fflch}/\textsc{usp}) em 2005; e a Alcides Villaça, que permitiu a reprodução de
  seu ensaio, publicado originalmente na revista \emph{Novos Estudos}
  (Cebrap), em 1998.}, docente aposentado da Faculdade de Filosofia,
Letras e Ciências Humanas da \textsc{usp} (onde colabora com o programa de
pós-graduação em \emph{Literatura Brasileira}). Atualmente, é professor
visitante do curso de pós em Letras da Unifesp, \emph{Campus} Guarulhos.
``A Cartomante'' é discutida por Alcides Villaça, que leciona
\emph{Literatura Brasileira} na Faculdade de Filosofia, Letras e
Ciências Humanas (\textsc{usp}) e é credenciado no programa de pós-graduação em
Literatura Brasileira.

Nesta edição, foi respeitada a pontuação empregada por Machado de Assis
por se acreditar que determinadas alterações, ainda que justificáveis da
perspectiva estritamente gramatical, poderiam modificar o entendimento
da prosódia e o uso estilístico da língua pelo escritor, em seu tempo e
lugar. Quanto à grafia, os ditongos ``ou'', que apareciam em ``dous'',
``cousa(s)'' e ``doudo(s)'' foram substituídos por ``oi''
{[}\emph{dois}, \emph{coisa}(s) e \emph{doido}(s){]}. Com vistas a
orientar a leitura fluente e estimular a fruição dos textos, as
narrativas de Machado contêm notas de rodapé que trazem dados históricos
e culturais. Quando julgado necessário, atualizam o significado de
algumas palavras empregadas no final do século \textsc{xix} --- hoje, em desuso.

Feitas essas breves considerações, esperamos que leitoras e leitores
possam extrair máximo proveito e deleite das páginas que seguem.

\bigskip

\hfill{}\emph{J.P.C.}

\part{O Alienista}

\chapter[Introdução, \emph{por Jean Pierre Chauvin}]{Introdução\subtitulo{De Maomé a Dom Pedro}}

\begin{flushright}
\textsc{jean pierre chauvin}
\end{flushright}

\epigraph{Amava muito fazer justiça com direito. E, assim como quem faz
correição, andava pelo Reino; e, visitada uma parte, não lhe esquecia de
ir ver a outra; em guisa que poucas vezes acabava um mês em cada lugar
de estada}{\textsc{fernão lopes}.\footnotemark}\footnotetext{Lopes, Fernão. \emph{Crónica de D.\,Pedro}. Lisboa: Portugália Editora Lisboa, 1967, p. 44.}

\epigraph{\emph{Dios los remedie --- dije el cura ---, y estemos a la mira:
veremos em lo que para esta máquina de disparates de tal caballero y de
tal escudero, que parece que los forjaron a los dos em una turquesa y
que las locuras del señor sin las necedades del criado no valian un
ardite}}{\textsc{miguel de cervantes saavedra}.\footnotemark}\footnotetext{``-- Que Deus os ajude --- disse o padre ---, e fiquemos nós de atalaia; veremos onde vai parar essa máquina de disparates de tal cavaleiro e qual escudeiro,
pois parece que os dois foram forjados num mesmo molde e que as
loucuras do senhor sem as necedades do criado não valeriam mealha''
(Cervantes Saavedra, Miguel de. \emph{O engenhoso cavaleiro D.\,Quixote
de La Mancha} --- 2º livro. 3ª ed.
Tradução: Sérgio Molina São Paulo: Editora 34, 2012, p. 64).}

\epigraph{\emph{There is not anything which contributes more to the reputation
of particular persons, or to the honour of a nation in general, than
erecting and endowing proper edifices for the reception of those who
labour under different kinds of distress}}{\textsc{jonathan swift}.\footnotemark}\footnotetext{``Não
  há nada que contribua mais para a reputação das pessoas particulares,
  ou para a honra de uma nação em geral, que erigir e aprovisionar
  edifícios próprios para a recepção daqueles que lidam com diferentes
  espécies de sofrimento'' (Swift, Jonathan. \emph{A serious and useful
  scheme to make an hospital of incurables}. Londres: J. Roberts, 1733,
  p. \textsc{I}. [trad. minha]}

Durante bom tempo, \emph{O Alienista} foi aceito pacificamente como mais
um dentre os duzentos e poucos contos de Joaquim Maria Machado de Assis
(1839--1908) --- ao lado de narrativas muito mais breves, objetivas e mais
simples, como ilustram os demais textos da própria coletânea em que
apareceu. É curioso que isso ainda aconteça, considerando-se a extensão
que o texto ocupa em livro (noventa páginas, na primeira edição de
\emph{Papéis Avulsos}, de 1882, que somava trezentas e quinze). A
questão pode parecer de somenos importância, mas não seria impossível
supor que Machado tenha recorrido a uma narrativa com extensão maior que
a habitual para sugerir um gênero intermediário entre o conto e o
romance.

Quer dizer, \emph{O Alienista} poderia ser aproximado do gênero novela,
como se se tratasse de um experimento ficcional que transparecesse na
forma em que foi estruturado. Hipótese não destituída de toda razão, em
vista de um enredo cheio de peripécias, respaldado na controversa figura
do Dr.\,Simão Bacamarte: legítimo e ambíguo representante da coroa
portuguesa, na modesta vila. Na síntese de John Gledson:

\begin{quote}
\emph{O Alienista} não é um conto, e é difícil dizer a que gênero
pertence. Como o Cândido de Voltaire, que talvez seja seu parente mais
chegado, é um ataque às ingenuidades filosóficas do momento. Junto com o
humanitismo de \emph{Memórias Póstumas} e \emph{Quincas Borba}, é uma
investida devastadora contra o ``bando de ideias novas'', contra o
cientificismo.\footnote{Gledson, John. ``\emph{Papéis Avulsos}: um livro
  brasileiro?'' In: \textsc{machado de assis}. \emph{Papéis Avulsos}. São Paulo:
  Penguin Classics/Companhia das Letras, 2011, p. 18.}
\end{quote}

Hélcio Martins, José Aderaldo Castello e muitos outros constataram que a
ambivalência é um dos fundamentos do texto machadiano. O retrato que faz
das personagens é reforçado pelo modo desajeitado como elas se
comportam, o teor dicotômico do que pensam e o estilo indeciso de sua
fala: ``Desse repúdio às afirmações categóricas nasce o que o próprio
Machado de Assis chamou seu estilo ébrio, que vai guinando à direita e à
esquerda, andando e parando, ao passo que o leitor deseja a `narração
direita e nutrida, o estilo regular e fluente'\footnote{Afirmação do
  defunto/autor Brás Cubas, nas \emph{Memórias Póstumas}.}.''\footnote{Martins,
  Hélcio. ``A litotes em Machado de Assis''. In: \_\_\_\_\_. \emph{A
  rima na poesia de Carlos Drummond de Andrade} e \emph{outros ensaios}.
  Rio de Janeiro: Topbooks, 2005, p. 318.}

Outro aspecto a ser relembrado é que a novela circulou em folhetim,
antes de ser transportada como ``conto'' de abertura ao livro. Aliás,
``todos os contos de \emph{Papéis Avulsos} apareceram em jornais e
revistas antes da publicação em volume'', lembra John Gledson (2011, p.
33). No caso de \emph{O Alienista}, as partes saíram aos pedaços, na
revista \emph{A Estação}\footnote{``\emph{A Estação} foi fundada em 1872
  por Henri Gustave Lombaerts (1845--1897) com o nome de \emph{La Saison
  -- Jornal de Modas Parisienses}. Abaixo do título, na primeira página,
  vinha a informação de que era `Dedicado às Senhoras Brasileiras'.
  Publicava-se no dia 15 e no dia 30 de cada mês. Em seus primeiros oito
  anos, o jornal limitava-se a traduzir matérias alemãs e francesas. Em
  15 de janeiro de 1879, foi criada uma `Parte Literária', redigida no
  Brasil. A partir daí, o periódico passou a se chamar \emph{A Estação
  -- Jornal Ilustrado para a Família}. Circulou normalmente até 1904''
  (Teixeira, Ivan. \emph{O Altar \& o trono: dinâmica do poder em O
  Alienista}. Cotia: Ateliê; Campinas: Editora Unicamp, 2010, p. 47).},
entre outubro de 1881 e março de 1882.

Por que afirmo que se trata de uma novela, ao lado de \emph{Casa Velha}?
Porque, embora o fio condutor do enredo seja a prisão ou a liberdade dos
itaguaienses --- considerados mais ou menos loucos pelo extravagante
cientista ---, \emph{O Alienista} reserva espaço para narrativas
paralelas, dentre as quais a viagem (e fausta recepção) da comitiva
local, que viajara para o Rio de Janeiro; a disputa do poder entre os
barbeiros Porfírio Caetano das Neves e João Pina; as mudanças de posição
do vereador Sebastião Freitas --- figura que lembra em muito certos tipos
da política brasileira vigente, a oscilar ao (des) sabor dos recentes
golpes.

Dito isso, passemos a diferente matéria. Comecemos por uma breve
apreciação do enredo. No final do século \textsc{xviii}, um brasileiro erudito,
com largos conhecimentos em medicina, ciência, filosofia e religião,
recusa o magnânimo convite do rei Dom João \textsc{v}, para ocupar altos postos
na administração portuguesa. Em vez disso, mete-se na vila de Itaguaí e,
com o apoio da câmara de vereadores, manda construir um faraônico asilo
para loucos, a que dá o nome de Casa Verde.

Embora a novela tenha sido escrita entre 1881 e 1882, Machado a recuou
para um século antes, quando a cultura e os modos traziam embutidos o
acento político português. A operação não era tão simples. Ao ambientar
a narrativa no final do século \textsc{xviii}, o escritor reconstituiu protocolos
da época. Os capítulos parodiam as crônicas produzidas durante o Antigo
Regime. O narrador estiliza a maneira formal de relatar acontecimentos
(daí as constantes referências a supostas ``crônicas da vila de
Itaguaí''). Mesmo as andanças de Simão Bacamarte pela Europa, antes de
retornar ao Estado do Brasil, aludem a um procedimento comum ao
Setecentos luso-brasileiro, que respaldava a carreira dos profissionais
responsáveis pela saúde. De acordo com Daniela Buono Calainho:

\begin{quote}
O que marcou a figura do médico no Antigo Regime foi sua formação
erudita, acadêmica instrumentalizando-o na sua prática clínica e
terapêutica, distinguindo-se de outras categorias: os médicos usavam as
suas credenciais de cavalheiros eruditos para se separar e distinguir
dos que estavam mais abaixo na hierarquia social, como os cirurgiões,
barbeiros, boticários ou as parteiras.\footnote{Calainho, Daniela Buono.
  ``Curas e hierarquias sociais no mundo luso-brasileiro do século
  \textsc{xviii}''. In: \textsc{monteiro}, Rodrigo Bentes et al (orgs.). \emph{Raízes do
  privilégio: hierarquias sociais no mundo ibérico do Antigo Regime.}
  Rio de Janeiro: Civilização Brasileira, 2011, p. 486.}
\end{quote}

Diversos poderes entram em imprevista consonância, na novela: o poder
político, o poder do saber, o poder da psiquiatria, o poder marital, o
poder da psiquiatria, o poder da ciência, o poder personificado pelo
médico. Não se esqueça que Simão Bacamarte contou, desde sempre, com o
aval da Câmara de Vereadores da vila de Itaguaí. Ao chamar a atenção
para as alianças escusas entre medicina e política, Machado de Assis
resgatava um dos traços marcantes da colonização portuguesa no Estado do
Brasil, que vigorou entre os séculos \textsc{xvi} e \textsc{xviii}. Como descreve Joaquim
Romero Magalhães:

\begin{quote}
Entre as estruturas profundas da América de colonização portuguesa uma
há que suponho de excepcional relevância: a organização municipal. Ao
criar, em 1532, a vila de São Vicente segundo a legislação ordinária do
Reino fixada nas Ordenações, o rei de Portugal estendia ao Brasil o
regime judicial e administrativo em vigor no espaço português da Europa
e Ilhas do Atlântico. Aí assento e se foi alargando além-mar a
instituição concelhia que viera a tomar forma até finais do século \textsc{xiv} e
que só o liberalismo derrubaria.\footnote{Magalhães, Joaquim Romero.
  \emph{Concelhos e organização municipal na época moderna}. Coimbra:
  Imprensa da Universidade de Coimbra, 2011, p. 121.}
\end{quote}

Do poder político para o poder da palavra. Eugênio Gomes viu, em \emph{O
Alienista}, forte remissão a um texto satírico, em que Jonathan Swift
``sugeria a criação de um hospital para incuráveis, incuráveis morais:
os escrevinhadores, os vadios, os incréus, os mentirosos e, além de
tantos outros, os que fossem incuravelmente vaidosos, fátuos e
impertinentes''\footnote{Gomes, Eugênio. ``Swift''. In: \_\_\_\_\_.
  \emph{Machado de Assis: influências inglesas}. Rio de Janeiro: Pallas;
  Brasília: \textsc{inl}, 1976, p. 40.}. Publicada um século e meio após o ensaio
do escritor irlandês, a novela machadiana parte de premissa similar.
Desde o princípio, o projeto de construção da Casa Verde patenteava a
desproporção entre a reduzida quantidade dos dementes que circulavam na
vila de Itaguaí e o custo social e financeiro envolvidos em sua
manutenção.

A dicção empregada por Swift irmana-se ao tom zombeteiro do narrador de
\emph{O Alienista}, que, em linguagem elegante e castiça, parodia o
vocabulário técnico e o modo como médicos e filósofos se expressavam em
tratados escritos a sério. ``Frio como um diagnóstico'', o cientista
brasileiro tende a ampliar seu repertório (muito particular) para
(re)definir as manias com que se depara. Poder-se-ia aplicar à novela o
que Dirce Côrtes Riedel afirmou sobre o conto ``O empréstimo'': ``O
próprio \emph{kitsch} é usado pelo narrador metaforicamente, para
configurar um personagem que vive das aparências, do efeito fácil
produzido sobre os outros. {[}\ldots{}{]} Nem falta ao narrador o
\emph{gran finale} de mau gosto''\footnote{Riedel, Dirce Côrtes.
  \emph{Metáfora, o espelho de Machado de Assis}. Rio de Janeiro:
  Francisco Alves, 1974, pp. 35--36.}. Na leitura de outro conto, ``O
imortal'', João Adolfo Hansen nota que a persona do narrador
``{[}\ldots{}{]} estiliza elementos microtextuais, como o léxico antigo:
o termo `aleivosia' é divertidamente típico. E frases inteiras, que é
impossível ler sem sorrir de cumplicidade, pois o \emph{kitsch} não é de
Machado de Assis''\footnote{Cf. Hansen, João Adolfo. \emph{``O Imortal''
  e a verossimilhança}, nesta edição.}.

Em paralelo com a desconstrução da linguagem desmesurada e postiça, a
convenção social é outro componente observado pelas criaturas
machadianas. Aparentemente reto e virtuoso --- como convém aos sábios ---,
Simão Bacamarte cumpre todos os protocolos de um sujeito bem posicionado
na ramificada hierarquia reinol. No plano pessoal, o douto cientista é
casado com Dona Evarista --- uma mulher sem sal, frívola e subserviente,
que parece ter sido escolhida segundo critérios puramente racionais. Ele
também conta com um grande aliado, o boticário Crispim Soares --- que
vive a secundar as falas do ``ilustre médico'' com um discurso
francamente bajulatório. À medida que Bacamarte investiga os casos de
fúria, depressão e mania, a quantidade de pacientes da ``Casa de
Orates'' aumenta exponencialmente, a ponto de abrigar ``quatro quintos''
da população de Itaguaí e arredores.

Porém, alheio às pressões da câmara, à grita dos populares e às lágrimas
da consorte, Simão Bacamarte segue o seu roteiro vigorosamente,
cumprindo-o à risca. Se na primeira fase das investigações sobre
patologias mentais, ele supunha que a demência fosse uma exceção em meio
à normalidade, no segundo estágio o médico inverte o critério e passa a
encarcerar justamente aqueles que demonstrassem comportamento mais
coerente consigo mesmos e os outros.

Algumas dentre as páginas mais fascinantes está nos capítulos finais.
Bacamarte recorre a diversos expedientes para (des)curar a já reduzida
população de seu asilo. Aqui, a mistura de vozes --- de Machado, do
narrador e das personagens --- relembra uma das características de sua
ficção. José Leme Lopes percebeu que ``Ao descrever os delírios dos
recolhidos à Casa Verde, notamos que o escritor parece caricaturar com
finura as observações famosas dos tratados contemporâneos sobre as
doenças mentais''\footnote{Lopes, José Leme. ``A propósito de \emph{O
  Alienista}''. In: \_\_\_\_\_. \emph{A psiquiatria de Machado de
  Assis}. 2\textsuperscript{a} ed. Rio de Janeiro: Agir, 1981, p. 27.}.

Dois internados, acometidos de rematada modéstia, são submetidos à cura
pelo alarido narcisista. Em um caso, o médico recorre à divulgação do
nome do paciente --- como se se tratasse de notável talento poético ---,
através da matraca; em outro, obtém autorização régia para fazer
funcionar uma agremiação literária (que lembra em muito algumas
Academias de vida efêmera que existiram no país, enquanto colônia de
Portugal).

Embora haja constantes alusões ao vice-rei Luís de Vasconcelos e Sousa
(evocado pela viagem da comitiva, ao Rio de Janeiro, deslumbrada com o
Chafariz das Marrecas, construído durante o seu mandato), o narrador
interpõe dados de época posterior no enredo. Ora remete ao profeta
Maomé, que viveu no século \textsc{vii}; ora avança até o primeiro reinado, com
paródias aos gestos de Dom Pedro \textsc{i}.

Repare-se que o ``Dia do Fico'' foi recuado cronologicamente, como se
``antecipasse'' o episódio em que Simão Bacamarte resiste aos revoltosos
que esperneavam e se espremiam em frente a sua residência. Ao tomar a
câmara de assalto, o barbeiro Porfírio assina uma carta dirigida aos
itaguaienses, como ``Protetor da vila'' --- em alusão ao título concedido
ao Imperador Pedro I, como ``Defensor Perpétuo do Brasil''\footnote{Cf.
  Monteiro, Tobias. \emph{História do Império -- O Primeiro Reinado}. 2
  vol. Belo Horizonte: Itatiaia, 1982; Viana, Hélio. \emph{História do
  Brasil: período colonial, monarquia e república}. 15ª ed. São Paulo:
  Melhoramentos, 1994; Lustosa, Isabel. \emph{D.\,Pedro \textsc{i}: um herói sem
  nenhum caráter}. São Paulo: Companhia das Letras, 2006.}, outorgado no
ano de 1822 (ou seja, trinta anos após os episódios transcorridos em
Itaguaí). De um lado, a novela remete ao tempo em que éramos colônia de
Portugal, situada no final do século \textsc{xviii}; de outro, a postura do
barbeiro Porfírio alude ao turbulento processo de Independência, com
todos os seus reveses e sob o jugo da Inglaterra.

Outro fator histórico. Sabe-se que o ano de 1822 foi marcado pela
assinatura de cartas públicas e manifestos por Dom Pedro \textsc{i}, através da
imprensa oficial. A carta de Porfírio Caetano das Neves pode ser lida
como paródia do discurso empolado e cheio de vênias que o protocolo real
exigia, ao escrever, especialmente quando se dirigisse ao povo:

\begin{quote}
O Congresso de Lisboa arrogando-se o direito tirânico de impor ao Brasil
um artigo de nova crença, firmado em um juramento parcial, e
promissório, e que de nenhum modo poderia envolver a aprovação da
própria ruína {[}\ldots{}{]}. Brasileiros! Para vós não é preciso
recordar todos os males, a que estáveis sujeitos, e que vos impeliram à
Representação, que me fez a Câmara, e o Povo desta cidade no dia 23 de
Maio, o que motivou meu Real Decreto de 3 de Junho do corrente ano
{[}\ldots{}{]}. A Minha Felicidade (convencei-vos) existe na vossa
felicidade: é Minha Glória Reger um povo brioso, e livre. Dai-me o
exemplo das Vossas Virtudes e da Vossa União. Serei Digno de vós.
Palácio do Rio de Janeiro em o primeiro d'Agosto de 1822.\footnote{Pedro
  \textsc{i}, Dom {[}Príncipe Regente{]}. \emph{Manifesto de S.\,A.\,R. O Príncipe
  Regente Constitucional e Defensor Perpétuo do Reino do Brasil aos
  Povos deste Reino}. Rio de Janeiro: Imprensa Nacional, 1822, pp. 1--4.}
\end{quote}

O narrador, aderente às múltiplas tonalidades das figuras, reproduz até
mesmo a intercalação e a repetição de palavras, no discurso do barbeiro,
quando deixa a infrutífera entrevista com o médico:

\begin{quote}
--- \ldots{}porque \emph{eu velo}, podeis estar certos disso, \emph{eu
velo} pela execução das vontades do povo. Confiai em mim; e tudo se fará
pela melhor maneira. \emph{Só vos recomendo ordem}. \emph{A ordem}, meus
amigos, é a base do governo\ldots{}

--- Viva o ilustre Porfírio! Bradaram as trinta vozes, agitando os
chapéus {[}grifos meus{]}.
\end{quote}

Bacamarte sedimenta o saber desinteressado da ciência, a postura
dogmática e a síndrome do poder --- concedido pela administração
portuguesa. Com um quê de pesquisador, religioso e político, desde seu
retorno ao Brasil ele age em relativa consonância com a câmara e obtém
tudo de que necessita para implementar a morada dos loucos e tratá-los
em acordo com a sua lógica delirante. Kátia Muricy detectou:

\begin{quote}
{[}\ldots{}{]} no humor de \emph{O Alienista} uma crítica perspicaz às
intenções controladoras da nascente psiquiatria em relação à população,
bem como uma compreensão exata das alianças recíprocas entre ela e o
poder político. Mas é especialmente na ironia à positividade
experimental, aos altos ideais humanitários do saber psiquiátrico e à
sua suposta vinculação com os princípios universais da razão --- vínculo
que legitimava, no discurso médico, a intervenção da psiquiatria social
--- que a narrativa ganha sua inteligência mais requintada.\footnote{Muricy,
  Kátia. ``As desventuras da razão''. In: \_\_\_\_\_. \emph{A razão
  cética: Machado de Assis e as questões de seu tempo}. São Paulo:
  Companhia das Letras, 1988, p. 36.}
\end{quote}

Conhecedor dos meandros que ligam a extensa cadeia administrativa do
reino, age de modo mais cauteloso em relação ao padre Lopes e reserva
para sua esposa as migalhas que sobram de sua intensa dedicação à
ciência e à ambição de servir a humanidade. Altruísmo ou megalomania?
Para Alfredo Bosi:

\begin{quote}
Ele {[}Simão Bacamarte{]} \emph{pode} executar os projetos da ciência
que o obseda. Seu status de nobre e portador do valimento régio
transforma-o em ditador da pobre vila de Itaguaí. A população sofre os
efeitos de um terrorismo do prestígio de que as relações entre médico e
doente, psiquiatra e louco, são apenas casos particulares. O eixo da
novela será, portanto, o arbítrio do poder antes de ser o capricho de um
cientista de olho metálico.\footnote{Bosi, Alfredo. \emph{Machado de
  Assis -- o enigma do olhar}. São Paulo: Ática, 1999, p. 89 {[}grifo do
  autor{]}.}
\end{quote}

A novela é orientada por tópicos recorrentes: o saber do médico; o poder
descomunal de que ele está investido; a sua incomum relação matrimonial
com Dona Evarista\footnote{``{[}\ldots{}{]} o fundamento inicialmente
  reconhecido pelo romancista é o amor como aspiração selecionada da
  existência humana; a segunda é alimentada pelas ambições do poder. Mas
  pode haver uma terceira, determinada pela procura de fusão daquelas
  `duas forças principais da terra'*, o amor e o poder, ou da
  sobreposição da segunda à primeira, equilibrando-se ou gerando
  conflitos, no mundo machadiano. A conciliação, porém, é muito pouco
  frequente'' (Castello, José Aderaldo. \emph{Realidade e ilusão em
  Machado de Assis}. São Paulo: Companhia Editora Nacional; Edusp, 1969,
  p. 63). *Crônica publicada por Machado na coluna ``A Semana'' (jornal
  \emph{Gazeta de Notícias}), em 26 de junho de 1892.}; as assimetrias
sociais, cravadas no asilo, na câmara e na barbearia; as terapêuticas
pouco ortodoxas, aplicadas pelo cientista. Diante dessa profusão de
referências à saúde mental, religião, filosofia, história e política, o
leitor não fica imune aos métodos do cientista. Ele duvida dos bons (e
maus) propósitos do médico, tendo em vista os critérios --- nem sempre
objetivos --- a que ele recorre.

De fato, Bacamarte comporta-se de modo tão obstinado que a ciência
assume ares de doutrina, e a terapêutica beira a charlatanice. A maneira
como atesta o acerto de suas hipóteses beira o dogmatismo. Entretanto,
seria um equívoco supor que se trata de um tirano qualquer --- como
proclama a turba dos revoltosos, liderados pelo barbeiro Porfírio. Os
paralelos com a Revolução Francesa são tão evidentes que precisamos
examiná-los mais de perto:

\begin{quote}
Observe-se antes de mais nada que os títulos dos capítulos reproduzem as
grandes articulações da Revolução Francesa. O capítulo 5 é o Terror; o
capítulo 6 é a Rebelião; o capítulo 10 é a Restauração. Mas a cronologia
está deslocada. Em Paris, a sequência foi a Rebelião (a revolta popular
que culminou com a tomada da Bastilha); o Terror (o banho de sangue
decretado por Robespierre e Marat); e a Restauração (o retorno dos
Bourbon). {[}\ldots{}{]} A inversão temporal pode ser uma forma sutil de
aludir ao caráter reativo, reflexo, dos movimentos populares no Brasil,
em contraste com a Europa, onde o povo tem um protagonismo originário, e
não derivado.\footnote{Rouanet, Sérgio Paulo. ``Machado de Assis e o
  mundo às avessas''. In: \_\_\_\_\_ et al. \emph{Machado de Assis:
  cinco contos comentados}. Rio de Janeiro: Edições Casa de Rui Barbosa,
  2008, p. 83.}
\end{quote}

Para enfrentar essa questão, sugiro que procedamos como o douto
cientista faria. Em lugar de nos atermos meramente às disputas pelo
poder e/ou às arbitrariedades implementadas pelo médico, examinemos um
dos recantos responsáveis pelo êxito da narrativa. Em vez de nos
restringirmos aos sintomas de sabedoria/loucura, manifestados por
Bacamarte e pelos itaguaienses, desviemos nossa atenção para a voz que
narra. Uma das melhores sínteses sobre o estilo machadiano foi sugerida
por Alcides Maya, em 1912:

\begin{quote}
Um tristonho chancear é o processo permanente de Machado de Assis, que
se deleita em revelar o ridículo ora numa paráfrase mordaz, ora numa
redução folgazã da natureza e quase sempre na realidade individual,
falha e má, indiretamente estudada, com profundez ímpia, sob aparências
de sensatez, de virtude e dever. Daí o ser classificado com razão entre
os grandes humoristas.\footnote{Maya, Alcides. \emph{Machado de Assis --
  algumas notas sobre o humour}. 3ª ed. Porto Alegre: Movimento; Santa
  Maria: Editora \textsc{ufsm}, 2007, p. 47.}
\end{quote}

Em que pese a feição moralista do crítico, podemos aproveitar muito do
que ele diz. É ao narrador (talvez mais do que às personagens) que
podemos atribuir a sensação de que tudo foi colocado em questão. O seu
relato é impreciso, embora respalde-se nos cronistas de
Itaguaí.\footnote{``O tom da narrativa é dado pelo truque de imaginar
  que as peripécias do conto se fundamentam na verdade das crônicas: `As
  crônicas da vila de Itaguaí dizem que em tempos remotos\ldots{}'. E que,
  portanto, ao narrador cabia a função de trazer à luz um episódio
  histórico soterrado no pó dos manuscritos. É nessa atmosfera de
  aparente respeito à letra dos documentos que encontramos logo às
  primeiras linhas a chave para lhes interpretar o conteúdo''
  (Moisés, Massaud. ``\emph{O
  Alienista}: paródia do \emph{Dom Quixote}?''. \emph{Machado de Assis:
  ficção e utopia}. São Paulo: Cultrix, 2001, p. 128).} Os vereadores
são referidos como homens ``dignos'', embora a quase totalidade deles
defenda a manutenção de seus privilégios, dentre os quais a
exclusividade de serem imunes aos diagnósticos do médico --- amparados
pelo braço curto da lei. A semelhança com determinados procedimentos
adotados pelo Judiciário, nos últimos anos, dá maior consistência a essa
blindagem, no plano da ficção.

O poder supremo não reside exclusivamente na erudição do médico;
tampouco na proteção do reino, menos ainda na chancela provisória
concedida pela vereança. Ele passa pelo crivo do narrador, que nos leva
e traz, puxa e arremete, também com o intuito de desestabilizar a
leitura e alargar os seus efeitos para além do entretenimento. Sua
dicção, ora vaga, ora assertiva, confunde-se com a voz autoral. Autor e
narrador parecem nos aconselhar a não levar a literatura tão a sério,
embora dramatizem os abusos praticados pelas autoridades da vila.

\emph{O Alienista} dispõe de um desses enredos que, em meio ao tom
jocoso, deixa-nos adivinhar que os episódios transcorridos em Itaguaí
não diferem grandemente de outros eventos, situados para além da
historiografia (aproximada da ficção), fosse em Vila Rica --- enquanto
durou a violenta perseguição e castigo dos conjurados mineiros ---; fosse
em Paris, quando o terceiro estado pegou em armas contra a realeza e seu
sistema de privilégios, perpetuado havia séculos, na França.

Uma das consequências é que a angústia vivenciada pela população da vila
contagia o leitor. Nada mais coerente, em termos estilísticos, tendo-se
em vista que estavam diante de um mundo em miniatura não só invertido,
mas sujeito à constante reversão. Como disse, o exame sobre o cientista
pode situar-se na linguagem empregada pelo narrador e demais
personagens. Maria Nazaré Lins Soares disse-o com acurácia: ``Um abismo
parece mediar entre a magra realidade e a linguagem hiperbólica de que
fazem uso os personagens para exprimi-la''.\footnote{Soares, Maria
  Nazaré Lins. \emph{Machado de Assis e a análise da expressão}. Rio de
  Janeiro: \textsc{inl}, 1968, p. 13.}

Para além das paródias a eventos sangrentos da história, e as constantes
alusões a celebridades do vice-reinado e do primeiro império, o narrador
recorre à \emph{Bíblia} e ao \emph{Corão}, em diversas ocasiões, a
sugerir que Simão Bacamarte seria um daqueles sábios autênticos em
extinção, já que bebe, em quantidade, de todas as fontes. A esse
respeito, assinalem-se as várias palavras de origem árabe e o possível
paralelo entre a trajetória do médico e de Maomé --- o que inclui as
esposas com que se casaram\footnote{``A alusão à cultura árabe é fato
  merecedor de atenção. {[}\ldots{}{]} o profeta tinha 25 anos, ao se
  casar com uma viúva rica, 15 anos mais velha que ele. Já a personagem
  {[}Bacamarte{]} é financeiramente estabelecida, tem 40 anos, e sua
  esposa, Evarista, tem 25. {[}\ldots{}{]} {[}Maomé{]} foi acusado de
  demente, ao longo da vida, em razão da maneira pela qual, embora
  semianalfabeto, fizesse referências escritas a Alá {[}\ldots{}{]}
  Maomé enfrentou violeta oposição de grande parcela da população de
  Meca, em sua jornada'' (Chauvin, Jean Pierre. O Alienista: \emph{a
  teoria dos contrastes em Machado de Assis}. São Paulo: Reis Editorial,
  2005, pp. 88--9).}. A alusão a Averróis, outro nome fundamental da
filosofia, também pode ser vista como índice de que o pensamento árabe
se mantém, feito ideia fixa, no horizonte do médico. Isso, sem contar as
ações que parodiam certos atos de Dom Pedro \textsc{i}, considerado louco por uma
parcela de seus biógrafos, como refere Tobias Monteiro.

Detentor de múltiplos saberes, Bacamarte oscila da frieza à
impassibilidade. De um lado, pesquisa e aplica técnicas estritas do
campo médico; de outro, escuta e fala sem demonstrar aparente
perturbação, obedecendo ao rito recomendado pelos filósofos gregos da
Antiguidade. Simão Bacamarte resulta de um misto de cálculo e
\emph{ataraxia}. Como percebeu Benedito Nunes, ``Em Machado, essa
transposição parodística se faz como um artifício de teorização apoiado
numa \emph{imitatio} da exposição racional, argumentativa, ou do
comentário de realce erudito, com o aparato das citações de
autoridade''\footnote{Nunes, Benedito. ``Machado de Assis e a
  Filosofia''. In: \_\_\_\_\_. \emph{No tempo do niilismo e outros
  ensaios}. São Paulo: Ática, 1993, p. 136.}.

Essa aparente tranquilidade do cientista, filósofo e médico contribui
para que o leitor até simpatize com ele e, porventura, aceite a maior
parte dos argumentos que ele sustenta com tamanha segurança e altivez.
Temos que convir que ele não parece ser mais um oportunista, nos moldes
do que viria a ser o Cristiano Palha, de \emph{Quincas Borba}, nem um
bajulador de ofício --- como será José Dias, agregado da família
Santiago, em \emph{Dom Casmurro.} Para Maria Nazaré Lins Soares, a
linguagem utilizada pelas personagens guardava íntima relação com o seu
\emph{ethos}:

\begin{quote}
Para merecer alguma consideração não só aos olhos dos outros, mas aos
seus próprios, busca compensar sua situação social esdrúxula,
apresentando-se sempre impecável na sua roupa engomada e buscando com
superlativos dar pernas compridas a suas ideias chochas. José Dias é
pusilânime, interesseiro e falso como a linguagem que usa.\footnote{Soares,
  Maria Nazaré Lins. \emph{Machado de Assis e a análise da expressão},
  ed. cit., 1968, p. 7.}
\end{quote}

Não será preciso grande esforço para constatar que o boticário Crispim
Soares é um parente mais novo de José Dias. Ao agir de modo bajulatório,
quase emudecido pela verborrágica genialidade do médico, Crispim anula a
sua personalidade. A cada mudança na teoria formulada pelo médico, o
boticário ressente o dilema de se manter fiel ao amo, ainda que isso
implique aderir a uma e outra ideia, ao sabor dos humores do médico. A
caracterização de José Dias ajusta-se quase inteiramente ao perfil de
Crispim Soares:

\begin{quote}
E porque o boticário se admirasse de uma tal promiscuidade, o alienista
disse-lhe que era tudo a mesma coisa, e até acrescentou
sentenciosamente:

--- A ferocidade, Sr.\,Soares, é o grotesco a sério.

--- Gracioso, muito gracioso! --- exclamou Crispim Soares levantando as
mãos ao céu.
\end{quote}

Os contrastes do boticário com o médico são evidentes. A despeito de sua
impassibilidade pseudofilosófica, o que não falta a Simão Bacamarte é
uma personalidade enérgica que se coaduna com o seu proceder intrépido e
modos distintos. O narrador (sempre ele) sugere essa conduta em diversos
momentos, particularmente quando médico e boticário regressam, com
ânimos opostos, após a partida da comissão que fora para o Rio de
Janeiro. De um lado, o representante típico do ``vulgo'', a padecer de
saudade, como acontece aos medíocres; de outro, a vivacidade do médico,
imperturbável diante de qualquer contrariedade. O que num é paixão e
adversidade, no outro é ciência e planejamento.

Simão Bacamarte combate três rivais em potencial: a minoria da vereança
de Itaguaí, o padre Lopes e os barbeiros (Porfírio e João Pina),
apoiados por uma parte do povo local. Apesar da eventual disputa de
poder, em nenhuma ocasião a câmara representou ameaça aos desígnios
interpostos pelo cientista. Admirado pelo rei Dom João \textsc{v}, Bacamarte
recorreu à tribuna para fazer valer a sua força e autoridade. Por sua
vez, o padre Lopes ocupa lugar intermediário: habituado ao púlpito, fez
uso da palavra para se contrapor aos pensamentos e atitudes que ocorriam
ao ilustre pesquisador. Ao agir desse modo, ele reproduz a postura do
padre concebido por Cervantes em \emph{Dom Quixote}: ``-- Quanto a mim,
tornou o vigário, só se pode explicar pela confusão das línguas na torre
de Babel, segundo nos conta a Escritura; provavelmente, confundidas
antigamente as línguas, é fácil trocá-las agora, desde que a razão não
trabalhe\ldots{}''.

A maior demonstração de força não reside nos vereadores, nem no vigário;
ela parte dos populares. Aquilo que a política e a religião não venceram
transformou-se em alvo primário dos Canjicas --- denominação que satiriza
os apelidos dados aos revolucionários, no âmbito da história: dos
\emph{sans-culotte} a Tiradentes. Nessa confluência de saber, discurso e
força, pairam algumas dúvidas. Teria o alienista agido como agiu
exclusivamente em função do poder desmedido de que supunha estar
investido? Ou a sua postura mais firme devia-se à convicção inabalável
de que poderia prestar o melhor serviço ao povo local --- como se os
experimentos na Casa Verde fossem balões de ensaio para a humanidade?

Se aceitarmos a investigação psiquiátrica como um gesto de vaidade de
Bacamarte, a sua conduta lembra em muito a do herdeiro (e ocioso) Brás
Cubas, que padeceu da ideia fixa de ver seu nome irradiado através dos
tempos, no emplastro com que pretendia curar a ``melancólica
humanidade''. O que um e outro têm em comum? A concepção totalizante do
universo, a partir do seu umbigo.

O impacto de \emph{O Alienista} não repousa apenas no teor mirabolante
criado por Machado. Tão importante quanto o enredo inventado é o fato de
ser uma narrativa ágil, o que não impede ao narrador semear digressões
--- como se estivesse a ridicularizar o próprio relato das desventuras
sucedidas na vila. Nos lances mais risíveis, escutamos a sua voz a
referendar (ou a alimentar a imprecisão) (n)o que conta, com base nos
cronistas de Itaguaí. Ivan Teixeira percebeu, como poucos, que:

\begin{quote}
{[}\ldots{}{]} a voz narrativa parece muito clara, mas é de fato um
tanto sinuosa, visto produzir uma gradação que vai da opinião ao boato e
do boato ao boato duvidoso. {[}\ldots{}{]} o texto procura a clareza por
meio da insinuação, como que sugerindo que a gravidade da matéria
desaconselha a objetividade. Daí a condição algo paradoxal de sua
clareza relativa, mas que é prevista pela tradição dos procedimentos
retóricos. O texto apresenta também uma espécie de mistura entre o
disparate e o solene, que lembra certa visualidade amaneirada de gestos
e feições, próximos do desenho caricatural.\footnote{Teixeira, Ivan.
  \emph{O altar \& o trono: dinâmica do poder em} O Alienista, ed. cit.,
  2010, pp. 265--267.}
\end{quote}

Como reparou Augusto Meyer: ``\emph{O Alienista} é um espanto. O feitio
objetivo do entrecho, o tempo lento da narração, a contenção da ironia
sem malabarismos inúteis, a serenidade superior, a graça irresistível,
mas apagada e modesta --- tudo concorre para dar ao leitor, por
contraste, uma impressão de espantosa vertigem''\footnote{Meyer,
  Augusto. ``Na Casa Verde''. In: \_\_\_\_\_. \emph{Machado de Assis}.
  Rio de Janeiro: Simões, 1952, p. 60.}.

Embora a crônica seja considerada uma fonte historiográfica, o narrador
distribui incertezas em seu relato, a sugerir que ficção e
historiografia\footnote{``Há passagens no conto {[}\emph{O Alienista}{]}
  que fazem pensar numa clarividência histórica assombrosa em Machado. É
  o caso daquela atitude da Câmara Municipal de Itaguaí, aprovada por
  instigação do cientista, que autorizava o uso de um anel de prata no
  polegar da mão esquerda a todo habitante que declarasse ter sangue
  godo nas veias. Ora, O Alienista é contemporâneo da teorização racista
  de Chamberlain, Gobineau e certos antropólogos alemães. {[}\ldots{}{]}
  O espantoso, realmente, é a sensibilidade com que Machado percebeu
  aonde essa preocupação `científica' poderia levar'' (Schnaiderman,
  Boris. ``\emph{O Alienista}, um conto dostoievskiano?'' \emph{Revista
  Teresa}, n. 6/7, 2006, p. 272).} se equivalem, já que não passam de um
recorte ambivalente da realidade, disposto de maneira não a convencer o
leitor de uma pretensa veracidade, mas a relembrar que lhe cabe o papel
maior de se divertir, enquanto reflete sobre os episódios. Por sinal, o
tom grandiloquente expresso pelo narrador e pelas personagens parece
escorar-se no próprio gênero, estilizado pelo narrador: espécie de
cronista de segunda linha.

A novela parodia gêneros discursivos, retira a credibilidade do registro
historiográfico e disseca a expressão calculada e artificial das
personagens. O furor taxonômico do cientista, como apontado por José
Leme Lopes (1981) e Pierre Brunel\footnote{``Machado de Assis, assim
  como vários escritores do século \textsc{xix}, escolheu uma terceira via, a da
  narrativa. Parece-nos que ela permite manter distância, sem
  {[}possuir{]} a frieza do discurso científico'' (Brunel, Pierre.
  ``Itaguaï, ou le grand théâtre du monde''. In: Machado de Assis.
  \emph{L'Aliéniste}. Paris: Éditions Métailié, 1984, p. 9. [trad. minha]} (1984), não deveria enganar o leitor. Um dos maiores triunfos
da novela reside no modo como Simão Bacamarte e narrador medem suas
palavras, provocando a impressão de que elas fluem com máximo cuidado e
precisão, embora a narrativa desdiga a si mesma, a cada parágrafo.

Porventura não se trate de dissecarmos a alma humana --- configurada na
reduzida escala dos caricatos cidadãos itaguaienses ---, mas de
desvendarmos a nós mesmos. A novela demanda que desconfiemos dos outros
e riamos de nossos limites, mal equilibrados entre vícios e virtudes.
Como se vê, nem mesmo Fernão Lopes teria pintado retrato mais fiel e
gracioso do douto, louco e extraordinário Simão Bacamarte: ambivalente
``Hipócrates forrado de Catão''.

\chapter{O Alienista\footnote[*]{O texto que se vai ler foi cotejado com a
  primeira versão em livro da novela que abre a coletânea \emph{Papéis
  Avulsos}, editada pela Garnier (Paris e Rio de Janeiro) em 1882.}}

\chapter{\textsc{i}. De como Itaguaí ganhou uma Casa de Orates\footnote{Casa de
  Loucos (o termo ``Orate'' é de origem espanhola).}}

As crônicas da vila de Itaguaí\footnote{Corresponde ao atual município
  de Itaguaí, no interior do Rio de Janeiro. A forma ``vila'' situa a
  localidade no final do século \textsc{xviii}.} dizem que em tempos remotos
vivera ali um certo médico, o Dr.\,Simão Bacamarte, filho da nobreza da
terra e o maior dos médicos do Brasil, de Portugal e das Espanhas.
Estudara em Coimbra e Pádua. Aos trinta e quatro anos regressou ao
Brasil, não podendo el-rei\footnote{``El-Rei'' era a forma respeitosa
  com que os súditos, em geral, referiam-se a Dom João \textsc{v} (1689--1750),
  rei de Portugal.} alcançar dele que ficasse em Coimbra, regendo a
universidade, ou em Lisboa, expedindo os negócios da monarquia.

--- A ciência, disse ele a Sua Majestade, é o meu emprego único; Itaguaí
é o meu universo.

Dito isto, meteu-se em Itaguaí, e entregou-se de corpo e alma ao estudo
da ciência, alternando as curas com as leituras, e demonstrando os
teoremas com cataplasmas. Aos quarenta anos casou com D.\,Evarista da
Costa e Mascarenhas, senhora de vinte e cinco anos, viúva de um juiz de
fora, e não bonita nem simpática. Um dos tios dele, caçador de pacas
perante o Eterno\footnote{Paródia de uma expressão que aparece no
  \emph{Gênesis} (``poderoso caçador perante o Eterno''), no cap. 10,
  vers. 9.}, e não menos franco, admirou-se de semelhante escolha e
disse-lho\footnote{Forma castiça que resulta de contração: ``Disse-o a
  ele''.}. Simão Bacamarte explicou-lhe que D.\,Evarista reunia condições
fisiológicas e anatômicas de primeira ordem, digeria com facilidade,
dormia regularmente, tinha bom pulso, e excelente vista; estava assim
apta para dar-lhe filhos robustos, sãos e inteligentes. Se além dessas
prendas, --- únicas dignas da preocupação de um sábio, D.\,Evarista era
mal composta de feições, longe de lastimá-lo, agradecia-o a Deus,
porquanto não corria o risco de preterir os interesses da ciência na
contemplação exclusiva, miúda e vulgar da consorte.

D.\,Evarista mentiu às esperanças do Dr.\,Bacamarte, não lhe deu filhos
robustos nem mofinos\footnote{Débeis, frágeis, fracos.}. A índole
natural da ciência é a longanimidade\footnote{Generosidade. Resistência
  frente as adversidades.}; o nosso médico esperou três anos, depois
quatro, depois cinco. Ao cabo desse tempo fez um estudo profundo da
matéria, releu todos os escritores árabes e outros, que trouxera para
Itaguaí, enviou consultas às universidades italianas e alemãs, e acabou
por aconselhar à mulher um regime alimentício especial. A ilustre dama,
nutrida exclusivamente com a bela carne de porco de Itaguaí, não atendeu
às admoestações\footnote{Conselho, aviso, repreensão.} do esposo; e à
sua resistência, --- explicável, mas inqualificável, --- devemos a total
extinção da dinastia dos Bacamartes.

Mas a ciência tem o inefável dom de curar todas as mágoas; o nosso
médico mergulhou inteiramente no estudo e na prática da medicina. Foi
então que um dos recantos desta lhe chamou especialmente a atenção, ---
o recanto psíquico, o exame da patologia cerebral. Não havia na colônia,
e ainda no reino, uma só autoridade em semelhante matéria, mal
explorada, ou quase inexplorada\footnote{Esse dado respalda-se na
  historiografia. As primeiras instituições reservadas para a reclusão e
  terapêutica dos loucos remontam à Idade Média; mas os estabelecimentos
  dessa natureza demoraram séculos para chegar a Portugal.}. Simão
Bacamarte compreendeu que a ciência lusitana, e particularmente a
brasileira, podia cobrir-se de ``louros imarcescíveis''\footnote{Expressão
  enfática que combina o símbolo da glória (``louros'') à sua perenidade
  (``imarcessível''). No contexto, sugere que as folhas (da vitória) não
  perderiam o viço (o vigor).}, --- expressão usada por ele mesmo, mas
em um arroubo de intimidade doméstica; exteriormente era modesto,
segundo convém aos sabedores.

--- A saúde da alma, bradou ele, é a ocupação mais digna do médico.

--- Do verdadeiro médico, emendou Crispim Soares, boticário\footnote{Responsável
  pela botica, estabelecimento onde se manipulavam remédios.} da vila, e
um dos seus amigos e comensais\footnote{Convidados para o banquete.}.

A vereança\footnote{Câmara de Vereadores.} de Itaguaí, entre outros
pecados de que é arguida pelos cronistas, tinha o de não fazer caso dos
dementes. Assim é que cada louco furioso era trancado em uma alcova, na
própria casa, e, não curado, mas descurado, até que a morte o vinha
defraudar\footnote{Contrariar a lei, fraudar.} do benefício da vida; os
mansos\footnote{Alusão de Machado à teoria dos humores, formulada por
  Hipócrates (460--370 a.C.). O médico grego classificava os homens em
  quatro tipos: irascíveis (sangue), fleumáticos (fleuma), coléricos
  (bile amarela) e melancólicos (bile negra).} andavam à solta pela rua.
Simão Bacamarte entendeu desde logo reformar tão ruim costume; pediu
licença à Câmara para agasalhar e tratar no edifício que ia construir
todos os loucos de Itaguaí e das demais vilas e cidades\footnote{``Vilas''
  e ``cidades'' eram localidades de diferente configuração. A comarca de
  Itaguaí passou ao estatuto de vila em 1720. Em 1818, foi alçada a
  município.}, mediante um estipêndio\footnote{Pagamento de uma
  gratificação.}, que a Câmara lhe daria quando a família do enfermo o
não pudesse fazer. A proposta excitou a curiosidade de toda a vila, e
encontrou grande resistência, tão certo é que dificilmente se
desarraigam hábitos absurdos, ou ainda maus. A ideia de meter os loucos
na mesma casa, vivendo em comum, pareceu em si mesma um sintoma de
demência, e não faltou quem o insinuasse à própria mulher do médico.

--- Olhe, D.\,Evarista, disse-lhe o padre Lopes, vigário do lugar, veja
se seu marido dá um passeio ao Rio de Janeiro. Isso de estudar sempre,
sempre, não é bom, vira o juízo.

D.\,Evarista ficou aterrada, foi ter com o marido, disse-lhe ``que estava
com desejos'', um principalmente, o de vir\footnote{Machado optou pelo
  verbo ``vir'' (e não ``ir'').} ao Rio de Janeiro\footnote{Capital do
  Estado do Brasil a partir de 1763.} e comer tudo o que a ele lhe
parecesse adequado a certo fim. Mas aquele grande homem, com a rara
sagacidade que o distinguia, penetrou a intenção da esposa e
redarguiu-lhe sorrindo que não tivesse medo. Dali foi à Câmara, onde os
vereadores debatiam a proposta, e defendeu-a com tanta eloquência, que a
maioria resolveu autorizá-lo ao que pedira, votando ao mesmo tempo um
imposto destinado a subsidiar o tratamento, alojamento e mantimento dos
doidos pobres. A matéria do imposto não foi fácil achá-la; tudo estava
tributado em Itaguaí. Depois de longos estudos, assentou-se em permitir
o uso de dois penachos nos cavalos dos enterros. Quem quisesse emplumar
os cavalos de um coche mortuário pagaria dois\footnote{Na novela, o
  autor alternou a grafia do número entre ``dous'' e ``dois''. As formas
  foram padronizadas para ``dois''. O mesmo vale para ``coisa'' e
  ``doido'' (aqui, grafadas ``coisa'' e ``doido'').} tostões à Câmara,
repetindo-se tantas vezes esta quantia quantas fossem as horas
decorridas entre a do falecimento e a da última bênção na sepultura. O
escrivão perdeu-se nos cálculos aritméticos do rendimento possível da
nova taxa; e um dos vereadores, que não acreditava na empresa do médico,
pediu que se relevasse o escrivão de um trabalho inútil.

--- Os cálculos não são precisos, disse ele, porque o Dr.\,Bacamarte não
arranja nada. Quem é que viu agora meter todos os doidos dentro da mesma
casa?

Enganava-se o digno magistrado; o médico arranjou tudo. Uma vez
empossado da licença começou logo a construir a casa. Era na Rua Nova, a
mais bela rua de Itaguaí naquele tempo, tinha cinquenta janelas por
lado, um pátio no centro, e numerosos cubículos para os hóspedes. Como
fosse grande arabista, achou no \emph{Corão}\footnote{O \emph{Corão},
  livro sagrado dos seguidores de Maomé, desde o século \textsc{vii}.} que
Maomé\footnote{Maomé (571--632), o mensageiro de Alá, a quem se atribui a
  autoria do \emph{Corão}.} declara veneráveis os doidos, pela
consideração de que Alá\footnote{Denominação de Deus em árabe. O
  ``achado'' de Simão Bacamarte não está no \emph{Corão}.} lhes tira o
juízo para que não pequem. A ideia pareceu-lhe bonita e profunda, e ele
a fez gravar no frontispício da casa; mas, como tinha medo ao vigário, e
por tabela ao bispo, atribuiu o pensamento a Benedito \textsc{viii}\footnote{Benedito
  \textsc{viii} (980--1024), centésimo quadragésimo terceiro papa romano.
  Enfrentou os sarracenos, na Itália. Dentre algumas regras que
  instituiu com vigor, está o celibato dos padres.}, merecendo com essa
fraude, aliás pia, que o padre Lopes lhe contasse, ao almoço, a vida
daquele pontífice eminente.

A Casa Verde foi o nome dado ao asilo, por alusão à cor das janelas, que
pela primeira vez apareciam verdes em Itaguaí. Inaugurou-se com imensa
pompa; de todas as vilas e povoações próximas, e até remotas, e da
própria cidade do Rio de Janeiro, correu gente para assistir às
cerimônias, que duraram sete dias. Muitos dementes já estavam
recolhidos; e os parentes tiveram ocasião de ver o carinho paternal e a
caridade cristã com que eles iam ser tratados. D.\,Evarista,
contentíssima com a glória do marido, vestira-se luxuosamente, cobriu-se
de joias, flores e sedas. Ela foi uma verdadeira rainha naqueles dias
memoráveis; ninguém deixou de ir visitá-la duas e três vezes, apesar dos
costumes caseiros e recatados do século, e não só a cortejavam como a
louvavam; porquanto, --- e este fato é um documento altamente honroso
para a sociedade do tempo, ---porquanto viam nela a feliz esposa de um
alto espírito, de um varão ilustre, e, se lhe tinham inveja, era a santa
e nobre inveja dos admiradores. Ao cabo de sete dias expiraram as festas
públicas; Itaguaí tinha finalmente uma casa de orates.

\chapter{\textsc{ii}. Torrente de Loucos}

Três dias depois, numa expansão íntima com o boticário Crispim Soares,
desvendou o alienista o mistério do seu coração.

--- A caridade, Sr.\,Soares, entra decerto no meu procedimento, mas entra
como tempero, como o sal das coisas, que é assim que interpreto o dito
de São Paulo aos Coríntios: ``Se eu conhecer quanto se pode saber, e não
tiver caridade, não sou nada''.\footnote{Cf. Epístola de Paulo aos
  Coríntios, capítulo 1, versículo 13.} O principal nesta minha obra da
Casa Verde é estudar profundamente a loucura, os seus diversos graus,
classificar-lhe os casos, descobrir enfim a causa do fenômeno e o
remédio universal. Este é o mistério do meu coração. Creio que com isto
presto um bom serviço à humanidade.

--- Um excelente serviço, corrigiu o boticário.

--- Sem este asilo, continuou o alienista, pouco poderia fazer; ele
dá-me, porém, muito maior campo aos meus estudos.

--- Muito maior, acrescentou o outro.

E tinham razão. De todas as vilas e arraiais vizinhos afluíam loucos à
Casa Verde. Eram furiosos, eram mansos, eram monomaníacos, era toda a
família dos deserdados do espírito. Ao cabo de quatro meses, a Casa
Verde era uma povoação. Não bastaram os primeiros cubículos; mandou-se
anexar uma galeria de mais trinta e sete. O padre Lopes confessou que
não imaginara a existência de tantos doidos no mundo, e menos ainda o
inexplicável de alguns casos. Um, por exemplo, um rapaz bronco e vilão,
que todos os dias, depois do almoço, fazia regularmente um discurso
acadêmico, ornado de tropos\footnote{Termos utilizados com sentido
  diverso do habitual (como a metáfora e a metonímia).}, de
antíteses\footnote{Aproximação de palavras de sentido oposto.}, de
apóstrofes\footnote{Interrupção, no discurso, para aludir a seres reais
  ou fictícios. Interpelação. Dito mordaz.}, com seus recamos\footnote{Ornamento,
  floreio discursivo.} de grego e latim, e suas borlas\footnote{Literalmente,
  pompons, enfeites. Metaforicamente, adornos, ornamentos no discurso.}
de Cícero, Apuleio e Tertuliano\footnote{Cícero (106--43 a.C.), orador;
  Apuleio (125--170 d.C.), filósofo; Tertuliano (160--220 d.C.), teólogo
  cristão.}. O vigário não queria acabar de crer. Quê! um rapaz que ele
vira, três meses antes, jogando peteca na rua!

--- Não digo que não, respondia-lhe o alienista; mas a verdade é o que
Vossa Reverendíssima está vendo. Isto é todos os dias.

--- Quanto a mim, tornou o vigário, só se pode explicar pela confusão
das línguas na torre de Babel, segundo nos conta a Escritura\footnote{Edificação
  construída pelos descendentes de Noé, após o dilúvio. Por punição de
  Deus, houve confusão entre as línguas ali praticadas. O episódio está
  no \emph{Gênesis}, capítulo 11.}; provavelmente, confundidas
antigamente as línguas, é fácil trocá-las agora, desde que a razão não
trabalhe\ldots{}

--- Essa pode ser, com efeito, a explicação divina do fenômeno,
concordou o alienista, depois de refletir um instante, mas não é
impossível que haja também alguma razão humana, e puramente científica,
e disso trato\ldots{}

--- Vá que seja, e fico ansioso. Realmente!

Os loucos por amor eram três ou quatro, mas só dois espantavam pelo
curioso do delírio. O primeiro, um Falcão, rapaz de vinte e cinco anos,
supunha-se estrela-d'alva, abria os braços e alargava as pernas, para
dar-lhes certa feição de raios, e ficava assim horas esquecidas a
perguntar se o sol já tinha saído para ele recolher-se. O outro andava
sempre, sempre, sempre, à roda das salas ou do pátio, ao longo dos
corredores, à procura do fim do mundo. Era um desgraçado, a quem a
mulher deixou por seguir um peralvilho. Mal descobrira a fuga, armou-se
de uma garrucha, e saiu-lhes no encalço; achou-os duas horas depois, ao
pé de uma lagoa, matou-os a ambos com os maiores requintes de crueldade.
O ciúme satisfez-se, mas o vingado estava louco. E então começou aquela
ânsia de ir ao fim do mundo à cata dos fugitivos.

A mania das grandezas tinha exemplares notáveis. O mais notável era um
pobre-diabo, filho de um algibebe, que narrava às paredes (porque não
olhava nunca para nenhuma pessoa) toda a sua genealogia, que era esta:

--- Deus engendrou um ovo, o ovo engendrou a espada, a espada engendrou
Davi\footnote{Rei de Israel, a quem se atribuem os salmos, na
  \emph{Bíblia}.}, Davi engendrou a púrpura, a púrpura engendrou o
duque, o duque engendrou o marquês, o marquês engendrou o conde, que sou
eu.\footnote{Possível paródia relacionada a Sem, filho de Noé, constante
  do capítulo onze, no \emph{Gênesis}.}

Dava uma pancada na testa, um estalo com os dedos, e repetia cinco, seis
vezes seguidas:

--- Deus engendrou um ovo, o ovo, etc.

Outro da mesma espécie era um escrivão, que se vendia por mordomo do
rei; outro era um boiadeiro de Minas, cuja mania era distribuir boiadas
a toda a gente, dava trezentas cabeças a um, seiscentas a outro, mil e
duzentas a outro, e não acabava mais. Não falo dos casos de monomania
religiosa; apenas citarei um sujeito que, chamando-se João de Deus,
dizia agora ser o deus João, e prometia o reino dos céus a quem o
adorasse, e as penas do inferno aos outros; e depois desse, o licenciado
Garcia, que não dizia nada, porque imaginava que no dia em que chegasse
a proferir uma só palavra, todas as estrelas se despegariam do céu e
abrasariam a terra; tal era o poder que recebera de Deus. Assim o
escrevia ele no papel que o alienista lhe mandava dar, menos por
caridade do que por interesse científico.

Que, na verdade, a paciência do alienista era ainda mais extraordinária
do que todas as manias hospedadas na Casa Verde; nada menos que
assombrosa. Simão Bacamarte começou por organizar um pessoal de
administração; e, aceitando essa ideia ao boticário Crispim Soares,
aceitou-lhe também dois sobrinhos, a quem incumbiu da execução de um
regimento que lhes deu, aprovado pela câmara, da distribuição da comida
e da roupa, e assim também da escrita, etc. Era o melhor que podia
fazer, para somente cuidar do seu ofício. --- A Casa Verde, disse ele ao
vigário, é agora uma espécie de mundo, em que há o governo temporal e o
governo \emph{espiritual}\footnote{De acordo com os principais teólogos
  da igreja católica, Deus regeria o mundo dos homens (temporal) e
  aquele que os transcende (espiritual).}. E o padre Lopes ria deste pio
trocado\footnote{Dito piedoso que foi alterado pelo médico.}, --- e
acrescentava, --- com o único fim de dizer também uma chalaça\footnote{Dito
  zombeteiro.}: --- Deixe estar, deixe estar, que hei de mandá-lo
denunciar ao papa.

Uma vez desonerado da administração, o alienista procedeu a uma vasta
classificação dos seus enfermos. Dividiu-os primeiramente em duas
classes principais: os furiosos e os mansos; daí passou às subclasses,
monomanias, delírios, alucinações diversas. Isto feito, começou um
estudo aturado\footnote{Grafado desta forma, no original (em lugar de
  ``apurado'').} e contínuo; analisava os hábitos de cada louco, as
horas de acesso, as aversões, as simpatias, as palavras, os gestos, as
tendências; inquiria da vida dos enfermos, profissão, costumes,
circunstâncias da revelação mórbida, acidentes da infância e da
mocidade, doenças de outra espécie, antecedentes na família, uma
devassa\footnote{Averiguação, investigação. Possível alusão aos autos da
  devassa, implementada durante a chamada Inconfidência Mineira, entre
  1789 e 1792.}, enfim, como a não faria o mais atilado\footnote{Sagaz,
  inteligente, cuidadoso.} corregedor\footnote{Magistrado posicionado
  acima de seus colegas de profissão. Fiscaliza o trabalho dos demais.}.
E cada dia notava uma observação nova, uma descoberta interessante, um
fenômeno extraordinário. Ao mesmo tempo estudava o melhor regime, as
substâncias medicamentosas, os meios curativos e os meios paliativos,
não só os que vinham nos seus amados árabes, como os que ele mesmo
descobria, à força de sagacidade e paciência. Ora, todo esse trabalho
levava-lhe o melhor e o mais do tempo. Mal dormia e mal comia; e, ainda
comendo, era como se trabalhasse, porque ora interrogava um texto
antigo, ora ruminava uma questão, e ia muitas vezes de um cabo a outro
do jantar sem dizer uma só palavra a D.\,Evarista.

\chapter{\textsc{iii}. Deus sabe o que faz!\footnote{Expressão recorrente tanto na
  \emph{Bíblia} quanto no \emph{Corão}.}}

A ilustre dama, no fim de dois meses, achou-se a mais desgraçada das
mulheres; caiu em profunda melancolia, ficou amarela, magra, comia pouco
e suspirava a cada canto. Não ousava fazer-lhe nenhuma queixa ou
reproche\footnote{Palavra de origem francesa, significa censura,
  repreensão.}, porque respeitava nele o seu marido e senhor, mas
padecia calada, e definhava a olhos vistos. Um dia, ao jantar, como lhe
perguntasse o marido o que é que tinha, respondeu tristemente\footnote{Grafado
  ``tritemente'', no original.} que nada; depois atreveu-se um pouco, e
foi ao ponto de dizer que se considerava tão viúva como dantes. E
acrescentou:

--- Quem diria nunca que meia dúzia de lunáticos\ldots{}

Não acabou a frase; ou antes, acabou-a levantando os olhos ao teto, ---
os olhos, que eram a sua feição mais insinuante, --- negros, grandes,
lavados de uma luz úmida, como os da aurora. Quanto ao gesto, era o
mesmo que empregara no dia em que Simão Bacamarte a pediu em casamento.
Não dizem as crônicas se D.\,Evarista brandiu aquela arma com o perverso
intuito de degolar de uma vez a ciência, ou, pelo menos, decepar-lhe as
mãos; mas a conjectura é verossímil. Em todo caso, o alienista não lhe
atribuiu outra intenção. E não se irritou o grande homem, não ficou
sequer consternado. O metal de seus olhos não deixou de ser o mesmo
metal, duro, liso, eterno, nem a menor prega veio quebrar a superfície
da fronte quieta como a água de Botafogo\footnote{Alusão à praia de
  Botafogo, no Rio de Janeiro, em sua porção sem ondas.}. Talvez um
sorriso lhe descerrou os lábios, por entre os quais filtrou esta palavra
macia como o óleo do \emph{Cântico}\footnote{Referência ao \emph{Cântico
  dos Cânticos}, livro do Antigo Testamento. A expressão ``óleo do
  \emph{Cântico}'' alude às metáforas de Salomão, que aproxima
  determinadas pessoas de perfumes característicos.}:

--- Consinto que vás dar um passeio ao Rio de Janeiro.

D.\,Evarista sentiu faltar-lhe o chão debaixo dos pés. Nunca dos nuncas
vira o Rio de Janeiro, que posto não fosse sequer uma pálida sombra do
que hoje é, todavia era alguma coisa mais do que Itaguaí. Ver o Rio de
Janeiro, para ela, equivalia ao sonho do hebreu cativo\footnote{Os
  ``hebreus'', enquanto cativos da Babilônia, sonhavam em retornar para
  a terra prometida.}. Agora, principalmente, que o marido assentara de
vez naquela povoação interior, agora é que ela perdera as últimas
esperanças de respirar os ares da nossa boa cidade; e justamente agora é
que ele a convidava a realizar os seus desejos de menina e moça. D.\,Evarista não pôde dissimular o gosto de semelhante proposta. Simão
Bacamarte pagou-lhe na mão e sorriu, --- um sorriso tanto ou quanto
filosófico, além de conjugal, em que parecia traduzir-se este
pensamento: --- ``Não há remédio certo para as dores da alma; esta
senhora definha, porque lhe parece que a não amo; dou-lhe o Rio de
Janeiro, e consola-se''. E porque era homem estudioso tomou nota da
observação.

Mas um dardo atravessou o coração de D.\,Evarista. Conteve-se,
entretanto; limitou-se a dizer ao marido, que, se ele não ia, ela não
iria também, porque não havia de meter-se sozinha pelas estradas.

--- Irá com sua tia, redarguiu o alienista.

Note-se que D.\,Evarista tinha pensado nisso mesmo; mas não quisera
pedi-lo nem insinuá-lo, em primeiro lugar porque seria impor grandes
despesas ao marido, em segundo lugar porque era melhor, mais metódico e
racional que a proposta viesse dele.

--- Oh! mas o dinheiro que será preciso gastar! suspirou D.\,Evarista sem
convicção.

--- Que importa? Temos ganho muito, disse o marido. Ainda ontem o
escriturário prestou-me contas. Queres ver?

E levou-a aos livros. D.\,Evarista ficou deslumbrada. Era uma via láctea
de algarismos. E depois levou-a às arcas, onde estava o dinheiro. Deus!
eram montes de ouro, eram mil cruzados sobre mil cruzados, dobrões sobre
dobrões; era a opulência. Enquanto ela comia o ouro com os seus olhos
negros, o alienista fitava-a, e dizia-lhe ao ouvido com a mais pérfida
das alusões.

--- Quem diria que meia dúzia de lunáticos\ldots{}

D.\,Evarista compreendeu, sorriu e respondeu com muita resignação:

--- Deus sabe o que faz!

Três meses depois efetuava-se a jornada. D.\,Evarista, a tia, a mulher do
boticário, um sobrinho deste, um padre que o alienista conhecera em
Lisboa, e que de aventura achava-se em Itaguaí, cinco ou seis
pajens\footnote{Criado, empregado (termo criado na Idade Média).},
quatro mucamas\footnote{Criada (durante a escravidão, também trabalhava
  como ama de leite).}, tal foi a comitiva que a população viu dali sair
em certa manhã do mês de maio. As despedidas foram tristes para todos,
menos para o alienista. Conquanto as lágrimas de D.\,Evarista fossem
abundantes e sinceras, não chegaram a abalá-lo. Homem de ciência, e só
de ciência, nada o consternava fora da ciência; e se alguma coisa o
preocupava naquela ocasião, se ele deixava correr pela multidão um olhar
inquieto e policial, não era outra coisa mais do que a ideia de que
algum demente podia achar-se ali misturado com a gente de juízo.

--- Adeus! soluçaram enfim as damas e o boticário.

E partiu a comitiva. Crispim Soares, ao tornar a casa, trazia os olhos
entre as duas orelhas da besta ruana\footnote{Pelagem branca com machas
  negras.} em que vinha montado; Simão Bacamarte alongava os seus pelo
horizonte adiante, deixando ao cavalo a responsabilidade do regresso.
Imagem vivaz do gênio e do vulgo\footnote{A descrição recupera a
  contrastante dupla Dom Quixote e Sancho Pança, cavaleiro e escudeiro
  do romance de Miguel de Cervantes Saavedra --- publicado em 1705 e
  1710.}! Um fita o presente, com todas as suas lágrimas e saudades,
outro devassa o futuro com todas as suas auroras.

\chapter{\textsc{iv}. Uma teoria nova }

Ao passo que D.\,Evarista, em lágrimas, vinha buscando o Rio de Janeiro,
Simão Bacamarte estudava por todos os lados uma certa ideia arrojada e
nova, própria a alargar as bases da psicologia. Todo o tempo que lhe
sobrava dos cuidados da Casa Verde, era pouco para andar na rua, ou de
casa em casa, conversando\footnote{Grafado desta forma no original, sem
  a preposição ``com''.} as gentes, sobre trinta mil assuntos, e
virgulando as falas de um olhar que metia medo aos mais heroicos.

Um dia de manhã, --- eram passadas três semanas, --- estando Crispim
Soares ocupado em temperar um medicamento, vieram dizer-lhe que o
alienista o mandava chamar.

--- Trata-se de negócio importante, segundo ele me disse, acrescentou o
portador.

Crispim empalideceu. Que negócio importante podia ser, se não alguma
triste notícia da comitiva, e especialmente da mulher? Porque este
tópico deve ficar claramente definido, visto insistirem nele os
cronistas: Crispim amava a mulher, e, desde trinta anos, nunca estiveram
separados um só dia. Assim se explicam os monólogos que ele fazia agora,
e que os fâmulos\footnote{Serviçal, bajulador.} lhe ouviam muita vez:
--- ``Anda, bem feito, quem te mandou consentir na viagem de Cesária?
Bajulador, torpe bajulador! Só para adular ao Dr.\,Bacamarte. Pois agora
aguenta-te; anda, aguenta-te, alma de lacaio, fracalhão, vil, miserável.
Dizes amém a tudo, não é? aí tens o lucro, biltre\footnote{Patife, que
  age de modo vil.}!'' --- E muitos outros nomes feios, que um homem não
deve dizer aos outros, quanto mais a si mesmo. Daqui a imaginar o efeito
do recado é um nada. Tão depressa ele o recebeu como abriu mão das
drogas e voou à Casa Verde.

Simão Bacamarte recebeu-o com a alegria própria de um sábio, uma alegria
abotoada de circunspecção até o pescoço.

--- Estou muito contente, disse ele.

--- Notícias do nosso povo? perguntou o boticário com a voz trêmula.

O alienista fez um gesto magnífico, e respondeu:

--- Trata-se de coisa mais alta, trata-se de uma experiência científica.
Digo experiência, porque não me atrevo a assegurar desde já a minha
ideia; nem a ciência é outra coisa, Sr.\,Soares, se não uma investigação
constante. Trata-se, pois, de uma experiência, mas uma experiência que
vai mudar a face da terra. A loucura, objeto dos meus estudos, era até
agora uma ilha perdida no oceano da razão; começo a suspeitar que é um
continente.

Disse isto, e calou-se, para ruminar o pasmo do boticário. Depois
explicou compridamente a sua ideia. No conceito dele a insânia abrangia
uma vasta superfície de cérebros; e desenvolveu isto com grande cópia de
raciocínios, de textos, de exemplos. Os exemplos achou-os na história e
em Itaguaí\footnote{Bacamarte se refere a Itaguaí, mas pode-se inferir
  que Machado aludisse a ilustres cidadãos que moravam no Rio de Janeiro
  em seu tempo. A cidade contava com o Hospício Pedro \textsc{ii}, desde 1852 ---
  tempo que não corresponde ao da novela.}; mas, como um raro espírito
que era, reconheceu o perigo de citar todos os casos de Itaguaí e
refugiou-se na história. Assim, apontou com especialidade alguns
personagens célebres, Sócrates, que tinha um demônio familiar\footnote{Sócrates
  (469--399 a.C.). Filósofo grego, foi acusado de desviar os jovens. Foi
  condenado ao suicídio por envenenamento.}, Pascal, que via um abismo à
esquerda\footnote{Blaise Pascal (1623--1662). Pensador francês, com
  conhecimentos em matemática, física e teologia. Tornou-se célebre pelo
  livro de pensamentos e máximas, publicado após a sua morte.}, Maomé,
Caracala\footnote{Caracala, nome como ficou mais conhecido o imperador
  romano Marco Aurélio Antonino (188--217 d.C.). Implacável nos combates
  de que participou, concedeu cidadania romana à quase totalidade dos
  povos livres, em 212 d.C.}, Domiciano\footnote{Tito Flávio Domiciano
  (51--96 d.C.) foi um imperador romano, considerado tirânico e violento.},
Calígula\footnote{Caio Júlio César Augusto Germânico (12--41 d.C.) foi um
  imperador romano, considerado extravagante e violento. Foi assassinado
  pela guarda pretoriana, a mando de Cássio, que tencionava reimplantar
  a república.}, etc., uma enfiada de casos e pessoas, em que de mistura
vinham entidades odiosas, e entidades ridículas. E porque o boticário se
admirasse de uma tal promiscuidade, o alienista disse-lhe que era tudo a
mesma coisa, e até acrescentou sentenciosamente:

--- A ferocidade, Sr.\,Soares, é o grotesco a sério.

--- Gracioso, muito gracioso! exclamou Crispim Soares levantando as mãos
ao céu.

Quanto à ideia de ampliar o território da loucura, achou-a o boticário
extravagante; mas a modéstia, principal adorno de seu espírito, não lhe
sofreu confessar outra coisa além de um nobre entusiasmo; declarou-a
sublime e verdadeira, e acrescentou que era ``caso de matraca''. Esta
expressão não tem equivalente no estilo moderno. Naquele tempo, Itaguaí,
que como as demais vilas, arraiais e povoações da colônia, não dispunha
de imprensa\footnote{Antes de a imprensa do reino chegar ao país,
  portanto, o que só veio a acontecer em 1808.}, tinha dois modos de
divulgar uma notícia: ou por meio de cartazes manuscritos e pregados na
porta da câmara e da matriz; --- ou por meio de matraca. Eis em que
consistia este segundo uso. Contratava-se um homem, por um ou mais dias,
para andar as ruas do povoado, com uma matraca na mão. De quando em
quando tocava a matraca, reunia-se gente, e ele anunciava o que lhe
incumbiam, --- um remédio para sezões\footnote{Febre intermitente.},
umas terras lavradias\footnote{Terras apropriadas para a lavoura.}, um
soneto\footnote{Gênero poético que teria sido criado na Itália, entre os
  séculos \textsc{xiii} e \textsc{xiv}. Essa forma conquistou grande popularidade, sendo
  cultivada até hoje.}, um donativo eclesiástico\footnote{Doação à
  igreja.}, a melhor tesoura da vila, o mais belo discurso do ano, etc.
O sistema tinha inconvenientes para a paz pública; mas era conservado
pela grande energia de divulgação que possuía. Por exemplo, um dos
vereadores, --- aquele justamente que mais se opusera à criação da Casa
Verde, --- desfrutava a reputação de perfeito educador de cobras e
macacos, e aliás nunca domesticara um só desses bichos; mas, tinha o
cuidado de fazer trabalhar a matraca todos os meses. E dizem as crônicas
que algumas pessoas afirmavam ter visto cascavéis dançando no peito do
vereador; afirmação perfeitamente falsa, mas só devida à absoluta
confiança no sistema. Verdade, verdade, nem todas as instituições do
antigo regime\footnote{O Antigo Regime, ou Monarquia (expressão
  originária do francês), durou até o final do século \textsc{xviii}, quando foi
  derrubado pela Revolução Francesa.} mereciam o desprezo do nosso
século.

--- Há melhor do que anunciar a minha ideia, é praticá-la, respondeu o
alienista à insinuação do boticário.

E o boticário, não divergindo sensivelmente deste modo de ver, disse-lhe
que sim, que era melhor começar pela execução.

--- Sempre haverá tempo de a dar à matraca, concluiu ele.

Simão Bacamarte refletiu ainda um instante, e disse:

--- Suponho o espírito humano uma vasta concha, o meu fim, Sr.\,Soares, é
ver se posso extrair a pérola, que é a razão; por outros termos,
demarquemos definitivamente os limites da razão e da loucura. A razão é
o perfeito equilíbrio de todas as faculdades; fora daí insânia, insânia
e só insânia.

O Vigário Lopes a quem ele confiou a nova teoria, declarou
lisamente\footnote{Honesto, sem desvios.} que não chegava a entendê-la,
que era uma obra absurda, e, se não era absurda, era de tal modo
colossal que não merecia princípio de execução.

--- Com a definição atual\footnote{Alusão ao chamado período das luzes
  (ao final do século \textsc{xviii}), em que se os enciclopedistas reivindicavam
  a supremacia da razão, como caminho para o esclarecimento e a
  liberdade, a justiça e a fraternidade.}, que é a de todos os tempos,
acrescentou, a loucura e a razão estão perfeitamente delimitadas.
Sabe-se onde uma acaba e onde a outra começa. Para que transpor a cerca?

Sobre o lábio fino e discreto\footnote{Desde o século \textsc{xvii}, circulavam
  na Europa manuais que ensinavam os nobres a se portar, de maneira a
  afetar superioridade e não serem confundidos com o povo, dono de
  hábitos vulgares e indiscretos.} do alienista roçou a vaga sombra de
uma intenção de riso, em que o desdém vinha casado à
comiseração\footnote{Compaixão, piedade.}; mas nenhuma palavra saiu de
suas egrégias\footnote{Distinta, magnífica.} entranhas. A ciência
contentou-se em estender a mão à teologia, --- com tal segurança, que a
teologia não soube enfim se devia crer em si ou na outra. Itaguaí e o
universo ficavam à beira de uma revolução.

\chapter{\textsc{v.} O Terror\footnote{Alusão ao período violento, sob os Jacobinos,
  que sucedeu na França entre 1793 e 1794 e que culminou na prisão do
  líder Robespierre.}}

Quatro dias depois, a população de Itaguaí ouviu consternada a notícia
de que um certo Costa fora recolhido à Casa Verde.

--- Impossível!

--- Qual impossível! foi recolhido hoje de manhã.

--- Mas, na verdade, ele não merecia\ldots{} Ainda em cima! depois de
tanto que ele fez\ldots{}

Costa era um dos cidadãos mais estimados de Itaguaí. Herdara
quatrocentos mil cruzados em boa moeda de el-rei D.\,João \textsc{V}\footnote{Dom
  João \textsc{v} (1689--1750), rei português conhecido como ``o Magnânimo''. Seu
  mandato foi marcado pela descoberta e exploração de ouro em Minas
  Gerais. Na novela, isso parece explicar a generosa herança que deixou
  para o sobrinho.}, dinheiro cuja renda bastava, segundo lhe declarou o
tio no testamento, para viver ``até o fim do mundo''. Tão depressa
recolheu a herança, como entrou a dividi-la em empréstimos, sem usura,
mil cruzados a um, dois mil a outro, trezentos a este, oitocentos
àquele, a tal ponto que, no fim de cinco anos, estava sem nada. Se a
miséria viesse de chofre, o pasmo de Itaguaí seria enorme; mas veio
devagar; ele foi passando da opulência à abastança, da abastança à
mediania, da mediania à pobreza, da pobreza à miséria, gradualmente. Ao
cabo daqueles cinco anos, pessoas que levavam o chapéu ao chão, logo que
ele assomava no fim da rua, agora batiam-lhe no ombro, com intimidade,
davam-lhe piparotes no nariz, diziam-lhe pulhas\footnote{Sem caráter.}.
E o Costa sempre lhano\footnote{Simples, humilde, gentil.}, risonho. Nem
se lhe dava de ver que os menos corteses eram justamente os que tinham
ainda a dívida em aberto; ao contrário, parece que os agasalhava com
maior prazer, e mais sublime resignação. Um dia, como um desses
incuráveis devedores lhe atirasse uma chalaça grossa, e ele se risse
dela, observou um desafeiçoado, com certa perfídia: --- ``Você suporta
esse sujeito para ver se ele lhe paga''. Costa não se deteve um minuto,
foi ao devedor e perdoou-lhe a dívida. --- ``Não admira, retorquiu o
outro; o Costa abriu mão de uma estrela, que está no céu''. Costa era
perspicaz, entendeu que ele negava todo o merecimento ao ato,
atribuindo-lhe a intenção de rejeitar o que não vinham meter-lhe na
algibeira. Era também pundonoroso\footnote{Que valorizada a honra;
  orgulhoso de si mesmo (palavra de origem castelhana, ``\emph{punt
  d'honor}'').} e inventivo; duas horas depois achou um meio de provar
que lhe não cabia um tal labéu\footnote{Infâmia, mancha, calúnia.}:
pegou de algumas dobras\footnote{Moeda de ouro, equivalente a 12.800
  réis (ou 40 patacas de prata). Foi utilizada entre os séculos \textsc{xviii} e
  \textsc{xix}, em Portugal e suas possessões, incluindo o Brasil.}, e mandou-as
de empréstimo ao devedor.

--- Agora espero que\ldots{} --- pensou ele sem concluir a frase.

Esse último rasgo do Costa persuadiu a crédulos e incrédulos; ninguém
mais pôs em dúvida os sentimentos cavalheirescos daquele digno cidadão.
As necessidades mais acanhadas saíram à rua, vieram bater-lhe à porta,
com os seus chinelos velhos, com as suas capas remendadas. Um verme,
entretanto, roía a alma do Costa: era o conceito do desafeto. Mas isso
mesmo acabou; três meses depois veio este pedir-lhe uns cento e vinte
cruzados com promessa de restituir-lhos daí a dois dias; era o resíduo
da grande herança, mas era também uma nobre desforra: Costa emprestou o
dinheiro logo, logo, e sem juros. Infelizmente não teve tempo de ser
pago; cinco meses depois era recolhido à Casa Verde.

Imagina-se a consternação de Itaguaí, quando soube do caso. Não se falou
em outra coisa, dizia-se que o Costa ensandecera, no almoço, outros que
de madrugada; e contavam-se os acessos, que eram furiosos, sombrios,
terríveis, --- ou mansos, e até engraçados, conforme as versões. Muita
gente correu à Casa Verde, e achou o pobre Costa, tranquilo, um pouco
espantado, falando com muita clareza, e perguntando por que motivo o
tinham levado para ali. Alguns foram ter com o alienista. Bacamarte
aprovava esses sentimentos de estima e compaixão, mas acrescentava que a
ciência era a ciência, e que ele não podia deixar na rua um
mentecapto\footnote{Aquele que não possui razão, clareza ou
  discernimento (palavra de origem latina, ``\emph{mente + captus}'').}.
A última pessoa que intercedeu por ele (porque depois do que vou contar
ninguém mais se atreveu a procurar o terrível médico) foi uma pobre
senhora, prima do Costa. O alienista disse-lhe confidencialmente que
esse digno homem não estava no perfeito equilíbrio das faculdades
mentais, à vista do modo como dissipara os cabedais que\ldots{}

--- Isso, não! isso, não! interrompeu a boa senhora com energia. Se ele
gastou tão depressa o que recebeu, a culpa não é dele.

--- Não?

--- Não, senhor. Eu lhe digo como o negócio se passou. O defunto meu tio
não era mau homem; mas quando estava furioso era capaz de nem tirar o
chapéu ao Santíssimo\footnote{Sinal de respeito a Deus. Denominação do
  local, nas igrejas, reservado ao Santíssimo Sacramento.}. Ora, um dia,
pouco tempo antes de morrer, descobriu que um escravo lhe roubara um
boi; imagine como ficou. A cara era um pimentão; todo ele tremia, a boca
escumava\footnote{Espumava.}; lembra-me como se fosse hoje. Então um
homem feio, cabeludo, em mangas de camisa, chegou-se a ele e pediu água.
Meu tio (Deus lhe fale n'alma!) respondeu que fosse beber ao rio ou ao
inferno. O homem olhou para ele, abriu a mão em ar de ameaça, e rogou
esta praga: --- ``Todo o seu dinheiro não há de durar mais de sete anos
e um dia, tão certo como isto ser o \emph{sino salamão}!\footnote{Símbolo
  (dois triângulos entrelaçados) inscrito no anel do rei Salomão.
  Acreditava-se que tinha poderes mágicos e poderia protegê-lo dos maus
  espíritos.}'' E mostrou o \emph{sino salamão} impresso no braço. Foi
isto, meu senhor; foi esta praga daquele maldito.

Bacamarte espetara na pobre senhora um par de olhos agudos como punhais.
Quando ela acabou, estendeu-lhe a mão polidamente, como se o fizesse à
própria esposa do vice-rei\footnote{``O vice-rei ou capitão-general era
  o delegado imediato do soberano, para quem unicamente se podia apelar
  de suas resoluções. {[}\ldots{}{]} Presidia a Junta da Fazenda e,
  quando havia relação, era o governador dela; onde simples junta de
  justiça, era dela o presidente'' (Francisco Adolfo de Varnhagen.
  \emph{História Geral do Brasil.} São Paulo, s/e, 1952, tomo \textsc{iv}, p.
  289).}, e convidou-a a ir falar ao primo. A mísera acreditou; ele
levou-a à Casa Verde e encerrou-a na galeria dos alucinados.

A notícia desta aleivosia\footnote{Perfídia, traição.} do ilustre
Bacamarte lançou o terror à alma da população. Ninguém queria acabar de
crer, que, sem motivo, sem inimizade, o alienista trancasse na Casa
Verde uma senhora perfeitamente ajuizada, que não tinha outro crime
senão o de interceder por um infeliz. Comentava-se o caso nas esquinas,
nos barbeiros; edificou-se um romance, umas finezas namoradas que o
alienista outrora dirigira à prima do Costa, a indignação do Costa e o
desprezo da prima. E daí a vingança. Era claro. Mas a austeridade do
alienista, a vida de estudos que ele levava, pareciam desmentir uma tal
hipótese. Histórias! Tudo isso era naturalmente a capa do
velhaco\footnote{Falsa aparência do espertalhão.}. E um dos mais
crédulos chegou a murmurar que sabia de outras coisas, não as dizia, por
não ter certeza plena, mas sabia, quase que podia jurar.

---Você, que é íntimo dele, não nos podia dizer o que há, o que houve,
que motivo\ldots{}

Crispim Soares derretia-se todo. Esse interrogar da gente inquieta e
curiosa, dos amigos atônitos, era para ele uma consagração pública. Não
havia duvidar; toda a povoação sabia enfim que o privado do alienista
era ele, Crispim, o boticário, o colaborador do grande homem e das
grandes coisas; daí a corrida à botica. Tudo isso dizia o carão jucundo
e o riso discreto do boticário, o riso e o silêncio, porque ele não
respondia nada; um, dois, três monossílabos, quando muito, soltos,
secos, encapados no fiel sorriso constante e miúdo, cheio de mistérios
científicos, que ele não podia, sem desdouro nem perigo, desvendar a
nenhuma pessoa humana.

--- ``Há coisa,'' pensavam os mais desconfiados.

Um desses limitou-se a pensá-lo, deu de ombros e foi embora. Tinha
negócios pessoais Acabava de construir uma casa suntuosa\footnote{Luxuosa,
  opulenta.}. Só a casa bastava para deter a chamar toda a gente; mas
havia mais, --- a mobília, que ele mandara vir da Hungria e da Holanda,
segundo contava, e que se podia ver do lado de fora, porque as janelas
viviam abertas, --- e o jardim, que era uma obra-prima de arte e de
gosto. Esse homem, que enriquecera no fabrico de albardas\footnote{Sela
  para animais de carga.}, tinha tido sempre o sonho de uma casa
magnífica, jardim pomposo, mobília rara. Não deixou o negócio das
albardas, mas repousava dele na contemplação da casa nova, a primeira de
Itaguaí, mais grandiosa do que a Casa Verde, mais nobre do que a da
câmara. Entre a gente ilustre da povoação havia choro e ranger de
dentes, quando se pensava, ou se falava, ou se louvava a casa do
albardeiro, --- um simples albardeiro, Deus do céu!

--- Lá está ele embasbacado, diziam os transeuntes, de manhã.

De manhã, com efeito, era costume do Mateus estatelar-se, no meio do
jardim, com os olhos na casa, namorado\footnote{Grafado desta forma no
  original (e não ``enamorado'').}, durante uma longa hora, até que
vinham chamá-lo para almoçar. Os vizinhos, embora o cumprimentassem com
certo respeito, riam-se por trás dele, que era um gosto. Um desses
chegou a dizer que o Mateus seria muito mais econômico, e estaria
riquíssimo, se fabricasse as albardas para si mesmo; epigrama\footnote{Dito
  mordaz, satírico.} ininteligível, mas que fazia rir às bandeiras
despregadas\footnote{Gargalhar, rir sem controle.}.

--- Agora lá está o Mateus a ser contemplado, diziam à tarde.

A razão deste outro dito era que, de tarde, quando as famílias safam a
passeio (jantavam cedo) usava o Mateus postar-se à janela, bem no
centro, vistoso, sobre um fundo escuro, trajado de branco, atitude
senhoril, e assim ficava duas e três horas até que anoitecia de todo.
Pode crer-se que a intenção do Mateus era ser admirado e invejado, posto
que ele não a confessasse a nenhuma pessoa, nem ao boticário, nem ao
padre Lopes, seus grandes amigos. E entretanto não foi outra a alegação
do boticário, quando o alienista lhe disse que o albardeiro talvez
padecesse do amor das pedras, mania que ele Bacamarte descobrira e
estudava desde algum tempo. Aquilo de contemplar a casa\ldots{}

--- Não, senhor, acudiu vivamente Crispim Soares.

--- Não?

--- Há de perdoar-me, mas talvez não saiba que ele de manhã examina a
obra, não a admira; de tarde, são os outros que o admiram a ele e à
obra. --- E contou o uso do albardeiro, todas as tardes, desde cedo até
o cair da noite.

Uma volúpia científica alumiou os olhos de Simão Bacamarte. Ou ele não
conhecia todos os costumes do albardeiro, ou nada mais quis,
interrogando o Crispim, do que confirmar alguma notícia incerta ou
suspeita vaga. A explicação satisfê-lo; mas como tinha as alegrias
próprias de um sábio, concentradas, nada viu o boticário que fizesse
suspeitar uma intenção sinistra. Ao contrário, era de tarde, e o
alienista pediu-lhe o braço para irem a passeio. Deus! era a primeira
vez que Simão Bacamarte dava ao seu privado tamanha honra; Crispim ficou
trêmulo, atarantado, disse que sim, que estava pronto. Chegaram duas ou
três pessoas de fora, Crispim mandou-as mentalmente a todos os diabos;
não só atrasavam o passeio, como podia acontecer que Bacamarte elegesse
alguma delas, para acompanhá-lo, e o dispensasse a ele. Que impaciência!
que aflição! Enfim, saíram. O alienista guiou para os lados da casa do
albardeiro, viu-o à janela, passou cinco, seis vezes por diante,
devagar, parando, examinando as atitudes, a expressão do rosto. O pobre
Mateus, apenas notou que era objeto da curiosidade ou admiração do
primeiro vulto de Itaguaí, redobrou de expressão, deu outro relevo às
atitudes\ldots{} Triste! triste! não fez mais do que condenar-se; no dia
seguinte, foi recolhido à Casa Verde.

--- A Casa Verde é um cárcere privado, disse um médico sem clínica.

Nunca uma opinião pegou e grassou\footnote{Espalhou-se.} tão
rapidamente. Cárcere privado: eis o que se repetia de norte a sul e de
leste a oeste de Itaguaí, --- a medo, é verdade, porque durante a semana
que se seguiu à captura do pobre Mateus, vinte e tantas pessoas, ---
duas ou três de consideração, --- foram recolhidas à Casa Verde. O
alienista dizia que só eram admitidos os casos patológicos, mas pouca
gente lhe dava crédito. Sucediam-se as versões populares. Vingança,
cobiça de dinheiro, castigo de Deus, monomania do próprio médico, plano
secreto do Rio de Janeiro com o fim de destruir em Itaguaí qualquer
germe de prosperidade que viesse a brotar, arvorecer, florir, com
desdouro e míngua daquela cidade, mil outras explicações, que não
explicavam nada, tal era o produto diário da imaginação pública.

Nisto chegou do Rio de Janeiro a esposa do alienista, a tia, a mulher do
Crispim Soares, e toda a mais comitiva, --- ou quase toda ---, que
algumas semanas antes partira de Itaguaí. O alienista foi recebê-la, com
o boticário, o padre Lopes, os vereadores, e vários outros magistrados.
O momento em que D.\,Evarista pôs os olhos na pessoa do marido é
considerado pelos cronistas do tempo como um dos mais sublimes da
história moral dos homens, e isto pelo contraste das duas naturezas,
ambas extremas, ambas egrégias. D.\,Evarista soltou um grito, balbuciou
uma palavra e atirou-se ao consorte, de um gesto que não se pode melhor
definir do que comparando-o a uma mistura de onça e rola. Não assim o
ilustre Bacamarte; frio como um diagnóstico, sem desengonçar por um
instante a rigidez científica, estendeu os braços à dona que caiu neles,
e desmaiou. Curto incidente; ao cabo de dois minutos, D.\,Evarista
recebia os cumprimentos dos amigos e o préstito\footnote{Procissão,
  cortejo.} punha-se em marcha.

D.\,Evarista era a esperança de Itaguaí; contava-se com ela para minorar
o flagelo da Casa Verde. Daí as aclamações públicas, a imensa gente que
atulhava as ruas, as flâmulas\footnote{Bandeira.}, as flores e damascos
às janelas. Com o braço apoiado no do padre Lopes, --- porque o eminente
Bacamarte confiara a mulher ao vigário, e acompanhava-os a passo
meditativo, --- D.\,Evarista voltava a cabeça a um lado e outro, curiosa,
inquieta, petulante. O vigário indagava do Rio de Janeiro, que ele não
vira desde o vice-reinado anterior\footnote{Referência ao vice-reinado
  de Luís de Almeida Portugal Soares de Alarcão d'Eça e Melo Silva
  Mascarenhas, 5º conde de Avintes e 2º Marquês do Lavradio, entre 1769
  e 1778 --- o que corrobora a localização histórica dos eventos
  transcorridos na novela.}; e D.\,Evarista respondia, entusiasmada, que
era a coisa mais bela que podia haver no mundo. O Passeio Público estava
acabado\footnote{O Passeio Público do Rio de Janeiro foi construído
  entre 1789 e 1793, durante o vice-reinado de Luís de Vasconcelos e
  Sousa. Representou profunda alteração na paisagem, já que foi
  construído sobre o aterro da Lagoa do Boqueirão.}, um paraíso onde ela
fora muitas vezes, e a Rua das Belas Noites\footnote{Atualmente, Rua das
  Marrecas.}, o chafariz das Marrecas\ldots{} Ah! o chafariz das
Marrecas! Eram mesmo marrecas, --- feitas de metal e despejando água
pela boca fora. Uma coisa galantíssima\footnote{O Chafariz das Marrecas
  era a atração principal do Passeio Público. A escultura, a cargo do
  célebre Mestre Valentim, foi inaugurada em 1783. O gosto de Dona
  Evarista é duvidoso e revela seu deslumbramento não com a cidade, mas
  com o fato de ter visitado um local que estava em evidência.}. O
vigário dizia que sim, que o Rio de Janeiro\footnote{Como a novela se
  passa no final do século \textsc{xviii}, o narrador se refere constantemente a
  Luís de Vasconcelos e Sousa (1742--1809), 4º conde de Figueiró --- o
  décimo segundo vice-rei do Brasil, entre 1778 e 1790, e também
  governador do Rio de Janeiro, cidade-sede do reino. De acordo com o
  Francisco Adolfo de Varnhagen, Vasconcelos distinguia-se pela
  ``moderação'' e ``prudência''. Sob seu governo, construiu-se o Passeio
  Público e se instituiu a Sociedade Literária (mais tarde, extinta). No
  plano político, reprimiu violentamente a Conjuração Mineira (entre
  1789 e 1792), o que resultou no enforcamento do alferes Tiradentes,
  durante o vice-reinado posterior.} devia estar agora muito mais
bonito. Se já o era noutro tempo! Não admira, maior do que Itaguaí, e de
mais a mais sede do governo\ldots{} Mas não se pode dizer que Itaguaí
fosse feio; tinha belas casas, a casa do Mateus, a Casa Verde\ldots{}

--- A propósito de Casa Verde, disse o padre Lopes escorregando
habilmente para o assunto da ocasião, a senhora vem achá-la muito cheia
de gente.

--- Sim?

--- É verdade. Lá está o Mateus\ldots{}

--- O albardeiro?

--- O albardeiro; está o Costa, a prima do Costa, e Fulano, e Sicrano,
e\ldots{}

---Tudo isso doido?

--- Ou quase doido, obtemperou o padre.

--- Mas então?

O vigário derreou os cantos da boca, à maneira de quem não sabe nada ou
não quer dizer tudo; resposta vaga, que se não pode repetir a outra
pessoa por falta de texto. D.\,Evarista achou realmente extraordinário
que toda aquela gente ensandecesse; um ou outro, vá; mas todos?
Entretanto custava-lhe duvidar; o marido era um sábio, não recolheria
ninguém à Casa Verde sem prova evidente de loucura.

--- Sem dúvida\ldots{} sem dúvida\ldots{} ia pontuando o vigário.

Três horas depois, cerca de cinquenta convivas\footnote{Comensais.}
sentavam-se em volta da mesa de Simão Bacamarte; era o jantar das
boas-vindas. D.\,Evarista foi o assunto obrigado dos brindes, discursos,
versos de toda a casta, metáforas, amplificações, apólogos\footnote{Metáfora
  é a figura de linguagem em que um termo é substituído por outro. A
  amplificação relaciona-se ao desdobramento de um tema, retomado com
  maior vigor ao longo do discurso. Apólogo é narrativa ficcional de
  premissa moral.}. Ela era a esposa do novo Hipócrates\footnote{Hipócrates
  é considerado o pai da medicina.}, a musa da ciência, anjo, divina,
aurora, caridade, vida, consolação; trazia nos olhos duas estrelas,
segundo a versão modesta de Crispim Soares, e dois sóis, no conceito de
um vereador. O alienista ouvia essas coisas um tanto enfastiado, mas sem
visível impaciência. Quando muito dizia ao ouvido da mulher, que a
retórica\footnote{Arte da persuasão. Neste caso, a palavra é empregada
  de modo pejorativo pelo médico.} permitia tais arrojos sem
significação. D.\,Evarista fazia esforços para aderir a esta opinião do
marido; mas, ainda descontando três quartas partes das louvaminhas,
ficava muito com que enfunar-lhe\footnote{Inflar-se, envaidecer-se.} a
alma. Um dos oradores, por exemplo, Martim Brito, rapaz de vinte e cinco
anos, pintalegrete\footnote{Presunçoso, vaidoso, exibido.} acabado,
curtido de namoros e aventuras, declamou um discurso em que o nascimento
de D.\,Evarista era explicado pelo mais singular dos reptos\footnote{Desafio,
  enfrentamento.}. ``Deus, disse ele, depois de dar ao universo o homem
e a mulher, esse diamante e essa pérola da coroa divina (e o orador
arrastava triunfalmente esta frase de uma ponta a outra da mesa), Deus
quis vencer a Deus, e criou D.\,Evarista''.

D.\,Evarista baixou os olhos com exemplar modéstia. Duas senhoras,
achando a cortesanice excessiva e audaciosa, interrogaram os olhos do
dono da casa; e, na verdade, o gesto do alienista pareceu-lhes nublado
de suspeitas, de ameaças e, provavelmente, de sangue. O atrevimento foi
grande, pensaram as duas damas. E uma e outra pediam a Deus que
removesse qualquer episódio trágico, --- ou que o adiasse, ao menos para
o dia seguinte. Sim, que o adiasse. Uma delas, a mais piedosa, chegou a
admitir, consigo mesma, que D.\,Evarista não merecia nenhuma
desconfiança, tão longe estava de ser atraente ou bonita. Uma simples
água-morna. Verdade é que, se todos os gostos fossem iguais, o que seria
do amarelo? E esta ideia fê-la tremer outra vez, embora menos; menos,
porque o alienista sorria agora para o Martim Brito e, levantados todos,
foi ter com ele e falou-lhe do discurso. Não lhe negou que era um
improviso brilhante, cheio de rasgos magníficos. Seria dele mesmo a
ideia relativa ao nascimento de D.\,Evarista, ou tê-la-ia encontrado em
algum autor que?\ldots{} Não, senhor; era dele mesmo; achou-a naquela
ocasião e parecera-lhe adequada a um arroubo oratório. De resto, suas
ideias eram antes arrojadas do que ternas ou jocosas\footnote{Divertida,
  cômica.}. Dava para o épico\footnote{Gênero elevado, assim como a
  Tragédia --- segundo a \emph{Poética} de Aristóteles (384--322 a.C.).}.
Uma vez, por exemplo, compôs uma ode à queda do Marquês de
Pombal\footnote{Título concedido a Sebastião José de Carvalho e Melo,
  braço direito do rei Dom José \textsc{i}, entre 1750 e 1777.}, em que dizia que
esse ministro era o ``dragão aspérrimo\footnote{Asperíssimo, muito
  áspero.} do Nada'', esmagado pelas ``garras vingadoras do Todo''; e
assim outras, mais ou menos fora do comum; gostava das ideias sublimes e
raras, das imagens grandes e nobres\ldots{}

--- Pobre moço! pensou o alienista. E continuou consigo: --- Trata-se de
um caso de lesão cerebral; fenômeno sem gravidade, mas digno de
estudo\ldots{}

D.\,Evarista ficou estupefata quando soube, três dias depois, que o
Martim Brito fora alojado na Casa Verde. Um moço que tinha ideias tão
bonitas! As duas senhoras atribuíram o ato a ciúmes do alienista. Não
podia ser outra coisa; realmente, a declaração do moço fora audaciosa de
mais.

Ciúmes? Mas como explicar que, logo em seguida, fossem recolhidos José
Borges do Couto Leme, pessoa estimável, o Chico das Cambraias,
folgazão\footnote{Brincalhão, galhofeiro.} emérito, o escrivão Fabrício
e ainda outros? O terror acentuou-se. Não se sabia já quem estava são,
nem quem estava doido. As mulheres, quando os maridos saíam, mandavam
acender uma lamparina a Nossa Senhora; e nem todos os maridos eram
valorosos; alguns não andavam fora sem um ou dois capangas.
Positivamente o terror. Quem podia, emigrava. Um desses fugitivos chegou
a ser preso a duzentos passos da vila. Era um rapaz de trinta anos,
amável, conversado, polido, tão polido que não cumprimentava alguém sem
levar o chapéu ao chão; na rua, acontecia-lhe correr uma distância de
dez a vinte braças\footnote{Antiga medida correspondente a dois braços
  abertos (aproximadamente 2 metros).} para ir apertar a mão a um homem
grave\footnote{Sisudo, comportado, digno.}, a uma senhora, às vezes a um
menino, como acontecera ao filho do juiz de fora. Tinha a vocação das
cortesias. De resto, devia as boas relações da sociedade, não só aos
dotes pessoais, que eram raros, como à nobre tenacidade com que nunca
desanimava diante de uma, duas, quatro, seis recusas, caras feias, etc.
O que acontecia era que, uma vez entrado numa casa, não a deixava mais,
nem os da casa o deixavam a ele, tão gracioso era o Gil Bernardes. Pois
o Gil Bernardes, apesar de se saber estimado, teve medo quando lhe
disseram um dia, que o alienista o trazia de olho; na madrugada seguinte
fugiu da vila, mas foi logo apanhado e conduzido à Casa Verde.

--- Devemos acabar com isto!

--- Não pode continuar!

--- Abaixo a tirania!

--- Déspota! violento! Golias!\footnote{Gigante violento que teria sido
  derrotado pelo rei Davi, em acordo com o livro de Samuel, na
  \emph{Bíblia}.}

Não eram gritos na rua, eram suspiros em casa, mas não tardava a hora
dos gritos. O terror crescia; avizinhava-se a rebelião. A ideia de uma
petição ao governo para que Simão Bacamarte fosse capturado e deportado,
andou por algumas cabeças, antes que o barbeiro Porfírio a expendesse na
loja, com grandes gestos de indignação. Note-se, --- e essa é uma das
laudas mais puras desta sombria história --- note-se que o Porfírio,
desde que a Casa Verde começara a povoar-se tão extraordinariamente, viu
crescerem-lhe os lucros pela aplicação assídua de sanguessugas que dali
lhe pediam; mas o interesse particular, dizia ele, deve ceder ao
interesse público. E acrescentava: ---é preciso derrubar o tirano!
Note-se mais que ele soltou esse grito justamente no dia em que Simão
Bacamarte fizera recolher à Casa Verde um homem que trazia com ele uma
demanda, o Coelho.

--- Não me dirão em que é que o Coelho é doido? bradou o Porfírio.

E ninguém lhe respondia; todos repetiam que era um homem perfeitamente
ajuizado. A mesma demanda\footnote{Pendência processual.} que ele trazia
com o barbeiro, acerca de uns chãos da vila, era filha da obscuridade de
um alvará e não da cobiça ou ódio. Um excelente caráter o Coelho. Os
únicos desafeiçoados que tinha eram alguns sujeitos que, dizendo-se
taciturnos\footnote{Melancólico, silencioso.}, ou alegando andar com
pressa, mal o viam de longe dobravam as esquinas, entravam nas lojas,
etc. Na verdade, ele amava a boa palestra, a palestra comprida, gostada
a sorvos\footnote{Goles. Figurativamente, palestra largamente
  aproveitada, absorvida.} largos, e assim é que nunca estava só,
preferindo os que sabiam dizer duas palavras, mas não desdenhando os
outros. O padre Lopes, que cultivava o Dante, e era inimigo do Coelho,
nunca o via desligar-se de uma pessoa que não declamasse e emendasse
este trecho:

\begin{quote}
La bocca sollevò dal fiero pasto

Quel \emph{``seccatore''\ldots{}}\footnote{Versos que aparecem no
  33\textsuperscript{o} capítulo do ``Inferno'' de Dante Alighieri. O
  termo ``seccatore'', em itálico no original, mostra que a substituição
  de ``peccator'' era intencional. (A tradução dos versos originais
  seria: ``A boca suspendeu do fero alimento / Aquele pecador'').}
\end{quote}

mas uns sabiam do ódio do padre, e outros pensavam que isto era uma
oração em latim.

\chapter{\textsc{vi}. A Rebelião}

Cerca de trinta pessoas ligaram-se ao barbeiro, redigiram e levaram uma
representação\footnote{Documento assinado coletivamente em que um
  indivíduo representa os demais signatários.} à câmara. A câmara
recusou aceitá-la, declarando que a Casa Verde era uma instituição
pública, e que a ciência não podia ser emendada por votação
administrativa, menos ainda por movimentos de rua.

--- Voltai ao trabalho, concluiu o presidente, é o conselho que vos
damos.

A irritação dos agitadores foi enorme. O barbeiro declarou que iam dali
levantar a bandeira da rebelião, e destruir a Casa Verde; que Itaguaí
não podia continuar a servir de cadáver aos estudos e experiências de um
déspota; que muitas pessoas estimáveis, algumas distintas, outras
humildes, mas dignas de apreço, jaziam nos cubículos da Casa Verde; que
o despotismo científico do alienista complicava-se do espírito de
ganância, visto que os loucos, ou supostos tais, não eram tratados de
graça: as famílias, e em falta delas a câmara, pagavam ao
alienista\ldots{}

--- É falso, interrompeu o presidente.

--- Falso?

--- Há cerca de duas semanas recebemos um ofício do ilustre médico, em
que nos declara que, tratando de fazer experiências de alto valor
psicológico, desiste do estipêndio votado pela câmara, bem como nada
receberá das famílias dos enfermos.

A notícia deste ato tão nobre, tão puro, suspendeu um pouco a alma dos
rebeldes. Seguramente o alienista podia estar em erro, mas nenhum
interesse alheio à ciência o instigava; e para demonstrar o erro era
preciso alguma coisa mais do que arruaças e clamores. Isto disse o
presidente, com aplauso de toda a câmara. O barbeiro, depois de alguns
instantes de concentração, declarou que estava investido de um mandato
público e não restituiria a paz a Itaguaí antes de ver por terra a Casa
Verde, --- ``essa Bastilha da razão humana'', --- expressão que ouvira a
um poeta local e que ele repetiu com muita ênfase. Disse, e a um sinal
todos saíram com ele.

Imagine-se a situação dos vereadores; urgia obstar ao ajuntamento, à
rebelião, à luta, ao sangue. Para acrescentar ao mal, um dos vereadores,
que apoiara o presidente, ouvindo agora a denominação dada pelo barbeiro
à Casa Verde --- ``Bastilha da razão humana'', --- achou-a tão elegante,
que mudou de parecer. Disse que entendia de bom aviso decretar alguma
medida que reduzisse a Casa Verde; e porque o presidente, indignado,
manifestasse em termos enérgicos o seu pasmo, o vereador fez esta
reflexão:

--- Nada tenho que ver com a ciência; mas, se tantos homens em quem
supomos juízo são reclusos por dementes, quem nos afirma que o alienado
não é o alienista?

Sebastião Freitas, o vereador dissidente, tinha o dom da palavra e falou
ainda por algum tempo com prudência\footnote{A prudência era considerada
  uma arte, pelo menos até o final do século \textsc{xviii}, período em que se
  passa a narrativa.}, mas com firmeza. Os colegas estavam atônitos; o
presidente pediu-lhe que, ao menos, desse o exemplo da ordem e do
respeito à lei, não aventasse as suas ideias na rua, para não dar corpo
e alma à rebelião, que era por ora um turbilhão de átomos dispersos.
Esta figura corrigiu um pouco o efeito da outra: Sebastião Freitas
prometeu suspender qualquer ação reservando-se o direito de pedir pelos
meios legais a redução da Casa Verde. E repetia consigo, namorado: ---
Bastilha da razão humana!

Entretanto, a arruaça crescia. Já não eram trinta, mas trezentas pessoas
que acompanhavam o barbeiro, cuja alcunha familiar deve ser mencionada,
porque ela deu o nome à revolta; chamavam-lhe o Canjica, --- e o
movimento ficou célebre com o nome de revolta dos Canjicas\footnote{Possível
  paródia da Conjuração Mineira.}. A ação podia ser restrita, --- visto
que muita gente, ou por medo, ou por hábitos de educação, não descia à
rua; mas o sentimento era unânime, ou quase unânime, e os trezentos que
caminhavam para a Casa Verde, --- dada a diferença de Paris a Itaguaí,
--- podiam ser comparados aos que tomaram a Bastilha\footnote{Nova
  alusão aos eventos relacionados à Revolução Francesa --- contemporânea
  da Conjuração Mineira ---, no final do século \textsc{xviii}.}.

D.\,Evarista teve notícia da rebelião antes que ela chegasse; veio
dar-lha\footnote{Contração de lhe + a (Veio lhe dar a notícia).} uma de
suas crias\footnote{Criadas.}. Ela provava nessa ocasião um vestido de
seda, --- um dos trinta e sete que trouxera do Rio de Janeiro, --- e não
quis crer.

--- Há de ser alguma patuscada, dizia ela mudando a posição de um
alfinete. Benedita, vê se a barra está boa.

--- Está, sinhá, respondia a mucama de cócoras no chão, está boa. Sinhá
vira um bocadinho. Assim. Está muito boa.

--- Não é patuscada, não, senhora; eles estão gritando: --- Morra o
Dr.\,Bacamarte! o tirano! dizia o moleque assustado.

--- Cala a boca, tolo! Benedita, olha aí do lado esquerdo; não parece
que a costura está um pouco enviesada? A risca azul não segue até
abaixo; está muito feio assim; é preciso descoser para ficar igualzinho
e\ldots{}

--- Morra o Dr.\,Bacamarte! morra o tirano! uivaram fora trezentas vozes.
Era a rebelião que desembocava na Rua Nova.

D.\,Evarista ficou sem pinga de sangue\footnote{Exangue, pálida, sem cor.}.
No primeiro instante não deu um passo, não fez um gesto; o terror
petrificou-a. A mucama correu instintivamente para a porta do fundo.
Quanto ao moleque, a quem D.\,Evarista não dera crédito, teve um instante
de triunfo súbito, um certo movimento súbito, imperceptível, entranhado,
de satisfação moral, ao ver que a realidade vinha jurar por ele.

--- Morra o alienista! bradavam as vozes mais perto.

D.\,Evarista, se não resistia facilmente às comoções de prazer, sabia
entestar\footnote{Confrontar.} com os momentos de perigo. Não desmaiou;
correu à sala interior onde o marido estudava. Quando ela ali entrou,
precipitada, o ilustre médico escrutava um texto de Averróis;\footnote{Averróis
  (1126--1198), filósofo e médico árabe que traduziu e comentou
  Aristóteles.} os olhos dele, empanados pela cogitação, subiam do livro
ao teto e baixavam do teto ao livro, cegos para a realidade exterior,
videntes para os profundos trabalhos mentais. D.\,Evarista chamou pelo
marido duas vezes, sem que ele lhe desse atenção; à terceira, ouviu e
perguntou-lhe o que tinha, se estava doente.

--- Você não ouve estes gritos? perguntou a digna esposa em lágrimas.

O alienista atendeu então; os gritos aproximavam-se, terríveis,
ameaçadores; ele compreendeu tudo. Levantou-se da cadeira de espaldar em
que estava sentado, fechou o livro, e, a passo firme e tranquilo, foi
depositá-lo na estante. Como a introdução do volume desconsertasse um
pouco a linha dos dois tomos contíguos, Simão Bacamarte cuidou de
corrigir esse defeito mínimo, e, aliás, interessante. Depois disse à
mulher que se recolhesse, que não fizesse nada.

--- Não, não, implorava a digna senhora, quero morrer ao lado de você\ldots{}

Simão Bacamarte teimou que não, que não era caso de morte; e ainda que o
fosse, intimava-lhe em nome da vida que ficasse. A infeliz dama curvou a
cabeça, obediente e chorosa.

--- Abaixo a Casa Verde! bradavam os Canjicas.

O alienista caminhou para a varanda da frente, e chegou ali no momento
em que a rebelião também chegava e parava, defronte, com as suas
trezentas cabeças rutilantes de civismo e sombrias de desespero. ---
Morra! morra! bradaram de todos os lados, apenas o vulto do alienista
assomou na varanda. Simão Bacamarte fez um sinal pedindo para falar; os
revoltosos cobriram-lhe a voz com brados de indignação. Então, o
barbeiro, agitando o chapéu, a fim de impor silêncio à turba, conseguiu
aquietar os amigos, e declarou ao alienista que podia falar, mas
acrescentou que não abusasse da paciência do povo como fizera até então.

--- Direi pouco, ou até não direi nada, se for preciso. Desejo saber
primeiro o que pedis.

--- Não pedimos nada, replicou fremente o barbeiro; ordenamos que a Casa
Verde seja demolida, ou pelo menos despojada dos infelizes que lá estão.

--- Não entendo.

--- Entendeis bem, tirano; queremos dar liberdade às vítimas do vosso
ódio, capricho, ganância\ldots{}

O alienista sorriu, mas o sorriso desse grande homem não era coisa
visível aos olhos da multidão; era uma contração leve de dois ou três
músculos, nada mais. Sorriu e respondeu:

--- Meus senhores, a ciência é coisa séria, e merece ser tratada com
seriedade. Não dou razão dos meus atos de alienista a ninguém, salvo aos
mestres e a Deus. Se quereis emendar a administração da Casa Verde,
estou pronto a ouvir-vos; mas se exigis que me negue a mim mesmo, não
ganhareis nada. Poderia convidar alguns de vós, em comissão dos outros,
a vir ver comigo os loucos reclusos; mas não o faço, porque seria
dar-vos razão do meu sistema, o que não farei a leigos nem a rebeldes.

Disse isto o alienista, e a multidão ficou atônita; era claro que não
esperava tanta energia e menos ainda tamanha serenidade. Mas o assombro
cresceu de ponto quando o alienista, cortejando a multidão com muita
gravidade, deu-lhe as costas e retirou-se lentamente para dentro. O
barbeiro tornou logo a si e, agitando o chapéu, convidou os amigos à
demolição da Casa Verde; poucas vozes e frouxas lhe responderam. Foi
nesse momento decisivo que o barbeiro sentiu despontar em si a ambição
do governo; pareceu-lhe então que, demolindo a Casa Verde e
derrocando\footnote{Demolir, aniquilar.} a influência do alienista,
chegaria a apoderar-se da câmara, dominar as demais autoridades e
constituir-se senhor de Itaguaí. Desde alguns anos que ele forcejava por
ver o seu nome incluído nos pelouros\footnote{Urna circular utilizada em
  votações da câmara.} para o sorteio dos vereadores, mas era recusado
por não ter uma posição compatível com tão grande cargo. A ocasião era
agora ou nunca. Demais fora tão longe na arruaça que a derrota seria a
prisão, ou talvez a forca, ou o degredo\footnote{Prisão, força e degredo
  são penas coincidentes com aquelas aplicadas aos revoltosos de Vila
  Rica, durante a chamada Conjuração Mineira.}. Infelizmente, a resposta
do alienista diminuíra o furor dos sequazes\footnote{Partidário, adepto.}.
O barbeiro, logo que o percebeu, sentiu um impulso de indignação, e quis
bradar-lhes: --- Canalhas! covardes! --- mas conteve-se e rompeu deste
modo:

--- Meus amigos, lutemos até o fim! A salvação de Itaguaí está nas
vossas mãos dignas e heroicas. Destruamos o cárcere de vossos filhos e
pais, de vossas mães e irmãs, de vossos parentes e amigos, e de vós
mesmos. Ou morrereis a pão e água, talvez a chicote, na masmorra daquele
indigno.

A multidão agitou-se, murmurou, bradou, ameaçou, congregou-se toda em
derredor do barbeiro. Era a revolta que tornava a si da ligeira síncope
e ameaçava arrasar a Casa Verde.

--- Vamos! bradou Porfírio agitando o chapéu.

--- Vamos! repetiram todos.

Deteve-os um incidente: era um corpo de dragões\footnote{Infantaria
  montada de prestígio que atuou durante os séculos \textsc{xviii} e \textsc{xix}.} que, a
marche-marche, entrava na Rua Nova.

\chapter{\textsc{vii}. O Inesperado}

Chegados os dragões em frente aos Canjicas, houve um instante de
estupefação; os Canjicas não queriam crer que a força pública fosse
mandada contra eles; mas o barbeiro compreendeu tudo e esperou. Os
dragões pararam, o capitão intimou à multidão que se dispersasse; mas,
conquanto uma parte dela estivesse inclinada a isso, a outra parte
apoiou fortemente o barbeiro, cuja resposta consistiu nestes termos
alevantados:

--- Não nos dispersaremos. Se quereis os nossos cadáveres, podeis
tomá-los; mas só os cadáveres; não levareis a nossa honra, o nosso
crédito, os nossos direitos, e com eles a salvação de Itaguaí.

Nada mais imprudente do que essa resposta do barbeiro; e nada mais
natural. Era a vertigem das grandes crises. Talvez fosse também um
excesso de confiança na abstenção das armas por parte dos dragões;
confiança que o capitão dissipou logo, mandando carregar sobre os
Canjicas. O momento foi indescritível. A multidão urrou furiosa; alguns,
trepando às janelas das casas ou correndo pela rua fora, conseguiram
escapar; mas a maioria ficou, bufando de cólera, indignada, animada pela
exortação do barbeiro. A derrota dos Canjicas estava iminente, quando um
terço dos dragões, --- qualquer que fosse o motivo, as crônicas não o
declaram, --- passou subitamente para o lado da rebelião. Este
inesperado reforço deu alma aos Canjicas, ao mesmo tempo que lançou o
desânimo às fileiras da legalidade. Os soldados fiéis não tiveram
coragem de atacar os seus próprios camaradas\footnote{Possível alusão de
  Machado à forma de tratamento com que os adeptos do Socialismo se
  referiam aos companheiros de trabalho, a partir da Revolução de 1848.
  O termo aparece no \emph{Manifesto Comunista}, de Marx e Engels.}, e,
um a um, foram passando para eles, de modo que ao cabo de alguns
minutos, o aspecto das coisas era totalmente outro. O capitão estava de
um lado, com alguma gente, contra uma massa compacta que o ameaçava de
morte. Não teve remédio, declarou-se vencido e entregou a espada ao
barbeiro.

A revolução triunfante não perdeu um só minuto; recolheu os feridos às
casas próximas, e guiou para a câmara. Povo e tropa fraternizavam, davam
vivas a el-rei, ao vice-rei, a Itaguaí, ao ``ilustre Porfírio''. Este ia
na frente, empunhando tão destramente a espada, como se ela fosse apenas
uma navalha um pouco mais comprida. A vitória cingia-lhe a fronte de um
nimbo\footnote{Auréola.} misterioso. A dignidade de governo começava a
enrijar-lhe os quadris.

Os vereadores, às janelas, vendo a multidão e a tropa, cuidaram que a
tropa capturara a multidão, e sem mais exame, entraram e votaram uma
petição ao vice-rei\footnote{Nova menção ao vice-rei Luís de Vasconcelos
  e Sousa.} para que mandasse dar um mês de soldo\footnote{Remuneração
  paga ao soldado.} aos dragões, ``cujo denodo\footnote{Ousadia,
  empenho, coragem.} salvou Itaguaí do abismo a que o tinha lançado uma
cáfila\footnote{Coletivo de camelos.} de rebeldes''. Esta frase foi
proposta por Sebastião Freitas, o vereador dissidente, cuja defesa dos
Canjicas tanto escandalizara os colegas. Mas bem depressa a ilusão se
desfez. Os vivas ao barbeiro, os morras aos vereadores e ao alienista
vieram dar-lhes notícia da triste realidade. O presidente não desanimou:
--- qualquer que seja a nossa sorte, disse ele, lembremo-nos de que
estamos ao serviço de Sua Majestade e do povo. --- Sebastião insinuou
que melhor se poderia servir à coroa e à vila saindo pelos fundos e indo
conferenciar com o juiz de fora, mas toda a câmara rejeitou esse
alvitre\footnote{Arbítrio.}.

Daí a nada o barbeiro, acompanhado de alguns de seus tenentes, entrava
na sala da vereança, e intimava à câmara a sua queda. A câmara não
resistiu, entregou-se, e foi dali para a cadeia. Então os amigos do
barbeiro propuseram-lhe que assumisse o governo da vila, em nome de Sua
Majestade. Porfírio aceitou o encargo, embora não desconhecesse
(acrescentou) os espinhos que trazia; disse mais que não podia dispensar
o concurso dos amigos presentes; ao que eles prontamente anuíram. O
barbeiro veio à janela e comunicou ao povo essas resoluções, que o povo
ratificou, aclamando o barbeiro. Este tomou a denominação de ---
``Protetor da vila em nome de Sua Majestade e do povo''\footnote{Machado
  de Assis parodia o título conferido ao príncipe Dom Pedro \textsc{i}, em maio
  de 1822: ``Protetor e Defensor Perpétuo do Brasil''.}. ---
Expediram-se logo várias ordens importantes, comunicações oficiais do
novo governo, uma exposição minuciosa ao vice-rei, com muitos protestos
de obediência às ordens de Sua Majestade; finalmente, uma proclamação ao
povo, curta, mas enérgica:

\begin{quote}
``Itaguaienses!

Uma câmara corrupta e violenta conspirava contra os interesses de Sua
Majestade e do povo. A opinião pública tinha-a condenado; um punhado de
cidadãos, fortemente apoiados pelos bravos dragões de Sua Majestade,
acaba de a dissolver ignominiosamente\footnote{Vergonhosamente,
  desonradamente.}, e por unânime consenso da vila, foi-me confiado o
mando supremo, até que Sua Majestade se sirva ordenar o que parecer
melhor ao seu real serviço. Itaguaienses! não vos peço se não que me
rodeeis de confiança, que me auxilieis em restaurar a paz e a fazenda
pública, tão desbaratada pela câmara que ora findou às vossas mãos.
Contai com o meu sacrifício, e ficai certos de que a coroa será por nós.

O Protetor da vila em nome de Sua Majestade e do povo
\end{quote}

\textsc{porfírio caetano das neves}''.

Toda a gente advertiu no absoluto silêncio desta proclamação acerca da
Casa Verde; e, segundo uns, não podia haver mais vivo indício dos
projetos tenebrosos do barbeiro. O perigo era tanto maior quanto que, no
meio mesmo desses graves sucessos, o alienista metera na Casa Verde umas
sete ou oito pessoas, entre elas duas senhoras, sendo um dos homens
aparentado com o Protetor. Não era um repto, um ato intencional; mas
todos o interpretaram dessa maneira, e a vila respirou com a esperança
de que o alienista dentro de vinte e quatro horas estaria a ferros, e
destruído o terrível cárcere.

O dia acabou alegremente. Enquanto o arauto\footnote{Mensageiro,
  porta-voz.} da matraca ia recitando de esquina em esquina a
proclamação, o povo espalhava-se nas ruas e jurava morrer em defesa do
ilustre Porfírio. Poucos gritos contra a Casa Verde, prova de confiança
na ação do governo. O barbeiro faz expedir um ato declarando feriado
aquele dia, e entabulou negociações com o vigário para a celebração de
um \emph{Te-Deum}, tão conveniente era aos olhos dele a conjunção do
poder temporal com o espiritual; mas o padre Lopes recusou abertamente o
seu concurso.

--- Em todo caso, Vossa Reverendíssima não se alistará entre os inimigos
do governo? disse-lhe o barbeiro, dando à fisionomia um aspecto
tenebroso.

Ao que o padre Lopes respondeu, sem responder:

--- Como alistar-me, se o novo governo não tem inimigos?

O barbeiro sorriu; era a pura verdade. Salvo o capitão, os vereadores e
os principais\footnote{Sujeito pertencente à elite local. Denominação
  que remonta à antiga ``homem-bom'', sujeito de posses e que exercia
  poder político e mando.} da vila, toda a gente o aclamava. Os mesmos
principais, se o não aclamavam, não tinham saído contra ele. Nenhum dos
almotacés\footnote{Inspetor de pesos e medidas, que fixava o valor dos
  alimentos.} deixou de vir receber as suas ordens. No geral, as
famílias abençoavam o nome daquele que ia enfim libertar Itaguaí da Casa
Verde e do terrível Simão Bacamarte.

\chapter{\textsc{viii}. As angústias do boticário}

Vinte e quatro horas depois dos sucessos narrados no capítulo anterior,
o barbeiro saiu do palácio do governo, --- foi a denominação dada à casa
da câmara, --- com dois ajudantes de ordens, e dirigiu-se à residência
de Simão Bacamarte. Não ignorava ele que era mais decoroso ao governo
mandá-lo chamar; o receio, porém, de que o alienista não obedecesse,
obrigou-o a parecer tolerante e moderado.

Não descrevo o terror do boticário ao ouvir dizer que o barbeiro ia à
casa do alienista. --- Vai prendê-lo, pensou ele. E redobraram-lhe as
angústias. Com efeito, a tortura moral do boticário naqueles dias de
revolução excede a toda a descrição possível. Nunca um homem se achou em
mais apertado lance: --- a privança do alienista chamava-o ao lado
deste, a vitória do barbeiro atraía-o ao barbeiro. Já a simples notícia
da sublevação\footnote{Revolta, levante, conjuração.} tinha-lhe sacudido
fortemente a alma, porque ele sabia a unanimidade do ódio ao alienista;
mas a vitória final foi também o golpe final. A esposa, senhora máscula,
amiga particular de D.\,Evarista, dizia que o lugar dele era ao lado de
Simão Bacamarte; ao passo que o coração lhe bradava que não, que a causa
do alienista estava perdida, e que ninguém, por ato próprio, se amarra a
um cadáver. Fê-lo Catão\footnote{Marco Pórcio Catão (234--149 a.C.), o
  Velho, que se tornou célebre como censor romano. A denominação é
  irônica, já que Hipócrates era grego e a cultura helenística
  constituía um dos alvos do romano Catão. A alcunha sugere que
  Bacamarte corresponderia a uma espécie de sábio e censor.}, é verdade,
\emph{sed victa Catoni}\footnote{Expressão latina extraída de um verso
  de Lucano (39--65 d.C): ``\emph{victrix causa diis placuit,} sed victa
  Catoni'' (A causa vencedora agradou aos deuses, mas a vencida a Catão)
  que integra o poema épico \emph{Farsália}, publicado séculos após a
  sua morte.}, pensava ele, relembrando algumas palestras habituais do
padre Lopes; mas Catão não se atou a uma causa vencida, ele era a
própria causa vencida, a causa da república; o seu ato, portanto, foi de
egoísta, de um miserável egoísta; minha situação é outra. Insistindo,
porém, a mulher, não achou Crispim Soares outra saída em tal crise senão
adoecer; declarou-se doente, e meteu-se na cama.

--- Lá vai o Porfírio à casa do Dr.\,Bacamarte, disse-lhe a mulher no dia
seguinte à cabeceira da cama; vai acompanhado de gente.

--- Vai prendê-lo, pensou o boticário.

Uma ideia traz outra; o boticário imaginou que, uma vez preso o
alienista, viriam também buscá-lo a ele, na qualidade de cúmplice. Esta
ideia foi o melhor dos vesicatórios\footnote{Que provoca bolhas,
  vesicante.}. Crispim Soares ergueu-se, disse que estava bom, que ia
sair; e apesar de todos os esforços e protestos da consorte, vestiu-se e
saiu. Os velhos cronistas são unânimes em dizer que a certeza de que o
marido ia colocar-se nobremente ao lado do alienista consolou
grandemente a esposa do boticário; e notam, com muita perspicácia, o
imenso poder moral de uma ilusão; porquanto, o boticário caminhou
resolutamente ao palácio do governo, não à casa do alienista. Ali
chegando, mostrou-se admirado de não ver o barbeiro, a quem ia
apresentar os seus protestos de adesão, não o tendo feito desde a
véspera por enfermo. E tossia com algum custo. Os altos funcionários que
lhe ouviam esta declaração, sabedores da intimidade do boticário com o
alienista, compreenderam toda a importância da adesão nova e trataram a
Crispim Soares com apurado carinho; afirmaram-lhe que o barbeiro não
tardava; Sua Senhoria tinha ido à Casa Verde, a negócio importante, mas
não tardava. Deram-lhe cadeira, refrescos, elogios; disseram-lhe que a
causa do ilustre Porfírio era a de todos os patriotas; ao que o
boticário ia repetindo que sim, que nunca pensara outra coisa, que isso
mesmo mandaria declarar Sua Majestade.

\chapter{\textsc{ix}. Dois lindos casos }

Não se demorou o alienista em receber o barbeiro; declarou-lhe que não
tinha meios de resistir, e portanto estava prestes a obedecer. Só uma
coisa pedia, é que o não constrangesse a assistir pessoalmente à
destruição da Casa Verde.

--- Engana-se Vossa Senhoria, disse o barbeiro depois de alguma pausa,
engana-se em atribuir ao governo intenções vandálicas. Com razão ou sem
ela, a opinião crê que a maior parte dos doidos ali metidos estão em seu
perfeito juízo, mas o governo reconhece que a questão é puramente
científica, e não cogita em resolver com posturas\footnote{Ordem emitida
  pela câmara municipal.} as questões científicas. Demais\footnote{Além
  disso, Ademais.}, a Casa Verde é uma instituição pública; tal a
aceitamos das mãos da câmara dissolvida. Há, entretanto, --- por força
que há de haver um alvitre intermédio que restitua o sossego ao espírito
público.

O alienista mal podia dissimular o assombro; confessou que esperava
outra coisa, o arrasamento do hospício, a prisão dele, o desterro, tudo,
menos\ldots{}

--- O pasmo de Vossa Senhoria, atalhou gravemente o barbeiro, vem de não
atender à grave responsabilidade do governo. O povo, tomado de uma cega
piedade que lhe dá em tal caso legítima indignação, pode exigir do
governo certa ordem de atos; mas este, com a responsabilidade que lhe
incumbe, não os deve praticar, ao menos integralmente, e tal é a nossa
situação. A generosa revolução que ontem derrubou uma câmara
vilipendiada e corrupta, pediu em altos brados o arrasamento da Casa
Verde; mas pode entrar no ânimo do governo eliminar a loucura? Não. E se
o governo não a pode eliminar, está ao menos apto para discriminá-la,
reconhecê-la? Também não; é matéria de ciência. Logo, em assunto tão
melindroso, o governo não pode, não deve, não quer dispensar o concurso
de Vossa Senhoria. O que lhe pede é que de certa maneira demos alguma
satisfação ao povo. Unamo-nos, e o povo saberá obedecer. Um dos alvitres
aceitáveis, se Vossa Senhoria não indicar outro, seria fazer retirar da
Casa Verde aqueles enfermos que estiverem quase curados, e bem assim os
maníacos de pouca monta, etc. Desse modo, sem grande perigo, mostraremos
alguma tolerância e benignidade.

--- Quantos mortos e feridos houve ontem no conflito? perguntou Simão
Bacamarte, depois de uns três minutos.

O barbeiro ficou espantado da pergunta, mas respondeu logo que onze
mortos e vinte e cinco feridos.

--- Onze mortos e vinte e cinco feridos! repetiu duas ou três vezes o
alienista.

E em seguida declarou que o alvitre lhe não parecia bom, mas que ele ia
catar algum outro, e dentro de poucos dias lhe daria resposta. E fez-lhe
várias perguntas acerca dos sucessos da véspera; ataque, defesa, adesão
dos dragões, resistência da câmara, etc., ao que o barbeiro ia
respondendo com grande abundância, insistindo principalmente no
descrédito em que a câmara caíra. O barbeiro confessou que o novo
governo não tinha ainda por si a confiança dos principais da vila, mas o
alienista podia fazer muito nesse ponto. O governo, concluiu o barbeiro,
folgaria se pudesse contar, não já com a simpatia senão com a
benevolência do mais alto espírito de Itaguaí, e seguramente do reino.
Mas nada disso alterava a nobre e austera fisionomia daquele grande
homem, que ouvia calado, sem desvanecimento nem modéstia, mas impassível
como um deus de pedra.

--- Onze mortos e vinte e cinco feridos, repetiu o alienista, depois de
acompanhar o barbeiro até a porta. Eis aí dois lindos casos de doença
cerebral. Os sintomas de duplicidade e descaramento deste barbeiro são
positivos. Quanto à toleima dos que o aclamaram não é preciso outra
prova além dos onze mortos e vinte e cinco feridos --- dois lindos
casos!

--- Viva o ilustre Porfírio! bradaram umas trinta pessoas que aguardavam
o barbeiro à porta.

O alienista espiou pela janela e ainda ouviu este resto de uma pequena
fala do barbeiro às trinta pessoas que o aclamavam.

--- \ldots{} porque eu velo, podeis estar certos disso, eu velo pela
execução das vontades do povo. Confiai em mim; e tudo se fará pela
melhor maneira. Só vos recomendo ordem. E ordem, meus amigos, é a base
do governo\ldots{}

--- Viva o ilustre Porfírio! bradaram as trinta vozes, agitando os
chapéus.

--- Dois lindos casos! murmurou o alienista.

\chapter{\textsc{x.} A Restauração\footnote{Nova alusão aos episódios transcorridos
  durante a Revolução Francesa.}}

Dentro de cinco dias, o alienista meteu na Casa Verde cerca de cinquenta
aclamadores do novo governo. O povo indignou-se. O governo, atarantado,
não sabia reagir. João Pina, outro barbeiro, dizia abertamente nas ruas,
que o Porfírio estava ``vendido ao ouro de Simão Bacamarte'', frase que
congregou em torno de João Pina a gente mais resoluta da vila. Porfírio,
vendo o antigo rival da navalha à testa da insurreição, compreendeu que
a sua perda era irremediável, se não desse um grande golpe; expediu dois
decretos, um abolindo a Casa Verde, outro desterrando o alienista. João
Pina mostrou claramente, com grandes frases, que o ato de Porfírio era
um simples aparato, um engodo, em que o povo não devia crer. Duas horas
depois caía Porfírio ignominiosamente, e João Pina assumia a difícil
tarefa do governo. Como achasse nas gavetas as minutas da proclamação,
da exposição ao vice-rei e de outros atos inaugurais do governo
anterior, deu-se pressa em os fazer copiar e expedir; acrescentam os
cronistas, e aliás subentende-se, que ele lhes mudou os nomes, e onde o
outro barbeiro falara de uma câmara corrupta, falou este de ``um intruso
eivado das más doutrinas francesas e contrário aos sacrossantos
interesses de Sua Majestade, etc''. Nisto entrou na vila uma força
mandada pelo vice-rei, e restabeleceu a ordem. O alienista exigiu desde
logo a entrega do barbeiro Porfírio, e bem assim a de uns cinquenta e
tantos indivíduos, que declarou mentecaptos; e não só lhe deram esses,
como afiançaram entregar-lhe mais dezenove sequazes do barbeiro, que
convalesciam das feridas apanhadas na primeira rebelião.

Este ponto da crise de Itaguaí marca também o grau máximo da influência
de Simão Bacamarte. Tudo quanto quis, deu-se-lhe; e uma das mais vivas
provas do poder do ilustre médico achamo-la na prontidão com que os
vereadores, restituídos a seus lugares, consentiram em que Sebastião
Freitas também fosse recolhido ao hospício. O alienista, sabendo da
extraordinária inconsistência das opiniões desse vereador, entendeu que
era um caso patológico, e pediu-o. A mesma coisa aconteceu ao boticário.
O alienista, desde que lhe falaram da momentânea adesão de Crispim
Soares à rebelião dos Canjicas, comparou-a à aprovação que sempre
recebera dele, ainda na véspera, e mandou capturá-lo. Crispim Soares não
negou o fato, mas explicou-o dizendo que cedera a um movimento de
terror, ao ver a rebelião triunfante, e deu como prova a ausência de
nenhum outro ato seu, acrescentando que voltara logo à cama, doente.
Simão Bacamarte não o contrariou; disse, porém, aos circunstantes que o
terror também é pai da loucura, e que o caso de Crispim Soares lhe
parecia dos mais caracterizados.

Mas a prova mais evidente da influência de Simão Bacamarte foi a
docilidade com que a câmara lhe entregou o próprio presidente. Este
digno magistrado tinha declarado em plena sessão, que não se contentava,
para lavá-lo da afronta dos Canjicas, com menos de trinta
almudes\footnote{Medida que correspondia a aproximadamente 32 litros.}
de sangue; palavra que chegou aos ouvidos do alienista por boca do
secretário da câmara, entusiasmado de tamanha energia. Simão Bacamarte
começou por meter o secretário na Casa Verde, e foi dali à câmara, à
qual declarou que o presidente estava padecendo da ``demência dos
touros'', um gênero que ele pretendia estudar, com grande vantagem para
os povos. A Câmara a princípio hesitou, mas acabou cedendo.

Daí em diante foi uma coleta desenfreada. Um homem não podia dar
nascença ou curso à mais simples mentira do mundo, ainda daquelas que
aproveitam ao inventor ou divulgador, que não fosse logo metido na Casa
Verde. Tudo era loucura. Os cultores de enigmas, os fabricantes de
charadas, de anagramas, os maldizentes, os curiosos da vida alheia, os
que põem todo o seu cuidado na tafularia\footnote{Casquilharia. Ação
  típica de um janota, almofadinha.}, um ou outro almotacé enfunado,
ninguém escapava aos emissários do alienista. Ele respeitava as
namoradas e não poupava as namoradeiras, dizendo que as primeiras cediam
a um impulso natural, e as segundas a um vício. Se um homem era avaro ou
pródigo, ia do mesmo modo para a Casa Verde; daí a alegação de que não
havia regra para a completa sanidade mental. Alguns cronistas creem que
Simão Bacamarte nem sempre procedia com lisura, e citam em abono da
afirmação (que não sei se pode ser aceita) o fato de ter alcançado da
câmara uma postura autorizando o uso de um anel de prata no dedo polegar
da mão esquerda, a toda a pessoa que, sem outra prova documental ou
tradicional, declarasse ter nas veias duas ou três onças de sangue
godo\footnote{Antigo povo germânico que invadiu o império romano no
  século \textsc{iii} d.C. No contexto, pode-se referir ao sangue impuro,
  pertencente a um povo rude, bárbaro.}. Dizem esses cronistas que o fim
secreto da insinuação à câmara foi enriquecer um ourives, amigo e
compadre dele; mas, conquanto seja certo que o ourives viu prosperar o
negócio depois da nova ordenação municipal\footnote{Decreto,
  regulamento. Recebe o nome de ordenação por aproximação com as
  Ordenações Reais, estendidas às possessões do reino português.}, não o
é menos que essa postura deu à Casa Verde uma multidão de inquilinos;
pelo que, não se pode definir, sem temeridade, o verdadeiro fim do
ilustre médico. Quanto à razão determinativa da captura e aposentação na
Casa Verde de todos quantos usaram do anel, é um dos pontos mais
obscuros da história de Itaguaí; a opinião mais verossímil é que eles
foram recolhidos por andarem a gesticular, à toa, nas ruas, em casa, na
igreja. Ninguém ignora que os doidos gesticulam muito. Em todo caso, é
uma simples conjectura; de positivo, nada há.

--- Onde é que este homem vai parar? diziam os principais da terra. Ah!
se nós tivéssemos apoiado os Canjicas\ldots{}

Um dia de manhã, --- dia em que a câmara devia dar um grande baile, ---
a vila inteira ficou abalada com a notícia de que a própria esposa do
alienista fora metida na Casa Verde. Ninguém acreditou; devia ser
invenção de algum gaiato. E não era: era a verdade pura. D.\,Evarista
fora recolhida às duas horas da noite. O padre Lopes correu ao alienista
e interrogou-o discretamente acerca do fato.

--- Já há algum tempo que eu desconfiava, disse gravemente o marido. A
modéstia com que ela vivera em ambos os matrimônios não podia
conciliar-se com o furor das sedas, veludos, rendas e pedras preciosas
que manifestou, logo que voltou do Rio de Janeiro. Desde então comecei a
observá-la. Suas conversas eram todas sobre esses objetos; se eu lhe
falava das antigas cortes, inquiria logo da forma dos vestidos das
damas; se uma senhora a visitava, na minha ausência, antes de me dizer o
objeto da visita, descrevia-me o trajo\footnote{Grafado desta forma, no
  original (e não ``traje'').}, aprovando umas coisas e censurando
outras. Um dia, creio que Vossa Reverendíssima há de lembrar-se,
propôs-se a fazer anualmente um vestido para a imagem de Nossa Senhora
da matriz. Tudo isto eram sintomas graves; esta noite, porém,
declarou-se a total demência. Tinha escolhido, preparado, enfeitado o
vestuário que levaria ao baile da câmara Municipal; só hesitava entre um
colar de granada e outro de safira. Anteontem perguntou-me qual deles
levaria; respondi-lhe que um ou outro lhe ficava bem. Ontem repetiu a
pergunta ao almoço; pouco depois de jantar fui achá-la calada e
pensativa. --- Que tem? perguntei-lhe. --- Queria levar o colar de
granada, mas acho o de safira tão bonito! --- Pois leve o de safira. ---
Ah! mas onde fica o de granada? --- Enfim, passou a tarde sem novidade.
Ceamos, e deitamo-nos. Alta noite, seria hora e meia, acordo e não a
vejo; levanto-me, vou ao quarto de vestir, acho-a diante dos dois
colares, ensaiando-os ao espelho, ora um, ora outro. Era evidente a
demência; recolhi-a logo.

O padre Lopes não se satisfez com a resposta, mas não objetou nada. O
alienista, porém, percebeu e explicou-lhe que o caso de D.\,Evarista era
de ``mania sumptuária''\footnote{Suntuosa, luxuosa, opulenta.}, não
incurável, e em todo caso digno de estudo.

--- Conto pô-la boa dentro de seis semanas, concluiu ele.

A abnegação do ilustre médico deu-lhe grande realce. Conjecturas,
invenções, desconfianças, tudo caiu por terra desde que ele não duvidou
recolher à Casa Verde a própria mulher, a quem amava com todas as forças
da alma. Ninguém mais tinha o direito de resistir-lhe, ---menos ainda o
de atribuir-lhe intuitos alheios à ciência. Era um grande homem austero,
Hipócrates forrado de Catão\footnote{A denominação é irônica, já que
  Hipócrates era grego e a cultura helenística constituía um dos alvos
  censurados pelo romano Catão. A alcunha sugere que Bacamarte
  corresponderia a uma espécie de sábio e censor, conhecedor versado na
  cultura greco-latino.}.

\chapter{\textsc{xi}. O assombro de Itaguaí}

E agora prepare-se o leitor para o mesmo assombro em que ficou a vila,
ao saber um dia que os loucos da Casa Verde iam todos ser postos na rua.

--- Todos?

--- Todos.

--- É impossível; alguns sim, mas todos\ldots{}

--- Todos. Assim o disse ele no ofício que mandou hoje de manhã à
câmara.

De fato, o alienista oficiara à câmara expondo: ---
1\textsuperscript{o}, que verificara das estatísticas da vila e da Casa
Verde, que quatro quintos da população estavam aposentados naquele
estabelecimento; 2°, que esta deslocação de população levara-o a
examinar os fundamentos da sua teoria das moléstias cerebrais, teoria
que excluía do domínio da razão todos os casos em que o equilíbrio das
faculdades, não fosse perfeito e absoluto; 3°, que desse exame e do fato
estatístico resultara para ele a convicção de que a verdadeira doutrina
não era aquela, mas a oposta, e portanto que se devia admitir como
normal e exemplar o desequilíbrio das faculdades, e como hipóteses
patológicas todos os casos em que aquele equilíbrio fosse ininterrupto;
4\textsuperscript{o}, que à vista disso, declarava à câmara que ia dar
liberdade aos reclusos da Casa Verde e agasalhar nela as pessoas que se
achassem nas condições agora expostas; 5°, que tratando de descobrir a
verdade científica, não se pouparia a esforços de toda a natureza,
esperando da câmara igual dedicação; 6º, que restituía à câmara e aos
particulares a soma do estipêndio recebido para alojamento dos supostos
loucos, descontada a parte efetivamente gasta com a alimentação, roupa,
etc.; o que a câmara mandaria verificar nos livros e arcas da Casa
Verde.

O assombro de Itaguaí foi grande; não foi menor a alegria dos parentes e
amigos dos reclusos. Jantares, danças, luminárias\footnote{No contexto,
  ``luminárias'' parece referir-se à iluminação pública, como forma de
  celebração, festividade.}, músicas, tudo houve para celebrar tão
fausto\footnote{Luxuoso, farto.} acontecimento. Não descrevo as festas
por não interessarem ao nosso propósito; mas foram esplêndidas, tocantes
e prolongadas.

E vão assim as coisas humanas! No meio do regozijo produzido pelo ofício
de Simão Bacamarte, ninguém advertia na frase final do § 4º, uma frase
cheia de experiências futuras.

\chapter{\textsc{xii}. O final do § 4º }

Apagaram-se as luminárias, reconstituíram-se as famílias, tudo parecia
reposto nos antigos eixos. Reinava a ordem, a câmara exercia outra vez o
governo, sem nenhuma pressão externa; o próprio presidente e o vereador
Freitas tornaram aos seus lugares. O barbeiro Porfírio, ensinado pelos
acontecimentos, tendo ``provado tudo'', como o poeta\footnote{Possível
  alusão ao poema ``Napoleão'', de Fagundes Varela, que integrou o seu
  livro \emph{Vozes da América}, de 1864.} disse de Napoleão, e mais
alguma coisa, porque Napoleão não provou a Casa Verde, o barbeiro achou
preferível a glória obscura da navalha e da tesoura às calamidades
brilhantes do poder; foi, é certo, processado; mas a população da vila
implorou a clemência de Sua Majestade; daí o perdão. João Pina foi
absolvido, atendendo-se a que ele derrocara um rebelde. Os cronistas
pensam que deste fato é que nasceu o nosso adágio: --- ladrão que furta
ladrão tem cem anos de perdão; --- adágio\footnote{Rifão, lugar-comum,
  frase-feita.} imoral, é verdade, mas grandemente útil. Não só findaram
as queixas contra o alienista, mas até nenhum ressentimento ficou dos
atos que ele praticara; acrescendo que os reclusos da Casa Verde, desde
que ele os declarara plenamente ajuizados, sentiram-se tomados de
profundo reconhecimento e férvido entusiasmo. Muitos entenderam que o
alienista merecia uma especial manifestação, e deram-lhe um baile, ao
qual se seguiram outros bailes e jantares. Dizem as crônicas que D.\,Evarista a princípio tivera ideia de separar-se do consorte, mas a dor
de perder a companhia de tão grande homem venceu qualquer ressentimento
de amor-próprio, e o casal veio a ser ainda mais feliz do que antes.

Não menos íntima ficou a amizade do alienista e do boticário. Este
concluiu do ofício de Simão Bacamarte que a prudência é a primeira das
virtudes em tempos de revolução e apreciou muito a magnanimidade do
alienista que, ao dar-lhe a liberdade, estendeu-lhe a mão de amigo
velho.

--- É um grande homem, disse ele à mulher, referindo aquela
circunstância.

Não é preciso falar do albardeiro, do Costa, do Coelho, do Martim Brito
e outros, especialmente nomeados neste escrito; basta dizer que puderam
exercer livremente os seus hábitos anteriores. O próprio Martim Brito,
recluso por um discurso em que louvara enfaticamente D.\,Evarista, fez
agora outro em honra do insigne médico --- ``cujo altíssimo gênio,
elevando as asas muito acima do sol, deixou abaixo de si todos os demais
espíritos da terra''.

--- Agradeço as suas palavras, retorquiu-lhe o alienista, e ainda me não
arrependo de o haver restituído à liberdade.

Entretanto, a câmara, que respondera ao ofício de Simão Bacamarte, com a
ressalva de que oportunamente estatuiria em relação ao final do § 4°,
tratou enfim de legislar sobre ele. Foi adotada, sem debate, uma postura
autorizando o alienista a agasalhar na Casa Verde as pessoas que se
achassem no gozo do perfeito equilíbrio das faculdades mentais. E porque
a experiência da câmara tivesse sido dolorosa, estabeleceu ela a
cláusula, de que a autorização era provisória, limitada a um ano, para o
fim de ser experimentada a nova teoria psicológica, podendo a câmara,
antes mesmo daquele prazo, mandar fechar a Casa Verde, se a isso fosse
aconselhada por motivos de ordem pública. O vereador Freitas propôs
também a declaração de que em nenhum caso fossem os vereadores
recolhidos ao asilo dos alienados: cláusula que foi aceita, votada e
incluída na postura, apesar das reclamações do vereador Galvão. O
argumento principal deste magistrado é que a câmara, legislando sobre
uma experiência científica, não podia excluir as pessoas dos seus
membros das consequências da lei; a exceção era odiosa e ridícula. Mal
proferira estas duas palavras, romperam os vereadores em altos brados
contra a audácia e insensatez do colega; este, porém, ouviu-os e
limitou-se a dizer que votava contra a exceção.

--- A vereança, concluiu ele, não nos dá nenhum poder especial nem nos
elimina do espírito humano.

Simão Bacamarte aceitou a postura com todas as restrições. Quanto à
exclusão dos vereadores, declarou que teria profundo sentimento se fosse
compelido a recolhê-los à Casa Verde; a cláusula, porém, era a melhor
prova de que eles não padeciam do perfeito equilíbrio das faculdades
mentais. Não acontecia o mesmo ao vereador Galvão, cujo acerto na
objeção feita, e cuja moderação na resposta dada às invectivas dos
colegas mostravam da parte dele um cérebro bem organizado; pelo que,
rogava à câmara que lho entregasse. A câmara, sentindo-se ainda
agravada\footnote{Que sofreu agravo ou foi injuriada.} pelo proceder do
vereador Galvão, estimou o pedido do alienista, e votou unanimemente a
entrega. Compreende-se que, pela teoria nova, não bastava um fato ou um
dito para recolher alguém à Casa Verde; era preciso um longo exame, um
vasto inquérito do passado e do presente. O padre Lopes, por exemplo, só
foi capturado trinta dias depois da postura, a mulher do boticário
quarenta dias. A reclusão desta senhora encheu o consorte de indignação.
Crispim Soares saiu de casa espumando de cólera, e declarando às pessoas
a quem encontrava que ia arrancar as orelhas ao tirano. Um sujeito,
adversário do alienista, ouvindo na rua essa notícia, esqueceu os
motivos de dissidência, e correu à casa de Simão Bacamarte a
participar-lhe o perigo que corria. Simão Bacamarte mostrou-se grato ao
procedimento do adversário, e poucos minutos lhe bastaram para conhecer
a retidão dos seus sentimentos, a boa-fé, o respeito humano, a
generosidade; apertou-lhe muito as mãos, e recolheu-o à Casa Verde.

--- Um caso destes é raro, disse ele à mulher pasmada. Agora esperemos o
nosso Crispim.

Crispim Soares entrou. A dor vencera a raiva, o boticário não arrancou
as orelhas ao alienista. Este consolou o seu privado, assegurando-lhe
que não era caso perdido; talvez a mulher tivesse alguma lesão cerebral;
ia examiná-la com muita atenção; mas antes disso não podia deixá-la na
rua. E parecendo-lhe vantajoso reuni-los, porque a astúcia e velhacaria
do marido poderiam de certo modo curar a beleza moral que ele descobrira
na esposa, disse Simão Bacamarte:

--- O senhor trabalhará durante o dia na botica, mas almoçará e jantará,
com sua mulher, e cá passará as noites, e os domingos e dias santos.

A proposta colocou o pobre boticário na situação do asno de
Buridan\footnote{Referência ao francês Jean Buridan (1295--1358), que
  teria concebido o impasse mortal enfrentado por um burro, dividido
  entre comer aveia ou tomar água.}. Queria viver com a mulher, mas
temia voltar à Casa Verde; e nessa luta esteve algum tempo, até que D.\,Evarista o tirou da dificuldade, prometendo que se incumbiria de ver a
amiga e transmitiria os recados de um para outro. Crispim Soares
beijou-lhe as mãos agradecido. Este último rasgo de egoísmo pusilânime
pareceu sublime ao alienista.

Ao cabo de cinco meses estavam alojadas umas dezoito pessoas; mas Simão
Bacamarte não afrouxava; ia de rua em rua, de casa em casa, espreitando,
interrogando, estudando; e quando colhia um enfermo, levava-o com a
mesma alegria com que outrora os arrebanhava às dúzias. Essa mesma
desproporção confirmava a teoria nova; achara-se enfim a verdadeira
patologia cerebral. Um dia, conseguiu meter na Casa Verde o juiz de
fora; mas procedia com tanto escrúpulo que o não fez senão depois de
estudar minuciosamente todos os seus atos, e interrogar os principais da
vila. Mais de uma vez esteve prestes a recolher pessoas perfeitamente
desequilibradas; foi o que se deu com um advogado, em quem reconheceu um
tal conjunto de qualidades morais e mentais que era perigoso deixá-lo na
rua. Mandou prendê-lo; mas o agente, desconfiado, pediu-lhe para fazer
uma experiência; foi ter com um compadre, demandado por um testamento
falso, e deu-lhe de conselho que tomasse por advogado o
Salustiano\footnote{Patriarca de Jerusalém que viveu no século V d.C.
  Saiu-se vitorioso de uma disputa com os monges liderados por São Sabas
  (439--532 d.C.).}; era o nome da pessoa em questão.

--- Então parece-lhe\ldots{}?

--- Sem dúvida: vá, confesse tudo, a verdade inteira, seja qual for, e
confie-lhe a causa.

O homem foi ter com o advogado, confessou ter falsificado o testamento,
e acabou pedindo que lhe tomasse a causa. Não se negou o advogado;
estudou os papéis, arrazoou longamente, e provou a todas as luzes que o
testamento era mais que verdadeiro. A inocência do réu foi solenemente
proclamada pelo juiz, e a herança passou-lhe às mãos. O distinto
jurisconsulto deveu a esta experiência a liberdade. Mas nada escapa a um
espírito original e penetrante. Simão Bacamarte, que desde algum tempo
notava o zelo, a sagacidade, a paciência, a moderação daquele agente,
reconheceu a habilidade e o tino com que ele levara a cabo uma
experiência tão melindrosa\footnote{Delicada, complexa.} e complicada, e
determinou recolhê-lo imediatamente à Casa Verde; deu-lhe, todavia, um
dos melhores cubículos.

Os alienados foram alojados por classes. Fez-se uma galeria de modestos;
isto é, os loucos em quem predominava esta perfeição moral; outra de
tolerantes, outra de verídicos, outra de símplices, outra de leais,
outra de magnânimos, outra de sagazes, outra de sinceros, etc.
Naturalmente, as famílias e os amigos dos reclusos bradavam contra a
teoria; e alguns tentaram compelir a câmara a cassar a licença. A câmara
porém, não esquecera a linguagem do vereador Galvão, e se cassasse a
licença, vê-lo-ia na rua, e restituído ao lugar; pelo que, recusou.
Simão Bacamarte oficiou aos vereadores, não agradecendo, mas
felicitando-os por esse ato de vingança pessoal.

Desenganados da legalidade, alguns principais da vila recorreram
secretamente ao barbeiro Porfírio e afiançaram-lhe todo o apoio de
gente, dinheiro e influência na corte, se ele se pusesse à testa de
outro movimento contra a câmara e o alienista. O barbeiro respondeu-lhes
que não; que a ambição o levara da primeira vez a transgredir as leis,
mas que ele se emendara, reconhecendo o erro próprio e a pouca
consistência da opinião dos seus mesmos sequazes; que a câmara entendera
autorizar a nova experiência do alienista, por um ano: cumpria, ou
esperar o fim do prazo, ou requerer ao vice-rei, caso a mesma Câmara
rejeitasse o pedido. Jamais aconselharia o emprego de um recurso que ele
viu falhar em suas mãos, e isso a troco de mortes e ferimentos que
seriam o seu eterno remorso.

--- O que é que me está dizendo? perguntou o alienista quando um agente
secreto lhe contou a conversação do barbeiro com os principais da vila.

Dois dias depois o barbeiro era recolhido à Casa Verde. --- Preso por
ter cão, preso por não ter cão! exclamou o infeliz.

Chegou o fim do prazo, a câmara autorizou um prazo suplementar de seis
meses para ensaio dos meios terapêuticos. O desfecho deste episódio da
crônica itaguaiense, é de tal ordem, e tão inesperado, que merecia nada
menos de dez capítulos de exposição; mas contento-me com um, que será o
remate da narrativa, e um dos mais belos exemplos de convicção
científica e abnegação humana.

\chapter{\textsc{xiii}. Plus Ultra!\footnote{Expressão latina que significa ``mais
  ainda''.}}

Era a vez da terapêutica. Simão Bacamarte, ativo e sagaz em descobrir
enfermos, excedeu-se ainda na diligência e penetração com que principiou
a tratá-los. Neste ponto todos os cronistas estão de pleno acordo: o
ilustre alienista fez curas pasmosas, que excitaram a mais viva
admiração em Itaguaí.

Com efeito, era difícil imaginar mais racional sistema terapêutico.
Estando os loucos divididos por classes, segundo a perfeição moral que
em cada um deles excedia às outras, Simão Bacamarte cuidou em atacar de
frente a qualidade predominante. Suponhamos um modesto. Ele aplicava a
medicação que pudesse incutir-lhe o sentimento oposto; e não ia logo às
doses máximas, --- graduava-as, conforme o estado, a idade, o
temperamento, a posição social do enfermo. Às vezes bastava uma casaca,
uma fita, uma cabeleira, uma bengala, para restituir a razão ao
alienado; em outros casos a moléstia era mais rebelde; recorria então
aos anéis de brilhantes, às distinções honoríficas, etc. Houve um
doente, poeta, que resistiu a tudo. Simão Bacamarte começava a
desesperar da cura, quando teve a ideia de mandar correr matraca, para o
fim de o apregoar como um rival de Garção\footnote{Correia Garção, poeta
  árcade português bastante lido no Brasil e em Portugal, durante os
  séculos \textsc{xviii} e \textsc{xix}.} e de Píndaro\footnote{Píndaro (517--438 a.C.),
  poeta grego, autor de hinos célebres em seu tempo.}.

--- Foi um santo remédio, contava a mãe do infeliz a uma comadre; foi um
santo remédio.

Outro doente, também modesto, opôs a mesma rebeldia à medicação; mas,
não sendo escritor (mal sabia assinar o nome), não se lhe podia aplicar
o remédio da matraca. Simão Bacamarte lembrou-se de pedir para ele o
lugar de secretário da Academia dos Encobertos\footnote{Machado de Assis
  alude às academias literárias que se tornaram numerosas na Europa e no
  Brasil, a partir do século \textsc{xvii}. O nome dado à agremiação, na novela,
  possivelmente parodia a Academia dos Esquecidos, que funcionou em
  Salvador entre 1724 e 1725, e teve como um de seus secretários o
  historiador Sebastião da Rocha Pita.} estabelecida em Itaguaí. Os
lugares de presidente e secretários eram de nomeação régia, por especial
graça do finado Rei Dom João \textsc{v}, e implicavam o tratamento de Excelência
e o uso de uma placa de ouro no chapéu. O governo de Lisboa recusou o
diploma; mas, representando o alienista que o não pedia como prêmio
honorífico ou distinção legítima, e somente como um meio terapêutico
para um caso difícil, o governo cedeu excepcionalmente à súplica; e
ainda assim não o fez sem extraordinário esforço do ministro da marinha
e ultramar, que vinha a ser primo do alienado. Foi outro santo remédio.

--- Realmente, é admirável! dizia-se nas ruas, ao ver a expressão sadia
e enfunada dos dois ex-dementes.

Tal era o sistema. Imagina-se o resto. Cada beleza moral ou mental era
atacada no ponto em que a perfeição parecia mais sólida; e o efeito era
certo. Nem sempre era certo. Casos houve em que a qualidade predominante
resistia a tudo; então o alienista atacava outra parte, aplicando à
terapêutica o método da estratégia militar, que toma uma fortaleza por
um ponto, se por outro o não pode conseguir.

No fim de cinco meses e meio estava vazia a Casa Verde; todos curados! O
vereador Galvão, tão cruelmente afligido de moderação e equidade, teve a
felicidade de perder um tio; digo felicidade, porque o tio deixou um
testamento ambíguo, e ele obteve uma boa interpretação, corrompendo os
juízes, e embaçando os outros herdeiros. A sinceridade do alienista
manifestou-se nesse lance; confessou ingenuamente que não teve parte na
cura: foi a simples \emph{vis medicatrix}\footnote{Trecho de sentença
  latina que significa ``poder de cura''. A expressão original (``vis
  medicatrix naturae'') é atribuída ao médico Hipócrates.} da natureza.
Não aconteceu o mesmo com o padre Lopes. Sabendo o alienista que ele
ignorava perfeitamente o hebraico e o grego, incumbiu-o de fazer uma
análise crítica da versão dos Setenta\footnote{Reunião de escritos que
  conteria a versão em grego das escrituras hebraicas, também conhecido
  como \emph{Septuaginta}, ou Versão dos Setenta. Remonta ao século \textsc{iii}
  a.C.}; o padre aceitou a incumbência, e em boa hora o fez; ao cabo de
dois meses possuía um livro e a liberdade. Quanto à senhora do
boticário, não ficou muito tempo na célula que lhe coube, e onde aliás
lhe não faltaram carinhos.

--- Por que é que o Crispim não vem visitar-me? dizia ela todos os dias.

Respondiam-lhe ora uma coisa, ora outra; afinal disseram-lhe a verdade
inteira. A digna matrona não pôde conter a indignação e a vergonha. Nas
explosões da cólera escaparam-lhe expressões soltas e vagas, como estas:

--- Tratante!\ldots{} velhaco!\ldots{} ingrato!\ldots{} Um patife que tem feito casas à
custa de unguentos\footnote{Medicamento em forma de pomada.}
falsificados e podres\ldots{} Ah! tratante!\ldots{}

Simão Bacamarte advertiu que, ainda quando não fosse verdadeira a
acusação contida nestas palavras, bastavam elas para mostrar que a
excelente senhora estava enfim restituída ao perfeito desequilíbrio das
faculdades; e prontamente lhe deu alta.

Agora, se imaginais que o alienista ficou radiante ao ver sair o último
hóspede da Casa Verde, mostrais com isso que ainda não conheceis o nosso
homem. \emph{Plus ultra!} era a sua divisa. Não lhe bastava ter
descoberto a teoria verdadeira da loucura; não o contentava ter
estabelecido em Itaguaí o reinado da razão. \emph{Plus ultra!} Não ficou
alegre, ficou preocupado, cogitativo; alguma coisa lhe dizia que a
teoria nova tinha, em si mesma, outra e novíssima teoria.

``Vejamos, pensava ele; vejamos se chego enfim à última verdade''.

Dizia isto, passeando ao longo da vasta sala\footnote{Simão Bacamarte
  age como os filósofos peripatéticos, à época de Aristóteles.}, onde
fulgurava a mais rica biblioteca dos domínios ultramarinos de Sua
Majestade. Um amplo chambre de damasco, preso à cintura por um cordão de
seda, com borlas de ouro (presente de uma Universidade) envolvia o corpo
majestoso e austero do ilustre alienista. A cabeleira cobria-lhe uma
extensa e nobre calva adquirida nas cogitações quotidianas da ciência.
Os pés, não delgados e femininos, não graúdos e mariolas\footnote{Mariola
  era um doce de frutas em formato de tablete. No contexto, parece se
  referir ao tamanho ou formato dos pés.}, mas proporcionados ao vulto,
eram resguardados por um par de sapatos cujas fivelas não passavam de
simples e modesto latão. Vede a diferença: --- só se lhe notava luxo
naquilo que era de origem científica; o que propriamente vinha dele
trazia a cor da moderação e da singeleza, virtudes tão ajustadas à
pessoa de um sábio.

Era assim que ele ia, o grande alienista, de um cabo a outro da vasta
biblioteca, metido em si mesmo, estranho a todas as coisas que não fosse
o tenebroso problema da patologia cerebral. Súbito, parou. Em pé, diante
de uma janela, com o cotovelo esquerdo apoiado na mão direita, aberta, e
o queixo na mão esquerda, fechada, perguntou ele a si:

--- Mas deveras estariam eles doidos, e foram curados por mim, --- ou o
que pareceu cura, não foi mais do que a descoberta do perfeito
desequilíbrio do cérebro?

E cavando por aí abaixo, eis o resultado a que chegou: os cérebros bem
organizados que ele acabava de curar, eram desequilibrados como os
outros. Sim, dizia ele consigo, eu não posso ter a pretensão de
haver-lhes incutido um sentimento ou uma faculdade nova; uma e outra
coisa existiam no estado latente\footnote{Sintoma oculto, ainda não
  manifesto.}, mas existiam.

Chegado a esta conclusão, o ilustre alienista teve duas sensações
contrárias, uma de gozo, outra de abatimento. A de gozo foi por ver que,
ao cabo de longas e pacientes investigações, constantes trabalhos, luta
ingente\footnote{Desmedido, descomunal.} com o povo, podia afirmar esta
verdade: --- não havia loucos em Itaguaí; Itaguaí não possuía um só
mentecapto. Mas tão depressa esta ideia lhe refrescara a alma, outra
apareceu que neutralizou o primeiro efeito; foi a ideia da dúvida. Pois
quê! Itaguaí não possuiria um único cérebro concertado? Esta conclusão
tão absoluta, não seria por isso mesmo errônea, e não vinha, portanto,
destruir o largo e majestoso edifício da nova doutrina psicológica?

A aflição do egrégio Simão Bacamarte é definida pelos cronistas
itaguaienses como uma das mais medonhas tempestades morais que têm
desabado sobre o homem. Mas as tempestades só aterram os fracos; os
forres enrijam-se contra elas e fitam o trovão. Vinte minutos depois
alumiou-se a fisionomia do alienista de uma suave claridade.

--- Sim, há de ser isso, pensou ele.

Isso é isto. Simão Bacamarte achou em si os característicos do perfeito
equilíbrio mental e moral; pareceu-lhe que possuía a sagacidade, a
paciência, a perseverança, a tolerância, a veracidade, o vigor moral, a
lealdade, todas as qualidades enfim que podem formar um acabado
mentecapto. Duvidou logo, é certo, e chegou mesmo a concluir que era
ilusão; mas, sendo homem prudente, resolveu convocar um conselho de
amigos, a quem interrogou com franqueza. A opinião foi afirmativa.

--- Nenhum defeito?

--- Nenhum, disse em coro a assembleia.

--- Nenhum vício?

--- Nada.

--- Tudo perfeito?

--- Tudo.

--- Não, impossível, bradou o alienista. Digo que não sinto em mim essa
superioridade que acabo de ver definir com tanta magnificência. A
simpatia é que vos faz falar. Estudo-me e nada acho que justifique os
excessos da vossa bondade.

A assembleia insistiu; o alienista resistiu; finalmente o padre Lopes
explicou tudo com este conceito digno de um observador:

--- Sabe a razão por que não vê as suas elevadas qualidades, que aliás
todos nós admiramos? É porque tem ainda uma qualidade que realça as
outras: --- a modéstia.

Era decisivo. Simão Bacamarte curvou a cabeça, juntamente alegre e
triste, e ainda mais alegre do que triste. Ato contínuo, recolheu-se à
Casa Verde. Em vão a mulher e os amigos lhe disseram que ficasse, que
estava perfeitamente são e equilibrado: nem rogos nem sugestões nem
lágrimas\footnote{Período grafado sem vírgulas, no original.} o
detiveram um só instante. A questão é científica, dizia ele; trata-se de
uma doutrina nova, cujo primeiro exemplo sou eu. Reúno em mim mesmo a
teoria e a prática.

--- Simão! Simão! meu amor! dizia-lhe a esposa com o rosto lavado em
lágrimas.

Mas o ilustre médico, com os olhos acesos da convicção científica,
trancou os ouvidos à saudade da mulher, e brandamente a repeliu. Fechada
a porta da Casa Verde, entregou-se ao estudo e à cura de si mesmo. Dizem
os cronistas que ele morreu dali a dezessete meses, no mesmo estado em
que entrou, sem ter podido alcançar nada. Alguns chegam ao ponto de
conjecturar que nunca houve outro louco, além dele, em Itaguaí; mas esta
opinião, fundada em um boato que correu desde que o alienista expirou,
não tem outra prova, senão o boato; e boato duvidoso, pois é atribuído
ao padre Lopes, que com tanto fogo realçara as qualidades do grande
homem. Seja como for, efetuou-se o enterro com muita pompa e rara
solenidade.

\part{O Imortal}

\chapter[Introdução, \emph{por João Adolfo Hansen}]{Introdução\subtitulo{``O Imortal'' e a verossimilhança}}

\begin{flushright}
\textsc{joão adolfo hansen}
\end{flushright}

Entre 15 de julho e 15 de setembro de 1882, Machado de Assis publicou os
seis capítulos do conto ``O Imortal'' em \emph{A Estação}, uma revista
feminina do Rio de Janeiro. No final dele, o narrador afirma: ``Tal é o
caso extraordinário, que há anos, com outro nome, e por outras palavras,
contei a este bom povo, que provavelmente já os esqueceu a
ambos''\footnote{Cf. Assis, Joaquim Maria Machado de. ``O Imortal''. In:
  \_\_\_\_. \emph{Obra Completa}. Org. Aluizio Leite \emph{et al}. Rio
  de Janeiro: Editora Nova Aguilar, 2015, v. 3 (Conto, Poesia, Teatro,
  Miscelânea, Correspondência), p. 77.}. Segundo Jean-Michel Massa, o
conto é ``a repetição de 'Rui de Leão', assinado Max, publicado no
\emph{Jornal das Famílias}, janeiro-fevereiro-março 1872 e republicado
por Magalhães Júnior nos \emph{Contos Recolhidos}, pp. 89--117''\footnote{Cf.
  Massa, Jean-Michel. \emph{A Juventude de Machado de Assis 1839--1870}.
  Ensaio de biografia intelectual. Ed. ilustrada. Trad. Marco Aurélio de
  Moura Matos. Rio de Janeiro: Civilização Brasileira; Conselho Nacional
  de Cultura, 1971, p. 534}. ``O Imortal'' começa com a fala do Dr.\,Leão,
médico homeopata, que relata a história da vida de seu pai para o
Coronel Bertioga e o tabelião João Linhares: ``--- Meu pai nasceu em
1600\ldots{}''. É uma noite chuvosa de novembro de 1855, numa vila fluminense,
``suponhamos que Itaboraí ou Sapucaia'', diz o narrador. Bertioga é o
proprietário da casa onde estão; Linhares é um ``espírito forte'',
expressão francesa que no século \textsc{xvii} significava ``libertino'', e, no
\textsc{xix}, ``livre pensador''. ``--- Perdão, em 1800, naturalmente\ldots{} --- Não,
senhor, replicou o Dr.\,Leão, de um modo grave e triste, foi em 1600''.

É provável que a história que o médico começa a contar para seus
ouvintes também não coincida ``naturalmente'' com as opiniões do leitor. O
leitor considera falsa a opinião de que um homem possa viver 255 anos,
pois não conhece nenhuma evidência empírica que a comprove como fato
biológico natural, habitual e normal. O narrador põe em cena essa mesma
opinião, quando faz o homeopata antecipar-se às objeções dos ouvintes:
``{[}\ldots{}{]} na verdade a história de meu pai não é fácil de crer''.
Realmente, não é. Vejamos três ou quatro coisas que permitam discuti-lo.

Comecemos pelo gênero do conto. Machado de Assis o escreveu elegendo uma
tradição antiga para ele, a de Luciano de Samósata, um grego do segundo
século da era cristã, autor de obras satíricas e paródicas relacionadas
à chamada \textsc{ii} Sofística\footnote{Cf. Rego, Enylton José de Sá. \emph{O
  Calundu e a panaceia}. A sátira menipeia e a tradição lucianesca. Rio
  de Janeiro: Forense Universitária, 1981.}. Caso de \emph{História
Verdadeira}, uma paródia, informa Henrique Murachco, das narrativas de
Odisseu na corte do rei Alcino nos cantos \textsc{ix} e seguintes da
\emph{Odisseia}.\footnote{Luciano- \emph{Diálogo dos Mortos}: versão
  bilingüe grego/português. Trad., introdução e notas: Henrique G.
  Murachco. São Paulo: Palas Ahena: Edusp, 1996, p. 14.} Como outros
textos de Luciano, \emph{História Verdadeira} se caracteriza pela
improbabilidade das ações e dos eventos narrados, improbabilidade que
hoje chamamos ``fantástico''. O gênero, que foi usado por autores
conhecidos de Machado de Assis, como Swift, de \emph{Viagens de
Gulliver}, Campanella, de \emph{A Cidade do Sol,} ou Cyrano de Bergerac,
de \emph{Viagem à Lua,} tem regras específicas: é uma ficção falsa, ou
seja, ficção sobre coisas impossíveis e improváveis. Para especificá-la,
podemos repetir a pergunta de Espinosa: a narração de um evento que não
ocorreu em parte alguma é \emph{falsa} ou \emph{fictícia}? Há dois tipos
de critérios para responder, o de existência e o de essência.

Quando a narração se refere a algo que realmente \emph{existe} e o
relaciona com um evento que não ocorreu em parte alguma, tem-se a
``\emph{ficção primeira}''. Por exemplo, com a referência à existência de
uma pessoa conhecida, Machado de Assis, inventa-se a ficção de algo que
nunca ocorreu, como uma viagem à Inglaterra, onde Joaquim Maria faz
contatos com uma leitora de \emph{Otelo} chamada Capitolina. Tem-se a
``\emph{ficção segunda}'' quando a narração se refere somente à
\emph{essência} dos seres; com a referência à essência, é possível
inventar uma ficção verdadeira, como \emph{vera fictio}, e uma ficção
falsa, como \emph{falsa fictio}. Como exemplo desta, imaginemos uma
história absurda, onde um inseto infinito voa num espaço que,
teoricamente, deverá estar todo ocupado por seu corpo; ou uma personagem
que tem uma alma quadrada. Ou, ainda, um homem imortal.

A distinção permite conceber operacionalmente a ficção verdadeira como a
narração que relaciona a existência ou a essência verdadeira de algo com
eventos que não aconteceram. E também definir a ficção de algo falso,
que não é nem existe, como história que relaciona o não-ser com
acontecimentos que nunca ocorreram. A \emph{falsa fictio} inventa algo
impossível de ser e, assim, de ocorrer. Em ambos os casos, verdadeiro e
falso, o termo \emph{ficção} define uma operação da imaginação, uma
técnica, uma forma e um efeito aplicados ora ao conhecimento de
existência, ora ao conhecimento de essência.

As duas espécies de ficção podem ser relacionadas com a passagem da
\emph{Poética} em que Aristóteles prescreve que o gênero histórico trata
do que efetivamente ocorreu, como uma narrativa de existência que conta
eventos particulares e verdadeiros, diferentemente da poesia, que figura
o possível ou o universal, como ficção de essência sem necessidade de se
referir a eventos particulares. Como se sabe, Aristóteles considera a
história inferior à poesia, porque a história é \emph{mímesis} parcial
que trabalha com o conhecimento de existência do passado ou um
conhecimento particular fornecido por testemunhos. Por isso mesmo, em
suas versões clássicas, a história consegue estabelecer a variante
``verdadeira'', quando estabelece ``fatos'' que permitem eliminar outras
variantes concorrentes. Como dizia Jean-Pierre Faye, o incêndio do
Reichstag não pode ter sido produzido ao mesmo tempo pelos comunistas e
Van der Lubbe, versão nazista, pelos \textsc{sa} de Göring, versão da
Internacional comunista, e por Van der Lubbe sozinho, versão de Tobias.
As três versões se excluem logicamente e é impossível que sejam
verdadeiras ao mesmo tempo. Com isso, a história se opõe à fantasia
poética que, mesmo ao tratar do passado, como a novela histórica, não
refere o que efetivamente foi, mas o que poderia ter sido.

O Dr.\,Leão afirma que sua história não é fácil de crer. Realmente não é,
se o fantástico do seu gênero, como ficção falsa, for avaliado com as
opiniões positivistas e realistas que o leitor, assim como João
Linhares, considera verdadeiras quando pensa em ``realidade'', pressupondo
que a ficção é uma imitação direta da mesma. O gênero fantástico é
explicitamente incrível: a descrença é seu pressuposto, não seu efeito,
pois sua matéria é não-ser. Seu destinatário deve saber que lê uma arte
de representar o inacreditável do não-ser e do não-existente, aceitando,
contudo, a realidade da convenção e do artifício. Na história
fantástica, nada existe em que acreditar, a não ser o bom desempenho
técnico e artístico das convenções de um gênero que trata do falso. O
gênero prevê que seus personagens vivam aventuras e situações
improváveis. Por exemplo, ser um morto que escreve. Ou ser pernambucano
e tornar-se rei da Inglaterra. O último exemplo é tipicamente
fantástico, na medida mesma em que os possíveis de uma vida apenas
mortal são por definição restritos e restritivos. Afinal, lembremos
Sartre, cada um é o que faz com o que fizeram dele. Nunca é suficiente,
embora o pouco quase sempre seja demasiado. O que acontece ao personagem
de ``O Imortal'' é ser e fazer muito, acumulando e vivendo demasiadamente
na sua os vários possíveis das vidas de outros homens: pernambucano,
religioso franciscano, amante de índia, amante de lady escocesa, guarda
papal, rei de Inglaterra, traficante de escravos, soldado, espião etc. A
história incomum de sua vida é efetivamente uma história esticada como
somatória, por assim dizer, de existências, escolhas e ações de muitos
homens. Nos diversos momentos dos seus 255 anos, variam enormemente as
pessoas e as experiências; no entanto, em todas as situações que vive,
ano após ano, entre 1639 e 1855, sempre se lê a mesma história básica.
Como se, vivendo o impossível da imortalidade, a cada nova experiência
estivesse condenado a efetivamente viver as possibilidades restritas de
uma vida só mortal, repetindo na longa extensão da sua as mesmas poucas
experiências da vida breve de todos, o amor, a aventura e a intriga. Não
há moral da história, pois quer divertir; no entanto, se quiser, o
leitor cioso de moralidade poderá concluir que estar livre da morte, mas
sujeito às contingências da condição humana, é tristemente tedioso, uma
vez que Pangloss é um estúpido e este aqui não é, com toda evidência, o
melhor dos mundos possíveis. Não é sem alívio que Rui de Leão vai enfim
para o outro lado do mistério, onde está Brás Cubas, que lá continua
escrevendo.

Obviamente, Machado de Assis é um mestre insuperado na sátira e na
paródia que caracterizam a tradição luciânica, podendo-se supor que,
sendo o monstro de perversidade que é, também deseja que seu leitor
descreia. E não do tema, ``imortalidade'', nem da história do personagem
que vive 255 anos, porque as inventa como o fantástico que diverte a
fantasia do ``bom povo'' de 1872, como poderá, talvez, divertir o de 2004.
Ao republicar o conto em 1882, provavelmente para ganhar uns cobres, já
poderia supor que seus leitores fossem como aqueles cinco do prólogo de
\emph{Memórias Póstumas de Brás Cubas} (1880/1881). Não eram,
evidentemente, porque ``O Imortal'' foi publicado entre anquinhas atiradas
para cima, toucados de \emph{ondulations} \emph{et} \emph{chutes} e o
escocês e o xadrez vitorianos que então, pasmo!, influenciavam até a
moda de Paris, segundo \emph{A estação}. Provavelmente foi lido, se é
que foi, como um \emph{fait divers} a mais. A história republicada em
1882 já tinha sido contada ``por outras palavras'' em 1872, explica o
narrador. Como um avô de Pierre Menard, no entanto, em 1882 o mesmo
conto já não era o mesmo de 1872: \emph{Memórias Póstumas de Brás Cubas}
evidencia que era já impossível ler como discurso sério o romanesco
romântico com que recheia os lugares-comuns da vida fantástica de Rui de
Leão. O romantismo continuaria a divertir o ``bom povo'', como agora, com
o \emph{kitsch} da ideologia do ideal, complicação sentimental,
aventuras e intrigas; em 1882, contudo, a mesma história que diverte
também perverte a diversão, pois subordina os lugares-comuns e os
efeitos fantásticos a outros fins.

Quando publicou ``O Imortal'', Machado de Assis continuava interessado na
paródia e no pastiche como gêneros literários. A partir de 1880, tinha
começado a usar narradores que escrevem textos improváveis ou
inconfiáveis. Com eles, transformando a matéria social de seu tempo,
passou a relativizar e a destruir a representação fundamentada no
pressuposto da adequação entre os signos da linguagem, os conceitos da
mente e as estruturas da realidade objetiva. A partir de 1880,
tornaram-se mais e mais frequentes nos seus textos as imagens da morte,
do falso e do nada, como a falta de memória, a equivalência de razão e
loucura, a ação do diabo, o acaso das semelhanças, o arbitrário do
encadeamento da narrativa, o duplo, a improbabilidade e a
indeterminação. Basta lembrar que o narrador de \emph{Memórias Póstumas}
é um defunto. Que é a morte? angustia-se o leitor. Nada, certamente,
pois não é dizível ou escriptível e nela não há nenhum fazer. Nada se
pode afirmar sobre ela e qualquer ideia de fazer seu conceito é
autocontraditória. Evidentemente, não é natural, habitual ou normal que
um morto escreva. O uso de um como autor sugere que Machado de Assis faz
da falta de ser o princípio alegórico do sentido da sua arte como
negação da representação tradicional. A delegação da escrita do romance
para o morto, que incrivelmente recorda e impossivelmente escreve, é
fantástica e desloca a autoria para uma liberdade arbitrária e
artificiosa que não é mais definível por unidades de sentido das quais o
discurso fosse uma semelhança adequada. A escrita do morto esvazia as
representações unitárias da subjetividade, do mundo objetivo e da
linguagem, que são as da vida e sua ideologia, descartando com elas o
ilusionismo baseado em opiniões dadas como naturalmente verdadeiras e
evidentes, como as do livre-pensador João Linhares. Por várias razões,
Machado de Assis é um grande escritor e sua consciência da historicidade
das formas literárias é uma das principais. Em 1882, o mesmo ponto de
vista sério da complicação sentimental romântica é repetido, mas agora
se evidencia como convenção histórica, ou seja, particularidade apenas
mortal. As leitoras de \emph{A estação} decerto não pensavam em
literatura como coisa séria nem que pudesse ter algum sentido crítico.
Provavelmente, queriam a literatura como se deseja uma cama mental: a
história lá longe, na tempestade que passa lá fora e, aqui, dentro, o
calor do suplemento de alma, a maciez do romanesco, o sono morno da
razão nos lençóis do passatempo. Por isso mesmo, ignorando que seu modo
de ler já era ruína, elas eram lidas por \emph{Memórias Póstumas}
\emph{de Brás Cubas} e pelo conto, ainda quando não os liam, e passavam,
como seu tempo passava, entre anquinhas e outras coisas mais altas do
Império que ameaçavam desabar e já ruíam. Mas vamos ao conto.

A narração do Dr.\,Leão repete três procedimentos básicos: a exposição
linear, a complicação e a explicação. Ele conta ações do mais passado
para o presente, linearizando a história da vida do pai de 1600 a 1855;
simultaneamente, amplifica e complica cada uma das ações com acidentes
ordenados como repetição dos mesmos lugares-comuns de aventura, intriga
e amor. Como o que conta é fantástico ou improvável, dá explicações aos
ouvintes, fornecendo causas que tentam tornar plausível e mesmo verídico
o que lhes diz. Vejamos.

Rui de Leão nasce no Recife em 1600; seu pai era da nobreza de Espanha e
a mãe, de grande casa do Alentejo. Entra aos 25 anos para a ordem
franciscana em Igaraçu, fica no convento até 1639, quando é aprisionado
pelos holandeses; recebe um salvo-conduto, vai para o mato, chega a uma
aldeia de gentio, junta-se com Maracujá, filha do chefe Pirajuá. Antes
de morrer, o chefe lhe revela o local de uma igaçaba enterrada; contém
um boião com um líquido amarelo; preparado por um pajé, garante a
imortalidade a quem o bebe. O chefe morre, Rui de Leão adoece, vai ao
local do vaso, bebe a substância, volta à tribo, sara; outros índios
atacam a aldeia, morre Maracujá, Rui é ferido, sara, decide voltar para
o Recife. Quando os holandeses são expulsos, em 1654, vai para Portugal,
casa-se, tem um filho; em março de 1661, seu filho e sua mulher morrem e
ele parte para a França e a Holanda.

Até aqui, o Dr.\,Leão narra sessenta e um anos de vida do pai aplicando
tópicas ou lugares-comuns da aventura e do amor típicos do \emph{kitsch}
romântico usual. Depois deles, aplica lugares-comuns também românticos
de intriga. Por exemplo, na Holanda, ``por motivo de uns amores secretos,
ou por ódio de alguns judeus descendentes ou naturais de Portugal, com
quem entreteve relações comerciais na Haia, ou enfim por outros motivos
desconhecidos'', Rui de Leão é preso; levam-no para a Alemanha, donde
passa à Hungria, a cidades italianas, à França e à Inglaterra.

Machado de Assis compõe a complicação romanesca da vida do personagem
até 1654 como estilização da história colonial. A começar pelo nome do
personagem, Rui de Leão ``ou antes Rui Garcia de Meireles e Castro
Azevedo de Leão''. A somatória de nomes de família tradicionais era
séria, nos tempos coloniais e imperiais, pois distintiva da prosápia dos
``homens bons'', ``gente de representação'' ou ``melhores''; em 1882, é uma
afetação burguesa de arrivistas, barões Joões do Império, proprietários
de escravos e funcionários públicos aspirantes a ministérios que
hiperboliza, com o ``e'' e o ``de'', as alianças de parentesco, compadrio e
favor das oligarquias do tempo. A estilização reescreve textos de
cronistas e jesuítas sobre os contatos com as tribos indígenas; retoma
relatos sobre as guerras holandesas, a situação dos judeus portugueses
refugiados na Holanda, as negociações pela posse de Pernambuco etc. O
trecho citado no parágrafo anterior estiliza as intrigas políticas que
envolveram a ação diplomática de Vieira em Haia e Amsterdam. Machado
também estiliza a ficção indianista e histórica de José de Alencar e
mais românticos: Rui, Maracujá e Pirajuá refiguram Martim, Iracema e
Araquém; o tipo do religioso que abandona o hábito, vai para o mato e
amiga-se com índia pode ser rastreado em personagens ex-padres de
\emph{O} \emph{Guarani}, \emph{As Minas de Prata} e \emph{O Jesuíta.} O
poema escatológico de Bernardo Guimarães, ``O Elixir do Pajé'', é
estilizado na referência ao pajé que fez a beberagem. Nas aventuras de
Rui de Leão posteriores a 1661, também estiliza personagens como D.\,Juan, o atleta do amor, e os mitos eróticos da vida de poetas românticos
mais ou menos descabelados, como Byron, Lamartine, Musset. Estiliza
ainda os segredos, as traições, os desesperos, o patético e o
sentimental de narrativas de românticos ingleses, escoceses e franceses,
Walter Scott, Trilby, Lamartine, George Sand, Eugene Sue, Morand, Carco,
Musset, Alexandre Dumas e um grande etc. Também estiliza elementos
microtextuais, como o léxico antigo: o termo ``aleivosia'' é
divertidamente típico. E frases inteiras, que é impossível ler sem
sorrir de cumplicidade, pois o \emph{kitsch} não é de Machado de Assis.
Em todos os casos, a estilização mantém as características originais do
estilo romanesco da arte e da vida dos românticos, para subordiná-las a
outro fim, transformando o sério do ideal num pastiche irônico.

Vejamos um pouco mais do texto. Em Londres, Rui estuda inglês; sabe o
latim do convento, o hebraico aprendido em Haia com um amigo (nada menos
que um polidor de lentes, o filósofo Espinosa), e o francês, o italiano,
parte do alemão e do húngaro, tornando-se objeto de curiosidade e
veneração de plebeus e cortesãos. A história acumula mais
lugares-comuns. A enumeração dos múltiplos ofícios de Rui de Leão
condensa em poucos segundos de leitura o tempo de muitos anos vividos
por ele --- soldado, advogado, sacristão, mestre de dança, comerciante,
livreiro, espião na Áustria, guarda pontifício, armador de navios,
letrado, gamenho. Na aceleração narrativa, de novo se associa a esses
lugares o lugar-comum do amor. Mais que as \emph{mille e tre} mulheres
de D.\,Juan, informa o Dr.\,Leão, seu pai teve não menos de cinco mil.
Outros lugares-comuns se acrescentam ao exagero improvável que diverte a
inveja erótica do leitor: os da beleza feminina e sua psicologia vária,
com alfinetadas nas leitoras de figurinos. Por exemplo, a gentil
descortesia que é dizer elogiosamente que a estupidez das mulheres é
graciosa, usando para isso um preceito retórico do gênero cômico --- ``Há
casos em que uma mulher estúpida tem o seu lugar''. Os ``casos'' e o ``lugar'' para uma mulher estúpida são tópicas cômicas e elas sempre prescrevem
que a estupidez é só isso: estúpida. De novo, complicação sentimental e
aventuras amplificadas: na Haia, entre os novos amores, Rui de Leão
torna-se amante de lady Ema Sterling, ``senhora inglesa, ou antes
escocesa''. A caracterização de Ema estiliza heroínas como Corinne,
Norma, Graziella, Geneviève, Cosette, Delphine, Aurélia, Diva, Helena,
Iaiá Garcia e outras, cuja solidão moral assombra o imaginário dos
leitores românticos: ``formosa, resoluta e audaz --- tão audaz que chegou
a propor ao amante uma expedição a Pernambuco para conquistar a
capitania, e aclamarem-se reis do novo Estado''. Apaixonadamente dedicada
ou dedicadamente apaixonada, Ema deseja alçar Rui a grande posição: ``--- Tu
serás rei ou duque\ldots{} --- Ou cardeal, acrescentava ele rindo. --- Por que
não cardeal?''. Os dois enunciados elencam lugares-comuns de alta posição
social dos romances capa-e-espada. ``--- Por que não cardeal?''. As três
posições são aplicáveis à história romanesca de Rui de Leão, pois, sendo
imortal, tem tempo para viver todos os lugares. Assim, lady Ema o faz
entrar na conspiração que resulta na guerra civil inglesa. Segundo o
Dr.\,Leão, ela tem uma ideia espantosa: afirmar que Rui de Leão é o pai do
Duque de Monmouth, suposto filho natural de Carlos \textsc{ii} e principal
caudilho dos rebeldes. A tal ideia causa nova complicação narrativa que
obriga o Dr.\,Leão a justificar porque lady Ema pôde tê-la: ``A verdade é
que eram parecidos como duas gotas d'água. Outra verdade é que lady Ema,
por ocasião da guerra civil, tinha o plano secreto de fazer matar o
duque, se ele triunfasse, e substituí-lo pelo amante, que assim subiria
ao trono da Inglaterra. O pernambucano, escusado é dizê-lo, não soube de
semelhante aleivosia, nem lhe daria seu consentimento''.

O Dr.\,Leão usa o termo ``verdade'' duas vezes para justificar o que torna
plausível a ideia de lady Ema: a semelhança. A narração do conto
acontece em 1852; esse é um tempo romântico, o Dr.\,Leão é homeopata e a
semelhança ainda é tudo. Desde \emph{Memórias Póstumas}, porém, as
identidades e unidades metafísicas que a fundamentavam foram criticadas
e a semelhança já está arruinada como critério de validação da verdade
dos discursos transformados pela ficção. Machado de Assis ainda iria
escrever o texto decisivo, cujo núcleo é o equívoco da semelhança,
\emph{Dom Casmurro,} de 1899. Por ora, fiquemos com a história do Dr.\,Leão.

Nas idas e vindas da revolta, sempre envolvido em aventuras, intrigas e
no eterno amor de lady Ema, o pernambucano é aclamado rei de Inglaterra.
De novo, em poucos segundos de leitura, o leitor fica sabendo que Rui
governa o país, reprime sedições, baixa leis, é preso quando a fraude é
revelada, julgado, condenado à morte na Torre de Londres. Duas vezes o
machado do carrasco lhe atinge o pescoço, sem cortá-lo; solto, é
admirado, temido, amado, odiado, comparado a Cristo. O ano é 1686, Rui
de Leão tem oitenta e seis, não aparenta mais que quarenta.

No início do capítulo \textsc{v}, o Dr.\,Leão adverte Bertioga e Linhares: ``Já
veem, pelo que lhes contei, que não acabaria hoje nem em toda esta
semana, se quisesse referir miudamente a vida inteira de meu pai. Algum
dia o farei, mas por escrito, e cuido que a obra dará cinco volumes, sem
contar os documentos\ldots{}''. Aqui, afirma mais duas coisas relativas à
probabilidade da história que conta: a primeira é que poderia
amplificá-la ilimitadamente, acumulando detalhes. Por exemplo, se
parasse para contar miudamente a história de cada um dos ofícios
exercidos pelo pai; ou se tratasse de cada um dos seus cinco mil amores.
Para fazê-lo, bastaria aplicar novamente os lugares-comuns de aventura,
amor e intriga, que espichariam a história pelos cinco volumes com que
felizmente só ameaça o leitor. A segunda coisa é que afirma ter
documentos que comprovam a veracidade da história: ``títulos, cartas,
traslados de sentenças, de escrituras, cópias de estatísticas\ldots{}''.

Na historiografia, o leitor sabe, provas documentais atestam a
existência dos eventos narrados, distinguindo a narração histórica da
narração ficcional. Alegando as provas documentais que tornam o gênero
histórico provável, o médico homeopata propõe que o fantástico da sua
história tem a autenticidade e a autoridade de um discurso verdadeiro
sobre coisas e eventos reais- ``fatos'', como diziam os positivistas
também no tempo de Machado de Assis. A suposta realidade dos ``fatos''
assim constituídos pelos supostos documentos permite separar e excluir
como ``ficção'', irrealidade, o discurso que não pode apresentá-los. Se a
opinião de que um homem possa viver 255 anos é considerada falsa, a
história do Dr.\,Leão sobre a vida do pai é improvável; mas ela tende a
ser recebida não só como plausível, mas principalmente como verídica,
quando declara aos ouvintes que tem documentos que a comprovam. Voltemos
a Rui de Leão.

Sempre entre os quarenta e os cinquenta anos, vivendo oito, dez ou doze
numa cidade e noutra, perde a herança de lady Ema em um lugar de
complicação. Com os dez mil cruzados que lhe restam, tem a ideia de
meter-se no negócio de escravos. Aqui, mais lugares-comuns de aventura.
Obtém privilégio, arma navio negreiro, transporta escravos para o
Brasil. Mas, ainda lugar-comum sentimental que lhe enche as horas vagas
do negócio negreiro, sofre de ``vazio interior'', alargado pelas ``solidões
do mar''. Isso em 1694. Em 1695, mais lugares de aventura, combate o
quilombo de Palmares, perde um amigo e salva um jovem, Damião, em um
lugar-comum de heroísmo, no qual recebe no peito a flecha desferida
contra o rapaz por um quilombola. Outros lugares-comuns sentimentais,
gratidão, modéstia, amizade: ``A pobre mãe do oficial quis beijar-lhe as
mãos: -Basta-me um prêmio, disse ele; a sua amizade e a do seu filho''.
Mas a murmuração do povo de Pernambuco o aborrece e vai para a Bahia,
onde casa com D.\,Helena. Repete-se a situação narrativa da união
amorosa: D.\,Helena agora, antes \emph{lady} Ema, anteriormente a mulher
portuguesa, a índia Maracujá no início. Das outras mulheres o Dr.\,Leão
felizmente nada conta, deixando-as para os cinco volumes prometidos; mas
o leitor pode imaginar o que seria a história se ele narrasse todos os
casos de amor do pai com o mesmo lugar-comum do amor romântico:
dedicação, adoração, paixão, traição. Adiante, ele falará ainda sobre
duas espanholas e sua mãe e o leitor poderá ter seu trabalho de
imaginação reduzido, pois serão só quatro mil novecentas e noventa e
três as restantes que não entram na história.

Damião vai à Bahia, leva uma madeixa dos cabelos da mãe morta e um colar
que a moribunda ofereceu a D.\,Helena, lugares-comuns de gratidão. Três
meses depois, Damião e D.\,Helena aplicam em Rui o lugar-comum da
traição: ``Meu pai soube da aleivosia por um comensal da casa. Quis
matá-los; mas o mesmo que os denunciou avisou-os do perigo, e eles
puderam evitar a morte. Meu pai voltou o punhal contra si, e enterrou-o
no coração''. Três ou quatro lugares patéticos se atropelam no trecho: a
revelação da aleivosia, o ultraje da honra, o indizível do desespero, o
tresloucado do ato suicida. Como na Torre de Londres, repete-se a
experiência fantástica: Rui de Leão não pode morrer; foge, vai para o
Sul; no princípio do século \textsc{xviii}, nova aventura, está na descoberta das
minas: ``Era um modo de afogar o desespero, que era grande, pois amara
muito a mulher, como um louco\ldots{}'', Machado faz o Dr.\,Leão ir sendo
falado pelo melhor do \emph{kitsch} romântico. Em 1713, Rui de Leão está
no Rio de Janeiro, rico com as minas e com ideias de ser feito
governador. Repete-se situação narrativa também conhecida do leitor: o
pernambucano que já foi rei da Inglaterra agora deseja governar o Rio.
Aqui, complica-se a complicação sentimental: D.\,Helena retorna,
lugar-comum, mostra-lhe uma carta escrita pelo comensal, outro lugar;
nela, o denunciante pede perdão pela calúnia, mais um, declarando que
mentiu, outro, por ``criminosa paixão'', mais outro, comuníssimo. D.\,Helena volta para ele, trazendo a mãe e o tio; o tempo passa, Rui é
sempre o mesmo, eles envelhecem, morrem. Segundo o Coronel Bertioga,
``vieram ao cheiro dos cobres''. Linhares, também sempre positivo, afirma
que D.\,Helena ``não estava tão inocente como dizia''. Mas faz uma
ressalva, em que novamente aparece o termo ``verdade'' como fundamento de
uma explicação provável: ``É verdade que a carta do denunciante\ldots{}''. Mas
o Dr.\,Leão é peremptório e, explicando a perfídia da ação de D.\,Helena
--- ``O denunciante foi pago para escrever a carta'' ---, também explica por
que pode dar essa explicação: ``{[}\ldots{}{]}meu pai soube disso, depois
da morte da mulher ao passar pela Bahia''.

É meia-noite, o médico tem sono e quer dormir, mas os ouvintes insistem
em que termine a história: ``Mas, senhores\ldots{} Só se for muito por alto.
--- Seja por alto''. Outra vez, os mesmos lugares de aventura: seu pai
deixa o Brasil, passa por Lisboa, vai para a Índia, onde fica cinco anos
fazendo estudos, volta a Portugal, publica-os, é chamado pelas
autoridades, que o nomeiam governador de Goa. Os mesmos lugares-comuns
de intriga, inveja, maledicência e aleivosia são aplicados à situação
narrativa repetida pela terceira vez: antes rei de Inglaterra, depois
quase governador do Rio, agora talvez governador de Goa. Um candidato ao
cargo encomenda a um latinista a falsificação de um texto latino da obra
de Rui de Leão, atribuindo-o a um frade agostinho. A tacha de plagiário
o faz perder o governo de Goa; perde também a consideração pessoal e,
mais aventura, vai para Madri, onde mais lugares, amores com fidalgas
espanholas (romanticamente, as espanholas são morenas como as mouras,
misteriosas como a noite e ardentes como a lava, leitor), ``uma delas
viúva e bonita como o sol, a outra casada, menos bela, porém amorosa e
terna como uma pomba-rola'' etc., são aplicados para de novo engatar-se
neles o lugar-comum da honra: o marido ultrajado da aleivosia não duela
com Rui de Leão para lavar a honra em sangue, mas, lugar-comum de
falsidade vingativa e baixeza de caráter, manda assassiná-lo. Três
punhaladas, quinze dias de cama; um tiro e, como na Torre de Londres e
nas tentativas de suicídio, nada. Novamente, com um lugar-comum de
intriga, o marido o denuncia ao Santo Ofício da Inquisição. O Dr.\,Leão
explica por que o denunciante pôde fazer a denúncia: tinha visto coisas
religiosas da Índia com seu pai e elas lhe forneceram pretexto para
acusá-lo de ser dado a práticas supersticiosas.

Nesse ponto, o leitor bem pode concluir que Machado de Assis trabalha
com poucas situações narrativas básicas e três espécies também básicas
de lugares-comuns, na verdade os mesmos, só lhes variando o recheio.
Como diz o Dr.\,Leão, seu pai acha ``todas as caras novas; e essa troca de
caras {[}\ldots{}{]} dava-lhe a impressão de uma peça teatral, em que o
cenário não muda, e só mudam os atores''. Assim, mudam os atores da
história, mas não a própria história e seus actantes. Nas narrativas,
como o leitor sabe, sempre há um problema, para que os personagens
possam agir superando-o ou sendo vencidos por ele. Nessa, os problemas
vão como que desabando em cascata sobre o personagem, para que ele possa
viajar e meter-se em novas aventuras e problemas nos novos lugares. Os
problemas são diversos, diversas as viagens, diversas as aventuras,
diversos os locais para onde vai, mas sempre há muitos problemas, várias
viagens, inúmeras aventuras e, obviamente, os mesmíssimos
lugares-comuns. O personagem está sempre de tal modo ocupado por eles
que não tem tempo para viver a imortalidade na sua ação sempre exterior.
O mesmo acontece quando ama, tem pretensões políticas ou é vítima da
intriga de inimigos. Com que fim?

O leitor poderá pensar que Machado de Assis aplica os lugares-comuns
funcionalmente, para espichar a história, pois afinal ela é sobre a vida
fantástica de um personagem ``imortal''. E pensará bem, pois o gênero
pressupõe essas complicações. Mas pensará melhor se observar que o
espichamento é produzido redundantemente com os mesmos elementos típicos
do patetismo do romanesco romântico, aventura, intriga, amores. Por
serem exageros aplicados com redundância, tornam cada ponto e o todo do
conto também redundantes e exagerados. O patetismo dessa contínua
agitação exterior é uma deformação; como deformação, as paixões intensas
--- a solidão moral, a paixão amorosa, a honra ultrajada, o desespero
suicida etc., --- que passavam por sublimes, digamos que entre 1830 e
1870, são efetivamente cômicas em 1882. Na estilização, a seriedade
romântica evidencia-se como mera convenção de seriedade tornada
objetivamente ridícula pela marcha das coisas. ``Na verdade'', como diria
o Dr.\,Leão, Machado de Assis pouco se importa com que o leitor creia ou
não na história de ``O Imortal'', pois a crença é um efeito determinado
pelo gênero e será problema só do leitor se não sabe ler e confunde
gêneros literários com a empiria e pensa que personagens são pessoas. Na
verdade, Machado está interessado em parodiar um gênero e um estilo
romanescos em que a complicação sentimental da aventura exterior era
séria, passando em revista sua legibilidade. Por assim dizer, é o conto
que lê os leitores, propondo-lhes que não se trata de saber se a
imortalidade é possível, nem de duvidar da história narrada, mas de
evidenciar que é improvável propô-la diretamente como tema sério, porque
são justamente os modos de escrever sobre ela ou o que for com
lugares-comuns românticos de aventura, intriga e amor fundamentados na
semelhança que se tornaram improváveis, ou seja, inverossímeis. Decerto
seria possível propor a imortalidade indiretamente, como acontece com o
defunto Brás Cubas, que escreve \emph{Memórias Póstumas,} ou o
desmemoriado Bento Santiago, que lembra em \emph{Dom Casmurro.} A
imortalidade seria então um tema sério, como uma metáfora ou alegoria
para outras coisas importantes, como a crítica da vida. Isso porque,
digamos de novo, os lugares ficaram para trás com a mudança histórica
das coisas. Em 1882, as audazes Emas fidalgas dos textos românticos
transformaram-se para valer em pacatas burguesas leitoras de figurinos.
O amor, que antes só queria o absoluto de si mesmo, rendeu-se de vez ao
valor-de-troca. As intrigas e as aleivosias não são as da honra, mas
tornaram-se a estrutura mesma da imprensa, da política e do grande
negócio. E, como no de hoje, nada há de heroico na vida do ``bom povo'' de
então. O conto degrada gêneros e estilos românticos: o conto, a novela,
o poema narrativo, o romance, o folhetim; ironiza personagens típicos: o
herói aventuresco, a heroína apaixonada, o vilão intrigante; critica uma
espécie de ação: a aventura exterior, a complicação; desqualifica o
ideal heroico-erótico, honra, amor, devoção, que os negócios de 1882
tornam improvável.

Nos seis capítulos curtos, a repetição dos lugares incha e deforma o
texto no cômico que o transforma objetivamente em meio para outra coisa.
Como um cabide ou varal sempre esticáveis onde é pendurada a roupa
velha, os estereótipos se dependuram na imortalidade, uns após outros, e
são o que são: mortais. Em 1882, pelo viés de \emph{Memórias Póstumas}
já é impossível não rir de enunciados como: ``Disse-lhe que não me
esquecesse, dei-lhe uma trança de cabelos, pedi-lhe que perdoasse o
carrasco''; ``Era um modo de afogar o desespero, que era grande, pois
amara muito a mulher, como um louco\ldots{}''. Bem antes de \emph{Orlando}, em
que o personagem troca de sexo durante quatrocentos anos para que
Virginia Woolf componha o romance como um painel das mudanças históricas
da vida e da arte desde a Inglaterra elisabetana, no século \textsc{xvi}, até a
vitoriana, no fim do \textsc{xix}, ``O Imortal'' propõe sub-repticiamente ao leitor
do seu tempo a experiência da historicidade dos modos de escrever e
consumir ficção.

Correndo ao lado do tema da imortalidade, que provavelmente é o que mais
chama a atenção porque a história é fantástica, o tema na verdade
principal de ``O Imortal'' é subterrâneo e decisivo, porque corrosivo e
destruidor da representação: a impossibilidade moderna de contar
histórias em que a aventura, o amor, a intriga e a intervenção de causas
maravilhosas, como a beberagem do pajé, sejam temas sérios e também
causas ou motivos propostos como explicações naturais, habituais e
normais. O verdadeiro tema de ``O Imortal'' é a verossimilhança.

O texto pode ser relacionado diretamente com passagens dos \emph{Tópicos
I}, da \emph{Retórica} e da \emph{Poética,} onde Aristóteles escreve
sobre a atividade do historiador e do poeta, prescrevendo que devem
compor imitando as opiniões tidas por verdadeiras pelos sábios ou pela
maioria deles. As opiniões tidas por verdadeiras fornecem causas e
explicações que tornam o discurso \emph{verossímil} ou semelhante ao
verdadeiro da opinião. A história e a antropologia demonstram, como o
leitor sabe, que os critérios de ``verdadeiro'' são variáveis ao longo dos
tempos. ``A ciência de um século não sabia tudo'', diz o Dr.\,Leão. Em
Roma, por exemplo, o amor de escravo por patrícia não encontrava para
apoiá-lo nenhuma opinião estabelecida que o definisse como algo
``verdadeiro'', por isso era tido como improvável, sendo proposto como
assunto do ridículo, o pequeno riso da comédia. Definido como
não-natural, não-habitual e não-normal, era classificado como
inverossímil. Mas a mesma inverossimilhança era adequada no gênero
cômico, pois fazia rir com a desproporção. A partir do século \textsc{xviii}, na
Inglaterra, o tema do inferior que se apaixona pelo superior passou a
gozar de grande prestígio romântico, tornando-se natural, habitual e
normal um gênero de conto, novela, romance e poema que tratam dele de
modo não-cômico, mas sério, ingênuo-patético- sentimental.

De todo modo, é útil lembrar: \emph{a verossimilhança é uma relação de
semelhança entre discursos}. Ou seja: a verossimilhança decorre da
relação do texto de ficção não com a realidade empírica da sociedade do
autor, mas da sua relação com outros discursos da sua cultura, que
funcionam como explicações ou causas da história narrada, tornando-a
adequada àquilo que se considera natural, habitual e normal que aconteça
\emph{na} realidade e \emph{como} realidade. A ficção é verossímil
quando o leitor reconhece os códigos que julga verdadeiros e que são
aplicados pelo autor para motivar as ações da história. O verossímil
\emph{motiva} a ficção, ou seja, fornece motivos para as ações.
Aristotelicamente, cada gênero tem uma verossimilhança específica,
aplicando motivos particulares como explicação e causa das ações. O
discurso da história sempre começa pelo início da ação narrada, indo do
mais recuado no passado em direção ao presente em que é escrito,
seguindo uma ordem definida como natural. A poesia épica começa com a
ação pela metade, seguindo uma ordem artificial. O fantástico narra
ações impossíveis. Na tragédia, os personagens devem ser melhores que o
espectador, ao passo que a história não tem que melhorar a vida dos
homens. Na comédia, os personagens devem ser piores que o público etc.
Tradicionalmente, usar os motivos específicos que conferem
verossimilhança a um gênero para compor o verossímil de outro era
definido como inépcia artística e inverossimilhança. Por exemplo,
aplicar a baixeza do caráter dos personagens da comédia para escrever
uma tragédia. Tal uso só era admitido como a ``licença poética'' pela qual
as incongruências fingidas tinham por finalidade parodiar as convenções
do gênero imitado e causar riso. Assim, segundo o preceito aristotélico
do ``semelhante ao verdadeiro'', num primeiro momento ``O Imortal'' aparece
como inverossímil, pois não pode ser comparado com nenhuma opinião sobre
o assunto ``morte'' que possa ser tida como verdadeira. Segundo a mesma
concepção, no entanto, o que é inverossímil em um gênero torna-se
adequado ou verossímil ao gênero fantástico, que se ocupa justamente de
narrar coisas falsas e improváveis no registro da ``licença poética''. É o
caso de ``O Imortal'', em que três critérios históricos de verossimilhança
aparecem superpostos.

Um deles é a verossimilhança do gênero fantástico, apropriada por
Machado de Assis da longa tradição satírica de Luciano de Samósata. No
caso, o verossímil é construído por meio de ações e eventos falsos,
improváveis e inverossímeis, pois esse é o ``verdadeiro'' da opinião que
se tem sobre as convenções do gênero fantástico. Nesse sentido, logo no
início do conto, quando um dos ouvintes corrige a data do nascimento do
pai do Dr.\,Leão, tenta mudar o registro da narração, transformando o
fantástico que começa a ser contado em gênero realista baseado em
opiniões tidas por ``naturais''. Mas o Dr.\,Leão insiste e mantém a
história como gênero fantástico. Assim, quando diz que ela não é fácil
de crer, joga com a dupla perspectiva da recepção, que já apareceu na
correção feita por um dos ouvintes: a história não é fácil de crer, se
for lida por meio da verossimilhança positivista-realista; mas é
totalmente crível se for lida como gênero fantástico, que aplica
convenções críveis para narrar o incrível.

Outra espécie de verossimilhança que organiza o texto é a da ficção
romântica, que exigia causas, explicações e motivos ideais e idealistas
para as ações. Romanticamente, o amor, o heroísmo, a honra etc. são
opiniões verdadeiras como causas alegáveis para explicar qualquer ação.
No final do mesmo século \textsc{xix}, num momento em que todos os sistemas de
representação foram abalados pelo capital, novos critérios de definição
de ``verdadeiro'' passaram a reger a legibilidade da literatura e a
visibilidade das artes plásticas, tornando a verossimilhança romântica
improvável, inverossímil e, logo, cômica. Obviamente, para as leitoras
de \emph{A estação,} que talvez tenham lido ``O Imortal'', os motivos
românticos estilizados no conto pareciam naturais, porque a cultura é a
única natureza possível para os homens. O que costuma ocorrer é que o
leitor de literatura geralmente sofre de etnocentrismo ingênuo, pois
quase nunca pensa que sua cultura não é natural, como uma
particularidade entre outras, tendendo a generalizá-la como critério
universal de avaliação, como se fosse ``verdadeira'' para todos os tempos
e lugares. Pode-se supor que a leitora de \emph{A estação} lia ``O
Imortal'' desse modo: naturalmente, como você e eu, com uma atenção
delicadamente flutuante voltada para o enredo, ``o que ele quis dizer'',
não para a técnica ou para a crítica que o texto efetua. Só quando é
flagrantemente inverossímil o leitor percebe o artifício da ficção,
podendo pensar, quando pensa, as duas coisas ditas antes: o escritor é
incompetente, não conhece as regras da sua arte e escreve mal ou a
inverossimilhança é tão óbvia que deve ser proposital, tendo um sentido
que ainda deve ser achado.

Para especificar essa inverossimilhança produzida programaticamente pelo
escritor, é útil insistir em que a ficção não é a vida empírica,
confusão naturalista, pois esta não tem nenhum sentido predeterminado. A
ficção imita outra coisa, os discursos que regulam a vida, devendo ser
absolutamente lógica no modo como os imita para fazer sentido mesmo
quando seu efeito é a total falta de sentido. Quando conta uma estória,
o narrador constrói sequências somando palavras: ``Meu'', ``Meu pai'', ``Meu
pai nasceu'', ``Meu pai nasceu em\ldots{}''. Evidentemente, como diziam os
formalistas russos, diz alguma coisa antes para relacioná-la
funcionalmente com outra que vem depois, por isso inúmeras
circunstâncias poderiam ser usadas pelo Dr.\,Leão com a preposição ``em'':
``em Recife'', ``em um lugar distante'', ``em 1800'' etc\ldots{} No caso, Machado o
faz dizer ``em 1600''; a data exclui todas as outras circunstâncias e, ao
mesmo tempo, produz a necessidade de dar continuidade a uma sequência
que vai diretamente de encontro a uma expectativa realista fundamentada
em opiniões ``verdadeiras'', como as de João Linhares e Bertioga, que
julgam falsa, e com razão, a ideia de que um homem tenha nascido em 1600
e seja pai do personagem que lhes fala em 1855. Se o narrador afirma:
``Meu pai nasceu em'', eles esperam, segundo a opinião que fazem sobre ``o
verdadeiro'', que apareça algo provável e, por isso mesmo, previsível ---
``em 1800, naturalmente''. Ou seja: acreditam que ouvir ou ler uma
história significa \emph{reconhecer} algo já ouvido ou lido antes,
\emph{naturalmente}. Para eles, a semelhança é tudo. Como o personagem
insiste em afirmar ``em 1600'', também o leitor acha que isso não é
``natural''. O improvável do ``não-natural'' é imprevisível, por isso o
leitor fica imediatamente avisado de que ou o personagem mente, ou o
personagem é inepto ou o personagem lhe está propondo outro esquema
retórico, outro gênero literário e outra legibilidade.

A literatura moderna, como a de Machado de Assis a partir de
\emph{Memórias Póstumas de Brás Cubas}, fez desse arbitrário da direção
narrativa um dos seus eixos principais, produzindo a imprevisibilidade
que \emph{desnaturaliza} os modos habituais de ler. A desnaturalização
incide diretamente sobre a verossimilhança. Gerard Genette propôs que
há, basicamente, três graus da verossimilhança aplicáveis às
narrativas\footnote{Cf. Genette, Gérard. ``Verossímil e Motivação''. In
  Barthes, Roland \emph{et alii}. \emph{Literatura e Semiologia}.
  Seleção de ensaios da revista ``Communications''. Trad. Célia Neves
  Dourado. Petrópolis: Vozes, 1971.}. Nenhum deles é melhor ou pior e
todos podem ocorrer, mas a literatura moderna prefere um deles, como se
verá adiante. O primeiro caso é o de um ``grau zero'' de marcas do
verossímil. O discurso não apresenta quase nenhuma explicação ou causa
das ações dos personagens e a ausência de explicação corresponde
justamente à suposição, partilhada pelo narrador e leitor, de que o
narrado é totalmente natural, habitual e normal. É o caso do exemplo de
início idiota de narrativa dado por Valéry, ``A marquesa saiu às 5h'',
que é um enunciado tido como natural, habitual e normal, não
necessitando de nenhuma explicação, pois a existência de marquesas é um
fato, existe o hábito de sair e a hora, 5 da tarde, não parece
extraordinária ou inconveniente. Da mesma maneira, se o Dr.\,Leão
dissesse ``Meu pai nasceu em 1800'', nenhuma explicação seria necessária e
nenhum dos ouvintes interviria para tentar corrigi-lo, como ocorre na
segunda fala do conto.

Genette propõe como segundo grau de verossimilhança aquele em que
aparecem explicações motivando o que é narrado. As explicações
particularizam ou generalizam os motivos da ação. Por exemplo, quando o
Dr.\,Leão explica por que o nobre espanhol não quis duelar com seu pai,
dá uma explicação generalizante, por assim dizer ``sociológica'', o
código de honra da aristocracia ibérica, que impede, em geral, que nobre
suje as mãos com sangue plebeu em duelos. Uma explicação
particularizante ocorre quando o Dr.\,Leão diz que o marido traído pôde
denunciar seu pai à Inquisição porque tinha visto os objetos que tinha
trazido da Índia. Ou, ainda, quando desautoriza a interpretação que João
Linhares faz do comportamento de D.\,Helena, afirmando que seu pai ficou
sabendo que ela havia contratado o comensal para escrever a carta etc.
Tais explicações funcionam bem, pois correspondem às opiniões do leitor
sobre ``aristocracia'', ``vingança'', ``honra'', ``aleivosia'', ``cartas
anônimas'', ``maridos traídos'' etc. etc., podendo-se dizer que o leitor as
espera para que o artifício narrativo seja ``natural''.

O terceiro grau pode ser o mais interessante, no caso de ``O Imortal''.
Novamente, vejamos a fala inicial do Dr.\,Leão: ``Meu pai nasceu em
1600\ldots{}''. Se o leitor se lembra, desde que o médico afirma que sua
história não é fácil de crer, outros enunciados sem explicação vão sendo
justapostos a ``Meu pai nasceu em 1600\ldots{}'', como é o caso dos enunciados
referentes à poção que torna imortal. A não ser a explicação de Pirajuá,
que diz a Rui de Leão que foi preparada por um pajé de longe, o leitor
lê sobre o efeito da bebida e sobre coisas, ações e acontecimentos sem
explicação, passando para outros também sem esclarecimentos. Antes de
começar a história, o Dr.\,Leão adverte seus ouvintes de que não pode
``entrar em pormenores'' e, com isso, motiva ou explica o arbitrário da
falta de explicação ou motivação para muitas ações e para o encadeamento
delas. Os formalistas russos do início do século \textsc{xx} chamavam de
``procedimento a nu'' a técnica que representa para o leitor o próprio ato
que constrói o discurso, ou seja, as decisões do narrador, que conta sem
explicar. Esse procedimento que narra o insólito sem explicação é
nuclear na literatura moderna, que o aplica para criticar, negar e
destruir os sistemas causais de interpretação que o leitor julga
naturais, evidenciando a particularidade e a arbitrariedade deles num
mundo em que ``opiniões verdadeiras'' são ideologia. ``Certa manhã, Gregor
Samsa acordou\ldots{}'', escreve Kafka, sem nenhuma explicação, o que faz com
que o texto seja literal. É verdade, no entanto, que o Dr.\,Leão fala
várias vezes a expressão ``na verdade'', quando alega causas com que tenta
tornar plausível a narração, adequando a história às opiniões
``verdadeiras'' dos seus ouvintes.

Historicamente, a noção aristotélica de verossimilhança teve vigência
enquanto se acreditou que existia adequação substancial entre os signos
da linguagem, os conceitos e as estruturas da realidade empírica. No
final do século \textsc{xix}, como disse, os sistemas de representação
considerados suficientes até então para estabelecer essa adequação, como
a linguagem, entraram em crise e, com eles, a literatura, que deixou de
ser uma reprodução previsível de opiniões tidas como verdadeiras.
Deixando de ser semelhante às opiniões tidas por ``verdadeiras'', passou a
ser escrita como transformação dos próprios meios técnicos de produzir
literatura. Produzindo efeitos de sentido a partir de si mesma, ela
passou a chamar a atenção do leitor para o seu próprio artifício,
evidenciando-se como produto arbitrário, sem relação necessária com o
que se entendia por ``verdadeiro''.

A partir de \emph{Memórias Póstumas de Brás Cubas}, Machado de Assis
passou a escrever ficção como dispositivo que dissolve o princípio de
causalidade da verossimilhança. É por isso que, nesse livro, sua arte
aparece para o leitor como funcionalidade do procedimento a nu, que
torna indecidível o sentido da história narrada. Logo, na obra de uma
imaginação sempre racionalmente controlada como é a sua, a própria noção
de ``realidade'' torna-se crítica, pois são abaladas as opiniões tidas
como ``verdadeiras'' que o leitor tem da mesma. Devemos lembrar que a
falta de sentido da dor humana, a loucura, o desencontro e a desarmonia
do universo são temas obsessivamente tratados por ele, que escreveu num
tempo em que a ideologia evolucionista de ``nação'', ``ordem'' e ``progresso''
afirmava o darwinismo social como verdade científica que classificava e
excluía a gente explorada branca, negra, índia e mestiça como sub-raça.
Ao destruir as semelhanças previsíveis que pressupõem a naturalidade e a
normalidade das ``opiniões verdadeiras'' como fundamento da ação, as
histórias contadas pelos narradores machadianos são como palcos onde se
encena a inversão sistemática das convenções ``verdadeiras'' do leitor.
Sistematicamente, seus narradores opõem e invertem os termos de
\emph{realidade/aparência}, \emph{razão/loucura},
\emph{ideal/interesse}, \emph{verdade/falsidade},
\emph{verossimilhança/inverossimilhança}, que organizavam a
racionalidade das práticas de seu tempo. Esse modo de neutralizar as
significações familiares e previsíveis, que é observável no seu
compromisso exclusivamente artístico com a forma, talvez pretendesse a
autonomia de uma liberdade estética que recusa a instrumentalização da
arte, inclusive a ideologia naturalista da literatura como semelhança
refletora da realidade empírica, que com ele se torna indeterminada. A
ficção escrita como questionamento da possibilidade da existência da
ficção é um dos temas privilegiados da sua arte inventada como uma
singular teoria da enunciação. Como ``A Chinela Turca'', ``Singular
Ocorrência'', ``A Cartomante'' e outros contos, ``O Imortal'' joga com o
arbitrário de direção narrativa, dissolvendo a verossimilhança
tradicional por meio da estilização e paródia da mesma como gênero
cômico.

Mas o que ocorreu com Rui de Leão? preso pela Inquisição espanhola,
temeu inicialmente ficar detido para sempre; depois, acreditou que o
Santo Ofício o soltaria quando descobrisse que não morria; finalmente,
que seria um alívio ficar livre do ``espetáculo exterior'' do mundo etc.
Para encurtar, ele finalmente morreu em 1855. Como? \emph{Similia
similibus curantur}: \emph{os semelhantes são curados com os
semelhantes}. A caracterização do Dr.\,Leão como homeopata, no início do
conto, tem sua função revelada. Um dia, ouvindo o filho falar sobre a
homeopatia, Rui de Leão tem a ideia de beber novamente a poção e morre.
Como Ella, a feiticeira apaixonada pelo Calícrates da novela de Haggard,
que também morre, quando entra pela segunda vez no fogo sagrado que lhe
deu a imortalidade.

Falta talvez explicar a própria narrativa do Dr.\,Leão. Por que ele
contou tal história? O narrador afirma que a seriedade do médico era tão
profunda que Bertioga e Linhares ``creram no caso, e creram também
definitivamente na homeopatia''. Aqui, o narrador pluraliza as
explicações para reconduzir o leitor à questão da verossimilhança:

``Narrada a história a outras pessoas, não faltou quem supusesse que o
médico era louco; outros atribuíram-lhe o intuito de tirar ao coronel e
ao tabelião o desgosto manifestado por ambos de poderem viver
eternamente, mostrando-lhes que a morte é, enfim, um benefício. Mas a
suspeita de que ele apenas quis propagar a homeopatia entrou em alguns
cérebros, e não era inverossímil''.

Três opiniões tidas pelo leitor como fundamentos válidos de explicações
``verdadeiras'' são mobilizadas pelo narrador para explicar a causa da
ação do homeopata: a irracionalidade, a generosidade, o interesse. O
narrador não toma partido de nenhuma delas, deixando a solução em
aberto, talvez porque há uma quarta: o Dr.\,Leão conta a história da
imortalidade do pai não \emph{porque} é louco, generoso ou interesseiro,
mas \emph{para} Machado de Assis demonstrar ao leitor que a
historicidade do artifício da verossimilhança é mortal.

\chapter{O Imortal\footnote[*]{O texto a seguir foi cotejado com a versão
  em livro de ``O Imortal'', que integra a coletânea \emph{Contos
  Avulsos}, disponibilizada na \emph{Obra Completa} de Machado de Assis,
  reeditada em 2015 pela Nova Aguilar.}}

\section*{Capítulo Primeiro\protect\footnote{``O
  Imortal'' foi publicado em livro na coletânea \emph{Contos Avulsos}
  (Cf. Machado de Assis. \emph{Obra Completa}. v. 3. Rio de Janeiro:
  Nova Aguilar, 2015, pp. 64--77), postumamente.}}

--- Meu pai nasceu em 1600\ldots{}

--- Perdão, em 1800, naturalmente\ldots{}

--- Não, senhor, replicou o Dr.\,Leão\footnote{Alusão ao conto ``Ruy de
  Leão'', publicado pelo próprio autor no \emph{Jornal das Famílias}, em
  1872.}, de um modo grave e triste; foi em 1600.

Estupefação dos ouvintes, que eram dois, o Coronel Bertioga, e o
tabelião da vila, João Linhares. A vila era na província fluminense;
suponhamos Itaboraí ou Sapucaia. Quanto à data, não tenho dúvida em
dizer que foi no ano de 1855, uma noite de novembro, escura como breu,
quente como um forno, passante de nove horas. Tudo silêncio. O lugar em
que os três estavam era a varanda que dava para o terreiro. Um lampião
de luz frouxa, pendurado de um prego, sublinhava a escuridão exterior.
De quando em quando, gania um seco e áspero vento, mesclando-se ao som
monótono de uma cachoeira próxima. Tal era o quadro e o momento, quando
o Dr.\,Leão insistiu nas primeiras palavras da narrativa.

--- Não, senhor; nasceu em 1600.

Médico homeopata, --- a homeopatia começava a entrar nos domínios da
nossa civilização, --- este Dr.\,Leão chegara à vila, dez ou doze dias
antes, provido de boas cartas de recomendação, pessoais e políticas. Era
um homem inteligente, de fino trato e coração benigno. A gente da vila
notou-lhe certa tristeza no gesto, algum retraimento nos hábitos, e até
uma tal ou qual sequidão de palavras, sem embargo da perfeita cortesia;
mas tudo foi atribuído ao acanho dos primeiros dias e às saudades da
Corte. Contava trinta anos, tinha um princípio de calva, olhar baço e
mãos episcopais. Andava propagando o novo sistema.

Os dois ouvintes continuavam pasmados. A dúvida fora posta pelo dono da
casa, o Coronel Bertioga, e o tabelião ainda insistiu no caso, mostrando
ao médico a impossibilidade de ter o pai nascido em 1600. Duzentos e
cinquenta e cinco anos antes! dois séculos e meio! Era impossível.
Então, que idade tinha ele? e de que idade morreu o pai?

--- Não tenho interesse em contar-lhes a vida de meu pai, respondeu o
Dr.\,Leão. Falaram-me no macróbio que mora nos fundos da matriz;
disse-lhes que, em negócio de macróbios, conheci o que há mais espantoso
no mundo, um homem imortal\ldots{}

--- Mas seu pai não morreu? disse o coronel.

--- Morreu.

--- Logo, não era imortal, concluiu o tabelião triunfante. Imortal se
diz quando uma pessoa não morre, mas seu pai morreu.

--- Querem ouvir-me?

--- Homem, pode ser, observou o coronel meio abalado. O melhor é ouvir
a história. Só o que digo é que mais velho do que o Capataz nunca vi
ninguém. Está mesmo caindo de maduro. Seu pai devia estar também muito
velho\ldots{}?

--- Tão moço como eu. Mas para que me fazem perguntas soltas? Para se
espantarem cada vez mais, porque na verdade a história de meu pai não é
fácil de crer. Posso contá-la em poucos minutos.

Excitada a curiosidade, não foi difícil impor-lhes silêncio. A família
toda estava acomodada, os três eram sós na varanda, o dr. Leão contou
enfim a vida do pai, nos termos em que o leitor vai ver, se se der o
trabalho de ler o segundo e os outros capítulos.



\section*{ii}



--- Meu pai nasceu em 1600, na cidade de Recife.

Aos vinte e cinco anos tomou o hábito franciscano, por vontade de minha
avó, que era profundamente religiosa. Tanto ela como o marido eram
pessoas de bom nascimento, --- ``bom sangue'', como dizia meu pai,
afetando a linguagem antiga.

Meu avô descendia da nobreza de Espanha, e minha avó era de uma grande
casa do Alentejo. Casaram-se ainda na Europa, e, anos depois, por
motivos que não vêm ao caso dizer, transportaram-se ao Brasil, onde
ficaram e morreram. Meu pai dizia que poucas mulheres tinha visto tão
bonitas como minha avó. E olhem que ele amou as mais esplêndidas
mulheres do mundo. Mas não antecipemos.

Tomou meu pai o hábito, no convento de Iguaraçu, onde ficou até 1639,
ano em que os holandeses, ainda uma vez, assaltaram a povoação. Os
frades deixaram precipitadamente o convento; meu pai, mais remisso do
que os outros (ou já com o intento de deitar o hábito às urtigas),
deixou-se ficar na cela, de maneira que os holandeses o foram achar no
momento em que recolhia alguns livros pios e objetos de uso pessoal. Os
holandeses não o trataram mal. Ele os regalou com o melhor da ucharia
franciscana, onde a pobreza é de regra. Sendo uso daqueles frades
alternarem-se no serviço da cozinha, meu pai entendia da arte, e esse
talento foi mais um encanto ao aparecer do inimigo.

No fim de duas semanas, o oficial holandês ofereceu-lhe um
salvo-conduto, para ir aonde lhe parecesse; mas meu pai não o aceitou
logo, querendo primeiro considerar se devia ficar com os holandeses, e,
à sombra deles desamparar a Ordem, ou se lhe era melhor buscar vida por
si mesmo. Adotou o segundo alvitre, não só por ter o espírito
aventureiro, curioso e audaz, como porque era patriota, e bom católico,
apesar da repugnância à vida monástica, e não quisera misturar-se com o
herege invasor. Aceitou o salvo-conduto e deixou Iguaraçu.

Não se lembrava ele, quando me contou essas coisas, não se lembrava
mais do número de dias que despendeu sozinho por lugares ermos, fugindo
de propósito ao povoado, não querendo ir a Olinda ou Recife, onde
estavam os holandeses. Comidas as provisões que levava, ficou dependente
de alguma caça silvestre e frutas. Deitara, com efeito, o hábito às
urtigas; vestia uns calções flamengos, que o oficial lhe dera, e uma
camisola ou jaquetão de couro. Para encurtar razões, foi ter a uma
aldeia de gentio, que o recebeu muito bem, com grandes carinhos e
obséquios. Meu pai era talvez o mais insinuante dos homens. Os índios
ficaram embeiçados por ele, mormente o chefe, um guerreiro velho, bravo
e generoso, que chegou a dar-lhe a filha em casamento. Já então minha
avó era morta, e meu avô desterrado para a Holanda, notícias que meu pai
teve, casualmente, por um antigo servo da casa. Deixou-se estar, pois,
na aldeia, o gentio, até o ano de 1642, em que o guerreiro faleceu. Este
caso do falecimento é que é maravilhoso: peço-lhes a maior atenção.

O coronel e o tabelião aguçaram os ouvidos, enquanto o Dr.\,Leão extraía
pausadamente uma pitada\footnote{O hábito de cheirar rapé era bem comum
  nos círculos sociais frequentados pela elite social, no século \textsc{xix}.} e
inseria-a no nariz, com a pachorra de quem está negaceando uma coisa
extraordinária.



\section*{iii}



Uma noite, o chefe indígena, --- chamava-se Pirajuá\footnote{Do tupi
  \emph{Pirá} (peixe) + \emph{Juá} (fruta de espinho).}, --- foi à rede
de meu pai, anunciou-lhe que tinha de morrer, pouco depois de nascer o
sol, e que ele estivesse pronto para acompanhá-lo fora, antes do momento
último. Meu pai ficou alvoroçado, não por lhe dar crédito, mas por
supô-lo delirante. Sobre a madrugada, o sogro veio ter com ele.

--- Vamos, disse-lhe.

--- Não, agora não: estás fraco, muito fraco\ldots{}

--- Vamos! repetiu o guerreiro.

E, à luz de uma fogueira expirante, viu-lhe meu pai a expressão
intimativa do rosto, e um certo ar diabólico, em todo caso
extraordinário, que o aterrou. Levantou-se, acompanhou-o na direção de
um córrego. Chegando ao córrego, seguiram pela margem esquerda, acima,
durante um tempo que meu pai calculou ter sido um quarto de hora. A
madrugada acentuava-se; a lua fugia diante dos primeiros anúncios do
sol. Contudo, e apesar da vida do sertão que meu pai levava desde alguns
tempos, a aventura assustava-o; seguia vigiando o sogro, com receio de
alguma traição. Pirajuá ia calado, com os olhos no chão, e a fronte
carregada de pensamentos, que podiam ser cruéis ou somente tristes. E
andaram, andaram, até que Pirajuá disse:

--- Aqui.

Estavam diante de três pedras, dispostas em triângulo. Pirajuá
sentou-se numa, meu pai noutra. Depois de alguns minutos de descanso:

--- Arreda aquela pedra, disse o guerreiro, apontando para a terceira,
que era a maior.

Meu pai levantou-se e foi à pedra. Era pesada, resistiu ao primeiro
impulso; mas meu pai teimou, aplicou todas as forças, a pedra cedeu um
pouco, depois mais, enfim foi removida do lugar.

--- Cava o chão, disse o guerreiro.

Meu pai foi buscar uma lasca de pau, uma taquara ou não sei que, e
começou a cavar o chão. Já então estava curioso de ver o que era.
Tinha-lhe nascido uma ideia, --- algum tesouro enterrado, que o
guerreiro, receoso de morrer, quisesse entregar-lhe. Cavou, cavou,
cavou, até que sentiu um objeto rijo; era um vaso tosco, talvez uma
igaçaba. Não o tirou, não chegou mesmo a arredar a terra em volta dele.
O guerreiro aproximou-se, desatou o pedaço de couro de anta que lhe
cobria a boca, meteu dentro o braço, e tirou um boião. Este boião tinha
a boca tapada com outro pedaço de couro.

--- Vem cá, disse o guerreiro.

Sentaram-se outra vez. O guerreiro tinha o boião sobre os joelhos,
tapado, misterioso, aguçando a curiosidade de meu pai, que ardia por
saber o que havia ali dentro.

--- Pirajuá vai morrer, disse ele; vai morrer para nunca mais. Pirajuá
ama guerreiro branco, esposo de Maracujá, sua filha; e vai mostrar um
segredo como não há outro.

Meu pai estava trêmulo. O guerreiro desatou lentamente o couro que
tapava o boião. Destapado, olhou para dentro, levantou-se, e veio
mostrá-lo a meu pai. Era um líquido amarelado, de um cheiro acre e
singular.

--- Quem bebe isto, um gole só, nunca mais morre.

--- Oh! bebe, bebe! exclamou meu pai com vivacidade.

Foi um movimento de afeto, um ato irrefletido de verdadeira amizade
filial, porque só um instante depois é que meu pai advertiu que não
tinha, para crer na notícia que o sogro lhe dava, senão a palavra do
mesmo sogro, cuja razão supunha perturbada pela moléstia. Pirajuá sentiu
o espontâneo da palavra de meu pai, e agradeceu-lha; mas abanou a
cabeça.

--- Não, disse ele; Pirajuá não bebe, Pirajuá quer morrer. Está
cansado, viu muita lua, muita lua. Pirajuá quer descansar na terra, está
aborrecido. Mas Pirajuá quer deixar este segredo a guerreiro branco;
está aqui; foi feito por um velho pajé de longe, muito longe\ldots{}
Guerreiro branco bebe, não morre mais.

Dizendo isto, tornou a tapar a boca do boião, e foi metê-lo outra vez
dentro da igaçaba. Meu pai fechou depois a boca da mesma igaçaba, e
repôs a pedra em cima. O primeiro clarão do sol vinha apontando.
Voltaram para casa depressa; antes mesmo de tomar a rede, Pirajuá
faleceu.

Meu pai não acreditou na virtude do elixir. Era absurdo supor que um tal
líquido pudesse abrir uma exceção na lei da morte. Era naturalmente
algum remédio, se não fosse algum veneno; e neste caso, a mentira do
índio estava explicada pela turvação mental que meu pai lhe atribuiu.
Mas, apesar de tudo, nada disse aos demais índios da aldeia, nem à
própria esposa. Calou-se; --- nunca me revelou o motivo do silêncio:
creio que não podia ser outro senão o próprio influxo do mistério.

Tempos depois, adoeceu, e tão gravemente que foi dado por perdido. O
curandeiro do lugar anunciou a Maracujá que ia ficar viúva. Meu pai não
ouviu a notícia, mas leu-a em uma página de lágrimas, no rosto da
consorte, e sentiu em si mesmo que estava acabado. Era forte, valoroso,
capaz de encarar todos os perigos; não se aterrou, pois, com a ideia de
morrer, despediu-se dos vivos, fez algumas recomendações e preparou-se
para a grande viagem.

Alta noite, lembrou-se do elixir, e perguntou a si mesmo se não era
acertado tentá-lo. Já agora a morte era certa, que perderia ele com a
experiência? A ciência de um século não sabia tudo; outro século vem e
passa adiante. Quem sabe, dizia ele consigo, se os homens não
descobrirão um dia a imortalidade, e se o elixir científico não será
esta mesma droga selvática\footnote{Da selva, selvagem --- a sugerir
  oposição entre natureza e ciência.}? O primeiro que curou a febre
maligna fez um prodígio. Tudo é incrível antes de divulgado. E, pensando
assim, resolveu transportar-se ao lugar da pedra, à margem do arroio;
mas não quis ir de dia, com medo de ser visto. De noite, ergueu-se, e
foi, trôpego, vacilante, batendo o queixo. Chegou à pedra, arredou-a,
tirou o boião, e bebeu metade do conteúdo. Depois sentou-se para
descansar. Ou o descanso, ou o remédio, alentou-o logo. Ele tornou a
guardar o boião; daí a meia hora estava outra vez na rede. Na seguinte
manhã estava bom\ldots{}

--- Bom de todo? perguntou o tabelião João Linhares, interrompendo o
narrador.

--- De todo.

--- Era algum remédio para febre\ldots{}

Foi isto mesmo o que ele pensou, quando se viu bom. Era algum remédio
para febre e outras doenças; e nisto ficou; mas, apesar do efeito da
droga, não a descobriu a ninguém. Entretanto, os anos passaram, sem que
meu pai envelhecesse; qual era no tempo da moléstia, tal ficou. Nenhuma
ruga, nenhum cabelo branco. Moço, perpetuamente moço. A vida do mato
começara a aborrecê-lo; ficara ali por gratidão ao sogro; as saudades da
civilização vieram tomá-lo. Um dia, a aldeia foi invadida por uma horda
de índios de outra, não se sabe por que motivo, nem importa ao nosso
caso. Na luta pereceram muitos, meu pai foi ferido, e fugiu para o mato.
No dia seguinte veio à aldeia, achou a mulher morta. As feridas eram
profundas; curou-as com o emprego de remédios usuais; e restabeleceu-se
dentro de poucos dias. Mas os sucessos confirmaram-no no propósito de
deixar a vida semi-selvagem e tornar à vida civilizada e cristã. Muitos
anos se tinham passado depois da fuga do convento de Iguaraçu; ninguém
mais o reconheceria. Um dia de manhã deixou a aldeia, com o pretexto de
ir caçar; foi primeiro ao arroio, desviou a pedra, abriu a igaçaba,
tirou o boião, onde deixara um resto do elixir. A ideia dele era fazer
analisar a droga na Europa, ou mesmo em Olinda ou no Recife, ou na
Bahia, por algum entendido em coisas de química e farmácia. Ao mesmo
tempo não podia furtar-se a um sentimento de gratidão; devia àquele
remédio a saúde. Com o boião ao lado, a mocidade nas pernas e a
resolução no peito, saiu dali, caminho de Olinda e da eternidade.





\section*{iv}



--- Não posso demorar-me em pormenores, disse o Dr.\,Leão aceitando o
café que o coronel mandara trazer. São quase dez horas\ldots{}

--- Que tem? perguntou o coronel. A noite é nossa; e, para o que temos
de fazer amanhã, podemos dormir quando bem nos parecer. Eu por mim não
tenho sono. E você, Sr.\,João Linhares?

--- Nem um pingo, respondeu o tabelião.

E teimou com o Dr.\,Leão para contar tudo, acrescentando que nunca ouvira
nada tão extraordinário. Note-se que o tabelião presumia ser lido em
histórias antigas, e passava na vila por um dos homens mais ilustrados
do Império; não obstante, estava pasmado. Ele contou ali mesmo, entre
dois goles de café, o caso de Matusalém, que viveu novecentos e sessenta
e nove anos, e o de Lameque\footnote{Matusalém e seu filho, Lameque, são
  personagens do ``Gênesis'', \emph{Antigo Testamento}. Teriam vivido
  centenas de anos.} que morreu com setecentos e setenta e sete; mas,
explicou logo, porque era um espírito forte, que esses e outros exemplos
da cronologia hebraica não tinham fundamento científico\ldots{}

--- Vamos, vamos ver agora o que aconteceu a seu pai, interrompeu o
coronel.

O vento, de esfalfado, morrera; e a chuva começava a rufar nas folhas
das árvores, a princípio com intermitências, depois mais contínua e
basta. A noite refrescou um pouco. O Dr.\,Leão continuou a narração, e,
apesar de dizer que não podia demorar-se nos pormenores, contou-os com
tanta miudeza, que não me atrevo a pô-los tais quais nestas páginas;
seria fastidioso. O melhor é resumi-lo.

Rui de Leão, ou antes Rui Garcia de Meireles e Castro Azevedo de Leão,
que assim se chamava o pai do médico, pouco tempo se demorou em
Pernambuco. Um ano depois, em 1654, cessava o domínio holandês. Rui de
Leão assistiu às alegrias da vitória, e passou-se ao reino, onde casou
com uma senhora nobre de Lisboa. Teve um filho; e perdeu o filho e a
mulher no mesmo mês de março de 1661. A dor que então padeceu foi
profunda; para distrair-se visitou a França e a Holanda. Mas na Holanda,
ou por motivo de uns amores secretos, ou por ódio de alguns judeus
descendentes ou naturais de Portugal, com quem entreteve relações
comerciais na Haia, ou enfim por outros motivos desconhecidos, Rui de
Leão não pôde viver tranquilo muito tempo; foi preso e conduzido para a
Alemanha, de onde passou à Hungria, a algumas cidades italianas, à
França, e finalmente à Inglaterra. Na Inglaterra estudou o inglês
profundamente; e, como sabia o latim, aprendido no convento, o hebraico,
que lhe ensinara na Haia o famoso Spinoza, de quem foi amigo, e que
talvez deu causa ao ódio que os outros judeus lhe criaram; --- o francês
e o italiano, parte do alemão e do húngaro, tornou-se em Londres objeto
de verdadeira curiosidade e veneração. Era buscado, consultado, ouvido,
não só por pessoas do vulgo ou idiotas, como por letrados, políticos e
personagens da Corte.

Convém dizer que em todos os países por onde andara tinha ele exercido
os mais contrários ofícios: soldado, advogado, sacristão, mestre de
dança, comerciante e livreiro. Chegou a ser agente secreto da Áustria,
guarda pontifício e armador de navios. Era ativo, engenhoso, mas pouco
persistente, a julgar pela variedade das coisas que empreendeu; ele,
porém, dizia que não, que a sorte é que sempre lhe foi adversa. Em
Londres, onde o vemos agora, limitou-se ao mister de letrado e gamenho;
mas não tardou que voltasse a Haia, onde o esperavam alguns dos amores
velhos, e não poucos recentes.

Que o amor, força é dizê-lo, foi uma das causas da vida agitada e
turbulenta do nosso herói. Ele era pessoalmente um homem galhardo,
insinuante, dotado de um olhar cheio de força e magia. Segundo ele mesmo
contou ao filho, deixou muito longe o algarismo dom-juanesco das
\emph{mille} e \emph{tre}\footnote{Don Juan é uma célebre personagem que
  aparece na Itália, durante o século \textsc{xvii}. Ele teria se relacionado
  intensa e fugazmente com mil e três mulheres, segundo a tradição.}.
Não podia dizer o número exato das mulheres a quem amara, em todas as
latitudes e línguas, desde a selvagem Maracujá de Pernambuco, até à bela
cipriota ou à fidalga dos salões de Paris e Londres; mas calculava em
não menos de cinco mil mulheres. Imagina-se facilmente que uma tal
multidão devia conter todos os gêneros possíveis da beleza feminil:
loiras, morenas, pálidas, coradas, altas, meãs, baixinhas, magras ou
cheias, ardentes ou lânguidas, ambiciosas, devotas, lascivas, poéticas,
prosaicas, inteligentes, estúpidas; --- sim, também estúpidas, e era
opinião dele que a estupidez das mulheres tinha o sexo feminino, era
graciosa, ao contrário da dos homens, que participava da aspereza viril.

--- Há casos, dizia ele, em que uma mulher estúpida tem o seu lugar.

Na Haia, entre os novos amores, deparou-se-lhe um que o prendeu por
longo tempo: \emph{lady} Emma Sterling, senhora inglesa, ou antes
escocesa, pois descendia de uma família de Dublin. Era formosa,
resoluta, e audaz; --- tão audaz que chegou a propor ao amante uma
expedição a Pernambuco para conquistar a capitania, e aclamarem-se reis
do novo Estado. Tinha dinheiro, podia levantar muito mais, chegou mesmo
a sondar alguns armadores e comerciantes, e antigos militares que ardiam
por uma desforra. Rui de Leão ficou aterrado com a proposta da amante, e
não lhe deu crédito; mas \emph{lady} Ema insistiu e mostrou-se tão de
rocha, que ele reconheceu enfim achar-se diante de uma ambiciosa
verdadeira. Era, todavia, homem de senso; viu que a empresa, por mais
bem organizada que fosse, não passaria de tentativa desgraçada;
disse-lho a ela; mostrou-lhe que, se a Holanda inteira tinha recuado,
não era fácil que um particular chegasse a obter ali domínio seguro, nem
ainda instantâneo. \emph{Lady} Ema abriu mão do plano, mas não perdeu a
ideia de o exalçar a alguma grande situação.

--- Tu serás rei ou duque\ldots{}

--- Ou cardeal, acrescentava ele rindo.

--- Por que não cardeal?

\emph{Lady} Ema fez com que Rui de Leão entrasse daí a pouco na
conspiração que deu em resultado a invasão da Inglaterra, a guerra
civil, e a morte enfim dos principais cabos da rebelião. Vencida esta,
\emph{lady} Ema não deu por vencida. Ocorreu-lhe então uma ideia
espantosa. Rui de Leão inculcava ser o próprio pai do duque de Monmouth,
suposto filho natural de Carlos \textsc{ii}, e caudilho principal dos rebeldes. A
verdade é que eram parecidos como duas gotas d'água. Outra verdade é que
\emph{lady} Ema, por ocasião da guerra civil, tinha o plano secreto de
fazer matar o duque, se ele triunfasse, e substituí-lo pelo amante, que
assim subiria ao trono de Inglaterra. O pernambucano, escusado é
dizê-lo, não soube de semelhante aleivosia, nem lhe daria o seu
assentimento. Entrou na rebelião, viu-a perecer ao sangue e no suplício,
e tratou de esconder-se. Ema acompanhou-o; e, como a esperança do cetro
não lhe saía do coração, passado algum tempo fez correr que o duque não
morrera, mas sim um amigo tão parecido com ele, e tão dedicado, que o
substituiu no suplício.

--- O duque está vivo, e dentro de pouco aparecerá ao nobre povo da
Grã-Bretanha, sussurrava ela aos ouvidos.

Quando Rui de Leão efetivamente apareceu, a estupefação foi grande, o
entusiasmo reviveu, o amor deu alma a uma causa, que o carrasco supunha
ter acabado na Torre de Londres. Donativos, presentes, armas,
defensores, tudo veio às mãos do audaz pernambucano, aclamado rei, e
rodeado logo de um troço de varões resolutos a morrer pela mesma causa.

--- Meu filho, --- disse ele, século e meio depois, ao médico
homeopata, --- dependeu de muito pouco não teres nascido príncipe de
Gales\ldots{} Cheguei a dominar cidades e vilas, expedi leis, nomeei
ministros, e, ainda assim, resisti a duas ou três sedições militares que
pediam a queda dos dois últimos gabinetes. Tenho para mim que as
dissensões internas ajudaram as forças legais, e devo-lhes a minha
derrota. Ao cabo, não me zanguei com elas; a luta fatigara-me; não minto
dizendo que o dia da minha captura foi para mim de alívio. Tinha visto,
além da primeira, duas guerras civis, uma dentro da outra, uma cruel,
outra ridícula, ambas insensatas. Por outro lado, vivera muito, e uma
vez que me não executassem, que me deixassem preso ou me exilassem para
os confins da terra, não pedia nada mais aos homens, ao menos durante
alguns séculos\ldots{} Fui preso, julgado e condenado à morte. Dos meus
auxiliares não poucos negaram tudo; creio mesmo que um dos principais
morreu na Câmara dos \emph{Lords}. Tamanha ingratidão foi um princípio
de suplício. Ema, não; essa nobre senhora não me abandonou; foi presa,
condenada, e perdoada; mas não me abandonou. Na véspera de minha
execução, veio ter comigo, e passamos juntos as últimas horas. Disse-lhe
que não me esquecesse, dei-lhe uma trança de cabelos, pedi-lhe que
perdoasse ao carrasco\ldots{} Ema prorrompeu em soluços; os guardas vieram
buscá-la. Ficando só, recapitulei a minha vida, desde Iguaraçu até a
Torre de Londres. Estávamos então em 1686; tinha eu oitenta e seis anos,
sem parecer mais de quarenta. A aparência era a da eterna juventude; mas
o carrasco ia destruí-la num instante. Não valia a pena ter bebido
metade do elixir e guardado comigo o misterioso boião, para acabar
tragicamente no cepo do cadafalso\ldots{} Tais foram as minhas ideias naquela
noite. De manhã preparei-me para a morte. Veio o padre, vieram os
soldados, e o carrasco. Obedeci maquinalmente. Caminhamos todos, subi ao
cadafalso, não fiz discurso; inclinei o pescoço sobre o cepo, o carrasco
deixou cair a arma, senti uma dor penetrante, uma angústia enorme, como
que a parada súbita do coração; mas essa sensação foi tão grande como
rápida; no instante seguinte tornara ao estado natural. Tinha no pescoço
algum sangue, mas pouco e quase seco. O carrasco recuou, o povo bramiu
que me matassem. Inclinaram-me a cabeça, e o carrasco, fazendo apelo a
todos os seus músculos e princípios, descarregou outro golpe, e maior,
se é possível, capaz de abrir-me ao mesmo tempo a sepultura, como já se
disse de um valente. A minha sensação foi igual à primeira na
intensidade e na brevidade; reergui a cabeça. Nem o magistrado nem o
padre consentiram que se desse outro golpe. O povo abalou-se, uns
chamaram-me santo, outros diabo, e ambas essas opiniões eram defendidas
nas tabernas à força de punho e de aguardente. Diabo ou santo, fui
presente aos médicos da corte. Estes ouviram o depoimento do magistrado,
do padre, do carrasco, de alguns soldados, e concluíram que, uma vez
dado o golpe, os tecidos do pescoço ligavam-se outra vez rapidamente, e
assim os mesmos ossos, e não chegavam a explicar um tal fenômeno. Pela
minha parte, em vez de contar o caso do elixir, calei-me; preferi
aproveitar as vantagens do mistério. Sim, meu filho; não imaginas a
impressão de toda a Inglaterra, os bilhetes amorosos que recebi das mais
finas duquesas, os versos, as flores, os presentes, as metáforas. Um
poeta chamou-me Anteu\footnote{Filho de Posídon e Gaia, com proporções
  de gigante.}. Um jovem protestante demonstrou-me que eu era o mesmo
Cristo.



\section*{V}



O narrador continuou:

--- Já veem, pelo que lhes contei, que não acabaria hoje nem em toda
esta semana, se quisesse referir miudamente a vida inteira de meu pai.
Algum dia o farei, mas por escrito, e cuido que a obra dará cinco
volumes, sem contar os documentos\ldots{}

--- Que documentos? perguntou o tabelião.

--- Os muitos documentos comprobatórios que possuo, títulos, cartas,
traslados de sentenças, de escrituras, cópias de estatísticas\ldots{} Por
exemplo, tenho uma certidão do recenseamento de um certo bairro de
Gênova, onde meu pai morreu em 1742; traz o nome dele, com declaração do
lugar em que nasceu\ldots{}

--- E com a verdadeira idade? perguntou o coronel.

--- Não. Meu pai andou sempre entre os quarenta e os cinquenta. Chegando
aos cinquenta, cinquenta e poucos, voltava para trás; --- e era-lhe
fácil fazer isto, porque não esquentava lugar; vivia cinco, oito, dez,
doze anos numa cidade, e passava a outra\ldots{} Pois tenho muitos documentos
que juntarei, entre outros, o testamento de \emph{lady} Ema, que morreu
pouco depois da execução gorada de meu pai. Meu pai dizia-me que entre
as muitas saudades que a vida lhe ia deixando, \emph{lady} Ema era das
mais fortes e profundas. Nunca viu mulher mais sublime, nem amor mais
constante, nem dedicação mais cega. E a morte confirmou a vida, porque o
herdeiro de \emph{lady} Ema foi meu pai. Infelizmente, a herança teve
outros reclamantes, e o testamento entrou em processo. Meu pai, não
podendo residir em Inglaterra, concordou na proposta de um amigo
providencial que veio a Lisboa dizer-lhe que tudo estava perdido; quando
muito poderia salvar um restozinho de nada, e ofereceu-lhe por esse
direito problemático uns dez mil cruzados. Meu pai aceitou-os; mas, tão
caipora que o testamento foi aprovado, e a herança passou às mãos do
comprador\ldots{}

--- E seu pai ficou pobre\ldots{}

--- Com os dez mil cruzados, e pouco mais que apurou. Teve então ideia
de meter-se no negócio de escravos; obteve privilégio, armou um navio, e
transportou africanos para o Brasil. Foi a parte da vida que mais lhe
custou; mas afinal acostumou-se às tristes obrigações de um navio
negreiro. Acostumou-se, e enfarou-se, que era outro fenômeno na vida
dele. Enfarava-se dos ofícios. As longas solidões do mar alargaram-lhe o
vazio interior. Um dia refletiu, e perguntou a si mesmo, se chegaria a
habituar-se tanto à navegação, que tivesse de varrer o oceano, por todos
os séculos dos séculos. Criou medo; e compreendeu que o melhor modo de
atravessar a eternidade era variá-la\ldots{}

--- Em que ano ia ele?

--- Em 1694; fins de 1694.

--- Veja só! Tinha então noventa e quatro anos, não era? Naturalmente,
moço\ldots{}

--- Tão moço que casou daí a dois anos, na Bahia, com uma bela senhora
que\ldots{}

--- Diga.

--- Digo, sim; porque ele mesmo me contou a história. Uma senhora que
amou a outro. E que outro! Imaginem que meu pai, em 1695, entrou na
conquista da famosa república dos Palmares\footnote{Uma das denominações
  do \emph{Quilombo dos Palmares}, localizado na então capitania de
  Pernambuco durante o século \textsc{xvii}.}. Bateu-se como um bravo, e perdeu
um amigo, um amigo íntimo, crivado de balas, pelado\ldots{}

--- Pelado?

--- É verdade; os negros defendiam-se também com água fervendo, e este
amigo recebeu um pote cheio; ficou uma chaga. Meu pai contava-me esse
episódio com dor, e até com remorso, porque, no meio da refrega, teve de
pisar o pobre companheiro; parece até que ele expirou quando meu pai lhe
metia as botas na cara\ldots{}

O tabelião fez uma careta; e o coronel, para disfarçar o horror,
perguntou o que tinha a conquista dos Palmares com a mulher que\ldots{}

--- Tem tudo, continuou o médico. Meu pai, ao tempo que via morrer um
amigo, salvara a vida de um oficial, recebendo ele mesmo uma flecha no
peito. O caso foi assim. Um dos negros, depois de derrubar dois
soldados, envergou o arco sobre a pessoa do oficial, que era um rapaz
valente e simpático, órfão de pai, tendo deixado a mãe em Olinda\ldots{} Meu
pai compreendeu que a flecha não lhe faria mal a ele, e então, de um
salto, interpôs-se. O golpe feriu-o no peito; ele caiu. O oficial,
Damião\ldots{} Damião de tal. Não digo o nome todo, porque ele tem alguns
descendentes para as bandas de Minas. Damião basta. Damião passou a
noite ao pé da cama de meu pai, agradecido, dedicado, louvando-lhe uma
ação tão sublime. E chorava. Não podia suportar a ideia de ver morrer o
homem que lhe salvara a vida por um modo tão raro. Meu pai sarou
depressa, com pasmo de todos. A pobre mãe do oficial quis beijar-lhe as
mãos: --- ``Basta-me um prêmio, disse ele; a sua amizade e a do seu
filho''. O caso encheu de pasmo Olinda inteira. Não se falava em outra
coisa; e daí a algumas semanas a admiração pública trabalhava em fazer
uma lenda. O sacrifício, como veem, era nenhum, pois meu pai não podia
morrer; mas o povo, que não sabia disso, buscou uma causa ao sacrifício,
uma causa tão grande como ele, e descobriu que o Damião devia ser filho
de meu pai, e naturalmente filho adúltero. Investigaram o passado da
viúva; acharam alguns recantos que se perdiam na obscuridade. O rosto de
meu pai entrou a parecer conhecido de alguns; não faltou mesmo quem
afirmasse ter ido a uma merenda, vinte anos antes, em casa da viúva, que
era então casada, e visto aí meu pai. Todas estas patranhas aborreceram
tanto a meu pai, que ele determinou passar à Bahia, onde casou\ldots{}

--- Com a tal senhora?

--- Justamente\ldots{} Casou com D.\,Helena, bela como o sol, dizia ele. Um
ano depois morria em Olinda a viúva, e o Damião vinha à Bahia trazer a
meu pai uma madeixa dos cabelos da mãe, e um colar que a moribunda pedia
para ser usado pela mulher dele. D.\,Helena soube do episódio da flecha,
e agradeceu a lembrança da morta. Damião quis voltar para Olinda; meu
pai disse-lhe que não, que fosse no ano seguinte. Damião ficou. Três
meses depois uma paixão desordenada\ldots{} Meu pai soube da aleivosia de
ambos, por um comensal da casa. Quis matá-los; mas o mesmo que os
denunciou avisou-os do perigo, e eles puderam evitar a morte. Meu pai
voltou o punhal contra si, e enterrou-o no coração.

``Filho, dizia-me ele, contando o episódio; dei seis golpes, cada um dos
quais bastava para matar um homem, e não morri.'' Desesperado saiu de
casa, e atirou-se ao mar. O mar restituiu-o à terra. A morte não podia
aceitá-lo: ele pertencia à vida por todos os séculos. Não teve outro
recurso mais do que fugir; veio para o Sul, onde alguns anos depois, no
princípio do século passado, podemos achá-lo na descoberta das minas.
Era um modo de afogar o desespero, que era grande, pois amara muito a
mulher, como um louco\ldots{}

--- E ela?

--- São contos largos, e não me sobra tempo. Ela veio ao Rio de Janeiro,
depois das duas invasões francesas; creio que em 1713\footnote{Os
  franceses ocuparam o território brasileiro, possessão de Portugal, em
  pelos menos três oportunidades: entre 1555 e 1567 (na região que hoje
  corresponde à Baía de Guanabara); entre 1612 e 1615 (em São Luís, no
  antigo Estado do Maranhão e Grão Pará); e entre 1710 e 1711 (atual
  cidade do Rio de Janeiro).}. Já então meu pai enriquecera com as
minas, e residia na cidade fluminense, benquisto, com ideias até de ser
nomeado governador. D.\,Helena apareceu-lhe, acompanhada da mãe e de um
tio. Mãe e tio vieram dizer-lhe que era tempo de acabar com a situação
em que meu pai tinha colocado a mulher. A calúnia pesara longamente
sobre a vida da pobre senhora. Os cabelos iam-lhe embranquecendo: não
era só a idade que chegava, eram principalmente os desgostos, as
lágrimas. Mostraram-lhe uma carta escrita pelo comensal denunciante,
pedindo perdão a D.\,Helena da calúnia que lhe levantara e confessando
que o fizera levado de uma criminosa paixão. Meu pai era uma boa alma;
aceitou a mulher, a sogra e o tio. Os anos fizeram o seu ofício; todos
três envelheceram, menos meu pai. Helena ficou com a cabeça toda branca;
a mãe e o tio voavam para a decrepitude; e nenhum deles tirava os olhos
de meu pai, espreitando as cãs que não vinham, e as rugas ausentes. Um
dia meu pai ouviu-lhes dizer que ele devia ter parte com o diabo. Tão
forte! E acrescentava o tio: ``De que serve o testamento, se temos de ir
antes?'' Duas semanas depois morria o tio; a sogra acabou pateta, daí a
um ano. Restava a mulher, que pouco mais durou.

--- O que me parece, aventurou o coronel, é que eles vieram ao cheiro
dos cobres\ldots{}

--- Decerto.

--- \ldots{} e que a tal D.\,Helena (Deus lhe perdoe!) não estava tão inocente
como dizia. É verdade que a carta do denunciante\ldots{}

--- O denunciante foi pago para escrever a carta, explicou o Dr.\,Leão;
meu pai soube disso, depois da morte da mulher, ao passar pela Bahia\ldots{}
Meia-noite! Vamos dormir; é tarde; amanhã direi o resto.

--- Não, não, agora mesmo.

--- Mas, senhores\ldots{} Só se for muito por alto.

--- Seja por alto.

O doutor levantou-se e foi espiar a noite, estendendo o braço para fora,
e recebendo alguns pingos de chuva na mão. Depois voltou-se e deu com os
dois olhando um para o outro, interrogativos. Fez lentamente um cigarro,
acendeu-o, e, puxadas umas três fumaças, concluiu a singular história.





\section*{vi}



--- Meu pai deixou pouco depois o Brasil, foi a Lisboa, e dali passou-se
à Índia, onde se demorou mais de cinco anos, e donde voltou a Portugal,
com alguns estudos feitos acerca daquela parte do mundo. Deu-lhes a
última lima, e fê-los imprimir, tão a tempo, que o governo mandou-o
chamar para entregar-lhe o governo de Goa\footnote{Estado situado, hoje,
  na região oeste da Índia. A personagem refere-se ao período histórico
  quando Goa era uma das possessões portuguesas, no globo.}. Um
candidato ao cargo, logo que soube do caso, pôs em ação todos os meios
possíveis e impossíveis. Empenhos, intrigas, maledicência, tudo lhe
servia de arma. Chegou a obter, por dinheiro, que um dos melhores
latinistas da península, homem sem escrúpulos, forjasse um texto latino
da obra de meu pai, e o atribuísse a um frade agostinho\footnote{Referência
  à católica Ordem de Santo Agostinho, fundada em 1243 pelo Papa
  Inocêncio \textsc{iv}.}, morto em Adém\footnote{Cidade que se localizaria,
  atualmente, no Iêmen.}. E a tacha de plagiário acabou de eliminar meu
pai, que perdeu o governo de Goa, o qual passou às mãos do outro;
perdendo também, o que é mais, toda a consideração pessoal. Ele escreveu
uma longa justificação, mandou cartas para a Índia, cujas respostas não
esperou, porque no meio desses trabalhos, aborreceu-se tanto, que
entendeu melhor deixar tudo, e sair de Lisboa. Esta geração passa, disse
ele, e eu fico. Voltarei cá daqui a um século, ou dois.

--- Veja isto, interrompeu o tabelião, parece coisa de caçoada! Voltar
daí a um século --- ou dois, como se fosse um ou dois meses. Que diz,
``seu'' coronel?

--- Ah! eu quisera ser esse homem! É verdade que ele não voltou um
século depois\ldots{} Ou voltou?

--- Ouça-me. Saiu dali para Madri, onde esteve de amores com duas
fidalgas, uma delas viúva e bonita como o sol, a outra casada, menos
bela, porém amorosa e terna como uma pomba-rola. O marido desta chegou a
descobrir o caso, e não quis bater-se com meu pai, que não era nobre;
mas a paixão do ciúme e da honra levou esse homem ofendido à prática de
uma aleivosia, igual à outra: mandou assassinar meu pai; os esbirros
deram-lhe três punhaladas e quinze dias de cama. Restabelecido,
deram-lhe um tiro; foi o mesmo que nada. Então, o marido achou um meio
de eliminar meu pai; tinha visto com ele alguns objetos, notas, e
desenhos de coisas religiosas da Índia, e denunciou-o ao Santo
Ofício\footnote{Na Espanha, a Inquisição, a cargo do Tribunal do Santo
  Ofício, foi autorizada mediante bula do Papa Sixto \textsc{iv} em 1478. Em
  Portugal, a instituição passou a atuar em 1536, autorizada pelo papa
  Paulo \textsc{iii}.}, como dado a práticas supersticiosas. O Santo Ofício, que
não era omisso nem frouxo nos seus deveres, tomou conta dele, e
condenou-o a cárcere perpétuo. Meu pai ficou aterrado. Na verdade, a
prisão perpétua para ele devia ser a coisa mais horrorosa do mundo.
Prometeu\footnote{O mito de Prometeu é assunto da tragédia
  \emph{Prometeu Acorrentado}, de Ésquilo. De acordo com a lenda, o titã
  foi punido por roubar o fogo primordial (que deu a vida aos homens) a
  Zeus. Acorrentado no alto de uma montanha inacessível, teve o fígado
  bicado por uma águia durante anos, até que Quíron (um centauro) o
  libertou das correntes que o aprisionavam.}, o mesmo Prometeu foi
desencadeado\ldots{} Não me interrompa, Sr.\,Linhares, depois direi quem foi
esse Prometeu. Mas, repito: ele foi desencadeado, enquanto que meu pai
estava nas mãos do Santo Ofício, sem esperança. Por outro lado, ele
refletiu consigo que, se era eterno, não o era o Santo Ofício. O Santo
Ofício há de acabar um dia, e os seus cárceres, e então ficarei livre.
Depois, pensou também que, desde que passasse um certo número de anos,
sem envelhecer nem morrer, tornar-se-ia um caso tão extraordinário, que
o mesmo Santo Ofício lhe abriria as portas. Finalmente, cedeu a outra
consideração. ``Meu filho, disse-me ele, eu tinha padecido tanto
naqueles longos anos de vida, tinha visto tanta paixão má, tanta
miséria, tanta calamidade, que agradeci a Deus, o cárcere e uma longa
prisão; e disse comigo que o Santo Ofício não era tão mau, pois que me
retirava por algumas dezenas de anos, talvez um século, do espetáculo
exterior\ldots{}''

--- Ora essa!

--- Coitado! Não contava com a outra fidalga, a viúva, que pôs em campo
todos os recursos de que podia dispor, e alcançou-lhe a fuga daí a
poucos meses. Saíram ambos de Espanha, meteram-se em França, e passaram
à Itália, onde meu pai ficou residindo por longos anos. A viúva
morreu-lhe nos braços; e, salvo uma paixão que teve em Florença, por um
rapaz nobre, com quem fugiu e esteve seis meses, foi sempre fiel ao
amante. Repito, morreu-lhe nos braços, e ele padeceu muito, chorou
muito, chegou a querer morrer também. Contou-me os atos de desespero que
praticou; porque, na verdade, amara muito a formosa madrilena.
Desesperado, meteu-se a caminho, e viajou por Hungria, Dalmácia,
Valáquia; esteve cinco anos em Constantinopla; estudou o turco a fundo,
e depois o árabe. Já lhes disse que ele sabia muitas línguas; lembra-me
de o ver traduzir o padre-nosso em cinquenta idiomas diversos. Sabia
muito. E ciências! Meu pai sabia uma infinidade de coisas: filosofia,
jurisprudência, teologia, arqueologia, química, física, matemáticas,
astronomia, botânica; sabia arquitetura, pintura, música. Sabia o diabo.

--- Na verdade\ldots{}

--- Muito, sabia muito. E fez mais do que estudar o turco; adotou o
maometanismo. Mas deixou-o daí a pouco. Enfim, aborreceu-se dos turcos:
era a sina dele aborrecer-se facilmente de uma coisa ou de um ofício.
Saiu de Constantinopla, visitou outras partes da Europa, e finalmente
passou-se a Inglaterra aonde não fora desde longos anos. Aconteceu-lhe
aí o que lhe acontecia em toda a parte: achou todas as caras novas; e
essa troca de caras no meio de uma cidade, que era a mesma deixada por
ele, dava-lhe a impressão de uma peça teatral, em que o cenário não
muda, e só mudam os atores. Essa impressão, que a princípio foi só de
pasmo, passou a ser de tédio; mas agora, em Londres, foi outra coisa
pior, porque despertou nele uma ideia, que nunca tivera, uma ideia
extraordinária, pavorosa\ldots{}

--- Que foi?

--- A ideia de ficar doido um dia. Imaginem: um doido eterno. A comoção
que esta ideia lhe dava foi tal que quase enlouqueceu ali mesmo. Então
lembrou-se de outra coisa. Como tinha o boião do elixir consigo, lembrou
de dar o resto a alguma senhora ou homem, e ficariam os dois imortais.
Sempre era uma companhia. Mas, como tinha tempo diante de si, não
precipitou nada; achou melhor esperar pessoa cabal. O certo é que essa
ideia o tranquilizou\ldots{} Se lhe contasse as aventuras que ele teve outra
vez na Inglaterra, e depois em França, e no Brasil, onde voltou no
vice-reinado do Conde de Resende\footnote{O Vice-Reinado do português
  José Luís de Castro (1744--1819), 2.º Conde de Resende, durou de 1790 a
  1801.}, não acabava mais, e o tempo urge, além do que o Sr.\,Coronel
está com sono\ldots{}

--- Qual sono!

--- Pelo menos está cansado.

--- Nem isso. Se eu nunca ouvi uma coisa que me interessasse tanto.
Vamos; conte essas aventuras.

--- Não; direi somente que ele achou-se em França por ocasião da
revolução de 1789\footnote{Iniciada em 1789, a Revolução Francesa se
  estendeu até 1799, quando Napoleão Bonaparte assume o poder.},
assistiu a tudo, à queda e morte do rei, dos girondinos, de Danton, de
Robespierre; morou algum tempo com Filinto Elísio\footnote{Cognome do
  poeta português Francisco Manuel do Nascimento (1734--1819), muito
  popular durante o século \textsc{xix}.}, o poeta, sabem? Morou com ele em
Paris; foi um dos elegantes do Diretório, deu-se com o primeiro
Cônsul\footnote{Napoleão Bonaparte, que assumiu o Consulado Francês em
  1799.}\ldots{} Quis até naturalizar-se e seguir as armas e a política;
podia ter sido um dos marechais do império, e pode ser até que não
tivesse havido Waterloo\footnote{Encerrada em 18 de junho de 1815, a
  batalha de Waterloo trouxe vitória aos ingleses e prussianos, que
  enfrentavam Napoleão. O combate representou o fim de seu império e
  assinala um novo período social e político na França.}. Mas ficou tão
enjoado de algumas apostasias políticas, e tão indignado, que recusou a
tempo. Em 1808 achamo-lo em viagem com a corte real para o Rio de
Janeiro. Em 1822 saudou a independência; e fez parte da Constituinte;
trabalhou no 7 de Abril\footnote{Grafado com inicial maiúscula, na
  edição original.}; festejou a maioridade\footnote{A maioridade de
  Pedro \textsc{ii} foi decretada em 23 de julho de 1840. Ele foi coroado
  Imperador em 18 de julho de 1841.}; há dois anos era deputado.

Neste ponto os dois ouvintes redobraram de atenção. Compreenderam que
iam chegar ao desenlace, e não quiseram perder uma sílaba daquela parte
da narração, em que iam saber da morte do imortal. Pela sua parte, o Dr.\,Leão parara um pouco; podia ser uma lembrança dolorosa; podia também ser
um recurso para aguçar mais o apetite. O tabelião ainda lhe perguntou,
se o pai não tinha dado a alguém o resto do elixir, como queria; mas o
narrador não lhe respondeu nada. Olhava para dentro; enfim, terminou
deste modo:

--- A alma de meu pai chegara a um grau de profunda melancolia. Nada o
contentava; nem o sabor da glória, nem o sabor do perigo, nem o do amor.
Tinha então perdido minha mãe, e vivíamos juntos, como dois solteirões.
A política perdera todos os encantos aos olhos dum homem que pleiteara
um trono, e um dos primeiros do universo. Vegetava consigo; triste,
impaciente, enjoado. Nas horas mais alegres fazia projetos para o século
\textsc{xx} e \textsc{xxiv}, porque já então me desvendara todo o segredo da vida dele.
Não acreditei, confesso; e imaginei que fosse alguma perturbação mental;
mas as provas foram completas, e demais a observação mostrou-me que ele
estava em plena saúde. Só o espírito, como digo, parecia abatido e
desencantado. Um dia, dizendo-lhe eu que não compreendia tamanha
tristeza, quando eu daria a alma ao diabo para ter a vida eterna, meu
pai sorriu com uma tal expressão de superioridade, que me enterrou cem
palmos abaixo do chão. Depois, respondeu que eu não sabia o que dizia;
que a vida eterna afigurava-se-me excelente, justamente porque a minha
era limitada e curta; em verdade, era o mais atroz dos suplícios. Tinha
visto morrer todas as suas afeições; devia perder-me um dia, e todos os
mais filhos que tivesse pelos séculos adiante. Outras afeições e não
poucas o tinham enganado; e umas e outras, boas e más, sinceras e
pérfidas, era-lhe forçoso repeti-las, sem trégua, sem um respiro ao
menos, porquanto, a experiência não lhe podia valer contra a necessidade
de agarrar-se a alguma coisa, naquela passagem rápida dos homens e das
gerações. Era uma necessidade da vida eterna; sem ela, cairia na
demência. Tinha provado tudo, esgotado tudo; agora era a repetição, a
monotonia, sem esperanças, sem nada. Tinha de relatar a outros filhos,
vinte ou trinta séculos mais tarde, o que me estava agora dizendo; e
depois a outros, e outros, e outros, um não acabar mais nunca. Tinha de
estudar novas línguas, como faria Aníbal\footnote{Aníbal, de Cartago
  (247--183 a.C.), estadista e general romano, reconhecido por ter
  desbravado o Mediterrâneo.}, se vivesse até hoje: e para quê? para
ouvir os mesmos sentimentos, as mesmas paixões\ldots{} E dizia-me tudo isso,
verdadeiramente abatido. Não parece esquisito? Enfim um dia, como eu
fizesse a alguns amigos uma exposição do sistema homeopático, vi reluzir
nos olhos de meu pai um fogo desusado e extraordinário. Não me disse
nada. De noite, vieram chamar-me ao quarto dele. Achei-o moribundo;
disse-me então, com a língua trôpega, que o princípio homeopático fora
para ele a salvação. \emph{Similia similibus curantur}\footnote{Sentença
  atribuída ao grego Hipócrates (460--370 a.C.), considerado pai da
  Medicina, que se poderia traduzir como cura de um mal pelo emprego de
  substância ou medida semelhante à doença.}. Bebera o resto do elixir,
e assim como a primeira metade lhe dera a vida, a segunda dava-lhe a
morte. E, dito isto, expirou.

O coronel e o tabelião ficaram algum tempo calados, sem saber que
pensassem da famosa história; mas a seriedade do médico era tão
profunda, que não havia duvidar. Creram no caso, e creram também
definitivamente na homeopatia. Narrada a história a outras pessoas, não
faltou quem supusesse que o médico era louco; outros atribuíram-lhe o
intuito de tirar ao coronel e ao tabelião o desgosto manifestado por
ambos de não poderem viver eternamente, mostrando-lhes que a morte é,
enfim, um benefício. Mas a suspeita de que ele apenas quis propagar a
homeopatia entrou em alguns cérebros, e não era inverossímil. Dou este
problema aos estudiosos. Tal é o caso extraordinário, que há anos, com
outro nome, e por outras palavras, contei a este bom povo, que
provavelmente já os esqueceu a ambos. 

\part{A cartomante}

\chapter[Introdução, \emph{por Alcides Villaça}]{Introdução\subtitulo{Machado de Assis, tradutor de si mesmo}}

\begin{flushright}
\textsc{alcides villaça}
\end{flushright}

\section*{I}

Não nos cansaremos de encontrar nos textos machadianos o empréstimo de
ideias e de formas, as incontáveis alusões, as fontes veladas ou
explícitas, as citações e as glosas, os lapsos forjados ou verdadeiros
--- todo um arsenal, enfim, de dispositivos intertextuais que encorajam
estudos comparatistas. Cada detalhe revelado, cada minúcia
caprichosamente ampliada pode provocar o falso dilema: será o texto
machadiano muito mais rico ou muito menos original do que se pensava? A
intertextualidade não traz, por si mesma, o critério de valor que decida
a questão. É óbvio que o comparativismo consequente deve levar em conta
os processos e os contextos de composição, que sobredeterminam as
pontualidades comparadas. Como, no caso de Machado de Assis, as questões
que contam derivam da complexidade da narração processada e do modo como
esta se articula com a cultura nacional, corre-se o risco de querer
comparar \emph{a} com \emph{x}, e não com \emph{b}. Considere-se ainda o
fato de que dificilmente os artistas excepcionais são, por isso mesmo,
``comparáveis'' entre si; sempre foi mais útil compreendê-los pelo modo
como consideraram, para superá-las, expectativas firmadas na tradição.
As homenagens que Machado está permanentemente (e a seu modo) prestando
às incontáveis ``fontes'' de seu repertório não são reverências ao valor
intrínseco deste, mas pontes para um outro valor que ao mestre interessa
estabelecer. Podemos e devemos voltar por elas, pois não, mas justamente
para apreciar no caminho a qualidade desse olhar moderno, que
transfigurou a antiga paisagem. É provável que um príncipe florentino
não se sinta bem na companhia de Janjão (``Teoria do medalhão''), que
Hamlet distribua empurrões para não ver Rita e Camilo ladeando-o num
patético retrato (``A cartomante''), ou que o Império Romano se recuse a
caber no bolso de Custódio (``O empréstimo''); do nosso lado, Império e
República são letreiros de confeitaria (\emph{Esaú e Jacó}), e os
partidos nacionais em alternância no poder sempre podem inspirar
buliçosas polcas (``Um homem célebre''). O racionalismo ilustrado talvez
não pare em pé no trapézio de Brás Cubas (\emph{Memórias póstumas}), e
não é fácil aceitar que um copista de estudos de teologia venha a
emendar as Escrituras (``O enfermeiro''). A falibilidade de Deus,
revelada por Seu gesto de desalento resignado, coloca-o apenas um degrau
acima da ingenuidade de um Lúcifer humilhado (``A Igreja do Diabo'').
Para que não adotemos, enfim, o mesmo ponto de vista vertiginoso do
delírio que leva a Pandora, e para que não nos estarreça cada elemento
minúsculo de um particularíssimo processo de remissões, resta tentar um
pouco de luz nos subterrâneos desse narrador, com a esperança de vermos
alguma coisa em torno.

Interessa-me, aqui, analisar um dos processos de composição acionados
por Machado, uma certa ``tradução'', que buscarei determinar. Tal
processo está representado de forma privilegiada (e à maneira sempre
oblíqua do autor) no conto ``A cartomante'', de \emph{Várias histórias}.
São páginas das mais sugestivas que Machado se permitiu, no limiar de
uma autorrevelação a que altivamente se furtou.

\section*{ii}

Na perseguição do processo representado, o leitor escorregará primeiro
pela superfície da historieta melodramática que termina em morte ---
provavelmente o piso único para muitos dos leitores da \emph{Gazeta de
Notícias}, em 1884, quando o conto se publicou. Um pouquinho mais abaixo
do caso de adultério que transpira e é vingado situa-se o plano das
superstições, dos vaticínios e do destino caprichoso. Descendo ainda um
palmo, o leitor encontrará estímulos para discutir o papel da
causalidade ou do acaso e suas consequências no rumo dos acontecimentos.
Nessa verticalidade de planos, tanto a boa dama fluminense quanto um
Augusto Meyer saberão fixar o que lhes aproveite (este último deteve-se,
como se sabe, no do homem subterrâneo). Eu gostaria de explorar, na
horizontal, uma insistente sucessão de considerandos e analogias que o
narrador vai tecendo aos poucos, ao longo da história.

A alusão que abre a narrativa poderia ter pouco peso, não fosse ela o
mote mesmo de um processo que se irá reiterar:

\begin{quote}
Hamlet observa a Horácio que há mais cousas no céu e na terra do que
sonha a nossa filosofia. Era a mesma explicação que dava a bela Rita ao
moço Camilo, numa sexta-feira de novembro de 1869, quando este ria dela,
por ter ido na véspera consultar uma cartomante; a diferença é que o
fazia por outras palavras.
\end{quote}

Tomando Shakespeare, Machado se vale da batida frase de Hamlet a
Horácio; apesar de gasta, é sempre uma referência culta. No período
seguinte, as duas personagens da tragédia dão rapidamente lugar à ``bela
Rita'' e ao ``moço Camilo'', numa cena risonha e bem datada. Parece que
a ``filosofia'' de que falava o atormentado Hamlet encontra paralelismo
nos sortilégios da cartomante que a crédula Rita foi consultar. Mas a
ponte se armou foi no eixo da ``mesma explicação'' que tanto Shakespeare
quanto Machado colocaram na boca de suas personagens. Uma ponte, aliás,
meio torta, pois a ``\emph{mesma} explicação'' logo surge ressalvada:
``a diferença é que o fazia por \emph{outras} palavras'' (grifos meus).
Em suma: descontada a questão da forma, Hamlet e Rita podem estar muito
próximos, não sendo impossível a convergência de seus pensamentos.

O ``mesmo'' pode atuar dentro da ``diferença'': as palavras podem ser
``outras'', sem prejuízo aparente para as ``explicações''. Tal
dissociação entre forma e conteúdo talvez escandalize os princípios de
um bacharel em Letras, mas não soará razoável para um leitor comum?

Um pouco mais adiante, o narrador esclarecerá que Rita, ``sem saber que
traduzia Hamlet em vulgar, disse-lhe que havia muita cousa misteriosa e
verdadeira neste mundo''. Aqui, a forma do chavão aparece em sua
inteireza, na condição de uma vulgata que traduzisse, por pura
coincidência, o discurso hamletiano. O efeito de despropósito é
relativizado pelo narrador com a autoridade de quem, íntimo tanto de
Hamlet quanto de Rita, pode perfeitamente ajuizar quanto à procedência
de uma comparação, sobretudo quando as diferenças, tidas por mínimas,
acabam por se eliminar no fundo do ``mesmo''.

O conto faz ver que Rita, além de ``formosa'', é também ``tonta'' --- o
que parece vir desenhado na expressão meio aérea da ``boca fina e
interrogativa''. Por sua vez, Camilo ``era um ingênuo na vida moral e
prática'', não tendo ``nem experiência, nem intuição''. É muito natural,
pois, que venham a se atrair: ``não tardou que o sapato se acomodasse ao
pé'' --- a forma de acomodação indicando também o nível terrestre em que
se dá. Como o leitor não quer se identificar nem com tontos nem com
ingênuos, acompanhará os acontecimentos com curiosidade e distância.
Lembremos ainda que, na origem da sedução, Camilo aniversariante
recebera de Rita um ``cartão com um vulgar cumprimento a lápis'',
bilhetinho de ``palavras vulgares'', sim, mas --- propõe o narrador ---
``há vulgaridades sublimes, ou, pelo menos, deleitosas''. Não há dúvida
de que a ideia do vulgar está se expandindo no conto, e que na outra
ponta da escala o narrador vai buscar o sublime para montar --- e
relativizar --- o paradoxo. Esse processo de eliminação das diferenças
ganha expressiva ilustração na referência à ``velha caleça de praça''
que, para os namoradinhos que lá dentro se apertem, ``vale o carro de
Apolo''. Portanto: Rita ``vale'' Hamlet, um vulgar cumprimento ``vale''
uma mensagem sublime, a mesma explicação ``vale'' em diferentes
palavras. De valor a valor, de tradução em tradução, as vulgatas valem o
original, o prosaico vale o mitológico, a curiosidade vale a metafísica,
a cartomancia vale o conhecimento. Ao promover essas traduções
aparentemente disparatadas, o narrador cria um critério para sua
narração; acompanhemo-la.

Não faltam ao conto ingredientes de melodrama romântico: o nó da intriga
se aperta com a atuação acusatória e ameaçadora das cartas anônimas que
vão chegando a Camilo. Saberá delas o amigo Vilela? Que fará, se souber?
O recurso é novelesco, e foi duramente apontado pelo próprio Machado
como uma das debilidades de \emph{O primo Basílio}, de Eça de Queirós.
Também não falta ao conto certa pimenta naturalista que, a princípio
abrandada pela melodia italiana da expressão ``\emph{odor di femmina}'',
revela todo o ardor neste período de alusão bíblica: ``Rita, como uma
serpente, foi-se acercando dele, envolveu-o todo, fez-lhe estalar os
ossos num espasmo, e pingou-lhe o veneno na boca''. Expediente
melodramático e sensualismo rastejante vão alimentando o texto com sua
vulgaridade, sem que o leitor, no entanto, possa imputá-la ao estilo
propriamente dito, que se conserva elegante e precavido como quem calça
finas luvas para lidar na cozinha. Além do que, já nos prevenira o autor
quanto à hipótese das ``vulgaridades sublimes''.

A ameaça das cartas anônimas é seguida por um lacônico bilhete do amigo
Vilela a Camilo: ``Vem já, já, à nossa casa; preciso falar-te sem
demora''. Que valerá esta mensagem: uma ordem raivosa de quem se soube
traído? um chamado para negócios urgentes? Como faltam a Camilo aqueles
``óculos de cristal, que a natureza põe no berço de alguns para adiantar
os anos'', sobra-lhe o dilema: ir ou não ir à casa do outro? \emph{To go
or not to go?} O caso amoroso, tão banal quanto os que o vivem, é
rondado pelo trágico. Sem o concurso da experiência ou da intuição, o
ingênuo Camilo deve tomar uma decisão em que pode estar a diferença
entre viver ou morrer. Busca algum recado de Rita, ``que lhe explicasse
tudo'', mas ``não achou nada, nem ninguém''. Depois de imaginar cenas de
drama, de cogitar em ir armado, de crer que de fato já estava vendo o
que iria acontecer, Camilo parece abandonar-se àquele mesmo movimento
pelo qual o narrador o apresentara no início do conto: ``diante do
mistério, contentou-se em levantar os ombros, e foi andando''. É verdade
que esse impulso de sua natureza ingênua, incrédula e indiferente leva,
agora, um coração batendo muito forte, e se ainda é ingênuo e incrédulo,
não mais se dá ao luxo de ser indiferente, e ``dar de ombros''. Tão
desconfortável quanto o medo da morte, o dilema obrigado à decisão faz
Camilo entrar logo num tílburi, para ir até Vilela. ``-- Quanto antes,
melhor, pensou ele; não posso estar assim''. A inconsistência desse
``melhor'' (?) dá a medida da irracionalidade do impulso que decide pela
personagem.

A ironia machadiana apoia-se com muita frequência nas simetrias,
traduzindo uma situação por outra num eixo de equivalências --- processo
nada estranho, como se vê, ao que está sendo comentado por mim e pelo
próprio conto. Se uma coisa vale outra, se Camilo vale Rita, por que não
irá ele parar na mesma cartomante? Faltando-lhe iniciativa, o narrador
faz o tílburi deter-se junto a uma carroça atravessada na rua, bem em
frente à casa da adivinha. A simetria revela, mais literalmente do que
nunca, a ironia da sorte: ``Dir-se-ia a morada do indiferente Destino''.
Supersticioso por necessidade, e inconscientemente obedecendo às ordens
dos homens que procuram safar a carroça (``Anda! agora! empurra! vá!
vá!'' --- as duas últimas parecendo reforçar eufonicamente o ``Vem já,
já, à nossa casa'' de Vilela), sobe Camilo a escada, sugestivamente
arcaica, de ``degraus comidos dos pés'' e de ``corrimão pegajoso'',
conduzido pela cartomante. A cena da consulta contrapõe a ingenuidade do
consulente à sagacidade da italiana ``com grandes olhos sonsos e
agudos'' que, arrancando-lhe o que ele quer ouvir, vaticina o futuro
bonançoso e o despede: ``Vá, \emph{ragazzo innamorato}\ldots{}'', ``vá,
vá tranquilo'' (outra eufonia, outra correspondência). Levado por mais
esse empurrão, o amante de Rita segue confiante para a casa de Vilela,
olhando o mar, ``até onde a água e o céu dão um abraço infinito, e teve
assim uma sensação do futuro, longo, longo, interminável''. Vilela o
recebe sem palavras e o leva a uma saleta onde, antes de pegá-lo pela
gola e abatê-lo com dois tiros, dá-lhe o tempo de ver Rita morta e
ensanguentada num canapé.

\section*{iii}

Esta a história, notável também por tantos e tantos outros detalhes de
construção caprichosa, expressiva e provocadora, que o interesse de
outros leitores poderá analisar. Retomo o meu, que vinha se orientando
no plano das traduções, um dos expedientes-chave da poética machadiana.
Vamos a elas.

Numa acepção corrente, o termo \emph{paródia} indica a retomada de um
texto, de uma forma, de um estilo, para efeito de seu deslocamento a
novo eixo morfológico-expressivo, onde o sentido original se transvia,
quase sempre rebaixado, servindo pelo avesso a uma outra posição
crítica. Em ``A cartomante'', Machado cria uma estranha relação entre a
tragédia shakespeariana, o carro de Apolo e o sublime, de um lado, e as
vulgaridades todas da história de Rita e Camilo, de outro. O método é o
de ir emparelhando elementos de uma tradição alta e elementos prosaicos
de uma gozosa existência burguesa (``Adeus, escrúpulos!''). Mas a
impressão de paródia não passaria disso, pois o final é agudo e
trágico\ldots{} Trágico? Não, não é este o efeito estético das duas
mortes violentas, tão abruptas quanto inglórias, que parecem abrir o fim
do conto para uma picante manchete na página policial do dia seguinte; o
efeito estético é o do grande descompasso entre o fato e a fatura
literária, tão elegante e precisa esta, tão vulgar aquele. Afinal de
contas, os amantes morreram sobretudo pela má administração dos
colóquios, e sempre lhes teria faltado qualquer vocação para o heroico.
Pode-se dizer que morreram de vulgaridade, o ingênuo Camilo e a tonta
Rita, tendo no entanto encontrado o seu Shakespeare, que se não os fez
Hamlet e Ofélia, nem Romeu e Julieta, soube compreender o carro de Apolo
que parecia estar em sua caleça de praça. ``Há vulgaridades sublimes'':
nessa perspectiva, um prisma da modernidade já permite fundir os gêneros
e os planos artísticos, as virtudes e os vícios humanos, de modo que o
escritor se libere para contar histórias prosaicas que não desmentem a
grandeza de um Shakespeare, apenas a ``atualizam''. Nessa tradução
burguesa está por certo o novo leitor, ávido das emoções fortes que as
subnovelas românticas ou vagamente realistas lhe ofereciam nos jornais,
a \emph{Gazeta de Notícias} entre eles. É como se os componentes
clássicos do trágico, do heroico e do sublime estivessem agora à
disposição num eclético bazar da época, adaptados a um consumo
cotidiano, bem à mão dos consumidores. Caiu um pouco, por anacrônica, a
aura original? Não há por que lamentá-lo: ela se faz representar agora
por seu valor nominal, apeada do seu Pégaso, mas firme na caleça de
praça. A astúcia de Machado está em reafirmar que uma coisa vale a
outra; mas o leitor mais desconfiado não parará por aí. E aí começam as
interpretações.

\section*{iv}

É próprio do pensamento mítico que um arquétipo viva de sua atualização
e recriação, ainda que sob forma aparentemente mais prosaica (caso do
``Recado do morro'', de Guimarães Rosa) ou mesmo degradada (caso do
\emph{Ulisses}, de Joyce). Sob o melodrama, ``A cartomante'' parece
apontar para uma tradição de sibilas, deuses e tragédias, tomando em
bloco a presença do sublime e os ecos do estilo alto, e
providenciando-lhes ``traduções'' que caibam no espírito e no espaço de
um conto despretensioso. O narrador se acautela quanto a este disparate
por meio da fórmula ``há vulgaridades sublimes'', e apresenta o
estratagema do ``isto \emph{vale} aquilo'' como argumento para as
alusões. Tal expediente é típico de Machado. Na ``Teoria do medalhão''
(\emph{Papéis avulsos}), o conto terminava com a frase: ``Guardadas as
proporções, a conversa desta noite vale \emph{O Príncipe}, de
Machiavelli''; em ``O enfermeiro'' (\emph{Várias histórias}) o narrador
finalizava com a seguinte ressalva: ``Se achar que esses apontamentos
\emph{valem} alguma coisa {[}\ldots{}{]}'' --- e acabava fazendo uma
``emenda'' ao divino sermão da montanha (grifos meus); a história
contada por Jacobina, em ``O espelho'', vale por um ``Esboço de uma nova
teoria da alma humana'', assim como ``D.\,Benedita'' (ambos de Papéis
avulsos) é a própria personificação da Veleidade; etc. etc. O expediente
é, na verdade, uma autêntica profissão de fé que faz Machado de seu
processo de relativização, amplo, geral e aberto. Nesse processo, o
autor simula conformar-se em emprestar a altura e o fôlego mais
limitados do conto realista à representação de matérias que outra
altitude teriam alcançado na tradição. E de fato nada lamenta, pois ao
mesmo tempo que reconhece a diferença entre o sublime e o vulgar,
dissolve-a, digamos assim, em nome de evidências da prática. A operação
irônica não se apresenta, é claro, como irônica, mas como decantação
pura do cotidiano em que estamos todos: o leitor (ou leitora) que diga
se um momento de intimidade apaixonada não vale o Olimpo\ldots{} Se
assim é, cabe Hamlet entre Rita e Camilo, como cabe ao Destino fazer sua
morada na casa da cartomante sonsa.

Resta ao leitor relativizar, por sua vez, o que faz o escritor (operação
com que por certo contava Machado). A relativização nossa bate sobre
esse fingimento do narrador, que insiste em traduzir pelo mesmo o que
também sabe reconhecer como diferenças. Há uma caprichosa ``tabela'' de
traduções em ``A cartomante'', que se poderia assim organizar:

\begin{quote}
frase de Hamlet a Horácio / explicação de Rita a Camilo

mitologia, religiosidade / superstições ou incredulidade

carro de Apolo / caleça de praça

estilo alto / palavras vulgares, mal compostas

personagens trágicas / tonta Rita, ingênuo Camilo

a morada do Destino / a casa da cartomante

sibila / italiana sonsa

dilema do ``ser ou não ser'' / dilema do ``ir ou não ir''

nivelamento das personagens / nivelamento das personagens

pelo sublime pela vulgaridade
\end{quote}

Por esse sistema de traduções, seria lícito concluir que Machado vale
Shakespeare, ``guardadas as proporções''. Ora, o narrador age exatamente
como um operador do desproporcional, tirando todo o efeito da ironia de
não admitir isso. A partir desse ângulo privilegiado da lucidez que não
tem compromisso com qualquer valor senão consigo mesma, a História vale
um delírio, uma ópera, um papel avulso ou uma folha sem data, um
pretexto qualquer para se recolherem as ``páginas amigas'', que nem por
serem amenas deixam de concentrar uma espécie de ``suma da vida''.
Traduzindo desproporções como equivalências, o narrador atrai o leitor
para o seu sistema, do qual não é fácil sair. Para consegui-lo, teremos
que ter precisão quanto aos nossos valores e suas diferenças; teremos
que definir antagonismos reais, contradições verdadeiras, e ser
consequentes --- exatamente as tarefas mais problemáticas que enfrenta o
pensamento crítico, quando resiste às diluições da modernidade eufórica.
Parece-me ser esse o desafio que, politicamente, Machado armou para si e
para seu público, de ontem e de hoje.

Ainda retornando à fórmula das ``vulgaridades sublimes'', tentemos
aprofundar suas implicações. Sem sair da lógica dessa fusão de opostos,
poderíamos igualmente reconhecer o corolário das ``tragédias
vulgarizadas'', e com ela fundar uma nova perspectiva para o conto. Tal
inversão em nada contradiz o jogo das traduções levado a efeito pelo
narrador, que aliás está sempre a estimulá-lo. A determinação
estilística do perfeccionismo, da elegância culta e do requinte retórico
é o único traço que o narrador não pode ocultar, e talvez seja o único
que de fato o revele. Esse lugar do estilo não surge ``vulgarizado''; se
já não é o sublime, ou o épico, ou o trágico, é por certo ainda um lugar
privilegiado, de cuja altura retórica nos é lançado um olhar
condescendente. Que lugar é esse, onde nasce o princípio absoluto das
relativizações, dos ``caprichos'' (Augusto Meyer), da ``volubilidade''
(Roberto Schwarz), das ``simetrias'' (Alfredo Bosi)?

A pergunta supõe alguma estabilidade do ponto de vista em que a
consciência do narrador se detém para elaborar-se e para promover o
diálogo com o mundo, no desejo de sua representação. Sem essa
estabilidade, ainda que dissimulada, precária ou mínima, não há autor,
estilo e forma consequentes. Creio que em ``A cartomante'', como num
sem-número de outros lugares, o narrador machadiano instala-se nesse
ângulo tão peculiar de ``tradutor'': um tradutor das tradições que
constituem seu repertório de cultura, que vem da Bíblia e de Homero, da
antiguidade clássica e dos teólogos medievais, que passa por Dante,
Maquiavel, Montaigne, Cervantes, Shakespeare, Pascal, pelos
enciclopedistas, por Schopenhauer, pela literatura brasileira --- e acaba
caindo no colo da dama fluminense ou num chapéu elegante da rua do
Ouvidor. Essa ``queda'' --- na verdade o já reconhecido salto crítico do
Machado particularizante e universalmente nacional --- é a marca de fogo
de sua fase madura, quando a ironia se torna princípio e a ``tradução''
uma rica possibilidade de composição. Multiplicado nessa liberdade
vertiginosa, o narrador rastreia quaisquer horizontes para selecionar
com a aparência do arbítrio o que de fato se origina e resulta
determinado. Assentado que está em nível retórico-estilístico de altura
indiscutível, pode-se permitir a fusão da galhofa e da melancolia sem
perder a reverência básica de uma linguagem a ser cultuada pela mesma
dama fluminense ou pelo dono do chapéu. É desse lugar a um tempo
dialético e cristalizado que se podem ver Rita, Camilo, Hamlet e Horácio
numa sequência que, se de um lado promove uma dissolução de valores, de
outro ainda os distingue enquanto singularidades aptas à ``tradução''.
\emph{Ainda} os distingue: é dessa frágil reserva de tempo que parece
anunciar-se uma liquidação geral e efetiva, hipótese em razão da qual o
narrador já se deita ao modo de um defunto e toma posse de uma
(pós)última decisão da consciência teimosamente ativa. O vazio íntimo de
Rita e de Camilo é indicado segundo um parâmetro que poderia supor uma
tão secreta quanto reprimida nostalgia do absoluto, recoberta pelo senso
do realismo e pela análise do cotidiano burguês. Nessa perspectiva, o
sentimento do trágico já se banalizou, vulgarizando-se por consequência
toda uma custosa tradição de expressões do sublime, que perderam o lugar
próprio. Fulminados, Rita e Camilo atualizam com seu próprio estilo de
viver e de morrer uma tragédia burguesa que Machado escreveu para
traduzir o seu tempo e a si mesmo.

\section*{V}

É nesse específico lugar de ``tradutor'' que tantas e tantas vezes se
instalará o narrador livre mas sistemático da fase madura. O humor
principal que daí se destila vive de um paradoxo que busca se negar
enquanto tal: a liberdade ronda o caos, ameaça promover o absurdo e o
nonsense, como a reedição do Gênesis no capítulo ``O delírio'' ou a
conversa gestual entre Adão e Eva (\emph{Memórias póstumas}) --- mas o
leitor sente que o narrador nunca afrouxará o punho firme que segura a
pena racional e elegante, deslizando pelo estilo inconfundível. Tais
``traduções'' tornam-se tema e processo, com direito a um sem-número de
variações que sabem se adaptar à diversidade das situações narradas. Nos
contos, há alguns que traduzem outros (caso de ``Um homem célebre'' e
``Cantiga de esponsais'', por exemplo), variando detalhes, ênfases e
tonalidades, que reparticularizam tudo. O efeito inicial pode ser a
sensação do \emph{mesmo} nas \emph{diferenças} (quando se busca
reconhecer o modo de narrar ou alguma ``ideologia'' sistemática), mas
modula-se no efeito da percepção de diferenças que alcançam alguma
emancipação do \emph{mesmo} (quando se privilegia na análise o
particularismo da expressão artística). Sim, uma coisa não vem sem a
outra, e parece nascer da junção delas o efeito geral de paradoxo, assim
resumível: o narrador vive na multiplicidade das situações criadas sem
se deixar levar por essa mesma multiplicação, antes subordinando-as ao
seu sistema de ``tradutor''. A variedade dos tempos históricos, dos
valores, dos desejos humanos, das lutas pelo poder, dos gêneros e dos
estilos é considerada, sim, por um minuto, para no minuto seguinte
passar pelo funil estreito da perspectiva do narrador, onde a qualidade
original aparece ``traduzida''. Assim é que a metafísica da alma humana
pode exemplificar-se numa laranja (``O espelho'') e a patologia de um
Calígula atualizar-se em escala reduzida num certo Fortunato, que aliás
passa por benemérito (``A causa secreta''). Tais ``traduções'' em nada
escandalizariam um Schopenhauer que considera inacessível aos mortais o
rosto mesmo da Vontade; que dá como limite o estatuto das
\emph{representações} do mundo; e que nos lembra o fato de que uma
circunferência de diâmetro descomunal tem as mesmas propriedades
geométricas da de um diâmetro diminuto. Também para Machado parece certo
que ``muitas vezes uma só hora é a representação de uma vida inteira''
(``O empréstimo''), e nesse caso o contista (e o autor de tantos
capítulos de romances) teria o privilégio de poder ``traduzir'' em
poucas palavras o essencial de uma existência --- e por que não o da
História mesma? No curto espaço de \emph{O Príncipe}, Maquiavel oferece
a Lorenzo de Médicis ``tudo aquilo que, em tantos anos e à custa de
tantos incômodos e perigos, hei conhecido''. Se em estreito molde se
fundava a moderna ciência política (deslocando-se o sentido da
\emph{virtù} do moralismo medieval para a do pragmatismo do poder), por
que não caberia a específica astúcia de um medalhão caboclo numa teoria
exposta em sessenta minutos (``Teoria do medalhão'')? As ``proporções''
a que se refere a personagem desse conto, remetendo-nos a Maquiavel, têm
o sentido duplo das traduções machadianas: tanto implicam o reducionismo
implícito ou explícito do modelo quanto a manutenção do mesmo sentido
básico do ``original'', atualizado e ``traduzido''. Uma universalização
tão descarada parece, no entanto, aguda e verdadeira, quando ela parece
confirmar-se em cada caso particular, transitando da narrativa para a
vida, da ficção para a experiência, num retorno coerente e exemplar, que
irritou Augusto Meyer e o levou à expressão ``monstro cerebral''. O
lastro de realismo considerado pelo autor a cada página é pesado o
bastante para que não nos deliciemos impunemente com a amenidade que
Machado adiciona ao tom; nem aceita o bruxo o poder da imaginação
indiscriminada, pela qual viesse a afastar-se um só milímetro do eixo de
seu realismo básico.

Machado de Assis, que admitia tudo, menos ``ser empulhado'', não admitiu
para si mesmo a hipótese de ser apenas um dissolvente ``tradutor'' de
tradições, um mestre do \emph{divertissement} ou da ironia só engenhosa.
Quando, por exemplo, na ``Teoria do medalhão'', usa e abusa da retórica
decorativa, ela já se faz exemplo prático e funcional da linguagem
recomendada ao tipo; quando numa crônica se vale da filosofia ao tratar
de um recém-publicado manual de confeiteiro, o abuso é declarado. Na
crônica intitulada ``O autor de si mesmo'' (\emph{Gazeta de Notícias},
16/06/1895), Schopenhauer comparece em pessoa, como um Artur familiar,
ratificando seu pessimismo à luz do caso trágico, recente e
dolorosíssimo, que parece ter atingido Machado em cheio. São mostras de
como as ``traduções'', em níveis e em consequências tão diversos,
constituem um autêntico processo de criação e de crítica,
irreversivelmente aberto a partir das \emph{Memórias póstumas}.

Lembremos ainda dois contos: ``O empréstimo'' (\emph{Papéis avulsos}) e
``Um homem célebre'' (\emph{Várias histórias}). O primeiro se apresenta
como anedota verídica, mas não dispensa a entrada de Carlyle, Pitágoras,
Sêneca e Balzac, antes de se contar: é o caso de um pobre-diabo chamado
Custódio, que busca fazer de um tabelião um sócio seu numa fábrica de
agulhas de padrão inglês, sociedade que não custaria ao felizardo
parceiro mais do que \emph{cinco contos}. Diante das negativas do
tabelião, Custódio vai reduzindo o valor do empréstimo, reduzindo,
reduzindo, até que a empreitada inicial se transforma num ``jantar
certo'', patrocinado pelos cinco mil-réis que o outro lhe concede.
Decepção para Custódio? Absolutamente não: ele sai com a nota ``como se
viesse de conquistar a Ásia Menor''. Conclui o narrador: ``ele apertava
amorosamente os \emph{cinco mil-réis}, resíduo de uma grande ambição,
que ainda há pouco saíra contra o sol, num ímpeto de águia, e ora batia
modestamente as asas de frango rasteiro''. Está claro que no frango
rasteiro há um resíduo de águia, e que o pobre Custódio alegrou-se de
qualquer modo porque, afinal de contas, foi pedir um empréstimo e obteve
outro empréstimo, mantendo-se na astronômica \emph{diferença} entre as
quantias a salvaguarda do \emph{mesmo} princípio.

Quanto à obra-prima que é ``Um homem célebre'', não vale a pena insistir
na já decantada sombra autobiográfica que ronda o conto (a experiência
de vida de Machado projeta-se muito mais poderosamente do que parece, em
suas histórias; está por se fazer um estudo meticuloso dessas projeções,
em consequência das quais restaria bastante relativizado o semidogma do
\emph{distanciamento} do narrador --- no final das contas um estratagema
contra a ampla confissão). Interessa aqui sublinhar a distância que vai
das sonatas de Beethoven às polcas de Pestana, tão diferentes entre si,
mas afinal tão próximas para quem tira muito prazer destas, sem que
desmereça aquelas --- possibilidade que não se ofereceu ao romântico
Pestana, condenado pelo autor a sofrer em plena consagração mundana. A
condenação só concede um gesto de simpatia \emph{in extremis}: à morte,
Pestana mostra-se lúcido e revela súbito senso de humor, aproximando-se
enfim do próprio tom da narração e comungando, nesse único instante, da
verve do narrador. Fosse Pestana um Custódio, ouviria suas polcas como
se sonatas fossem, conquistando-as como se representassem a Ásia Menor,
valendo-se delas para ampliar até alguma Roma imperial a sensação que
vinha do sucesso fluminense. Já no plano coletivo da política, ao qual
Pestana servia a seu modo incauto, dedicando-lhe polcas, tudo também
estaria em ninguém se deixar impressionar com a queda ou ascensão dos
liberais ou dos conservadores, sabendo-se que nesse processo as
\emph{diferenças}, mais do que nunca, redundavam no \emph{mesmo}. Por
certo Machado preferiu tratar dessa questão política espelhando-a em seu
próprio processo de criação, processo político em sentido mais ativo e
problemático, resguardado na ampla ironia de tão provocadoras
``traduções''.

Arrisquemos, por fim, um paralelo. No \emph{Ulisses} de Joyce há, sim, o
paroxismo linguístico de quem crê em seu poder final de destruição e
criação, de quem é épico e pedestre na simultaneidade com que opera
tanto as linguagens faladas em Dublin como uma máquina poética
sofisticadíssima, em que os códigos se baralham e se reinventam a cada
momento, atingindo no coração o preceito da unidade estilística. Moderno
e modernista, Joyce encarna com esses pesos a função demiúrgica, própria
de quem tem deuses e demônios a reproduzir e a exorcizar com empenho
máximo. Machado de Assis, diferentemente, considera com impecável apuro
o triunfo da vulgaridade, e só a deixa fracassar quando ainda mais
decisiva do que ela é a falta de malícia ou do poder de adaptação do
sujeito em que ela se encarna. Esse foco objetivante, que de modo algum
quer se deixar confundir com seu objeto, precisa para isso de recursos
estáveis, independentes e autossuficientes, sem os quais o narrador se
arriscaria a escorregar entre valores, identificando-se com as fórmulas
já batidas do pessimismo, do cinismo, do niilismo, para nem falar das
posições afirmativas do moralismo, do cientificismo, do idealismo.
Diante desse estoque de tão aliciantes possibilidades, o narrador
machadiano faz com que elas se traduzam umas pelas outras, vivendo ele
próprio da estabilidade estilística desse lugar aparentemente imune a
qualquer contradição, que é o lugar do puro observador, ou, quando não
apenas isso, o do velho, o do morto, o do diplomata aposentado. Mas só
\emph{aparentemente} imune: quem quisesse suprimir as contradições,
traduzindo-as pelo que não seria mais do que a sempre mesma \emph{verità
effetuale delle cose} (Maquiavel), não abriria nunca o espaço político
da ironia e da análise lúcida, que se definem na diferença pela qual se
constitui, em definitivo, o sujeito que se recusa a ser traduzido pela
perspectiva das \emph{coisas-mesmas}.

\textsc{villaça}, Alcides. ``Machado de Assis, tradutor de si mesmo''.
Originalmente publicado em: \emph{Revista Novos Estudos Cebrap.} São
Paulo, 1998, n. 51, pp. 3--14.

\chapter{A Cartomante\footnote[*]{O conto ``A Cartomante'' foi incluído por
  Machado de Assis na coletânea \emph{Várias Histórias}, publicada em
  1896. O texto desta edição foi cotejado com o daquela.}}

Hamlet observa a Horácio que há mais coisas no céu e na terra do que
sonha a nossa filosofia\footnote{Alusão à peça \emph{Hamlet}, de William
  Shakespeare (1564--1616), publicada em 1603.}. Era a mesma explicação
que dava a bela Rita ao moço Camilo, numa sexta-feira de novembro de
1869, quando este ria dela, por ter ido na véspera consultar uma
cartomante; a diferença é que o fazia por outras palavras.

--- Ria, ria. Os homens são assim; não acreditam em nada. Pois saiba que
fui, e que ela adivinhou o motivo da consulta, antes mesmo que eu lhe
dissesse o que era. Apenas começou a botar as cartas, disse-me: ``A
senhora gosta de uma pessoa\ldots{}''. Confessei que sim, e então ela
continuou a botar as cartas, combinou-as, e no fim declarou-me que eu
tinha medo de que você me esquecesse, mas que não era verdade\ldots{}

--- Errou! --- interrompeu Camilo, rindo.

--- Não diga isso, Camilo. Se você soubesse como eu tenho andado, por sua
causa. Você sabe; já lhe disse. Não ria de mim, não ria\ldots{}

Camilo pegou-lhe nas mãos, e olhou para ela sério e fixo. Jurou que lhe
queria muito, que os seus sustos pareciam de criança; em todo o caso,
quando tivesse algum receio, a melhor cartomante era ele mesmo. Depois,
repreendeu-a; disse-lhe que era imprudente andar por essas casas. Vilela
podia sabê-lo, e depois\ldots{}

--- Qual saber! Tive muita cautela, ao entrar na casa.

--- Onde é a casa?

--- Aqui perto, na rua da Guarda Velha; não passava ninguém nessa
ocasião. Descansa; eu não sou maluca.

Camilo riu outra vez:

--- Tu crês deveras nessas coisas? perguntou-lhe.

Foi então que ela, sem saber que traduzia Hamlet em vulgar, disse-lhe
que havia muita coisa misteriosa e verdadeira neste mundo. Se ele não
acreditava, paciência; mas o certo é que a cartomante adivinhara tudo.
Que mais? A prova é que ela agora estava tranquila e satisfeita.

Cuido que ele ia falar, mas reprimiu-se. Não queria arrancar-lhe as
ilusões. Também ele, em criança, e ainda depois, foi supersticioso, teve
um arsenal inteiro de crendices, que a mãe lhe incutiu e que aos vinte
anos desapareceram. No dia em que deixou cair toda essa vegetação
parasita, e ficou só o tronco da religião, ele, como tivesse recebido da
mãe ambos os ensinos, envolveu-os na mesma dúvida, e logo depois em uma
só negação total. Camilo não acreditava em nada. Por quê? Não poderia
dizê-lo, não possuía um só argumento; limitava-se a negar tudo. E digo
mal, porque negar é ainda afirmar, e ele não formulava a incredulidade;
diante do mistério, contentou-se em levantar os ombros, e foi andando.

Separaram-se contentes, ele ainda mais que ela. Rita estava certa de ser
amada; Camilo não só o estava, mas via-a estremecer e arriscar-se por
ele, correr às cartomantes, e, por mais que a repreendesse, não podia
deixar de sentir-se lisonjeado. A casa do encontro era na antiga rua dos
Barbonos, onde morava uma coprovinciana de Rita. Esta desceu pela rua
das Mangueiras, na direção de Botafogo, onde residia; Camilo desceu pela
da Guarda Velha, olhando de passagem para a casa da cartomante.

Vilela, Camilo e Rita, três nomes, uma aventura, e nenhuma explicação
das origens. Vamos a ela. Os dois primeiros eram amigos de infância.
Vilela seguiu a carreira de magistrado. Camilo entrou no funcionalismo,
contra a vontade do pai, que queria vê-lo médico; mas o pai morreu, e
Camilo preferiu não ser nada, até que a mãe lhe arranjou um emprego
público. No princípio de 1869, voltou Vilela da província, onde casara
com uma dama formosa e tonta; abandonou a magistratura e veio abrir
banca de advogado. Camilo arranjou-lhe casa para os lados de Botafogo, e
foi a bordo recebê-lo.

--- É o senhor? exclamou Rita, estendendo-lhe a mão. Não imagina como meu
marido é seu amigo, falava sempre do senhor.

Camilo e Vilela olharam-se com ternura. Eram amigos deveras. Depois,
Camilo confessou de si para si que a mulher do Vilela não desmentia as
cartas do marido. Realmente, era graciosa e viva nos gestos, olhos
cálidos, boca fina e interrogativa. Era um pouco mais velha que ambos:
contava trinta anos, Vilela, vinte e nove e Camilo, vinte e seis.
Entretanto, o porte grave de Vilela fazia-o parecer mais velho que a
mulher, enquanto Camilo era um ingênuo na vida moral e prática.
Faltava-lhe tanto a ação do tempo, como os óculos de cristal, que a
natureza põe no berço de alguns para adiantar os anos. Nem experiência,
nem intuição.

Uniram-se os três. Convivência trouxe intimidade. Pouco depois morreu a
mãe de Camilo, e nesse desastre, que o foi, os dois mostraram-se grandes
amigos dele. Vilela cuidou do enterro, dos sufrágios e do inventário;
Rita tratou especialmente do coração, e ninguém o faria melhor.

Como daí chegaram ao amor, não o soube ele nunca. A verdade é que
gostava de passar as horas ao lado dela; era a sua enfermeira moral,
quase uma irmã, mas principalmente era mulher e bonita. \emph{Odor di
femmina}\footnote{Do italiano, aroma de mulher.}: eis o que ele aspirava
nela, e em volta dela, para incorporá-lo em si próprio. Liam os mesmos
livros, iam juntos a teatros e passeios. Camilo ensinou-lhe as damas e o
xadrez e jogavam às noites; --- ela, mal, --- ele, para lhe ser agradável,
pouco menos mal. Até aí as coisas. Agora a ação da pessoa, os olhos
teimosos de Rita, que procuravam muita vez os dele, que os consultavam
antes de o fazer ao marido, as mãos frias, as atitudes insólitas. Um
dia, fazendo ele anos, recebeu de Vilela uma rica bengala de presente, e
de Rita apenas um cartão com um vulgar cumprimento a lápis, e foi então
que ele pôde ler no próprio coração; não conseguia arrancar os olhos do
bilhetinho. Palavras vulgares; mas há vulgaridades sublimes, ou, pelo
menos, deleitosas. A velha caleça\footnote{Carruagem alugada (nas praças
  do Rio de Janeiro).} de praça, em que pela primeira vez passeaste com
a mulher amada, fechadinhos ambos, vale o carro de Apolo\footnote{Veículo
  que seria utilizado pelo Deus Apolo, segundo a mitologia grega, na
  Antiguidade.}. Assim é o homem, assim são as coisas que o cercam.

Camilo quis sinceramente fugir, mas já não pôde. Rita, como uma
serpente, foi-se acercando dele, envolveu-o todo, fez-lhe estalar os
ossos num espasmo, e pingou-lhe o veneno na boca. Ele ficou atordoado e
subjugado. Vexame, sustos, remorsos, desejos, tudo sentiu de mistura;
mas a batalha foi curta e a vitória, delirante. Adeus, escrúpulos! Não
tardou que o sapato se acomodasse ao pé, e aí foram ambos, estrada fora,
braços dados, pisando folgadamente por cima de ervas e pedregulhos, sem
padecer nada mais que algumas saudades, quando estavam ausentes um do
outro. A confiança e estima de Vilela continuavam a ser as mesmas.

Um dia, porém, recebeu Camilo uma carta anônima, que lhe chamava imoral
e pérfido, e dizia que a aventura era sabida de todos. Camilo teve medo,
e, para desviar as suspeitas, começou a rarear as visitas à casa de
Vilela. Este notou-lhe as ausências. Camilo respondeu que o motivo era
uma paixão frívola de rapaz. Candura gerou astúcia. As ausências
prolongaram-se, e as visitas cessaram inteiramente. Pode ser que
entrasse também nisso um pouco de amor-próprio, uma intenção de diminuir
os obséquios do marido, para tornar menos dura a aleivosia do ato.

Foi por esse tempo que Rita, desconfiada e medrosa, correu à cartomante
para consultá-la sobre a verdadeira causa do procedimento de Camilo.
Vimos que a cartomante restituiu-lhe a confiança, e que o rapaz
repreendeu-a por ter feito o que fez. Correram ainda algumas semanas.
Camilo recebeu mais duas ou três cartas anônimas, tão apaixonadas, que
não podiam ser advertência da virtude, mas despeito de algum
pretendente; tal foi a opinião de Rita, que, por outras palavras mal
compostas, formulou este pensamento: --- a virtude é preguiçosa e avara,
não gasta tempo nem papel; só o interesse é ativo e pródigo.

Nem por isso Camilo ficou mais sossegado; temia que o anônimo fosse ter
com Vilela, e a catástrofe viria então sem remédio. Rita concordou que
era possível.

--- Bem, disse ela; eu levo os sobrescritos para comparar a letra com a
das cartas que lá aparecerem; se alguma for igual, guardo-a e
rasgo-a\ldots{}

Nenhuma apareceu; mas daí a algum tempo Vilela começou a mostrar-se
sombrio, falando pouco, como desconfiado. Rita deu-se pressa em dizê-lo
ao outro, e sobre isso deliberaram. A opinião dela é que Camilo devia
tornar à casa deles, tatear o marido, e pode ser até que lhe ouvisse a
confidência de algum negócio particular. Camilo divergia; aparecer
depois de tantos meses era confirmar a suspeita ou denúncia. Mais valia
acautelarem-se, sacrificando-se por algumas semanas. Combinaram os meios
de se corresponderem, em caso de necessidade, e separaram-se com
lágrimas.

No dia seguinte, estando na repartição, recebeu Camilo este bilhete de
Vilela: ``Vem já, já, à nossa casa; preciso falar-te sem demora''. Era
mais de meio-dia. Camilo saiu logo; na rua, advertiu que teria sido mais
natural chamá-lo ao escritório; por que em casa? Tudo indicava matéria
especial, e a letra, fosse realidade ou ilusão, afigurou-se-lhe trêmula.
Ele combinou todas essas coisas com a notícia da véspera.

--- Vem já, já, à nossa casa; preciso falar-te sem demora --- repetia ele
com os olhos no papel.

Imaginariamente, viu a ponta da orelha de um drama, Rita subjugada e
lacrimosa, Vilela indignado, pegando da pena e escrevendo o bilhete,
certo de que ele acudiria, e esperando-o para matá-lo. Camilo
estremeceu, tinha medo: depois sorriu amarelo, e em todo caso
repugnava-lhe a ideia de recuar, e foi andando. De caminho, lembrou-se
de ir a casa; podia achar algum recado de Rita, que lhe explicasse tudo.
Não achou nada, nem ninguém. Voltou à rua, e a ideia de estarem
descobertos parecia-lhe cada vez mais verossímil; era natural uma
denúncia anônima, até da própria pessoa que o ameaçara antes; podia ser
que Vilela conhecesse agora tudo. A mesma suspensão das suas visitas,
sem motivo aparente, apenas com um pretexto fútil, viria confirmar o
resto.

Camilo ia andando inquieto e nervoso. Não relia o bilhete, mas as
palavras estavam decoradas, diante dos olhos, fixas; ou então --- o que
era ainda pior --- eram-lhe murmuradas ao ouvido, com a própria voz de
Vilela. ``Vem já, já, à nossa casa; preciso falar-te sem demora''. Ditas
assim, pela voz do outro, tinham um tom de mistério e ameaça. Vem, já,
já, para quê? Era perto de uma hora da tarde. A comoção crescia de
minuto a minuto. Tanto imaginou o que se iria passar, que chegou a
crê-lo e vê-lo. Positivamente, tinha medo. Entrou a cogitar em ir
armado, considerando que, se nada houvesse, nada perdia, e a precaução
era útil. Logo depois rejeitava a ideia, vexado de si mesmo, e seguia,
picando o passo, na direção do largo da Carioca, para entrar
num tílburi\footnote{Carruagem para um passageiro, puxada por um cavalo.}.
Chegou, entrou e mandou seguir a trote largo.

\begin{itemize}
\item
  Quanto antes, melhor, pensou ele; não posso estar assim\ldots{}
\end{itemize}

Mas o mesmo trote do cavalo veio agravar-lhe a comoção. O tempo voava, e
ele não tardaria a entestar com o perigo. Quase no fim da rua da Guarda
Velha, o tílburi teve de parar; a rua estava atravancada com uma
carroça, que caíra. Camilo, em si mesmo, estimou o obstáculo, e esperou.
No fim de cinco minutos, reparou que ao lado, à esquerda, ao pé
do tílburi, ficava a casa da cartomante, a quem Rita consultara uma vez,
e nunca ele desejou tanto crer na lição das cartas. Olhou, viu as
janelas fechadas, quando todas as outras estavam abertas e pejadas de
curiosos do incidente da rua. Dir-se-ia a morada do
indiferente Destino\footnote{Grafado com inicial maiúscula, no original.
  Possível alusão às reflexões sobre a vida como uma sucessão de causas
  e consequências, conforme concebidas pelos antigos filósofos gregos e
  latinos.}.

Camilo reclinou-se no tílburi, para não ver nada. A agitação dele era
grande, extraordinária, e do fundo das camadas morais emergiam alguns
fantasmas de outro tempo, as velhas crenças, as superstições antigas. O
cocheiro propôs-lhe voltar a primeira travessa, e ir por outro caminho;
ele respondeu que não, que esperasse. E inclinava-se para fitar a
casa\ldots{} Depois fez um gesto incrédulo: era a ideia de ouvir a
cartomante, que lhe passava ao longe, muito longe, com vastas asas
cinzentas; desapareceu, reapareceu, e tornou a esvair-se no cérebro; mas
daí a pouco moveu outra vez as asas, mais perto, fazendo uns giros
concêntricos\ldots{} Na rua, gritavam os homens, safando a carroça:

--- Anda! agora! empurra! vá! vá!

Daí a pouco estaria removido o obstáculo. Camilo fechava os olhos,
pensava em outras coisas; mas a voz do marido sussurrava-lhe às orelhas
as palavras da carta: ``Vem, já, já\ldots{}'' E ele via as contorções do
drama e tremia. A casa olhava para ele. As pernas queriam descer e
entrar\ldots{} Camilo achou-se diante de um longo véu opaco\ldots{}
pensou rapidamente no inexplicável de tantas coisas. A voz da mãe
repetia-lhe uma porção de casos extraordinários, e a mesma frase do
príncipe de Dinamarca reboava-lhe dentro: ``Há mais coisas no céu e na
terra do que sonha a nossa filosofia\ldots{}'' Que perdia ele,
se\ldots{}?

Deu por si na calçada, ao pé da porta; disse ao cocheiro que esperasse,
e rápido enfiou pelo corredor, e subiu a escada. A luz era pouca, os
degraus, comidos dos pés, o corrimão, pegajoso; mas ele não viu nem
sentiu nada. Trepou e bateu. Não aparecendo ninguém, teve ideia de
descer; mas era tarde, a curiosidade fustigava-lhe o sangue, as fontes
latejavam-lhe; ele tornou a bater uma, duas, três pancadas. Veio uma
mulher; era a cartomante. Camilo disse que ia consultá-la, ela fê-lo
entrar. Dali subiram ao sótão, por uma escada ainda pior que a primeira
e mais escura. Em cima, havia uma salinha, mal alumiada por uma janela,
que dava para o telhado dos fundos. Velhos trastes, paredes sombrias, um
ar de pobreza, que antes aumentava do que destruía o prestígio.

A cartomante fê-lo sentar diante da mesa, e sentou-se do lado oposto,
com as costas para a janela, de maneira que a pouca luz de fora batia em
cheio no rosto de Camilo. Abriu uma gaveta e tirou um baralho de cartas
compridas e enxovalhadas. Enquanto as baralhava, rapidamente, olhava
para ele, não de rosto, mas por baixo dos olhos. Era uma mulher de
quarenta anos, italiana, morena e magra, com grandes olhos sonsos e
agudos. Voltou três cartas sobre a mesa, e disse-lhe:

--- Vejamos primeiro o que é que o traz aqui. O senhor tem um grande
susto\ldots{}

Camilo, maravilhado, fez um gesto afirmativo.

--- E quer saber, continuou ela, se lhe acontecerá alguma coisa ou
não\ldots{}

--- A mim e a ela, explicou vivamente ele.

A cartomante não sorriu; disse-lhe só que esperasse. Rápido pegou outra
vez das cartas e baralhou-as, com os longos dedos finos, de unhas
descuradas; baralhou-as bem, transpôs os maços, uma, duas, três vezes;
depois começou a estendê-las. Camilo tinha os olhos nela, curioso e
ansioso.

--- As cartas dizem-me\ldots{}

Camilo inclinou-se para beber uma a uma as palavras. Então ela
declarou-lhe que não tivesse medo de nada. Nada aconteceria nem a um nem
a outro; ele, o terceiro, ignorava tudo. Não obstante, era indispensável
muita cautela: ferviam invejas e despeitos. Falou-lhe do amor que os
ligava, da beleza de Rita\ldots{} Camilo estava deslumbrado. A
cartomante acabou, recolheu as cartas e fechou-as na gaveta.

--- A senhora restituiu-me a paz ao espírito, disse ele estendendo a mão
por cima da mesa e apertando a da cartomante.

Esta levantou-se, rindo.

--- Vá, disse ela; vá, \emph{ragazzo innamorato}\ldots{}\footnote{Do
  italiano, rapaz enamorado.}

E de pé, com o dedo indicador, tocou-lhe na testa. Camilo
estremeceu, como se fosse a mão da própria sibila\footnote{Mulher que
  pertencia ao oráculo, em acordo com a antiga mitologia grega.}, e
levantou-se também. A cartomante foi à cômoda, sobre a qual estava um
prato com passas, tirou um cacho destas, começou a despencá-las e
comê-las, mostrando duas fileiras de dentes que desmentiam as unhas.
Nessa mesma ação comum, a mulher tinha um ar particular. Camilo, ansioso
por sair, não sabia como pagasse; ignorava o preço.

--- Passas custam dinheiro, disse ele afinal, tirando a carteira. Quantas
quer mandar buscar?

--- Pergunte ao seu coração, respondeu ela.

Camilo tirou uma nota de dez mil-réis, e deu-lha. Os olhos da cartomante
fuzilaram. O preço usual era dois mil-réis.

--- Vejo bem que o senhor gosta muito dela\ldots{} E faz bem; ela gosta
muito do senhor. Vá, vá tranquilo. Olhe a escada, é escura; ponha o
chapéu\ldots{}

A cartomante tinha já guardado a nota na algibeira, e descia com ele,
falando, com um leve sotaque. Camilo despediu-se dela embaixo, e desceu
a escada que levava à rua, enquanto a cartomante, alegre com a paga,
tornava acima, cantarolando uma barcarola\footnote{Composição muito
  comum na Itália, desde o século \textsc{xv}.}. Camilo achou
o tílburi esperando, a rua estava livre. Entrou e seguiu a trote largo.

Tudo lhe parecia agora melhor, as outras coisas traziam outro aspecto, o
céu estava límpido e as caras joviais. Chegou a rir dos seus receios,
que chamou pueris; recordou os termos da carta de Vilela e reconheceu
que eram íntimos e familiares. Onde é que ele lhe descobrira a ameaça?
Advertiu também que eram urgentes, e que fizera mal em demorar-se tanto;
podia ser algum negócio grave e gravíssimo.

--- Vamos, vamos depressa, repetia ele ao cocheiro.

E consigo, para explicar a demora ao amigo, engenhou qualquer coisa;
parece que formou também o plano de aproveitar o incidente para tornar à
antiga assiduidade\ldots{} De volta com os planos, reboavam-lhe na alma
as palavras da cartomante. Em verdade, ela adivinhara o objeto da
consulta, o estado dele, a existência de um terceiro; por que não
adivinharia o resto? O presente que se ignora vale o futuro. Era assim,
lentas e contínuas, que as velhas crenças do rapaz iam tornando ao de
cima, e o mistério empolgava-o com as unhas de ferro. Às vezes queria
rir, e ria de si mesmo, algo vexado; mas a mulher, as cartas, as
palavras secas e afirmativas, a exortação: --- Vá, vá, \emph{ragazzo
innamorato}; e no fim, ao longe, a barcarola da despedida, lenta e
graciosa, tais eram os elementos recentes, que formavam, com os antigos,
uma fé nova e vivaz.

A verdade é que o coração ia alegre e impaciente, pensando nas horas
felizes de outrora e nas que haviam de vir. Ao passar pela
Glória\footnote{Bairro do Rio de Janeiro.}, Camilo olhou para o mar,
estendeu os olhos para fora, até onde a água e o céu dão um abraço
infinito, e teve assim uma sensação do futuro, longo, longo,
interminável.

Daí a pouco chegou à casa de Vilela. Apeou-se, empurrou a porta de ferro
do jardim e entrou. A casa estava silenciosa. Subiu os seis degraus de
pedra, e mal teve tempo de bater, a porta abriu-se, e apareceu-lhe
Vilela.

--- Desculpa, não pude vir mais cedo; que há?

Vilela não lhe respondeu: tinha as feições decompostas; fez-lhe sinal, e
foram para uma saleta interior. Entrando, Camilo não pôde sufocar um
grito de terror: --- ao fundo, sobre o canapé\footnote{Assento parecido
  com o sofá, que permite sentar ou se deitar de lado.}, estava Rita
morta e ensanguentada. Vilela pegou-o pela gola, e, com dois tiros de
revólver, estirou-o morto no chão.
