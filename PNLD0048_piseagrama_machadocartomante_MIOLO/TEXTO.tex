\chapter{Apresentação}

A coletânea \emph{O Alienista — O Imortal — A Cartomante} contém três
narrativas de Machado de Assis: as duas primeiras saíram na revista
\emph{A Estação}, periódico voltado para o público feminino que circulou
no Rio de Janeiro entre 1879 e 1904, e a terceira na \emph{Gazeta de
Notícias}. Nelas, reconfigura a matéria social do seu tempo,
desestabiliza a verossimilhança e desnaturaliza os modos habituais de
ler.

\emph{O Alienista}, conto"-novela de Machado de Assis publicado em
\emph{A Estação} entre outubro de 1881 e março de 1882 e incluído em
\emph{Papéis avulsos} em 1892, nos abala a pensar em múltiplas questões
atuais: a importância e os limites da ciência; o exercício tirânico do
poder, que se apropria segundo as conveniências da ciência; os conflitos
e sofrimentos decorrentes da nossa morada nos hábitos instituídos e da
recusa ao diferente; a prevalência de interesses particulares e do
capricho, acima de princípios éticos e coletivos.

``Itaguaí é o meu universo.'' Essa afirmação do alienista Simão
Bacamarte e as estratégias do narrador em terceira pessoa de evocar
crônicas itaguaienses para atestar a verossimilhança de sua história,
bem como de estabelecer um paralelo de episódios locais, como o Terror e
a Revolta dos Canjicas, com as fases da Revolução Francesa, compõem uma
forma peculiar machadiana, de \emph{humour} crítico: ele nos
possibilita, jocosa e melancolicamente, ver semelhanças e desproporções
entre realidades brasileiras e estrangeiras, criando uma compreensão
quanto às iniquidades conservadas desde a formação colonial brasileira,
e quanto aos motores egoístas das ações.

Essa forma específica de reflexão também se manifesta em ``O Imortal'',
que chama a atenção por aproximar"-se do gênero do conto fantástico.
Aqui, o leitor encontrará a história de Rui de Leão, homem que viveu
mais de duzentos anos depois de tomar um elixir indígena. Esse
protagonista participa da Invasão Holandesa e da brutal aniquilação do
Quilombo dos Palmares, no Brasil, além de correr mundo e contribuir para
eventos como a Rebelião de Monmouth, na Inglaterra, e daí de volta ao
Brasil, no tráfico de escravos, na chegada da corte portuguesa em 1808 e
na Independência, em 1822. Mais uma vez, Machado chama o leitor, de
forma galhofeira e soturna, a articular as contradições da realidade
brasileira a um contexto mais amplo.

Justamente ``A Cartomante'', conto presente também na antologia,
configura essa forma peculiar da ironia machadiana. Na aparência, o
enredo é bastante simples: Rita, casada com Vilela, tem um caso amoroso
com Camilo, amigo do marido. Os amantes se assustam com ameaças
recebidas por correspondência anônima, e, depois de muita tensão
crescente, o conto se conclui com uma tragédia final, só revelada no
último parágrafo. Em um dos ensaios contidos nesta edição, chamado
``Machado de Assis, tradutor de si mesmo'', o professor e crítico
Alcides Villaça afirma que o processo de composição desse conto, apoiado
frequentemente em simetrias, traduz ``desproporções como
equivalências'', distingue e dissolve valores, promovendo a
relativização de tudo, num processo que inclui ``o dialético e o
cristalizado'', de forma a deixar ao leitor a tarefa de perceber
``antagonismos reais'' e resistir às diluições. De maneira geral, no
conjunto da obra machadiana, o narrador, diante de múltiplas
possibilidades de ``traduções'', cria para si a ``estabilidade
estilística'' do lugar de observador, aparentemente imune a
contradições. Mas afinal ele abre ``o espaço político da ironia e da
análise lúcida'', por meio das quais se constitui como sujeito,
recusando"-se a ser traduzido pela ``perspectiva das coisas"-mesmas''.
Essas traduções, marcas em especial dos contos machadianos aqui
incluídos, constituem um processo crítico de criação concebido desde as
\emph{Memórias póstumas de Brás Cubas}.

Nos três textos de Machado de Assis que compõem esta antologia, o leitor
encontrará, portanto, exemplos marcantes da agudeza analítica do autor a
respeito do seu tempo, intrinsecamente associada à forma específica que
o consagrou, instrumento crítico e estético cujo alcance se faz notar
até hoje.


\chapter{Advertência}

Esta coletânea contém três narrativas de Machado de Assis, duas delas
publicadas na Revista \emph{A Estação} --- periódico voltado para o
público feminino que circulou no Rio de Janeiro entre 1879 e 1904 --- e
uma na \emph{Gazeta de Notícias}. Os textos foram rigorosamente
cotejados com as coletâneas em que apareceram após sua disseminação em
folhetim.

A novela ``O Alienista'' é introduzida por Jean Pierre Chauvin,
professor de \emph{Cultura e Literatura Brasileira} da Escola de
Comunicações e Artes e credenciado nos programas de pós-graduação
Estudos Comparados de Literaturas de Língua Portuguesa, na Faculdade de
Filosofia, Letras e Ciências Humanas (\textsc{usp}) e no programa de
pós-graduação em Letras da Escola de Filosofia, Letras e Ciências
Humanas (Unifesp). O conto ``O Imortal'' é apresentado por João Adolfo
Hansen\footnote{Agradeço a João Adolfo Hansen, que autorizou a
  reprodução de seu ensaio, publicado originalmente na revista
  \emph{Teresa} (Departamento de Letras Clássicas e Vernáculas,
  \textsc{fflch}/\textsc{usp}) em 2005; e a Alcides Villaça, que permitiu a reprodução de
  seu ensaio, publicado originalmente na revista \emph{Novos Estudos}
  (Cebrap), em 1998.}, docente aposentado da Faculdade de Filosofia,
Letras e Ciências Humanas da \textsc{usp} (onde colabora com o programa de
pós-graduação em \emph{Literatura Brasileira}). Atualmente, é professor
visitante do curso de pós em Letras da Unifesp, \emph{Campus} Guarulhos.
``A Cartomante'' é discutida por Alcides Villaça, que leciona
\emph{Literatura Brasileira} na Faculdade de Filosofia, Letras e
Ciências Humanas (\textsc{usp}) e é credenciado no programa de pós-graduação em
Literatura Brasileira.

Nesta edição, foi respeitada a pontuação empregada por Machado de Assis
por se acreditar que determinadas alterações, ainda que justificáveis da
perspectiva estritamente gramatical, poderiam modificar o entendimento
da prosódia e o uso estilístico da língua pelo escritor, em seu tempo e
lugar. Quanto à grafia, os ditongos ``ou'', que apareciam em ``dous'',
``cousa(s)'' e ``doudo(s)'' foram substituídos por ``oi''
{[}\emph{dois}, \emph{coisa}(s) e \emph{doido}(s){]}. Com vistas a
orientar a leitura fluente e estimular a fruição dos textos, as
narrativas de Machado contêm notas de rodapé que trazem dados históricos
e culturais. Quando julgado necessário, atualizam o significado de
algumas palavras empregadas no final do século \textsc{xix} --- hoje, em desuso.

Feitas essas breves considerações, esperamos que leitoras e leitores
possam extrair máximo proveito e deleite das páginas que seguem.

\bigskip

\hfill{}\emph{J.P.C.}

\part[o alienista — o imortal — a cartomante]{\textsc{o alienista — o imortal —\break a cartomante}}

\chapter{O alienista\footnote[*]{O texto que se vai ler foi cotejado com a
  primeira versão em livro da novela que abre a coletânea \emph{Papéis
  Avulsos}, editada pela Garnier (Paris e Rio de Janeiro) em 1882.}}


\section*{\textsc{i}. De como Itaguaí ganhou\break uma Casa de Orates\protect\footnote[\dagger]{\MakeUppercase{C}asa de \MakeUppercase{L}oucos (o termo ``\MakeUppercase{O}rate'' é de origem espanhola).}}

\noindent{}As crônicas da vila de Itaguaí\footnote{Corresponde ao atual município
  de Itaguaí, no interior do Rio de Janeiro. A forma ``vila'' situa a
  localidade no final do século \textsc{xviii}.} dizem que em tempos remotos
vivera ali um certo médico, o Dr.\,Simão Bacamarte, filho da nobreza da
terra e o maior dos médicos do Brasil, de Portugal e das Espanhas.
Estudara em Coimbra e Pádua. Aos trinta e quatro anos regressou ao
Brasil, não podendo el-rei\footnote{``El-Rei'' era a forma respeitosa
  com que os súditos, em geral, referiam-se a Dom João \textsc{v} (1689--1750),
  rei de Portugal.} alcançar dele que ficasse em Coimbra, regendo a
universidade, ou em Lisboa, expedindo os negócios da monarquia.

--- A ciência, disse ele a Sua Majestade, é o meu emprego único; Itaguaí
é o meu universo.

Dito isto, meteu-se em Itaguaí, e entregou-se de corpo e alma ao estudo
da ciência, alternando as curas com as leituras, e demonstrando os
teoremas com cataplasmas. Aos quarenta anos casou com D.\,Evarista da
Costa e Mascarenhas, senhora de vinte e cinco anos, viúva de um juiz de
fora, e não bonita nem simpática. Um dos tios dele, caçador de pacas
perante o Eterno\footnote{Paródia de uma expressão que aparece no
  \emph{Gênesis} (``poderoso caçador perante o Eterno''), no cap. 10,
  vers. 9.}, e não menos franco, admirou-se de semelhante escolha e
disse-lho\footnote{Forma castiça que resulta de contração: ``Disse-o a
  ele''.}. Simão Bacamarte explicou-lhe que D.\,Evarista reunia condições
fisiológicas e anatômicas de primeira ordem, digeria com facilidade,
dormia regularmente, tinha bom pulso, e excelente vista; estava assim
apta para dar-lhe filhos robustos, sãos e inteligentes. Se além dessas
prendas, --- únicas dignas da preocupação de um sábio, D.\,Evarista era
mal composta de feições, longe de lastimá-lo, agradecia-o a Deus,
porquanto não corria o risco de preterir os interesses da ciência na
contemplação exclusiva, miúda e vulgar da consorte.

D.\,Evarista mentiu às esperanças do Dr.\,Bacamarte, não lhe deu filhos
robustos nem mofinos\footnote{Débeis, frágeis, fracos.}. A índole
natural da ciência é a longanimidade\footnote{Generosidade. Resistência
  frente as adversidades.}; o nosso médico esperou três anos, depois
quatro, depois cinco. Ao cabo desse tempo fez um estudo profundo da
matéria, releu todos os escritores árabes e outros, que trouxera para
Itaguaí, enviou consultas às universidades italianas e alemãs, e acabou
por aconselhar à mulher um regime alimentício especial. A ilustre dama,
nutrida exclusivamente com a bela carne de porco de Itaguaí, não atendeu
às admoestações\footnote{Conselho, aviso, repreensão.} do esposo; e à
sua resistência, --- explicável, mas inqualificável, --- devemos a total
extinção da dinastia dos Bacamartes.

Mas a ciência tem o inefável dom de curar todas as mágoas; o nosso
médico mergulhou inteiramente no estudo e na prática da medicina. Foi
então que um dos recantos desta lhe chamou especialmente a atenção, ---
o recanto psíquico, o exame da patologia cerebral. Não havia na colônia,
e ainda no reino, uma só autoridade em semelhante matéria, mal
explorada, ou quase inexplorada\footnote{Esse dado respalda-se na
  historiografia. As primeiras instituições reservadas para a reclusão e
  terapêutica dos loucos remontam à Idade Média; mas os estabelecimentos
  dessa natureza demoraram séculos para chegar a Portugal.}. Simão
Bacamarte compreendeu que a ciência lusitana, e particularmente a
brasileira, podia cobrir-se de ``louros imarcescíveis''\footnote{Expressão
  enfática que combina o símbolo da glória (``louros'') à sua perenidade
  (``imarcessível''). No contexto, sugere que as folhas (da vitória) não
  perderiam o viço (o vigor).}, --- expressão usada por ele mesmo, mas
em um arroubo de intimidade doméstica; exteriormente era modesto,
segundo convém aos sabedores.

--- A saúde da alma, bradou ele, é a ocupação mais digna do médico.

--- Do verdadeiro médico, emendou Crispim Soares, boticário\footnote{Responsável
  pela botica, estabelecimento onde se manipulavam remédios.} da vila, e
um dos seus amigos e comensais\footnote{Convidados para o banquete.}.

A vereança\footnote{Câmara de Vereadores.} de Itaguaí, entre outros
pecados de que é arguida pelos cronistas, tinha o de não fazer caso dos
dementes. Assim é que cada louco furioso era trancado em uma alcova, na
própria casa, e, não curado, mas descurado, até que a morte o vinha
defraudar\footnote{Contrariar a lei, fraudar.} do benefício da vida; os
mansos\footnote{Alusão de Machado à teoria dos humores, formulada por
  Hipócrates (460--370 a.C.). O médico grego classificava os homens em
  quatro tipos: irascíveis (sangue), fleumáticos (fleuma), coléricos
  (bile amarela) e melancólicos (bile negra).} andavam à solta pela rua.
Simão Bacamarte entendeu desde logo reformar tão ruim costume; pediu
licença à Câmara para agasalhar e tratar no edifício que ia construir
todos os loucos de Itaguaí e das demais vilas e cidades\footnote{``Vilas''
  e ``cidades'' eram localidades de diferente configuração. A comarca de
  Itaguaí passou ao estatuto de vila em 1720. Em 1818, foi alçada a
  município.}, mediante um estipêndio\footnote{Pagamento de uma
  gratificação.}, que a Câmara lhe daria quando a família do enfermo o
não pudesse fazer. A proposta excitou a curiosidade de toda a vila, e
encontrou grande resistência, tão certo é que dificilmente se
desarraigam hábitos absurdos, ou ainda maus. A ideia de meter os loucos
na mesma casa, vivendo em comum, pareceu em si mesma um sintoma de
demência, e não faltou quem o insinuasse à própria mulher do médico.

--- Olhe, D.\,Evarista, disse-lhe o padre Lopes, vigário do lugar, veja
se seu marido dá um passeio ao Rio de Janeiro. Isso de estudar sempre,
sempre, não é bom, vira o juízo.

D.\,Evarista ficou aterrada, foi ter com o marido, disse-lhe ``que estava
com desejos'', um principalmente, o de vir\footnote{Machado optou pelo
  verbo ``vir'' (e não ``ir'').} ao Rio de Janeiro\footnote{Capital do
  Estado do Brasil a partir de 1763.} e comer tudo o que a ele lhe
parecesse adequado a certo fim. Mas aquele grande homem, com a rara
sagacidade que o distinguia, penetrou a intenção da esposa e
redarguiu-lhe sorrindo que não tivesse medo. Dali foi à Câmara, onde os
vereadores debatiam a proposta, e defendeu-a com tanta eloquência, que a
maioria resolveu autorizá-lo ao que pedira, votando ao mesmo tempo um
imposto destinado a subsidiar o tratamento, alojamento e mantimento dos
doidos pobres. A matéria do imposto não foi fácil achá-la; tudo estava
tributado em Itaguaí. Depois de longos estudos, assentou-se em permitir
o uso de dois penachos nos cavalos dos enterros. Quem quisesse emplumar
os cavalos de um coche mortuário pagaria dois\footnote{Na novela, o
  autor alternou a grafia do número entre ``dous'' e ``dois''. As formas
  foram padronizadas para ``dois''. O mesmo vale para ``coisa'' e
  ``doido'' (aqui, grafadas ``coisa'' e ``doido'').} tostões à Câmara,
repetindo-se tantas vezes esta quantia quantas fossem as horas
decorridas entre a do falecimento e a da última bênção na sepultura. O
escrivão perdeu-se nos cálculos aritméticos do rendimento possível da
nova taxa; e um dos vereadores, que não acreditava na empresa do médico,
pediu que se relevasse o escrivão de um trabalho inútil.

--- Os cálculos não são precisos, disse ele, porque o Dr.\,Bacamarte não
arranja nada. Quem é que viu agora meter todos os doidos dentro da mesma
casa?

Enganava-se o digno magistrado; o médico arranjou tudo. Uma vez
empossado da licença começou logo a construir a casa. Era na Rua Nova, a
mais bela rua de Itaguaí naquele tempo, tinha cinquenta janelas por
lado, um pátio no centro, e numerosos cubículos para os hóspedes. Como
fosse grande arabista, achou no \emph{Corão}\footnote{O \emph{Corão},
  livro sagrado dos seguidores de Maomé, desde o século \textsc{vii}.} que
Maomé\footnote{Maomé (571--632), o mensageiro de Alá, a quem se atribui a
  autoria do \emph{Corão}.} declara veneráveis os doidos, pela
consideração de que Alá\footnote{Denominação de Deus em árabe. O
  ``achado'' de Simão Bacamarte não está no \emph{Corão}.} lhes tira o
juízo para que não pequem. A ideia pareceu-lhe bonita e profunda, e ele
a fez gravar no frontispício da casa; mas, como tinha medo ao vigário, e
por tabela ao bispo, atribuiu o pensamento a Benedito \textsc{viii}\footnote{Benedito
  \textsc{viii} (980--1024), centésimo quadragésimo terceiro papa romano.
  Enfrentou os sarracenos, na Itália. Dentre algumas regras que
  instituiu com vigor, está o celibato dos padres.}, merecendo com essa
fraude, aliás pia, que o padre Lopes lhe contasse, ao almoço, a vida
daquele pontífice eminente.

A Casa Verde foi o nome dado ao asilo, por alusão à cor das janelas, que
pela primeira vez apareciam verdes em Itaguaí. Inaugurou-se com imensa
pompa; de todas as vilas e povoações próximas, e até remotas, e da
própria cidade do Rio de Janeiro, correu gente para assistir às
cerimônias, que duraram sete dias. Muitos dementes já estavam
recolhidos; e os parentes tiveram ocasião de ver o carinho paternal e a
caridade cristã com que eles iam ser tratados. D.\,Evarista,
contentíssima com a glória do marido, vestira-se luxuosamente, cobriu-se
de joias, flores e sedas. Ela foi uma verdadeira rainha naqueles dias
memoráveis; ninguém deixou de ir visitá-la duas e três vezes, apesar dos
costumes caseiros e recatados do século, e não só a cortejavam como a
louvavam; porquanto, --- e este fato é um documento altamente honroso
para a sociedade do tempo, ---porquanto viam nela a feliz esposa de um
alto espírito, de um varão ilustre, e, se lhe tinham inveja, era a santa
e nobre inveja dos admiradores. Ao cabo de sete dias expiraram as festas
públicas; Itaguaí tinha finalmente uma casa de orates.

\section*{\textsc{ii}. Torrente de Loucos}

Três dias depois, numa expansão íntima com o boticário Crispim Soares,
desvendou o alienista o mistério do seu coração.

--- A caridade, Sr.\,Soares, entra decerto no meu procedimento, mas entra
como tempero, como o sal das coisas, que é assim que interpreto o dito
de São Paulo aos Coríntios: ``Se eu conhecer quanto se pode saber, e não
tiver caridade, não sou nada''.\footnote{Cf. Epístola de Paulo aos
  Coríntios, capítulo 1, versículo 13.} O principal nesta minha obra da
Casa Verde é estudar profundamente a loucura, os seus diversos graus,
classificar-lhe os casos, descobrir enfim a causa do fenômeno e o
remédio universal. Este é o mistério do meu coração. Creio que com isto
presto um bom serviço à humanidade.

--- Um excelente serviço, corrigiu o boticário.

--- Sem este asilo, continuou o alienista, pouco poderia fazer; ele
dá-me, porém, muito maior campo aos meus estudos.

--- Muito maior, acrescentou o outro.

E tinham razão. De todas as vilas e arraiais vizinhos afluíam loucos à
Casa Verde. Eram furiosos, eram mansos, eram monomaníacos, era toda a
família dos deserdados do espírito. Ao cabo de quatro meses, a Casa
Verde era uma povoação. Não bastaram os primeiros cubículos; mandou-se
anexar uma galeria de mais trinta e sete. O padre Lopes confessou que
não imaginara a existência de tantos doidos no mundo, e menos ainda o
inexplicável de alguns casos. Um, por exemplo, um rapaz bronco e vilão,
que todos os dias, depois do almoço, fazia regularmente um discurso
acadêmico, ornado de tropos\footnote{Termos utilizados com sentido
  diverso do habitual (como a metáfora e a metonímia).}, de
antíteses\footnote{Aproximação de palavras de sentido oposto.}, de
apóstrofes\footnote{Interrupção, no discurso, para aludir a seres reais
  ou fictícios. Interpelação. Dito mordaz.}, com seus recamos\footnote{Ornamento,
  floreio discursivo.} de grego e latim, e suas borlas\footnote{Literalmente,
  pompons, enfeites. Metaforicamente, adornos, ornamentos no discurso.}
de Cícero, Apuleio e Tertuliano\footnote{Cícero (106--43 a.C.), orador;
  Apuleio (125--170 d.C.), filósofo; Tertuliano (160--220 d.C.), teólogo
  cristão.}. O vigário não queria acabar de crer. Quê! um rapaz que ele
vira, três meses antes, jogando peteca na rua!

--- Não digo que não, respondia-lhe o alienista; mas a verdade é o que
Vossa Reverendíssima está vendo. Isto é todos os dias.

--- Quanto a mim, tornou o vigário, só se pode explicar pela confusão
das línguas na torre de Babel, segundo nos conta a Escritura\footnote{Edificação
  construída pelos descendentes de Noé, após o dilúvio. Por punição de
  Deus, houve confusão entre as línguas ali praticadas. O episódio está
  no \emph{Gênesis}, capítulo 11.}; provavelmente, confundidas
antigamente as línguas, é fácil trocá-las agora, desde que a razão não
trabalhe\ldots{}

--- Essa pode ser, com efeito, a explicação divina do fenômeno,
concordou o alienista, depois de refletir um instante, mas não é
impossível que haja também alguma razão humana, e puramente científica,
e disso trato\ldots{}

--- Vá que seja, e fico ansioso. Realmente!

Os loucos por amor eram três ou quatro, mas só dois espantavam pelo
curioso do delírio. O primeiro, um Falcão, rapaz de vinte e cinco anos,
supunha-se estrela-d'alva, abria os braços e alargava as pernas, para
dar-lhes certa feição de raios, e ficava assim horas esquecidas a
perguntar se o sol já tinha saído para ele recolher-se. O outro andava
sempre, sempre, sempre, à roda das salas ou do pátio, ao longo dos
corredores, à procura do fim do mundo. Era um desgraçado, a quem a
mulher deixou por seguir um peralvilho. Mal descobrira a fuga, armou-se
de uma garrucha, e saiu-lhes no encalço; achou-os duas horas depois, ao
pé de uma lagoa, matou-os a ambos com os maiores requintes de crueldade.
O ciúme satisfez-se, mas o vingado estava louco. E então começou aquela
ânsia de ir ao fim do mundo à cata dos fugitivos.

A mania das grandezas tinha exemplares notáveis. O mais notável era um
pobre-diabo, filho de um algibebe, que narrava às paredes (porque não
olhava nunca para nenhuma pessoa) toda a sua genealogia, que era esta:

--- Deus engendrou um ovo, o ovo engendrou a espada, a espada engendrou
Davi\footnote{Rei de Israel, a quem se atribuem os salmos, na
  \emph{Bíblia}.}, Davi engendrou a púrpura, a púrpura engendrou o
duque, o duque engendrou o marquês, o marquês engendrou o conde, que sou
eu.\footnote{Possível paródia relacionada a Sem, filho de Noé, constante
  do capítulo onze, no \emph{Gênesis}.}

Dava uma pancada na testa, um estalo com os dedos, e repetia cinco, seis
vezes seguidas:

--- Deus engendrou um ovo, o ovo, etc.

Outro da mesma espécie era um escrivão, que se vendia por mordomo do
rei; outro era um boiadeiro de Minas, cuja mania era distribuir boiadas
a toda a gente, dava trezentas cabeças a um, seiscentas a outro, mil e
duzentas a outro, e não acabava mais. Não falo dos casos de monomania
religiosa; apenas citarei um sujeito que, chamando-se João de Deus,
dizia agora ser o deus João, e prometia o reino dos céus a quem o
adorasse, e as penas do inferno aos outros; e depois desse, o licenciado
Garcia, que não dizia nada, porque imaginava que no dia em que chegasse
a proferir uma só palavra, todas as estrelas se despegariam do céu e
abrasariam a terra; tal era o poder que recebera de Deus. Assim o
escrevia ele no papel que o alienista lhe mandava dar, menos por
caridade do que por interesse científico.

Que, na verdade, a paciência do alienista era ainda mais extraordinária
do que todas as manias hospedadas na Casa Verde; nada menos que
assombrosa. Simão Bacamarte começou por organizar um pessoal de
administração; e, aceitando essa ideia ao boticário Crispim Soares,
aceitou-lhe também dois sobrinhos, a quem incumbiu da execução de um
regimento que lhes deu, aprovado pela câmara, da distribuição da comida
e da roupa, e assim também da escrita, etc. Era o melhor que podia
fazer, para somente cuidar do seu ofício. --- A Casa Verde, disse ele ao
vigário, é agora uma espécie de mundo, em que há o governo temporal e o
governo \emph{espiritual}\footnote{De acordo com os principais teólogos
  da igreja católica, Deus regeria o mundo dos homens (temporal) e
  aquele que os transcende (espiritual).}. E o padre Lopes ria deste pio
trocado\footnote{Dito piedoso que foi alterado pelo médico.}, --- e
acrescentava, --- com o único fim de dizer também uma chalaça\footnote{Dito
  zombeteiro.}: --- Deixe estar, deixe estar, que hei de mandá-lo
denunciar ao papa.

Uma vez desonerado da administração, o alienista procedeu a uma vasta
classificação dos seus enfermos. Dividiu-os primeiramente em duas
classes principais: os furiosos e os mansos; daí passou às subclasses,
monomanias, delírios, alucinações diversas. Isto feito, começou um
estudo aturado\footnote{Grafado desta forma, no original (em lugar de
  ``apurado'').} e contínuo; analisava os hábitos de cada louco, as
horas de acesso, as aversões, as simpatias, as palavras, os gestos, as
tendências; inquiria da vida dos enfermos, profissão, costumes,
circunstâncias da revelação mórbida, acidentes da infância e da
mocidade, doenças de outra espécie, antecedentes na família, uma
devassa\footnote{Averiguação, investigação. Possível alusão aos autos da
  devassa, implementada durante a chamada Inconfidência Mineira, entre
  1789 e 1792.}, enfim, como a não faria o mais atilado\footnote{Sagaz,
  inteligente, cuidadoso.} corregedor\footnote{Magistrado posicionado
  acima de seus colegas de profissão. Fiscaliza o trabalho dos demais.}.
E cada dia notava uma observação nova, uma descoberta interessante, um
fenômeno extraordinário. Ao mesmo tempo estudava o melhor regime, as
substâncias medicamentosas, os meios curativos e os meios paliativos,
não só os que vinham nos seus amados árabes, como os que ele mesmo
descobria, à força de sagacidade e paciência. Ora, todo esse trabalho
levava-lhe o melhor e o mais do tempo. Mal dormia e mal comia; e, ainda
comendo, era como se trabalhasse, porque ora interrogava um texto
antigo, ora ruminava uma questão, e ia muitas vezes de um cabo a outro
do jantar sem dizer uma só palavra a D.\,Evarista.

\section*{\textsc{iii}. Deus sabe o que faz!\protect\footnote[*]{\MakeUppercase{E}xpressão recorrente tanto na \emph{\MakeUppercase{B}íblia} quanto no \emph{\MakeUppercase{C}orão}.}}

A ilustre dama, no fim de dois meses, achou-se a mais desgraçada das
mulheres; caiu em profunda melancolia, ficou amarela, magra, comia pouco
e suspirava a cada canto. Não ousava fazer-lhe nenhuma queixa ou
reproche\footnote{Palavra de origem francesa, significa censura,
  repreensão.}, porque respeitava nele o seu marido e senhor, mas
padecia calada, e definhava a olhos vistos. Um dia, ao jantar, como lhe
perguntasse o marido o que é que tinha, respondeu tristemente\footnote{Grafado
  ``tritemente'', no original.} que nada; depois atreveu-se um pouco, e
foi ao ponto de dizer que se considerava tão viúva como dantes. E
acrescentou:

--- Quem diria nunca que meia dúzia de lunáticos\ldots{}

Não acabou a frase; ou antes, acabou-a levantando os olhos ao teto, ---
os olhos, que eram a sua feição mais insinuante, --- negros, grandes,
lavados de uma luz úmida, como os da aurora. Quanto ao gesto, era o
mesmo que empregara no dia em que Simão Bacamarte a pediu em casamento.
Não dizem as crônicas se D.\,Evarista brandiu aquela arma com o perverso
intuito de degolar de uma vez a ciência, ou, pelo menos, decepar-lhe as
mãos; mas a conjectura é verossímil. Em todo caso, o alienista não lhe
atribuiu outra intenção. E não se irritou o grande homem, não ficou
sequer consternado. O metal de seus olhos não deixou de ser o mesmo
metal, duro, liso, eterno, nem a menor prega veio quebrar a superfície
da fronte quieta como a água de Botafogo\footnote{Alusão à praia de
  Botafogo, no Rio de Janeiro, em sua porção sem ondas.}. Talvez um
sorriso lhe descerrou os lábios, por entre os quais filtrou esta palavra
macia como o óleo do \emph{Cântico}\footnote{Referência ao \emph{Cântico
  dos Cânticos}, livro do Antigo Testamento. A expressão ``óleo do
  \emph{Cântico}'' alude às metáforas de Salomão, que aproxima
  determinadas pessoas de perfumes característicos.}:

--- Consinto que vás dar um passeio ao Rio de Janeiro.

D.\,Evarista sentiu faltar-lhe o chão debaixo dos pés. Nunca dos nuncas
vira o Rio de Janeiro, que posto não fosse sequer uma pálida sombra do
que hoje é, todavia era alguma coisa mais do que Itaguaí. Ver o Rio de
Janeiro, para ela, equivalia ao sonho do hebreu cativo\footnote{Os
  ``hebreus'', enquanto cativos da Babilônia, sonhavam em retornar para
  a terra prometida.}. Agora, principalmente, que o marido assentara de
vez naquela povoação interior, agora é que ela perdera as últimas
esperanças de respirar os ares da nossa boa cidade; e justamente agora é
que ele a convidava a realizar os seus desejos de menina e moça. D.\,Evarista não pôde dissimular o gosto de semelhante proposta. Simão
Bacamarte pagou-lhe na mão e sorriu, --- um sorriso tanto ou quanto
filosófico, além de conjugal, em que parecia traduzir-se este
pensamento: --- ``Não há remédio certo para as dores da alma; esta
senhora definha, porque lhe parece que a não amo; dou-lhe o Rio de
Janeiro, e consola-se''. E porque era homem estudioso tomou nota da
observação.

Mas um dardo atravessou o coração de D.\,Evarista. Conteve-se,
entretanto; limitou-se a dizer ao marido, que, se ele não ia, ela não
iria também, porque não havia de meter-se sozinha pelas estradas.

--- Irá com sua tia, redarguiu o alienista.

Note-se que D.\,Evarista tinha pensado nisso mesmo; mas não quisera
pedi-lo nem insinuá-lo, em primeiro lugar porque seria impor grandes
despesas ao marido, em segundo lugar porque era melhor, mais metódico e
racional que a proposta viesse dele.

--- Oh! mas o dinheiro que será preciso gastar! suspirou D.\,Evarista sem
convicção.

--- Que importa? Temos ganho muito, disse o marido. Ainda ontem o
escriturário prestou-me contas. Queres ver?

E levou-a aos livros. D.\,Evarista ficou deslumbrada. Era uma via láctea
de algarismos. E depois levou-a às arcas, onde estava o dinheiro. Deus!
eram montes de ouro, eram mil cruzados sobre mil cruzados, dobrões sobre
dobrões; era a opulência. Enquanto ela comia o ouro com os seus olhos
negros, o alienista fitava-a, e dizia-lhe ao ouvido com a mais pérfida
das alusões.

--- Quem diria que meia dúzia de lunáticos\ldots{}

D.\,Evarista compreendeu, sorriu e respondeu com muita resignação:

--- Deus sabe o que faz!

Três meses depois efetuava-se a jornada. D.\,Evarista, a tia, a mulher do
boticário, um sobrinho deste, um padre que o alienista conhecera em
Lisboa, e que de aventura achava-se em Itaguaí, cinco ou seis
pajens\footnote{Criado, empregado (termo criado na Idade Média).},
quatro mucamas\footnote{Criada (durante a escravidão, também trabalhava
  como ama de leite).}, tal foi a comitiva que a população viu dali sair
em certa manhã do mês de maio. As despedidas foram tristes para todos,
menos para o alienista. Conquanto as lágrimas de D.\,Evarista fossem
abundantes e sinceras, não chegaram a abalá-lo. Homem de ciência, e só
de ciência, nada o consternava fora da ciência; e se alguma coisa o
preocupava naquela ocasião, se ele deixava correr pela multidão um olhar
inquieto e policial, não era outra coisa mais do que a ideia de que
algum demente podia achar-se ali misturado com a gente de juízo.

--- Adeus! soluçaram enfim as damas e o boticário.

E partiu a comitiva. Crispim Soares, ao tornar a casa, trazia os olhos
entre as duas orelhas da besta ruana\footnote{Pelagem branca com machas
  negras.} em que vinha montado; Simão Bacamarte alongava os seus pelo
horizonte adiante, deixando ao cavalo a responsabilidade do regresso.
Imagem vivaz do gênio e do vulgo\footnote{A descrição recupera a
  contrastante dupla Dom Quixote e Sancho Pança, cavaleiro e escudeiro
  do romance de Miguel de Cervantes Saavedra --- publicado em 1705 e
  1710.}! Um fita o presente, com todas as suas lágrimas e saudades,
outro devassa o futuro com todas as suas auroras.

\section*{\textsc{iv}. Uma teoria nova }

Ao passo que D.\,Evarista, em lágrimas, vinha buscando o Rio de Janeiro,
Simão Bacamarte estudava por todos os lados uma certa ideia arrojada e
nova, própria a alargar as bases da psicologia. Todo o tempo que lhe
sobrava dos cuidados da Casa Verde, era pouco para andar na rua, ou de
casa em casa, conversando\footnote{Grafado desta forma no original, sem
  a preposição ``com''.} as gentes, sobre trinta mil assuntos, e
virgulando as falas de um olhar que metia medo aos mais heroicos.

Um dia de manhã, --- eram passadas três semanas, --- estando Crispim
Soares ocupado em temperar um medicamento, vieram dizer-lhe que o
alienista o mandava chamar.

--- Trata-se de negócio importante, segundo ele me disse, acrescentou o
portador.

Crispim empalideceu. Que negócio importante podia ser, se não alguma
triste notícia da comitiva, e especialmente da mulher? Porque este
tópico deve ficar claramente definido, visto insistirem nele os
cronistas: Crispim amava a mulher, e, desde trinta anos, nunca estiveram
separados um só dia. Assim se explicam os monólogos que ele fazia agora,
e que os fâmulos\footnote{Serviçal, bajulador.} lhe ouviam muita vez:
--- ``Anda, bem feito, quem te mandou consentir na viagem de Cesária?
Bajulador, torpe bajulador! Só para adular ao Dr.\,Bacamarte. Pois agora
aguenta-te; anda, aguenta-te, alma de lacaio, fracalhão, vil, miserável.
Dizes amém a tudo, não é? aí tens o lucro, biltre\footnote{Patife, que
  age de modo vil.}!'' --- E muitos outros nomes feios, que um homem não
deve dizer aos outros, quanto mais a si mesmo. Daqui a imaginar o efeito
do recado é um nada. Tão depressa ele o recebeu como abriu mão das
drogas e voou à Casa Verde.

Simão Bacamarte recebeu-o com a alegria própria de um sábio, uma alegria
abotoada de circunspecção até o pescoço.

--- Estou muito contente, disse ele.

--- Notícias do nosso povo? perguntou o boticário com a voz trêmula.

O alienista fez um gesto magnífico, e respondeu:

--- Trata-se de coisa mais alta, trata-se de uma experiência científica.
Digo experiência, porque não me atrevo a assegurar desde já a minha
ideia; nem a ciência é outra coisa, Sr.\,Soares, se não uma investigação
constante. Trata-se, pois, de uma experiência, mas uma experiência que
vai mudar a face da terra. A loucura, objeto dos meus estudos, era até
agora uma ilha perdida no oceano da razão; começo a suspeitar que é um
continente.

Disse isto, e calou-se, para ruminar o pasmo do boticário. Depois
explicou compridamente a sua ideia. No conceito dele a insânia abrangia
uma vasta superfície de cérebros; e desenvolveu isto com grande cópia de
raciocínios, de textos, de exemplos. Os exemplos achou-os na história e
em Itaguaí\footnote{Bacamarte se refere a Itaguaí, mas pode-se inferir
  que Machado aludisse a ilustres cidadãos que moravam no Rio de Janeiro
  em seu tempo. A cidade contava com o Hospício Pedro \textsc{ii}, desde 1852 ---
  tempo que não corresponde ao da novela.}; mas, como um raro espírito
que era, reconheceu o perigo de citar todos os casos de Itaguaí e
refugiou-se na história. Assim, apontou com especialidade alguns
personagens célebres, Sócrates, que tinha um demônio familiar\footnote{Sócrates
  (469--399 a.C.). Filósofo grego, foi acusado de desviar os jovens. Foi
  condenado ao suicídio por envenenamento.}, Pascal, que via um abismo à
esquerda\footnote{Blaise Pascal (1623--1662). Pensador francês, com
  conhecimentos em matemática, física e teologia. Tornou-se célebre pelo
  livro de pensamentos e máximas, publicado após a sua morte.}, Maomé,
Caracala\footnote{Caracala, nome como ficou mais conhecido o imperador
  romano Marco Aurélio Antonino (188--217 d.C.). Implacável nos combates
  de que participou, concedeu cidadania romana à quase totalidade dos
  povos livres, em 212 d.C.}, Domiciano\footnote{Tito Flávio Domiciano
  (51--96 d.C.) foi um imperador romano, considerado tirânico e violento.},
Calígula\footnote{Caio Júlio César Augusto Germânico (12--41 d.C.) foi um
  imperador romano, considerado extravagante e violento. Foi assassinado
  pela guarda pretoriana, a mando de Cássio, que tencionava reimplantar
  a república.}, etc., uma enfiada de casos e pessoas, em que de mistura
vinham entidades odiosas, e entidades ridículas. E porque o boticário se
admirasse de uma tal promiscuidade, o alienista disse-lhe que era tudo a
mesma coisa, e até acrescentou sentenciosamente:

--- A ferocidade, Sr.\,Soares, é o grotesco a sério.

--- Gracioso, muito gracioso! exclamou Crispim Soares levantando as mãos
ao céu.

Quanto à ideia de ampliar o território da loucura, achou-a o boticário
extravagante; mas a modéstia, principal adorno de seu espírito, não lhe
sofreu confessar outra coisa além de um nobre entusiasmo; declarou-a
sublime e verdadeira, e acrescentou que era ``caso de matraca''. Esta
expressão não tem equivalente no estilo moderno. Naquele tempo, Itaguaí,
que como as demais vilas, arraiais e povoações da colônia, não dispunha
de imprensa\footnote{Antes de a imprensa do reino chegar ao país,
  portanto, o que só veio a acontecer em 1808.}, tinha dois modos de
divulgar uma notícia: ou por meio de cartazes manuscritos e pregados na
porta da câmara e da matriz; --- ou por meio de matraca. Eis em que
consistia este segundo uso. Contratava-se um homem, por um ou mais dias,
para andar as ruas do povoado, com uma matraca na mão. De quando em
quando tocava a matraca, reunia-se gente, e ele anunciava o que lhe
incumbiam, --- um remédio para sezões\footnote{Febre intermitente.},
umas terras lavradias\footnote{Terras apropriadas para a lavoura.}, um
soneto\footnote{Gênero poético que teria sido criado na Itália, entre os
  séculos \textsc{xiii} e \textsc{xiv}. Essa forma conquistou grande popularidade, sendo
  cultivada até hoje.}, um donativo eclesiástico\footnote{Doação à
  igreja.}, a melhor tesoura da vila, o mais belo discurso do ano, etc.
O sistema tinha inconvenientes para a paz pública; mas era conservado
pela grande energia de divulgação que possuía. Por exemplo, um dos
vereadores, --- aquele justamente que mais se opusera à criação da Casa
Verde, --- desfrutava a reputação de perfeito educador de cobras e
macacos, e aliás nunca domesticara um só desses bichos; mas, tinha o
cuidado de fazer trabalhar a matraca todos os meses. E dizem as crônicas
que algumas pessoas afirmavam ter visto cascavéis dançando no peito do
vereador; afirmação perfeitamente falsa, mas só devida à absoluta
confiança no sistema. Verdade, verdade, nem todas as instituições do
antigo regime\footnote{O Antigo Regime, ou Monarquia (expressão
  originária do francês), durou até o final do século \textsc{xviii}, quando foi
  derrubado pela Revolução Francesa.} mereciam o desprezo do nosso
século.

--- Há melhor do que anunciar a minha ideia, é praticá-la, respondeu o
alienista à insinuação do boticário.

E o boticário, não divergindo sensivelmente deste modo de ver, disse-lhe
que sim, que era melhor começar pela execução.

--- Sempre haverá tempo de a dar à matraca, concluiu ele.

Simão Bacamarte refletiu ainda um instante, e disse:

--- Suponho o espírito humano uma vasta concha, o meu fim, Sr.\,Soares, é
ver se posso extrair a pérola, que é a razão; por outros termos,
demarquemos definitivamente os limites da razão e da loucura. A razão é
o perfeito equilíbrio de todas as faculdades; fora daí insânia, insânia
e só insânia.

O Vigário Lopes a quem ele confiou a nova teoria, declarou
lisamente\footnote{Honesto, sem desvios.} que não chegava a entendê-la,
que era uma obra absurda, e, se não era absurda, era de tal modo
colossal que não merecia princípio de execução.

--- Com a definição atual\footnote{Alusão ao chamado período das luzes
  (ao final do século \textsc{xviii}), em que se os enciclopedistas reivindicavam
  a supremacia da razão, como caminho para o esclarecimento e a
  liberdade, a justiça e a fraternidade.}, que é a de todos os tempos,
acrescentou, a loucura e a razão estão perfeitamente delimitadas.
Sabe-se onde uma acaba e onde a outra começa. Para que transpor a cerca?

Sobre o lábio fino e discreto\footnote{Desde o século \textsc{xvii}, circulavam
  na Europa manuais que ensinavam os nobres a se portar, de maneira a
  afetar superioridade e não serem confundidos com o povo, dono de
  hábitos vulgares e indiscretos.} do alienista roçou a vaga sombra de
uma intenção de riso, em que o desdém vinha casado à
comiseração\footnote{Compaixão, piedade.}; mas nenhuma palavra saiu de
suas egrégias\footnote{Distinta, magnífica.} entranhas. A ciência
contentou-se em estender a mão à teologia, --- com tal segurança, que a
teologia não soube enfim se devia crer em si ou na outra. Itaguaí e o
universo ficavam à beira de uma revolução.

\section*{\textsc{v.} O Terror\protect\footnote[*]{\MakeUppercase{A}lusão ao período violento, sob os \MakeUppercase{J}acobinos, que sucedeu na \MakeUppercase{F}rança entre 1793 e 1794 e que culminou na prisão do líder \MakeUppercase{R}obespierre.}}

Quatro dias depois, a população de Itaguaí ouviu consternada a notícia
de que um certo Costa fora recolhido à Casa Verde.

--- Impossível!

--- Qual impossível! foi recolhido hoje de manhã.

--- Mas, na verdade, ele não merecia\ldots{} Ainda em cima! depois de
tanto que ele fez\ldots{}

Costa era um dos cidadãos mais estimados de Itaguaí. Herdara
quatrocentos mil cruzados em boa moeda de el-rei D.\,João \textsc{V}\footnote{Dom
  João \textsc{v} (1689--1750), rei português conhecido como ``o Magnânimo''. Seu
  mandato foi marcado pela descoberta e exploração de ouro em Minas
  Gerais. Na novela, isso parece explicar a generosa herança que deixou
  para o sobrinho.}, dinheiro cuja renda bastava, segundo lhe declarou o
tio no testamento, para viver ``até o fim do mundo''. Tão depressa
recolheu a herança, como entrou a dividi-la em empréstimos, sem usura,
mil cruzados a um, dois mil a outro, trezentos a este, oitocentos
àquele, a tal ponto que, no fim de cinco anos, estava sem nada. Se a
miséria viesse de chofre, o pasmo de Itaguaí seria enorme; mas veio
devagar; ele foi passando da opulência à abastança, da abastança à
mediania, da mediania à pobreza, da pobreza à miséria, gradualmente. Ao
cabo daqueles cinco anos, pessoas que levavam o chapéu ao chão, logo que
ele assomava no fim da rua, agora batiam-lhe no ombro, com intimidade,
davam-lhe piparotes no nariz, diziam-lhe pulhas\footnote{Sem caráter.}.
E o Costa sempre lhano\footnote{Simples, humilde, gentil.}, risonho. Nem
se lhe dava de ver que os menos corteses eram justamente os que tinham
ainda a dívida em aberto; ao contrário, parece que os agasalhava com
maior prazer, e mais sublime resignação. Um dia, como um desses
incuráveis devedores lhe atirasse uma chalaça grossa, e ele se risse
dela, observou um desafeiçoado, com certa perfídia: --- ``Você suporta
esse sujeito para ver se ele lhe paga''. Costa não se deteve um minuto,
foi ao devedor e perdoou-lhe a dívida. --- ``Não admira, retorquiu o
outro; o Costa abriu mão de uma estrela, que está no céu''. Costa era
perspicaz, entendeu que ele negava todo o merecimento ao ato,
atribuindo-lhe a intenção de rejeitar o que não vinham meter-lhe na
algibeira. Era também pundonoroso\footnote{Que valorizada a honra;
  orgulhoso de si mesmo (palavra de origem castelhana, ``\emph{punt
  d'honor}'').} e inventivo; duas horas depois achou um meio de provar
que lhe não cabia um tal labéu\footnote{Infâmia, mancha, calúnia.}:
pegou de algumas dobras\footnote{Moeda de ouro, equivalente a 12.800
  réis (ou 40 patacas de prata). Foi utilizada entre os séculos \textsc{xviii} e
  \textsc{xix}, em Portugal e suas possessões, incluindo o Brasil.}, e mandou-as
de empréstimo ao devedor.

--- Agora espero que\ldots{} --- pensou ele sem concluir a frase.

Esse último rasgo do Costa persuadiu a crédulos e incrédulos; ninguém
mais pôs em dúvida os sentimentos cavalheirescos daquele digno cidadão.
As necessidades mais acanhadas saíram à rua, vieram bater-lhe à porta,
com os seus chinelos velhos, com as suas capas remendadas. Um verme,
entretanto, roía a alma do Costa: era o conceito do desafeto. Mas isso
mesmo acabou; três meses depois veio este pedir-lhe uns cento e vinte
cruzados com promessa de restituir-lhos daí a dois dias; era o resíduo
da grande herança, mas era também uma nobre desforra: Costa emprestou o
dinheiro logo, logo, e sem juros. Infelizmente não teve tempo de ser
pago; cinco meses depois era recolhido à Casa Verde.

Imagina-se a consternação de Itaguaí, quando soube do caso. Não se falou
em outra coisa, dizia-se que o Costa ensandecera, no almoço, outros que
de madrugada; e contavam-se os acessos, que eram furiosos, sombrios,
terríveis, --- ou mansos, e até engraçados, conforme as versões. Muita
gente correu à Casa Verde, e achou o pobre Costa, tranquilo, um pouco
espantado, falando com muita clareza, e perguntando por que motivo o
tinham levado para ali. Alguns foram ter com o alienista. Bacamarte
aprovava esses sentimentos de estima e compaixão, mas acrescentava que a
ciência era a ciência, e que ele não podia deixar na rua um
mentecapto\footnote{Aquele que não possui razão, clareza ou
  discernimento (palavra de origem latina, ``\emph{mente + captus}'').}.
A última pessoa que intercedeu por ele (porque depois do que vou contar
ninguém mais se atreveu a procurar o terrível médico) foi uma pobre
senhora, prima do Costa. O alienista disse-lhe confidencialmente que
esse digno homem não estava no perfeito equilíbrio das faculdades
mentais, à vista do modo como dissipara os cabedais que\ldots{}

--- Isso, não! isso, não! interrompeu a boa senhora com energia. Se ele
gastou tão depressa o que recebeu, a culpa não é dele.

--- Não?

--- Não, senhor. Eu lhe digo como o negócio se passou. O defunto meu tio
não era mau homem; mas quando estava furioso era capaz de nem tirar o
chapéu ao Santíssimo\footnote{Sinal de respeito a Deus. Denominação do
  local, nas igrejas, reservado ao Santíssimo Sacramento.}. Ora, um dia,
pouco tempo antes de morrer, descobriu que um escravo lhe roubara um
boi; imagine como ficou. A cara era um pimentão; todo ele tremia, a boca
escumava\footnote{Espumava.}; lembra-me como se fosse hoje. Então um
homem feio, cabeludo, em mangas de camisa, chegou-se a ele e pediu água.
Meu tio (Deus lhe fale n'alma!) respondeu que fosse beber ao rio ou ao
inferno. O homem olhou para ele, abriu a mão em ar de ameaça, e rogou
esta praga: --- ``Todo o seu dinheiro não há de durar mais de sete anos
e um dia, tão certo como isto ser o \emph{sino salamão}!\footnote{Símbolo
  (dois triângulos entrelaçados) inscrito no anel do rei Salomão.
  Acreditava-se que tinha poderes mágicos e poderia protegê-lo dos maus
  espíritos.}'' E mostrou o \emph{sino salamão} impresso no braço. Foi
isto, meu senhor; foi esta praga daquele maldito.

Bacamarte espetara na pobre senhora um par de olhos agudos como punhais.
Quando ela acabou, estendeu-lhe a mão polidamente, como se o fizesse à
própria esposa do vice-rei\footnote{``O vice-rei ou capitão-general era
  o delegado imediato do soberano, para quem unicamente se podia apelar
  de suas resoluções. {[}\ldots{}{]} Presidia a Junta da Fazenda e,
  quando havia relação, era o governador dela; onde simples junta de
  justiça, era dela o presidente'' (Francisco Adolfo de Varnhagen.
  \emph{História Geral do Brasil.} São Paulo, s/e, 1952, tomo \textsc{iv}, p.
  289).}, e convidou-a a ir falar ao primo. A mísera acreditou; ele
levou-a à Casa Verde e encerrou-a na galeria dos alucinados.

A notícia desta aleivosia\footnote{Perfídia, traição.} do ilustre
Bacamarte lançou o terror à alma da população. Ninguém queria acabar de
crer, que, sem motivo, sem inimizade, o alienista trancasse na Casa
Verde uma senhora perfeitamente ajuizada, que não tinha outro crime
senão o de interceder por um infeliz. Comentava-se o caso nas esquinas,
nos barbeiros; edificou-se um romance, umas finezas namoradas que o
alienista outrora dirigira à prima do Costa, a indignação do Costa e o
desprezo da prima. E daí a vingança. Era claro. Mas a austeridade do
alienista, a vida de estudos que ele levava, pareciam desmentir uma tal
hipótese. Histórias! Tudo isso era naturalmente a capa do
velhaco\footnote{Falsa aparência do espertalhão.}. E um dos mais
crédulos chegou a murmurar que sabia de outras coisas, não as dizia, por
não ter certeza plena, mas sabia, quase que podia jurar.

---Você, que é íntimo dele, não nos podia dizer o que há, o que houve,
que motivo\ldots{}

Crispim Soares derretia-se todo. Esse interrogar da gente inquieta e
curiosa, dos amigos atônitos, era para ele uma consagração pública. Não
havia duvidar; toda a povoação sabia enfim que o privado do alienista
era ele, Crispim, o boticário, o colaborador do grande homem e das
grandes coisas; daí a corrida à botica. Tudo isso dizia o carão jucundo
e o riso discreto do boticário, o riso e o silêncio, porque ele não
respondia nada; um, dois, três monossílabos, quando muito, soltos,
secos, encapados no fiel sorriso constante e miúdo, cheio de mistérios
científicos, que ele não podia, sem desdouro nem perigo, desvendar a
nenhuma pessoa humana.

--- ``Há coisa,'' pensavam os mais desconfiados.

Um desses limitou-se a pensá-lo, deu de ombros e foi embora. Tinha
negócios pessoais Acabava de construir uma casa suntuosa\footnote{Luxuosa,
  opulenta.}. Só a casa bastava para deter a chamar toda a gente; mas
havia mais, --- a mobília, que ele mandara vir da Hungria e da Holanda,
segundo contava, e que se podia ver do lado de fora, porque as janelas
viviam abertas, --- e o jardim, que era uma obra-prima de arte e de
gosto. Esse homem, que enriquecera no fabrico de albardas\footnote{Sela
  para animais de carga.}, tinha tido sempre o sonho de uma casa
magnífica, jardim pomposo, mobília rara. Não deixou o negócio das
albardas, mas repousava dele na contemplação da casa nova, a primeira de
Itaguaí, mais grandiosa do que a Casa Verde, mais nobre do que a da
câmara. Entre a gente ilustre da povoação havia choro e ranger de
dentes, quando se pensava, ou se falava, ou se louvava a casa do
albardeiro, --- um simples albardeiro, Deus do céu!

--- Lá está ele embasbacado, diziam os transeuntes, de manhã.

De manhã, com efeito, era costume do Mateus estatelar-se, no meio do
jardim, com os olhos na casa, namorado\footnote{Grafado desta forma no
  original (e não ``enamorado'').}, durante uma longa hora, até que
vinham chamá-lo para almoçar. Os vizinhos, embora o cumprimentassem com
certo respeito, riam-se por trás dele, que era um gosto. Um desses
chegou a dizer que o Mateus seria muito mais econômico, e estaria
riquíssimo, se fabricasse as albardas para si mesmo; epigrama\footnote{Dito
  mordaz, satírico.} ininteligível, mas que fazia rir às bandeiras
despregadas\footnote{Gargalhar, rir sem controle.}.

--- Agora lá está o Mateus a ser contemplado, diziam à tarde.

A razão deste outro dito era que, de tarde, quando as famílias safam a
passeio (jantavam cedo) usava o Mateus postar-se à janela, bem no
centro, vistoso, sobre um fundo escuro, trajado de branco, atitude
senhoril, e assim ficava duas e três horas até que anoitecia de todo.
Pode crer-se que a intenção do Mateus era ser admirado e invejado, posto
que ele não a confessasse a nenhuma pessoa, nem ao boticário, nem ao
padre Lopes, seus grandes amigos. E entretanto não foi outra a alegação
do boticário, quando o alienista lhe disse que o albardeiro talvez
padecesse do amor das pedras, mania que ele Bacamarte descobrira e
estudava desde algum tempo. Aquilo de contemplar a casa\ldots{}

--- Não, senhor, acudiu vivamente Crispim Soares.

--- Não?

--- Há de perdoar-me, mas talvez não saiba que ele de manhã examina a
obra, não a admira; de tarde, são os outros que o admiram a ele e à
obra. --- E contou o uso do albardeiro, todas as tardes, desde cedo até
o cair da noite.

Uma volúpia científica alumiou os olhos de Simão Bacamarte. Ou ele não
conhecia todos os costumes do albardeiro, ou nada mais quis,
interrogando o Crispim, do que confirmar alguma notícia incerta ou
suspeita vaga. A explicação satisfê-lo; mas como tinha as alegrias
próprias de um sábio, concentradas, nada viu o boticário que fizesse
suspeitar uma intenção sinistra. Ao contrário, era de tarde, e o
alienista pediu-lhe o braço para irem a passeio. Deus! era a primeira
vez que Simão Bacamarte dava ao seu privado tamanha honra; Crispim ficou
trêmulo, atarantado, disse que sim, que estava pronto. Chegaram duas ou
três pessoas de fora, Crispim mandou-as mentalmente a todos os diabos;
não só atrasavam o passeio, como podia acontecer que Bacamarte elegesse
alguma delas, para acompanhá-lo, e o dispensasse a ele. Que impaciência!
que aflição! Enfim, saíram. O alienista guiou para os lados da casa do
albardeiro, viu-o à janela, passou cinco, seis vezes por diante,
devagar, parando, examinando as atitudes, a expressão do rosto. O pobre
Mateus, apenas notou que era objeto da curiosidade ou admiração do
primeiro vulto de Itaguaí, redobrou de expressão, deu outro relevo às
atitudes\ldots{} Triste! triste! não fez mais do que condenar-se; no dia
seguinte, foi recolhido à Casa Verde.

--- A Casa Verde é um cárcere privado, disse um médico sem clínica.

Nunca uma opinião pegou e grassou\footnote{Espalhou-se.} tão
rapidamente. Cárcere privado: eis o que se repetia de norte a sul e de
leste a oeste de Itaguaí, --- a medo, é verdade, porque durante a semana
que se seguiu à captura do pobre Mateus, vinte e tantas pessoas, ---
duas ou três de consideração, --- foram recolhidas à Casa Verde. O
alienista dizia que só eram admitidos os casos patológicos, mas pouca
gente lhe dava crédito. Sucediam-se as versões populares. Vingança,
cobiça de dinheiro, castigo de Deus, monomania do próprio médico, plano
secreto do Rio de Janeiro com o fim de destruir em Itaguaí qualquer
germe de prosperidade que viesse a brotar, arvorecer, florir, com
desdouro e míngua daquela cidade, mil outras explicações, que não
explicavam nada, tal era o produto diário da imaginação pública.

Nisto chegou do Rio de Janeiro a esposa do alienista, a tia, a mulher do
Crispim Soares, e toda a mais comitiva, --- ou quase toda ---, que
algumas semanas antes partira de Itaguaí. O alienista foi recebê-la, com
o boticário, o padre Lopes, os vereadores, e vários outros magistrados.
O momento em que D.\,Evarista pôs os olhos na pessoa do marido é
considerado pelos cronistas do tempo como um dos mais sublimes da
história moral dos homens, e isto pelo contraste das duas naturezas,
ambas extremas, ambas egrégias. D.\,Evarista soltou um grito, balbuciou
uma palavra e atirou-se ao consorte, de um gesto que não se pode melhor
definir do que comparando-o a uma mistura de onça e rola. Não assim o
ilustre Bacamarte; frio como um diagnóstico, sem desengonçar por um
instante a rigidez científica, estendeu os braços à dona que caiu neles,
e desmaiou. Curto incidente; ao cabo de dois minutos, D.\,Evarista
recebia os cumprimentos dos amigos e o préstito\footnote{Procissão,
  cortejo.} punha-se em marcha.

D.\,Evarista era a esperança de Itaguaí; contava-se com ela para minorar
o flagelo da Casa Verde. Daí as aclamações públicas, a imensa gente que
atulhava as ruas, as flâmulas\footnote{Bandeira.}, as flores e damascos
às janelas. Com o braço apoiado no do padre Lopes, --- porque o eminente
Bacamarte confiara a mulher ao vigário, e acompanhava-os a passo
meditativo, --- D.\,Evarista voltava a cabeça a um lado e outro, curiosa,
inquieta, petulante. O vigário indagava do Rio de Janeiro, que ele não
vira desde o vice-reinado anterior\footnote{Referência ao vice-reinado
  de Luís de Almeida Portugal Soares de Alarcão d'Eça e Melo Silva
  Mascarenhas, 5º conde de Avintes e 2º Marquês do Lavradio, entre 1769
  e 1778 --- o que corrobora a localização histórica dos eventos
  transcorridos na novela.}; e D.\,Evarista respondia, entusiasmada, que
era a coisa mais bela que podia haver no mundo. O Passeio Público estava
acabado\footnote{O Passeio Público do Rio de Janeiro foi construído
  entre 1789 e 1793, durante o vice-reinado de Luís de Vasconcelos e
  Sousa. Representou profunda alteração na paisagem, já que foi
  construído sobre o aterro da Lagoa do Boqueirão.}, um paraíso onde ela
fora muitas vezes, e a Rua das Belas Noites\footnote{Atualmente, Rua das
  Marrecas.}, o chafariz das Marrecas\ldots{} Ah! o chafariz das
Marrecas! Eram mesmo marrecas, --- feitas de metal e despejando água
pela boca fora. Uma coisa galantíssima\footnote{O Chafariz das Marrecas
  era a atração principal do Passeio Público. A escultura, a cargo do
  célebre Mestre Valentim, foi inaugurada em 1783. O gosto de Dona
  Evarista é duvidoso e revela seu deslumbramento não com a cidade, mas
  com o fato de ter visitado um local que estava em evidência.}. O
vigário dizia que sim, que o Rio de Janeiro\footnote{Como a novela se
  passa no final do século \textsc{xviii}, o narrador se refere constantemente a
  Luís de Vasconcelos e Sousa (1742--1809), 4º conde de Figueiró --- o
  décimo segundo vice-rei do Brasil, entre 1778 e 1790, e também
  governador do Rio de Janeiro, cidade-sede do reino. De acordo com o
  Francisco Adolfo de Varnhagen, Vasconcelos distinguia-se pela
  ``moderação'' e ``prudência''. Sob seu governo, construiu-se o Passeio
  Público e se instituiu a Sociedade Literária (mais tarde, extinta). No
  plano político, reprimiu violentamente a Conjuração Mineira (entre
  1789 e 1792), o que resultou no enforcamento do alferes Tiradentes,
  durante o vice-reinado posterior.} devia estar agora muito mais
bonito. Se já o era noutro tempo! Não admira, maior do que Itaguaí, e de
mais a mais sede do governo\ldots{} Mas não se pode dizer que Itaguaí
fosse feio; tinha belas casas, a casa do Mateus, a Casa Verde\ldots{}

--- A propósito de Casa Verde, disse o padre Lopes escorregando
habilmente para o assunto da ocasião, a senhora vem achá-la muito cheia
de gente.

--- Sim?

--- É verdade. Lá está o Mateus\ldots{}

--- O albardeiro?

--- O albardeiro; está o Costa, a prima do Costa, e Fulano, e Sicrano,
e\ldots{}

---Tudo isso doido?

--- Ou quase doido, obtemperou o padre.

--- Mas então?

O vigário derreou os cantos da boca, à maneira de quem não sabe nada ou
não quer dizer tudo; resposta vaga, que se não pode repetir a outra
pessoa por falta de texto. D.\,Evarista achou realmente extraordinário
que toda aquela gente ensandecesse; um ou outro, vá; mas todos?
Entretanto custava-lhe duvidar; o marido era um sábio, não recolheria
ninguém à Casa Verde sem prova evidente de loucura.

--- Sem dúvida\ldots{} sem dúvida\ldots{} ia pontuando o vigário.

Três horas depois, cerca de cinquenta convivas\footnote{Comensais.}
sentavam-se em volta da mesa de Simão Bacamarte; era o jantar das
boas-vindas. D.\,Evarista foi o assunto obrigado dos brindes, discursos,
versos de toda a casta, metáforas, amplificações, apólogos\footnote{Metáfora
  é a figura de linguagem em que um termo é substituído por outro. A
  amplificação relaciona-se ao desdobramento de um tema, retomado com
  maior vigor ao longo do discurso. Apólogo é narrativa ficcional de
  premissa moral.}. Ela era a esposa do novo Hipócrates\footnote{Hipócrates
  é considerado o pai da medicina.}, a musa da ciência, anjo, divina,
aurora, caridade, vida, consolação; trazia nos olhos duas estrelas,
segundo a versão modesta de Crispim Soares, e dois sóis, no conceito de
um vereador. O alienista ouvia essas coisas um tanto enfastiado, mas sem
visível impaciência. Quando muito dizia ao ouvido da mulher, que a
retórica\footnote{Arte da persuasão. Neste caso, a palavra é empregada
  de modo pejorativo pelo médico.} permitia tais arrojos sem
significação. D.\,Evarista fazia esforços para aderir a esta opinião do
marido; mas, ainda descontando três quartas partes das louvaminhas,
ficava muito com que enfunar-lhe\footnote{Inflar-se, envaidecer-se.} a
alma. Um dos oradores, por exemplo, Martim Brito, rapaz de vinte e cinco
anos, pintalegrete\footnote{Presunçoso, vaidoso, exibido.} acabado,
curtido de namoros e aventuras, declamou um discurso em que o nascimento
de D.\,Evarista era explicado pelo mais singular dos reptos\footnote{Desafio,
  enfrentamento.}. ``Deus, disse ele, depois de dar ao universo o homem
e a mulher, esse diamante e essa pérola da coroa divina (e o orador
arrastava triunfalmente esta frase de uma ponta a outra da mesa), Deus
quis vencer a Deus, e criou D.\,Evarista''.

D.\,Evarista baixou os olhos com exemplar modéstia. Duas senhoras,
achando a cortesanice excessiva e audaciosa, interrogaram os olhos do
dono da casa; e, na verdade, o gesto do alienista pareceu-lhes nublado
de suspeitas, de ameaças e, provavelmente, de sangue. O atrevimento foi
grande, pensaram as duas damas. E uma e outra pediam a Deus que
removesse qualquer episódio trágico, --- ou que o adiasse, ao menos para
o dia seguinte. Sim, que o adiasse. Uma delas, a mais piedosa, chegou a
admitir, consigo mesma, que D.\,Evarista não merecia nenhuma
desconfiança, tão longe estava de ser atraente ou bonita. Uma simples
água-morna. Verdade é que, se todos os gostos fossem iguais, o que seria
do amarelo? E esta ideia fê-la tremer outra vez, embora menos; menos,
porque o alienista sorria agora para o Martim Brito e, levantados todos,
foi ter com ele e falou-lhe do discurso. Não lhe negou que era um
improviso brilhante, cheio de rasgos magníficos. Seria dele mesmo a
ideia relativa ao nascimento de D.\,Evarista, ou tê-la-ia encontrado em
algum autor que?\ldots{} Não, senhor; era dele mesmo; achou-a naquela
ocasião e parecera-lhe adequada a um arroubo oratório. De resto, suas
ideias eram antes arrojadas do que ternas ou jocosas\footnote{Divertida,
  cômica.}. Dava para o épico\footnote{Gênero elevado, assim como a
  Tragédia --- segundo a \emph{Poética} de Aristóteles (384--322 a.C.).}.
Uma vez, por exemplo, compôs uma ode à queda do Marquês de
Pombal\footnote{Título concedido a Sebastião José de Carvalho e Melo,
  braço direito do rei Dom José \textsc{i}, entre 1750 e 1777.}, em que dizia que
esse ministro era o ``dragão aspérrimo\footnote{Asperíssimo, muito
  áspero.} do Nada'', esmagado pelas ``garras vingadoras do Todo''; e
assim outras, mais ou menos fora do comum; gostava das ideias sublimes e
raras, das imagens grandes e nobres\ldots{}

--- Pobre moço! pensou o alienista. E continuou consigo: --- Trata-se de
um caso de lesão cerebral; fenômeno sem gravidade, mas digno de
estudo\ldots{}

D.\,Evarista ficou estupefata quando soube, três dias depois, que o
Martim Brito fora alojado na Casa Verde. Um moço que tinha ideias tão
bonitas! As duas senhoras atribuíram o ato a ciúmes do alienista. Não
podia ser outra coisa; realmente, a declaração do moço fora audaciosa de
mais.

Ciúmes? Mas como explicar que, logo em seguida, fossem recolhidos José
Borges do Couto Leme, pessoa estimável, o Chico das Cambraias,
folgazão\footnote{Brincalhão, galhofeiro.} emérito, o escrivão Fabrício
e ainda outros? O terror acentuou-se. Não se sabia já quem estava são,
nem quem estava doido. As mulheres, quando os maridos saíam, mandavam
acender uma lamparina a Nossa Senhora; e nem todos os maridos eram
valorosos; alguns não andavam fora sem um ou dois capangas.
Positivamente o terror. Quem podia, emigrava. Um desses fugitivos chegou
a ser preso a duzentos passos da vila. Era um rapaz de trinta anos,
amável, conversado, polido, tão polido que não cumprimentava alguém sem
levar o chapéu ao chão; na rua, acontecia-lhe correr uma distância de
dez a vinte braças\footnote{Antiga medida correspondente a dois braços
  abertos (aproximadamente 2 metros).} para ir apertar a mão a um homem
grave\footnote{Sisudo, comportado, digno.}, a uma senhora, às vezes a um
menino, como acontecera ao filho do juiz de fora. Tinha a vocação das
cortesias. De resto, devia as boas relações da sociedade, não só aos
dotes pessoais, que eram raros, como à nobre tenacidade com que nunca
desanimava diante de uma, duas, quatro, seis recusas, caras feias, etc.
O que acontecia era que, uma vez entrado numa casa, não a deixava mais,
nem os da casa o deixavam a ele, tão gracioso era o Gil Bernardes. Pois
o Gil Bernardes, apesar de se saber estimado, teve medo quando lhe
disseram um dia, que o alienista o trazia de olho; na madrugada seguinte
fugiu da vila, mas foi logo apanhado e conduzido à Casa Verde.

--- Devemos acabar com isto!

--- Não pode continuar!

--- Abaixo a tirania!

--- Déspota! violento! Golias!\footnote{Gigante violento que teria sido
  derrotado pelo rei Davi, em acordo com o livro de Samuel, na
  \emph{Bíblia}.}

Não eram gritos na rua, eram suspiros em casa, mas não tardava a hora
dos gritos. O terror crescia; avizinhava-se a rebelião. A ideia de uma
petição ao governo para que Simão Bacamarte fosse capturado e deportado,
andou por algumas cabeças, antes que o barbeiro Porfírio a expendesse na
loja, com grandes gestos de indignação. Note-se, --- e essa é uma das
laudas mais puras desta sombria história --- note-se que o Porfírio,
desde que a Casa Verde começara a povoar-se tão extraordinariamente, viu
crescerem-lhe os lucros pela aplicação assídua de sanguessugas que dali
lhe pediam; mas o interesse particular, dizia ele, deve ceder ao
interesse público. E acrescentava: ---é preciso derrubar o tirano!
Note-se mais que ele soltou esse grito justamente no dia em que Simão
Bacamarte fizera recolher à Casa Verde um homem que trazia com ele uma
demanda, o Coelho.

--- Não me dirão em que é que o Coelho é doido? bradou o Porfírio.

E ninguém lhe respondia; todos repetiam que era um homem perfeitamente
ajuizado. A mesma demanda\footnote{Pendência processual.} que ele trazia
com o barbeiro, acerca de uns chãos da vila, era filha da obscuridade de
um alvará e não da cobiça ou ódio. Um excelente caráter o Coelho. Os
únicos desafeiçoados que tinha eram alguns sujeitos que, dizendo-se
taciturnos\footnote{Melancólico, silencioso.}, ou alegando andar com
pressa, mal o viam de longe dobravam as esquinas, entravam nas lojas,
etc. Na verdade, ele amava a boa palestra, a palestra comprida, gostada
a sorvos\footnote{Goles. Figurativamente, palestra largamente
  aproveitada, absorvida.} largos, e assim é que nunca estava só,
preferindo os que sabiam dizer duas palavras, mas não desdenhando os
outros. O padre Lopes, que cultivava o Dante, e era inimigo do Coelho,
nunca o via desligar-se de uma pessoa que não declamasse e emendasse
este trecho:

\begin{quote}
La bocca sollevò dal fiero pasto

Quel \emph{``seccatore''\ldots{}}\footnote{Versos que aparecem no
  33\textsuperscript{o} capítulo do ``Inferno'' de Dante Alighieri. O
  termo ``seccatore'', em itálico no original, mostra que a substituição
  de ``peccator'' era intencional. (A tradução dos versos originais
  seria: ``A boca suspendeu do fero alimento / Aquele pecador'').}
\end{quote}

mas uns sabiam do ódio do padre, e outros pensavam que isto era uma
oração em latim.

\section*{\textsc{vi}. A Rebelião}

Cerca de trinta pessoas ligaram-se ao barbeiro, redigiram e levaram uma
representação\footnote{Documento assinado coletivamente em que um
  indivíduo representa os demais signatários.} à câmara. A câmara
recusou aceitá-la, declarando que a Casa Verde era uma instituição
pública, e que a ciência não podia ser emendada por votação
administrativa, menos ainda por movimentos de rua.

--- Voltai ao trabalho, concluiu o presidente, é o conselho que vos
damos.

A irritação dos agitadores foi enorme. O barbeiro declarou que iam dali
levantar a bandeira da rebelião, e destruir a Casa Verde; que Itaguaí
não podia continuar a servir de cadáver aos estudos e experiências de um
déspota; que muitas pessoas estimáveis, algumas distintas, outras
humildes, mas dignas de apreço, jaziam nos cubículos da Casa Verde; que
o despotismo científico do alienista complicava-se do espírito de
ganância, visto que os loucos, ou supostos tais, não eram tratados de
graça: as famílias, e em falta delas a câmara, pagavam ao
alienista\ldots{}

--- É falso, interrompeu o presidente.

--- Falso?

--- Há cerca de duas semanas recebemos um ofício do ilustre médico, em
que nos declara que, tratando de fazer experiências de alto valor
psicológico, desiste do estipêndio votado pela câmara, bem como nada
receberá das famílias dos enfermos.

A notícia deste ato tão nobre, tão puro, suspendeu um pouco a alma dos
rebeldes. Seguramente o alienista podia estar em erro, mas nenhum
interesse alheio à ciência o instigava; e para demonstrar o erro era
preciso alguma coisa mais do que arruaças e clamores. Isto disse o
presidente, com aplauso de toda a câmara. O barbeiro, depois de alguns
instantes de concentração, declarou que estava investido de um mandato
público e não restituiria a paz a Itaguaí antes de ver por terra a Casa
Verde, --- ``essa Bastilha da razão humana'', --- expressão que ouvira a
um poeta local e que ele repetiu com muita ênfase. Disse, e a um sinal
todos saíram com ele.

Imagine-se a situação dos vereadores; urgia obstar ao ajuntamento, à
rebelião, à luta, ao sangue. Para acrescentar ao mal, um dos vereadores,
que apoiara o presidente, ouvindo agora a denominação dada pelo barbeiro
à Casa Verde --- ``Bastilha da razão humana'', --- achou-a tão elegante,
que mudou de parecer. Disse que entendia de bom aviso decretar alguma
medida que reduzisse a Casa Verde; e porque o presidente, indignado,
manifestasse em termos enérgicos o seu pasmo, o vereador fez esta
reflexão:

--- Nada tenho que ver com a ciência; mas, se tantos homens em quem
supomos juízo são reclusos por dementes, quem nos afirma que o alienado
não é o alienista?

Sebastião Freitas, o vereador dissidente, tinha o dom da palavra e falou
ainda por algum tempo com prudência\footnote{A prudência era considerada
  uma arte, pelo menos até o final do século \textsc{xviii}, período em que se
  passa a narrativa.}, mas com firmeza. Os colegas estavam atônitos; o
presidente pediu-lhe que, ao menos, desse o exemplo da ordem e do
respeito à lei, não aventasse as suas ideias na rua, para não dar corpo
e alma à rebelião, que era por ora um turbilhão de átomos dispersos.
Esta figura corrigiu um pouco o efeito da outra: Sebastião Freitas
prometeu suspender qualquer ação reservando-se o direito de pedir pelos
meios legais a redução da Casa Verde. E repetia consigo, namorado: ---
Bastilha da razão humana!

Entretanto, a arruaça crescia. Já não eram trinta, mas trezentas pessoas
que acompanhavam o barbeiro, cuja alcunha familiar deve ser mencionada,
porque ela deu o nome à revolta; chamavam-lhe o Canjica, --- e o
movimento ficou célebre com o nome de revolta dos Canjicas\footnote{Possível
  paródia da Conjuração Mineira.}. A ação podia ser restrita, --- visto
que muita gente, ou por medo, ou por hábitos de educação, não descia à
rua; mas o sentimento era unânime, ou quase unânime, e os trezentos que
caminhavam para a Casa Verde, --- dada a diferença de Paris a Itaguaí,
--- podiam ser comparados aos que tomaram a Bastilha\footnote{Nova
  alusão aos eventos relacionados à Revolução Francesa --- contemporânea
  da Conjuração Mineira ---, no final do século \textsc{xviii}.}.

D.\,Evarista teve notícia da rebelião antes que ela chegasse; veio
dar-lha\footnote{Contração de lhe + a (Veio lhe dar a notícia).} uma de
suas crias\footnote{Criadas.}. Ela provava nessa ocasião um vestido de
seda, --- um dos trinta e sete que trouxera do Rio de Janeiro, --- e não
quis crer.

--- Há de ser alguma patuscada, dizia ela mudando a posição de um
alfinete. Benedita, vê se a barra está boa.

--- Está, sinhá, respondia a mucama de cócoras no chão, está boa. Sinhá
vira um bocadinho. Assim. Está muito boa.

--- Não é patuscada, não, senhora; eles estão gritando: --- Morra o
Dr.\,Bacamarte! o tirano! dizia o moleque assustado.

--- Cala a boca, tolo! Benedita, olha aí do lado esquerdo; não parece
que a costura está um pouco enviesada? A risca azul não segue até
abaixo; está muito feio assim; é preciso descoser para ficar igualzinho
e\ldots{}

--- Morra o Dr.\,Bacamarte! morra o tirano! uivaram fora trezentas vozes.
Era a rebelião que desembocava na Rua Nova.

D.\,Evarista ficou sem pinga de sangue\footnote{Exangue, pálida, sem cor.}.
No primeiro instante não deu um passo, não fez um gesto; o terror
petrificou-a. A mucama correu instintivamente para a porta do fundo.
Quanto ao moleque, a quem D.\,Evarista não dera crédito, teve um instante
de triunfo súbito, um certo movimento súbito, imperceptível, entranhado,
de satisfação moral, ao ver que a realidade vinha jurar por ele.

--- Morra o alienista! bradavam as vozes mais perto.

D.\,Evarista, se não resistia facilmente às comoções de prazer, sabia
entestar\footnote{Confrontar.} com os momentos de perigo. Não desmaiou;
correu à sala interior onde o marido estudava. Quando ela ali entrou,
precipitada, o ilustre médico escrutava um texto de Averróis;\footnote{Averróis
  (1126--1198), filósofo e médico árabe que traduziu e comentou
  Aristóteles.} os olhos dele, empanados pela cogitação, subiam do livro
ao teto e baixavam do teto ao livro, cegos para a realidade exterior,
videntes para os profundos trabalhos mentais. D.\,Evarista chamou pelo
marido duas vezes, sem que ele lhe desse atenção; à terceira, ouviu e
perguntou-lhe o que tinha, se estava doente.

--- Você não ouve estes gritos? perguntou a digna esposa em lágrimas.

O alienista atendeu então; os gritos aproximavam-se, terríveis,
ameaçadores; ele compreendeu tudo. Levantou-se da cadeira de espaldar em
que estava sentado, fechou o livro, e, a passo firme e tranquilo, foi
depositá-lo na estante. Como a introdução do volume desconsertasse um
pouco a linha dos dois tomos contíguos, Simão Bacamarte cuidou de
corrigir esse defeito mínimo, e, aliás, interessante. Depois disse à
mulher que se recolhesse, que não fizesse nada.

--- Não, não, implorava a digna senhora, quero morrer ao lado de você\ldots{}

Simão Bacamarte teimou que não, que não era caso de morte; e ainda que o
fosse, intimava-lhe em nome da vida que ficasse. A infeliz dama curvou a
cabeça, obediente e chorosa.

--- Abaixo a Casa Verde! bradavam os Canjicas.

O alienista caminhou para a varanda da frente, e chegou ali no momento
em que a rebelião também chegava e parava, defronte, com as suas
trezentas cabeças rutilantes de civismo e sombrias de desespero. ---
Morra! morra! bradaram de todos os lados, apenas o vulto do alienista
assomou na varanda. Simão Bacamarte fez um sinal pedindo para falar; os
revoltosos cobriram-lhe a voz com brados de indignação. Então, o
barbeiro, agitando o chapéu, a fim de impor silêncio à turba, conseguiu
aquietar os amigos, e declarou ao alienista que podia falar, mas
acrescentou que não abusasse da paciência do povo como fizera até então.

--- Direi pouco, ou até não direi nada, se for preciso. Desejo saber
primeiro o que pedis.

--- Não pedimos nada, replicou fremente o barbeiro; ordenamos que a Casa
Verde seja demolida, ou pelo menos despojada dos infelizes que lá estão.

--- Não entendo.

--- Entendeis bem, tirano; queremos dar liberdade às vítimas do vosso
ódio, capricho, ganância\ldots{}

O alienista sorriu, mas o sorriso desse grande homem não era coisa
visível aos olhos da multidão; era uma contração leve de dois ou três
músculos, nada mais. Sorriu e respondeu:

--- Meus senhores, a ciência é coisa séria, e merece ser tratada com
seriedade. Não dou razão dos meus atos de alienista a ninguém, salvo aos
mestres e a Deus. Se quereis emendar a administração da Casa Verde,
estou pronto a ouvir-vos; mas se exigis que me negue a mim mesmo, não
ganhareis nada. Poderia convidar alguns de vós, em comissão dos outros,
a vir ver comigo os loucos reclusos; mas não o faço, porque seria
dar-vos razão do meu sistema, o que não farei a leigos nem a rebeldes.

Disse isto o alienista, e a multidão ficou atônita; era claro que não
esperava tanta energia e menos ainda tamanha serenidade. Mas o assombro
cresceu de ponto quando o alienista, cortejando a multidão com muita
gravidade, deu-lhe as costas e retirou-se lentamente para dentro. O
barbeiro tornou logo a si e, agitando o chapéu, convidou os amigos à
demolição da Casa Verde; poucas vozes e frouxas lhe responderam. Foi
nesse momento decisivo que o barbeiro sentiu despontar em si a ambição
do governo; pareceu-lhe então que, demolindo a Casa Verde e
derrocando\footnote{Demolir, aniquilar.} a influência do alienista,
chegaria a apoderar-se da câmara, dominar as demais autoridades e
constituir-se senhor de Itaguaí. Desde alguns anos que ele forcejava por
ver o seu nome incluído nos pelouros\footnote{Urna circular utilizada em
  votações da câmara.} para o sorteio dos vereadores, mas era recusado
por não ter uma posição compatível com tão grande cargo. A ocasião era
agora ou nunca. Demais fora tão longe na arruaça que a derrota seria a
prisão, ou talvez a forca, ou o degredo\footnote{Prisão, força e degredo
  são penas coincidentes com aquelas aplicadas aos revoltosos de Vila
  Rica, durante a chamada Conjuração Mineira.}. Infelizmente, a resposta
do alienista diminuíra o furor dos sequazes\footnote{Partidário, adepto.}.
O barbeiro, logo que o percebeu, sentiu um impulso de indignação, e quis
bradar-lhes: --- Canalhas! covardes! --- mas conteve-se e rompeu deste
modo:

--- Meus amigos, lutemos até o fim! A salvação de Itaguaí está nas
vossas mãos dignas e heroicas. Destruamos o cárcere de vossos filhos e
pais, de vossas mães e irmãs, de vossos parentes e amigos, e de vós
mesmos. Ou morrereis a pão e água, talvez a chicote, na masmorra daquele
indigno.

A multidão agitou-se, murmurou, bradou, ameaçou, congregou-se toda em
derredor do barbeiro. Era a revolta que tornava a si da ligeira síncope
e ameaçava arrasar a Casa Verde.

--- Vamos! bradou Porfírio agitando o chapéu.

--- Vamos! repetiram todos.

Deteve-os um incidente: era um corpo de dragões\footnote{Infantaria
  montada de prestígio que atuou durante os séculos \textsc{xviii} e \textsc{xix}.} que, a
marche-marche, entrava na Rua Nova.

\section*{\textsc{vii}. O Inesperado}

Chegados os dragões em frente aos Canjicas, houve um instante de
estupefação; os Canjicas não queriam crer que a força pública fosse
mandada contra eles; mas o barbeiro compreendeu tudo e esperou. Os
dragões pararam, o capitão intimou à multidão que se dispersasse; mas,
conquanto uma parte dela estivesse inclinada a isso, a outra parte
apoiou fortemente o barbeiro, cuja resposta consistiu nestes termos
alevantados:

--- Não nos dispersaremos. Se quereis os nossos cadáveres, podeis
tomá-los; mas só os cadáveres; não levareis a nossa honra, o nosso
crédito, os nossos direitos, e com eles a salvação de Itaguaí.

Nada mais imprudente do que essa resposta do barbeiro; e nada mais
natural. Era a vertigem das grandes crises. Talvez fosse também um
excesso de confiança na abstenção das armas por parte dos dragões;
confiança que o capitão dissipou logo, mandando carregar sobre os
Canjicas. O momento foi indescritível. A multidão urrou furiosa; alguns,
trepando às janelas das casas ou correndo pela rua fora, conseguiram
escapar; mas a maioria ficou, bufando de cólera, indignada, animada pela
exortação do barbeiro. A derrota dos Canjicas estava iminente, quando um
terço dos dragões, --- qualquer que fosse o motivo, as crônicas não o
declaram, --- passou subitamente para o lado da rebelião. Este
inesperado reforço deu alma aos Canjicas, ao mesmo tempo que lançou o
desânimo às fileiras da legalidade. Os soldados fiéis não tiveram
coragem de atacar os seus próprios camaradas\footnote{Possível alusão de
  Machado à forma de tratamento com que os adeptos do Socialismo se
  referiam aos companheiros de trabalho, a partir da Revolução de 1848.
  O termo aparece no \emph{Manifesto Comunista}, de Marx e Engels.}, e,
um a um, foram passando para eles, de modo que ao cabo de alguns
minutos, o aspecto das coisas era totalmente outro. O capitão estava de
um lado, com alguma gente, contra uma massa compacta que o ameaçava de
morte. Não teve remédio, declarou-se vencido e entregou a espada ao
barbeiro.

A revolução triunfante não perdeu um só minuto; recolheu os feridos às
casas próximas, e guiou para a câmara. Povo e tropa fraternizavam, davam
vivas a el-rei, ao vice-rei, a Itaguaí, ao ``ilustre Porfírio''. Este ia
na frente, empunhando tão destramente a espada, como se ela fosse apenas
uma navalha um pouco mais comprida. A vitória cingia-lhe a fronte de um
nimbo\footnote{Auréola.} misterioso. A dignidade de governo começava a
enrijar-lhe os quadris.

Os vereadores, às janelas, vendo a multidão e a tropa, cuidaram que a
tropa capturara a multidão, e sem mais exame, entraram e votaram uma
petição ao vice-rei\footnote{Nova menção ao vice-rei Luís de Vasconcelos
  e Sousa.} para que mandasse dar um mês de soldo\footnote{Remuneração
  paga ao soldado.} aos dragões, ``cujo denodo\footnote{Ousadia,
  empenho, coragem.} salvou Itaguaí do abismo a que o tinha lançado uma
cáfila\footnote{Coletivo de camelos.} de rebeldes''. Esta frase foi
proposta por Sebastião Freitas, o vereador dissidente, cuja defesa dos
Canjicas tanto escandalizara os colegas. Mas bem depressa a ilusão se
desfez. Os vivas ao barbeiro, os morras aos vereadores e ao alienista
vieram dar-lhes notícia da triste realidade. O presidente não desanimou:
--- qualquer que seja a nossa sorte, disse ele, lembremo-nos de que
estamos ao serviço de Sua Majestade e do povo. --- Sebastião insinuou
que melhor se poderia servir à coroa e à vila saindo pelos fundos e indo
conferenciar com o juiz de fora, mas toda a câmara rejeitou esse
alvitre\footnote{Arbítrio.}.

Daí a nada o barbeiro, acompanhado de alguns de seus tenentes, entrava
na sala da vereança, e intimava à câmara a sua queda. A câmara não
resistiu, entregou-se, e foi dali para a cadeia. Então os amigos do
barbeiro propuseram-lhe que assumisse o governo da vila, em nome de Sua
Majestade. Porfírio aceitou o encargo, embora não desconhecesse
(acrescentou) os espinhos que trazia; disse mais que não podia dispensar
o concurso dos amigos presentes; ao que eles prontamente anuíram. O
barbeiro veio à janela e comunicou ao povo essas resoluções, que o povo
ratificou, aclamando o barbeiro. Este tomou a denominação de ---
``Protetor da vila em nome de Sua Majestade e do povo''\footnote{Machado
  de Assis parodia o título conferido ao príncipe Dom Pedro \textsc{i}, em maio
  de 1822: ``Protetor e Defensor Perpétuo do Brasil''.}. ---
Expediram-se logo várias ordens importantes, comunicações oficiais do
novo governo, uma exposição minuciosa ao vice-rei, com muitos protestos
de obediência às ordens de Sua Majestade; finalmente, uma proclamação ao
povo, curta, mas enérgica:

\begin{quote}
``Itaguaienses!

Uma câmara corrupta e violenta conspirava contra os interesses de Sua
Majestade e do povo. A opinião pública tinha-a condenado; um punhado de
cidadãos, fortemente apoiados pelos bravos dragões de Sua Majestade,
acaba de a dissolver ignominiosamente\footnote{Vergonhosamente,
  desonradamente.}, e por unânime consenso da vila, foi-me confiado o
mando supremo, até que Sua Majestade se sirva ordenar o que parecer
melhor ao seu real serviço. Itaguaienses! não vos peço se não que me
rodeeis de confiança, que me auxilieis em restaurar a paz e a fazenda
pública, tão desbaratada pela câmara que ora findou às vossas mãos.
Contai com o meu sacrifício, e ficai certos de que a coroa será por nós.

O Protetor da vila em nome de Sua Majestade e do povo
\end{quote}

\textsc{porfírio caetano das neves}''.

Toda a gente advertiu no absoluto silêncio desta proclamação acerca da
Casa Verde; e, segundo uns, não podia haver mais vivo indício dos
projetos tenebrosos do barbeiro. O perigo era tanto maior quanto que, no
meio mesmo desses graves sucessos, o alienista metera na Casa Verde umas
sete ou oito pessoas, entre elas duas senhoras, sendo um dos homens
aparentado com o Protetor. Não era um repto, um ato intencional; mas
todos o interpretaram dessa maneira, e a vila respirou com a esperança
de que o alienista dentro de vinte e quatro horas estaria a ferros, e
destruído o terrível cárcere.

O dia acabou alegremente. Enquanto o arauto\footnote{Mensageiro,
  porta-voz.} da matraca ia recitando de esquina em esquina a
proclamação, o povo espalhava-se nas ruas e jurava morrer em defesa do
ilustre Porfírio. Poucos gritos contra a Casa Verde, prova de confiança
na ação do governo. O barbeiro faz expedir um ato declarando feriado
aquele dia, e entabulou negociações com o vigário para a celebração de
um \emph{Te-Deum}, tão conveniente era aos olhos dele a conjunção do
poder temporal com o espiritual; mas o padre Lopes recusou abertamente o
seu concurso.

--- Em todo caso, Vossa Reverendíssima não se alistará entre os inimigos
do governo? disse-lhe o barbeiro, dando à fisionomia um aspecto
tenebroso.

Ao que o padre Lopes respondeu, sem responder:

--- Como alistar-me, se o novo governo não tem inimigos?

O barbeiro sorriu; era a pura verdade. Salvo o capitão, os vereadores e
os principais\footnote{Sujeito pertencente à elite local. Denominação
  que remonta à antiga ``homem-bom'', sujeito de posses e que exercia
  poder político e mando.} da vila, toda a gente o aclamava. Os mesmos
principais, se o não aclamavam, não tinham saído contra ele. Nenhum dos
almotacés\footnote{Inspetor de pesos e medidas, que fixava o valor dos
  alimentos.} deixou de vir receber as suas ordens. No geral, as
famílias abençoavam o nome daquele que ia enfim libertar Itaguaí da Casa
Verde e do terrível Simão Bacamarte.

\section*{\textsc{viii}. As angústias do boticário}

Vinte e quatro horas depois dos sucessos narrados no capítulo anterior,
o barbeiro saiu do palácio do governo, --- foi a denominação dada à casa
da câmara, --- com dois ajudantes de ordens, e dirigiu-se à residência
de Simão Bacamarte. Não ignorava ele que era mais decoroso ao governo
mandá-lo chamar; o receio, porém, de que o alienista não obedecesse,
obrigou-o a parecer tolerante e moderado.

Não descrevo o terror do boticário ao ouvir dizer que o barbeiro ia à
casa do alienista. --- Vai prendê-lo, pensou ele. E redobraram-lhe as
angústias. Com efeito, a tortura moral do boticário naqueles dias de
revolução excede a toda a descrição possível. Nunca um homem se achou em
mais apertado lance: --- a privança do alienista chamava-o ao lado
deste, a vitória do barbeiro atraía-o ao barbeiro. Já a simples notícia
da sublevação\footnote{Revolta, levante, conjuração.} tinha-lhe sacudido
fortemente a alma, porque ele sabia a unanimidade do ódio ao alienista;
mas a vitória final foi também o golpe final. A esposa, senhora máscula,
amiga particular de D.\,Evarista, dizia que o lugar dele era ao lado de
Simão Bacamarte; ao passo que o coração lhe bradava que não, que a causa
do alienista estava perdida, e que ninguém, por ato próprio, se amarra a
um cadáver. Fê-lo Catão\footnote{Marco Pórcio Catão (234--149 a.C.), o
  Velho, que se tornou célebre como censor romano. A denominação é
  irônica, já que Hipócrates era grego e a cultura helenística
  constituía um dos alvos do romano Catão. A alcunha sugere que
  Bacamarte corresponderia a uma espécie de sábio e censor.}, é verdade,
\emph{sed victa Catoni}\footnote{Expressão latina extraída de um verso
  de Lucano (39--65 d.C): ``\emph{victrix causa diis placuit,} sed victa
  Catoni'' (A causa vencedora agradou aos deuses, mas a vencida a Catão)
  que integra o poema épico \emph{Farsália}, publicado séculos após a
  sua morte.}, pensava ele, relembrando algumas palestras habituais do
padre Lopes; mas Catão não se atou a uma causa vencida, ele era a
própria causa vencida, a causa da república; o seu ato, portanto, foi de
egoísta, de um miserável egoísta; minha situação é outra. Insistindo,
porém, a mulher, não achou Crispim Soares outra saída em tal crise senão
adoecer; declarou-se doente, e meteu-se na cama.

--- Lá vai o Porfírio à casa do Dr.\,Bacamarte, disse-lhe a mulher no dia
seguinte à cabeceira da cama; vai acompanhado de gente.

--- Vai prendê-lo, pensou o boticário.

Uma ideia traz outra; o boticário imaginou que, uma vez preso o
alienista, viriam também buscá-lo a ele, na qualidade de cúmplice. Esta
ideia foi o melhor dos vesicatórios\footnote{Que provoca bolhas,
  vesicante.}. Crispim Soares ergueu-se, disse que estava bom, que ia
sair; e apesar de todos os esforços e protestos da consorte, vestiu-se e
saiu. Os velhos cronistas são unânimes em dizer que a certeza de que o
marido ia colocar-se nobremente ao lado do alienista consolou
grandemente a esposa do boticário; e notam, com muita perspicácia, o
imenso poder moral de uma ilusão; porquanto, o boticário caminhou
resolutamente ao palácio do governo, não à casa do alienista. Ali
chegando, mostrou-se admirado de não ver o barbeiro, a quem ia
apresentar os seus protestos de adesão, não o tendo feito desde a
véspera por enfermo. E tossia com algum custo. Os altos funcionários que
lhe ouviam esta declaração, sabedores da intimidade do boticário com o
alienista, compreenderam toda a importância da adesão nova e trataram a
Crispim Soares com apurado carinho; afirmaram-lhe que o barbeiro não
tardava; Sua Senhoria tinha ido à Casa Verde, a negócio importante, mas
não tardava. Deram-lhe cadeira, refrescos, elogios; disseram-lhe que a
causa do ilustre Porfírio era a de todos os patriotas; ao que o
boticário ia repetindo que sim, que nunca pensara outra coisa, que isso
mesmo mandaria declarar Sua Majestade.

\section*{\textsc{ix}. Dois lindos casos }

Não se demorou o alienista em receber o barbeiro; declarou-lhe que não
tinha meios de resistir, e portanto estava prestes a obedecer. Só uma
coisa pedia, é que o não constrangesse a assistir pessoalmente à
destruição da Casa Verde.

--- Engana-se Vossa Senhoria, disse o barbeiro depois de alguma pausa,
engana-se em atribuir ao governo intenções vandálicas. Com razão ou sem
ela, a opinião crê que a maior parte dos doidos ali metidos estão em seu
perfeito juízo, mas o governo reconhece que a questão é puramente
científica, e não cogita em resolver com posturas\footnote{Ordem emitida
  pela câmara municipal.} as questões científicas. Demais\footnote{Além
  disso, Ademais.}, a Casa Verde é uma instituição pública; tal a
aceitamos das mãos da câmara dissolvida. Há, entretanto, --- por força
que há de haver um alvitre intermédio que restitua o sossego ao espírito
público.

O alienista mal podia dissimular o assombro; confessou que esperava
outra coisa, o arrasamento do hospício, a prisão dele, o desterro, tudo,
menos\ldots{}

--- O pasmo de Vossa Senhoria, atalhou gravemente o barbeiro, vem de não
atender à grave responsabilidade do governo. O povo, tomado de uma cega
piedade que lhe dá em tal caso legítima indignação, pode exigir do
governo certa ordem de atos; mas este, com a responsabilidade que lhe
incumbe, não os deve praticar, ao menos integralmente, e tal é a nossa
situação. A generosa revolução que ontem derrubou uma câmara
vilipendiada e corrupta, pediu em altos brados o arrasamento da Casa
Verde; mas pode entrar no ânimo do governo eliminar a loucura? Não. E se
o governo não a pode eliminar, está ao menos apto para discriminá-la,
reconhecê-la? Também não; é matéria de ciência. Logo, em assunto tão
melindroso, o governo não pode, não deve, não quer dispensar o concurso
de Vossa Senhoria. O que lhe pede é que de certa maneira demos alguma
satisfação ao povo. Unamo-nos, e o povo saberá obedecer. Um dos alvitres
aceitáveis, se Vossa Senhoria não indicar outro, seria fazer retirar da
Casa Verde aqueles enfermos que estiverem quase curados, e bem assim os
maníacos de pouca monta, etc. Desse modo, sem grande perigo, mostraremos
alguma tolerância e benignidade.

--- Quantos mortos e feridos houve ontem no conflito? perguntou Simão
Bacamarte, depois de uns três minutos.

O barbeiro ficou espantado da pergunta, mas respondeu logo que onze
mortos e vinte e cinco feridos.

--- Onze mortos e vinte e cinco feridos! repetiu duas ou três vezes o
alienista.

E em seguida declarou que o alvitre lhe não parecia bom, mas que ele ia
catar algum outro, e dentro de poucos dias lhe daria resposta. E fez-lhe
várias perguntas acerca dos sucessos da véspera; ataque, defesa, adesão
dos dragões, resistência da câmara, etc., ao que o barbeiro ia
respondendo com grande abundância, insistindo principalmente no
descrédito em que a câmara caíra. O barbeiro confessou que o novo
governo não tinha ainda por si a confiança dos principais da vila, mas o
alienista podia fazer muito nesse ponto. O governo, concluiu o barbeiro,
folgaria se pudesse contar, não já com a simpatia senão com a
benevolência do mais alto espírito de Itaguaí, e seguramente do reino.
Mas nada disso alterava a nobre e austera fisionomia daquele grande
homem, que ouvia calado, sem desvanecimento nem modéstia, mas impassível
como um deus de pedra.

--- Onze mortos e vinte e cinco feridos, repetiu o alienista, depois de
acompanhar o barbeiro até a porta. Eis aí dois lindos casos de doença
cerebral. Os sintomas de duplicidade e descaramento deste barbeiro são
positivos. Quanto à toleima dos que o aclamaram não é preciso outra
prova além dos onze mortos e vinte e cinco feridos --- dois lindos
casos!

--- Viva o ilustre Porfírio! bradaram umas trinta pessoas que aguardavam
o barbeiro à porta.

O alienista espiou pela janela e ainda ouviu este resto de uma pequena
fala do barbeiro às trinta pessoas que o aclamavam.

--- \ldots{} porque eu velo, podeis estar certos disso, eu velo pela
execução das vontades do povo. Confiai em mim; e tudo se fará pela
melhor maneira. Só vos recomendo ordem. E ordem, meus amigos, é a base
do governo\ldots{}

--- Viva o ilustre Porfírio! bradaram as trinta vozes, agitando os
chapéus.

--- Dois lindos casos! murmurou o alienista.

\section*{\textsc{x.} A Restauração\protect\footnote[*]{\MakeUppercase{N}ova alusão aos episódios transcorridos durante a \MakeUppercase{R}evolução \MakeUppercase{F}rancesa.}}

Dentro de cinco dias, o alienista meteu na Casa Verde cerca de cinquenta
aclamadores do novo governo. O povo indignou-se. O governo, atarantado,
não sabia reagir. João Pina, outro barbeiro, dizia abertamente nas ruas,
que o Porfírio estava ``vendido ao ouro de Simão Bacamarte'', frase que
congregou em torno de João Pina a gente mais resoluta da vila. Porfírio,
vendo o antigo rival da navalha à testa da insurreição, compreendeu que
a sua perda era irremediável, se não desse um grande golpe; expediu dois
decretos, um abolindo a Casa Verde, outro desterrando o alienista. João
Pina mostrou claramente, com grandes frases, que o ato de Porfírio era
um simples aparato, um engodo, em que o povo não devia crer. Duas horas
depois caía Porfírio ignominiosamente, e João Pina assumia a difícil
tarefa do governo. Como achasse nas gavetas as minutas da proclamação,
da exposição ao vice-rei e de outros atos inaugurais do governo
anterior, deu-se pressa em os fazer copiar e expedir; acrescentam os
cronistas, e aliás subentende-se, que ele lhes mudou os nomes, e onde o
outro barbeiro falara de uma câmara corrupta, falou este de ``um intruso
eivado das más doutrinas francesas e contrário aos sacrossantos
interesses de Sua Majestade, etc''. Nisto entrou na vila uma força
mandada pelo vice-rei, e restabeleceu a ordem. O alienista exigiu desde
logo a entrega do barbeiro Porfírio, e bem assim a de uns cinquenta e
tantos indivíduos, que declarou mentecaptos; e não só lhe deram esses,
como afiançaram entregar-lhe mais dezenove sequazes do barbeiro, que
convalesciam das feridas apanhadas na primeira rebelião.

Este ponto da crise de Itaguaí marca também o grau máximo da influência
de Simão Bacamarte. Tudo quanto quis, deu-se-lhe; e uma das mais vivas
provas do poder do ilustre médico achamo-la na prontidão com que os
vereadores, restituídos a seus lugares, consentiram em que Sebastião
Freitas também fosse recolhido ao hospício. O alienista, sabendo da
extraordinária inconsistência das opiniões desse vereador, entendeu que
era um caso patológico, e pediu-o. A mesma coisa aconteceu ao boticário.
O alienista, desde que lhe falaram da momentânea adesão de Crispim
Soares à rebelião dos Canjicas, comparou-a à aprovação que sempre
recebera dele, ainda na véspera, e mandou capturá-lo. Crispim Soares não
negou o fato, mas explicou-o dizendo que cedera a um movimento de
terror, ao ver a rebelião triunfante, e deu como prova a ausência de
nenhum outro ato seu, acrescentando que voltara logo à cama, doente.
Simão Bacamarte não o contrariou; disse, porém, aos circunstantes que o
terror também é pai da loucura, e que o caso de Crispim Soares lhe
parecia dos mais caracterizados.

Mas a prova mais evidente da influência de Simão Bacamarte foi a
docilidade com que a câmara lhe entregou o próprio presidente. Este
digno magistrado tinha declarado em plena sessão, que não se contentava,
para lavá-lo da afronta dos Canjicas, com menos de trinta
almudes\footnote{Medida que correspondia a aproximadamente 32 litros.}
de sangue; palavra que chegou aos ouvidos do alienista por boca do
secretário da câmara, entusiasmado de tamanha energia. Simão Bacamarte
começou por meter o secretário na Casa Verde, e foi dali à câmara, à
qual declarou que o presidente estava padecendo da ``demência dos
touros'', um gênero que ele pretendia estudar, com grande vantagem para
os povos. A Câmara a princípio hesitou, mas acabou cedendo.

Daí em diante foi uma coleta desenfreada. Um homem não podia dar
nascença ou curso à mais simples mentira do mundo, ainda daquelas que
aproveitam ao inventor ou divulgador, que não fosse logo metido na Casa
Verde. Tudo era loucura. Os cultores de enigmas, os fabricantes de
charadas, de anagramas, os maldizentes, os curiosos da vida alheia, os
que põem todo o seu cuidado na tafularia\footnote{Casquilharia. Ação
  típica de um janota, almofadinha.}, um ou outro almotacé enfunado,
ninguém escapava aos emissários do alienista. Ele respeitava as
namoradas e não poupava as namoradeiras, dizendo que as primeiras cediam
a um impulso natural, e as segundas a um vício. Se um homem era avaro ou
pródigo, ia do mesmo modo para a Casa Verde; daí a alegação de que não
havia regra para a completa sanidade mental. Alguns cronistas creem que
Simão Bacamarte nem sempre procedia com lisura, e citam em abono da
afirmação (que não sei se pode ser aceita) o fato de ter alcançado da
câmara uma postura autorizando o uso de um anel de prata no dedo polegar
da mão esquerda, a toda a pessoa que, sem outra prova documental ou
tradicional, declarasse ter nas veias duas ou três onças de sangue
godo\footnote{Antigo povo germânico que invadiu o império romano no
  século \textsc{iii} d.C. No contexto, pode-se referir ao sangue impuro,
  pertencente a um povo rude, bárbaro.}. Dizem esses cronistas que o fim
secreto da insinuação à câmara foi enriquecer um ourives, amigo e
compadre dele; mas, conquanto seja certo que o ourives viu prosperar o
negócio depois da nova ordenação municipal\footnote{Decreto,
  regulamento. Recebe o nome de ordenação por aproximação com as
  Ordenações Reais, estendidas às possessões do reino português.}, não o
é menos que essa postura deu à Casa Verde uma multidão de inquilinos;
pelo que, não se pode definir, sem temeridade, o verdadeiro fim do
ilustre médico. Quanto à razão determinativa da captura e aposentação na
Casa Verde de todos quantos usaram do anel, é um dos pontos mais
obscuros da história de Itaguaí; a opinião mais verossímil é que eles
foram recolhidos por andarem a gesticular, à toa, nas ruas, em casa, na
igreja. Ninguém ignora que os doidos gesticulam muito. Em todo caso, é
uma simples conjectura; de positivo, nada há.

--- Onde é que este homem vai parar? diziam os principais da terra. Ah!
se nós tivéssemos apoiado os Canjicas\ldots{}

Um dia de manhã, --- dia em que a câmara devia dar um grande baile, ---
a vila inteira ficou abalada com a notícia de que a própria esposa do
alienista fora metida na Casa Verde. Ninguém acreditou; devia ser
invenção de algum gaiato. E não era: era a verdade pura. D.\,Evarista
fora recolhida às duas horas da noite. O padre Lopes correu ao alienista
e interrogou-o discretamente acerca do fato.

--- Já há algum tempo que eu desconfiava, disse gravemente o marido. A
modéstia com que ela vivera em ambos os matrimônios não podia
conciliar-se com o furor das sedas, veludos, rendas e pedras preciosas
que manifestou, logo que voltou do Rio de Janeiro. Desde então comecei a
observá-la. Suas conversas eram todas sobre esses objetos; se eu lhe
falava das antigas cortes, inquiria logo da forma dos vestidos das
damas; se uma senhora a visitava, na minha ausência, antes de me dizer o
objeto da visita, descrevia-me o trajo\footnote{Grafado desta forma, no
  original (e não ``traje'').}, aprovando umas coisas e censurando
outras. Um dia, creio que Vossa Reverendíssima há de lembrar-se,
propôs-se a fazer anualmente um vestido para a imagem de Nossa Senhora
da matriz. Tudo isto eram sintomas graves; esta noite, porém,
declarou-se a total demência. Tinha escolhido, preparado, enfeitado o
vestuário que levaria ao baile da câmara Municipal; só hesitava entre um
colar de granada e outro de safira. Anteontem perguntou-me qual deles
levaria; respondi-lhe que um ou outro lhe ficava bem. Ontem repetiu a
pergunta ao almoço; pouco depois de jantar fui achá-la calada e
pensativa. --- Que tem? perguntei-lhe. --- Queria levar o colar de
granada, mas acho o de safira tão bonito! --- Pois leve o de safira. ---
Ah! mas onde fica o de granada? --- Enfim, passou a tarde sem novidade.
Ceamos, e deitamo-nos. Alta noite, seria hora e meia, acordo e não a
vejo; levanto-me, vou ao quarto de vestir, acho-a diante dos dois
colares, ensaiando-os ao espelho, ora um, ora outro. Era evidente a
demência; recolhi-a logo.

O padre Lopes não se satisfez com a resposta, mas não objetou nada. O
alienista, porém, percebeu e explicou-lhe que o caso de D.\,Evarista era
de ``mania sumptuária''\footnote{Suntuosa, luxuosa, opulenta.}, não
incurável, e em todo caso digno de estudo.

--- Conto pô-la boa dentro de seis semanas, concluiu ele.

A abnegação do ilustre médico deu-lhe grande realce. Conjecturas,
invenções, desconfianças, tudo caiu por terra desde que ele não duvidou
recolher à Casa Verde a própria mulher, a quem amava com todas as forças
da alma. Ninguém mais tinha o direito de resistir-lhe, ---menos ainda o
de atribuir-lhe intuitos alheios à ciência. Era um grande homem austero,
Hipócrates forrado de Catão\footnote{A denominação é irônica, já que
  Hipócrates era grego e a cultura helenística constituía um dos alvos
  censurados pelo romano Catão. A alcunha sugere que Bacamarte
  corresponderia a uma espécie de sábio e censor, conhecedor versado na
  cultura greco-latino.}.

\section*{\textsc{xi}. O assombro de Itaguaí}

E agora prepare-se o leitor para o mesmo assombro em que ficou a vila,
ao saber um dia que os loucos da Casa Verde iam todos ser postos na rua.

--- Todos?

--- Todos.

--- É impossível; alguns sim, mas todos\ldots{}

--- Todos. Assim o disse ele no ofício que mandou hoje de manhã à
câmara.

De fato, o alienista oficiara à câmara expondo: ---
1\textsuperscript{o}, que verificara das estatísticas da vila e da Casa
Verde, que quatro quintos da população estavam aposentados naquele
estabelecimento; 2°, que esta deslocação de população levara-o a
examinar os fundamentos da sua teoria das moléstias cerebrais, teoria
que excluía do domínio da razão todos os casos em que o equilíbrio das
faculdades, não fosse perfeito e absoluto; 3°, que desse exame e do fato
estatístico resultara para ele a convicção de que a verdadeira doutrina
não era aquela, mas a oposta, e portanto que se devia admitir como
normal e exemplar o desequilíbrio das faculdades, e como hipóteses
patológicas todos os casos em que aquele equilíbrio fosse ininterrupto;
4\textsuperscript{o}, que à vista disso, declarava à câmara que ia dar
liberdade aos reclusos da Casa Verde e agasalhar nela as pessoas que se
achassem nas condições agora expostas; 5°, que tratando de descobrir a
verdade científica, não se pouparia a esforços de toda a natureza,
esperando da câmara igual dedicação; 6º, que restituía à câmara e aos
particulares a soma do estipêndio recebido para alojamento dos supostos
loucos, descontada a parte efetivamente gasta com a alimentação, roupa,
etc.; o que a câmara mandaria verificar nos livros e arcas da Casa
Verde.

O assombro de Itaguaí foi grande; não foi menor a alegria dos parentes e
amigos dos reclusos. Jantares, danças, luminárias\footnote{No contexto,
  ``luminárias'' parece referir-se à iluminação pública, como forma de
  celebração, festividade.}, músicas, tudo houve para celebrar tão
fausto\footnote{Luxuoso, farto.} acontecimento. Não descrevo as festas
por não interessarem ao nosso propósito; mas foram esplêndidas, tocantes
e prolongadas.

E vão assim as coisas humanas! No meio do regozijo produzido pelo ofício
de Simão Bacamarte, ninguém advertia na frase final do § 4º, uma frase
cheia de experiências futuras.

\section*{\textsc{xii}. O final do § 4º }

Apagaram-se as luminárias, reconstituíram-se as famílias, tudo parecia
reposto nos antigos eixos. Reinava a ordem, a câmara exercia outra vez o
governo, sem nenhuma pressão externa; o próprio presidente e o vereador
Freitas tornaram aos seus lugares. O barbeiro Porfírio, ensinado pelos
acontecimentos, tendo ``provado tudo'', como o poeta\footnote{Possível
  alusão ao poema ``Napoleão'', de Fagundes Varela, que integrou o seu
  livro \emph{Vozes da América}, de 1864.} disse de Napoleão, e mais
alguma coisa, porque Napoleão não provou a Casa Verde, o barbeiro achou
preferível a glória obscura da navalha e da tesoura às calamidades
brilhantes do poder; foi, é certo, processado; mas a população da vila
implorou a clemência de Sua Majestade; daí o perdão. João Pina foi
absolvido, atendendo-se a que ele derrocara um rebelde. Os cronistas
pensam que deste fato é que nasceu o nosso adágio: --- ladrão que furta
ladrão tem cem anos de perdão; --- adágio\footnote{Rifão, lugar-comum,
  frase-feita.} imoral, é verdade, mas grandemente útil. Não só findaram
as queixas contra o alienista, mas até nenhum ressentimento ficou dos
atos que ele praticara; acrescendo que os reclusos da Casa Verde, desde
que ele os declarara plenamente ajuizados, sentiram-se tomados de
profundo reconhecimento e férvido entusiasmo. Muitos entenderam que o
alienista merecia uma especial manifestação, e deram-lhe um baile, ao
qual se seguiram outros bailes e jantares. Dizem as crônicas que D.\,Evarista a princípio tivera ideia de separar-se do consorte, mas a dor
de perder a companhia de tão grande homem venceu qualquer ressentimento
de amor-próprio, e o casal veio a ser ainda mais feliz do que antes.

Não menos íntima ficou a amizade do alienista e do boticário. Este
concluiu do ofício de Simão Bacamarte que a prudência é a primeira das
virtudes em tempos de revolução e apreciou muito a magnanimidade do
alienista que, ao dar-lhe a liberdade, estendeu-lhe a mão de amigo
velho.

--- É um grande homem, disse ele à mulher, referindo aquela
circunstância.

Não é preciso falar do albardeiro, do Costa, do Coelho, do Martim Brito
e outros, especialmente nomeados neste escrito; basta dizer que puderam
exercer livremente os seus hábitos anteriores. O próprio Martim Brito,
recluso por um discurso em que louvara enfaticamente D.\,Evarista, fez
agora outro em honra do insigne médico --- ``cujo altíssimo gênio,
elevando as asas muito acima do sol, deixou abaixo de si todos os demais
espíritos da terra''.

--- Agradeço as suas palavras, retorquiu-lhe o alienista, e ainda me não
arrependo de o haver restituído à liberdade.

Entretanto, a câmara, que respondera ao ofício de Simão Bacamarte, com a
ressalva de que oportunamente estatuiria em relação ao final do § 4°,
tratou enfim de legislar sobre ele. Foi adotada, sem debate, uma postura
autorizando o alienista a agasalhar na Casa Verde as pessoas que se
achassem no gozo do perfeito equilíbrio das faculdades mentais. E porque
a experiência da câmara tivesse sido dolorosa, estabeleceu ela a
cláusula, de que a autorização era provisória, limitada a um ano, para o
fim de ser experimentada a nova teoria psicológica, podendo a câmara,
antes mesmo daquele prazo, mandar fechar a Casa Verde, se a isso fosse
aconselhada por motivos de ordem pública. O vereador Freitas propôs
também a declaração de que em nenhum caso fossem os vereadores
recolhidos ao asilo dos alienados: cláusula que foi aceita, votada e
incluída na postura, apesar das reclamações do vereador Galvão. O
argumento principal deste magistrado é que a câmara, legislando sobre
uma experiência científica, não podia excluir as pessoas dos seus
membros das consequências da lei; a exceção era odiosa e ridícula. Mal
proferira estas duas palavras, romperam os vereadores em altos brados
contra a audácia e insensatez do colega; este, porém, ouviu-os e
limitou-se a dizer que votava contra a exceção.

--- A vereança, concluiu ele, não nos dá nenhum poder especial nem nos
elimina do espírito humano.

Simão Bacamarte aceitou a postura com todas as restrições. Quanto à
exclusão dos vereadores, declarou que teria profundo sentimento se fosse
compelido a recolhê-los à Casa Verde; a cláusula, porém, era a melhor
prova de que eles não padeciam do perfeito equilíbrio das faculdades
mentais. Não acontecia o mesmo ao vereador Galvão, cujo acerto na
objeção feita, e cuja moderação na resposta dada às invectivas dos
colegas mostravam da parte dele um cérebro bem organizado; pelo que,
rogava à câmara que lho entregasse. A câmara, sentindo-se ainda
agravada\footnote{Que sofreu agravo ou foi injuriada.} pelo proceder do
vereador Galvão, estimou o pedido do alienista, e votou unanimemente a
entrega. Compreende-se que, pela teoria nova, não bastava um fato ou um
dito para recolher alguém à Casa Verde; era preciso um longo exame, um
vasto inquérito do passado e do presente. O padre Lopes, por exemplo, só
foi capturado trinta dias depois da postura, a mulher do boticário
quarenta dias. A reclusão desta senhora encheu o consorte de indignação.
Crispim Soares saiu de casa espumando de cólera, e declarando às pessoas
a quem encontrava que ia arrancar as orelhas ao tirano. Um sujeito,
adversário do alienista, ouvindo na rua essa notícia, esqueceu os
motivos de dissidência, e correu à casa de Simão Bacamarte a
participar-lhe o perigo que corria. Simão Bacamarte mostrou-se grato ao
procedimento do adversário, e poucos minutos lhe bastaram para conhecer
a retidão dos seus sentimentos, a boa-fé, o respeito humano, a
generosidade; apertou-lhe muito as mãos, e recolheu-o à Casa Verde.

--- Um caso destes é raro, disse ele à mulher pasmada. Agora esperemos o
nosso Crispim.

Crispim Soares entrou. A dor vencera a raiva, o boticário não arrancou
as orelhas ao alienista. Este consolou o seu privado, assegurando-lhe
que não era caso perdido; talvez a mulher tivesse alguma lesão cerebral;
ia examiná-la com muita atenção; mas antes disso não podia deixá-la na
rua. E parecendo-lhe vantajoso reuni-los, porque a astúcia e velhacaria
do marido poderiam de certo modo curar a beleza moral que ele descobrira
na esposa, disse Simão Bacamarte:

--- O senhor trabalhará durante o dia na botica, mas almoçará e jantará,
com sua mulher, e cá passará as noites, e os domingos e dias santos.

A proposta colocou o pobre boticário na situação do asno de
Buridan\footnote{Referência ao francês Jean Buridan (1295--1358), que
  teria concebido o impasse mortal enfrentado por um burro, dividido
  entre comer aveia ou tomar água.}. Queria viver com a mulher, mas
temia voltar à Casa Verde; e nessa luta esteve algum tempo, até que D.\,Evarista o tirou da dificuldade, prometendo que se incumbiria de ver a
amiga e transmitiria os recados de um para outro. Crispim Soares
beijou-lhe as mãos agradecido. Este último rasgo de egoísmo pusilânime
pareceu sublime ao alienista.

Ao cabo de cinco meses estavam alojadas umas dezoito pessoas; mas Simão
Bacamarte não afrouxava; ia de rua em rua, de casa em casa, espreitando,
interrogando, estudando; e quando colhia um enfermo, levava-o com a
mesma alegria com que outrora os arrebanhava às dúzias. Essa mesma
desproporção confirmava a teoria nova; achara-se enfim a verdadeira
patologia cerebral. Um dia, conseguiu meter na Casa Verde o juiz de
fora; mas procedia com tanto escrúpulo que o não fez senão depois de
estudar minuciosamente todos os seus atos, e interrogar os principais da
vila. Mais de uma vez esteve prestes a recolher pessoas perfeitamente
desequilibradas; foi o que se deu com um advogado, em quem reconheceu um
tal conjunto de qualidades morais e mentais que era perigoso deixá-lo na
rua. Mandou prendê-lo; mas o agente, desconfiado, pediu-lhe para fazer
uma experiência; foi ter com um compadre, demandado por um testamento
falso, e deu-lhe de conselho que tomasse por advogado o
Salustiano\footnote{Patriarca de Jerusalém que viveu no século V d.C.
  Saiu-se vitorioso de uma disputa com os monges liderados por São Sabas
  (439--532 d.C.).}; era o nome da pessoa em questão.

--- Então parece-lhe\ldots{}?

--- Sem dúvida: vá, confesse tudo, a verdade inteira, seja qual for, e
confie-lhe a causa.

O homem foi ter com o advogado, confessou ter falsificado o testamento,
e acabou pedindo que lhe tomasse a causa. Não se negou o advogado;
estudou os papéis, arrazoou longamente, e provou a todas as luzes que o
testamento era mais que verdadeiro. A inocência do réu foi solenemente
proclamada pelo juiz, e a herança passou-lhe às mãos. O distinto
jurisconsulto deveu a esta experiência a liberdade. Mas nada escapa a um
espírito original e penetrante. Simão Bacamarte, que desde algum tempo
notava o zelo, a sagacidade, a paciência, a moderação daquele agente,
reconheceu a habilidade e o tino com que ele levara a cabo uma
experiência tão melindrosa\footnote{Delicada, complexa.} e complicada, e
determinou recolhê-lo imediatamente à Casa Verde; deu-lhe, todavia, um
dos melhores cubículos.

Os alienados foram alojados por classes. Fez-se uma galeria de modestos;
isto é, os loucos em quem predominava esta perfeição moral; outra de
tolerantes, outra de verídicos, outra de símplices, outra de leais,
outra de magnânimos, outra de sagazes, outra de sinceros, etc.
Naturalmente, as famílias e os amigos dos reclusos bradavam contra a
teoria; e alguns tentaram compelir a câmara a cassar a licença. A câmara
porém, não esquecera a linguagem do vereador Galvão, e se cassasse a
licença, vê-lo-ia na rua, e restituído ao lugar; pelo que, recusou.
Simão Bacamarte oficiou aos vereadores, não agradecendo, mas
felicitando-os por esse ato de vingança pessoal.

Desenganados da legalidade, alguns principais da vila recorreram
secretamente ao barbeiro Porfírio e afiançaram-lhe todo o apoio de
gente, dinheiro e influência na corte, se ele se pusesse à testa de
outro movimento contra a câmara e o alienista. O barbeiro respondeu-lhes
que não; que a ambição o levara da primeira vez a transgredir as leis,
mas que ele se emendara, reconhecendo o erro próprio e a pouca
consistência da opinião dos seus mesmos sequazes; que a câmara entendera
autorizar a nova experiência do alienista, por um ano: cumpria, ou
esperar o fim do prazo, ou requerer ao vice-rei, caso a mesma Câmara
rejeitasse o pedido. Jamais aconselharia o emprego de um recurso que ele
viu falhar em suas mãos, e isso a troco de mortes e ferimentos que
seriam o seu eterno remorso.

--- O que é que me está dizendo? perguntou o alienista quando um agente
secreto lhe contou a conversação do barbeiro com os principais da vila.

Dois dias depois o barbeiro era recolhido à Casa Verde. --- Preso por
ter cão, preso por não ter cão! exclamou o infeliz.

Chegou o fim do prazo, a câmara autorizou um prazo suplementar de seis
meses para ensaio dos meios terapêuticos. O desfecho deste episódio da
crônica itaguaiense, é de tal ordem, e tão inesperado, que merecia nada
menos de dez capítulos de exposição; mas contento-me com um, que será o
remate da narrativa, e um dos mais belos exemplos de convicção
científica e abnegação humana.

\section*{\textsc{xiii}. Plus Ultra!\protect\footnote[*]{\MakeUppercase{E}xpressão latina que significa ``mais ainda''.}}

Era a vez da terapêutica. Simão Bacamarte, ativo e sagaz em descobrir
enfermos, excedeu-se ainda na diligência e penetração com que principiou
a tratá-los. Neste ponto todos os cronistas estão de pleno acordo: o
ilustre alienista fez curas pasmosas, que excitaram a mais viva
admiração em Itaguaí.

Com efeito, era difícil imaginar mais racional sistema terapêutico.
Estando os loucos divididos por classes, segundo a perfeição moral que
em cada um deles excedia às outras, Simão Bacamarte cuidou em atacar de
frente a qualidade predominante. Suponhamos um modesto. Ele aplicava a
medicação que pudesse incutir-lhe o sentimento oposto; e não ia logo às
doses máximas, --- graduava-as, conforme o estado, a idade, o
temperamento, a posição social do enfermo. Às vezes bastava uma casaca,
uma fita, uma cabeleira, uma bengala, para restituir a razão ao
alienado; em outros casos a moléstia era mais rebelde; recorria então
aos anéis de brilhantes, às distinções honoríficas, etc. Houve um
doente, poeta, que resistiu a tudo. Simão Bacamarte começava a
desesperar da cura, quando teve a ideia de mandar correr matraca, para o
fim de o apregoar como um rival de Garção\footnote{Correia Garção, poeta
  árcade português bastante lido no Brasil e em Portugal, durante os
  séculos \textsc{xviii} e \textsc{xix}.} e de Píndaro\footnote{Píndaro (517--438 a.C.),
  poeta grego, autor de hinos célebres em seu tempo.}.

--- Foi um santo remédio, contava a mãe do infeliz a uma comadre; foi um
santo remédio.

Outro doente, também modesto, opôs a mesma rebeldia à medicação; mas,
não sendo escritor (mal sabia assinar o nome), não se lhe podia aplicar
o remédio da matraca. Simão Bacamarte lembrou-se de pedir para ele o
lugar de secretário da Academia dos Encobertos\footnote{Machado de Assis
  alude às academias literárias que se tornaram numerosas na Europa e no
  Brasil, a partir do século \textsc{xvii}. O nome dado à agremiação, na novela,
  possivelmente parodia a Academia dos Esquecidos, que funcionou em
  Salvador entre 1724 e 1725, e teve como um de seus secretários o
  historiador Sebastião da Rocha Pita.} estabelecida em Itaguaí. Os
lugares de presidente e secretários eram de nomeação régia, por especial
graça do finado Rei Dom João \textsc{v}, e implicavam o tratamento de Excelência
e o uso de uma placa de ouro no chapéu. O governo de Lisboa recusou o
diploma; mas, representando o alienista que o não pedia como prêmio
honorífico ou distinção legítima, e somente como um meio terapêutico
para um caso difícil, o governo cedeu excepcionalmente à súplica; e
ainda assim não o fez sem extraordinário esforço do ministro da marinha
e ultramar, que vinha a ser primo do alienado. Foi outro santo remédio.

--- Realmente, é admirável! dizia-se nas ruas, ao ver a expressão sadia
e enfunada dos dois ex-dementes.

Tal era o sistema. Imagina-se o resto. Cada beleza moral ou mental era
atacada no ponto em que a perfeição parecia mais sólida; e o efeito era
certo. Nem sempre era certo. Casos houve em que a qualidade predominante
resistia a tudo; então o alienista atacava outra parte, aplicando à
terapêutica o método da estratégia militar, que toma uma fortaleza por
um ponto, se por outro o não pode conseguir.

No fim de cinco meses e meio estava vazia a Casa Verde; todos curados! O
vereador Galvão, tão cruelmente afligido de moderação e equidade, teve a
felicidade de perder um tio; digo felicidade, porque o tio deixou um
testamento ambíguo, e ele obteve uma boa interpretação, corrompendo os
juízes, e embaçando os outros herdeiros. A sinceridade do alienista
manifestou-se nesse lance; confessou ingenuamente que não teve parte na
cura: foi a simples \emph{vis medicatrix}\footnote{Trecho de sentença
  latina que significa ``poder de cura''. A expressão original (``vis
  medicatrix naturae'') é atribuída ao médico Hipócrates.} da natureza.
Não aconteceu o mesmo com o padre Lopes. Sabendo o alienista que ele
ignorava perfeitamente o hebraico e o grego, incumbiu-o de fazer uma
análise crítica da versão dos Setenta\footnote{Reunião de escritos que
  conteria a versão em grego das escrituras hebraicas, também conhecido
  como \emph{Septuaginta}, ou Versão dos Setenta. Remonta ao século \textsc{iii}
  a.C.}; o padre aceitou a incumbência, e em boa hora o fez; ao cabo de
dois meses possuía um livro e a liberdade. Quanto à senhora do
boticário, não ficou muito tempo na célula que lhe coube, e onde aliás
lhe não faltaram carinhos.

--- Por que é que o Crispim não vem visitar-me? dizia ela todos os dias.

Respondiam-lhe ora uma coisa, ora outra; afinal disseram-lhe a verdade
inteira. A digna matrona não pôde conter a indignação e a vergonha. Nas
explosões da cólera escaparam-lhe expressões soltas e vagas, como estas:

--- Tratante!\ldots{} velhaco!\ldots{} ingrato!\ldots{} Um patife que tem feito casas à
custa de unguentos\footnote{Medicamento em forma de pomada.}
falsificados e podres\ldots{} Ah! tratante!\ldots{}

Simão Bacamarte advertiu que, ainda quando não fosse verdadeira a
acusação contida nestas palavras, bastavam elas para mostrar que a
excelente senhora estava enfim restituída ao perfeito desequilíbrio das
faculdades; e prontamente lhe deu alta.

Agora, se imaginais que o alienista ficou radiante ao ver sair o último
hóspede da Casa Verde, mostrais com isso que ainda não conheceis o nosso
homem. \emph{Plus ultra!} era a sua divisa. Não lhe bastava ter
descoberto a teoria verdadeira da loucura; não o contentava ter
estabelecido em Itaguaí o reinado da razão. \emph{Plus ultra!} Não ficou
alegre, ficou preocupado, cogitativo; alguma coisa lhe dizia que a
teoria nova tinha, em si mesma, outra e novíssima teoria.

``Vejamos, pensava ele; vejamos se chego enfim à última verdade''.

Dizia isto, passeando ao longo da vasta sala\footnote{Simão Bacamarte
  age como os filósofos peripatéticos, à época de Aristóteles.}, onde
fulgurava a mais rica biblioteca dos domínios ultramarinos de Sua
Majestade. Um amplo chambre de damasco, preso à cintura por um cordão de
seda, com borlas de ouro (presente de uma Universidade) envolvia o corpo
majestoso e austero do ilustre alienista. A cabeleira cobria-lhe uma
extensa e nobre calva adquirida nas cogitações quotidianas da ciência.
Os pés, não delgados e femininos, não graúdos e mariolas\footnote{Mariola
  era um doce de frutas em formato de tablete. No contexto, parece se
  referir ao tamanho ou formato dos pés.}, mas proporcionados ao vulto,
eram resguardados por um par de sapatos cujas fivelas não passavam de
simples e modesto latão. Vede a diferença: --- só se lhe notava luxo
naquilo que era de origem científica; o que propriamente vinha dele
trazia a cor da moderação e da singeleza, virtudes tão ajustadas à
pessoa de um sábio.

Era assim que ele ia, o grande alienista, de um cabo a outro da vasta
biblioteca, metido em si mesmo, estranho a todas as coisas que não fosse
o tenebroso problema da patologia cerebral. Súbito, parou. Em pé, diante
de uma janela, com o cotovelo esquerdo apoiado na mão direita, aberta, e
o queixo na mão esquerda, fechada, perguntou ele a si:

--- Mas deveras estariam eles doidos, e foram curados por mim, --- ou o
que pareceu cura, não foi mais do que a descoberta do perfeito
desequilíbrio do cérebro?

E cavando por aí abaixo, eis o resultado a que chegou: os cérebros bem
organizados que ele acabava de curar, eram desequilibrados como os
outros. Sim, dizia ele consigo, eu não posso ter a pretensão de
haver-lhes incutido um sentimento ou uma faculdade nova; uma e outra
coisa existiam no estado latente\footnote{Sintoma oculto, ainda não
  manifesto.}, mas existiam.

Chegado a esta conclusão, o ilustre alienista teve duas sensações
contrárias, uma de gozo, outra de abatimento. A de gozo foi por ver que,
ao cabo de longas e pacientes investigações, constantes trabalhos, luta
ingente\footnote{Desmedido, descomunal.} com o povo, podia afirmar esta
verdade: --- não havia loucos em Itaguaí; Itaguaí não possuía um só
mentecapto. Mas tão depressa esta ideia lhe refrescara a alma, outra
apareceu que neutralizou o primeiro efeito; foi a ideia da dúvida. Pois
quê! Itaguaí não possuiria um único cérebro concertado? Esta conclusão
tão absoluta, não seria por isso mesmo errônea, e não vinha, portanto,
destruir o largo e majestoso edifício da nova doutrina psicológica?

A aflição do egrégio Simão Bacamarte é definida pelos cronistas
itaguaienses como uma das mais medonhas tempestades morais que têm
desabado sobre o homem. Mas as tempestades só aterram os fracos; os
forres enrijam-se contra elas e fitam o trovão. Vinte minutos depois
alumiou-se a fisionomia do alienista de uma suave claridade.

--- Sim, há de ser isso, pensou ele.

Isso é isto. Simão Bacamarte achou em si os característicos do perfeito
equilíbrio mental e moral; pareceu-lhe que possuía a sagacidade, a
paciência, a perseverança, a tolerância, a veracidade, o vigor moral, a
lealdade, todas as qualidades enfim que podem formar um acabado
mentecapto. Duvidou logo, é certo, e chegou mesmo a concluir que era
ilusão; mas, sendo homem prudente, resolveu convocar um conselho de
amigos, a quem interrogou com franqueza. A opinião foi afirmativa.

--- Nenhum defeito?

--- Nenhum, disse em coro a assembleia.

--- Nenhum vício?

--- Nada.

--- Tudo perfeito?

--- Tudo.

--- Não, impossível, bradou o alienista. Digo que não sinto em mim essa
superioridade que acabo de ver definir com tanta magnificência. A
simpatia é que vos faz falar. Estudo-me e nada acho que justifique os
excessos da vossa bondade.

A assembleia insistiu; o alienista resistiu; finalmente o padre Lopes
explicou tudo com este conceito digno de um observador:

--- Sabe a razão por que não vê as suas elevadas qualidades, que aliás
todos nós admiramos? É porque tem ainda uma qualidade que realça as
outras: --- a modéstia.

Era decisivo. Simão Bacamarte curvou a cabeça, juntamente alegre e
triste, e ainda mais alegre do que triste. Ato contínuo, recolheu-se à
Casa Verde. Em vão a mulher e os amigos lhe disseram que ficasse, que
estava perfeitamente são e equilibrado: nem rogos nem sugestões nem
lágrimas\footnote{Período grafado sem vírgulas, no original.} o
detiveram um só instante. A questão é científica, dizia ele; trata-se de
uma doutrina nova, cujo primeiro exemplo sou eu. Reúno em mim mesmo a
teoria e a prática.

--- Simão! Simão! meu amor! dizia-lhe a esposa com o rosto lavado em
lágrimas.

Mas o ilustre médico, com os olhos acesos da convicção científica,
trancou os ouvidos à saudade da mulher, e brandamente a repeliu. Fechada
a porta da Casa Verde, entregou-se ao estudo e à cura de si mesmo. Dizem
os cronistas que ele morreu dali a dezessete meses, no mesmo estado em
que entrou, sem ter podido alcançar nada. Alguns chegam ao ponto de
conjecturar que nunca houve outro louco, além dele, em Itaguaí; mas esta
opinião, fundada em um boato que correu desde que o alienista expirou,
não tem outra prova, senão o boato; e boato duvidoso, pois é atribuído
ao padre Lopes, que com tanto fogo realçara as qualidades do grande
homem. Seja como for, efetuou-se o enterro com muita pompa e rara
solenidade.


\chapter{O imortal\footnote[*]{O texto a seguir foi cotejado com a versão
  em livro de ``O imortal'', que integra a coletânea \emph{Contos
  Avulsos}, disponibilizada na \emph{Obra Completa} de Machado de Assis,
  reeditada em 2015 pela Nova Aguilar.}}

\section*{i\protect\footnote[\dagger]{``\MakeUppercase{O}
  imortal'' foi publicado em livro na coletânea \emph{\MakeUppercase{C}ontos \MakeUppercase{A}vulsos} (\MakeUppercase{C}f. \MakeUppercase{M}achado de \MakeUppercase{A}ssis. \emph{\MakeUppercase{O}bra completa}. v.\,3. \MakeUppercase{R}io de \MakeUppercase{J}aneiro:
  \MakeUppercase{N}ova \MakeUppercase{A}guilar, 2015, pp. 64--77), postumamente.}}

--- Meu pai nasceu em 1600\ldots{}

--- Perdão, em 1800, naturalmente\ldots{}

--- Não, senhor, replicou o Dr.\,Leão\footnote{Alusão ao conto ``Ruy de
  Leão'', publicado pelo próprio autor no \emph{Jornal das Famílias}, em
  1872.}, de um modo grave e triste; foi em 1600.

Estupefação dos ouvintes, que eram dois, o Coronel Bertioga, e o
tabelião da vila, João Linhares. A vila era na província fluminense;
suponhamos Itaboraí ou Sapucaia. Quanto à data, não tenho dúvida em
dizer que foi no ano de 1855, uma noite de novembro, escura como breu,
quente como um forno, passante de nove horas. Tudo silêncio. O lugar em
que os três estavam era a varanda que dava para o terreiro. Um lampião
de luz frouxa, pendurado de um prego, sublinhava a escuridão exterior.
De quando em quando, gania um seco e áspero vento, mesclando-se ao som
monótono de uma cachoeira próxima. Tal era o quadro e o momento, quando
o Dr.\,Leão insistiu nas primeiras palavras da narrativa.

--- Não, senhor; nasceu em 1600.

Médico homeopata, --- a homeopatia começava a entrar nos domínios da
nossa civilização, --- este Dr.\,Leão chegara à vila, dez ou doze dias
antes, provido de boas cartas de recomendação, pessoais e políticas. Era
um homem inteligente, de fino trato e coração benigno. A gente da vila
notou-lhe certa tristeza no gesto, algum retraimento nos hábitos, e até
uma tal ou qual sequidão de palavras, sem embargo da perfeita cortesia;
mas tudo foi atribuído ao acanho dos primeiros dias e às saudades da
Corte. Contava trinta anos, tinha um princípio de calva, olhar baço e
mãos episcopais. Andava propagando o novo sistema.

Os dois ouvintes continuavam pasmados. A dúvida fora posta pelo dono da
casa, o Coronel Bertioga, e o tabelião ainda insistiu no caso, mostrando
ao médico a impossibilidade de ter o pai nascido em 1600. Duzentos e
cinquenta e cinco anos antes! dois séculos e meio! Era impossível.
Então, que idade tinha ele? e de que idade morreu o pai?

--- Não tenho interesse em contar-lhes a vida de meu pai, respondeu o
Dr.\,Leão. Falaram-me no macróbio que mora nos fundos da matriz;
disse-lhes que, em negócio de macróbios, conheci o que há mais espantoso
no mundo, um homem imortal\ldots{}

--- Mas seu pai não morreu? disse o coronel.

--- Morreu.

--- Logo, não era imortal, concluiu o tabelião triunfante. Imortal se
diz quando uma pessoa não morre, mas seu pai morreu.

--- Querem ouvir-me?

--- Homem, pode ser, observou o coronel meio abalado. O melhor é ouvir
a história. Só o que digo é que mais velho do que o Capataz nunca vi
ninguém. Está mesmo caindo de maduro. Seu pai devia estar também muito
velho\ldots{}?

--- Tão moço como eu. Mas para que me fazem perguntas soltas? Para se
espantarem cada vez mais, porque na verdade a história de meu pai não é
fácil de crer. Posso contá-la em poucos minutos.

Excitada a curiosidade, não foi difícil impor-lhes silêncio. A família
toda estava acomodada, os três eram sós na varanda, o dr. Leão contou
enfim a vida do pai, nos termos em que o leitor vai ver, se se der o
trabalho de ler o segundo e os outros capítulos.



\section*{ii}



--- Meu pai nasceu em 1600, na cidade de Recife.

Aos vinte e cinco anos tomou o hábito franciscano, por vontade de minha
avó, que era profundamente religiosa. Tanto ela como o marido eram
pessoas de bom nascimento, --- ``bom sangue'', como dizia meu pai,
afetando a linguagem antiga.

Meu avô descendia da nobreza de Espanha, e minha avó era de uma grande
casa do Alentejo. Casaram-se ainda na Europa, e, anos depois, por
motivos que não vêm ao caso dizer, transportaram-se ao Brasil, onde
ficaram e morreram. Meu pai dizia que poucas mulheres tinha visto tão
bonitas como minha avó. E olhem que ele amou as mais esplêndidas
mulheres do mundo. Mas não antecipemos.

Tomou meu pai o hábito, no convento de Iguaraçu, onde ficou até 1639,
ano em que os holandeses, ainda uma vez, assaltaram a povoação. Os
frades deixaram precipitadamente o convento; meu pai, mais remisso do
que os outros (ou já com o intento de deitar o hábito às urtigas),
deixou-se ficar na cela, de maneira que os holandeses o foram achar no
momento em que recolhia alguns livros pios e objetos de uso pessoal. Os
holandeses não o trataram mal. Ele os regalou com o melhor da ucharia
franciscana, onde a pobreza é de regra. Sendo uso daqueles frades
alternarem-se no serviço da cozinha, meu pai entendia da arte, e esse
talento foi mais um encanto ao aparecer do inimigo.

No fim de duas semanas, o oficial holandês ofereceu-lhe um
salvo-conduto, para ir aonde lhe parecesse; mas meu pai não o aceitou
logo, querendo primeiro considerar se devia ficar com os holandeses, e,
à sombra deles desamparar a Ordem, ou se lhe era melhor buscar vida por
si mesmo. Adotou o segundo alvitre, não só por ter o espírito
aventureiro, curioso e audaz, como porque era patriota, e bom católico,
apesar da repugnância à vida monástica, e não quisera misturar-se com o
herege invasor. Aceitou o salvo-conduto e deixou Iguaraçu.

Não se lembrava ele, quando me contou essas coisas, não se lembrava
mais do número de dias que despendeu sozinho por lugares ermos, fugindo
de propósito ao povoado, não querendo ir a Olinda ou Recife, onde
estavam os holandeses. Comidas as provisões que levava, ficou dependente
de alguma caça silvestre e frutas. Deitara, com efeito, o hábito às
urtigas; vestia uns calções flamengos, que o oficial lhe dera, e uma
camisola ou jaquetão de couro. Para encurtar razões, foi ter a uma
aldeia de gentio, que o recebeu muito bem, com grandes carinhos e
obséquios. Meu pai era talvez o mais insinuante dos homens. Os índios
ficaram embeiçados por ele, mormente o chefe, um guerreiro velho, bravo
e generoso, que chegou a dar-lhe a filha em casamento. Já então minha
avó era morta, e meu avô desterrado para a Holanda, notícias que meu pai
teve, casualmente, por um antigo servo da casa. Deixou-se estar, pois,
na aldeia, o gentio, até o ano de 1642, em que o guerreiro faleceu. Este
caso do falecimento é que é maravilhoso: peço-lhes a maior atenção.

O coronel e o tabelião aguçaram os ouvidos, enquanto o Dr.\,Leão extraía
pausadamente uma pitada\footnote{O hábito de cheirar rapé era bem comum
  nos círculos sociais frequentados pela elite social, no século \textsc{xix}.} e
inseria-a no nariz, com a pachorra de quem está negaceando uma coisa
extraordinária.



\section*{iii}



Uma noite, o chefe indígena, --- chamava-se Pirajuá\footnote{Do tupi
  \emph{Pirá} (peixe) + \emph{Juá} (fruta de espinho).}, --- foi à rede
de meu pai, anunciou-lhe que tinha de morrer, pouco depois de nascer o
sol, e que ele estivesse pronto para acompanhá-lo fora, antes do momento
último. Meu pai ficou alvoroçado, não por lhe dar crédito, mas por
supô-lo delirante. Sobre a madrugada, o sogro veio ter com ele.

--- Vamos, disse-lhe.

--- Não, agora não: estás fraco, muito fraco\ldots{}

--- Vamos! repetiu o guerreiro.

E, à luz de uma fogueira expirante, viu-lhe meu pai a expressão
intimativa do rosto, e um certo ar diabólico, em todo caso
extraordinário, que o aterrou. Levantou-se, acompanhou-o na direção de
um córrego. Chegando ao córrego, seguiram pela margem esquerda, acima,
durante um tempo que meu pai calculou ter sido um quarto de hora. A
madrugada acentuava-se; a lua fugia diante dos primeiros anúncios do
sol. Contudo, e apesar da vida do sertão que meu pai levava desde alguns
tempos, a aventura assustava-o; seguia vigiando o sogro, com receio de
alguma traição. Pirajuá ia calado, com os olhos no chão, e a fronte
carregada de pensamentos, que podiam ser cruéis ou somente tristes. E
andaram, andaram, até que Pirajuá disse:

--- Aqui.

Estavam diante de três pedras, dispostas em triângulo. Pirajuá
sentou-se numa, meu pai noutra. Depois de alguns minutos de descanso:

--- Arreda aquela pedra, disse o guerreiro, apontando para a terceira,
que era a maior.

Meu pai levantou-se e foi à pedra. Era pesada, resistiu ao primeiro
impulso; mas meu pai teimou, aplicou todas as forças, a pedra cedeu um
pouco, depois mais, enfim foi removida do lugar.

--- Cava o chão, disse o guerreiro.

Meu pai foi buscar uma lasca de pau, uma taquara ou não sei que, e
começou a cavar o chão. Já então estava curioso de ver o que era.
Tinha-lhe nascido uma ideia, --- algum tesouro enterrado, que o
guerreiro, receoso de morrer, quisesse entregar-lhe. Cavou, cavou,
cavou, até que sentiu um objeto rijo; era um vaso tosco, talvez uma
igaçaba. Não o tirou, não chegou mesmo a arredar a terra em volta dele.
O guerreiro aproximou-se, desatou o pedaço de couro de anta que lhe
cobria a boca, meteu dentro o braço, e tirou um boião. Este boião tinha
a boca tapada com outro pedaço de couro.

--- Vem cá, disse o guerreiro.

Sentaram-se outra vez. O guerreiro tinha o boião sobre os joelhos,
tapado, misterioso, aguçando a curiosidade de meu pai, que ardia por
saber o que havia ali dentro.

--- Pirajuá vai morrer, disse ele; vai morrer para nunca mais. Pirajuá
ama guerreiro branco, esposo de Maracujá, sua filha; e vai mostrar um
segredo como não há outro.

Meu pai estava trêmulo. O guerreiro desatou lentamente o couro que
tapava o boião. Destapado, olhou para dentro, levantou-se, e veio
mostrá-lo a meu pai. Era um líquido amarelado, de um cheiro acre e
singular.

--- Quem bebe isto, um gole só, nunca mais morre.

--- Oh! bebe, bebe! exclamou meu pai com vivacidade.

Foi um movimento de afeto, um ato irrefletido de verdadeira amizade
filial, porque só um instante depois é que meu pai advertiu que não
tinha, para crer na notícia que o sogro lhe dava, senão a palavra do
mesmo sogro, cuja razão supunha perturbada pela moléstia. Pirajuá sentiu
o espontâneo da palavra de meu pai, e agradeceu-lha; mas abanou a
cabeça.

--- Não, disse ele; Pirajuá não bebe, Pirajuá quer morrer. Está
cansado, viu muita lua, muita lua. Pirajuá quer descansar na terra, está
aborrecido. Mas Pirajuá quer deixar este segredo a guerreiro branco;
está aqui; foi feito por um velho pajé de longe, muito longe\ldots{}
Guerreiro branco bebe, não morre mais.

Dizendo isto, tornou a tapar a boca do boião, e foi metê-lo outra vez
dentro da igaçaba. Meu pai fechou depois a boca da mesma igaçaba, e
repôs a pedra em cima. O primeiro clarão do sol vinha apontando.
Voltaram para casa depressa; antes mesmo de tomar a rede, Pirajuá
faleceu.

Meu pai não acreditou na virtude do elixir. Era absurdo supor que um tal
líquido pudesse abrir uma exceção na lei da morte. Era naturalmente
algum remédio, se não fosse algum veneno; e neste caso, a mentira do
índio estava explicada pela turvação mental que meu pai lhe atribuiu.
Mas, apesar de tudo, nada disse aos demais índios da aldeia, nem à
própria esposa. Calou-se; --- nunca me revelou o motivo do silêncio:
creio que não podia ser outro senão o próprio influxo do mistério.

Tempos depois, adoeceu, e tão gravemente que foi dado por perdido. O
curandeiro do lugar anunciou a Maracujá que ia ficar viúva. Meu pai não
ouviu a notícia, mas leu-a em uma página de lágrimas, no rosto da
consorte, e sentiu em si mesmo que estava acabado. Era forte, valoroso,
capaz de encarar todos os perigos; não se aterrou, pois, com a ideia de
morrer, despediu-se dos vivos, fez algumas recomendações e preparou-se
para a grande viagem.

Alta noite, lembrou-se do elixir, e perguntou a si mesmo se não era
acertado tentá-lo. Já agora a morte era certa, que perderia ele com a
experiência? A ciência de um século não sabia tudo; outro século vem e
passa adiante. Quem sabe, dizia ele consigo, se os homens não
descobrirão um dia a imortalidade, e se o elixir científico não será
esta mesma droga selvática\footnote{Da selva, selvagem --- a sugerir
  oposição entre natureza e ciência.}? O primeiro que curou a febre
maligna fez um prodígio. Tudo é incrível antes de divulgado. E, pensando
assim, resolveu transportar-se ao lugar da pedra, à margem do arroio;
mas não quis ir de dia, com medo de ser visto. De noite, ergueu-se, e
foi, trôpego, vacilante, batendo o queixo. Chegou à pedra, arredou-a,
tirou o boião, e bebeu metade do conteúdo. Depois sentou-se para
descansar. Ou o descanso, ou o remédio, alentou-o logo. Ele tornou a
guardar o boião; daí a meia hora estava outra vez na rede. Na seguinte
manhã estava bom\ldots{}

--- Bom de todo? perguntou o tabelião João Linhares, interrompendo o
narrador.

--- De todo.

--- Era algum remédio para febre\ldots{}

Foi isto mesmo o que ele pensou, quando se viu bom. Era algum remédio
para febre e outras doenças; e nisto ficou; mas, apesar do efeito da
droga, não a descobriu a ninguém. Entretanto, os anos passaram, sem que
meu pai envelhecesse; qual era no tempo da moléstia, tal ficou. Nenhuma
ruga, nenhum cabelo branco. Moço, perpetuamente moço. A vida do mato
começara a aborrecê-lo; ficara ali por gratidão ao sogro; as saudades da
civilização vieram tomá-lo. Um dia, a aldeia foi invadida por uma horda
de índios de outra, não se sabe por que motivo, nem importa ao nosso
caso. Na luta pereceram muitos, meu pai foi ferido, e fugiu para o mato.
No dia seguinte veio à aldeia, achou a mulher morta. As feridas eram
profundas; curou-as com o emprego de remédios usuais; e restabeleceu-se
dentro de poucos dias. Mas os sucessos confirmaram-no no propósito de
deixar a vida semi-selvagem e tornar à vida civilizada e cristã. Muitos
anos se tinham passado depois da fuga do convento de Iguaraçu; ninguém
mais o reconheceria. Um dia de manhã deixou a aldeia, com o pretexto de
ir caçar; foi primeiro ao arroio, desviou a pedra, abriu a igaçaba,
tirou o boião, onde deixara um resto do elixir. A ideia dele era fazer
analisar a droga na Europa, ou mesmo em Olinda ou no Recife, ou na
Bahia, por algum entendido em coisas de química e farmácia. Ao mesmo
tempo não podia furtar-se a um sentimento de gratidão; devia àquele
remédio a saúde. Com o boião ao lado, a mocidade nas pernas e a
resolução no peito, saiu dali, caminho de Olinda e da eternidade.





\section*{iv}



--- Não posso demorar-me em pormenores, disse o Dr.\,Leão aceitando o
café que o coronel mandara trazer. São quase dez horas\ldots{}

--- Que tem? perguntou o coronel. A noite é nossa; e, para o que temos
de fazer amanhã, podemos dormir quando bem nos parecer. Eu por mim não
tenho sono. E você, Sr.\,João Linhares?

--- Nem um pingo, respondeu o tabelião.

E teimou com o Dr.\,Leão para contar tudo, acrescentando que nunca ouvira
nada tão extraordinário. Note-se que o tabelião presumia ser lido em
histórias antigas, e passava na vila por um dos homens mais ilustrados
do Império; não obstante, estava pasmado. Ele contou ali mesmo, entre
dois goles de café, o caso de Matusalém, que viveu novecentos e sessenta
e nove anos, e o de Lameque\footnote{Matusalém e seu filho, Lameque, são
  personagens do ``Gênesis'', \emph{Antigo Testamento}. Teriam vivido
  centenas de anos.} que morreu com setecentos e setenta e sete; mas,
explicou logo, porque era um espírito forte, que esses e outros exemplos
da cronologia hebraica não tinham fundamento científico\ldots{}

--- Vamos, vamos ver agora o que aconteceu a seu pai, interrompeu o
coronel.

O vento, de esfalfado, morrera; e a chuva começava a rufar nas folhas
das árvores, a princípio com intermitências, depois mais contínua e
basta. A noite refrescou um pouco. O Dr.\,Leão continuou a narração, e,
apesar de dizer que não podia demorar-se nos pormenores, contou-os com
tanta miudeza, que não me atrevo a pô-los tais quais nestas páginas;
seria fastidioso. O melhor é resumi-lo.

Rui de Leão, ou antes Rui Garcia de Meireles e Castro Azevedo de Leão,
que assim se chamava o pai do médico, pouco tempo se demorou em
Pernambuco. Um ano depois, em 1654, cessava o domínio holandês. Rui de
Leão assistiu às alegrias da vitória, e passou-se ao reino, onde casou
com uma senhora nobre de Lisboa. Teve um filho; e perdeu o filho e a
mulher no mesmo mês de março de 1661. A dor que então padeceu foi
profunda; para distrair-se visitou a França e a Holanda. Mas na Holanda,
ou por motivo de uns amores secretos, ou por ódio de alguns judeus
descendentes ou naturais de Portugal, com quem entreteve relações
comerciais na Haia, ou enfim por outros motivos desconhecidos, Rui de
Leão não pôde viver tranquilo muito tempo; foi preso e conduzido para a
Alemanha, de onde passou à Hungria, a algumas cidades italianas, à
França, e finalmente à Inglaterra. Na Inglaterra estudou o inglês
profundamente; e, como sabia o latim, aprendido no convento, o hebraico,
que lhe ensinara na Haia o famoso Spinoza, de quem foi amigo, e que
talvez deu causa ao ódio que os outros judeus lhe criaram; --- o francês
e o italiano, parte do alemão e do húngaro, tornou-se em Londres objeto
de verdadeira curiosidade e veneração. Era buscado, consultado, ouvido,
não só por pessoas do vulgo ou idiotas, como por letrados, políticos e
personagens da Corte.

Convém dizer que em todos os países por onde andara tinha ele exercido
os mais contrários ofícios: soldado, advogado, sacristão, mestre de
dança, comerciante e livreiro. Chegou a ser agente secreto da Áustria,
guarda pontifício e armador de navios. Era ativo, engenhoso, mas pouco
persistente, a julgar pela variedade das coisas que empreendeu; ele,
porém, dizia que não, que a sorte é que sempre lhe foi adversa. Em
Londres, onde o vemos agora, limitou-se ao mister de letrado e gamenho;
mas não tardou que voltasse a Haia, onde o esperavam alguns dos amores
velhos, e não poucos recentes.

Que o amor, força é dizê-lo, foi uma das causas da vida agitada e
turbulenta do nosso herói. Ele era pessoalmente um homem galhardo,
insinuante, dotado de um olhar cheio de força e magia. Segundo ele mesmo
contou ao filho, deixou muito longe o algarismo dom-juanesco das
\emph{mille} e \emph{tre}\footnote{Don Juan é uma célebre personagem que
  aparece na Itália, durante o século \textsc{xvii}. Ele teria se relacionado
  intensa e fugazmente com mil e três mulheres, segundo a tradição.}.
Não podia dizer o número exato das mulheres a quem amara, em todas as
latitudes e línguas, desde a selvagem Maracujá de Pernambuco, até à bela
cipriota ou à fidalga dos salões de Paris e Londres; mas calculava em
não menos de cinco mil mulheres. Imagina-se facilmente que uma tal
multidão devia conter todos os gêneros possíveis da beleza feminil:
loiras, morenas, pálidas, coradas, altas, meãs, baixinhas, magras ou
cheias, ardentes ou lânguidas, ambiciosas, devotas, lascivas, poéticas,
prosaicas, inteligentes, estúpidas; --- sim, também estúpidas, e era
opinião dele que a estupidez das mulheres tinha o sexo feminino, era
graciosa, ao contrário da dos homens, que participava da aspereza viril.

--- Há casos, dizia ele, em que uma mulher estúpida tem o seu lugar.

Na Haia, entre os novos amores, deparou-se-lhe um que o prendeu por
longo tempo: \emph{lady} Emma Sterling, senhora inglesa, ou antes
escocesa, pois descendia de uma família de Dublin. Era formosa,
resoluta, e audaz; --- tão audaz que chegou a propor ao amante uma
expedição a Pernambuco para conquistar a capitania, e aclamarem-se reis
do novo Estado. Tinha dinheiro, podia levantar muito mais, chegou mesmo
a sondar alguns armadores e comerciantes, e antigos militares que ardiam
por uma desforra. Rui de Leão ficou aterrado com a proposta da amante, e
não lhe deu crédito; mas \emph{lady} Ema insistiu e mostrou-se tão de
rocha, que ele reconheceu enfim achar-se diante de uma ambiciosa
verdadeira. Era, todavia, homem de senso; viu que a empresa, por mais
bem organizada que fosse, não passaria de tentativa desgraçada;
disse-lho a ela; mostrou-lhe que, se a Holanda inteira tinha recuado,
não era fácil que um particular chegasse a obter ali domínio seguro, nem
ainda instantâneo. \emph{Lady} Ema abriu mão do plano, mas não perdeu a
ideia de o exalçar a alguma grande situação.

--- Tu serás rei ou duque\ldots{}

--- Ou cardeal, acrescentava ele rindo.

--- Por que não cardeal?

\emph{Lady} Ema fez com que Rui de Leão entrasse daí a pouco na
conspiração que deu em resultado a invasão da Inglaterra, a guerra
civil, e a morte enfim dos principais cabos da rebelião. Vencida esta,
\emph{lady} Ema não deu por vencida. Ocorreu-lhe então uma ideia
espantosa. Rui de Leão inculcava ser o próprio pai do duque de Monmouth,
suposto filho natural de Carlos \textsc{ii}, e caudilho principal dos rebeldes. A
verdade é que eram parecidos como duas gotas d'água. Outra verdade é que
\emph{lady} Ema, por ocasião da guerra civil, tinha o plano secreto de
fazer matar o duque, se ele triunfasse, e substituí-lo pelo amante, que
assim subiria ao trono de Inglaterra. O pernambucano, escusado é
dizê-lo, não soube de semelhante aleivosia, nem lhe daria o seu
assentimento. Entrou na rebelião, viu-a perecer ao sangue e no suplício,
e tratou de esconder-se. Ema acompanhou-o; e, como a esperança do cetro
não lhe saía do coração, passado algum tempo fez correr que o duque não
morrera, mas sim um amigo tão parecido com ele, e tão dedicado, que o
substituiu no suplício.

--- O duque está vivo, e dentro de pouco aparecerá ao nobre povo da
Grã-Bretanha, sussurrava ela aos ouvidos.

Quando Rui de Leão efetivamente apareceu, a estupefação foi grande, o
entusiasmo reviveu, o amor deu alma a uma causa, que o carrasco supunha
ter acabado na Torre de Londres. Donativos, presentes, armas,
defensores, tudo veio às mãos do audaz pernambucano, aclamado rei, e
rodeado logo de um troço de varões resolutos a morrer pela mesma causa.

--- Meu filho, --- disse ele, século e meio depois, ao médico
homeopata, --- dependeu de muito pouco não teres nascido príncipe de
Gales\ldots{} Cheguei a dominar cidades e vilas, expedi leis, nomeei
ministros, e, ainda assim, resisti a duas ou três sedições militares que
pediam a queda dos dois últimos gabinetes. Tenho para mim que as
dissensões internas ajudaram as forças legais, e devo-lhes a minha
derrota. Ao cabo, não me zanguei com elas; a luta fatigara-me; não minto
dizendo que o dia da minha captura foi para mim de alívio. Tinha visto,
além da primeira, duas guerras civis, uma dentro da outra, uma cruel,
outra ridícula, ambas insensatas. Por outro lado, vivera muito, e uma
vez que me não executassem, que me deixassem preso ou me exilassem para
os confins da terra, não pedia nada mais aos homens, ao menos durante
alguns séculos\ldots{} Fui preso, julgado e condenado à morte. Dos meus
auxiliares não poucos negaram tudo; creio mesmo que um dos principais
morreu na Câmara dos \emph{Lords}. Tamanha ingratidão foi um princípio
de suplício. Ema, não; essa nobre senhora não me abandonou; foi presa,
condenada, e perdoada; mas não me abandonou. Na véspera de minha
execução, veio ter comigo, e passamos juntos as últimas horas. Disse-lhe
que não me esquecesse, dei-lhe uma trança de cabelos, pedi-lhe que
perdoasse ao carrasco\ldots{} Ema prorrompeu em soluços; os guardas vieram
buscá-la. Ficando só, recapitulei a minha vida, desde Iguaraçu até a
Torre de Londres. Estávamos então em 1686; tinha eu oitenta e seis anos,
sem parecer mais de quarenta. A aparência era a da eterna juventude; mas
o carrasco ia destruí-la num instante. Não valia a pena ter bebido
metade do elixir e guardado comigo o misterioso boião, para acabar
tragicamente no cepo do cadafalso\ldots{} Tais foram as minhas ideias naquela
noite. De manhã preparei-me para a morte. Veio o padre, vieram os
soldados, e o carrasco. Obedeci maquinalmente. Caminhamos todos, subi ao
cadafalso, não fiz discurso; inclinei o pescoço sobre o cepo, o carrasco
deixou cair a arma, senti uma dor penetrante, uma angústia enorme, como
que a parada súbita do coração; mas essa sensação foi tão grande como
rápida; no instante seguinte tornara ao estado natural. Tinha no pescoço
algum sangue, mas pouco e quase seco. O carrasco recuou, o povo bramiu
que me matassem. Inclinaram-me a cabeça, e o carrasco, fazendo apelo a
todos os seus músculos e princípios, descarregou outro golpe, e maior,
se é possível, capaz de abrir-me ao mesmo tempo a sepultura, como já se
disse de um valente. A minha sensação foi igual à primeira na
intensidade e na brevidade; reergui a cabeça. Nem o magistrado nem o
padre consentiram que se desse outro golpe. O povo abalou-se, uns
chamaram-me santo, outros diabo, e ambas essas opiniões eram defendidas
nas tabernas à força de punho e de aguardente. Diabo ou santo, fui
presente aos médicos da corte. Estes ouviram o depoimento do magistrado,
do padre, do carrasco, de alguns soldados, e concluíram que, uma vez
dado o golpe, os tecidos do pescoço ligavam-se outra vez rapidamente, e
assim os mesmos ossos, e não chegavam a explicar um tal fenômeno. Pela
minha parte, em vez de contar o caso do elixir, calei-me; preferi
aproveitar as vantagens do mistério. Sim, meu filho; não imaginas a
impressão de toda a Inglaterra, os bilhetes amorosos que recebi das mais
finas duquesas, os versos, as flores, os presentes, as metáforas. Um
poeta chamou-me Anteu\footnote{Filho de Posídon e Gaia, com proporções
  de gigante.}. Um jovem protestante demonstrou-me que eu era o mesmo
Cristo.



\section*{V}



O narrador continuou:

--- Já veem, pelo que lhes contei, que não acabaria hoje nem em toda
esta semana, se quisesse referir miudamente a vida inteira de meu pai.
Algum dia o farei, mas por escrito, e cuido que a obra dará cinco
volumes, sem contar os documentos\ldots{}

--- Que documentos? perguntou o tabelião.

--- Os muitos documentos comprobatórios que possuo, títulos, cartas,
traslados de sentenças, de escrituras, cópias de estatísticas\ldots{} Por
exemplo, tenho uma certidão do recenseamento de um certo bairro de
Gênova, onde meu pai morreu em 1742; traz o nome dele, com declaração do
lugar em que nasceu\ldots{}

--- E com a verdadeira idade? perguntou o coronel.

--- Não. Meu pai andou sempre entre os quarenta e os cinquenta. Chegando
aos cinquenta, cinquenta e poucos, voltava para trás; --- e era-lhe
fácil fazer isto, porque não esquentava lugar; vivia cinco, oito, dez,
doze anos numa cidade, e passava a outra\ldots{} Pois tenho muitos documentos
que juntarei, entre outros, o testamento de \emph{lady} Ema, que morreu
pouco depois da execução gorada de meu pai. Meu pai dizia-me que entre
as muitas saudades que a vida lhe ia deixando, \emph{lady} Ema era das
mais fortes e profundas. Nunca viu mulher mais sublime, nem amor mais
constante, nem dedicação mais cega. E a morte confirmou a vida, porque o
herdeiro de \emph{lady} Ema foi meu pai. Infelizmente, a herança teve
outros reclamantes, e o testamento entrou em processo. Meu pai, não
podendo residir em Inglaterra, concordou na proposta de um amigo
providencial que veio a Lisboa dizer-lhe que tudo estava perdido; quando
muito poderia salvar um restozinho de nada, e ofereceu-lhe por esse
direito problemático uns dez mil cruzados. Meu pai aceitou-os; mas, tão
caipora que o testamento foi aprovado, e a herança passou às mãos do
comprador\ldots{}

--- E seu pai ficou pobre\ldots{}

--- Com os dez mil cruzados, e pouco mais que apurou. Teve então ideia
de meter-se no negócio de escravos; obteve privilégio, armou um navio, e
transportou africanos para o Brasil. Foi a parte da vida que mais lhe
custou; mas afinal acostumou-se às tristes obrigações de um navio
negreiro. Acostumou-se, e enfarou-se, que era outro fenômeno na vida
dele. Enfarava-se dos ofícios. As longas solidões do mar alargaram-lhe o
vazio interior. Um dia refletiu, e perguntou a si mesmo, se chegaria a
habituar-se tanto à navegação, que tivesse de varrer o oceano, por todos
os séculos dos séculos. Criou medo; e compreendeu que o melhor modo de
atravessar a eternidade era variá-la\ldots{}

--- Em que ano ia ele?

--- Em 1694; fins de 1694.

--- Veja só! Tinha então noventa e quatro anos, não era? Naturalmente,
moço\ldots{}

--- Tão moço que casou daí a dois anos, na Bahia, com uma bela senhora
que\ldots{}

--- Diga.

--- Digo, sim; porque ele mesmo me contou a história. Uma senhora que
amou a outro. E que outro! Imaginem que meu pai, em 1695, entrou na
conquista da famosa república dos Palmares\footnote{Uma das denominações
  do \emph{Quilombo dos Palmares}, localizado na então capitania de
  Pernambuco durante o século \textsc{xvii}.}. Bateu-se como um bravo, e perdeu
um amigo, um amigo íntimo, crivado de balas, pelado\ldots{}

--- Pelado?

--- É verdade; os negros defendiam-se também com água fervendo, e este
amigo recebeu um pote cheio; ficou uma chaga. Meu pai contava-me esse
episódio com dor, e até com remorso, porque, no meio da refrega, teve de
pisar o pobre companheiro; parece até que ele expirou quando meu pai lhe
metia as botas na cara\ldots{}

O tabelião fez uma careta; e o coronel, para disfarçar o horror,
perguntou o que tinha a conquista dos Palmares com a mulher que\ldots{}

--- Tem tudo, continuou o médico. Meu pai, ao tempo que via morrer um
amigo, salvara a vida de um oficial, recebendo ele mesmo uma flecha no
peito. O caso foi assim. Um dos negros, depois de derrubar dois
soldados, envergou o arco sobre a pessoa do oficial, que era um rapaz
valente e simpático, órfão de pai, tendo deixado a mãe em Olinda\ldots{} Meu
pai compreendeu que a flecha não lhe faria mal a ele, e então, de um
salto, interpôs-se. O golpe feriu-o no peito; ele caiu. O oficial,
Damião\ldots{} Damião de tal. Não digo o nome todo, porque ele tem alguns
descendentes para as bandas de Minas. Damião basta. Damião passou a
noite ao pé da cama de meu pai, agradecido, dedicado, louvando-lhe uma
ação tão sublime. E chorava. Não podia suportar a ideia de ver morrer o
homem que lhe salvara a vida por um modo tão raro. Meu pai sarou
depressa, com pasmo de todos. A pobre mãe do oficial quis beijar-lhe as
mãos: --- ``Basta-me um prêmio, disse ele; a sua amizade e a do seu
filho''. O caso encheu de pasmo Olinda inteira. Não se falava em outra
coisa; e daí a algumas semanas a admiração pública trabalhava em fazer
uma lenda. O sacrifício, como veem, era nenhum, pois meu pai não podia
morrer; mas o povo, que não sabia disso, buscou uma causa ao sacrifício,
uma causa tão grande como ele, e descobriu que o Damião devia ser filho
de meu pai, e naturalmente filho adúltero. Investigaram o passado da
viúva; acharam alguns recantos que se perdiam na obscuridade. O rosto de
meu pai entrou a parecer conhecido de alguns; não faltou mesmo quem
afirmasse ter ido a uma merenda, vinte anos antes, em casa da viúva, que
era então casada, e visto aí meu pai. Todas estas patranhas aborreceram
tanto a meu pai, que ele determinou passar à Bahia, onde casou\ldots{}

--- Com a tal senhora?

--- Justamente\ldots{} Casou com D.\,Helena, bela como o sol, dizia ele. Um
ano depois morria em Olinda a viúva, e o Damião vinha à Bahia trazer a
meu pai uma madeixa dos cabelos da mãe, e um colar que a moribunda pedia
para ser usado pela mulher dele. D.\,Helena soube do episódio da flecha,
e agradeceu a lembrança da morta. Damião quis voltar para Olinda; meu
pai disse-lhe que não, que fosse no ano seguinte. Damião ficou. Três
meses depois uma paixão desordenada\ldots{} Meu pai soube da aleivosia de
ambos, por um comensal da casa. Quis matá-los; mas o mesmo que os
denunciou avisou-os do perigo, e eles puderam evitar a morte. Meu pai
voltou o punhal contra si, e enterrou-o no coração.

``Filho, dizia-me ele, contando o episódio; dei seis golpes, cada um dos
quais bastava para matar um homem, e não morri.'' Desesperado saiu de
casa, e atirou-se ao mar. O mar restituiu-o à terra. A morte não podia
aceitá-lo: ele pertencia à vida por todos os séculos. Não teve outro
recurso mais do que fugir; veio para o Sul, onde alguns anos depois, no
princípio do século passado, podemos achá-lo na descoberta das minas.
Era um modo de afogar o desespero, que era grande, pois amara muito a
mulher, como um louco\ldots{}

--- E ela?

--- São contos largos, e não me sobra tempo. Ela veio ao Rio de Janeiro,
depois das duas invasões francesas; creio que em 1713\footnote{Os
  franceses ocuparam o território brasileiro, possessão de Portugal, em
  pelos menos três oportunidades: entre 1555 e 1567 (na região que hoje
  corresponde à Baía de Guanabara); entre 1612 e 1615 (em São Luís, no
  antigo Estado do Maranhão e Grão Pará); e entre 1710 e 1711 (atual
  cidade do Rio de Janeiro).}. Já então meu pai enriquecera com as
minas, e residia na cidade fluminense, benquisto, com ideias até de ser
nomeado governador. D.\,Helena apareceu-lhe, acompanhada da mãe e de um
tio. Mãe e tio vieram dizer-lhe que era tempo de acabar com a situação
em que meu pai tinha colocado a mulher. A calúnia pesara longamente
sobre a vida da pobre senhora. Os cabelos iam-lhe embranquecendo: não
era só a idade que chegava, eram principalmente os desgostos, as
lágrimas. Mostraram-lhe uma carta escrita pelo comensal denunciante,
pedindo perdão a D.\,Helena da calúnia que lhe levantara e confessando
que o fizera levado de uma criminosa paixão. Meu pai era uma boa alma;
aceitou a mulher, a sogra e o tio. Os anos fizeram o seu ofício; todos
três envelheceram, menos meu pai. Helena ficou com a cabeça toda branca;
a mãe e o tio voavam para a decrepitude; e nenhum deles tirava os olhos
de meu pai, espreitando as cãs que não vinham, e as rugas ausentes. Um
dia meu pai ouviu-lhes dizer que ele devia ter parte com o diabo. Tão
forte! E acrescentava o tio: ``De que serve o testamento, se temos de ir
antes?'' Duas semanas depois morria o tio; a sogra acabou pateta, daí a
um ano. Restava a mulher, que pouco mais durou.

--- O que me parece, aventurou o coronel, é que eles vieram ao cheiro
dos cobres\ldots{}

--- Decerto.

--- \ldots{} e que a tal D.\,Helena (Deus lhe perdoe!) não estava tão inocente
como dizia. É verdade que a carta do denunciante\ldots{}

--- O denunciante foi pago para escrever a carta, explicou o Dr.\,Leão;
meu pai soube disso, depois da morte da mulher, ao passar pela Bahia\ldots{}
Meia-noite! Vamos dormir; é tarde; amanhã direi o resto.

--- Não, não, agora mesmo.

--- Mas, senhores\ldots{} Só se for muito por alto.

--- Seja por alto.

O doutor levantou-se e foi espiar a noite, estendendo o braço para fora,
e recebendo alguns pingos de chuva na mão. Depois voltou-se e deu com os
dois olhando um para o outro, interrogativos. Fez lentamente um cigarro,
acendeu-o, e, puxadas umas três fumaças, concluiu a singular história.





\section*{vi}



--- Meu pai deixou pouco depois o Brasil, foi a Lisboa, e dali passou-se
à Índia, onde se demorou mais de cinco anos, e donde voltou a Portugal,
com alguns estudos feitos acerca daquela parte do mundo. Deu-lhes a
última lima, e fê-los imprimir, tão a tempo, que o governo mandou-o
chamar para entregar-lhe o governo de Goa\footnote{Estado situado, hoje,
  na região oeste da Índia. A personagem refere-se ao período histórico
  quando Goa era uma das possessões portuguesas, no globo.}. Um
candidato ao cargo, logo que soube do caso, pôs em ação todos os meios
possíveis e impossíveis. Empenhos, intrigas, maledicência, tudo lhe
servia de arma. Chegou a obter, por dinheiro, que um dos melhores
latinistas da península, homem sem escrúpulos, forjasse um texto latino
da obra de meu pai, e o atribuísse a um frade agostinho\footnote{Referência
  à católica Ordem de Santo Agostinho, fundada em 1243 pelo Papa
  Inocêncio \textsc{iv}.}, morto em Adém\footnote{Cidade que se localizaria,
  atualmente, no Iêmen.}. E a tacha de plagiário acabou de eliminar meu
pai, que perdeu o governo de Goa, o qual passou às mãos do outro;
perdendo também, o que é mais, toda a consideração pessoal. Ele escreveu
uma longa justificação, mandou cartas para a Índia, cujas respostas não
esperou, porque no meio desses trabalhos, aborreceu-se tanto, que
entendeu melhor deixar tudo, e sair de Lisboa. Esta geração passa, disse
ele, e eu fico. Voltarei cá daqui a um século, ou dois.

--- Veja isto, interrompeu o tabelião, parece coisa de caçoada! Voltar
daí a um século --- ou dois, como se fosse um ou dois meses. Que diz,
``seu'' coronel?

--- Ah! eu quisera ser esse homem! É verdade que ele não voltou um
século depois\ldots{} Ou voltou?

--- Ouça-me. Saiu dali para Madri, onde esteve de amores com duas
fidalgas, uma delas viúva e bonita como o sol, a outra casada, menos
bela, porém amorosa e terna como uma pomba-rola. O marido desta chegou a
descobrir o caso, e não quis bater-se com meu pai, que não era nobre;
mas a paixão do ciúme e da honra levou esse homem ofendido à prática de
uma aleivosia, igual à outra: mandou assassinar meu pai; os esbirros
deram-lhe três punhaladas e quinze dias de cama. Restabelecido,
deram-lhe um tiro; foi o mesmo que nada. Então, o marido achou um meio
de eliminar meu pai; tinha visto com ele alguns objetos, notas, e
desenhos de coisas religiosas da Índia, e denunciou-o ao Santo
Ofício\footnote{Na Espanha, a Inquisição, a cargo do Tribunal do Santo
  Ofício, foi autorizada mediante bula do Papa Sixto \textsc{iv} em 1478. Em
  Portugal, a instituição passou a atuar em 1536, autorizada pelo papa
  Paulo \textsc{iii}.}, como dado a práticas supersticiosas. O Santo Ofício, que
não era omisso nem frouxo nos seus deveres, tomou conta dele, e
condenou-o a cárcere perpétuo. Meu pai ficou aterrado. Na verdade, a
prisão perpétua para ele devia ser a coisa mais horrorosa do mundo.
Prometeu\footnote{O mito de Prometeu é assunto da tragédia
  \emph{Prometeu Acorrentado}, de Ésquilo. De acordo com a lenda, o titã
  foi punido por roubar o fogo primordial (que deu a vida aos homens) a
  Zeus. Acorrentado no alto de uma montanha inacessível, teve o fígado
  bicado por uma águia durante anos, até que Quíron (um centauro) o
  libertou das correntes que o aprisionavam.}, o mesmo Prometeu foi
desencadeado\ldots{} Não me interrompa, Sr.\,Linhares, depois direi quem foi
esse Prometeu. Mas, repito: ele foi desencadeado, enquanto que meu pai
estava nas mãos do Santo Ofício, sem esperança. Por outro lado, ele
refletiu consigo que, se era eterno, não o era o Santo Ofício. O Santo
Ofício há de acabar um dia, e os seus cárceres, e então ficarei livre.
Depois, pensou também que, desde que passasse um certo número de anos,
sem envelhecer nem morrer, tornar-se-ia um caso tão extraordinário, que
o mesmo Santo Ofício lhe abriria as portas. Finalmente, cedeu a outra
consideração. ``Meu filho, disse-me ele, eu tinha padecido tanto
naqueles longos anos de vida, tinha visto tanta paixão má, tanta
miséria, tanta calamidade, que agradeci a Deus, o cárcere e uma longa
prisão; e disse comigo que o Santo Ofício não era tão mau, pois que me
retirava por algumas dezenas de anos, talvez um século, do espetáculo
exterior\ldots{}''

--- Ora essa!

--- Coitado! Não contava com a outra fidalga, a viúva, que pôs em campo
todos os recursos de que podia dispor, e alcançou-lhe a fuga daí a
poucos meses. Saíram ambos de Espanha, meteram-se em França, e passaram
à Itália, onde meu pai ficou residindo por longos anos. A viúva
morreu-lhe nos braços; e, salvo uma paixão que teve em Florença, por um
rapaz nobre, com quem fugiu e esteve seis meses, foi sempre fiel ao
amante. Repito, morreu-lhe nos braços, e ele padeceu muito, chorou
muito, chegou a querer morrer também. Contou-me os atos de desespero que
praticou; porque, na verdade, amara muito a formosa madrilena.
Desesperado, meteu-se a caminho, e viajou por Hungria, Dalmácia,
Valáquia; esteve cinco anos em Constantinopla; estudou o turco a fundo,
e depois o árabe. Já lhes disse que ele sabia muitas línguas; lembra-me
de o ver traduzir o padre-nosso em cinquenta idiomas diversos. Sabia
muito. E ciências! Meu pai sabia uma infinidade de coisas: filosofia,
jurisprudência, teologia, arqueologia, química, física, matemáticas,
astronomia, botânica; sabia arquitetura, pintura, música. Sabia o diabo.

--- Na verdade\ldots{}

--- Muito, sabia muito. E fez mais do que estudar o turco; adotou o
maometanismo. Mas deixou-o daí a pouco. Enfim, aborreceu-se dos turcos:
era a sina dele aborrecer-se facilmente de uma coisa ou de um ofício.
Saiu de Constantinopla, visitou outras partes da Europa, e finalmente
passou-se a Inglaterra aonde não fora desde longos anos. Aconteceu-lhe
aí o que lhe acontecia em toda a parte: achou todas as caras novas; e
essa troca de caras no meio de uma cidade, que era a mesma deixada por
ele, dava-lhe a impressão de uma peça teatral, em que o cenário não
muda, e só mudam os atores. Essa impressão, que a princípio foi só de
pasmo, passou a ser de tédio; mas agora, em Londres, foi outra coisa
pior, porque despertou nele uma ideia, que nunca tivera, uma ideia
extraordinária, pavorosa\ldots{}

--- Que foi?

--- A ideia de ficar doido um dia. Imaginem: um doido eterno. A comoção
que esta ideia lhe dava foi tal que quase enlouqueceu ali mesmo. Então
lembrou-se de outra coisa. Como tinha o boião do elixir consigo, lembrou
de dar o resto a alguma senhora ou homem, e ficariam os dois imortais.
Sempre era uma companhia. Mas, como tinha tempo diante de si, não
precipitou nada; achou melhor esperar pessoa cabal. O certo é que essa
ideia o tranquilizou\ldots{} Se lhe contasse as aventuras que ele teve outra
vez na Inglaterra, e depois em França, e no Brasil, onde voltou no
vice-reinado do Conde de Resende\footnote{O Vice-Reinado do português
  José Luís de Castro (1744--1819), 2.º Conde de Resende, durou de 1790 a
  1801.}, não acabava mais, e o tempo urge, além do que o Sr.\,Coronel
está com sono\ldots{}

--- Qual sono!

--- Pelo menos está cansado.

--- Nem isso. Se eu nunca ouvi uma coisa que me interessasse tanto.
Vamos; conte essas aventuras.

--- Não; direi somente que ele achou-se em França por ocasião da
revolução de 1789\footnote{Iniciada em 1789, a Revolução Francesa se
  estendeu até 1799, quando Napoleão Bonaparte assume o poder.},
assistiu a tudo, à queda e morte do rei, dos girondinos, de Danton, de
Robespierre; morou algum tempo com Filinto Elísio\footnote{Cognome do
  poeta português Francisco Manuel do Nascimento (1734--1819), muito
  popular durante o século \textsc{xix}.}, o poeta, sabem? Morou com ele em
Paris; foi um dos elegantes do Diretório, deu-se com o primeiro
Cônsul\footnote{Napoleão Bonaparte, que assumiu o Consulado Francês em
  1799.}\ldots{} Quis até naturalizar-se e seguir as armas e a política;
podia ter sido um dos marechais do império, e pode ser até que não
tivesse havido Waterloo\footnote{Encerrada em 18 de junho de 1815, a
  batalha de Waterloo trouxe vitória aos ingleses e prussianos, que
  enfrentavam Napoleão. O combate representou o fim de seu império e
  assinala um novo período social e político na França.}. Mas ficou tão
enjoado de algumas apostasias políticas, e tão indignado, que recusou a
tempo. Em 1808 achamo-lo em viagem com a corte real para o Rio de
Janeiro. Em 1822 saudou a independência; e fez parte da Constituinte;
trabalhou no 7 de Abril\footnote{Grafado com inicial maiúscula, na
  edição original.}; festejou a maioridade\footnote{A maioridade de
  Pedro \textsc{ii} foi decretada em 23 de julho de 1840. Ele foi coroado
  Imperador em 18 de julho de 1841.}; há dois anos era deputado.

Neste ponto os dois ouvintes redobraram de atenção. Compreenderam que
iam chegar ao desenlace, e não quiseram perder uma sílaba daquela parte
da narração, em que iam saber da morte do imortal. Pela sua parte, o Dr.\,Leão parara um pouco; podia ser uma lembrança dolorosa; podia também ser
um recurso para aguçar mais o apetite. O tabelião ainda lhe perguntou,
se o pai não tinha dado a alguém o resto do elixir, como queria; mas o
narrador não lhe respondeu nada. Olhava para dentro; enfim, terminou
deste modo:

--- A alma de meu pai chegara a um grau de profunda melancolia. Nada o
contentava; nem o sabor da glória, nem o sabor do perigo, nem o do amor.
Tinha então perdido minha mãe, e vivíamos juntos, como dois solteirões.
A política perdera todos os encantos aos olhos dum homem que pleiteara
um trono, e um dos primeiros do universo. Vegetava consigo; triste,
impaciente, enjoado. Nas horas mais alegres fazia projetos para o século
\textsc{xx} e \textsc{xxiv}, porque já então me desvendara todo o segredo da vida dele.
Não acreditei, confesso; e imaginei que fosse alguma perturbação mental;
mas as provas foram completas, e demais a observação mostrou-me que ele
estava em plena saúde. Só o espírito, como digo, parecia abatido e
desencantado. Um dia, dizendo-lhe eu que não compreendia tamanha
tristeza, quando eu daria a alma ao diabo para ter a vida eterna, meu
pai sorriu com uma tal expressão de superioridade, que me enterrou cem
palmos abaixo do chão. Depois, respondeu que eu não sabia o que dizia;
que a vida eterna afigurava-se-me excelente, justamente porque a minha
era limitada e curta; em verdade, era o mais atroz dos suplícios. Tinha
visto morrer todas as suas afeições; devia perder-me um dia, e todos os
mais filhos que tivesse pelos séculos adiante. Outras afeições e não
poucas o tinham enganado; e umas e outras, boas e más, sinceras e
pérfidas, era-lhe forçoso repeti-las, sem trégua, sem um respiro ao
menos, porquanto, a experiência não lhe podia valer contra a necessidade
de agarrar-se a alguma coisa, naquela passagem rápida dos homens e das
gerações. Era uma necessidade da vida eterna; sem ela, cairia na
demência. Tinha provado tudo, esgotado tudo; agora era a repetição, a
monotonia, sem esperanças, sem nada. Tinha de relatar a outros filhos,
vinte ou trinta séculos mais tarde, o que me estava agora dizendo; e
depois a outros, e outros, e outros, um não acabar mais nunca. Tinha de
estudar novas línguas, como faria Aníbal\footnote{Aníbal, de Cartago
  (247--183 a.C.), estadista e general romano, reconhecido por ter
  desbravado o Mediterrâneo.}, se vivesse até hoje: e para quê? para
ouvir os mesmos sentimentos, as mesmas paixões\ldots{} E dizia-me tudo isso,
verdadeiramente abatido. Não parece esquisito? Enfim um dia, como eu
fizesse a alguns amigos uma exposição do sistema homeopático, vi reluzir
nos olhos de meu pai um fogo desusado e extraordinário. Não me disse
nada. De noite, vieram chamar-me ao quarto dele. Achei-o moribundo;
disse-me então, com a língua trôpega, que o princípio homeopático fora
para ele a salvação. \emph{Similia similibus curantur}\footnote{Sentença
  atribuída ao grego Hipócrates (460--370 a.C.), considerado pai da
  Medicina, que se poderia traduzir como cura de um mal pelo emprego de
  substância ou medida semelhante à doença.}. Bebera o resto do elixir,
e assim como a primeira metade lhe dera a vida, a segunda dava-lhe a
morte. E, dito isto, expirou.

O coronel e o tabelião ficaram algum tempo calados, sem saber que
pensassem da famosa história; mas a seriedade do médico era tão
profunda, que não havia duvidar. Creram no caso, e creram também
definitivamente na homeopatia. Narrada a história a outras pessoas, não
faltou quem supusesse que o médico era louco; outros atribuíram-lhe o
intuito de tirar ao coronel e ao tabelião o desgosto manifestado por
ambos de não poderem viver eternamente, mostrando-lhes que a morte é,
enfim, um benefício. Mas a suspeita de que ele apenas quis propagar a
homeopatia entrou em alguns cérebros, e não era inverossímil. Dou este
problema aos estudiosos. Tal é o caso extraordinário, que há anos, com
outro nome, e por outras palavras, contei a este bom povo, que
provavelmente já os esqueceu a ambos. 


\chapter{A cartomante\footnote[*]{O conto ``A cartomante'' foi incluído por
  Machado de Assis na coletânea \emph{Várias Histórias}, publicada em
  1896. O texto desta edição foi cotejado com o daquela.}}

Hamlet observa a Horácio que há mais coisas no céu e na terra do que
sonha a nossa filosofia\footnote{Alusão à peça \emph{Hamlet}, de William
  Shakespeare (1564--1616), publicada em 1603.}. Era a mesma explicação
que dava a bela Rita ao moço Camilo, numa sexta-feira de novembro de
1869, quando este ria dela, por ter ido na véspera consultar uma
cartomante; a diferença é que o fazia por outras palavras.

--- Ria, ria. Os homens são assim; não acreditam em nada. Pois saiba que
fui, e que ela adivinhou o motivo da consulta, antes mesmo que eu lhe
dissesse o que era. Apenas começou a botar as cartas, disse-me: ``A
senhora gosta de uma pessoa\ldots{}''. Confessei que sim, e então ela
continuou a botar as cartas, combinou-as, e no fim declarou-me que eu
tinha medo de que você me esquecesse, mas que não era verdade\ldots{}

--- Errou! --- interrompeu Camilo, rindo.

--- Não diga isso, Camilo. Se você soubesse como eu tenho andado, por sua
causa. Você sabe; já lhe disse. Não ria de mim, não ria\ldots{}

Camilo pegou-lhe nas mãos, e olhou para ela sério e fixo. Jurou que lhe
queria muito, que os seus sustos pareciam de criança; em todo o caso,
quando tivesse algum receio, a melhor cartomante era ele mesmo. Depois,
repreendeu-a; disse-lhe que era imprudente andar por essas casas. Vilela
podia sabê-lo, e depois\ldots{}

--- Qual saber! Tive muita cautela, ao entrar na casa.

--- Onde é a casa?

--- Aqui perto, na rua da Guarda Velha; não passava ninguém nessa
ocasião. Descansa; eu não sou maluca.

Camilo riu outra vez:

--- Tu crês deveras nessas coisas? perguntou-lhe.

Foi então que ela, sem saber que traduzia Hamlet em vulgar, disse-lhe
que havia muita coisa misteriosa e verdadeira neste mundo. Se ele não
acreditava, paciência; mas o certo é que a cartomante adivinhara tudo.
Que mais? A prova é que ela agora estava tranquila e satisfeita.

Cuido que ele ia falar, mas reprimiu-se. Não queria arrancar-lhe as
ilusões. Também ele, em criança, e ainda depois, foi supersticioso, teve
um arsenal inteiro de crendices, que a mãe lhe incutiu e que aos vinte
anos desapareceram. No dia em que deixou cair toda essa vegetação
parasita, e ficou só o tronco da religião, ele, como tivesse recebido da
mãe ambos os ensinos, envolveu-os na mesma dúvida, e logo depois em uma
só negação total. Camilo não acreditava em nada. Por quê? Não poderia
dizê-lo, não possuía um só argumento; limitava-se a negar tudo. E digo
mal, porque negar é ainda afirmar, e ele não formulava a incredulidade;
diante do mistério, contentou-se em levantar os ombros, e foi andando.

Separaram-se contentes, ele ainda mais que ela. Rita estava certa de ser
amada; Camilo não só o estava, mas via-a estremecer e arriscar-se por
ele, correr às cartomantes, e, por mais que a repreendesse, não podia
deixar de sentir-se lisonjeado. A casa do encontro era na antiga rua dos
Barbonos, onde morava uma coprovinciana de Rita. Esta desceu pela rua
das Mangueiras, na direção de Botafogo, onde residia; Camilo desceu pela
da Guarda Velha, olhando de passagem para a casa da cartomante.

Vilela, Camilo e Rita, três nomes, uma aventura, e nenhuma explicação
das origens. Vamos a ela. Os dois primeiros eram amigos de infância.
Vilela seguiu a carreira de magistrado. Camilo entrou no funcionalismo,
contra a vontade do pai, que queria vê-lo médico; mas o pai morreu, e
Camilo preferiu não ser nada, até que a mãe lhe arranjou um emprego
público. No princípio de 1869, voltou Vilela da província, onde casara
com uma dama formosa e tonta; abandonou a magistratura e veio abrir
banca de advogado. Camilo arranjou-lhe casa para os lados de Botafogo, e
foi a bordo recebê-lo.

--- É o senhor? exclamou Rita, estendendo-lhe a mão. Não imagina como meu
marido é seu amigo, falava sempre do senhor.

Camilo e Vilela olharam-se com ternura. Eram amigos deveras. Depois,
Camilo confessou de si para si que a mulher do Vilela não desmentia as
cartas do marido. Realmente, era graciosa e viva nos gestos, olhos
cálidos, boca fina e interrogativa. Era um pouco mais velha que ambos:
contava trinta anos, Vilela, vinte e nove e Camilo, vinte e seis.
Entretanto, o porte grave de Vilela fazia-o parecer mais velho que a
mulher, enquanto Camilo era um ingênuo na vida moral e prática.
Faltava-lhe tanto a ação do tempo, como os óculos de cristal, que a
natureza põe no berço de alguns para adiantar os anos. Nem experiência,
nem intuição.

Uniram-se os três. Convivência trouxe intimidade. Pouco depois morreu a
mãe de Camilo, e nesse desastre, que o foi, os dois mostraram-se grandes
amigos dele. Vilela cuidou do enterro, dos sufrágios e do inventário;
Rita tratou especialmente do coração, e ninguém o faria melhor.

Como daí chegaram ao amor, não o soube ele nunca. A verdade é que
gostava de passar as horas ao lado dela; era a sua enfermeira moral,
quase uma irmã, mas principalmente era mulher e bonita. \emph{Odor di
femmina}\footnote{Do italiano, aroma de mulher.}: eis o que ele aspirava
nela, e em volta dela, para incorporá-lo em si próprio. Liam os mesmos
livros, iam juntos a teatros e passeios. Camilo ensinou-lhe as damas e o
xadrez e jogavam às noites; --- ela, mal, --- ele, para lhe ser agradável,
pouco menos mal. Até aí as coisas. Agora a ação da pessoa, os olhos
teimosos de Rita, que procuravam muita vez os dele, que os consultavam
antes de o fazer ao marido, as mãos frias, as atitudes insólitas. Um
dia, fazendo ele anos, recebeu de Vilela uma rica bengala de presente, e
de Rita apenas um cartão com um vulgar cumprimento a lápis, e foi então
que ele pôde ler no próprio coração; não conseguia arrancar os olhos do
bilhetinho. Palavras vulgares; mas há vulgaridades sublimes, ou, pelo
menos, deleitosas. A velha caleça\footnote{Carruagem alugada (nas praças
  do Rio de Janeiro).} de praça, em que pela primeira vez passeaste com
a mulher amada, fechadinhos ambos, vale o carro de Apolo\footnote{Veículo
  que seria utilizado pelo Deus Apolo, segundo a mitologia grega, na
  Antiguidade.}. Assim é o homem, assim são as coisas que o cercam.

Camilo quis sinceramente fugir, mas já não pôde. Rita, como uma
serpente, foi-se acercando dele, envolveu-o todo, fez-lhe estalar os
ossos num espasmo, e pingou-lhe o veneno na boca. Ele ficou atordoado e
subjugado. Vexame, sustos, remorsos, desejos, tudo sentiu de mistura;
mas a batalha foi curta e a vitória, delirante. Adeus, escrúpulos! Não
tardou que o sapato se acomodasse ao pé, e aí foram ambos, estrada fora,
braços dados, pisando folgadamente por cima de ervas e pedregulhos, sem
padecer nada mais que algumas saudades, quando estavam ausentes um do
outro. A confiança e estima de Vilela continuavam a ser as mesmas.

Um dia, porém, recebeu Camilo uma carta anônima, que lhe chamava imoral
e pérfido, e dizia que a aventura era sabida de todos. Camilo teve medo,
e, para desviar as suspeitas, começou a rarear as visitas à casa de
Vilela. Este notou-lhe as ausências. Camilo respondeu que o motivo era
uma paixão frívola de rapaz. Candura gerou astúcia. As ausências
prolongaram-se, e as visitas cessaram inteiramente. Pode ser que
entrasse também nisso um pouco de amor-próprio, uma intenção de diminuir
os obséquios do marido, para tornar menos dura a aleivosia do ato.

Foi por esse tempo que Rita, desconfiada e medrosa, correu à cartomante
para consultá-la sobre a verdadeira causa do procedimento de Camilo.
Vimos que a cartomante restituiu-lhe a confiança, e que o rapaz
repreendeu-a por ter feito o que fez. Correram ainda algumas semanas.
Camilo recebeu mais duas ou três cartas anônimas, tão apaixonadas, que
não podiam ser advertência da virtude, mas despeito de algum
pretendente; tal foi a opinião de Rita, que, por outras palavras mal
compostas, formulou este pensamento: --- a virtude é preguiçosa e avara,
não gasta tempo nem papel; só o interesse é ativo e pródigo.

Nem por isso Camilo ficou mais sossegado; temia que o anônimo fosse ter
com Vilela, e a catástrofe viria então sem remédio. Rita concordou que
era possível.

--- Bem, disse ela; eu levo os sobrescritos para comparar a letra com a
das cartas que lá aparecerem; se alguma for igual, guardo-a e
rasgo-a\ldots{}

Nenhuma apareceu; mas daí a algum tempo Vilela começou a mostrar-se
sombrio, falando pouco, como desconfiado. Rita deu-se pressa em dizê-lo
ao outro, e sobre isso deliberaram. A opinião dela é que Camilo devia
tornar à casa deles, tatear o marido, e pode ser até que lhe ouvisse a
confidência de algum negócio particular. Camilo divergia; aparecer
depois de tantos meses era confirmar a suspeita ou denúncia. Mais valia
acautelarem-se, sacrificando-se por algumas semanas. Combinaram os meios
de se corresponderem, em caso de necessidade, e separaram-se com
lágrimas.

No dia seguinte, estando na repartição, recebeu Camilo este bilhete de
Vilela: ``Vem já, já, à nossa casa; preciso falar-te sem demora''. Era
mais de meio-dia. Camilo saiu logo; na rua, advertiu que teria sido mais
natural chamá-lo ao escritório; por que em casa? Tudo indicava matéria
especial, e a letra, fosse realidade ou ilusão, afigurou-se-lhe trêmula.
Ele combinou todas essas coisas com a notícia da véspera.

--- Vem já, já, à nossa casa; preciso falar-te sem demora --- repetia ele
com os olhos no papel.

Imaginariamente, viu a ponta da orelha de um drama, Rita subjugada e
lacrimosa, Vilela indignado, pegando da pena e escrevendo o bilhete,
certo de que ele acudiria, e esperando-o para matá-lo. Camilo
estremeceu, tinha medo: depois sorriu amarelo, e em todo caso
repugnava-lhe a ideia de recuar, e foi andando. De caminho, lembrou-se
de ir a casa; podia achar algum recado de Rita, que lhe explicasse tudo.
Não achou nada, nem ninguém. Voltou à rua, e a ideia de estarem
descobertos parecia-lhe cada vez mais verossímil; era natural uma
denúncia anônima, até da própria pessoa que o ameaçara antes; podia ser
que Vilela conhecesse agora tudo. A mesma suspensão das suas visitas,
sem motivo aparente, apenas com um pretexto fútil, viria confirmar o
resto.

Camilo ia andando inquieto e nervoso. Não relia o bilhete, mas as
palavras estavam decoradas, diante dos olhos, fixas; ou então --- o que
era ainda pior --- eram-lhe murmuradas ao ouvido, com a própria voz de
Vilela. ``Vem já, já, à nossa casa; preciso falar-te sem demora''. Ditas
assim, pela voz do outro, tinham um tom de mistério e ameaça. Vem, já,
já, para quê? Era perto de uma hora da tarde. A comoção crescia de
minuto a minuto. Tanto imaginou o que se iria passar, que chegou a
crê-lo e vê-lo. Positivamente, tinha medo. Entrou a cogitar em ir
armado, considerando que, se nada houvesse, nada perdia, e a precaução
era útil. Logo depois rejeitava a ideia, vexado de si mesmo, e seguia,
picando o passo, na direção do largo da Carioca, para entrar
num tílburi\footnote{Carruagem para um passageiro, puxada por um cavalo.}.
Chegou, entrou e mandou seguir a trote largo.

\begin{itemize}
\item
  Quanto antes, melhor, pensou ele; não posso estar assim\ldots{}
\end{itemize}

Mas o mesmo trote do cavalo veio agravar-lhe a comoção. O tempo voava, e
ele não tardaria a entestar com o perigo. Quase no fim da rua da Guarda
Velha, o tílburi teve de parar; a rua estava atravancada com uma
carroça, que caíra. Camilo, em si mesmo, estimou o obstáculo, e esperou.
No fim de cinco minutos, reparou que ao lado, à esquerda, ao pé
do tílburi, ficava a casa da cartomante, a quem Rita consultara uma vez,
e nunca ele desejou tanto crer na lição das cartas. Olhou, viu as
janelas fechadas, quando todas as outras estavam abertas e pejadas de
curiosos do incidente da rua. Dir-se-ia a morada do
indiferente Destino\footnote{Grafado com inicial maiúscula, no original.
  Possível alusão às reflexões sobre a vida como uma sucessão de causas
  e consequências, conforme concebidas pelos antigos filósofos gregos e
  latinos.}.

Camilo reclinou-se no tílburi, para não ver nada. A agitação dele era
grande, extraordinária, e do fundo das camadas morais emergiam alguns
fantasmas de outro tempo, as velhas crenças, as superstições antigas. O
cocheiro propôs-lhe voltar a primeira travessa, e ir por outro caminho;
ele respondeu que não, que esperasse. E inclinava-se para fitar a
casa\ldots{} Depois fez um gesto incrédulo: era a ideia de ouvir a
cartomante, que lhe passava ao longe, muito longe, com vastas asas
cinzentas; desapareceu, reapareceu, e tornou a esvair-se no cérebro; mas
daí a pouco moveu outra vez as asas, mais perto, fazendo uns giros
concêntricos\ldots{} Na rua, gritavam os homens, safando a carroça:

--- Anda! agora! empurra! vá! vá!

Daí a pouco estaria removido o obstáculo. Camilo fechava os olhos,
pensava em outras coisas; mas a voz do marido sussurrava-lhe às orelhas
as palavras da carta: ``Vem, já, já\ldots{}'' E ele via as contorções do
drama e tremia. A casa olhava para ele. As pernas queriam descer e
entrar\ldots{} Camilo achou-se diante de um longo véu opaco\ldots{}
pensou rapidamente no inexplicável de tantas coisas. A voz da mãe
repetia-lhe uma porção de casos extraordinários, e a mesma frase do
príncipe de Dinamarca reboava-lhe dentro: ``Há mais coisas no céu e na
terra do que sonha a nossa filosofia\ldots{}'' Que perdia ele,
se\ldots{}?

Deu por si na calçada, ao pé da porta; disse ao cocheiro que esperasse,
e rápido enfiou pelo corredor, e subiu a escada. A luz era pouca, os
degraus, comidos dos pés, o corrimão, pegajoso; mas ele não viu nem
sentiu nada. Trepou e bateu. Não aparecendo ninguém, teve ideia de
descer; mas era tarde, a curiosidade fustigava-lhe o sangue, as fontes
latejavam-lhe; ele tornou a bater uma, duas, três pancadas. Veio uma
mulher; era a cartomante. Camilo disse que ia consultá-la, ela fê-lo
entrar. Dali subiram ao sótão, por uma escada ainda pior que a primeira
e mais escura. Em cima, havia uma salinha, mal alumiada por uma janela,
que dava para o telhado dos fundos. Velhos trastes, paredes sombrias, um
ar de pobreza, que antes aumentava do que destruía o prestígio.

A cartomante fê-lo sentar diante da mesa, e sentou-se do lado oposto,
com as costas para a janela, de maneira que a pouca luz de fora batia em
cheio no rosto de Camilo. Abriu uma gaveta e tirou um baralho de cartas
compridas e enxovalhadas. Enquanto as baralhava, rapidamente, olhava
para ele, não de rosto, mas por baixo dos olhos. Era uma mulher de
quarenta anos, italiana, morena e magra, com grandes olhos sonsos e
agudos. Voltou três cartas sobre a mesa, e disse-lhe:

--- Vejamos primeiro o que é que o traz aqui. O senhor tem um grande
susto\ldots{}

Camilo, maravilhado, fez um gesto afirmativo.

--- E quer saber, continuou ela, se lhe acontecerá alguma coisa ou
não\ldots{}

--- A mim e a ela, explicou vivamente ele.

A cartomante não sorriu; disse-lhe só que esperasse. Rápido pegou outra
vez das cartas e baralhou-as, com os longos dedos finos, de unhas
descuradas; baralhou-as bem, transpôs os maços, uma, duas, três vezes;
depois começou a estendê-las. Camilo tinha os olhos nela, curioso e
ansioso.

--- As cartas dizem-me\ldots{}

Camilo inclinou-se para beber uma a uma as palavras. Então ela
declarou-lhe que não tivesse medo de nada. Nada aconteceria nem a um nem
a outro; ele, o terceiro, ignorava tudo. Não obstante, era indispensável
muita cautela: ferviam invejas e despeitos. Falou-lhe do amor que os
ligava, da beleza de Rita\ldots{} Camilo estava deslumbrado. A
cartomante acabou, recolheu as cartas e fechou-as na gaveta.

--- A senhora restituiu-me a paz ao espírito, disse ele estendendo a mão
por cima da mesa e apertando a da cartomante.

Esta levantou-se, rindo.

--- Vá, disse ela; vá, \emph{ragazzo innamorato}\ldots{}\footnote{Do
  italiano, rapaz enamorado.}

E de pé, com o dedo indicador, tocou-lhe na testa. Camilo
estremeceu, como se fosse a mão da própria sibila\footnote{Mulher que
  pertencia ao oráculo, em acordo com a antiga mitologia grega.}, e
levantou-se também. A cartomante foi à cômoda, sobre a qual estava um
prato com passas, tirou um cacho destas, começou a despencá-las e
comê-las, mostrando duas fileiras de dentes que desmentiam as unhas.
Nessa mesma ação comum, a mulher tinha um ar particular. Camilo, ansioso
por sair, não sabia como pagasse; ignorava o preço.

--- Passas custam dinheiro, disse ele afinal, tirando a carteira. Quantas
quer mandar buscar?

--- Pergunte ao seu coração, respondeu ela.

Camilo tirou uma nota de dez mil-réis, e deu-lha. Os olhos da cartomante
fuzilaram. O preço usual era dois mil-réis.

--- Vejo bem que o senhor gosta muito dela\ldots{} E faz bem; ela gosta
muito do senhor. Vá, vá tranquilo. Olhe a escada, é escura; ponha o
chapéu\ldots{}

A cartomante tinha já guardado a nota na algibeira, e descia com ele,
falando, com um leve sotaque. Camilo despediu-se dela embaixo, e desceu
a escada que levava à rua, enquanto a cartomante, alegre com a paga,
tornava acima, cantarolando uma barcarola\footnote{Composição muito
  comum na Itália, desde o século \textsc{xv}.}. Camilo achou
o tílburi esperando, a rua estava livre. Entrou e seguiu a trote largo.

Tudo lhe parecia agora melhor, as outras coisas traziam outro aspecto, o
céu estava límpido e as caras joviais. Chegou a rir dos seus receios,
que chamou pueris; recordou os termos da carta de Vilela e reconheceu
que eram íntimos e familiares. Onde é que ele lhe descobrira a ameaça?
Advertiu também que eram urgentes, e que fizera mal em demorar-se tanto;
podia ser algum negócio grave e gravíssimo.

--- Vamos, vamos depressa, repetia ele ao cocheiro.

E consigo, para explicar a demora ao amigo, engenhou qualquer coisa;
parece que formou também o plano de aproveitar o incidente para tornar à
antiga assiduidade\ldots{} De volta com os planos, reboavam-lhe na alma
as palavras da cartomante. Em verdade, ela adivinhara o objeto da
consulta, o estado dele, a existência de um terceiro; por que não
adivinharia o resto? O presente que se ignora vale o futuro. Era assim,
lentas e contínuas, que as velhas crenças do rapaz iam tornando ao de
cima, e o mistério empolgava-o com as unhas de ferro. Às vezes queria
rir, e ria de si mesmo, algo vexado; mas a mulher, as cartas, as
palavras secas e afirmativas, a exortação: --- Vá, vá, \emph{ragazzo
innamorato}; e no fim, ao longe, a barcarola da despedida, lenta e
graciosa, tais eram os elementos recentes, que formavam, com os antigos,
uma fé nova e vivaz.

A verdade é que o coração ia alegre e impaciente, pensando nas horas
felizes de outrora e nas que haviam de vir. Ao passar pela
Glória\footnote{Bairro do Rio de Janeiro.}, Camilo olhou para o mar,
estendeu os olhos para fora, até onde a água e o céu dão um abraço
infinito, e teve assim uma sensação do futuro, longo, longo,
interminável.

Daí a pouco chegou à casa de Vilela. Apeou-se, empurrou a porta de ferro
do jardim e entrou. A casa estava silenciosa. Subiu os seis degraus de
pedra, e mal teve tempo de bater, a porta abriu-se, e apareceu-lhe
Vilela.

--- Desculpa, não pude vir mais cedo; que há?

Vilela não lhe respondeu: tinha as feições decompostas; fez-lhe sinal, e
foram para uma saleta interior. Entrando, Camilo não pôde sufocar um
grito de terror: --- ao fundo, sobre o canapé\footnote{Assento parecido
  com o sofá, que permite sentar ou se deitar de lado.}, estava Rita
morta e ensanguentada. Vilela pegou-o pela gola, e, com dois tiros de
revólver, estirou-o morto no chão.
