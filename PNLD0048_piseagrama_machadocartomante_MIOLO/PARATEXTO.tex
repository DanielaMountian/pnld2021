\part{\textsc{paratexto}}

\chapter{Vida e obra de Machado de Assis}

\begin{flushright}
\textsc{carlos rogério duarte barreiros}
\end{flushright}\medskip

\section{Sobre o autor}

Machado de Assis (1839--1908) é frequentemente considerado o maior escritor brasileiro de todos os tempos. Em sua obra, expõe, com prosa brilhante e ironia fina, a espinha dorsal das relações sociais da sociedade brasileira de seu tempo, da qual carregamos hoje muitas heranças. É autor de grandes romances, com destaque para a chamada trilogia realista composta por \emph{Memórias póstumas de Brás Cubas} (1881), \emph{Quincas Borba} (1891) e \emph{Dom Casmurro} (1899), e também de um sem número de contos, novelas e folhetins. Em ambos os casos, os escritos de Machado, para além do incalculável valor literário, permanecem atuais como grandes reflexões acerca do Brasil do século \textsc{xix} e do atual.

\section{Sobre a obra}

As três narrativas de Machado de Assis aqui organizadas foram, originalmente,
publicadas em revista. ``O alienista'' saiu na Revista \emph{A Estação} --- periódico voltado para o público feminino que circulou no Rio de Janeiro entre 1879 e 1904 --- entre 15 de outubro de 1881 e 15 de março de 1882. Foi posteriormente coletado em \textit{Papéis avulsos}.
Nessa mesma revista saiu ``O imortal'' em 1882.
Dois anos depois, em 28 de novembro de 1884, ``A cartomante'' foi publica na revistsa \emph{Gazeta de Notícias}, e depois coletada no livro de contos \emph{Várias histórias}.

\subsection{«O Alienista»: entre a insanidade e a~lucidez,~entre~o~fato~e~a~ficção}

``O Alienista'' --- texto de difícil classificação, entre o conto e a
novela, como vamos observar a seguir --- é das obras mais conhecidas de
Machado de Assis. Numa Itaguaí do século \textsc{xviii}, a construção de uma Casa
de Orates --- um hospício --- causa convulsões sociais que são comparáveis
às da Revolução Francesa, tudo isso contado por um narrador ambivalente.
Desnecessário frisar a atualidade do texto, especialmente se
considerarmos a luta antimanicomial, como se Machado já pudesse prever,
em pleno 1882, ano da publicação de ``O Alienista'' no livro
\emph{Papéis Avulsos}, que as pesquisas e terapias psiquiátricas podem
chegar ao extremo da \emph{medicalização} dos comportamentos, servindo a
interesses de caráter pessoal, muito distantes da suposta objetividade
neutra da ciência.

Tudo é ambivalente em ``O Alienista'', a começar do gênero a que
pertence esse texto. Jean Pierre Chuavin, ao analisá"-lo na introdução
desta antologia, afirma que durante muito tempo ele foi considerado um
conto, mas que essa classificação está longe de ser pacífica: embora a
pesquisa de Simão Bacamarte na Casa Verde seja o fio condutor do texto,
articulando"-o como todo, há diversas narrativas paralelas que ganham
alguma autonomia no conjunto. Eis aí um dos desafios de trabalhar com
Machado de Assis em sala de aula: a impossibilidade, muitas vezes, de
apresentar uma classificação fechada a respeito de seus textos. Só para
citar mais uma dessas dificuldades, lembremo"-nos do famoso prólogo da
quarta edição das \emph{Memórias Póstumas de Brás Cubas}, em que o autor
transcreve uma pergunta de Capistrano de Abreu: esse livro é um romance?
No caso de ``O Alienista'', se ele tivesse prólogo, Machado poderia
perguntar: ``O Alienista'' é conto ou novela?

Mas poderíamos questionar ainda mais: quais são os efeitos obtidos por
meio dessa ambivalência? Talvez o primeiro deles seja a sugestão de que
o texto, da mesma forma que os eventos nele relatados, é
inclassificável, em boa medida. O narrador, por exemplo, nos conta
eventos ocorridos em Itaguaí no fim do século \textsc{xviii}, mas não nos
apresenta datas específicas, apenas as sugere, por meio de referências
esparsas. Não deixa de ser curioso que o texto comece aludindo às
``crônicas de Itaguaí'', na mesma linha, refira"-se a ``tempos remotos''
e se conclua com alusões a boatos; ora, se a referência são as crônicas,
nada mais natural que elas contenham informações mais ou menos claras
sobre a localização dos fatos no tempo; mas aqui estamos em pleno oceano
machadiano, em que homens, coisas e fatos são elas e seu inverso,
\emph{são e não são simultaneamente} --- e por isso o narrador fala em
tempos distantes, sem especificá"-los. Estamos entre fatos e ficção,
entre história e literatura --- e é nesse espaço intervalar e indefinido
que Machado de Assis transita. Habituar"-se a lê"-lo é, de certa maneira,
aprender a lidar com esse interstício e a habitá"-lo, como leitor. Essa
é, aliás, uma das linhas de força de Machado.

Coloquemo"-nos no lugar de nossos estudantes: o que é que se ganha, eles
nos perguntariam, com tanta indefinição? Para começo de conversa, um
ponto de vista privilegiado para contar a história. Aparentemente ao
lado de Simão Bacamarte, o narrador não se restringe à perspectiva dele,
e transita por muitas outras, de forma a dar ao leitor uma visão mais
abrangente do que aparenta à primeira vista. Além disso, as incertezas
obrigam a novas leituras.

Nosso debate em sala de aula pode concentrar"-se, por exemplo, na
personagem marcante de Simão Bacamarte, espécie de encarnação brasileira
do médico à moda do século \textsc{xix}, mais cientista do que clínico, que usa
as experiências do cotidiano terapêutico para avançar as hipóteses que
vai compondo ao sabor dos resultados. Há nessa personagem uma certa
desumanização, como se ele fosse mais cientista do que gente,
especialmente quando escolhe a esposa pelas ``condições fisiológicas e
anatômicas de primeira ordem'' que ela apresenta, sem apaixonar"-se por
ela. Essa desumanização é também flagrante nos desdobramentos da teoria
da razão, especialmente quando Bacamarte aprisiona os habitantes
supostamente normais de Itaguaí. O leitor já entendeu, nessa passagem do
texto, que o conceito de normalidade se expandiu e se refere, agora,
também a comportamentos de ordem moral ou ética. O projeto final de
Simão Bacamarte é, trocando em miúdos, arruinar qualquer manifestação de
altruísmo, como se o tratamento proposto apresentasse a canalhice e o
egoísmo como remédios para o respeito aos outros. Somente uma pessoa
\emph{desumanizada} como Bacamarte poderia propor essa terapêutica às
avessas.

Outra análise claramente fértil de ``O Alienista'' é a do tempo. Já
citamos acima as diversas alusões à Revolução Francesa que se fazem
presentes no texto de maneira bastante explícita. A questão fundamental
aqui --- e também no conto ``O Imortal'' --- é tentar entender por que
Machado de Assis dá tanta importância à virada do século \textsc{xviii}, tanto no
Brasil quanto fora dele. É evidente que esse momento histórico contém
uma das mais marcantes viradas sociais, econômicas, políticas e
culturais da História humana: a derrocada final das hierarquias e
dependências características dos séculos \textsc{xvi}, \textsc{xvii} e \textsc{xviii}, em que se
formava o capitalismo, que dão lugar às liberdades de inspiração
burguesa, emergentes precisamente naquela virada. Para nós, é importante
perceber que Machado de Assis reconhecia semelhanças entre esse processo
no Brasil e fora dele, como se em Itaguaí, em escala distinta, se
reproduzissem os acontecimentos da Revolução na França, com diferenças
marcantes, que é preciso registrar. Para reconhecer essas diferenças e
semelhanças, leia o artigo de artigo ``Machado de Assis e a (sua)
Revolução Francesa'', de André Dutra Boucinhas, indicado na bibliografia
comentada.

Note"-se, ainda, que, embora os eventos narrados em ``O Alienista'' se
passem no fim do século \textsc{xviii}, a mentalidade científica de Simão
Bacamarte lembra muito o processo de \emph{medicalização} dos
comportamentos, que foi iniciado no século \textsc{xix} e que dura até hoje: a
obsessão pelo ``exame da patologia cerebral'' tem o sabor quase sádico
das pesquisas médicas que se desenvolveram nos oitocentos, cometeram
horrores no século \textsc{xx} contra pessoas e animais e seguem até hoje. O
desejo da classificação e sua consequente discriminação entre os homens,
por meio de pesquisas que ferem limites éticos, é um dos temas marcantes
desse texto, com potencial de exploração bastante grande em aula.

O debate a respeito dos limites entre razão e loucura interessa bastante
a Machado de Assis --- veja a importância desse assunto, por exemplo, nas
\emph{Memórias Póstumas de Brás Cubas} e \emph{Quincas Borba}, no
filósofo que dá título ao segundo romance. Simão Bacamarte e Quincas
Borba são parecidos em alguns aspectos: são personagens cuja mentalidade
e comportamento são guiados por estudos de prestígio --- a medicina, no
primeiro caso; a filosofia, no segundo --- numa indicação clara de que a
especialização crescente dos estudos da ciência, apesar de amparada em
critérios rigorosos, leva à perda da compreensão do real. Ora, há poucos
assuntos tão atuais quanto esse: que dizer dos médicos especialistas que
nos mandam de um consultório a outro, sob a desculpa de que nosso
problema não faz parte da especialidade deles? Qual é o limite entre
causas genéticas e fisiológicas, de um lado, e ambientais e sociais, de
outro, dos transtornos de ordem mental? Afinal: quais de nós podemos
dizer que estamos em plenas condições de apreensão do real, se ele
próprio se apresenta adoecido e insano, muitas vezes?

Costumamos atribuir essas perguntas às circunstâncias de nosso tempo ---
as tecnologias digitais, a internet, as redes sociais e suas
consequências sociais, econômicas, políticas e culturais --- mas Machado
de Assis já fazia perguntas similares às nossas no século \textsc{xix}. ``O
Alienista'' pode ser lido como uma pesquisa de Machado de Assis a
respeito de um evento histórico marcante para a constituição do mundo
contemporâneo --- a Revolução Francesa --- à procura das bases sobre as
quais foi erigida a sociedade moderna, em que vivemos, especialmente o
discurso médico"-científico.

\subsection{A fantástica história de «O Imortal»}

``O Imortal'' é um conto especialmente impressionante, em primeiro lugar
por aproximar"-se do gênero do conto fantástico. Nele, em 1855, Dr. Leão
relata a dois outros homens a vida do pai, nascido em 1600. Esse
despropósito --- a distância de 255 anos entre o nascimento do pai e a
vida adulta do filho --- abre o conto, cujo ponto de partida é uma lenda
indígena da imortalidade, repleta de referências históricas: Rui de Leão
teria participado da Invasão Holandesa e da brutal aniquilação do
Quilombo dos Palmares, no Brasil, além de correr mundo e contribuir para
eventos como a Rebelião de Monmouth, na Inglaterra, e daí de volta ao
Brasil, no tráfico de escravos, na chegada da corte portuguesa em 1808 e
na Independência, em 1822. Da mesma maneira que as \emph{Memórias
Póstumas de Brás Cubas} surpreendem pela tagarelice de um morto que
conta a própria vida, ``O Imortal'' chama a atenção pela fantástica
imortalidade do protagonista; e, em ambos os textos, quase sem que o
leitor perceba, o misterioso e o insólito desviam"-lhe a atenção,
impedindo"-o de observar o desfile de eventos históricos que lhe passam
pelos olhos. Nesse sentido, a leitura de Machado de Assis é formativa ---
para professores e alunos --- porque requer de nós que suspeitemos sempre
de quem nos conta a história.

Contos fantásticos são sempre fascinantes para as gerações mais jovens.
Aqui, para preparar"-se para o debate a respeito deles, o professor pode
ler a \emph{Introdução à Literatura Fantástica}, de Tzvetan Todorov, e
partir do pressuposto de que o conto desse gênero se caracteriza pela
incerteza: num mundo como o nosso, sem criaturas maravilhosas, dominado
pelo \emph{discurso científico que se propõe a explicar tudo}, ocorre um
evento inacreditável; ou a ciência ainda não descobriu as explicações
desse evento, ou ele é falso, ilusório, imaginário. É precisamente nesse
interstício que se produz o fantástico, que é, para Todorov, ``a
hesitação experimentada por um ser que só conhece as leis naturais, face
a um acontecimento aparentemente sobrenatural''.\footnote{\textsc{todorov},
  Tzvetan. \textit{Introdução à Literatura Fantástica}. São Paulo:
  Perspectiva, 2012. p.\,31.}

De fato, uma vida humana de mais de duzentos anos é posta em xeque desde
as primeiras páginas do conto: os ouvintes de Dr. Leão inicialmente não
acreditam que isso seja possível, mas se deixam envolver pela capacidade
narrativa desse médico homeopata, pelo fascínio da imortalidade e pela
multiplicidade das experiências vividas por Rui de Leão no tempo e no
espaço. Eis aí um dos atrativos de ``O Imortal'': a possibilidade de
participar de eventos históricos muito distintos, em espaços remotos.
Numa perspectiva histórica --- e uma das atividades que propusemos
aprofunda essa interpretação --- os anos que correm de 1600 até à morte
do pai do narrador, em 1855, correspondem ao período de formação do
capitalismo: primeiramente, assistimos aos anos do século \textsc{xvii}, no qual
o mundo observou a circulação do primeiro produto de massas da história
--- o açúcar --- e no qual os ingleses deram os primeiros passos no
sentido da modernização de suas estruturas políticas, que culminariam na
Revolução Industrial do século \textsc{xix}. O pai do narrador participou, de
certa forma, desses dois eventos, por estar no Brasil quando ocorriam as
Invasões Holandesas ao Brasil e por contribuir em alguma medida nos
eventos que levaram à Revolução Gloriosa, na Inglaterra. Da mesma
maneira, ele também traficou pessoas escravizadas no século \textsc{xviii},
marcando presença no comércio que é o pressuposto de todas as atividades
econômicas que, no Brasil, estão ligadas à Independência e, no exterior,
à formação da sociedade industrial. Rui de Leão pode ser entendido,
assim, como um privilegiado que teria assistido a dois dos séculos
supostamente mais férteis da história humana: aqueles que abriram as
portas à Revolução Industrial.

Mas essa euforia não corresponde à avaliação final: cansado das
decepções pessoais, prevendo infinitas perdas de entes queridos, que
inevitavelmente acabaria por enterrar, Rui de Leão está desesperançado.
Considerando tudo que viu, poderíamos concluir, numa primeira análise,
que esse personagem encarna o que a tradição crítica chamou de
\emph{negatividade} em Machado de Assis: a humanidade pode avançar em
termos científicos e tecnológicos, mas é a mesma em termos das
barbaridades que inflige a si mesma; diante dos afetos, o desfile dos
grandes eventos históricos se apequena. O pai de Dr. Leão prefere a
morte à vida perpétua, porque esta se tornara mera repetição
interminável.

Não deixemos de notar que o ano da morte dele é 1855: apenas cinco anos
depois da extinção do tráfico de escravos no Brasil, que finalmente
cedera às pressões inglesas. Também podemos ver aí mais uma virada
histórica: depois de duzentos e cinquenta e cinco anos de comércio de
africanos, o Brasil se preparava para abolir a ignomínia --- mas a
experiência de nosso protagonista não lhe permitia ter a esperança de
dias melhores, como se nos ensinasse que a humanidade sabe bem repetir e
aprofundar as brutalidades que comete contra si mesma e contra a
natureza.

\subsection{Adivinhação e realidade em «A Cartomante»}

``A Cartomante'' é, finalmente, um dos mais conhecidos contos de Machado
de Assis --- e, mais uma vez, se nos deixarmos levar pela superfície do
enredo, talvez não percebamos a riqueza de detalhes que costuram a
tragédia de um crime ambientado no Rio de Janeiro do século \textsc{xix}. Podemos
arriscar dizer que as matrizes patriarcais da sociedade brasileira estão
claramente manifestas na passagem final do conto, em que um marido
assassina a própria a esposa que o traía com o melhor amigo, num retrato
explícito da violência que recai sobre as mulheres em um país como o
nosso. Mas não se trata apenas disso: também é preciso desvendar, no
conjunto do conto, a importância da alusão imprecisa de uma das
personagens ao \emph{Hamlet}, de Shakespeare, e o debate por trás da
cartomante que dá título ao conto. Nas primeiras páginas dessa história
(que poderia perfeitamente figurar em um dos programas sensacionalistas
de fim de tarde, na televisão), Rita, a esposa que será morta pelo
marido, relata ao amante Camilo a visita que fizera a uma cartomante;
também ele fará a mesma coisa, antes de ser assassinado --- e o leitor se
pergunta: que desejará Machado dizer ao flagrar cartomantes que não
alertam seus clientes da tragédia por vir?

Certamente, ``A Cartomante'' é dos melhores contos da Literatura
Brasileira. A alusão imprecisa a uma passagem de ninguém menos do
William Shakespeare, nas primeiras linhas, revela a distância que vai
entre o que se poderia chamar de \emph{céu} --- a literatura
grandiloquente do passado, da tragédia ocorrida entre nobres da corte do
Reino da Dinamarca --- e a \emph{terra} --- os eventos familiares que se
restringem a três personagens e que também terminarão em desgraça; aqui
as proporções são menores, certamente, sem implicações no xadrez
político da Europa, mas clímax do conto, condensado em seu último
parágrafo, causa no leitor, especialmente pela estrutura do conjunto,
uma forte impressão. A alusão a Shakespeare, a conversa inicial, o
\emph{flash"-back} no qual o leitor toma contato com a história do
adultério, o curtíssimo bilhete de Vilela a Camilo, a hesitação deste em
atender ao chamado e sua consulta à cartomante --- todos esses elementos
convergem para o parágrafo final, cuja violência surpreende até mesmo o
mais atento e experiente dos leitores.

Se nos permitirmos uma divagação, podemos nos perguntar quais serão as
consequências do assassinato de Camilo e Rita para Vilela --- e talvez
elas não fossem das piores. É curioso saber que o Código Penal
Brasileiro que vigorou de 1890 a 1940 previa que crimes cometidos por
uma pessoa privada dos próprios sentidos e da inteligência poderiam não
levar à prisão ou a outras penas. Em resumo: quem cometia crime fora de
si não era responsabilizado por ele, o que nos leva a crer que Vilela
talvez não fosse preso pelos dois assassinatos. ``A Cartomante'' foi
publicado na Gazeta de Notícias em 1884 --- isto é, apenas seis anos
antes de esse código entrar em vigor. Crimes como o cometido por Vilela
eram --- e infelizmente ainda são --- comuns na sociedade brasileira. O
conto de Machado de Assis pode ser lido, assim, como um documento de
época a respeito da violência cotidiana, que nos ensina bastante a
respeito de nosso próprio tempo.

Do ponto de vista da escolha do tema, além de dialogar com as grandes
tragédias familiares, como a de Hamlet, apesar das diferenças evidentes
de contexto, ``A Cartomante'' também dialoga com o jornalismo
sensacionalista: as páginas policiais do ``fait divers'', expressão
francesa que, traduzida ao pé da letra, significa ``fatos diversos'' e
que se refere às notícias cotidianas, que correm mundo; são os crimes
que chocam --- e que guardam pelo menos um laço com as tragédias:
Aristóteles dizia que o efeito catártico causado por elas advinha do
\emph{temor} e da \emph{pena} do público. O temor, especificamente,
derivava da percepção de que \emph{a tragédia ocorrida com as
personagens pode acontecer também comigo}. Talvez esteja aí a força da
impressão causada pelo final de ``A Cartomante'': também nós podemos
acabar cedendo aos desejos e traindo nossos amigos, esposas e maridos;
também nós podemos não querer ver a realidade como ela é, tentando
escapar dela pelos charlatães em geral --- porque não se trata de
analisar os fatos, mas de querer \emph{acreditar} nesses embusteiros;
finalmente, também nós podemos sofrer consequências graves daquilo que
fizemos, depois de tentar tapar o sol com a peneira.

É muito curioso saber que a Gazeta de Notícias, o jornal em que Machado
de Assis publicou ``A Cartomante'', tinha colunas de ``fait divers''. As
seções de nomes ``Ocorrências da Rua'', iniciada em 1878,
``Acontecimentos Notáveis'', em 1880, e ``Casos Policiais'', em 1890,
dividiam o espaço com notícias, editoriais e contos como o de Machado de
Assis --- e noticiavam crimes chocantes, no que se pode considerar o
nascedouro da imprensa sensacionalista brasileira. ``A Cartomante'' pode
ser lido, assim, como texto em que Machado de Assis aproximou a tragédia
de Shakespeare das tragédias cariocas, com a mediação da imprensa.

A título de conclusão dessa análise, cabem aqui duas reflexões.

A primeira, a respeito desses elementos todos --- a citação a
Shakespeare, a estrutura do conto e a semelhança com as colunas de
ocorrências violentas e policiais: talvez Machado de Assis esteja
fazendo, em ``A Cartomante'', um extenso exercício de \emph{linguagem
literária} produzindo mediações entre o céu e a terra, expressões
tomadas aqui, evidentemente, no sentido figurado. De forma geral, o
autor nos mostra que as tragédias familiares se repetem, seja na corte
da Dinamarca, seja no Rio de Janeiro, apesar das diferenças; que essas
tragédias impressionam porque nos identificamos com elas, isto é, porque
todos sabemos que podemos incorrer naqueles erros e sofrer
consequências; e finalmente: na época de Machado de Assis, e também na
nossa, essas tragédias são mercantilizáveis, seja na forma de notícias
sensacionalistas, seja na forma de contos. Nesse sentido, desponta mais
uma vez a capacidade que Machado tem de antever as coisas: programas de
televisão que só noticiam crimes brutais; canais virtuais com vídeos
sangrentos; jornais sensacionalistas --- todos eles seguem transformando
a tragédia alheia em notícia, raras vezes respeitando o sofrimento das
pessoas envolvidas, expondo"-as severamente.

A segunda diz respeito ao nosso desejo de não ver o que está diante de
nossos olhos. É o que costuma acontecer às vítimas de tragédias: elas
não percebem que já estão enredadas pelo destino, à beira da própria
desgraça. Camilo não percebe, pouco antes de ser morto por Vilela, que
será assassinado e prefere acreditar no embuste da cartomante. Não
seremos todos nós, ao menos um pouco, como Camilo? Mais ainda: não somos
nós, leitores, levados pelo narrador a preferir acreditar que nada vai
acontecer? Provavelmente sim --- e por isso tomamos aquele susto ao ler o
último parágrafo de ``A Cartomante'': porque continuamos acreditando que
existe ``muita coisa misteriosa e verdadeira neste mundo'', apesar dos
fatos que são noticiados todos os dias na imprensa.

\section{Sobre o gênero}

\begin{quote}
O conto é uma narrativa unívoca, univalente: constitui
uma \textit{unidade dramática}, uma \textit{célula dramática}, visto gravitar ao
redor de um só conflito, um só drama, uma só ação. Caracteriza"-se,
assim, por conter \textit{unidade de ação}, tomada esta como a sequência de atos praticados pelos protagonistas, ou de acontecimentos de
que participam. A ação pode ser externa, quando as personagens se
deslocam no espaço e no tempo, e interna, quando o conflito se
localiza em sua mente.\footnote{\textsc{moisés}, Massaud. \textit{A criação literária}. São Paulo: Cultrix, 2006, p.\,40.}
\end{quote}

Partindo da definição de Massaud Moisés sobre o conto, evidencia"-se a principal característica desse gênero literário: a unidade de conflito, condensada em ações que se completam em um único enredo. Ao conto, ainda seguindo Moisés, aborrecem as divagações e os excessos, pois há uma concentração de efeitos e pormenores essenciais, em sua brevidade, para o bom funcionamento do conto.
Cada construção, cada palavra nesse gênero tem sua razão de existir, pois integra a economia global da narrativa.

Apesar da brevidade de sua forma, o conto desdobra"-se em muitas direções e implicações, e o faz a partir de elementos restritos: a unidade dramática, como já mencionada, assim como a presença de poucas personagens e a limitação espacial e temporal. Um ótimo exemplo é o conto ``Missa do galo'', de Machado de Assis, em que o narrador, Nogueira, conta a sua experiência de uma única noite na companhia de sua hospedeira, D.\,Conceição. Apesar de unidade temporal (a noite que antecede a Missa do galo), espacial (uma sala na casa de D.\,Conceição) e da redução dramática, basicamente, à interação entre duas personagens, Conceição e Nogueira, esse conto desdobra"-se em muitas direções. A companhia de Conceição desperta a sexualidade de Nogueira, e seu impacto é tão profundo que o narrador relembra aos leitores esse acontecimento de sua juventude. As intenções da anfitriã, narradas e, logo, distorcidas pela memória de Nogueira, também são ambíguas, levantando as mais diversas questões e interpretações.

Como reflete o escritor argentino Julio Cortázar, o conto consegue, de forma muito concisa, despertar ``uma realidade infinitamente mais vasta que a do seu mero argumento'', influindo ``em nós com uma força que nos faria suspeitar da modéstia do seu conteúdo aparente, da brevidade do seu texto''.\footnote{\textsc{cortázar}, Julio. \textit{Valise de cronópio}. São Paulo: Editora Perspectiva, 2008, p.\,155.}
Apesar da aparente banalidade do argumento, o conto abre essa possibilidade de desenvolver o tema em profundidade, em contraposição à aparente concisão narrativa.

Já a novela está entre o conto e o
romance em termos de extensão. Enquanto o primeiro se caracteriza pela
presença de poucos personagens, uma única ação, um só conflito e um
drama, além de uma limitação do tempo e do espaço, a novela é mais ampla
e plural quanto a esses elementos.

Segundo o crítico Massaud Moisés, a novela, em comparação com o conto, é essencialmente multívoca, polivalente: ``constitui"-se de uma série de unidades ou células dramáticas. De onde se segue que a primeira característica estrutural da novela é sua pluralidade dramática: ao invés do conto, que gira em torno de um conflito, a novela focaliza vários. E cada um deles apresenta começo, meio e fim''.\footnote{\textsc{moisés}, Massaud. \textit{A criação literária}. São Paulo: Cultrix, 2006, p.\,113.}

Ainda nas palavras do crítico literário:

\begin{quote}
O novelista não esgota por completo o conteúdo de uma unidade para depois efetuar o mesmo com as seguintes: no fim de cada episódio, procura deixar sementes de mistério ou conflito para manter aceso o interesse do leitor. É raro que esvazie o recheio dramático duma célula antes de prosseguir, pois frustraria a curiosidade do leitor.
Em suma multiplicidade dramática, numa corrente horizontal. Por isso, o número de páginas pode crescer à vontade: a pluralidade pressupõe uma estrutura aberta, de modo que novos episódios possam adicionar"-se numa cadeia sucessiva, assim como o fim provisório da narrativa implica a multivocidade.\footnote{Ibid., p.\,114.}
\end{quote}

Após esclarecer a ação na novela, Moisés coloca em perspectiva o tempo novelesco. Afirma que a estrutura linear e plural da novela lhe impõe uma limitação temporal, que faz com que esse gênero não aborde a história da personagem desde seu nascimento, mas reduz"-lhe o passado a poucas linhas, essenciais para compreender"-lhe as ações e seu modo de ser, supreendendo a personagem no momento em que está madura par agir.

Por fim, ao abordar o espaço da novela, Moisés ressalta o dinamismo acelerado desse gênero literário, causado pela sucessão de episódios, que implica a ausência de uma unidade espacial e uma pluralidade de espaços que distingue a novela:

\begin{quote}
À semelhança do conto, a estrutura da novela caracteriza"-se por ser plástica, concreta, horizontal. Assumindo as mais das vezes a perspectiva da terceira pessoa, o autor se coloca fora dos acontecimentos, ou concede a uma personagem a direção da narrativa. A vida imaginária sobrepõe"-se à vida observada: o novelista concentra"-se em multiplicar os expedientes narrativos, formulando sucessivas células dramáticas, sem atentar para os imperativos da verossimilhança. O enredo, além de visível, não esconde nada, não dissimula profundidades dramáticas ou psicológicas: com o predomínio da ação, tudo o mais se torna menos significativo.\footnote{Ibid., p.\,117--118.}
\end{quote}

