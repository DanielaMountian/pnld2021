\paginabranca
\chapter{O homem das multidões}
\hedramarkboth{o homem das multidões}{}

%\begin{flushright}
%\textsc{dirceu villa}
%\end{flushright}

%\section*{baudelaire, l'homme des foules\protect\footnote{ “\uppercase{O} homem das
%multidões”, como o título do conto de \uppercase{E}dgar \uppercase{A}llan \uppercase{P}oe.}}

\section{Sobre o autor}

\noindent{}Charles Baudelaire (1821--1867) é um autor famoso e quase
incompreensível para a mentalidade politicamente correta e
espontaneísta de nossos dias. É preciso começar dizendo isso porque
Baudelaire ataca convenções e é um verdadeiro artesão da palavra, e
sobretudo estes \textit{Pequenos poemas em prosa} são uma coleção de
textos em que provocou (e não apenas pelo gosto de provocar) o então
chamado “pequeno burguês”, aquela pessoa odiosa que se conformava a uma
vida mesquinha, porque modestamente confortável, e alheia aos poderes
transformadores da arte. Um grupo de pessoas que, aliás, vinha
aumentando. Mas de que transformação estaríamos falando? Baudelaire
teria algum tipo de engajamento?

Não, na verdade. Ele é um daqueles casos literários indefiníveis, mesmo
porque não tinha simpatia por nenhuma causa, e a “ideia de arte”
em seu pensamento possui antes um sentido absoluto, que a separa num
corte nítido da “ideia de natureza”: esse é, em breves palavras,
o cerne de sua poesia, e daquilo que Victor Hugo chamou de \textit{frisson
nouveau} ao lê"-lo. Essa distinção entre arte e natureza lhe
conferiu também a peculiaridade de sua \textit{persona} literária, que
se desdobrava em poeta, crítico de arte e crítico literário. \textit{Krino},
\textit{krinein}: o “escolher”, que está na raiz grega da palavra
 “crítico”. Baudelaire tinha o discernimento necessário para
produzir cortes nítidos, um dom raro em poetas.

%O título desta parte da introdução o põe como “o homem das multidões”.
Poderíamos dizer de Baudelaire que foi um ``homem das multidões'', retomando o título de um conto de Edgar A.\,Poe.
É uma ideia até certo ponto bizarra dizer desse quase misantropo que fosse
um “homem das multidões”, mas a imersão na massa das cidades grandes
pouco tinha realmente de um encontro: era, poderíamos dizer, um
 “perpassar”, quando não fosse explícita repulsa; produtiva, mas
repulsa. Lembremos que Edgar Allan Poe, um dos heróis de Baudelaire,
apôs de epígrafe a seu conto acima mencionado, chamado “The Man of the
Crowd”, a seguinte queixa de La Bruyère: “Ce grand malheur, de ne
pouvoir être seul”.\footnote{ “Essa grande infelicidade, a de não poder
estar só”, em Edgar Allan Poe, \textit{Sixty"-Seven Tales}, Londres: Leopard
Books, 1995, p. 240.}

O conto de Poe, que narra o encontro de um convalescente com a turba que
ele espia da janela de um café, num contínuo fluxo de deformidades, é
datado de 1840; em “Les foules” (“As massas”), o décimo segundo destes
\textit{Poemas em prosa}, e publicado pela primeira vez em 1861 na
\textit{Revue fantaisiste}, Baudelaire parece responder engenhosamente
ao pequenino comentário de La Bruyère. Ele nos dá uma fórmula de
impressionante perspicácia, já que as multidões eram coisa
razoavelmente nova: “Multidão, solidão”. 

\hyphenation{poe-ta}
\hyphenation{poe-ma}
\hyphenation{poe-sia}
\hyphenation{tem-pes-tuoso}
\begin{quote}
Multidão, solidão: termos iguais e permutáveis para o poeta ativo e
fecundo. Quem não sabe povoar sua solidão, tampouco sabe estar só em
meio a uma massa atarefada.
\end{quote}

“Povoar sua solidão” e “estar só em meio a uma massa atarefada”. Esses
dois movimentos se sucedem por todo o livro, às vezes imbricados: o
solitário imerge na multidão; o observador detecta, na massa, um
personagem distinto e repleto de significado existencial (como o velho
saltimbanco, por exemplo); a solidão é acossada por estranhos que
surgem como que do nada; há o ruído da dispersão e o silêncio
introspectivo que sonha paraísos artificiais. Não há o “não poder estar
só” que, por princípio, se exclui: o poeta é efetivamente um solitário,
um diferente, um pária, aquele que, em \textit{Les fleurs du mal}, será
comparado ao albatroz apanhado num navio, onde os marujos lhe põem um
cachimbo, riem e zombam daquela ave que, antes, voava elegantemente:
“Exilado no chão em meio às turbas, / Suas asas de gigante impedem"-no
de andar”. Em “À une heure du matin” (“A uma hora da manhã”), deste
nosso livro, escreve: “Enfin! seul!”, isto é, \textit{Enfim! só!} Ou,
o artista é “um solitário de imaginação ativa”, como Baudelaire
definiria o desenhista Constantin Guys, em “O pintor da vida
moderna”.\footnote{ Charles Baudelaire, \textit{Sobre a modernidade}
(org. e trad. Teixeira Coelho), Rio de Janeiro: Paz e
Terra, 1996, p. 24.}

Crítico muito requintado --- que reconheceu nos desenhos de fina
observação de Constantin Guys, na arte caricatural e nervosa de
Daumier, na música tempestuosa de Wagner e nos textos matematicamente
febris de Poe os caminhos da arte, e deles tirou cristalizações que
definem seus procedimentos com clareza ---, era de se esperar que uma tal
mente não se afinasse pelo diapasão de seu meio. Também por isso seu
gosto pela frieza da alegoria, que tem um aspecto analítico, mas
esconde dentro de si aquela pulsação que encontramos na definição de
Emmanuele Tesauro para a força da metáfora, quando dizia que, pela
compressão do sentido --- que se opõe à sucessividade das imagens ---, a
metáfora é perspectiva, arrojada, pulsante.

É nisso que vamos encontrar aquele poder transformador, aludido acima. A
voz que nos fala no livro cede a impulsos para compreender seus
limites, para atender a um apetite passageiro, para realizar uma
experiência, e então observa essa experiência quase como faria um
pesquisador, para depois fornecer o resultado que se junta à percepção
geral já desenvolvida sobre seu tema, necessariamente um modo de
flagrar um mundo em transformação, seja no aspecto da paisagem, dos
agrupamentos humanos, da cultura ou “da própria percepção”. Não          
por acaso, enquanto Poe era ignorado nos \textsc{eua} (quando não fosse 
considerado simplesmente um sujeito esquisito), ele já havia se tornado
um autor da predileção de Baudelaire, que traduziu vários de seus
textos ficcionais e críticos para o francês. Baudelaire trazia consigo
uma tradição antiga (Tertuliano, Petrônio e Ausônio, a mística de
Swedenborg, a poesia tecnicamente impecável de Gautier), mas renovada,
e discernia onde algo novo estava surgindo, algo com que ele tinha mais
de uma afinidade, e afinidades mais do que superficiais.

Baudelaire é assim uma encruzilhada: nele se encontram enfeixados os
princípios anteriores do rigor formal e gélida superfície da arte
poética parnasiana, de Théophile Gautier (a quem não por acaso dedica
 \textit{Les fleurs du mal}), a retomada de lugares"-comuns do
romantismo (o satanismo, o \textit{spleen} etc.)\footnote{ Se o leitor
se espanta com a afirmação, basta ler em sequência poemas de Byron,
Álvares de Azevedo e Baudelaire, onde não raro encontrará vocabulário
específico comum e certas poses previamente calculadas para o efeito
que produzem. E o efeito é o da separação em relação a uma ideia de
“vulgo”. Temos o princípio desse artista que desafia as convenções de
seu meio social.}~e os primeiros exemplos da arte posterior, de
decadentistas e simbolistas (como leríamos em poemas como
“Correspondances”, no qual articula pela primeira vez a palavra
\textit{symboles}, “símbolos”, assim como apareceria depois no
movimento que tomou esse nome, aplicando"-o à sinestesia mística
colhida por Baudelaire na tradição do ocultismo).\footnote{ \textit{Símbolos}: 
palavra que não aparece na tradução brasileira de Ivan Junqueira,
substituída por “segredos” pelo motivo incômodo da rima. Percebendo a
necessidade de manter a palavra e tendo de recorrer a um \textit{enjambement}
discutível, Jamil Almansur Haddad conseguiu inseri"-la, iniciando com
ela o quarto verso da primeira estrofe do poema. Em Charles Baudelaire,
 \textit{As flores do mal} (trad., pref. e notas de Jamil A.
Haddad), São Paulo: Difel, 1964, p. 92.}~E, como veremos, está muito
longe de ser um exagero assinalar em sua obra poética indícios de uma
transformação de sensibilidade que teve desenvolvimento depois, no
modernismo.

O que, por outro lado, o afastou do mesmo modernismo\footnote{ No~manifesto 
milanês de Guillaume Apollinaire, “L’antitradition
futuriste”, de 1913, há uma amostra disso. Dividido em \textit{merde}/\textit{roses}, era
merda para os imprestáveis da tradição (Baudelaire, D’Annunzio,
Rostand, Shakespeare, “Dandismos” etc.), e, naturalmente, rosas para
Marinetti, Picasso, Boccioni, Apollinaire etc. Em Gilberto Mendonça Teles,
\textit{Vanguarda europeia e modernismo brasileiro}, Petrópolis:
Vozes, 1999, pp. 118--121.} foi menos sua atitude francamente
moderna (“é preciso ser absolutamente moderno”\footnote{ 
\textit{“Il faut être absolument moderne”}}, diria Rimbaud mais
tarde) do que seu esteticismo místico, perverso e preciso, coisas que ---
excetuando"-se a desagradável e complicada palavra “preciso” ---
atraíram um culto generalizado de imitadores péssimos em toda parte do
mundo. 

Então, Baudelaire foi de uma só vez o maldito, o recessivo e também o
estigmatizado por ter sido copiado, no mínimo por 50 anos, por quase
todo indivíduo que arriscou um verso naquela última metade do século
\textsc{xix} e no começo do \textsc{xx}. E mesmo um pouco depois, 
já que nunca devemos ignorar a disposição dos retardatários.

%\section*{o esteta, o dândi, o flâneur}
\section{Sobre a obra}

Nós temos, imagino, certa dificuldade em apreender o que terá sido um
\textit{esteta}, alguém desvinculado da preocupação imediata com as
coisas imediatas, particularmente converter seu trabalho em
dinheiro.\footnote{ A propósito do dandismo, Baudelaire escreve: “O
dândi não aspira ao dinheiro como a uma coisa essencial; um crédito
ilimitado poderia lhe bastar: ele deixa essa grosseira paixão aos
vulgares mortais”. Em Charles Baudelaire, op. cit., p. 49.} 
Teremos de fazer um pequeno esforço de
transposição mental para situar uma atitude razoavelmente orgulhosa na
arte, e razoavelmente devedora de um comportamento
aristocrático,\footnote{ 
\textit{De hoi aristoi}, em grego, significando “os
poucos”, oposto a \textit{hoi polloi}, isto é, “os muitos”.}
hoje bastante rarefeito.~O esteta tinha para si não a \textit{torre de marfim},\footnote{ Termo que Baudelaire
na época considerava um nobilitante, e achava que torre de marfim era apenas um repúdio ao vulgo.} 
onde o puseram os críticos quase sempre infelizes em suas definições
apressadas, mas o cuidado meticuloso do artifício. Baudelaire tinha o
sentido desse primado absoluto da estética, do desejo da perfeição na
beleza e no horror, mas sabia também que “o estudo do belo é um duelo
em que o artista grita de pavor antes de ser vencido”.\footnote{ Em “O
confiteor do artista”, neste livro.}

Seria subestimar demais a influência desses códigos em tudo o que se
seguiu depois se não traçássemos uma linha divisória, mostrando como
isso gerou uma crise no pensamento estético majoritariamente hegeliano
da época, a partir do qual se presumia que a arte buscava o belo e o
verdadeiro, o aspecto sublime. Tiveram de produzir o construto meio
absurdo “belo"-horrível” para descrever poemas como “La charogne” (“A
carniça”), ou, deste nosso livro, um texto como “Le chien et le flacon”
(“O cão e o frasco”), porque não era possível negar o artesanato verbal
diligentíssimo que se encarregava de esmiuçar feiúras. Com isso se pôs
em xeque a leitura idealizante, e se produziu um dos primeiros aspectos
que iriam separar a literatura do chamado modernismo do início do
século \textsc{xx} daquela do romantismo, porque onde Hugo era conciliatório e
melodramático (o feio de alma bela que era o sensível corcunda
Quasímodo), Baudelaire é intransigente: o leitor é um semelhante, um
irmão e, sobretudo, um hipócrita.

Mas não é apenas dessa matéria dura sem e contrastes que é feita a obra de
Baudelaire, claro. A leitora e o leitor perceberão que, se não cedem
às convenções de moral da sociedade, os \textit{Pequenos poemas em
prosa} se revelam muito morais num outro sentido, menos prescritivo: há
a simpatia pelo infortúnio em “Le gâteau” (“O bolo”) --- assim como desespero
em relação à camada brutal da psique humana ---, e em “Le joujou du pauvre”
(“O brinquedinho do pobre”), e até mesmo naqueles que se pudesse achar
mais terríveis, como “Le mauvais vitrier” (“O mau vidraceiro”), no qual
chega a matizar sua fúria, reconhecida como impulso inexplicável, e
“Assomons les pauvres!” (“Espanquemos os pobres!”), um tipo de
\textit{reductio ad absurdum}.

Ao esteta e ao \textit{dândi}, na figura de Baudelaire, junta"-se o
\textit{flâneur}, que é aquele que flana, ou vive a deambular, caminhar
sem rumo pela cidade. Isso é importante: o \textit{flâneur} não está
indo a lugar algum, ele está de passagem, e isso se reflete no modo
como esses textos devem ser lidos, pois eles resvalam na ideia da
cidade, que figura inúmeras sugestões ao observador atento e estético.
Baudelaire ainda dinamiza aquela incrível e insuspeita força moral
sobre seus objetos de atenção. São cenas passageiras, na rua ou dentro
de um teatro da imaginação, e, ao mesmo tempo, exibem uma fixidez de
eternidade, numa possível (e bastante óbvia) analogia com
a fotografia, que tem por objetivo recortar e congelar o momento. 

O \textit{flâneur} capta um momento anedótico, exemplar, e o poeta
Baudelaire o transforma num objeto no qual se encenam complexas visões
de estados permanentes da mente. Estados que, como vimos, aplicam"-se diretamente
à vida nas então recentes
metrópoles e, em particular, em Paris. E Paris, no livro, é mais (ou
menos) que uma cidade: é, como dito, um estado mental.

%O poeta nos dá um vislumbre de como opera essa transformação através de uma das palavras que intitula o livro, \textit{spleen}, um “tédio existencial”, mais ou %menos; porque, afinal de contas, Baudelaire poderia ter chamado o livro
%\textit{L’ennui de Paris} se quisesse dizer simplesmente “tédio”, como
%uma chateação qualquer.
%
%%\section*{spleen}
%%Esta é a palavra do título
%
%Termo tão específico e difícil de traduzir que Baudelaire, como tantos
%outros, preferiu deixar mesmo em inglês, assinalando não só o que
%queria dizer com a palavra específica, mas também sua filiação. Numa
%edição francesa do \textit{Spleen de Paris}\footnote{ Charles Baudelaire,
%\textit{Petits poèmes en prose --- Le Spleen de Paris} (préf. et
%comm. Pierre"-Louis Rey), Paris: Pocket, 1998. Diderot: “Le
%spline, ou les vapeurs anglaises”, pp. 156--7.} oferece"-se como
%apoio um texto de Diderot no qual o autor diz encontrar um escocês que
%lhe explica o significado da palavra, levando pouco mais de uma página
%para tanto, a partir da qual Diderot conclui sem muita esperança: “Sua
%tristeza é original, e não é triste”. Pode nos lembrar um pouco, de
%fato, a “boa melancolia”, a imaginativa, aquela valorizada na
%Renascença e pouco depois, como vemos na célebre gravura de Albrecht
%Dürer, \textit{Melencolia \textsc{i}} (1514), mas isso ainda não resolve o
%problema.
%
%Voltaire, como sempre, tinha algo a dizer sobre o assunto. Escreveu o
%seguinte: “Os ingleses, com efeito, chamam a essa enfermidade
%\textit{spleen}, que eles pronunciam splin, palavra que significa baço.
%Nossas mulheres outrora sofriam dessa afecção do baço… Os ingleses têm
%o splin, ou a splin, e se matam por humor”.\footnote{ “Dossier”, em
%\textit{Petits poèmes en prose}, op. cit., p. 156.} No antigo esquema dos quatro
%humores (sanguíneo, melancólico, colérico e fleumático), o baço
%secretaria, por influência de Saturno, a bile negra do melancólico. Não
%por acaso, um dos mais famosos e baudelairianos livros de Paul Verlaine
%iria se chamar \textit{Poèmes Saturniens} (Poemas
%saturnianos).\footnote{ É possível até mesmo ver a imagem do crepúsculo
%que Verlaine iria popularizar na literatura decadentista já desenhada
%em “Le Crépuscule du soir” (“O crepúsculo da tarde”), neste livro.}
%
%Mas quem codificou o significado dessa palavra usada largamente durante
%o período romântico e por todos os imitadores de Baudelaire foi George
%Gordon Byron, ou Lord Byron, como ficou conhecido o belo barão que
%diabolicamente mancava de uma perna. Temos de lembrar qual era o código
%implícito: diz"-se que o poeta é um gênio; que, por sê"-lo, deve
%suportar o estar cercado de imbecis; que, cercado por eles,
%refugia"-se na extravagância de seus hábitos inexplicáveis
%socialmente; e que esse refúgio desenvolve como consequência uma
%espécie de melancolia pensante, o \textit{spleen}. E daí 
%Baudelaire chegará até mesmo, neste nosso livro, a derivar um
%adjetivo,\textit{ spleenétique}.\footnote{ Artur Azevedo
%(1855--1903) seguiu o exemplo escrevendo “esplenético” num curioso
%poeminha chamado “Que horror”.}
%

%\section*{spleen de paris}
%Para concluir, tendo comentado brevemente Charles Baudelaire, a forma
%deste seu livro e as imediações literárias em que se moveu, vamos nos
%voltar um pouco para certos aspectos das peças literárias que o
%compõem. 

Característico da experiência de uma metrópole, esse livro comporta de tudo um pouco: ele começa com “L’étranger” (“O estrangeiro”), um
brevíssimo intróito dialogado, lembrando que as pessoas são estranhas
quando você é um estranho; passa por pequenos poemas narrativos de
momentos como uma espécie de iluminação às avessas (quero dizer,
normalmente incluindo alguma crueldade\footnote{ Crueldade que, tendo
começado em Sade, atinge esse ponto peculiar em Baudelaire, que dará
depois em Villiers de L’Isle Adam e seus \textit{Contes Cruels}, em Huysmans e
sua perversidade metálica e mineral, como qualifica Gaston Bachelard em
\textit{A terra e os devaneios da vontade: ensaios sobre a imaginação das
forças} (trad. Paulo Neves da Silva), São Paulo: Martins Fontes,
1990. “O devaneio petrificante”, pp. 165--177, partes \textsc{i} e \textsc{ii}.}); pela
alegoria cinzenta de “Chacun à sa chimère” (“Cada qual com sua quimera”)
e pelo conto de fadas de “Les dons des fées” (“Os dons das fadas”); e
termina num epílogo em verso, o único pedaço realmente em verso do
livro, e que, não por acaso, está escrito em \textit{terza
rima},\footnote{ Tercetos de rima encadeada.} como a \textit{Divina
Commedia} dantesca --- recuperando aquela relação infernal estabelecida
por Eliot em seu binômio Dante"-Baudelaire ---, onde a cidade por fim
recebe uma declaração de amor evocando o grande patrono Satã, as
prostitutas, a cidade que é “hospital, lupanar, purgatório, inferno,
prisão”.

Não é um percurso como por dentro da cidade de Paris, nem também do
drama à poesia em algumas páginas, embora possa ser isso se você
desejar ver assim; mas é sobretudo um percurso dentro da mente de
Baudelaire, pelas cristalizações de seu pensamento. Uso a palavra
\textit{cristalizações} para acentuar o fato de que cada um desses
poemas em prosa se abre para a comunhão natural de ideias e enfoques
com qualquer outro texto de Baudelaire, seja sua crítica de arte, os
diários íntimos e fragmentários das \textit{Fusées} ou de “Mon
c\oe ur mis à nu” (“Meu coração a nu”, mais explícito, impossível) e
mesmo, como vimos, em \textit{Les fleurs du mal}. 

%\section*{o poema em prosa}

\section{Sobre o gênero}
\textit{Le Spleen de Paris} praticamente inaugura um gênero,\footnote{
Sei que é uma afirmação razoavelmente polêmica. Max Milner, da edição
do \textit{Gaspard de la nuit} da Gallimard, afirma que foi Bertrand, sem sombra
de dúvida, quem criou o gênero, e foi “publicado, para a indiferença
geral, em 1842”.} se descontarmos o \textit{Gaspard de la nuit}, de
Aloysius Bertrand, que Baudelaire diz na carta a Arsène Houssaye ter
lido uma boa porção de vezes.\footnote{ Mallarmé também adorava o livro
de Bertrand.}~É fato que o livro de Bertrand possui seu charme, mas é
igualmente fato que não se propõe ao rigor da concisão poética, e
funciona mais como um regalo para o diletante educado, que encontrará
ali uma prosa de certa elegância e cheia de reminiscências e anedotas.
%como a do surgimento de uma herética barba pontuda numa sinagoga:

%\begin{quote}
%E eis que, de repente, em meio a tantas barbas redondas, ovais,
%quadradas, em flocos, ou frisadas, que exalavam âmbar e benjoim,
%notou"-se uma barba aparada em ponta.
%
%Um doutor chamado Elébotham, enfeitado de uma meda de flanela que
%brilhava de pedrinhas, se levanta e diz: “Profanação! Há aqui uma barba
%pontuda!”\footnote{ Aloysius Bertrand, \textit{Gaspard de la nuit: Fantasies a
%la manière de Rembrandt et Callot} (édition présenté, établie et annotée
%par Max Milner), Paris: Gallimard, 1980, “École flamande”, \textsc{iv}, p. 93.}
%\end{quote}
%
%Esse episódio, que culminará com uma queda no Reno, ou o mercador de tulipas,
%o alquimista, “a velha Paris” de cenas pitorescas, são todos
%tratados como minicontos, pequenas histórias inteligentes, mordazes
%e curiosas, como aquelas com as quais Apollinaire iria mais tarde
%deleitar seu círculo de amigos famosos e um grupo de leitores sempre
%seleto, de paladar literário cosmopolita.

Já o livro de Baudelaire perfaz um todo --- apesar de ele alegar, com certa
justiça, que não tem “pé nem cabeça” --- se notamos que a diversidade dos
textos remete a mais do que a mera curiosidade exótica, ou caráter
anedótico de um episódio qualquer: há a singularidade da voz que os
enuncia; o permanente e tenso equilíbrio (por vezes mesmo
\textit{oposição}) entre multidão / indivíduo; e até mesmo o contato
profundo com poemas de \textit{As flores do mal}, com o qual partilha
títulos e mesmo o desenvolvimento de alguns textos, que ganham duas
abordagens e podem ser lidos com proveito, comparativamente, pelo
leitor e pela leitora curiosos.\looseness=-1

Um exemplo é “L’invitation au voyage” (“O convite à viagem”), que não
divide apenas o título com seu poema"-irmão em \textit{As flores do
mal}: em ambos há o sonho da viagem para um lugar mirífico de prazeres
contínuos. Nos \textit{Pequenos poemas em prosa}, ele tem o nome daquele
topônimo fantasioso da Idade Média, \textit{Cocagne} (Cocanha), “onde
o luxo tem prazer de se mirar na ordem”, é “uma China ocidental”,
repleta de adornos, riquezas, prazer; e o poema em versos se transforma
num verdadeiro convite a esse sonho orientalizante e perfumado, um
poema musical e inesquecível (\textit{“Là, tout n’est qu’ordre et
beauté, / Luxe, calme et volupté.”}\footnote{ “Lá, tudo é ordem e beleza, / 
Luxo, calma e volúpia.”}) que recebeu uma imitação abrasileirada de
Manuel Bandeira no “Vou"-me embora pra Pasárgada”. Nos dois poemas de
Baudelaire esse país extraordinário parece com sua convidada, “minha
criança, minha irmã”.

Em todo caso, e não obstante as diferenças entre as obras, a de
Baudelaire e a de Bertrand, estavam as duas nas luxuosas estantes de
livros de Des Esseintes, personagem emblemático do decadentismo, do
romance \textit{A rebours} (1884), de J."-K. Huysmans, que escreve no
capítulo \textsc{xiv} sobre Des Esseintes folheando: “uma outra plaquete que ele
fizera imprimir para seu uso, uma antologia do poema em prosa, um
pequeno santuário posto sob a invocação de Baudelaire, e aberto sobre o
átrio de seus poemas. Essa antologia compreendia uma seleta do
\textit{Gaspard de la nuit}, desse fantasioso Aloysius Bertrand, que
transferiu os procedimentos de Leonardo para a prosa e pintou, com
óxidos metálicos, pequenos painéis em que as cores vivas cintilam,
assim como aqueles esmaltes lúcidos”.\footnote{ J."-K. Huysmans,  \textit{A
rebours} (avec un préface de l’auteur écrite vingt ans après le roman),
Paris: Fasquelle, 1947, p. 263.}  E este é apenas um trecho da
longa descrição do amor de Des Esseintes pelo poema em prosa.

O que é, enfim, o poema em prosa, que praticamente se autonomizou como
um gênero? Tentemos uma explicação por contraste: Jean"-Jacques
Rousseau escreveu um livro, publicado postumamente (1782), chamado
\textit{Les rêveries du promeneur solitaire}, ou “Os devaneios do
caminhante solitário”: dele se disse que poderia ser entendido como
“prosa poética”, porque utiliza um grupo de recursos mais comuns na
poesia, como aliterações, ecos (rimas), um ritmo demarcado mais
claramente. Mas não tem, por outro lado, preocupação alguma de
restringir o que se escreve muito prosaicamente, quase \textit{manu
currente}.~Empregaria, assim, efeitos superficiais da poesia, mas não
se escreveria como tal. Vamos ler o início da “Première Promenade”,
onde peço que o leitor preste atenção às copiosas aliterações, ao ritmo
\textit{flamboyant}, à rima (itálicos, negritos e sublinhados para
destacar esses usos):


\begin{quote}
Me \underline{voici} donc \underline{seul sur} la \textit{terre}, n’ayant plus
de \textit{frère}, de prochain, d’\textbf{ami}, de \underline{société} que
\textbf{moi"-même}. Le plus \underline{sociable} et le plus \textbf{aimant} des \textbf{humains} en a été
proscrit par un accord \textbf{unanime}.\footnote{ “E eis que estou nessa vida sem pessoa amiga, irmão, um próximo ou
sociedade fora eu"-mesmo. O mais sociável e o mais amorável dos homens,
proscrito em unânime acordo.” Jean"-Jacques Rousseau,  \textit{Les
Revêries du Promeneur Solitaire} (édition critique publié d’après les
manuscrits autographes par John S. Spink), Paris: Marcel
Didier, 1948, p. 3.}
\end{quote}

Os \textit{Pequenos poemas em prosa} são o contrário da prosa poética de Rousseau,
 e assim a ordem das palavras na definição “pequenos poemas em prosa” 
deve ser observada: são antes de mais nada poemas, escritos utilizando 
aquela outra forma, a prosa, como veículo; da poesia guardam o inevitável 
específico da alta compressão de significado, e emprestam da prosa da \textit{clarté}
 francesa seu desenho de nitidez, da exposição direta, quase sem ornamento; 
portanto, o hábito delicado que alguns críticos têm de chamar 
seus parágrafos de “estrofes”, embora simpático, não faz lá muito sentido. E
em “Mademoiselle Bistouri” (“Senhorita Bisturi”), a atitude irônica a
respeito do prosaísmo na poesia francesa pode ser lida quando escreve,
diante da casa da mulher que acompanhava: “Omito a descrição do
casebre; pode ser encontrada em diversos velhos poetas franceses bem
conhecidos”. E isso nos dá também o teor de sua escrita.

%
%\begin{quote}
%Rose couleur de cuivre, plus frauduleuse que nos joies, rose couleur de
%cuivre, embaume"-nous dans tes mensonges, fleur hypocrite, fleur du
%silence.
%
%Rose au visage peint comme une fille d’amour, rose au
%cœur prostitué, rose au visage peint, fais semblant
%d’être pitoyable, fleur hypocrite, fleur du
%silence.\footnote{ “Rosa cor de cobre, mais fraudulenta que os nossos gozos, rosa cor de
%cobre, embalsama"-nos em tuas mentiras, hipócrita flor, flor do
%silêncio. / Rosa de rosto pintado como uma cortesã, rosa de peito prostituído, rosa
%de rosto pintado, finges ser digna de pena, hipócrita flor, flor do
%silêncio.” Remy de Gourmont,  \textit{Le pélerin du silence}, Paris:
%Mercure de France, [1920], “Livre des Litanies”, p. 151.}
%\end{quote}
%
%\textls[-15]{E assim por diante, no mesmo esquema, por treze páginas.}

Um amálgama, por assim dizer, de Baudelaire e Rousseau daria em algo
como os longos versículos das ``Litanies de la rose'', de Remy de
Gourmont, publicadas numa plaquete de luxo em 1892.
Walt Whitman já havia, então, circulado no original e em traduções
feitas por poetas franceses, como Jules Laforgue; e Gustave Kahn já
havia publicado suas experiências e opiniões inteligentes sobre o
\textit{vers libre}.\footnote{ Num artigo notável para a \textit{Revue
indépendante}, em 1888.}

Não há dúvida, portanto, de que o caso dos poemas em prosa é um
daqueles que oferecem ao futuro um caminho praticável. Há
neles inclusive um sentido de \textit{fragmento}, como Schlegel já
começara a perceber na literatura romântica, que opera depois nas
breves narrativas deste livro, e é o próprio Baudelaire que nos situa
quanto a esse aspecto, quando escreve a Houssaye: 

\begin{quote}
Podemos interromper
onde quisermos, eu meu devaneio, você o manuscrito, o leitor sua
leitura; pois não mantenho suspensa a recalcitrante vontade deste
último ao fio interminável de uma intriga supérflua. Retire uma
vértebra, e os dois pedaços desta tortuosa fantasia irão se juntar sem
dificuldade. Lacere"-a em diversos fragmentos, e verá que cada um
deles pode existir à parte.\footnote{ Tradução de Dorothée de Bruchard.
Daí a razão de Baudelaire em dizer o “sem pé nem cabeça”.}\looseness=-1
\end{quote}

O que significa a percepção fragmentária da realidade no momento em que
Baudelaire escreve? Num ensaio excelente, “Sobre alguns temas em
Baudelaire”, Walter Benjamin de certa forma aborda a questão, mostrando
que está, no fundo, intimamente ligada às massas das regiões
superpopulosas.~Apresenta a propósito disso um dos melhores sonetos de
\textit{As flores do mal}, “A une passante” (“A uma passante”), no qual uma
 mulher belíssima é flagrada em meio ao “frenético alarido”\footnote{ A
expressão é de Ivan Junqueira, traduzindo o primeiro verso que diz, no
original, “La rue assourdissante autour de moi hurlait”, em
Charles Baudelaire, \textit{As flores do mal} (bilíngue; trad.,
intr. e notas de Ivan Junqueira), Rio de Janeiro: Nova Fronteira,
1985, pp. 344--5.} das multidões na rua. Mas aquele instantâneo
deslumbramento é desfeito em meio à turba que os separa. O terceto
final diz:


\begin{quote}
Longe daqui! tarde demais! nunca \textit{talvez}!\\ 
Pois de ti já me fui, de mim já tu fugiste,\\
Tu que eu teria amado, ó tu que bem o viste!\footnote{Ibid., p.\,345.}
\end{quote}

E Benjamin comenta: “O significado do soneto numa frase é o seguinte: a
aparição que fascina o habitante da metrópole --- longe de ter na
multidão somente a sua antítese, somente um elemento hostil --- é
proporcionada a ele unicamente pela multidão. O êxtase do citadino é um
amor não já à primeira vista, e sim à última. É uma despedida para
sempre que, na poesia, coincide com o instante do enlevo”.\footnote{
 Walter Benjamin, “Sobre alguns temas em Baudelaire” (trad. Edson
Araújo Cabral e José Benedito de Oliveira Damião), em \textit{Os pensadores}
(volume \textsc{xlviii}), São Paulo: Abril Cultural, 1975, p. 44.} É assim que
funciona a percepção fragmentária, porque ela anota que o ritmo das
coisas foi modificado, que muitas delas ficam então pelo caminho como
pedaços a que não se poderia dar um desenho conclusivo e redondo; que
não seria exato como representação, também, prover essa percepção com a
elasticidade continuada de um romance, por exemplo. Quando chegamos ao
século \textsc{xx} e a obras como \textit{Ulysses}, de James Joyce, o fragmento
é praticamente a regra, e se destaca como componente estrutural do
próprio romance,\footnote{ Joyce iria também aprofundar a proximidade
entre prosa e poesia no \textit{Finnegans Wake}.} revolucionando"-o.

Baudelaire será seguido nessa forma dos poemas em prosa por dois dos
mais importantes poetas franceses do fim do século \textsc{xix} (seus
confessos admiradores): Stéphane Mallarmé e Arthur Rimbaud. E deles a
árvore ganhará novos ramos.

\hyphenation{fo-lhean-do}
\hyphenation{de-li-cio-sas}
\hyphenation{real-men-te}
\hyphenation{bea-ti-tu-de}
%\hyphenation{ex-pe-riên-cia}
\hyphenation{in-con-ve-nien-tes}
\hyphenation{mis-te-rio-sas}
\hyphenation{si-len-cio-sa-men-te}
\hyphenation{sua-ve-men-te}
%\hyphenation{crian-ça}
\hyphenation{re-crea-va}

\textls[-10]{Mallarmé escreverá alguns poemas em prosa também, claramente baseados no
modelo dos \textit{Pequenos poemas em prosa}: basta lermos “O fenômeno futuro”, “Lamento de outono”, “Pobre criança pálida”, “\textit{Frisson} de inverno” e “O demônio da
analogia”. É claro, já não são mais como os de Baudelaire, porque
Mallarmé mergulha num mundo que as palavras reconstroem com arbítrio
próprio. Sobretudo em “O demônio da analogia”, recuperando ideias de
Poe, Mallarmé aponta inclusive para a técnica inovadora daquele que
será seu texto mais famoso e influente, o “Coup de dés”,
distanciando"-se da referência a Baudelaire e propiciando um novo
artesanato construtivo que seria explorado mais tarde pelas vanguardas.}

Vindo também do mesmo \textit{poème en prose} de Baudelaire e indo parar
bastante longe do desenvolvimento\footnote{ Que o leitor entenda
“desenvolvimento” como um fazer diferente, e não melhor. Não é, em
nenhum desses três casos, questão de valor, mas um bem prático “a
partir de”.} que lhe aplicou Mallarmé, Rimbaud escreverá
\textit{Uma temporada no inferno} e as
\textit{Iluminações}, ou \textit{Iluminuras}. Rimbaud daria o passo decisivo, nesses
dois textos (juntamente com os \textit{Cantos de Maldoror}, de
Lautréamont), para aquilo que veio a se chamar surrealismo, pelo modo
brusco como levou imagens a colidirem através do “total desregramento
dos sentidos”, o que o diferencia claramente de Baudelaire, borrando
com mais insistência os limites entre prosa e poesia, entre linguagem
inteligível e uma fala poética cujas regras são quase imperceptíveis,
estão no limiar do sonho (ou pesadelo).

Além deles, todos os decadentistas e simbolistas mencionados aqui
derivam muita coisa das descobertas e da sensibilidade de Baudelaire;
num contínuo de sentido, em muitas de suas particularidades, portanto,
também o modernismo internacional.

A multidão ganha visões a partir do desbravamento de Baudelaire. Em \textit{The
Waste Land}, de T. S. Eliot, uma das referências na primeira parte do
poema, chamada “The Burial of the Dead”\footnote{ Eliot escreve: “A
crowd flowed over London Bridge, so many, / I had no thought death had
undone so many”, ou “Multidões inundam a Ponte de
Londres, tantos, / Eu não imaginava que a morte tivesse aniquilado
tantos”. T. S. Eliot, \textit{Collected Poems} (1909--1962), Londres: Faber \&
Faber, 1974, p. 65.} (“O enterro dos mortos”), é justamente a
\textit{cidade formigante, cidade repleta de sonhos}\footnote{
“Fourmillante cité, cité pleine de rêves”, no original.} de “Les sept
vieillards” (“Os sete velhos”), de \textit{As flores do mal}, mas numa
combinação particularmente curiosa: a referência surge da visão de
multidões na ponte de Londres e é cruzada com uma paráfrase de versos
de Dante Alighieri na \textit{Divina commedia}, mais especificamente no
Canto \textsc{iii} do “Inferno”, que dizem: […] \textit{ si lunga tratta / di gente,
ch’io non averei creduto / che morte tanta n’avesse disfatta}, ou “[…]
enorme multidão surgia, / tantos que eu não podia imaginar / tivesse a
morte aniquilado um dia”\footnote{ Dante Alighieri,  \textit{A Divina comédia}
(trad., coment. e notas de Cristiano Martins), São Paulo/Belo
Horizonte: Edusp/Itatiaia, 1979, p. 122.}.  Eliot já vê a turba como
uma massa inumana, infernal, e estabelece esse circuito bastante
fecundo de Dante a Baudelaire.

É essa, num aperitivo que espera não estragar o apetite, a refeição que espera os leitores. Como dirá o próprio Baudelaire: \textit{Enivrez-vous!} Embriaguem-se.


