\chapterspecial{Sob o pôr do sol}{Do gótico ao \emph{Drácula}}{Rafael Rocca dos Santos}

A literatura gótica floresceu na Inglaterra como uma resposta às ideias
e movimentos decorrentes do Iluminismo oriundo da França no século
\versal{XVIII}, cuja característica principal consistia em afirmar a primazia da
razão sobre as demais formas de pensamento e a religião em geral. O~movimento propôs uma análise da sociedade tendo como ponto de partida a
observação empírica dos costumes, das leis, do comportamento, entre
outros, o que influenciou sobremaneira a literatura da virada do século
\versal{XVIII} ao \versal{XIX}. Assim, à medida que ascendia a instâncias cada vez mais
altas nos planos econômico, político e científico"-intelectual, a
burguesia veio a encontrar no romance sua forma de expressão literária
por excelência.

No entanto, alguns autores ingleses desviaram"-se do afã provocado pelo
nascimento das ideias iluministas de esclarecimento intelectual e
progresso. Tais autores, cujos principais representantes são Ann
Radcliffe\footnote{1729--1807, autora de \emph{O~velho barão
inglês} (1777). É~digno de nota que um grande número de novelas do
gênero gótico foi escrito por mulheres, algo até então incomum.},
Horace Walpole\footnote{1717--1797, autor de \emph{O~castelo de
Otranto} (1764).}
 e Matthew Gregory
 Lewis\footnote{1775--1818, autor de \emph{O~monge} (1796).}, 
ao mesmo tempo que aceitavam as mudanças
provocadas pelo pensamento racionalista, colocavam"-no em xeque,
valorizando e explorando dimensões sombrias e sobrenaturais da
experiência, que seriam indevassáveis pelas luzes da razão. Os autores
do gênero gótico combinam a modernidade da medicina e dos transportes
com a atmosfera medieval de castelos frios, cheios de salas secretas e
passagens subterrâneas sombrias; o refinamento dos novos costumes com o
barbarismo e a excentricidade; a descrição realista das ações e dos
ambientes com sentimentos de desolação e abandono. Atingindo seu auge na
década de 1790\footnote{\versal{VASCONCELOS} (2002), p\,130.}, a literatura gótica influenciou
muitos autores de gerações posteriores que não se dedicaram ao gênero.
Por exemplo, Jane Austen, que em \emph{A~abadia de Northanger} (1818)
conta uma história na qual a protagonista é leitora de romances góticos
(tal como a própria Austen). Centrando sua narrativa na discussão desse
tipo de literatura, a autora busca realçar seus aspectos interessantes e
criticar seus pontos fracos. Já entre os continuadores do gótico,
destaca"-se justamente Bram Stoker: seu \emph{Conde Drácula}, ainda que
produzido bem depois do auge do gênero, é considerado o romance gótico
por excelência.

Bram Stoker nasceu em 1847, em Dublin, Irlanda. Logo após seu
nascimento, foi acometido por uma doença desconhecida, que o deixou
acamado e afastado do convívio social até os sete anos de idade. Durante
esse período de convalescença e isolamento, seus familiares liam para
ele histórias, contos de fadas e breves narrativas sobre diversos
assuntos que lhe pudessem interessar. Já então Stoker teve um primeiro e
intenso contato com a literatura gótica, que surgira recentemente e era
bastante consumida. Aos sete anos, e sem qualquer explicação, o menino
Stoker recuperou sua saúde e iniciou os estudos. A~respeito dessa nova
fase, Stoker escreve, em uma obra biográfica sobre seu amigo Henry
Irving, um ator muito famoso e seu amigo próximo: ``Eu era naturalmente
pensativo, e o ócio de uma longa doença forneceu a oportunidade a muitos
pensamentos que foram frutíferos, de acordo com sua qualidade, em anos
posteriores''. Esse dado biográfico é relevante e está refletido em
diversos temas tratados ao longo de toda a sua obra, tais como a
impossibilidade da fala e dos movimentos e a oposição entre os ambientes
fechados (claustro) e o mundo externo.

Jovem adulto, Stoker inicia sua carreira como crítico de teatro no
periódico \emph{Dublin Evening Mail}. Em uma de suas críticas, escreve
sobre uma montagem de \emph{Hamlet} estrelada por Henry Irving, fato que
deu ensejo à duradoura amizade entre ambos. Literariamente, seu
\emph{début} se deu com contos publicados em jornais a partir de 1872.
Seguiram"-se romances e coletâneas de contos, destacando"-se a presente
obra, \emph{Sob o pôr do sol} (publicada em 1882), e sua obra máxima,
\emph{Drácula}, publicada em 1897.

\emph{Drácula} é um dos romances mais conhecidos da literatura inglesa
do século \versal{XIX}, tendo sido desde então adaptado para teatro, cinema e
usado como referência ou ponto de partida para outras obras literárias.
O~tema central do romance, o vampirismo, que está presente em lendas e
mitos antigos, encontrou um grande desenvolvimento no gótico literário
inglês. Bram Stoker partiu de tais lendas e de referências literárias
para desenvolver sua própria concepção de vampirismo, acrescentando"-lhe
características peculiares e inéditas. \emph{Drácula} possuiu uma longa
gestação, cerca de sete anos, durante os quais o autor reuniu mais de
cem páginas manuscritas com informações acerca de costumes, modos,
atitudes, localizações geográficas, reações médicas, relatos de viagem e
referências literárias, todo um material de pesquisa que foi em larga
medida processado e incorporado à obra.

Literariamente, Stoker bebeu das mais diversas fontes, em especial da
poesia e da prosa inglesas, norte"-americanas e alemãs. Neste breve texto
introdutório, porém, seria inoportuna a análise de toda essa gama de
influências. Aqui, basta que se aponte para o aspecto fundamental da
ficção do autor, que está presente tanto em \emph{Sob o pôr do sol} (uma
obra de juventude) quanto em Drácula (uma obra de maturidade): a
articulação da narrativa com base em dualidades de extremos como ``bem e
mal'', ``luz e trevas'', ``amor e ódio''.

O livro \emph{Sob o pôr do sol} inicia com um conto cujo título é
justamente ``Sob o Pôr do Sol''. Trata"-se de um conto"-moldura, que cria
uma espécie de quadro, com características próprias, ``dentro'' do qual
deve ser feita a leitura dos contos subsequentes. Por meio dessa técnica
literária, Stoker estabelece um fio condutor que alinhava todas as
narrativas do livro, definindo o lugar e o tema de todas as histórias,
respectivamente: o País Sob o Pôr do Sol e as lutas travadas entre o
``bem'' e o ``mal''.

O País Sob o Pôr do Sol tem em seu centro um castelo, é governado por um
Rei bom e povoado por pessoas puras, pias e dedicadas ao bem. Na região
fronteiriça por que passa a única estrada que liga o país ao exterior, a
terra é mais seca, com menos vida e menos luz. É~nessa fronteira que se
ergue o Portal guardado por dois anjos celestes. Para além dele há outro
país, de terra desolada, em cujo centro habita outro Rei (personagem do
conto ``O Castelo do Rei''), que é exatamente o oposto do Rei bondoso e
caridoso do País sob o Pôr do Sol. Nessa outra terra, a paisagem é
assustadora: somente trevas e vales escuros. Os habitantes são todos
seres os mais repugnantes: desde animais peçonhentos a fantasmas e
espíritos de não mortos e não vivos. Enfim, além do Portal, está tudo o
que há de perigoso e repugnante às pessoas puras do País Sob o Pôr do
Sol. A~oposição entre ``bem'' e ``mal'', que confere uma base religiosa
cristã para o conto (Stoker era ele mesmo um homem religioso) já está
delimitada no primeiro dos contos da coletânea e retornará, em maior ou
menor grau, em todas as demais narrativas\footnote{Segundo uma leitura psicanalítica, o conto
``País sob o Pôr do Sol'' representa um \emph{claustrum}, um lugar
análogo ao útero materno, um estado primitivo de conforto e nutrição, do
qual estariam afastados os perigos do mundo exterior (v. \versal{BIERMAN}, 1988,
p\,167).}.

O segundo conto do volume, ``O Príncipe da Rosa'', trata da batalha do
filho do Rei do País sob o Pôr do Sol contra um gigante que foi enviado
de fora para castigar as pessoas que se desviam do caminho justo e
incorrem em pecados. Com essa história, Stoker parece sugerir que toda
disputa violenta sempre tem um efeito nocivo sobre a povo, que sofre com
suas consequências mesmo quando a luta é vencida pelo lado virtuoso e
não exige a participação direta da população. Por mais que não pareça
ter uma influência gótica tão acentuada, ``O Príncipe da Rosa''
apresenta uma característica típica do gênero: a fabulação de costumes e
perversões de uma perspectiva maniqueísta.

Os elementos do gótico propriamente dito começam a aparecer no terceiro
conto do livro, ``O Gigante Invisível''. No País sob o Pôr do Sol, com o
argumento de que o mal não tem capacidade de triunfar nesse mundo, as
pessoas relevam a boa conduta e a atenção à solidariedade. Em meio a
esse ambiente de displicência moral, uma menina órfã, Zaya, enxerga
aproximando"-se da cidade um Gigante imenso, que traz consigo os males do
mundo como castigo às atitudes perversas do povo. No entanto, ele é
visto somente pela menina, que tenta em vão avisar a todos de sua
aproximação funesta\footnote{O Gigante simboliza as pragas bíblicas do
\emph{Velho Testamento}. É~interessante notar o paralelo entre a
personagem de Zaya, neste conto, e a de Cassandra, na mitologia. Segundo
algumas versões do mito grego, o dom da profecia foi dado a Cassandra
por Apolo. Entretanto, uma vez que ela não cedia aos avanços do deus,
este lhe retira o poder do convencimento mas preserva o da profecia.
Como resultado, Cassandra passa a alertar a todos acerca dos males por
vir sem jamais ser escutada.}. O~conto então se reveste de
uma atmosfera sombria, e o elemento da morte começa a aparecer com mais
intensidade, conforme mostram as belas ilustrações originais que também
estão presentes nesta edição.

É no quarto conto da obra, ``O Construtor de Sombras''\footnote{A história foi adaptada para o cinema em
1988, sob o título \emph{Shadow Builder} (em português, \emph{O~senhor
das sombras}).}, que o gótico está mais bem representado. A~personagem que dá nome à narrativa é um ser sombrio, esquivo, solitário,
que habita um local incerto. Ele constrói sombras das vidas de seres
humanos que passarão por um limiar entre a vida e a morte, viverão e
enfim retornarão a uma procissão de sombras, que permanece a vagar sem
rumo, eternamente. O~terror, o horror e o medo, sentimentos arraigados
no mais íntimo do ser humano, estão explicitados na atitude do
Construtor de Sombras e no lugar sobrenatural em que atua e habita. Em
vez do sonho presente no conto ``Sob o Pôr do Sol'', agora o pesadelo se
torna protagonista. Tal clima de pesadelo é intensificado ao máximo na
passagem em que uma mãe se esforça por salvar seu bebê, que atravessou o
limiar entre a vida e a morte\footnote{Essa passagem em que uma mãe parte em busca
de seu filho parece inspirada pelo romance \emph{Robinson Crusoe}
(1719), de Daniel Defoe.}. Trata"-se de um
dos momentos mais belos e poéticos do todo o livro. O~tratamento
literário desse trecho é perceptível até mesmo na pontuação, manejada
por Stoker com maestria: as frases tornam"-se mais curtas e o andamento é
acelerado, de modo a criar um clima de suspense e preparar o desfecho.

O interesse que Stoker possuía pelas ciências exatas, que em sua época
avançavam a plenos pulmões, é bastante perceptível no quinto conto da
coletânea, ``Como o 7 ficou louco''. A~história lida de maneira
emblemática com o mais importante e característico tema da literatura
gótica: o conflito entre o racional e o irracional. No entrecho, um
professor conta uma história acerca do número 7, ressaltando a
importância do aprendizado da matemática. Pouco depois, um de seus
alunos desdenha desse número e desaparece misteriosamente do cotidiano
de seus colegas. Em uma virada inesperada, o fim do conto contém
elementos do grotesco e do estranho, que realçam justamente a tensão
entre o racional e irracional.

O sexto conto, ``Mentiras e lírios'', está ligado ao tema da mentira,
recorrente na obra de Bram Stoker. A~história relata como a menina
Claribel\footnote{O nome evoca a pureza da personagem,
relacionando"-a tanto à ``luz'' e à ``claridade'' quanto à ``beleza''.} sofre a influência de um ``Espírito
Mau'', Skooro (que já aparece no primeiro conto como influência negativa
no mundo) e de um espírito bom, Chiaro\footnote{Ambos os nomes compõem mais uma das
múltiplas dualidades presentes em \emph{Sob o pôr do sol}: ``Chiaro''
evoca o termo homógrafo italiano para ``claro, luminoso, ilustre'',
representando o ``bem''; já ``Skooro'' evoca ``escuro, obscuro, triste,
sombrio'', representando o ``mau''. Também é digno de nota que ambas as
palavras estejam presentes no termo \emph{chiaroscuro}, usado para
designar os efeitos de luz e sombra da pintura renascentista.}. Sua
professora prega que os mentirosos sejam condenados ao ostracismo, uma
pena que consiste em denegar"-lhes as benesses da vida ao bani"-los do
País Sob o Pôr do Sol\footnote{Tal como a professora neste conto, o Sr.~Swales, no \emph{Drácula}, também prega contra a mentira, condenando as
falsidades inscritas nas lápides dos túmulos da abadia de Whitby. Ambas
as personagens também se assemelham pela personalidade doce e amigável.}. Aqui, a oposição entre
``dentro'' e ``fora'' retoma o contraste bíblico entre o ``digno'' e o
``indigno'' e, por extensão, entre o ``bom'' (pessoa que fala a verdade)
e o ``mau'' (pessoa que mente).

Assim como em ``O Construtor de Sombras'', o sétimo conto da coletânea,
``O Castelo do Rei'', traz os elementos do gótico ao primeiro plano. Já
na frase de abertura, a morte e a loucura aparecem de modo explícito,
dando ao conto um tom sombrio que gradualmente se intensifica. O~protagonista é chamado de ``Poeta'', alegoria de um tipo humano
associado ao ato da criação, ao desbravamento de distâncias físicas e
imaginárias por meio da poesia. Desde o mito grego de Tirésias, os
poetas são tidos ``como os cegos, podem ver na escuridão'' -- comparação
que se ajusta perfeitamente ao conto. Desejando resgatar (ou se unir à)
sua amada no temível castelo do Rei da Morte, que fica além do Portal do
País Sob o Pôr do Sol e simboliza um lugar de antítese completa aos
valores que ali reinam, o Poeta parte em uma jornada por terras
sombrias. Atravessa o Vale das Sombras\footnote{Clara referência bíblica ao Salmo 23,
intitulado ``O bom Pastor'', relacionado à tradição de Davi.
},
escutando a Música das Esferas\footnote{A concepção de uma ``música das esferas''
provém de Pitágoras, segundo o qual os astros celestes emitem uma música
em harmonia com o cosmos.}, passando por
diversos perigos e dificuldades, a fim de alcançar o castelo do Rei da
Morte e enfrentá"-lo frente a frente. A~narrativa dá a entender que
semelhante aventura só poderia ser empreendida por um poeta, indivíduo
excepcional que se distingue pela coragem de criar e ousar novas formas.

A paisagem que o Poeta encontra em sua viagem é um rarear de plantas, um
suceder de cavernas, montanhas e abismos, um desfile de bestas ferozes e
animais peçonhentos, em especial, de cobras. À~medida que se aproxima da
raiz de todo o mal, o castelo do Rei da Morte, o poeta testemunha o
terror e o horror absolutos, que afugentam até mesmo aqueles animais
rudes e nocivos. No entanto, mesmo exausto e com pés corroídos pela
jornada, o Poeta prosseguirá com sua missão para, talvez, conseguir
resgatar sua esposa, presa nas garras da morte.

Como se vê, o clima sombrio, os frequentes adjetivos relacionados à
morte e a presença de animais peçonhentos não só dão a esse conto a
atmosfera própria das narrativas góticas como também revelam afinidades
entre a obra de Stoker e as de Edgar Allan Poe e H\, P\, Lovecraft, dois
dos maiores expoentes da literatura de terror. E~tal como em Poe e
Lovecraft, Stoker também sabe dar aos temas mórbidos e sombrios um forte
teor poético. Valendo"-se de um trato sugestivo da linguagem na descrição
de sons, cores e lugares, a narrativa consegue a façanha de combinar a
sensação de suspense com um tom ameno, a despeito de toda a dureza e
toda a crueldade que permeiam a jornada do Poeta. As ilustrações
originais também sugerem uma atmosfera completamente enegrecida diante
do horizonte, transmitindo um clima de desolação e isolamento, em
contraste com o mundo cheio de luz presente em outros contos.

O ambiente noturno e sombrio desaparece no oitavo e último conto do
livro, ``A Criança Maravilhosa''. Numa atmosfera diametralmente oposta à
de ``O Castelo do Rei'', a história narra a descoberta de um bebê por
duas crianças, os irmãos Sibold e May, que se aventuram em um mundo
paralelo ao transpassar a fenda de um salgueiro, árvore miticamente
associada à pureza e à imaginação. O~bebê que encontram nessa outra
dimensão, apesar de possuir traços de divindade benfazeja, demonstrar
necessita de cuidados e de atenção, pois sua bondade o expõe a diversos
perigos, sugerindo a associação simbólica com a figura do menino Jesus.
Um tom sapiencial permeia esse último conto do livro, pois o bebê
explica às crianças como o comportamento delas deve ser para que possam
levar a vida de uma maneira justa e feliz. Após esse período de
aprendizagem, os dois irmãos retornam ao mundo real, e a história
termina com ambos dormindo rodeados de
papoulas\footnote{A presença das papoulas nas descrições de
Stoker pode estar associada ao ópio, substância muito consumida à época
como fonte de inspiração criadora.}.

\asterisc{}


As anotações dos diários de Bram Stoker, descobertos há pouco tempo,
elucidam de modo particularmente penetrante muitas semelhanças entre Sob
o pôr do sol e os romances posteriores do autor. Nesses diários é
possível encontrar ideias centrais de seu projeto literário e até mesmo
a semente de certos escritos. Por exemplo:

 

\begin{quote}
Um homem constrói uma sombra em uma parede, pedaço a pedaço,
acrescentando nela sua substância. De repente, a sombra se torna viva.

 

\end{quote}
Em nota marginal a esse apontamento, Stoker escreve: ``Ideia usada em
Sob o pôr do sol'', numa evidentemente alusão a ``O Construtor de
Sombras''.

Já outra passagem dos diários diz o seguinte:

 

\begin{quote}
Anot.{[}Anotação{]} para história para crianças: Palácio da Fada Rainha.
Criança vai dormir e o palácio cresce -- céu muda para cortinas azuis de
seda etc.

 

\end{quote}
A ideia expressa nesse apontamento aparentemente tem a intenção de
descrever um processo da imaginação infantil pelo qual a criança se
transporta do mundo real para o onírico. Ora, semelhante transporte
psicológico pode ser identificado em vários momentos de Sob o pôr do
sol, especialmente em ``Como o 7 Ficou Louco'' e ``Mentiras e Lírios''.
O~universo infantil, com efeito, permeia toda a obra de Stoker: as
crianças que povoam os contos de Sob o pôr do sol como exemplos de
candura serão, no romance Drácula, as vítimas preferidas de Lucy,
exatamente por serem puras de alma e de sangue.

Mas não é apenas nos diários de Stoker que se podem constatar as
estreitas relações entre Sob o pôr do sol e Drácula. Como já sugerido, é
patente o paralelo entre a centralidade do tema da mentira no conto
``Mentiras e Lírios'' e as invectivas do Sr.~Swales contra as mentiras
em Drácula (ver nota 12). Também é perceptível a proximidade entre o
Médico de Alfabeto, no conto ``Como o 7 Ficou Louco'', e o médico John
Seward, em \emph{Drácula}.

No entanto, a mais clara semelhança entre a coletânea de contos e o
romance é dada pela comparação entre a descrição da paisagem que o Poeta
vê ao aproximar"-se do castelo do Rei da Morte, em ``O Castelo do Rei'',
e a descrição da paisagem que Jonathan Harker vê ao aproximar"-se do
castelo do Conde Drácula. Em ``O Castelo do Rei'', depois de ser
aconselhado a não se aventurar em um mundo perigoso e desconhecido, o
Poeta inicia uma longa travessia em busca de sua Amada, que está nos
braços da morte. A~caminhada é interrompida pelo encontro com animais
que representam grande perigo; estradas vicinais levam a cavernas e a
passagens estreitas por entre montanhas, que têm a função de desviar a
atenção de quem passa por lá; a vegetação, em vez de fornecer alguma
espécie de alimento, torna"-se seca e cada vez mais rara; brumas e névoas
atravancam os sentidos, obnubilando o caminho e minando a disposição do
poeta em continuar sua missão. Esses elementos de uma natureza sombria
que acompanha externamente o ânimo inflamado do protagonista também
estão presentes em \emph{Drácula}. Ao se dirigir pela primeira vez ao
castelo do conde, construção oculta atrás de uma cadeia montanhosa,
Jonathan Harker é aconselhado pelos locais da Transilvânia a não viajar
com a carruagem enviada por Drácula. Harker, no entanto, embarca na
carruagem, que é guiada por um estranho condutor. Observando a paisagem
ao seu redor, à medida que sobe os estreitos caminhos pelas montanhas,
Harker percebe mudanças incomuns: brumas e névoas envolvem a noite densa
e o fazem acreditar que está andando em círculos; no meio do nevoeiro,
surgem pequenos clarões, que, como depois se saberá, são os espíritos
dos mortos de uma antiga batalha que ali acontecera; as montanhas
tornam"-se cada vez mais íngremes e os caminhos cada vez mais perigosos,
ameaçando a carruagem com seus desfiladeiros; por fim, divisa"-se o
castelo do Conde. Está coberto de escuridão e névoas, envolto por sons
estranhos e uivos de lobos, circundando por uma paisagem desolada. As
jornadas do Poeta e de Harker até o lugar do mau, além de muito
semelhantes, também indicam um claro paralelismo entre o Rei da Morte,
em ``O Castelo do Rei'', e o Conde Drácula no romance homônimo.

De resto, cabe dizer que a proximidade entre as obras não se refere
apenas às semelhanças de enredo, mas também à comunhão dos grandes
temas. Com efeito, tanto os contos de \emph{Sob o pôr do sol} quanto o
romance \emph{Drácula} se estruturam sobre as oposições ``bem e mal'',
``claro e escuro'', ``verdade e mentira'', ``razão e desrazão''. Além
disso, também se caracterizam pela forte presença do universo infantil,
do moralismo e da religiosidade. Ambos os livros se complementam formal
e tematicamente, mostrando o que há de melhor no riquíssimo mundo
narrativo de Bram Stoker. \emph{Sob o pôr do sol}, em especial, não
obstante ser uma das primeiras obras do autor, revela com ainda maior
força a poeticidade de sua prosa, com trechos de grande lirismo e
maestria na arte de contar histórias. A~presente tradução, a primeira e
única para o português, buscou ser fiel às nuances do original e à
qualidade da linguagem de Stoker, vindo a público para mostrar ao leitor
brasileiro que o talento de Bram Stoker não se resume a um livro só.

\asterisc{}
  



\section{Breve bibliografia e sugestões de leitura}


\begin{itemize}
\small
\item
  \versal{BIERMAN}, Joseph S.: ``A Crucial Stage in the Writing of Dracula''; in
  \versal{HUGHES}, William \& \versal{SMITH}, Andrew (orgs.): \emph{Bram Stoker: History,
  Phychoanalisys and the Gothic}, MacMillan, Londres, 1988, pp\,  151--172.
\item
  \versal{ROGERS}, David: ``Introduction''; in \versal{STOKER}, Bram: \emph{Dracula},
  Wordsworth Editions, Londres, 2000, pp\, v"-xix.
\item
  \versal{SENF}, Carol A.: \emph{Science and Social Science in Bram Stoker's
  Fiction}, Greenwood Press, Londres, 2002.
\item
  \versal{STOKER}, Bram \& \versal{MILLER}, Elizabeth \& \versal{STOKER}, Dacre (eds.): \emph{The
  lost Journal of Bram Stoker: The Dublin Years}, Biteback Publishing,
  Londres, 2013.
\item
  \versal{VASCONCELOS}, Sandra Guardini: \emph{Dez lições sobre o romance inglês
  do século \versal{XVIII}}, Boitempo Editorial, São Paulo, 2002, pp\,118--135.
\end{itemize}
