\chapterspecial{Sob o pôr do sol}{}{}
 

Longe, muito longe, há um belo País o qual nenhum olho humano jamais viu
em vigília. Sob o Pôr do Sol ele jaz, onde o horizonte distante amarra o
dia, e onde as nuvens, esplêndidas em luz e cor, dão uma promessa da
glória e da beleza que o cerca.

Algumas vezes, é"-nos concedido vê"-lo em sonhos.

De vez em quando vêm Anjos ternamente, abanando com suas grandes asas
brancas os cenhos ansiosos, e repousam mãos frescas sobre os olhos
dormentes. Então, o espírito do adormecido levanta voo. Do ofuscamento e
das trevas da temporada noturna surge. Para longe, através das nuvens
púrpuras, veleja. Apressa"-se pelo vasto espaço da luz e do ar. Pelo azul
escuro da abóbada celeste voa e, estendendo"-se pelo longínquo horizonte
repousa no belo Reino Sob o Pôr do Sol.

Esse País é como o nosso de muitas formas. Tem homens e mulheres, reis e
rainhas, ricos e pobres; tem casas, e árvores, e campos, e pássaros, e
flores. Há ali dia e também noite, e calor e frio, e doença e saúde. Os
corações dos homens e mulheres, e de garotos e garotas, batem como os
daqui. Há as mesmas tristezas e as mesmas alegrias; e as mesmas
esperanças e os mesmos medos.

Se uma criança daquele País estivesse ao lado de uma criança daqui, você
não poderia apontar a diferença entre elas, exceto que somente as roupas
são diferentes. Elas falam a mesma língua que nós falamos. Elas não
sabem que são diferentes de nós, e não sabemos que somos diferentes
delas. Quando elas vêm a nós em seus sonhos, não sabemos que são
estranhos; e quando vamos ao País delas em nossos sonhos, parecemos
estar em casa. Talvez isso ocorra porque as casas de pessoas boas estão
em seus corações; e, em qualquer lugar em que possam estar, terão paz.

O País Sob o Pôr do Sol foi por longas eras um Reino fantástico e
agradável. Nada havia que não fosse belo e doce e agradável. Foi somente
quando o pecado veio que as coisas começaram a perder sua perfeita
beleza. Até mesmo agora é uma terra fantástica e agradável.

Porque lá o sol é forte, às margens de todas as estradas estão plantadas
grandes árvores que espalham seus galhos grossos. Assim, os viajantes
têm abrigo quando passam. Os marcos são fontes de água fria e agradável,
tão clara e cristalina que quando o viajante chega a uma delas ele se
senta na pedra entalhada a seu lado e dá um suspiro de alívio, pois ele
sabe que há descanso.

Quando é pôr do sol aqui, lá é o meio do dia. As nuvens reúnem"-se e
protegem o Reino do grande calor. Então, por um curto tempo, tudo dorme.

Essa hora agradável e pacífica é chamada de Hora do Descanso.

Quando ela chega, os pássaros param seu canto, e repousam sob as amplas
calhas das casas ou em galhos das árvores, no lugar onde eles se juntam
aos caules. Os peixes param de nadar rápido e descansam sob as pedras,
com suas barbatanas e caudas tão imóveis como se estivessem mortos. A~ovelha e o gado descansam sob as árvores. Os homens e as mulheres
deitam"-se em redes estendidas entre as árvores ou sob as varandas de
suas casas. Então, quando o sol para de resplandecer intensamente e as
nuvens derretem, todas as coisas vivas acordam.

As únicas coisas vivas que não dormem na Hora do Descanso são os
cachorros. Eles ficam deitados, muito quietos, somente meio dormindo,
com um olho aberto e uma orelha levantada, mantendo vigilância o tempo
todo. Assim, se algum estranho chega durante o tempo do Descanso, os
cães levantam"-se e olham para ele, suavemente, sem latir, para que não
atrapalhem ninguém. Eles sabem se o recém"-chegado é inofensivo; e, se
for o caso, deitam"-se novamente, e o estranho também se deita até que a
Hora do Descanso termine.

Mas se os cães pensam que o estranho veio para causar malefícios, eles
latem alto e rosnam. As vacas começam a mugir e as ovelhas a balir, e os
pássaros a gorjear e a cantar suas notas mais altas, mas sem música
nelas; e até mesmo os peixes começam a nadar rápido e a espirrar água.
Os homens acordam e saltam de suas redes, e agarram suas armas. Então, o
momento fica ruim para o intruso. Imediatamente ele é levado à Corte e
julgado, e, se considerado culpado, ele é sentenciado e colocado na
prisão ou banido.

Depois os homens voltam para suas redes e todas as coisas vivas
novamente se retiram até que a Hora do Descanso termine.

À noite acontece igual à Hora do Descanso caso um intruso venha causar
malefícios. À~noite, somente os cães e os doentes e suas enfermeiras
estão acordados.

Ninguém pode deixar o País Sob o Pôr do Sol exceto para uma direção.
Aqueles que vão para lá em sonhos, ou que vêm em sonhos para nosso
mundo, vêm e vão não sabem como. Mas, se um habitante tenta deixá"-lo,
ele não consegue exceto de uma maneira. Se ele tenta outras maneiras,
ele vaga infinitamente, fazendo curvas sem saber, até que chega a um
lugar de onde somente ele pode partir.

Esse lugar é chamado de Portal, e ali os Anjos mantêm guarda.

Exatamente no meio do País fica o palácio do Rei, e as estradas
estendem"-se a partir dele por todos os lados. Quando o Rei está em pé no
topo da torre, que se ergue a uma grande altura no meio de seu palácio,
ele pode observar as estradas, que são todas bem retas.

Elas parecem se tornar mais e mais estreitas à medida que seguem
adiante, até que por fim se perdem totalmente a menor distância.

Em volta do palácio do Rei estão reunidas casas de grandes nobres, cada
uma proporcionalmente próxima ao posto de seu dono. Ao lado dessas vêm
as casas dos nobres menores; e depois aquelas de todas as outras
pessoas, tornando"-se cada vez menores à medida que se vai mais adiante.

Toda casa, grande e pequena, situa"-se no meio de um jardim que tem uma
fonte e um curso d'água, e grandes árvores, e canteiros de belas flores.

Mais além, em direção ao Portal, o país torna"-se cada vez mais selvagem.
Para além, há densas florestas e grandes montanhas repletas de cavernas
profundas, tão escuras quanto a noite. Aqui, animais selvagens e todas
as coisas cruéis têm seu lar.

Então vêm pântanos e brejos e lamaçais profundos e instáveis, e densas
selvas. Depois tudo se torna tão selvagem que a estrada some
completamente.

Nos lugares selvagens além dali nenhum homem sabe o que há. Alguns dizem
que os Gigantes, que ainda existem, vivem ali, e que todas as plantas
venenosas ali crescem. Dizem que lá há um vento iníquo que revela as
sementes de todas as coisas más e espalha"-as sobre o solo. Alguns há que
dizem que o mesmo vento iníquo também espalha as Doenças e as Pragas que
existem ali. Outros dizem que a Fome vive lá nos pântanos, e que ela se
aproxima silente quando os homens são malévolos -- tão malévolos que os
Espíritos que guardam a terra choram tão amargamente que não a veem
passar.

Murmura"-se que a Morte tem seu reino nas Solidões para além dos
pântanos, e vive em um castelo tão terrível de se olhar que ninguém o
vira e viveu para contar como ele é. Também se diz que todas as coisas
más que vivem nos pântanos são os desobedientes Filhos da Morte que
deixaram seus lares e não conseguem mais encontrar o caminho de volta.

Mas nenhum homem sabe onde está o Castelo do Rei Morte. Todos os homens
e mulheres, garotos e garotas, e mesmo crianças pequenas devem viver
para que assim, quando tiverem de entrar no Castelo e ver o macabro Rei,
possam não ter medo ao contemplar seu rosto.

Por muito tempo, a Morte e seus Filhos permaneceram fora do Portal e
tudo dentro dele era alegre.

Mas veio um tempo quando tudo mudou. Os corações dos homens se tornaram
frios e duros de orgulho com sua prosperidade, e eles não prestaram
atenção nas lições que lhes foram ensinadas. Então, quando lá dentro
houve frieza e indiferença e desdenho, os Anjos em guarda viram nos
terrores lá de fora o meio de punição e a lição que poderia fazer bem.

As boas lições vieram -- como muito frequentemente vêm as coisas boas --
depois de dor e de provação, e elas ensinaram muito. A~história de sua
vinda guarda uma lição para o bom entendedor.

No Portal, dois Anjos eternamente vigiavam e mantinham guarda. Esses
anjos eram tão majestosos e tão vigilantes, e sempre tão inflexíveis em
sua guarda, que havia somente um nome para ambos. Qualquer um ou ambos
seriam, ao falar com eles, chamados pelo nome completo. Um deles
conhecia tanto quanto o outro sobre qualquer coisa que pudesse ser
conhecida. Isso não era tão estranho, pois ambos conheciam tudo. O~nome
deles era Fid"-Def.

Fid"-Def estavam de guarda no Portal. Ao lado deles havia uma
Criança"-Anjo, mais bela do que a luz do sol. A~silhueta de sua bela
forma era tão suave que sempre parecia estar derretendo no ar; parecia
uma luz viva e sagrada.

Ela não ficava em pé como os outros Anjos, mas flutuava para cima e para
baixo e por todo lado. Algumas vezes era somente uma pequena mancha, e,
então, repentinamente, sem parecer haver qualquer mudança, tornava"-se
maior do que os grandes Espíritos Guardiões que eram os mesmos desde
sempre.

Fid"-Def amavam a Criança"-Anjo e, à medida que ela se erguia de vez em
quando, eles abriam suas grandes asas brancas, e ela subia às vezes em
cima delas. Suas próprias asas belas e frágeis arejavam seus rostos
suavemente quando se viravam para falar.

Mas a Criança"-Anjo nunca cruzava o limiar. Ela olhava para a vastidão ao
longe, mas nunca colocava nem mesmo a ponta de sua asa para além do
Portal.

Ela estava fazendo perguntas sobre Fid"-Def, e parecia querer saber o que
havia lá fora, e como tudo lá diferia de tudo daqui.

As perguntas e as respostas dos Anjos não eram como nossas perguntas e
respostas, pois não havia necessidade de fala. No momento em que vinha
um pensamento de querer saber alguma coisa, a pergunta era feita e a
resposta era dada. Mas, ainda assim, a pergunta era feita pela
Criança"-Anjo e respondida por Fid"-Def; e se conhecêssemos a não língua
que os Anjos estavam não falando, teríamos ouvido assim. Fid"-Def estava
falando com Fid"-Def:

``Chiaro não é belo?''

``Ele é muito belo. Ele será um novo poder no Reino.''

Aqui Chiaro, que estava apoiado com um pé na pluma da asa de Fid"-Def,
disse:

``Diga"-me, Fid"-Def, o que são aqueles Seres de aparência horrível para
além do Portal?''

Fid"-Def respondeu:

``Eles são os Filhos do Rei Morte. O~mais horrível de todos, envolto em
trevas, é Skooro, um Espírito Mau.''

``Como eles parecem horríveis!''

``Muito horríveis, caro Chiaro. E~esses Filhos da Morte querem cruzar o
Portal e entrar no Reino''.

Chiaro, diante da terrível notícia, ergueu"-se para o alto, e ficou tão
grande que todo o País Sob o Pôr do Sol ficou claro. Logo, entretanto,
ele se tornou cada vez menor até que se virou somente uma mancha, como o
feixe colorido visto em um quarto escuro quando o sol entra por uma
fresta. Ele perguntou aos Anjos do Portal:

``Digam"-me, Fid"-Def, por que os Filhos da Morte querem entrar?''

``Porque, querida Criança, eles são malvados, e querem corromper os
corações dos moradores do Reino''.

``Mas me digam, Fid"-Def, eles conseguem entrar? Certamente, se o Supremo
diz `Não!', eles devem ficar para sempre fora do Reino.''

Depois de uma pausa veio a resposta dos Anjos do Portal:

``O Supremo é mais sábio do que até mesmo os Anjos podem conceber. Ele
expulsou os malvados com seus próprios truques, e ele prendeu o caçador
em sua própria armadilha. Os Filhos da Morte, quando entram -- como
estão prestes a fazer --farão muitas coisas boas ao Reino que querem
fazer mal. Pois, veja!, os corações das pessoas estão corrompidos. Eles
esqueceram as lições que lhes foram ensinadas. Eles não sabem o quanto
deveriam ser gratos por sua sorte, pois a tristeza eles não conhecem.
Deve haver alguma dor ou pesar ou tristeza para que possam ver o erro de
seus caminhos''.

Enquanto falavam, os Anjos choraram de dor pelos pecados do povo e pela
dor que devem suportar.

A Criança"-Anjo respondeu estupefata:

``Então esse horrível Ser está também para entrar no Reino. Ai! Ai!''

``Querida Criança'', disseram os Espíritos Guardiães enquanto a
Criança"-Anjo deslizou para seus peitos, ``a você é dado um grande dever.
Os Filhos da Morte estão prestes a entrar. A~você foi confiada a guarda
contra esse Ser horrível, Skooro. Onde quer que ele vá, lá você deverá
estar também; assim, nada de mal pode acontecer -- exceto somente o que
é pretendido ou permitido''.

A Criança"-Anjo, maravilhada pela grandeza da confiança, resolveu que sua
tarefa deveria ser bem"-feita. Fid"-Def continuaram:

``Você deve saber, querida Criança, que sem a escuridão não há medo
algum do invisível; e nem mesmo a escuridão da noite pode assustar caso
haja luz dentro da alma. Ao bom e ao puro não há medo seja das coisas
más da terra, seja dos Poderes que são invisíveis. A~você é confiada a
guarda do puro e do verdadeiro. Skooro irá encobri"-los com sua sombra;
mas a você é dado penetrar em seus corações e por sua própria luz
gloriosa tornar a sombra dos Filhos da Morte invisível e desconhecida.

``Mas dos facínoras -- dos maus, e dos mal"-agradecidos, e dos
implacáveis, e dos impuros, e dos falsos você se manterá afastado; e
assim, quando eles o procurarem para lhes dar conforto -- como sempre
haverão de fazer -- eles não irão ver você. Eles verão somente a sombra
que a sua luz distante fará parecer ainda mais escura, pois a sombra
estará em suas próprias almas.

``Mas, oh!, Criança, nosso Pai é bom para além do acreditável. Ele
ordena que caso alguém que seja mau se arrependa, você voará
imediatamente a ele e o confortará, e irá ajudá"-lo, e animá"-lo, e
forçará a sombra para longe. Caso eles apenas finjam se arrepender,
desejando ser novamente maus quando o perigo passar, ou caso eles ajam
somente devido ao medo, então você esconderá sua claridade para que a
sombra possa se avolumar mais escura sobre eles. Agora, querido Chiaro,
torne"-se invisível. A~hora se aproxima quando será permitido aos Filhos
da Morte entrar no Reino. Ele tentará entrar sorrateiramente, e
deveremos deixá"-lo, pois devemos trabalhar invisíveis e incógnitos para
desempenharmos nossa função''.

Então a Criança"-Anjo desvaneceu"-se lentamente a fim de que nenhum olho
-- nem mesmo os de Fid"-Def -- pudesse vê"-lo;e os Espíritos Guardiães se
postaram como sempre ao lado do Portal.

A Hora do Descanso chegou, e tudo estava quieto no Reino.

Quando os Filhos da Morte, bem ao longe, nos pântanos, viram que nada
estava em movimento, e cuidando que os Anjos estavam como sempre de
guarda, eles resolveram fazer outra tentativa de entrar no Reino.

Consequentemente, eles se separaram em muitas partes. Cada parte tomou
uma forma diferente, mas todos juntos se moveram em direção ao Portal.
Assim, os Filhos da Morte aproximaram"-se do limiar do Reino.

Sobre a asa de um pássaro que passava eles vieram, em uma nuvem que
deslizava lentamente no céu, nas cobras que rastejavam sobre aterra --
nos vermes, e ratos, e toupeiras que se arrastavam sob ela, nos peixes
que nadavam e nos insetos que voavam. Por terra e água e ar eles vieram.

Então, sem embargos ou atrasos, e de muitas formas, os Filhos da Morte
entraram no país Sob o Pôr do Sol; e a partir daquela hora tudo naquele
belo Reino mudou.

Os Filhos da Morte não se fizeram conhecidos imediatamente. Um a um, os
espíritos mais arrojados entre eles, espreitando com passadas cruéis
pelo Reino, preencheram todos os corações com terror à medida que
avançavam.

Entretanto, cada um e todos eles deixaram uma lição para o bem nos
corações dos moradores do Reino.
