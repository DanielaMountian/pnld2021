\chapterspecial{Introdução}{}{Christian Werner}

\noindent{}Diferentemente dos poemas de Homero, os de Hesíodo se associam, eles
próprios, a um poeta e a um lugar como espaço de sua gestação: o poeta
de \emph{Trabalhos e dias} e  \emph{Teogonia} se nomeia e se vincula ao entorno do monte Hélicon na
Beócia (22--23). Trata"-se de uma região no centro da Grécia, cuja cidade
principal, no passado e hoje, é Tebas. Suas montanhas principais são o
Parnasso, junto a Delfos, o Citéron (onde Édipo foi exposto) e o
Hélicon, com sua fonte Hipocrene (``Fonte do Cavalo''), estes dois
mencionados no início do poema em associação às Musas (1--8).

Indicações temporais, porém, estão virtualmente ausentes do poema, o que
permite reconstituições diversas, todas elas imprecisas e sujeitas a
críticas. Uma delas, feita pelos antigos, é associar Hesíodo a outros
poetas da tradição hexamétrica grega arcaica --- Museu, Orfeu e,
sobretudo, Homero --- e estabelecer uma cronologia relativa, para o que
um critério poderia ser a autoridade: a maior seria a do poeta mais
velho (Koning 2010). Modernamente, a cronologia relativa reaparece
fundamentada no exame linguístico"-estatístico do \emph{corpus}
hexamétrico restante (Andersen \& Haug 2012). Assim, Janko (1982), um
trabalho seminal, definiu como sequência cronológica de composição
\emph{Ilíada}, \emph{Odisseia}, \emph{Teogonia} e \emph{Trabalhos e
dias}.

Outra forma de contextualizar os poemas no tempo está ligado a
tentativas de reconstituir os séculos \textsc{viii} e \textsc{vii} a.C. como a época na
qual se sedimentaram uma série de fenômenos culturais e políticos que
acabaram por definir as sociedades gregas, em especial o surgimento da
\emph{polis} como principal organização política e social, o templo de
Apolo e seu oráculo em Delfos como um santuário de todos os gregos,
festivais de cunho religioso, como os Jogos Olímpicos, que passaram a
atrair participantes de uma ampla gama de territórios grego, a
reintrodução da escrita, o culto aos heróis etc. Trata"-se de fenômenos
que definem o que Gregory Nagy (1999), na esteira de Snodgrass (1971),
chama de pan"-helenismo,\footnote{Moraes (2019, p. 12) entende ``o
  Pan"-helenismo como um discurso político capaz de prover uma sensação
  de pertencimento às comunidades de língua grega, baseado em critérios
  simultaneamente culturais e políticos de caráter aglutinador e que
  atuou na produção e reprodução da identidade helênica''.} e do qual
faria parte a produção e recepção da \emph{Teogonia}.

A introdução paulatina, com adaptações, nos territórios gregos, nos
quais se falavam dialetos diversos, de um alfabeto de origem fenícia em
torno do século \textsc{viii} a.C. foi um dos responsáveis pela modificação
gradual de diversas práticas sociais, entre elas, a produção e recepção
de poesia. Os poemas podiam ser cantados ou recitados, e, quando
cantados por um coro (o que não é o caso da poesia épica como a
\emph{Teogonia} e \emph{Trabalhos e dias}), esse produzia figuras de dança. É exatamente assim que as Musas são representadas no início de \emph{Teogonia} (1--11), em contraste com o
cantor individual Hesíodo. Composições corais eram apresentadas em
ocasiões específicas, muitas vinculadas ao calendário religioso de
determinadas localidades. Quanto à poesia hesiódica, o contexto de
performance é desconhecido por nós. De fato, como se verá mais abaixo
por meio do nome de Hesíodo, é necessário tratar com cuidado os
elementos que parecem atar o poema à realidade.


Em \emph{Trabalhos e dias}, encena"-se uma relação de aprendizagem, em que
o narrador do poema é alguém que entende tudo por si mesmo e dessa
forma pode dar conselhos aos outros (293--95). Nesse sentido, ele pode
ser comparado a Quelemém, incluindo"-se o uso que faz de imagens
didáticas e frases que soam como provérbios para transmitir seu
conhecimento. Entretanto, por mais que a moralidade de sua visão de
mundo, como a de Quelemém, pareça ser transparente --- há os juntos e os
injustos, e estes, em algum momento, serão punidos ---, Hesíodo
arma enigmas que exigem a colaboração do
leitor para entregarem uma resposta, sendo que esta não elimina certa
desconfiança geral em relação à capacidade da linguagem de transmitir a
verdade (Pucci 1977, Vergados 2018).

Já \emph{Teogonia} move"-se entre o arcaico e o moderno, o mítico e o racional,
principalmente em relação à sua personagem central, Zeus.
Esse poema explora os meandros da
justiça e da soberania como idealizações dependentes da astúcia
(\emph{mētis}), essa qualidade ou habilidade excencialmente múltipla e
imanente, focada no aqui e agora da experiência sempre cambiante (``no
tremeluz\ldots{} muito veloz''). Mito e razão não
se revelam formas de pensamento opostas ou incompatíveis, em particular,
pela modo como, em Hesíodo, a linguagem e a narrativa desvelam o cosmo.

\section*{Trabalhos e dias}

\subsection{O poema e sua voz autoral}


A voz autoral que enuncia os \emph{Trabalhos} certamente fala de si, ou
seja, de um poeta ao qual ela está associada. Por uma questão de
economia, ao longo desta introdução, usarei ``Hesíodo'' como sinônimo
dessa voz, embora este nome não seja, ele mesmo, mencionado nenhuma vez
ao longo do poema. Como destacaram Ford (1997), Scodel (2011) e outros,
o que particularmente distingue essa voz, diferenciando"-a daquela que
percorre os poemas homéricos, é ela se dirigir diversas vezes ao longo
do poema a um interlocutor específico com o qual o poeta mantém uma
relação estreita, seu irmão Perses. Estamos longe do público impessoal
pressuposto por um poema épico como a \emph{Ilíada} mas bastante
próximos de certos tipos de poemas que podemos chamar, de forma bem
genérica, de líricos, os quais, no universo grego arcaico, podiam ter um
ou mais destinatários explícitos (indivíduos e/ou a população de uma
cidade).

Para muitos estudiosos, porém, faz mais sentido se pensar que a
representação daquilo que parece dizer respeito à vida dessa voz autoral
--- e o mesmo vale para a \emph{Teogonia} --- seja intrínseco à tradição
poética da qual esse poema é dependente e não um recorte da biografia de
seu autor, por mais que a escassez de material transmitido torne difícil
reconstruir"-se a tradição. Nesse sentido, Hesíodo seria um mito (Nagy
1990). Os eventos que perfazem certo pano de fundo do poema (a briga
entre Hesíodo e Perses) não são, porém, uma ``mentira'' ou uma
``ficção'', mas, como defende Gregory Nagy, uma componente que contribui
para a autoridade do discurso como um todo, provido, portanto, de
eficácia no âmbito das sociedades onde ele foi apresentado e pelas quais
foi assimilado até se tornar parte de um cânone. Dessa forma, Hesíodo e
Perses não seriam nem realidades históricas nem ficções, mas elementos
de uma tradição mitopoética em uma sociedade tradicional, na qual o
conhecimento dependia de formas específicas, ligadas à oralidade, para
ser apresentado e transmitido de geração em geração. Em última
instância, e parafraseando Nagy, dificilmente saberemos um dia, com
segurança, se foi um Hesíodo histórico que deu vida aos poemas (todos?
quais?) que dele falam ou se são poemas e tradições poéticas que deram
vida à figura que passou a ser conhecida como Hesíodo.

%\section{A unidade do poema}


\subsection{Proêmio: Zeus e o poeta}

Como em outras composição gregas compostas na época deste poema e que
também usam sua unidade métrica, o hexâmetro datílico, os
\emph{Trabalhos} iniciam com um proêmio (1--10). Esse trecho é uma das
façanhas poéticas mais notáveis que conhecemos deste período (caso não
tenha sido uma adição bem posterior, como já defendido na Antiguidade).
Uma sucessão de figuras de linguagem comprimidas em uma sequência sonora
quase encantatória, abundante em rimas, externas e internas, e outros
tipos de repetição (Watkins 1995, Macedo 2010, Werner 2016), tornam o
trecho virtualmente intraduzível. A exibição do domínio desses recursos
poéticos reforça a ação de Hesíodo, a demarcação de sua voz e de sua
autoridade, em suma, o seu discurso, em relação a duas outras
autoridades geralmente presentes na poesia hexamétrica, as Musas e Zeus.

Assim, a densidade da linguagem pode ser o ponto de partida para uma
interpretação do trecho, sobretudo se aceitarmos, como defendido por
Watkins (1995), que a primeira palavra do proêmio (\emph{Mousai},
``Musas'') como que está inscrita em sua última
(\emph{\emph{muthēsaimēn}}, ``quero por em um discurso''). Tal façanha
linguística nos fornece um elemento fundamental para pensar o proêmio
como um todo. Esses dois termos são altamente marcados, o que se
evidencia a partir do seguinte trecho do outro poema atribuído a Hesíodo
(\emph{Teogonia}, 24--28):

\begin{verse}
Este discurso (\emph{muthos}), primeiríssimo ato, dirigiram"-me as deusas,\\
as Musas do Olimpo, filhas de Zeus porta"-égide:\\
``Pastores rústicos, infâmias vis, ventres somente,\\
sabemos falar muito fato enganoso como genuíno,\\
e sabemos, quando queremos, proclamar verdades''.
\end{verse}

\emph{Mousai} diz respeito à instância metafísica sobre a qual se apoia
o cantor oral grego para justificar a veracidade, a autoridade do seu
canto, mas a forma verbal \emph{muthēsaimēn} (o verbo é cognato de
\emph{muthos}) refere"-se a um discurso cuidadosamente elaborado por meio
do qual o falante quer fazer valer a sua autoridade, geralmente em um
ambiente público (Martin 1989).

Entre a autoridade das Musas e a sua própria, o poeta assume um discurso
que ele aproxima ao máximo da atuação de Zeus no mundo dos homens e, com
isso, no mínimo parcialmente, se afasta das Musas, aquelas encarregadas
de preservar as façanhas dos heróis épicos, guerreiros do passado (1:
``que com cantos glorificam''). Os versos 3 a 8 por certo celebram Zeus,
mas quase que exclusivamente por meio do modo como ele age no mundo dos
homens, culminando em sua administração da justiça (7 e 9). A
contiguidade, no grego, entre os dois pronomes que abrem o verso 10,
``tu'' (Zeus) e ``eu'' (poeta), garantem ao receptor que a eficácia do
discurso deriva do conhecimento de Hesíodo e apoia"-se na atuação de
Zeus, a quem o poeta pede ajuda em um momento de tensão, marcado pelo
imperativo ``atende"-me'' (9). Ignorar as Musas seria impensável para um
poeta do período; mas aqui elas parecem ser deslocadas para uma posição
marginal, e sua menção, do modo como é feita, talvez também sirva para
reforçar que neste poema não serão glorificados os heróis do passado.

%\section{O início do poema: Brigas e trabalho}

\subsection{Mitos: Prometeu, Pandora e a as linhages do homem}

A figura do irmão serve sobretudo de intermediário para Hesíodo
fazer o seu público decifrar seu poema e assimilar seu raciocínio. Se
Perses não escutar o irmão, ele cometerá um erro tão tolo quanto aquele
de Epimeteu, o irmão de Prometeu (47--105). A história do roubo do fogo
pelo Titã e a vingança de Zeus, que criou a primeira mulher, Pandora,
também é mencionada na \emph{Teogonia}, mas em \emph{Trabalhos} ela é
reconfigurada tendo em vista o discurso que Hesíodo dirige ao irmão.

Tanto a história de Prometeu e Pandora quanto o mito das linhagems do
homem (106--201) giram em torno da separação entre homens e deuses, mais
especificamente, da necessidade de os homens trabalharem para obter seu
sustento. Na primeira história, Zeus e os demais deuses não surgem como
figuras simpáticas aos mortais, mas, como notou Martin (2004), é o
desnível de conhecimento no âmbito de duas duplas --- Zeus/Prometeu e
Prometeu/Epimeteu --- que encaminha a história, no que se reproduz a
moldura discursiva do poema, a sapiência de Hesíodo sobrepondo"-se à
tolice de Perses. Vale dizer que Perses não é considerado um caso
perdido, não é desprezado por Hesíodo. Se assim o fosse, não haveria
razão para o discurso existir. Hesíodo por certo se apresenta como
sumamente sábio, mas o problema de seu irmão, assim como o de Epimeteu e
o dos heróis homéricos que costumam ser chamados de \emph{nēpios}
(``tolo''), é não se darem conta, na hora correta, das consequências
danosas futuras de suas ações.

Voltando à história de Prometeu, em um primeiro nível, o fogo escondido
por Zeus e trazido de volta pelo Titã aponta para as conquistas técnicas
que tornam a agricultura uma atividade mais simples e produtiva. O fogo,
porém, também é análogo à destruição consumidora consubstanciada na
primeira mulher. Tecnicamente impecável e muito sedutora, a mulher é um
mal que se opõe ao homem trabalhador. Nesse sentido, ela está do lado de
Perses e dos reis que aceitam serem subornados. Seu nome, porém, compõe
um enigma: ``Pan"-dora'' pode ser interpretado como, ``todos os
presentes'', ou seja, as dádivas que cada deus confere à criatura
moldada por Hefesto ou ``presentes de todos (os deuses)''. Com efeito,
tais presentes são enganadores, eles não trazem bem algum para os homens
e, assim como os presentes que ganham os reis ``come"-presente'' (39) e a
parte do patrimônio que Perses tentou (e/ou tenta) conquistar
injustamente, eles têm um efeito danoso no futuro. Os males futuros,
porém, não se anunciam mais aos homens; Zeus lhes tirou a voz, o
contrário do que determinou fosse feito com Pandora. Entre os homens, o
que resta é dar ouvidos ao sábio que conhece a origem desses males
desconhecidos. Essa é uma das faces da ``esperança'' (\emph{elpis}),
essa entidade naturalmente ambivalente que também dá conta da
expectativa depositada pelos reis e Perses no lucro que podem ganhar sem
trabalho. Para quem percebe a falácia dessa ``expectativa'' (outra
tradução de \emph{elpis}), resta enunciar e defender que Zeus
``endireita o torto'' (7).

Se o foco da história de Prometeu é a dificuldade da vida humana, em
especial, a necessidade do trabalho como garantia do sustento, uma
diferença essencial entre a vida humana e a divina, na história seguinte
acerca das linhagens ou espécies do homem, a relação principal não é
mais vertical, mas horizontal, os modos diversos dos homens agirem
reciprocamente: ou bem a justiça (\emph{dikē}) ou bem a violência
(\emph{hubris}) tem primazia numa determinada linhagem. Esse, pelo
menos, é um dos eixos da influente interpretação de Vernant (2002), para
quem estas categorias dão sentido ao mito.

A polarização mais clara que perpassa a história é, porém, aquela mesma
que separa os irmãos Hesíodo e Perses, o conflito (nos seus modos
diversos) e sua ausência. A linhagem presente (de ferro) tem diante de
si uma dupla alternativa: a auto"-destruição, o que ocorreu com todas as
outras, salvo a de ouro, ou uma situação que a poria muito perto da
linhagem de ouro (225--47), a saber, uma cidade onde o máximo de justiça
garantirá o máximo de prosperidade para todo aquele que quiser trabalhar
sua terra. A tese do poema, portanto, torna"-se clara: o trabalho é um
mal; entretanto e paradoxalmente, se todos se dedicarem a ele, a
humanidade poderá se reaproximar daquilo que perdeu para sempre, uma
vida sem trabalho e outras aflições. Nesse tipo de sociedade, o espaço
para injustiças geradoras de conflitos é mínimo. É uma tolice da parte
daquele que se acredita o mais forte em dada situação buscar vantagens
baseadas em sua força: uma série de mecanismos atrás dos quais estão
Zeus e Justiça fará com que o usurpador seja punido. Ou não?

\subsection{Justiça e os desígnios de Zeus}

``Justiça'' traduz o grego \emph{dikē}, termo cuja polissemia é
explorada por Hesíodo ao longo do poema. Na tradução do termo e seus
cognatos, utilizou"-se sobretudo ``juízo'' e ``justiça'', aquele, para
abarcar procedimentos legais, em particular, julgamentos e vereditos,
que podem ser qualificados como ``retos'' ou ``tortos''; este, para
punicação ou compensação, ordem cósmica, como personificação ou como um
ideal ou princípio (Vergados 2020, p. 169--75).

O proêmio do poema e outros trechos, sobretudo aqueles que indicam que
não basta ao homem trabalhar sua terra seguindo todas as regras, pois o
clima depende, em última instância, dos desígnios de Zeus, indicam que o
poder do deus é por vezes arbitrário. Esse talvez seja o mistério --- e
fascínio --- último do poema, entender como pode ser Zeus arbitrário e
justo. Por um lado, é responsável a ignorância humana: ``Sempre
cambiante é a mente de Zeus porta"-égide,/ e para os homens mortais é
difícil entendê"-la'' (483--84). Por outro lado, alguns homens conseguem
chegar muito perto da verdade; um deles é Hesíodo, e todo aquele que por
ele for persuadido seguirá seus conselhos, mesmo que seu sucesso não
seja cem por cento garantido (293--97):

\begin{verse}
Este o melhor de todos, quem por si tudo entender\\
ao refletir no que será melhor, depois e no fim;\\
nobre também o que atende a quem dá bons conselhos;\\
mas quem por si não entender nem, de outro ouvindo,\\
lançar no ânimo, esse é um varão infrutífero.
\end{verse}

Esta passagem ajuda a explicar por que Hesíodo não enuncia claramente a
moral da estória (\emph{ainos}) do falcão e do rouxinol (202--12): por um
lado, a estória aponta para o sucesso do mais forte, algo que todo o
homem, ao longo de sua vida, tende a testemunhar. Se atentarmos para o
fato de que o rouxinol vale como um poeta e que a voz de Hesíodo se faz
notar explicitamente no verso 213 ao se dirigir a Perses, então os
versos que seguem à estória, nos quais se sobressai a oposição entre a
Cidade Justa e a Cidade Injusta, compõem, na verdade, a resposta,
alongada, dada pelo rouxinol"-poeta ao falcão. Não vale uma resposta
presa a uma só dimensão temporal, assim como só entendemos como funciona
uma das linhagems do homem se a colocarmos em oposição às outras; nem
mesmo o fim da linhagem de ferro garantirá a vitória inequívoca de um ou
outro dos animais da fábula. Para Hesíodo, porém, \emph{no contexto da
idade de ferro}, a resposta indireta do rouxinol é a boa. A Justiça, uma
filha de Zeus (256--57), é um presente do deus (279--80) e, como tal, a
contrapartida de Pandora. A diferença entre as duas é que Pandora e o
que ela representa irremediavelmente estão entre os homens; Justiça só
fica entre eles se os homens a cultivarem.

\subsection{Trabalhos, estações e os dias}

A partir do verso 286, Hesíodo paulatinamente abandona o tema da
justiça, que é necessária, coletivamente, para o bem de todos, e passa a
abordar, de forma mais específica, o tema do trabalho, que é, em
primeiro lugar, uma ocupação individual. Nessa dobra do texto, a figura
negativa deixa de ser a criatura violenta, que só confia na própria
força, e passa a ser o inativo e/ou mendigo; Perses passa a ser pensado
no intervalo entre duas figuras, aquele que quer tomar a riqueza alheia
e o inativo faminto com vergonha de trabalhar (298--319). O trabalho
opõe"-se à inação e ao roubo, uma das ações moralmente vis para a qual
Zeus garante compensação (320--34) e que, mais adiante, também merecerá
um trocadilho: o homem ``sono"-diurno'' é o ladrão (605). Ao fazer o
elogio do trabalho em oposição à inação, o pessimismo fica de lado
quanto à ordem garantida por Zeus: para uma sociedade na qual o trabalho
é a opção da maioria, Zeus está presente e não vale o que enuncia o
falcão ao rouxinol.

O trecho que inicia no verso 286 e vai até o verso 380 e os versos
694--764 apresentam, na forma de catálogos, uma série de sentenças
morais, preceitos e conselhos, típicos da literatura sapiencial, os
quais flanqueiam o calendário no qual se desdobram as atividades
próprias da agricultura e da navegação. O ``tu'' ao qual se dirige o
poeta não é mais necessariamente apenas Perses, mas, em muitas
passagens, é um ``tu'' genérico (Schmidt 1986). Não é difícil entender a
função do trecho que vai de 286--380, pois, são enumerados uma série de
preceitos que, se forem seguidos, permitirão ao homem uma vida melhor no
que diz respeito aos deuses, sua sociedade e sua propriedade. Entre a
justiça, uma conquista coletiva, e a riqueza, pessoal, há uma série de
ações que o homem deve realizar para o seu sucesso material e moral
(289--92).

Erro comum na apreciação do poema é achar que, ao iniciar o que amiúde
se denomina ``almanaque agrícola'' (381--617), um elenco de atividades
agrícolas associadas a certos fenômenos celestes e climáticos, o
objetivo do poeta passasse a ser compor um manual para um agricultor
inexperiente. Obviamente, esse não devia ser o uso do poema, assim como
não precisamos pensar em Hesíodo como um poeta"-agricultor, alguém que de
dia geria sua propriedade e à noite ou nas horas vagas --- por exemplo,
durante o calor do verão, ouvindo as cigarras --- pensava nos seus
poemas. Mas então para que serve essa passagem propriamente agrícola?

Em primeiro lugar, tanto nela quanto no curioso trecho que é o último do
poema (os ``dias''), há temas que reaparecem alhures (Lardinois 1998).
Segundo Nelson (1996, p. 53), por exemplo, essa passagem não ensina
``como ser um agricultor, mas o que o ciclo do ano, com seu equilíbrio
entre bem e mal, lucro e risco, ansiedade e descanso, implica acerca da
vontade de Zeus'' De fato, Zeus continua tão presente quando antes; sua
atuação garante uma ordem sem a qual a agricultura seria impossível
(414--16). Se os trabalhos são sazonais, as mudanças entre as estações
(565) devem"-se a Zeus, que, não por acaso, é seu pai (\emph{Teogonia}
901--3): ``A segunda{[}esposa{]}, {[}Zeus{]} fez conduzir a luzidia
Norma, mãe das Estações,/ Decência, Justiça e a luxuriante Paz,/ elas
que zelam pelos trabalhos (\emph{erga}) dos homens mortais''.

A fartura parece advir da vontade de Zeus (465), mas esta está sujeita a
mudanças bruscas (483), imprevisíveis (488). A luta do homem, porém, não
é contra a vontade de Zeus, mas a partir dela. Por isso a descrição dos
fenômenos climáticos é tão importante: o bom agricultor é aquele que
nunca deixa passar o momento correto de realizar certas atividades
precisas. Além disso, cada estação tem particularidades que ultrapassam
a mera contextualização dos trabalhos do agricultor e que se relacionam
com outros trechos do poema; exemplifiquemos isso através do trecho
sobre o inverno (493--563).

Em primeiro lugar, observe"-se que o trecho mais longo dedicado a uma
estação trata da parte do ano durante a qual menos se trabalha. Trata"-se
apenas da exibição, por parte do poeta, de seus recursos descritivos? É
clara a oposição com o verão, já que são essas as duas estações nas
quais prepondera a ausência do trabalho, sendo que numa, por outro lado,
o sofrimento é máximo, ao passo que na outra abundam situações
prazerosas (Manakidou 2006).

A linguagem enigmática também não está ausente dessa parte do poema. Se
aceitarmos a leitura de Bagordo (2009) e outros, que, porém, não
convence a todos, então parte do episódio invernal é marcada por
metáforas sexuais, provavelmente ecoando a ausência de fertilidade da
estação, a começar pelo vento Bóreas, que, com a sua potência, consegue
penetrar em quase todos os espaços. Reitera"-se o frio onipresente
através de um catálogo de espaços e criaturas que a ele se submetem (a
infertilidade da estação por meio do seu contrário), e tanto mais
surpreende o vento, um notório raptor de virgens moças, não conseguir
atingir uma jovem que parece ignorar a dureza do inverno por ter uma
série de luxos à sua disposição (519--23). Figuras femininas, sobretudo
mortais, não costumam ser retratadas de forma positiva no poema, e não é
por acaso que se menciona que a moça é inexperiente sexualmente,
sobretudo se aceitarmos que a expressão ``sem"-osso'' alude ao pênis. As
mudanças de estação não exigem ações distintas apenas na roça, mas
também em relação às criaturas femininas, às quais as figuras masculinas
do poema encontram"-se necessariamente ligadas. No episódio invernal,
quem não trabalha ou é um velho alquebrado, um indolente cujo único
prazer é a masturbação ou um pobre coitado que, devido à solidão,
precisa refrear seu impulso sexual. Quando a mulher estiver no auge da
excitação no verão, os homens, por sua vez, estarão na situação oposta.

Após concluir a parte referente à agricultura, Hesíodo também trata
brevemente da navegação, sobretudo das épocas para se viajar com menores
riscos (618--94). Nesse trecho, quase que como \emph{pendant} ao início
do poema, a família do poeta volta ao primeiro plano, não somente
através da explicitação de que Perses é irmão do poeta, mas também que
seu pai é um emigrante que adotou a navegação como forma de tentar fugir
da pobreza. O trecho mais notável, porém, diz respeito a uma disputa
poética na qual Hesíodo foi vitorioso. A passagem é repleta de ironia e
humor: Hesíodo fala de navegação sem nunca ter navegado, com exceção de
um trecho de menos de cem metros, entre o continente e uma ilha próxima.
O conhecimento que carece para compor seu canto é dado pelas Musas, e
parece ser construída uma comparação com a poesia épica heroica em
relação à qual a tradição representada por Hesíodo é vitoriosa, um feito
que a posteridade representará de diversas formas (Torrano 2005, Koning
2010).

Como acontece alhures na poesia hexamétrica arcaica, o trecho final
desse poema representa como que um anticlímax, ou melhor, ele retoma
temas ou formas já explorados, só que em uma chave menor, informando o
público ouvinte, desse modo, de que a performance poética está chegando
ao fim (Kelly 2007). Os ``dias'' referem"-se a alguns dias fastos e
nefastos nos meses em geral, e sua menção reforça para o ouvinte que o
sucesso de um homem não depende apenas do seu próprio trabalho, mas de
uma série de fatores que estão além de sua vontade, entre eles, a
arbitrariedade dos deuses (823--28).


\section*{Teogonia}

%\subsection{Hino às Musas: o proêmio do poema}

Por certo é significativo que o narrador da \emph{Teogonia} --- ao
contrário do narrador dos poemas homéricos --- se nomeie no início do
poema (mas apenas uma única vez!) no momento mesmo em que é narrado seu
encontro singular com a entidade religiosa tradicional que confere
autoridade a seu canto e garante a precisão de seu conteúdo, as Musas.
Os primeiros 115 versos do poema compõem um proêmio, no qual se celebram
essas divindades (1--103) e se demarca explicitamente o conteúdo do canto
a seguir (104--15). O trecho se assemelha a uma forma poético"-religiosa
tradicional em várias sociedades antigas, o canto que celebra as
honrarias ou áreas de atuação (\emph{timē} no singular) de um deus e
que, mais tarde, passou a ser denominado ``hino''
(\emph{humnos}).\footnote{O substantivo (que aparece uma vez na
  \emph{Odisseia}) e o verbo cognato (diversas vezes na
  \emph{Teogonia}), que traduzi por ``louvar'' ou ``cantar''
  (Torrano1992 traduz consistentemente pelo neologismo ``hinear''), não
  parecem ser associados primordialmente a deuses nesses textos.} Com
efeito, tal tipo de canto ganhou na Grécia Antiga, em algum momento, uma
versão narrativa no contexto da tradição hexamétrica: são os hinos
homéricos longos ou médios (Ribeiro Antunes \emph{et al.} 2011; Antunes
2015). O que há de muito particular nesse hino da \emph{Teogonia},
porém, é que somente os gregos conheceram essas divindades coletivas
responsáveis por uma esfera cultural que podemos chamar de poesia, mas
que envolvia também música e dança.

Ao celebrar as Musas antes de apresentar o canto que elas propiciam, ou
seja, a cosmogonia e teogonia que começam no verso 116, o poeta também
fala da relação que há entre ele próprio e essas divindades, pois o
valor (de verdade), ou seja, a autoridade do canto que apresenta depende
dessa relação. Como pode um mortal falar de eventos pretensamente reais
que não presenciou --- o surgimento do mundo conhecido e de todas as
divindades, bem como dos mortais que com elas dormiram --- se não
apresentar e fundamentar sua relação com certa autoridade transcendente,
já que não há uma tradição (textual) canônica e uniforme independente do
poema? Nesse sentido, não é mais possível, para nós, saber com certeza
se algum dia ouve um poeta chamado Hesíodo e que foi o autor do poema
que conhecemos, ou se ``Hesíodo'' teria sido uma autoridade
\emph{mítica} inseparável de certa tradição poética e que seria
reencarnada a cada apresentação do poema, um pouco como o ator que
reencarnaria, com uma máscara ritual, nas apresentações teatrais
atenienses no século V a.C., as figuras tradicionais do mito (Nagy
1990). Nesse diapasão, a iniciação no canto, conduzida pelas Musas, pela
qual teria passado o poeta Hesíodo (9--34) também faria parte desse
contexto mítico.

Isso pode ser exemplificado pelo nome ``Hesíodo''. Por certo não é
possível \emph{provar} que não tenha existido uma figura histórica com
esse nome responsável pela composição de um ou mais poemas associados ao
nome (Cingano 2009). Além disso, a etimologia do nome não é segura e tem
sido interpretada de diferentes modos (Most 2006, p. xiv"-xvi).
Meier"-Brügger (1990), por exemplo, rediscutiu todas as hipóteses e
defendeu que ``Hesíodo'' significa ``aquele que se compraz com
caminhos'', o que pode ser interpretado metapoeticamente. Contudo, o
contexto imediato da única vez em que o nome é mencionado no poema
parece indicar que a expressão \emph{ossan hieisai} (``voz emitindo''),
repetida diversas vezes no proêmio (10, 43, 65 e 67), seria uma glosa de
``Hesíodo'' (Nagy 1990, p. 47--48; Vergados 2020, p. 43--46), um exemplo
entre vários (veremos outros abaixo) do que Vergados (2020) define como
o pensamento etimológico do autor.

Outro elemento saliente no proêmio é Zeus. Na verdade, como soberano dos
deuses e dos homens, ou seja, como deus responsável pela estrutura
sociopolítica final do cosmo e, dessa forma, também pela manutenção de
sua dimensão física, não é raro Zeus desempenhar algum papel nos hinos
aos deuses que conhecemos, sobretudo, os hinos homéricos maiores. Sua
presença no proêmio da \emph{Teogonia}, porém, é ubíqua, e não apenas
como pai das Musas e seu público primeiro e principal (não nessa ordem
na sequência do poema!), mas também como o deus que, em vista do que
representa, é particularmente associado ao poder político exercido pelos
reis (\emph{basileus} no singular) no mundo humano. Não surpreende,
assim, que, no final do proêmio, as Musas sejam apresentadas como
sombremaneira ligadas não só aos poetas (94--103), mas também aos reis
(80--93), uma figura que, no contexto hesiódico, não representa um
monarca com amplos poderes, mas uma figura que, na esfera pública, age
sobretudo na função de um juiz (Gagarin 1992). O tipo de poder real
exercido por Zeus no poema --- o poder é absoluto e hereditário --- não é
homólogo àquele dos líderes políticos da época. O rei humano é antes de
tudo um aristocrata com prestígio local que participa da administração
da justiça. Que reis e poetas, porém, são figuras dissociáveis, isso
fica claro no destaque dado a Apolo nessa passagem; de qualquer forma, o
proêmio sugere que, entre os homens, poetas são figuras bastante
próximas dos reis (Laks 1996).

\subsection{Genealogias divinas}

No poema, teogonia e cosmogonia são inseparáveis à medida que o espaço
se constitui e as genealogias divinas se sucedem. As divindades que
passam pelo poema --- mais de 300 --- são de diversos tipos no que diz
respeito a cultos e mitos (West 1966): (1) os deuses do panteão
(sobretudo os Olímpicos, como Zeus, Apolo, Atena e Ártemis), cultuados
pela Hélade mas de uma forma mais específica que aquela com que aparecem
no poema (por exemplo, vinculados a certo lugar ou templo específicos);
(2) deuses presentes nas histórias míticas, mas que provavelmente nunca
foram exatamente objetos de culto (Atlas e, provavelmente, como
coletividade, os Titãs); (3) partes do cosmo divinizados (Terra, Noite,
Montanhas; alguns eram cultuados); (4) personificações (elementos que,
para nós, são abstratos, mas não o eram para os gregos); (5) aqueles
sobre os quais nada sabemos fora de Hesíodo, ou seja, podem ser parte de
um recurso típico dessa tradição, que permitiria a ``criação'' de
divindades para compor catálogos ou expressar caraterísticas de uma
linhagem (isso não deve ser confundido com ficção nem com inovação).

Essa tipologia, porém, não deve ser tomada como algo estático e
invariável. Eros, por exemplo, pode ser pensado como um deus de culto ou
não. Com efeito, o poema não pode ser om retrato de uma estrutura
religiosa fixa, pois essa não existia. Pelo contrário, ele e a tradição
da qual faz parte deveriam ser antes pensados como uma tentativa de
enquadrar, de dar certa forma a uma vivência religiosa que é
essencialmente plural no tempo e no espaço. O lance astuto incorporado
pela tradição --- ou pelo autor do poema --- é justamente procurar
apresentar como um sistema obviamente fixo algo que é necessariamente
variável. A isso está ligado seu sucesso pan"-helênico.

\subsection{Afrodite}

Um dos modos do poeta expressar o que cada divindade tem de específico é
a derivação do seu nome e de seus epítetos. Uma das construções mais
desenvolvidas que exemplificam é a que trata do nascimento, a partir do
esperma de Céu (\emph{Ouranos}), de Afrodite (192--200):

\begin{verse}
(\ldots{}) primeiro da numinosa Citera achegou"-se,\\
e então de lá atingiu o oceânico Chipre.\\
E saiu a respeitada, bela deusa, e grama em volta\\
crescia sob os pés esbeltos: a ela Afrodite \num{195}\\
espumogênita e Citereia bela"-coroa\\
chamam deuses e varões, porque na espuma (\emph{aphros})\\
foi criada; Citereia, pois alcançou Citera;\\
cipriogênita, pois nasceu em Chipre cercado"-de"-mar;\\
e ama"-sorriso (\emph{philommeidea}), pois da genitália (\emph{mēdōn}) surgiu.
\end{verse}

Ora, à medida que o narrador, devido ao encontro que teve com as Musas,
garante estar falando a verdade, ao mostrar, através do próprio nome
(aceito em toda a Hélade) do deus que as histórias que ele conta como
que estão inscritas na identidade verbal mesma do deus, ele confronta
outras histórias --- de outras tradições --- que não revelariam o mesmo
conhecimento profundo e inequívoco da realidade por ele dominado. A
filiação da Afrodite de Homero --- ela é filha de Zeus e de Dione --- como
que sucumbe às ``provas'' dadas na \emph{Teogonia}, cuja lógica só tem
espaço para uma Afrodite, a filha de Céu.

O surgimento de Afrodite é um dos nascimentos que marcam o fim da
supremacia de Céu sobre o cosmo incipiente, ou seja, um momento de crise
que antecede o equilíbrio cosmológico verificado ainda hoje pelos
ouvintes do poema no seu cotidiano. Depois de Céu, também Crono, seu
herdeiro como deus patriarca detentor do poder soberano, é derrotado;
somente Zeus, como rei dos deuses e homens, sempre tem sucesso nos
conflitos que enfrenta. No século \textsc{xx} percebeu"-se que o chamado ``mito de
sucessão'', fundamental para o entendimento do poema, composto por três
gerações de deuses e seus ``patriarcas'' (Céu, Crono e Zeus) e os
conflitos principais que cada uma enfrenta --- a castração de Céu, o
nascimento de Zeus possibilitado pelo truque da pedra aplicado por Reia
e o combate de Zeus contra os Titãs e, posteriormente, Tifeu --- guarda
semelhanças em graus diversoscom mitos equivalentes transmitidos por
outras culturas antigas do Oriente, como a babilônia e hurro"-hitita
(Rutherford 2009, Kelly 2019). O intercâmbio verificado entre essas
culturas problematiza, assim, a origem necessariamente nebulosa mas
certamente não helenocêntrica do poema (ou pelo menos de parte dele). A
maioria dos intérpretes concorda, hoje, que, de Homero e Hesíodo a
Platão, não deve ter havido nada parecido com um ``milagre grego'',
ainda que não possamos sempre rastrear com precisão como teriam ocorrido
os diversos casos de intercâmbio entre as culturas orientais e a grega
(Burkert 1992, West 1997, Rutherford 2009, Haubold 2013).

\subsection{Astúcia «versus» força e as criaturas prodigiosas}

Os eventos do mito de sucessão são permeados por um par de opostos
complementares fundamental na mitologia, vale dizer, na cultura grega,
``astúcia'' e ``força'' (Detienne \& Vernant 2008). É ele, por exemplo,
que subjaz à oposição entre os heróis máximos dos dois poemas homéricos,
Odisseu e Aquiles, o primeiro, o astuto por excelência, o segundo, o
herói grego mais temido pelos troianos devido à sua força. Também é essa
oposição que mostra, em diversas fábulas, animais mais fracos
fisicamente derrotando os mais fortes ou velozes. No caso da
\emph{Teogonia}, desde o início a astúcia tem a particularidade de ser
uma característica essencialmente feminina. É de Terra o plano ardiloso
que permite a derrota de Céu; Farsa (\emph{Apatē}) é filha de Noite; e
Astúcia (\emph{Mētis}) --- além de Persuasão (\emph{Peithō}) --- é uma das
dezenas de filhas de Oceano. No mito de sucesso, a divindade que usar
apenas uma das qualidades ou a usar de modo desproporcional em relação à
outra sempre sucumbe a adversários que combinam as duas de forma mais
eficaz.

Por outro lado, é a Terra que está ligada a geração dos seres
tradicionalmente chamados de monstros (270--335), Équidna, Hidra de
Lerna, Leão de Nemeia, Medusa, Pégaso, Cérbero, Quimera etc. O que
caracteriza tais criaturas como uma coletividade é que elas não se
assemelham nem a deuses, nem aos homens, nem aos animais, mas são sempre
seres estranhamente mistos, dotados --- assim como sua ancestral primeira
--- de um inominável, enorme poder, algo que faz deles seres incapazes de
serem conquistados pelos mortais, ou seja, ``impossíveis''
(\emph{amêkhanos}). Nesse sentido, e tendo em vista a história do termo
``monstro'', Zanon (2018) mostrou ser mais apropriado chamar essas
criaturas de ``prodígios''. Os únicos que as superaram foram certos
heróis, homens muito superiores em força e astúcia que os homens de hoje
e que, além disso, foram auxiliados por deuses.

Pela lógica da narrativa, as criaturas prodigiosas parecem ser uma
espécie de tentativa mal sucedida de continuar o desenvolvimento do
cosmo (Clay 2003), já que, em sua maioria, não têm função alguma salvo
contribuírem para a fama do herói que os derrotou. Além disso, por meio
delas se mostra que, assim como, no plano humano, mortais comuns se
opõem a heróis, no divino, deuses se opõem a monstros. Além disso, como
notou Pucci (2009), alguns deuses da geração de Zeus utilizam, eles
próprios, uma criatura para obter determinado fim pessoal, o que
sinaliza que o equilíbrio cósmico continua instável. Os monstros
presentes no poema indicam, para o leitor do presente, que, por ora, a
fertilidade feminina consubstanciada em Terra e que, na sua forma mais
frenética e disforme, gerou tais criaturas --- veja que nos versos 319 e
326 não fica claro quem é a mãe do respectivo monstro, o que parece
acentuar o desregramento ---, foi dominada e regrada por um elemento
masculino, mas esse não será, necessariamente, o fim da história. No
século \textsc{xx} e \textsc{xxi}, ``monstros'' continuam a assombrar a fantasia humana,
seja na forma de ameaças espaciais ou da guerra atômica, seja como
consequência da forma com que o homem trata o planeta em que habita ---
ou seja, novamente é Gaia quem parece deter a palavra final e, desta
vez, inalienável.

\subsection{Estige e Hécate}

Como que a contrabalançar o peso negativo dessas criaturas, na sequência
nascem duas coletividades benfazejas, os Rios e as Oceaninas (337--70),
e, entre essas últimas, destacam"-se duas figuras femininas, Estige e
Hécate (383--452). Ambas aparecem na narrativa, de forma anacrônica, para
serem cooptadas por Zeus, cujo nascimento ainda não ocorreu. Isso se
deve, como já foi mencionado acima, pela lógica própria do poema. As
duas divindades femininas não só se opõem à negatividade essencialmente
feminina dos monstros, mas também preparam a narrativa por vir. Estige
ela mesma e seus filhos antecipam a vitória cósmica de Zeus e o novo
equilíbrio que ele vai instaurar e manter. Esse equilíbrio, porém, não é
resultado de uma tábula rasa, mas dá continuidade ao que já estivera
equilibrado durante a supremacia de Crono.

Hécate, por sua vez, é a deusa que permite a primeira irrupção mais
substancial dos homens no poema. Como o objetivo do poema é revelar a
ordem do cosmo e as prerrogativas dos deuses e celebrá"-los, é esperada a
posição absolutamente marginal que o gênero humano ocupa no poema (Clay
2003). Os homens e seu modo de vida são os protagonistas de outro poema
atribuído a Hesíodo, \emph{Trabalhos e dias}. Isso não significa, porém,
que, do ponto de vista dos próprios deuses, ou seja, em última análise,
da própria \emph{Teogonia}, as características da fronteira que separa
deuses e homens não sejam relevantes. Essas aparecem com clareza em dois
episódios que emolduram o nascimento de Zeus, a celebração de Hécate e a
história de Prometeu.

Se aos heróis --- esses humanos (mortais) que, vale assinalar, estão no
meio do caminho entre deuses e homens --- é dada uma razão de ser durante
o catálogo de monstros, a relação entre Zeus e Hécate, num momento do
poema em que se enfatiza o equilíbrio cósmico resultante das
responsabilidades diversas atribuídas a cada deus, revela que esse
equilíbrio é indissociável da presença, na terra, dos homens. Dito de
outro modo: para pensar"-se, figurar"-se o modo como os deuses são no
mundo por meio da sequência de eventos que levou à ordem presene,
utiliza"-se também um retrato simplificado e razoavelmente genérico das
práticas cultuais humanas. Deuses, cosmo e homens não existem um sem o
outro. O trecho dedicado a Hécate, porém, revela também que a vida
humana, mais que marcada por certo equilíbrio, é permeada pelo
imponderável: por mais que os homens propiciem os deuses, nada garante
que serão auxiliados por eles.

Não possuímos nenhum testemunho histórico independente da
\emph{Teogonia} que aponte para a importância cultual, mesmo que apenas
local, de Hécate sugerida pelo destaque que lhe dado no poema. Isso é um
forte indício de que comentadores como Clay (2003) estão corretos ao
defender que a figura dessa deusa é usada para se falar de Zeus e da
relação entre os homens e os deuses inaugurada por ele. Menos certa é a
relação entre o nome de Hécate, a maneira como o poeta se refere ao seu
modo de atuação (``se ela quiser'' etc.) e o acaso.

\subsection{Zeus e Prometeu}

O nascimento de Zeus narrado logo depois (453--91) é o evento que permite
a queda de Crono e a ascensão do terceiro soberano dos deuses. A astúcia
de Terra é a responsável pela castração de Céu, a libertação
(\textasciitilde{}nascimento) de seus filhos, os Titãs, e a tomada de
poder por parte do filho mais novo, Crono. De forma homóloga, é a
astúcia da esposa de Crono, Reia, auxiliada pelos conselhos de Céu e
Terra, que permite que seus filhos vejam a luz do Sol e Zeus destrone o
pai. Desta vez, porém, há uma verdadeira competição entre astutos: como
todo bom rei, Crono é previdente, e, ao aprender (parcialmente) com o
erro de seu pai, decide engolir todos os filhos \emph{após} esses serem
paridos por sua esposa, com o que, porém, ainda exercita de uma forma
arbitrária sua força. Reia, porém, o ludibria no nascimento de Zeus, de
sorte que esse, por meio de uma série de manobras contadas rapidamente
no poema, pode ocupar a regência do cosmo. Ainda que, pelo menos em
parte, nesse momento da narrativa Zeus não seja representado como um
agente deliberando sozinho, seu poder é de pronto ligado às duas esferas
mencionadas acima, astúcia e força. Por enquanto, sua astúcia ainda é
aquela da mãe e da avó; sua força, porém, está ligada ao seu primeiro
ato como soberano --- do ponto de vista da lógica da narrativa: a
libertação dos Ciclopes (501--6), aqueles que lhe fornecerão os raios e o
trovão, atributos que, por certo, funcionam como armas mas também são
simbólicos, já que apontam para sua ligação com o céu.

O primeiro conflito resolvido por Zeus, porém, envolve a astúcia
(507--616). Trata"-se do momento em que deuses e homens se distinguiram,
se separaram em definitivo por ocasião de um banquete festivo para o
qual Prometeu separou a carne de um boi. Marcam esse evento a origem do
sacrifício, a conquista do fogo e a criação da mulher humana. O texto
não procura descrever detalhadamente a linhagem humana que não dominava
o fogo, ainda compartilhava da companhia dos deuses e não conhecia a
reprodução sexual; isso é feito, sob viés distinto, em \emph{Trabalhos e
dias}. Todavia, como o narrador deixa claro que Zeus aceita a repartição
da carne do boi feita por Prometeu para o banquete porque ele tinha em
mente males destinados \emph{aos homens mortais} (551--52), podemos supor
que, nesse momento de sua regência, quando Zeus ainda precisa consolidar
seu poder, os homens (os versos 50 e 185--87 talvez sugiram que esses
fossem guerreiros gigantes nascidos da terra, figuras que conhecemos de
outros relatos), em conluio com Prometeu, representam uma ameaça que
precisa ser dominada antes que seja tarde demais. A previdência é um
atributo indispensável do soberano que quiser manter seu poder. Ao
contrário de Zeus, que antecipa o movimento do provável inimigo, Crono
falhou em sua tática de engolir os filhos: bastou que um escapasse para
ele ser destronado.

Outro momento fundamental da história de Prometeu é a criação da
primeira mulher. Ao contrário do que ocorre em \emph{Trabalhos e dias},
aqui o narrador não informa seu nome (lá, Pandora). Como em todos os
eventos que marcam o episódio de Prometeu, bem e mal estão
indissociavelmente ligados (Vernant 1992 e 2002): nessa etiologia do
sacrifício, os ossos, que não podem ser digeridos (mal), são encobertos
pela gordura que solta delicioso aroma (bem), ao passo que a carne (bem)
é disfarçada sob o repelente estômago (mal). Assinale"-se que o disfarce
--- e, consequentemente, a habilidade de reconhecer o que está disfarçado
--- também faz parte do domínio da astúcia: se Prometeu é astuto, Zeus o
é em ainda mais alto grau. Os ossos, bem como o aroma da gordura
queimada, são, por outro lado, sinais da imortalidade divina (bem), ao
passo que a carne deliciosa (o alimento perecível) comida pelos homens
aponta para sua mortalidade (mal). A a adoção da carne em sua dieta,
escondida no estômago do boi, deixa claro que os homens são escravos de
seu próprio estômago e precisam satisfazê"-lo se não quiserem perecer.

No caso da Mulher, ela é dada aos homens em troca do fogo: ao passo que
o fogo permite que os homens sejam civilizados e não comam carne crua, a
mulher terá que ser por eles alimentada, caso queiram sobreviver por
intermédio de um herdeiro. De fato, fogo e mulher precisam ser
constantemente alimentados para que o homem não pereça. O sacrifício, o
fogo e a bela mulher, portanto, indicam que há elementos que apontam
para uma presença do divino no centro da vida humana, mas eles são tão
tênues como a fumaça que sobe do sacrifício para o céu e tão artificiais
quanto os enfeites da coroa da primeira mulher, contra a qual o homem
não tem defesa alguma.

\subsection{Titanomaquia}

Após essa separação entre deuses e homens levada a cabo por Zeus graças
à astúcia, a separação seguinte, entre os deuses da geração de seu pai,
os Titãs, e os da sua própria, os Olímpicos, é conseguida devido à
supremacia alcançada sobretudo por meio da força. O episódio conhecido
como Titanomaquia (617--720) mostra que o cosmo ficou mais complexo que
quando sobre ele regia Céu, pois se, para vencer seu pai, num primeiro
momento, Zeus contou com pelo menos dois ardis arquitetados pela mãe e
pela avó --- entregar a Crono uma pedra no lugar do bebê Zeus e,
posteriormente, fazê"-lo vomitar todos os irmãos de Zeus que com ele por
fim lutariam contra os deuses mais velhos ---, num segundo momento, a
astúcia deixa de ser suficiente.

É de novo Terra quem aconselha ao neto libertar aqueles que haviam sido
presos por Céu e assim mantidos por Crono abaixo da terra, os
Cem"-Braços. Trata"-se de uma força descomunal que os dois soberanos
anteriores acharam por bem simplesmente manter paralisada, paralisia
homóloga àquela que tentaram, sem sucesso, implementar contra seus
filhos. Zeus, porém, consegue convencê"-los a serem seus aliados e eles
se mostram decisivos no combate contra os Titãs, gratos por serem
trazidos de volta à luz.

Luz e trevas: essa polaridade marca toda a Titanomaquia, pois os Titãs,
uma vez vencidos, passam a ocupar o espaço subterrâneo onde antes
estiveram os Cem"-Braços que, porém, agora tem uma honra, uma função no
cosmo, a de serem os eternos guardas dos deuses outrora poderosos, os
Titãs. Essa polaridade, ademais, também prepara o episódio seguinte,
pois o esforço de Zeus para vencer os Titãs como que traz o cosmo de
volta ao seu estado inicial: terra, céu, mar e Tártaro, todos os espaços
são atingidos pelo fogo dos raios de Zeus, o que representa uma
recriação do mundo por meio da força. Não é por acaso que Abismo volta à
cena (700 e 814) e que as imagens e sons desse conflito cataclísmico
sejam amplificadas para o ouvinte por meio de uma imagem que remete à
união primordial entre Terra e Céu (700--5).

Uma vez finalizada a guerra, o narrador nos narra, pela primeira vez,
como é a geografia das terras ínferas (721--819). Não que antes nada lá
houvesse. Com o aprisionamento dos Titãs, porém, à essa parte do cosmo é
conferida sua estabilidade e Zeus pode finalmente aparecer como o
organizador último de todos os espaços. É por essa razão que de deuses
como Sono e Morte e Noite e Dia, cujas funções cósmicas os ligam ao
Tártaro, finalmente se fala mais longamente, uma vez mais se mostrando
de que forma pólos positivos e negativos da realidade estão
interligados. É precisamente por isso que também nesse momento do poema
descreve"-se a função de Estige, ligada a uma jura divina que, quando
quebrada por um deus, o leva a uma morte virtual por dez anos. A ligação
entre Estige e Zeus mostra que também o juramento --- uma instituição
social fundamental também entre os homens --- é instituído pelo rei dos
deuses e homens para bem administrar o mundo divino onde conflitos não
são excepcionais.

\subsection{Zeus e Tifeu}

Curiosamente, porém, Zeus ainda terá que enfrentar mais um conflito
belicoso, a luta contra Tifeu (820--80). Por um lado, como nos dois
poemas épicos que conhecemos, a \emph{Ilíada} e a \emph{Odisseia}, o
maior herói se revela quando um derrota inimigo poderoso com suas
próprias mãos. Por outro lado, esse inimigo é, estranhamente, filho do
próprio Tártaro com Terra. Que a fertilidade exacerbada desta tenho
gerado um ser para destronar o novo senhor do cosmo, isto não
surpreende, pois a eminência parda feminina foi peça fundamental na
deposição de Céu e Crono; que aquele seja o pai, isto sim é curioso,
pois até este momento da narrativa dele apenas se falou como um espaço.
É como se, pela lógica da narrativa hesiódica, só agora ele tivesse
adquirido o estatuto pleno de divindade e precisasse se envolver em um
conflito que garanta que sua forma não se alterará.

Tifeu, por sua vez, adquire, devido à lógica da narrativa, o lugar de
filho de Zeus, pois todo rei anterior fora deposto por seu filho (sempre
ligado à Terra). O conflito contra os Titãs, porém, já mostrou que a
manipulação da astúcia e da força, no grau superlativo em que o faz
Zeus, não deixa espaço para a possibilidade de derrota, mesmo que o
adversário também seja muito forte --- Tifeu tem cabeças com olhos de
onde sai fogo --- e muito astuto --- suas cem cabeças produzem todo tipo
de som, sendo que a metamorfose é um elemento mítico típico do universo
da astúcia. Além disso, esse combate singular entre a criatura
monstruosa e Zeus também permite que Terra, derradeiramente, seja
derrotada e esterilizada. O fogo de Zeus como que a derrete: de criadora
de metal e artífice metalúrgica, Terra como que se transforma, graças ao
fogo aniquilador de Zeus, no metal que é manipulado por artesãos machos
(861--67).

\subsection{Zeus e suas esposas}

Uma vez derrotada a astuta Terra, que imediatamente se torna aliada de
Zeus (891), a primeira providência do soberano é casar com Astúcia e,
antes de essa parir seu primeiro filho, devorá"-la, não esperando que
essa gerasse um deus macho mais forte que ele (886--900). Muellner (1996)
mostrou como esse episódio arremata todos os conflitos dinásticos
narrados até então: Zeus não devora seu primeiro filho (como Crono) ou
obriga que sua esposa o guarde no ventre (como Céu), mas assimila o
elemento feminino em si mesmo, (Astúcia) e o gera como aliado (Atena).
Com isso, Zeus se torna um andrógino perfeito (do ponto de vista grego:
muito mais masculino que feminino), e não um disforme emasculado (como
Céu). A astúcia revela"-se mais uma vez essencialmente feminina, mas para
sempre assimilada pelo próprio rei. A filha produzida pelo rei não só
não é um macho --- e foram sempre jovens machos que derrotaram seus pais
---, mas é uma virgem, ou seja, uma deusa que não irá produzir uma ameaça
ao \emph{status quo}. Por fim, ao ingerir a esposa grávida do primeiro
filho, ele bloqueou a previsão de que, depois de Atena, Astúcia geraria
um filho mais forte que o pai. Pela primeira vez, o rei dos deuses
consegue ``desparir'' de forma perfeita e acabada.

E somente agora nasce, de Zeus e várias de suas esposas, uma linhagem de
deuses responsáveis pelo que há de bom no cosmo propriamente humano, ou
seja, na sociedade (901--17): Norma, Decência, Justiça, Paz, as Musas,
Radiância, Alegria e Festa, notável prole antípoda aos filhos de Noite e
Briga. A última esposa de Zeus, Hera, é aquela que, de acordo com a
lógica do poema, representa a maior ameaça a Zeus, mas tanto o filho
mais perigoso que os dois têm juntos, quanto aquele que Hera, como que
emulando Zeus no caso de Atena, tem sozinha, Hefesto, não representam
adversários fortes o suficiente contra a filha que mais se assemelha ao
pai e está completamente alinhada com ele, Atena, senhora da guerra mas
também da astúcia (921--29).

É nesse sentido que se deve entender o longo catálogo que finaliza o
poema e que tem três partes: os casamentos de Zeus e os filhos deles
resultantes (901--29); um catálogo mais abrangente de casamentos divinos
(930--61), que revelam, de forma sumária, um panteão muito bem organizado
e potencialmente harmônico (como que servindo de epítome, o casamento
entre Ares e Afrodite produz, por um lado, os machos Terror e Pânico,
mas, por outro, Harmonia); e finalmente um catálogo de deusas que se
uniram a mortais (962--1020). Ora, com as deusas fêmeas que se unem a
machos mortais, o princípio de ruptura que vigorara ao longo do poema
agora se desloca para o mundo dos homens, mais precisamente, o mundo dos
heróis: nesse mundo, filhos poderão ser mais fortes que os pais,
podendo, no limite, o que atesta Telégono, o filho de Circe e Odisseu,
matá"-lo.

Para concluir, mencione"-se que há uma discussão inconclusa sobre onde a
versão ``original'' da \emph{Teogonia} teria terminado. Autores como
Clay (2003) e Kelly (2007) mostraram que os catálogos tal como
analisados acima compõe um final muito adequado ao poema; assim,
provavelmente somente os quatro ou possivelmente os dois últimos versos
foram acrescentados ao poema em um certo momento de sua transmissão para
introduzir um outro poema atribuído a Hesíodo, o \emph{Catálogo das
mulheres} (obra que chegou a nós por meio de fragmentos), que procurava
dar uma visão geral da idade dos heróis a partir das mulheres que com
deuses dormiram por toda a Grécia, catálogo este que, possivelmente, era
concluído pelo catálogo de pretendentes de Helena, cujo casamento
redundou no grande cataclisma que foi a guerra de Troia, que
metonimicamente podia ser pensada, na Antiguidade, como o fim da época
dos heróis.

\section*{Da tradução}

Para definir o texto grego aqui traduzido, cotejaram"-se as seguintes
edições: West (1966), Most (2006), Ercolani (2010) e Ricciardelli (2018). Também foram
muito úteis para se definir a opção por determinada leitura ou
interpretação, bem como para compor as notas, diversos textos citados na
bibliografia, especialmente Marg (1970), Renehan (1980), Verdenius (1972), Arrighetti
(2007), Pucci (2007), Canevaro (2015) e Vergados (2020). Para a tradução, também foi
fundamental o léxico organizado por Snell \emph{et al.} (1955--2010).

Um dos principais problemas enfrentados pelo tradutor de Hesíodo
diz respeito ao nomes das divindades. Não se buscou
nenhum tipo de padronização muito rígida, ou seja, ficou"-se entre os
extremos de traduzir (quase) todos os nomes e (quase) nenhum nome. De
forma geral, os principais critérios foram o bom senso, o conhecimento
do leitor e a sonoridade. Além disso, as notas apresentam a
transliteração de todos os nomes, bem como explicitam algumas figuras
etimológicas. Recuperaram"-se nas notas também as inevitáveis perdas na tradução, como, por exemplo, algumas figuras etimológicas.

Para facilitar a leitura, no caso de \textit{Teogonia},
optou"-se por seguir o que fazem a maioria dos
editores em sua forma de propor uma divisão do poema em partes
distintas. O recuo de parágrafo, ainda que estranho em um poema, deve
ser pensado como equivalente a um novo parágrafo em uma narrativa em
prosa. Não é possível saber, entretanto, se tais marcações são
equivalentes a pausa nas performances orais originais dos poemas.
Trata"-se, portanto, de um recurso eminentemente didático.
Já em \textit{Trabalhos e dias} essas marcações foram deixadas de lado,
para que cada leitor procure ele mesmo sua forma de estabelecer vínculos, unidades e cortes.

Algumas soluções que adotei nas minhas traduções de Homero (2018a) e
(2018b) nortearam certas modificações nesta edição da tradução dos poemas
hesiódicos. Uma delas é evitar excessos no uso da ordem sintática
indireta.

Por fim, gostaria de agradecer àqueles que compartilharam comigo seu
conhecimento de Hesíodo, em especial, da \emph{Teogonia} e \textit{Trabalhos e dias}, desde a 1ª edição deste volume ou me apontaram o que nele poderia ser melhorado ou
corrigido: Bruno Palavro, Juarez (Jota) Oliveira,
Lilah"-Grace Canevaro, Camila Zanon, Thanassis Vergados, Jim Marks, Adrian Kelly,
Teodoro Assunção, André Malta, os membros da minha banca de
livre"-docência (Jaa Torrano, Zélia de Almeida Cardoso, Jacyntho L.
Brandão, Pedro Paulo Funari e Maria Beatriz Florenzano) e
Antonio"-Orlando Dourado Lopes.


\begin{bibliohedra}
\tit{ALLAN}, W. Divine justice and cosmic order in early Greek Epic.
\emph{Journal of Hellenic Studies} v. 126, 2006, p. 1--35.

\tit{ANDERSEN}, Ø.; \textsc{haug}, D. T. T. (org.) \emph{Relative chronology in early
Greek epic poetry}. Cambridge: Cambridge University Press, 2012.

\tit{ANTUNES}, C. L. B. 26 hinos homéricos. \emph{Cadernos de literatura em
tradução} v. 15, p. 13--23, 2015.

\tit{ARNOULD}, D. Les noms des dieux dans la \emph{Théogonie} d'Hésiode:
étymologies et jeux de mots. \emph{Revue des études grecques} v. 122,
2009, p. 1--14.

\tit{ARRIGHETTI}, G. \emph{Esiodo opere}. Introdução, tradução e comentário.
Milano: Mondadori, 2007.

\tit{BAGORDO}, A. Zum \emph{anósteos} bei Hesiod (\emph{Erga} 524):
Griechische Zoologie, indogermanische Dichtersprache oder etwas
anderes?. \emph{Glotta} v. 85, 2009, p. 31--58.

\tit{BAKKER}, E. J. Hesiod in performance. In: \textsc{loney}, A. C.; \textsc{scully}, S. (org.) \emph{The Oxford Handbook of Hesiod}. Oxford: Oxford, 2018.

\tit{BEALL}, E. F. Notes on Hesiod's \emph{Works and days} 383--828.
\emph{American Journal of Philology} v. 122, 2001, p. 155--71. {[}pdf{]}

\titidem. The plow that broke the plain epic tradition: Hesiod's \emph{Works
and days} vv. 414--503. \emph{Classical Antiquity} v. 23, 2004, p. 1--31.

\tit{BLAISE}, F.; \textsc{judet de la combe}, P.; \textsc{rousseau}, P. (org.) \emph{Le métier
du mythe: lectures d' Hésiode}. Lille: Presses Universitaires du
Septentrion, 1996.

\tit{BLÜMER}, W. \emph{Interpretation archaischer Dichtung. Die mythologischen
Partien der Erga} \emph{Hesiods}. 2 vol. Münster: Aschendorff, 2001.

\tit{BOYS"-STONES}, G. R.; \textsc{haubold}, J. H. \emph{Plato and Hesiod}. Oxford:
Oxford University Press, 2010.

\tit{BRADFORD WELLES}, C. Hesiod's attitude towards labor. \emph{\textsc{grbs}} v. 8,
1967, p. 5--25.

\tit{BRANDÃO}, J. L. \emph{Antiga Musa (arqueologia da ficção)}. 2ª edição.
Belo Horizonte: Relicário, 2015.

\tit{BURKERT}, W. \emph{The Orientalizing revolution: Near Eastern influence
on Greek culture in the early archaic age}. Cambridge, Mass.: Harvard
University Press, 1992.

\titidem. \emph{Religião grega na época clássica e arcaica}. Lisboa: Fundação
Calouste Gulbenkian, 1993.

\tit{CALAME}, C. Succession des âges et pragmatique poétique de la justice: le
récit hésiodique des cinq espèces humaines. \emph{Kernos} v. 17, 2004,
p. 67--102.

\tit{CANEVARO}, L. G. Genre and authority in Hesiod's \emph{Works and Days}.
In: \textsc{werner}, C.; \textsc{sebastiani}, B.; \textsc{dourado"-lopes}, A. ; (org.) \emph{Gêneros
poéticos na Grécia Antiga: fronteiras e confluências}. São Paulo:
Humanitas, 2014, p. 23--48.

\titidem. \emph{Hesiod's Works and days}: how to teach self"-sufficiency.
Oxford: Oxford University Press, 2015.

\tit{CINGANO}, E. The Hesiodic corpus. In: \textsc{montanari}, F.; \textsc{rengakos}, A.;
\textsc{tsagalis}, C. (org.) \emph{Brill's companion to Hesiod}. Leiden/Boston:
Brill, 2009, p. 91--130.

\tit{CLAY}, J. S. \emph{Hesiod's cosmos}. Cambridge: Cambridge University
Press, 2003.

\tit{COLONNA}, A. \emph{Opere di Esiodo}. Torino: Unione Tipografico"-Editrice,
1977.

\tit{CURRIE}, B. Heroes and holy men in early Greece: Hesiod's \emph{theios
aner}. In: \textsc{coppola}, A. (org.) \emph{Eroi, eroismi, eroizzazioni dalla
Grecia antica a Padova e Venezia}. Padova: S.A.R.G.O.N., 2007, p.
162--92.

\titidem. Hesiod on human history. In: \textsc{marincola}, J. et al. (org.) \emph{Greek
notions of the past in the Archaic and Classical eras: history without
historians}. Edinburgh: Edinburgh University Press, 2012.

\tit{DETIENNE}, M. \emph{Os mestres da verdade na Grécia} arcaica. Trad.: A.
Daher. Rio de Janeiro: Jorge Zahar, 1988.

\titidem; \textsc{vernant}, J.-P. \emph{Métis}: As astúcias da inteligência.
Trad. F. Hirata. São Paulo: Odysseus, 2008.

\tit{ERCOLANI}, A. \emph{Esiodo: Opere e giorni}. Introduzione, traduzione e
commento. Roma: Carocci, 2010.

\tit{FORD}, A. Epic as genre. In: \textsc{morris}, I.; \textsc{powell}, B. (Org.) \emph{A new
companion to Homer}. Leiden: Brill, 1997, p. 396--414.

\tit{GAGARIN}, M. \emph{Dike} in the \emph{Works and days}. \emph{Classical
Philology} v. 68, 1973, p. 81--94.

\titidem. The poetry of justice: Hesiod and the origins of Greek law.
\emph{Ramus} v. 21, 1992, p. 61--78.

\tit{HAUBOLD}, J. \emph{Greece and Mesopotamia}: dialogues in literature.
Cambridge: Cambridge University Press, 2013.

\tit{HEATH}, M. Hesiod' didactic poetry. \emph{Classical Quarterly} v. 36,
1985, p. 245--63.

\tit{HOMERO}. \emph{Ilíada}. Tradução e ensaio introdutório: C. Werner. São
Paulo: Ubu/\textsc{sesi}, 2018a.

\titidem. \emph{Odisseia}. Tradução e introdução: C. Werner. Apresentação: R.
Martin. São Paulo: Ubu, 2018b.

\tit{HUNTER}, R. \emph{Hesiodic voices: studies in the ancient reception of
Hesiod's} Works and Days\emph{.} Cambridge; New York: ~Cambridge
University Press, 2014.

\tit{JANDA}, M. \emph{Über `Stock und Stein': die indogermanischen Variationen
eines universalen Phraseologismus}. Röll: Dettelbach, 1997.

\tit{JANKO}, R. \emph{Homer, Hesiod and the Hymns:} diachronic development in
epic diction. Cambridge: Cambridge University Press, 1982.

\tit{JUDET DE LA COMBE}, P. Le mythe hésiodique des races, oeuvre de langage:
Jean"-Pierre Vernant et après. \emph{L'Homme: Revue française
d'anthropologie} 218, 2016, p. 239--52.

\tit{KELLY}, A. How to end an orally"-derived epic poem? \emph{Transactions of
the American Philological Association} v. 137, 2007, p. 371--402.

\titidem. Gendrificando o mito de sucessão em Hesíodo e no antigo Oriente
Próximo. Trad.: C. A. Zanon. \emph{Classica} v. 32, n. 2, 2019, p.
119--38.

\tit{KONING}, H. \emph{Hesiod}: \emph{the other poet}: ancient reception of a
cultural icon. Leiden: Brill, 2010.

\tit{LAFER}, M. C. \emph{Hesíodo: Os trabalhos e os dias. Tradução, introdução
e comentários}. São Paulo: Iluminuras, 2002.

\tit{LAKS}, A. Le doublé du roi: remarques sur les antécédents hésiodiques du
philosophe"-roi. In~: \textsc{blaise}, F.; \textsc{judet de la combe}, P.; \textsc{rousseau}, P.
(org.) \emph{Le métier du mythe}: lectures d' Hésiode. Lille: Presses
Universitaires du Septentrion, 1996.

\tit{LAMBERTON}, R. \emph{Hesiod}. New Haven: Yale University Press, 1988.

\tit{LARDINOIS}, A. P. M. H. How the days fit the works in Hesiod's
\emph{Works and days. American Journal of Philology} v. 119, 1998, p.
319--36.

\tit{LECLERC}, M.-C. \emph{La parole chez Hésiode: à la recherche de
l'harmonie perdue}. Paris: Belles Lettres, 1993.

\tit{LEDBETTER}, G. M. \emph{Poetics before Plato: interpretation and
authority in early Greek theories of poetry}. Princeton: Princeton
University Press, 2003.

\tit{LONEY}, A. C. Hesiod's temporalities. In: \textsc{loney}, A. C.; \textsc{scully}, S. (org.)
\emph{The Oxford Handbook of Hesiod}. Oxford: Oxford, 2018.

\tit{MACEDO}, J. M. \emph{A palavra ofertada}: um estudo retórico dos hinos
gregos e indianos. Campinas: Edunicamp, 2010.

\tit{MANAKIDOU}, F. P. Autour de la structure des \emph{Travaux et les Jours}
d'Hésiode: la ``dualité'' du cosmos hésiodique. \emph{Gaia} v. 10, 2006,
p. 149--67.

\tit{MARG}, W. \emph{Hesiod}: Sämtliche Gedichte. Artemis: Zürich/Stuttgart,
1970.

\tit{MARSILIO}, M. S. Hesiod's winter maiden. \emph{Helios} v. 24, 1997, p.
101--111.

\titidem. \emph{Farming and poetry in Hesiod's Works and Days}. Boston:
University Press of America, 2000.

\tit{MARTIN}, R. P. \emph{The language of heroes}: speech and performance in
the \emph{Iliad}. Ithaca: Cornell University Press, 1989.

\titidem. Hesiod's metanastic poetics. \emph{Ramus} v. 21, 1992, p. 11--33.

\titidem. Hesiod and the didactic double. \emph{Synthesis} v. 11, 2004, p.
31--54.

\titidem. Hesiod, Odysseus, and the instruction of princes.
\emph{Transactions of the American Philological Association} v. 114,
1984, p. 29--48.

\titidem. Hesiodic theology. In: \textsc{loney}, A. C.; \textsc{scully}, S. (org.) \emph{The
Oxford Handbook of Hesiod}. Oxford: Oxford, 2018.

\tit{MEIER"-BRÜGGER}, M. Zu Hesiods Namen. \emph{Glotta} v. 68, 1990, p. 62--67.

\tit{MORAES}, A. S. de. História e etnicidade: Homero à vizinhança do
pan"-helenismo. \emph{Hélade} v. 5, n. 1, 2019, p. 12--36.

\tit{MORDINE}, M. J. Speaking to kings: Hesiod's \emph{ainos} and the rhetoric
of allusion in the \emph{Works and Days}. \emph{Classical Quarterly} v.
56, p. 363--73, 2006.

\tit{MOST}, G. W. Hesiod and the textualization of personal temporality. In:
\textsc{montanari}, F.; \textsc{arrighetti}, G. (org.) \emph{La componente autobiografica
nella poesia greca e latina}. Pisa: Giardini, 1993, p. 73--91.

\titidem. \emph{Hesiod}: \emph{Theogony, Works and Days, Testimonia}.
Cambridge, \textsc{ma}: Harvard University Press, 2006.

\titidem. Hesiod's myth of the five (or three or four) races. \emph{Proceedings
of the Cambridge Philological Society} v. 43, 1997, p. 104--27.

\tit{MOURA}, A. R. \emph{Hesíodo:} Os trabalhos e dias. Edição, tradução,
introdução e notas. Curitiba: Segesta, 2012.

\tit{MUELLNER}, L. C. \emph{The anger of Achilles}: mēnis \emph{in Greek
epic}. Ithaca: Cornell University Press, 1996.

\tit{MURRAY}, P. Poetic inspiration in early Greece. \emph{Journal of Hellenic
Studies} v. 101, 1981, p. 87--100.

\tit{NAGY}, G. Hesiod and the poetics of Pan"-Hellenism. In: ―. \emph{Greek
mythology and poetics}. Ithaca: Cornell Univesity Press, 1990, p. 36--82.

\titidem. \emph{The best of the Achaeans: concepts of the hero in archaic Greek
poetry}. 2ª ed. Baltimore: Johns Hopkins University Press, 1999.

\tit{NELSON}, S. A. The drama of Hesiod's farm. \emph{Classical Philology} v.
91, 1996, p. 45--53. {[}pdf{]}

\titidem. \emph{God and the land: the metaphysics of farming in Hesiod and
Vergil}. With a translation of \emph{Works and Days} by David Greene.
New York/Oxford: Oxford University Press, 1998.

\tit{NOORDEN}, H. van. \emph{Playing Hesiod}: the `myth of the races' in
classical antiquity. Cambrdige: Cambridge University Press, 2015.

\tit{OLIVEIRA}, J. ``Áurea Afrodite'' e a ordem cósmica de Zeus na poesia
hesiódica. \emph{Codex} -- Revista de estudos clássicos. Rio de Janeiro,
v. 7, n. 2, 2019, p. 69--80.

\titidem. A linhagem dos heróis na cosmologia hesiódica. \emph{Rónai} v. 8, n.
2, 2020, p. 353--374.

\tit{PUCCI}, P. \emph{Hesiod and the language of poetry}. Baltimore/London:
Johns Hopkins University Press, 1977.

\titidem. \emph{Inno alle Muse (Esiodo,} Teogonia\emph{, 1--115): texto,
introduzione, traduzione e commento}. Pisa: Fabrizio Serra, 2007.

\titidem. The poetry of the \emph{Theogony}. In: \textsc{montanari}, F.; \textsc{rengakos}, A.; \textsc{tsagalis}, C. (org.) \emph{Brill's Companion to Hesiod}. Leiden/Boston:
Brill, 2009, p. 37--70.

\tit{RENEHAN}, R. Progress in Hesiod. \emph{Classical Philology} n. 75, 1980,
p. 339--58.

\tit{RICCIARDELLI}, G. \emph{Esiodo: Teogonia}. Milano: Fondazione Lorenzo
Valla / Mondadori, 2018.

\tit{RIJKSBARON}, A. Discourse cohesion in the proem of Hesiod's
\emph{Theogony}. In: \textsc{bakker}, S.; \textsc{wakker}, G. (org.) \emph{Discourse
cohesion in Ancient Greek}. Leiden: Brill, 2009.

\tit{RIBEIRO Jr.}, W. A. \emph{et al.} \emph{Hinos homéricos: tradução, notas
e estudo}. São Paulo: Edunesp, 2011.

\tit{RODRÍGUEZ ADRADOS}, F. La composición de los poemas hesiódicos.
\emph{Emerita} v. 69, 2001, p. 197--223.

\tit{ROSEN}, R. M. Poetry and sailing in Hesiod's \emph{Works and days}.
\emph{Classical Antiquity} v. 9, 1990, p. 99--113.

\tit{ROUSSEAU}, P. Instruir Persès. Notes sur l'ouverture des \emph{Travaux}
d'Hésiode. In: \textsc{blaise}, F.; \textsc{judet de la combe}, P.; \textsc{rousseau}, P. (org.)
\emph{Le métier du mythe}: lectures d' Hésiode. Lille: Presses
Universitaires du Septentrion, 1996, p. 93--168.

\tit{ROWE}, C. J. `Archaic thought' in Hesiod. \emph{Journal of Hellenic
Studies} v. 103, p. 124--35, 1983.

\tit{RUTHERFORD}, I. Hesiod and the literary traditions of the Near East. In:
\textsc{montanari}, F.; \textsc{rengakos}, A.; \textsc{tsagalis}, C. (org.) \emph{Brill's companion to Hesiod}. Leiden: Brill, 2009.

\tit{SCHMIDT}, J.-U. \emph{Adressat und Paraineseform}: zur Intention von
Hesiods `Werken und Tagen'. Göttingen: Vandenhoeck \& Ruprecht, 1986.

\tit{SCODEL}, R. \emph{Works and days} as performance. In: \textsc{minchin}, E. (org.)
\emph{Orality, literacy and performance in the ancient world}. Leiden:
Brill, 2011, p. 111--26.

\tit{SCULLY}, S. \emph{Hesiod's~}Theogony\emph{:~from Near Eastern
creation myths to}~Paradise Lost\emph{.} Oxford and New York: Oxford
University Press,~2015.

\tit{SNELL}, B. O mundo dos deuses em Hesíodo. In: ―. \emph{A cultura grega e
as origens do pensamento}. São Paulo: Perspectiva, 2001.

\titidem. \emph{et al.} (orgs.) \emph{Lexikon des frühgriechischen
Epos}. 4 vol. Göttingen: Vandenhoeck \& Ruprecht, 1955--2010.

\tit{SNODGRASS}, A. M. \emph{The Dark Age of Greece}. Edinburgh: Edinburgh
University Press, 1971.

\tit{STOCKING}, C. Hesiod in Paris: justice, truth, and power between past and
present. \emph{Arethusa} v. 50, n. 3, 2017, p. 385--427.

\tit{THALMANN}, W. G. \emph{Conventions of form and thought in early Greek
epic}. Baltimore/ London: Johns Hopkins University Press, 1984.

\tit{TORRANO}, J. A. A. \emph{Hesíodo: Teogonia}. A origem dos deuses. Estudo
e tradução. 2\textsuperscript{a} edição. São Paulo: Iluminuras, 1992.

\titidem. \emph{O certame Homero"-Hesíodo} (texto integral).
\emph{Letras clássicas} 9, p. 215--24, 2005.

\tit{TSAGALIS}, C. Poetry and poetics in the Hesiodic corpus. In: \textsc{montanari},
F.; \textsc{rengakos}, A.; \textsc{tsagalis}, C. (org.) \emph{Brill's companion to
Hesiod}. Leiden: Brill, 2009, p. 131--78.

\tit{VERDENIUS}, W. J. Notes on the proem of Hesiod's \emph{Theogony}.
\emph{Mnemosyne} v. 25, 1972, p. 225--60.

\tit{VERGADOS}, A. Stitching narratives: unity and episod in Hesiod. In:
\textsc{werner}, C.; \textsc{DOURADO"-LOPES}, A.; \textsc{werner}, E. (org.) \emph{Tecendo
narrativas}: unidade e episódio na literatura grega antiga. São Paulo:
Humanitas, 2015, p. 29--54.

\titidem. \emph{Hesiod's verbal craft}: studies in Hesiod's conception of
language and its ancient reception. Oxford: Oxford University Press,
2020.

\tit{VERDENIUS}, W. J. Works and Days by M. L. West. \emph{Mnemosyne} v. 33,
1980, p. 377--89.

\titidem. \emph{A commentary on Hesiod}: Works and Days vv.1--382. Leiden:
Brill, 1985.

\tit{VERNANT}, J.-P. \emph{Mito e sociedade na Grécia antiga}. Rio de Janeiro:
José Olympio, 1992.

\titidem. \emph{Mito e pensamento entre os gregos}. Rio de Janeiro: Paz e
Terra, 2002.

\tit{VERSNEL}, H. S. (2011) \emph{Coping with the gods: wayward readings in
Greek theology}. Leiden: Brill

\tit{VOLK}, K. \emph{The poetics of Latin didactic}. Oxford: Oxford University
Press, 2002.

\tit{WATKINS}, C. ἀνόστεος ὃν πόδα τένδει. \emph{Étrennes de septantaine:
mélanges Michel Lejeune}. Paris: Klincksieck, 1978.

\titidem. \emph{How to kill a dragon: aspects of Indo"-European poetics}. New
York: Oxford University Press, 1995.

\tit{WERNER}, C. A fábula do falcão e do rouxinol e a épica heroica em
`Trabalhos e dias' de Hesíodo. \emph{Philia \& Filia} v. 3, n. 2, 2012,
p. 98--118.

\titidem. Futuro e passado da linhagem de ferro em \emph{Trabalhos e dias}: o
caso da guerra justa. \emph{Classica} v. 27, n. 1, 2014.

\titidem. Poeticidade em proêmios hexamétricos: \emph{Trabalhos e dias} e
\emph{Odisseia}. \emph{Organon} v. 31, n. 60, 2016, p. 31--45.

\tit{WEST}, M. L. \emph{Hesiod Works \& days:} edited with prolegomena and
commentary. Oxford: Oxford University Press, 1978.

\titidem. \emph{The east face of Helicon: West Asiatic elements in Greek poetry
and myth}. Oxford: Oxford University Press, 1997.

\tit{WOODWARD}, R. D. Hesiod and Greek myth. In: ―. (org.) \emph{The Cambridge
companion to Greek mythology}. Cambridge: Cambridge University Press,
2007.

\tit{ZANON}, C. A. \emph{Onde vivem os monstros: criaturas prodigiosas na
poesia de Homero e Hesíodo}. São Paulo: Humanitas, 2018.
\end{bibliohedra}

