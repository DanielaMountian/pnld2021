\chapter{Introdução}

\epigraph{Riobaldo, a colheita é comum,
mas o capinar é sozinho\ldots{}}{\textsc{j. guimarães rosa}, \textit{Grande sertão: veredas}}


Essa frase do romance de Rosa é atribuída pelo narrador Riobaldo a seu
compadre Quelemén. Ela soa como um provérbio: fácil de entender e
memorizar, com duas orações que se espelham sintaticamente e são unidas
pela tripla repetição do fonema /k/ em início de palavra. Quelemém, pelo
seu caráter (``homem de mansa lei, coração tão branco e grosso de
bom,'') e certa sapiência, especialmente em assuntos religiosos e
morais, é admirado por Riobaldo. Para este, porém, o mundo é mais
complexo que as regras claras propostas pelo discurso religioso daquela,
de sorte que Riobaldo percebe os limites daquilo que o amigo tem para
lhe ensinar: ``Compadre meu Quelemém nunca fala vazio, não subtrata. Só
que isto a ele não vou expor. A gente nunca deve de declarar que aceita
inteiro o alheio -- essa é que é a regra do rei!''.

\emph{Trabalhos e dias} também encena uma relação de aprendizagem, mas o
narrador do poema, diferente de Riobaldo, personagem marcado mais pelas
dúvidas que certezas, é alguém que entende tudo por si mesmo e dessa
forma pode dar conselhos aos outros (293-95). Nesse sentido, ele pode
ser comparado a Quelemém, incluindo-se o uso que faz de imagens
didáticas e frases que soam como provérbios para transmitir seu
conhecimento. Entretanto, por mais que a moralidade de sua visão de
mundo, como a de Quelemém, pareça ser transparente -- há os juntos e os
injustos, e estes, em algum momento, serão punidos --, Hesíodo, assim
como Riobaldo (ou Rosa), também arma enigmas que exigem a colaboração do
leitor para entregarem uma resposta, sendo que esta não elimina certa
desconfiança geral em relação à capacidade da linguagem de transmitir a
verdade (Pucci 1977, Vergados 2018).

\section{O poema e sua voz autoral}

\emph{Trabalhos e dias} é o poema grego no qual se mencionam Pandora e
seu pito (``caixa'' é uma invenção renascentista), as linhagens, raças
ou idades do homem e uma poética representação das estações do ano
(três) e das atividades agrícolas a cada uma delas associadas. Além de
trechos que definiríamos coo mitológicos e de uma figuração
ético-poética da vida agrícola, tópicos morais, políticos e religiosos
compõe esse poema que, somente até certo, utiliza a mesma linguagem das
narrativas épicas de Homero. Nele, porém, não é de quase-super-homens
como Aquiles e Odisseu que se fala, mas de outros tipos de heróis: o
poeta que de tudo sabe; o bom rei, que zela para que a justiça se faça
presente em sua comunidade; e o agricultor bem-sucedido, que, para
produzir riqueza por meio de sua propriedade ou ``fazenda''
(\emph{oikos}), deve não só trabalhar arduamente, mas atentar a uma
série enorme de regras climáticas, morais e religiosas, sendo que aquilo
que nós chamamos de acaso também espreita.

Além de \emph{Trabalhos e dias}, somente chegaram inteiros até nós os
poemas \emph{Teogonia} e \emph{Escudo de Héracles} entre aqueles
atribuídos na Antiguidade ao grego Hesíodo, poeta que teria vivido entre
os séculos VIII-VII a.C., ou seja, mais ou menos na mesma época que
Homero, tido pelos antigos como o autor da \emph{Ilíada} e da
\emph{Odisseia}. Todavia, inúmeros aspectos relacionados à cultura grega
coeva (por exemplo, a introdução e expansão do uso da escrita) que podem
ser reconstruídos hoje com uma margem de erro que não deixa de ser
razoavelmente grande fazem muitos pesquisadores duvidar que tenha havido
um poeta histórico chamado Hesíodo e que ele tenha composto por escrito
os poemas associados a seu nome. Para analisarmos as condições que
propiciaram poemas como os citados, a arqueologia e a história do
Mediterrâneo Oriental em geral e da Grécia em particular são uma
importante ajuda, mas, para entender um poema como \emph{Trabalhos e
dias}, o próprio texto ainda é nossa principal ferramenta, de sorte que
muitas das questões a ele pertinentes precisarão continuar sem uma
resposta categórica, por exemplo, quando, por quem e para que ele teria
sido composto. Isso não nos impede de procedermos a uma investigação
talvez ainda mais interessante, qual seja, o funcionamento do próprio
texto.

A voz autoral que enuncia os \emph{Trabalhos} certamente fala de si, ou
seja, de um poeta ao qual ela está associada. Por uma questão de
economia, ao longo desta introdução, usarei ``Hesíodo'' como sinônimo
dessa voz, embora este nome não seja, ele mesmo, mencionado nenhuma vez
ao longo do poema. Como destacaram Ford (1997), Scodel (2011) e outros,
o que particularmente distingue essa voz, diferenciando-a daquela que
percorre os poemas homéricos, é ela se dirigir diversas vezes ao longo
do poema a um interlocutor específico com o qual o poeta mantém uma
relação estreita, seu irmão Perses. Estamos longe do público impessoal
pressuposto por um poema épico como a \emph{Ilíada} mas bastante
próximos de certos tipos de poemas que podemos chamar, de forma bem
genérica, de líricos, os quais, no universo grego arcaico, podiam ter um
ou mais destinatários explícitos (indivíduos e/ou a população de uma
cidade).

Para muitos estudiosos, porém, faz mais sentido se pensar que a
representação daquilo que parece dizer respeito à vida dessa voz autoral
-- e o mesmo vale para a \emph{Teogonia} -- seja intrínseco à tradição
poética da qual esse poema é dependente e não um recorte da biografia de
seu autor, por mais que a escassez de material transmitido torne difícil
reconstruir-se a tradição. Nesse sentido, Hesíodo seria um mito (Nagy
1990). Os eventos que perfazem certo pano de fundo do poema (a briga
entre Hesíodo e Perses) não são, porém, uma ``mentira'' ou uma
``ficção'', mas, como defende Gregory Nagy, uma componente que contribui
para a autoridade do discurso como um todo, provido, portanto, de
eficácia no âmbito das sociedades onde ele foi apresentado e pelas quais
foi assimilado até se tornar parte de um cânone. Dessa forma, Hesíodo e
Perses não seriam nem realidades históricas nem ficções, mas elementos
de uma tradição mitopoética em uma sociedade tradicional, na qual o
conhecimento dependia de formas específicas, ligadas à oralidade, para
ser apresentado e transmitido de geração em geração. Em última
instância, e parafraseando Nagy, dificilmente saberemos um dia, com
segurança, se foi um Hesíodo histórico que deu vida aos poemas (todos?
quais?) que dele falam ou se são poemas e tradições poéticas que deram
vida à figura que passou a ser conhecida como Hesíodo.

\section{A unidade do poema}

De forma alguma o problema ligado à voz autoral e a tradição na qual se
inscreve o poema deve ser subestimado ou, o que seria ainda pior, tomado
como impedimento para uma interpretação do poema; muito pelo contrário:
ele é um ótimo ponto de partida! Para o leitor moderno, a primeira
impressão ao ler o poema pode ser a de que se está diante de uma obra
carente de qualquer tipo de unidade, uma colcha de retalhos composta ao
longo de muito tempo no âmbito de uma tradição. O desafio ao qual somos
direcionados pelos nossos hábitos de leitura é procurar entender se há
uma unidade por trás de partes nitidamente distintas, algum núcleo
formal e/ou temático que dê conta do todo que é um poema.

Como primeira baliza na busca de algum princípio unificador, podemos
invocar o modo como na primeira metade do século XX d. C. muitos
eruditos entenderam a literatura grega arcaica, ou seja, aquela composta
antes do século V a. C.: na precisa síntese de Versnel (2011, p. 214),
ela foi descrita como ``marcada por uma dicção paratática, `aditiva' ou
`aglutinadora', que contém tais qualidades como abundância
(\emph{poikilia}), autonomia e predominância de partes separadas, a
qualidade funcional delas, a ligação entre partes disparatadas e não
raramente contraditórias ou incompatíveis e a carência (aparente?) de um
conceito ou tema central unificador ou ligante''. Certamente podemos
concordar que essa descrição estilística se encaixa no poema em questão.
Ressalte-se, porém, a discreta interrogação acerca do princípio
unificador, que não é totalmente deixado de lado.

Outra observação preliminar é que não estamos diante de uma narrativa: a
querela entre os dois irmãos da qual fala o poema (versos 27-39) não é
seu fio condutor do mesmo modo que o retorno de Odisseu o é na
\emph{Odisseia}, pois o poema está bem longe de desenvolver uma sucessão
de eventos no tempo; aliás, ele praticamente nada diz dessa briga. Jenny
Clay (2003, p. 34) foi feliz ao adotar o termo ``monólogo dramático''
para dar conta do tipo de interação comunicativa que está em jogo: de um
lado, o sujeito performador que assimila e apresenta a voz autoral; de
outro, seu público, ou seja, nós. Nessa interação, a briga entre o poeta
e seu irmão certamente fornece um enquadramento logo no início do poema,
mas esse evento é minimamente abordado de forma direta, embora seja
incorporado e aflore, embora nem sempre explicitamente, no fluxo
contínuo composto gêneros discursivos diversos, como mito, fábula e
alegoria (Canevaro 2014). A briga, portanto, é um dos princípios
unificadores, mas é bastante tênue como elemento de uma construção
lógica que permite ao receptor interligar trechos que antes parecem
discretos. Mesmo que considerássemos que o público primeiro do poema
conheceria detalhes de uma suposta briga real entre duas figuras
históricas, isso provavelmente não valeria para a recepção posterior do
poema, que logo e durante muito tempo foi popular.

Para entender esse fenômemo, Canevaro (2015) propõe que Hesíodo tenha
utilizado duas estratégias principais na construção de seu poema. Por um
lado, ele constrói passagens, que podem ser de diferentes tamanhos
(desde um verso até um mito), facilmente destacáveis do poema, por
exemplo, frases que soam como provérbios e podem ser utilizadas pelo
receptor do poema como tais (287-92):

\begin{verse}
Miséria (\emph{kakotēs}) é possível, aos montes, agarrar\\
facilmente: é plano o caminho, e mora bem perto.\\
Mas diante de Excelência (\emph{aretē}) suor puseram os deuses\\
imortais: longa e íngreme é a via até ela,\\
e áspera no início; quando se chega ao topo,\\
fácil então ela é, embora sendo difícil.
\end{verse}

Quem ouve este trecho ou partes dele, não precisa conhecer o poema para
lhe conferir um sentido. Na verdade, mais de um, pois o vocabulário
usado é assaz indefinido, entre moral e socio-econômico. Por outro lado,
muitas dessas mesmas passagens costumam ter o que Canevaro chama de
``selo hesiódico'', ou seja, particularidades que as vinculam, nos
processos de recepção, a um poema específico e à figura autoral a ele
associado. Acima, a personificação dos dois substantivos principais,
Miséria e Excelência, faz mais sentido levando-se em conta a passagem
como um todo, bem como a ligação deles com a agenda econômica
desenvolvida no poema.

\section{Proêmio: Zeus e o poeta}

Como em outras composição gregas compostas na época deste poema e que
também usam sua unidade métrica, o hexâmetro datílico, os
\emph{Trabalhos} iniciam com um proêmio (1-10). Esse trecho é uma das
façanhas poéticas mais notáveis que conhecemos deste período (caso não
tenha sido uma adição bem posterior, como já defendido na Antiguidade).
Uma sucessão de figuras de linguagem comprimidas em uma sequência sonora
quase encantatória, abundante em rimas, externas e internas, e outros
tipos de repetição (Watkins 1995, Macedo 2010, Werner 2016), tornam o
trecho virtualmente intraduzível. A exibição do domínio desses recursos
poéticos reforça a ação de Hesíodo, a demarcação de sua voz e de sua
autoridade, em suma, o seu discurso, em relação a duas outras
autoridades geralmente presentes na poesia hexamétrica, as Musas e Zeus.

Assim, a densidade da linguagem pode ser o ponto de partida para uma
interpretação do trecho, sobretudo se aceitarmos, como defendido por
Watkins (1995), que a primeira palavra do proêmio (\emph{Mousai},
``Musas'') como que está inscrita em sua última
(\emph{\emph{muthēsaimēn}}, ``quero por em um discurso''). Tal façanha
linguística nos fornece um elemento fundamental para pensar o proêmio
como um todo. Esses dois termos são altamente marcados, o que se
evidencia a partir do seguinte trecho do outro poema atribuído a Hesíodo
(\emph{Teogonia}, 24-28):

\begin{verse}
Este discurso (\emph{muthos}), primeiríssimo ato, dirigiram-me as deusas,\\
as Musas do Olimpo, filhas de Zeus porta-égide:\\
``Pastores rústicos, infâmias vis, ventres somente,\\
sabemos falar muito fato enganoso como genuíno,\\
e sabemos, quando queremos, proclamar verdades''.
\end{verse}

\emph{Mousai} diz respeito à instância metafísica sobre a qual se apoia
o cantor oral grego para justificar a veracidade, a autoridade do seu
canto, mas a forma verbal \emph{muthēsaimēn} (o verbo é cognato de
\emph{muthos}) refere-se a um discurso cuidadosamente elaborado por meio
do qual o falante quer fazer valer a sua autoridade, geralmente em um
ambiente público (Martin 1989).

Entre a autoridade das Musas e a sua própria, o poeta assume um discurso
que ele aproxima ao máximo da atuação de Zeus no mundo dos homens e, com
isso, no mínimo parcialmente, se afasta das Musas, aquelas encarregadas
de preservar as façanhas dos heróis épicos, guerreiros do passado (1:
``que com cantos glorificam''). Os versos 3 a 8 por certo celebram Zeus,
mas quase que exclusivamente por meio do modo como ele age no mundo dos
homens, culminando em sua administração da justiça (7 e 9). A
contiguidade, no grego, entre os dois pronomes que abrem o verso 10,
``tu'' (Zeus) e ``eu'' (poeta), garantem ao receptor que a eficácia do
discurso deriva do conhecimento de Hesíodo e apoia-se na atuação de
Zeus, a quem o poeta pede ajuda em um momento de tensão, marcado pelo
imperativo ``atende-me'' (9). Ignorar as Musas seria impensável para um
poeta do período; mas aqui elas parecem ser deslocadas para uma posição
marginal, e sua menção, do modo como é feita, talvez também sirva para
reforçar que neste poema não serão glorificados os heróis do passado.

\section{O início do poema: Brigas e trabalho}

Os versos que se seguem ao proêmio (11-26) desenvolvem um tema
tradicional, a oposição entre as atividades ligadas à guerra e aquelas
ligadas à paz, tema que aparece neste trecho da \emph{Odisseia} de
Homero (\emph{Od.} 14.216-28):

\begin{verse}
Por certo audácia me deram Ares e Atena,\\
e força rompe-batalhão. Quando escolhia para tocaia\\
varões excelentes, engendrando males a inimigos,\\
nunca o ânimo orgulhoso pressentia minha morte,\\
ao contrário: após bem na frente saltar, com lança matava\\
quem, dentre os varões inimigos, recuasse com os pés.\\
Esse eu era na guerra (\emph{polemos}); mas o trabalho (\emph{ergon})\\
não me era caro,\\
tampouco o senso doméstico que cria radiantes crianças;\\
sempre me foram caras naus com remos,\\
guerras, dardos bem-polidos e flechas --\\
coisas funestas, que para os outros horripilantes são.\\
Mas isso era-me caro, o que o deus pôs no juízo;\\
cada varão se deleita em trabalhos (\emph{erga}) distintos.
\end{verse}

Na passagem hesiódica, todo o valor positivo é deslocado para o trabalho
agrícola; para quem persegue a riqueza desse modo, conflitos parecem
ausentes. Com efeito, há duas deusas \emph{Eris}, Briga; elas são
gêmeas, no que este poema corrige uma genealogia informada na
\emph{Teogonia}. A Briga ruim é a da guerra, aquela que preside os
combates na \emph{Ilíada}. A boa é aquela que faz um agricultor superar
a produção de seu vizinho.

O modelo de vida oferecido pelos \emph{Trabalhos} não permite empate;
ele se pretende o melhor, e seu herói é o agricultor que domina uma
pletora de diferentes atividades ao longo do ano para superar seu
vizinho. Hesíodo menciona poetas que competem com poetas (26) para
sugerir que seu poema também tem essa pretensão, a de mostrar que a vida
de seus heróis é mais digna de ser admirada e emulada que aquela dos
heróis homéricos?

Todo aquele que se envolve com atividades infrutíferas, estéreis,
contribui para o fortalecimento da Briga censurável. O antídoto é um só:
trabalho, que, no poema, é essencialmente o trabalho agrícola, e,
secundariamente, com reservas, a navegação. Não estamos diante de algo
nem mesmo próximo de uma ética valorizadora do trabalho (e da riqueza)
em si. Uma interpretação possível do enigmático verso 41 (``quão grande
valia há na malva e no asfódelo'') é que os reis que talvez aceitem ser
corrompidos por Perses em sua disputa com Hesíodo (38-39) não percebem o
valor de uma vida que se satisfaz com o mínimo (a malva e o asfódelo
compõem uma alimentação assaz deficiente, suficiente, talvez, apenas
para um asceta). Quanto ao trabalho, ele é um mal necessário (42-44,
90-92), mas, ao mesmo tempo, aliado da justiça. Quando os homens se
dedicam a atividades cujo ganho se dá sem trabalho (o ganho do parasita
e do corrupto) toda a sociedade perde, inclusive os reis que deveriam
zelar pelo seu bom funcionamento.\footnote{``Reis'' traduz o plural do
  substantivo grego \emph{basileus}; não se trata de um rei no sentido
  estrito, mas de um nobre cujo poder político em uma comunidade o torna
  uma figura central na administração da justiça.} O poema é, em boa
parte, um discurso persuasivo; o irmão de Hesíodo deve ser movido a
adotar um modo de vida mais benéfico para si e para a sociedade da qual
ela participa.

Um dos recursos retóricos que Hesíodo manobra ao longo do poema são os
enunciados mais ou menos enigmáticos. Um exemplo é o verso mencionado no
último parágrafo: ele deve ser entendido ironicamente? Literalmente,
talvez propondo um saber do conhecimento de poucos, ou seja, paradoxal
para a maioria? Enigmaticamente? Outras expressões, por não remeterem
diretamente ao referente e pelo seu provável grau de estranheza
(``cinco-galhos'' por mão, por exemplo, no verso 742), pedem para ser
decifrados pelos ouvintes. É possível que a tradição ao qual pertencia o
poema já tivesse condicionado aqueles com ela familiarizados a ouvirem e
interpretarem tais enunciados. No trecho inicial do poema, por exemplo,
o ouvinte é convidado a estranhar a existência de duas famílias de
Brigas. Ademais, nem sempre é claro quando certos tipos de construção
têm a finalidade de ser principalmente bem-humorados.

\section{Mitos: Prometeu, Pandora e a as linhages do homem}

A figura do irmão, assim, serve sobretudo de intermediário para Hesíodo
fazer o seu público decifrar seu poema e assimilar seu raciocínio. Se
Perses não escutar o irmão, ele cometerá um erro tão tolo quanto aquele
de Epimeteu, o irmão de Prometeu (47-105). A história do roubo do fogo
pelo Titã e a vingança de Zeus, que criou a primeira mulher, Pandora,
também é mencionada na \emph{Teogonia}, mas em \emph{Trabalhos} ela é
reconfigurada tendo em vista o discurso que Hesíodo dirige ao irmão.

Tanto a história de Prometeu e Pandora quanto o mito das linhagems do
homem (106-201) giram em torno da separação entre homens e deuses, mais
especificamente, da necessidade de os homens trabalharem para obter seu
sustento. Na primeira história, Zeus e os demais deuses não surgem como
figuras simpáticas aos mortais, mas, como notou Martin (2004), é o
desnível de conhecimento no âmbito de duas duplas -- Zeus/Prometeu e
Prometeu/Epimeteu -- que encaminha a história, no que se reproduz a
moldura discursiva do poema, a sapiência de Hesíodo sobrepondo-se à
tolice de Perses. Vale dizer que Perses não é considerado um caso
perdido, não é desprezado por Hesíodo. Se assim o fosse, não haveria
razão para o discurso existir. Hesíodo por certo se apresenta como
sumamente sábio, mas o problema de seu irmão, assim como o de Epimeteu e
o dos heróis homéricos que costumam ser chamados de \emph{nēpios}
(``tolo''), é não se darem conta, na hora correta, das consequências
danosas futuras de suas ações.

Voltando à história de Prometeu, em um primeiro nível, o fogo escondido
por Zeus e trazido de volta pelo Titã aponta para as conquistas técnicas
que tornam a agricultura uma atividade mais simples e produtiva. O fogo,
porém, também é análogo à destruição consumidora consubstanciada na
primeira mulher. Tecnicamente impecável e muito sedutora, a mulher é um
mal que se opõe ao homem trabalhador. Nesse sentido, ela está do lado de
Perses e dos reis que aceitam serem subornados. Seu nome, porém, compõe
um enigma: ``Pan-dora'' pode ser interpretado como, ``todos os
presentes'', ou seja, as dádivas que cada deus confere à criatura
moldada por Hefesto ou ``presentes de todos (os deuses)''. Com efeito,
tais presentes são enganadores, eles não trazem bem algum para os homens
e, assim como os presentes que ganham os reis ``come-presente'' (39) e a
parte do patrimônio que Perses tentou (e/ou tenta) conquistar
injustamente, eles têm um efeito danoso no futuro. Os males futuros,
porém, não se anunciam mais aos homens; Zeus lhes tirou a voz, o
contrário do que determinou fosse feito com Pandora. Entre os homens, o
que resta é dar ouvidos ao sábio que conhece a origem desses males
desconhecidos. Essa é uma das faces da ``esperança'' (\emph{elpis}),
essa entidade naturalmente ambivalente que também dá conta da
expectativa depositada pelos reis e Perses no lucro que podem ganhar sem
trabalho. Para quem percebe a falácia dessa ``expectativa'' (outra
tradução de \emph{elpis}), resta enunciar e defender que Zeus
``endireita o torto'' (7).

Se o foco da história de Prometeu é a dificuldade da vida humana, em
especial, a necessidade do trabalho como garantia do sustento, uma
diferença essencial entre a vida humana e a divina, na história seguinte
acerca das linhagens ou espécies do homem, a relação principal não é
mais vertical, mas horizontal, os modos diversos dos homens agirem
reciprocamente: ou bem a justiça (\emph{dikē}) ou bem a violência
(\emph{hubris}) tem primazia numa determinada linhagem. Esse, pelo
menos, é um dos eixos da influente interpretação de Vernant (2002), para
quem estas categorias dão sentido ao mito.

A polarização mais clara que perpassa a história é, porém, aquela mesma
que separa os irmãos Hesíodo e Perses, o conflito (nos seus modos
diversos) e sua ausência. A linhagem presente (de ferro) tem diante de
si uma dupla alternativa: a auto-destruição, o que ocorreu com todas as
outras, salvo a de ouro, ou uma situação que a poria muito perto da
linhagem de ouro (225-47), a saber, uma cidade onde o máximo de justiça
garantirá o máximo de prosperidade para todo aquele que quiser trabalhar
sua terra. A tese do poema, portanto, torna-se clara: o trabalho é um
mal; entretanto e paradoxalmente, se todos se dedicarem a ele, a
humanidade poderá se reaproximar daquilo que perdeu para sempre, uma
vida sem trabalho e outras aflições. Nesse tipo de sociedade, o espaço
para injustiças geradoras de conflitos é mínimo. É uma tolice da parte
daquele que se acredita o mais forte em dada situação buscar vantagens
baseadas em sua força: uma série de mecanismos atrás dos quais estão
Zeus e Justiça fará com que o usurpador seja punido. Ou não?

\section{Justiça e os desígnios de Zeus}

``Justiça'' traduz o grego \emph{dikē}, termo cuja polissemia é
explorada por Hesíodo ao longo do poema. Na tradução do termo e seus
cognatos, utilizou-se sobretudo ``juízo'' e ``justiça'', aquele, para
abarcar procedimentos legais, em particular, julgamentos e vereditos,
que podem ser qualificados como ``retos'' ou ``tortos''; este, para
punicação ou compensação, ordem cósmica, como personificação ou como um
ideal ou princípio (Vergados 2020, p. 169-75).

O proêmio do poema e outros trechos, sobretudo aqueles que indicam que
não basta ao homem trabalhar sua terra seguindo todas as regras, pois o
clima depende, em última instância, dos desígnios de Zeus, indicam que o
poder do deus é por vezes arbitrário. Esse talvez seja o mistério -- e
fascínio -- último do poema, entender como pode ser Zeus arbitrário e
justo. Por um lado, é responsável a ignorância humana: ``Sempre
cambiante é a mente de Zeus porta-égide,/ e para os homens mortais é
difícil entendê-la'' (483-84). Por outro lado, alguns homens conseguem
chegar muito perto da verdade; um deles é Hesíodo, e todo aquele que por
ele for persuadido seguirá seus conselhos, mesmo que seu sucesso não
seja cem por cento garantido (293-97):

\begin{verse}
Este o melhor de todos, quem por si tudo entender\\
ao refletir no que será melhor, depois e no fim;\\
nobre também o que atende a quem dá bons conselhos;\\
mas quem por si não entender nem, de outro ouvindo,\\
lançar no ânimo, esse é um varão infrutífero.
\end{verse}

Esta passagem ajuda a explicar por que Hesíodo não enuncia claramente a
moral da estória (\emph{ainos}) do falcão e do rouxinol (202-12): por um
lado, a estória aponta para o sucesso do mais forte, algo que todo o
homem, ao longo de sua vida, tende a testemunhar. Se atentarmos para o
fato de que o rouxinol vale como um poeta e que a voz de Hesíodo se faz
notar explicitamente no verso 213 ao se dirigir a Perses, então os
versos que seguem à estória, nos quais se sobressai a oposição entre a
Cidade Justa e a Cidade Injusta, compõem, na verdade, a resposta,
alongada, dada pelo rouxinol-poeta ao falcão. Não vale uma resposta
presa a uma só dimensão temporal, assim como só entendemos como funciona
uma das linhagems do homem se a colocarmos em oposição às outras; nem
mesmo o fim da linhagem de ferro garantirá a vitória inequívoca de um ou
outro dos animais da fábula. Para Hesíodo, porém, \emph{no contexto da
idade de ferro}, a resposta indireta do rouxinol é a boa. A Justiça, uma
filha de Zeus (256-57), é um presente do deus (279-80) e, como tal, a
contrapartida de Pandora. A diferença entre as duas é que Pandora e o
que ela representa irremediavelmente estão entre os homens; Justiça só
fica entre eles se os homens a cultivarem.

\section{Trabalhos, estações e os dias}

A partir do verso 286, Hesíodo paulatinamente abandona o tema da
justiça, que é necessária, coletivamente, para o bem de todos, e passa a
abordar, de forma mais específica, o tema do trabalho, que é, em
primeiro lugar, uma ocupação individual. Nessa dobra do texto, a figura
negativa deixa de ser a criatura violenta, que só confia na própria
força, e passa a ser o inativo e/ou mendigo; Perses passa a ser pensado
no intervalo entre duas figuras, aquele que quer tomar a riqueza alheia
e o inativo faminto com vergonha de trabalhar (298-319). O trabalho
opõe-se à inação e ao roubo, uma das ações moralmente vis para a qual
Zeus garante compensação (320-34) e que, mais adiante, também merecerá
um trocadilho: o homem ``sono-diurno'' é o ladrão (605). Ao fazer o
elogio do trabalho em oposição à inação, o pessimismo fica de lado
quanto à ordem garantida por Zeus: para uma sociedade na qual o trabalho
é a opção da maioria, Zeus está presente e não vale o que enuncia o
falcão ao rouxinol.

O trecho que inicia no verso 286 e vai até o verso 380 e os versos
694-764 apresentam, na forma de catálogos, uma série de sentenças
morais, preceitos e conselhos, típicos da literatura sapiencial, os
quais flanqueiam o calendário no qual se desdobram as atividades
próprias da agricultura e da navegação. O ``tu'' ao qual se dirige o
poeta não é mais necessariamente apenas Perses, mas, em muitas
passagens, é um ``tu'' genérico (Schmidt 1986). Não é difícil entender a
função do trecho que vai de 286-380, pois, são enumerados uma série de
preceitos que, se forem seguidos, permitirão ao homem uma vida melhor no
que diz respeito aos deuses, sua sociedade e sua propriedade. Entre a
justiça, uma conquista coletiva, e a riqueza, pessoal, há uma série de
ações que o homem deve realizar para o seu sucesso material e moral
(289-92).

Erro comum na apreciação do poema é achar que, ao iniciar o que amiúde
se denomina ``almanaque agrícola'' (381-617), um elenco de atividades
agrícolas associadas a certos fenômenos celestes e climáticos, o
objetivo do poeta passasse a ser compor um manual para um agricultor
inexperiente. Obviamente, esse não devia ser o uso do poema, assim como
não precisamos pensar em Hesíodo como um poeta-agricultor, alguém que de
dia geria sua propriedade e à noite ou nas horas vagas -- por exemplo,
durante o calor do verão, ouvindo as cigarras -- pensava nos seus
poemas. Mas então para que serve essa passagem propriamente agrícola?

Em primeiro lugar, tanto nela quanto no curioso trecho que é o último do
poema (os ``dias''), há temas que reaparecem alhures (Lardinois 1998).
Segundo Nelson (1996, p. 53), por exemplo, essa passagem não ensina
``como ser um agricultor, mas o que o ciclo do ano, com seu equilíbrio
entre bem e mal, lucro e risco, ansiedade e descanso, implica acerca da
vontade de Zeus'' De fato, Zeus continua tão presente quando antes; sua
atuação garante uma ordem sem a qual a agricultura seria impossível
(414-16). Se os trabalhos são sazonais, as mudanças entre as estações
(565) devem-se a Zeus, que, não por acaso, é seu pai (\emph{Teogonia}
901-3): ``A segunda{[}esposa{]}, {[}Zeus{]} fez conduzir a luzidia
Norma, mãe das Estações,/ Decência, Justiça e a luxuriante Paz,/ elas
que zelam pelos trabalhos (\emph{erga}) dos homens mortais''.

A fartura parece advir da vontade de Zeus (465), mas esta está sujeita a
mudanças bruscas (483), imprevisíveis (488). A luta do homem, porém, não
é contra a vontade de Zeus, mas a partir dela. Por isso a descrição dos
fenômenos climáticos é tão importante: o bom agricultor é aquele que
nunca deixa passar o momento correto de realizar certas atividades
precisas. Além disso, cada estação tem particularidades que ultrapassam
a mera contextualização dos trabalhos do agricultor e que se relacionam
com outros trechos do poema; exemplifiquemos isso através do trecho
sobre o inverno (493-563).

Em primeiro lugar, observe-se que o trecho mais longo dedicado a uma
estação trata da parte do ano durante a qual menos se trabalha. Trata-se
apenas da exibição, por parte do poeta, de seus recursos descritivos? É
clara a oposição com o verão, já que são essas as duas estações nas
quais prepondera a ausência do trabalho, sendo que numa, por outro lado,
o sofrimento é máximo, ao passo que na outra abundam situações
prazerosas (Manakidou 2006).

A linguagem enigmática também não está ausente dessa parte do poema. Se
aceitarmos a leitura de Bagordo (2009) e outros, que, porém, não
convence a todos, então parte do episódio invernal é marcada por
metáforas sexuais, provavelmente ecoando a ausência de fertilidade da
estação, a começar pelo vento Bóreas, que, com a sua potência, consegue
penetrar em quase todos os espaços. Reitera-se o frio onipresente
através de um catálogo de espaços e criaturas que a ele se submetem (a
infertilidade da estação por meio do seu contrário), e tanto mais
surpreende o vento, um notório raptor de virgens moças, não conseguir
atingir uma jovem que parece ignorar a dureza do inverno por ter uma
série de luxos à sua disposição (519-23). Figuras femininas, sobretudo
mortais, não costumam ser retratadas de forma positiva no poema, e não é
por acaso que se menciona que a moça é inexperiente sexualmente,
sobretudo se aceitarmos que a expressão ``sem-osso'' alude ao pênis. As
mudanças de estação não exigem ações distintas apenas na roça, mas
também em relação às criaturas femininas, às quais as figuras masculinas
do poema encontram-se necessariamente ligadas. No episódio invernal,
quem não trabalha ou é um velho alquebrado, um indolente cujo único
prazer é a masturbação ou um pobre coitado que, devido à solidão,
precisa refrear seu impulso sexual. Quando a mulher estiver no auge da
excitação no verão, os homens, por sua vez, estarão na situação oposta.

Após concluir a parte referente à agricultura, Hesíodo também trata
brevemente da navegação, sobretudo das épocas para se viajar com menores
riscos (618-94). Nesse trecho, quase que como \emph{pendant} ao início
do poema, a família do poeta volta ao primeiro plano, não somente
através da explicitação de que Perses é irmão do poeta, mas também que
seu pai é um emigrante que adotou a navegação como forma de tentar fugir
da pobreza. O trecho mais notável, porém, diz respeito a uma disputa
poética na qual Hesíodo foi vitorioso. A passagem é repleta de ironia e
humor: Hesíodo fala de navegação sem nunca ter navegado, com exceção de
um trecho de menos de cem metros, entre o continente e uma ilha próxima.
O conhecimento que carece para compor seu canto é dado pelas Musas, e
parece ser construída uma comparação com a poesia épica heroica em
relação à qual a tradição representada por Hesíodo é vitoriosa, um feito
que a posteridade representará de diversas formas (Torrano 2005, Koning
2010).

Como acontece alhures na poesia hexamétrica arcaica, o trecho final
desse poema representa como que um anticlímax, ou melhor, ele retoma
temas ou formas já explorados, só que em uma chave menor, informando o
público ouvinte, desse modo, de que a performance poética está chegando
ao fim (Kelly 2007). Os ``dias'' referem-se a alguns dias fastos e
nefastos nos meses em geral, e sua menção reforça para o ouvinte que o
sucesso de um homem não depende apenas do seu próprio trabalho, mas de
uma série de fatores que estão além de sua vontade, entre eles, a
arbitrariedade dos deuses (823-28).

\section{Da tradução}

Para definir o texto grego aqui traduzido, cotejaram-se as seguintes
edições: West (1978), Most (2006) e Ercolani (2010). Também foram muito
úteis para se definir a opção por determinada leitura ou interpretação,
bem como para compor as notas, diversos textos citados na bibliografia,
especialmente Marg (1970), Renehan (1980), Verdenius (1980) e (1985),
Canevaro (2015) e Vergados (2020). Para a tradução, também foi
fundamental o léxico organizado por Snell \emph{et al.} (1955-2010).

Algo do que se perde na tradução -- por exemplo, algumas figuras
etimológicas -- é recuperado nas notas.

Na maioria das edições, há marcações que separam certas unidades do
poema \emph{grosso modo} equivalentes a parágrafos e/ou capítulos. Tais
marcações pressupõem um modo de composição e recepção e, quiçá, de
performance. Todavia, como se tratam de propostas dos intérpretes
modernos, foram deixadas de lado para que cada leitor procure ele mesmo
sua forma de estabelecer vínculos, unidades e cortes.

Algumas soluções que adotei nas minhas traduções de Homero (2018a) e
(2018b) nortearam certas modificações nesta edição da tradução do poema
hesiódico. Uma delas é evitar excessos no uso da ordem sintática
indireta.

A numeração das notas de rodapé em forma de lemas segue o número que
indica um verso ou um conjunto de versos do poema.

Por fim, gostaria de agradecer àqueles que compartilharam comigo seu
conhecimento de Hesíodo, em especial, de \emph{Trabalhos e dias}, desde
a 1ª edição deste volume ou me apontaram o que nele poderia ser
melhorado ou corrigido: Bruno Palavro, Juarez (Jota) Oliveira,
Lilah-Grace Canevaro, Thanassis Vergados, Jim Marks, Adrian Kelly,
Teodoro Assunção, os membros da minha banca de livre-docência (Jaa
Torrano, Zélia de Almeida Cardoso, Jacyntho L. Brandão, Pedro Paulo
Funari e Maria Beatriz Florenzano) e Antonio-Orlando Dourado Lopes.


\begin{bibliohedra}
\tit{ARRIGHETTI}, G. \emph{Esiodo opere}. Introdução, tradução e comentário.
Milano: Mondadori, 2007.

\tit{BAGORDO}, A. Zum \emph{anósteos} bei Hesiod (\emph{Erga} 524):
Griechische Zoologie, indogermanische Dichtersprache oder etwas
anderes?. \emph{Glotta} v. 85, 2009, p. 31-58.

\tit{BAKKER}, E. J. Hesiod in performance. In: LONEY, A. C.; SCULLY, S. (org.)
\emph{The Oxford Handbook of Hesiod}. Oxford: Oxford, 2018.

\tit{BEALL}, E. F. Notes on Hesiod's \emph{Works and days} 383-828.
\emph{American Journal of Philology} v. 122, 2001, p. 155-71. {[}pdf{]}

\titidem. The plow that broke the plain epic tradition: Hesiod's \emph{Works
and days} vv. 414-503. \emph{Classical Antiquity} v. 23, 2004, p. 1-31.

\tit{BLÜMER}, W. \emph{Interpretation archaischer Dichtung. Die mythologischen
Partien der Erga} \emph{Hesiods}. 2 vol. Münster: Aschendorff, 2001.

\tit{BOYS-STONES}, G. R.; HAUBOLD, J. H. \emph{Plato and Hesiod}. Oxford:
Oxford University Press, 2010.

\tit{BRADFORD WELLES}, C. Hesiod's attitude towards labor. \emph{GRBS} v. 8,
1967, p. 5-25.

\tit{CALAME}, C. Succession des âges et pragmatique poétique de la justice: le
récit hésiodique des cinq espèces humaines. \emph{Kernos} v. 17, 2004,
p. 67-102.

\tit{CANEVARO}, L. G. Genre and authority in Hesiod's \emph{Works and Days}.
In: WERNER, C.; SEBASTIANI, B.; DOURADO-LOPES, A. ; (org.) \emph{Gêneros
poéticos na Grécia Antiga: fronteiras e confluências}. São Paulo:
Humanitas, 2014, p. 23-48.

\titidem. \emph{Hesiod's Works and days}: how to teach self-sufficiency.
Oxford: Oxford University Press, 2015.

\tit{CLAY}, J. S. \emph{Hesiod's cosmos}. Cambridge: Cambridge University
Press, 2003.

\tit{CURRIE}, B. Heroes and holy men in early Greece: Hesiod's \emph{theios
aner}. In: COPPOLA, A. (org.) \emph{Eroi, eroismi, eroizzazioni dalla
Grecia antica a Padova e Venezia}. Padova: S.A.R.G.O.N., 2007, p.
162-92.

\titidem. Hesiod on human history. In: MARINCOLA, J. et al. (org.) \emph{Greek
notions of the past in the Archaic and Classical eras: history without
historians}. Edinburgh: Edinburgh University Press, 2012.

\tit{DETIENNE}, M. \emph{Os mestres da verdade na Grécia} arcaica. Trad.: A.
Daher. Rio de Janeiro: Jorge Zahar, 1988.

\tit{ERCOLANI}, A. \emph{Esiodo: Opere e giorni}. Introduzione, traduzione e
commento. Roma: Carocci, 2010.

\tit{FORD}, A. Epic as genre. In: MORRIS, I.; POWELL, B. (Org.) \emph{A new
companion to Homer}. Leiden: Brill, 1997, p. 396-414.

\tit{GAGARIN}, M. \emph{Dike} in the \emph{Works and days}. \emph{Classical
Philology} v. 68, 1973, p. 81-94.

\tit{HEATH}, M. Hesiod' didactic poetry. \emph{Classical Quarterly} v. 36,
1985, p. 245-63.

\tit{HOMERO}. \emph{Ilíada}. Tradução e ensaio introdutório: C. Werner. São
Paulo: Ubu/SESI, 2018a.

\titidem. \emph{Odisseia}. Tradução e introdução: C. Werner. Apresentação: R.
Martin. São Paulo: Ubu, 2018b.

\tit{HUNTER}, R. \emph{Hesiodic voices: studies in the ancient reception of
Hesiod's} Works and Days\emph{.} Cambridge; New York: ~Cambridge
University Press, 2014.

\tit{JUDET DE LA COMBE}, P. Le mythe hésiodique des races, oeuvre de langage:
Jean-Pierre Vernant et après. \emph{L'Homme: Revue française
d'anthropologie} 218, 2016, p. 239-52.

\tit{KELLY}, A. How to end an orally-derived epic poem? \emph{Transactions of
the American Philological Association} v. 137, 2007, p. 371-402.

\tit{KONING}, H. \emph{Hesiod}: \emph{the other poet}: ancient reception of a
cultural icon. Leiden: Brill, 2010.

\tit{LAFER}, M. C. \emph{Hesíodo: Os trabalhos e os dias. Tradução, introdução
e comentários}. São Paulo: Iluminuras, 2002.

\tit{LARDINOIS}, A. P. M. H. How the days fit the works in Hesiod's
\emph{Works and days. American Journal of Philology} v. 119, 1998, p.
319-36.

\tit{MACEDO}, J. M. \emph{A palavra ofertada}: um estudo retórico dos hinos
gregos e indianos. Campinas: Edunicamp, 2010.

\tit{MANAKIDOU}, F. P. Autour de la structure des \emph{Travaux et les Jours}
d'Hésiode: la ``dualité'' du cosmos hésiodique. \emph{Gaia} v. 10, 2006,
p. 149-67.

\tit{MARG}, W. \emph{Hesiod}: Sämtliche Gedichte. Artemis: Zürich/Stuttgart,
1970.

\tit{MARSILIO}, M. S. Hesiod's winter maiden. \emph{Helios} v. 24, 1997, p.
101-111.

\titidem. \emph{Farming and poetry in Hesiod's Works and Days}. Boston:
University Press of America, 2000.

\tit{MARTIN}, R. P. \emph{The language of heroes}: speech and performance in
the \emph{Iliad}. Ithaca: Cornell University Press, 1989.

\titidem. Hesiod's metanastic poetics. \emph{Ramus} v. 21, 1992, p. 11-33.

\titidem. Hesiod and the didactic double. \emph{Synthesis} v. 11, 2004, p.
31-54.

\tit{MORDINE}, M. J. Speaking to kings: Hesiod's \emph{ainos} and the rhetoric
of allusion in the \emph{Works and Days}. \emph{Classical Quarterly} v.
56, p. 363-73, 2006.

\tit{MOST}, G. W. \emph{Hesiod}: \emph{Theogony, Works and Days, Testimonia}.
Cambridge, MA: Harvard University Press, 2006.

\titidem. Hesiod's myth of the five (or three or four) races. \emph{Proceedings
of the Cambridge Philological Society} v. 43, 1997, p. 104-27.

\tit{MOURA}, A. R. \emph{Hesíodo:} Os trabalhos e dias. Edição, tradução,
introdução e notas. Curitiba: Segesta, 2012.

\tit{NAGY}, G. Hesiod and the poetics of Pan-Hellenism. In: ―. \emph{Greek
mythology and poetics}. Ithaca: Cornell Univesity Press, 1990, p. 36-82.

\tit{NELSON}, S. A. The drama of Hesiod's farm. \emph{Classical Philology} v.
91, 1996, p. 45-53. {[}pdf{]}

\titidem. \emph{God and the land: the metaphysics of farming in Hesiod and
Vergil}. With a translation of \emph{Works and Days} by David Greene.
New York/Oxford: Oxford University Press, 1998.

\tit{NOORDEN}, H. van. \emph{Playing Hesiod}: the `myth of the races' in
classical antiquity. Cambrdige: Cambridge University Press, 2015.

\tit{PUCCI}, P. \emph{Hesiod and the language of poetry}. Baltimore/London:
Johns Hopkins University Press, 1977.

\tit{RENEHAN}, R. Progress in Hesiod. \emph{Classical Philology} n. 75, 1980,
p. 339-58.

\tit{RODRÍGUEZ ADRADOS}, F. La composición de los poemas hesiódicos.
\emph{Emerita} v. 69, 2001, p. 197-223.

\tit{ROSEN}, R. M. Poetry and sailing in Hesiod's \emph{Works and days}.
\emph{Classical Antiquity} v. 9, 1990, p. 99-113.

\tit{ROUSSEAU}, P. Instruir Persès. Notes sur l'ouverture des \emph{Travaux}
d'Hésiode. In: BLAISE, F.; JUDET DE LA COMBE, P.; ROUSSEAU, P. (org.)
\emph{Le métier du mythe}: lectures d' Hésiode. Lille: Presses
Universitaires du Septentrion, 1996, p. 93-168.

\tit{SCHMIDT}, J.-U. \emph{Adressat und Paraineseform}: zur Intention von
Hesiods `Werken und Tagen'. Göttingen: Vandenhoeck \& Ruprecht, 1986.

\tit{SCODEL}, R. \emph{Works and days} as performance. In: MINCHIN, E. (org.)
\emph{Orality, literacy and performance in the ancient world}. Leiden:
Brill, 2011, p. 111-26.

\tit{SNELL}, B. \emph{et al.} (orgs.) \emph{Lexikon des frühgriechischen
Epos}. 4 vol. Göttingen: Vandenhoeck \& Ruprecht, 1955-2010.

\tit{STOCKING}, C. Hesiod in Paris: justice, truth, and power between past and
present. \emph{Arethusa} v. 50, n. 3, 2017, p. 385-427.

\tit{THALMANN}, W. G. \emph{Conventions of form and thought in early Greek
epic}. Baltimore/ London: Johns Hopkins University Press, 1984.

\tit{TORRANO}, J. A. A. \emph{O certame Homero-Hesíodo} (texto integral).
\emph{Letras clássicas} 9, p. 215-24, 2005.

\tit{TSAGALIS}, C. Poetry and poetics in the Hesiodic corpus. In: MONTANARI,
F.; RENGAKOS, A.; TSAGALIS, C. (org.) \emph{Brill's companion to
Hesiod}. Leiden: Brill, 2009, p. 131-78.

\tit{VERGADOS}, A. Stitching narratives: unity and episod in Hesiod. In:
WERNER, C.; DOURADO-LOPES, A.; WERNER, E. (org.) \emph{Tecendo
narrativas}: unidade e episódio na literatura grega antiga. São Paulo:
Humanitas, 2015, p. 29-54.

\titidem. \emph{Hesiod's verbal craft}: studies in Hesiod's conception of
language and its ancient reception. Oxford: Oxford University Press,
2020.

\tit{VERDENIUS}, W. J. Works and Days by M. L. West. \emph{Mnemosyne} v. 33,
1980, p. 377-89.

\titidem. \emph{A commentary on Hesiod}: Works and Days vv.1-382. Leiden:
Brill, 1985.

\tit{VERNANT}, J.-P. \emph{Mito e sociedade na Grécia antiga}. Rio de Janeiro:
José Olympio, 1992.

\titidem. \emph{Mito e pensamento entre os gregos}. Rio de Janeiro: Paz e
Terra, 2002.

\tit{VERSNEL}, H. S. (2011) \emph{Coping with the gods: wayward readings in
Greek theology}. Leiden: Brill

\tit{VOLK}, K. \emph{The poetics of Latin didactic}. Oxford: Oxford University
Press, 2002.

\tit{WATKINS}, C. ἀνόστεος ὃν πόδα τένδει. \emph{Étrennes de septantaine:
mélanges Michel Lejeune}. Paris: Klincksieck, 1978.

\titidem. \emph{How to kill a dragon: aspects of Indo-European poetics}. New
York: Oxford University Press, 1995.

\tit{WERNER}, C. A fábula do falcão e do rouxinol e a épica heroica em
`Trabalhos e dias' de Hesíodo. \emph{Philia \& Filia} v. 3, n. 2, 2012,
p. 98-118.

\titidem. Futuro e passado da linhagem de ferro em \emph{Trabalhos e dias}: o
caso da guerra justa. \emph{Classica} v. 27, n. 1, 2014.

\titidem. Poeticidade em proêmios hexamétricos: \emph{Trabalhos e dias} e
\emph{Odisseia}. \emph{Organon} v. 31, n. 60, 2016, p. 31-45.

\tit{WEST}, M. L. \emph{Hesiod Works \& days:} edited with prolegomena and
commentary. Oxford: Oxford University Press, 1978.
\end{bibliohedra}

