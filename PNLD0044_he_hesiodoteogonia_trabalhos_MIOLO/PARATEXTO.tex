\chapter{Hesíodo e a Grécia Antiga}

\begin{flushright}
\textsc{christian werner}
\end{flushright}\bigskip

\section{Sobre o autor}

\noindent{}Hesíodo foi um poeta grego arcaico e, assim como ocorre com Homero,
não é possível provar que ele tenha realmente existido. Segundo certa tradição,
porém, teria vivido por volta dos anos 750 e 650 a.C.  Supõe"-se, a partir de
passagens do poema \textit{Trabalhos e dias}, que o pai de Hesíodo tenha
nascido no litoral da Ásia e viajado até a Beócia, para instalar"-se num
vilarejo chamado Ascra, onde teria nascido o poeta; supõe"-se também que ele
tenha tido um irmão, Perses, que teria tentado se apropriar, por meios ilegais,
de uma parte maior da herança paterna do que a que lhe cabia, exigindo ainda
ajuda de Hesíodo. Acredita"-se que a única viagem que Hesíodo teria realizado
tenha sido a Cálcis, com o objetivo de participar dos jogos funerários em honra
de Anfidamas, dos quais teria sido o ganhador e recebido um tripé pelo
desempenho na competição de cantos. Apenas três das obras atribuídas a Hesíodo
resistiram ao tempo e chegaram às nossas mãos: são elas os \textit{Trabalhos e
dias}, a \textit{Teogonia} e \textit{O escudo de Héracles}.

Acredita"-se que Hesíodo tenha vivido mais ou menos na mesma época que Homero,
tido pelos antigos como o autor da \emph{Ilíada} e da
\emph{Odisseia}. Todavia, inúmeros aspectos relacionados à cultura grega
coeva (por exemplo, a introdução e expansão do uso da escrita) que podem
ser reconstruídos hoje com uma margem de erro que não deixa de ser
razoavelmente grande fazem muitos pesquisadores duvidar que tenha havido
um poeta histórico chamado Hesíodo e que ele tenha composto por escrito
os poemas associados a seu nome. Para analisarmos as condições que
propiciaram poemas como os citados, a arqueologia e a história do
Mediterrâneo Oriental em geral e da Grécia em particular são uma
importante ajuda, mas, para entender um poema como \emph{Trabalhos e
dias}, o próprio texto ainda é nossa principal ferramenta, de sorte que
muitas das questões a ele pertinentes precisarão continuar sem uma
resposta categórica, por exemplo, quando, por quem e para que ele teria
sido composto. Isso não nos impede de procedermos a uma investigação
talvez ainda mais interessante, qual seja, o funcionamento do próprio
texto.


\section{Sobre a obra}

\subsection{Os trabalhos e dias}

\emph{Trabalhos e dias} é o poema grego no qual se mencionam Pandora e
seu pito (``caixa'' é uma invenção renascentista), as linhagens, raças
ou idades do homem e uma poética representação das estações do ano
(três) e das atividades agrícolas a cada uma delas associadas. Além de
trechos que definiríamos como mitológicos e de uma figuração
ético"-poética da vida agrícola, tópicos morais, políticos e religiosos
compõe esse poema que, somente até certo ponto, utiliza a mesma linguagem das
narrativas épicas de Homero. Nele, porém, não é de quase"-super"-homens
como Aquiles e Odisseu que se fala, mas de outros tipos de heróis: o
poeta que de tudo sabe; o bom rei, que zela para que a justiça se faça
presente em sua comunidade; e o agricultor bem"-sucedido, que, para
produzir riqueza por meio de sua propriedade ou ``fazenda''
(\emph{oikos}), deve não só trabalhar arduamente, mas atentar a uma
série enorme de regras climáticas, morais e religiosas, sendo que aquilo
que nós chamamos de acaso também espreita.

Após o proêmio do poema (1-10) --- versos iniciais que demarcam o discurso de Hesíodo em relação a duas outras autoridades presentes na poesia hexamétrica, as Musas e Zeus --- os versos seguintes (11--26) desenvolvem um tema
tradicional, a oposição entre as atividades ligadas à guerra e aquelas
ligadas à paz, tema que aparece neste trecho da \emph{Odisseia} de
Homero (\emph{Od.} 14.216--28):

\begin{verse}
Por certo audácia me deram Ares e Atena,\\
e força rompe"-batalhão. Quando escolhia para tocaia\\
varões excelentes, engendrando males a inimigos,\\
nunca o ânimo orgulhoso pressentia minha morte,\\
ao contrário: após bem na frente saltar, com lança matava\\
quem, dentre os varões inimigos, recuasse com os pés.\\
Esse eu era na guerra (\emph{polemos}); mas o trabalho (\emph{ergon})\\
não me era caro,\\
tampouco o senso doméstico que cria radiantes crianças;\\
sempre me foram caras naus com remos,\\
guerras, dardos bem"-polidos e flechas --\\
coisas funestas, que para os outros horripilantes são.\\
Mas isso era"-me caro, o que o deus pôs no juízo;\\
cada varão se deleita em trabalhos (\emph{erga}) distintos.
\end{verse}

Na passagem hesiódica, todo o valor positivo é deslocado para o trabalho
agrícola; para quem persegue a riqueza desse modo, conflitos parecem
ausentes. Com efeito, há duas deusas \emph{Eris}, Briga; elas são
gêmeas, no que este poema corrige uma genealogia informada na
\emph{Teogonia}. A Briga ruim é a da guerra, aquela que preside os
combates na \emph{Ilíada}. A boa é aquela que faz um agricultor superar
a produção de seu vizinho.

O modelo de vida oferecido pelos \emph{Trabalhos} não permite empate;
ele se pretende o melhor, e seu herói é o agricultor que domina uma
pletora de diferentes atividades ao longo do ano para superar seu
vizinho. Hesíodo menciona poetas que competem com poetas (26) para
sugerir que seu poema também tem essa pretensão, a de mostrar que a vida
de seus heróis é mais digna de ser admirada e emulada que aquela dos
heróis homéricos?

Todo aquele que se envolve com atividades infrutíferas, estéreis,
contribui para o fortalecimento da Briga censurável. O antídoto é um só:
trabalho, que, no poema, é essencialmente o trabalho agrícola, e,
secundariamente, com reservas, a navegação. Não estamos diante de algo
nem mesmo próximo de uma ética valorizadora do trabalho (e da riqueza)
em si. Uma interpretação possível do enigmático verso 41 (``quão grande
valia há na malva e no asfódelo'') é que os reis que talvez aceitem ser
corrompidos por Perses em sua disputa com Hesíodo (38--39) não percebem o
valor de uma vida que se satisfaz com o mínimo (a malva e o asfódelo
compõem uma alimentação assaz deficiente, suficiente, talvez, apenas
para um asceta). Quanto ao trabalho, ele é um mal necessário (42--44,
90--92), mas, ao mesmo tempo, aliado da justiça. Quando os homens se
dedicam a atividades cujo ganho se dá sem trabalho (o ganho do parasita
e do corrupto) toda a sociedade perde, inclusive os reis que deveriam
zelar pelo seu bom funcionamento.\footnote{``Reis'' traduz o plural do
  substantivo grego \emph{basileus}; não se trata de um rei no sentido
  estrito, mas de um nobre cujo poder político em uma comunidade o torna
  uma figura central na administração da justiça.} O poema é, em boa
parte, um discurso persuasivo; o irmão de Hesíodo deve ser movido a
adotar um modo de vida mais benéfico para si e para a sociedade da qual
ela participa.

Um dos recursos retóricos que Hesíodo manobra ao longo do poema são os
enunciados mais ou menos enigmáticos. Um exemplo é o verso mencionado no
último parágrafo: ele deve ser entendido ironicamente? Literalmente,
talvez propondo um saber do conhecimento de poucos, ou seja, paradoxal
para a maioria? Enigmaticamente? Outras expressões, por não remeterem
diretamente ao referente e pelo seu provável grau de estranheza
(``cinco"-galhos'' por mão, por exemplo, no verso 742), pedem para ser
decifrados pelos ouvintes. É possível que a tradição ao qual pertencia o
poema já tivesse condicionado aqueles com ela familiarizados a ouvirem e
interpretarem tais enunciados. No trecho inicial do poema, por exemplo,
o ouvinte é convidado a estranhar a existência de duas famílias de
Brigas. Ademais, nem sempre é claro quando certos tipos de construção
têm a finalidade de ser principalmente bem"-humorados.


\subsection{Teogonia}
%\section{Estrutura do poema}

Há diferentes maneiras de conceber a estrutura da \emph{Teogonia}. A de
Thalmann (1984, p. 38--39), traduzida abaixo, tem a vantagem de
identificar em sua sequência de partes singulares e mais ou menos
independentes, uma moldura em anel (ainda que incompleta), marcada pela
repetição das letras em ordem inversa.

\begin{itemize}
\item
a. 1--115: proêmio

\item 
b. 116--210: os primeiros deuses e os Titãs; primeiro estágio do mito de sucessão

\item 
c. 211--32: prole de Noite

\item 
d-1. 233--336: prole de Mar, incluindo as Nereidas

\item 
d-2. 337--70: prole de Oceano, incluindo as Oceânides

\item 
e-1. 371--403: uniões de outros Titãs e o episódio de Estige

\item 
e-2. 404--52: uniões de outros Titãs e o episódio de Hécate

\item 
f-1. 453--506: união de Reia e Crono; segundo estágio do mito de sucessão

\item 
g. 507--616: prole de Jápeto e o episódio de Prometeu

\item 
f-2. 617--720: batalha com os Titãs (Titanomaquia) e fim do segundo estágio

\item 
c. 721--819: descrição do Tártaro

\item 
b. 820--80: batalha com Tifeu, o último inimigo de Zeus

\item 
a. 881--929 (?): Zeus torna"-se rei e divide as honras; união com Astúcia
e demais\footnote{O ponto de interrogação indica que não há consenso que
  verso nos manuscritos do poema marcaria o fim da composição (Kelly
  2007).}
\end{itemize}

A estrutura em anel, na qual se retomam léxico e temas, é uma forma
retórica assaz trivial na poesia grega. Em Homero, por exemplo, o final
de um discurso pode retomar o tópico do início, indicando ao receptor
que o discurso está chegando ao fim.

Repare"-se que proêmio do poema é longo se comparado com o início de
outras composições hexamétricas arcaicas identificado como tal. Nele,
\emph{grosso modo}, o aedo costuma estabelecer algum tipo de vínculo com
a Musa, a divindidade da qual depende a performance de seu canto, e a
definir o tema geral do poema. Isto \emph{também} é feito na
\emph{Teogonia}, mas, de um modo bastante sofisticado, o tema principal
do poema --- a autoridade as ações de Zeus --- são interligadas àquelas
das Musas e do aedo.

Com isso, o corte entre o chamado proêmio e o restante do poema é bem
menos abrupto que aquele que se verifica na \emph{Ilíada} e na
\emph{Odisseia}: no proêmio nós podemos ver Zeus sendo celebrado como
deus supremo pelas Musas, e isso, de fato, é o que faz o poema como um
todo, pois, embora Zeus não seja o \emph{primeiro} deus, do ponto de
vista da sequência do poema, é como se ele fosse, já que nenhum deus é
tão poderoso ou merece ser tão celebrado como ele.

%\section{Abismo («Khaos») e o início do cosmo}

Para chegar a Zeus e o modo como esse controla o cosmo, o tema central
do poema, Hesíodo inicia do começo, ou seja, de Abismo (116), um espaço
vazio cuja delimitação primeira surge na sequência, Terra (\emph{Gaia}).
Não se trata, porém, da Terra tal qual a conhecemos, mas de uma espaço
físico ainda descaracterizado, ou melhor, marcado pela sua função
futura, ser o espaço de atuação dos deuses responsáveis pelo equilíbrio
cósmico, que vai, imageticamente, do Olimpo ao ínfero Tártaro. Antes de
Terra começar a gerar suas formas particulares (Montanhas e Mar) e das
divindades aparecerem, duas coisas fundamentais são necessárias, a
presença de Eros (120), o desejo sem o qual não há geração, e as
potências que permitem a sucessão temporal, Escuridão, Noite, Éter e Dia
(123--25).

Todos os deuses descendem de duas linhagens principais, a de Abismo e a
de Terra, mas entre elas não há nenhuma união. Os descendentes de Abismo
são, em sua maioria, potências cuja essência é negativa (Noite, Morte,
Agonia etc.); várias delas, além disso, expressam ações e emoções que
permeiam os eventos violentos narrados na sucessão de gerações da
linhagem de Terra (Briga, Disputas, Batalhas etc.). A linhagem de
Abismo, portanto, através da descendência de Noite (\emph{Nux}) e Briga
(\emph{Eris}), revela que a separação entre Terra e Abismo nunca é total
(as ações e emoções representadas como descendência de Abismo são
executadas ou sentidas pelos descendentes de Terra), e assim ilustra uma
constante no poema: o encadeamento das linhagens entre si e também delas
com as histórias que se sucedem mostram um poema no qual os catálogos
dos deuses nascentes e as narrativas nas quais os deuses estão
envolvidos não devem ser separados. Trata"-se de uma articulação de
imagens, ações e ideias que pressupõe uma temporalidade própria --- ou
melhor, diversas temporalidades (Loney 2018) --- que revela uma mescla
entre o tempo da narrativa genealógica, o tempo da sucessão de um
deus"-rei para o seguinte e o tempo da narração. É a partir disso que o
leitor deve entender, por exemplo, que um deus às vezes já apareça como
personagem no poema antes de o narrador mencionar seu nascimento
propriamente dito.

\section{Sobre o gênero}

De forma alguma o problema ligado à voz autoral e a tradição na qual se
inscreve o poema deve ser subestimado ou, o que seria ainda pior, tomado
como impedimento para uma interpretação do poema; muito pelo contrário:
ele é um ótimo ponto de partida! Para o leitor moderno, a primeira
impressão ao ler o poema pode ser a de que se está diante de uma obra
carente de qualquer tipo de unidade, uma colcha de retalhos composta ao
longo de muito tempo no âmbito de uma tradição. O desafio ao qual somos
direcionados pelos nossos hábitos de leitura é procurar entender se há
uma unidade por trás de partes nitidamente distintas, algum núcleo
formal e/ou temático que dê conta do todo que é um poema.

Como primeira baliza na busca de algum princípio unificador, podemos
invocar o modo como na primeira metade do século \textsc{xx} d.\,C. muitos
eruditos entenderam a literatura grega arcaica, ou seja, aquela composta
antes do século \textsc{v} a.\,C.: na precisa síntese de Versnel (2011, p. 214),
ela foi descrita como ``marcada por uma dicção paratática, `aditiva' ou
`aglutinadora', que contém tais qualidades como abundância
(\emph{poikilia}), autonomia e predominância de partes separadas, a
qualidade funcional delas, a ligação entre partes disparatadas e não
raramente contraditórias ou incompatíveis e a carência (aparente?) de um
conceito ou tema central unificador ou ligante''. Certamente podemos
concordar que essa descrição estilística se encaixa nos poemas em questão.
Ressalte"-se, porém, a discreta interrogação acerca do princípio
unificador, que não é totalmente deixado de lado.

Outra observação preliminar é que não estamos diante de uma narrativa: a
querela entre os dois irmãos, no caso de \textit{Trabalhos e dias}, da qual fala o poema (versos 27--39) não é
seu fio condutor do mesmo modo que o retorno de Odisseu o é na
\emph{Odisseia}, pois o poema está bem longe de desenvolver uma sucessão
de eventos no tempo; aliás, ele praticamente nada diz dessa briga. Jenny
Clay (2003, p. 34) foi feliz ao adotar o termo ``monólogo dramático''
para dar conta do tipo de interação comunicativa que está em jogo: de um
lado, o sujeito performador que assimila e apresenta a voz autoral; de
outro, seu público, ou seja, nós. Nessa interação, a briga entre o poeta
e seu irmão certamente fornece um enquadramento logo no início do poema,
mas esse evento é minimamente abordado de forma direta, embora seja
incorporado e aflore, embora nem sempre explicitamente, no fluxo
contínuo composto gêneros discursivos diversos, como mito, fábula e
alegoria (Canevaro 2014). A briga, portanto, é um dos princípios
unificadores, mas é bastante tênue como elemento de uma construção
lógica que permite ao receptor interligar trechos que antes parecem
discretos. Mesmo que considerássemos que o público primeiro do poema
conheceria detalhes de uma suposta briga real entre duas figuras
históricas, isso provavelmente não valeria para a recepção posterior do
poema, que logo e durante muito tempo foi popular.

Para entender esse fenômemo, Canevaro (2015) propõe que Hesíodo tenha
utilizado duas estratégias principais na construção de seu poema. Por um
lado, ele constrói passagens, que podem ser de diferentes tamanhos
(desde um verso até um mito), facilmente destacáveis do poema, por
exemplo, frases que soam como provérbios e podem ser utilizadas pelo
receptor do poema como tais (287--92):

\begin{verse}
Miséria (\emph{kakotēs}) é possível, aos montes, agarrar\\
facilmente: é plano o caminho, e mora bem perto.\\
Mas diante de Excelência (\emph{aretē}) suor puseram os deuses\\
imortais: longa e íngreme é a via até ela,\\
e áspera no início; quando se chega ao topo,\\
fácil então ela é, embora sendo difícil.
\end{verse}

Quem ouve este trecho ou partes dele, não precisa conhecer o poema para
lhe conferir um sentido. Na verdade, mais de um, pois o vocabulário
usado é assaz indefinido, entre moral e socio"-econômico. Por outro lado,
muitas dessas mesmas passagens costumam ter o que Canevaro chama de
``selo hesiódico'', ou seja, particularidades que as vinculam, nos
processos de recepção, a um poema específico e à figura autoral a ele
associado. Acima, a personificação dos dois substantivos principais,
Miséria e Excelência, faz mais sentido levando"-se em conta a passagem
como um todo, bem como a ligação deles com a agenda econômica
desenvolvida no poema.