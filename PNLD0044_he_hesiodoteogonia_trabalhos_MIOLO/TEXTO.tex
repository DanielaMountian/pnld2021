\chapter*{}

\begingroup\parindent=0em

Musas\footnote{No grego, a repetição fônica em ``Musas'' (\emph{m\emph{ousai}}) e
``glorificam'' (\emph{klei\emph{ousai}}), os termos que abrem e fecham o
verso, confere destaque às deusas e à sua função. ``Da Piéria'' pode
funcionar como um adjunto adverbial conectado a ``para cá'' e pressupor
uma ação verbal como ``vinde''.} da Piéria, que com cantos glorificam,

vinde e narrai de Zeus, vosso pai louvando:

ele torna igualmente homens mortais ignotos e insignes,

falados e ignorados devido ao grande Zeus.\footnote{Entre os versos 2--4, as formas gregas de ``Zeus'' são \emph{Di'} e
\emph{Dios}, com o que se estabelece um trocadilho (ou figura
etimológica) com a preposição \emph{dia} (``graças a'', ``por meio de'')
no verso 3.}

Pois fácil fortifica e fácil ao forte limita, \num{5}

fácil ao ilustre diminui e ao sem lustre avulta,

fácil endireita o torto e ao arrogante seca

Zeus troveja-no-alto, que a morada superior habita.

Atende-me, vendo e ouvindo, e com justiça\footnote{\emph{Dikē} é um termo polissêmico aqui traduzido por ``justiça'' mas
logo abaixo por ``juízo''.} endireita \qb{}sentenças

tu; eu a Perses\footnote{O nome do irmão de Hesíodo evoca o verbo grego \emph{perthein}
(``pilhar, destruir''), usado no segundo verso da \emph{Odisseia} para
caracterizar o responsável pela conquista de Troia, Odisseu.} o genuíno quero por num discurso. \num{10}

Ei, uma só família de Brigas não havia, mas na terra\footnote{Aqui se corrige o que é afirmado na \emph{Teogonia}, v. 225-32, onde
há uma única Briga e sua prole engloba Labor, Esquecimento, Fome,
Aflições, Batalhas, Combates, Matanças, Carnificinas, Disputas,
Embustes, Contos, Contendas, Má-Norma, Desastre e Jura, elementos que
pertencem à espinha dorsal de \emph{Trabalhos e dias}.}

há duas: a uma aprovaria quem a entendesse,

e a outra é censurável; são de ânimo bem distinto.

Pois esta a guerra danosa e a discórdia amplia,

terrível; mortal algum dela gosta, mas, constrangidos, \num{15}

devido a desígnios de imortais, honram a pesada Briga.

Àquela outra gerou primeiro a escura Noite,

e a pôs o Cronida trono-no-alto,\footnote{O epíteto grego traduzido por ``trono-no-alto'' tem sentido incerto.} morador do céu,

nas raízes da terra, e é bem melhor para os homens:\footnote{Tendo em vista as possíveis conotações cosmogônicas do trecho, em
particular, se aceitarmos que toda essa passagem inicial remete à
\emph{Teogonia}, talvez pudéssemos grafar Terra, ou seja, Gaia.}

ela até mesmo o inapto instiga ao trabalho.\footnote{``Trabalho'' traduz \emph{ergon}, termo recorrente no poema e traduzido também por ``feito'' e ``lavoura''.} \num{20}

Alguém, precisando trabalhar, olhando para outro,

rico, que almeja arar e plantar

e arrumar a fazenda,\footnote{``Fazenda'' traduz \emph{oikos}, que engloba o patrimônio (casa, terra, animais, instrumentos) e as pessoas sob o domínio de um só
indivíduo.} emularia, vizinho, o vizinho

que fartura almeja: boa essa Briga para os mortais.\footnote{Canevaro (p. 107, n. 78) assim resume as dificuldades do trecho:
``os versos 21-4 contêm um aparente anacoluto, talvez indicando que o
texto é corrupto ou que, ao agregar elementos tradicionais, Hesíodo não
arrumou tudo''.}

Ceramista ressente ceramista, e carpinteiro, carpinteiro, \num{25}

mendigo inveja mendigo, e cantor, cantor.

Tu, ó Perses, isso deposita em teu ânimo,

e a Briga sádica\footnote{Ao invés de ``sádica'' (o exagero anacrônico é proposital), talvez apenas ``que tem prazer com os males (dos outros)''.} não te afaste o ânimo do trabalho,

espreitando contendas e sendo ouvinte de assembleia.\footnote{``Assembleia'' se refere a atividades jurídicas na ágora.}

Pouco se preocupa\footnote{``Pouco se preocupa'' traduz \emph{ōrē gar t' oligē peletai}, a lição preferida pela maioria dos estudiosos. Caso se ler \emph{hōrē}
(``é curto/a o tempo/a estação de\ldots{}''), o trocadilho com
``sazonal'' (\emph{hōraios}, 32) e a referência ao tema do ``tempo''
(cf. ``dias'') no poema é acentuado.} com contendas e assembleias \num{30}

quem dentro não encerra suficiente sustento

sazonal que a terra produz, o grão de Deméter.

Tendo-se disso fartado, contendas e discórdia ampliarias

mirando bens alheios. Outra vez não te será possível

agir assim; mas agora decidamos a contenda \num{35}

com retos juízos, os que, de Zeus, são os melhores.

Pois já dividimos o patrimônio, e muita outra coisa

tentavas tomar e levar, enaltecendo bastante os reis

come-presente, eles que querem esse juízo pronunciar.

Tolos, não sabem quão maior é a metade que o todo \num{40}

nem quão grande valia há na malva e no asfódelo.\footnote{A passagem muito provavelmente tem ou bem um sentido crítico,
apontando para formas precárias de alimentação (a pobreza em oposição ao
modo de vida dos reis) ou místico, aludindo a uma dieta ``divina'',
parte de uma vida de privação com benefícios após a morte (ao modo
órfico-pitagórico)}

Pois deuses ocultaram e seguram o sustento dos homens;

de outro modo, fácil trabalharias até um só dia

de sorte a teres bastante, mesmo inativo,\footnote{``Inativo'' é a tradução de \emph{aergos}, formado a partir do radical de ``trabalho'' (\emph{ergon}) e ``trabalhar'' (\emph{ergazomai}).} até por um ano;

ligeiro irias o leme sobre a fumaça depositar,\footnote{A lareira.} \num{45}

e findaria o trabalho dos bois e das mulas robustas.

Mas Zeus o ocultou, com raiva em seu juízo,\footnote{O objeto direto do verbo não é explicitado; faz mais sentido ser ``sustento'' (47) que ``fogo'' (50).}

pois enganou-o Prometeu curva-astúcia.

Por isso para os homens armou agruras funestas

e ocultou o fogo; a esse, então, o brioso filho de Jápeto \num{50}

roubou para os homens do astucioso Zeus

em cavo funcho-gigante sem Zeus prazer-no-raio notar.

Com raiva, disse-lhe Zeus junta-nuvens:

``Filho de Jápeto, supremo mestre em planos,

regozijas após o fogo roubar e meu juízo iludir, \num{55}

para ti mesmo e homens vindouros, grande desgraça.

Eu a eles, pelo fogo, darei um mal com que todos

se deleitarão no ânimo, seu mal abraçando.''

Assim falou, e gargalhou o pai de varões e deuses.

Ordenou ao gloriosíssimo Hefesto bem rápido \num{60}

terra com água molhar, inserir humanas voz

e força e na face assemelhara deusas imortais

a bela aparência atraente de uma moça; e a Atena,

ensinar os trabalhos, tramar urdidura muito adornada;

e à dourada Afrodite, graça verter em torno da cabeça, \num{65}

anseio aflitivo e preocupações devora-membro;\footnote{O desejo é o dos homens por causa dela, e não o dela própria (West).}

inserir um espírito canino e um modo finório

a Hermes impôs, o condutor Matador-da-Serpente.

Assim falou, e obedeceram a Zeus, o senhor Cronida.

De pronto da terra modelou o glorioso Duas-Curvas,\footnote{Duas-Curvas é Hefesto; os antigos associavam o epíteto à deficiência
do deus na locomoção (literalmente, algo como torto em ambas as pernas),
mas é possível que etimologicamente o termo se referisse a uma
ferramenta do deus-artesão.} \num{70}

pelos desígnios do Cronida, a imagem de uma moça \qb{}respeitada;

a ela cinturou e adornou a deusa, Atena olhos-de-coruja;

em volta dela, as divinas Graças e a senhora Persuasão

colares de ouro puseram sobre a pele; coroaram-na

as Estações bela-coma com flores primaveris; \num{75}

e todo adorno ajustou a seu corpo Palas Atena.

Em seu peito, o condutor Matador-da-Serpente

mentiras, contos solertes e um modo finório

arranjou pelos desígnios de Zeus grave-ressoo; nela \qb{}discurso

pôs o arauto dos deuses, e nomeou essa mulher \num{80}

Pandora, porque todos que têm morada olímpia

deram-lhe um dom,\footnote{``Deram-lhe \emph{como} dom'' é outra possibilidade de tradução dessa frase ambígua.} desgraça dos varões come-grão.

Mas após o íngreme ardil, impossível, completarem,

a Epimeteu o pai enviou o glorioso Matador-da-Serpente,

veloz mensageiro dos deuses, com o dom. E Epimeteu \num{85}

não pensou no que lhe dissera Prometeu, nunca um dom

aceitar de Zeus Olímpio, mas reenviá-lo

de volta para não se tornar um mal aos mortais.

Mas após o receber, já de posse do mal, entendeu.

Pois antes viviam as tribos de homens sobre a terra \num{90}

afastadas de males e longe de duro labor

e aflitivas doenças, as que dão morte aos varões.

{[}Pois rápido, na miséria, envelhecem os mortais.{]}\footnote{Praticamente todos comentadores optam por deletar o verso (idêntico
a \emph{Odisseia} 19.360), considerando-o uma glosa.}

Mas a mulher tirou à mão a grande tampa do pito\footnote{``Pito'' (em grego, \emph{pithos}) é um grande vaso de cerâmica
usado para armazenar diversos tipos de alimento (grãos, azeite e vinho,
sobretudo). A imagem de uma caixa sedimentou-se a partir do
Renascimento.}

e os espalhou;\footnote{Espalhou os males.} para os homens, armou agruras funestas. \num{95}

Lá mesmo só Esperança, na casa inquebrável,

ficou, dentro do pito sob as bordas, e não porta

afora voou: deixou tombar antes a tampa do pito

pelos desígnios do porta-égide, Zeus junta-nuvens.

Outras mil coisas funestas erram entre os homens, \num{100}

pois plena é a terra de males, pleno, o mar:

doenças para os homens de dia, outras, de noite,

espontâneas, vagam, levando males aos homens

em silêncio, pois tirou-lhes a voz o astucioso Zeus.

Assim, é impossível escapar da mente de Zeus. \num{105}

Se queres, o outro relato para ti vou esboçar

bem e com destreza, e tu o lança em teu juízo,

como de origem igual são deuses e homens mortais.

De ouro a primeiríssima linhagem de homens mortais

foi feita pelos imortais que têm morada olímpia. \num{110}

Existiram na época de Crono, quando reinava no céu;

como deuses viviam com ânimo serenoo,

afastados de labores, longe de agonia: nem a infeliz

velhice havia, e, sempre iguais nos pés e mãos,

apraziam-se em festejos, longe de todos os males; \num{115}

morriam como por sono subjugados. Toda benesse

possuíam: o fruto, que produzia o solo fértil,

espontâneo, era farto e irrestrito; contentes

e em paz gozavam dos grãos\footnote{``Grãos'' traduz \emph{erga}, o plural de \emph{ergon} (``trabalho''), portanto, ``fruto do trabalho''.} com muita benesse.

\{Eram ricos em ovelhas, caros aos deuses ditosos.\}\footnote{A maioria dos editores considera esse verso espúrio por estar
ausente em todos os manuscritos e ser citado apenas por um autor antigo.} \num{120}

Mas depois que a terra encobriu essa linhagem,

eles são numes, pelos desígnios do grande Zeus,

nobres, terrestres, guardiões dos homens mortais:

eles vigiam juízos e feitos terríveis,

envoltos em neblina, vagando por toda a terra,\footnote{Alguns editores rejeitam esses versos por serem repetidos \emph{verbatim} em 254-5 ``Guardiões'' e ``vigias'' traduzem palavras com a mesma raiz.} \num{125}

dadores de riqueza; e essa honraria real receberam.

Uma segunda linhagem, muito pior, de prata

depois fizeram os que têm morada olímpia,

à de ouro semelhante nem no físico nem na ideia.

Mas por cem anos o menino, junto à mãe devotada, \num{130}

era criado, divertindo-se, grande tolo, em sua casa;

mas ao se tornar jovem e alcançar o pico da juventude,

pouquíssimo tempo viviam, com aflições

pela insensatez: violência iníqua eram incapazes

de conter, recíproca, nem servir aos imortais \num{135}

queriam ou sacrificar nos sacros altares dos ditosos,

norma dos homens pelos costumes. Esses então

Zeus Cronida, ocultou-os, com raiva, pois honras

não deram aos deuses ditosos que ocupam o Olimpo.

Mas depois que a terra também encobriu essa linhagem, \num{140}

esses, subterrâneos, são chamados mortais ditosos,

segundos, e, mesmo assim, honra também lhes segue.

E Zeus pai a terceira, outra linhagem de homens mortais,

de bronze, fez, em nada semelhante à de prata,

dos freixos, terrível e ponderosa: a eles os feitos \num{145}

de Ares importavam, funestos, e atos violentos. Pão

não comiam, com ânimo decidido de adamanto,

inabordáveis: grande força e mãos intocáveis

dos ombros nasceram sobre os membros robustos.

Deles eram brônzeas as armas, brônzeas as casas, \num{150}

e com bronze obravam: negro ferro não existia.

E eles, subjugados pelas mãos uns dos outros,

rumaram à casa bolorenta do Hades gelado,

anônimos: embora assombrosos, a eles a morte

negra levou, e largaram a fúlgida luz do sol. \num{155}

Mas depois que a terra também encobriu essa linhagem,

de novo ainda outra, a quarta, sobre o solo nutre-muitos

Zeus Cronida produziu, mais justa e melhor,

a divina linhagem de varões heróis, esses chamados

semideuses, a estirpe anterior sobre a terra sem-fim. \num{160}

Destruíram-lhes guerra danosa e prélio terrível,

a uns sob Tebas sete-portões, na terra cadmeia,

ao combaterem pelos rebanhos de Édipo,

a outros, nas naus, sobre o grande abismo do mar,

levando a Troia por conta de Helena belas-tranças. \num{165}

Lá em verdade a alguns encobriu o fim que é a morte,

e a outros, longe dos homens, ofertou sustento e casa

o pai, Zeus Cronida, e os alocou nos limites da terra. \num{168} 

E eles habitam com ânimo sereno \num{170}

nas Ilhas dos Venturosos junto a Oceano fundo-redemunho,

heróis afortunados, aos quais fruto doce como mel,

que três vezes ao ano floresce, produz o solo fértil.

Não mais, então, eu devia viver entre os quintos

varões, mas ter antes morrido ou depois nascido. \num{175}

De fato agora a linhagem é de ferro: nunca, de dia,

se livrarão da fadiga e da agonia, nem à noite,

extenuando-se: os deuses darão duros tormentos.

Todavia, para eles benesses se juntarão aos males.

Zeus destruirá também essa linhagem de homens mortais \num{180}

quando, ao nascer, tiverem cãs nas têmporas.

Nem o pai semelhante aos filhos, nem os filhos a ele,

nem anfitrião a hóspede, e companheiro a companheiro,

nem irmão será querido assim como no passado.

Aos genitores, tão logo envelhecerem, desonrarão: \num{185}

usando palavras duras, deles se queixarão,

terríveis, ignorantes do olhar dos deuses; nem eles

aos genitores idosos vão retribuir a criação.

Justos nas mãos: um aniquilará a cidade do outro;

não haverá gratidão pelo honesto nem pelo justo \num{190}

nem pelo bom, e mais ao feitor de males, à violência

do varão irão honrar; justiça estará nas mãos,

e respeito não haverá, e o mau lesará o homem melhor,

enunciando discursos tortos sobre os quais jurará.

E inveja\footnote{``Inveja'' traduz \emph{zēlos}, termo ambivalente cujo verbo
cognato é alhures traduzido por ``emular''; aqui o substantivo tem um
sentido claramente negativo.} a todos os homens agoniados, \num{195}

cacofônica, seguirá, sádica, horripilante.

Então rumo ao Olimpo, da terra largas-rotas,

após encobrir a bela pele com capas brancas,

para junto da tribo de imortais irão, deixando os homens,

Respeito e Indignação: isto deixarão, aflições funestas, \num{200}

aos homens mortais; defesa não haverá contra o mal.

Agora estória aos reis contarei, eles conscientes.

assim dirigiu-se falcão a rouxinol pescoço-variegado,

ao qual nas garras levava bem no alto, entre nuvens;

este, tristemente, transpassado por garras recurvas, \num{205}

chorava, e aquele, sobranceiro, dirigiu-lhe o discurso:

``Insano, por que guinchas? Tem-te um muito melhor;

irás aonde eu te levar, embora sendo um cantor;\footnote{Entre os versos 203--8, há um jogo de palavras entre \emph{aedōn} (``rouxinol'') e \emph{aoidós} (``cantor'').Entre os versos 203--8, há um jogo de palavras entre \emph{aedōn} (``rouxinol'') e \emph{aoidós} (``cantor'').}

de ti farei refeição, se eu quiser, ou te libertarei.

Insensato quem quer a superiores se contrapor: \num{210}

privado da vitória, além do vexame sofre aflições.''

Assim falou o falcão voa-veloz, ave asa-longa.

Tu, ó Perses, escuta a Justiça e não fomentes violência:

violência é nociva num mortal pobre; nem o nobre

é capaz de suportá-la fácil, mas a ela sucumbe \num{215}

ao topar desastres. Caminho distinto de percorrer

é melhor, o rumo ao justo: justiça sobrepuja violência

ao se consumar; após sofrer o tolo aprende.

De pronto corre Jura ao lado de juízos tortos;

e Justiça causa tumulto ao ser arrastada, quando a levam \qb{}varões\footnote{Não há consenso entre os editores sobre em que momento devem ser
personanificados, entre os versos 213--20, Justiça, Violência e Ruína,
respectivamente, \emph{Dikē}, \emph{Hubris} e o plural de \emph{Atē}.} \num{220}

come-presente e com tortos juízos escolhem sentenças:

ela segue, chorando a cidade e os costumes das gentes,

envolta em neblina, levando um mal aos homens,

eles que a expelem e não um juízo reto atribuem.

Os que dão juízos a estrangeiros e conterrâneos, \num{225}

retos, e de modo algum se desviam do justo,

para eles a cidade viceja, nela as gentes florescem;

Paz nutre-moços vai pela terra, e nunca a eles

destina guerra aflitiva Zeus ampla-visão;

nunca a varões reto-juízo segue Fome \num{230}

nem Desastre, e em festas repartem o fruto da lida.

A terra lhes produz muito sustento, e nos morros carvalho

produz bolotas no alto e, no meio, abelhas;

ovelhas lanosas sentem o peso dos velos;

e as mulheres parem filhos semelhantes aos pais. \num{235}

Vicejam com coisas boas direto: para os barcos

não vão, e o solo fértil produz fruto.

A quem importa nociva violência e feitos terríveis,

o Cronida lhes destina justiça , Zeus ampla-visão.

Amiúde até urbe inteira perde com um homem vil, \num{240}

um que comete ofensa e arma iniquidades.

Sobre eles, do céu o Cronida envia grande desgraça,

fome e peste, e as gentes perecem:

as mulheres não parem e as fazendas fenecem

pelo plano de Zeus Olímpico; e outra vez \num{245}

destrói seu amplo exército ou sua muralha

ou de suas naus o Cronida se vinga no mar.

Ó reis, também vós mesmos ponderai a fundo

essa justiça: estando perto entre os homens,

os imortais ponderam quantos com tortos juízos \num{250}

se ralam uns aos outros ignorando o olhar dos deuses.

Pois três vezes dez mil há pelo solo nutre-muitos,

imortais guardiões de Zeus sobre os homens mortais,

eles que vigiam juízos e feitos terríveis,

envoltos em neblina, vagando por toda a terra. \num{255}

Ela é uma moça, Justiça, de Zeus nascida,

majestosa e respeitada pelos deuses que ocupam o Olimpo,

e quando alguém a lesa, aviltando-a tortuosamente,

de pronto se senta ao lado de Zeus pai, o Cronida,

e proclama a mente dos homens injustos para que pague \num{260}

o povo pela iniquidade dos reis, que, com ideias funestas,

inclinam seus juizos para outro lado, enunciando-os tortos.

Vigiando essas coisas, endireitai os discursos, reis

come-presente, e de todo esqueçais dos juízos tortos.

A si mesmo faz mal o varão que faz mal a outro, \num{265}

e o desígnio vil, para quem decide, é o pior.

O olho de Zeus, que tudo vê e tudo entende,

também isso, se quiser, observa e não ignora

o jaez dessa justiça\footnote{Ou ``desse juízo''.} que a cidade dentro encerra.

Agora nem eu mesmo, entre os homens, justo \num{270}

fosse, nem meu filho, pois é ruim ser um homem

justo, se o mais injusto receber justiça maior.

Mas isso não espero que consuma o astucioso Zeus.

Tu, ó Perses, lança estas coisas em teu juízo

e ouve a Justiça e esquece a força de todo. \num{275}

Pois este costume para os homens o Cronida apontou,

para peixes, feras e aladas aves de rapina

se entredevorarem, pois não há justiça entre eles;

e aos homens deu justiça, que é de longe o melhor:

se alguém quiser anunciar o que é justo \num{280}

ao discerni-lo, fortuna lhe daria Zeus ampla-visão;

mas quem, no testemunho, de bom grado perjura

e mente, lesa Justiça e se fere sem cura,

e, mais débil, sua estirpe fica atrás no futuro;

e a estirpe do varão honesto, no futuro, será melhor. \num{285}

A ti eu direi o que penso de bom, Perses, grande tolo:

Miséria é possível, aos montes, agarrar

facilmente: é plano o caminho, e mora bem perto.

Mas diante de Excelência suor puseram os deuses

imortais: longa e íngreme é a via até ela, \num{290}

e áspera no início; quando se chega ao topo,

fácil então ela é, embora sendo difícil.\footnote{``Miséria'' e ``Excelência'' traduzem \emph{kakotēs} e
\emph{aretē}, que podem ser personificados. Seu sentido (social ou
moral) é aberto, não precisando ser limitado pelo contexto (Canevaro
2015, p. 70).}

Este é o melhor de todos,\footnote{``Melhor de todos'' traduz o termo poético \emph{panaristōs}, cunhado por Hesíodo, no qual reverbera \emph{aretē} (289).} quem por si tudo entender

ao refletir no que será melhor, depois e no fim;

nobre também o que atende a quem dá bons conselhos; \num{295}

mas quem por si não entender nem, de outro ouvindo,

lançar no ânimo, esse é um varão infrutífero.

Mas tu, sempre lembrando de nosso preceito,

trabalha, Perses de linhagem divina, para Fome a ti

odiar, e te querer bem Deméter bela-coroa, \num{300}

respeitada, e encher teu celeiro de sustento:

Fome é de todo companheira do varão inativo.

Contra este se indignam deuses e varões, quem inativo

vive, semelhante, no caráter, a zangões sem-ferrão,\footnote{O que se traduziu por ``sem-ferrão'' é, na verdade, o sentido mais
provável de um adjetivo grego de sentido desconhecido.}

eles que esgotam a faina das abelhas, inativos, \num{305}

comendo; que te seja caro ordenar o trabalho com precisão

para de sustento sazonal se encherem os celeiros.

Pelo trabalho são os varões muita-ovelha e ricos,

e, ao trabalhar, muito mais caro aos imortais

{[}e mortais serás: demais abominam inativos{]}.\footnote{Verso considerado espúrio pela maioria dos editores.} \num{310}

Trabalho não é insulto algum; inação, um insulto.

Se trabalhares, logo te emulará o inativo

ao enricares; à riqueza acompanham excelência e glória.

Seja o que o destino te tornar, trabalhar é melhor,\footnote{Verso de difícil interpretação; outra possibilidade seria ``Tal
como eras pelo destino, trabalhar é melhor'', ou seja, ``como eras (e
ainda és) pobre''.}

se, para longe de posses alheias o ânimo insano \num{315}

dirigindo ao trabalho, tratares do sustento como te peço.

Respeito não bom cuida de varão carente,\footnote{Seguimos Vergados (2020, p. 161-62). ``Vergonha não é boa em cuidar
de varão carente'' é a tradução com o infinitivo κομίζειν
\emph{preferido por alguns editores}.}

respeito, ele que muito lesa e beneficia varões:

respeito,\footnote{``Respeito'' (ou ``vergonha'') é \emph{aidōs}, termo cuja
conotação geralmente é positiva.} sim, é da desfortuna; audácia, da fortuna.

Bens não são para se tomar; dado por deus é melhor: \num{320}

se um à força, no braço, adquire grande fortuna

ou pilha por meio da língua, o que amiúde

ocorre quando lucro engana o espírito

dos homens, e Desrespeito oprime Respeito,\footnote{West e Most defendem a personificação.}

fácil debilitam-no os deuses, degradam a fazenda \num{325}

do varão, e pouco tempo fortuna o acompanha.

Age igual quem prejudica suplicante ou estranho,

quem sobe no leito do próprio irmão

às esconsas para sexo com sua esposa, ação imprópria,\footnote{Diversos editores (West, Ercolani) rejeitam o verso.}

quem, insensato, ofende os filhos órfãos de alguém, \num{330}

quem com o genitor idoso no vil umbral da velhice

briga, abordando-ο com duras palavras:

com ele o próprio Zeus se irrita e no fim,

pelas ações injustas, impõe-lhe dura retribuição.

Mas tu, disso mantem longe, de todo, o ânimo insano. \num{335}

Faze aos deuses imortais, quanto puderes, sacrifícios

santos e limpos, e em cima queima coxas radiantes;

outra vez, propicia-os com libações e incensos

quando fores dormir e quando a sacra luz retornar

para que tenham coração e ânimo propício a ti \num{340}

a fim de adquirires\footnote{``Adquirir'' traduz um verbo grego que implica uma troca por mercadorias e não uma transação pecuniária.} a gleba de outros, não outro, a tua.

Chama o amigo ao banquete, e o inimigo, deixa estar;

sobretudo chama quem mora perto de ti:

se também evento outro na vila te atingir,

vizinhos vêm sem cinto, mas cinturam-se os parentes.\footnote{Alguns comentadores, como West, defendem que a referência é à
velocidade da ajuda.} \num{345}

Vizinho ruim é desgraça, tanto quanto o bom é de grande \qb{}valia:

partilha de honraria quem partilha de ótimo vizinho;

nem um boi se perderia se o vizinho não fosse ruim.

O que vem do vizinho seja bem medido, e devolve bem

com a mesma medida ou mais, se fores capaz, \num{350}

para, se necessitado, também no futuro achares algo certo.

Não lucres de forma vil; lucros vis são como desastres.

Sê amigo dos amigos e visita quem te visita.

E dá a quem der e não dês a quem não der:

ao dador alguém dá, ao não dador ninguém dá. \num{355}

Dádiva é boa, Usurpa é má, doadora de morte:

se um varão de bom grado der algo, mesmo grande,

compraz-se com a dádiva e deleita-se em seu ânimo;

se alguém, confiante no desrespeito, pegar ele mesmo

ainda que pouco, isso congela seu coração.\footnote{Não fica claro no texto grego se o coração é da vítima ou do autor do roubo.} \num{360}

Pois se depositares, mesmo pouco, sobre pouco,

e amiúde o fizeres, muito isso logo se tornaria:

quem adiciona ao que há, esse evitará a fome ardente.

O depositado na casa não aflige o varão;

é melhor estar na casa, pois o que sai corre risco. \num{365}

É bom tomar do que está aí, e desgraça para o ânimo

carecer do distante; peço que isso ponderes.

Do pito iniciado ou em seu final te farta,

se, na metade, poupa; poupança na base é pífio.

Que paga a um caro varão, acordada, seja certa; \num{370}

e, ao sorrir para o irmão, traze uma testemunha:

tanto confiar quanto desconfiar destruiu varões.\footnote{Os versos 370--72 são de origem polêmica (o verso 372 tem também um problema
métrico na sua primeira metade), ausentes de parte dos papiros que
contêm o entorno da passagem; talvez tenham sido acrescentados alguns
séculos depois da composição do poema e rejeitados pelos seus primeiros
editores antigos.}

Que mulher balanç'-a-bunda\footnote{Adjetivo de difícil interpretação, talvez dissesse respeito ao modo de a mulher se vestir (destacando, aumentando as nádegas).} não te engane a ideia,

tagarelando, solerte, esquadrinhando teu celeiro:

quem confia em mulher, esse confia em larápios.\footnote{``Larápio'' traduz o grego \emph{philētēs}, que, pelo contexto,
talvez evoque termos cognatos do verbo \emph{philein} (``amar''); este
pode ser empregado em contexto erótico, mas acima é utilizado para
amigos (353).} \num{375}

Que haja um filho único para este a casa paterna

manter: assim riqueza crescerá nos salões;

que morra velho, deixando para trás outro filho.\footnote{Menção a um segundo filho (como no caso de Hesíodo e Perses) ou a
um neto? Outra dificuldade é que alguns manuscritos atestam a forma
``morras'', defendida por West e Verdenius. Canevaro (p. 95) argumenta
que um único filho se coaduna melhor com a ideia da autossufiência;
dois, com a razão do próprio poema, qual seja, instruir o irmão, o que
explicaria a menção da maior riqueza a ser produzida por mais gente no
verso seguinte.}

Fácil para mais gente\footnote{``Mais gente'' talvez implique ``mais filhos'' (Ercolani).} Zeus daria fortuna incontável:

mais empenho de mais gente, maior o excedente. \num{380}

Se riqueza te almeja o ânimo em teu íntimo,

age assim, e trabalha trabalho sobre trabalho.

Ao subirem as Plêiades, filhas de Atlas,\footnote{A passagem entre os versos 383--413  pode ser compreendida como uma abertura da longa
sessão centrada nas atividades agrícolas.}

iniciai a colheita, e arada e semeadura ao se porem.

Essas, se sabe, por quarenta noites e dias \num{385}

estão ocultas, e de novo, após o ano volver-se,\footnote{As posições da constelação referem-se à 1\textsuperscript{a}
metade de maio, no caso da época de colheita (antes, por 40 dias, estão
invisíveis), e à época que vai do fim de outubro ao início de novembro
para a arada.}

aparecem assim que o ferro é amolado.

Este é costume das planícies, e dos que do mar

perto habitam e dos que vales profundos,

longe do mar undoso, fértil região, \num{390}

habitam: semeia desnudo,\footnote{``Desnudo'' é a interpretação mais provável; a outra é ``só de túnica''. Pode apontar para uma compreensão religiosa da atividade do
agricultor; Terra, claro, é uma deusa. Mas também já se propôs uma
equivalência ao nosso ``arregaçar as mangas'' (Beall 2004, p. 7, n. 20).} ara desnudo,

colhe desnudo, se quiseres, na estação, de todo trabalho

de Deméter te ocupar, para cada um deles

te crescer na estação,\footnote{``Na estação'' (\emph{hōri}'; cf. v. 30) remete a um
\emph{leitmotiv} importante desta parte do poema, o tempo correto.} e que não, mais tarde carente,

mendigues em alheias casas e nada realizes --- \num{395}

como também agora vieste a mim: mais não te darei,

nada mais irei medir; trabalha, tolo Perses,

os trabalhos que, aos homens, deuses assinalaram

para um dia, com filhos e mulher, aflito no ânimo,

não buscares sustento nos vizinhos, que irão te ignorar. \num{400}

Duas ou três vezes talvez arranjes; se ainda molestares,

nada irás obter, e dirás muita coisa inútil,

e infrutífero será o pasto de palavras. Mas te peço

ponderares como livrar-te das dívidas e evitar a fome:

primeiríssima, a fazenda, e mulher e boi de arar --- \num{405}

adquirida, não desposada, uma que seguiria os bois.\footnote{O verso 406 como que corrige a ideia de que em 405 se fala da
esposa; a esposa, porém, era imprescindível para que a propriedade
passasse de pai para filho. Propriedade e esposa como elementos
fundamentais para a vida de um homem é uma ideia comum em textos do
Mediterrâneo Oriental. Mais adiante, Hesíodo dirá que a idade ideal para
o casamento é 30 anos, ou seja, depois de o homem garantir sua
propriedade.}

Os instrumentos, na fazenda, deixa todos preparados,

para a outro não pedires, ele recusar, e tu careceres,

e a estação transcorrer, e fenecer tua lavoura.

Não adies para amanhã e o dia seguinte; \num{410}

o varão lida-inútil não enche o celeiro

nem o que adia: empenho fomenta o trabalho,

e o varão lida-adiada sempre luta com desastres.

Quando o ímpeto do sol lancinante se abranda\footnote{Os versos 414--22 referem-se ao período entre setembro e outubro, ou seja, a época que precede a arada, o outono.}

na queimação sudorífera, manda chuva outonal \num{415}

o potente Zeus, e muda a pele do mortal,

muito aliviada: isso é quando o astro Sírio

ligeiro sobre a cabeça de homens cria-p'ra-morte

move-se de dia e colhe maior porção da noite ---

nisso é menos sujeita a vermes a por ferro cortada \num{420}

madeira, e folhas tombam no solo, e cessa de brotar:

então corta madeira cônscio, trabalho sazonal.

Corta um graal de três pés, um pilão de três cúbitos

e um eixo de sete pés, pois assim o encaixe é bom;

se fosse de oito pés, marreta dela também cortarias. \num{425}

Corta roda de três palmos para carro de dez mãos.

Há muita madeira curva: quando achares, leva o apo\footnote{Na sequência, mencionam-se as partes de um arado de garganta (ou
arado curvo). ``Apo'' é um termo genérico que seria mais propriamente
traduzido por ``garganta''.}

para casa, procurando na montanha e no campo ---

de azinheiro: esse, p'ra arar com bois, é o mais resistente,

quando o servo de Atena, após fixá-lo no dental\footnote{O carpinteiro.} \num{430}

com pregos, o aproxima e prende ao timão.

Labuta e coloca dois tipos de arado na fazenda,

com apo natural e articulado, pois muito melhor assim:

se quebrasses um, o outro sobre os bois lançarias.

Timões de loureiro ou de olmo são os mais sem vermes, \num{435}

o dental, de carvalho, o apo, de azinheiro. Bois nove-anos,

possui dois; de fato, nada fraca é sua força,

ainda na juventude: os melhores para trabalhar.

Os dois não quebrariam o arado, na trilha do sulco

brigando, e deixariam o trabalho, inútil, lá mesmo. \num{440}

Que a eles seguisse um varão quadragenário

após almoçar pão oito-partes, quadripartível,\footnote{Dois adjetivos obscuros nesse verso; provável referência a um pão muito grande e ao tamanho --- grande --- da porção comida pelo lavrador.}

que, empenhado na tarefa, dirigiria um sulco reto,

não mais focado nos amigos e com o ânimo

na tarefa; outro mais jovem que ele não é melhor \num{445}

para repartir sementes e não esbanjar no plantio:

varão mais jovem se excita atrás dos amigos.

Fica atento para quando ouvires o som do grou\footnote{Entre o final de outubro e o início de novembro.}

que do alto das nuvens guincha anualmente,

ele que traz o sinal da arada, a estação do inverno \num{450}

chuvoso indica e morde o coração do homem sem boi;

nisso dentro alimenta os lunados bois.

Pois é fácil dizer ``dá dois bois e um carro'';

e é fácil recusar: ``Há trabalho para meus bois''.

E pensa o homem rico no juízo em compor um carro; \num{455}

tolo, não sabe isto: cem pedaços tem um carro;

antes te empenha em deles dispor em casa.

Tão logo aparecer a arada aos mortais,

nisso te lança, igualmente os servos e tu mesmo,

arando a seca e a úmida na estação da arada,\footnote{O sentido não é totalmente claro; é possível suprir-se o substantivo ``terra''.} \num{460}

bem cedo te animando, para se encherem as lavouras.

Na primavera revolve a terra; renovada, no verão não te \qb{}trairá:

pousio semeia quando ainda for fofa a lavoura.\footnote{Segundo West (1978: 274), o que Hesíodo quer dizer é ``a terra
que semeias deveria ser terra de pousio que araste na primavera e, de
preferência, de novo no verão, e deveria ser arada na hora certa antes
que chuva demais tenha caído''; muita chuva deixa a terra pesada.}

Pousio afasta-dano apazigua crianças.\footnote{A menção parece ser a crianças que não passam fome.}

Reza a Zeus terreno e à pura Deméter \num{465}

para o sacro grão de Deméter, maduro, pesar

tão logo iniciares a lavra, quando, a ponta da rabiça

na mão, atingires o dorso dos bois com vara,

os quais, no jugo, puxam o timão com a correia. Pouco atrás,

outro servo, com enxada, faça as aves se esfalfarem, \num{470}

escondendo a semente: organização é o melhor

para os homens mortais, e desorganização, o pior.

Assim, maduras, ao solo as espigas se inclinariam

se o próprio Olímpio um bom termo depois ofertasse,

e teias de aranha expulsarias das vasilhas: espero que tu \num{475}

jubiles ao pegares dos víveres que dentro estiverem.

Bem-sucedido chegarás à luzente primavera e outros

não fitarás; mas de ti outro homem precisará.

E se nos solstícios de inverno arares a terra divina,

colherás sentado, pouco encerrando na mão, \num{480}

amarrando em paralelo, empoeirado, não muito alegre,

e num cesto levarás; poucos irão te contemplar.

Sempre cambiante é a mente de Zeus porta-égide,

e para os homens mortais é difícil entendê-la.

Se arares tarde, isto te poderia ser um remédio: \num{485}

quando um cuco cucula nas folhas do carvalho

pela prima vez e compraz mortais na terra sem-fim,

que Zeus deixe chover três dias sem parar,\footnote{Ou ``no terceiro dia''; trata-se de março.}

nem cobrindo a marca do casco do boi nem faltando;

assim o arador tardio com o precoce competiria. \num{490}

No ânimo a tudo atenta bem: que não te escape

a vinda da luzente primavera nem a chuva sazonal.

Passa ao largo do banco do ferreiro e do galpão protetor

na estação invernal, quando a varões da lavoura o frio

afasta: então inabalável varão muito a casa fomentaria; \num{495}

temo te capturem a Impotência do inverno vil

mais Pobreza, e com mão leve apertes o pé encorpado.\footnote{Sinais de desnutrição ou (menos provável?), sugestão de masturbação; ``pé'' pode se referir também no verso 524 ao \emph{membro}
masculino.}

Amiúde varão inativo, que fica junto à vã esperança,

carecendo de recursos, de vilezas fala ao ânimo.

Esperança não é boa em cuidar de varão carente \num{500}

sentado no galpão sem sustento o bastante.

Aponta aos servos no meio do verão ainda:

``Não será sempre verão; fazei-vos cabanas''.

O mês Lenaion,\footnote{Mês entre janeiro e fevereiro.} dias ruins, todos para couro de boi,

evita-o e também geadas, que, sobre a terra, \num{505}

quando sopra Bóreas, são implacáveis,

ele que, pela Trácia nutre-cavalo, no amplo mar

sopra e o agita; e mugem terra e mato:

a muitos carvalhos copa-alta e abetos encorpados

faz nos vales dos montes tocar o solo nutre-muitos, \num{510}

neles caindo, e então grita toda a vasta floresta;

tremem os bichos e põem os rabos sob os genitais,

até os com lanugem sombreando a pele; mas por eles

sopra, frio, mesmo que tenham peitos hirsutos.

Atravessa couro de boi, que não o contém, \num{515}

e sopra através da cabra longo-pelo; mas não ovelhas,

pois têm pelos fartos e por elas não sopra

a força do vento Bóreas; faz o ancião se recurvar\footnote{Sentido incerto; ``recurvar-se (como uma roda)'' ou ``correr (como
uma roda ou sobre rodas)''.}

e através da moça pele-macia não sopra,

ela que dentro da casa fica junto à cara mãe, \num{520}

ainda não versada nos feitos de Afrodite muito-ouro:

após bem lavar a pele delicada e com óleo à farta

se ungir, dentro se protege, no recesso da casa

em dia invernal, quando o sem-osso amolenta\footnote{Há dúvidas sobre o sentido do verbo traduzido por ``amolentar'' e o
referente do termo ``sem-osso''; para esse último, ``polvo'' foi a opção
dos comentadores antigos (seguida por vários modernos: durante o
inverno, o polvo comeria seus próprios tentáculos de acordo com alguns
antigos) e ``caracol'' e ``pênis'', propostas modernas.} seu pé

na casa sem fogo, deplorável espaço, \num{525}

pois o sol não lhe mostra campo para avançar,

mas à terra, à cidade dos homens negros

visita e a todos os helenos mais tarde brilha.

Então os chifrudos e os sem-chifre toca-no-mato\footnote{A referência não é, necessariamente, somente a veados machos (com cornos) e fêmeas (sem).}

rangem funestamente e pelos capões com ravinas \num{530}

fogem, e isto no juízo a todos preocupa,

os que procuram abrigos e têm covis bem-cercados

em furna pétrea; então são iguais a trípode mortal\footnote{Outra adivinha; aqui, anciãos. Não podemos excluir uma alusão ao
enigma da Esfinge decifrado por Édipo.}

com as costas alquebradas, cuja cabeça olha o chão:

semelhantes a ele vagam, evitando a neve branca. \num{535}

Então veste uma defesa do corpo, como te peço,

manto macio e túnica com franjas:

trama com muito fio em miúda urdidura;

disso te reveste, para teus pelos ficarem imóveis

e não, retos, tremerem de pé pelo corpo. \num{540}

Em volta dos pés, sandálias de boi morto à força,

ajustadas, prende, no interior cobertas com feltro;

de filhotes primogênitos, quando o frio sazonal vier,

cose as peles com tendão de boi para nas costas

lançares, proteção contra chuva; usa sobre a cabeça \num{545}

esmerado boné de feltro, para não molhar as orelhas.

Pois a aurora é gelada quando Bóreas cai;\footnote{``Quando Bóreas perde sua intensidade'' é uma possibilidade de interpretação.}

matutina, vinda do céu estrelado, sobre a terra

bruma se estende sobre trabalhos tritícolas\footnote{Alguns poucos filólogos defendem a leitura dos manuscritos, segundo
a qual ``tritícola'' (\textasciitilde{}fértil) qualifica ``bruma''.} de ditosos,

ela que é extraída dos rios perenes, \num{550}

ao alto, sobre a terra, erguida por lufada de vento,

e ora ela chove ao anoitecer, ora se dispersa,

quando o trácio Bóreas nuvens espessas aglomera.

Antes dela\footnote{``Dela'' parece referir-se a ``bruma''.} finda teu trabalho e à casa retorna,

para nuvem escura, vinda do céu, não te encobrir, \num{555}

deixar a pele molhada e encharcar as roupas ---

evita! --- pois o mês mais difícil é esse,

invernoso, difícil aos rebanhos, difícil aos homens.

Nisso a metade dá aos bois, e ao homem a maior parte\footnote{No inverno, o homem deve comer menos que a média de seu consumo diário no resto do ano, mas mais que 50\% dela.}

de sua ração: longas são as noites\footnote{No grego, no lugar de ``noite'' lê-se literalmente ``aquela que alivia o juízo'', ``benévola'', uma expressão metafórica relativamente comum.} auxiliadoras. \num{560}

Atento a isso até se completar o ano,\footnote{Não confundir com nosso calendário civil; no imaginário grego de
Hesíodo, o ano --- ou seja, o ciclo da agricultura --- começa quando os
depósitos de comida estão cheios.}

contrabalança noites e dias até que de novo\footnote{Ao longo do ano, à medida que as noites forem diminuindo, pode-se aumentar a ração diária.}

Terra, mãe de tudo, trouxer fruto variado.

Quando, após o solstício, o sexagésimo\footnote{A 2\textsuperscript{a} metade de fevereiro.}

dia invernal Zeus completar, então o astro \num{565}

Arcturo, deixando a sacra corrente de Oceano,

começa a raiar, brilhando todo no lusco-fusco;

depois a Pandionida geme-na-aurora,\footnote{Outro epíteto transmitido é ``vero-pranto''. Trata-se da Filomela
do mito de Tereu e Procne (Rouxinol).} a andorinha, se \qb{}lança

à luz para os homens, e a primavera de novo começa.

Antes dela poda as vinhas, pois assim é melhor. \num{570}

Mas quando o leva-casa\footnote{O caracol; talvez idêntico ao ``sem-osso'' (que, de acordo com alguns, também pode ser interpretado como ``sem-retorno'': quem carrega
a própria casa nunca retorna para ela).} subir nas plantas, vindo da terra,

das Plêiades fugindo, então não mais caves junto às vinhas,\footnote{Em meados de maio.}

não, amola a foice e desperta os escravos;

foge de bancos sombreados e sono até a aurora

na estação da colheita, quando o sol seca a pele: \num{575}

nessa época te apressa e leva o fruto p'ra casa,

de pé no nascente, para o sustento te bastar.

Aurora tem do trabalho a terça parte,

Aurora incita na jornada, incita no trabalho,

Aurora, ao surgir, faz muitos homens \num{580}

pegar a estrada e em muitos bois põe o jugo.

Quando enflora o cardo-de-ouro, e a soante cigarra,\footnote{Trata-se, provavelmente, do \emph{scolymus hispanicus}.}

em árvore sentada, entorna soante canto

sem parar de sob as asas, na estação do fadigoso estio,\footnote{Os versos 582--84 referem-se a meados de julho.}

então as cabras, gordíssimas, o vinho, excelente, \num{585}

as mulheres, lascivíssimas, os homens, esgotadíssimos\footnote{Hesíodo talvez aluda, por meio de uma figura etimológica com o
adjetivo referente aos homens, a um esgotamento de sêmen.}

estão, pois Sírio seca cabeça e joelhos,

ressequida a pele com o calor: mas que nisso

haja rochedo umbroso e vinho de Biblos,\footnote{Distrito na Trácia.}

pão feito com leite, leite de cabras que não aleitam \num{590}

e carne de vaca pasto-no-mato\footnote{A referência é a gado selvagem.} que nunca pariu

ou de filhotes primogênitos; junto bebe fulgente vinho

sentado à sombra, saciado, no coração, de comida,

após a face voltar para Zéfiro sopra-do-alto;

de fonte perene, corrente e límpida, \num{595}

três partes de água tira, e lança quarta de vinho.

Ao sagrado grão de Deméter incita os escravos

a debulhar na primeira aparição da força de Órion\footnote{Em torno de 20 de junho.}

num espaço ventilado, uma eira nivelada.

Medido, traze-o direito em vasilhas. E após \num{600}

todo o sustento, trancado, pores dentro de casa,

empregado sem casa\footnote{``Empregado\ldots{}'': outra interpretação é ``afasta o empregado de tua casa''.} arranja e criada sem filho

procura, eu te peço: problema é criada que aleita;\footnote{``Que aleita'': literalmente ``com bezerro a tiracolo''.}

e cuida do cão dente-afiado --- não poupes na comida ---

para nunca varão sono-diurno arrancar teus bens. \num{605}

Traze para dentro ração e limpadura, a fim de bastar

o ano todo para os bois e as mulas. E depois,

escravos refresquem seus joelhos e soltem os bois.

Quando Órion e Sírio chegarem ao meio\footnote{Meados de setembro.}

do céu, e a Arcturo vir Aurora dedos-róseos, \num{610}

Perses, colhe todo cacho de uva e leva p'ra casa,

expõe-nos ao sol por dez dias e dez noites,

por cinco, na sombra, e verte em vasilhas no sexto

o dom de Dioniso muito-júbilo. E quando

as Plêiades, as Híades e a força de Órion \num{615}

se puserem, nisso então te lembra da lavra

na estação: que o ano\footnote{``Ano'' --- talvez mais especificamente ``ano rural'' --- traduz
\emph{pleiōn}, mas é incerto que esse sentido já fosse corrente na época
de Hesíodo; como se trata de algo dentro do solo, ``semente, cereal'' é
outra opção. Seja como for, parece haver um trocadilho com \emph{pleōn /
ploos} (``navegar''/''navegação''), o tema da nova seção.} à terra se adeque.

E se te toma o desejo pela encrespada navegação:

quando as Plêiades, da força ponderosa de Órion,\footnote{Novembro.}

fogem e caem no mar embaciado, \num{620}

então grassam rajadas de todos os ventos;

então não mais o barco mantenhas no mar vinoso

mas, cônscio, trabalha a terra como te peço;

puxa teu barco à terra e o abriga com pedras

em volta para conterem o ímpeto dos ventos de úmido \qb{}sopro, \num{625}

e drena a água para a chuva de Zeus não o apodrecer.

Todo o material, trancado, põe em tua fazenda,

após dispor, bem ordenadas, as asas da nau cruza-mar;

e o leme bem-feito sobre a fumaça pendura.

Tu mesmo espera até vir o tempo da navegação; \num{630}

então nau veloz puxa ao mar e nela carga

arranjada apronta para lucro granjear à fazenda,

como meu e teu pai, Perses, seu grande tolo,

com barcos navegava, carente de vida nobre:

ele um dia também veio aqui após cruzar muito mar, \num{635}

tendo deixado a Cime eólia em negra nau,

não fugindo de abastança, riqueza e fortuna,

mas da vil pobreza, a que Zeus dá aos homens.

Fixou-se perto do Hélicon em miserável vilarejo,

Ascra, ruim no inverno, cruel no verão, boa jamais. \num{640}

Tu, Perses, sê cônscio dos trabalhos

todos na estação, sobremodo os da navegação.

Louva nau pequena, e em grande põe a carga:

carga, maior, lucro --- sobre lucro --- maior

será, se os ventos contiverem suas rajadas vis. \num{645}

Quando ao comércio voltares teu ânimo insano

e quiseres de dívidas fugir e da fome detestável,

irei te mostrar as medidas do mar ressoante,

de modo algum sábio em navegação ou barcos.

Pois nunca, com barco, naveguei pelo amplo mar, \num{650}

exceto de Áulis à Eubeia,\footnote{O trecho em questão é de 65m., o que aponta para certa comicidade na passagem.} onde um dia os aqueus,

esperando no inverno, grande tropa reuniram

para ir da sacra Hélade a Troia belas-mulheres.

Lá eu, atrás de prêmios pelo aguerrido Anfidamas,

até Cálquis cruzei --- em profusão esses anunciados \num{655}

prêmios fixaram os filhos do enérgico ---, onde afirmo eu

com um canto ter vencido e trazido trípode orelhuda.

Quanto a mim, essa dediquei às Musas do Hélicon,

onde no início me puseram na via do canto soante.

Tanta é minha experiência em naus muito-prego; \num{660}

mesmo assim direi a mente de Zeus porta-égide,

pois as Musas me ensinaram a cantar canto excepcional.

Por cinquenta dias depois do solstício,\footnote{Do final de junho até agosto. ``Por volta do quinquagésimo dia após o solstício'' também é uma tradução possível.}

quando chega ao fim o verão, estação da fadiga,

para os mortais é hora de navegar: nem ao barco \num{665}

destroçarias, nem aos varões aniquilaria o mar,

se, de propósito, nem Posêidon treme-terra

ou Zeus, rei dos imortais, quiserem destruir,

pois neles igualmente está o termo de males e o de bens.

Quando as brisas são definidas, e o mar, seguro, \num{670}

nisso, tranquilo, confia a nau veloz aos ventos,

puxa-a ao mar e nela põe toda a carga.

Apressa-te ao máximo em retornar de novo a casa

e não esperes vinho novo e a chuva outonal,\footnote{Os versos 674--77 refererem-se à parte final de setembro.}

o inverno chegando e as terríveis rajadas de Noto, \num{675}

ele que agita o oceano, acompanhando a chuva de Zeus

abundante no outono, e deixa o mar terrível.

Há outra navegação, primaveril, para os homens:\footnote{Final de abril.}

quando chega a hora --- o corvo, ao andar, tão grande

pegada faz quanto ao homem parece a folha \num{680}

no alto da figueira ---, nisso o mar é navegável;

é primaveril essa navegação. Eu não a ela

louvaria pois não compraz meu espírito;

é capciosa: difícil escapares de um mal; também isso,

porém, homens fazem com mente ignorante, \num{685}

pois bens são vida para os pobres mortais.

Terrível é morrer entre as ondas; mas te peço:

pondera no juízo tudo isso como exponho.

E nas cavas naus não ponhas todo o sustento,

mas deixa a maior parte, e a menor, carrega: \num{690}

é terrível, entre as ondas do mar, atingir desgraças

e, terrível, se para cima do carro ergueres peso brutal,

romperes o eixo, e a carga se degradar.

Atenta às medidas; oportunidade\footnote{``Oportunidade'' traduz \emph{kairos}, a mais antiga aparição do
termo conhecida por nós. Por isso mesmo, seu sentido na passagem depende
de interpretação. Acredito que ele se refira, ao mesmo tempo, ao momento
exato em que tanto as atividades agrícolas quanto as de navegação devem
ser realizadas no que diz respeito ao contexto imediato do conselho que
segue.} é o melhor em tudo.

Na tua hora, faze conduzir esposa à fazenda, \num{695}

nem muitos anos te faltando para os trinta,

nem com muitos a mais: essas te são bodas na hora;

e a mulher deve ser moça por quatro, e no quinto, casar.

Desposa uma virgem para ensinares hábitos de afeição,

e desposa em especial uma que more perto de ti \num{700}

após olhar tudo ao redor: não cases alegrando vizinhos.\footnote{No caso de a esposa ser infiel.}

Nada melhor conquista o varão que uma esposa

boa; que uma ruim nada há mais arrepiante,

a embosca-jantar, que ao esposo, embora altivo,

queima sem tição e o entrega à crua velhice.\footnote{Versos 704--5: uma das inúmeras passagens bem-humoradas de Hesíodo; ela faz do
marido uma refeição, mas crua. Para alguns, alusão ao modo como
Agamêmnon, no canto 11 da \emph{Odisseia}, conta a Odisseu como foi
morto por Clitemnestra.} \num{705}

Sê bem atento ao olhar\footnote{Esse olhar (\emph{opis}) diz respeito a uma punição retributiva.} dos deuses ditosos.

Não trates o camarada igual ao irmão;

se o tratares, não lhe faças mal primeiro

nem mintas pela graça da língua; se ele iniciar,

dizendo-te palavra detestável ou agindo, \num{710}

cônscio pune-o com o dobro; se de volta

for levado à amizade e quiser dar compensação,

aceita: reles o varão que ora um, ora outro seu amigo

torna; que em nada a mente infame tua aparência.

Que não te chamem muito-hóspede nem sem-hóspede, \num{715}

nem companheiro de vis nem vituperador de bons.

Nunca ouses por sua nefasta pobreza tira-vida

insultar um varão, dom dos ditosos sempre vivos.

O melhor, entre os homens, é o tesouro da língua

parca, e o máximo de graça está na comedida: \num{720}

se falares algo vil, logo tu mesmo mais ouvirás.

E não seja destemperado em banquete muito-hóspede:

na comunhão, a graça é máxima, a despesa, mínima.

Nunca, na aurora, libes a Zeus com vinho fulgente

tendo mãos não lavadas, nem aos outros imortais, \num{725}

pois não atendem e cospem as preces.

Não urines de pé voltado para o sol;

quando se pôr, cônscio, até ele raiar,

não o faças desnudado: as noites são dos ditosos. \num{730}

Nem na via nem fora dela, caminhando, urines; \num{729}

agacha-se o varão divino, versado no inteligente,

ou se aproxima do muro de pátio bem cercado.

Nem as vergonhas, salpicado de esperma, em casa

exibas nas proximidades do fogo-lar, mas evita.

Nem ao retornar de um funeral mau-agouro \num{735}

semeies prole, mas de um banquete de imortais.

Nem dos rios permanentes a água belo-fluxo

cruzes com os pés antes de rezar, mirando a bela corrente,

as mãos tendo lavado com água lúcida e benquista.

Quem um rio atravessa sem lavar o mal e as mãos, \num{740}

deuses se indignam contra ele e dão aflições no futuro.

Nem da cinco-galhos, no farto banquete dos deuses,

cortes o seco do verdejante com fúlgido ferro.\footnote{Metáfora vegetal para uma interdição relativa ao corte das unhas.}

Nem nunca ponhas jarra sobre ânfora\footnote{Enquanto se bebe, a jarra é usada para tirar vinho da ânfora e servi-lo.}

enquanto bebem: destino ruinoso para isso ocorre. \num{745}

Nem, ao fazer a casa, a deixes rugosa em cima,

para um corvo, grasnando, lá pousado não crocitar.

Nem tires de caldeirão com pés, não consagrado,

para comer ou se lavar, pois nisso recai pena.

Nem assentes no que é imóvel,\footnote{Significa ``túmulo'' mas também pode ser interpretado como ``altar''.} pois não é melhor, \num{750}

um filho de doze dias, o que torna o varão desviril,

nem um de doze meses: ocorre igual evento.

Nem com água de mulher limpe o corpo

o varão: por um tempo, também nisso recairá triste

pena. Nem, ao topar sacrifícios chamejantes, censura \num{755}

o que é consumido:\footnote{A saber, pelo fogo.} também com isso se indigna o deus.

Nem nas bocas dos rios que seguem ao mar

nem em fontes urines, mas de todo evita,

nem neles soltes ar, pois isso não é melhor.\footnote{``Não defeques'' é outra interpretação possível.}

Age assim; e evita a assombrosa reputação dos mortais: \num{760}

a reputação vil é leve para alguém erguer,

muito fácil, difícil de suportar e dura de afastar.

Reputação nunca de todo perece, qualquer que muitas

gentes reputarem: também ela é certo deus.

Os dias que vêm de Zeus, atenta a eles, bem e \qb{}ordenadamente, \num{765}

e os aponta aos escravos: o trigésimo do mês é o melhor

para vistoriar os trabalhos e distribuir a ração,

quando as gentes, julgando a verdade, o\footnote{``O'' refere-se ao 30º dia, mas não é explícito no grego. Outra
possibilidade de construção é deslocar o verso para depois de 769 e
traduzi-lo por ``se as gentes, julgando a verdade, os observam''
(Ercolani).} observam.

Estes são os dias que vêm do astucioso Zeus:

começando, dia sacro, o primeiro, o quarto, o sétimo --- \num{770}

nesse Leto gerou Apolo espada-de-ouro ---,

o oitavo e o nono. Todavia, dois dias do mês

crescente se destacam para a faina de trabalhos mortais,

o dia onze e o dia doze: ambos são ótimos

para tosquiar ovelhas e recolher o grão que alegra; \num{775}

e o dia doze é muito melhor que o dia onze,

pois nele trama seus fios a aranha voa-alto

em pleno dia, quando a sábia recolhe o monte:\footnote{A formiga (que no verão junta alimento para consumir no inverno).}

nele mulher deveria armar urdidura e pôr-se a obrar.

O mês crescendo, evita o décimo terceiro \num{780}

como início do plantio: é o melhor para plantas encanteirar.

O sexto do meio é às plantas bastante desfavorável\footnote{Os dias do mês são referidos de três formas, numericamente (do 1º
ao 30º dia), pela lua crescente e decrescente ou por uma divisão em três
partes, a 1ª começando no dia 1º, a 2ª no dia 14 (por exemplo, ``no nono
do meio'' quer dizer ``o nono dia no segundo período de dez dias'') e a
3ª no dia 21, quando, na Grécia histórica (em Hesíodo também?), os dias
eram referidos de forma decrescente: ``o quarto dia do que míngua''
(798), nesse caso, seria o 27º, não o 24º.}

e bom para nascerem varões; a moças não é favorável,

nem, em primeiro lugar, para nascer nem para casar.

Nem o primeiro sexto dia para uma menina nascer \num{785}

é adequado, mas para castrar cabritos e carneiros

e dia gentil para cercar um curral de rebanhos;

e p'ra nascerem varões é ótimo: ele amaria enunciar \qb{}provocações,

mentiras, contos solertes e conversas escusas.

No oitavo do mês, javali e touro muito-mugido \num{790}

castra, e no décimo segundo, mulas robustas.

No grande vigésimo, em pleno dia, homem sábio

nasça: na mente será muito arguto.

P'ra nascerem varões é ótimo o décimo; meninas, o quarto

no meio: nesse, ovelhas, lunadas vacas trôpegas, \num{795}

cão dente-afiado e mulas robustas,

sobre esses põe a mão e amansa. Atenta, no ânimo,

a evitar o quarto dia do que míngua e do que cresce

com aflições come-ânimo: é um dia assaz qualificado.

No quarto do mês, faze conduzir a noiva para casa, \num{800}

após discernir as aves, as melhores para esse feito.

Evita os dias cinco, pois duros e terríveis:

dizem que no quinto as Erínias acudiram\footnote{O 5º dia de cada parte? O dia 5 de cada mês? A expressão grega deveria ser traduzida pelo singular ``o dia cinco''?}

Jura ao nascer, desgraça aos perjuros, rebento de Briga.

No sétimo do meio, o sagrado grão de Deméter \num{805}

muito bem observa e na eira nivelada

joeira-o, e corte o lenhador madeira para o tálamo

e muita lenha de barco que seja adequada a naus.

No quarto,\footnote{Provavelmente deve-se subentender ``no quarto do meio''.} começa a construir naus pequenas.

No nono do meio, o dia é melhor no entardecer; \num{810}

o primeiro nono é de todo inofensivo aos homens,

pois é ótimo para serem plantados e gerados

homem e mulher, e nunca é um dia de todo ruim.

Poucos sabem que o três vezes nono é o melhor dia do mês\footnote{Provavelmente o dia 27, mas o 29 também é uma boa possibilidade (os
outros ``nonos'' sendo o 9º e o 19º dia)}

para iniciar um pito,\footnote{Começar o armazenamento em um cântaro vazio.} por o jugo no pescoço \num{815}

de bois, mulas e cavalos pés-ligeiros

e veloz nau muito-calço ao mar vinoso

puxar; poucos o chamam pelo nome verdadeiro.

No quarto do meio, abre um pito: mais que todos, dia

sagrado. Poucos sabem ser o vigésimo primeiro o melhor \num{820}

no momento da aurora; no entardecer, é pior.

Esses dias são de grande valia aos mortais;

os outros são incertos, sem destino, nada trazem.

Cada um louva dia distinto, e poucos sabem.

Ora um dia é uma sogra, ora uma mãe. \num{825}

Nisto é venturoso e afortunado, quem, tudo isso

conhecendo, trabalha de nada culpado para com imortais,

discernindo aves e evitando transgressões.

\endgroup