\chapter*{}

\begingroup\parindent=0em

Pelas Musas do Hélicon\footnote{Montanha próxima ao vilarejo de Ascra, na Beócia, mencionado
em \emph{Trabalhos e dias} como a localidade para onde emigrara o pai do
poeta.} comecemos a cantar,

elas que o Hélicon ocupam, monte grande e numinoso,

e em volta de fonte violácea com pés macios

dançam,\footnote{As Musas dançam em conjunto como um coro feminino, prática
músico"-ritual comum em várias ocasiões sócio"-religiosas específicas nas
comunidades gregas arcaicas.} e do altar do possante Cronida;

tendo a pele delicada no Permesso banhado, \num{5}

na Fonte do Cavalo ou no Olmeio numinoso,

no cimo do Hélicon compõem danças corais

belas, desejáveis, e fluem com os pés.

\quad{}De lá se lançando, ocultas por densa neblina,

de noite avançavam, belíssima voz emitindo, \num{10}

louvando Zeus porta-égide, a soberana Hera

argiva, que pisa com douradas sandálias,

a filha de Zeus porta-égide, Atena olhos-de-coruja,\footnote{Olhos-de-coruja é provavelmente o sentido cultual original desse
epíteto, que, na época histórica, em algum momento passou a ser
reinterpretado como ``com olhar brilhante (glauco)''.}

Febo\footnote{Epíteto de Apolo de origem desconhecida, talvez ligado à luz
ou à pureza.} Apolo e Ártemis verte-setas,

Posêidon, Treme-Solo sustém-terra,\footnote{Epítetos de Posêidon.} \num{15}

respeitada Norma e Afrodite olhar-vibrante,\footnote{Não há segurança sobre o sentido do epíteto de Afrodite; ``de olhos
negros'' é outra possibilidade.}

Juventude coroa-dourada e a bela Dione,\footnote{Dione: em Homero, é a mãe de Afrodite, mas não em Hesíodo.}

Leto, Jápeto e Crono curva-astúcia,

Aurora, o grande Sol e a reluzente Lua,

Terra, o grande Oceano e a negra Noite, \num{20}

e a sacra linhagem dos outros imortais sempre vivos.

\quad{}Sim, então essas a Hesíodo o belo canto ensinaram,\footnote{A arte de cantar em geral e não um canto específico.}

quando apascentava cordeiros sob o Hélicon numinoso.

Este discurso, primeiríssimo ato, dirigiram-me as deusas,

as Musas do Olimpo, filhas de Zeus porta-égide: \num{25}

``Pastores rústicos, infâmias vis, ventres somente,

sabemos falar muito fato enganoso como genuíno,

e sabemos, quando queremos, proclamar verdades''.

Assim falaram as filhas do grande Zeus, as \qb{}palavra-ajustada,\footnote{``Palavra-ajustada'' traduz \emph{artiepēs}, que na \emph{Ilíada}
22.281 tem sentido negativo (Aquiles censura a ladina manipulação
discursiva de Heitor).}

e me deram o cetro,\footnote{O cetro costuma ser associado a Zeus e a reis, mas, como aqui é de
um loureiro, o vínculo com Apolo também é possível.} galho vicejante de louro, \num{30}

após o colher, admirável; e sopraram-me voz

inspirada para eu glorificar o que será e foi,

pedindo que louvasse a linhagem dos ditosos sempre vivos

e a elas mesmas primeiro e por último sempre cantasse.

\quad{}Mas por que disso falo em torno do carvalho e da pedra?\footnote{O uso que Hesíodo faz dessa expressão é controverso; independente do
contexto (poético), uma análise comparativa (indo-europeia) propõe que o
sentido da fórmula utilizado aqui é ``de forma geral, de tudo um
pouco''. Hesíodo, portanto, se perguntaria: ``por que divago''?} \num{35}

Ei tu, comecemos pelas Musas, que para Zeus pai

cantam e deleitam sua grande mente no Olimpo,

dizendo o que é, o que será e o que foi antes,

harmonizando com o som, e, incansável, flui sua voz

das bocas, doce; sorri a morada do pai \num{40}

Zeus altissoante com a voz de lírio das deusas,

irradiante; ressoam o cume do Olimpo nevado

e as casas dos imortais: elas, imorredoura voz emitindo,

dos deuses a respeitada linhagem primo glorificam no canto

dês o início, estes que Terra e amplo Céu pariram, \num{45}

e estes que deles nasceram, os deuses oferentes de bens;

na sequência, a Zeus, pai de deuses e homens,

que elas louvam ao iniciar e cessar o canto,

pois é o mais forte dos deuses e supremo em poder;

depois, a linhagem dos homens e dos poderosos Gigantes \num{50}

louvando, deleitam a mente de Zeus no Olimpo

as Musas do Olimpo, filhas de Zeus porta-égide.

\quad{}Pariu-lhes na Piéria,\footnote{Piéria: região logo ao norte do Olimpo.} após se unir ao pai, o Cronida,

Memória, regente das ladeiras de Eleuteros,

como esquecimento\footnote{Em grego, o par memória \emph{versus} esquecimento é marcado fonicamente (\emph{mnēmosunēe} x \emph{lēsmosunēō}).} de males e suspensão de afãs. \num{55}

Por nove noites com ela se uniu o astucioso Zeus

longe dos imortais, subindo no sacro leito;

mas quando o ano chegou, e as estações deram a volta,

os meses finando, e muitos dias passaram,

ela gerou nove filhas concordes, que do canto \num{60}

no peito se ocupam com ânimo sereno,

perto do mais alto pico do Olimpo nevado:

lá têm reluzentes pistas de dança e belas moradas,

e junto delas as Graças e Desejo habitam

em festas; pela boca amável voz emitindo, \num{65}

cantam e dançam e os costumes e usos sábios de todos

os imortais glorificam, amável voz emitindo.

Nisso iam ao Olimpo, gozando a bela voz,

com música imortal; rugia a negra terra em volta

ao cantarem, e amável ressoo subia dos pés \num{70}

ao retornarem a seu pai: ele reina no céu,

ele mesmo segurando trovão e raio chamejante,

pois no poder venceu o pai Crono; bem cada coisa

apontou aos imortais por igual e indicou suas honrarias.

\quad{}Isso cantavam as Musas, que têm morada olímpia, \num{75}

as nove filhas geradas do grande Zeus,

Glória, Aprazível, Festa, Cantarina,

Dançapraz, Saudosa, Muitacanção, Celeste

e Belavoz: essa é a superior entre todas.

Pois essa também a respeitados reis acompanha. \num{80}

Quem quer que honrem as filhas do grande Zeus

e o veem ao nascer, um dos reis criados por Zeus,

para ele, sobre a língua, vertem doce orvalho,

e da boca dele fluem palavras amáveis; as gentes

todas o miram quando decide entre sentenças \num{85}

com retos juízos: falando com segurança,

de pronto até disputa grande interrompe destramente;

por isso reis são sensatos, pois às gentes

prejudicadas completam na ágora ações reparatórias

fácil, induzindo com palavras macias; \num{90}

ao se mover na praça, como um deus o propiciam

com respeito amável, e destaca-se na multidão.

\quad{}Tal é a sacra dádiva das Musas aos homens.

Pois das Musas, vê, e de Apolo lança-de-longe

vêm os varão cantores sobre a terra e os citaredos, \num{95}

e de Zeus, os reis: este é afortunado, quem as Musas

amam; doce é a voz que flui de sua boca.

Pois se alguém, com pesar no ânimo recém-afligido,

seca no coração, angustiado, mas um cantor,

assistente das Musas, glórias de homens de antanho \num{100}

e deuses ditosos, que ocupam o Olimpo, cantar,

de pronto ele esquece as tristezas e de aflição alguma

se lembra: rápido as desviam os dons das deusas.

\quad{}Felicidades, filhas de Zeus, e dai canto desejável;

glorificai a sacra linhagem dos imortais sempre vivos, \num{105}

os que de Terra nasceram, do estrelado Céu

e da escura Noite, e esses que criou o salso Mar.

Dizei como no início os deuses e Terra nasceram,

os Rios e o Mar sem-fim, furioso nas ondas,

os Astros fulgentes e o amplo Céu acima, \num{110}

e esses que deles nasceram, os deuses oferentes de bens;

como dividiram a abastança, repartiram as honrarias,

e também como no início ocuparam o Olimpo de muitos \qb{}vales.

Disso me narrem, Musas que têm morada olímpia,

do princípio, e dizei qual deles primeiro nasceu. \num{115}

\quad{}Bem no início, Abismo\footnote{Abismo = \emph{Khaos}, segundo a interpretação mais aceita, um
vazio sem forma, e não uma matéria indistinta.} nasceu; depois,

Terra largo-peito, de todos assento sempre firme,

dos imortais que possuem o pico do Olimpo nevado

e o Tártaro brumoso no recesso da terra largas-rotas,

e Eros, que é o mais belo entre os deuses imortais,\footnote{A leitura mais aceita é que Terra e Eros são divindades, e o
Tártaro, um espaço abaixo da superfície terrestre. Alguns optam pelo
Tártaro, nesta passagem, como uma divindade, colocando uma vírgula no
final do verso 118.} \num{120}

o solta-membros, e de todos os deuses e todos os homens

subjuga, no peito, mente e desígnio refletido.

\medskip

De Abismo nasceram Escuridão\footnote{Escuridão = \emph{Erebos}, lugar escuro, amiúde associado ao Hades.} e a negra Noite;

de Noite, então, Éter e Dia nasceram,

que gerou, grávida, após com Escuridão unir-se em amor. \num{125}

\medskip

Terra primeiro gerou, igual a ela,

o estrelado Céu, a fim de encobri-la por inteiro

para ser, dos deuses venturosos, assento sempre firme;

gerou as enormes Montanhas, refúgios graciosos de deusas,

as Ninfas, que habitam montanhas matosas; \num{130}

pariu também o ruidoso pélago, furioso nas ondas,

Mar, sem amor desejante; e então

deitou-se com Céu e pariu Oceano fundo-redemunho,

Coio, Creio, Hipérion,\footnote{Na poesia grega arcaica, Hipérion sempre aparece em conexão com o
Sol.} Jápeto,

Teia, Reia, Norma, Memória, \num{135}

Febe coroa-dourada e a atraente Tetís.

Depois deles, o mais novo nasceu, Crono curva-astúcia,

o mais fero dos filhos; e odiou o viçoso pai.

\quad{}Então gerou os Ciclopes, que têm brutal coração,

Trovão, Relâmpago e Clarão ânimo-ponderoso, \num{140}

eles que o trovão deram a Zeus e fabricaram o raio.

Quanto a eles, de resto assemelhavam-se aos deuses,

mas um só olho no meio da fronte jazia;

Ciclopes eram seu nome epônimo, porque deles

circular o olho, um só, que na fronte jazia;\footnote{No grego, o jogo etimológico é ainda mais saliente
(\emph{Kuklōpes} --- \emph{kukloterēs}).} \num{145}

energia, força e engenho havia em seus feitos.

\quad{}E outros então de Terra e Céu nasceram,

três filhos grandes e ponderosos, inomináveis,

Coto, Briareu e Giges, rebentos insolentes.

Cem braços de seus ombros se lançavam, \num{150}

inabordáveis, e cabeças, em cada um, cinquenta

dos ombros nasceram sobre os membros robustos;

a energia imensa era brutal na grande figura.

\quad{}Pois tantos quantos de Terra e Céu nasceram,

os mais feros dos filhos, por seu pai foram odiados \num{155}

desde o princípio: assim que nascesse um deles,

a todos ocultava, não os deixava à luz subir,

no recesso de Terra, e com o feito vil se regozijava

Céu; ela dentro gemia, a portentosa Terra,

constrita, e planejou ardiloso, nocivo estratagema. \num{160}

De pronto criou a espécie do cinzento adamanto,

fabricou grande foice e mostrou-a aos caros filhos.

\quad{}Atiçando-os, disse, agastada no caro coração:

``Filhos meus e de pai iníquo, caso quiserdes,

obedecei: nos vingaríamos da vil ofensa do pai \num{165}

vosso, o primeiro a armar feitos ultrajantes''.

\quad{}Assim falou; e o medo pegou a todos, e nenhum deles

falou. Com audácia, o grande Crono curva-astúcia

de pronto com um discurso respondeu à mãe devotada:

``Mãe, isso sob promessa eu cumpriria, \num{170}

o feito, pois desconsidero o inominável pai

nosso, o primeiro a armar feitos ultrajantes''.

\quad{}Assim falou; muito alegrou-se no juízo a portentosa \qb{}Terra.

Escondeu-o numa tocaia, colocou em suas mãos

a foice serridêntea e instruiu-o em todo o ardil. \num{175}

Veio, trazendo a noite, o grande Céu, e em torno de Terra

estendeu-se, desejoso de amor, e estirou-se em toda

direção. O outro, o filho, da tocaia a mão esticou,

a esquerda, e com a direita pegou a foice portentosa,

grande, serridêntea, os genitais do caro pai \num{180}

com avidez ceifou e lançou para trás, que fossem

embora. Mas, ao escapar da mão, não ficaram sem efeito:

tantas gotas de sangue quantas escapuliram,

Terra a todas recebeu; após os anos volverem-se,

gerou as Erínias brutais e os grandes Gigantes, \num{185}

luzidios em armas, com longas lanças nas mãos,

e as Ninfas que chamam Mélias\footnote{Mélias, ninfas ligadas a árvores.} na terra sem-fim.

Os genitais, quando primeiro os cortou com adamanto,

lançou-os para baixo, da costa ao mar encapelado,

levou-os o pélago muito tempo, e em volta, branca \num{190}

espuma lançou-se da carne imortal; e nela moça

foi criada: primeiro da numinosa Citera\footnote{Em Citera, ilha na ponta \textsc{so} do Peloponeso, ficava um templo de Afrodite. } achegou-se,

e então de lá atingiu o oceânico Chipre.\footnote{É em Chipre que os gregos costumavam representar a origem de Afrodite; lá ficavam seus centros cultuais mais importantes.}

E saiu a respeitada, bela deusa, e grama em volta

crescia sob os pés esbeltos: a ela Afrodite \num{195}

espumogênita e Citereia bela-coroa

chamam deuses e varões, porque na espuma\footnote{Jogo etimológico entre \emph{Aphrodite} e \emph{aphros} (``espuma'').}

foi criada; Citereia, pois alcançou Citera;

cipriogênita, pois nasceu em Chipre cercado-de-mar;

e ama-sorriso, pois da genitália surgiu.\footnote{Jogo etimológico entre \emph{philommeidēs} (``ama-sorriso'') e
\emph{mēdea} (``genitália masculina''), homófono de um termo que
significa ``planos ardilosos'', cujo radical é o mesmo do verbo
``armar'' (v. 166).} \num{200}

Eros acompanhou-a e Desejo a seguiu, belo,

quando ela nasceu e dirigiu-se à tribo dos deuses.

Tem esta honra desde o início e granjeou

quinhão entre homens e deuses imortais,

flertes de meninas, sorrisos e farsas, \num{205}

delicioso prazer, amor e afeto.

\quad{}Àqueles o pai chamava, por apelido, Titãs,

o grande Céu brigando com filhos que ele mesmo gerou;

dizia que, iníquos, se esticaram para efetuar enorme

feito, pelo qual haveria vingança\footnote{Jogo etimológico entre \emph{Titēnas} (``Titãs''),
\emph{titainontas} (de \emph{titainein}, ``estender, esticar'') e
\emph{tisis} (``vingança'').} depois no futuro. \num{210}

\quad{}E Noite pariu a medonha Sina, Perdição negra

e Morte, e pariu Sono, e pariu a tribo de Sonhos;

sem se deitar com um deus, pariu a escura Noite.\footnote{``Escura'' (\emph{erebennē}) parece remeter a Escuridão (\emph{Erebos}), parceiro sexual de Noite no início da cosmogonia.}

Em seguida, Pecha e aflitiva Agonia,

e Hespérides, que, para lá do glorioso Oceano, de belas \num{215}

maçãs de ouro cuidam e das árvores que trazem o fruto;

e gerou as Moiras e Perdições castigo-implacável,

Fiandeira, Sorteadora e Inflexível, elas que aos mortais,

ao nascerem, lhes concedem bem e mal como seus,

e elas que alcançam transgressões de homens e deuses \num{220}

e nunca desistem, as deusas, da raiva assombrosa

até retribuir com maligna punição àquele que errar.\footnote{A maioria dos críticos considera os versos 218--19
(905--6) como interpolados. Preferi considerar que
218--19 referem-se às Moiras, e 220-22, às Perdições.}

Também pariu Indignação, desgraça aos humanos mortais,

a ruinosa Noite; depois pariu Farsa e Amor

e a destrutiva Velhice, e pariu Briga ânimo-potente. \num{225}

\quad{}E a odiosa Briga pariu o aflitivo Labor,

Esquecimento, Fome, Aflições lacrimosas,

Batalhas, Combates, Matanças, Carnificinas,

Disputas, Embustes, Contos, Contendas,

Má-Norma e Desastre, vizinhas recíprocas, \num{230}

e Jura, ela que demais aos homens mortais

desgraça se alguém, de bom grado, perjura.

\medskip

A Nereu, probo e verdadeiro, gerou Mar,

ao mais velho dos filhos: chamam-no ``ancião''

porque é veraz e gentil e das regras \num{235}

não se esquece, mas planos justos e gentis conhece;

e então ao grande Taumas e ao orgulhoso Fórcis,

a Terra unido, e a Cetó bela-face

e Amplaforça com ânimo de adamanto no íntimo.

E de Nereu nasceram numerosas filhas de deusas, \num{240}

no mar ruidoso, com Dóris belas-tranças,\footnote{Dóris: o seu nome também remete à raiz verbal de ``dar'', elemento presente em algumas de suas filhas.}

filha do circular rio Oceano:

Propele, Completriz, Salva, Anfitrite,

Tétis, Dadivosa, Calmaria, Azúlis,

Ondacélere, a veloz Gruta, a desejável Festa,\footnote{Alguns críticos (Mazon, Ricciardelli) defendem, para a segunda
metade do verso, ``\ldots{} Gruta, Veloz e a desejável ``Marinha''.} \num{245}

Admiradíssima, Saudosa, Belarrixa braço-róseo,

a graciosa Amelada, Enseada, Resplende,

Doadora, Inicia, Levadora, Poderosa,

Ilhoa, Costeira, Primazia,

Dóris, Tudovê, a benfeita Galateia, \num{250}

a desejável Hipocorre, Hipomente braço-róseo,

Seguronda, que ondas no mar embaciado

e rajadas de ventos bravios com Cessonda

e Anfitrite de belo tornozelo fácil apazigua,

Ondina, Praiana, Mandamar bela-coroa, \num{255}

Partilhazúlis ama-sorriso, Viajamar,

Juntapovo, Juntabem, Cuidapovo,

Espirituosa, Cônscia, Compensadora,

Rebanhosa, desejável no físico, impecável na forma,

Areiana, graciosa de corpo, a divina Forcequina, \num{260}

Ilheia, Benconduz, Normativa, Previdente

e Veraz, que tem o espírito do pai imortal.

Essas nasceram do impecável Nereu,

cinquenta filhas, peritas em impecáveis trabalhos.

\quad{}E Taumas a filha de Oceano funda-corrente \num{265}

desposou, Brilhante; e ela pariu Íris veloz

e as Hárpias belas-tranças, Tempesta e Voaveloz,

que, com rajadas de ventos e aves, junto seguem

com asas velozes, pois disparam, altaneiras.

\medskip

E Cetó pariu para Fórcis velhas bela-face, \num{270}

grisalhas de nascença, que chamam Velhas

os deuses imortais e homens que andam na terra,

Penfredó belo-peplo, Enió peplo-açafrão

e as Górgonas, que habitam para lá do glorioso Oceano

no limite, rumo à noite, onde estão as Hespérides \qb{}clara-voz --- \num{275}

Estenó, Euríale e Medusa, que sofreu o funesto:

esta era mortal, as outras, imortais e sem velhice,

as duas; e só junto a ela deitou-se Juba-Cobalto\footnote{Juba-Cobalto é Posêidon.}

num prado macio com flores primaveris.

Dela, quando Perseu a cabeça cortou do pescoço, \num{280}

p'ra fora pularam o grande Espadouro e o cavalo Pégaso.

Ele tinha esse epônimo pois pegado às fontes\footnote{O nome é ligado a \emph{pēgas}, ``fontes''.} de Oceano

nasceu, e aquele, com espada de ouro nas caras mãos.

Pégaso alçou vôo, após deixar a terra, mãe de ovelhas,

e dirigiu-se aos imortais; a casa de Zeus habita \num{285}

e leva trovão e raio ao astucioso Zeus.

E Espadouro gerou Gerioneu três-cabeças,

unido a Bonflux, filha do famoso Oceano:

eis que a esse matou a força de Héracles,

junto a bois passo-arrastado na oceânica Eriteia \num{290}

naquele dia em que tangeu os bois fronte-larga

até a sacra Tirinto, após cruzar o estreito de Oceano

e ter matado Orto e o pastor Euritíon

na quinta brumosa p'ra lá do famoso Oceano.

\quad{}Ela gerou outro ser portentoso, impossível, nem parecido \num{295}

com homens mortais nem com deuses imortais,

em cava gruta, a divina Équidna juízo-forte,

metade moça olhar-luzente, bela-face,

metade serpente portentosa, terrível e grande,

dardejante come-cru, sob os confins da numinosa terra. \num{300}

Lá fica sua caverna, para baixo, sob cava pedra,

longe de deuses imortais e homens mortais,

onde os deuses lhe atribuíram casa gloriosa p'ra morar.

\quad{}Ela fica nos Arimos\footnote{Não se sabe o que são (cadeia de montanhas? povo?) nem onde ficavam.} sob a terra, a funesta Équidna,

moça imortal e sem velhice para todos os dias. \num{305}

Com ela, dizem, Tifeu uniu-se em amor,

o violento, terrível e ímpio com a moça olhar-luzente;

ela, após engravidar, gerou rebentos juízo-forte.

Orto primeiro ela gerou, um cão para Gerioneu;

depois pariu o impossível, de todo impronunciável, \num{310}

Cérbero come-cru, o cão bronzissonante de Hades,

cinquenta-cabeças, insolente e brutal;

como terceiro, gerou Hidra, versada no funesto,

de Lerna, a quem nutriu a divina Hera alvo-braço,

com imenso rancor da força de Héracles. \num{315}

A ela matou o filho de Zeus com bronze impiedoso,

o filho de Anfitríon, com Iolau caro-a-Ares ---

Héracles --- pelos desígnios de Atena guia-tropa.

E ela\footnote{Não fica claro quem é ``ela'', Cetó, Hidra ou Équidna. ``Quimera'',
em grego, é ``cabra''.} pariu Quimera, que fogo indômito soprava,

terrível, grande, pé-ligeiro, brutal. \num{320}

Tinha três cabeças: uma, de leão olhar-cobiçoso,

outra, de cabra, a terceira, de cobra, brutal serpente.

Na frente, leão, atrás, serpente, no meio, cabra,

soprando o fero ímpeto do fogo chamejante.\footnote{Como esses versos repetem dois versos da \emph{Ilíada} e estão
--- ou parecem estar --- em contradição com os dois versos anteriores, são
deletados por diversos editores.}

A ela pegaram Pégaso e o valoroso Belerofonte. \num{325}

E ela\footnote{Não fica claro quem é ``ela'', Cetó, Quimera ou Équidna.} pariu a ruinosa Esfinge, ruína dos cadmeus,

após ser subjugada por Orto, e o leão de Nemeia,

do qual Hera cuidou, a majestosa consorte de Zeus,

e o alocou nos morros de Nemeia, desgraça dos homens.

Ele, lá habitando, encurralava a linhagem de homens, \num{330}

dominando Tretos, na Nemeia, e Apesas;

mas a ele subjugou o vigor da força de Héracles.

\quad{}Cetó, unida em amor a Fórcis, como o mais jovem

gerou terrível serpente, que nos confins da terra lúgubre,

nos grandes limites, guarda um rebanho todo de ouro. \num{335}

\medskip

E essa é a linhagem de Ceto e Fórcis.

E Tetís para Oceano pariu rios vertiginosos,

Nilo, Alfeios e Eridanos fundo-redemunho,

Estrímon, Maiandros e Istros bela-corrente,

Fásis, Resos e Aquelôo argênteo-redemunho, \num{340}

Nessos, Ródios, Haliácmon, Heptaporos,

Grenicos, Esepos e o divino Simoente,

Peneios, Hermos e Caícos bem-fluente,

grande Sangarios, Ládon e Partênios,

Euenos, Aldescos e o divino Escamandro. \num{345}

E pariu sacra linhagem de moças, que, pela terra,

a meninos tornam varões com o senhor Apolo

e com os rios, e de Zeus tem esse quinhão,

Persuasão, Indomada, Violeta e Brilhante,

Dóris, Sopé e a divinal Celeste, \num{350}

Equina, Famosa, Rósea e Bonflux,

Zeuxó, Gloriosa, Sapiente e Admiradíssima,

Plexaure, Galaxaure e a encantadora Dione,

Ovelheira, Veloz e Muitadádiva bela-aparência,

a atraente Lançadeira e Riqueza olho-bovino, \num{355}

Perseís, Iáneira, Acaste e Loira,

a apaixonante Pétrea, Potência e Europa,

Astúcia, Eurínome e Círcula peplo-açafrão,

Criseís, Ásia e a desejável Calipso,\footnote{Calipso, transliteração de \emph{Kalipso}, algo como ``Encobre''.}

Beladádiva, Fortuna, Tornoflux e Celereflux, \num{360}

e Estige, essa que é a superior entre todas.

Essas nasceram de Oceano e Tetís,

as moças mais velhas. Também muitas outras há:

três mil são as Oceaninas tornozelo-fino,

elas que, bem-espalhadas, terra e profundas do mar, \num{365}

todo lugar por igual, frequentam, filhas radiantes de deusas.

E tantos e distintos os rios que fluem estrepitantes,

filhos de Oceano, aos quais gerou a senhora Tetís:

deles, o nome de todos custa ao varão mortal narrar,

e estes o respectivo conhecem, os que moram no entorno. \num{370}

\quad{}E Teia ao grande Sol, à fúlgida Lua

e à Aurora, que brilha para todos os mortais

e aos deuses imortais que do amplo céu dispõem,

gerou-os, subjugada em amor por Hipérion.

E para Creio Euribie pariu, unida em amor, \num{375}

diva entre as deusas, o grande Estrelado, Palas

e Perses, que entre todos sobressaía pela sapiência.

Para Estrelado Aurora pariu ventos ânimo-potente,

o clareante Zéfiro, Bóreas rota-ligeira

e Noto, em amor a deusa com o deus deitada. \num{380}

Depois deles, Nasce-Cedo pariu Estrela da Manhã\footnote{``Nasce-Cedo'' = Aurora; ``Estrela da Manhã'' = \emph{Heōsphoros}, ``traz-aurora''.}

e astros fulgentes, com os quais o céu se coroa.

\quad{}E Estige, filha de Oceano, pariu, unida a Palas,

Emulação e Vitória linda-canela no palácio

e Poder e Força gerou, filhos insignes. \num{385}

Não fica longe de Zeus nem sua casa nem seu assento,

nem via por onde o deus na frente deles não vá,

mas sempre junto a Zeus grave-ressoo se assentam.

Pois assim Estige planejou, a Oceanina eterna,

no dia em que o relampejante olímpico a todos \num{390}

os deuses imortais chamou ao grande Olimpo,

e disse que todo deus que com ele combatesse os Titãs,

desse não arrancaria suas mercês, e cada um a honra

teria tal como antes entre os deuses imortais.

Disse que quem não obtivera honra e mercê devido a Crono, \num{395}

esse receberia honra e mercês, como é a norma.

Eis que veio por primeiro ao Olimpo a eterna Estige

com seus filhos devido aos projetos do caro pai;

a ela Zeus honrou e deu-lhe dons prodigiosos.

Pois dela fez a grande jura dos deuses, \num{400}

e a seus filhos, por todos os dias, tornou seus coabitantes.

Assim como prometera para todos, sem exceção,

realizou; e ele mesmo tem grande poder e rege.

\quad{}E dirigiu-se Foibe ao desejável leito de Coio;

então engravidou a deusa em amor pelo deus \num{405}

e gerou Leto peplo-negro, sempre amável,

gentil para com os homens e deuses imortais,

amável dês o início, a mais suave dentro do Olimpo.

E gerou a auspiciosa Astéria bom-nome, que um dia Perses

fez conduzir à grande casa para ser chamada sua esposa. \num{410}

\quad{}Ela engravidou e pariu Hécate, a quem, mais que a todos,

Zeus Cronida honrou; e deu-lhe dádivas radiantes

para ela ter porção da terra e do mar ruidoso.

Ela também partilhou a honra do céu estrelado,

e pelos deuses imortais é sumamente honrada. \num{415}

Também agora, quando um homem mortal

faz belos sacrifícios regrados para os propiciar,

invoca Hécate: bastante honra segue aquele,

fácil, de quem, benévola, a deusa aceita preces,

e a ele oferta fortuna, pois a potência está a seu lado. \num{420}

Tantos quantos de Terra e Céu nasceram

e granjearam honraria, de todos ela tem uma porção

e com ela o Cronida em nada foi violento nem usurpou

daquilo que granjeou entre os Titãs, primevos deuses,

mas possui como foi, dês o início, a divisão original. \num{425}

Nem, sendo filha única, tem menor porção de honra

e de mercês na terra, no céu e no mar,

mas ainda muito mais, pois Zeus a honra.

Para quem ela quiser, magnificente, fica ao lado e favorece:

na assembleia, entre o povo se destaca quem ela quiser; \num{430}

e quando rumo à batalha aniquiladora se armam

os varões, a deusa ao lado fica daquele a quem quer,

benevolente, vitória ofertar e glória estender.

Num julgamento senta-se junto a reis respeitáveis,

e valorosa é sempre que varões disputam uma prova: \num{435}

aí a deusa também fica ao lado deles e os favorece,

e, tendo vencido pela força e vigor, belo prêmio

ele fácil leva, alegre, e aos pais oferta a glória.

É valorosa ao se por junto a cavaleiros, aos que quer,

e para estes que trabalham o glauco encrespado \num{440}

e fazem prece a Hécate e a Treme-Solo ressoa-alto,\footnote{``Treme-Solo'' e ``ressoa-alto'' são epítetos de Posêidon e geralmente identificam o deus neste poema.}

fácil a deusa majestosa oferta muita presa,

e fácil a tira quando aparece, se no ânimo quiser.

Valorosa é com Hermes, nas quintas, no aumentar os bens:

rebanhos de gado, amplos rebanhos de cabras, \num{445}

rebanhos de ovelhas lanosas, se ela no ânimo quiser,

de poucos, os fortalece, e de muitos, torna menores.

Assim, embora sendo filha única da mãe,

entre todos os imortais é honrada com mercês.

O Cronida tornou-a nutre-jovem dos que, depois dela, \num{450}

com os olhos veem a luz de Aurora muito-observa.

Assim, dês o início é nutre-jovem, e essas, as honras.

\medskip

E Reia, subjugada por Crono, pariu filhos insignes,

Héstia, Deméter e Hera sandália-dourada,

e o altivo Hades, que sob a terra habita sua casa \num{455}

com coração impiedoso, e Treme-Solo ressoa-alto,

e o astuto Zeus, pai de deuses e homens,

cujo raio sacode a ampla terra.

A esses engolia o grande Crono, quando cada um

se dirigisse do sacro ventre aos joelhos da mãe, \num{460}

pensando isso para nenhum ilustre celeste,

um outro entre os imortais, obter a honraria real.

Pois escutara de Terra e do estrelado Céu

que lhe estava destinado ser subjugado por seu filho ---

embora mais poderoso, pelos desígnios do grande Zeus. \num{465}

Por isso não mantinha vigia cega, mas, observador,

engolia seus filhos; e a Reia dominava aflição inesquecível.

Mas quando iria a Zeus, pai de deuses e homens,

parir, nisso ela então suplicou aos caros genitores,

aos seus próprios, Terra e Céu estrelado, \num{470}

com ela planejarem ardil para, sem ser notada, parir

o caro filho e fazer Crono pagar às erínias\footnote{Erínias são espíritos de vingança.} do pai

e dos filhos que ele engolia, o grande Crono curva-astúcia.

Eles à cara filha ouviram bem e obedeceram

e lhe apontaram tudo destinado a ocorrer \num{475}

acerca do rei Crono e do filho ânimo-potente.

Enviaram-na a Lictos, à fértil região de Creta,

quando iria parir o mais novo dos filhos,

o grande Zeus; a esse recebeu a portentosa Terra

na ampla Creta para criar e alimentar. \num{480}

Lá ela chegou, levando-o pela negra noite veloz,

primeiro a Lictos; pegou-o nos braços e escondeu

em gruta rochosa, sob os recessos numinosos da terra,

na montanha Egeia, coberta de mato cerrado.

Em grande pedra pôs um cueiro e àquele o estendeu, \num{485}

ao grande senhor filho de Céu, rei dos deuses primevos.

Pegou-a então com as mãos e em seu ventre depositou,

o terrível, e não notou no juízo que para ele, no futuro,

ao invés da pedra seu filho invencível e sereno

ficou, quem logo o iria subjugar com força e braços, \num{490}

o despojaria de sua honra e entre os imortais regeria.

\quad{}Eis que celeremente ímpeto e membros insignes

do senhor cresceram; e após um ano passar,

ludibriado pela sugestão refletida de Terra,

sua prole regurgitou o grande Crono curva-astúcia, \num{495}

vencido pela arte e força do próprio filho.

Primeiro vomitou a pedra, que por último engolira;

a ela Zeus fixou na terra largas-rotas

na divina Pitó, embaixo nas reentrâncias do Parnasso,\footnote{Ou seja, em Delfos.}

sinal aos vindouros, assombro aos homens mortais. \num{500}

\quad{}E soltou os irmãos do pai de seus laços ruinosos,

filhos de Céu, que prendera o pai devido a cego juízo;

eles, pela boa ação, retribuíram com um favor,

e deram-lhe trovão, raio chamejante

e relâmpago: antes a portentosa Terra os mantivera ocultos; \num{505}

com o apoio desses, ele rege sobre mortais e imortais.

\quad{}E Jápeto a moça linda-canela, a Oceanina

Famosa, fez ser conduzida e subiu no leito comum.

Ela gerou-lhe, como filho, Atlas juízo-forte

e pariu Menoitio super-majestoso, Prometeu, \num{510}

o variegado astúcia-cintilante, e o equivocado Epimeteu;

um mal foi esse, dês o início, aos homens come-grão:

recebeu originalmente, modelada, uma mulher

moça. E ao violento Menoitio Zeus ampla-visão

à escuridão abaixo enviou ao acertá-lo com raio fumoso \num{515}

por causa de iniquidade e insolente virilidade.

Atlas sustém o amplo céu, sob imperiosa necessidade,

nos limites da terra ante as Hespérides clara-voz

parado, com a cabeça e incansáveis braços:

esse quinhão lhe atribuiu o astuto Zeus. \num{520}

Prendeu a grilhões Prometeu desígnio-variegado,

a laços aflitivos, pelo meio puxando um pilar.

Contra ele instigou águia asa-longa; essa ao fígado

imortal comia, e esse crescia por completo, igual,

de noite, o que de dia comeria a ave asa-longa. \num{525}

Eis que a ela o bravo filho de Alcmena linda-canela,

Héracles, matou, e afastou a praga vil

do filho de Jápeto e libertou-o das amarguras

não contra o olímpico Zeus que do alto rege,

para que o tebano Héracles tivesse fama \num{530}

ainda mais que no passado sobre o solo nutre-muitos.

Assim, respeitando-o, Zeus honrava o insigne filho;

embora irado, cessou a raiva que antes tinha,

pois desafiara os desígnios do impetuoso Cronida.

\quad{}De fato, quando deuses e homens mortais se distinguiam \num{535}

em Mecone, nisso grande boi, com ânimo resoluto,

Prometeu dividiu e dispôs, tentando enganar o espírito de \qb{}Zeus.

Pois, para um, carne e entranhas fartas em gordura

na pele colocou, escondendo no ventre bovino;

para os outros, brancos ossos do boi com arte ardilosa \num{540}

arrumou e dispôs, escondendo com branca gordura.

\quad{}Então lhe disse o pai de varões e deuses:

``Filho de Jápeto, insigne entre todos os senhores,

meu caro, que modo parcial de dividir as porções''.

\quad{}Assim provocou- Zeus, mestre em ideias imperecíveis; \num{545}

e a ele retrucou Prometeu curva-astúcia,

de leve sorriu e não esqueceu a arte ardilosa:

``Majestoso Zeus, maior dos deuses sempiternos,

dessas escolhe a que no íntimo o ânimo te ordena''.

\quad{}Falou ardilosamente; Zeus, mestre em ideias imperecíveis, \num{550}

atentou, não desatento ao ardil; olhou com males no ânimo

contra os homens mortais, os quais iriam se cumprir.

Com ambas as mãos, pegou a gordura branca

e irou-se no juízo, e raiva alcançou seu ânimo

quando viu brancos os ossos do boi, fruto da arte ardilosa. \num{555}

Daí, aos imortais as tribos de homens sobre a terra

queimam brancos ossos sobre altares fragrantes.

\quad{}Muito perturbado, disse-lhe Zeus junta-nuvens:

``Filho de Jápeto, supremo mestre em planos,

meu caro, pois não esqueceste a arte ardilosa''. \num{560}

\quad{}Assim falou, irado, Zeus, mestre em ideias imperecíveis.

Depois disso, então, da raiva sempre se lembrando,

não dava aos freixos o ímpeto do fogo incansável

para os homens mortais, que sobre a terra habitam.\footnote{Os versos 563--564 são problemáticos; uma pequena alteração poderia redundar em
``não dava o ímpeto do fogo incansável para os homens mortais (nascidos
das ninfas) dos freixos''.}

Mas a ele enganou o brioso filho de Jápeto \num{565}

ao roubar o clarão visto-ao-longe do fogo incansável

em cavo funcho-gigante: isso mordeu o ânimo

de Zeus troveja-no-alto, e enraiveceu-se em seu coração

ao fitar entre os homens o clarão visto-ao-longe do fogo.

De pronto, pelo fogo fabricou um mal para os homens: \num{570}

da terra modelou o gloriosíssimo Duas-Curvas,\footnote{Epíteto que identifica Hefesto.}

pelos desígnios do Cronida, a imagem de uma moça \qb{}respeitada.

A ela cinturou e adornou a deusa, Atena olhos-de-coruja,

com veste argêntea; cabeça abaixo um véu

adornado, com as mãos, fez pender, assombro à visão; \num{575}

em volta dela, coroas broto-novo de flores do prado,

desejáveis, pôs Palas Atena em sua a cabeça.\footnote{Os versos 576--577 são deletados por muitos editores (Marg, West); Most os mantém.}

Em volta dela, pôs coroa dourada na cabeça,

que ele próprio fizera, o gloriosíssimo Duas-Curvas,

ao labutar com as palmas, comprazendo ao pai Zeus. \num{580}

Nela muito adorno foi fabricado, assombro à visão,

tantos animais terríveis quantos nutrem terra e mar;

muitos desses nela pôs, e graça sobre todos soprou,

admiráveis, semelhantes a criaturas com voz.

\quad{}E após fabricar o belo mal pelo bem, \num{585}

levou-a aonde estavam os outros deuses e homens,

ela feliz com o adorno da Olhos-de-Coruja de pai \qb{}ponderoso.\footnote{Dois epítetos comuns de Atena, filha de Zeus.}

Assombro tomou os deuses imortais e os homens mortais

quando viram o íngreme ardil, impossível para os homens.

Pois dela vem a linhagem das bem femininas mulheres, \num{590}

pois é dela a linhagem ruinosa, as tribos de mulheres,\footnote{Os versos 590--591 são muito parecidos, o que faz a maioria dos editores optar
por um ou outro.}

grande desgraça aos mortais, morando com varões,

camaradas não da ruinosa Pobreza, mas de Abundância.

Como quando abelhas, em colmeias arqueadas,

alimentam zangões, parceiros de feitos vis: \num{595}

elas, o dia inteiro até o sol se pôr,

todo dia se apressam e favos luzidios depositam,

e eles ficam dentro nas colmeias salientes

e a faina alheia para o próprio ventre recolhem ---

bem assim as mulheres, mal aos homens mortais, \num{600}

Zeus troveja-no-alto impôs, parceiras de feitos

aflitivos. E outro mal forneceu pelo bem:

quem das bodas fugir e dos feitos devastadores\footnote{``Devastadores'' busca traduzir \emph{mermera}, um termo de sentido algo incerto.} das mulheres

e não quiser casar, atingirá velhice ruinosa

carente de quem o cuide; não privado de sustento \num{605}

vive, mas, ao perecer, dividem seus recursos

parentes distantes. Já quem partilhar do casamento

e obtiver consorte devotada, ajustada em suas ideias,

para ele, dês a juventude, o mal contrabalança o bem

sempre; e quem encontrar espécie insultante, \num{610}

vive com irritação incessante no íntimo,

no ânimo e no coração, e o mal é incurável.

\quad{}Assim não se pode lograr nem ultrapassar a mente de \qb{}Zeus.

Pois nem o filho de Jápeto, o benéfico Prometeu,\footnote{O sentido do epíteto grego traduzido por ``benéfico'' é, na verdade, obscuro.}

se esquivou de sua raiva pesada, mas, sob coação, \num{615}

embora multi-perspicaz, grande laço o subjuga.

\medskip

Assim que o pai teve ódio no ânimo por Obriareu,

Coto e Giges, prendeu-os em laço forte,

irritado com a virilidade insolente, a aparência

e a altura; e alocou-os embaixo da terra largas-rotas. \num{620}

Lá eles, que sofriam habitando sob a terra,

estavam sentados na ponta, nos limites da grande terra,

há muito angustiados com grande pesar no coração.

Mas a eles o Cronida e outros deuses imortais,

os que Reia belas-tranças pariu em amor por Crono, \num{625}

graças ao plano de Terra, levaram de volta à luz:

ela tudo lhes contara, do início ao fim,

como com aqueles obter vitória e triunfo radiante.

Pois muito tempo lutaram em pugna aflige-ânimo,

uns contra os outros em batalhas brutais, \num{631}

os deuses Titãs e todos os que nasceram de Crono, \num{630}

aqueles a partir do alto Otris, os ilustres Titãs, \num{632}

estes a partir do Olimpo, os deuses oferentes de bens,

os que pariu Reia belas-tranças deitada com Crono.

Eles então entre si, em pugna aflige-ânimo, \num{635}

sem parar pelejaram dez anos inteiros;

solução não havia para a dura briga, nem fim\footnote{A saber, Coto, Obriareu e Giges.}

para lado algum, e o remate da guerra se equilibrava.

\quad{}Mas quando, vê, ofertou-lhes tudo que é adequado,

néctar e ambrosia, o que comem os próprios deuses, \num{640}

e no íntimo de todos avolumou-se o ânimo arrogante

quando comeram o néctar e a desejável ambrosia,\footnote{Diversos editores deletam o verso.}

nisso então entre eles falou o pai de deuses e homens:

``Ouvi-me, filhos radiantes de Terra e Céu,

para eu dizer o que o ânimo no peito me ordena. \num{645}

Já muito tempo uns contra os outros

pela vitória e poder combatemos todo dia,

os deuses Titãs e todos os que nascemos de Crono.

Vós grande força e mãos intocáveis

mostrai em oposição aos Titãs no prélio funesto \num{650}

ao se lembrar da amizade afável, quanto sofreram

e de novo a luz alcançaram, soltos do laço tenebroso

graças a nossos desígnios, vindos das trevas brumosas''.

\quad{}Assim falou; logo lhe respondeu o impecável Coto:

``Honorável, não anuncias algo ignoto, mas também nós \num{655}

sabemos que sobressais no discernimento e na ideia,

e te tornaste protetor dos imortais contra dano gelado,

e com tua sagacidade, vindos das trevas brumosas,

de volta de novo, dos laços inamáveis,

viemos, senhor Cronida, após sofrer o inesperado. \num{660}

Assim também agora, com ideia tenaz e ânimo resoluto,

protegeremos vosso poder na refrega terrível,

combatendo os Titãs nas batalhas brutais''.

Assim falou; e aprovaram os deuses oferentes de bens

o discurso após o ouvir: à peleja almejou seu ânimo \num{665}

mais ainda que antes; e à luta não invejável acordaram

todos, fêmeas e machos, naquele dia,

os deuses Titãs e todos os que nasceram de Crono,

e os que Zeus da escuridão, sob a terra, à luz enviou,

terríveis e brutais, com força insolente. \num{670}

De seus ombros cem braços se lançavam,

igual para todos, e cabeças, em cada um, cinquenta

nasceram dos ombros sobre os membros robustos.

Contra os Titãs então se postaram no prélio funesto

com rochas alcantiladas nas mãos robustas; \num{675}

os Titãs, do outro lado, revigoraram suas falanges

com afã: ação conjunta de braços e de força mostraram

ambos, e o mar sem-fim em volta rugia, terrível,

e a terra, alto, ribombava, e gemia o amplo céu

sacudido, e tremia do fundo o enorme Olimpo \num{680}

com o arremesso dos imortais, e tremor atingia, pesado,

dos pés, o Tártaro brumoso, bem como agudo zunido

do fragor indizível e dos arremessos brutais.

Assim uns nos outros lançavam projéteis desoladores;

alcançava o céu estrelado o som de ambas as partes , \num{685}

das exortações; e se chocaram com grande algaraviada.

\quad{}E Zeus não mais conteve seu ímpeto, mas dele agora

de pronto o peito se encheu de ímpeto, e toda

a força mostrou. Ao mesmo tempo, do céu e do Olimpo

relampejando, progrediu sem parar, e os raios \num{690}

em profusão, com trovão e relâmpago, voavam

de sua mão robusta, revolvendo a sagrada chama,

em massa. Em volta, ribombava a terra traz-víveres,

queimando, e, no entorno, alto chiava mato incontável.

Todo o solo fervia, as correntes de Oceano \num{695}

e o mar ruidoso; a eles rodeava o bafo quente,

aos terrestres Titãs, e chama alcançou a bruma divina,

indizível, e aos olhos deles, embora altivos, cegou

a luz cintilante do raio e do relâmpago.

Prodigiosa queimada ocupou o abismo;\footnote{Ou ``Abismo''.} parecia, em face \num{700}

olhando-se com olhos e com ouvidos ouvindo-se o rumor,

assim como quando Terra e o amplo Céu acima

se reuniram: tal ressoo, enorme, subiu,

ela pressionada e ele, do alto, pressionando ---

tamanho baque quando os deuses se chocaram na briga. \num{705}

Junto, ventos engrossavam o tremor, a poeira,

trovão, raio e relâmpago em fogo,

setas do grande Zeus, e levavam grito e assuada

ao meio de ambas as partes: veio imenso clangor

da briga aterrorizante, e o feito do poder se mostrou. \num{710}

\quad{}E a batalha se inclinou; antes, com avanços recíprocos,

pelejavam sem cessar em batalhas audazes.

Estes, entre os da frente, acordaram peleja lancinante,

Coto, Briareu e Giges, insaciável na guerra:

eles trezentas pedras de suas mãos robustas \num{715}

enviavam em sucessão, e com os projéteis sombrearam

os Titãs; e a eles para baixo da terra largas-rotas

enviaram e com laços aflitivos prenderam,

após vencê-los no braço, embora autoconfiantes,

tão longe abaixo da terra quanto o céu está da terra. \num{720}

\medskip

Tal a distância da terra até o Tártaro brumoso.

Pois por nove noites e dias bigorna de bronze,

caindo do céu, no décimo a terra alcançaria;

{[}por sua vez, igual da terra até o Tártaro brumoso.{]}\footnote{A maioria dos editores rejeita esse verso.} \num{723a}

De novo, por nove noites e dias bigorna de bronze,

da terra caindo, no décimo o Tártaro alcançaria. \num{725}

Em volta dele, corre muro de bronze; no entorno, noite

camada-tripla derrama-se em volta da garganta; acima,

crescem as raízes da terra e do mar ruidoso.

\quad{}Para lá os deuses Titãs, sob brumosa escuridão,

foram removidos pelos desígnios de Zeus junta-nuvem, \num{730}

em região bolorenta, extremos da terra portentosa.

É-lhes impossível sair, Posêidon fixou portões

de bronze, e muralha corre para os dois lados.

\quad{}Lá Giges, Coto e o animoso Obriareu

habitam, fiéis guardiões de Zeus porta-égide. \num{735}

\quad{}Lá da terra escura, do Tártaro brumoso,

do mar ruidoso e do céu estrelado

as fontes e limites, de tudo, em ordem estão,

aflitivos, bolorentos, aos quais até os deuses odeiam;

grande fenda, e nem no ciclo de um ano inteiro \num{740}

alguém atingiria o chão, os portões uma vez cruzados,

mas p'ra lá e p'ra cá o levaria rajada após rajada,

aflitiva: assombroso é também para deuses imortais

esse prodígio; e a morada assombrosa de Noite

está de pé, escondida em nuvem cobalto. \num{745}

\quad{}Na frente, o filho de Jápeto sustém o amplo céu,

parado, com a cabeça e braços incansáveis,

imóvel, onde Noite e Dia passam perto

e falam entre si ao cruzarem o grande umbral

de bronze: uma entra e a outra pela porta \num{750}

vai, e nunca a ambas a casa dentro encerra,

mas sempre uma delas deixa a casa

e à terra se dirige, e a outra na casa fica

e, até aquela chegar, aguarda a sua hora de ir.

Uma, para os mortais na terra, tem luz muito-observa; \num{755}

a outra tem nas mãos Sono, irmão de Morte,

a ruinosa Noite, escondida em nuvem embaçada.

\quad{}Lá habitam os filhos da lúgubre Noite,

Sono e Morte, deuses terríveis; nunca a eles

Sol, alumiando, observa com os raios \num{760}

quando sobe ao céu nem quando desce do céu.

Deles, um à terra e ao largo dorso do mar,

calmo, se dirige, amável para os homens,

e do outro o ânimo é de ferro, e de bronze, seu coração

impiedoso no peito: segura assim que pega algum \num{765}

dos homens; é odioso até aos deuses imortais.

\quad{}Lá na frente, a morada ruidosa do deus terrestre,

o altivo Hades, e da atroz Perséfone

está de pé, e terrível cão vigia na frente,

impiedoso, com arte vil: para quem entra, \num{770}

abana por igual o rabo e as duas orelhas

e não permite que de volta saia, mas, ao perceber,

come quem pegar saindo pelos portões

do altivo Hades e da atroz Perséfone.\footnote{Os versos 768 e 774 são iguais; ambos são prováveis interpolações.}

\quad{}Lá habita a deusa, estigma\footnote{``Estigma'' procura reproduzir a sugestão poética de que
``Estige'' (\emph{Stux}) derivaria de ``odioso'' (\emph{stugeros}); no
grego, ``odioso para os imortais''.} para os imortais, \num{775}

a terrível Estige, filha de Oceano flui-de-volta,

primogênita: longe dos deuses, habita casa gloriosa

com abóboda de grandes pedras; em todo seu entorno,

colunas de prata a sustentam rumo ao céu.

Raramente a filha de Taumas, a velocípede Íris, \num{780}

vem com mensagem sobre o largo dorso do mar.

Quando briga e disputa se instaura entre imortais,

e se mente um dos que têm morada olímpia,

Zeus envia Íris para trazer a grande jura dos deuses

de longe, em jarra de ouro, a renomada água, \num{785}

gelada, que goteja de rocha alcantilada,

elevada: do fundo da terra largas-rotas, muito

flui do sacro rio através da negra noite ---

braço de Oceano, e a décima parte a ela foi atribuída;

nove partes, em torno da terra e do largo dorso do mar, \num{790}

com remoinho prateado ele gira e cai no mar,

e ela, uma só, da rocha flui, grande aflição dos deuses.

Quem, com ela tendo libado, jurar em falso,

um imortal dos que possuem o pico do Olimpo nevado,

esse jaz sem respirar até um ano se completar; \num{795}

nunca de ambrosia e néctar se aproxima

quanto à comida, mas jaz sem fôlego e sem voz

num leito estendido, e sono vil o encobre.

Após cumprir a praga no grande dia ao fim do ciclo,

a essa prova segue outra ainda mais cruel: \num{800}

por nove anos, é privado dos deuses sempre vivos,

e nunca se junta a eles em conselho ou banquete

por nove anos inteiros; no décimo, se junta de novo

às reuniões dos imortais que têm morada olímpia.

Tal jura os deuses fizeram da água eterna de Estige, \num{805}

primeva; e ela flui através da terra escarpada.

\quad{}Lá da terra escura, do Tártaro brumoso,

do mar ruidoso e do céu estrelado

as raízes e limites, de tudo, em ordem estão,

aflitivos, bolorentos, aos quais até os deuses odeiam. \num{810}

\quad{}Lá ficam os portões luzidios e o umbral de bronze,

ajustados, imóveis, com raízes contínuas,

naturais; na frente, longe de todos os deuses,

habitam os Titãs, para lá do abismo penumbroso.

E os gloriosos aliados de Zeus troveja-alto \num{815}

habitam casas nos fundamentos de Oceano,

Coto e Giges; quanto a Briareu, sendo valoroso,

fez dele seu genro Agita-a-Terra grave-ressoo,

e deu-lhe Flanonda, sua filha, para desposar.

%\medskip

Mas depois que Zeus expulsou os Titãs do céu, \num{820}

pariu Tifeu, o filho mais novo, a portentosa Terra

em amor por Tártaro devido à dourada Afrodite:

dele, os braços \dagger{}façanhas seguram sobre a energia\dagger{},\footnote{Verso corrupto.}

e são incansáveis os pés do deus brutal; de seus ombros

havia cem cabeças de cobra, brutal serpente, \num{825}

movendo escuras línguas; de seus olhos,

nas cabeças prodigiosas, fogo sob as celhas luzia,

e de toda a cabeça fogo queimava ao fixar o olhar.

Vozes havia em toda cabeça assombrosa,

som de todo tipo emitindo, ilimitado: ora \num{830}

soavam como se para deuses entenderem, ora

voz de touro guincho-alto, ímpeto incontido, altivo,

ora, por sua vez, a de leão de ânimo insolente,

ora semelhante a cachorrinhos, assombro de se ouvir,

ora sibilava, e, abaixo, grandes montanhas ecoavam. \num{835}

Feito impossível teria havido naquele dia,

e ele de mortais e imortais teria se tornado senhor,

se não tivesse notado, arguto, o pai de varões e deuses:

trovejou de forma dura e ponderosa, em torno a terra

ecoou, aterrorizante, e também, acima, o amplo céu, \num{840}

o mar, as correntes de Oceano e o Tártaro da terra.

Sob os pés imortais, o grande Olimpo foi sacudido

quando o senhor se lançou; e a terra gemia em resposta.

Queimada abaixo dos dois tomou conta do mar violeta

vinda do trovão, do raio e do fogo desse portento, \num{845}

dos ventos de ígneos tornados e do relâmpago ardente;\footnote{Sintaxe ambígua; ``dos ventos de ígneos tornados'' pode referir-se às armas de Zeus ou ao modo de combater de Tifeu.}

todo o solo fervia, e o céu e o mar:

grandes ondas grassavam no entorno das praias

com o jato dos imortais, e tremor inextinguível se fez;

Hades, que rege os ínferos finados, amedrontou-se, \num{850}

e os Titãs, embaixo no Tártaro, em volta de Crono,

com o inextinguível zunido e a refrega apavorante.

\quad{}Zeus, após rematar seu ímpeto, pegou as armas,

trovão, raio e o chamejante relâmpago,

e golpeou-o arremetendo do Olimpo; em volta, todas \num{855}

as cabeças prodigiosas do terrível portento queimou.

Após subjugá-lo, tendo-o com golpes fustigado,

o outro tombou, aleijado, e gemeu a portentosa Terra;

e a chama fugiu desse senhor, relampejado,

nos vales da montanha escura, escarpada, \num{860}

ao ser atingido, e a valer queimou a terra portentosa

com o bafo prodigioso, e fundiu-se como estanho,

em cadinhos bem furados, com arte por varões

aquecido, ou ferro, que é a coisa mais forte,

nos vales de montanha subjugado por fogo ardente \num{865}

funde-se em solo divino pelas mãos de Hefesto\footnote{Os versos 859--866 mantiveram tradicionalmente certa obscuridade da sintaxe arrevesada do original. Na comparação, estanho e ferro são coordenados: a terra
fundiu-se como o estanho trabalhado por jovens metalúrgicos ou o ferro
fundido por Hefesto.} ---

assim fundiu-se\footnote{Pucci (2009) nota que o verbo ``fundir'', nos versos 862 e 867, guarda paralelos sonoros com o verbo ``parir'' (v. 821) que abre o
episódio, respecitivamente, (\emph{e})\emph{tēketo} e \emph{teke}.} a terra com a fulgência do fogo \qb{}chamejante.

E arremessou-o, atormentado no ânimo, no largo Tártaro.

\quad{}De Tifeu é o ímpeto dos ventos de úmido sopro,

exceto Noto, Bóreas e o clareante Zéfiro, \num{870}

que são de cepa divina, de grande valia aos mortais.

As outras brisas à toa sopram no oceano;

quanto à elas, caindo no mar embaçado,

grande desgraça aos mortais, correm com rajada má:

sopram p'ra cá depois p'ra lá, despedaçam naus \num{875}

e nautas destroem; contra o mal não há defesa

para homens que com elas se deparam no mar.

Essas também, na terra sem-fim, florida,

lavouras amadas destroem dos homens na terra nascidos,

enchendo-as de poeira e confusão aflitiva. \num{880}

\medskip

Mas após a pugna cumprirem os deuses venturosos

e com os Titãs as honrarias separarem à força,

então instigaram a ser rei e senhor,

pelo plano de Terra, ao olímpico Zeus ampla-visão ---

dos imortais; e ele bem distribuiu suas honrarias. \num{885}

\quad{}Zeus, rei dos deuses, fez de Astúcia a primeira esposa,

a mais inteligente entre os deuses e homens mortais.

Mas quando ela iria à deusa, Atena olhos-de-coruja,

parir, nisso, com um truque, ele enganou seu juízo

e com contos solertes depositou-a em seu ventre \num{890}

graças ao plano de Terra e do estrelado Céu:

assim lhe aconselharam, para a honraria real

outro dos deuses sempiternos, salvo Zeus, não ter.

Pois dela foi-lhe destinado gerar filhos bem-ajuizados:

primeiro a filha olhos-de-coruja, a Tritogênia,\footnote{Tritogênia é um termo de significado desconhecido, possivelmente
aludindo a um lugar (mítico?) onde Atena teria nascido.} \num{895}

com ímpeto igual ao do pai e desígnio refletido,

e eis que então um filho, rei dos deuses e varões,

possuindo brutal coração, iria gerar;

mas Zeus depositou-a antes em seu ventre

para a deusa lhe aconselhar sobre o bem e o mal. \num{900}

\quad{}A segunda, fez conduzir a luzidia Norma, mãe das \qb{}Estações,

Decência, Justiça e a luxuriante Paz,

elas que zelam\footnote{``Zelar'' (\emph{ōrein}) ecoa \emph{Hōra} (``estação'');
\emph{erga}, aqui traduzido por ``trabalhos'', também pode se referir a
``lavouras'', como no verso 879. O conjunto --- trabalho agrícola e
virtudes cívicas --- é como que uma síntese das ideias desenvolvidas por
Hesíodo em \emph{Trabalhos e dias}.} pelos trabalhos dos homens mortais,

e as Moiras,\footnote{As Moiras também são filhas da Noite; a dupla origem parece indicar que as ações das deusas podiam ser pensadas de formas distintas e/ou
remeter a tradições (locais) diversas.} a quem deu suma honraria o astuto Zeus,

Fiandeira, Sorteadora e Inflexível, que concedem \num{905}

aos homens mortais bem e mal como seus.

\quad{}Três Graças bela-face lhe pariu Eurínome,

a filha de Oceano, com aparência desejável,

Radiância, Alegria e a atraente Festa:

de suas pálpebras, quando olham, pinga desejo \num{910}

solta-membros; belo é o olhar sob as celhas.

\quad{}E dirigiu-se ao leito de Deméter multinutriz:

ela pariu Perséfone alvos-braços, que Aidoneu\footnote{Aidoneu é Hades.}

raptou de junto da mãe, e deu-lha o astuto Zeus.

\quad{}Por Memória então se enamorou, a belas-tranças, \num{915}

e dela as Musas faixa-dourada lhe nasceram,

nove, às quais agradam festas e o prazer do canto.

\quad{}E Leto a Apolo e Ártemis verte-setas,

prole desejável mais que todos os Celestes,

gerou, após unir-se em amor com Zeus porta-égide. \num{920}

\quad{}Como última, de Hera fez sua viçosa consorte:

ela pariu Juventude, Ares e Eilêitia,

unida em amor com o rei dos deuses e homens.

\quad{}Ele próprio da cabeça gerou Atena olhos-de-coruja,

terrível atiça-peleja, conduz-exército, infatigável,\footnote{Embora aqui traduzido por ``infatigável'', o sentido original do
adjetivo \emph{atrutonē}, utilizado somente para Atena, é desconhecido.
``Infatigável'' e ``invencível'' eram as glosas mais comuns na
antiguidade.} \num{925}

senhora a quem agradam gritaria, guerras e combates.

E Hera ao glorioso Hefesto, não unida em amor,

gerou, pois, enfurecida, brigou com seu marido:\footnote{Um caso de \emph{husteron proteron}, ou seja, o recurso
estilístico-narrativo no qual o que acontece antes é mencionado em
segundo lugar. A conjunção ``pois'' não está em grego; é acrescentada
para não tornar a frase incompreensível para o leitor da tradução.}

aquele nas artes supera todos os Celestes.

\quad{}E de Anfitrite e de Treme-Solo ressoa-alto \num{930}

nasceu o grande Tríton ampla-força, que do mar

a base ocupa e junto à cara mãe e ao senhor pai

habita casa dourada, o deus terrível. E para Ares

fura-pele\footnote{``Fura-pele'': pode dizer respeito à pele do herói ferido ou ao
couro do escudo.} Citereia pariu Terror e Pânico,\footnote{Na \emph{Odisseia}, Afrodite é representada como amante de Ares,
mas casada com Hefesto, que, por sua vez, na \emph{Teogonia} e em outros
textos, é representado casado com uma Graça.}

terríveis, que tumultuam cerradas falanges de varões \num{935}

com Ares arrasa-urbe em sinistra batalha,

e Harmonia,\footnote{Harmonia é um termo grego.} a quem o auto-confiante Cadmo desposou.

\quad{}Para Zeus a filha de Atlas, Maia, pariu o glorioso Hermes,

arauto dos deuses, após subir no sacro leito.

\quad{}E a filha de Cadmo, Semele, gerou-lhe filho insigne, \num{940}

unida em amor, Dioniso muito-júbilo,

a mortal ao imortal: ambos agora são deuses.

\quad{}E Alcmena pariu a força de Héracles,

unida em amor com Zeus junta-nuvem.

\quad{}E de Radiância o esplêndido Hefesto duas-curvas, \num{945}

da mais nova das Graças, fez sua viçosa consorte.

\quad{}E Dioniso juba-dourada da loira Ariadne,

a filha de Minos, fez sua viçosa consorte:

a ela, para ele, imortal e sem velhice tornou o Cronida.

\quad{}E de Juventude o bravo filho de Alcmena linda-canela, \num{950}

o vigor de Héracles, após findar tristes provas,

da filha do grande Zeus e de Hera sandália-dourada

fez sua esposa, respeitada no Olimpo nevado:

afortunado, que grande feito realizou entre os imortais,

e habita sem miséria e velhice por todos os dias. \num{955}

\quad{}A gloriosa filha de Oceano pariu ao incansável Sol

Perseís, Circe e o rei Eetes.

Eetes, o filho de Sol ilumina-mortal,

à filha do circular rio Oceano

desposou, Sapiente bela-face, pelos desígnios dos deuses: \num{960}

ela gerou-lhe Medeia belo-tornozelo,

em amor subjugada devido à dourada Afrodite.

\medskip

Agora, felicidades, vós que tendes moradas olímpia,

ilhas, continentes e, no interior, o salso mar;

mas agora a tribo das deusas cantai, doce-palavra \num{965}

Musas do Olimpo, filhas de Zeus porta-égide,

tantas quantas junto a varões mortais deitaram

e, imortais, geraram filhos semelhantes a deuses.

\quad{}Deméter a Pluto gerou, diva entre as deusas,

unida ao herói Iasíon em desejável amor, \num{970}

em pousio com três sulcos, na fértil região de Creta,

ao valoroso, que vai pelas amplas costas do mar e terra

inteira: a quem ao acaso topa e alcança suas mãos,

a esse torna rico e lhe dá grande fortuna.

\quad{}Para Cadmo Harmonia, filha de dourada Afrodite, \num{975}

a Ino, Semele, Agave bela-face,

Autônoe, a quem desposou Aristaio cabeleira-farta,

e também Polidoro gerou em Tebas bem-coroada.\footnote{Bem-coroada: referência às famosas muralhas da cidade.}

\quad{}A filha de Oceano, após ao destemido Espadouro \qb{}ânimo-potente

unir-se em amor de Afrodite muito-ouro, \num{980}

Bonflux, pariu o filho mais vigoroso de todos os mortais,

Gerioneu, a quem matou a força de Héracles

pelos bois passo-arrastado na oceânica Eriteia.

\quad{}E para Títono Aurora gerou Mêmnon elmo-brônzeo,

rei dos etíopes,\footnote{``Etíopes'', tribo mítica ainda não associada à região
posteriormente conhecida como Etiópia; diz respeito ao norte da África
de forma geral.} e o senhor Emátion. \num{985}

E para Céfalo gerou um filho insigne,

o altivo Faéton, varão semelhante a deuses:

ao jovem na suave flor da gloriosa juventude,

garoto imaturo, Afrodite ama-sorriso

lançou-se e o carregou, e de seus templos numinosos \num{990}

fez dele o servo bem no fundo, divo espírito.

\quad{}E à filha de Eetes o rei criado-por-Zeus,

o Esonida, pelos desígnios dos deuses sempiternos,\footnote{Trata-se de Jasão e Medeia}

levou de junto de Eetes, após findar tristes provas,

muitas, que lhe impôs o grande rei arrogante, \num{995}

o violento e iníquo Pélias ação-ponderosa:

quando as findou, chegou a Iolco, após muito sofrer,

sobre rápida nau levando a jovem olhar-luzente

o Esonida, e dela fez sua viçosa consorte.

E ela, subjugada por Jasão, pastor de tropa, \num{1000}

gerou o filho Medeio, de quem Quíron cuidou nos morros,

o filho de Filira; e a ideia do grande Zeus foi completada.

\quad{}E as filhas de Nereu, o velho do mar,

a Focos, por um lado, Areiana pariu, diva entre as deusas,

em amor por Eaco devido à dourada Afrodite; \num{1005}

e a Peleu subjugada, a deusa Tétis pés-de-prata

gerou Aquiles rompe-batalhão, de ânimo leonino.

\quad{}E a Eneias pariu Citereia bela-coroa,

após ao herói Anquises se unir em desejável amor

nos picos do ventoso Ida muito-vale. \num{1010}

\quad{}E Circe, a filha do Hiperionida Sol,

gerou, em amor por Odisseu juizo-paciente,

Ágrio e Latino, impecável e forte;

e a Telégono\footnote{O nome Telégono --- ``filho (nascido) longe'' --- remete ao outro filho de Odisseu, Telêmaco.} pariu devido à dourada Afrodite:

quanto a eles, bem longe, no recesso de sacras ilhas, \num{1015}

regiam todos os esplêndidos tirrenos.

\quad{}E Nauveloz para Odisseu Calipso, diva entre as deusas,

e Náutico gerou, unida em desejável amor.

\quad{}Essas deitaram junto a varões mortais

e, imortais, geraram filhos semelhantes a deuses. \num{1020}

Agora cantai a tribo das mulheres, doce-palavra

Musas do Olimpo, filhas de Zeus porta-égide.

\endgroup