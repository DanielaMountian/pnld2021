
\chapter[Introdução, \emph{por Clara C.~Santos e Ricardo M.~Valle}]{Introdução}
\hedramarkboth{introdução}{Clara C.~Santos e Ricardo M.~Valle}

\begin{flushright}
\textsc{clara c.~santos\\ricardo m.~valle}
\end{flushright}

\section{um livro e um nome}

\noindent{}Impressa em Lisboa, em 1576, na oficina de Antonio Gonçalves, a edição
da \textit{História da província Santa Cruz a que vulgarmente chamamos Brasil
feita por Pero de Magalhães de Gandavo}\footnote{ Nesta edição
seguimos o exemplar pertencente à Biblioteca Nacional de Lisboa da
\textit{História da província Santa Cruz a que vulgarmente chamamos Brasil
feita por Pero de Magalhães de Gandavo}. Lisboa: Officina de Antonio
Gouvea, 1576. Infelizmente a Biblioteca Nacional brasileira não tem
programa de digitalização fotográfica do acervo de obras raras,
dificultando as possibilidades de cotejo, para resolver problemas como
as aparentes irregularidades na página de Aprovação.} é um tratado da terra, isto
é, um laudo documental dos domínios do soberano, com tudo o que neles
houvesse, oferecido neste caso a um vassalo do Rei de Portugal como
louvor dos domínios do mesmo Rei. Pensado até aí, o documento parece
perfeitamente conformado no interior das instituições e regulamentos
institucionais a que então um impresso tinha de submeter"-se. Contudo, o
livro parece ter saído de circulação e o nome do autor praticamente
desaparece por quase dois séculos, principalmente em âmbito português.
A obscuridade do livro nos séculos seguintes à sua publicação é tanto
mais estranha se se tem em vista que, por intermédio de uma elegia e um
soneto de Camões, o livro é dedicado a um varão de armas em carreira
promissora nas Índias portuguesas, tendo sido impresso pela mesma
oficina impressora d'\textit{Os Lusíadas} (1572), que obtivera
alvará de Privilégio real para sair, apenas quatro anos antes.

Falando com rigor, nada efetivamente se sabe a respeito de seu autor,
além de que possivelmente tenha escrito o livro (mais uma ortografia e
alguns outros manuscritos), e de que o tenha feito assinando com este
nome, Pero de Magalhães de Gandavo. Estamos certos, aliás, de que
sabemos até menos do que isso. A designação ``de Gandavo'' 
aparentemente não foi herdada como sobrenome, mas incluído pela pessoa 
do autor e pode ser que para constituir tradição familiar, com vistas a alguma 
fidalguia. Daí se poderia supor também que fosse um imigrado em Portugal 
e que a si passasse a designar pelo patronímico, em verdade ilustríssimo naquele tempo, 
porque sabemos que Gandavo também era Carlos \textsc{v}, nascido em Guantes, 
Flandres, cidade distinguida justamente por esse evento de enorme 
significação para o catolicismo europeu.

Conhecendo os trâmites político"-institucionais a que estava submetido na
hierarquia e valendo"-se dos verossímeis da invenção no gênero
histórico, Barbosa Machado inventou em meados do século 	\textsc{xviii} elementos
da vida do autor. E a recepção moderna tantas vezes apenas acreditou
como efetividade, deixando de lado as possibilidades de pensar, não o
fato particular, que pouco importa, mas as implicações institucionais
supostas a esse quase nada de que se tem notícia, um nome e um
livro.\footnote{ Ver as questões e categorias discutidas por João Adolfo
Hansen em ``Um nome por fazer'', acerca de Gregório de Matos. In: \textit{A sátira e o Engenho}. 
São Paulo/Campinas: Ateliê/Unicamp, 2004, e em “Autor”. In: \textit{José Luís Jobim}. Palavras da
crítica. Rio de Janeiro: Imago, 1992.} Contudo, desde a publicação
francesa, que é do início do século \textsc{xix}, tem"-se reiterado a mesma
informação biográfica, posteriormente acrescentada, ainda que sempre de
poucas notícias.

É verdade que se poderia inventar o verossímil mentiroso de um Gandavo
anônimo, de família portuguesa com ascendência marrana, cristãos"-novos
fugidos da Península Ibérica no tempo de Dona Isabel e Dom Fernando, e
das consequências inclusive jurídicas da ação política dos piedosos e
violentos \textit{reyes católicos de España}, pais de Carlos \textsc{v}. A família
marrana, em processo de limpeza de sangue, serve o Império Católico em
Flandres no tempo do imperador, numa sabidamente lenta ascensão
estamental, por acumulação de dignidades em ofícios letrados. Em 1568,
com a revolução de Orange nos Países Baixos --- com a revisão dos pactos,
as reformas institucionais e as alterações no direito e costume ---, a
família portuguesa de origem judaica, e letrada em âmbito católico, sai
dali para o Brasil, procurando postos subalternos para homens com
algumas letras. Na província portuguesa, Pero de Magalhães (ou como
quer que se tenha chamado) ocupa alguma função anônima na província
portuguesa desta costa do Brasil, onde enriquece e forja documentos
para abreviar a carreira na volta à Europa. Em Lisboa, muito rico e
provavelmente ignorante, ou ao menos cheio de maus acentos no uso da
língua, torna"-se um adulador para conseguir nomeação de escrivão, ou
cronista na Torre do Tombo, seguindo talvez tradição paterna. Paga um
poeta ilustre, um varão de armas sem dinheiro, um impressor e um
historiador para forjar uma obra que lhe conferisse a autoridade de
historiador que lhe ajudaria a receber o cargo tornando"-se mais próximo
de um título de fidalguia. Descoberta a fraude, por irregularidades com
a licença do Paço, desenrola"-se o que é presumível. E com isso, a
memória de seu nome e de seu livro praticamente desaparecem, até que no
tempo da Academia Real de História o livro não fosse reconhecido como
irregular pelas instâncias de chancelaria, passando a ser mencionado,
mas pouco; até que, na Biblioteca  Lusitana, Diogo Barbosa Machado
cumprisse o lugar de bibliógrafo redigindo uma linha de sua vida, aqui
entendida como a espécie do gênero histórico cuja \textit{auctoritas} é
principalmente Plutarco. Com melhor engenho, seria possível inventar
outros particulares verossímeis para a vida de Gandavo, não sendo outra
coisa porém do que a aplicação dos procedimentos da arte de Diogo
Barbosa Machado, cuja notícia a crítica histórica raras vezes deixou de
considerar como o ``pouco que se sabe''.

Pero de Magalhães de Gandavo  tornou"-se um nome tão obscuro quanto o seu
livro. Desde a \textit{Bibliotheca portuguesa} de Barbosa Machado, que é de
meados do século \textsc{xviii}, diz"-se que fora natural de Braga, que o pai era
flamengo, que foi moço"-de"-câmara de Dom Sebastião, que trabalhou na
Torre do Tombo como copista, que permaneceu alguns anos no Brasil cuja
história escreveu, e que, após a publicação do livro, é nomeado
provedor da fazenda da cidade de Salvador, na Bahia, cargo que, diz"-se,
não exerceu. Versado pelo menos nas artes do \textit{trivium} e autor de uma
ortografia portuguesa, teria aberto uma escola na província, na região
entre o Douro e o Minho, onde também casara. A maior parte das ações e
funções institucionais que se lhe atribuíram, contudo, foram atos ou
ofícios que presumivelmente um homem de letras tinha dignidade para
exercer. São provavelmente verossímeis narrativos do gênero histórico,
inventados por tradições biblio e historiográficas de escrita de \textit{vit\ae}
de poetas; ou são notícias derivadas desses verossímeis em vertentes
historiográficas do século \textsc{xix}. O mistério que ronda o desaparecimento
de seu livro, aplica"-se a seu estranho sobrenome, que, sem ascendência
nem descendência certas, não se sabe hoje sequer a sílaba sobre a qual recai o acento.

Desde o século \textsc{xix}, a \textit{História da província Santa Cruz} foi lido como
``relato de viajantes'', ``literatura de informação'' ou como \textit{Nossa
primeira história}, título que recebeu na edição de 1921--1922, entendido
como testemunho de impressões antigas dos portugueses nas terras
d'além"-mar. Contudo, esta simples história, ou tratado
descritivo, da costa do Brasil teve uma circulação muito restrita no
seu século, fazendo parecer que o livro foi recolhido após sua
impressão, não se sabe precisamente por quê. Segundo uma linha de
interpretação da recepção da \textit{História da província Santa Cruz},
sustenta"-se que o livro teria sido recolhido por revelar segredos de
Estado sobre a província portuguesa, como a posição de rios e cidades
da costa do Brasil, segundo o que os historiadores chamaram de
``política do segredo'' de Dom Manuel \textsc{i}.\footnote{ Cf.~Sheila 
Moura Hue e Ronaldo Menegaz.
``Introdução''. In:  \textit{Primeira História do
Brasil. História da província Santa Cruz a que vulgarmente chamamos
Brasil}. Rio de Janeiro: Jorge Zahar, 2004, p.~14.} O certo é que
permaneceu praticamente ignorado até a primeira metade do século \textsc{xix},
quando foi reconsiderado na edição e tradução de M. Henri Ternaux, em
Paris, em 1837; e no século seguinte foi ainda vertido para o inglês
por John B. Stetson Jr., em 1969. No início do século \textsc{xviii}, com o
fomento à divulgação das navegações e feitos portugueses pelas
Academias de História no reinado de Dom João \textsc{v}, o opúsculo de Gandavo
consta dos documentos então exumados pela Real Academia, o que se prova
com a ocorrência secundária no Bluteau e logo na Biblioteca de Diogo
Barbosa Machado, em meados do século \textsc{xviii}, fazendo a essa regra raras
exceções, sendo mantido ainda em ampla obscuridade até a edição
francesa na primeira metade do século \textsc{xix}.

Diferente de um testemunho empírico, o livro é composto como gênero
histórico, retoricamente regrado, em que o historiador, apoiado pelo
aconselhamento ético da Igreja Católica, tem por últimos fins exaltar,
pelo discurso, ações virtuosas de pessoas de caráter elevado e eventos
providenciais; levar adiante a fama do monarca e dos homens de armas a
quem é dedicado; e legitimar sua autoridade e senhorio, com direito de
propriedade, sobre terras, rios, espécies animais e vegetais, pedras,
metais etc., que se podem usar e dispor como próprios, conforme aos
grandes axiomas e aos anátemas que regulavam estas disposições na forma
das leis civis e eclesiásticas. Estes fins deveriam atingir"-se pela
aplicação de procedimentos e lugares retóricos, entre os quais a
amplificação da beleza, utilidade, fertilidade, abundância etc., como
louvor dos novos domínios da Cristandade portuguesa. Neste sentido,
relata, ou historia, as particularidades da terra e de sua conquista
como louvor do feito português, visando à perpetuação da empresa
marítima lusitana a partir de uma narrativa que segue principalmente os
modelos preceptivos de Menandro, retor, e Plínio, o Velho, entre outras
autoridades de escrita histórica.

\section{a história como\break demonstração da sujeição}

Seja como for, num tempo em que a conservação da fama do nome,
transmitido pela família, é fundamental para a aquisição de favores,
dignidades, benefícios de estado, e sabendo ainda que o livro impresso
era instrumento de perpetuação de nomes e dignidades institucionais, é
sem dúvida extraordinário um tal desaparecimento. Conforme à aplicação
da tópica da perenidade das letras, o ``Prólogo ao
Leitor'' (parte do exórdio que se dirige ao auditório
retoricamente constituído como um gênero de homens ao menos letrados em
boas letras), o autor da \textit{História da província} repõe que
``a escritura seja vida da memória, e a memória uma
semelhança da imortalidade a que todos devemos aspirar''.

Como não interessa aqui encontrar a afirmação da verdade particular da
História, o provável, ou verossímil, infortúnio de Gandavo permite"-nos
pensar procedimentos de representação institucional que, como demonstra
João Adolfo Hansen, estão muito diretamente ligados às práticas
letradas, entre as quais a história e a poesia, como a teologia moral e
a jurisprudência etc. conforme os ofícios em questão:

\begin{hedraquote}
Na fronteira das categorias do imaginário social mais amplo e das
categorias dos grupos cultivados, situava"-se então a imagem do
sábio"-letrado, padre ou funcionário, muitas vezes poeta. A imagem era
equívoca, pois nela convergiam as representações do letrado"-artesão,
escrevente que simplesmente repetia a tradição, fazendo cópias
manuscritas de textos, e do letrado sábio, que reinventava em novas
ocasiões. [\ldots{}] As práticas dos letrados portugueses não se
autonomizavam da hierarquia, e, não sendo mais escrivães medievais, mas
também não sendo escritores, no sentido iluminista dado ao termo a
partir da segunda metade do século \textsc{xviii}, eles se identificavam com a
imagem social da profissão que exerciam e essa era, obviamente,
profissão subordinada ao poder real.\footnote{ João Adolfo Hansen,
``Fênix renascida \& Postilhão de Apolo''. In: \textit{Poesia seiscentista}. São Paulo: Hedra, 2002.} 
\end{hedraquote}

Com efeito, antes de imprimir sua \textit{História da província Santa Cruz},
Gandavo dedicou mais de um tratado da terra aos principais do reino,
declarando fidelidade no uso que faz das letras. Trata"-se do manuscrito
intitulado \textit{Tratado da terra do Brasil no qual se contem a informação
das cousas que há nestas partes feito por Pº de magalhães},\footnote{ Para 
um cotejo que explicita as diferenças do texto
do \textit{Tratado da terra do Brasil} e da \textit{História da província Santa Cruz},
ver notas à edição de Sheila Moura Hue e Ronaldo Menegaz, op.~cit. Vale
lembrar, porém, que o \textit{Tratado} manuscrito e a \textit{História} impressa são
dois textos, no todo, diversos.} também pertencente à Biblioteca
Nacional de Lisboa, que os disponibiliza no Acervo Digital. A versão
impressa é dedicada a dom Lionis Pereira, varão de armas de altura
ainda mediana na vassalagem portuguesa, e de quem sabemos que foi
capitão em Malaca, em 1564, e em Ceuta, em 1580, tendo passado
certamente por outros postos que acumularam mérito oficial para que de
tão longe passasse a tão perto, o que era evidentemente uma melhora no
interior do estamento de que participava. O manuscrito, por sua vez, é
dedicado ao Príncipe Cardeal"-Infante dom Henriques, o futuro regente
interino do reino e províncias portugueses, entre o desaparecimento de
Dom Sebastião, em 1578, e a ascensão de Filipe \textsc{ii} de Espanha, em 1580.
Na posição de um homem de letras, que demonstra sua fidelidade e
aptidão, o manuscrito assinado por Pero de magalhães (com minúscula,
como quase sempre assinava a gente sem fidalguia ou alguma distinção no
século \textsc{xvi} em Portugal) dirige"-se a um dos dois maiores postos do reino
num documento sem data. Esse manuscrito também não leva o obscuro
sobrenome supostamente paterno, Gandavo, e que só mesmo o século \textsc{xix}
francês elevou a alguma fama.

Neste ``sumário da terra do Brasil'' --- como é
designado pelo autor ---, o vassalo alega já ter feito e dedicado uma
cópia do seu tratado a Dom Sebastião, passando agora a fazê"-lo ao seu
tio, sucessor imediato, numa carta dedicatória ``Ao mui
alto e sereníssimo Príncipe dom Henrique Cardeal Infante de Portugal'':
\begin{hedraquote}
Posto que os dias passados apresentei outro sumário da terra do Brasil a
el"-Rei nosso senhor, foi por cumprir primeiro com esta obrigação de
vassalo que todos devemos a nosso Rei: e por esta razão me pareceu
cousa mui necessária (muito Alto e Sereníssimo señor) oferecer também
este a \textsc{v.a.} a quem se devem referir os louvores e acrescentamento das
terras que nestes Reinos florecem: pois sempre desejou tanto
aumentá"-las e conservar seus súditos e vassalos em paz.
\end{hedraquote}

Dada a altura elevadíssima das pessoas a quem o pequeno vassalo se
dirigia, é muito provável que seja verdadeira a informação acerca da
existência de outra versão manuscrita anterior à que conhecemos. A
informação, porém, aqui interessa para lembrar que a cópia conhecida,
tendo sido dedicada ao Cardeal"-Infante Dom Henriques, não poderia ser
considerada uma primeira redação ou esboço, como se o manuscrito
necessariamente estivesse destinado a receber letra de forma. Talvez
seja apenas uma outra espécie de documento. De qualquer modo, bastante
diferente de um rascunho, as cópias do \textit{Tratado da terra do Brasil}
cumprem certamente um papel institucional, por meio de uma
representação, ou ostentação, de fidelidade à soberania das ordens
superiores que governam o Estado; trata"-se de uma declaração de
reconhecimento, por parte do súdito, da necessidade sagrada da
hierarquia dos homens. Uma vez então que seja mais seguramente o
cumprimento de um dever de ``subdito e vassallo'' e ainda a demonstração de sua (humilde)
utilidade em ofícios que requeiram boas letras, o manuscrito não tem
uma natureza mais verídica no relato, ou história, da terra, nem
significa uma observação empírica mais autêntica, como foi interpretado
pela crítica historiográfica no século \textsc{xx} que procurou ler esses
documentos como ``História do Brasil'', ``literatura de informação'' ou
``de viajantes'', sem considerar muitas vezes os âmbitos institucionais, 
as disposições jurídicas e os procedimentos discursivos que estavam supostos.

\section{o poder constituído das mesas censórias}

Na segunda metade do século \textsc{xvi}, as aprovações do Ordinário, da
Inquisição e do Paço atestavam a verdade da instrução que o livro
continha, segundo as Leis católicas e do Império português, que em
grande medida são extensão umas das outras. Sejam livros úteis, como os
de Lei e os Sagrados, sejam os de recreação, como os de história ou
poesia,\footnote{ Empregamos aqui a classificação que se lê nas páginas
iniciais do \textit{Corte na aldeia} (1619), de Rodrigues Lobo, que inclui como
livros de recreação, de um lado, os de histórias fingidas, compreendendo
as cavalarias de \textit{res ficta}, isto é, de matérias fingidas, análogos neste
sentido de muitos gêneros da poesia, e de outro lado os de história
verdadeira, como podem ser pensadas as crônicas, décadas, tratados da
terra, etc. (Cf.~\textit{Obras políticas morais e métricas do insigne Portugues
Francisco Rodrigues Lobo. Natural da Cidade de Leyria. Nesta última
impressão novamente correcta, e postas por ordem. Offerecidas à
Magestade sempre augusta do Sereníssimo Rey de Portugal. D. João \textsc{v},
nosso senhor}. Lisboa Oriental: Na Officina Ferreyriana. 1723;
pp.~3--4.)} somente a concórdia entre as mesas conferia a completa
autorização para a impressão e circulação do livro, cruzando"-o com a
tradição das autoridades e axiomas de doutrina ou instrução verdadeira;
entendendo por tradição aqui o sentido com que o Concílio de Trento
distiguia boas e más tradições discursivas, tendo em vista que 

\begin{hedraquote}
esta verdade, e disciplina se contém em livros escritos, e sem escritos,
nas Tradições, que recebidas pelos Apóstolos da boca de Cristo, ou
ditadas pelo Espírito Santo, dos mesmos Apóstolos como de mão em mão
chegaram até nós; seguindo o exemplo dos Padres Ortodoxos [\ldots{}]; como
ditadas pela boca de Cristo, ou pelo Espírito Santo, e por uma
contínua sucessão, conservadas na Igreja Católica [\ldots{}].\footnote{ Sessão 
\textsc{iv}, do Concílio de Trento. Citamos pela edição bilíngue do
século 	\textsc{xviii}: \textit{O sacrosanto, e ecumenico Concilio de Trento Em Latim, e
Portuguez}. Trad. João Baptista Reycend. Lisboa: Na officina Patriarcal
de Luiz Ameno, 1781, Tomo \textsc{i}, p.~55.}
\end{hedraquote}

Como ficou dito, os destinatários das duas versões manuscritas (a
primeira delas apenas hipotética) são as duas mais poderosas
representações políticas daquelas décadas. Um é o jovem rei, que tem
representação primordialmente jurídico"-militar --- e, no caso particular
de Dom Sebastião, sobretudo militar e piedosa, segundo a invenção da
fama do nome na instituição histórica, cujo fim, entre as funções
bibliotecárias do reino desde antes de Fernão Lopes, era conferir
legitimidade à autoridade das potestades instituídas no presente.
Independentemente da feição particular da representação tipológica que
assume, o rei devia ser reconhecido como autoridade que reúne a
soberania sobre a totalidade das potestades, ofícios, artes e demais
serviços do reino, uma vez que seja monarca. Já o destinatário do
\textit{Tratado da terra do Brasil} que efetivamente nós podemos ler é o tio do
mesmo monarca, o velho Inquisidor"-mor, Cardeal dominicano, Dom
Henriques, que tinha autoridade direta sobre as três instâncias de
censura: do Ordinário, da Inquisição e do Paço. A mesa do Ordinário dá
vista e aprovação, ou a nega, normalmente pela autoridade de um doutor
em leis ou teologia escolhido entre os bispos da diocese para a função
\textit{ad hoc} de vedor ou qualificador. Como junta ou conselho episcopal do
reino, os membros do Ordinário estão necessariamente sob a autoridade
do Cardeal"-Infante, ainda mais quando o príncipe"-sacerdote é o próprio
Inquisidor do Paço. Além disso, sabemos que mesmo o Conselho Geral da
Inquisição havia sido instalado em Portugal sob a ação do mesmo
Cardeal"-Infante, feito Inquisidor"-mor por seu irmão, Dom João \textsc{iii},
décadas antes. 

Segundo o Concílio de Trento, os livros de matérias sagradas e, por
necessidade, toda matéria sagrada contida em livros de qualquer
espécie, deveriam ser conferidos, sob pena, com o ``verdadeiro sentido, 
e interpretação das Escrituras'', isto é, ``o unânime consenso
dos padres''. Assim, decreta que 

\begin{hedraquote}
a ninguém é lícito imprimir, nem mandar imprimir Livros alguns de
matérias sagradas sem nome do Autor, nem vendê"-los daqui em diante;
nem também tê"-los em seu poder, sem serem primeiro examinados, e
aprovados pelo Ordinário, sob pena de excomunhão [\ldots{}]. E se forem
Regulares (os Autores) além deste exame, e aprovação, estarão
obrigados a impetrar também Licença dos seus Superiores, sendo por
eles examinados os livros, na forma das suas Constituições.\footnote{ Idem, ibidem. pp.~61--63.}
\end{hedraquote}

Além de valer para qualquer documento manuscrito, as penas se estendiam
aos impressores, vendedores, possuidores e leitores da obra que
ilegalmente fosse posta em circulação. A censura do Ordinário em
Portugal, como na Espanha, era já exercida antes do Concílio --- e como
mostra a própria folha de ``Aprovação'' desta
\emph{História da província Santa Cruz} ---, a licença do Ordinário
subordinava"-se à petição do Conselho Geral da Inquisição. E assim como
os clérigos regulares deveriam ter ainda a permissão dos superiores da
Ordem a que pertencessem, os súditos deveriam ter a aprovação de seu
rei, o que em Portugal se fazia pela censura do Paço.

No livro de Gandavo em particular, a Licença do Conselho Geral do Santo
Ofício da Inquisição acredita o testemunho do vedor de livros do
Ordinário; tudo feito no interior de rígidas praxes, instituídas e
acostumadas por décadas. A aprovação do Santo Ofício apenas ratifica,
aparentemente sem exame, a informação de que o livro não continha coisa
que se pudesse considerar inimiga da santa fé e do bom costume
cristãos, não colocando em suspeição nem os mistérios nem os usos da
fé, sempre segundo o costume apostólico romano e terminantemente
revogadas pelo Concílio quaisquer tradições de opinião contrária ou
fundadas na própria prudência e livre inteligência das coisas sagradas.
Segundo o Decreto aposto à mesma Sessão do Concílio:
\begin{hedraquote}
para refrear engenhos petulantes, determina: que ninguém confiado na sua
prudência, em matérias de Fé, e costumes, e edificação da Doutrina
Cristã, torça a sagrada Escritura para os seus conceitos particulares,
contra aquele sentido que abraçou, e abraça a Santa Madre Igreja, a
quem pertence julgar o verdadeiro sentido e interpretação das
Escrituras; nem se atreva a interpretar a mesma Escritura contra o
unânime consenso dos Padres; ainda que estas interpretações nunca hajam
de se dar a luz. Os que a isto contravierem, sejam pelos Ordinários
declarados, e castigados com as penas estabelecidas em
direito.\footnote{ Idem, ibidem. pp.~59--61.}
\end{hedraquote}

O livro de Gandavo aparentemente não se choca com o Decreto, mas não
deixa de ser notável que, no exemplar que transcrevemos, pertencente à
Biblioteca Nacional de Lisboa, que o publica digitalmente, a caixa de
texto da licença do Paço apresenta recuo maior de ambos os lados e o
alinhamento horizontal é ligeiramente pendente à esquerda, o que pode
indicar que a inserção foi feita posteriormente. A hipótese talvez
fosse excessiva, não houvesse notícia pela Biblioteca John Carter Brown
de que há exemplares com e sem a terceira caixa de texto.\footnote{ A
informação é de Rubens Borba de Moraes, que compara os exemplares da
Biblioteca \textsc{jcb} com o exemplar da Biblioteca Nacional do Rio de Janeiro.
Não pudemos, contudo, auferir a autenticidade da informação (Cf.
\textit{Bibliografia brasileira do período colonial}. São Paulo: Instituto de
Estudos Brasileiros, \textsc{usp}, 1969). A pesquisadora Valéria Gauz, que
trabalhou tanto na Biblioteca Nacional quanto na John Carter Brown,
menciona esse dado no texto ``Materialidade de livros -- \textsc{ii}'', disponível no site da \textsc{info}\textit{home}. 
O acervo da Biblioteca Nacional brasileira, contudo, não é
disponibilizado digitalmente.} Tudo isso, que é nada, na mera hipótese
sobre o particular, apenas permite insinuar problemas na tese que
explica diretamente o desaparecimento do livro por violação de segredo
de Estado, ainda que a hipótese continue válida. É mesmo muito
plausível que o desaparecimento do livro tenha sido causado por razão
de Estado, mesmo porque o acaso raras vezes é tão violento quanto as
instituições normativas. Esses particulares conhecíveis não teriam
maior interesse, se o manuscrito conservado não fosse dedicado
justamente à altíssima autoridade censória do reino, ao mesmo tempo que
sua versão impressa, recomendada por Camões, parece ter sido destruída
por força estatal, do contrário não teria sido elidido na quase
totalidade de seus exemplares. À aparente irregularidade da página de
``Aprovação'', se se confirmar que há volumes
sem a terceira licença, que é a do Paço, podem"-se fazer suposições
inúmeras que talvez tenham \mbox{interesse se} forem pensadas a partir dos
mecanismos institucionais e discursivos que constituem historicamente
essas relações de poder.

\section{aos empobrecidos do reino}

Ainda na Dedicatória do \textit{Tratado da terra}, Pero de Magalhães, de
Gandavo, afeta modéstia, como tem de ser, para produzir retoricamente
benevolência no destinatário; para isso, elogia a utilidade da matéria
do seu tratado, com o que se quer demonstrar também a utilidade do
serviço do vassalo fiel em ofícios de letras:

\begin{hedraquote}
achei que não se podia dum fraco homem esperar maior serviço (ainda que
tal não pareça) que lançar mão desta informação da terra do Brasil
%Jorge: empreendeu
(cousa que até agora não empreendeu pessoa alguma) para que nestes Reinos
se divulgue sua fertilidade e provoque a muitas pessoas pobres que se
vão viver a esta província, que nisso consiste a felicidade e aumento dela.
\end{hedraquote}

Duas vezes repetida nesta carta dedicatória ao Cardeal"-Infante, bem como
no ``Prólogo ao Leitor'' da \textit{História da província Santa Cruz}, 
a recomendação da costa do Brasil para os
portugueses que estejam em pobreza na pátria é também empregada no
exórdio do livro impresso, sempre como argumento da utilidade do
sumário da terra, bem como dos serviços de quem o escreve, exercendo,
pois, dupla função, ao mesmo tempo retórica e política. Deveríamos
talvez supor que, sendo no mínimo gente de letras os seus leitores, a
condição de ``pobres e desamparados'' não designa o que as sinuosas 
classificações econômicas reconheceriam como
\textsc{c, d ou e}, nos sistemas de regulação por poder aquisitivo. Num corpo
político constituído por estados, ou ordens civis, os empobrecidos eram
mais provavelmente gente de baixa fidalguia, militares e letrados em
geral, cristãos"-velhos de ofício livre, possivelmente até
cristãos"-novos em processo de limpeza de sangue e lenta ascensão na
fidelidade institucional. A plebe mais baixa até estava implicada nesta
solução que o encarecimento da matéria do livro oferecia para a pobreza
dos pobres, mas estava suposta apenas como criadagem e companhia de
gente mediana, ou clientela de gente semiarruinada, ou ainda oficiais
mecânicos, mercenários, negociantes, exilados ou bandidos fugidos,
todos os pobres que por suposto também viriam, mas aos quais o livro de
Gandavo dificilmente se destinava, a não ser por exceção.

Tendo em vista os mecanismos da imensa hierarquia do império português,
o autor supõe demonstrar a utilidade de seu discurso colocando"-o a
serviço do necessário encorajamento às carreiras militares, jurídicas,
fiscais etc., na costa do Brasil, ainda designada província Santa
Cruz. Dirige"-se, portanto, à gente de carreira, isto é, gente
minimamente remediada para as instituições civis, educadas para o
preenchimento de funções direta ou indiretamente administrativas na
povoação e defesa dos novos domínios do reino, conforme o caso, mas
acometida por toda sorte de infortúnio que não fosse causado por crime.
Em outro uso, Antonio de Guevara, grande autoridade letrada do Império
Habsburgo no século \textsc{xvi}, avisava dos perigos da corte se dirigindo a
homens dignos de alguma distinção mas também empobrecidos por má
fortuna, recomendando os benefícios da recolha à província.\footnote{ Antonio 
de Guevara. \textit{Menosprecio de corte y alabanza de aldea} (1539).
Edición y notas de M. Martínez de Burgos. Madrid: Espasa"-Calpe, 1942.}
As diferenças entre os argumentos do elogio da aldeia, de Guevara, e da
recomendação feita nos exórdios dos breves tratados históricos da
terra, de Gandavo, são proporcionais aos decoros específicos, próprios
ao assunto e finalidade de cada livro. Em ambos os casos (e muitos
outros poderiam ser referidos), recomenda"-se o afastamento da corte
como um caminho difícil mas seguro para o exercício das boas virtudes
cristãs ou dos úteis serviços da fidelidade monárquica, com promessas
de prêmios nesta e na outra vida. Preenchendo o lugar retórico da
captação da benevolência --- no caso, pela defesa  e encarecimento da
matéria tratada ---, a recomendação de Gandavo, ainda que tenha
significação política, não precisaria ser lida com mais graves
consequências, ao menos não para que se pensem hipóteses sobre a origem
da pobreza do Brasil. A constituição política que está suposta nesta
exortação às carreiras do Novo Mundo ``dá sentido'', e talvez maior interesse, 
à descrição dos mantimentos da terra, bem como da barbárie dos nativos. 
Mas o sentido só ``é dado'' a nós que lemos o texto hoje, com outros hábitos de leitura. 
No seu tempo, o sentido já estava dado, salvo dissensos, que sempre existiram 
apesar dos sistemas de controle dos atos discursivos. 

``Primeiramente tratarei da planta e raiz de que os
moradores fazem seus mantimentos que lá comem em lugar de
pão.'' Assim começa o passo em que tratará da mandioca, de
que se extrai a farinha cuja técnica de uso Gandavo relata no
``Capítulo \textsc{v} --- Das plantas, mantimentos, e frutas que há
nesta província''. A mandioca é o análogo do trigo,
aproximado da raiz por sua finalidade, segundo modos de classificação
reconhecidos pelo português. Conforme a redação do \textit{Tratado} manuscrito:
\begin{hedraquote}
Nestas partes do Brasil não semeiam trigo nem se dá outro mantimento algum
deste Reino; o que lá se come em lugar de pão é farinha de pão: esta
se faz da raiz duma pranta que se chama mandioca, a qual é como inhame.
\end{hedraquote}

Como se dispõe a falar ``principalmente daquelas [plantas,
frutas e ervas], de cuja virtude e fruto participam os
Portugueses'' [Capítulo \textsc{v}], sempre segundo preceituam os
antigos retores do gênero histórico, a esta preeminência está suposto
que de trigo se faz pão, que é base da constituição alimentar do corpo
físico do vassalo cristão e base material para a consagração do Corpo
espiritual da comunidade de Cristo na eucaristia. Não por acaso a
mandioca, seu análogo, é o primeiro mantimento referido tanto na
\textit{História da província}, quanto no manuscrito do \textit{Tratado} da terra. Por
essa razão, também nas crônicas, tratados e histórias desta mesma
terra, escritos em português posteriormente, a mesma planta e seus usos
terão similar descrição e preeminência, sendo muitas vezes reiterada
sua analogia com o trigo, mesmo que o texto de Gandavo não tenha se
tornado fonte para a maior parte delas, devido a seu desaparecimento. O
caso da mandioca evidencia, portanto, que as descrições da fauna,
flora, costumes etc., eram redigidas segundo procedimentos
convencionais, orientados retórica, política e teologicamente.

O discurso de aconselhamento aos empobrecidos da pátria demonstra a
utilidade da matéria inventada, e por isso mesmo ocupa lugar
preeminente na escolha das matérias particulares de que tratará --- como
a abundância de mantimento, de água, de terra, de caça etc., e a
promessa de muitas pedras e metais ---, e que hoje se leem como elenco de
curiosidades especiosas da terra, por uma espécie de contaminação
turística de nossos tempos tristes, como outrora havia sido argumento
de ``porque me ufano'' de ser brasileiro, nas leituras românticas e
modernistas do texto, muitas vezes associadas institucionalmente a
jubileus comemorativos, como os cem anos da Independência, e aos
eventos bibliográficos, ou já editoriais, a eles ligados, direta ou
indiretamente. O aconselhamento aos empobrecidos não se contradiz pela
horrenda descrição dos índios e seus perigos, que se leem nas
terríveis, e exemplares, cenas de antropofagia que Gandavo produz na
última terça parte do livro. A exortação à carreira no ultramar não
seria de modo verossímil promissora em demonstração de virtude e heroísmo,
não houvesse obstáculos que a fé e a obediência do súdito deveriam
vencer para efetivamente melhorar de vida, isto é, adquirir dignidade
no interior da efetividade da instituição armada do Estado, e suas
demandas.

\section{gentilidade e justa dominação}

As terras que louvando se descrevem são entendidas como territórios
desertos, ou seja, ainda largamente desocupados de convívio civil.
Segundo o consenso dos padres e maiores potestades da Terra na
hierarquia do cristianismo romano, os naturais da terra viviam pouco
acima dos animais, isto entendido aristotelicamente como uso restrito,
precário ou torpe das potências superiores da alma. Em alguns casos,
como parece ser o de Gandavo, em geral a gentilidade da terra é
representada como incivil, algumas de suas nações são descritas, por
seus costumes, como bárbaras ou quase bestas, isso entendido como
deformidade da alma, causada pelos maus usos por séculos acostumados,
como se dizia.
\begin{hedraquote}
Já que tratamos da terra, e das coisas que nela foram criadas para o
homem, razão parece que demos aqui notícia dos naturais dela; a qual
posto que não seja de todos em geral, será especialmente daqueles que
habitam pela costa, e em partes pelo sertão dentro muitas léguas com
que temos comunicação. Os quais ainda que estejam divisos, e haja entre
eles diversos nomes de nações, todavia na semelhança, condição,
costumes, e ritos gentílicos todos são uns. E se em alguma maneira diferem
nesta parte, é tão pouco, que se não pode fazer caso disso, nem
particularizar cousas semelhantes, entre outras mais notáveis [\ldots{}]. 
Pela maior parte são bem dispostos, rijos e de boa estatura. Gente mui
esforçada e que estima pouco morrer, temerária na guerra e de muito
pouca consideração. São desagradecidos em grã maneira, e mui desumanos
e cruéis inclinados a pelejar e vingativos por extremo. Vivem todos mui
descansados sem terem outro pensamentos, senão de comer, beber, e matar
gente [\ldots{}].  São mui desonestos e dados à sensualidade, e assim se
entregam aos vícios como se neles não houvera razão de homens. Ainda
que todavia em seu ajuntamento os machos com as fêmeas tem o devido
resguardo, e nisto mostram ter alguma vergonha.

A língua de que usam,
toda pela costa é uma: ainda que em certos vocábulos difere em algumas
partes [\ldots{}]. Alguns vocábulos há nela de que não usam senão as fêmeas  e outros
que não o servem senão pera os machos. Carece de três letras, convém a
saber, não se acha nela, \textit{f}, nem, \textit{l}, nem, \textit{r} cousa digna de espanto, porque
assim não têm Fé, nem Lei, nem Rei: e desta maneira vivem
desordenadamente sem terem além disto conta, nem peso, nem medido.\footnote{ Página~\pageref{feleirei}.}
\end{hedraquote}

De saída, logo na introdução do novo assunto --- isto é, as coisas
relativas às gentes, sobre as quais discorrerá entre os Capítulos \textsc{x} e
\textsc{xiii} ---, a descrição exemplar da barbárie supõe indicativamente o modelo
teológico sobre o qual se assenta; por suposto, implica também o
reconhecimento da validade da alma imortal aristotélica e, por essa
razão, representa o índio como um ser humano, conforme a bula papal e o
Concílio de Trento, mas conduzido à animalidade pela destituição ou
descontrole das potências eternas da alma, ``como se neles
não houvera razão de homens'', a que se faz a ressalva de
que no coito seu costume demonstra pudor, vestígio de alma. Na
perspectiva católica que articula a descrição histórica do natural da
terra, só a louca fantasia e as voracidades do corpo poderiam, neste
sentido, governar os homens que por exemplo comessem o semelhante, e
mesmo o familiar, por costume e, pior, respeitante a leis. Em alguns
casos como o dos Aimorés, são considerados bestas, mas contra a
natureza, portanto por impiedade deliberada deles mesmos, assim
tornados, em termos católicos, pelo mau uso do livre"-arbítrio, que
levando a más práticas depois acostumadas por más tradições, segundo
uma interpretação que considere o texto de Gandavo como ortodoxo, do
ponto de vista dos direitos.

Numa interpretação diferente desta, João Adolfo Hansen entende a
descrição de Gandavo como adoção da tese herética sobre o gentio, que
foi opinião comum entre colonos portugueses no Brasil, partidários da
escravização indígena. Em ``A servidão natural do selvagem
e a guerra justa contra o bárbaro'', Hansen trata
basicamente dos modos de legitimação da sujeição dos indígenas levada a
termo por cronistas, viajantes, missionários, teólogos, juristas etc.
Ao descrever a pintura do indígena, em Gandavo, como um vegetal,
``uma erva má que afoga as boas ervas cristãs na passagem
em que declara ser impossível numerar e compreender a multidão de
bárbaro gentio que a natureza semeou pela terra do
Brasil'', Hansen propõe que essa interpretação, apoiada
nos interesses dos colonos, é contrária à Bula papal de 1537 que
decreta que os gentios e demais povos colonizados possuem alma,
``ou seja, eram gente como os católicos e que era vedado
escravizá"-los''. Hansen lembra principalmente que a tese
da animalidade dos ameríndios é ainda invalidada pelo Concílio de
Trento e que, em nome dela, se batia a Companhia de Jesus, nas diversas
reduções que estabeleceram os jesuítas na costa do Brasil.\footnote{ Ver João 
Adolfo Hansen ``A servidão natural do selvagem e
a guerra justa contra o bárbaro. In: Adauto Novaes (org.), \textit{A descoberta
do homem e do mundo}. São Paulo: Companhia das Letras, 1998, p.~354.}

De qualquer modo, como são quase bestas, o ato horrendo da antropofagia
que praticam não se representava, por exemplo, como a ação trágica da
ceia de Tiestes,\footnote{ Tiestes seduz a mulher de seu irmão Atreu. Atreu
mata os filhos de Tiestes e os serve num jantar. Tiestes amaldiçoa Atreu e sua descendência:
Agamênon e seu sobrinho Orestes. [\versal{N.}~do \versal{E.}]} também porque não se tratava de uma única
atrocidade exemplar para os séculos, sendo pelo contrário prática
acostumada por leis não escritas. Pela amplificação retórica do horror
dos maus costumes estranhos repunha"-se aí a naturalidade do costume
próprio, devendo o afeto produzido conduzir os leitores à virtude nos
usos e ofícios do mundo, estes sim bem acostumados às boas leis porque
sujeitos às verdadeiras fé e lei, segundo a doutrina. Esta costa do
Brasil é formada por terras incultas, tomadas por diversas nações que
variavam entre a dócil incivilidade, semelhante por vezes à pureza
original, e a horrenda barbárie, dessemelhante de toda outra conhecida
dos europeus dentre os piores exemplos das guerras pérsicas às
arábicas; tendo as amplificações por finalidade legitimar a conquista,
ainda que violenta. Assim pensado o texto de Gandavo, talvez tivéssemos
de reconhecer que sua leitura deveria ter pouquíssimo interesse
antropológico ou etnográfico, e sendo assim deveria ter pouco interesse
para até mesmo a História do Brasil assim entendida. A partir desses
desenhos da gentilidade das novas terras da costa do Brasil o que se
pode conhecer são modos de uma escrita e de legitimação dessa escrita,
que legitimam, por sua vez, o domínio sobre terras, homens, e mais
espécies. Como possessões recentes da República Cristã universal,
missionária, herdeira da função evangélica da instituição apostólica,
eram fundamentalmente terras em processo de redução à obediência dos
monarcas cristãos, em disputa na Europa ocidental. Em outras palavras,
uma empresa marítima de conquista do Novo Mundo como a monarquia
portuguesa era constituída por um estado"-maior em armas que entendia a
si próprio como a atualização autorizada dos domínios de certa tradição
de Pedro e Paulo.

Especificamente a empresa marítima lusitana pode ser entendida a partir
de uma união indissolúvel da Cruz e da Coroa, amplamente definida como
uma combinação de direitos, privilégios e deveres concedidos pelo
papado à Coroa de Portugal como patrona das missões civis e
eclesiásticas católicas em vastas regiões da África, da Ásia e Brasil.
São concedidos privilégios eclesiásticos à Ordem de Cristo (1455--56)
que legitimavam a obtenção da jurisdição espiritual sobre terras,
ilhas, lugares há pouco conquistados ou a se conquistar. Fundada em
1319 por Dom Dinis em substituição à Ordem dos Cavaleiros Templários,
possuía por chefia um membro da família real desde o tempo do Conde Dom
Henriques, e estava formalmente incorporada à Coroa, com o apoio de
outras duas ordens militares portuguesas, Santiago e Avis, pela bula
papal \textit{Praeclara Charissimi} (1551). Na dupla condição de reis de
Portugal e de governadores perpétuos da Ordem de Cristo, Dom Manuel e
seus sucessores tinham o direito do padroado sobre todos os postos,
cargos, benefícios e funções eclesiásticas nos territórios ultramarinos
confiados ao padroado depois que as terras não descobertas tivessem
sido, de fato, divididas entre as Coroas de Portugal e de Castela pelo
Tratado de Tordesilhas (1494),\footnote{ Cf.~Charles Boxer. \textit{O império
marítimo português}. São Paulo: Companhia das Letras, 2002.} já que
``nestes Reinos de Portugal trazem a Cruz no peito por
insígnia da ordem e cavalaria de Cristo'', segundo a
redação do Capítulo \textsc{i} desta \textit{História da província Santa Cruz}, de Gandavo.

Assim, nos atos do descobrimento português da nova terra e sua
reconquista das mãos dos gentios, dever"-se"-ia reconhecer que,
formalmente, o governo dos novos domínios foi entregue pelo Papa a Dom
Manuel \textsc{i}. Sua descendência herda a mesma disposição jurídica na
condição de que seus feitos fizessem jus àquele mérito herdado, isto é,
desde que as novas gerações de príncipes e seus principais varões
permanecessem reconhecidas na sujeição ao Papa e na soberania sobre o
corpo político do Estado. Uma vez reconhecidos os sinais divinos e os
pactos entre os varões da Terra, os territórios que se creem
restituídos à Cristandade recebem ordens que constituem as hierarquias
políticas, éticas, disciplinares, hagiológicas etc., cujo discurso é
homologado entre vária espécie de formulação útil e agradável de doutrina. 

\section{o império universal de cristo}

Segundo a rede de pactos vigentes ou em disputa entre as potestades
cristãs, o domínio da terra em âmbito português era prerrogativa do
Sumo Pontífice, que, herdeiro de Pedro, era só quem tinha suficiente
dignidade para distribuir o poder sobre a Terra entre
``aqueles Príncipes, a quem Deus fez como árbitros de
todos os negócios'', como está na Bula papal de publicação
do Concílio de Trento (1542).\footnote{ Na edição citada, p.~5.} Cada
monarca, por sua parte, deveria atuar como ministro, ou instrumento, do
vice"-Cristo apostólico e, para gerir as terras e nações postas sob sua
tutela, constitui corpos administrativos, militares, jurídicos,
universitários etc., que, na unidade que formam como corpo do Estado,
são sempre instituições mais ou menos diretamente responsáveis pela
própria manutenção das ordens e ofícios que compõem a mesma Monarquia,
na concórdia do bem"-comum. Como sabemos, com Kantarowicz, Hansen e
outros, a sustentação das ordens não é bem como a de um edifício mas a
do corpo humano, que é a matriz principal da analogia; mantido em pé,
ou instruído, pela alma. Todo o discurso de metafísica que existe na
base das formulações jurídicas que legitimam as ordens instituintes das
corporações no corpo do Estado funcionava, neste sentido, como uma
máquina de fazer almas; em outros termos, um discurso que, operando
categorias dadas como conhecidas a partir das autoridades filosóficas e
sapientes antigas, produz controle abstrato sobre os seres
particulares, por meio das disposições discursivas normativas feitas
para a sustentação das instituições que as redigem.

O rei católico devia gerir os estados, ou estamentos, sujeitos ao seu em
dignidade, isto é, as ordens de famílias inferiores à sua, que no
conjunto somam a totalidade de seus súditos leais e eventuais
revoltosos de todo tipo, subordinados pela força que o Estado detém
sobre todos os corpos particulares dos homens. Ao longo das sucessivas
gerações de príncipes e varões ilustres, e das tradições de antigos
contratos de fidelidade renovados entre os homens em meio a muita
guerra, diplomacia, ordenações, direito comum, depois ainda direito
natural e das gentes --- e desde antes de qualquer direito internacional,
princípios metafísicos de uma Lei que se nomeia Universal, Católica,
constituída sobretudo no Direito Canônico, em sucessivas unificações e
correspondências de códigos ---, os estamentos ordenados, bem como cada
cargo, função, distinção, privilégio, favor etc., eram em seu tempo
atualizações particulares da potência outorgada no rei, cabeça do corpo
político, pelas mãos do Papa.

A Igreja Católica faz, com isso, imitação do ato apostólico que
circunscrevia os ofícios de Deus aos doze, ao mesmo tempo que instituía
os sete diáconos, estado de homens destinados a funções instrumentais,
ou ministeriais, como diplomacia, leis comuns, proselitismo, martírio
missionário (Atos, \textsc{vi}, 1--7). Assumindo os encargos que os sacerdotes
por costume não deveriam assumir, como a guerra e a regência dos povos,
os reis instituíam"-se na hierarquia católica romana como potestades
armadas, assinaladas por Deus e reconhecidos por mérito, sempre segundo
a doutrina e jurisprudência que homologam o exercício de ofícios
letrados dignatários de alguma honra ou benefício, segundo graus,
funções, distinções de várias espécies nas hierarquias administrativas,
militares, acadêmicas, eclesiásticas de cada reino. 

Dentre as diversas carreiras abertas para a administra\-ção dos novos
impérios da Terra, esses homens de ofício livre, que, tendo adquirido
bem as letras, e tendo dignidade familiar e pessoal para isto, poderiam
fazer"-se autores de uma história,
como esta história descritiva do Brasil. Ainda dentro do mesmo gênero
de homens (ou melhor, de ofícios e dignidades civis) e dentro do mesmo
gênero discursivo, outros homens poderiam ser autores não de uma
história descritiva, mas de uma história dos atos heroicos e feitos
ilustres numa guerra e pacificação, ou a vida de reis, varões de armas,
de homens santos ou de homens sábios, segundo o modelo de Plutarco
principalmente. Conforme a inclinação natural, ou engenho, homens como
esses poderiam, ser autores de um panegírico bucólico para núpcias
reais, um auto sacramental para as exéquias da rainha, uma tragédia com
tema pátrio para a recreação e aviso da corte, uma epopeia para
infundir vontade heroica na mocidade sobretudo fidalga, e cristã velha;
coisas assim pensadas ao menos nas formulações de princípios deste
discurso. Assim, a posição do discurso histórico era neste sentido
análoga à da autoria de poemas laudatórios, romances pastoris, comédias
piedosas, sátiras morais, poemas heroicos etc. etc., ou ainda
preceptivas e tratados também de variadíssimas formas; cada espécie
discursiva obedecendo quase sempre, mesmo nos desvios, a decoros
políticos, segundo a natureza de seu assunto e de sua destinação institucional.

Neste sentido, tanto a invenção d'\textit{Os Lusíadas} (1572),
de Camões, como a da \textit{História da província Santa Cruz}, de Gandavo,
escolhem seu assunto entre matérias de um mesmo gênero, a saber, o
conjunto das coisas notáveis relativas ao descobrimento e conquista de
novas terras para os domínios da Cristandade pelos varões de Portugal.
O primeiro, contudo, é redigido em gênero de elocução elevado, próprio
para a invenção heroica do poema épico, mais ornado já por ser poesia,
e menos afeito ao especioso da história. A invenção histórica do
segundo, por sua vez, é especiosa, porque a história, quase que por
definição, perscruta sua matéria entre os particulares. No caso, uma
vez que se tratasse de uma história desta terra tão
``pouco sabida'', que há pouco não tinha vestígio de fé, lei ou rei, 
e que apenas recentemente vinha sendo
ocupada por ordens cristãs, era necessariamente ainda muito pouco
noticiosa de feitos heroicos para encarecer a narração. Por isso, os
particulares e específicos que se relatam e sobretudo se louvam no
tratado demonstrativo da terra são quase sempre seres inferiores na
hierarquia dos seres, segundo as apropriações católicas da física
aristotélica por intermédio da história natural latina. Trata, pois, de
minerais, vegetais, animais, gentios; e entre estes, nações mansas e
bravas, dóceis e danosamente bárbaras. Em meio disso, alguns vassalos
do rei, homens de armas, imitados como da espécie dos heróis, que é o
sujeito por excelência do louvor no gênero histórico. Tal é o caso de
Fernão de Sá, filho do Capitão"-general da província e morto em
fidelidade na mão dos índios, constituindo pelo heroísmo distinção
familiar oficial, salvo desgraça ou infortúnio dos herdeiros.

Segundo os modelos do Capítulo \textsc{i} --- a saber, o capitão Pedro Álvares Cabral
e o historiador de seus feitos, ``aquele ilustre e famoso
escritor João de Barros'' ---, o herói bem como a autoridade
letrada eram, genericamente, o tipo do vassalo fiel, zeloso de renome,
numa representação política efetuada por meio dos principais lugares
demonstrativos da fidelidade estamental, do reconhecimento dos pactos,
da hierarquia Universal, e assim por diante. E antes de mais nada, o
mais elevado herói político da história de que tratamos, que não por
acaso está na primeira linha do primeiro capítulo da história da
província: ``Reinando aquele mui católico e sereníssimo
Príncipe el"-Rei Dom Manuel, fez"-se uma frota para a Índia de que ia por
capitão mór Pedro Álvarez Cabral; que foi a segunda navegação que fizeram
os portugueses para aquelas partes do Oriente''.

Já que a ``escritura seja vida da memória'',
como já sabemos, o autor arrazoa retoricamente (e citando Cícero) os
motivos que o levaram a fazer esta breve história, 

\begin{hedraquote}
para cujo ornamento não busquei epítetos
esquisitos, nem outra formosura de vocábulos de que os eloquentes
oradores costumam usar, para com artifício de palavras engrandecerem
suas obras. Somente procurei escrever esta na verdade, por um estilo
fácil e chão, como meu fraco engenho me ajudou, desejoso de agradar a
todos os que dela quiserem ter notícia.
\end{hedraquote}

O autor aí define para a ``breve história'' o
ornamento da sentença escrita em estilo simples, ou natural, no que
está suposto que se trate provavelmente de uma opção de escola que
define o decoro elocutivo da prosa histórica mantido não longe de uma
dicção média, por isso clara, não muito ornada, própria à instrução,
que é o fim das espécies do discurso didático, como a história
descritiva da terra de que tratamos.

Camões e Gandavo, como homens de Letras, desfrutam certo grau de
paridade nas posições hierárquicas que ocupam, com igualmente provável
diferença de fidalguia que os distingue para as funções que cada um
comprovasse ser digno por nascimento e mérito. Com efeito, o já então
autor de \textit{Os Lusíadas} recomendava um protetor para a \textit{História da
província Santa Cruz} na posição civil de um amigo. E é provável mesmo,
por hipótese, que Camões exerça algum tipo de autoridade efetiva na
obtenção de alguma vantagem, ainda que seja apenas, por exemplo, a da
publicação impressa da dita obra. O certo é que, como temos visto,
ambas as obras saem pela mesma oficina tipográfica, esta em 1576,
aquela em 1572, em pleno reinado de Dom Sebastião. A primeira, contudo,
é publicada com privilégio real e é dedicada à pessoa do monarca,
enquanto a segunda é apenas aprovada e recomendada pela censura, sem
privilégio, sendo dedicada a um ilustre. A diferença de posição entre
os dois autores no interior da paridade que os aproxima provavelmente
dava a cada um dignidades de acesso diferentes entre os estados, ou
estamentos, da hierarquia, parecendo suposto n'\textit{Os Lusíadas} 
que a persona que a si encena na autoria tem algum privilégio
de privança real, o que não se vê na autoria da \textit{História da província
Santa Cruz}; autoria entendida como uma cena autoral, que é tanto
retórica quanto política. A diferença de posição é proporcional à
encenação de um certo caráter, ou \textit{\=ethos}, histórico, diverso de um 
ethos heroico, poético, que constituem justamente as máscaras das
\textit{person\ae} que os autores de história e de poesia épica encenam. Seja
como for, a terça"-rima alegórica de Camões ``Ao muito
ilustre senhor Dom Lionis Pereira'' (impresso como Elegia
\textsc{iv} nas edições camonianas) encena a antiga tópica das letras e armas,
louvando por Hermes aqueles homens de armas, como o capitão, governador
em Malaca a quem o livro é dedicado, que cuidam da posteridade da
própria fama promovendo as letras, ciências e artes livres; e com o
louvor destes varões ilustres, ao menos no esquema moral da recepção
coetânea da obra, pretende"-se exortar os então atuais atores da
potência do Império cristão a providenciar a posteridade de seus feitos
com a cultura das letras que o perpetuam.

\begin{bibliohedra}
\tit{Anônimo}\ [Cícero]. \textit{Retórica a Herênio}. Edição bilíngue, tradução
e introdução de Ana Paula Calestino Faria e Adriana Seabra. São Paulo:
Hedra, 2005.
\tit{Arendt}, Hannah. \textit{O conceito de história: antigo e
moderno.} In: Entre o passado e o futuro. São Paulo: Perspectiva,
1988. p.~43--68. 
\tit{Arroyo}, Leonardo. \textit{A carta de Pero Vaz de Caminha: ensaio de
informação à procura de constantes válidas de método.} 2\ai edição. São
Paulo: Melhoramentos;  Brasília: \textsc{inl}, 1976.
\tit{Barros}, João de \& \textsc{Couto}, Diogo. \textit{Da Asia de João de Barros e de
Diogo de Couto Nova Edição offerecida D. Maria \textsc{i} rainha fidelíssima
\&c. \&c. \&c}. Lisboa: Regia Officina Typographica, 1778.
\tit{Boxer}, Charles. \textit{O império marítimo português, 1415--1825. } São
Paulo: Companhia das Letras, 2002.
\tit{Camões}, Luiz de. \textit{Obras de Luiz de Camões precedidas de um ensaio
biographico no qual se relatam alguns factos não conhecidos da sua vida
pelo visconde de Juromenha}, volume \textsc{i}. Lisboa: Imprensa Nacional, 1860.
\tit{Correa}, Gaspar. \textit{Lendas da India}.
Tomo \textsc{iii}, parte \textsc{ii}. Lisboa: Typographia da Academia Real
das Sciencias, 1863.
\tit{Couto}, Diogo do. \textit{Decada setima da Asia dos feitos que os
portugueses fizeraõ no descobrimento dos mares}. Lisboa: Pedro Craesbeeck, 1616. 
\tit{Curtius}, Ernst. \textit{Literatura Europeia e Idade Média Latina}. São
Paulo: Hucitec/Ed\textsc{usp}, 1996.
\tit{Elias}, Norbert. \textit{A sociedade de Corte: investigação sobre a
sociologia da realeza e da aristocracia de corte}. Tradução de Pedro
Sussekind; pref.~Roger Chartier. Rio de Janeiro: Jorge Zahar
Editor, 2001.
\tit{Guevara}, Antonio de. \textit{Menosprecio de corte y alabanza de aldea}.
Edición y notas de M. Martínez de Burgos. Madri: Espasa Calpe, 1942.
\tit{Hansen}, J.A. \textit{A sátira e o engenho: Gregório de Matos e a Bahia
do século \textsc{xvii}}. 2\ai edição. São Paulo: Ateliê Editorial/Unicamp,
2004.
\titidem. ``Agudezas seiscentistas''. In: \textit{Floema
Especial}, ano \textsc{ii} 2A, Vitória da Conquista: Edições Uesb, 2006. 
\titidem. \textit{Autor, obra e público nas letras luso-brasileiras 
dos séculos \textsc{xvi, xvii} e \textsc{xviii}.} (Workshop: Universidade
Estadual do Sudoeste da Bahia, Vitória da Conquista, outubro 2006).
\titidem. ``A servidão natural do selvagem e a
guerra justa contra o bárbaro'' In:  \textit{A descoberta do homem e do mundo}. 
Adauto Novaes (org.). São Paulo: Companhia das Letras, 1998.
\titidem. ``Introdução'' In: \textit{Poesia seiscentista.} Alcir Pécora (org.). São Paulo:
Hedra, 2002.
\tit{Kantarowicz}, Ernest H\textit{. Os dois corpos do rei.} Tradução de Cid
Knipel Moreira. São Paulo: Companhia das Letras, 1998.
\tit{Koselleck}, Reinhart. \textit{Futuro Passado: contribuição à
semântica dos tempos históricos.}  Rio de Janeiro: Editora da
Pontifícia Universidade Católica do Rio de Janeiro, 2006.
\tit{Lobo}, Francisco Rodrigues. \textit{Corte na aldeia e noites de inverno}.
Edição de Afonso Lopes Vieira. Lisboa: Sá da
Costa, 1972.
\tit{Mendonça}, Hieronimo. \textit{Iornada de Africa}.
Lisboa: Off. de Joze da Silva Nazareth, 1785.
\tit{Pécora}, Alcir. ``A arte das cartas jesuíticas do Brasil''. In: \textit{Máquina de
Gêneros}. São Paulo: Editora da Universidade de São Paulo, 2001. 
\titidem. ``Vieira, o índio e o corpo místico''. In: \textit{A
descoberta do homem e do mundo}. Adauto Novaes (org.). São Paulo:
Companhia das Letras, 1998.
\tit{Pliny the Elder}. \textit{The Natural History}. John Bostock, \textsc{m.d., f.r.s.
h.t.} Riley, Esq., B.A. London: Taylor and Francis, Red Lion Court,
Fleet Street. 1855.
\titidem. \textit{Naturalis Historia}. Karl Friedrich Theodor
Mayhoff. Lipsiae. Teubner, 1906.
\tit{Santos}, Manoel. \textit{Historia Sebastica}. Lisboa: Officina de Antonio Pedrozo Galram, 1735. 
\tit{Seed}, Patricia. \textit{Cerimônias de posse na conquista europeia do novo mundo}, 1492--1640; 
Tradução de Lenita R. Esteves. São Paulo: Editora
Unesp, 1999.
\tit{Todorov}, Tzvetan. \textit{A conquista da América: a questão do Outro}; Tradução de 
Beatriz Perrone-Moisés. São Paulo: Martins Fontes, 2003.
\tit{White}, Hayden. \textit{Meta"-história: a imaginação histórica do século \textsc{xix}};
tradução José Laurêncio de Melo. São Paulo: Ed\textsc{usp}, 1995.
\tit{Zumthor}, Paul. \textit{A letra e a voz: a ``literatura'' medieval.}; Tradução de Amálio
Pinheiro, Jerusa Pires Ferreira. São Paulo: Companhia das Letras, 1993.

\medskip
\tit{\textbf{dicionários e periódicos}}

\tit{Collecção}\ \textit{de opúsculos reimpressos relativos a historia
das navegações, viagens e conquistas dos portugueses pela Academia Real
das Sciencias}, Tomo \textsc{i}, n\oi \textsc{iii}, 1858.
\tit{Revista}\ \textit{Trimensal do Instituto Historico Geographico e
Ethnographico do Brasil, fundado no Rio de Janeiro debaixo da immediata
protecção de \textsc{s.~m.~i.} o senhor D.~Pedro \textsc{ii}}. Tomo \textsc{xxii}. Rio de Janeiro:
Typ.~Imparcial de J.~M.~N.~Garcia, 1859.
\tit{Bluteau}. \textit{Vocabulario portuguez e latino}. Coimbra: Collegio das Artes da Companhia de
Jesus, 1712. \enlargethispage{\baselineskip}
\tit{Pereyra}, D. Bento. \textit{Thesouro da Lingoa portuguesa}. Lisboa: Paulo Craesbeeck, 1647.
\end{bibliohedra}
