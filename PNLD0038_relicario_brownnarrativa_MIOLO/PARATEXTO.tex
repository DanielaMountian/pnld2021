\part{Paratexto}

\chapter{\textsc{W.\,W.}\,Brown e o tráfico negreiro}

\section{Sobre o autor}

William Wells Brown (1814--1884) foi um abolicionista, romancista, dramaturgo e historiador afro"-americano. Nascido escravo, no Kentucky, fugiu para a liberdade aos 20 anos de idade e, aos 33, publicou essa \textit{Narrativa}, já estabelecido em Ohio, na qual conta a história de sua vida nos estados do Kentucky e Missouri, onde trabalhou como aprendiz em um jornal, transportando escravos para a venda em Nova Orleans e em diversas outras atividades. A \textit{Narrativa} reconta os horrores da escravidão, o tráfico interno de escravos nos Estados Unidos da América e a relação de Brown com seus donos e familiares. Seu autor, no entanto, não hesita em revelar seus vícios e defeitos, destacando assim a individualidade que se desenvolveu sob uma instituição totalizante e desumanizadora, que via em homens e mulheres apenas braços para a lavoura e ventres para uma nova geração de cativos. A \textit{Narrativa} é uma crítica à ganância e à hipocrisia religiosa, ao preconceito e à violência, mas, acima de tudo, é uma proclamação da humanidade do seu autor e de todos que sofreram ao seu lado.

Através de sua narrativa, o leitor tem uma aproximação de dentro, em perspectiva única, do que foi o escravismo sulista norte-americano, que às vésperas da abolição da escravatura contava com 4 milhões de escravizados em seus campos de trabalho. Nesse relato o leitor vai encontrar a descrição do universo familiar de Brown e suas resistências diárias, mas também a exploração do trabalho e a violência simbólica e física cometida pelos senhores brancos. Seu testemunho sugere que o objetivo da escravidão negra nas Américas era análogo à utopia autoritária do capital no século \textsc{xxi}: desumanizar o ser humano até reduzi"-lo à condição inanimada e sedutora de uma mercadoria.
Brown faleceu no ano de 1884, em Chelsea, Massachusetts.

\section{Sobre a obra}

Assim como outros livros de escravizados fugidos do sul para o norte dos
Estados Unidos, publicados no século \textsc{xix}, a \emph{Narrativa de William
Wells Brown, escravo fugitivo. Escrita por ele mesmo} também se situa no contexto social e
econômico norte"-americano do período.
A obra contém um apelo ao fim da escravidão nos Estados Unidos
e no resto do mundo. Também nos é mostrado todo o ambiente de violência
física e psicológica que o sistema escravista imprimia sobre as pessoas
por ele vitimadas.

Nascido escravo, um dos pontos mais marcantes da trajetória de William
Wells Brown ocorreu quando ele tinha apenas 16 anos. Com essa idade, foi
vendido por seu senhor para um traficante de escravos no Rio Mississipi
e acabou trabalhando com brancos escravizadores, participando do tráfico
de pessoas para o trabalho forçado. Esse fato acabou sendo crucial para
o desenvolvimento das experiências relatadas por Brown enquanto
escritor, por permitir compreender o funcionamento do sistema escravista
nos Estados Unidos, embora tenha sido polêmico para sua reputação.

Nove anos após sua fuga, William Wells Brown começou a dar palestras
para o público abolicionista dos estados do Norte, contando sua história
com uma boa retórica e já iniciando os preparativos para a escrita deste
livro, publicado originalmente em 1847.

Logo em seguida, ele se fixou em Boston, local que, no século \textsc{xix}, era o
grande centro do abolicionismo estadunidense. Lá ele se engajou à causa
abolicionista, estudou e se tornou um grande escritor. Além da causa da
abolição, Brown também defendeu outras pautas sociais, como o sufrágio
feminino, a redução do consumo de álcool e a reforma penitenciária,
entre outras questões. Em meados de 1847, publica sua grande obra,
\emph{Narrativa de William Wells Brown, escravo fugitivo. Escrita por ele mesmo}.

Em suas páginas podemos ler sobre toda a crueldade praticada
contra os escravizados e a brutalidade do tráfico de pessoas que foram presenciadas pessoalmente pelo autor.
Brown testemunhou também outra triste característica da escravidão
norte-americana, marcada, entre outras tragédias, pela indiferença com
que eram realizadas as separações de familiares. O próprio autor foi
separado de sua mãe, irmão e irmã, um dos motivos que mais lhe doía para tentar fugir.

Essas experiências tornam pungente o relato de Brown e revelam a
violência por trás do sistema escravista nos Estados Unidos do século
\textsc{xix}.

É, portanto, fundamental observar a obra como o relato da vida de um
ex"-escravizado norte"-americano que conseguiu fugir de um sistema
escravista cruel e estruturado. O relato autobiográfico de Brown permite
refletir sobre o universo da escravidão a partir do testemunho de uma
das vítimas do regime. No prefácio à primeira edição, J.\,C. Hathaway
assinalou que William Wells Brown vivenciou quase que todos os aspectos
do sistema escravagista norte"-americano, revelando como todas as
instituições do país estavam voltadas para sustentar a submissão racial
e servil dos cativos.

Além disso, é uma obra que teve grande peso na luta abolicionista.
Deve-se lembrar que os Estados do Sul chegaram a ter uma população de
negros escravizados orbitando entre 51 e 54\% do total populacional.
Esse número é extremamente alto e demonstra a forte estruturação do
sistema escravista. Além do mais, nesse período, o valor de um negro
escravizado nos Estados Unidos era tão elevado que perdia apenas para a
propriedade de terras.

Vale ressaltar também que Brown sempre acreditou nas tradições
anglo"-americanas e cristãs protestantes, vindas de séculos anteriores,
enquanto buscava o imediato fim da escravidão e da subordinação racial
nos Estados Unidos. Por isso, o autor foi visto como um reformista e não
exatamente como um revolucionário, e essa classificação impactou a
difusão de suas ideias.

O relato de William Wells Brown surge na metade do século \textsc{xix},
um período em que a escravidão
norte"-americana estava consolidada, anteriormente à Guerra Civil
(1860--1865) que romperia com a estrutura escravagista do país. Nesse
período, outras obras com a mesma temática foram publicadas, como as de
Frederick Douglass --- que chegou a discutir publicamente com Brown --- e
a de Harriet Ann Jacobs --- autora de \textit{Incidentes da vida de uma escrava}
---, em 1861. Portanto, vemos que William Wells Brown foi um homem
atuante em seu tempo, defendendo a liberdade dos negros e contribuindo
com o conjunto de publicações de autobiografias de ex"-escravizados.

Além de autobiografia, o livro de Brown, assim como outros do mesmo
gênero, é uma excelente fonte de documentação histórica e revela a
relação opressiva e a subordinação racial e servil imposta pelos brancos
escravistas aos negros escravizados no sul dos Estados Unidos do século \textsc{xix}.

Outro elemento que merece destaque é o fato de que a obra, assim como
outras autobiografias de negros escravizados, visavam combater a
literatura sulista, branca e escravocrata, a qual descrevia a vida nas
fazendas do Sul como algo idílico e bom, tanto para os brancos senhores
de terras quanto para os negros escravizados. Até mesmo um termo foi
criado para caracterizar esse movimento contrário à literatura
escravista: ``a escola literária do fugitivo heroico norte"-americano''.
Sobre essa vertente literária, cabe ressaltar que a maioria dos
escritores a ela relacionados saíam dos estados escravistas mais
próximos ao Norte do país, ou seja, esses escritores fugidos da
escravidão não estavam nas fazendas de algodão do extremo sul dos
Estados Unidos. Esse dado geográfico promove um recorte histórico e
social na realidade retratada pela maior parte desses textos.


\section{Sobre o gênero}

O gênero do relato, em linhas fundamentais, define"-se como a narrativa em que um sujeito, inscrito em um determinado tempo histórico, debruça"-se sobre fatos, descrições e interpretações desse momento histórico no qual vive. Para o historiador francês Paul Veyne, o relato histórico segue uma forma similar à forma tradicional de escrever história, seguindo um \textit{continuum} espaço"-temporal.

Apesar dessa relativa unidade, o relato, considerado como uma forma de fazer história,
é parcial e subjetivo, pois não consegue apreender a globalidade dos acontecimentos, apenas
aquilo que está ao alcance do narrador e, mesmo isso, não de uma forma pura, mas filtrado pela sua subjetividade e pelos objetivos de seu relato.
Para Veyne, estaríamos assim quase próximos do romance:

\begin{quote}
A história é uma narrativa de eventos: todo o resto resulta disso. Já que é, de fato, uma narrativa, ela não faz reviver esses eventos, assim como tampouco o faz o romance; o vivido, tal como ressai das mãos do historiador, não é o dos atores; é uma narração,
o que permite evitar alguns falsos problemas. Como o romance, a
história seleciona, simplifica, organiza, faz com que um século
caiba numa página.\footnote{\textsc{veyne}, Paul. \textit{Como se escreve a história}. Brasília: Editora Universidade de Brasília, 1999, p.\,18.}
\end{quote}

Seguindo nessa linha de pensamento, podemos observar, com o historiador francês Marc Bloch, que o relato é apenas um ``vestígio'' da história, um pequeno pedaço do factual que, pela pena de um narrador, pôde"-se cristalizar no tempo e ser transmitido a gerações posteriores, sendo apenas uma das infinitas possibilidades de apreensão e compreensão de determinados fenômenos:

\begin{quote}
Quer se trate das ossadas
emparedadas nas muralhas da Síria, de uma palavra cuja forma ou emprego revele um
costume, de um relato escrito pela testemunha de uma cena antiga [ou recente], o que
entendemos efetivamente por documentos senão um ``vestígio'' quer dizer, a marca,
perceptível aos sentidos, deixada por um fenômeno em si mesmo impossível de captar?\footnote{\textsc{bloch}, Marc. \textit{Apologia da história}. Rio de Janeiro: Zahar, 2002, p.\,73.}
\end{quote}

No caso desta \textit{Narrativa de William~Wells Brown, escravo fugitivo}, evidencia"-se seu caráter de relato pois Brown estava circunscrito a um determinado momento da escravidão norte"-americana e pôde transmitir à posteridade o que viu e vivenciou nessa circunstância.
É um relato, dado o seu caráter histórico, mas também uma autobiografia, pois perpassa toda a trajetória do autor, da separação dos pais para ser escravizado à emancipação no Norte.
Dentro do gênero biográfico, a narrativa ainda se inscreve na categoria, muito popular nos Estados Unidos, das ``narrativas de escravos''
(\textit{slave narratives}), autobiografias de ex"-escravizados focadas nas histórias de liberdade que detalhavam a reação de seus autores à escravidão e seus caminhos para a liberdade. No Estados Unidos, surgiram em princípios da década de 1820, quando escritores começaram a compor obras contra a emergente ficção regionalista do Sul, que representava a vida dos brancos e negros nas fazendas escravistas como um jardim das delícias.

Com a publicação de várias autobiografias de ex"-escravizados que fugiram do cativeiro --- como as de Harriet Jacobs e Frederick Douglass, e os relatos do \textsc{fwp} --- as narrativas de escravos consolidaram"-se como um gênero literário em meados do século \textsc{xx}, quando começaram a despontar mais narrativas, e passaram gradativamente a ser aceitas pelos historiadores e acadêmicos como peças fundamentais para entender a escravidão nas Américas.

Na definição do historiador William L.\,Andrews, professor da Universidade da Carolina do Norte, as ``narrativas de escravos'' podem ser entendidas como ``qualquer relato da vida, ou uma parte importante da vida, de um fugitivo ou ex"-escravo, escrito ou relatado oralmente pelo próprio escravo''.\footnote{Disponível em \emph{http://nationalhumanitiescenter.org/tserve/freedom/1609-1865/essays/slavenarrative.htm}.}

Para Calvin Schermerhorn, pesquisador e professor da Universidade Estadual do Arizona,
a \emph{Narrativa} de William Brown segue exatamente as regras básicas do gênero
das narrativas de escravos:

\begin{quote}
Ela começa com uma nota de agradecimento ao
Quaker branco que o salvou do frio e da fome durante a sua fuga, o Wells
Brown de quem ganhou o nome. Dois prefácios adicionais atestam a
veracidade e o caráter do autor. Um foi escrito por Joseph C. Hathaway,
um Quaker de Farmington, Nova York, participante ativo da Ferrovia
Subterrânea, a rede de refúgios e meios de transporte que tiravam
escravizados do Sul, e apoiador da Sociedade Antiescravista do Oeste de Nova
York. Brown dera palestras em Farminton e morara na cidade em meados da
década de 1840. Suas filhas estudaram lá e Hathaway estava posicionado
para dar credibilidade a Brown junto a um público que não o conhecia. O
segundo endosso, mais prestigioso, foi escrito pelo abolicionista Edmund
Quincy, da Sociedade Antiescravista de Massachusetts. Quincy declara que
o texto da \emph{Narrativa} foi mesmo escrito por Brown e apresentado a
ele para ser revisado, admitindo até mesmo que editou o texto original
para ``corrigir alguns equívocos (\ldots{}) e sugerir alguns
abreviamentos''.\footnote{Página \pageref{ref1} desta edição.} Juntos, os prefácios
preparam o público para acreditar nos relatos do autor. Em uma ironia
que fortaleceu ainda mais a autenticidade de Brown, seu último
proprietário, Enoch Price, de St.\,Louis, enviou uma carta oferecendo"-se
para vender Brown no início de 1848, na qual ele supostamente
``reconheceu a veracidade substancial {[}da \emph{Narrativa}{]}''.\footnote{\textsc{schermerhorn}, Calvin. ``Introdução''. In: \textsc{brown}, William Wells. \textit{Narrativa de William Wells Brown,
escravo fugitivo. Escrita por ele mesmo}. São Paulo: Hedra, 2020, p.\,14--15.}
\end{quote}

A primeira narrativa de escravo que se tem história é a 
\textit{Interesting Narrative of the Life of Olaudah Equiano, or Gustavus Vassa, the African}
(1789), em que Olaudah Equiano narra a sua trajetória de vida desde sua infância na África Ocidental, passando pelo tráfico transatlântico, sua situação de cativo, e encerrando"-se com sua liberdade e o sucesso financeiro como um cidadão britânico.
A história de Equiano, publicada na Inglaterra, chocou o público ao revelar os horrores da escravidão e balançou o senso"-comum da época que via na escravidão uma bem aceita e estabelecida instituição socioeconômica inglesa.

Já nos Estados Unidos da América, a primeira narrativa de escravo foi a \emph{Life of William Grimes, the Runaway Slave, Written by Himself}, de 1825.
Com seu relato, William Grimes revelou aos leitores do Norte o verdadeiro terror do cativeiro nos estados do Sul e as injustiças raciais da Nova Inglaterra.
Essas autobiografias que deram início ao gênero eram, na análise de Schermerhorn, 
``um híbrido de gêneros diversos, incluindo narrativas de cativeiro, literatura de protesto, confissão religiosa e relatos de viagem''.\footnote{\textsc{schermerhorn}, Calvin, op.\,cit., p.\,9.}

No entanto, como ressalta Schermerhorn, a maioria dessas biografias foram publicadas com auxílio editorial de brancos e para públicos brancos.  “Assim, desde os primeiros momentos da autobiografia negra na América, domina a pressuposição de que o narrador negro precisa de um leitor branco para completar o seu texto, para construir uma
hierarquia de significância abstrata referente ao simples
conjunto dos seus fatos, para oferecer uma presença onde
antes havia apenas um `Negro', uma ausência escura.”\footnote{\textsc{andrews}, William L. \textit{To Tell A Free Story: The First Century of Afro-American Autobiography, 1760–1865}, 1988, p.\,32--33 \textit{apud} \textsc{schermerhorn}, Calvin, op.\,cit., p.\,9--10.}

Desse conflito entre um gênero escrito por negros no meio de uma sociedade branca e racista, desponta outra característica das ``narrativas de escravo'', a modelação de sua história para ser bem"-aceita pelo público leitor branco:

\begin{quote}
O testemunho era a pedra fundamental da autobiografia de um ex-escravizado, e um dos principais desafios artísticos enfrentados pelos autores negros foi como atrair a
simpatia do leitor para que enxergassem as cenas de subjugação da maneira apresentada. Muitas vezes, isso significava aceitar boa parte da cultura anglo-americana como
normativa, sugerindo que a civilidade dos brancos seria a
razão por que era inaceitável a crueldade dos escravizadores. Os americanos civilizados não deveriam tolerar o barbarismo da escravidão. Para os escritores negros, isso
significou atenuar a importância das tradições, culturas, religiões e idiomas da África e dar preferência aos valores e tradições dos povos descendentes de europeus. Significou atenuar o radicalismo e a militância que caracterizou líderes descendentes de escravizados como Toussaint ou Jean-Jacques Dessalines no Haiti.\footnote{Ibidem, p.\,10.}
\end{quote}

Essa adequação aos valores e tradições eurocêntricos pode ser vista mesmo na forma de se portar e vestir. Após adotar o nome do Quaker branco que o salvou em Ohio (Wells Brown) e retomar o William, com o qual sua mãe o chamava --- um importante ato de autopossessão, na definição de Calvin Schermerhorn ---, William Wells Brown passou por uma transformação externa. Apesar do cabelo afro, podemos ver na imagem de frontispício da 1ª edição de sua \emph{Narrativa}\footnote{Página \pageref{front} dessa edição.} que ele se portava como um cavalheiro americano, destacando em sua aparência e até em sua assinatura
``uma afiliação de classe com os cidadãos
brancos americanos de classe média''.\footnote{Ibidem, p.\,18.}

Como relembra Schermerhorn, recuperando uma análise do
filósofo norte"-americano \textsc{w.e.b.} Du Bois, esse movimento inauguraria uma ``dupla
consciência'', marcada pela divisão do sujeito entre sua identidade negra norte"-americana e, em projeção para o exterior, uma atenção grande à forma como os negros são vistos pelos brancos, enxergando"-se da mesma forma que os brancos. 
Para Schermerhorn, no entanto, essa ``dupla consciência'' é mais profunda e significativa do que aparenta na conceituação de Du Bois:

\begin{quote}
A \emph{Narrativa} de Brown tem elementos de dupla consciência quando
explora o emaranhado de identidades e perspectivas envolvidas em ser um
ex"-escravizado e uma pessoa de ascendência africana que tenta salvar o
projeto da democracia americana, expressando"-se para um público que
suspeita da sua aparência, tradição e legitimidade. O filósofo Cornel
West chamou isso de ``a crise tripla da autoconsciência'', que era um
anseio por ser parte da alta cultura cosmopolita dominante ao mesmo
tempo que se tinha um histórico provinciano, uma herança de migração
forçada e a identidade social de um homem negro sem status.\footnote{Cornel
  West, \emph{Prophecy Deliverance!} \emph{An Afro"-American
  Revolutionary Christianity} (Louisville, Ky: Westminster John Knox
  Press, 2002, p.~31).} Durante toda a sua carreira, Brown, assim como
muitos outros abolicionistas negros, teve dificuldade para ser
reconhecido como um cavalheiro, um homem letrado, culto e refinado. Na
\emph{Narrativa}, ele enfrenta esse drama ao apresentar"-se como um
afrodescendente escravizado, mas também como um americano que acreditava
em liberdade pessoal, na democracia e no Cristianismo protestante. Ele
argumenta que os Estados Unidos eram uma república de escravizadores,
alicerçada na supremacia branca. Em vez de conclamar pela sua
destruição, no entanto, ele insistia na sua reforma. A escravidão
precisava ter fim, e a subordinação racial, também.\footnote{\textsc{schermerhorn}, Calvin, op.\,cit., p.\,19.}
\end{quote}