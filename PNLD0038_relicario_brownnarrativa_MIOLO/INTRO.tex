\chapter*{W.\,W.\,Brown e o tráfico negreiro}
\addcontentsline{toc}{chapter}{W.\,W.\,Brown e o tráfico negreiro, \emph{por Calvin Schermerhorn}}

\begin{flushright}
\versal{CALVIN SCHERMERHORN}\\\vspace*{-3pt}
\versal{ARIZONA STATE UNIVERSITY}
\end{flushright}

%\section{Escravismo e tráfico negreiro na~República~do~Algodão}
\section{Sobre o autor}
%Além dos seus triunfos literários, a \emph{Narrativa} de Brown é uma
%fonte histórica excepcionalmente sofisticada sobre a escravidão e a vida
%dos afro"-americanos.

Nascido no Kentucky em 1814, Brown era o mais novo
de sete filhos de Elizabeth, que o deu à luz quando tinha 20 ou 25 anos.
A escravidão o privou de ter pai. Seu pai biológico foi um homem branco,
George W. Higgins (1785--1835), que se recusou a reconhecê"-lo como
herdeiro. Deserdar os filhos era uma prática comum entre os homens
brancos que estupravam e engravidavam mulheres afrodescendentes. Era
mais comum do que se imagina a ascendência europeia de negros escravizados. O virginiano John Brown lembra que, em criança, conheceu seu avô
igbo (africano) ``quando ele veio visitar minha mãe. Ele era bem
negro''.\footnote{John Brown, \emph{Slave Life in Georgia: A Narrative
  of the Life, Sufferings, and Escape of John Brown, A Fugitive Slave},
  ed. L. A. Chamerozow (London: o autor, 1855, p.~2).} Mas a prática dos
pais brancos de não reconhecerem como filhos os frutos de seus abusos sexuais criava famílias secretas, compostas de parentes afrodescendentes e familiares brancos.

Assim como o resto dos americanos, os afrodescendentes preferiam
famílias estáveis, com um pai e uma mãe, mas raramente atingiam esse
ideal. A escravidão roubou da mãe de Brown algumas partes essenciais da
maternidade. Desprotegida do estupro, Elizabeth também não podia cuidar
dos seus bebês sem a interferência dos escravizadores, e cada um dos
seus sete filhos teve um pai diferente. Perturbações familiares
envolvendo violência sexual e deslocamento geográfico eram a norma, não
a exceção. Como todos os filhos nascidos de mulheres escravizadas eram
cativos por lei, Brown lembra, o dono de sua mãe, John Young
(1779--1833), ``me roubou assim que nasci''.\footnote{Brown,
  \emph{Narrative of William W. Brown}, p.~13. {[}Página \pageref{ref5} desta
  edição.{]}}

A população escrava crescia rapidamente nos Estados Unidos no início
do século \versal{XIX}, e os escravizadores do Sul Profundo, aqueles que moravam
nas regiões algodoeiras e açucareiras que abrangiam os estados da
Luisiana, Mississippi, Alabama, Geórgia e Carolina do Sul, demandavam
trabalhadores cativos para mourejar na produção das culturas de
rendimento. Quando Brown viveu na condição de escravo, a população cativa dos
Estados Unidos representava cerca de 15\% da população nacional total.
No Missouri em 1830, os escravizados eram 18\% da população do estado.
Mas em locais onde ocorria a produção intensiva de algodão, as
populações escravas eram muito mais densas, atingindo 51\% na
Luisiana e 54\% na Carolina do Sul, por exemplo. No Alabama, Mississippi
e outros estados do Extremo Sul, o trabalho econômico mais importante
dos escravizados era produzir algodão. Os fardos de algodão que
produziam corriam para as fábricas do Noroeste da Inglaterra e dos
estados da Nova Inglaterra, onde eram processados em fios e tecidos que
vestiam pessoas ao redor do mundo.

Durante as primeiras décadas da vida de Brown, o cultivo de algodão se
tornou o principal interesse econômico dos Estados Unidos. O algodão
exportado anualmente aumentou de alguns milhares de fardos em 1790 para
quatro milhões em 1860. Sua participação no total das exportações
americanas quase quintuplicou entre 1800 e 1820. Em 1840, o algodão
respondia por metade do valor das exportações do país, número que chegou
a 57\% em 1860. E a população escrava também cresceu rapidamente. Os
quase 1,2 milhão de cativos contados em 1810 se tornaram os ancestrais
de 2 milhões em 1830, 3 milhões em 1850 e quase 4 milhões em 1860.
Em 1830, a soma total da
propriedade de escravos valia 577 milhões de dólares, ou 15\% do
patrimônio nacional. Em 1860, esse valor ultrapassou os 3 bilhões de
dólares, ou quase 19\% do patrimônio total dos \versal{EUA}.
Os escravizados valiam mais do que
todos os bancos, meios de transporte e manufaturas juntos. A propriedade
de escravos era o segundo tipo mais valioso, perdendo apenas para a
terra. Eles trabalhavam no que um historiador chamava de ``campos de
trabalho escravo'', e a produtividade dos trabalhadores nos algodoais
aumentou 400\% entre 1800 e 1860.\footnote{Edward E. Baptist, \emph{The
  Half Has Never Been Told: Slavery and the Making of American
  Capitalism} (New York: Basic, 2014), xiv (citação); p.~246.} A escravidão
moldou a paisagem e alterou o contorno de uma parcela significativa do
país.

E sempre que um fazendeiro de algodão do Alabama ou Mississippi queria
expandir seus negócios ou substituir trabalhadores perdidos, ele os
comprava de outros senhores. A partir de 1808, os Estados Unidos baniram
a importação de cativos estrangeiros. As economias coloniais de
Maryland, Virgínia, Carolina do Sul e Geórgia dependiam da importação de
cativos africanos para o cultivo de arroz e tabaco, mas quando Brown
nasceu (1814), os descendentes desses africanos já haviam se tornado
afro"-americanos. Em estados como a Virgínia e o Kentucky, essas famílias
de afrodescendentes muitas vezes se estendiam por várias gerações.
Remoções forçadas deserdavam as crianças escravizadas do acesso à
sabedoria dos seus ancestrais, pois desarticulavam suas próprias
famílias. As demandas da produção algodoeira no Sul Profundo afetavam os
estados mais ao Norte, incluindo o Missouri e o Kentucky.

Além de atuar em diversas funções de trabalho doméstico e de serviços,
incluindo assistente de tipógrafo, auxiliar de alfaiate, lavrador,
barbeiro, cavalariço e cocheiro, Brown também foi assistente de um
traficante negreiro, responsável por preparar pessoas escravizadas
para a venda em Natchez, Mississippi, e Nova Orleans, Luisiana. Essa
experiência, que Brown vivenciou quando tinha meros 16 anos, se destaca
como um dos episódios mais emocionantes e aterroradores da sua
\emph{Narrativa}. Walker, o traficante negreiro, alugou Brown de John
Young em 1831 e o obrigou a ajudá"-lo no transporte de três comboios de
pessoas escravas em barcos a vapor no Rio Mississippi, vendendo
alguns em Vicksburg e Natchez, Mississippi, e os outros em Nova Orleans.
Um ``comboio'' era uma caravana humana. Na \emph{Narrativa}, ser
submetido a Walker é parte de um processo mais amplo de migração
forçada. Assim como outros americanos afrodescendentes, Brown foi
forçado a sair da terra onde nascera, o Kentucky, e levado ao Missouri.
De lá ele foi então forçado a descer o rio até a terra de morte e
miséria que era o mercado de escravos em Nova Orleans, que alimentava as
regiões de produção intensiva de algodão e açúcar no Sul Profundo dos
Estados Unidos. Mais tarde, Brown explicaria que este era uma extensão
do tráfico negreiro transatlântico, no qual os negros foram
``colocados em uma viagem experimental'' que durou duzentos
anos.\footnote{\emph{National Anti"-Slavery Standard}, 28 de outubro de
  1854, p.~2--3, citado em Greenspan, p.~3.} Foi uma sequência da Travessia
Atlântica (``Middle Passage''), nome dado nos Estados Unidos ao
deslocamento forçado dos africanos para as Américas.

A participação de Brown no tráfico negreiro interno foi relutante, e
ele registrou o seu horror em ter que vestir e guardar dezenas de
afro"-americanos enquanto eram transportados à força por centenas de
quilômetros em direção ao Sul, para ser vendidos e separados das suas
famílias. Além de vendas privadas, Brown testemunhou o espetáculo dos
leilões público sob o teto cupulado de um café chamado Hewlett's
Exchange, localizado na Rua Chartres, no que hoje é o Bairro Francês de
Nova Orleans. Ele viu uma mulher saltar do convés inferior do vapor
\emph{Enterprise} e se afogar nas águas turvas do Mississippi para não
ser vendida. Brown viu Walker roubar um bebê dos braços da mãe e outra
forçada à escravidão sexual, transformada em amante de Walker, uma
prática comum entre os traficantes. Ele viu um cativo ser
assassinado em Nova Orleans, seu corpo atirado na carroça de lixo, e um
afrodescendente livre ser escravizado e vendido. E quando Brown implorou
para ser retirado das mãos de Walker, seu dono se recusou, insistindo
que o contrato precisava ser honrado. Uma década e meia depois,
refletindo sobre a experiência e as cenas de subjugação que assistira,
Brown escreveu: ``Não tenho palavras para expressar meus
sentimentos''.\footnote{Brown, \emph{Narrative of William W. Brown}, p.~40.
  {[}Página \pageref{ref6} desta edição.{]}}

Brown viu de perto a destruição de inúmeras famílias afro"-americanas, e
ainda foi forçado a ser cúmplice das práticas de negócios fraudulentas
do seu senhor quando Walker lhe mandou ``preparar os escravos
idosos para o mercado''. Brown explica que ele precisava ``raspar as
barbas e bigodes dos velhos e arrancar os cabelos grisalhos quando estes
não eram por demais numerosos; caso contrário, ele possuía uma mistura
de graxa preta e um pincel para aplicá"-la''.\footnote{Brown,
  \emph{Narrative of William W. Brown}, p.~43. {[}Página \pageref{ref7} desta
  edição.{]}} Esse trabalho cosmético era parte do processo de apagar
identidades individuais. O tráfico negreiro roubava as histórias dos
familiares e colocava uma propaganda inventada no seu lugar. Brown tinha
ordens de colocar ``alguns (\ldots{}) a dançar, alguns a pular, alguns a
cantar e alguns a jogar cartas'' para que fingissem que estavam
contentes e obedeceriam aos seus novos proprietários.\footnote{Brown,
  \emph{Narrative of William W. Brown}, p.~46. {[}Página \pageref{ref8} desta
  edição.{]}}

Ajudar Walker a transportar três turmas de escravos do
Missouri até o Mississippi e a Luisiana foi ``o ano mais longo de toda a
minha vida''. Apesar dos horrores que testemunhou, ser alugado deu a
Brown a oportunidade de conhecer boa parte do Vale do Rio Mississippi
nos mil quilômetros que separam St.\,Louis de Nova Orleans. Ele ganhou
gorjetas trabalhando de camareiro, dinheiro que pôde usar na sua
tentativa de fuga, além de conhecer e interagir com diversas outras
pessoas, tanto livres quanto cativas. A descrição de Brown do
tráfico negreiro visto de dentro é uma das mais esclarecedoras
oferecida por uma testemunha que fora escravizada.

Entre as atrocidades que reconta para o leitor está o caso da mãe de um
bebê doente que não consegue impedir que os choros da criança
incomodassem Walker. O traficante, lembra Brown, ``pegou a criança por
um braço, como se pegaria um gato pela perna'' e o deu para uma mulher
branca, ignorando as súplicas da mãe, que em seguida ``foi acorrentada à
turma''.\footnote{Brown, \emph{Narrative of William W. Brown}, p.~49--51.
  {[}Página \pageref{ref9} desta edição.{]}} As separações da escravidão dissolviam
cerca de um terço dos primeiros casamentos dos escravos nos estados
sulistas mais ao Norte. Cerca de metade das crianças escravas
perdia seus pais e mães, geralmente os primeiros. Traficantes negreiros como Walker não queriam os muito jovens ou os muito velhos.
Bebês e avôs não geravam boas vendas. Em vez disso, eles exigiam
trabalhadores no auge da fertilidade. O roubo entre gerações era
implacável, separando uma geração da outra durante várias décadas e
espalhando os membros das famílias afrodescendentes por todo o Sul dos
Estados Unidos.

Ter um filho arrancado dos braços da mãe era o pesadelo de todas as
pessoas escravizadas que enfrentavam o destino terrível de serem
removidas da terra do algodão. Brown era atormentado pela lembrança de
como a irmã e a mãe haviam sido mandadas para o Sul Profundo. ``{[}E{]}u
imaginava ver minha mãe querida no algodoal, seguida por um capataz
impiedoso, sem ninguém para lhe oferecer uma palavra de consolo! Eu
enxergava minha irmã nas mãos de um feitor, forçada a se submeter à sua
crueldade!''\footnote{Brown, \emph{Narrative of William W. Brown}, p.~94.
  {[}Página \pageref{ref10} desta edição.{]}} Tais separações representavam uma
ameaça existencial e, na \emph{Narrativa}, a perda da mãe e da irmã para
o tráfico negreiro deixou Brown livre para fugir. Os escravizadores
haviam rompido seus laços com seus entes queridos, e sua resposta foi
aproveitar essa liberdade enquanto indivíduo.


\section{Sobre a obra}

Publicada no bastião abolicionista de Boston
pela Sociedade Antiescravista de Massachusetts em julho de 1847, a
\emph{Narrativa de William Wells Brown, escravo fugitivo. Escrita
por ele mesmo} é um apelo exaltado à abolição da escravidão nos Estados
Unidos e, por consequência, em todo o mundo. Redigido pelo ex"-escravizado
William Brown,\footnote{No aparato crítico dessa edição empregou"-se a palavra ``escravizado'' no lugar de ``escravo''. Essa mudança lexical desnaturaliza o processo de escravização e a existência social do escravismo, pois a alteração do sufixo transforma o substantivo ``escravo'', que conota \emph{status} ou condição permanente, no verbo ``escravizar'', evidenciando o dinamismo da construção social da pessoa em situação de escravidão. Em inglês, o particípio e a função adjetiva do particípio se distinguem pela posição do termo em relação ao nome (``\emph{enslaved person}'', ``\emph{person enslaved}''). Em português, o valor posicional do termo não produz distinção semântica com a mesma clareza que o inglês. Nos casos em que o emprego da forma nominal ``escravizado'' gerasse ambiguidade, esta edição optou excepcionalmente pelo uso vernacular dos vocábulos. [\versal{N.~O.}]} o livro comoveu os leitores e incitou o debate na época
da sua primeira edição, vendendo 8000 exemplares em dois anos e sendo
reeditado nove vezes em quatro décadas. A \emph{Narrativa} é, e continua
a ser, um texto poderoso, pois representa um testemunho em primeira
pessoa da escravidão americana e coloca os leitores cara a cara com o
ambiente de violência social que impactava radicalmente a personalidade,
a família e o desenvolvimento moral entre os escravizados. O narrador
não era um cativo qualquer. Brown precisou transcender as dificuldades
impostas pela escravidão para conseguir se expressar enquanto orador e
escritor americano. O subtítulo, ``escrita por ele mesmo'', insiste que
ele fala autenticamente, nas próprias palavras, sob a própria orientação
moral e intelectual.

A \emph{Narrativa} de Brown
lançou a carreira literária de um dos maiores autores afro"-americanos da
história. As primeiras resenhas chamaram o livro de ``forte, emocionante
e arrebatador (\ldots{}) e contém muitas passagens sensacionais''.\footnote{\emph{The} {[}\emph{Boston}{]} \emph{Liberator}, 23 de julho de 1847, p.~118.} Estimulado
pelo sucesso do seu primeiro livro, Brown, que já atuava como
palestrante abolicionista, publicaria obras de vulto em diversos
gêneros. Além de ser provavelmente o primeiro romancista afro"-americano
com \emph{Clotel; or, The President's Daughter: A Narrative of Slave
Life in the United States} (1853), ele também escreveu obras de
não"-ficção importantes, como \emph{The Negro in the American Rebellion:
His Heroism and his Fidelity} (1867), sobre a participação dos
afrodescendentes na Revolução Americana (1775--1783), e \emph{My Southern
Home; or, The South and its People} (1880), que amplia seus textos
autobiográficos ao mesmo tempo que antecipa a antropologia cultural.
Quando da sua morte em 1884, Brown havia publicado mais do que qualquer
outro escritor afro"-americano contemporâneo. Apesar de não ter recebido
tanta atenção quanto Frederick Douglass,\footnote{William L. Andrews,
  Introduction, \emph{From Fugitive Slave to Free Man: The
  Autobiographies of William Wells Brown} (Columbia and London:
  University of Missouri Press, 2003).} de quem foi contemporâneo, seus
talentos para comunicar a experiência negra nos \versal{EUA} prefiguraram as
gerações seguintes dos intelectuais e artistas afro"-americanos, como W.
E. B. Du Bois, James Baldwin e Langston Hughes.

%\section{O Gênero das Narrativas de Escravos}
\section{Sobre o gênero}

Geralmente chamadas de \emph{narrativas de escravos} (\emph{slave narratives}), as autobiografias
de ex"-escravizados como a \emph{Narrativa} eram em sua essência histórias de
liberdade que detalhavam as respostas dos autores à escravidão (tivessem
eles nascido nela ou sido forçados a essa condição) e seus caminhos para
a liberdade. As primeiras autobiografias de ex"-escravizados eram um híbrido
de gêneros diversos, incluindo narrativas de cativeiro, literatura de
protesto, confissão religiosa e relatos de viagem. A maioria foi escrita
e publicada com auxílio editorial de brancos, e quase todas foram
publicadas para públicos brancos. ``Assim, desde os primeiros momentos
da autobiografia negra na América'', argumenta um crítico literário,
``domina a pressuposição de que o narrador negro precisa de um leitor
branco para completar o seu texto, para construir uma hierarquia de
significância abstrata referente ao simples conjunto dos seus fatos,
para oferecer uma presença onde antes havia apenas um `Negro', uma
ausência escura''.\footnote{William L. Andrews, \emph{To Tell A Free
  Story: The First Century of Afro"-American Autobiography, 1760--1865}
  (Urbana and Chicago: University of Illinois Press, 1988, p.~32--33).}
Coletivamente, elas exigiam que os leitores brancos testemunhassem as
injustiças contra os afrodescendentes cometidas por outros brancos e
reagissem de acordo.

O testemunho era a pedra fundamental da autobiografia de um ex"-escravizado,
e um dos principais desafios artísticos enfrentados pelos autores negros
foi como atrair a simpatia do leitor para que enxergassem as cenas de
subjugação da maneira apresentada. Muitas vezes, isso significava
aceitar boa parte da cultura anglo"-americana como normativa, sugerindo
que a civilidade dos brancos seria a razão por que era inaceitável a
crueldade dos escravizadores. Os americanos civilizados não deveriam
tolerar o barbarismo da escravidão. Para os escritores negros, isso
significou atenuar a importância das tradições, culturas, religiões e
idiomas da África e dar preferência aos valores e tradições dos povos
descendentes de europeus. Significou atenuar o radicalismo e a
militância que caracterizou líderes descendentes de escravizados como
Toussaint ou Jean"-Jacques Dessalines no Haiti. Ex"-cativos autobiógrafos
como Brown eram reformadores radicais, não revolucionários violentos,
semelhantes a pessoas como o brasileiro Luiz Gama. O gênero representado
pela \emph{Narrativa} defendia concepções anglo"-americanas de liberdade
e do Cristianismo ocidental, argumentando que a escravidão desviava
delas e as corrompia. Eram radicais dentro dessa tradição, não
contrários a ela. Essa atitude caracteriza algumas das primeiras
autobiografias de ex"-escravizados anglo"-americanos mais famosas, como
\emph{The Interesting Narrative of the Life of Olaudah Equiano, or
Gustavus Vassa, The African} (1789), publicada na
Inglaterra. Mas à medida que o gênero avançou, as críticas dos autores à
cultura dominante ganharam destaque, com a maior sofisticação do páthos
e da natureza interior do sujeito escravizado.

Quando o gênero da autobiografia de ex"-escravos norte"-americano foi
ganhando forma na década de 1820, os autores começaram a escrever contra
o gênero emergente da ficção regionalista do Sul, que representava a
vida nas fazendas como uma experiência igualmente idílica para os
brancos e para os negros. Os
americanos embelezavam o seu caráter coletivo através da ficção sobre as
peculiaridades de suas regiões natais. Quando participavam desse
empreendimento artístico, intelectual e político, os autores brancos
sulistas tropeçavam na escravidão, pois era óbvio para os visitantes que
as relações sociais da escravidão estavam carregas de violência e
crueldade. Assim, os regionalistas brancos, entre os quais os mais
famosos foram William Gilmore Simms e John Pendleton Kennedy,
construíram um roteiro que invertia esses pressupostos. Eles afirmavam
que, se a escravidão diferenciava o Sul, ela também era um bem positivo,
não um mal necessário. Em sua visão, a escravidão beneficiava os
afrodescendentes e dava aos brancos uma obrigação paternalista que
reforçava a supremacia branca e a servidão negra.

Com os romances de autores brancos articulando uma defesa da escravidão
a jorrar das editoras, os abolicionistas negros reagiram. Essa disputa
despontou na década de 1840, no que um crítico literário chamou de ``a
escola do `fugitivo heroico' da literatura americana''.\footnote{Benjamin
  Quarles, ``Narrative of the Life of Frederick Douglass,'' \emph{em
  Landmarks of American Writing}, ed. Hennig Cohen (New York: Basic
  Books, 1969, p.~91).} A \emph{Narrativa da vida de Frederick Douglass, um
escravo americano, escrita por ele mesmo}, de 1845, é o exemplo mais
conhecido e mais respeitado dessa fase do gênero. Brown juntou"-se a
Douglass e outros ex"-escravizados, como Henry Bibb e James W. C. Pennington,
na publicação de autobiografias nas quais o autor se
torna o herói da própria narrativa. A maioria veio dos estados mais ao
Norte (incluindo, mas não apenas, os estados do Kentucky, Maryland,
Missouri, Carolina do Norte e Virgínia) e havia se alfabetizado pelo
menos parcialmente durante o seu cativeiro.

Na competição literária entre os brancos pró"-escravidão e os autores
negros ex"-escravizados, os brancos começaram a defender que eram eles, e
não os afrodescendentes, que apresentavam um retrato mais fiel da sua
sociedade. Nessa disputa, a autenticidade dos autores negros passou a
ser questionada. Porque tantas autobiografias foram
escritas com a ajuda ou o auxílio editorial de brancos, o ônus de
demonstrar a sua autenticidade caiu sobre os autores afrodescendentes.
Um jornal do Mississippi, por exemplo, afirmou em 1847 que Henry Bibb
seria ``algum habitante seboso de Five Points {[}bairro
nova"-iorquino{]}, mas se apresenta como um criado autoemancipado do
Kentucky''.\footnote{\emph{Mississippi Free Trader}, 16 de junho de 1847,
  p.~2.} Às vezes, provava"-se que alguns autores negros inventaram as suas
histórias, como aconteceu com James Williams. Seu livro, \emph{The
Authentic Narrative of James Williams, An American Slave} (1838), na
verdade foi escrito pelo poeta abolicionista John Greenleaf Whittier. A
descoberta do fato causou escândalo. Os autores negros precisavam
demonstrar que eram os autores das suas narrativas, não apenas os
sujeitos delas. Muitas vezes, isso significava que as autobiografias
vinham prefaciadas por defensores brancos. Segundo um estudioso, isso
era como colocar uma mensagem negra dentro de um envelope
branco.\footnote{John Sekora, ``Black Message/White Envelope: Genre,
  Authenticity, and Authority in the Antebellum Slave Narrative'',
  \emph{Callaloo} 32, 1987, p.~482--515.} Mas isso mostra os desafios que
os autores afrodescendentes enfrentavam em uma nação política e um
mercado literário controlados pelos brancos.

O gênero do fugitivo heroico das narrativas de escravos conta uma
história estilizada e elegante de como é crescer sob o jugo da
escravidão e testemunhá"-la ao mesmo tempo que narra o próprio
desenvolvimento moral do autor que o leva a livrar"-se dela. Para
preservar a autenticidade e eficácia para os públicos brancos do Norte
que liam esses livros, os ex"-escravizados autobiógrafos enfatizam a
masculinidade negra circunscrita pela moralidade. O rompimento de
Douglass com a escravidão ocorre em fases, através da alfabetização e
então uma luta com um escravista mesquinho para o qual ele fora alugado,
um combate que vence, e então como se salva miraculosamente dos castigos
severos que geralmente eram dados aos homens negros que revidavam contra
agressores brancos. Em vez de pegar em armas, entretanto, Douglass
organiza uma ``escola sabatina'' para pregar o evangelho cristão para os
seus irmãos escravizados. Ele usa a escola para organizar uma fuga, o
que enfatiza a humanidade daqueles que se reuniam com ele. ``Eu os amava
com um amor mais forte do que tudo que vivenciei desde então'', Douglass
lembra sobre as pessoas que tentou (sem sucesso) levar à liberdade com
uma fuga para o Norte.\footnote{Frederick Douglass, \emph{Narrative of
  the Life of Frederick Douglass, An American Slave, Written by Himself}
  (Boston: The Anti"-slavery Office, 1845, p.~81, 83).}

%\section{A «Narrativa» de William Brown}

\chapterspecial{Para saber mais}{}{Calvin Schermerhorn}

A \emph{Narrativa} de William Brown segue as regras básicas do gênero
das narrativas de escravos. Ela começa com uma nota de agradecimento ao
Quaker branco que o salvou do frio e da fome durante a sua fuga, o Wells %nota explicando Quaker?
Brown de quem ganhou o nome. Dois prefácios adicionais atestam a
veracidade e o caráter do autor. Um foi escrito por Joseph C. Hathaway,
um Quaker de Farmington, Nova York, participante ativo da Ferrovia
Subterrânea, a rede de refúgios e meios de transporte que tiravam
escravizados do Sul, e apoiador da Sociedade Antiescravista do Oeste de Nova
York. Brown dera palestras em Farminton e morara na cidade em meados da
década de 1840. Suas filhas estudaram lá e Hathaway estava posicionado
para dar credibilidade a Brown junto a um público que não o conhecia. O
segundo endosso, mais prestigioso, foi escrito pelo abolicionista Edmund
Quincy, da Sociedade Antiescravista de Massachusetts. Quincy declara que
o texto da \emph{Narrativa} foi mesmo escrito por Brown e apresentado a
ele para ser revisado, admitindo até mesmo que editou o texto original
para ``corrigir alguns equívocos (\ldots{}) e sugerir alguns
abreviamentos''.\footnote{Ezra Greenspan, \emph{William Wells Brown: An
  African American Life} (New York: W. W. Norton, 2014, capítulos 4--5);
  William Wells Brown, \emph{Narrative of William W. Brown, A Fugitive
  Slave, Written by Himself} (Boston: The Anti"-slavery office, 1847), vi
  (citação). {[}Página \pageref{ref1} desta edição.{]}} Juntos, os prefácios
preparam o público para acreditar nos relatos do autor. Em uma ironia
que fortaleceu ainda mais a autenticidade de Brown, seu último
proprietário, Enoch Price, de St.\,Louis, enviou uma carta oferecendo"-se
para vender Brown no início de 1848, na qual ele supostamente
``reconheceu a veracidade substancial {[}da
\emph{Narrativa}{]}''.\footnote{\emph{Salem} {[}\emph{Massachusetts}{]}
  \emph{Observer}, 12 de fevereiro de 1848, p.~2.}

Como outras obras nesse gênero, a \emph{Narrativa} de Brown é uma
representação literária. Após fugir da escravidão em 1834 e trabalhar
nos vapores dos Grandes Lagos da América do Norte, ele começou a dar
palestras para públicos antiescravidão em 1843 e acabaria por contar sua
história nos palcos. A oratória era um gênero que exigia que o
palestrante cativasse o público, e a experiência de Brown em recontar
seus primeiros anos nos palcos o ajudou a moldar a narrativa quando esta
foi se transformando em um projeto literário. Os abolicionistas também
lhe forneceram um vocabulário que formava uma retórica de denúncia dos
escravizadores de acordo com os temas da hipocrisia religiosa, crueldade
lasciva e separações desalmadas das famílias afrodescendentes. Esse
vocabulário foi uma maneira de enquadrar suas reflexões pessoais e
ligá"-las aos apelos dos abolicionistas por reforma moral. A
\emph{Narrativa} funde a retórica abolicionista com as memórias
reconstruídas dos seus anos no Missouri e no Rio Mississippi, culminando
com a sua fuga em 1834, após o Ano Novo.

Apesar de pertencer ao gênero do fugitivo heroico das autobiografias de
ex"-escravizados, Brown foi ao mesmo tempo herói e anti"-herói. Quando era
criado do traficante negreiro Walker em 1832, Brown foi mandado para
ser açoitado em Vicksburg, Mississippi. Walker lhe deu um dólar e um
bilhete com a instrução de castigá"-lo e mandou que os entregasse ao
carcereiro; temendo o conteúdo do bilhete (já que não sabia lê"-lo),
Brown pagou outro homem negro para levar o papel até a cadeia, onde seu
substituto recebeu ``vinte chibatadas nas costas
nuas''.\footnote{Brown, \emph{Narrative of William W. Brown}, p.~56.
  {[}Página \pageref{ref2} desta edição.{]}} Brown se expõe ao ridículo e à
humilhação, ou até à perfídia, por enganar outro negro para que sofresse
no seu lugar. Ele também praticou um embuste contra Almira Price, a
esposa do seu último proprietário. Enquanto morava em St.\,Louis e
trabalhava de cocheiro, Brown foi sujeitado à ``armadilha que a Sra.\,Price criara para me deixar satisfeito com meu novo lar ao me obter uma
esposa''. Ele fingiu interesse em casar"-se com uma jovem chamada Eliza,
também escravizada, e que Enoch Price adquirira para que formassem um
par. A compra teve dois motivos. Primeiro, seria muito menos provável
que Brown fugisse caso se casasse, e todos os filhos de Eliza também
seriam escravos e se tornariam propriedade dos Prices. Mas Brown
estava determinado a fugir e permaneceu solteiro; ``mas esse segredo eu
era forçado a guardar de todos'', ele escreveu.\footnote{Brown,
  \emph{Narrative of William W. Brown}, p.~88. {[}Página \pageref{ref3} desta
  edição.{]}} Mais uma vez, Brown sobreviveu graças à dissimulação, o
que desta vez envolveu Eliza também e não apenas a família Price. Em um
mundo no qual a moralidade deveria ser a expressão pública dos valores
mais íntimos, Brown admitia suas próprias deficiências.\footnote{Clay M.
  Hooper, ```It Is Good to Be Shifty': William Wells Brown's Trickster
  Critique of Black Autobiography'', \emph{Modern Language Studies} 38.2,
  2009, p.~28--45.}

Mas a \emph{Narrativa} de Brown não perde a oportunidade de transformar
a sua duplicidade em lição. ``Esse incidente mostra como a
escravidão transforma suas vítimas em mentirosos mesquinhos'', ele
escreve, ``vícios pelos quais ela os censura depois e usa como argumento
para provar que não merecem sina melhor do que essa''. Mas Brown se
desculpa, escrevendo: ``(\ldots{}) muito lamentei e me arrependi
profundamente do logro que perpetrei contra esse pobre rapaz; é meu
desejo sincero que, um dia, esteja ao meu alcance ressarci"-lo pela
tortura que sofreu em meu nome''.\footnote{Brown, \emph{Narrative of
  William W. Brown}, p.~57--58. {[}Página \pageref{ref4} desta edição.{]}} Ao admitir
que era um trapaceiro, ele também repudia essa imoralidade depois de
liberto. Quando escapou da escravidão, ele se transformou de um homem
imoral em um homem moral. A liberdade foi transformadora. Brown assumiu
um novo nome e insistiu que se possuísse a moralidade de um homem livre,
jamais teria se rebaixado a um embuste como aquele.\footnote{Andrews,
  \emph{To Tell a Free Story}, p.~146--51.}

Após a fuga, Brown se reinventou, tanto no palco quanto como figura
literária. Antes da fuga, ele era conhecido pelo nome de Sandford
Higgins, mas assumiu o nome de Wells Brown, um Quaker branco de Ohio que
o ajudou no seu momento de necessidade, e retomou o nome de William, que
era como Elizabeth, sua mãe, o chamava. A escolha do nome era
importante, pois representava um ato de autopossessão. Essa reinvenção
foi acompanhada por uma segunda transformação externa. A imagem no
frontispício da sua \emph{Narrativa} é a de um cavalheiro americano.
William Wells Brown usava um penteado afro"-americano, mas sua aparência
e até sua assinatura destaca uma afiliação de classe com os cidadãos
brancos americanos de classe média. No contexto dos Estados Unidos, o
filósofo americano W. E. B. Du Bois batizaria isso de ``dupla
consciência'', uma divisão da identidade negra norte"-americana em um
sujeito negro voltado para o exterior que, ao mesmo tempo, se enxerga da
mesma forma que os brancos, dando a mais estrita atenção à importância
de como os negros são vistos pelos brancos.\footnote{W. E. B. Du Bois,
  \emph{The Souls of Black Folk: Essays and Sketches,} Third Edition
  (Chicago: A. C. McClurg and Co., 1903, p.~2).} Usando o conceito
sociopsicológico de um véu, Du Bois argumentou que os afro"-americanos
viam o mundo de dentro de um véu, enquanto afrodescendentes, e também de
fora dele, espiando como faziam os brancos.

A \emph{Narrativa} de Brown tem elementos de dupla consciência quando
explora o emaranhado de identidades e perspectivas envolvidas em ser um
ex"-escravizado e uma pessoa de ascendência africana que tenta salvar o
projeto da democracia americana, expressando"-se para um público que
suspeita da sua aparência, tradição e legitimidade. O filósofo Cornel
West chamou isso de ``a crise tripla da autoconsciência'', que era um
anseio por ser parte da alta cultura cosmopolita dominante ao mesmo
tempo que se tinha um histórico provinciano, uma herança de migração
forçada e a identidade social de um homem negro sem status.\footnote{Cornel
  West, \emph{Prophecy Deliverance!} \emph{An Afro"-American
  Revolutionary Christianity} (Louisville, Ky: Westminster John Knox
  Press, 2002, p.~31).} Durante toda a sua carreira, Brown, assim como
muitos outros abolicionistas negros, teve dificuldade para ser
reconhecido como um cavalheiro, um homem letrado, culto e refinado. Na
\emph{Narrativa}, ele enfrenta esse drama ao apresentar"-se como um
afrodescendente escravizado, mas também como um americano que acreditava
em liberdade pessoal, na democracia e no Cristianismo protestante. Ele
argumenta que os Estados Unidos eram uma república de escravizadores,
alicerçada na supremacia branca. Em vez de conclamar pela sua
destruição, no entanto, ele insistia na sua reforma. A escravidão
precisava ter fim, e a subordinação racial, também.

A \emph{Narrativa} de Brown foi elogiada pelo público leitor. Um
correspondente do jornal \emph{The Boston Whig}, escreveu: ``As ideias e
sentimentos que emergem naturalmente da leitura desta pequena narrativa
fazem com que todas as questões das rivalidades partidárias e sectárias
pareçam absolutamente insignificantes''. A sinceridade do texto era
evidente, fossem quais fossem as preferências partidárias do leitor.
``Quem dera que toda a nossa literatura pudesse tornar o vício odioso em
todas as suas formas, como esta tornará a escravidão''.\footnote{\emph{The
  Liberator}, 26 de novembro de 1847, p.~189 {[}reprinted from the Boston
  \emph{Whig}{]}.} Um item publicado no \emph{Liberator}, de Boston, o
jornal abolicionista mais famoso do país, elogiou a \emph{Narrativa} e
profetizou que ela causaria ``uma impressão profunda e duradoura nas
mentes da geração que desponta''.\footnote{\emph{The Liberator}, 23 de
  julho de 1847, p.~118.}



%\section{Considerações Finais}
%
%A conquista suprema da \emph{Narrativa} de Brown é o seu retrato nítido
%e expressivo da vida na escravidão americana às margens do Rio
%Mississippi, contada com o páthos e os sentimentos de uma testemunha
%ocular eloquente. Sua composição austera e o ritmo acelerado dos eventos
%permite que o leitor imagine muitos dos detalhes e simpatize com um
%sujeito que sofre uma série cada vez maior de privações. As viagens de
%Brown em busca da liberdade são parte do seu processo de autoconstrução,
%e o leitor sente a tensão se acumulando à medida que ele cresce e
%aprende. E essa conexão se aprofunda com os seus lapsos, seus truques e
%mentiras, que ajudam a formar um ser humano completo ao mesmo tempo que
%o enche de remorso e angústia. Brown, além de oferecer um retrato fiel
%da vida nas décadas de 1820 e 1830, com detalhes históricos cruciais
%confirmados pelo trabalho independente dos estudiosos, pinta um quadro
%vivo do Vale do Rio Mississippi, retratando"-o como o coração sombrio da
%América.

\pagebreak
\thispagestyle{empty}