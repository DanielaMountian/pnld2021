\part{Poesia completa}

\part{Transposição {[}1966--1967{]}}

\chapter*{}
\thispagestyle{empty}
\mbox{}
\vfill
\hfill\emph{A um passo de meu próprio espírito}

\hfill\emph{A um passo impossível de Deus.}

\hfill\emph{Atenta ao real: aqui.}

\hfill\emph{Aqui aconteço.}

\part*{\textsc{i}\\\textsc{base}}

\chapter{Transposição}

\begin{verse}
Na manhã que desperta\\
o jardim não mais geometria\\
é gradação de luz e aguda\\
descontinuidade de planos.

Tudo se recria e o instante\\
varia de ângulo e face\\
segundo a mesma vidaluz\\
que instaura jardins na amplitude

que desperta as flores em várias\\
coresinstantes e as revive\\
jogando-as lucidamente\\
em transposição contínua.
\end{verse}

\chapter{Tempo}

\begin{verse}
O fluxo obriga\\
qualquer flor\\
a abrigar-se em si mesma\\
sem memória.

O fluxo onda ser\\
impede qualquer flor\\
de reinventar-se em\\
flor repetida.

O fluxo destrona\\
qualquer flor\\
de seu agora vivo\\
e a torna em sono.

O universofluxo\\
repele\\
entre as flores estes\\
cantosfloresvidas.

--- Mas eis que a palavra\\
cantoflorvivência\\
re-nascendo perpétua\\
obriga o fluxo

cavalga o fluxo num milagre\\
de vida.
\end{verse}

\chapter{Arabesco}

\begin{verse}
A geometria em mosaico\\
cria o texto labirinto\\
intrincadíssimos caminhos\\
complexidades nítidas.

A geometria em florido\\
plano de minúcias vivas\\
a geometria toda em fuga\\
e o texto como em primavera.

A ordem transpondo-se em beleza\\
além dos planos no infinito\\
e o texto pleno indecifrado\\
em mosaico flor ardendo.

O caos domado em plenitude\\
\hfill{}a primavera.
\end{verse}

\chapter{Pedra}

\begin{verse}
A pedra é transparente:\\
o silêncio se vê\\
em sua densidade.

(Clara textura e verbo\\
definitivo e íntegro\\
a pedra silencia).

O verbo é transparente:\\
o silêncio o contém\\
em pura eternidade.
\end{verse}

\chapter{Poema \textsc{i}}

\begin{verse}
O sol novifluente\\
transfigura a vivência:\\
outra figura nasce\\
e subsiste, plena.

É um renascer contínuo\\
que nela se inaugura:\\
vida nunca acabada\\
tentando o absoluto.

Espírito nascido\\
das águas intranquilas\\
verbo fixado: sol\\
novifluente.
\end{verse}

\chapter{Meada}

\begin{verse}
Uma trança desfaz-se:\\
calmamente as mãos\\
soltam os fios\\
inutilizam\\
o amorosamente tramado.

Uma trança desfaz-se:\\
as mãos buscam o fundo\\
da rede inesgotável\\
anulando a trama\\
e a forma.

Uma trança desfaz-se:\\
as mãos buscam o fim\\
do tempo e o início\\
de si mesmas, antes\\
da trama criada.

As mãos\\
destroem, procurando-se\\
antes da trança e da memória.
\end{verse}

\chapter{Ludismo}

\begin{verse}
Quebrar o brinquedo\\
é mais divertido.

As peças são outros jogos\\
construiremos outro segredo.\\
Os cacos são outros reais\\
antes ocultos pela forma\\
e o jogo estraçalhado\\
se multiplica ao infinito\\
e é mais real que a integridade: mais lúcido.

Mundos frágeis adquiridos\\
no despedaçamento de um só.\\
E o saber do real múltiplo\\
e o sabor dos reais possíveis\\
e o livre jogo instituído\\
contra a limitação das coisas\\
contra a forma anterior do espelho.

E a vertigem das novas formas\\
multiplicando a consciência\\
e a consciência que se cria\\
em jogos múltiplos e lúcidos\\
até gerar-se totalmente:\\
no exercício do jogo\\
esgotando os níveis do ser.

Quebrar o brinquedo ainda\\
é mais brincar.
\end{verse}

\chapter{Mãos}

\begin{verse}
Com as mãos nuas\\
lavrar o campo:

as mãos se ferindo\\
nos seres, arestas\\
da subjacente unidade

as mãos desenterrando\\
luzesfragmentos\\
do anterior espelho

Com as mãos nuas\\
lavrar o campo:

desnudar a estrela essencial\\
sem ter piedade do sangue.
\end{verse}

\chapter{Salto}

\section{I}

\begin{verse}
Momento\\
despreendido da forma

salto buscando\\
o além\\
do momento.
\end{verse}

\section{II}

\begin{verse}
Desvitalizar a forma\\
des - fazer\\
des - membrar

e - além da estrutura -\\
viver o puro ato\\
inabitável.
\end{verse}

\chapter{Laboratório}

\begin{verse}
Des - armamos o fato\\
para - pacientemente -\\
re - generarmos a estrutura

ser nascido do que\\
apenas acontece.

Re - fazemos a vida.
\end{verse}

\chapter{Tato}

\begin{verse}
Mãos tateiam\\
palavras\\
tecido\\
de formas.

Tato no escuro das palavras\\
mãos capturando o fato\\
texto e textura: afinal\\
matéria.
\end{verse}

\chapter{Núcleo}

\begin{verse}
Aprender a ser terra\\
e, mais que terra, pedra\\
nuclear diamante\\
cristalizando a palavra.

A palavra definitiva.\\
A palavra áspera e não plástica.
\end{verse}

\chapter{Desafio}

\begin{verse}
Contra as flores que vivo\\
contra os limites\\
contra a aparência a atenção pura\\
constrói um campo sem mais jardim\\
que a essência.
\end{verse}

\chapter{Poema \textsc{ii}}

\begin{verse}
Ser em espelho\\
fluxo detido\\
ante si mesmo

lucidez.
\end{verse}

\chapter{Diálogo}

\begin{verse}
Variável asa lúcida\\
tramando verbos véus\\
de sentido humano nas\\
coisas

lúcida sede in\\
expressa inesgotável\\
prospecção infecunda\\
do segredo

texto ato humanidade\\
variável asa diálogo\\
entre o verbo e o real\\
inefável.
\end{verse}

\chapter{Quadros}

\section{i}

\begin{verse}
O círculo em torno\\
do ato:

lisa superfície\\
da esfera\\
do oceano concreto\\
impenetrável

--- a verbalização do sangue.
\end{verse}

\medskip
\section{ii}

\begin{verse}
Um nódulo cego\\
e a luz destacando-o\\
num espaço total\\
vivo e infinito.

Um nódulo cego\\
e a luz contornando-o\\
luz densa gerando um plano\\
cruel e nítido.

Um nódulo cego\\
e a luz que o transpassa\\
definindo seu ser\\
sem diluí-lo.
\end{verse}

\medskip
\section{iii}

\begin{verse}
Livres fragmentos:\\
cores sons figuras\\
em dispersão lúcida

vertigem

Livres fragmentos:\\
constelações em fuga\\
dissonância.

Livres fragmentos\\
e a livre unidade\\
livremente aceita

(jogo maior\\
além da infância).
\end{verse}

\chapter{Série}

\begin{verse}
Primeiro\\
o apelo\\
(paralela a palavra\\
ao universo).

Depois\\
invocadas potências\\
formas se tramam puro\\
mapa lúdico.

Enfim\\
conclusão do ato\\
o amor ser possível\\
amanhece\\
lúcido.
\end{verse}

\part*{\textsc{ii}\\ (---)}

\chapter{Fala}

\begin{verse}
Tudo\\
será difícil de dizer:\\
a palavra real\\
nunca é suave.

Tudo será duro:\\
luz impiedosa\\
excessiva vivência\\
consciência demais do ser.

Tudo será\\
capaz de ferir. Será\\
agressivamente real.\\
Tão real que nos despedaça.

Não há piedade nos signos\\
e nem no amor: o ser\\
é excessivamente lúcido\\
e a palavra é densa e nos fere.

(Toda palavra é crueldade.)
\end{verse}

\chapter{Pouso}

\begin{verse}
Ó pássaro, em minha mão\\
encontram-se\\
tua liberdade intacta\\
minha aguda consciência.

Ó pássaro, em minha mão\\
teu canto\
de vitalidade pura\\
encontra a minha humanidade.

Ó pássaro, em minha mão\\
pousado\\
será possível cantarmos\\
em uníssono

se és o raro pouso\\
do sentimento vivo\\
e eu, pranto vertido\\
na palavra?
\end{verse}

\chapter{Rosa}

\begin{verse}
Eu assassinei o nome\\
da flor\\
e a mesma flor forma complexa\\
simplifiquei-a no símbolo\\
(mas sem elidir o sangue).

Porém se unicamente\\
a palavra \textsc{flor} --- a palavra\\
em si é humanidade\\
como expressar mais o que\\
é densidade inverbal, viva?

(A ex-rosa, o crepúsculo\\
o horizonte.)

Eu assassinei a palavra\\
e tenho as mãos vivas em sangue.
\end{verse}

\chapter{Meio-dia}

\begin{verse}
Ao meio-dia a vida\\
é impossível.

A luz destrói os segredos:\\
a luz é crua contra os olhos\\
ácida para o espírito.

A luz é demais para os homens.\\
(Porém como o saberias\\
quando vieste à luz\\
de ti mesmo?)

Meio-dia! Meio-dia!\\
A vida é lúcida e impossível.
\end{verse}

\chapter{Revelação}

\begin{verse}
A porta está aberta\\
como se hoje fosse infância\\
e as coisas não guardassem pensamentos\\ formas de nós nelas inscritas.

A porta está aberta. Que sentido\\
tem o que é original e puro?\\
Para além do que é humano o ser se integra\\
e a porta fica aberta. Inutilmente.
\end{verse}

\chapter{Ode \textsc{i}}

\begin{verse}
O real? A palavra\\
coisa humana\\
humanidade\\
penetrou no universo e eis que me entrega\\ tão-somente uma rosa.
\end{verse}

\chapter{Destruição}

\begin{verse}
A coisa contra a coisa:\\
a inútil crueldade\\
da análise. O cruel\\
saber que despedaça\\
o ser sabido.

A vida contra a coisa:\\
a violentação\\
da forma, recriando-a\\
em sínteses humanas\\
sábias e inúteis.

A vida contra a vida:\\
a estéril crueldade\\
da luz que se consome\\
desintegrando a essência\\
inutilmente.
\end{verse}

\chapter{Torres}

\begin{verse}
Construir torres abstratas\\
porém a luta é real. Sobre a luta\\
nossa visão se constrói. O real\\
nos doerá para sempre.
\end{verse}

\chapter{Coros}

\begin{verse}
Coros pungentes\\
cores\\
do crepúsculo

ser perdido em\\
vozesfragmentos

arestas

violação\\
de um só silêncio\\
lúcido.
\end{verse}

\chapter{Círculos}

\begin{verse}
Há uma lua\\
luz\\
além\\
do círculo dia

há uma lua\\
outro círculo.
\end{verse}

\chapter{Claustro}

\begin{verse}
Célula\\
onde nenhuma palma\\
contempla\\
na fria aridez o rosto\\
pacificado

solidão sem imagens\\
para sempre.
\end{verse}

\chapter{Múmia}

\begin{verse}
Liana\\
liame\\
linho.

Voltas e mais voltas\\
apertadas voltas\\
concêntricas.

Brancas espirais\\
tela branca\\
unguento incenso contundentes\\
aromas.

Lianas\\
liames da espera\\
incubando o sono.

Linho indizível\\
branco:\\
branco arcaico em torno\\
de nada.
\end{verse}

\chapter{Caramujo}

\begin{verse}
A superfície\\
suave convexa\\
não revela seu dentro:\\
apenas brilha.

A entrada\\
estreita abóbada\\
é sóbria sombria\\
gruta.

A sequência\\
rampa enovelada\\
se estreita num pasmo\\
labiríntico.

O fim\\
limite íntimo\\
nada é além de si mesmo\\
ponto último.

A saída\\
é a volta.
\end{verse}

\chapter{Rota}

\begin{verse}
Há um rumo intacto, uma\\
absoluta aridez\\
na ave que repousa. Nela\\
o repouso é a rota: não há mais\\
necessidade de voo.
\end{verse}

\chapter{Notícia}

\begin{verse}
Não mais sabemos do barco\\
mas há sempre um náufrago:\\
um que sobrevive\\
ao barco e a si mesmo\\
para talhar na rocha\\
a solidão.
\end{verse}

\chapter{Acalantos}

\section{i}

\begin{verse}
Perde-se a forma no silêncio\\
e a cor não é mais palavra\\
da plasticidade viva:\\
coisas que eram reais e belas.

O sono\\
oblitera o real: o olho se cala\\
na indistinção final dos rumos.
\end{verse}

\medskip
\section{ii}

\begin{verse}
Não saber não saber não saber não saber\\
ser consumida\\
por tempo neutro\\
espaço arrítmico\\
onde o sangue do ser\\
não me pertence.
\end{verse}

\medskip
\section{iii}

\begin{verse}
Água constelada\\
entre as mãos incertas

e as estrelas derramadas no tempo.
\end{verse}

\medskip
\section{iv}

\begin{verse}
Um pequeno lago\\
sem sabor de forma\\
um centro repouso\\
sem nada\\
sem fundo\\
lago olho oculto\\
no sono.
\end{verse}

\part*{\textsc{iii}\\(+)}

\chapter{Ode \textsc{ii}}

\begin{verse}
O amor, imor\\
talidade do instante\\
totalização da forma\\
em ato vivo: obscura\\
força refazendo o ser.

O amor, momen\\
to do ser refletido\\
eternamente pelo espírito.
\end{verse}

\chapter{Ode \textsc{iii}}

\begin{verse}
Pouco é viver\\
mas pesa\\
como todo o ser\\
como toda a luz\\
como a concentração do tempo.
\end{verse}

\chapter{Lavra}

\begin{verse}
A semente em seu sulco\\
e o tempo vivo.

A semente em seu sulco\\
e a vida rítmica fluindo\\
para a realização do fruto.
\end{verse}

\chapter{Voo}

\begin{verse}
Flecha ato não verbo\\
impulso puro\\
corta o instante\\
e faz-se a vida\\
em acontecer tão frágil

lucidez breve\\
do movimento\\
acontecido.
\end{verse}

\chapter{Vermelho}

\begin{verse}
Tensão da rosa em\\
sábia maturidade\\
vermelhocéu contido\\
no máximo horizonte.

Tensão do horizonte em\\
vermelho rosa transposto\\
sábia rosa em seu\\
maduro silêncio.
\end{verse}

\chapter{Girassol}

\begin{verse}
Quero expressar a flor\\
e o girassol me escolhe:\\
helianto bizâncio ouro luz\\
\hfill{}ouro ouro

Variando de horizonte\\
porém sempre\\
audazmente fiel\\
fitando a luz intensamente

o girassol me escolhe:\\
adoração dourada\\
fixação tranquila\\
calor lúcido.

Flor para sempre e muito mais\\
que flor.
\end{verse}

\chapter{Gesto}

\begin{verse}
Palma\\
imóvel\\
verde

insistente verde\\
ânsia verde calma.

Silêncio insistindo\\
na unidade cega.

Existência em frio\\
esplendor aberto

Gesto na luz fixo\\
ânsia verde calma

Palma\\
imóvel\\
vida

Palma imóvel. Palma.
\end{verse}

\chapter{Sensação}

\begin{verse}
Vejo cantar o pássaro\\
toco este canto com meus nervos\\
seu gosto de mel. Sua forma\\
gerando-se da ave\\
como aroma.

Vejo cantar o pássaro e através\\
da percepção mais densa\\
ouço abrir-se a distância\\
como rosa\\
em silêncio.
\end{verse}

\chapter{Fronde}

\begin{verse}
Vida aberta sem ritmo\\
multiplicada em\\
mil lâminas abertas\\
mil lâminas vivendo a luz

lâminas sob a luz\\
como sentidos.
\end{verse}

\chapter{Luz}

\begin{verse}
A lâmpada sus\\
pensa, milagre

inatingível suspensa\\
horizonte.

Nós a olhamos fascinados.
\end{verse}

\chapter{Aurora}

\begin{verse}
Madrugada\\
negação da vertigem\\
redescoberta infinita\\
da luz.

Madrugada\\
figura limpa da unidade.
\end{verse}

\chapter{Média}

\begin{verse}
Meia lua.\\
Meia palavra.\\
Meia vida.

Não basta?
\end{verse}

\chapter{Reflexo}

\begin{verse}
O lago em círculo\\
círculo água\\
céu apreendido\\
eternidade no tempo.
\end{verse}

\part*{\textsc{iv}\\\textsc{fim}}


\chapter{Questões}

\begin{verse}
a)\\
O\\
fruto\\
arquitetado:\\
como o sermos?

b)\\
Difícil o real.\\
O real fruto.\\
Como, através\\
da forma\\
distingui-lo?

c)\\
Aguda\\
a\\
luz\\
sem forma\\
do que somos.\\
Como, sem vacilações\\
vivê-la?
\end{verse}

\chapter{Sede}

\section{i}

\begin{verse}
Beber a hora\\
beber a água\\
embriagar-se\\
com água apenas.
\end{verse}

\medskip
\section{ii}

\begin{verse}
Água? É só isso\\
que purifica.
\end{verse}

\medskip
\section{iii}

\begin{verse}
Fonte maior\\
e não oculta\\
fonte sem Narciso\\
nem flores.
\end{verse}

\medskip
\section{iv}

\begin{verse}
Bendita a sede\\
por arrancar nossos olhos\\
da pedra.

Bendita a sede\\
por ensinar-nos a pureza\\
da água.

Bendita a sede\\
por congregar-nos em torno\\
da fonte.
\end{verse}

\chapter{Fluxo}

\begin{verse}
A gênese das águas\\
é secreta e infinita\\
entre as pedras se esconde\\
de toda contemplação.\\
A gênese das águas\\
e em si mesma.

\dotfill

O movimento das águas\\
é caminho inconsciente\\
mutação contínua\\
nunca terminada.

É caminho vital\\
de si mesma.

\dotfill

O fim das águas\\
é dissolução e espelho\\
morte de todo o ritmo\\
em contemplação viva.

Consciencialização\\
de si mesma.
\end{verse}

\chapter{Rebeca}

\begin{verse}
A moça de cântaro e seu\\
gesto essencial: dar água.
\end{verse}

\chapter{O nome}

\begin{verse}
A escolha do nome: eis tudo.

O nome circunscreve\\
o novo homem: o mesmo,\\
repetição do humano\\
no ser não nomeado.

O homem em branco, virgem\\
da palavra\\
é ser acontecido:\\
sua existência nua\\
pede o nome.

Nome\\
branco sagrado que não\\
define, porém aponta:\\
que o aproxima de nós\\
marcado do verbo humano.

A escolha do nome: eis\\
o segredo.
\end{verse}

\chapter{O equilibrista}

\begin{verse}
Essencialmente equilíbrio:\\
nem máximo nem mínimo.

Caminho determinado\\
movimentos precisos sempre\\
medo controlado máscara\\
de serenidade difícil.

Atenção dirigida olhar reto\\
pés sobre o fio sobre a lâmina\\
ser numa só ideia nítida\\
equilíbrio. Equilíbrio.

Acaba a prova? Só quando\\
o trapézio oferece o voo\\
e a queda possível desafia\\
a precisão do corpo todo.

Acaba a prova se a aventura\\
inda mais aguda se mostra\\
mortal intensa desumana\\
desequilíbrio essencialmente.
\end{verse}

\chapter{Advento}

\begin{verse}
Deste tempo múltiplo\\
o que nascerá?

Da onda\\
rítmica\\
amplitude\\
da intensidade\\
amorfa\\
ritmicamente esfacelada

do múltiplo que um\\
mais que tempo virá\\
e que luz haverá além\\
do tempo?
\end{verse}

\chapter{Estrela}

\begin{verse}
Sobre a paisagem um ponto\\
de luz cósmica completa\\
e cena fixa\\
que não a encerra.

A estrela completa\\
a unidade em que\\
não habita.
\end{verse}

\chapter{Dispersão}

\begin{verse}
As aves se dispersaram\\
em céus mais infinitos

criaram distâncias exatas\\
linhas puras de ser no tempo

fugiram em palpitações\\
de nitidez absoluta

além da aparência perderam-se\\
intactas, na existência.
\end{verse}

\chapter{A estátua jacente}

\section{i}

\begin{verse}
Contido\\
em seu livre abandono\\
um dinamismo se alimenta\\
de sua contenção pura.

Jacente\\
uma atmosfera cerca\\
de tal força o silêncio

como se jacente guardasse\\
o gesto total do segredo.
\end{verse}

\medskip
\section{ii}

\begin{verse}
O jacente\\
é mais que um morto: habita\\
tempos não sabidos\\
de mortos e vivos.

O jacente\\
ressuscitado para o silêncio\\
possui-se no ser\\
e nos habita.
\end{verse}

\medskip
\section{iii}

\begin{verse}
Vemos somente o repouso\\
como uma face neutra\\
além de tudo o que\\
significa.

(Mas se nos víssemos\\
no verbo totalizado\\
--- forma que se concentra\\
além de nós ---

(Mas se nos víssemos\\
na contenção do ser\\
o repouso seria\\
expressão nítida.)

Vemos apenas\\
repouso:\\
contenção da palavra\\
no silêncio.
\end{verse}

\medskip
\section{iv}

\begin{verse}
Jaz\\
sobre o real o gesto\\
inútil: esta palma.

A palavra vencida\\
e para sempre inesgotável.
\end{verse}

\part{Helianto {[}1973{]}}

\mbox{}
\vfill
\thispagestyle{empty}

\hfill\emph{A}

\hfill\emph{Antonio Candido}

\hfill\emph{com amizade e reconhecimento}

\pagebreak
\vfill
\thispagestyle{empty}

\hfill\emph{Menina, minha menina}

\hfill\emph{Faz favor de entrar na roda} 

\hfill\emph{Cante um verso bem bonito}

\hfill\emph{Diga adeus e vá-se embora}

\hfill\textsc{cantiga de roda}

\chapter{Helianto}

\begin{verse}
Cânon\\
da flor completa\\
metro / valência / rito\\
da flor\\
verbo

círculo\\
exemplar de helianto\\
flor e\\
mito

ciclo\\
do complexo espelho\\
flor e\\
ritmo

cânon\\
da luz perfeita\\
capturada fixa\\
na flor\\
verbo.
\end{verse}

\chapter{Alvo}

\begin{verse}
Miro e disparo:\\
o alvo\\
o al\\
o a

centro exato dos círculos\\
concêntricos\\
branco do a\\
a branco\\
\quad{}ponto\\
\quad{}branco\\
atraindo todo o impacto

(Fixar o voo\\
da luz na\\
\quad{}forma\\
firmar o canto\\
em preciso\\
silêncio

--- confirmá-lo no centro\\
\quad\quad\quad{}do silêncio.)

Miro e disparo:\\
o a\\
o al\\
o alvo.
\end{verse}

\chapter{Rosácea}

\begin{verse}
Rosa primária quíntupla\\
abstrato vitral\\
das figuras do ser.

Ritmo em círculo, cinco\\
tempos de um mesmo ponto\\
interno, que se acende\\
no infinito. Rosa\\
não rosa: arquitetura\\
corforma do possível.

Abstrato vitral\\
das figuras do ser.
\end{verse}

\chapter{Sob a língua}

\begin{verse}
Sob a língua

palavras beijos alimentos\\
alimentos beijos palavras.

O saber que a boca prova\\
O sabor mortal da palavra.
\end{verse}

\chapter{Impressões}

\begin{verse}
Cimo\\
de palmeira rubra:\\
\quad\quad\quad{}``vida''.

Lago\\
de amarelo turvo:\\
\quad\quad\quad{}``tempo''.

Cubo\\
de metal opaco:\\
\quad\quad\quad``Deus''.
\end{verse}

\chapter{Marca}

\begin{verse}
As Leis olham do alto:\\
arcanjos de basalto\\
sobrevoam pesadas\\
quadriculam o humano.

Nem olham: subsistem.\\
Além do instante e do sangue\\
as Leis puramente\\
\hfill{}reinam.

--- Na estela\\
\quad{}a primeira marca\\
\quad{}na mente\\
\quad{}a primeira forma

configurando --- violentando ---\\
\quad\quad\quad{}o humano.
\end{verse}

\chapter{Herança}

\begin{verse}
O que o tempo descura\\
e que transfixa

o que o tempo transmite\\
e subverte

o que o tempo desmente\\
e mitifica.
\end{verse}

\chapter{Minério}

\begin{verse}
O metal e seu pálido\\
\hfill{}horizonte

o seu fulgor apenas\\
\hfill{}superfície

--- sua presençatempo\\
\quad{}erigida em silente\\
\quad{}espaço neutro.

O metal tempo opondo-se\\
ao olhar vivo: o metal adensando\\
e horizonte em fronteira\
\hfill{}inviolada.

O metal presença\\
\hfill{}íntegra\\
opondo às águas seu frio\\
e incorruptível núcleo.
\end{verse}

\chapter{Tela}

\section{i}

\begin{verse}
O\\
tecido:\\
não sabemos qual\\
a trama.
\end{verse}

\medskip
\section{ii}

\begin{verse}
Avesso\\
ou\\
direito:

como julgar o denso\\
amor vivido?
\end{verse}

\medskip
\section{iii}

\begin{verse}
Figuras.

Realmente\\
figuras?

Intencionalmente\\
impressas

ou acidentes\\
--- face nossa\\
\quad{}ao espelho?
\end{verse}

\medskip
\section{iv}

\begin{verse}
O\\
tecido:\\
como subtrairmo-nos\\
à trama?
\end{verse}

\chapter{Escultura}

\begin{verse}
O aço não desgasta\\
seus espelhos múltiplos\\
curvas\\
arestas\\
apocalíptica fera.

O aço não se entrega\\
e nem se estraga é\\
\hfill{}forma\\
--- presença imposta sem signos.

O aço ameaça\\
--- imóvel ---\\
com a aspereza total\\
de seu frio.

Ó forma\\
violenta pura\\
como emprestar-te algo\\
\hfill{}humano

uma vivência\\
um nome?
\end{verse}

\chapter{Forma}

\begin{verse}
Forma\\
como envolver-te\\
se dispões os seres\\
em composição plena?

Forma\\
como abraçar-te\\
se abraças o ser\\
em estrutura e plenitude?

Forma\\
densamente forma\\
como revelar-te\\
se me revelas?
\end{verse}

\chapter{Poemas do leque}

\section{i}

\begin{verse}
O\\
leque\\
fechado:\\
ausência.
\end{verse}

\medskip
\section{ii}

\begin{verse}
A plumagem das sombras\\
a textura\\
do silêncio brunido\\
(viva espera)

a plumagem espanto\\
a pupila\\
da atenção cega densa\\
(sombra
\quad\quad{}viva).
\end{verse}

\medskip
\section{iii}

\begin{verse}
O\\
leque\\
fechado:\\
espera.
\end{verse}

\medskip
\section{iv}

\begin{verse}
Grau a grau\\
(leque abrindo-se)\\
gesto por gesto\\
(leque abrindo-se) trama-se\\
a antirrosa e seu brilho\\
\quad\quad\quad{}gesto\\
\quad\quad\quad{}pleno.
\end{verse}

\medskip
\section{v}

\begin{verse}
Cultiva-se (cultua-se)\\
em ato extremo\\
a antirrosa\\
esplêndida

apresenta-se (apreende-se)\\
o árido ápice\\
luz vertical\\
extrema.
\end{verse}

\medskip
\section{iv}

\begin{verse}
Re-descoberta:\\
o olharamor\\
apreende o
\hfill\textsc{que}
\end{verse}

\medskip
\section{vii}

\begin{verse}
Leque aberto. O\\
real\\
---  o insolúvel real\\
\quad{}presença apenas.
\end{verse}

\chapter{Caleidoscópio}

\begin{verse}
Acontece: um\\
\quad{}giro\\
\quad{}e a forma brilha.

Espelhos do instante\\
\quad{}filtram\\
a ordem pura cores forma\\
\quad{}brilho

(e sem nenhuma\\
\quad{}palavra).

Acontece: outro\\
\quad{}giro\\
\quad{}outra forma e o mesmo\\
\hfill{}brilho.

Ó espelho dos instantes\\
\hfill{}fragmentos\\
estruturados em reflexos\\
\hfill{}fúlgidos!

Acontece: novo\\
\hfill{}giro\ldots{}\\
O caleidoscópio quebra-se.
\end{verse}

\chapter{Sol}

\begin{verse}
Sol.\\
Sol maiormente. Alucinado.

Sol\\
trespassante: há aberturas\\
\quad\quad\quad{}no sangue\\
\quad\quad\quad{}há janelas de vidro\\
\quad\quad\quad{}na mente.

Que mito subsiste\\
---  que infância ---\\
sob este Sol que ternura\\
\quad\quad{}nos resta?

Só o mito maior\\
deste Sol\\
puro.

Sol\\
sem nenhuma sombra\\
\quad\quad{}possível.
\end{verse}

\chapter{Prata}

\section{i}

\begin{verse}
Luz pesando sobre a\\
\quad\quad\quad{}prata.\\
Luz\quad\quad tensa\\
dói\quad\quad fere\\
\quad nota argêntea se\\
\quad oferece.

Luz vibrando sobre\\
\quad\quad\quad o espelho

forma \quad luz modulando-se \quad madura\\cristal \quad prata luzindo \quad fixada.
\end{verse}

\medskip
\section{ii}

\begin{verse}
De prata o campo --- o escudo ---\\
\hfill{}e através\\
\hfill{}dele\\
o silêncio imprevisto (o antigo)\\
\hfill{}pleno.

Águas transparentes\hfill silêncio\\
--- o escudo erguido no silêncio --- campo\\ de transparência e de prata.
\end{verse}

\medskip
\section{iii}

\begin{verse}
Vieram barcos. Planícies\\
de ondulação e sal coagularam-se\\
em austero silêncio (prata): o tempo\\
anula-se nesta vinda.

Vieram\\
nunc et semper\\
velas vivas.

Todos os modos do silêncio (mesmo\\
o mais austero) \hfill coalharam-se nas\\
\hfill asas\\
dos barcos vivos (prata) \hfill e o não\\
\hfill tempo\\
é horizonte porto\hfill fulgor\\
\hfill branco\\
das planícies abertas\ldots{} Campo em prata.
\end{verse}

\medskip
\section{iv}

\begin{verse}
Cristal \hfill página branca\\
nudez? \hfill nada?\\
Campo de possíveis. Aurora.\\
E o helianto\\
\quad\quad\quad--- pleno ---\\
\hfill sobre a prata.
\end{verse}

\chapter{Onde a fonte?}

\begin{verse}
Onde a\\
\quad fonte?

\quad Secas mãos conchas\\
\quad plasmam-se\\
\quad receptivos leitos\\
\quad a seu fluxo

\quad Vasos aguardam\\
\quad\quad\quad pacientes.

Onde a\\
\quad fonte? Na sede\\
\quad um frescor nascituro
\quad\quad\quad se acentua.
\end{verse}

\chapter{As sereias}

\begin{verse}
Atraídas e traídas\\
atraímos e traímos

Nossa tarefa: fecundar\\
\hfill atraindo\\
nossa tarefa: ultrapassar\\
\hfill traindo\\
o acontecer puro\\
que nos vive.

Nosso crime: a palavra.\\
Nossa função: seduzir mundos.

Deixando a água original\\
cantamos\\
sufocando o espelho\\
do silêncio.
\end{verse}

\chapter{Fera}

\begin{verse}
Na imóvel floresta \hfill um ritmo\\
oculto pelo Sol \hfill pelos ramos\\
no meio-dia \quad o medo armado \hfill o salto

(o tempo irá deflagrar\\
\quad\quad\quad o seu raio\\
anulando o limbo\\
\quad\quad\quad a ausência\\
o emboscado poder\\
\quad\quad\quad irá ferir\\
o branco centro do eterno).

A fera: ritmo em cor\\
\quad\quad luz e sombra\\
a fera: ritmo em voo\\
\quad\quad melodia.

O perigo da fera: falsa ausência\\
no desarmado silêncio.

Intensa fera. De súbito, na\\
\hfill selva\\
o medo salta! Mas aparece o sentido.
\end{verse}

\chapter{Aurora (\textsc{ii})}

\begin{verse}
Instaura-se a forma\\
num só ato

a luz da forma é um único\\
ápice

o fruto é uma única forma\\
instaurada plenamente

(o amor é unicamente\\
quando in-forma).

\ldots{} mas custa o Sol a atravessar o deserto\\
\quad mas custa amadurecer a luz\\
\quad mas custa o sangue a pressentir o horizonte
\end{verse}

\chapter{Tato (\textsc{ii})}

\begin{verse}
Revivo a exata\\
tensão da forma:\\
a pele, o pelo\\
o pêssego.

Textura viva:\\
plano pulsando\\
face\\
sob o gesto

mãos revivendo-se\\
na aguda\\
tela mítica.
\end{verse}

\chapter{Sete poemas do pássaro}

\section{i}

\begin{verse}
O pássaro é definitivo\\
por isso não o procuremos:\\
ele nos elegerá.
\end{verse}

\medskip
\section{ii}

\begin{verse}
Se for esta a hora do pássaro\\
abre-te e saberás\\
o instante eterno.
\end{verse}

\medskip
\section{iii}

\begin{verse}
Nunca será mais a mesma\\
nossa atmosfera\\
pois sustentamos o voo\\
que nos sustenta.
\end{verse}

\medskip
\section{iv}

\begin{verse}
O pássaro é lúcido\\
e nos retalha.\\
Sangramos. Nunca haverá\\
cicatrização possível\\
para este rumo.
\end{verse}

\medskip
\section{v}

\begin{verse}
Este pássaro é reto;\\
arquiteta o real e é o real mesmo.
\end{verse}

\medskip
\section{vi}

\begin{verse}
Nunca saberemos\\
tanta pureza:\\
pássaro devorando-nos\\
enquanto o cantamos.
\end{verse}

\medskip
\section{vii}

\begin{verse}
Na luz do voo profundo\\
existiremos neste pássaro:\\
ele nos vive.
\end{verse}

\chapter{Para fixar}

\begin{verse}
Para fixar\\
a flor\\
não nos serve o espaço\\
de pauta

ela des\\
liza pre\\
cede-nos\\
no horizonte duração\\
\quad\quad\quad aberta

elaestrela nada\\
a fixa\\
mas elaflor nos fixa\\
em seu\\
voo

flor\\
que nos vive no puro\\
tempo.
\end{verse}

\chapter{Oscila}

\begin{verse}
Oscila\\
a florinstante solta\\
entre norte\\
e oriente

\quad\quad\quad\quad{}oscila o instante\\
\quad\quad\quad\quad{}leve\\
\quad\quad\quad\quad{}flor entre o oriente\\
\quad\quad\quad\quad{}e o norte

oscilafulge\\
única\\
centrando\\
na oscilação a\\
fuga\\
que a transporta

\quad\quad\quad\quad{}oscila oscila única\\ \quad\quad\quad\quad{}inocência\\
\quad\quad\quad\quad{}pupila solta entre o norte\\
\quad\quad\quad\quad{}e o nada.
\end{verse}

\chapter{As estações}

\begin{verse}
Anuncia-se a luz

e o puro Sol\\
o Sol informe\\
verte-se

\quad\quad\quad{}desencantando cores\\
\quad\quad\quad{}frutos vivos\\
\quad\quad\quad{}--- força em ciclo descobrindo-se.

\quad\quad\quad\quad{}\ldots{} mas

\quad\quad\quad\quad\quad{}há o estar da pedra\\
\quad\quad\quad\quad\quad{}há o estar do corpo\\
\quad\quad\quad\quad\quad{}há peso e forma: os frutos\\
\hfill{}apodrecem.
\end{verse}

\chapter{Ciclo}

\begin{verse}
Sob o Sol \hfill sob o tempo\\
\quad\quad\quad(em seu próprio agudo\\
\quad\quad\quad\quad\quad\quad ritmo)\\
dispersam-se \hfill intercruzam-se\\
\quad\quad\quad--- em ciclo implacável ---\\
 \quad\quad\quad\quad\quad\quad pássaros.\\
Sob o Sol \hfill sob o tempo\\
\quad\quad\quad reinventa-se\\
\quad\quad\quad (esplendor cruel) o\\
\quad\quad\quad ritmo.\\
Sob o Sol \hfill sob o tempo\\
\quad\quad\quad automáticas flores\\
\quad\quad inauguram-se.\\
Sob o Sol \hfill sob o tempo\\
\quad\quad\quad a vida se cumpre\\
\quad\quad\quad autônoma.
\end{verse}

\chapter{Estrelas}

\begin{verse}
Fixar estrelas\\
no mapa móvel\\
zodíaco.

Jogar com astros\\
e fixar-se\\
no próprio jogo.

Nomear constelações\\
---  submeter os astros\\
à palavra.

Buscar estrelas. Viver estrelas\\
\hfill --- animal siderado\\
\hfill e siderante.
\end{verse}

\chapter{Jogo}

\begin{verse}
Atira\\
a bola\\
alto\\
\textsc{mais alto}

cristal acima do universo.
\end{verse}

\chapter{São Sebastião}

\begin{verse}
As setas\\
--- cruas --- no corpo

as setas\\
no fresco sangue

as setas\\
na nudez jovem

as setas\\
--- firmes --- confirmando\\
\hfill a carne.
\end{verse}

\chapter{Gigantomaquia}

\begin{verse}
Gigantes armados. Feros\\
braços girando\\
moendo o tempo.

Armados girando. Círculos.\\
Cavalo e cavaleiro\\
em voo.

Feros braços. Gigantes\\
mais duros do que o\\
delírio.

Braços implacáveis. Giros\\
implacáveis da loucura\\
destroçando o tempo. Luta\\
feroz e triste em seu ciclo. Moinhos.

(Cavalo e cavaleiro\\
em voo.)
\end{verse}

\chapter{Astronauta}

\begin{verse}
Astro\\
nauta

corpo \quad nave \quad liberta\\
corpo \quad nave \quad memória\\
descolada do grave\\
tempoinfância

corpo \quad plexo \quad vogando\\
em campo\\
nulo

corponave memória\\
no vazio

perdido livre\\
corpo\\
despreendidamente\\
nave.

Onde o horizonte? Astro\\
\hfill cai\\
\hfill em\\
\hfill órbita.
\end{verse}

\chapter{A rosa (atualmente)}

\begin{verse}
A rosa reta\\
não a rosa\\
\quad rosa

rosa de raiva\\
não a rara\\
\quad rosa

rosa de plástico\\
não a plástica\\
\quad rosa.
\end{verse}

\chapter{Tabela}

\begin{verse}
Existe

resiste\\
persiste\\
insiste

\quad\quad Desiste
\end{verse}

\chapter{O canto}

\begin{verse}
O canto\\
e o conto:

O canto! O canto!

O conto canta. O canto\\
\hfill faz de conta.

O conto\\
o canto

A \textsc{conta}\\
A \textsc{conta}.
\end{verse}

\chapter{Stop}

\begin{verse}
Estado de sítio\\
estado de sido\\
estase.
\end{verse}

\chapter{Nau}

\begin{verse}
Flutua\\
baila\\
aladamente baila\\
sobre o fluxo.

Flutua\\
fere\\
o espelho\\
puro\\
--- insinua-se, móvel,\\
na água\\
viva.

Flutua: avança\\
(bailado e\\
luta)\\
aladamente viva\\
contra o fluxo.
\end{verse}

\chapter{Oposição}

\begin{verse}
Na oposição se completam\\
os arcanjos contrários\\
sendo a mesma existência\\
em dois sentidos.

(Um, severo e nítido\\
na negação pura\\
de seu ser. O outro\\
em adoração firmado.)

Não se contemplam e se sabem\\
um mesmo enigma cindido\\
combatem-se, mas abraçando-se\\
na unidade da essência.

Interfecundam-se no mesmo\\
bloco de ser e de silêncio\\
coluna viva em que a memória\\
cindiu-se em dois horizontes.

(Sim e não no mesmo\\
abismo do espírito.)
\end{verse}

\chapter{Odes}

\section{i}

\begin{verse}
O verbo?\\
Embebê-lo de denso\\
\hfill vinho.

A vida?\\
Dissolvê-la no intenso\\
\hfill júbilo.
\end{verse}

\medskip
\section{ii}

\begin{verse}
Sonho vivido desde sempre\\
--- real buscado até o sangue.
\end{verse}

\medskip
\section{iii}

\begin{verse}
O Sol cai até o solo\\
a árvore dói até o cerne\\
a vida pulsa até o centro

\ldots{} o arco se verga\\
até o extremo limite.
\end{verse}

\medskip
\section{iv}

\begin{verse}
Lavro a figura\\
não em pedra:\\
\hfill em silêncio.

Lavro a figura\\
não na pedra (inda plástica) mas no\\
inumano vazio\\
do silêncio.
\end{verse}

\medskip
\section{v}

\begin{verse}
A flor abriu-se.\\
A flor mostrou-se em sua\\
\hfill inteireza:

--- Tragamos, ouro, incenso, mirra!
\end{verse}

\chapter{Eros}

\begin{verse}
Cego?\\
Não: livre.\\
Tão livre que não te importa\\
a direção da seta.

Alado? Irradiante.\\
Feridas multiplicadas\\
nascidas de um só

\hfill abismo.

Disseminas pólens e aromas.\\
És talvez a

\hfill primavera?

Supremamente livre\\
\quad\quad\quad --- violento ---\\
não és estátua: és pureza\\
\hfill oferta.

Que forma te conteria?\\
Tuas setas armam\\
\hfill o mundo\\
enquanto --- aberto --- és abismo\\
\hfill inflamadamente vivo.
\end{verse}

\chapter{Templo}

\begin{verse}
A severa arquitetura\\
serenamente prende-nos.

As linhas vivas. Os refolhos\\
\quad\quad\quad\quad barrocos\\
\quad\quad(o céu íntimo)\\
a bela ordem aquietando-nos.

Ó interior matriz\\
(humano e sacro)\\
em que tudo é nascente\\
\quad\quad\quad\quad e brilha\\
como mistério entre nichos\\
\quad\quad\quad\quad de sombra

ó tempo\\
divinizado em luz\\
que cresce e vive.
\end{verse}

\chapter{Voo (\textsc{ii})}

\begin{verse}
Asas de\\
neve\\
buscam o\\
branco\\
cume perfeito.

Asas contra o\\
azul\\
montanha contra o azul

azul --- e --- branco.

A terra muito\\
\hfill abaixo.\\
Muito abaixo o odor\\
\hfill do sangue.
\end{verse}

\chapter{Composição}

\begin{verse}
Cavalo branco em campo verde\\
parado\\
sereno\\
branco corcel ao longe\\
realidade\\
e miragem.

\ldots{} numa viagem branca, através\\
de todos os verdes\\
a forma se tornava\\
em ritmo, delírio\\
de forças desatadas\\
impulso leve e forte\\
que saltava horizontes\\
que rasgava as tormentas\\
\hfill e as dores\ldots{}

Mas agora, parado,\\
o ser cristalizou-se\\
na imagem de si mesmo\\
realidade lúcida\\
e plácida miragem.

\dotfill

Cavalo branco em campo verde\\
parado\\
sereno\\
branco corcel ao longe\\
realidade\\
e miragem.
\end{verse}

\chapter{O gato}

\begin{verse}
Na casa\\
inefavelmente\\
circulam olhos\\
de ouro

vibre (em ouro) a\\
\quad\quad volúpia\\
o escuro tenso\\
vulto do deus sutil\\
indecifrado

na casa\\
o imperecível mito\\
se aconchega

quente (macio) ei-lo\\
em nossos braços:\\
visitante de um tempo sacro (ou de um não tempo).
\end{verse}

\chapter{Gênesis}

\begin{verse}
Um pássaro arcaico\\
(com sabor de\\
\quad \quad origem)\\
pairou (pássaro arcano)\\
\quad sobre os mares.

Um pássaro\\
movendo-se\\
espelhando-se\\
em águas plenas, desvelou\\
o sangue.

Um pássaro silente\\
abriu\\
as\\
asas\\
--- plenas de luz profunda ---\\
sobre as águas.

Um pássaro\\
invocou mudamente\\
o abismo.
\end{verse}

\chapter{Figuras}

\begin{verse}
a) \emph{repuxo}\\
\quad A água fragmentada ascende\\
\quad em brancura dinâmica\\
\quad e no ápice de si constrói o arco\\
\quad de que perenamente cai\\
\quad regressando à unidade de seu ser.

b) \emph{estátua}\\
\quad Equilíbrio\\
\quad branco

\quad momento\\
\quad dançante\\
\quad da forma.

\quad Fluência detida do ser; forma\\
\quad --- apenas equilíbrio de ritmos.

c) \emph{esfera}\\
\quad O mundo\\
\quad preciso

\quad o mundo\\
\quad conciso\\
\quad o espaço concreto\\
\quad o tempo perfeito\\
\quad a presença íntegra

\quad o infinito\\
\quad lúdico.
\end{verse}

\chapter{Paraíso}

\begin{verse}
Animais sob o céu.\\
Puramente visíveis.\\
Postos num tempo íntegro\\
\qquad\qquad\quad sem trauma.

Os animais --- visíveis ---\\
sobre o campo\\
entre o florir suprarreal\\
\qquad\qquad\quad da aurora.

Os animais na origem.\\
Fixados\\
--- como num quadro --- inda sem voz\\
\qquad\qquad\qquad\qquad\qquad\qquad alguma

só a atenção tranquila\\
ao céu\\
que baixa\ldots{}

O céu fecundantemente.
\end{verse}

\chapter{Sonho}

\begin{verse}
O ar irreal que cai\\
compõe um nítido campo\\
onde os ritmos \qquad os tempos\\
interfecundam-se \qquad plenos.

Imagens --- ó cores puras! --- sem peso\\
amplitude intangível \quad claros pomos\\
peixes sutis na água viva \qquad peixes\\
deslizando --- secretos --- no silêncio.

O ar irreal que cai \qquad a queda lúcida\\
dentro do sono \quad a grande flor aberta\\
o íntimo tempo que se instaura \mbox{ }mito\\
rápidos peixes \mbox{ }e pássaros \mbox{ }e campos.


O ar irreal que cai\\
e se constela\\
--- o absoluto no horizonte\\
\quad do tempo.
\end{verse}

\chapter{Repouso}

\begin{verse}
Basta o profundo ser\\
em que a rosa descansa.

Inúteis o perfume\\
e a cor: apenas signos\\
de uma presença oculta\\
inútil mesmo a forma\\
claro espelho da essência

inútil mesmo a rosa.

Basta o ser. O escuro\\
mistério vivo, poço\\
em que a lâmpada é pura\\
e humilde o esplendor\\
das mais cálidas flores.

Na rosa basta o ser:\\
nele tudo descansa.
\end{verse}

\chapter{Poemetos}

\begin{verse}
a) \emph{manhã}\\
\quad Ninguém ainda. As rosas me saúdam\\
\quad e eu saúdo o silêncio\\
\quad das rosas.

b) \emph{ausência}\\
\quad Aqui ninguém\\
\quad e nuvens.

c) \emph{ave}\\
\quad Asas suspensas em
\quad instanteluz.

d) \emph{lua}\\
\quad Integralidade.\\
\quad Fixidez.

e) \emph{Narciso}\\
\quad A flor a água a face\\
\quad a flor a água\\
\quad a flor.

f) \emph{primavera}\\
\quad Da não-espera\\
\quad acontecem as\\
\quad flores.

g) \emph{lago}\\
\quad Tensão\\
\quad fria\\
\quad da água: paz --- em --- ser.

h) \emph{espera}\\
\quad As janelas abertas.\\
\quad A porta apenas encostada\ldots{}

i) \emph{vaso}\\
\quad mas incomunicante.

j) \emph{fim}\\
\quad A ausência das rosas. O caminho\\
\quad já sem ninguém, para o silêncio.
\end{verse}

\chapter{Ode}

\begin{verse}
E enquanto mordemos\\
frutos vivos\\
declina a tarde.

E enquanto fixamos\\
claros signos\\
flui o silêncio.

E enquanto sofremos\\
a hora intensa

lentamente o tempo\\
perde-nos.
\end{verse}

\chapter{Elegia (\textsc{i})}

\begin{verse}
Mas para que serve o pássaro?\\
Nós o contemplamos inerte.\\
Nós o tocamos no mágico fulgor das penas.\\
De que serve o pássaro se\\
desnaturado o possuímos?

O que era voo e eis\\
que é concreção letal e cor\\
paralisada, íris silente, nítido,\\
o que era infinito e eis\\
que é peso e forma, verbo fixado, lúdico

o que era pássaro e é\\
o objeto: jogo\\
de uma inocência que\\
o contempla e revive\\
--- criança que tateia\\
no pássaro um esquema\\
de distâncias ---

mas para que serve o pássaro?

O pássaro não serve. Arrítmicas\\
brandas asas repousam.
\end{verse}

\chapter{Elegia (\textsc{ii})}

\begin{verse}
Os extremos do vento\\
sons\\
partidos.

Os extremos os\\
mais\\
agudos cumes\\
da tensão viva amor\\
--- criação viva ---

agora par\\
\qquad tidos\\
luz e lira\\
inertes.

Os extremos do amor:\\
áridos\\
restos.
\end{verse}

\chapter{A paisagem em círculo}

\begin{verse}
Os plátanos as pombas estas fontes\\
as frondes, longe; e, de novo, os\\
\qquad\qquad\qquad\qquad\qquad\qquad\quad plátanos.

As pombas estes plátanos as frondes\\
as fontes, longe; e, de novo, as\\
\qquad\qquad\qquad\qquad\qquad\qquad\quad pombas.

As fontes estas frondes estas pombas\\
plátanos, longe; e, de novo, as\\
\qquad\qquad\qquad\qquad\qquad\qquad\quad fontes.

Estas frondes os plátanos as fontes\\
as pombas, longe; e, de novo, as\\
\qquad\qquad\qquad\qquad\qquad\qquad\quad frondes.
\end{verse}

\chapter{Claustro (\textsc{ii})}

\begin{verse}
Antigo\\
jardim fechado:\\
águas, azulejos\\
\qquad e sombra.

Macular esta paz?\\
\qquad Proibido.\\
Só leves pensamentos\\
\qquad transitam

--- leves, tão\\
\qquad leves\\
que agravam mais o silêncio.

E o jardim se aprofunda\\
\qquad\qquad\qquad\qquad espelho\\
verde do abismo: céu\\
nas águas claras

e este chão não existe\\
\quad --- tudo é abismo ---\\
e esta paz é vertigem\\
\quad --- puro abismo ---\\
e o pensamento fixo\\
\quad --- mudo abismo ---

tudo amplia mais o silêncio.
\end{verse}

\chapter{Estrada}

\begin{verse}
A estrada percorre\\
\qquad\qquad o bosque\\
entre árvores mudas\\
entre pedras opacas\\
entre jogos de luz\\
\qquad\qquad e sombra.

A estrada caminha\\
e o seu solo\\
(ancestralmente fundo)\\
não tem som.

A estrada prossegue\\
e seu silêncio\\
fixa presenças densas\\
e embriaga\\
sufocando toda a\\
\qquad\qquad memória\ldots{}
\end{verse}

\chapter{Termos}

\begin{verse}
Despreende-se a seta \qquad\qquad alvo alcançado\\
apreende-se o tempo \qquad\qquad flor colhida

não mais além \qquad\qquad\qquad\quad só isto

\quad\qquad\qquad\qquad\qquad --- é

\quad\qquad\qquad\qquad\qquad\quad tudo ---

\qquad\qquad concentrado fruto e fonte.

Flor alcançada \qquad\qquad\qquad vida exata

\qquad\qquad\qquad\qquad\textsc{é}

\qquad\qquad\qquad\qquad\textsc{tudo}

elimina-se a meta \qquad\qquad\quad jogo findo.
\end{verse}

\part{\textsc{alba} {[}1983{]}}

\thispagestyle{empty}

\mbox{}
\vfill
Para

Davi

Haquira

Lucia

Ana Maria

\pagebreak
\thispagestyle{empty}

\mbox{}
\vfill
\setlength{\epigraphwidth}{.3\textwidth}
\epigraph{\emph{Que bien sé yo la fonte}\\
\emph{que mana y corre,}\\
\emph{aunque es de noche.}}{\textsc{san juan de la cruz}}

\pagebreak
\thispagestyle{empty}

\mbox{}\vfill
\begin{verse}
A um passo\\
do pássaro\\
res\\
piro.
\end{verse}

\chapter{Alba}

\forceindent\textsc{i}

\begin{verse}
Entra furtivamente\\
a luz\\
surpreende o sonho inda imerso\\
\qquad\qquad\qquad\qquad\qquad na carne.
\end{verse}

\medskip
\textsc{ii}

\begin{verse}
Abrir os olhos.\\
Abri-los\\
como da primeira vez\\
--- e a primeira vez\\
\quad é sempre.
\end{verse}

\medskip
\textsc{iii}\\

\begin{verse}
Toque\\
de um raio breve\\
e a violência das imagens\\
no tempo.
\end{verse}

\medskip
\textsc{iv}

\begin{verse}
Branco\\
sinal oferto\\
e a resposta do\\
sangue:\\
\textsc{agora!}
\end{verse}


\chapter{Poema}

\begin{verse}
Saber de cor o silêncio\\
diamante e/ou espelho\\
o silêncio além\\
do branco.

Saber seu peso\\
seu signo\\
--- habitar sua estrela\\
\quad impiedosa.

Saber seu centro: vazio\\
esplendor além\\
da vida\\
e vida além\\
da memória.

Saber de cor o silêncio

--- e profaná-lo, dissolvê-lo\\
\qquad\qquad\qquad em palavras.
\end{verse}

\chapter{Vigília}

\begin{verse}
Momento\\
\qquad\qquad pleno:\\
pássaro vivo\\
atento a.

Tenso no\\
\qquad\qquad instante\\
--- imóvel voo ---\\
plena presença\\
pássaro e\\
\qquad\qquad signo

(atenção branca\\
aberta e\\
\qquad\qquad vívida).

Pássaro imóvel.\\
Pássaro vivo\\
atento\\
a.
\end{verse}

\chapter{Clima}

\begin{verse}
Neste lugar marcado: campo onde\\
uma árvore única\\
se alteia

e o alongado\\
gesto\\
absorvendo\\
todo o silêncio --- ascende e\\
\qquad\qquad imobiliza-se

(som antes da voz\\
pré-vivo\\
ou além da voz\\
e vida)

neste lugar marcado: campo\\
\qquad\qquad imoto\\
segredo \quad cio \quad cisma\\
o ser\\
celebra-se

--- mudo eucalipto\\
\quad elástico\\
\quad e elíptico.
\end{verse}

\chapter{Pouso (\textsc{ii})}

\begin{verse}
Difícil para o pássaro\\
\qquad\qquad pousar\\
\qquad\qquad manso\\
em nossa mão --- mesmo\\
\qquad\qquad aberta.

Difícil difícil\\
para a livre\\
\qquad vida\\
repousar em quietude\\
\qquad\qquad limpa\\
\qquad\qquad densa

e inda mais\\
\qquad difícil\\
--- contendo o\\
\qquad voo\\
imprevisível ---

maturar o seu canto\\
no alvo seio\\
de nosso aberto\\
mas opaco

silêncio.
\end{verse}

\chapter{Cisne}

\begin{verse}
Humanizar o cisne\\
é violentá-lo. Mas\\
também quem nos dirá\\
o arisco esplendor\\
--- a presença do cisne?

Como dizê-lo? Densa\\
a palavra fere\\
o branco\\
expulsa a presença e --- humana ---\\
é esplendor memória\\
\qquad\qquad\qquad e sangue.

\qquad\qquad\qquad E\\
\qquad\qquad\qquad resta

não o cisne: a

\qquad\qquad\qquad palavra

--- a palavra mesmo

\qquad\qquad\qquad cisne.
\end{verse}

\chapter{Composição}

\begin{verse}
Compor os pomos\\
\qquad\qquad\qquad --- exatamente ---\\
\qquad\qquad\qquad\qquad\qquad\qquad até

que os signos\\
\qquad\qquad\qquad --- deiscentes ---\\
\qquad\qquad\qquad\qquad\qquad\qquad transfigurem-se.

Compor os pomos\\
\qquad\qquad\qquad\qquad até\\
a anárquica primavera.

Compor \qquad transpor\\
\quad até\\
a rosa única\\
--- múltiplo\\
\quad espanto.
\end{verse}

\chapter{Trama}

\begin{verse}
Tecem-se voos\\
campos dóceis\\
esperas

tecem-se verbos\\
atentas claras\\
luzes

tecem-se formas\\
jogos maduros\\
redes

tecem-se tempos\\
para um só ato\\
infindo.
\end{verse}

\chapter{Bdas de caná}

\forceindent\textsc{i}
\begin{verse}
Da pura água\\
criar o vinho\\
do puro tempo extrair\\
o verbo.
\end{verse}

\medskip
\textsc{ii}

\begin{verse}
Milagre (anti-\\
milagre)\\
era tornar em água\\
o vinho\\
vivo.
\end{verse}

\medskip
\textsc{iii}

\begin{verse}
A água embriaga\\
mas para além do humano: no amor\\
simples.
\end{verse}

\medskip
\textsc{iv}

\begin{verse}
Para os anjos a\\
água. Para nós\\
o vinho encarnado\\
sempre.
\end{verse}

\chapter{As trocas}

\begin{verse}
Um fruto por um\\
\qquad\qquad\qquad ácido\\
um sol por um\\
\qquad\qquad\qquad sigilo\\
o oceano por um\\
\qquad\qquad\qquad núcleo

o espaço por uma\\
\qquad\qquad\qquad fuga\\
a fuga por um\\
\qquad\qquad\qquad silêncio

--- riquezas por uma\\
\qquad\qquad\qquad nudez.
\end{verse}

\chapter{Caça}

\begin{verse}
Visar o centro\\
ou, pelo menos,\\
o melhor lado\\
(o mais frágil).

Astúcia e tempo\\
(paciência armada)\\
e --- na surpresa\\
do golpe rápido ---

colher a coisa\\
que, apreendida,\\
rende-se?

Não: desnatura-se\\
ao nosso ato\ldots{}\\
Ou foge.
\end{verse}

\chapter{A mão}

\forceindent\textsc{i}

\begin{verse}
A mão destrói imagens\\
descristaliza signos

e a luz de novo\\
desabitada\\
pulsa serenamente\\
em frio ímpeto.
\end{verse}

\medskip
\textsc{ii}

\begin{verse}
A mão destrói-se\\
furtando-se\\
à textura do ser\\
e do silêncio

e --- naufragada a forma ---\\
subsiste uma estrela\\
sobre as águas.
\end{verse}

\chapter{Trovões}

\begin{verse}
Trovões invadem\\
casas\\
coisas\\
quebram\\
louças gráficos\\
\qquad\qquad\qquad vidros.

Anulam o supérfluo: articulam\\
um campo para o destino.

Trovões transportam raízes\\
a altas distâncias nuas\\
tentando armar uma flor\\
com o que resta --- ainda ---\\
do silêncio.
\end{verse}

\chapter{Prometeu}

\begin{verse}
A Lei\\
cinzenta --- ave de\\
\qquad\qquad\qquad\quad rapina ---

voa mas\\
pesa: desce e\\
\qquad\qquad\quad busca\\
\qquad\qquad\quad o Sangue

o Sangue: agravo\\
o Sangue: gravidade.

Peso da\\
Lei\\
peso do\\
Sangue

--- destruição rubro-cinza.
\end{verse}

\chapter{Touro}

\forceindent\textsc{i}

\begin{verse}
No verde campo\\
o touro\\
qual noite exposta\\
em claro\\
dia

no verde chão\\
da irrealidade\\
a violência:\\
o sangue contido\\
(ainda).

\medskip
\textsc{ii}

No verde dia\\
(fábula)\\
a morte? A\\
\textsc{vida}

--- tão brutalmente\\
\quad \textsc{vida}\\
\quad que a tememos.
\end{verse}

\chapter{Centauros}

\begin{verse}
Centauros derrubam ídolos

centauros derrubam-se\\
centauros centauros.

Mas a memória\\
--- texto pul\\
\qquad\qquad sante ---\\
\quad mas a memória\\
--- rito do\\
\quad\quad sangue ---

mas a memória\\
--- sempre a\\
\qquad\qquad memória ---

absorvendo o ímpeto\\
floresce.
\end{verse}

\chapter{Peixe}

\begin{verse}
Gira\\
forma oblíqua no espelho\\
cor\\
capturada em fria\\
plenitude.

Gira\\
na transparência a\\
\qquad\qquad\qquad\qquad forma\\
apenas forma: sem\\
\qquad\qquad\qquad\qquad fuga.

Apenas forma: ciclo\\
ritmo submerso\\
sem asas para o tempo.
\end{verse}

\chapter{Uvas}

\begin{verse}
Mesclados: o mel\\
\qquad\qquad\quad e o mal

a vida: madura\\
\qquad\quad impura

doces-pobres\\
bagos

em que o gozo\\
do mel\\
inclui o mal

em que o gosto\\
de podre\\
aguça o fruto.
\end{verse}

\chapter{Mito}

\begin{verse}
Bizâncio:\\
grande céu dourado\\
sem pássaros.

Bizâncio:\\
os mosaicos sem tempo\\
luz\\
imota.

Bizâncio:\\
o eterno helianto\\
--- a estrela\\
\quad fixa.
\end{verse}

\chapter{Mosaico}

\begin{verse}
Os anjos fortes eretos.

Faces\\
neutras\\
vestes\\
claras\\
asas tranquilas\\
imotas.

Os anjos.\\
Inamovíveis.
\end{verse}

\chapter{Penélope}

\begin{verse}
O que faço des\\
\qquad\qquad faço\\
o que vivo des\\
\qquad\qquad vivo\\
o que amo des\\
\qquad\qquad amo

(meu ``sim'' traz o ``nao''\\
\qquad\qquad no seio).
\end{verse}

\chapter{Relógio}

\begin{verse}
Hora dos\\
peixes\\
hora dos\\
náufragos\\
hora do es\\
pesso\\
concreto abismo

hora das\\
algas\\
lentas flu\\
tuantes\\
hora das\\
ondas\\
brandas in\\
findas

hora dos\\
peixes\\
densos\\
obscuros\\
na obscuridade líquida.
\end{verse}

\chapter{As Parcas}

\begin{verse}
As Parcas\\
fiam\\
nada\\
tecendo

tecendo o\\
\qquad\quad nada\\
em lento fio\\
branco? Nem\\
branco:

apenas pura\\
perda, sussurro\\
de lento canto\\
que autoesvazia-se

e --- inútil ---\\
\qquad tomba\\
evanescendo-se\\
na transparência.

\dotfill

Apenas\\
isto:\\
Parcas vigilam.\\
Cintila o\\
mar.
\end{verse}

\chapter{Odes}

\forceindent{i}

\begin{verse}
No mar interior em que\\
olhosvivências se tramam\\
no mar estruturado\\
de olhares em que a vida\\
se adensa, não há falhas

e onde tudo é vivo nenhum\\
barco furtivo se aventura
\end{verse}

\medskip
\textsc{ii}

\begin{verse}
Retezo o arco e o\\
\qquad\qquad\qquad sonho

espero:

nada mais é preciso.
\end{verse}

\medskip
\textsc{iii}

\begin{verse}
O fruto arde e se consome\\
o vinho sustenta os pássaros\\
a liberdade das águas\\
dissolve-nos.

Bebemos profundamente\ldots{}\\
Não é preciso renascer.
\end{verse}

\chapter{Poemetos (\textsc{ii})}

\begin{verse}
\emph{Brejo}\\
Água parada água parada água pa\\
rando\\
sob a cintilação dos lírios.

\medskip
\emph{O azul}\\
o exílio.

\medskip
\emph{Fonte}\\
As águas levando\\
as palmas\\
as águas lavando\\
os olhos\\
as águas livrando\\
tudo.

\medskip
\emph{A estrela próxima}\\
Próxima: mas ainda\\
estrela\\
--- muito mais estrela\\
\quad que próxima.

\medskip
\emph{Sal}\\
ritmo\\
flama\\
ciclo\\
--- rio absoluto\\
\quad do sangue.

\medskip
\emph{Centro}\\
O que é tão puro que enlouquece as flores\\
o que é tão puro que magnetiza o deserto\\
o que é tão puro que nem simplesmente existe.

\medskip
\emph{Reflexos}\\
No olho --- espelho ---\\
na água --- espelho ---\\
no tempo --- espelho ---

espelhos nos\\
\qquad\qquad\quad espelhos nos\\
\qquad\qquad\quad\qquad\qquad\quad espelhos

--- infinito irreal --- o sonho\\
\qquad\qquad\quad\qquad\qquad\qquad flui.
\end{verse}

\chapter{Murmúrio}

\begin{verse}
O murmúrio não cessa. Nunca a\\
\quad\qquad\qquad\qquad\qquad\quad fonte

deixará de cantar\\
oculta

e oculto mesmo\\
o canto\\
soterrado em cansaço\\
hábito e olvido

e tudo oculto sob árida\\
lápide\\
sob o contínuo deslizar\\
das formas

e tudo\\
oculto\\
mas água\\
sempre

pulsação\\
viva\\
centrando\\
o\\
tempo.
\end{verse}

\chapter{Mapa}

\begin{verse}
Eis a carta dos céus:\\
as distâncias vivas\\
indicam apenas\\
roteiros\\
os astros não se interligam\\
e a distância maior\\
é olhar apenas.

A estrela\\
voo e luz somente\\
sempre nasce agora:\\
desconhece as irmãs\\
e é sem espelho.

Eis a carta dos céus: tudo\\
indeterminado e imprevisto\\
cria um amor fluente\\
e sempre vivo.

Eis a carta dos céus: tudo\\
\qquad\qquad\qquad\qquad se move.
\end{verse}

\chapter{Noturno}

\begin{verse}
O silêncio sem cor nem peso\\
(vacuidade) sustenta\\
agudas sementes --- júbilo ---\\
da lucidez nunca\\
\qquad\qquad extinta.

Grandes estrelas fixas.
\end{verse}

\chapter{Alba (\textsc{ii})}

\begin{verse}
A estrela d'alva --- puríssimo\\
centro da aurora --- sidera-me\\
penetra-me até à vertigem.
\end{verse}

\chapter{Alba (\textsc{iii})}

\begin{verse}
Ó rosa face\\
emergente:\\
puro gosto de luz\\
branca.
\end{verse}

\chapter{Antártida}

\begin{verse}
O campo branco (nenhum mapa) intenso.

Os passos consomem-se\\
o espaço introverte-se\\
branco branco

asséptico/absoluto.

Norte nenhum\\
noite nenhuma\\
--- branco sobre\\
\qquad\qquad\quad o branco.
\end{verse}

\chapter{Espelho}

\begin{verse}
O espelho\\
lúcido branco\qquad silente\\
imóvel lâmina\qquad fluxo\\
o espelho:\qquad\qquad corola\\
\qquad\qquad\qquad\qquad branca

o espelho\\
branco centro da

enorme corola

forma vazia do branco

o espelho

vertigem áspera

flor sem memória fluência
\end{verse}

\begin{itemize}
\item
  intensa corola branca.
\end{itemize}

\begin{quote}
 ()



Fita-nos o cristal, vácuo de onde emergem rosas pássaros.

Fita-nos o tempo. Viva

a infância nos rememora.



Aves

disparam no espelho vívidas

aves

lucidamente navegam no puro cristal

do tempo.

 ()

Doce perfume des falecente, rosa

mais-que-perfeita: solta em voo

puro.

Doces pétalas vivas.

 ()

Um barco

fende --- tranquilo --- o mar (o amor) transporta
\end{quote}

\begin{itemize}
\item
  voo profundo --- o esplendor do silêncio.
\end{itemize}

\begin{quote}
Um barco

fende o rumor do mar transporta
\end{quote}

\begin{itemize}
\item
  silente ânfora --- a imortal lucidez
\end{itemize}

\begin{quote}
do branco

a

siderante impossível primavera.

 ()

Os pássaros retornam sempre e sempre.

O tempo cumpre-se. Constrói-se a evanescente forma

ser e

ritmo.

Os pássaros retornam. Sempre os pássaros.

A infância volta devagarinho.



A fonte (oculta) ignora-se.

Escamas: sóis

intranquilos torrente: luz

que se quebra oferta multi

plicada .

\ldots{} mas na escura gruta intacta

a fonte --- serena --- expande-se.
\end{quote}











A madrugada.

Seu coração de silêncio.

O silêncio cheio de peixes

de irisados peixes úmidos.

Grandes árvores ânforas

transbordantes de silêncio.

Galos

no alto silêncio impressos

seda

translúcida do silêncio.

\begin{quote}


Ainda há maior nudez: barreira ininterrupta do silêncio

guardando em si a evidência das formas.

Ainda há maior nudez: evidência sem mais sinais

exata em sua luz interna.

Ainda há maior nudez: a luz infinda simplicidade

sem apoio além de si mesma.



Do leste vieram pássaros rápidos leves

nem sombra nem rastro deixam:

apenas passam. Não pousam.





Há um caminho solitário construído a cada
\end{quote}





passo: não leva a lugar algum.

Na floresta um branco

\begin{quote}
pássaro
\end{quote}

oculta-se em seu

\begin{quote}
silêncio.
\end{quote}

No alto

--- jubilosamente --- uma estrela apenas.







 ()

Águas não cantam:

fluem suaves fogem.

Fresco silêncio:

a flor não fala.

Nenhum ruído. Apenas brancas pétalas

da flor que navega nas águas esplêndidas.

\begin{quote}


Tensa uma flama

no denso silêncio

vela

imóvel brilha

intensa vigília

áurea

esfera cálida

--- brilho e sigilo ---

no intenso silêncio vibra e

vela.



O início? O mesmo fim. O fim? O mesmo início.

Não há fim nem início. Sem história o ciclo dos dias

vive-nos.

 ()

O instante-surpresa: pássaros atravessando o silêncio

o instante

surpreso: conchas esmaltadas imóveis

o instante

esta pedra tranquila.



No espelho a vida

a pura vida já sem

palavras.

A vida viva.

A vida quem?

A vida

em branco espelho puro:

ninguém ninguém.



Ó rio

subterrâneo ao ritmo do sangue

ó água frígida clara

que elimina toda a sede

ó água abissal sem gosto

nem vestígio algum de tempo

ó fonte

sem mais música audível: água

densa

que nos limpa de todas as palavras.
\end{quote}

\section{Rosácea {[}1986{]}}\label{rosuxe1cea}

\begin{quote}
\emph{in memoriam de meus pais}

\emph{Coisas varridas e ao acaso mescladas}

\emph{--- o mais belo universo}


\end{quote}

\subsection{}\label{section-4}

\begin{quote}


Rosa, rosas. A primeira cor. Rosas que os cavalos esmagam.



Se vens a uma terra estranha curva-te

se este lugar é esquisito curva-te

se o dia é todo estranheza submete-te

--- és infinitamente mais estranho.



Só porque erro encontro

o que não se procura

só porque erro invento

o labirinto

a busca a coisa

a causa da procura

só porque erro acerto: me construo.

Margem de erro: margem de liberdade.



Voo onde ninguém mais --- vivo em luz

mínima ouço o mínimo arfar --- farejo o

sangue

e capturo a presa

em pleno escuro.



A tranquila explosão fria

fora do tempo e nos olhos

esplêndida solitária

no ápice do amor tremeluzia.



Da avó materna:

uma toalha (de batismo).

Do pai:

um martelo um alicate uma torquês duas flautas.

Da mãe: um pilão

um caldeirão um lenço.

 ()

Duas coisas admiro: a dura lei cobrindo-me

e o estrelado céu dentro de mim.

  ()

As ignotas (des)razões do espanto.

 

Há um tempo para desarmar os presságios

há um tempo para desamar os frutos

há um tempo para desviver

o tempo.



Matar o pássaro eterniza o silêncio

matar a luz elimina o limite

matar o amor instaura a liberdade.



É importante acordar a tempo

é importante penetrar o tempo

é importante vigiar

o desabrochar do destino.
\end{quote}







  ( ó)

O relógio horologium a hora

o logos.

Os peixes estão no aquário

o touro está na balança

e a virgem parindo

os gêmeos.

Os relógios estão na eternidade.

\begin{quote}


Ei-la

dor de milhares força de humanidade anônima

(do faraó nem cinzas).



Os que nascem de noite e, entre ossos, vigiam

o fogo os que olham os astros e, oprimidos, respiram

em cavernas

os que vão viver apesar da escuridão e nos olhos a luz clandestina

acendem

os que não sonham, os que nascem

de noite não vieram brincar: seu peito guarda uma só palavra.

  

--- a firme montanha o mar indomável

o ardente silêncio ---

em tudo pulsa e penetra

o clamor

do indomesticável destino.



A madrugada futura

já existindo na lembrança a memória in

chando o tempo vivo pin

(Tudo

gando dos olhos.

contaminado de tudo.)



amargas cobrem o barco

as águas salobras trazem

o dilúvio, o naufrágio, o necessário batismo.

Através do silêncio cai a

água

filtra-se através do ser a inextinguível água

do silêncio.

 

O

espelho: atra vés

de seu líquido nada me des

dobro.

Ser quem me olha

e olhar seus olhos

nada de nada duplo mistério.

Não amo

o espelho: temo-o.



Viajar mas não para

viajar mas sem onde

sem rota sem ciclo sem círculo sem finalidade possível.

Viajar

e nem sequer sonhar-se esta viagem.
\end{quote}

\subsection{ú}\label{section-5}

\begin{quote}


O peixe é a ave do mar

a ave

o peixe do ar

e só o homem

nem peixe nem ave

não é daquém

e nem de além e nem

o que será

já em nenhum lugar.

 

O selvagem não aprende

o selvagem não se emenda

o selvagem não se curva

(o mitológico selvagem).



A mente mente

e o corpo (ah) consente.

 

Não vim. Não vi.

Não havia guerra alguma.

  (  )

O que é o que é?

ó !

Publicitária. Mitológica. Andrógina.

 ()

Ó vida, triste vida!

Se eu me chamasse Aparecida dava na mesma.

 ()

Caio ver

ticalmente

e me transformo.





No meio

do caminho a flor nasceu.



A rosa só (mas que calor danado!)

A estrela d'alva, o escândalo

a vontade de morrer

(mas era um calor danado!)

J. J. Rousseau

*** les riches très sensibles

dans toutes les parties de leurs biens.

(Du contrat social)

 

\emph{Mário Quintana}



Exorcizar os ventos anular as estátuas recuperar os anjos

--- instaurar a alegria.

Para instaurar jardins: desencantar as fadas dissolver os rochedos
devorar as esfinges.

  ó

A poesia é impossível

o amor é mais que impossível

a vida, a morte loucamente impossíveis.

Só a estrela, só a estrela

existe

--- só existe o impossível.
\end{quote}

\subsection{ó}\label{section-6}

\begin{quote}


O vento, a chuva, o Sol, o frio tudo vai e vem, tudo vem e vai. Tenho a
ilusão de estar sonhando.

Tenho o manto de Buda, que é nenhum.

 

(Meu nome como leiga Zen-budista)

O pássaro ines

perado

O pássaro agreste. O som

silencioso, vivo, dul

císsimo.

!

!

Bem te vi, sim --- leve

pousado

no último --- altíssimo --- no fragílimo galho.

Pássaro no paraíso

dos pássaros.

Bem-te-vi (vendo-me?) desenho

vivo

no último andar de um sonho.



A cor

alada: borboleta ou pétala?

Fresca asa per passa

as mãos abertas.

Sussurro orelha caramujo antena

os cabelos ao vento.

ó

Vaca

mansamente pesada

vaca

lacteamente morna

vaca

densamente materna inocente grandeza: vaca

vaca no pasto (ai, vida, simples vaca).



Lua pen

dente lua tre

mente nas águas

vivas.



Neste tudo tudo falta

(neblina)

e nesta falta: eis tudo.



As rosas (brancas)

as claras rosas calam-se

e floresce o silêncio.

Flor terra silêncio vento

ausência de pensamento.

Encanto e espanto;

o adorável adorante helianto.

Simples a água

o amor

mais simples.

Luz

fria. Pelos caminhos as rosas brancas

em lágrimas.

A chuva lavou-me toda

sem deixar vestígios de ontem.

Pedrinha redonda fria

estrela branca nas águas.

Noite vaso negro

e o silêncio uma flor branca.



Asa sem pássaro se vai ou vem

se vem ou vai quem sabe?

Leve vazia branca.

A flor do céu. A forma do silêncio.



Frescas sombras de bruçam-se

nas águas

fontes

jorram pedras

calam-se

brilho das flores:

incendiada doçura.

Tão ácida a sede

e a água tão breve.

* * *

Tão instantâneo

o pássaro: nem mesmo

o voo é captado pelas águas.

Chegam os pássaros

devoram frutos picam grãos vivos

e álacres partem

(sonho de pássaros)

eternos aéreos livres

Semeio sóis e sons

na terra viva

afundo os pés

no chão: semeio e passo.

Não me importa a colheita.
\end{quote}

\subsection{ó}\label{section-7}

\begin{quote}


Canto o mar púrpura e as insolentes naves

que sangraram horizontes canto o mar cor de vinho o vinho

púrpura púrpura o puro púrpura canto.



Na raiz cega deste espanto há um cristal: quem o fitar

ah, quem o fitar

com os olhos em sangue com as mãos em sangue com o sangue vivo

quem o fitar não dormirá mas será cristal de espanto

--- ficará lúcido para sempre.

 

És filho do desejo e do espírito

e (como a carne é impureza) a loucura não te salva de ser, e cais

Triste Figura mesmo se o delírio te eleva

à potência do abismo

Triste Figura mesmo na alta planície em que eternizado, morres

herdeiro do desejo e do espírito.



Do amor sem fundo

--- do

inominável ---

o dragão: raio

densa energia ascende

e ao sacro ímpeto que amor resiste?

Rasgam-se os véus

do inominado.

 ()

A moça do cântaro e seu

silêncio de água e de barro.

 

O profeta prevê um extenso silêncio.

A queda

se deu: na pedra nenhuma interrogação.

(O profeta sorri tranquilamente.)



Abóbada par tida

os céus

se rompem.

Terra solvida. Vida finda. O Sopro

reabsorve-se

e a escuríssima água

bebe a luz.



Não há culpa não há desculpa não há perdão.



Vai-te, Selene, vai-te daqui vampira

Diana estéril selvagem assassina

vai-te, vai-te daqui, noiva do Hades Perséfone

vai-te caveira pedra morta Medusa

vai-te, Medeia feiticeira, Circe, dona do abismo amargo do mar doido

dona do mênstruo, vai!

Vai-te daqui, cadela Helena infame

vai-te, luz falsa, vai-te puta virgem

infernal Hécate! Vai-te daqui

!



Não há perguntas. Selvagem o silêncio cresce, difícil.
\end{quote}

\subsection{}\label{section-8}

\begin{quote}


Nem flor nem folha mas raíz

absoluta. Amarga.

Nem ramos nem botões. Raiz íntegra. Sórdida.

Nem tronco ou

caule. Nem sequer planta

--- só a raiz é o fruto.

 ()

O querubim arde no adro, erguendo sua alada cabeça, essência

apenas

o querubim arde no adro, e a pedra

em que se encarna, arde no fogo

de sua integridade absoluta.

  ()

Deserta é a praia, e grande.

Estéreis os coqueiros, inúteis

e, na areia, demarcação de água e terra, paz

vazia a concha: nem mesmo a espera a fecunda.

A tarde em mim se repete num tempo irreal, decadência obstinada, onde o

silêncio

nunca é completamente treva

A tarde em mim se repete configurando uma distância irrealizada,
evanescência onde nunca anoitece.

A tarde em mim se repete

e nunca surgem as estrelas.



Cristal envol vendo o ramo

sal envol vendo o sal

sal envolvendo-se

cristal único estéril mar em branco.



Da planta tiro a flor:

cor estruturada em torno da origem.

Da flor tiro as várias

vestes

as sépalas e as pétalas

--- proteções e ornamentos.

Do núcleo floral puro

retiro o androceu

e o indefeso estigma

e --- anulada a flor ---

eis-nos de volta à planta

pelo fruto.

 

não reconstrói: elide a trama e o verbo.

A Paz

não organiza: explode o núcleo-tempo.

A Paz

não é letal: vivifica.

A Paz

não apazigua: fere.

A Paz

não acalma: renova o ser e o sangue.



A mesa, todos interligados

pela realidade do alimento pelo universo único

do ser

a mesa, todos coexistem no júbilo

comungando a oferta pura das coisas.



Partilharemos somente o que em nós se continua:

a singeleza a luta

a esperança.

Partilharemos somente esta maior intensidade: absoluta palavra

que nos pertence integralmente.

Partilharemos somente o pão unificado

e a água sem face.

    (nascida morta)

No opaco silêncio estátuas virgens

de sal e luz tombaram, desmembradas, no abismo das lúcidas origens
dormem nomes e formas olvidadas.

Dormem --- não se levantam --- primitivas ideias puras no limbo
fenecidas

pulcras estátuas virgens, mas cativas, à luz total do ser não
prometidas.

Na memória elas pesam como puro tormento, arremessadas neste escuro poço
das coisas frustras, não nascidas:

assim vives em mim, irmã, singela pulsas em mim como a visão mais bela
entre rosas sepultas e queridas.

26.2.62

Alta agonia é ser, difícil prova: entre metamorfoses superar-se

e --- essência viva em pureza suprema --- despir os sortilégios, brumas,
mitos.

Alta agonia é esta raiz, pureza

de contingência extrema a abeberar-se nos mares do Ser pleno e,
arrebatada, fazer-se única em seu lúcido fruto.

Alta agonia é ser: essencial

tarefa humana e sobre-humana graça de renascermos em solidão vera

e em solidão --- dor suportada e glória --- em nossa contingência
suportarmos

o peso essencial do amor profundo.

25.10.63

Cansa-me ser. A chaga inumerável de mim cintila; sem palavras, úmida
fonte rubra do ser, anseio e tédio

de prosseguir, inabitada, viva.

Prosseguir. Ai, presença ignorada

do ser em mim, segredo e contingência, espelho, cristal raso, submerso

na eternidade do existir, tranquilo.

Cansa-me ser. Ai chaga e antigo sonho de áureas transmutações e vidas
outras além de mim, além de uma outra vida!

Mas amolda-me o ser. Prende-me a essência (raiz profunda e vera) a
imutável

condição de ser fonte e ser ferida.

23.7.64

Casa lúcida, habitada de denso vazio vivo altas janelas cerradas

na madrugada e no sonho.

Lâmpada estrita, contida, área densa, iluminada, agudas metamorfoses

de sentido amargo, estranho.

Intimidade velada ácida e intensa do ser

com a pura imagem, vera.

Contato de si consigo,

Casa lúcida. Visão

na madrugada. E no sonho.

1.11.64

Olhos vertidos no desamparo extremo o ser se exaure. Céu de desamparo.

Olhos concisos de ser. Nem mais as lágrimas mas ser exato, em nítido
cansaço

vertido, extremo e puro, no silêncio do maior desamparo se exaurindo.

Chama despida, extremo ser. Nem forma nem instante o contém. Lúcido
abismo.

Noite abstrata do desamparo extremo em que o ser se desnuda, exato e
nítido!

Que morte é mais vital, que a extrema chaga

do ser: silêncio em desamparo abrindo? Céu extremo de ser, a chama
exausta sua própria luz consome, e vai florindo.

4.2.65

Lentamente ferido de consentido sono

o pensamento é cúmplice de estrangeiro universo.

Visões sem tempo o cercam e as deformadas lâmpadas sensibilizam mundos

a uma luz mais antiga

um onírico raio de desejo incriado

que o penetra de ser

que lentamente o fere de um sono essencial entre o mistério.

20.4.65

Os domínios da luz, onde as potências se integram no equilíbrio de uma
forma os domínios da luz, parque tranquilo sobrevoado de aspiração viva.

O jardim de água lúdica, palavra, totalidade branca sobre os cimos
abstrato jardim onde tramamos nossa morte sutil, esplendor branco.

Os domínios da luz: ápice incerto além da luta, do áspero segredo

que nos habita e nos espanta. Parque

raramente fitado entre gradis

e jamais visitado entre a dolência

a intensa angústia da fronteira em que estamos.

17.3.67

Inútil a ternura pelo leve

momento a desprender-se do infinito: frágil, a construção do tempo é
morte do que se atualiza. Mais fecundo

é secundar o pássaro buscando o momento possível, voo pleno.

Mais fecundo é voar. Mas a ternura (este pássaro morto abandonado

como forma perdida de nós mesmos) nos alimenta em sua sombra. Torna-nos
em sombras sem alento. E sofremos

como pássaros frágeis: desprendidos do voo pleno nos cristalizamos
realizando a morte em que vivemos.

4.4.67

O branco é campo para o desespero é quando sem infância persistimos e
nos fita de face a luz sem pausa

da memória suspensa (tempo em branco).

O branco é luz aberta: existimos

sem sombra de segredo, sem mais causa sem mais infância em nós. É
desespero nos fixando (puro campo branco).

O branco é branco apenas. Sem refúgio insistimos na luz. A luz constrói

a flor em nós (sua rosácea branca).

O branco é campo para a crueldade onde nos encontramos: tenso espaço na
luz vivente (branco apenas, branco).

17.5.68
\end{quote}

\section{Teia {[}1996{]}}\label{teia}

\begin{quote}
\emph{Para}

\emph{Roswitha Kempf In memoriam}

\emph{A lucidez alucina}

\emph{``Todas as grandes coisas}

\emph{são difíceis e raras''}


\end{quote}

\subsection{}\label{section-9}

\begin{quote}


A teia, não mágica

mas arma, armadilha

a teia, não morta

mas sensitiva, vivente

a teia, não arte

mas trabalho, tensa

a teia, não virgem

mas intensamente

prenhe:

no centro

a aranha espera.



Falo de agrestes pássaros de sóis

que não se apagam de inamovíveis pedras

de sangue

vivo de estrelas que não cessam.

Falo do que impede o sono.



mescladas a esmo:

o fim o infinito

o mesmo

a hora e sua seta

o limite e o após a meta

o justo e o demais também

--- a beleza e seu além.



Platão

fixando as formas

Heráclito cultuando o fogo

Sócrates

fiel ao seu Daimon.



Foi de poesia lição primeira:

``a arara morreu na

aroeira''.



Gerar é escura lenta

forma in

forme

gerar é força

silenciosa firme

gerar é trabalho opaco:

só o nascimento grita.
\end{quote}







De barro o operário e a casa

(de barro o nome

e a obra).

O pássaro-operário madruga:

construir a casa construir o canto

ganhar --- construir --- o dia.

O pássaro faz o seu trabalho

e o trabalho faz o pássaro.

\begin{quote}


O duro impuro

labor: construir-se.



O canto é anterior ao pássaro

a casa é anterior ao barro

o nome é anterior à vida.
\end{quote}









 

O boi é só. O boi é só. O

boi.

Que século, meu Deus! disseram os ratos.

Perdi o bonde

(e a esperança), porém garanto

que uma flor nasceu.

Ôpa, carlos: desconfio que escrevi um poema!

\begin{quote}




Mais vale um pássaro

na mão pou

sado que o voo da ave além

do sangue.



Mais vale o canto agreste

do que o vívido silêncio branco além do humano sangue.
\end{quote}





Mais vale a luz

aberta

do que austera

noite primeva para além do sangue.

Mais vale o pássaro mais vale o sangue.

\subsection{}\label{section-10}

\begin{quote}


Sempre é melhor saber

que não saber.

Sempre é melhor sofrer

que não sofrer.

Sempre é melhor desfazer

que tecer.

Sem mão não acorda a pedra

sem língua não ascende o canto

sem olho não existe o sol.



Ter asas

é não ter cérebro.

ter cérebro é não ter asas.

 (  )



A maçã cai

e os astros dançam.



O abismo atrai o abismo: caio

em mim.



Sol inconsciente

Sol

de negro cerne

Sol

aureolado de luz.

ó

Quando pousa o pássaro

quando acorda o espelho

quando amadurece a hora.



Da vida

não se espera resposta.

 ú

--- é proibido voltar atrás e chorar.



Culpados ou cúmplices

nunca temos álibi:

por força, estamos aqui.



\ldots{} mais filosofias que coisas!
\end{quote}

\subsection{ }\label{section-11}

\begin{quote}
 

Um pássaro

seu ninho é pedra

seu grito metal cinza

dói no espaço seu olho.

Um pássaro pesa

e caça entre lixo e tédio.

Um pássaro resiste aos

céus. E perdura. Apesar.



\ldots{} fatos

são pedras duras. Não há como fugir.

Fatos são palavras ditas pelo mundo.

(\emph{Extraído de} A hora da Estrela\emph{, de Clarice Lispector.})



Nudez. O corpo

denso amargamente impuro.

Nudez. A febre

a totalidade informe.

Nudez até o cerne

o grito o lixo

o ignóbil.

Nudez até o osso

até a impossível verdade.

Comer o vinho beber o pão

nesta luz (natural?) da desrazão.



O que é impalpável mas

pesa

o que é sem rosto mas

fere

o que é invisível mas

dói.

ó

A cicatriz, talvez não indelével

o sangue agora estigma.



Ver

o avesso do sol o ventre

do caos os ossos.

Ver. Ver-se. Não dizer nada.

 

Dentes: positivos.

Presas a preendem incisivos cortam.

Dentes: decisivos.

 

O amor não vê

o amor não ouve

o amor não age

o amor não.



Não sou um deus, Graças a todos os deuses!

Sou carne viva e sal. Posso morrer.
\end{quote}

\subsection{ {[}{]}}\label{section-12}

\begin{quote}


Canta o galo e a noite

se aprofunda em plena meia noite: o galo

é negro.

Galo abissal --- galo invisível canta

e tudo o mais se cala. No vazio

só --- opaco --- per siste

o galo negro.





Os gatos secretos saltam

somem no abstrato escuro.



Gatos no negro

fluem: fosforecem

arranham vidros destroçam espectros

farejam todos os rumos.
\end{quote}



No vácuo

insone na meia-noite lúcida

cuidado: gatos agindo.

Numa hora secreta

as águas dormem

(rios detidos fontes inertes

\begin{quote}
introvertido oceano)
\end{quote}

numa hora impossível cessa o

\begin{quote}
fluxo

e eis a

estrela: amor cristalizado.



Flores

negras no negro inéditas

flores

opacas (nenhuma estrela).

Nunca

irão saber que são flores.
\end{quote}









Ultrapassar a face: negro amor consteladamente vivo.

Acolher o

vazio. Dissolver-se. Refugiar-se no abismo.

E anulado

o espelho: eis

o infinito.

\begin{quote}


Esconder (esquecer) a face

soterrar (ocultar) a luz

escurecer o amor dormir.

Aguardar o que nasce.



Senhora das feras e esferas

Senhora

do sangue

e do abismo

Senhora do grito

e da angústia

Senhora noturna e eterna

--- escuta-nos!
\end{quote}

\subsection{}\label{section-13}

\begin{quote}


Meio-dia cristal ácido

meio-dia amor sem sombra.



O círculo é astuto: enrola-se envolve-se

autofagicamente.

Depois explode

--- galáxias! ---

abre-se vivo pulsa

multiplica-se

divindadecírculo perplexa (perversa?)

o unicírculo devorando tudo.



Estrela esplendor estéril

selvática solitude

estrela inútil ímpeto energia

amor casto ab soluto

estrela estrela lúcida demência

dura estrela explosão pura.



Os metais nascem da paciência surda da terra fundem-se em

silêncio.

Os metais crescem ferozmente

(cristais vibrantes se

acasalam).

Os metais pulsam cruelmente

nunca dormem nem sonham
\end{quote}

\begin{itemize}
\item
  meditantes.
\end{itemize}

\begin{quote}
Os metais se entretecem fundamente
\end{quote}

\begin{itemize}
\item
  metais cantam no âmago
\end{itemize}

\begin{quote}
do tempo.



Arrulhos cio céus pertur

bados

asas cin zentas

cinza Afrodite ave!

(amor

cegueira exata).

 

ó cadela: libérrima

despetalada e eterna.

 Invisível teia

de vento de luz de névoa?

teia viva senha e signo

a mente une todas as coisas.



O brilho feliz

da gema

a luz concreta

do cristal: ordem

viva.



Violeta signo

violeta limiar

violeta ultra

passagem.



Um gato

e um girassol feliz.

Uma nudez sem nome.

Um imaculado vinho.



Casulo: trabalho

oculto trabalho do sono.

Seda: trabalho

borboleta futura.

Sono: trabalho

ardente trama da meta morfose.

 ()

Tudo acontece no espelho.

A fonte

deságua na própria fonte.

Leio

minha mão: livro

único.

Um deus olho ôlho no ôlho.

A vida é que nos tem: nada mais

temos.

A luz está

em nós: iluminamos.

A aventura

--- a

ventura --- fluir

sempre.

Nunca amar o que não vibra

nunca crer no que não canta.

Vemos por espelho e enigma

(mas haverá outra forma de ver?)

O espelho dissolve o tempo

o espelho aprofunda

o enigma

o espelho devora a face.
\end{quote}

\subsection{ {[}{]}}\label{section-14}

\begin{quote}


Como a túnica é uma só

os dados rolam no verde.

Como a túnica é única

são necessários os dados.

Seis faces brancas e os signos

que decidirão a posse:

um movimento, um risco

e a decisão no verde impressa.

A túnica permanecerá intacta.



Pano branco. Integralmente branco.

(Material mas suspenso

na brancura).

Branco

Que as formas nascem\ldots{} ah, tão branco

véu

para receber o sangue de todas

as coisas.



O estranho bate:

na amplitude interior não há resposta.

É o estranho (o irmão) que bate mas nunca haverá

resposta:

muito além é o país do acolhimento



Ouvir um pássaro

é agora ou nunca

é infância ou puro momento?

Ouvir um pássaro

é sempre

(dói fundo no pensamento).
\end{quote}









A beira do rio o silêncio dos peixes

a beira rio nem a espera.

A água não cessa e o rio

nunca passa.

A beira rio a lucidez

a pedra

e a pedra é

pedra: não germina. Basta-se.







  

A teoria

azul dos montes longe.

As montanhas arcaicas, ventre de um Sol perfeito

de uma infinita Lua

e os ventos de agosto, a névoa

elidindo montanhas sóis e

tudo.

Esta estrada\ldots{} Névoas

Nenhum murmúrio. Nada.

Passamos (e o Sol fenece).

Jamais haverá volta.

\begin{quote}


Os anjos são livres.

Podemos sofrer podemos viver o acontecer único
\end{quote}

\begin{itemize}
\item
  os anjos são livres ---
\end{itemize}

\begin{quote}
podemos morrer inocentemente
\end{quote}

\begin{itemize}
\item
  e os anjos são livres
\end{itemize}

\begin{quote}
até da inocência.



Uma estrela atrai

a luz

uma estrela suga

o resto do resto, o silêncio

elide os deuses, im plode

acaba morre finalíssi mamente.





Um anjo é fogo:

consome-se.

Um anjo é olhar:

introverte-se.



Um anjo é cristal:

dissolve-se.

Um anjo é luz

e se apaga.



A estrela da tarde está madura

e sem nenhum perfume.

A estrela da tarde é infecunda

e altíssima:

depois dela só há o silêncio.
\end{quote}

\section{Poemas inéditos {[}1997--1998{]}}\label{poemas-inuxe9ditos}

\begin{quote}


um vento brusco

sacudiu palmas varreu a

vida

um vento elidiu as manchas da vida.
\end{quote}



\begin{quote}
O

ovo

em silêncio trabalha espera trans muda-se

O

ovo silêncio vivo

O

ovo vibra

preparando o

voo
\end{quote}









Poema: casa ao contrário

o exato in verso

do abrigo.

Avisos. Perigos. Fugas-

Alta tensão nas

\begin{quote}
torres.

Poema: abrigo im

possível

casa jamais habitada



Sus

pense entre

o chão e o

signo

névoa o agora

e o próximo pas so in

certo

ser --- horizonte --- continua

mente em aberto.
\end{quote}







ó   ontem

porém amanhã tem circo.

Paz?

no futuro. Glória?

no passado.

Nunca há paraíso aqui e

agora

--- mas amanhã tem circo!

\begin{quote}


Resta uma sombra soçobro

a memória sem porque

resta um ovo

oco

talvez lenda

pobre nome vazio.

   raro

pássaro

protegê-lo com meu sangue

integrá-lo no meu tempo?

Ah como é livre.

Que fazer do raro

pássaro

liberá-lo no infinito

no azul friamente ingrato?

Ah como é frágil.

Frágil leve livre.

O que fazer:

soltá-lo engaiolá-lo comê-lo?

 

Um

gato tenso tocaiando o silêncio

  vive

chaga e/ou estrela

é eterno.

O aberto brilha

destrói muros amor intenso e livre.

 : 

alimenta-me mas foge

e inaugura o aberto do tempo.

  depois? o depois.

O que é certo? o mais incerto

o indefinido o aberto.



A tarde o vinho

nada esclarece

e mais tarde não há lua.

A tarde os espelhos sangram

nada se profetiza

e é certo: não haverá lua.

(A tarde

já é muito, muito tarde).



.   canta

e não entende o que canta



No canto o pássaro vive

sem compreender que canta.

  rumina

emburradíssimo burro: Burrinho burramente inocente.

. . \ldots{}

E

no entanto nosso século fez tudo

pra merecer --- demais --- o Apocalipse.



Espelho anterior aos olhos

fonte sem nenhuma imagem

água infinita da infância.

  é pássaro em voo

um pássaro vive

no voo

um pássaro vale

se voa

um pássaro voa

voa.

 

Derramar um cântaro

um canto deixar fluir o novíssimo encanto.



Equilibrar-se em vermelho.

Evitar o rosa. Despetalar o amarelo.

Transcender-se em violeta.

Colher algum azul se possível.



Por ser cego e irrefletido

meu espelho disse a verdade:

quebrei-o.

Sete anos sete anos sete anos de enganos!

   cria: cuida.

Deixa florescer o instante

e transparecer o núcleo.

 

Deus existir

ou não: o mesmo escândalo.

  vem

do abismo mais fundo

a onda vem

e se quebra: um refundo

(a onda dura

um mundo).



   incolor giros

Melhor é se é vida agasalha

galáxias e germes.

Mas só o dia vibra.



A noite é austera refúgio

Melhor útero quantas

essências. Silêncios Mas só o dia vibra.

 

pensa-se o pensamento voa.

 

--- abriga o círculo

   mantém: a eternidade é intacta.
\end{quote}

