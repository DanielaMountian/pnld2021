\chapterspecial{Vida e obra de Júlia Lopes de Almeida}{}{Rodrigo Jorge Ribeiro Neves}
\hedramarkboth{vida e obra}{}

\section{Sobre a autora}

Em um casarão na rua do Lavradio, no Centro do Rio de Janeiro, nasceu
Júlia Lopes de Almeida em 24 de setembro de 1862. Com uma produção
literária expressiva em gêneros diversos, Júlia foi romancista,
contista, cronista e dramaturga. Em 1881, aos dezenove anos, publicou
seus primeiros textos em \emph{A Gazeta de Campinas}, jornal da cidade
para onde se mudara com a família ainda na infância. Aos 22 anos, em
1884, começou a escrever para um dos principais periódicos brasileiros,
\emph{O País}, colaboração que se estendeu por mais de trinta anos. A
atividade literária e jornalística em importantes veículos da imprensa
da época exerceu influência decisiva na sua atuação intelectual e
artística.

Filha de um casal de portugueses, Valentim José da Silveira Lopes e
Adelina Pereira Lopes, Júlia mudou"-se, em 1886, para Lisboa, onde deu
início a sua carreira de escritora. No ano seguinte, com a irmã Adelina
Lopes Vieira, publicou \emph{Contos infantis}. Em 1887, casou"-se com o
também escritor Filinto de Almeida, então diretor do periódico carioca
\emph{A Semana Illustrada}, que contou com a frequente colaboração de
Júlia Lopes. Por meio de folhetins em \emph{O País}, lançou, em 1888,
seu primeiro romance, \emph{Memórias de Marta}, quando retornou ao
Brasil. Desde então, foi uma escritora prolífica e engajada, abordando
temas como a República, a escravidão e o papel da mulher nas esferas
pública e privada da sociedade, com o Rio de Janeiro como um de seus
principais cenários. Dentre seus livros, destaca"-se o romance \emph{A
falência}, de 1901, retrato contundente de um país que mudava de regime
e se modernizava, mas permanecia preso a estruturas arcaicas de
exploração e desigualdades.

Júlia Lopes de Almeida foi uma das escritoras mais importantes da virada
do século \textsc{xix} para o \textsc{xx}, sendo um dos principais nomes da \emph{Belle
Époque} carioca. Esteve entre os idealizadores da Academia Brasileira de
Letras, mas foi preterida a assumir uma das cadeiras entre os fundadores
por ser mulher, já que a maioria dos membros decidiu acompanhar a
tradição da Academia Francesa de Letras, modelo seguido pela agremiação
no Brasil, que contava apenas com homens no quadro. Seu marido, Filinto
de Almeida, ao contrário, ocupou a cadeira de número 3, embora
reconhecesse, em entrevista a João do Rio, que quem deveria estar na
Academia era Júlia, e não ele.

A escritora chegou a morar novamente em Portugal, onde publicou suas
primeiras peças teatrais, e depois na França, onde sua obra foi
traduzida e divulgada. Participou ativamente de diversas associações
femininas e discutiu temas relacionados ao Brasil e à mulher em
conferências no país e no exterior, bem como em alguns de seus livros.
Faleceu no Rio de Janeiro em 30 de maio de 1934.

Mesmo sendo uma das autoras mais importantes de seu tempo e admirada
pelos seus pares, o nome de Júlia Lopes de Almeida não resistiu aos
mecanismos de apagamento do cânone. No entanto, sua obra vem sendo
resgatada nos últimos anos por estudiosos de diversas áreas das
humanidades, com reedições de seus principais livros. Além disso, a
atualidade das questões discutidas em sua obra e a moderna sofisticação
de sua escrita são também fatores determinantes para que sua leitura
seja cada vez mais necessária.

\section{Sobre a obra}

Esta coletânea reúne algumas narrativas curtas de Júlia Lopes de
Almeida, dividida em duas seções: ``Contos'' e ``Novelas''. Os livros
dos quais foram extraídos os textos são, respectivamente, \emph{Ânsia
eterna} (1903) e \emph{A isca} (1922). Embora não sejam os únicos
volumes de narrativas curtas da escritora, foram selecionados por
apresentarem algumas das características da narrativa de Júlia Lopes e
dos temas que permeiam sua obra. Por isso, este livro não se propõe a
ser uma síntese ou um panorama da multifacetada e expressiva produção
literária da autora, mas um convite à discussão sobre questões presentes
em suas temáticas, bem como um estímulo a conhecer suas demais obras.

\emph{Ânsia eterna} foi publicado pela primeira vez, no Rio de Janeiro,
pela H. Garnier. Em 1938, foi lançada uma reedição póstuma pela editora
A Noite, com correções feitas pela autora. Uma das principais
influências desse livro, e de outros que marcam o estilo de Júlia Lopes
de Almeida, são os contos do escritor francês Guy de Maupassant
(1850-1893). Embora os textos de \emph{Ânsia eterna} fujam um pouco do
universo da obra de Júlia Lopes, ao abordar o insólito e o fantástico, a
começar pelo título do volume, eles não deixam de discutir as questões
caras à escritora, como o papel da mulher e o retrato da sociedade
escravocrata. Para esta coletânea, foram selecionados dez contos: ``O
caso de Rute'', ``A rosa branca'', ``Os porcos'', ``A caolha'',
``Incógnita'', ``A morte da velha'', ``Perfil de preta (Gilda)'', ``A
nevrose da cor'', ``As três irmãs'' e ``O futuro presidente''. Muitos
deles são dedicados a escritores e intelectuais de sua geração, como
Arthur Azevedo e Machado de Assis.

Já a edição de \emph{A isca} foi um trabalho da Livraria Leite Ribeiro,
também no Rio de Janeiro. O livro é constituído de quatro novelas, das
quais selecionamos duas para esta coletânea, ``O laço azul'' e ``O dedo
do velho''. Com o subtítulo ``novela romântica'', a primeira traz à tona
o lugar da mulher na constituição familiar, sua posição em tempos de
guerra e as dinâmicas das relações entre seus membros. E isto através da
questão do duplo, representada por duas irmãs gêmeas, um dos temas
recorrentes da prosa de ficção moderna. A segunda novela foi publicada
pela primeira vez em \emph{A Illustração Brazileira}, em 1909, com o
subtítulo ``romance''. Assim como em alguns contos de \emph{Ânsia
eterna}, ``O dedo do velho'' também se reveste do insólito no
desenvolvimento de sua história, além de apresentar alguns índices da
modernidade, nas referências ao automóvel e à urbanização.

Para esta edição, foi atualizada a grafia segundo o Novo Acordo
Ortográfico da Língua Portuguesa. Palavras como ``oiro'', ``coiro'',
``doiradas'', ``loiça'', ``óptica'' e ``cousa'', embora contempladas no
Vocabulário Ortográfico da Língua Portuguesa, foram substituídas pelas
suas formas contemporâneas do Português Brasileiro, como ``ouro'',
``couro'', ``douradas'', ``louça'', ``ótica'' e ``coisa''. A pontuação
da autora também foi conservada, salvo em casos que podem levar a
ambiguidades ou estejam em desacordo com regras sintáticas, como a
exclusão de vírgulas separando sujeito e predicado. Decidimos manter
ainda colocações pronominais, como próclises, mesóclises e ênclises,
empregadas pela autora. Expressões em língua estrangeira foram grifadas
em itálico.

\section{Sobre o gênero}