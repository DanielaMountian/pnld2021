\chapter[Apresentação, por Beatriz Kopschitz Bastos]{Apresentação}
\hedramarkboth{apresentação}{beatriz kopschitz bastos}


\textsc{Romance de estreia} do escritor irlandês James Joyce, \textit{Um retrato do artista quando jovem} 
descreve a trajetória de um dos personagens mais célebres
da literatura do século \textsc{xx}, Stephen Dedalus, em direção ao autoconhecimento,
bem como sua libertação das redes da religião, da nação e da língua. O romance
narra a infância, adolescência e juventude do personagem central, que acaba por
deixar a Irlanda, partindo para o continente em busca de perfeição artística.
Semiautobiográfico, o \textit{Retrato} lança um olhar oblíquo sobre a vida de
Joyce e representa seu amadurecimento como escritor e sua relação conflituosa
com questões centrais para a sociedade irlandesa da época, como o nacionalismo,
a religião católica e o renascimento cultural irlandês. 

Embora o \textit{Retrato} tenha sido publicado pela primeira vez em Nova York
em 1916, seguido por uma edição inglesa de 1917, sua gênese é bem anterior,
processo que levou mais de dez anos até a publicação da obra. No início de
1904, Joyce submeteu ao periódico \textit{Dana}, editado por John Eglinton e
Fred Ryan em Dublin, um ensaio autobiográfico intitulado “Um retrato do
artista”, hoje considerado o começo da maturidade de sua obra ficcional.
Apesar de recusado, o ensaio veio a ser o embrião do vasto romance
autobiográfico \textit{Stephen Herói}, escrito entre 1904 e 1907 e publicado
postumamente em 1944. Por conselho do escritor americano Ezra Pound, Joyce
abandonou esse primeiro romance e dedicou"-se a retrabalhá"-lo em versão que se
tornaria \textit{Um} \textit{retrato do artista quando jovem.} Diferentemente
do que este viria a ser, \textit{Stephen Herói} é escrito em ordem
absolutamente cronológica e corresponde, em assunto, ao último capítulo do
\textit{Retrato}. Considerado por muitos um romance ainda imaturo dentro da
obra completa de Joyce, \textit{Stephen Herói} pode ser considerado texto de
grande valor como um relato da época e da sociedade do autor, além de
constituir o germe do amadurecimento literário do romance seguinte. 

O \textit{Retrato} começou a ser publicado em episódios na prestigiada revista
literária londrina \textit{The Egoist} em 1914, mesmo ano da publicação da
coleção de contos \textit{Dublinenses.} Aproxima"-se desta em aspectos como
assunto e técnica narrativa, especialmente no caso do último conto do volume,
“Os mortos”, que já exibe parte do caráter de biografia lírica que é o
\textit{Retrato.} A publicação em livro deu"-se em 1916, ironicamente o
ano do Levante de Páscoa, um dos momentos de maior apelo emocional na história
da Irlanda, extremamente significativo na iconografia do nacionalismo
revolucionário irlandês que Joyce rejeitava.

\textit{Um retrato do artista quando jovem} é amplamente considerado um romance
de formação, ou, em alemão, \textit{Bildungsroman}, tipo de romance que expõe o
processo de desenvolvimento físico, moral, psicológico e social de um
personagem, passando pelas várias fases de sua vida: infância, adolescência,
idade adulta e maturidade. O gênero, que se espalhou na Europa, surgiu na
Alemanha, possivelmente com \textit{Os sofrimentos do jovem Werther}, de
Goethe, que Joyce conhecia bem. Esse tipo de romance é, de maneira geral,
psicológico, mais dedicado aos pensamentos e sentimentos do protagonista,
normalmente de sexo masculino, do que ao que acontece ao seu redor. Isso acaba
tornando o personagem um tanto solitário, como o é Stephen, que pode ser visto
como um típico herói de \textit{Bildungsroman}: sensível, inteligente, sempre
pronto a procurar respostas para seus questionamentos e a viver novas
experiências.  

O nome de Stephen Dedalus, o personagem central, é uma referência ao
mito grego de Dédalo, obreiro astucioso, artífice lendário de Atenas,
cujas estátuas podiam mover"-se. Receoso de que seu sobrinho, Talos,
pudesse ultrapassá"-lo em engenhosidade, Dédalo decidiu lançá"-lo do alto
da Acrópole. Condenado, fugiu para Creta, onde construiu o Labirinto
para o rei Minos. Após ajudar na fuga de Teseu, Dédalo acabou confinado
no próprio Labirinto, do qual tentou escapar construindo asas de cera
para si mesmo e para seu filho, Ícaro. Ícaro voou muito alto e suas
asas foram derretidas pelo Sol, levando à sua queda. O romance de Joyce
narra o confinamento de Stephen e, de forma oblíqua, do próprio Joyce,
nas redes de sua vida; seu recurso final, assim como o de Dédalo para
livrar"-se do confinamento do Labirinto, é, literalmente, fugir, deixar
o país, e, metaforicamente, voar rumo à liberdade, rumo ao
desconhecido, com as asas da arte. O caminho para as artes
desconhecidas já é apontado na epígrafe do livro, extraída das
\textit{Metamorfoses}, de Ovídio: “\textit{Et ignotas animum dimittit
in artes}” (“E ele aplica a mente às artes ocultas”). E na última linha
do romance, quando está prestes a partir, e assim assumindo o papel de
filho espiritual, Stephen evoca a proteção de Dédalo: “Velho pai, velho
artífice, ampara"-me agora e sempre em sã jornada”. 

A autodescoberta de Stephen e de sua missão como artista desenrola"-se
paralelamente à evolução da própria escrita do livro. Joyce havia lido
são Tomás de Aquino e \textit{De Anima}, de
Aristóteles, e aplicou os princípios adquiridos com as leituras na
descoberta dos estilos que compõem o \textit{Retrato}. Cada capítulo é
escrito em estilo que corresponde ao estágio da vida em que o
personagem se encontra: a escrita imita sua psicologia interior. Da
infância, passando pela adolescência, até a juventude, enquanto Stephen
toma consciência de sua missão como artista, a prosa do livro progride
e evolui em estilo. A força do \textit{Retrato} está exatamente na
riqueza da variedade de estilos e na densidade da organização
simbólica, em um texto que combina a sequência de episódios conectados
cronologicamente com a intervenção de momentos de epifania e clímax
triunfal. Enquanto o primeiro capítulo é escrito praticamente em estilo
realista, com algum uso de monólogo interior, a partir do segundo 
intensificam"-se as reações interiores do protagonista aos
acontecimentos internos; nos capítulos subsequentes predomina cada vez
mais a técnica do fluxo de consciência, que antecipa a prosa
característica do próximo romance de Joyce, sua obra"-prima modernista:
\textit{Ulisses}. A paixão pelo pensamento de Stephen no
\textit{Retrato} dá lugar à centralidade da própria atividade do pensar
em \textit{Ulisses}, publicado em 1922.  

A obra ficcional de Joyce a partir do \textit{Retrato} até
\textit{Finnegans Wake}, publicado em 1939, é um dos pilares do
Modernismo na literatura ocidental. O \textit{Retrato}, entretanto, foi
algumas vezes considerado antimoderno, uma vez que o próprio gênero
que o constituiu, o \textit{Bildungsroman}, traz em si características
de um tradicionalismo literário oposto às rupturas propostas pelos
modernistas. O argumento contrário, entretanto, também procede. Se
olharmos somente para a literatura irlandesa e deixarmos um pouco de
lado o Modernismo europeu, encontraremos no \textit{Retrato} uma
síntese e, ao mesmo tempo, uma transgressão de movimentos anteriores,
como o Realismo, o Decandentismo, o Esteticismo e o Simbolismo. Isso o
torna uma obra de ruptura em relação a romances autobiográficos e
autobiografias irlandesas do século \textsc{xix}, como \textit{O retrato de
Dorian Gray}, de Oscar Wilde, e \textit{Confessions of A Young Man}, de
George Moore: 

\begin{quote}
Do ponto de vista temático, Joyce reescreveu a tradição;
do ponto de vista da maneira de representação da realidade na arte, foi
muito além de suas raízes --- fez a síntese dos grandes movimentos
literários que conviveram e batalharam por seus princípios estéticos no
século \textsc{xix}.\footnote{ Mutran, p.~234.}
\end{quote}


\section{O Retrato de Joyce}

Se o \textit{Retrato} pode ser visto como um romance semiautobiográfico
e Stephen Dedalus como o alter ego de James Joyce, como sugerem vários
críticos, alguns dados biográficos a respeito do autor, principalmente
sobre o período que corresponde à sua infância e juventude no romance,
tornam"-se também relevantes para o leitor que deseja acompanhar esse
aspecto do livro.  

James Augustine Aloysius Joyce, o primogênito de uma família de dez
filhos, nasceu em Rathgar, um bairro respeitável de Dublin, em 2 de
fevereiro de 1882. Seu pai, John Stanislaus Joyce, pertencia à classe
média católica descendente dos Joyces de Galway e herdou propriedades e
bens em Cork, patrimônio que dilapidou ao longo da vida, assim como
Simon Dedalus no romance. Dissipando sua riqueza, submeteu a família a
uma longa série de mudanças de residência. A cada mudança, a
localização das quatorze casas alugadas ia decrescendo em prestígio. O
senso de humor ácido de seu pai, seu amor pela música, o álcool e um
certo anticlericalismo viriam a ser influências permanentes em James
Joyce. Sua mãe, Mary Jane Murray, dez anos mais jovem que o marido,
católica devota e natural de Longford, com certeza também foi
inspiração direta para a recriação da mãe de Stephen no romance.

Joyce ingressou na escola Jesuíta Conglowes Wood, um internato no
Condado de Kildare, em 1888; depois passou a frequentar a escola dos
Christian Brothers, em Dublin, quando seu pai não podia mais pagar a
mensalidade de Clongowes. Em 1893, foi aceito no Belvedere College em
Dublin, também Jesuíta, com uma bolsa de estudos. Nessa época, Joyce
foi encorajado a juntar"-se à Ordem de Jesus, mas rejeitou a proposta e
voltou"-se cada vez mais para a literatura como uma alternativa à
religião. Em 1895, Joyce começou a cursar a University College, em
Dublin, também com uma bolsa de estudos, e dedicou"-se ao estudo de
línguas, além das matérias obrigatórias como Matemática e Filosofia.
Sua formação, entretanto, foi também em parte a de um autodidata,
leitor assíduo nas bibliotecas de Dublin. Como estudante, Joyce gostava
de se ver e ser visto como artista e poeta. 

Após formar"-se, James Joyce conheceu os líderes do Renascimento
irlandês, movimento de resgate da herança cultural irlandesa da virada
do século e início do século \textsc{xx}, mas, desde então, passou a nutrir um
verdadeiro antagonismo aos ideais do movimento, embora tivesse
admiração por W.B.~Yeats. Também na área da política, Joyce rejeitava
a ideologia nacionalista católica vigente, desenvolvendo uma antipatia
particular por Patrick Pearse, poeta e um dos líderes do Levante de
Páscoa de 1916.  Tudo isso gerou o isolamento de James Joyce em relação
àqueles que promoviam os ideais predominantes nos meios culturais e
políticos da Irlanda de seu tempo.  

Necessidades financeiras fizeram com que Joyce, a exemplo de Oliver 
St.~John Gogarty, procurasse o curso de Medicina, primeiramente na Irlanda
e, em 1902, em Paris, quando Joyce deixa Dublin pela primeira vez. Com
dificuldades para ingressar nas universidades francesas e saudoso,
Joyce retorna à Irlanda para o Natal. Já de volta a Paris, Joyce
retorna com a notícia da morte de sua mãe, vítima do câncer, em 1903. 

O ano de 1904 foi decisivo na vida e na carreira de James Joyce: 
publicou “Um retrato do artista”, ensaio escrito em um
dia e que deu origem a \textit{Stephen Herói}, e os primeiros
contos, no \textit{The Irish Homestead}, que mais tarde seriam reunidos em \textit{Dublinenses}.
Depois de empreitadas literárias preliminares, Joyce inicia então o
amadurecimento de sua carreira. Em junho desse mesmo ano, conhece Nora
Barnacle, de Galway, sua companheira, esposa e inspiração por toda a
vida. Em outubro, o casal deixa Dublin rumo a Paris, em uma atitude de
exílio voluntário. Posteriormente Joyce e Nora mudam"-se para Pula, hoje
na Croácia, e fixam residência em Trieste, hoje na Itália, com algumas
incursões à Irlanda.  

A vida em Trieste, onde Joyce trabalhava como professor de inglês, foi
marcada pelo nascimento de seus filhos, Giorgio, em 1905, e Lucia, em
1907, e por dificuldades financeiras, para as quais os Joyce e Nora
procuraram alternativas diversas: a ajuda do irmão de James,
Stanislaus, um emprego temporário em um banco em Roma e doações de
amigos e benfeitores, como Harriet Shaw Weaver, editora do \textit{The
Egoist}, por exemplo. A renda insuficiente de Joyce e seu alcoolismo
fizeram com que a família passasse por inúmeros endereços em Trieste,
até que, devido à Primeira Guerra Mundial, não podendo permanecer no
então Império Austro"-Húngaro por serem britânicos, mudaram"-se para
Zurique, onde residiram até 1919. 

Em 1919, Joyce retorna a Trieste, mas no ano seguinte, muda"-se, a
conselho de Ezra Pound, para Paris, onde reside até o início da Segunda
Guerra Mundial. Todas as grandes obras de Joyce foram publicadas nesse
período, incluindo episódios dos romances em periódicos, que saíram
antes dos livros propriamente ditos, em meio a inúmeras batalhas com
editores e censores. Embora o autor tenha morado fora da Irlanda a
maior parte de sua vida adulta, sua terra natal e sua relação com a
mesma forneceram o principal material literário para sua obra. 

Com o advento da guerra, a família mudou"-se para uma pequena cidade
próxima a Vichy, em seguida retornando à Suíça, com vistos especiais.
Acometido de sérios problemas de vista e tomado por fortes dores
estomacais, Joyce morreu em 13 de janeiro de 1941, após ser
hospitalizado e submetido a uma cirurgia no aparelho digestivo. Seu
legado para a literatura universal tem sido material de pesquisa e
estudo em universidades em todo o mundo e é celebrado anualmente no dia
16 de junho, o \textit{Bloomsday}, dia em que Joyce passeou por Dublin
pela primeira vez com Nora e dia que marca a perambulação de Leopold
Bloom pela mesma Dublin em \textit{Ulisses}.


\begin{bibliohedra}

\tit{CRONIN}, John. \textit{Irish Fiction --- 1900-1940}. Belfast: Appletree
Press, 1992.

\tit{DEANE}, Seamus. \textit{Celtic Revivals}. London: Faber \& Faber, 1985.

\tit{ELMANN}, Richard. James Joyce. Trad. Lya Luft. Rio de Janeiro: Editora
Globo, 1989.

\tit{FOSTER}, John Wilson (ed.). \textit{The Cambridge Companion to The Irish
Novel}.  Cambridge: Cambridge University Press, 2006.

\tit{HARVEY}, Paul. Trad. Mário da Gama Kury. \textit{Dicionário Oxford de
\mbox{Literatura} Clássica}. Rio de Janeiro: Jorge Zahar Editor, 1987.

\tit{MUTRAN}, Munira H. \textit{Álbum de Retratos}. São Paulo:
Humanitas/\textsc{fapesp}, 2002.

\tit{THORNTON}, Weldon. \textit{The Antimodernism of Joyce’s}
\textit{Portrait of The Artist as A Young Man.} Syracuse,
N.Y.: Syracuse University Press, 1994.

\end{bibliohedra}

