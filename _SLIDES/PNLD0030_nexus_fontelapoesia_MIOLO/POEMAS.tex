\part{\textsc{transposição} {[}1966--1967{]}}

\chapter*{}
\thispagestyle{empty}
\mbox{}
\vfill
\hfill\emph{A um passo de meu próprio espírito}

\hfill\emph{A um passo impossível de Deus.}

\hfill\emph{Atenta ao real: aqui.}

\hfill\emph{Aqui aconteço.}

\part*{\textsc{i}\\\textsc{base}}

\chapter{Transposição}

\begin{verse}
Na manhã que desperta\\
o jardim não mais geometria\\
é gradação de luz e aguda\\
descontinuidade de planos.

Tudo se recria e o instante\\
varia de ângulo e face\\
segundo a mesma vidaluz\\
que instaura jardins na amplitude

que desperta as flores em várias\\
coresinstantes e as revive\\
jogando-as lucidamente\\
em transposição contínua.
\end{verse}

\chapter{Tempo}

\begin{verse}
O fluxo obriga\\
qualquer flor\\
a abrigar-se em si mesma\\
sem memória.

O fluxo onda ser\\
impede qualquer flor\\
de reinventar-se em\\
flor repetida.

O fluxo destrona\\
qualquer flor\\
de seu agora vivo\\
e a torna em sono.

O universofluxo\\
repele\\
entre as flores estes\\
cantosfloresvidas.

--- Mas eis que a palavra\\
cantoflorvivência\\
re-nascendo perpétua\\
obriga o fluxo

cavalga o fluxo num milagre\\
de vida.
\end{verse}

\chapter{Pedra}

\begin{verse}
A pedra é transparente:\\
o silêncio se vê\\
em sua densidade.

(Clara textura e verbo\\
definitivo e íntegro\\
a pedra silencia).

O verbo é transparente:\\
o silêncio o contém\\
em pura eternidade.
\end{verse}

\chapter{Meada}

\begin{verse}
Uma trança desfaz-se:\\
calmamente as mãos\\
soltam os fios\\
inutilizam\\
o amorosamente tramado.

Uma trança desfaz-se:\\
as mãos buscam o fundo\\
da rede inesgotável\\
anulando a trama\\
e a forma.

Uma trança desfaz-se:\\
as mãos buscam o fim\\
do tempo e o início\\
de si mesmas, antes\\
da trama criada.

As mãos\\
destroem, procurando-se\\
antes da trança e da memória.
\end{verse}

\chapter{Ludismo}

\begin{verse}
Quebrar o brinquedo\\
é mais divertido.

As peças são outros jogos\\
construiremos outro segredo.\\
Os cacos são outros reais\\
antes ocultos pela forma\\
e o jogo estraçalhado\\
se multiplica ao infinito\\
e é mais real que a integridade: mais lúcido.

Mundos frágeis adquiridos\\
no despedaçamento de um só.\\
E o saber do real múltiplo\\
e o sabor dos reais possíveis\\
e o livre jogo instituído\\
contra a limitação das coisas\\
contra a forma anterior do espelho.

E a vertigem das novas formas\\
multiplicando a consciência\\
e a consciência que se cria\\
em jogos múltiplos e lúcidos\\
até gerar-se totalmente:\\
no exercício do jogo\\
esgotando os níveis do ser.

Quebrar o brinquedo ainda\\
é mais brincar.
\end{verse}

\chapter{Mãos}

\begin{verse}
Com as mãos nuas\\
lavrar o campo:

as mãos se ferindo\\
nos seres, arestas\\
da subjacente unidade

as mãos desenterrando\\
luzesfragmentos\\
do anterior espelho

Com as mãos nuas\\
lavrar o campo:

desnudar a estrela essencial\\
sem ter piedade do sangue.
\end{verse}

\chapter{Núcleo}

\begin{verse}
Aprender a ser terra\\
e, mais que terra, pedra\\
nuclear diamante\\
cristalizando a palavra.

A palavra definitiva.\\
A palavra áspera e não plástica.
\end{verse}


\chapter{Série}

\begin{verse}
Primeiro\\
o apelo\\
(paralela a palavra\\
ao universo).

Depois\\
invocadas potências\\
formas se tramam puro\\
mapa lúdico.

Enfim\\
conclusão do ato\\
o amor ser possível\\
amanhece\\
lúcido.
\end{verse}

\part*{\textsc{ii}\\ (---)}

\chapter{Fala}

\begin{verse}
Tudo\\
será difícil de dizer:\\
a palavra real\\
nunca é suave.

Tudo será duro:\\
luz impiedosa\\
excessiva vivência\\
consciência demais do ser.

Tudo será\\
capaz de ferir. Será\\
agressivamente real.\\
Tão real que nos despedaça.

Não há piedade nos signos\\
e nem no amor: o ser\\
é excessivamente lúcido\\
e a palavra é densa e nos fere.

(Toda palavra é crueldade.)
\end{verse}

\chapter{Meio-dia}

\begin{verse}
Ao meio-dia a vida\\
é impossível.

A luz destrói os segredos:\\
a luz é crua contra os olhos\\
ácida para o espírito.

A luz é demais para os homens.\\
(Porém como o saberias\\
quando vieste à luz\\
de ti mesmo?)

Meio-dia! Meio-dia!\\
A vida é lúcida e impossível.
\end{verse}

\chapter{Revelação}

\begin{verse}
A porta está aberta\\
como se hoje fosse infância\\
e as coisas não guardassem pensamentos\\ formas de nós nelas inscritas.

A porta está aberta. Que sentido\\
tem o que é original e puro?\\
Para além do que é humano o ser se integra\\
e a porta fica aberta. Inutilmente.
\end{verse}

\chapter{Ode \textsc{i}}

\begin{verse}
O real? A palavra\\
coisa humana\\
humanidade\\
penetrou no universo e eis que me entrega\\ tão-somente uma rosa.
\end{verse}

\chapter{Destruição}

\begin{verse}
A coisa contra a coisa:\\
a inútil crueldade\\
da análise. O cruel\\
saber que despedaça\\
o ser sabido.

A vida contra a coisa:\\
a violentação\\
da forma, recriando-a\\
em sínteses humanas\\
sábias e inúteis.

A vida contra a vida:\\
a estéril crueldade\\
da luz que se consome\\
desintegrando a essência\\
inutilmente.
\end{verse}

\chapter{Notícia}

\begin{verse}
Não mais sabemos do barco\\
mas há sempre um náufrago:\\
um que sobrevive\\
ao barco e a si mesmo\\
para talhar na rocha\\
a solidão.
\end{verse}


\part*{\textsc{iii}\\(+)}

\chapter{Ode \textsc{ii}}

\begin{verse}
O amor, imor\\
talidade do instante\\
totalização da forma\\
em ato vivo: obscura\\
força refazendo o ser.

O amor, momen\\
to do ser refletido\\
eternamente pelo espírito.
\end{verse}

\chapter{Lavra}

\begin{verse}
A semente em seu sulco\\
e o tempo vivo.

A semente em seu sulco\\
e a vida rítmica fluindo\\
para a realização do fruto.
\end{verse}

\chapter{Voo}

\begin{verse}
Flecha ato não verbo\\
impulso puro\\
corta o instante\\
e faz-se a vida\\
em acontecer tão frágil

lucidez breve\\
do movimento\\
acontecido.
\end{verse}

\chapter{Girassol}

\begin{verse}
Quero expressar a flor\\
e o girassol me escolhe:\\
helianto bizâncio ouro luz\\
\qquad\qquad\qquad\quad ouro ouro

Variando de horizonte\\
porém sempre\\
audazmente fiel\\
fitando a luz intensamente

o girassol me escolhe:\\
adoração dourada\\
fixação tranquila\\
calor lúcido.

Flor para sempre e muito mais\\
que flor.
\end{verse}

\chapter{Sensação}

\begin{verse}
Vejo cantar o pássaro\\
toco este canto com meus nervos\\
seu gosto de mel. Sua forma\\
gerando-se da ave\\
como aroma.

Vejo cantar o pássaro e através\\
da percepção mais densa\\
ouço abrir-se a distância\\
como rosa\\
em silêncio.
\end{verse}

\part*{\textsc{iv}\\\textsc{fim}}


\chapter{Sede}

\forceindent\textsc{i}

\begin{verse}
Beber a hora\\
beber a água\\
embriagar-se\\
com água apenas.
\end{verse}

\medskip
\textsc{ii}

\begin{verse}
Água? É só isso\\
que purifica.
\end{verse}

\medskip
\textsc{iii}

\begin{verse}
Fonte maior\\
e não oculta\\
fonte sem Narciso\\
nem flores.
\end{verse}

\medskip
\textsc{iv}

\begin{verse}
Bendita a sede\\
por arrancar nossos olhos\\
da pedra.

Bendita a sede\\
por ensinar-nos a pureza\\
da água.

Bendita a sede\\
por congregar-nos em torno\\
da fonte.
\end{verse}

\chapter{Fluxo}

\begin{verse}
A gênese das águas\\
é secreta e infinita\\
entre as pedras se esconde\\
de toda contemplação.\\
A gênese das águas\\
e em si mesma.

\dotfill

O movimento das águas\\
é caminho inconsciente\\
mutação contínua\\
nunca terminada.

É caminho vital\\
de si mesma.

\dotfill

O fim das águas\\
é dissolução e espelho\\
morte de todo o ritmo\\
em contemplação viva.

Consciencialização\\
de si mesma.
\end{verse}

\chapter{O nome}

\begin{verse}
A escolha do nome: eis tudo.

O nome circunscreve\\
o novo homem: o mesmo,\\
repetição do humano\\
no ser não nomeado.

O homem em branco, virgem\\
da palavra\\
é ser acontecido:\\
sua existência nua\\
pede o nome.

Nome\\
branco sagrado que não\\
define, porém aponta:\\
que o aproxima de nós\\
marcado do verbo humano.

A escolha do nome: eis\\
o segredo.
\end{verse}

\chapter{O equilibrista}

\begin{verse}
Essencialmente equilíbrio:\\
nem máximo nem mínimo.

Caminho determinado\\
movimentos precisos sempre\\
medo controlado máscara\\
de serenidade difícil.

Atenção dirigida olhar reto\\
pés sobre o fio sobre a lâmina\\
ser numa só ideia nítida\\
equilíbrio. Equilíbrio.

Acaba a prova? Só quando\\
o trapézio oferece o voo\\
e a queda possível desafia\\
a precisão do corpo todo.

Acaba a prova se a aventura\\
inda mais aguda se mostra\\
mortal intensa desumana\\
desequilíbrio essencialmente.
\end{verse}

\chapter{A estátua jacente}

\forceindent\textsc{i}

\begin{verse}
Contido\\
em seu livre abandono\\
um dinamismo se alimenta\\
de sua contenção pura.

Jacente\\
uma atmosfera cerca\\
de tal força o silêncio

como se jacente guardasse\\
o gesto total do segredo.
\end{verse}

\medskip
\textsc{ii}

\begin{verse}
O jacente\\
é mais que um morto: habita\\
tempos não sabidos\\
de mortos e vivos.

O jacente\\
ressuscitado para o silêncio\\
possui-se no ser\\
e nos habita.
\end{verse}

\medskip
\textsc{iii}

\begin{verse}
Vemos somente o repouso\\
como uma face neutra\\
além de tudo o que\\
significa.

(Mas se nos víssemos\\
no verbo totalizado\\
--- forma que se concentra\\
além de nós ---

(Mas se nos víssemos\\
na contenção do ser\\
o repouso seria\\
expressão nítida.)

Vemos apenas\\
repouso:\\
contenção da palavra\\
no silêncio.
\end{verse}

\medskip
\textsc{iv}

\begin{verse}
Jaz\\
sobre o real o gesto\\
inútil: esta palma.

A palavra vencida\\
e para sempre inesgotável.
\end{verse}

\part{\textsc{helianto} {[}1973{]}}

\mbox{}
\vfill
\thispagestyle{empty}

\hfill\emph{A}

\hfill\emph{Antonio Candido}

\hfill\emph{com amizade e reconhecimento}

\pagebreak
\vfill
\thispagestyle{empty}

\hfill\emph{Menina, minha menina}

\hfill\emph{Faz favor de entrar na roda} 

\hfill\emph{Cante um verso bem bonito}

\hfill\emph{Diga adeus e vá-se embora}

\hfill\textsc{cantiga de roda}

\chapter{Helianto}

\begin{verse}
Cânon\\
da flor completa\\
metro / valência / rito\\
da flor\\
verbo

círculo\\
exemplar de helianto\\
flor e\\
mito

ciclo\\
do complexo espelho\\
flor e\\
ritmo

cânon\\
da luz perfeita\\
capturada fixa\\
na flor\\
verbo.
\end{verse}

\chapter{Alvo}

\begin{verse}
Miro e disparo:\\
o alvo\\
o al\\
o a

centro exato dos círculos\\
concêntricos\\
branco do a\\
a branco\\
\quad{}ponto\\
\quad{}branco\\
atraindo todo o impacto

(Fixar o voo\\
da luz na\\
\quad{}forma\\
firmar o canto\\
em preciso\\
silêncio

--- confirmá-lo no centro\\
\quad\quad\quad{}do silêncio.)

Miro e disparo:\\
o a\\
o al\\
o alvo.
\end{verse}

\chapter{Sob a língua}

\begin{verse}
Sob a língua

palavras beijos alimentos\\
alimentos beijos palavras.

O saber que a boca prova\\
O sabor mortal da palavra.
\end{verse}

\chapter{Herança}

\begin{verse}
O que o tempo descura\\
e que transfixa

o que o tempo transmite\\
e subverte

o que o tempo desmente\\
e mitifica.
\end{verse}

\chapter{Escultura}

\begin{verse}
O aço não desgasta\\
seus espelhos múltiplos\\
curvas\\
arestas\\
apocalíptica fera.

O aço não se entrega\\
e nem se estraga é\\
\hfill{}forma\\
--- presença imposta sem signos.

O aço ameaça\\
--- imóvel ---\\
com a aspereza total\\
de seu frio.

Ó forma\\
violenta pura\\
como emprestar-te algo\\
\hfill{}humano

uma vivência\\
um nome?
\end{verse}

\chapter{Caleidoscópio}

\begin{verse}
Acontece: um\\
\quad{}giro\\
\quad{}e a forma brilha.

Espelhos do instante\\
\quad{}filtram\\
a ordem pura cores forma\\
\quad{}brilho

(e sem nenhuma\\
\quad{}palavra).

Acontece: outro\\
\quad{}giro\\
\quad{}outra forma e o mesmo\\
\hfill{}brilho.

Ó espelho dos instantes\\
\hfill{}fragmentos\\
estruturados em reflexos\\
\hfill{}fúlgidos!

Acontece: novo\\
\hfill{}giro\ldots{}\\
O caleidoscópio quebra-se.
\end{verse}

\chapter{Sol}

\begin{verse}
Sol.\\
Sol maiormente. Alucinado.

Sol\\
trespassante: há aberturas\\
\quad\quad\quad{}no sangue\\
\quad\quad\quad{}há janelas de vidro\\
\quad\quad\quad{}na mente.

Que mito subsiste\\
---  que infância ---\\
sob este Sol que ternura\\
\quad\quad{}nos resta?

Só o mito maior\\
deste Sol\\
puro.

Sol\\
sem nenhuma sombra\\
\quad\quad{}possível.
\end{verse}

\chapter{As sereias}

\begin{verse}
Atraídas e traídas\\
atraímos e traímos

Nossa tarefa: fecundar\\
\hfill atraindo\\
nossa tarefa: ultrapassar\\
\hfill traindo\\
o acontecer puro\\
que nos vive.

Nosso crime: a palavra.\\
Nossa função: seduzir mundos.

Deixando a água original\\
cantamos\\
sufocando o espelho\\
do silêncio.
\end{verse}


\chapter{Sete poemas do pássaro}

\forceindent\textsc{i}

\begin{verse}
O pássaro é definitivo\\
por isso não o procuremos:\\
ele nos elegerá.
\end{verse}

\medskip
\textsc{ii}

\begin{verse}
Se for esta a hora do pássaro\\
abre-te e saberás\\
o instante eterno.
\end{verse}

\medskip
\textsc{iii}

\begin{verse}
Nunca será mais a mesma\\
nossa atmosfera\\
pois sustentamos o voo\\
que nos sustenta.
\end{verse}

\medskip
\textsc{iv}

\begin{verse}
O pássaro é lúcido\\
e nos retalha.\\
Sangramos. Nunca haverá\\
cicatrização possível\\
para este rumo.
\end{verse}

\medskip
\textsc{v}

\begin{verse}
Este pássaro é reto;\\
arquiteta o real e é o real mesmo.
\end{verse}

\medskip
\textsc{vi}

\begin{verse}
Nunca saberemos\\
tanta pureza:\\
pássaro devorando-nos\\
enquanto o cantamos.
\end{verse}

\medskip
\textsc{vii}

\begin{verse}
Na luz do voo profundo\\
existiremos neste pássaro:\\
ele nos vive.
\end{verse}

\chapter{Estrelas}

\begin{verse}
Fixar estrelas\\
no mapa móvel\\
zodíaco.

Jogar com astros\\
e fixar-se\\
no próprio jogo.

Nomear constelações\\
---  submeter os astros\\
à palavra.

Buscar estrelas. Viver estrelas\\
\hfill --- animal siderado\\
\hfill e siderante.
\end{verse}

\chapter{São Sebastião}

\begin{verse}
As setas\\
--- cruas --- no corpo

as setas\\
no fresco sangue

as setas\\
na nudez jovem

as setas\\
--- firmes --- confirmando\\
\hfill a carne.
\end{verse}


\chapter{Stop}

\begin{verse}
Estado de sítio\\
estado de sido\\
estase.
\end{verse}

\chapter{Oposição}

\begin{verse}
Na oposição se completam\\
os arcanjos contrários\\
sendo a mesma existência\\
em dois sentidos.

(Um, severo e nítido\\
na negação pura\\
de seu ser. O outro\\
em adoração firmado.)

Não se contemplam e se sabem\\
um mesmo enigma cindido\\
combatem-se, mas abraçando-se\\
na unidade da essência.

Interfecundam-se no mesmo\\
bloco de ser e de silêncio\\
coluna viva em que a memória\\
cindiu-se em dois horizontes.

(Sim e não no mesmo\\
abismo do espírito.)
\end{verse}

\chapter{Eros}

\begin{verse}
Cego?\\
Não: livre.\\
Tão livre que não te importa\\
a direção da seta.

Alado? Irradiante.\\
Feridas multiplicadas\\
nascidas de um só

\hfill abismo.

Disseminas pólens e aromas.\\
És talvez a

\hfill primavera?

Supremamente livre\\
\quad\quad\quad --- violento ---\\
não és estátua: és pureza\\
\hfill oferta.

Que forma te conteria?\\
Tuas setas armam\\
\hfill o mundo\\
enquanto --- aberto --- és abismo\\
\hfill inflamadamente vivo.
\end{verse}

\chapter{Repouso}

\begin{verse}
Basta o profundo ser\\
em que a rosa descansa.

Inúteis o perfume\\
e a cor: apenas signos\\
de uma presença oculta\\
inútil mesmo a forma\\
claro espelho da essência

inútil mesmo a rosa.

Basta o ser. O escuro\\
mistério vivo, poço\\
em que a lâmpada é pura\\
e humilde o esplendor\\
das mais cálidas flores.

Na rosa basta o ser:\\
nele tudo descansa.
\end{verse}


\chapter{Elegia (\textsc{i})}

\begin{verse}
Mas para que serve o pássaro?\\
Nós o contemplamos inerte.\\
Nós o tocamos no mágico fulgor das penas.\\
De que serve o pássaro se\\
desnaturado o possuímos?

O que era voo e eis\\
que é concreção letal e cor\\
paralisada, íris silente, nítido,\\
o que era infinito e eis\\
que é peso e forma, verbo fixado, lúdico

o que era pássaro e é\\
o objeto: jogo\\
de uma inocência que\\
o contempla e revive\\
--- criança que tateia\\
no pássaro um esquema\\
de distâncias ---

mas para que serve o pássaro?

O pássaro não serve. Arrítmicas\\
brandas asas repousam.
\end{verse}

\chapter{Elegia (\textsc{ii})}

\begin{verse}
Os extremos do vento\\
sons\\
partidos.

Os extremos os\\
mais\\
agudos cumes\\
da tensão viva amor\\
--- criação viva ---

agora par\\
\qquad tidos\\
luz e lira\\
inertes.

Os extremos do amor:\\
áridos\\
restos.
\end{verse}

\chapter{Estrada}

\begin{verse}
A estrada percorre\\
\qquad\qquad o bosque\\
entre árvores mudas\\
entre pedras opacas\\
entre jogos de luz\\
\qquad\qquad e sombra.

A estrada caminha\\
e o seu solo\\
(ancestralmente fundo)\\
não tem som.

A estrada prossegue\\
e seu silêncio\\
fixa presenças densas\\
e embriaga\\
sufocando toda a\\
\qquad\qquad memória\ldots{}
\end{verse}


\part{\textsc{alba} {[}1983{]}}

\thispagestyle{empty}

\mbox{}
\vfill
Para

Davi

Haquira

Lucia

Ana Maria

\pagebreak
\thispagestyle{empty}

\mbox{}
\vfill
\setlength{\epigraphwidth}{.3\textwidth}
\epigraph{\emph{Que bien sé yo la fonte}\\
\emph{que mana y corre,}\\
\emph{aunque es de noche.}}{\textsc{san juan de la cruz}}

\pagebreak
\thispagestyle{empty}

\mbox{}\vfill
\begin{verse}
A um passo\\
do pássaro\\
res\\
piro.
\end{verse}

\chapter{Alba}

\forceindent\textsc{i}

\begin{verse}
Entra furtivamente\\
a luz\\
surpreende o sonho inda imerso\\
\qquad\qquad\qquad\qquad\qquad na carne.
\end{verse}

\medskip
\textsc{ii}

\begin{verse}
Abrir os olhos.\\
Abri-los\\
como da primeira vez\\
--- e a primeira vez\\
\quad é sempre.
\end{verse}

\medskip
\textsc{iii}\\

\begin{verse}
Toque\\
de um raio breve\\
e a violência das imagens\\
no tempo.
\end{verse}

\medskip
\textsc{iv}

\begin{verse}
Branco\\
sinal oferto\\
e a resposta do\\
sangue:\\
\textsc{agora!}
\end{verse}


\chapter{Poema}

\begin{verse}
Saber de cor o silêncio\\
diamante e/ou espelho\\
o silêncio além\\
do branco.

Saber seu peso\\
seu signo\\
--- habitar sua estrela\\
\quad impiedosa.

Saber seu centro: vazio\\
esplendor além\\
da vida\\
e vida além\\
da memória.

Saber de cor o silêncio

--- e profaná-lo, dissolvê-lo\\
\qquad\qquad\qquad em palavras.
\end{verse}

\chapter{Trovões}

\begin{verse}
Trovões invadem\\
casas\\
coisas\\
quebram\\
louças gráficos\\
\qquad\qquad\qquad vidros.

Anulam o supérfluo: articulam\\
um campo para o destino.

Trovões transportam raízes\\
a altas distâncias nuas\\
tentando armar uma flor\\
com o que resta --- ainda ---\\
do silêncio.
\end{verse}

\chapter{Prometeu}

\begin{verse}
A Lei\\
cinzenta --- ave de\\
\qquad\qquad\qquad\quad rapina ---

voa mas\\
pesa: desce e\\
\qquad\qquad\quad busca\\
\qquad\qquad\quad o Sangue

o Sangue: agravo\\
o Sangue: gravidade.

Peso da\\
Lei\\
peso do\\
Sangue

--- destruição rubro-cinza.
\end{verse}

\chapter{Centauros}

\begin{verse}
Centauros derrubam ídolos

centauros derrubam-se\\
centauros centauros.

Mas a memória\\
--- texto pul\\
\qquad\qquad sante ---\\
\quad mas a memória\\
--- rito do\\
\quad\quad sangue ---

mas a memória\\
--- sempre a\\
\qquad\qquad memória ---

absorvendo o ímpeto\\
floresce.
\end{verse}

\chapter{Relógio}

\begin{verse}
Hora dos\\
peixes\\
hora dos\\
náufragos\\
hora do es\\
pesso\\
concreto abismo

hora das\\
algas\\
lentas flu\\
tuantes\\
hora das\\
ondas\\
brandas in\\
findas

hora dos\\
peixes\\
densos\\
obscuros\\
na obscuridade líquida.
\end{verse}

\chapter{Mapa}

\begin{verse}
Eis a carta dos céus:\\
as distâncias vivas\\
indicam apenas\\
roteiros\\
os astros não se interligam\\
e a distância maior\\
é olhar apenas.

A estrela\\
voo e luz somente\\
sempre nasce agora:\\
desconhece as irmãs\\
e é sem espelho.

Eis a carta dos céus: tudo\\
indeterminado e imprevisto\\
cria um amor fluente\\
e sempre vivo.

Eis a carta dos céus: tudo\\
\qquad\qquad\qquad\qquad se move.
\end{verse}


\chapter{Espelho}

\begin{verse}
O espelho\\
lúcido branco\qquad silente\\
imóvel lâmina\qquad fluxo\\
o espelho:\qquad\qquad corola\\
\qquad\qquad\qquad\qquad branca

o espelho\\
branco centro da\\
\qquad\qquad\qquad\qquad vertigem\\
enorme corola\\
\qquad\qquad\qquad\qquad áspera\\
forma vazia\\
do branco

o espelho\\
flor sem memória\qquad fluência\\
--- intensa corola\\
\quad branca.
\end{verse}


\chapter{Fonte}

\begin{verse}
A fonte (oculta) ignora-se.

Escamas: sóis\\
\qquad\qquad intranquilos\\
torrente: luz\\
que se quebra\\
oferta multi\\
\qquad\qquad plicada.

\ldots{} mas na escura gruta\\
\qquad intacta

a fonte --- serena --- expande-se.
\end{verse}

\chapter{Silêncio}

\forceindent\textsc{i}

\begin{verse}
A madrugada.\\
Seu coração de silêncio.
\end{verse}

\medskip
\textsc{ii}

\begin{verse}
O silêncio cheio\\
de peixes\\
de irisados peixes\\
úmidos.
\end{verse}

\medskip
\textsc{iii}

\begin{verse}
Grandes árvores\\
ânforas\\
transbordantes de silêncio.
\end{verse}

\medskip
\textsc{iv}

\begin{verse}
Galos\\
no alto silêncio\\
impressos

seda\\
translúcida do silêncio.
\end{verse}

\chapter{Nudez}

\begin{verse}
Ainda há maior nudez: barreira\\
ininterrupta do silêncio\\
guardando em si a evidência das formas.

Ainda há maior nudez: evidência\\
sem mais sinais\\
exata em sua luz interna.

Ainda há maior nudez: a luz\\
infinda simplicidade\\
sem apoio além de si mesma.
\end{verse}

\chapter{Migração}

\begin{verse}
Do leste vieram pássaros\\
rápidos leves\\
nem sombra nem rastro\\
deixam:\\
apenas passam. Não pousam.
\end{verse}

\chapter{Via}

\forceindent\textsc{i}

\begin{verse}
Há um caminho solitário\\
construído a cada\\
passo:\\
não leva a lugar algum.
\end{verse}

\medskip
\textsc{ii}

\begin{verse}
Na floresta um branco\\
pássaro\\
oculta-se em seu\\
silêncio.
\end{verse}

\medskip
\textsc{iii}

\begin{verse}
No alto\\
--- jubilosamente ---\\
uma estrela\\
apenas.
\end{verse}

\chapter{Flama}

\begin{verse}
Tensa\\
uma flama\\
no denso silêncio\\
\qquad\qquad\quad vela

imóvel\\
brilha\\
intensa vigília\\
\qquad\qquad áurea

esfera\\
cálida\\
--- brilho e\\
\qquad sigilo ---

no intenso\\
silêncio\\
vibra e\\
\qquad vela.
\end{verse}

\chapter{Ode}

\begin{verse}
O início? O mesmo fim.\\
O fim? O mesmo início.

Não há fim nem início. Sem história\\
o ciclo dos dias\\
vive-nos.
\end{verse}

\chapter{Reflexos}

\begin{verse}
No espelho\\
a vida

a pura\\
vida\\
já sem\\
palavras.

A vida viva.

A vida\\
quem?

A vida\\
em branco\\
espelho\\
puro:

ninguém\\
ninguém.
\end{verse}

\chapter{Letes}

\begin{verse}
Ó rio\\
subterrâneo ao ritmo\\
do sangue

ó água\\
frígida clara\\
que elimina toda a\\
sede

ó água abissal\\
sem gosto\\
nem vestígio algum\\
de tempo

ó fonte\\
sem mais música audível: água\\
\qquad\qquad\qquad\qquad densa\\
que nos limpa de todas\\
as palavras.
\end{verse}

\part{\textsc{rosácea} {[}1986{]}}

\chapter*{}
\thispagestyle{empty}
\mbox{}
\vspace*{\fill}
\hfill Coisas varridas e\\
\hfill ao acaso\\
\hfill mescladas\\
\hfill --- o mais belo universo\\
\hfill \textsc{heráclito}
\vspace*{\fill}

\part*{\textsc{novos}}

\chapter{Iniciação}

\begin{verse}
Se vens a uma terra estranha\\
curva-te

se este lugar é esquisito\\
curva-te

se o dia é todo estranheza\\
submete-te

--- és infinitamente mais estranho.
\end{verse}

\chapter{Errância}

\begin{verse}
Só porque\\
erro\\
encontro\\
o que não se\\
procura

só porque\\
erro\\
invento\\
o labirinto

a busca\\
a coisa\\
a causa da\\
procura

só porque\\
erro\\
acerto: me\\
construo.

Margem de\\
erro: margem\\
de liberdade.
\end{verse}

\chapter{Herança}

\begin{verse}
Da avó materna:\\
uma toalha (de batismo).

Do pai:\\
um martelo\\
um alicate\\
uma torquês\\
duas flautas.

Da mãe:\\
um pilão\\
um caldeirão\\
um lenço.
\end{verse}

\chapter{Kant (relido)}

\begin{verse}
Duas coisas admiro: a dura lei\\
cobrindo-me\\
e o estrelado céu\\
dentro de mim.
\end{verse}

\chapter{O coração (Pascal)}

\begin{verse}
As ignotas\\
(des)razões\\
do\\
espanto.
\end{verse}

\chapter{Do Eclesiastes}

\begin{verse}
Há um tempo para\\
desarmar os presságios

há um tempo para\\
desamar os frutos

há um tempo para\\
desviver\\
o tempo.
\end{verse}

\chapter{Pirâmide}

\begin{verse}
Ei-la\\
dor de milhares\qquad força\\
de humanidade\\
anônima\\
(do faraó\\
nem cinzas).
\end{verse}

\chapter{As coisas selvagens}

\begin{verse}
--- a firme montanha\\
\qquad o mar indomável\\
\qquad o ardente\\
\qquad silêncio ---

em tudo pulsa\\
e penetra\\
o clamor\\
do indomesticável destino.
\end{verse}

\chapter{Viagem}

\begin{verse}
Viajar\\
mas não\\
para

viajar\\
mas sem\\
onde

sem rota\qquad\quad sem ciclo\qquad\quad sem círculo\\
sem finalidade possível.

Viajar\\
e nem sequer sonhar-se\\
esta viagem.
\end{verse}

\part*{Lúdicos}

\chapter{Habitat}

\begin{verse}
O peixe\\
é a ave\\
do mar

a ave\\
o peixe\\
do ar

e só o\\
homem\\
nem peixe nem\\
ave

não é\\
daquém\\
e nem de além\\
e nem

o que será\\
já em nenhum\\
lugar.
\end{verse}

\chapter{O Anti-César}

\begin{verse}
Não vim.\\
Não vi.\\
Não havia guerra alguma.
\end{verse}

\chapter{\textsc{cda} (imitado)}

\begin{verse}
Ó vida, triste vida!\\
Se eu me chamasse Aparecida\\
dava na mesma.
\end{verse}

\chapter{\textsc{cda} (relido)}

\begin{verse}
Caio ver\\
\qquad\quad ticalmente\\
\qquad\quad e me transformo.
\end{verse}

\chapter{Homenagens}

\begin{verse}
\textsc{cda}\\
No meio\\
do caminho a flor\\
nasceu.

\textsc{mb}\\
A rosa só\\
(mas que calor\\
danado!)

A estrela d'alva, o\\
escândalo\\
a vontade de morrer

(mas era um calor\\
danado!)

J. J. Rousseau\\
*** les riches\\
\qquad très sensibles\\
\qquad dans toutes les parties\\
\qquad de leurs biens.\\
\qquad (Du contrat social)
\end{verse}

\part*{Bucólicos}

\chapter{Gatha}

\begin{verse}
O vento, a chuva, o Sol, o frio\\
tudo vai e vem, tudo vem e vai.\\
Tenho a ilusão de estar sonhando.\\
Tenho o manto de Buda, que é nenhum.

Myosen Xingue\\
(Meu nome como leiga Zen-budista)
\end{verse}

\chapter{Bucólica}

\begin{verse}
Vaca\\
mansamente pesada

vaca\\
lacteamente morna

vaca\\
densamente materna

inocente grandeza: vaca

vaca no pasto (ai, vida,\\
simples vaca).
\end{verse}

\chapter{Rosas}

\begin{verse}
As rosa\\
(brancas\\
as claras rosa\\
calam-s\\
e floresce o silêncio.

Flor\\
terra\\
silêncio\\
vento

ausência de\\
pensamento.

Encanto\\
e\\
espanto;

o adorável\\
adorante\\
helianto.

Simples\\
a água

o amor\\
mais simples.

Luz\\
fria. Pelos caminhos\\
as rosas brancas\\
em lágrimas.

A chuva\\
lavou-me\\
toda\\
sem deixar vestígios\\
de ontem.

Pedrinha\\
redonda\\
fria

estrela branca nas\\
águas.

Noite\\
vaso\\
negro

e o silêncio uma flor\\
branca.
\end{verse}

\chapter{Nuvem}

\begin{verse}
Asa sem\\
pássaro\\
se vai ou\\
vem\\
se vem ou\\
\quad vai\\
\quad quem\\
\quad sabe?

Leve\quad vazia\quad branca.

A flor do\\
céu. A forma\\
do silêncio.
\end{verse}

\part*{Mitológicos}

\chapter{Lenda}

\begin{verse}
Na raiz cega deste espanto\\
há um cristal: quem o fitar

ah, quem o fitar\\
com os olhos em sangue\\
com as mãos em sangue\\
com o sangue vivo

quem o fitar não dormirá\\
mas será cristal de espanto

--- ficará lúcido para sempre.
\end{verse}

\chapter{Dom Quixote}

\begin{verse}
És filho do desejo e do espírito\\
e (como a carne é impureza) a loucura\\
não te salva de ser, e cais

Triste Figura mesmo\\
se o delírio te eleva\\
à potência do abismo

Triste Figura mesmo\\
na alta planície em que\\
eternizado, morres

herdeiro do desejo e do espírito.
\end{verse}

\chapter{Antigênesis}

\begin{verse}
Abóbada par\\
tida\\
os céus\\
se rompem.\\
Terra solvida. Vida finda. O\\
Sopro\\
reabsorve-se

e a escuríssima\\
água\\
bebe\\
a\\
luz.
\end{verse}

\chapter{Esconjuro}

\begin{verse}
Vai-te, Selene, vai-te daqui\\
vampira\\
Diana estéril selvagem\\
assassina

vai-te, vai-te daqui, noiva do Hades\\
Perséfone\\
vai-te caveira pedra morta\\
Medusa

vai-te, Medeia feiticeira, Circe,\\
dona do abismo amargo do mar\\
doido\\
dona do mênstruo, vai!

Vai-te daqui, cadela\\
Helena infame\\
vai-te, luz falsa, vai-te\\
puta virgem

infernal Hécate! Vai-te daqui\\
\textsc{vai}!
\end{verse}

\chapter{Esfinge}

\begin{verse}
Não há perguntas. Selvagem\\
o silêncio cresce, difícil.
\end{verse}

\part*{Antigos}

\chapter{Origem}

\begin{verse}
Nem flor nem folha mas\\
raíz\\
absoluta. Amarga.

\qquad Nem ramos nem botões. Raiz\\
\qquad\quad íntegra. Sórdida.

\qquad\quad Nem tronco ou\\
\qquad\quad caule. Nem sequer planta\\
\qquad\quad --- só a raiz\\
\qquad\quad\quad é o fruto.
\end{verse}

\chapter{A paz}

\begin{verse}
não reconstrói: elide\\
a trama e o verbo.

A Paz\\
não organiza: explode\\
o núcleo-tempo.

A Paz\\
não é letal: vivifica.

A Paz\\
não apazigua: fere.

A Paz\\
não acalma: renova\\
o ser e o sangue.
\end{verse}

\chapter{Ceia}

\begin{verse}
A mesa, todos\\
interligados\\
pela realidade do alimento\\
pelo universo único\\
do ser\\
a mesa, todos\\
coexistem no júbilo\\
comungando a oferta pura das coisas.
\end{verse}

\chapter{Partilha}

\begin{verse}
Partilharemos somente\\
o que em nós se\\
continua:\\
a singeleza\\
a luta\\
a esperança.

Partilharemos somente\\
esta maior intensidade:\\
absoluta palavra\\
que nos pertence integralmente.

Partilharemos somente\\
o pão unificado\\
e a água sem face.
\end{verse}

\part{\textsc{teia} {[}1996{]}}

\chapter*{}
\thispagestyle{empty}
\mbox{}

\vspace*{\fill}
\hfill A lucidez\\
\hfill alucina

\bigskip

\epigraph{Todas as grandes\\
coisas\\
são difíceis\\
e raras}{\textsc{spinoza}}
\vspace*{\fill}

\part*{Fala}

\chapter{Teia}

\begin{verse}
A teia, não\\
mágica\\
mas arma, armadilha

a teia, não\\
morta\\
mas sensitiva, vivente

a teia, não\\
arte\\
mas trabalho, tensa\\

a teia, não\\
virgem\\
mas intensamente\\
\qquad\qquad prenhe:

no\\
centro\\
a aranha espera.
\end{verse}

\chapter{Fala}

\begin{verse}
Falo de agrestes\\
pássaros\qquad\qquad de sóis\\
\qquad que não se apagam\\
\qquad de inamovíveis\\
\qquad pedras

\qquad de sangue\\
\qquad vivo de estrelas\\
\qquad que não cessam.

\qquad Falo do que impede\\
\qquad o sono.
\end{verse}

\chapter{Exemplos}

\begin{verse}
Platão\\
fixando as formas

Heráclito\\
cultuando o fogo

Sócrates\\
fiel ao seu Daimon.
\end{verse}

\chapter{Maiêutica}

\begin{verse}
Gerar é escura\\
lenta\\
forma in\\
\qquad\qquad forme

gerar é\\
força\\
silenciosa\\
firme

gerar é\\
trabalho\\
opaco:

só o nascimento\\
grita.
\end{verse}

\chapter{Para \textsc{cda}}

\forceindent\textsc{i}

\begin{verse}
O boi é só. O boi é\\
só. O\\
boi.
\end{verse}

\medskip
\textsc{ii}

\begin{verse}
Que século, meu Deus! disseram\\
os ratos.
\end{verse}

\medskip
\textsc{iii}

\begin{verse}
Perdi o bonde\\
(e a esperança), porém\\
garanto\\
que uma flor nasceu.
\end{verse}

\medskip
\textsc{iv}

\begin{verse}
Ôpa, carlos: desconfio\\
que escrevi um poema!
\end{verse}

\part*{Axiomas}

\chapter{Axiomas}

\begin{verse}
Sempre é melhor\\
saber\\
\qquad que não saber.

\qquad Sempre é melhor\\
\qquad sofrer\\
\qquad que não sofrer.

\qquad Sempre é melhor\\
\qquad desfazer\\
\qquad que tecer.

\qquad Sem mão\\
\qquad não acorda\\
\qquad a pedra

\qquad sem língua\\
\qquad não ascende\\
\qquad o canto

\qquad sem olho\\
\qquad não existe\\
\qquad o sol.
\end{verse}

\chapter{Newton (ou a gravidade)}

\forceindent\textsc{i}

\begin{verse}
A maçã\\
cai\\
e os astros\\
dançam.
\end{verse}

\medskip
\textsc{ii}

\begin{verse}
O abismo atrai\\
o abismo: caio\\
em\\
mim.
\end{verse}

\chapter{Kairós}

\begin{verse}
Quando pousa\\
o pássaro

quando acorda\\
o espelho

quando amadurece\\
a hora.
\end{verse}

\chapter{Carta}

\begin{verse}
Da\\
vida\\
não se espera resposta.
\end{verse}

\part*{O antipássaro}

\chapter{O antipássaro}

\begin{verse}
Um pássaro\\
seu ninho é pedra

seu grito\\
metal cinza

dói no espaço\\
seu olho.

Um pássaro\\
pesa\\
e caça\\
entre lixo\\
e tédio.

Um pássaro\\
resiste aos\\
céus. E perdura.\\
Apesar.
\end{verse}

\chapter{Fatos}

\begin{verse}
\ldots{} fatos\\
são pedras duras.

Não há como fugir.

Fatos são palavras\\
ditas pelo mundo.

(\emph{Extraído de} A hora da Estrela\emph{, de Clarice Lispector}.)
\end{verse}

\chapter{Memória}

\begin{verse}
A cicatriz, talvez\\
não indelével

o sangue\\
agora\\
estigma.
\end{verse}

\chapter{Do poder}

\begin{verse}
Dentes: positivos.

Presas a\\
preendem\\
incisivos\\
cortam.

Dentes: decisivos.
\end{verse}

\chapter{Teologia}

\begin{verse}
Não sou um deus, Graças a todos\\
os deuses!\\
Sou carne viva e\\
sal. Posso morrer.
\end{verse}

\part*{Galo\\ {[}Noturnos{]}}

\chapter{Galo}

\begin{verse}
Canta o galo e a\\
noite\\
se aprofunda\\
em plena meia\\
noite: o galo\\
é negro.

Galo abissal --- galo invisível\\
canta\\
e tudo o mais se cala. No\\
vazio\\
só --- opaco --- per\\
siste\\
o galo\\
negro.
\end{verse}

\chapter{Gatos}

\forceindent\textsc{i}

\begin{verse}
Os gatos\\
secretos\\
saltam

somem no abstrato\\
escuro.
\end{verse}

\medskip
\textsc{ii}

\begin{verse}
Gatos no\\
negro\\
fluem: fosforecem

arranham vidros\qquad\qquad\quad destroçam\\
espectros\\
farejam todos\\
os rumos.
\end{verse}

\medskip
\textsc{iii}

\begin{verse}
No vácuo\\
insone\qquad\qquad\quad na meia-noite\\
lúcida\\
cuidado: gatos\\
agindo.

Numa hora\\
secreta\\
as águas\\
dormem

(rios detidos\\
\quad fontes inertes\\
\qquad introvertido oceano)

numa hora\\
impossível\\
cessa o\\
\qquad\quad fluxo

e eis a\\
estrela: amor\\
cristalizado.
\end{verse}

\chapter{Noite}

\begin{verse}
Esconder (esquecer)\\
a face

soterrar (ocultar)\\
a luz

escurecer o\\
amor\\
dormir.

Aguardar o que nasce.
\end{verse}

\part*{Figuras}

\chapter{Círculo}

\begin{verse}
O círculo\\
é astuto:\\
enrola-se\\
envolve-se

autofagicamente.

Depois\\
explode\\
--- galáxias! ---

abre-se\\
vivo\\
pulsa

multiplica-se

divindadecírculo\\
perplexa\\
(perversa?)

o unicírculo\\
devorando\\
tudo.
\end{verse}

\chapter{Narciso (jogos)}

\begin{verse}
Tudo\\
acontece no\\
espelho.
\end{verse}

\pagebreak
\begin{verse}
A vida é que nos tem: nada mais\\
\qquad\qquad\qquad\quad temos.
\end{verse}

\pagebreak
\begin{verse}
Nunca amar\\
o que não\\
vibra

nunca crer\\
no que não\\
canta.
\end{verse}

\pagebreak
\begin{verse}
O espelho dissolve\\
o tempo

o espelho aprofunda\\
o enigma

o espelho devora\\
a face.
\end{verse}

\part*{Vésper\\ {[}Finais{]}}

\chapter{Porta}

\begin{verse}
O estranho\\
bate:\\
na amplitude interior\\
não há resposta.

É o estranho (o irmão) que bate\\
mas nunca haverá\\
resposta:

muito além é o país\\
do acolhimento
\end{verse}

\chapter{Cantiga}

\begin{verse}
Ouvir um\\
pássaro\\
é agora ou\\
nunca

é infância ou\\
puro\\
momento?

Ouvir um\\
pássaro\\
é sempre

(dói fundo no\\
pensamento).
\end{verse}

\chapter{Balada}

\begin{verse}
Os anjos são\\
livres.

Podemos sofrer\\
podemos viver\\
o acontecer\\
único

--- os anjos são\\
livres ---

podemos morrer\\
inocentemente

--- e os anjos são\\
livres\\
até da inocência.
\end{verse}

\chapter{Vésper}

\begin{verse}
A estrela da tarde está\\
madura\\
e sem nenhum perfume.

A estrela da tarde é\\
infecunda\\
e altíssima:

depois dela só há\\
o silêncio.
\end{verse}

\part{\textsc{poemas inéditos} {[}1997--1998{]}}

\chapter{Utopia}

\forceindent\textsc{i}

\begin{verse}
Poema: casa\\
ao contrário

o exato in\\
verso\\
do abrigo.
\end{verse}

\medskip
\textsc{ii}

\begin{verse}
Avisos. Perigos. Fugas-\\
Alta tensão nas\\
\qquad\qquad torres.
\end{verse}

\medskip
\textsc{iii}

\begin{verse}
Poema: abrigo\\
im\\
possível

casa jamais\\
habitada
\end{verse}

\pagebreak
\forceindent\textsc{i}

\begin{verse}
\qquad\quad \textsc{só é paraíso}\\
ontem\\
\qquad\quad porém amanhã\\
tem circo.
\end{verse}

\medskip
\textsc{ii}

\begin{verse}
Paz?\\
no futuro.\\
Glória?\\
no passado.
\end{verse}

\medskip
\textsc{iii}

\begin{verse}
Nunca há paraíso\\
aqui e\\
agora\\
--- mas amanhã tem circo!
\end{verse}


\chapter{Da poesia}

\begin{verse}
Um\\
gato tenso\\
tocaiando o silêncio
\end{verse}

\chapter{Autoimagem}

\begin{verse}
Por ser cego e\\
irrefletido\\
meu espelho disse\\
a verdade:

quebrei-o.

Sete anos\\
sete anos\\
sete anos de\\
enganos!
\end{verse}

\chapter{Teologia \textsc{ii}}

\begin{verse}
Deus existir\\
ou não: o mesmo\\
escândalo.
\end{verse}
