\chapterspecial{A poesia selvagem de Orides Fontela}{}{Rodrigo Ribeiro Neves}

\section{Sobre a autora}

Selvagem, assim como o silêncio que cresceu difícil de um de seus
versos, nasceu Orides Fontela, em 21 de abril de 1940, na cidade de São
João da Boa Vista, interior de São Paulo. Filha de pai analfabeto,
recebeu da mãe suas lições ainda cedo, chegando a escrever seus
primeiros versos entre os seis e sete anos de idade. Em entrevistas,
Orides comentou que compunha, nessa época, umas quadrinhas infantis, a
mesma fonte de que muitos grandes poetas beberam, como Manuel Bandeira,
um de seus mestres na fase adulta. Na sua cidade natal, frequentou o
ginásio e a escola normal, formando"-se como professora primária, em
1955.

No ano seguinte, publicou seus primeiros poemas, no jornal \emph{O
Município}. Com apenas 16 anos, saía seu primeiro escrito no cabeçalho das colunas socinais d'\emph{O Município}:

\begin{quote}
Noite calma de outono. Céu límpido e estrelado. Uma menina caminhava rapidamente pelas ruas paralelepípedos, de braços dados com seu pai e agarrada a um guarda"-chuva. Usava um vestido branco com mangas bufantes e muitos saiotes para aumentar o volume de seu vestido, tentando esconder um corpo muito magro. Não olhavam pra ninguém. Viviam num mundo fechado, só deles. Iam ao cinema, como faziam todas as semanas. Este era o programa preferido dos dois\ldots{} Seu nome? Orides Fontela.\footnote{Disponível em: \emph{http://www.mulheresdesaojoao.com.br/index\_arquivos/OridesFontelaBiografia.htm}.}
\end{quote}

Alguns anos mais tarde, se mudou para a capital paulista,
para estudar filosofia na \textsc{usp}, formação que influenciou muito na seu
fazer poético. Os poemas iniciais, publicados naquele periódico
paulista, chamaram a atenção de um conterrâneo, o crítico Davi Arrigucci
Junior, de quem Orides se tornou amiga. Sobre ela relembra o crítico:

\begin{quote}
A conheci ainda menina em São João da Boa Vista. Ela foi minha companheira de período no Grupo Escolar ``Coronel Joaquim José''. Era uma menina um pouco peculiar, como foi a vida inteira, pois não sabia controlar os risos e as lágrimas. Os anos foram passando\ldots{} Algumas vezes eu a via no consultório médico de meu pai. Orides ia com o pai, com quem se dava muitíssimo bem. Eu também o conheci. Era um marceneiro, homem bom, simples, modesto e pobre, com uma inteligência viva, arguta, perguntadora e também muito engraçado e, neste sentido, ela era bastante parecida com ele. Às vezes eu a via no cinema acompanhada do pai. Gostava de fazer comentários altos, sem se incomodar se atrapalhava as pessoas. Ela não tinha muito a medida do outro.\footnote{Disponível em: \emph{http://www.mulheresdesaojoao.com.br/index\_arquivos/OridesFontelaBiografia.htm}.}
\end{quote}


Arrigucci Junior foi um dos
principais responsáveis por trazer a obra de Orides Fontela para uma
comunidade maior de leitores, além de intelectuais e críticos paulistas,
coordenando a publicação de seu primeiro livro, \emph{Transposição}, de
1969. Seu terceiro livro, \emph{Alba}, de 1983, contou com prefácio de
Antonio Candido e ganhou o prêmio Jabuti, o que sugeria a consolidação
de uma carreira literária. Muitos a viam como uma continuidade do
modernismo de João Cabral de Melo Neto, outros, como uma renovadora da
fase heroica do movimento, e outros ainda, como uma voz necessária em um
período em que já se esboçava um esgotamento das vanguardas dos anos
1950.

Na verdade, Orides Fontela aparece como uma voz única e dissonante das
tendências que se formavam em seu tempo, por mais dispersivas e
múltiplas que, em geral, elas sejam caracterizadas. A influência do
modernismo na formação intelectual e artística de Orides Fontela foi
fundamental, no entanto, ela também dialogou com preceitos de seu tempo,
como a visualidade na construção dos poemas, à maneira dos poetas
concretos. A dimensão dos símbolos com o qual ela trabalhou fez com que
alguns a enxergassem como neossimbolista, na esteira de uma Cecília
Meireles, mas os vestígios dessa estética a aproximam mais de outra
poeta modernista, a mineira Henriqueta Lisboa. Entre os poetas de sua
geração, ela não se alinha a nenhum grupo.

E essa ausência de relação reflete não apenas a inserção de sua obra,
mas também o seu lugar como intelectual. Infelizmente as dificuldades
financeiras e o seu temperamento considerado difícil a lançou em
situações de privação. Morou de favor em uma residência estudantil na
região central de São Paulo e se isolou cada vez mais dos amigos. Passou
os últimos dias de sua vida em um sanatório, em Campos do Jordão, onde
faleceu em 2 de novembro de 1998. Dois anos antes, havia publicado sua
última obra, \emph{Teia}, conquistando o prêmio da \textsc{apca}, a Associação
Paulista de Críticos de Arte. Uma obra, assim como todas as demais,
tecida com os fios do tempo, que amadurece diante da nossa espera.

\pagebreak

\section{Sobre a obra}

Esta edição reúne poemas escolhidos da obra de Orides Fontela a partir
dos livros \emph{Transposição} (1969), \emph{Helianto} (1973),
\emph{Alba} (1983), \emph{Rosácea} (1986), \emph{Teia} (1996) e inéditos
escritos entre 1997 e 1998, publicados em sua \emph{Poesia completa}
(2015). Os textos estão organizados na ordem cronológica de lançamento
dos livros e foram selecionados de acordo com o que consideramos
representativo para a compreensão dos principais aspectos da poesia de
Orides Fontela e de sua importância para a literatura brasileira.

Seu livro de estreia, \emph{Transposição}, contém poemas escritos entre
1966 e 1967. A sua publicação teve a colaboração do crítico Davi
Arrigucci Junior, conterrâneo de Orides e de quem se tornou grande
incentivador e amigo. O livro se divide em quatro seções, ``Base'',
``(--)'', ``(+)'' e ``Fim'', numerada com algarismos romanos. A primeira
sugere o estabelecimento dos fundamentos construtivos de sua poética,
como se fossem os alicerces de uma edificação. A partir do primeiro
poema, que intitula a obra, somos levados a acompanhar um movimento
entre o real e o símbolo, em um jogo de perda e de alcance do que é
possível. Os sinais negativo e positivo indicam essa tendência: ora o
impasse e a dor, ora a lucidez e a contemplação. É a transposição do
ser, daquilo que ele é e daquilo que falta, para a dimensão simbólica,
que o sustenta e o resgata.

\emph{Helianto} traz poemas de 1973. Enquanto no primeiro havia uma
noção de deslocamento de um ponto a outro, aqui surge a noção de
movimento circular, também expresso na epígrafe com uma cantiga de roda.
A palavra que intitula a obra é uma designação, em Botânica, para se
referir às plantas do gênero \emph{Helianthus}, das quais, uma das mais
conhecidas é o girassol, portanto, os signos de lucidez permanecem como
elementos de sua poética, mas, desta vez, eles giram, atando as pontas
entre a palavra e o ser que ela busca representar. É um dos livros em
que Orides mais explora a geometria e materialidade visual de seus
poemas.

Os textos de \emph{Alba} são de 1983 e acompanham um prefácio de Antonio
Candido, conferindo uma estatura consolidada na recepção crítica de sua
poesia. Não por acaso, o livro rendeu a Orides o Prêmio Jabuti. O título
tanto remete à primeira claridade da manhã quanto a um gênero lírico
medieval em que há a despedida de dois amantes nesse mesmo instante do
dia. De uma certa maneira, podemos pensar nesses dois amantes como o ser
e o símbolo, a claridade é a poesia. O instante da despedida, no
entanto, não é a separação consumada, mas o que a antecede e a suspende,
tornando"-os conscientes, lúcidos, de que o momento seguinte os divide,
mas aquele em que expressam isso, o presente, ainda os mantém. Não por
acaso, \emph{Alba} é um livro mais maduro em relação à linguagem poética
e à destinação da própria poesia como uma atividade do espírito. A
epígrafe de San Juan de la Cruz dialoga com o aspecto místico"-religioso
que também aparece nos poemas, não de forma temática, mas na sobriedade
e gravidade na representação das paixões.

A composição de \emph{Rosácea}, de 1986, é mais heterogênea em relação
aos anteriores e se divide em seções, ``Novos'', ``Lúdicos'',
``Bucólicos'', ``Mitológicos'' e ``Antigos''. As questões que permeiam
sua obra até aqui estão presentes, mas os movimentos circulares e de
transposição possuem referências intelectuais, culturais e afetivas, com
nomes de poetas, filósofos, familiares, entre outros. A sigla de Carlos
Drummond de Andrade (\textsc{cda}), por exemplo, uma das grandes influências para
a poeta, é uma das mais recorrentes. Essas referências não são um mero
repositório de informações biográficas e bibliográficas. A poesia de
Orides Fontela rejeita qualquer tom confessional ou de reconstituição
nostálgica de algum passado. O tempo é o da experiência poética, que
atualiza essas referências em face do inevitável destino da poesia.

Por fim, Orides lançou \emph{Teia}, de 1996, premiado pela \textsc{apca}, a
Associação Paulista de Críticos de Arte. O livro se divide nas seções
``Fala'', ``Axiomas'', ``O antipássaro'', ``Galo {[}Noturnos{]}'',
``Figuras'' e ``Vésper {[}Finais{]}''. Em seu último livro, a poeta
retoma as questões e as imagens que a acompanharam em seu fazer poético,
expressão do silêncio extraído dos símbolos em seu encontro com o real.
A imagem da teia reforça a atividade literária, artifício de quem tece o
seu texto à espera de agarrar o que resta do tempo, das coisas e da
vida. Ou de quem sabe da permanência da própria espera, de quem está à
espreita, a um passo do pássaro e de acontecer.

Acrescentamos ainda, nesta edição, cinco poemas dentre os seus inéditos,
escritos no fim da vida da poeta, mas só publicados postumamente na
reunião de toda sua obra. Neles também estão presentes certa
religiosidade, a lucidez, a visualidade, os deslocamentos e giros entre
o ser e os símbolos, além da consciência da poesia como destino e do
tempo da experiência poética.

Para o estabelecimento do texto, utilizamos a edição de \emph{Poesia
completa}, lançada em 2015 pela editora Hedra e organizada por Luis
Dolhnikoff. Foram mantidas todas as epígrafes e as seções dos
respectivos livros, pois funcionam como elementos compositivos na
estruturação dos poemas em cada edição.

\pagebreak
\section{Sobre o gênero}

Aos 25 anos, quando foi descoberta por Davi Arrigucci Jr., Orides Fontela viu"-se subitamente no centro de um grande embate nas letras brasileiras, talvez o mais duro desde as repercussões da Semana de Arte Moderna de 1922.
Como escreve o escritor e crítico literário Luis Dolhnikoff:

\begin{quote}
Tratava-se da luta de vida e morte pela herança modernista, no palco
armado pelos adeptos das novas vanguardas visualistas do final dos anos
1950 (das quais o concretismo era a mais visível), que pretendiam tirar
de cena os defensores do verso \emph{moderno}. Pois nada do verso
tradicional restara depois da revolução modernista, que, entre outras
coisas, aproximou a linguagem poética da sintaxe brasileira
contemporânea, além de implodir, explodir e repudiar todas as formas
antigas, incluindo estruturas (como a do soneto) e ritmos (como as
medidas tradicionais de versos).

As novas vanguardas, então, vieram para
dizer que o que o modernismo fizera ao revolucionar o verso estava
feito e bem feito, mas não mais bastava. A revolução tinha de continuar.
E, para tanto, sacrificar no altar do Novo o próprio verso modernista,
que resultara, de um modo ou de outro, por caminhos e vieses
múltiplos, na maior e melhor poesia brasileira, incluindo nomes como
Carlos Drummond, João Cabral, Murilo Mendes, Ferreira Gullar e Vinicius
de Moraes.\footnote{João Cabral de Melo Neto iniciou sua obra próximo do surrealismo e da ``geração de 45''. Em mais de um aspecto, portanto, muito longe do modernismo. Mas
apenas se entendido como o Modernismo de 22. Pois Cabral não se
engajou em um ``\emph{retour à l'ordre}'' (para lembrar a expressão de
Picasso ao voltar à figuração depois de ajudar a criar o abstracionismo, via cubismo), caso em que se deveriam encontrar, por exemplo,
traços parnasianos em sua linguagem. Cabral, o ``antilírico'' por
excelência, ou, dito de outro modo, um representante do
``objetivismo'', que engloba obras tão variadas como as de Wallace
Stevens, Carlos Williams e Francis Ponge, é por isso mesmo um dos
grandes representantes do modernismo internacional. No caso de Gullar,
bastaria citar o \emph{Poema sujo}, uma ``lição de coisas'' das
incontáveis possibilidades do verso modernista. Vinicius, naturalmente,
além dos justamente famosos sonetos, também se dedicou ao verso livre (e
escreveu sonetos como ``A pera'').} O verso modernista, porém, apesar da potência então ainda inconclusa dessas obras, estava repentinamente morto --- ao menos
segundo as novas vanguardas visualistas. A poesia
futura seria visual ou não seria.\footnote{\textsc{dolhnikoff}, Luis. ``Introdução''. In: \textsc{fontela}, Orides. \emph{Poesia completa}. São Paulo: Hedra, 2015, pp.7-88.}
\end{quote}

Em meio a esse embate, como bem descreve Dolhnikoff, a crítica ``ou rejeitava as novas obras e as novas teses ou, ao menos, rejeitava a nova tese de que a nova poesia visual não podia ou não devia conviver com o `velho' verso modernista''.\footnote{Ibid, p.\,8.} Orides Fontela surgiu no meio desse quadro e, com sua poesia ousada, revitalizou o verso do modernismo brasileiro. 

Muitos críticos descrevem sua poesia como ``antilírica'', no sentido de que o eu"-lírico, ao longo de seus versos, perde espaço para a própria palavra, que ganha protagonismo e parece se sustentar autônoma, questionando seus limites de apreensão e representação da realidade e da experiência. É uma característica que a aproxima das vanguardas visualistas, como o concretismo, movimento que, em certa medida, parece também ter influenciado Orides na composição de seus versos que jogam e brincam com a espacialidade branca do papel, fazendo da posição da palavra no poema outro elemento para a definição de seu sentido.

No entanto, ao contrário dos visualistas, a poesia de Orides não pode ser considerada ``abstratizante''. Na definição de Dolhnikoff, a poesia ``abstratizante'' é ``aquela que busca, consciente ou inconscientemente, restringir a condição referente da palavra'' através de vários mecanismos, ``como o uso cifrado, idiossincrático, para
`iniciados', o vocabulário esdrúxulo, os neologismos injustificados, as
apropriações extraculturais, os estrangeirismos `de butique' (ou da
moda) etc''.\footnote{Ibid., p.\,10.} Orides, no entanto, considerava esse tipo de poesia algo próximo do barroco, sem sentido ou força poética no século \textsc{xx}.

Nem visualista, nem ``abstratizante'', a poeta pode ser considerada uma renovadora do modernismo, o que reafirma a força incontornável de sua obra poética.