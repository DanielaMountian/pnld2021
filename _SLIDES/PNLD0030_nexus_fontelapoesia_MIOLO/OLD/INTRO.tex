\chapterspecial{Introdução}{A áspera beleza da poesia que renovou o modernismo brasileiro}{Luis Dolhnikoff}

A poeta paulista Orides Fontela (1940--1998) surgiu na cena literária
brasileira da segunda metade do século \textsc{xx} por meio de alguns dos nomes
mais influentes das críticas literária e acadêmica (a começar de Antonio
Candido). E se revelaria, afinal, a poeta mais importante de sua geração,
que reúne autores mais conhecidos, ou menos desconhecidos, como Hilda
Hilst, Adélia Prado, Roberto Piva e Paulo Leminski.\footnote{Nascidos nos anos 1930--40.} Entender os motivos
da dissintonia entre sua importância e seu reconhecimento pode revelar
algo ou muita coisa sobre o estado da poesia brasileira contemporânea,
sua recepção pública e sua crítica.\footnote{Este volume reúne todos os livros de poemas de Orides Fontela
publicados, assim como uma pequena mas significativa coleção de 27 poemas
inéditos, resgatada pelo biógrafo Gustavo de Castro em \emph{O enigma
Orides} (São Paulo, Hedra, 2015 {[}ganhador do Rumos Itaú Cultural
2013--2014{]}). Daí se tratar de sua poesia completa, salvo algum
improvável novo achado póstumo. Cabe acrescentar que os dados sobre a
vida e a vida literária de Orides referidos nesta introdução devem seu
mérito ao trabalho de Gustavo de Castro. Os eventuais equívocos são
todos do autor deste texto.}

ando descoberta por Davi Arrigucci Jr. através de um poema publicado
no jornal de sua cidade, São João da Boa Vista, em 1965 (o que pouco
depois resultaria em seu primeiro livro, \emph{Transposição},
coorganizado por ele), Orides Fontela, sem o saber, e à mais completa
revelia de seus 25 anos, estava ou foi posta no centro do embate mais
duro travado nas letras brasileiras desde as primeiras reações e
rejeições ao Modernismo de 22.

Tratava-se da luta de vida e morte pela herança modernista, no palco
armado pelos adeptos das novas vanguardas visualistas do final dos anos
1950 (das quais o concretismo era a mais visível), que pretendiam tirar
de cena os defensores do verso \emph{moderno}. Pois nada do verso
tradicional restara depois da revolução modernista, que, entre outras
coisas, aproximou a linguagem poética da sintaxe brasileira
contemporânea, além de implodir, explodir e repudiar todas as formas
antigas, incluindo estruturas (como a do soneto) e ritmos (como as
medidas tradicionais de versos).

As novas vanguardas, então, vieram para
dizer que o que o modernismo fizera ao revolucionar o verso estava
feito e bem feito, mas não mais bastava. A revolução tinha de continuar.
E, para tanto, sacrificar no altar do Novo o próprio verso modernista,
que resultara, de um modo ou de outro, por caminhos e vieses
múltiplos, na maior e melhor poesia brasileira, incluindo nomes como
Carlos Drummond, João Cabral, Murilo Mendes, Ferreira Gullar e Vinicius
de Moraes.\footnote{João Cabral de Melo Neto iniciou sua obra próximo do surrealismo e da ``geração de 45''. Em mais de um aspecto, portanto, muito longe do modernismo. Mas
apenas se entendido como o Modernismo de 22. Pois Cabral não se
engajou em um ``\emph{retour à l'ordre}'' (para lembrar a expressão de
Picasso ao voltar à figuração depois de ajudar a criar o abstracionismo, via cubismo), caso em que se deveriam encontrar, por exemplo,
traços parnasianos em sua linguagem. Cabral, o ``antilírico'' por
excelência, ou, dito de outro modo, um representante do
``objetivismo'', que engloba obras tão variadas como as de Wallace
Stevens, Carlos Williams e Francis Ponge, é por isso mesmo um dos
grandes representantes do modernismo internacional. No caso de Gullar,
bastaria citar o \emph{Poema sujo}, uma ``lição de coisas'' das
incontáveis possibilidades do verso modernista. Vinicius, naturalmente,
além dos justamente famosos sonetos, também se dedicou ao verso livre (e
escreveu sonetos como ``A pera'').} O verso modernista, porém, apesar da potência então ainda inconclusa dessas obras, estava repentinamente morto --- ao menos
segundo as novas vanguardas visualistas. A poesia
futura seria visual ou não seria.

A voz das novas vanguardas era alta. Suas teses, altissonantes. Suas
obras, cada vez mais visíveis e vistosas. Mas ainda não bastava para calar parte importante da crítica, que ou rejeitava as novas obras e as
novas teses ou, ao menos, rejeitava a nova tese de que a nova poesia
visual não podia ou não devia conviver com o ``velho'' verso modernista,
cuja longevidade fora mostrada, demonstrada e mantida pelas gerações posteriores à
de 22.

Nesse quadro, um jovem poeta que, além de jovem em si, rejuvenescesse,
revitalizasse ou renovasse o verso modernista seria, quase irresistivelmente, recebido e incensado por boa parte da mais importante crítica literária e acadêmica --- que apesar de sua força e influência,
sentia os duros golpes das novas e vigorosas vanguardas. Esse aguardado
e, de alguma forma, buscado revitalizador do verso modernista brasileiro
foi Orides Fontela.\footnote{Outro que parecia capaz de ocupar esse lugar foi Bruno Tolentino
(1940--2007), notadamente em seu livro de estreia, \emph{Anulação e
outros reparos} (São Paulo, Massao Ohno, 1963, apresentação José
Guilherme Merquior). Mas Tolentino, à diferença de Cabral, operou em
seguida um verdadeiro ``retorno à ordem'' (literalmente): à ordem, ou
estrutura, do soneto. Isto não torna sua obra menor, ao contrário, pois
ele seria, ao lado de Nelson Ascher e Glauco Mattoso (sem esquecer,
obviamente, Drummond, Vinicius e Jorge de Lima), um vigoroso renovador
da longa tradição do soneto em português. Não poucos poemas de \emph{A
balada do cárcere} (Rio de Janeiro, Topbooks, 1996) e de \emph{A
imitação do amanhecer} (São Paulo, Globo, 2006) estão entre os grandes
sonetos da língua portuguesa. E sua língua é o português brasileiro
contemporâneo.}

Toda uma pequena mitologia envolve a obra de Orides. Em parte alimentada por ela própria, ao se declarar uma adepta da ``inspiração'',
isto é, do espontaneísmo ou da espontaneidade (e não, portanto, da
concepção, da construção e da elaboração do poema) e em parte por sua
biografia intelectual, que tem dois marcos principais: o curso de
filosofia na \textsc{usp}, o que a tornaria uma poeta filósofa, ou de uma poesia
filosofante ou filosófica, e a dedicação ao budismo nos anos 1970, o que
daria à sua obra um viés metafísico-oriental. Nada disso, porém, resiste
minimamente aos fatos, ou seja, aos fatos de linguagem de sua poesia.

Como representante (maior) da última geração ainda modernista, ou seja,
como uma renovadora do modernismo em seu final (não porque nos tenhamos
tornado, depois, ``pós-modernistas'', mas porque depois nos tornamos
diluidores de todas as conquistas poéticas do século \textsc{xx}; daí a fraqueza da poesia e dos poetas atuais), e como contemporânea das novas
vanguardas surgidas nos anos 1950, Orides Fontela foi uma poeta antilírica, ao menos no sentido em que, se em sua poesia o eu lírico ainda tem
vez, no entanto tem pouca voz, trocado pelo protagonismo da palavra.
Isto a aproxima, afinal, das vanguardas visualistas, de que o fato de ser
uma renovadora do modernismo deveria afastá-la. Ocorre que a história
das formas não é linear, nem pura. Se há algo que Orides Fontela não
faz, em todo caso, é uma poesia ``abstratizante''.

A poesia ``abstratizante'' é aquela que busca, consciente ou inconscientemente, restringir a condição referente da palavra. Há vários mecanismos para isso, como o uso cifrado, idiossincrático, para
``iniciados'', o vocabulário esdrúxulo, os neologismos injustificados, as
apropriações extraculturais, os estrangeirismos ``de butique'' (ou da
moda) etc.\footnote{``Abstratizante'' porque, apesar dos esforços, não há uma verdadeira
poesia abstrata, à diferença da pintura. A palavra \emph{casa} refere-se
a uma casa ou a todas as casas ou à ideia de casa, mas não pode se
referir a nada, ou melhor, a si mesma, ao contrário de uma mancha de
tinta, que é assemântica, ou nula de qualquer \emph{referência} a algo
que lhe seja externo (\emph{re-ferir}, do grego \emph{foréo}, levar, voltar-se para); e se não
há referência, há abstração, isto é, a separação, o isolamento do signo
(do latim \emph{abstrahere}, separar).} Orides Fontela chama esse tipo de linguagem poética, já
muito difundida em sua época (e hoje dominante na poesia brasileira) de
``barroquista''. Com qualquer nome, trata-se de marca e sintoma de uma
poesia que perdeu a força e o sentido das poéticas ``duras'' das muitas
vanguardas do século passado, daí se tornando incapaz, ou pouco capaz,
de dar conta poeticamente das complexidades do mundo contemporâneo; daí,
enfim, sua relativa irrelevância.

Se Orides Fontela não era ``abstratizante'' nem era visualista, reforça-se a percepção de ter sido uma renovadora do modernismo. O que por sua
vez reforça a importância incontornável de sua obra, pois depois dela e
de sua geração, seria o dilúvio da diluição em meio à grande confusão
contemporânea.\footnote{Naturalmente, sempre há exceções, condição mesma para haver a regra.
Aqui, bastaria citar um nome, Régis Bonvicino, caso quase único de um
poeta cuja linguagem encara de frente a feiura complexa do mundo
contemporâneo. A vasta maioria, por outro lado, decaiu em certo
``autismo'' satisfeito (a ``poesia para mim mesmo'' --- e para os meus
pares), ou nessa outra forma de ``autismo'' poético que é a poesia
``abstratizante''.}

Apesar do referido protagonismo da palavra, há ainda em sua poesia uma
perceptível presença do eu-lírico --- e também de um ``nós'' arriscado.
Pois o fragmentário mundo contemporâneo já não reconhece facilmente
nenhum plural comum, tampouco uma individualidade segura: tudo o que era
sólido se desmanchou no ar, para lembrar a famosa frase de Marx. Daí
esse \emph{eu} e esse \emph{nós}, na poesia de Orides Fontela,
conviverem com uma presença ainda mais forte de imagens de rarefação e
desfazimento, de que a reiterada palavra ``ar'', ao lado de outros
signos de dissolução e antissolidez, como ``pássaro'', são marcas
constantes. Na verdade, se alguma coisa é de fato sólida em sua poesia,
para além da densidade de sua linguagem, é a crua clareza da lucidez.

``Destruição'' aparece já em seu livro de estreia, de 1969:

\begin{verse}
A coisa contra a coisa:\\
a inútil crueldade\\
da análise. O cruel\\
saber que despedaça\\
o ser sabido.

A vida contra a coisa:\\
a violentação\\
da forma, recriando-a\\
em sínteses humanas\\
sábias e inúteis.

A vida contra a vida:\\
a estéril crueldade\\
da luz que se consome\\
desintegrando a essência\\
inutilmente.\footnote{\emph{Transposição} (São Paulo, edição autoral, 1969).}
\end{verse}

No outro extremo de sua obra, um de seus últimos poemas, ``Da poesia''
(1996), sintetiza fortemente sua poética:

\begin{verse}
um\\
gato tenso\\
tocaiando o silêncio\footnote{Poema inédito, resgatado por Gustavo de Castro (ver nota 2).}
\end{verse}

Orides Fontela era uma leitora de Heidegger (a quem lia, segundo suas
próprias palavras, não como filósofo, mas como uma espécie de ``poeta em
prosa''). Mas se for de fato inevitável, ou necessário, aproximar sua
poesia de alguma filosofia, é preciso pensar em Wittgenstein. Pois um dos
temas mais caros a essa poesia é o da relação da palavra com o calar,
com o calado, além daquela das palavras com as coisas e das coisas com o
silêncio: a poesia é, seria ou deveria ser uma possibilidade de
trânsito, de transporte entre tudo isso (daí o referido título de seu
primeiro livro, \emph{Transposição}).

\begin{verse}
Não há perguntas. Selvagem\\
o silêncio cresce, difícil.\footnote{``Esfinge''. In \emph{Rosácea}. São Paulo, Roswita Kempf, 1986.}
\end{verse}

Para além do modernismo brasileiro, a obra de Orides também se reporta
ao alto modernismo internacional, de que William Carlos Williams é um
dos nomes mais fortes. Ao se ler ``Da poesia'', é inevitável pensar em
``\emph{Poem}'': ``\emph{as the cat / climbed over / the top of // the
jamcloset / first the right / forefoot // carefully / then the hind /
stepped down // into the pit of / the empty / flower pot}''.\footnote{Acessível em \textless{}http://goo.gl/A1yeLo\textgreater{} (``o gato
/ ao agarrar-se / no alto do // armário / primeiro a / pata / direita // com cuidado / depois a traseira / depôs // no cavo do / vazio / vaso de flores'' {[}trad. L. D.{]}).} O gato da poesia, para Williams, é altamente
articulado,e aciona uma trama complexa de ressonâncias em seu silêncio
factual. Por exemplo, a palavra \emph{jamcloset}, que abre a segunda
estrofe, ecoa os vários \emph{tt} finais mudos dos versos (a partir da própria palavra-tema, \emph{cat}), para então ecoar na tripla aliteração final,
\emph{pit-empty-pot}. Ao mesmo tempo, a estrutura e o ritmo do poema,
definidos por seus cortes, são também defi- nidos por essas recorrências
sonoras, que os cortes dos versos expõem. Já o gato poético de Orides é
mais tenso, mais contido, mais potência do que
verdadeira possibilidade de realização. Poucos poemas metalinguísticos
poderiam ser mais (poderosamente) sintéticos. E poucos trariam em tão
parca matéria vocabular (apenas seis palavras) tal matéria poética
redivivamente moderna --- isto é, construtivista. A trama sonora não é
menos densa que a de Williams:  está em anagrama em
\emph{iando}, e \emph{tenso} é uma assonância forte de
\emph{silêncio}. Mas Orides diz mais com menos, ou seja, depura as
lições do alto modernismo. É verdade que este, assim como o próprio
modernismo brasileiro, atingiu eventualmente os limites máximos da
síntese, como no famoso ``Amor // humor'' de Oswald, ou como nos poemas
de palavras desmontadas de Cummings. Mas, em primeiro lugar, a síntese
de Orides não fica nada a dever à máxima tensão sintética dos
modernismos; em segundo lugar, ela usa essa tensão/contenção em uma
poesia cuja matéria formal informa e conforma densamente o material
semântico. Em suma, Orides Fontela se apropria das lições mais radicais
dos modernismos e com elas cria uma poesia cuja ambiência não é mais a
da irônica iconoclastia ainda antiburguesa (apesar das simpatias pela
técnica) do início do século \textsc{xx}, mas a do amargo e duro ceticismo do
final desse mesmo século, cujo centro fora dominado pela catástrofe.

\begin{verse}
Deus existir\\
ou não: o mesmo\\
escândalo.
\end{verse}

Notar o corte polissêmico em \emph{mesmo}: pois a pausa evoca a frase
``{[}dá no{]} mesmo'' (Deus existir ou não), antes de a palavra afinal
resultar em adjetivo de ``escândalo''.\footnote{``Teologia \textsc{ii}'', poema inédito.}

O domínio do corte significativo e significante (e por isso tantas vezes
polissêmico) é outra marca da maestria poética de Orides Fontela, e
outra lição do modernismo que ela retoma, revitaliza e repotencializa.
Se o verso livre, o abandono da métrica constante, e mesmo de toda
métrica, dando afinal lugar ao prosaísmo (presente e evidente tanto em
Drummond
quanto em Bandeira, além de dominante em Mário de Andrade, entre outros), foi um dos principais mecanismos da revolução poética modernista,
sua vulgarização e sua diluição se tornaram parte importante do
\emph{laissez-faire} da poesia contemporânea. Entre um e outro, a poesia e a
poética de Orides emergem como um ponto equidistante de adensamento.

``Como as palavras permanecem as mesmas, é por sua disposição que o
estilo é construído. A disposição das palavras é: sintaxe''. Não há nada
de especialmente inovador nesta afirmação de Mallarmé.\footnote{``\emph{Comme les mots demeurent les mêmes, c'est par leur
disposition que le style se construit. Le disposition des mots est: la
syntaxe}''. In Pamela Marie Hoffer. \emph{Reflets Réciproques --- A Prismatic Reading of Stéphane Mallarmé and Hélène Cixous}.
New York, Peter Lang Publishing, 2006, p. 18.} Na verdade, ele
está sendo tão somente etimológico: \emph{sintaxe}, do grego
\emph{syntáxis} (\emph{σύνταξις}), significa ordenação (\emph{táxis})
conjunta (\emph{syn}) --- ou seja, coordenação, organização,
disposição. Daí se entende outra famosa afirmação de Mallarmé: ``Sou
profunda e escrupulosamente sintaxeador''. O poeta é um organizador,
na verdade, um reorganizador das palavras --- e um grande poeta, em
consequência, é um renovador da sintaxe. Pois não se trata mais, ou não
se tratou jamais, de meramente construir belas frases de medida e ritmo
precisos, mas de fazer com as palavras o que a língua e a prosa
correntes não fazem --- sem, no entanto, tornar-se por isso obscuro, o
que seria fácil e, provavelmente, inútil. O próprio Mallarmé conclui
que, por ser ``escrupulosamente sintaxeador'', sua poesia é
``desprovida de obscuridade''.\footnote{``\emph{On s'aperçoit que je suis profondément et scrupuleusement syntaxier, que mon écriture est dépourvue d´obscurité}''. Idem, p. 17.} A renovação criteriosa da sintaxe faz a
poesia, e faz da poesia um ato de renovação compreensível da sintaxe.

\begin{verse}
Mas para que serve o pássaro?\\
Nós o contemplamos inerte.\\
Nós o tocamos no mágico fulgor das penas.\\
De que serve o pássaro se\\
Desnaturado o possuímos?

O que era voo e eis\\
que é concreção letal e cor\\
paralisada, íris silente, nítido,\\
o que era infinito e eis\\
que é peso e forma, verbo fixado, lúdico

O que era pássaro e é\\
o objeto: jogo\\
de uma inocência que\\
o contempla e revive\\
--- criança que tateia\\
no pássaro um\\
esquema de distâncias ---

Mas para que serve o pássaro?

O pássaro não serve. Arrítmicas\\
Brandas asas repousam.
\end{verse}

``O que era pássaro e é / o objeto'': o esperado, ou ``natural'', seria
que se mantivesse a mesma categoria genérica, \emph{pássaro} x
\emph{objeto}, um pássaro tornado um objeto, porém o poema contrapõe o
genérico ao particular, ao antepor o artigo definido ao segundo termo da
equação verbal, \emph{pássaro} x \emph{o objeto}, em uma múltipla
transmutação, de ser vivo em coisa, de condição (ser pássaro) a ser uma
coisa em particular, de ter sido no passado a ser no presente.

A Orides Fontela da última fase, apesar de tudo, talvez reduzisse esse
poema aos seus versos finais: ``O pássaro não serve. Arrítmicas //
Brandas asas repousam''. Mas o fato é que o poema como publicado já
continha, em 1965, todos os elementos de uma linguagem poética potente, feita de um
grande domínio da frase, do ritmo, do vocabulário e das imagens, a ponto
de não poderem ser facilmente aproximados ou subsumidos a qualquer
grande poeta anterior, que a poeta estreante estivesse tentando emular
ou superar. E como tal foram prontamente reconhecidos por Davi Arrigucci
Jr., que leu por acaso o poema em um jornal, e por Antonio Candido, a
quem Arrigucci, então estudante em São Paulo, mostrou-o em seguida. A
publicação do poema pelo jornal \emph{O município}, de São João, levaria
em seguida a jovem poeta para as páginas do Suplemento Literário de
\emph{O Estado de São Paulo}, então editado por Décio de Almeida Prado,
e para a edição de seu primeiro livro. Também a levaria para São Paulo,
para a , para o curso de filosofia, para o contato e a convivência com
parte importante da intelectualidade e da crítica paulistanas e para o
pronto reconhecimento de sua grande relevância para a poesia brasileira
contemporânea (seu terceiro livro, \emph{Alba}, de 1983, traria um
prefácio particularmente elogioso de Antonio Candido --- além de,
incidentalmente, ganhar o Jabuti de 1984).

Nada disso, no entanto, tornaria sua obra minimamente popular, mesmo
considerando o mínimo de popularidade que é o parâmetro habitual da
poesia. Basta comparar seu nome aos de seus contemporâneos Adélia Prado,
Hilda Hilst, Roberto Piva e, principalmente, Paulo Leminski.

Uma das
diferenças é que cada um, a seu modo, soube ou pôde mobilizar um menor ou maior círculo de cultores, não de todo incompreensivelmente. Adélia Prado é uma espécie de palatável Drummond de saias, ou uma
``drummondiana'' feminina-levemente-feminista, e tanto o ser uma coisa
como a outra é um tanto quanto simpático para parte do público e da
crítica; Hilda Hilst construiu, malgrado ela mesma, certo mito de poeta
incompreendida, mas não necessariamente incompreensível, ao contrário,
pois afinal se reportava ao tardorromantismo, com sua mistura de poemas
de amor, misticismo \& mistério, e o tardorromantismo, para parte
importante do público, é, de certa forma, a ``verdadeira'' poesia, a
mais ``genuína'', porque mais próxima da ``alma'' do poeta --- portanto,
da ``essência'' da poesia; Roberto Piva é um caso de certo modo
semelhante, ainda que em chave militantemente contracultural (e também
por isso). Paulo Leminski é um caso à parte e, de alguma maneira, o mais
antitético (e o mais antiteticamente esclarecedor) ao de Orides Fontela.
Há muitas semelhanças entre as duas obras, como certo predomínio do
poema curto ou curtíssimo, a forte presença da metalinguagem, as
referências orientais, a frase lapidar. Mas as diferenças são ainda
maiores, a ponto de torná-los, de fato, verdadeiros antípodas. Para citar um aspecto fundamental,
se uma enunciava a dificuldade, o outro cultuava, ao final das contas, a
facilidade. Orides: ``tudo / será difícil de dizer: / a palavra real /
nunca é suave''. Leminski: ``inverno / primavera / poeta / é quem se
considera''.\footnote{``Leminski hesitaria a vida e a obra inteiras entre o `capricho' e o `relaxo', a densidade e o raso, a verdadeira inteligência e suas
exigências e a pseudoesperteza \emph{pop} e sedutora. {[}\ldots{}{]} Sua
obra é, afinal, dominada pela segunda vertente --- o que, por sua vez,
explica e apoia sua recente popularidade, tanto via internet quanto via
antologia''. L.D. ``Paulo Leminski, o Paulo Coelho da poesia''.
Acessível em \textless{}http://bit.ly/1rw8E9U\textgreater{}.}

Orides Fontela pode, então, ser talvez descrita ou
compreendida como uma espécie de anti-Leminski. Não apenas no rigor e na
relativa dificuldade de sua poesia --- em todo caso, em sua
não-facilidade, apesar da lucidez e da clareza da expressão --- mas
também, e não por acaso, na sua oposta popularidade.

Além disso, Orides Fontela era ainda mais difícil no trato pessoal do
que na própria poesia, como sua biografia demonstra à exaustão.\footnote{Ver nota 2.} E isto
afinal explica seu posterior afastamento de todos os mesmos nomes influentes da crítica que de início prontamente abriram, de forma pouco
usual, os caminhos do meio literário para a jovem poeta do interior de
São Paulo. Se se soma, então, a secura da linguagem à aspereza da
personalidade, tanto a popularidade (minimamente) possível a um poeta
quanto sua presença crítica, humanamente também sensível ao trato (ou
destrato) pessoal --- e não apenas aos intangíveis estratos dos
argumentos ---, o relativo silêncio que hoje cerca, não sem certa
ironia, a obra de uma das mais fortes e importantes poetas brasileiras
contemporâneas, torna-se grandemente explicável --- ainda que
inversamente justificável.

