\textbf{Heinrich Heine} (Düsseldorf, 1797--Paris, 1856) é um dos maiores nomes
da literatura alemã. Seus primeiros poemas são publicados já
em 1817, num jornal de Hamburgo. Em 1824 surge a coletânea lírica
\textit{Trinta e três poemas}, na qual se inclui a canção “Loreley”,
uma das mais célebres de toda a literatura alemã. Neste mesmo ano faz
uma viagem a pé pela região do Harz (norte da Alemanha), e em seguida
visita Goethe em Weimar. Dois anos mais tarde publica a narrativa
\textit{Viagem pelo Harz}, elaboração poética das observações,
experiências e reflexões feitas durante a caminhada. Em outubro de 1827
vem a lume novo volume lírico, \textit{Livro das canções}, acolhido
entusiasticamente pela juventude alemã, e que se torna, ao longo dos
anos, uma fonte de inspiração para vários compositores de
\textit{Lieder} (canções).
Em 1831, após concluir a quarta e última parte de seus \textit{Quadros
de viagem}, emigra para Paris, de onde passa a enviar artigos para um
influente jornal liberal alemão (\textit{Augsburger Allgemeine
Zeitung}). Com suas obras, artigos e intervenções busca promover o
intercâmbio cultural e a aproximação entre a França e a Alemanha, como
atestam as seguintes palavras de Balzac: “Heine representa em Paris o
espírito e a poesia da Alemanha, assim como encarna na Alemanha a
crítica francesa mais viva e espirituosa”. Depois de 12 anos de
ausência, retorna à Alemanha para visitar sua mãe em Hamburgo. Elabora
as impressões de viagem no longo poema \textit{Alemanha, um conto de
inverno}, obra considerada por muitos como a maior sátira da literatura moderna.
Em 1840, o governo francês concede"-lhe uma pensão no valor de 400 francos
mensais, o equivalente ao salário de um professor universitário bem remunerado,
mas apesar de seu prestígio na França, Heine tem a prisão decretada em vários
estados da Alemanha e suas obras são cada vez mais visadas pela censura.
Em 1848, a já abalada saúde do autor piora sensivelmente
e a partir de então se vê preso ao que chamou de sua “cripta de
colchões”, vítima de uma doença degenerativa que provoca dores atrozes,
obrigando"-o a tomar altas doses de morfina. Contudo, sua produção
literária prossegue intensa até os últimos dias de vida. Falece em 17
de fevereiro de 1856 e três dias depois é sepultado no cemitério de
Montmartre.

\textbf{O Rabi de Bacherach} foi concebido inicialmente como projeto de romance histórico,
no início de 1824, durante um período de bastante
contato com a história e a cultura do povo judeu. O ensejo imediato
para a narrativa foi a sua intenção de contrapor"-se à escalada de
antissemitismo que se manifestava então na Alemanha e promover o
intercâmbio e o diálogo entre as culturas alemã e judaica. O enredo
de \textit{O Rabi de Bacherach} é situado no final do século \textsc{xv}, mas o
narrador remonta também a séculos anteriores para tocar nas raízes
históricas do antissemitismo na Alemanha. Apesar de seu entusiasmo
pelo projeto, Heine não conseguiu vencer a amplitude e a aspereza do
assunto, publicando a narrativa, como “fragmento de romance”, em 1840
no 4\oi\,volume da série intitulada \textit{Salon}. Entretanto, mesmo em seu
caráter fragmentário, a obra constitui expressivo exemplo da
arte narrativa de Heine e é o documento mais elucidativo de sua
flutuante relação com o judaísmo.
Os três textos publicados como apêndice enfocam a questão do
fanatismo religioso e foram extraídos do volume \textit{Lutetia}, em
que Heine enfeixou 61 artigos escritos em Paris entre fevereiro de 1840
e maio de 1844, bem como quatro artigos dos anos 1843 a 1846. Esse
volume, que constitui verdadeira obra"-prima da prosa jornalística,
foi publicado em alemão em 1854, com o subtítulo \textit{Relatos sobre
arte, política e vida social}, que recebeu, poucos meses depois, uma edição
francesa (\textit{Lutèce. Lettres sur la Vie politique, artistique et
sociale en France}), para a qual Heine escreve o célebre prefácio em
que comenta o advento do comunismo.

\textbf{Marcus Vinicius Mazzari} é professor de Teoria Literária na Universidade
de São Paulo. Traduziu para o português textos de Walter Benjamin, Bertolt
Brecht, Adelbert von Chamisso, Thomas Mann, Günter Grass, Goethe, entre outros.
Organizou uma edição bilíngue da primeira e segunda partes de \textit{Fausto},
de Goethe (Trad. J.~K.~Segall, Editora 34, 2004/2007). Publicou ainda, entre
outros trabalhos, \textit{Romance de formação em perspectiva histórica}
(Ateliê, 1999) e \textit{Die Danziger Trilogie von Günter Grass. Erzählen gegen
die Dämonisierung deutscher Geschichte} (Berlim, 1994).


