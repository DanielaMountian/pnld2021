\textbf{O HERÓI DA ABOLIÇÃO}

\textbf{A LUTA DE LUIZ GAMA NO IMPÉRIO DO BRASIL}

\textbf{2021}

\textbf{UMA AUTOBIOGRAFIA}

\textbf{*didascália*}

\emph{Quatro textos constituem essa seção: o primeiro é um pequeno
bilhete escrito por Luiz Gama. O segundo, a famosa carta que Gama
escreveu a Lucio de Mendonça. Em terceiro lugar, encontra-se um poema de
Gama dedicado à sua mãe, Luiza Mahin. Por fim, lê-se uma resposta de
Lucio de Mendonça, que é uma transformação da carta em um perfil
biográfico, imediatamente publicado na imprensa. Há muitas nuances para
se debater o conteúdo da carta, entre elas, a razão que levou Gama a
escolher Mendonça como portador da mensagem, digna das melhores páginas
da história do Brasil. No entanto, deixemos debates que poderiam
descambar para pormenores acadêmicos para outra ocasião. Procuremos
aqui, num exercício de criatividade e imaginação, ler a carta como se
fossemos nós mesmos os destinatários dela. Assim, a dimensão privada do
endereçamento perde fôlego. Além disso, resta o que o narrador brilhante
talvez intentasse lá atrás, vislumbrando quiçá a perenidade do texto: a
escrita autobiográfica da experiência de vida, dos tempos, angústias,
sonhos, frustrações, provações, dilemas, conquistas e lutas, do
sofrimento, em suma, de um autor. Em síntese: embora tecnicamente uma
correspondência particular, a carta -- enigmática, cifrada e luminosa
feito "trovão dentro da mata" -- pode ter sido concebida para ganhar,
com o tempo, a dimensão autobiográfica que possui, quando leitor se
permite receber a carta como real destinatário dela. O pacto
escritor-leitor, portanto, ganha novo e original sentido. Tal a mandinga
da carta. E como já disse Gama: "Quem não tem peito não toma mandinga!"
O convite, portanto, é para lermos bilhete, carta e poema, todos de
Gama, e o perfil biográfico produzido pelo primeiro leitor da carta,
Mendonça, como pedrinhas de um mesmo fio de contas. Afinal de contas,
todos nós, quando leitores de uma autobiografia, podemos, em misterioso
vaivém, tomar parte da vida dela, assim como ela toma assento em nossa
própria.}

\textbf{1. BILHETE PARA LÚCIO DE MENDONÇA}

\textbf{*didascália*}

\emph{A famosíssima carta a Lucio de Mendonça era antecedida por esse
bilhete, até hoje desconhecido do grande público. O bom humor abre-alas
para a correspondência histórica. }

***

1880 -- Julho 26 -- à noite.

Lucio.

Abraça-te, e beija-te (sem sacrilégio) o teu,

Luiz Gonzaga Pinto da Gama.

\textbf{2. CARTA A LUCIO DE MENDONÇA}\footnote{. In: Biblioteca
  Nacional, Carta a Lucio de Mendonça, Documento textual, Manuscritos -
  I-2-11,018, São Paulo, 25/07/1880.}

\textbf{*didascália*}

\emph{Sob a perspectiva biográfica, essa é a carta mais significativa da
produção intelectual de Luiz Gama. Repleta de declarações impactantes e
minúcias finíssimas que o mais diligente leitor pode sem querer deixar
escapar -- ao que antecipadamente alerto em vista de redobrar a atenção
--, a Carta a Lucio de Mendonça é uma obra de arte da literatura
brasileira. A narrativa da jornada épica do menino baiano é das coisas
mais impressionantes da história do Brasil. Para se ter uma ideia, Gama,
aos dez anos de idade, atravessa o país no porão de um navio infestado
de ratos e apinhado de mercadorias e pessoas escravizadas, chegando ao
Rio de Janeiro. De lá, ruma, acorrentado em um navio, para Santos,
depois vai à pé para Jundiaí, Campinas e finalmente São Paulo. Luiz Gama
passa, então, oito anos barbaramente escravizado no centro da capital
paulista. De modo enigmático, foge do cativeiro, alcança provas de sua
liberdade e assenta praça na Força Pública, espécie de regimento
policial da época. De lá, o que já era épico tem sua marca confirmada
pelos eventos sincrônicos e seguintes. Insurge-se contra o abuso de
autoridade uma, duas, três -- diversas! -- vezes, aprende a ler e
escrever com maestria, toma posse de empregos públicos reservados
àqueles que possuíam sólido conhecimento normativo e administrativo. E
mais, revela-se um homem de letras -- poeta e jornalista -- e, entre
múltiplas expertises, torna-se um dos mais importantes advogados -- e
juristas! -- já conhecidos no Brasil. A carta, que pode ser lida como
autobiografia, é um monumento à criatividade, à luta e à perseverança da
humanidade negra que, nas palavras do poeta, "fez e faz história
segurando esse país no braço". }

\emph{***}

1880. Julho 25.

Meu caro Lucio,

Recebi o teu cartão com a data de 28 do pretérito.

Não me posso negar ao teu pedido, porque antes quero ser
acoimado\footnote{. Tachado.} de ridículo, em razão de referir verdades
pueris\footnote{. Ingênuas.}, que me dizem respeito, do que de vaidoso e
fátuo\footnote{. Presunçoso.}, pelas ocultar, de envergonhado: aí tens
os apontamentos que me pedes, e que sempre eu os trouxe de memória.

Nasci na cidade de São Salvador, capital da província da Bahia, em um
sobrado da rua do Bângla\footnote{. Optei em grafar exatamente como no
  original, mesmo que a atualização para o português corrente
  requisitasse a mudança para "Bângala", tal como hoje se acha o nome da
  rua, na região do centro histórico de Salvador. A razão para isso é
  porque Gama narra alguns apontamentos que ele "sempre trouxe de
  memória", logo, o nome da rua para ele, tão meticuloso no manejo das
  palavras, seria como trazia de cabeça: "Bângla". Além do mais, tal
  forma de grafar/pronunciar tem implicações para se compreender as
  minúcias e variações das muitas línguas do grupo Bantu, do qual
  possivelmente provenha a palavra.}, formando ângulo interno, em a
quebrada\footnote{. Esquina.}, lado direito de quem parte do adro da
Palma\footnote{. Refere-se à Igreja de Nossa Senhora da Palma, na antiga
  freguesia de Sant'Anna, hoje bairro da Mouraria, Salvador, Bahia.}, na
Freguesia de Sant'Ana, a 21 de Junho de 1830, por as 7 horas da manhã, e
fui batizado, 8 anos depois, na Igreja Matriz do Sacramento, da cidade
de Itaparica\footnote{. A pedido de Sud Mennucci, o cônego Aníbal Matta,
  secretário da Cúria de Salvador, e o padre Clodoaldo Barbosa, além da
  famosa educadora Anfrísia Santiago, reviraram os livros de
  assentamento de batismo da matriz de Itaparica sem, no entanto,
  encontrar "nenhuma criança de oito anos, com o nome de Luiz ou Luiz
  Gonzaga, entre os registros." Eu mesmo revirei linha por linha os
  livros dos arquivos da Cúria de Salvador sem obter maior sucesso que
  Mennucci e sua turma. As muitas hipóteses de análise, que inclusive em
  nada desmerecem a afirmativa de Gama, tornando-a, antes, apenas mais
  complexa de se examinar, são bem mapeadas por Mennucci. Dentre tantas
  conjecturas, algumas possuem verossimilhança maior, sem, contudo,
  serem conclusivas a toda prova. A exata certidão de batismo, defende o
  biógrafo, "só se poderia verificar mediante uma batida completa nos
  livros da Cúria, e referentes a todas as freguesias existentes na
  época, não só da cidade do Salvador, mas também das cidades vizinhas.
  Trabalho para anos...".}.

Sou filho natural de uma negra, Africana-livre\footnote{. Aqui Gama
  provavelmente utiliza uma noção ampla do conceito de africano-livre
  enquanto o africano não-escravizado. Em muitos contextos, tal conceito
  restringe-se aos domínios do campo jurídico, indicando estritamente
  aquele que desembarcou no Brasil após norma proibitiva.}, da
Costa-da-Mina (Nagô de Nação)\footnote{. Nesse contexto, nagô remete a
  um dos povos de língua iorubá e a costa da Mina à região geográfica do
  continente africano, atualmente situada no litoral dos países de Gana,
  Togo e Benim.}, de nome Luiza Mahin\footnote{. A expressão de Lígia
  Ferreira de que Luiz Gama é um "filho que dá luz à mãe" me parece a
  mais acertada possível, afinal, é a partir da \emph{Carta a Lúcio de
  Mendonça} e da poesia \emph{Minha mãe}, que lhe vai anexa, que Gama
  conta os detalhes que se conhece sobre a vida de sua mãe, Luiza Mahin.
  A imaginação histórica que sucede o relato vivo de seu filho é, sem
  dúvida, tema dos mais instigantes, dentre outros campos, da fortuna
  crítica de Gama e da história das lutas populares no Brasil.}, pagã,
que sempre recusou o batismo e a doutrina cristã.

Minha mãe era baixa de estatura, magra, bonita, a cor era de um preto
retinto e sem lustro, tinha os dentes alvíssimos como a neve, era muito
altiva, geniosa, insofrida e vingativa.

Dava-se ao comércio -- era quitandeira --, muito laboriosa; e mais de
uma vez, na Bahia, foi presa, como suspeita de envolver-se em planos de
insurreições de escravos, que não tiveram efeito\footnote{. A década de
  1830 foi especialmente agitada e revoltosa na cidade da Bahia, como
  então era chamada Salvador, a hoje capital do estado da Bahia. O
  Levante dos Malês (1835), por exemplo, um dos maiores e mais perigosos
  para a ordem escravista socialmente constituída, bem expressa a tensão
  dos conflitos políticos da época. Embora não haja citação direta a
  esse evento, o fato de Gama viver na cidade da Bahia justamente nesse
  período, a poucos metros da Ladeira da Praça, epicentro do Levante dos
  Malês, sugere que essa seja uma das "insurreições de escravos" a que
  faz menção em sentido amplo.}.

Era dotada de atividade. Em 1837, depois da Revolução do dr.
Sabino\footnote{. A "revolução do dr. Sabino", também conhecida por
  "Sabinada" em razão da liderança do médico Francisco Sabino
  (1796-1846), possuía pautas republicanas e reivindicava maior
  autonomia da então província da Bahia frente ao Rio de Janeiro, sede
  da administração do Império, assim como a redivisão de poderes locais,
  incluindo grupos com baixa ou nenhuma representação política.}, na
Bahia, veio ela ao Rio de Janeiro, e nunca mais voltou. Procurei-a em
1847, em 1856 e em 1861, na Corte, sem que a pudesse encontrar. Em 1862,
soube, por uns pretos minas que conheciam-na e que deram-me sinais
certos, que ela, apanhada com malungos\footnote{. Companheiros,
  camaradas. No contexto, também pode significar conterrâneo, africano
  da mesma nação.} desordeiros, em uma {casa de dar fortuna}\footnote{.
  Espaço de reunião social, política e religiosa de africanos e negros
  brasileiros. As casas de dar fortuna eram fortemente reprimidas pelas
  polícias locais, como a da Corte, Rio de Janeiro, que devassavam esses
  ambientes por representarem potencial subversão da ordem escravista
  constituída.}, em 1838, fora posta em prisão; e que tanto ela como os
companheiros desapareceram. Era opinião dos meus informantes que esses
{amotinadores}\footnote{. Que provoca motins, revoltas, agitações.}
fossem mandados pôr fora, pelo Governo, que, nesse tempo, tratava
rigorosamente os Africanos-livres, tidos como provocadores.

Nada mais pude alcançar a respeito dela. Nesse ano, de 1861, voltando a
São Paulo, e estando em comissão do Governo, na vila de Caçapava,
dediquei-lhe os versos que, com esta carta, envio-te\footnote{. Trata-se
  do poema \emph{Minha Mãe}, que se lê a seguir.}.

Meu pai, não ouso afirmar que fosse branco, porque tais afirmativas,
neste país, constituem grave perigo perante a verdade, no que concerne à
melindrosa presunção das cores humanas: era fidalgo; e pertencia a uma
das principais famílias da Bahia, de origem portuguesa.

Devo poupar à sua infeliz memória uma injúria dolorosa, e o faço
ocultando o seu nome.

Ele foi rico; e, nesse tempo, muito extremoso para mim: criou-me em seus
braços. Foi revolucionário em 1837. Era apaixonado por a diversão da
pesca e da caça; muito apreciador de bons cavalos; jogava bem as armas,
e muito melhor de baralho, amava as súcias\footnote{. Festanças, farras.}
e os divertimentos: esbanjou uma boa herança, obtida de uma tia em 1836;
e, reduzido à pobreza extrema, a 10 de Novembro de 1840, em companhia de
Luiz Candido Quintella, seu amigo inseparável e hospedeiro, que vivia
dos proventos de uma casa de tavolagem\footnote{. Casa de jogos,
  usualmente de cartas, dados e tabuleiros.}, na cidade da Bahia,
estabelecida em um sobrado de quina, ao largo da praça, vendeu-me, como
seu escravo, a bordo do patacho {Saraiva}..........

Remetido para o Rio de Janeiro nesse mesmo navio, dias depois, que
partiu carregado de escravos, fui, com muitos outros, para a casa de um
cerieiro português de nome Vieira, dono de uma loja de velas, à rua da
Candelária, canto da do Sabão. Era um negociante de estatura baixa,
circunspecto e enérgico, que recebia escravos da Bahia, à comissão.
Tinha um filho aperaltado, que estudava em colégio; e creio que três
filhas já crescidas, muito bondosas, muito meigas, e muito compassivas,
principalmente a mais velha. A senhora Vieira era uma perfeita matrona,
exemplo de candura e piedade. Tinha eu 10 anos. Ela e as filhas
afeiçoaram-se de mim imediatamente.

Eram 5 horas da tarde quando entrei em sua casa.

Mandaram lavar-me; vestiram-me uma camisa e uma saia da filha mais nova,
deram-me de cear e mandaram-me dormir com uma mulata de nome Felícia,
que era mucamba\footnote{. Aparentemente, Gama grafou mucama, mas, como
  se nota em exame mais detalhado, ele próprio corrigiu para mucamba.
  Ambas expressões serviam para designar a função de criada doméstica.}
da casa.

Sempre que me lembro desta boa senhora e das suas filhas, vêm-me as
lágrimas aos olhos; porque tenho saudade do amor e dos cuidados com que
afagaram-me por alguns dias.

Dali saí derramando copioso\footnote{. Abundante.} pranto, e também
todas elas, sentidas de verem-me partir.

Oh, eu tenho lances doridos em minha vida, que valem mais do que as
lendas sentidas da vida armargurada dos mártires.

Nesta casa, em dezembro de 1840, fui vendido ao negociante e
contrabandista alferes\footnote{. Antiga patente militar, abaixo do
  tenente.} Antônio Pereira Cardozo\footnote{. Antonio Pereira Cardozo
  (1791-1861), português, fazendeiro, proprietário da fazenda Cachoeira,
  Lorena (SP), registrado como morador do distrito norte da freguesia da
  Sé, capital, já em 1837. Cf. \emph{O Novo Farol Paulistano},
  08/02/1837, p. 1. Por mais que Gama indique de modo expresso o recorte
  temporal do suicídio de Cardozo como sendo "há oito ou dez anos", o
  fato ocorreu em 1861. Diferente de outras ocasionais passagens em que,
  por lapso ou descuido, Gama confunde datas, as razões para ele indicar
  uma data em mais de dez anos distante da factual não parecem ter sido
  por erro fortuito. Exploro essa questão decisiva para a formação de
  Gama em minha tese de doutorado.}, o mesmo que, há 8 ou 10 anos, sendo
fazendeiro no município de Lorena, nesta Província, no ato de o
prenderem por ter morto alguns escravos à fome, em cárcere privado, e já
na idade maior de 60 a 70 anos, suicidou-se com um tiro de pistola, cuja
bala atravessou-lhe o crânio.

Este alferes Antônio Pereira Cardozo comprou-me em um lote de cento e
tantos escravos; e trouxe-nos a todos, pois que era este o seu negócio,
para vender nesta província.

Como já disse, tinha eu apenas 10 anos; e, à pé, fiz toda a viagem de
Santos até Campinas.

Fui escolhido por muitos compradores, nesta cidade, em Jundiaí\footnote{.
  Jundiaí, município paulista que fica 50 km distante de São Paulo (SP),
  era a principal cidade ao limite norte da capital.} e Campinas; e por
todos repelido, como se repelem as cousas ruins, pelo simples fato de
ser eu {baiano}...

Valeu-me a pecha!...

O último recusante foi o venerando e simpático ancião Francisco Egydio
de Souza Aranha\footnote{. Francisco Egydio de Souza Aranha (1778-1860),
  santista, senhor de engenho em Campinas, foi um dos introdutores da
  cultura cafeeira naquela cidade. Em seu testamento, datado do ano de
  1859, Francisco Egydio declarava ser proprietário de 356 escravos. Cf.
  Maria Alice Rosa Ribeiro. \emph{Açúcar, café, escravos e dinheiro a
  prêmio: Campinas, 1817-1861}, 2015.}, pai do exmo. conde de Três Rios,
meu respeitável amigo.

Este, depois de haver-me escolhido, afagando-me, disse:

-- Há de ser um bom pajem para os meus meninos; dize-me: onde nasceste?

-- Na Bahia, respondi eu.

-- {Baiano}!?... exclamou, admirado, o excelente velho. Nem de graça o
quero. Já não foi por bom que o venderam tão pequeno!....

Repelido {como refugo}, com outro escravo da Bahia, de nome José,
sapateiro, voltei para casa do sr. Cardozo, nesta cidade, à rua do
Comércio\footnote{. Antiga rua do centro de São Paulo, atualmente
  denominada de rua Álvares de Azevedo.}, nº 2, sobrado, perto da Igreja
da Misericórdia\footnote{. A Igreja da Misericórdia, situada no antigo
  largo da Misericórdia, foi construída em 1716 e demolida em 1886. Foi
  um ponto nevrálgico de circulação, comércio e abastecimento de água da
  cidade de São Paulo dos séculos XVIII e XIX.}.

Aí aprendi a copeiro\footnote{. Indivíduo que se ocupa do serviço da
  copa, serve a mesa e faz outros serviços domésticos.}, a sapateiro, a
lavar e a engomar roupa, e a costura.

Em 1847, contava eu 17 anos, quando para a casa do sr. Cardozo veio
morar, como hóspede, para estudar humanidades, tendo deixado a cidade de
Campinas, onde morava, o menino Antônio Rodrigues do Prado Júnior, hoje
doutor em direito, ex-magistrado de elevados méritos, e residente em
Mogi Guaçu\footnote{. Município do interior paulista, distante 160 km da
  capital que, ao final do século XIX, possuía grandes fazendas de café
  e concentração de gente escravizada.}, onde é fazendeiro.

Fizemos amizade íntima, de irmãos diletos, e ele começou de ensinar-me
as primeiras letras.

Em 1848, sabendo eu ler e contar alguma cousa, e tendo obtido ardilosa e
secretamente provas inconcussas\footnote{. Incontestáveis, irrefutáveis.}
de minha liberdade, retirei-me fugido da casa do alferes Antônio Pereira
Cardozo, que aliás votava-me a maior estima, e fui assentar praça. Servi
até 1854, seis anos; cheguei a cabo-de-esquadra graduado\footnote{.
  Antiga patente militar que comandava um coletivo de soldados, cabos e
  recrutas.}, e tive baixa do serviço, depois de responder a conselho
por atos de suposta insubordinação, quando eu tinha limitado-me a
ameaçar um oficial insolente, que me havia insultado, e que soube
conter-se.

Estive então preso 39 dias, de 1º de julho a 9 de agosto\footnote{. Ver,
  nesse volume, \emph{Carta -- Recreio D'Amizade}.}. Passava os dias
lendo e às noites: sofria de insônias; e, de contínuo, tinha diante dos
olhos a imagem de minha querida mãe. Uma noite, eram mais de duas horas;
eu dormitava; e, em sonho, vi que a levavam presa. Pareceu-me ouvi-la
distintamente, que chamava por mim.

Dei um grito, espavorido saltei fora da tarimba; os companheiros
alvorotaram-se; corri à grade, enfiei a cabeça pelo xadrez\footnote{.
  Cela, cadeia.}.

Era solitário e silencioso o longo e lôbrego\footnote{. Diz-se do lugar
  sombrio, escuro, em que quase não há claridade.} corredor da prisão,
mal alumiado, e do seio do qual pendia a luz amarelenta de enfumaçada
lanterna.

Voltei para minha esteira, narrei a ocorrência aos curiosos colegas;
eles narraram-me fatos semelhantes; eu caí em nostalgia, chorei e dormi.

Durante o meu tempo de praça, nas horas vagas, fiz-me copista; escrevia
para o escritório do Escrivão Major Benedicto Antônio Coelho Netto, que
tornou-se meu Amigo; e que hoje, pelo seu merecimento, desempenha o
cargo de oficial-maior da Secretaria do Governo; e, como
amanuense\footnote{. Funcionário de repartição pública que geralmente
  fazia cópias, registros e tratava da correspondência.}, no gabinete do
exmo. sr. conselheiro Francisco Maria de Sousa Furtado de
Mendonça\footnote{. Francisco Maria de Sousa Furtado de Mendonça
  (1812-1890), nascido em Luanda, Angola, foi subdelegado, delegado,
  chefe de polícia e secretário de polícia da província de São Paulo ao
  longo de quatro décadas. Foi, também, professor catedrático de Direito
  Administrativo da Faculdade de Direito de São Paulo. A relação de Luiz
  Gama com Furtado de Mendonça é bastante complexa, escapando, em muito,
  aos limites dos eventos da demissão de Gama do cargo de amanuense da
  secretaria de polícia, em 1869. Para que se ilustre temporalmente a
  relação, tenhamos em vista que à época do rompimento público, aos
  finais da década de 1860, ambos já se conheciam e trabalhavam juntos
  há coisa de duas décadas; e, mais, Gama não rompeu definitivamente com
  Furtado de Mendonça, como erroneamente indica a historiografia, visto
  que em 1879 publicou o artigo \emph{Aos homens de bem}, defesa moral e
  política explícita do legado de Furtado de Mendonça.}, que aqui
exerceu, por muitos anos, com aplausos e admiração do público em geral,
altos cargos de administração, polícia e judicatura, e que é catedrático
da Faculdade de Direito, fui seu ordenança\footnote{. Nesse caso,
  soldado às ordens pessoais de uma autoridade a quem acompanha durante
  as horas do expediente.}; por meu caráter, por minha atividade e por
meu comportamento, conquistei a sua estima e a sua proteção; e as boas
lições de letras e de civismo, que conservo com orgulho.

Em 1856, depois de haver servido como escrivão perante diversas
autoridades policiais, fui nomeado amanuense da Secretaria de Polícia,
onde servi até 1869\footnote{. Por equívoco de datas, no original se lê
  1868, quando a demissão de fato ocorreu em 1869.}, época em que, por
{turbulento} e {sedicioso}\footnote{. Insubordinado, indisciplinado.}{,}
fui demitido {a bem do serviço público}, pelos conservadores, que então
haviam subido ao poder. A portaria de demissão foi lavrada pelo dr.
Antônio Manuel dos Reis, meu particular amigo, então secretário da
polícia, e assinada pelo exmo. dr. Vicente Ferreira da Silva
Bueno\footnote{. Vicente Ferreira da Silva Bueno (1815-1873) teve longa
  carreira administrativo-judiciária, exercendo cargos de delegado de
  polícia, juiz municipal, juiz dos órfãos, juiz de direito e
  desembargador em diversas províncias, como Bahia, Paraná, São Paulo e
  Rio de Janeiro. Em 1869, era chefe de polícia interino da província de
  São Paulo, cabendo a ele papel de algoz no espetáculo da demissão de
  Luiz Gama do cargo de amanuense da Secretaria de Polícia.}, que, por
este e outros atos semelhantes, foi nomeado desembargador da Relação da
Corte\footnote{. Refere-se ao Tribunal da Relação da Corte, equivalente
  à segunda instância judiciária da antiga jurisdição da Corte.}.

A turbulência consistia em fazer eu parte do Partido Liberal; e, pela
imprensa e pelas urnas, pugnar pela vitória das suas e minhas ideias; e
promover processos em favor de pessoas livres, criminosamente
escravizadas; e auxiliar licitamente, na medida de meus esforços,
alforrias de escravos, porque detesto o cativeiro e todos os senhores,
principalmente os Reis.

Desde que fiz-me soldado, comecei a ser homem; porque até os 10 anos fui
criança; dos 10 anos aos 18 fui \sout{escravo} soldado.

Fiz versos; escrevi para muitos jornais; colaborei em outros, literários
e políticos, e redigi alguns.

Agora chego ao período em que, meu caro Lucio, nos encontramos no
\emph{Ypiranga}, à rua do Carmo\footnote{. Antiga rua do centro de São
  Paulo.}, tu como tipógrafo\footnote{. Indivíduo que faz serviços
  tipográficos de composição, paginação ou impressão.}, poeta, tradutor,
folhetinista\footnote{. Que escreve folhetins -- novelas ou crítica de
  literatura e artes -- para jornais.} principiante; e eu como simples
aprendiz-compositor\footnote{. Encarregado de compor originais de texto
  em tipografia.}, de onde saí para o foro e para a tribuna, onde ganho
o pão para mim e para os meus, que são todos os pobres, todos os
infelizes; e para os míseros escravos, que, em número superior a 500,
tenho arrancado às garras do crime.

Eis o que te posso dizer, às pressas, sem importância e sem valor; menos
para ti, que me estimas deveras.

Teu Luiz.

\textbf{3. MINHA MÃE}\footnote{. O poema \emph{Minha Mãe}, conforme
  conta Gama, foi escrito em 1861, quando ele se encontrava em trabalho
  na vila de Caçapava, vale do Paraíba (SP). Esse poema foi publicado já
  em 1861, na segunda edição das \emph{Primeiras Trovas Burlescas de
  Getulino}.}

\textbf{*didascália*}

\emph{Como dito na carta, Gama juntou esse poema à histórica mensagem
que ganharia o mundo como sua autobiografia. }

\emph{***}

\emph{Minha mãe era mui bela}

\emph{-- Eu me lembro tanto d'ela}

\emph{De tudo quanto era seu!}

\emph{Tenho em meu peito guardadas,}

\emph{Suas palavras sagradas}

\emph{C'os risos que ela me deu.}

Junqueira Freire\footnote{. Luís José Junqueira Freire (1832--1855),
  natural de Salvador (BA), monge beneditino, sacerdote e poeta. A
  epígrafe confere exatamente com trecho do poema \emph{A órfã na
  costura}, publicado no livro \emph{Inspirações do claustro} (1855).}

Era mui bela e formosa,

Era a mais linda pretinha,

Da adusta\footnote{. Quente, fervente.} Líbia rainha,

E no Brasil pobre escrava!

Oh, que saudade que eu tenho

Dos seus mimosos carinhos,

Quando c'os tenros filhinhos

Ela sorrindo brincava.

Éramos dois -- seus cuidados,

Sonhos de sua alma bela;

Ela a palmeira singela,

Na fulva\footnote{. De cor amarelo-ouro.} areia nascida.

Nos roliços braços de ébano,

De amor o fruto apertava,

E à nossa boca juntava

Um beijo seu, que era vida.

Quando o prazer entreabria

Seus lábios de roixo lírio,

Ela fingia o martírio

Nas trevas da solidão.

Os alvos dentes nevados

Da liberdade eram mito,

No rosto a dor do aflito,

Negra a cor da escravidão.

Os olhos negros, altivos,

Dois astros eram luzentes;

Eram estrelas cadentes

Por corpo humano sustidas.

Foram espelhos brilhantes

Da nossa vida primeira,

Foram a luz derradeira

Das nossas crenças perdidas.

Tão terna como a saudade

No frio chão das campinas,

Tão meiga como as boninas\footnote{. Flores também conhecidas por
  maravilha e bela-margaridas.}

Aos raios do sol de Abril.

No gesto grave e sombria,

Como a vaga que flutua,

Plácida a mente -- era a Lua

Refletindo em Céus de Anil.

Suave o gênio, qual rosa

Ao despontar da alvorada,

Quando treme enamorada

Ao sopro d'aura fagueira.

Brandinha a voz sonorosa,

Sentida como a Rolinha,

Gemendo triste sozinha,

Ao som da aragem faceira.

Escuro e ledo o semblante,

De encantos sorria a fronte,

-- Baça\footnote{. Fosca, sem brilho.} nuvem no horizonte

Das ondas surgindo à flor;

Tinha o coração de santa,

Era seu peito de Arcanjo,

Mais pura n'alma que um Anjo,

Aos pés de seu Criador.

Se junto à Cruz penitente,

A Deus orava contrita,

Tinha uma prece infinita

Como o dobrar do sineiro;

As lágrimas que brotavam

Eram pérolas sentidas,

Dos lindos olhos vertidas

Na terra do cativeiro.

\textbf{4. LUIZ GAMA {[}POR LUCIO DE MENDONÇA{]}}\footnote{. In.
  \emph{Gazeta da Tarde} (RJ), Folhetim, 15/12/1880, pp. 1-2.}

\textbf{*didascália*}

\emph{Por cinco décadas, o perfil biográfico escrito por Mendonça foi o
único relato conhecido da história de vida de Luiz Gama. Publicado num
almanaque paulistano como "Biografia" e num jornal do Rio de Janeiro
como "Folhetim", ambas publicações ganharam as ruas no final do ano de
1880. A homenagem, portanto, deu-se com Luiz Gama em vida, que nada
censurou ou emendou no conteúdo lançado. O texto de Mendonça repete
algumas passagens da carta de Gama, reelabora outras e acrescenta
pontualmente informações que ele próprio testemunhou. É um documento de
primeira importância para os estudos sobre a vida e a obra de Luiz Gama.
}

\emph{***}

I

Os republicanos brasileiros, a toda a hora abocanhados pela recordação
injuriosa de meia dúzia de apostasias\footnote{. Espécie de desistência,
  abandono de uma causa política, no contexto, da defesa da bandeira
  republicana.}, das que negrejam na crônica de todos os partidos, se
quisessem com um nome só, que é um alto exemplo de honrada perseverança,
tapar a boca aos detratores, podia lançar-lhes o belo e puro nome que
coroa esta página. Quantos outros iguais oferecem porventura, desde o
começo de sua existência, os nossos velhos partidos
monárquicos?\footnote{. Refere-se ao Partido Conservador e ao Partido
  Liberal que, subordinados ao imperador, se alternavam no exercício do
  poder político do Executivo e do Legislativo.}

Faz-se em duas palavras o elogio deste homem verdadeiramente grande,
grande neste tempo em que só o podem ser os amigos da humanidade;
nascido e criado escravo até a primeira juventude, tem depois alcançado
a liberdade a mais de quinhentos escravos!

À nobre província de S. Paulo, que hoje o estima entre os seus melhores
cidadãos, e que ele preza com o entusiasmo que lhe inspiram todas as
grandezas democráticas, presumo que há de ser grato ler, em um livro que
é particularmente seu, a biografia, já hoje gloriosa, deste bom
republicano.

Se chegar a cumprir-se, como eu espero e desejo, o seu elevado destino,
possam ser estas linhas obscuras fiel subsídio para cronistas de
melhores dias!

II

Nasceu Luiz Gonzaga Pinto da Gama na cidade de S. Salvador da Bahia, à
rua do Bângala\footnote{. A rua do Bângala está localizada no bairro da
  Mouraria, próxima do centro histórico de Salvador (BA).}, em 21 de
Junho de 1830, pelas 7 horas da manhã; e foi batizado, oito anos depois,
na igreja matriz do Sacramento, da cidade de Itaparica.\footnote{. A
  pedido de Sud Mennucci, o cônego Aníbal Matta, secretário da Cúria de
  Salvador, e o padre Clodoaldo Barbosa, além da famosa educadora
  Anfrísia Santiago, reviraram os livros de assentamento de batismo da
  matriz de Itaparica sem, no entanto, encontrar "nenhuma criança de
  oito anos, com o nome de Luiz ou Luiz Gonzaga, entre os registros." Eu
  mesmo revirei linha por linha os livros dos arquivos da Cúria de
  Salvador sem obter maior sucesso que Mennucci e sua turma. As muitas
  hipóteses de análise, que inclusive em nada desmerecem a afirmativa de
  Gama, tornando-a, antes, apenas mais complexa de se examinar, são bem
  mapeadas por Mennucci. Dentre tantas conjecturas, algumas possuem
  verossimilhança maior, sem, contudo, serem conclusivas a toda prova. A
  exata certidão de batismo, defende o biógrafo, "só se poderia
  verificar mediante uma batida completa nos livros da Cúria, e
  referentes a todas as freguesias existentes na época, não só da cidade
  do Salvador, mas também das cidades vizinhas. Trabalho para anos...".}

É filho natural de uma negra, africana livre\footnote{. Em sentido
  amplo, trata-se da africana não-escravizada ou liberta. Em muitos
  contextos, tal conceito restringe-se aos domínios do campo jurídico,
  indicando estritamente o africano que desembarcou no Brasil após norma
  proibitiva do tráfico de escravos.}, da Costa de Mina, de nação
Nagô\footnote{. Nesse contexto, nagô remete a um dos povos de língua
  iorubá e a costa da Mina à região geográfica do continente africano,
  atualmente situada no litoral dos países de Gana, Togo e Benim.}, de
nome Luiza Mahin\footnote{. A expressão de Lígia Ferreira de que Luiz
  Gama é um "filho que dá luz à mãe" me parece a mais acertada possível,
  afinal, é a partir da \emph{Carta a Lúcio de Mendonça} e da poesia
  \emph{Minha mãe}, que lhe vai anexa, que Gama conta os detalhes que se
  conhece sobre a vida de sua mãe, Luiza Mahin. A imaginação histórica
  que sucede o relato vivo de seu filho é, sem dúvida, tema dos mais
  instigantes, dentre outros campos, da fortuna crítica de Gama e da
  história das lutas populares no Brasil.}, pagã: recusou esta sempre
batizar-se e de modo algum converter-se ao cristianismo. Era mulher
baixa de estatura, magra, bonita, de um preto retinto e sem lustro;
tinha os dentes alvíssimos; era imperiosa, de gênio violento, insofrida,
e vingativa; de

..... olhos negros, altivos,

..................................

No gesto grave e sombria.

Era quitandeira, muito laboriosa. Mais de uma vez, na Bahia, foi presa,
por suspeita de envolver-se em planos de insurreições de escravos, que
não tiveram efeito.\footnote{. A década de 1830 foi especialmente
  agitada e revoltosa na cidade da Bahia, como então era chamada
  Salvador, a hoje capital do estado da Bahia. O Levante dos Malês
  (1835), por exemplo, um dos maiores e mais perigosos para a ordem
  escravista socialmente constituída, bem expressa a tensão dos
  conflitos políticos da época. Embora não haja citação direta a esse
  evento, o fato de Gama viver na cidade da Bahia justamente nesse
  período, a poucos metros da Ladeira da Praça, epicentro do Levante dos
  Malês, sugere que essa seja uma das "insurreições de escravos" a que
  faz menção em sentido amplo.} Em 1837, depois da revolução do dr.
Sabino\footnote{. A "revolução do dr. Sabino", também conhecida por
  "Sabinada" em razão da liderança do médico Francisco Sabino
  (1796-1846), possuía pautas republicanas e reivindicava maior
  autonomia da então província da Bahia frente ao Rio de Janeiro, sede
  da administração do Império, assim como a redivisão de poderes locais,
  incluindo grupos com baixa ou nenhuma representação política.},
naquela província, veio ao Rio de Janeiro, e nunca mais voltou.
Procurou-a o filho em 1847, em 1856 e 1861, na Corte, sem que a pudesse
encontrar; em 1862 soube, por uns pretos minas que a conheciam e dela
deram sinais certos, que, apanhada com malungos desordeiros, em uma
\emph{casa de dar fortuna}\footnote{. Espaço de reunião social, política
  e religiosa de africanos e negros brasileiros. As casas de dar fortuna
  eram fortemente reprimidas pelas polícias locais, como a da Corte, Rio
  de Janeiro, que devassavam esses ambientes por representarem potencial
  subversão da ordem escravista constituída.}, em 1838, fora posta em
prisão, e que tanto ela como os companheiros desapareceram. Era opinião
dos informantes que os amotinadores houvessem sido deportados pelo
governo, que nesse tempo tratava rigorosamente os africanos livres,
tidos como provocadores.

Nada mais, até hoje, pôde Luiz alcançar a respeito de sua mãe. Naquele
mesmo ano de 1861, voltando a S. Paulo, e estando em comissão do
governo, na então vila de Caçapava, consagrou à mãe perdida os saudosos
versos que se leem, como nota de um sentimentalismo dissonante, no
risonho livro das \emph{Trovas Burlescas}, que deu à lume com o
pseudônimo de Getulino\footnote{. Em 1859, Luiz Gama publicou suas
  \emph{Primeiras Trovas Burlescas}, obra corrigida e aumentada em 1861,
  quando incluiu, na segunda edição, o poema \emph{Minha mãe}. A adoção
  do pseudônimo Getulino remete provavelmente à Getúlia, território ao
  norte da África.}.

Vê-se que é hereditário em Luiz Gama o profundo sentimento de
insurreição e liberdade. Abençoado sejas, nobre ventre africano, que
deste ao mundo um filho predestinado, em quem transfundiste, com o teu
sangue selvagem, a energia indômita que havia de libertar centenas de
cativos!

O pai de Luiz -- outra analogia deste com Spartacus\footnote{. Spartacus
  (109 a.C-71 a.C) foi um gladiador-general, estrategista e líder
  popular que escapou da escravidão a que era submetido e, num levante
  de grandes proporções, organizou um exército que enfrentou o poder
  central de Roma na Terceira Guerra Servil (73 a.C-71 a.C). Gama citou
  Spartacus por diversos escritos, o que revelava sua admiração e até
  mesmo veneração pela história do mártir que venceu o cativeiro e lutou
  pelo fim da escravidão na Roma Antiga.} -- era nobre, fidalgo, de uma
das principais famílias baianas, de origem portuguesa. Foi rico, e,
nesse tempo, extremoso para o filho: criou-o nos braços. Foi
revolucionário em 1837. Era apaixonado pela pesca e pela caça; gostava
dos bons cavalos; jogava bem as armas, e melhor as cartas: comprazia-se
em folguedos e orgias: esbanjou uma boa herança, havida de uma tia, em
1836. Reduzido à pobreza extrema, em 10 de Novembro de 1840, em
companhia de Luiz Candido Quintella, seu amigo inseparável, que vivia
dos proventos de uma casa de tavolagem na Bahia, vendeu o filho, como
seu escravo, a bordo do patacho \emph{Saraiva}.

Não sei se o desgraçado ainda vive, nem lhe conheço o nome, que Luiz
oculta generoso aos amigos mais íntimos; mas, ainda que jogador e
fidalgo, a recordação da monstruosa infâmia deve ter-lhe esbofeteado, em
todo o resto de seus dias, a velhice desonrada.

III

Remetido dias depois para o Rio de Janeiro, no mesmo navio que partiu
carregado de escravos, foi Luiz, com muitos outros, para a casa de um
cerieiro português, de nome Vieira, estabelecido com loja de velas à rua
da Candelária, esquina da do Sabão. Era um negociante de estatura baixa,
circunspecto e enérgico, que recebia escravos da Bahia, à comissão.
Tinha, além de um filho peralta que estudava em colégio, umas filhas já
crescidas, muito compassivas e meigas; a senhora de Vieira era uma
perfeita matrona, cheia de piedade. Tinha então Luiz 10 anos. Todas as
mulheres da casa se lhe afeiçoaram imediatamente. Eram 5 horas da tarde
quando lhes entrou em casa; mandaram-o lavar; vestiram-lhe uma camisa e
uma saia da filha mais nova, deram-lhe de cear, e mandaram-o dormir em
boa cama.

Ainda hoje Luiz Gama, que é um dos melhores corações que eu conheço,
lembra-se comovido daquela boa gente que o recebeu com tanto afago.

Mas foi por poucos dias: dali saiu logo depois, chorando amargamente e
deixando as suas boas amigas chorosas também de o verem ir.

Era em 1840; foi vendido, naquela casa, ao negociante e contrabandista
alferes Antonio Pereira Cardozo\footnote{. Antonio Pereira Cardozo
  (1791-1861), português, fazendeiro, proprietário da fazenda Cachoeira,
  Lorena (SP), registrado como morador do distrito norte da freguesia da
  Sé, capital, já em 1837. Cf. \emph{O Novo Farol Paulistano},
  08/02/1837, p. 1. Por mais que Gama indique de modo expresso o recorte
  temporal do suicídio de Cardozo como sendo "há oito ou dez anos", o
  fato ocorreu em 1861. Diferente de outras ocasionais passagens em que,
  por lapso ou descuido, Gama confunde datas, as razões para ele indicar
  uma data em mais de dez anos distante da factual não parecem ter sido
  por erro fortuito. Exploro essa questão decisiva para a formação de
  Gama em minha tese de doutorado.}, o mesmo que, há oito ou dez anos,
sendo fazendeiro no município de Lorena, na província de S. Paulo, no
ato de o prenderem, por haver matado à fome alguns escravos em cárcere
privado, já velho de setenta anos, suicidou-se, atravessando o crânio
com uma bala de pistola.\footnote{. Foi na fazenda Cachoeira o cenário
  do suicídio do alferes Cardozo, crime que marcou a história do
  município e a memória de Gama, conforme ele conta na \emph{Carta a
  Lucio de Mendonça}.}

O alferes Cardozo comprou Luiz em um lote de cento e tantos escravos, e
levou-os todos, pois tal era o seu comércio, a vender para a província
de S. Paulo.

À pé, com 10 anos de idade, fez Luiz toda a viagem de Santos até
Campinas. Escravo, saído de uma infância trágica, descalço, desamparado,
faminto, subiu entre um bando de escravos aquela áspera serra do
Cubatão, por onde, anos depois, não há muitos anos, lembra-me que
passamos juntos os dois, eu estudante que voltava para as aulas, ele
advogado que voltava da Corte, abastado, jovial e forte, com um cesto de
frutas para a família, repotreado no assento macio de um dos ricos
vagões da companhia inglesa.

Foi escolhido por muitos compradores, na capital paulista, em
Jundiaí\footnote{. Jundiaí, município paulista que fica 50 km distante
  de São Paulo (SP), era a principal cidade ao limite norte da capital.},
em Campinas, e por todos rejeitado, como se rejeitam as cousas ruins,
pela circunstância de ser \emph{bahiano.}

O último que o enjeitou foi o respeitável ancião Francisco Egydio de
Souza Aranha, pai do sr. conde de Três Rios\footnote{. Francisco Egydio
  de Souza Aranha (1778-1860), santista, senhor de engenho em Campinas,
  foi um dos introdutores da cultura cafeeira naquela cidade. Em seu
  testamento, de 1859, Francisco Egydio declarava ser proprietário de
  356 escravos. Cf. Maria Alice Rosa Ribeiro. \emph{Açucar, café,
  escravos e dinheiro a prêmio: Campinas, 1817-1861}, 2015.}. Depois de
o haver escolhido, afagou-o, dizendo:

-- Está um bom pajem para os meus pequenos.

E perguntou-lhe:

-- Onde nasceste?

-- Na Bahia.

-- \emph{Baiano!...} exclamou, admirado, o excelente velho. Nem de
graça! Já não foi por bom que o venderam tão pequeno!...

O sr. conde de Três Rios, que esteve a ponto de ter Luiz para pajem,
tem-no hoje como um de seus amigos mais considerados.

Enjeitado como \emph{refugo}, com outro escravo baiano, de nome José,
sapateiro, voltou para a casa de Cardoso, na cidade de S. Paulo, à rua
do Comércio, nº 2, sobrado, perto da igreja da Misericórdia.

Ali aprendeu a copeiro, a sapateiro, a lavar e engomar, e a costura.

Em 1847, tinha Luiz 17 anos, quando para a casa de Cardoso veio morar
como hóspede, para estudar humanidades, o menino Antonio Rodrigues do
Prado Junior, hoje doutor em direito, o qual já foi magistrado de muito
mérito, e reside agora em Mogi-Guaçu, onde é fazendeiro.

Travaram amizade estreita, de irmãos, e com o estudante entrou Luiz a
aprender as primeiras letras. Em 1848, sabendo ler, escrever e contar
alguma cousa, e havendo obtido ardilosa e secretamente provas
inconcussas de sua liberdade, retirou-se, fugindo, da casa do alferes
Cardoso, que aliás o tinha na maior estima, e foi assentar praça.

Termina aqui o período do seu cativeiro.

IV

Serviu como soldado até 1854, seis anos; chegou a cabo de esquadra
graduado, e teve baixa do serviço, depois de responder a conselho, por
atos de suposta insubordinação, quando limitara-se a ameaçar um oficial
insolente, que o insultara, e que soube conter-se. Esteve preso o cabo
de esquadra Luiz Gama, de 1º de Julho a 9 de Agosto, trinta e nove dias,
que passou em leitura constante.

Durante o seu tempo de praça, nas horas vagas, fez-se copista; escrevia
para o cartório do escrivão major Benedicto Antonio Coelho Netto, que se
tornou seu amigo; e daí, sem dúvida, lhe nasceu a inclinação para o
foro.

Serviu também como amanuense\footnote{. Funcionário de repartição
  pública que geralmente fazia cópias, registros e tratava da
  correspondência.} no gabinete do conselheiro Francisco Maria de Souza
Furtado de Mendonça\footnote{. Francisco Maria de Sousa Furtado de
  Mendonça (1812-1890), nascido em Luanda, Angola, foi subdelegado,
  delegado, chefe de polícia e secretário de polícia da província de São
  Paulo ao longo de quatro décadas. Foi, também, professor catedrático
  de Direito Administrativo da Faculdade de Direito de São Paulo. A
  relação de Luiz Gama com Furtado de Mendonça é bastante complexa,
  escapando, em muito, aos limites dos eventos da demissão de Gama do
  cargo de amanuense da secretaria de polícia, em 1869. Para que se
  ilustre temporalmente a relação, tenhamos em vista que à época do
  rompimento público, aos finais da década de 1860, ambos já se
  conheciam e trabalhavam juntos há coisa de duas décadas; e, mais, Gama
  não rompeu definitivamente com Furtado de Mendonça, como erroneamente
  indica a historiografia, visto que em 1879 publicou o artigo \emph{Aos
  homens de bem}, defesa moral e política explícita do legado de Furtado
  de Mendonça.}, que por longos anos exerceu na capital de S. Paulo
altos cargos administrativos, e é ainda hoje catedrático na Faculdade de
Direito. Luiz foi sempre seu ordenança\footnote{. Nesse caso, soldado às
  ordens pessoais de uma autoridade a quem acompanha durante as horas do
  expediente.}, e pelo seu vivo talento, pela sua atividade e bom
proceder, mereceu-lhe toda a estima e proteção, e dele recebeu
proveitosas lições de letras.

Em 1856, depois de haver servido como escrivão perante diversas
autoridades policiais, foi nomeado amanuense da secretaria da polícia,
onde esteve até 1868, época em que, \emph{turbulento} e
\emph{sedicioso}\footnote{. Insubordinado, indisciplinado.}, foi
demitido, \emph{a bem do serviço público}, pela reação conservadora. A
portaria de demissão foi lavrada pelo dr. Antonio Manoel dos Reis, seu
dedicado amigo e ainda mais dedicado católico, então secretário da
polícia, e assinada pelo dr. Vicente Ferreira da Silva Bueno\footnote{.
  Vicente Ferreira da Silva Bueno (1815-1873) teve longa carreira
  administrativo-judiciária, exercendo cargos de delegado de polícia,
  juiz municipal, juiz dos órfãos, juiz de direito e desembargador em
  diversas províncias, como Bahia, Paraná, São Paulo e Rio de Janeiro.
  Em 1869, era chefe de polícia interino da província de São Paulo,
  cabendo a ele papel de algoz no espetáculo da demissão de Luiz Gama do
  cargo de amanuense da Secretaria de Polícia.}, que, por este e
semelhantes atos, foi escolhido desembargador da Relação da
Corte.\footnote{. Refere-se ao Tribunal da Relação da Corte, equivalente
  à segunda instância judiciária da antiga jurisdição da Corte.}

A turbulência de Luiz Gama consistia em ser liberal exaltado e
militante, em promover pelos meios judiciais a liberdade de pessoas
livres reduzidas a criminoso cativeiro, e auxiliar alforrias de
escravos, na medida de suas posses, e às vezes, além delas, na medida de
sua dedicação à causa santa dos oprimidos.

V

Nesse ano de 1868, conheci Luiz Gama. Vi-o, se bem me lembra, a primeira
vez, na tipografia do diário liberal \emph{O Ypiranga}, de propriedade e
redação de meu irmão Salvador de Mendonça e do dr. José Maria de
Andrade\footnote{. José Maria de Andrade (s.d.-s.d.), nascido em São
  Paulo (SP), foi escrivão do Tribunal da Relação, promotor, juiz
  municipal e secretário de polícia da província de São Paulo. Como
  registra a crônica da academia de direito paulistana, e o parecer
  supra indica, Andrade foi sócio do escritório dos Andradas.}. Ali era
eu revisor de provas, e empregava os ócios do estudo em aprender a arte
tipográfica; também Luiz Gama era aprendiz de compositor\footnote{.
  Encarregado de compor originais de texto em tipografia.}, praticante
do foro, e colaborador da folha, onde assinava com o pseudônimo
\emph{Afro}.

No ano seguinte, lembro-me dele entre os redatores do \emph{Radical
Paulistano}, que eram Ruy Barbosa, Bernardino Pamplona de Menezes, o dr.
Eloy Ottoni e outros, e entre os oradores do Club Radical. Foi
aplaudidíssima uma conferência sua no salão Joaquim Elias, à rua Nova de
S. José.

Os radicais foram, nos nossos últimos anos políticos, os precursores dos
republicanos. À exceção de meia dúzia de estacionários ou retrógrados,
entre os quais Silveira Martins\footnote{. Gaspar da Silveira Martins
  (1835-1901), natural de Cerro Largo, Uruguai, foi advogado, magistrado
  e político. Eleito deputado e senador por sucessivos mandatos, também
  foi ministro da Fazenda (1878-1879) e presidente da província de São
  Pedro do Rio Grande do Sul (1889).}, Silveira da Motta\footnote{.
  Arthur Silveira da Motta (1843-1914) foi escritor, historiador e
  militar. Considerado herói na Guerra do Paraguai (1865-1870),
  reformado como almirante, foi também membro da Academia Brasileira de
  Letras (1907).} e Ruy Barbosa, em fins de 1869\footnote{. No original,
  por evidente erro tipográfico, está 1879.} e começo de 1871, os
radicais declararam-se abertamente pela República.

Por esse tempo, ou proximamente, fazia Luiz Gama a todo transe a
propaganda abolicionista: a sua advocacia era o terror dos senhores de
escravos. Sei que teve a cabeça posta a prêmio por fazendeiros de S.
Paulo, e tempo houve em que não poderia ir da capital à Campinas sem
risco de vida.

Há 8 ou 10 anos, foi Luiz Gama à barra do júri de S. Paulo, processado
por crime de injúrias contra uma autoridade judiciária; defendeu-se por
si mesmo, brilhantemente; teve de referir grande parte de sua vida
passada; a sala do tribunal, apinhada de assistentes, onde estava quase
toda a mocidade da Academia de Direito, a todo o momento cobria de
aplausos a voz do réu, a despeito da campainha do presidente; o júri o
absolveu por voto unânime, e foi Luiz levado em triunfo até à casa.

Como defensor de escravos, perante o júri, foi mais de uma vez chamado à
ordem pelo presidente do tribunal, por pregar francamente o direito de
insurreição:

-- Todo escravo que mata o senhor, afirmava Luiz Gama, seja em que
circunstâncias for, mata em legítima defesa!

Em uma causa célebre no foro de Santos, em que o advogado contrário era
ninguém menos que o seu grande amigo José Bonifácio\footnote{. José
  Bonifácio de Andrade e Silva, o Moço (1827-1886), nasceu em Bordeaux,
  França, e viveu grande parte da vida em São Paulo, onde se graduou e
  foi professor de Direito. Poeta, literato, foi na política que
  alcançou maior notoriedade, como deputado, ministro e senador em
  sucessivos mandatos desde o início da década de 1860.}, ganhou Luiz
Gama a liberdade de mais de cem escravos.

Recordo-me, como testemunha presencial, de outra solene ocasião em que o
nobre vulto de Luiz Gama destacou-se a toda à luz. Estava reunido em S.
Paulo, num palacete da rua de Miguel Carlos, em 2 de Julho de 1873, o
primeiro congresso republicano da província, presidido pelo austero
cidadão dr. Américo Braziliense\footnote{. Américo Braziliense de
  Almeida e Mello (1833-1896), nascido em Sorocaba (SP), foi político,
  advogado, professor catedrático de Direito Romano na Faculdade de
  Direito de São Paulo, juiz e ministro da Supremo Tribunal Federal. Foi
  vereador e deputado em São Paulo, presidente das províncias da Paraíba
  (1866-1867) e do Rio de Janeiro (1868) e o primeiro governador do
  estado de São Paulo (1891) no período republicano.}.

Era uma assembleia imponente. Verificados os poderes na sessão da
véspera, estavam presentes vinte e sete representantes de municípios --
agricultores, advogados, jornalistas, um engenheiro, todos os membros do
congresso, moços pela maior parte, compenetrados da alta significação do
mandato que cumpriam, tinham, na sobriedade do discurso e na gravidade
do aspecto, a circunspecção de um senado romano.

Lidas, discutidas e aprovadas as bases oferecidas pela \emph{Convenção
de Itu}\footnote{. A famosa Convenção de Itu, realizada em 18 de Abril
  de 1873, foi um marco do movimento republicano brasileiro e selou a
  fundação do Partido Republicano Paulista. Não há registro da
  participação de Gama na convenção.} para a constituição do congresso,
e depois de outros trabalhos, foi, por alguns representantes, submetido
ao congresso, e afinal aprovado, um manifesto à província relativamente
à questão do estado servil. No manifesto, em que se atendia mais às
conveniências políticas do partido do que à pureza de seus princípios,
anunciava-se que, se tal problema fosse entregue à deliberação dos
republicanos, estes resolveriam que cada província da União Brasileira
realizaria a reforma de acordo com seus interesses peculiares \emph{mais
ou menos lentamente}, conforme a maior ou menor facilidade na
substituição do trabalho escravo pelo trabalho livre; e que, \emph{em
respeito aos direitos adquiridos} e para conciliar a propriedade de fato
com o princípio da liberdade, a reforma se faria tendo por base a
indenização e o resgate.

Posto em discussão o manifesto, tomou a palavra Luiz Gama, representante
do município de S. José dos Campos.\footnote{. São José dos Campos (SP)
  é um município localizado na parte paulista do Vale do Paraíba. Embora
  Gama não tenha morado na cidade, as regras estatutárias do Partido
  Republicano para eleger representantes locais deveriam prever a
  possibilidade de delegação independente do município de residência ou
  domicílio.} Protestou contra as ideias do manifesto, contra as
concessões que nele se faziam à opressão e ao crime; propugnava
ousadamente pela abolição completa, imediata e incondicional do elemento
servil.

Crescia na tribuna o vulto do orador: o gesto, a princípio frouxo,
alargava-se, acentuava-se, enérgico e inspirado; estava quebrada a calma
serenidade da sessão; os representantes, quase todos de pé, mas
dominados e mudos, ouviam a palavra fogosa, vingadora e formidável do
tribuno negro.

Não era já um homem, era um princípio que falava... digo mal: não era um
princípio, era uma paixão absoluta, era a paixão da igualdade que rugia,
ali estava na tribuna, envergonhando os tímidos, verberando os
prudentes, ali estava, na rude explosão da natureza primitiva, o neto
d'África, o filho de Luiza Mahin!

A sua opinião caiu vencida e única; mas não houve também ali um coração
que se não alvoroçasse de entusiasmo pelo defensor dos escravos.

Dir-te-hei sempre, meu nobre amigo, que não estás isolado, no partido
republicano, na absoluta afirmação da liberdade humana. Também como tu,
eu proclamo que não há condições para a reivindicação deste imortal
princípio, que não há contra ele nem direitos nem fatos que se
respeitem. \emph{Pereat mundus, fiat justitia!}\footnote{. "Que o mundo
  pereça, mas faça-se justiça!"} E é ignorar essencialmente a natureza
das \emph{leis da instituição}, querer que elas respeitem \emph{direitos
adquiridos}. Não é para Victor Hugo\footnote{. Victor-Marie Hugo
  (1802-1885), poeta, dramaturgo e romancista de renome mundial, lançou
  clássicos como \emph{Os trabalhadores do mar} e \emph{O Corcunda de
  Notre-Dame}. Além da obra literária, que marcou profundamente diversas
  gerações de leitores, Hugo teve marcante militância política a favor
  dos direitos humanos e da democracia.}, nem para Castelar\footnote{.
  Emilio Castelar (1832-1899) foi jornalista, escritor, romancista,
  político e presidente da Espanha (1873-1874).} que apelamos: é para
Savigny, o histórico.\footnote{. Friderich Carl von Savigny (1779-1861),
  nascido em Frankfurt am Main, Alemanha, foi um dos mais influentes
  juristas e historiadores do direito do século XIX. Foi professor de
  direito civil, direito penal e direito romano, tendo publicado obras
  em todos esses campos do conhecimento jurídico. Mendonça faz
  referência indireta à Escola Histórica do Direito, movimento
  intelectual alemão do século XIX do qual Savigny foi um dos principais
  doutrinadores.}

VI

Aí está, em meia dúzia de pálidos traços, o perfil do grande homem que
se chama Luiz Gama.

Filho de uma província que, com razão ou sem ela, não é simpática aos
brasileiros do Sul; emancipador tenaz, violento, inconciliável, numa
província inundada de escravos; sem outra família a não ser a que
constituiu por si; sem outros elementos que não fossem o seu forte
caráter e o seu grande talento; atirado só a todas as vicissitudes do
destino, ignorante, pobre, perseguido, vendido como escravo por seu
próprio pai, enjeitado pelos próprios compradores de negros, Luiz Gama é
hoje em S. Paulo um advogado de muito crédito e um cidadão
estimadíssimo. É mais do que isso: é um nome de que se ufana a
democracia brasileira.

O seu passado é, como se viu, dos mais interessantes; o seu futuro, se
se der em vida sua o grande momento político desta terra, há de ler-se
-- sem a menor dúvida o vaticínio -- nas laudas de nossa história.

Seja como for, e ainda que mais não faça, é já um nome que merece um
lugar na gratidão humana, entre Spartacus e John Brown.\footnote{. John
  Brown (1800-1859) foi um abolicionista radical que liderou
  insurreições armadas contra a escravidão. Foi condenado à pena de
  morte e passou à história como mártir da Abolição nos Estados Unidos
  da América.}

LUCIO DE MENDONÇA.

S. Gonçalo, Minas\footnote{. São Gonçalo do Sapucaí é um município
  localizado no sul de Minas Gerais e teve Lúcio de Mendonça como seu
  primeiro prefeito (1883-1885).}, 21 de Agosto de 1880.

\textbf{DIREITO E LIBERDADE EM TEMPOS DE ESCRAVIDÃO}

\textbf{*didascália*}

\emph{Esta série é composta de quatro textos. Todos eles tratam da
produção de liberdade em tempos de escravidão. Não só as linhas gerais,
mas verdadeiras minúcias de um raciocínio jurídico sofisticado podem ser
lidas em cada um dos casos que Gama sustenta. Engana-se, todavia, quem
supõe encontrar um jurista em formação atado ou ao legalismo raso ou à
confusão de categorias afeitas ao domínio da moral e da política. Gama
revela-se como jurista de primeira grandeza já na articulação de prática
judicial intransigente e formulação de resposta normativa baseada no uso
criativo das fontes do direito. As defesas -- e as teses! --
relacionadas às causas de liberdade de Rita, Lucinda, Benedicto,
Jacyntho e Anna, por exemplo, inauguravam um modo de intervenção na
esfera pública. Mais do que um estilo de ativismo difuso, a literatura
normativo-pragmática de Gama falava com as urgências do tempo presente e
projetava uma obra intelectual de liberdade para o futuro.}

\textbf{5. QUESTÃO DE LIBERDADE}\footnote{. In: \emph{Correio
  Paulistano} (SP), A Pedido, Foro da Capital, 13/03/1869, pp. 2-3.}

\textbf{*didascália*}

\emph{Na defesa da parda Rita, Gama inaugura um novo modo de intervenção
na esfera pública. O argumento pela liberdade da escravizada Rita revela
o método que se seguiria, com adaptações, por toda a carreira de Gama.
Tem a mofina, a exposição do erro jurídico do titular do juízo e a
defesa de uma resposta normativa amparada na doutrina que outorgue o
melhor direito. É nesse espaço de arremate, possibilitado pela exposição
pública da má fé ou criminosa desídia do julgador, que se cria e
desenvolve uma literatura normativo-pragmática original. Por se tratar
do primeiro, o caso de Rita faz as vezes de prólogo, introdutório de uma
peça que se desdobraria em muitos atos até o fim da carreira de Gama.
Tem um desfecho inusitado. Aposta no anticlímax para comover a
audiência. Causa estranhamento. Coisa que se afigura própria, sem
embargo, de alguém que se inicia nos caminhos da advocacia oriundo da
formação poética e teatral. }

***

Homem obscuro por nascimento e condição social, e de apoucada
inteligência, jamais cogitei, no meu exílio natural, que a cega
fatalidade pudesse um dia arrastar-me à imprensa, nestes afortunados
tempos de venturas constitucionais, para, diante de uma população
ilustrada, como é seguramente a desta moderna Atenas brasileira,
sustentar os direitos conculcados\footnote{. Pisoteados, espezinhados,
  tratados com desprezo.} de pobres infelizes, vítimas arrastadas ao
bárbaro sacrifício do cativeiro pelos ingênuos caprichos e pela paternal
caridade dos civilizados cristãos de hoje, em face de homens notáveis,
jurisconsultos reconhecidos e acreditados legalmente, a quem o supremo e
quase divino governo do país, em hora abençoada, confiou o sagrado
sacerdócio da honrosa judicatura.

É por sem dúvida dificílima tarefa, sobremodo árdua, a que submeti meus
fracos ombros. Luta irrisória e talvez insensata é esta em que venho
intrometer-me; eu o conheço e confesso compungido e crente do mesquinho
espetáculo a que me ofereço: pigmeu nos páramos\footnote{. Planalto.} do
direito, desafiando ousado os gigantes ulpiânicos da
jurisprudência\footnote{. Comparação evidentemente irônica entre os
  juristas paulistas e a figura lendária de Eneu Domício Ulpiano
  (150-223), jurista romano de enorme importância para o desenvolvimento
  do direito civil, da praxe processual, bem como da filosofia do
  direito na Antiguidade.}!...

A força invencível do destino quis, porém, que os cegos mendicantes
esmolassem o óbolo\footnote{. Donativo de pouca monta.} da caridade
arrimados\footnote{. Sustentados.} à fraca puerícia\footnote{. Pelo
  contexto, a expressão correta seria perícia.} e às mãos protetoras dos
seus irmãos de infortúnio.

Ninguém jamais viu a indigência apoiada ao braço da fortuna. Os
andrajos\footnote{. Trapos, farrapos.} da miséria escandalizariam a
nobreza e os brilhos rutilantes da fidalguia.

Eis a razão porque tomei a mim voluntariamente a proteção, se bem que
fraquíssima, dos que litigam pela sua emancipação.

Valho tanto como eles; estou no meu posto de honra, embora açoitado no
patíbulo\footnote{. Lugar, geralmente um palanque montado a céu aberto,
  onde se erguia o instrumento de tortura (forca, garrote ou guilhotina)
  para a execução dos condenados à pena capital.} da difamação pelo
azorrague\footnote{. Chicote, chibata formada por várias correias
  entrelaçadas presas num cabo de pau. Instrumento de tortura.} pungente
dos escárnios da opulenta grandeza.

\_\_\_\_\_\_\_\_\_

Afirmam contestes os mestres da ciência, e provoco desde já a que se me
prove o contrário, que nas causas de liberdade enceta-se\footnote{.
  Inicia-se.} o pleito pela alegação preliminar, em juízo, dos direitos
do manumitente\footnote{. Alforriando, que demanda a liberdade.};
alegação que deve ser feita por escrito e conforme o que se acha
estabelecido e prescrito por abalizados praxistas\footnote{. Indivíduo
  versado nas praxes do foro, especialista em direito processual.}.

Cumprido, pelo juiz, o dever da aceitação em juízo da alegação do
manumitente, quando juridicamente feita, segue-se o depósito judicial do
mesmo e a nomeação de curador\footnote{. Aquele que está, em virtude de
  lei ou por ordem de juiz, incumbido de cuidar dos interesses e bens de
  quem se acha judicialmente incapacitado de fazê-lo.} idôneo, a quem é
incumbida a obrigação de velar e defender os direitos e interesses
inerentes à causa de que se trata.

E isto assim se faz porque o escravo, não tendo pessoalidade jurídica,
não pode requerer em juízo, principalmente contra seu senhor, e menos
ainda ser considerado \emph{autor}, enquanto legalmente, por meio de
curatela\footnote{. Aqui como efeito dos encargos da curadoria.} e de
depósito, não estiver mantido, protegido e representado.

O depósito, espécie de manutenção, neste caso importa dupla garantia
que, assegurando ao detentor, de modo judicial, o seu domínio, quando
para isso lhe assistam causas razoáveis, oferece ao detento os meios
precisos para desassombradamente e isento de coação fazer valer os seus
direitos; direitos que veste o curador, atenta a incapacidade do escravo
para sustentá-los.

É só depois destas diligências preliminares, ou antes garantias pessoais
indispensáveis, que o escravo, simples impetrante, pode ser considerado
pessoa e admitido, por seu curador, a figurar de \emph{autor} em juízo
para regularmente \emph{pedir} que se lhe declare um direito, que por
outrem é contestado.

Isto é o que ensina o distinto advogado dr. Cordeiro\footnote{. Lopo
  Diniz Cordeiro (1834-1919), natural de Angra dos Reis (RJ), foi
  advogado, promotor de resíduos e capelas, juiz e deputado. Na época
  desse artigo, Diniz Cordeiro era advogado de entidades filantrópicas
  na Corte, como a Caixa de Socorros Pedro IV, o que indica que era uma
  autoridade jurídica abalizada em matérias de curadoria daqueles
  judicialmente incapazes de se representarem.}, firmado nas opiniões
esclarecidas dos mais cultos decanos da jurisprudência e na prática
inalterável adotada e seguida no ilustrado Foro da Corte, à face, e com
assentimento dos primeiros e mais respeitáveis tribunais do país.

É o que está escrito em obras importantes, vulgaríssimas, que por aí
correm ao alcance de todas as pessoas que lidam no Foro, e ao acesso das
mais acanhadas inteligências, não só pela linguagem clara, como pelo
estilo fácil da textura.

Do que fica expendido, como ainda do que ensina o egrégio jurisconsulto,
exmo. conselheiro Ramalho\footnote{. Joaquim Ignacio Ramalho
  (1809-1902), nascido em São Paulo (SP), foi presidente da província de
  Goiás (1845-1848) e diretor da Faculdade de Direito de São Paulo
  (1891-1902). Professor reconhecido, publicou obras jurídicas, a
  exemplo de \emph{Elementos de processo criminal para uso das
  Faculdades de Direito do Império} (1856) e \emph{Praxe brasileira}
  (1869), que Gama com frequência citava em suas petições.}, estribado
na douta opinião dos mais acreditados comentadores do direito civil
pátrio e subsidiário, e do que escreveram conceituados praxistas,
aceitos e seguidos, evidencia-se o modo preciso e incontroverso de
auspiciar as causas de liberdade perante às autoridades competentes do
país.

Neste sentido, e sem a menor discrepância de um só ponto, requeri do
meritíssimo juiz municipal desta cidade, o respeitável sr. dr. Felicio
Ribeiro dos Santos Camargo\footnote{. Felicio Ribeiro dos Santos Camargo
  (?-?), nascido em São Paulo (SP), foi um político e juiz que, a
  exemplo de Rego Freitas, foi um dos principais adversários de Luiz
  Gama.}, em nome da parda Rita, alforriada pelo meu prezado amigo dr.
Rodrigo José Maurício, o depósito da mesma e nomeação de curador idôneo
para judicialmente intentar a competente ação de liberdade.

Confiado inteiramente na sua reconhecida retidão e imparcialidade
aguardava eu, com segurança, benigno deferimento da petição oferecida,
quando fui surpreendido com o seguinte despacho exarado pelo eminente
magistrado:

"J{[}unte{]} neste a Suplicante os documentos que provam o direito que
tem à sua liberdade, a fim de ser ordenado o depósito e tudo o mais que
de direito for.

S. Paulo, 18 de Janeiro de 1869.

\emph{Santos Camargo}".

A exibição de documentos confirmativos da alforria alegada, antes da
garantia de segurança pessoal requerida, importa exigência extralegal de
prova prévia, quando, conforme o nosso direito, é no andamento da causa
e em ocasião oportuna, que se ela exige.

O despacho do benemérito juiz foi uma tortura imposta à desvalida
impetrante, que, para fazer valer o seu direito, implorava segurança de
pessoa, perante a justiça do libérrimo\footnote{. Superlativo de livre,
  algo como muitíssimo livre, muitíssimo liberal.} país em que ela
desgraçadamente sofre ignominiosa\footnote{. Humilhante, desonrosa.}
escravidão.

É uma violação flagrante dos preceitos característicos do julgador
porque, com semelhante despacho, foi desfavorecida com
desabrimento\footnote{. Desprezo, desaforo.} notável a suplicante, e, se
bem que sem malícia, largamente protegido o detentor, quando é certo que
o juiz \emph{não pode tolher os meios legítimos que tem cada um de usar
de seu direito, nem favorecer mais à uma do que à outra parte
litigante}.

Ao digno Magistrado corria o imperioso dever de atender
incontinenti\footnote{. Imediatamente, sem demora.} à impetrante porque,
segundo os princípios invariáveis do direito natural, devem os
magistrados considerar como procedentes, por serem intuitivas, as
alegações de liberdade, e só admitir como válidas as de escravidão,
quando cabalmente provadas; visto como a escravidão, que constitui
direito anômalo, baseando-se em exceção odienta, embora sancionada por
ordenação civil, não se presume, e só se aceita depois de prova
completa.

O honrado sr. dr. juiz municipal, sem forma de processo, parece ter
condenado a degredo os princípios de direito natural: trocou as
lindes\footnote{. Raias, limites.} e transpôs os contendores e, assim
disposta a cena a seu talante\footnote{. Arbítrio.}, antes que se
tivesse encetado o pleito, visto como tratava-se de uma diligência
preliminar obrigatória, exigiu da mísera manumitente\footnote{. Relativo
  ao que demanda liberdade.} prova antecipada de sua liberdade,
colocando-a, deste modo, em posição visivelmente desfavorável. Fato este
irregularíssimo que podia e pode ainda proporcionar ao detentor a livre
disposição da detenta e a sua retirada desta cidade para lugar longínquo
ou desconhecido, onde jamais possa incomodá-lo.

E deste modo concorre o exímio juiz direta, se bem que
involuntariamente, para a perpetração de uma grave e escandalosa
extorsão.

Entretanto, para pôr termo ao singular capricho do respeitável juiz,
curvei-me respeitoso diante do seu venerando despacho, não como cidadão
perante as aras da justiça de um país livre, mas como subalterno diante
do superior.

Satisfiz o arbitrário ditame e, por meio de réplica, exibi um documento
hológrafo\footnote{. Documento inteiramente escrito à mão pelo próprio
  autor.}, do próprio detentor, por meio do qual mostra-se claramente a
concessão de alforria feita à peticionária.

Deste modo estava cortado o nó gordiano\footnote{. Remete à passagem
  lendária em que Alexandre, o Grande (356-323 a.C.), cortou o nó da
  corda que atava a carroça do antigo rei Górdio a uma das colunas do
  templo de Zeus. Nesse caso, a metáfora significa que a exibição do
  documento, prova irrefutável, cortaria qualquer possibilidade de
  continuidade do problema.}.

Submetida, porém, a petição a despacho, o sr. dr. Antonio Pinto do Rego
Freitas\footnote{. Antonio Pinto do Rego Freitas (1835-1886), nascido em
  São Paulo (SP), foi um político e juiz de destaque no cenário local.
  Durante as décadas de 1860 e 1880, foi presidente da Câmara Municipal
  de São Paulo, juiz municipal, inspetor do tesouro provincial e diretor
  de banco. Como ficará patente mais à frente, Rego Freitas foi um dos
  mais encarniçados adversários que Luiz Gama encontrou.}, que então
ilustrava a segunda cadeira magistrática da capital, como presidente que
é da ilustríssima Edilidade\footnote{. Isto é, dos vereadores da Câmara
  Municipal de São Paulo.}, proferiu o despacho que segue-se:

"Justifique.

S. Paulo, 25 de Fevereiro de 1869.

Rego Freitas".

Ao ser-me apresentado este novo assalto jurídico, que outro nome mais
adequado não me ocorre de pronto para dar-lhe, assalto que, conquanto
diversifique do primeiro, segundo a forma, lhe é, em fundo,
completamente idêntico, ocorreram-me à enfraquecida memória estes versos
do satírico lusitano:

Na forma diferentes se mostravam,\\
Mas, em fundo, a clamar similcadentes\footnote{. Diz-se da palavra que
  tem pronúncia e/ou grafia quase homônima à de outra palavra.},

Peregrinas doutrinas expendendo,

Transportavam de espanto às cultas gentes."

Confesso que, com este meditado despacho, julguei deslumbrado e
confundido o meu bom senso e, homem orgulhoso, jurei, por tal decepção,
vingar-me do seu preclaro autor. E ora o faço muito de caso pensado, mas
sem torturar a lei, sem menosprezar o direito e sem ofender o nobre
caráter e imaculada sensatez do severo jurista, mas dizendo-lhe em face
e diante do público que nos observa verdades que S. S., ainda que
nimiamente\footnote{. Demasiadamente, excessivamente.} modesto, jamais
será capaz de contestar, porque a verdade não se contesta.

Será lícito ao escravo demandar o senhor antes de manutenido\footnote{.
  No contexto, aquele que está em posse provisória de sua liberdade.}?

Será aceitável a justificação como prova legal, sem a citação do senhor?

Poderá requerer em juízo o indivíduo a quem o direito nega pessoalidade,
e sem que esta haja sido homologada?

A nova jurisprudência dos Doroteus hodiernos assim o afirmam\footnote{.
  Referência a Doroteu, historiador e jurista que viveu no VI século,
  tendo passado à história como um dos principais codificadores do
  direito romano e compiladores dos cânones jurídicos publicados sob o
  império de Justiniano I (483-565). A menção, contudo, se dirigia
  sarcasticamente aos "Doroteus hodiernos", isto é, por metonímia, aos
  juristas modernos.}.

Se em tais causas deve ser prévia a exibição das provas, creio que de
hoje em diante, por esta nova doutrina, estão elas proibidas.

Um dia, nos Estados Unidos da federação norte-americana, um homem
apareceu perante o magistrado territorial reclamando com altivez a
entrega de outrem que dizia seu escravo, e comprovava a sua alegação com
testemunhas.

-- O juiz ouvia-as; e depois de breve meditação exclamou: Não estou
satisfeito, isto não basta!

--~O que mais exiges de mim, senhor? Redarguiu o reclamante.

-- Que mostreis o título pelo qual Deus vos fez senhor de vosso irmão.

E voltando-se para o paciente, acrescentou:

-- Ide-vos daí: e se alguém tentar contra a vossa segurança,
defendei-vos como homem acometido por salteadores.

Este singular magistrado, por este ato de moral sublime, foi acusado
como violador dos direitos de propriedade nos tribunais superiores, que
o absolveram, declarando: Que ninguém pode ser compelido à obediência de
leis iníquas\footnote{. Perversas.} que o barbarizem e degradem perante
Deus e a moral.

Lamento sinceramente que o procedimento dos juízes brasileiros seja
diametralmente oposto ao daquele benemérito magistrado, verdadeiro
sacerdote da justiça.

Ao terminar este artigo devo declarar que aconselhei à impetrante Rita o
abandono da causa, até que melhores tempos a favoreçam.

Escrevendo estas linhas visei tão somente a sustentação do direito de
uma infeliz, que tem contra si até a animadversão\footnote{. Aversão
  intensa, ódio.} da justiça, e nunca foi, nem é intenção minha
molestar, ainda que de leve, dois respeitáveis jurisconsultos,
carácteres altamente considerados, que tenho em conta e prezo como
excelentes amigos.

São Paulo, 11 de Março de 1869.

LUIZ GAMA

\textbf{6. APONTAMENTOS BIOGRÁFICOS}\footnote{. In: \emph{Radical
  Paulistano} (SP), A Pedido, 24/05/1869, pp. 2-3.}

\textbf{*didascália*}

\emph{Luiz Gama elabora um dos mais interessantes estudos sobre direito
e escravidão de que se tem notícia no Brasil. A defesa causa de
liberdade de Lucinda e sua família articula três frentes distintas:
discussão doutrinária de direito civil, crítica política à Igreja
Católica e a legitimidade e legalidade de demandas de liberdade na
tradição civilista luso-brasileira. Tudo isso numa linguagem criativa
que transita da ironia ao sarcasmo, sem deixar de asseverar com
sobriedade a razão jurídica da causa que discute. Gama atribui a si
mesmo o papel de "historiógrafo do presente" para contar alguns traços
da história da família de Lucinda e do seu "legítimo proprietário", o
padre (que depois foi nomeado bispo) Mello. Divide a história em três
datas distintas: 1828, 1840 e 1869. Na primeira delas, em fevereiro de
1828, o então pároco de Itu registrava no Livro de Notas do cartório
local uma promessa condicional de liberdade para seus quatro
escravizados e a potencial descendência que deles adviria. Impunha
algumas condições, mas, no fundamental, concedia a liberdade. Ocorre que
doze anos se passaram e, em junho de 1840, o padre Mello voltou ao
cartório de Itu com o juízo arrependido e revogou a promessa de
liberdade, alegando, para isso, "desregramentos e ingratidão". Em maio
de 1869, contudo, mês em que "Apontamentos biográficos" era publicado,
Lucinda estava escravizada e separada de sua família. Mas tinha ao seu
lado Luiz Gama. Como o amanuense abolicionista conheceu a história é um
enigma. Seja como for, Gama pensou uma estratégia de liberdade que
passava pela imprensa, por sete diferentes professores e advogados (que
serviram de pareceristas de uma consulta feita por Gama), pelos juízes
de Itu e Jundiaí e, também, pelo presidente da província de São Paulo.
Mais do que um perfil biográfico, portanto, os "Apontamentos" reúnem
lições de liberdade através da doutrina do direito civil. }

***

O BISPO D. A. JOAQUIM DE MELLO\footnote{. Antonio Joaquim de Mello
  (1791-1861), nascido em Itu, foi um bispo católico e conde romano de
  grande influência na antiga província de São Paulo.}, CONDE ROMANO,
CONFESSOR DE S. SANTIDADE, DO CONSELHO DE S. MAJESTADE O IMPERADOR,
ETC., ETC.

Os grandes homens não são do passado

Nem serão jamais do futuro. Pertencem à eternidade.

V. DURUY\footnote{. Victor Duruy (1811-1894) foi um historiador e
  político francês que ocupou o ministério da educação da França entre
  1863-1869, sendo o ministro em exercício no período da carta que se
  lê. Duruy foi um dos defensores do ensino primário gratuito e laico,
  ideias que Luiz Gama elaborou na série de artigos sobre a Instrução
  Pública na província de São Paulo.}.

A história dos grandes homens e os seus atos são exemplos vivos de
moralidade e civismo, perante os quais edificam-se os homens, elevam-se
os povos e glorificam-se as nações.

Recontar às gerações por vir os feitos notáveis dos grandes homens é o
primeiro dever dos historiógrafos do presente; é este o meio de
perpetuar na memória dos séculos os atos heroicos dos mártires do
socialismo.\footnote{. A expressão deve ser lida no contexto satírico
  que o início do artigo, espécie de prólogo de uma peça, usualmente
  carregava.}

Nesta importante província não há quem ignore os relevantes serviços
prestados à magna causa da santa religião do Crucificado, pelo nunca
assaz\footnote{. Suficientemente, bastante.} chorado bispo diocesano d.
Antonio Joaquim de Mello.

Feitos notáveis, porém, abundam nas trevas do mistério, encobertos pela
tímida mão da esquiva modéstia, que, para a glória da igreja paulistana
e honra de tão preclaro varão, devem ser postos a lume.

Os fatos que vamos referir são a prova irrecusável e cabal da nobreza
d'alma, retidão de consciência, ingenuidade de intenções, vastidão de
munificência\footnote{. Generosidade, magnanimidade.},
acrisolamento\footnote{. Aqui no sentido de sublimação, de purificação
  pelo amor às coisas religiosas.} de piedade e clareza de razão, que
distinguiram sempre, no mais subido grau, a egrégia pessoa do nosso
carinhoso pai apostólico, por cujos lábios de contínuo emanavam os
ditames sublimes da divina providência.

ANO DE 1828

Inspirado pelo padre Diogo Antonio Feijó\footnote{. Diogo Antonio Feijó
  (1784-1843) foi um sacerdote católico e estadista do Império. Teve
  destacada atuação na burocracia do estado, ocupando posições como
  deputado, ministro, presidente do Senado. Como ministro da Justiça,
  assinou a Lei que marcou seu nome na história legislativa brasileira,
  proibindo o tráfico de escravos para o Brasil (1831).}, então uma das
mais fortes colunas do Partido Republicano do Brasil, o digno padre
Antonio Joaquim de Mello, servindo-se do púlpito, onde era ouvido com
profunda consideração pelos bons ituanos\footnote{. Naturais de Itu,
  cidade do interior paulista.}, pregou não só contra a introdução de
escravos africanos no Brasil, como ainda contra o elemento servil, cuja
abolição impunha em nome de Deus, da moral e da religião. E para dar ao
povo uma prova inequívoca da sua íntima sinceridade, começou o árduo
tirocínio\footnote{. Exercício prático.} evangélico libertando os seus
escravos, como demonstra o seguinte documento:

"Eu, o padre Antonio Joaquim de Mello, que possuo quatro escravos --
João e sua mulher, Rita; Paulo e sua mulher, Lucina, com eles tratei o
seguinte:

Prometo-lhes, como prometido tenho, que todos os filhos que lhes
nascerem de legítimo matrimônio serão libertos desde o dia de seu
nascimento, mas ficando sujeitos a viverem debaixo da minha tutela até
terem 25 anos de idade, e então, tendo juízo suficiente para se regerem,
poderão sair de minha companhia: acrescento que, a terem vícios de
bêbados, ladrões ou inquietos, ficarão privados de viver sobre si, até
mostrarem emenda de dois anos.

Prometi mais, que, tendo eles idade de 17 anos, começarão a ganhar (os
homens) dobra\footnote{. Antiga unidade monetária.} por ano; e as
mulheres oito mil réis, o que serei obrigado a entregar, para junto,
quando estiverem nas circunstâncias de viver sobre si, como acima fiz
menção; que se eu morrer antes que os ditos filhos de meus escravos
tenham inteirado a idade mencionada, irão para outra tutela, que, em
testamento, eu declarar, tudo debaixo das mesmas condições.

Aos escravos nomeados prometi e DOU o seguinte:

João, que agora terá 30 anos de idade, me servirá até ter 45, findos os
quais fica liberto;

Paulo, que agora terá 32 anos de idade, me servirá até ter 50;

Rita, que terá 16, me servirá até ter 45 anos;

Lucina, que terá 13 anos, me servirá até ter 40.

Se eu morrer antes deles terem preenchido o tempo de seu cativeiro, irão
preencher o dito tempo em outro poder, e lhes darei a escolha, entre
três senhorios, isto em testamento, ou aí declararei cousa que lhes seja
mais vantajosa.

Se por algum motivo houver pessoa que possa ter direito a meus bens, não
poderá jamais apreender os ditos escravos; eles estarão no poder que
lhes parecer, e esse que tiver direito o terá \emph{só sobre o valor de
seus serviços}, para cuja avaliação haverá dois árbitros, um de cada
parte, e se atenderá ao sustento e enfermidades.

Se algum dos ditos meus escravos, no tempo de gozar de sua liberdade,
tiver vícios de bebedice, continuará a estar debaixo de senhorio, até
ter emenda de dois anos.

Se quiserem mudar de cativeiro, enquanto são obrigados a me servir,
\emph{fica de nenhum vigor a doação que lhes faço}. A respeito dos
quatro nomeados eis o que lhes prometi e eles aceitaram, debaixo das
condições declaradas.

Para mais firmeza, este documento será escrito no livro público
competente.

Itu, 5 de fevereiro de 1828.

Antonio Joaquim de Mello."

(Foi a firma reconhecida e o documento registrado no Livro de Notas).

ANO DE 1840

No ano de 1840, porém, despersuadido o virtuoso padre Antonio Joaquim de
Mello das utopias pueris\footnote{. Ingênuas.} que sugerira-lhe o
sonhador republicano padre Feijó\footnote{. Diogo Antonio Feijó
  (1784-1843) foi um sacerdote católico e estadista do Império. Teve
  destacada atuação na burocracia do estado, ocupando posições como
  deputado, ministro, presidente do Senado. Como ministro da Justiça,
  assinou a Lei que marcou seu nome na história legislativa brasileira,
  proibindo o tráfico de escravos para o Brasil (1831). O uso do
  adjetivo republicano é controverso, no que o autor, interessado em
  polemizar, teria razão a mais para lançá-lo.}, e nobremente inspirado
por algumas beatas senhoras, às quais rendia a mais sincera homenagem,
no intuito religioso de beneficiá-las, escravizou alguns dos seus
libertos e os vendeu.

Nem é para admirar tão estranho procedimento da parte do muito caridoso
padre Antonio Joaquim de Mello, pois sabe toda a província de S. Paulo,
e até o Imperador, que o nomeou bispo, que ele tinha fama de santo. E
ninguém ousará contestar que os erros dos santos valem mais perante os
homens do que os acertos dos míseros pecadores.

Eis, pois, o 2º documento comprobatório das santas e misteriosas
virtudes do nosso bem-aventurado ex-bispo:

"Pela presente declaro que revogo e dou por nenhum efeito a promessa que
tinha feito a meus escravos de os libertar depois de passados certos
anos; e, \emph{bem que eu soubesse que eles, segundo as leis, não podiam
contratar comigo, os encorajava, por este modo, a melhor se conterem no
dever, não só para com Deus, como para comigo}; eles, apesar desta
promessa, têm sempre se portado com indiferença, infidelidade e mesmo
imoralidade, por isso, tendo já revogado a respeito de Lucina, a vendi,
não podendo mais suportar desregramentos e ingratidão para comigo;
quando também incluído seu marido, que tem sido tão mau escravo, que tem
levado até meses sem dar serviço, por manhas muito conhecidas.

Restam João e Rita, para com os quais presentemente revogo, tendo o dito
João cada vez se tornado mais negligente no seu serviço, deixando
perder-se o que ele deve vigiar, furtando e deixando furtar o que é de
seu senhor, além disto queixando-se e imputando caluniosamente o que não
faço, como dizer que é meu costume ocupar em dias de guarda\footnote{.
  Ou seja, cobrar trabalho em dias guardados ao descanso, como domingos
  e feriados.}; sua mulher, Rita, jamais querendo prestar serviço que
satisfaça, sem jamais fazer ato em que reconheça o bem que lhe fiz,
libertando seus filhos, dos quais existem três libertos.

Atendendo, pois, à ingratidão destes, tendo consultado a jurisconsultos,
certo de que em consciência posso fazê-lo, ficam para sempre sujeitos,
salvo uma nova graça que possam merecer.

Os filhos que libertei libertos ficam, menos o que prometi na idade de
17 anos até 25, por ser muito oneroso e nem se achar quem os cure, na
minha falta, com tal ônus.

Prolongo mais a tutoria até a idade de 32 anos, emendo o viverem sobre
si desde os 25, pois é classe de gente que com muito mais custo se torna
pesada. E claro é, que nenhum contrato houve entre mim e eles, mesmo
quando houvesse, podia revogar.

Esta será lançada no Livro de Notas, onde está lançada essa promessa que
eu lhes tinha feito e que torno de nenhum vigor.

Itu, 18 de junho de 1840.

Antonio Joaquim de Mello."

A despeito do que encerra este precioso documento, cuja textura alça em
relevo a Santidade do seu preclaríssimo autor, e certo de que
ilegalmente foram os libertos escravizados, escrevi a seguinte consulta,
que foi respondida satisfatoriamente por jurisconsultos de superior
conceito.

PERGUNTA-SE:

1º: Em virtude do que se acha disposto na primeira escritura, são livres
João e sua mulher, Rita; Paulo e sua mulher, Luciana {[}Lucinda{]}, uma
vez que não tenham eles de motu próprio\footnote{. Iniciativa própria,
  espontaneamente.} faltado aos deveres a que se obrigaram por prazo
determinado, para com o benfeitor?

2º: Sendo livres podiam ser revocados\footnote{. Retornados. No sentido
  de retroagir ao estado anterior.} à escravidão em face do direito
pátrio?

3º: Na hipótese afirmativa, são bastantes para determinar a
revocação\footnote{. Efeito de revocar, anular, revogar.} as simples
alegações aduzidas pelo benfeitor, sem audiência judicial dos
revocados\footnote{. A definição agora se aplica a uma das partes, isto
  é, aos que seriam reescravizados pelo bispo Antônio Joaquim de Mello.}?

RESPOSTA

Ao 1º: Respondemos afirmativamente: os indivíduos mencionados no 1º
quesito são forros\footnote{. Nesse contexto, significa alforriado,
  liberto, que saiu da escravidão.}, por força da escritura que
concedeu-lhes a liberdade, tanto mais quanto claríssima é a intenção do
senhor, tentando, pela segunda escritura, revogar a primeira.

Ao 2º: Respondemos negativamente: a Ord{[}enações{]}, Liv{[}ro{]} 4,
Tít{[}ulo{]} 63, § 7 º, não pode subsistir, por incompatível com os
princípios constitucionais -- Const{[}ituição{]}, art. 6º, § 1º, e art.
94, § 2º.\footnote{. O título 63 tratava das "doações e alforrias que se
  podem revogar por ingratidão" e o seu § 7º, descontando referências
  internas, dispunha que: "se alguém forrar seu escravo, livrando-o de
  toda a servidão, e depois que for forro cometer contra que o forrou
  alguma ingratidão pessoal em sua presença, ou em ausência, quer seja
  verbal, quer de feito e real, poderá esse patrono revogar a liberdade
  que deu a esse liberto e reduzi-lo à servidão em que estava. E bem
  assim por cada uma das outras causas de ingratidão, porque o doador
  pode revogar a doação feita ao donatário". O art. 6º da Constituição
  de 1824, por sua vez, definia quem eram os cidadãos brasileiros, sendo
  o § 1º assim redigido: "Os que no Brasil tiverem nascido, quer sejam
  ingênuos, ou libertos, ainda que o pai seja estrangeiro, uma vez ue
  este não resida por serviço de sua Nação". O art. 94 qualificava quem
  seriam os eleitores aptos a votar nas eleições legislativas de todos
  os níveis -- desde a assembleia paroquial até o Senado. O § 2º do art.
  94, contudo, excluía de forma taxativa os libertos de poderem votar e,
  por decorrência óbvia, de poderem ser eleitos. O raciocínio dos
  pareceristas, em síntese, defendia que a possibilidade de revogação da
  alforria seria de todo incompatível com a eventual aquisição de
  cidadania por parte do liberto. Ou seja, uma vez adquiridos direitos
  inscritos na Constituição, não poderia ter esse estatuto retroagido
  por liberalidade particular.}

E com tanto mais fundamento deve ser aceita esta nossa opinião contra a
que sustenta a possibilidade de revogação da alforria, quanto, sendo a
escravidão um fato contrário à natureza, a liberdade uma vez adquirida
nunca mais deve perder-se. Arouc. Ed. Lib.1, Tit{[}ulo{]} 5, de Stat.
Hom. L. 4, § 1º, nº 20.\footnote{. António Mendes Arouca (1610-1680) foi
  um jurista português e advogado na Casa da Suplicação de Lisboa. Não
  se sabe, até o momento, qual obra de Arouca os pareceristas citavam.}

Ao 3º: A revogação da liberdade, ainda quando estivesse em vigor a
Ord{[}enações{]}, Liv{[}ro{]} 4, Tít{[}ulo{]} 63, § 7º, não se dava
\emph{ipso jure}\footnote{. De acordo com o direito.}; a lei concedeu
uma ação pessoal ao doador contra o donatário {[}Lima ad.
Ord{[}enações{]}, Liv{[}ro{]} 4, Tít{[}ulo{]} 63, § 9º{]}. Dependia,
portanto, de uma sentença regularmente proferida {[}Donel., Tomo 1º,
Cap{[}ítulo{]} 24, n{[}úmeros{]} 3 e 4.\footnote{. O título 63 tratava
  das "doações e alforrias que se podem revogar por ingratidão" e o seu
  § 9º, descontando referências internas, dispunha que: "E se o doador
  (...) ou patrono, que por sua vontade livrou o escravo da servidão em
  que era posto não revogou e sua vida a doação feita ao donatário, ou a
  liberdade que deu ao liberto, por razão da ingratidão contra ele
  cometida, ou não moveu em sua vida demanda em juízo para revogar a
  doação ou liberdade, não poderão depois de sua morte seus herdeiros
  fazer tal revogação (...)". Como suporte ao parecer, seus redatores
  citam dois breves comentários nesse parágrafo: o primeiro, do jurista
  português Amaro Lima, provavelmente da obra "Commentaria ad
  Ordinationes Regni Portugalliae" (1740); e o segundo, do jurista
  francês Hugo Donnelus (1527-1591), professor de direito da
  Universidade de Altdorf, Alemanha. É de se supor que ambos os
  comentários tenham sido retirados das riquíssimas notas de rodapé que
  acompanhavam algumas das versões mais detalhadas das Ordenações.}

A Ord{[}enações{]}, Liv{[}ro{]} 4, Tít{[}ulo{]} 63, § 7º, diz: "Poderá
ser revogada", e as causas, constituindo fatos, que a lei não presume,
dependem de prova em juízo. {[}Masc. De probate cons. 898, n{[}úmeros{]}
1 e 18.\footnote{. Para o § 7º do título 63, ver nota acima. O
  comentário que arremata, no entanto, cita o jurista ligurês do século
  XVI, Josephi Mascardi, e o primeiro volume de sua obra \emph{De
  Probationibus: conclusiones probationum omnium} (1585).}

É este nosso parecer, salvo melhor juízo.

S. Paulo, 4 de março de 1869.

José Bonifácio\footnote{. José Bonifácio de Andrade e Silva, o Moço
  (1827-1886), nasceu em Bordeaux, França, e viveu grande parte da vida
  em São Paulo, onde se graduou e foi professor de Direito. Poeta,
  literato, foi na política que alcançou maior notoriedade, como
  deputado, ministro e senador em sucessivos mandatos desde o início da
  década de 1860.}.

Antonio Carlos R. de A. M. e Silva\footnote{. Antonio Carlos Ribeiro de
  Andrada Machado e Silva (1830-1902) nasceu em Santos (SP) e pertence à
  segunda geração dos Andradas, sendo sobrinho de José Bonifácio, "O
  Patriarca", e filho de pai homônimo. Foi político, professor de
  Direito Comercial e advogado, profissão que exerceu como sócio de Luiz
  Gama por aproximadamente uma década.}.

José Maria de Andrade.\footnote{. José Maria de Andrade (s.d.-s.d.),
  nascido em São Paulo (SP), foi escrivão do Tribunal da Relação,
  promotor, juiz municipal e secretário de polícia da província de São
  Paulo. Como registra a crônica da academia de direito paulistana, e o
  parecer supra indica, Andrade foi sócio do escritório dos Andradas.}

"Concordo completamente em todos os pontos do jurídico parecer neste
exarado.

S. Paulo, 8 de março de 1869.

Dr. \emph{Francisco Justino Gonçalves de Andrade}\footnote{. Francisco
  Justino Gonçalves de Andrade (1821-1902), nascido na Ilha da Madeira,
  Portugal, formou-se e fez carreira jurídica em São Paulo. Foi
  professor de Direito Natural e Direito Civil, alcançando notoriedade
  nesse último campo como autor de diversos livros doutrinários.}."

"Em todas as suas partes concordo com o parecer.

S. Paulo, 9 de março de 1869.

\emph{Vicente Mamede de Freitas}\footnote{. Vicente Mamede de Freitas
  (?-1908), paulista da capital, foi deputado provincial, promotor,
  professor de Direito Civil e diretor da Faculdade de Direito de São
  Paulo.}."

"Concordo.

S. Paulo, 11 de março de 1869.

\emph{Cryspiniano}\footnote{. José Cryspiniano Soares (1809-1876),
  nascido em Guarulhos (SP), foi político, advogado e professor de
  Direito Romano da Faculdade de Direito de São Paulo. Figura de
  destaque na política, foi presidente de quatro províncias do Império,
  respectivamente: Mato Grosso (1847-1848), Minas Gerais (1863-1864),
  Rio de Janeiro (1864) e São Paulo (1864-1865).}."

"Curvo-me perante os venerandos pareceres supra exarados.

S. Paulo, 11 de março de 1869.

\emph{Lins de Vasconcellos}\footnote{. Luiz de Oliveira Lins de
  Vaconcellos (1853-1916), nascido em Maceió (AL), foi um advogado,
  promotor público e político, chegando a exercer a presidência da
  província do Maranhão (1879-1880). Na advocacia foi um colaborador em
  diversas demandas de liberdade junto a Luiz Gama, muito embora também
  tenha atuado, em matéria comercial, no polo oposto de Gama.}."

Tenho consciência de haver prestado relevante serviço à esta heroica
província e ao país inteiro, com o mais vivo contentamento dos sinceros
amigos do exmo. finado bispo d. Antonio Joaquim de Mello, publicando
estes dois preciosos documentos. Nem era preciso a inserção que venho de
fazer, de cinco pareceres jurídicos, para consolidar a justa fama de
sábio e virtuoso que foi sempre o mais resplandecente laurel\footnote{.
  Coroa de louros.} de tão pio varão.

É, pois, certo que os anciãos respeitáveis que comparavam-no ao egrégio
pregador, padre Antonio Vieira\footnote{. Antonio Vieira (1608-1697) foi
  um sacerdote católico, filósofo e escritor português que exerceu
  imensa influência no mundo religioso e político do século XVII e
  seguintes. Há muitas citações diretas de Vieira nos escritos de Gama,
  o que revela que este era um de seus autores prediletos.}, não se
enganaram no conceito.

Resta-me agora um duplo dever, que, com indizível prazer, passo a
cumprir.

Implorar a Deus que ilumine os Pontífices e os reis para que felicitem
as dioceses com a nomeação de bispos iguais ao sempre chorado d. Antonio
Joaquim de Mello, e reclamar perante os tribunais a emancipação de
\emph{sete} infelizes, que se acham em cativeiro, como vítimas da
santidade do nosso finado e adorado bispo.

S. Paulo, 26 de abril de 1869.

L. GAMA.

\textbf{7. FORO DO BELÉM DE JUNDIAÍ}\footnote{. In: \emph{Radical
  Paulistano} (SP), {[}editorial{]}, 30/09/1869, p. 2.}

\textbf{*didascália*}

\emph{O "indigno e escandaloso" fato que Luiz Gama narra ao público é
realmente cruel. Benedicto pertencia ao espólio de Anna Francisca Moraes
e, no curso da ação inventariante, "foi alforriado pelo herdeiro
reconhecido", José Bueno do Amaral. A alforria possuía uma condição: o
pagamento de "quantia complementar do preço de avaliação", que seria
direcionada aos demais herdeiros de Anna Francisca de Moraes, que não se
opunham ao pagamento. Sendo assim, Benedicto, já na "condição de estatu
liber, requereu, como devia, ao juiz inventariante, para que ordenasse o
recebimento" do valor estipulado pela avaliação. O juiz, porém, deu
andamento diverso à causa de liberdade: "resolveu a questão indeferindo
o requerimento, e mandando vender, em hasta pública, o peticionário,
quando ele já não era escravo!...". Gama sustentava o "fato legal e
incontestável" da concessão da alforria como momento-chave não só para a
possibilidade de ação do ex-escravizado, mas para a defesa de seus
direitos enquanto alguém de indisputável estatuto jurídico de pessoa
livre. Restava ao juiz, portanto, "passar carta de liberdade" em
benefício do peticionário. Indignado com o escancarado procedimento em
desfavor da causa de liberdade, Gama subiu o tom da crítica ao julgador,
dizendo, entre outros protestos, que era evidente "a completa
incapacidade intelectual desse cidadão {[}o juiz Soares Muniz{]} para o
desempenho das importantíssimas funções inerentes à magistratura".}

***

Acaba de dar-se um fato contristador\footnote{. Desolador, que
  entristece.}, senão indigno e escandaloso, no importante termo de
Belém de Jundiaí\footnote{. Atual município de Itatiba, situado
  aproximadamente 90 km da capital paulista.}, fato para o qual ouso
invocar a benigna atenção das pessoas sensatas.

Benedicto, pertencente ao espólio\footnote{. Herança, conjunto de bens
  que formam o patrimônio do morto, a ser partilhado no inventário entre
  herdeiros ou legatários.} de d. Anna Francisca de Moraes, foi
alforriado pelo herdeiro reconhecido -- José Bueno do Amaral.

Posto, por este fato legal e incontestável, na condição de \emph{estatu
liber}\footnote{. Há muitas variações, mas nesse contexto designa alguém
  que está livre sob condição.}, requereu, como devia, ao juiz
inventariante\footnote{. Juiz responsável pelo processo de inventário.},
para que ordenasse o recebimento, na estação competente, da quantia
complementar do preço de avaliação pertencente aos demais herdeiros, que
a isto não se opuseram.

Atendida esta justa providência e cumpridos os demais preceitos
jurídicos, dever-se-ia, em prol do peticionário, passar carta de
liberdade.

O estólido\footnote{. Estúpido, desprovido de discernimento.} juiz,
porém, resolveu a questão indeferindo o requerimento, e mandando vender,
em hasta pública\footnote{. Leilão público.}, o peticionário, quando ele
já não era escravo!...

Esta lamentável ocorrência é nada menos que um grave atentado, cometido
bruscamente pela autoridade ignorante, contra uma vítima desprotegida.

É mais uma prova eloquente, exibida, em nome do bom senso revoltado,
contra o fatal sistema de confiar-se cargos de judicatura a pessoas
nimiamente\footnote{. Demasiadamente, excessivamente.} ignorantes,
despidas até dos mais comezinhos\footnote{. Corriqueiros, simples.}
rudimentos de direito, como é seguramente o sr. Florencio Soares Muniz,
suplente do juízo municipal no Belém de Jundiaí.

Em homenagem à verdade, que muito prezo, sou forçado a declarar que,
escrevendo estas linhas, não tenho o intento de pôr em dúvida ou
desabonar a nobreza de caráter, a honradez, ou a influência política,
que hão de, por certo, sobejar\footnote{. Sobrar.} ao sr. Soares Muniz,
mas patentear, diante do público judicioso, a completa incapacidade
intelectual desse cidadão para o desempenho das importantíssimas funções
inerentes à magistratura.

É meu fim discutir um fato real, e sobremodo contrário aos direitos
incontestáveis de um indivíduo, que teve a infelicidade de pretender
mantê-los perante tão desazado\footnote{. Inoportuno, inábil,
  negligente.} juiz.

Quero que a lei seja uma verdade respeitada no município de Belém e não
um joguete pernicioso posto fortuitamente nas mãos da imbecilidade.

Ao exmo. Governo da Província requeri providências em favor da esbulhada
vítima do sr. Soares Muniz, e conto que justiça ser-lhe-á feita.

S. Paulo, 27 de setembro de 1869.

LUIZ GAMA.

\textbf{8. FORO DA CAPITAL {[}CASO JACYNTHO E ANNA{]}}\footnote{. In:
  \emph{Radical Paulistano} (SP), Radical Paulistano {[}editorial{]},
  13/11/1869, p. 1.}

\textbf{*didascália*}

\emph{A defesa do caso dos africanos Jacyntho e Anna foi paradigmática
para a carreira de Gama. Foi esse caso que detonou a crise política que
o atirou para fora da administração pública. Após doze anos de serviço
público regular, Gama foi demitido do cargo de amanuense a mando do
presidente da província de São Paulo justamente alguns dias após
publicar esse artigo. Foi, portanto, o estopim que o tirou da polícia.
Em minuciosa descrição e denúncia dos eventos e agentes criminosos, Gama
preparou o ousado pedido de liberdade do casal Jacyntho e Anna --
extensivo aos seus dez filhos e dois netos --, baseado na
multinormatividade do contrabando. Os eventos se passam em três
jurisdições diferentes: Jaguari, na província de Minas Gerais; Amparo e
São Paulo, essas duas na província paulista. A narrativa de Gama se
sustenta em três pontos fundamentais: i) a comprovação do desembarque de
Jacyntho e Anna ocorrido dentro da vigência da lei proibitiva do
desembarque de africanos em regime de escravidão no Brasil; ii) a fraude
dos títulos dos supostos senhores; e iii) a competência do foro da
capital para conhecer e decidir da causa de liberdade. A estratégia de
liberdade e o domínio do repertório semântico do direito são admiráveis.
Gama une, portanto, estratégia e erudição, e as dirige para a causa de
liberdade de Jacyntho, Anna e mais dez familiares. É, em suma, uma aula
de direito e justiça.}

\textbf{***}

Época difícil é a que atravessamos para as causas judiciárias.

Muito longe vai o tempo dos rotineiros emperrados\footnote{. Teimosos,
  obtusos.} do VI século; agora brilham com esplendor deslumbrante os
sábios juristas da moderna jurisprudência dedinatória das
\emph{incompetências} que tanto tem de \emph{cômoda} como de
\emph{agradável}.

Para mim, principalmente, mísero capa-em-colo da ciência\footnote{. No
  sentido de vadio, pobretão. Em português antigo indica um homem que
  não tem nada de seu a não ser a capa.}, que não pertenço ao luminoso
grêmio dos divinos purpurados\footnote{. Por metonímia, aquele que foi
  investido de grande dignidade.} da egrégia Faculdade, torna-se
inextricável\footnote{. Que não se pode desembaraçar, desemaranhar, que
  não se pode desatar.} a gordiana\footnote{. A gordiana, ou nó górdio,
  remete à passagem lendária em que Alexandre, o Grande (356-323 a.C),
  cortou o nó da corda que atava a carroça do antigo rei Górdio a uma
  das colunas do templo de Zeus. A metáfora, adaptada para esse caso,
  indica um problema complexo e impossível de desatar.}
urdidura\footnote{. Por sentido figurado, a maquinação que se tramou
  contra alguém. Enredo, trama ardilosa.} jurídica de que fazem alardo
os preclaríssimos doutores.

Creio que bem perto está o tempo almejado em que os \emph{leigos}
tarelos\footnote{. O mesmo que tagarela.} serão lançados fora dos
átrios\footnote{. Possui muitos significados, sendo os dois mais
  adequados para o contexto os que indicam a entrada exterior de um
  tribunal, ou seu pátio interno, geralmente cercado de arcadas e
  galerias.} da justiça pelos seus perreiros\footnote{. Por extensão de
  sentido, porteiro ou guarda de local público, sendo, nesse caso, dos
  espaços judiciários.} de roupeta\footnote{. Batina.} e cândida
gargantilha, para neles imperarem soberanos os tardos\footnote{. Lerdos,
  patetas.} Pandectas\footnote{. A expressão, oriunda do grego antigo e
  referente aos livros que codificaram o direito dos romanos, indica,
  nesse caso, alguém que domina profundamente o conhecimento jurídico.
  Pela notória carga de ironia da metonímia, pode-se compreender que seu
  emprego subverte a ideia de erudição.} de cabeleira
empolvilhada\footnote{. Coberta de polvilho. A expressão possui uma
  ironia sutil sobre modos e estilos da liturgia jurídica da época, em
  que a "cabeleira empolvilhada" servia de metonímia para uma peruca ou
  mesmo um penteado solene.}.

Perto está o tempo feliz em que o direito moderno, livre dos atrevidos
impertinentes rábulas\footnote{. Pessoa habilitada para solicitar causas
  no foro. A expressão tem conotações variadas, a depender do contexto,
  muito embora quase todas carreguem carga pejorativa.}, se expandirá em
chamas no cenáculo\footnote{. Local solene para refeições e comunhões de
  uma confraria, nesse particular, da comunidade acadêmica e/ou do
  direito.} das Academias, por sobre as frontes predestinadas dos
inspirados Doroteus\footnote{. Referência a Doroteu, historiador e
  jurista que viveu no VI século, tendo passado à história como um dos
  principais codificadores do direito romano e compiladores dos cânones
  jurídicos publicados sob a autoridade do imperador Justiniano I
  (483-565).}.

Enquanto, porém, não chega a suspirada idade do ouro, conveniente é que
eu me aproveite do ensejo para tasquinhar, com incontestável
\emph{competência}, sem embargos da \emph{incompetência}, oposta pelos
Doutos Magistrados desta cidade, nos seus memoráveis e
\emph{competentíssimos} despachos.

Ao eminente jurisconsulto sr. dr. Antonio Pinto do Rego
Freitas\footnote{. Antonio Pinto do Rego Freitas (1835-1886), nascido em
  São Paulo (SP), foi um político e juiz de destaque no cenário local.
  Durante as décadas de 1860 e 1880, foi presidente da Câmara Municipal
  de São Paulo, juiz municipal, inspetor do tesouro provincial e diretor
  de banco. Como ficará patente mais à frente, Rego Freitas foi um dos
  mais encarniçados adversários que Luiz Gama encontrou.}, juiz
municipal suplente da capital em exercício, dirigi eu a seguinte
petição.

"Ilmo. Sr. Dr. Juiz Municipal.

Acha-se nesta cidade o preto Jacyntho, africano, Congo de Nação,
importado no Rio de Janeiro em o ano de 1848, e levado para a cidade de
Jaguari\footnote{. A antiga cidade de Jaguary, extremo sul de Minas
  Gerais, passou a ser chamada de Camanducaia nas primeiras décadas do
  século XX, nome que até hoje conserva.}, província de Minas Gerais, no
ano de 1849, por Antonio da Cunha.

Tendo falecido este Antonio da Cunha, foi o preto Jacyntho
\emph{arrematado em praça, sendo ainda visivelmente boçal}\footnote{. O
  negro recém-chegado da África, que ainda não falava o português.}, por
Antonio Gonçalves Pereira.

Em poder deste, casaram-no com a preta Anna, de Nação Cabinda, importada
no Brasil, em o ano de 1850, e vendida, em Jaguary, por Aureliano
Furquim de Almeida, que levou-a do Rio de Janeiro para ali, ao mesmo
Antonio Gonçalves Pereira.

Tanto Jacyntho como Anna, sua mulher, foram batizados na cidade de
Jaguary pelo finado padre Joaquim José de Mello.

Não existe, porém, nos livros competentes, assentamento algum a
respeito, seguramente para evitar-se conhecimento da fraude com que
procedera o referido padre, batizando como escravos africanos
livres.\footnote{. Para forjar a legalidade da compra de escravizados,
  padres registravam no livro de batismo paroquial os africanos
  ilegalmente introduzidos no Brasil que eram apresentados à sua
  autoridade. Esta prática notarial-eclesiástica foi largamente
  difundida por todo o Brasil e serviu ardilosamente para justificar a
  propriedade ilegal de centenas de milhares de africanos escravizados.}

Foram padrinhos de Jacyntho, Manoel da Rosa, já falecido; e de Anna,
Beralda de Tal, que ainda vive em Jaguari.

Ultimamente, Antonio Gonçalves Pereira, sabendo que a propriedade que
tinha de tais indivíduos era ilegal, e que corria iminente perigo de
perdê-la, veio cautelosamente a esta província e, no município do
Amparo\footnote{. Cidade paulista que dista 140 km da capital.}, vendeu
o africano Jacyntho e sua mulher a Ignacio Preto, trazendo-os amarrados
e escoltados por José de Lima Oliveira, e Pedro, filho deste, fato que
foi observado por Francisco de Assis Fleminge, pela mulher deste, e por
José Ribeiro de Moraes.

Sabem da importação ilegal e criminosa destes africanos, porque
viram-nos chegar a Jaguary, ainda completamente boçais, nos anos de 1849
e 1850:

-- João Pedro Ribeiro de Sá;

-- José Ribeiro de Moraes;

-- Tenente Francisco José Lourenço;

-- Bernardo da Cunha e Souza; e sua mulher, Maria Custodia;

-- Tenente Manoel Luiz Pinto Monteiro;

-- Francisco Ponciano;

-- D. Anna Ponciano;

-- José Custódio (das Antas);

-- Francisco do Prado;

-- Alferes Francisco Gonçalves Barboza;

-- José Mariano da Silva (do Morro).

São todos do município de Jaguary.

Em vista do que exposto fica, vem o abaixo assinado perante V. S.
requerer que se digne mandar pôr incontinenti\footnote{. Imediatamente,
  sem demora.} em depósito o africano Jacyntho; requisitar, com
urgência, a apreensão e remessa da mulher do mesmo, de nome Anna, do
Amparo para esta cidade, para ser igualmente depositada; e, por
precatória\footnote{. Carta precatória. Instrumento pelo qual um juiz de
  uma jurisdição pede ao juiz de outra jurisdição que cumpra um mandado
  ou sentença sua.}, mandar ouvir as testemunhas indicadas; e, afinal,
declarando livres os ditos africanos nos termos da Lei de 7 de Novembro
de 1831, Regulamento de 12 de Abril de 1832 e mais disposições em vigor,
oficiar ao juiz municipal de Jaguary para que reconheça e mantenha em
liberdade, pelos meios judiciais, os filhos dos mencionados africanos,
de nomes, Joanna, Catharina, Ignacia, Benedicto, Agostinho, Rita, João,
Sabino, Eva e Sebastião; e os seus netos, Marianna e
Marcellino.\footnote{. Considerada uma lei vazia de força normativa,
  recebendo até o apelido de "lei para inglês ver", a conhecida "Lei de
  1831" previa punição para traficantes de escravizados e, de maneira
  não tão assertiva como a historiografia crava, declarava livres os
  escravizados que chegassem ao Brasil após a vigência da lei. Por sua
  vez, o decreto de 12/04/1832 regulava a execução da Lei de 7 de
  Novembro de 1831. Gama fazia referência indireta ao art. 10 do decreto
  que reconhecia de modo bastante enfático a capacidade jurídica do
  preto (sublinhe-se, não escravizado) requerer sua liberdade com base
  no tráfico ilegal. Gama equipara categorias jurídicas que sabia
  bastante distintas -- "preto" e "escravo" -- para reforçar seu
  argumento, isto é, a formação e extensão de um direito de ação ao
  escravizado, assim como discutir a questão nos termos da lógica
  senhorial a um só tempo escravista e racista. Dada a força normativa
  do artigo, que Gama exploraria outras vezes, vejamos seu conteúdo na
  íntegra desde já. Art. 10. "Em qualquer tempo, em que o preto requerer
  a qualquer juiz, de paz ou criminal, que veio para o Brasil depois da
  extinção do tráfico, o juiz o interrogará sobre todas as
  circunstâncias que possam esclarecer o fato, e oficialmente procederá
  a todas as diligências necessárias para certificar-se dele, obrigando
  o senhor a desfazer todas as dúvidas que se suscitarem a tal respeito.
  Havendo presunções veementes de ser o preto livre, o mandará depositar
  e proceder nos mais termos da lei."}

O abaixo assinado jura a boa fé com que dá a presente denúncia e
compromete-se a acompanhá-la em juízo, prestando os esclarecimentos que
forem necessários.

P{[}ede{]} à V{[}ossa{]} S{[}enhoria{]} deferimento de direito.

E. R. M.

S. Paulo, 13 de outubro de 1869.

\emph{Luiz Gama}"

Neste requerimento todo firmado em lei, e sem período ou frase alguma
que possa oferecer controvérsia, pôs\footnote{. O mesmo que opôs.} o
meritíssimo juiz este inqualificável despacho.

"Constando da presente \emph{alegação} (aliás, denúncia, sapientíssimo
sr. doutor) que o \emph{senhor do escravo Jacyntho é morador no Termo do
Amparo}, não estando, por isso, debaixo da jurisdição deste juízo,
\emph{requeira ao juízo competente}.

São Paulo, 25 de outubro de 1869.

REGO FREITAS."

\emph{E doze dias} estudou o sábio jurisconsulto para lavrar este
inconcebível despacho que faria injúria à inteligência mais humilde!

REQUEIRA AO JUÍZO COMPETENTE?!...

Consinta o imponente juiz, sem ofensa do seu amor próprio, que muito
respeito, e da reconhecida ilustração dos seus venerandos mestres, que
eu lhe dê uma proveitosa lição de direito, para que não continue a
enxovalhar em público o pergaminho de bacharel que foi-lhe concedido
pela mais distinta das faculdades jurídicas do império.

Esta lição está contida e escrita com maior clareza na seguinte
disposição de Lei, que o meritíssimo juiz parece ou finge ignorar.

"Em qualquer tempo em que o preto requerer A QUALQUER JUIZ DE PAZ, ou
criminal, que veio para o Brasil depois da extinção do tráfico, o juiz o
interrogará sobre todas as circunstâncias que possam esclarecer o fato,
e OFICIALMENTE PROCEDERÁ A TODAS AS DILIGÊNCIAS NECESSÁRIAS PARA
CERTIFICAR-SE DELE, obrigando o senhor a desfazer as dúvidas que
suscitarem a tal respeito.

HAVENDO PRESUNÇÕES VEEMENTES DE SER O PRETO LIVRE, O MANDARÁ DEPOSITAR e
proceder nos mais termos da Lei."

Nesta disposição é que devera o sr. dr. Rego Freitas estribar\footnote{.
  Firmar, apoiar, fundamentar.} o seu despacho, como juiz íntegro, e não
em sofismas fúteis, que bem revelam a intenção de frustrar o direito de
um miserável africano, que não possui brasões nem títulos honoríficos
para despertar as simpatias e a veia jurídica do eminente e
amestrado\footnote{. Doutrinador, aquele que se tornou mestre em seu
  ofício.} jurisconsulto.

Descanse, porém, o sr. dr. Rego Freitas, porque eu protesto perante o
país inteiro de obrigá-lo à cingir-se\footnote{. Restringir-se,
  limitar-se.} à lei, respeitar o direito e cumprir estritamente o seu
dever para o que é pago com o suor do povo, que é o ouro da Nação.

1869 -- Outubro 26.

LUIZ GAMA.

\textbf{A DEMISSÃO DE LUIZ GAMA}

\textbf{*didascália*}

\emph{Esse conjunto de quatro textos relata um dos mais importantes
eventos da vida de Luiz Gama. Paradoxalmente, sua exoneração da
Secretaria de Polícia significou a transição para uma nova etapa de sua
militância abolicionista e republicana, que o inseriu definitivamente
nas páginas da história do Brasil e das Américas como um jurista que
concebeu uma estratégia de liberdade original. Sendo todos os textos
assinados em seu nome próprio, esse bloco trata exclusivamente de sua
demissão do cargo de amanuense da Secretaria de Polícia da província de
São Paulo. Transitando entre a literatura normativo-pragmática -- haja
vista a discussão de solução normativa aplicada ao caso concreto, ainda
que fosse ele o protagonista da causa -- e a propaganda política, Gama
conta a história que o levou a ser exonerado da administração pública e,
para a incredulidade de seus algozes, persistir no caminho do direito.}

\textbf{9. UM NOVO ALEXANDRE}\footnote{. In: \emph{Correio Paulistano}
  (SP), A Pedido, 20/11/1869, p. 2.}

\textbf{*didascália*}

\emph{Gama dá a conhecer ao público o fato de sua demissão e as razões
formais e reais que a embasaram. Gama recorda eventos, revela segredos,
comenta a portaria de exoneração, o ato solene, afinal, que o demitiu, e
discute a fundo as razões políticas que levaram o presidente da
província a intervir na Secretaria de Polícia e determinar sua demissão
do cargo de amanuense. Antes mesmo de uma defesa pessoal, o leitor
poderá ler "Um Novo Alexandre" como um ato político performático que
denunciava o abuso de poder, a fragilidade normativa diante da vontade
política e, em última instância, uma sociedade inteiramente capturada
pelos interesses privados da escravidão sobre a administração pública. O
autor relaciona de modo bastante convincente que sua participação na
causa de liberdade de Jacyntho e Anna foi o estopim para a sua demissão.
Isso fica claro não só pela sincronia de datas, mas, sobretudo, pela
concorrência de agentes graúdos interessados em pôr um fim na ação
forense e jornalística de um funcionário público que se mostrava
crescentemente influente. Gama já havia avisado nos jornais, por
exemplo, que um dos juízes interessados em constranger sua atividade
judiciária o havia alertado do perigo que corria em patrocinar de modo
enérgico a defesa de escravizados. Como se lê nesse artigo, o aviso de
perigo não era só uma reprimenda genérica. A demissão se consumava. E,
como Gama abria ao público, a demissão seria só a primeira etapa, que
poderia se desdobrar por outras violências de maior impacto e gravidade.
Sem meneios, portanto, Gama colocou suas cartas na mesa, discutindo, por
um lado, a ilegalidade do administrativo, e por outro lado, as condições
fáticas e os interesses envolvidos na sua demissão. }

***

\emph{Para os déspotas a violência é o principal meio de convencer os
recalcitrantes}. Alfieri\footnote{. Vittorio Alfieri (1749-1803) foi um
  dramaturgo, poeta e escritor italiano com obras sobre política,
  filosofia e crítica de costumes. As ideias anticlericais,
  antimonárquicas e jacobinas, temperadas pela sátira afiada, fizeram
  dele um autor lido e relido por gerações de pensadores, entre eles
  Luiz Gama, que o cita com frequência.}

I

Honro-me com a demissão que acabo de receber do cargo de amanuense da
repartição de polícia desta província, porque para autorizá-la o muito
digno e ilustrado chefe de polícia interino, exmo. dr. Vicente Ferreira
da Silva Bueno\footnote{. Vicente Ferreira da Silva Bueno (1815-1873)
  teve longa carreira administrativo-judiciária, exercendo cargos de
  delegado de polícia, juiz municipal, juiz dos órfãos, juiz de direito
  e desembargador em diversas províncias, como Bahia, Paraná, São Paulo
  e Rio de Janeiro. Em 1869, era chefe de polícia interino da província
  de São Paulo, cabendo a ele papel de algoz no espetáculo da demissão
  de Luiz Gama do cargo de amanuense da Secretaria de Polícia.}, teve
precisão de procurar motivo em fatos inteiramente alheios aos deveres
que solenemente contraí perante a lei, e como empregado jurei cumprir.

Sou empregado público há 12 anos e ufano-me de que neste longo e não
interrompido período de tempo, se não encontrasse um só fato para
galvanizar-se\footnote{. No sentido de dar causa, provocar, suscitar.} a
\emph{violenta e ilegal} demissão com que fui calculadamente fulminado.

S. Ex., o respeitável sr. dr. chefe de polícia, dignou-se a registrar as
razões que obrigaram-no a exonerar-me, mas atilado\footnote{.
  Escrupuloso, cuidadoso. O emprego do termo possui evidente conotação
  sarcástica.} e cauteloso não julgou conveniente exibir todas as causas
que influíram no seu experimentado espírito, pelo que vou dar-me ao
trabalho de mencionar o principal fundamento que ele hipocritamente
ocultara.

No dia 2 do corrente (foi no dia da comemoração dos mortos!) um ancião
venerando, a quem presto a mais profunda consideração, procurou-me, com
empenho, na secretaria de polícia e, chamando-me de parte, intimou-me
formalmente, em nome de s. ex. o sr. presidente da província, dr.
Antonio Candido da Rocha\footnote{. Antonio Candido da Rocha
  (1821-1882), nascido em Resende (RJ), foi promotor público, juiz
  municipal, juiz de direito, desembargador e político que, à época da
  demissão de Gama do cargo de amanuense da Secretaria de Polícia,
  exercia a presidência da província de São Paulo. Gama e Candido Rocha
  se encontrariam em muitos embates após a demissão que a série de
  artigos sobre o "novo Alexandre" reporta. Muitas ações judiciais de
  Gama foram julgadas pelo desembargador do Tribunal da Relação de São
  Paulo, Candido da Rocha.}, \emph{para que deixasse eu de promover e
patrocinar causas de manumissão de escravos, sob pena de, continuando,
ser demitido do lugar de amanuense da secretaria de polícia, além de
outras graves... coerções pendentes da vontade presidencial}!...

Eu não sei transigir com a infâmia. Entre mim e o governo da província
seria impossível o acordo proposto.

Sou da escola de Poredorax\footnote{. Citação tão contundente quanto
  intrincada: remete aparentemente a um dos quarenta gauleses,
  ancestrais dos franceses, condenados à morte sem enterro pelo déspota
  sanguinário Mitrídates do Ponto VI (132 a.C-63 a.C). A ordem, no
  entanto, não foi aplicada por inteiro, visto que o gaulês Poredorax,
  embora morto, foi enterrado, em sinal de dignidade. Cf: John
  Lempriere, \emph{Lempriere's Classical Dictionary for Schools and
  Academies}, 1832, p. 321.}: o homem honesto sofre, mas não se
corrompe, nem se vende.

Ao estimado amigo que interpelava-me, declarei que prosseguiria sempre,
a despeito da \emph{demissão}, da \emph{prisão} e da \emph{deportação}
que, mais de uma vez, fora objeto de íntimos colóquios no gabinete
presidencial...

Eu advogo de graça, por dedicação sincera, às causas dos desgraçados;
não pretendo lucros, nem temo violências.

A minha demissão foi por modo sobremaneira escandaloso imposta pelo
presidente ao sr. dr. chefe de polícia interino; porque o sr. dr.
Antonio Candido da Rocha protege às ocultas e toma vivo interesse contra
uma causa de liberdade que eu defendo com pertinácia, e continuarei a
defender.

É a causa do infeliz africano Jacyntho, acintosamente contrariada pelo
dr. Antonio Pinto do Rego Freitas\footnote{. Antonio Pinto do Rego
  Freitas (1835-1886), nascido em São Paulo (SP), foi um político e juiz
  de destaque no cenário local. Durante as décadas de 1860 e 1880, foi
  presidente da Câmara Municipal de São Paulo, juiz municipal, inspetor
  do tesouro provincial e diretor de banco. Como ficará patente mais à
  frente, Rego Freitas foi um dos mais encarniçados adversários que Luiz
  Gama encontrou.}, como juiz municipal desta cidade.

S. Ex. neste negócio há sido o principal assessor daquele \emph{dócil}
juiz cujos despachos, manifestamente contrários à evidência da lei, hão
sido por mim publicados pela imprensa, com espanto das pessoas sensatas.

Para minha completa justificação basta-me a singularíssima portaria de
exoneração que foi-me endereçada.

Admirem-na:

"O dr. Vicente Ferreira da Silva Bueno, chefe de polícia interino desta
província, etc.

Chegando oficialmente ao meu conhecimento (\emph{por comunicação
oficiosa que lhe fizera o presidente da província})\footnote{. Conforme
  se lê no original, este é um comentário de Gama na portaria de
  exoneração.} a maneira \emph{inconveniente} e \emph{desrespeitosa} com
a qual o amanuense da secretaria da polícia Luiz Gonzaga Pinto da Gama
tem tratado ao dr. juiz municipal suplente em exercício, do termo desta
capital, em requerimentos sobre não verificados direitos de escravos,
que, subtraindo-se ao poder de seus senhores \emph{encontram apoio no
mesmo amanuense} e, sendo por isso inconveniente a sua conservação na
repartição da polícia, demito-o do lugar de amanuense. (!!!) \footnote{.
  Conforme se lê no original, anotação de Gama na portaria de
  exoneração.}

Secretaria de Polícia de S. Paulo, 18 de Novembro de 1869.

O chefe de polícia interino

VICENTE FERREIRA DA SILVA BUENO."

Mentira!

Dentro do prazo de um ano tenho conseguido a manutenção judicial de 30
pessoas que achavam-se em cativeiro indébito. Nenhuma delas fugiu da
casa dos seus senhores ou detentores. Foram todas por mim arrancadas,
por meios legais, do poder da usurpação moral.

Assiste-me o direito de perguntar ao exmo. sr. dr. chefe de polícia:
quem são esses escravos aos quais ele se refere em sua memorável
portaria?

De onde e quando vieram eles?\\
A quem pertencem?\\
Qual o lugar em que os acoutei?\\
A falta de verdade em um alto funcionário é uma nódoa\footnote{. Mácula,
  desonra.} inapagável.\\
Há um africano, um só, que veio da província de Minas Gerais em procura
dos meus minguados esforços.\\
Em favor desse infeliz, requeri eu, no mesmo dia em que ele aqui chegou,
ao celebérrimo\footnote{. Superlativo de célebre, algo como muitíssimo
  célebre.} sr. dr. juiz municipal suplente desta cidade, as
providências ordenadas pela Lei de 7 de Novembro de 1831 e Decreto de 12
Abril de 1832.\footnote{. Considerada uma lei vazia de força normativa,
  recebendo até o apelido de "lei para inglês ver", a conhecida "Lei de
  1831" previa punição para traficantes de escravizados e, de maneira
  não tão assertiva como a historiografia crava, declarava livres os
  escravizados que chegassem ao Brasil após a vigência da lei. Por sua
  vez, o decreto de 12/04/1832 regulava a execução da Lei de 7 de
  Novembro de 1831. Gama fazia referência indireta ao art. 10 do decreto
  que reconhecia de modo bastante enfático a capacidade jurídica do
  preto (sublinhe-se, não escravizado) requerer sua liberdade com base
  no tráfico ilegal. Gama equipara categorias jurídicas que sabia
  bastante distintas -- "preto" e "escravo" -- para reforçar seu
  argumento, isto é, a formação e extensão de um direito de ação ao
  escravizado, assim como discutir a questão nos termos da lógica
  senhorial a um só tempo escravista e racista. Dada a força normativa
  do artigo, que Gama exploraria outras vezes, vejamos seu conteúdo na
  íntegra desde já. Art. 10. "Em qualquer tempo, em que o preto requerer
  a qualquer juiz, de paz ou criminal, que veio para o Brasil depois da
  extinção do tráfico, o juiz o interrogará sobre todas as
  circunstâncias que possam esclarecer o fato, e oficialmente procederá
  a todas as diligências necessárias para certificar-se dele, obrigando
  o senhor a desfazer todas as dúvidas que se suscitarem a tal respeito.
  Havendo presunções veementes de ser o preto livre, o mandará depositar
  e proceder nos mais termos da lei."}

A petição que então escrevi já é conhecida do público; foi por mim
publicada no \emph{Radical Paulistano}\footnote{. Cf: Foro da Capital
  {[}Caso Jacyntho e Anna{]}, in: \emph{Radical Paulistano} (SP),
  Radical Paulistano {[}editorial{]}, 13/11/1869, p. 1.}.

Nessa petição, depois de 12 dias de reiteradas conferências, pôs o sr.
dr. Rego Freitas um despacho inepto, ofensivo da lei e todo inspirado
pelo exmo. sr. dr. presidente da província.

Se eu presto criminoso abrigo a escravos fugitivos, deixe o sr. dr.
chefe de polícia o indigno ardil das sancadilhas\footnote{. Rasteiras,
  sacanagens, tramoias.} em que se envolve, sem consciência do risível
papel que com sobeja\footnote{. Excessiva, demasiada.} inópia\footnote{.
  Pobreza. Pode ser entendida, pelo contexto, como fraqueza moral.}
representa. Processe-me ou mande processar-me; cumpra o seu dever,
porque eu saberei manter ileso o meu direito.

Creia o exmo. sr. dr. Vicente Ferreira da Silva Bueno que o meu nome
jamais servirá de pancárpia\footnote{. Coroa de flores.} para
galardoar-se\footnote{. Premiar-se.} as prevaricações\footnote{.
  Corrupções.} ingênuas do adiposo\footnote{. Por sentido figurado,
  pegajoso, nojento.} sr. dr. Rego Freitas.

Agora duas palavras ao público judicioso.

Há seis anos, quando o sr. dr. José Ignacio Gomes Guimarães\footnote{.
  José Ignacio Gomes Guimarães (?-?) foi advogado, chefe de polícia,
  juiz na comarca de Limeira e desembargador do Tribunal da Relação de
  São Paulo, onde serviu como presidente (1888-1890).} exercia o lugar
de juiz municipal no termo de Limeira\footnote{. Município do interior
  paulista, distante 140 km da capital.}, à propósito de um discurso
ultramontano\footnote{. Relativo ao ultramontanismo, doutrina
  conservadora que sustentava a autoridade absoluta e a infalibilidade
  do papa, tanto em assuntos civis como em matérias de fé.} que ali
pronunciara em pública reunião, escrevi, como democrata sincero, alguns
artigos estigmatizando as doutrinas desse respeitável magistrado;
artigos que, pelo ferino da sátira e forte energia de linguagem que
encerravam, molestaram-no de algum modo.

Há quatro anos, sendo acre\footnote{. Em sentido figurado, ríspido,
  áspero, e/ou também ácido, mordaz.} e violentamente acometido pela
tribuna e pela imprensa o sr. dr. Vicente Ferreira da Silva Bueno, como
juiz de direito da comarca de Campinas, por decisões que dera no
processo crime instaurado contra os culpados no homicídio do dr.
Bernardino José de Campos\footnote{. Bernardino José de Campos
  (1806-1864), nascido na Bahia, foi advogado e juiz de direito em Minas
  Gerais e São Paulo. Foi pai de Bernardino José de Campos Júnior e
  Américo de Campos, figuras proeminentes no cenário político paulista e
  colaboradores bastante próximos de Luiz Gama ao tempo do Club Radical
  Paulistano. É provável que os "amigos íntimos" de que Gama fala nesse
  parágrafo sejam os próprios Bernardino Júnior e Américo.}, resistindo
eu obstinadamente às rogativas e reflexões dos meus amigos íntimos,
inspirado tão somente por princípios de justiça, a mim tomei,
espontaneamente e por mera simpatia, a causa digna do
encanecido\footnote{. Debilitado, envelhecido.} juiz. E tenho plena
convicção de havê-la defendido com louvável independência nas colunas da
\emph{Revista Commercial}.

Há pouco tempo foi chefe de polícia desta província o sr. dr. José
Ignacio Gomes Guimarães que, durante a sua administração, desfez-se em
provas de estima e sincera consideração para comigo, ao ponto de opor-se
obstinadamente à minha demissão, que, por motivos políticos, fora-lhe
formal e tenazmente imposta!

Serve, hoje, interinamente de chefe de polícia desta província o sr. dr.
Vicente Ferreira da Silva Bueno, que acaba de demitir-me, \emph{segundo
ele próprio declarou-me, de ordem do presidente da província, por
inconveniente e desrespeitoso procedimento para com o exmo.
jurisconsulto do Arouche}\footnote{. Referência ao Largo do Arouche,
  provável local de moradia do juiz de direito Rego Freitas.}\emph{!}...

Esta triste ocorrência é prova cabal de que a honra e a dignidade não
pertencem exclusivamente aos magistrados.

Entre eles há homens de bem, assim como há miseráveis togados.

S. Paulo, 18 de Novembro de 1869.

LUIZ GAMA.

\textbf{10. O NOVO ALEXANDRE}\footnote{. In: \emph{Correio Paulistano}
  (SP), A Pedido, 21/11/1869, p. 2.}

\textbf{*didascália*}

\emph{Gama dá continuidade ao artigo que revelou sua demissão do cargo
de amanuense da Secretaria de Polícia. Segue a estratégia de defesa por
duas frentes. No plano formal, debate a ilegalidade do ato
administrativo que consumou sua demissão, demonstrando de modo
cristalino a forma grosseira com a qual o chefe de polícia resolveu a
questão. Por outro lado, discute as razões fundantes da exoneração.
Nesse artigo, Gama traz à baila o papel do juiz municipal Rego Freitas e
seu conluio político com o presidente da província e o chefe de polícia.
Gama ainda revela ter sido informado de instruções e ordens do
presidente da província para que apreendesse seu cliente, o africano
Jacyntho, e o entregasse preso aos capangas daquele que se pretendia
senhor. Ordem secreta, aliás, que o chefe de polícia teria ilegalmente
cumprido. Foi o interesse escravocrata, ou a política da escravidão,
portanto, o ponto determinante para dobrar a fraqueza moral da
autoridade da jurisdição competente. No fundo, sugere o autor, o chefe
de polícia, Silva Bueno, e o presidente da província, Candido da Rocha,
apenas executavam ordens de outra esfera de poder que era, a rigor,
muito mais poderosa que a administração pública. "Quanto ao sr. dr. Rego
Freitas", o juiz municipal competente para julgar a causa de Jacyntho e
Anna, fulminava Gama: "direi apenas que é um pobre de espírito, para
quem Deus aparelhou o reino do céu". }

***

\emph{A rigorosa observância das leis constitui a sólida reputação dos
magistrados. }

Des. M. F. Tomás\footnote{. Manuel Fernandes Tomás (1771-1822), juiz,
  desembargador e político português que teve atuação destacada nas
  Cortes Constituintes (1820) e na Revolução Liberal do Porto (1820).}

II

Como empregado da secretaria de polícia, tinha os meus deveres marcados
no Código de Processo Criminal, na Lei nº 261 de 3 de Dezembro de 1841,
no Regulamento nº 120 de 31 de Janeiro de 1842 e nos Decretos nº 1.746
de 16 de Abril de 1856 e nº 1.898 de 21 de Fevereiro de 1857\footnote{.
  Referência geral ao Código de Processo Criminal (1832), à lei que o
  reformou em 1841, e a dois decretos regulamentares das funções das
  secretarias de polícia -- o primeiro da polícia da Corte e o segundo
  da polícia de algumas províncias, entre elas, a de São Paulo. É de se
  notar que Gama organiza uma hierarquia normativa sobre os deveres de
  um empregado de polícia, começando pelo Código de Processo Criminal,
  principal legislação de processo criminal do país, seguindo até um
  decreto específico que regia a organização policial em São Paulo.}.

Em nenhuma destas disposições acha-se estabelecida a obrigação de
tratarem os empregados subalternos com subserviente vassalagem\footnote{.
  Carrega o sentido de resignação, extrema subalternidade, servilismo.}
os seus superiores. E, menos ainda, a qualquer outro funcionário ou
magistrado de diversa hierarquia.

O juiz municipal nenhuma interferência tem, quer como autoridade
judiciária, quer como funcionário administrativo, nas repartições de
polícia.

Nem eu, tampouco, na qualidade de amanuense da secretaria de polícia,
tinha dever algum que cumprir em tal juízo.

Nas petições que firmei, a ele endereçadas, exerci um direito
incontestável, como qualquer do povo ou simples cidadão.

Se no exercício imperturbável de semelhante direito cometi algum delito,
é porque tive liberdade para perpetrá-lo.

Por tais atos, à ninguém devo satisfazer senão às autoridades
competentes.

De tais atos, só podem conhecer as autoridades por meio de sumário
criminal e nos termos da lei.

Assim, pois, o arbitrário procedimento do exmo. sr. dr. chefe de polícia
para comigo, encerraria uma indignidade revoltante se a miopia fatal que
lhe obscurece os olhos já lhe não tivesse penetrado a consciência
rostida\footnote{. Moída, maltratada, surrada.} pelos anos e pelas
mesuradas\footnote{. Atenciosas, rigorosas.} homenagens à fraqueza.

S. Excia. é jurisconsulto abalizado; não pode, de boa fé, infringir
grosseiramente a lei para violar os direitos sagrados dos seus
concidadãos.

S. Excia. declarou-me que foi compelido pelo governo a demitir-me e que
o fizera contra a sua vontade!... E podia acrescentar: contra o seu
dever.

Ao confessar, porém, esta vergonhosa fraqueza, esqueceu-se da disposição
do art. 45 do Regulamento nº 120 de 31 de Janeiro de 1842:

"Os amanuenses da repartição de polícia são \emph{livremente} nomeados e
demitidos pelo chefe de polícia."

Se o presidente da província foi bastante iníquo\footnote{. Perverso.}
para impor tão estranho arbítrio ao sr. dr. chefe de polícia, cônscio
dos seus deveres, cabia repeli-lo com energia, não por amor dos meus
interesses, mas em consideração do seu próprio pudor.

O presidente exigiu!

Se aprouver amanhã ao sr. presidente da província, \emph{o que não será
novidade}, mandar recolher-me à prisão e, se para satisfazer o seu
malévolo capricho, tiver a inspiração de escolher, para instrumento, o
bondoso sr. dr. Vicente Ferreira\footnote{. Vicente Ferreira da Silva
  Bueno (1815-1873) teve longa carreira administrativo-judiciária,
  exercendo cargos de delegado de polícia, juiz municipal, juiz dos
  órfãos, juiz de direito e desembargador em diversas províncias, como
  Bahia, Paraná, São Paulo e Rio de Janeiro. Em 1869, era chefe de
  polícia interino da província de São Paulo, cabendo a ele papel de
  algoz no espetáculo da demissão de Luiz Gama do cargo de amanuense da
  Secretaria de Polícia.}, nutro a segurança de que o
integérrimo\footnote{. Extremamente íntegro, o que, dada a escancarada
  ironia, sugere exatamente o oposto.} chefe de polícia, depois de
recalcitrar\footnote{. Resistir obstinadamente ao cumprimento de uma
  ordem.} um pouco, por nímia\footnote{. Demasiada, excessiva.}
modéstia, mandará submisso executar o firmã\footnote{. O mesmo que
  firmão, decreto vindo de soberano ou autoridade máxima. Carrega
  sentido pejorativo, que assinala ato despótico, como se aplica ao caso
  em vista.}; feito o quê, sairá contrito\footnote{. Arrependido,
  pesaroso.} e opado\footnote{. Por sentido figurado, soberbo,
  orgulhoso.}, de porta em porta, mussitando\footnote{. Murmurando,
  resmungando.} aos seus fiéis amigos: "que o presidente usou e abusou
da sua pudicícia\footnote{. Probidade, decência.}; e que, perverso, o
arrastou à perpetração da hórrida monstruosidade!..."

Que governo, santo Deus, e que magistrados!\\
São estes os garantidores da honra, dos direitos e da segurança dos
cidadãos!!\\
\emph{Proh pudor}\footnote{. Do latim, "ó, vergonha".}\emph{!!!}\\
Os superiores, sr. dr. Vicente Ferreira, as autoridades altamente
colocadas pela vontade nacional, quando não estão poluídas pela
morféia\footnote{. O mesmo que lepra, doença crônica e contagiosa.} da
desídia\footnote{. Negligência, irresponsabilidade.}, só devem exigir o
rigoroso cumprimento do que as leis determinam.

O país paga para ter juízes honestos; os algozes depravados procuram-se
nos cárceres, entre os abomináveis criminosos.

Mandar o contrário é um crime, é provocar, com desazo\footnote{.
  Despropósito, inépcia.}, a indignação dos empregados sisudos.

O ato de minha demissão encerra uma miséria inqualificável, que tornaria
réu de prevaricação\footnote{. A expressão ganha sentido
  jurídico-processual de abuso de poder, quando o funcionário público
  pratica ato de ofício contra expressa disposição legal, visando
  satisfazer interesse pessoal e/ou partidário.} o seu autor, se de há
muito a idade e os dissabores políticos lhe não houvessem arrebatado o
fardel\footnote{. O fardo. No contexto, o substantivo assume o sentido
  figurado de sérias responsabilidades.} oneroso da imputabilidade.

Ousa dizer o exmo. sr. dr. chefe de polícia que eu prestei abrigo
indevido a escravos subtraídos domínio senhorial!

Quão gasta pelas tricas\footnote{. Intrigas, artimanhas.} inconfessáveis
vai de tropel\footnote{. A todo vapor, por livre extensão de sentido.} a
enfraquecida memória de S. Excia!...

Eu requeri ao sr. juiz municipal suplente, dr. Rego Freitas,
\emph{depósito judicial do africano Jacyntho, importado no Brasil depois
da lei proibitiva do tráfico}.

O sr. dr. Rego Freitas, assessorado juiz, por excelência, inspirado pelo
honrado presidente da província, nega-se obstinadamente ao cumprimento
da lei.

Entretanto, enquanto eu sustentava, com tenacidade e energia, o direito
desse infeliz, o exmo. sr. dr. chefe de polícia, por misterioso acordo
com o presidente, expedia ordem secreta ao exmo. conselheiro delegado da
capital para mandar apreender clandestinamente o desgraçado africano, e
entregá-lo manietado ao reclamante, suposto senhor, a fim de conduzi-lo
para a província de Minas, por dois expressos\footnote{. O que
  transporta rapidamente, sem escalas. Pelo contexto, pode significar
  dois veículos ou dois capangas encarregados de escoltar o africano
  Jacyntho.} postos à espera nas cercanias desta cidade!...

E ousa afirmar o exmo. sr. chefe de polícia que eu dou a escravos
proteção ilegal!...

S. Excia. sofre da vista e tem a simplicidade de crer que o mundo é
composto de cegos.

Digamos a verdade sem rebuço.\\
A minha demissão era um nó górdio\footnote{. Remete à passagem lendária
  em que Alexandre, o Grande (356-323 a.C.), cortou o nó da corda que
  atava a carroça do antigo rei Górdio a uma das colunas do templo de
  Zeus. A metáfora, adaptada nesse caso, indica alguém que resolve um
  problema complexo de modo simplório.} que há tempos preocupava muitos
espíritos. E para cortá-lo, achou-se, ao fim, um inculpado Alexandre de
cataratas!\footnote{. Por metonímia, a referência a Alexandre, o Grande
  (356-323 a.C.), assume contornos burlescos e substitui o todo-poderoso
  chefe de polícia que assinou a portaria de demissão, Vicente Ferreira
  da Silva Bueno (1815-1873). Enfurecido, o autor insinua que Bueno
  "sofre da vista" e portava "cataratas", não se sabendo, contudo, se
  empregava, uma vez mais, o recurso retórico da metáfora de que o chefe
  de polícia não enxergava bem, ou se explorava uma condição física
  desfavorável.}\\
Consta-me que a horda esfaimada\footnote{. Esfomeada.} de garimpeiros
políticos e de refalsados\footnote{. Desleais, hipócritas.}
estelionatários que por aí se arrastam atidos\footnote{. Extremamente
  apegado.} à fímbria\footnote{. Barra das calcas.} dos
taumaturgos\footnote{. Por derivação de sentido, charlatães,
  trapaceiros.} de partido, mendigando sinecuras\footnote{. Benesses,
  mamatas ou cargos rendosos que exigem pouco ou nenhum trabalho.} e
depredações, começa de exercer contra mim a sua costumeira maledicência.

O meu primeiro artigo, inserto no \emph{Correio} de hoje, na opinião
sáfara\footnote{. Tosca, grosseira.} destes gastos polinetores do
governo, não é uma expansão da moral revoltada perante o cinismo
autocrático da administração; é a cólera do despeito exacerbada pela
perda do emprego!

Míseros turcomanos\footnote{. Relativo ao indivíduo turcomano, povo
  asiático originário da Sibéria oriental cujos descendentes se
  encontram entre os turcos, búlgaros e cazaques.} despudorados, para
quem mais vale o dinheiro do que a honra!

E jactam-se\footnote{. Gabam-se, vangloriam-se.} de cidadãos brasileiros
indivíduos que, sem corar, põem acima da razão e do direito os
preconceitos sociais e as conveniências imorais de alguns funcionários
prevaricadores!

Descansem, porém, os turiferários\footnote{. Bajuladores, aduladores.}
do escândalo e da corrupção, que eu hei de continuar impávido na tarefa
encetada\footnote{. Iniciada, em desenvolvimento.}, se bem que sobremodo
árdua, ainda que pese o arbítrio desfaçado com que pretendem vencer-me.

Agora aguardo o processo cuja instauração foi requisitada ao sr. dr.
promotor público da comarca, pela calúnia que irroguei\footnote{. Impus,
  acusei.} ao sr. dr. juiz municipal suplente.

Do Tribunal do Júri darei aos meus concidadãos conta completa dos meus
atos.\footnote{. Refere-se à acusação de que trataria de se defender e
  da qual sairia inocentado por decisão unânime dos jurados, em dezembro
  de 1870.}

Quanto ao sr. dr. Rego Freitas, direi apenas que é um pobre de espírito,
para quem Deus aparelhou o reino do céu.

S. Paulo, 20 de Novembro de 1869.

LUIZ GAMA.

P. S.: Consta-me que algumas pessoas julgaram apócrifa a portaria de
minha demissão e forjada por gaiatice, para injuriar-se o bom senso e a
ilustração do sr. dr. chefe de polícia.

Este fato obriga-me a deixá-la em exposição na tipografia do
\emph{Correio Paulistano}, para desilusão dos \emph{Thomés}\footnote{.
  Remete à "Dúvida de Tomé", passagem bíblica narrada em João 20:24-29.
  A expressão indica que a dúvida só pode ser sanada com o contato
  direto, visual.} incrédulos.

\textbf{11. AINDA O NOVO ALEXANDRE}\footnote{. In: \emph{Correio
  Paulistano} (SP), A Pedido, 27/11/1869, p. 1.}

\textbf{*didascália*}

\emph{Como o título sugere, Gama continua a discussão pública sobre as
causas que levaram à sua demissão. Acrescenta, nessa oportunidade, um
excerto do noticiário, oriundo da própria Secretaria de Polícia, no qual
o ex-delegado Furtado de Mendonça, seu "mestre" e "dedicado protetor",
dava sua versão da conversa que tiveram em 02/11/1869, na qual Gama
havia sido intimado a deixar de agir em "questões de liberdade". Ocorre
que, como Gama explora com a notável habilidade que lhe é
característica, o que deveria servir para isentar o governo de
responsabilidade política tornou-se uma admissão indireta dos interesses
que guiaram a consumação da demissão. A livre confissão de que o
envolvimento de Gama em causas de liberdade era como "estar mexendo em
um vulcão" não poderia ser mais ilustrativa. Contudo, o que já estava
explícito ganhava ainda mais nitidez com a continuação do teor da
conversa, mesmo na versão publicada pela Secretaria de Polícia. Nela, o
ex-chefe de polícia avisou Gama com todas as letras que "o podiam
demitir e perseguir". Isso mesmo: Furtado de Mendonça avisara Gama que,
em razão em sua ação forense e jornalística em questões de liberdade,
dimensões do exercício pessoal da cidadania fora da competência da
administração, o governo poderia demiti-lo e persegui-lo. Gama tomaria a
frase de Furtado de Mendonça e a discutiria com o público.}

***

Tinha-se representado a tragédia e o sr. dr. Vicente Ferreira\footnote{.
  Vicente Ferreira da Silva Bueno (1815-1873) teve longa carreira
  administrativo-judiciária, exercendo cargos de delegado de polícia,
  juiz municipal, juiz dos órfãos, juiz de direito e desembargador em
  diversas províncias, como Bahia, Paraná, São Paulo e Rio de Janeiro.
  Em 1869, era chefe de polícia interino da província de São Paulo,
  cabendo a ele papel de algoz no espetáculo da demissão de Luiz Gama do
  cargo de amanuense da Secretaria de Polícia.} bem desempenhado o seu
papel de Alexandre.

Eu estava demitido e a propriedade servil acautelada.

Os salteadores da liberdade dormiam o sono dos justos e a Lei de 1831
estava esmagada pela rocha\footnote{. Referência indireta ao presidente
  da província de São Paulo, artífice da demissão de Gama, Antonio
  Candido da Rocha.} presidencial.

O espetáculo, porém, não se havia completado.

Houve surpresa. O público não estava prevenido. Deu-se a representação
sem programa; a curiosidade fora tomada de assalto.

A fé! Em urdiduras de bastidores o nosso amável governo é o primeiro!

Sentemo-nos de novo na arquibancada, distintos leitores: vai continuar a
interrompida representação.

Agora também faço eu parte dos espectadores.

Comprei, por bom preço, bilhete de \emph{segunda ordem}, mas deram-me
assento na plateia!...

Isto, porém, acontece impunemente, porque o sr. chefe interino de
polícia, inspetor do \emph{teatro}, é gerente secreto da empresa...

Nada reclamo, entretanto; porque comprei o direito de patear\footnote{.
  Bater os pés em sinal de protesto, que aqui serve tanto em linguagem
  teatral, quando o público reage em desagrado, quanto como protesto
  jurídico perante a opinião pública, a qual o autor, não podemos
  esquecer, não perde de vista.} ao meu sabor.

Para dissipar as impressões veementes, causadas pela exibição da
tragédia policial, vão deleitar-nos com festival comédia.

Ouçamo-la:

"\textbf{Secretaria de Polícia} -- Dessa repartição comunicam-nos o
seguinte:

Não tendo estado com o ex-amanuense Luiz Gama depois de sua demissão,
por incômodos de saúde que me tem privado de sair, por isso, tendo lido
a exposição que ele fez, não sei se a mim se refere no 4º §. \emph{Se
é}, cumpre-me retificá-lo, em honra da verdade.

Pela amizade que a ele tenho há vinte e dois anos, tendo ido à
secretaria da polícia, em dia que não tenho presente, para informar-me
de quantas licenças tivera o carcereiro Taborda, encontrando o mesmo
Luiz Gama disse-lhe \emph{que mais uma vez e a última lhe dizia
terminantemente deixasse de envolver-se em questões de liberdade, e que
era estar mexendo em um vulcão,} e que eu achava inconveniente\emph{,
bem como o dr. Antonio Candido da Rocha}, assim proceder ele,
\emph{sendo empregado de polícia} (!!!); e \emph{de minha conta}
acrescentei: QUE O PODIAM DEMITIR E PERSEGUIR. Eu não podia
\emph{intimar em nome de quem pela lei não podia demitir} e que
semelhante recomendação me não fizera.

\emph{F. M. S. Furtado de Mendonça}\footnote{. Francisco Maria de Sousa
  Furtado de Mendonça (1812-1890), nascido em Luanda, Angola, foi
  subdelegado, delegado, chefe de polícia e secretário de polícia da
  província de São Paulo ao longo de quatro décadas. Foi, também,
  professor catedrático de Direito Administrativo da Faculdade de
  Direito de São Paulo. A relação de Luiz Gama com Furtado de Mendonça é
  bastante complexa, escapando, em muito, aos limites dos eventos da
  demissão de Gama do cargo de amanuense da secretaria de polícia, em
  1869. Para que se ilustre temporalmente a relação, tenhamos em vista
  que à época do rompimento público, aos finais da década de 1860, ambos
  já se conheciam e trabalhavam juntos há coisa de duas décadas; e,
  mais, Gama não rompeu definitivamente com Furtado de Mendonça, como
  erroneamente indica a historiografia, visto que em 1879 publicou o
  artigo \emph{Aos homens de bem}, defesa moral e política explícita do
  legado de Furtado de Mendonça.}."

Este precioso documento, aliás escrito com ingenuidade mui notável e, ao
que parece, no seguro intuito de não ver a luz da imprensa, foi com
irrisório ardil extraído das \emph{partes oficiais} da delegacia à
secretaria de polícia, pelo deslumbrado sr. dr. Vicente Ferreira da
Silva Bueno.

Admirável originalidade!...

A sua especiosa\footnote{. Enganosa, com aparência de verdade.} e
meditada publicação pela imprensa é um depoimento inconcusso\footnote{.
  Indiscutível, incontestável.} da culposa prevaricação\footnote{.
  Corrupção, descumprimento do~dever~por~interesse~ou~má-fé.} do chefe
de polícia, e da refalsada\footnote{. Desleal, hipócrita.} conivência do
presidente nesta questão.

Depois da inserção importante deste documento, os exmos. srs. presidente
e chefe de polícia, para serem coerentes e mostrarem-se dignos dos
lugares que ocupam, deveriam ter requerido carta de guia para o hospício
de Dom Pedro II.

Estou plenamente justificado perante os homens honestos do meu país.

A eloquência incontrastável da sentença, proferida por juiz competente,
dispensa-me de ociosos arrazoados.

Resta-me agradecer ao exmo. sr. conselheiro F. M. S. Furtado de
Mendonça, meu ilustre mestre, honrado amigo e dedicado protetor, a
sátira pungente com que acaba de fulminar a corruptora administração dos
srs. Antonio Candido da Rocha e Vicente Ferreira da Silva Bueno.

"\emph{Deixa-te de patrocinares a causa dos infelizes, postos
ilegalmente em cativeiro, porque o governo, protetor do crime e da
imoralidade,} DEMITIR-TE-Á \emph{do emprego que exerces, e te}
PERSEGUIRÁ!!!...".

E a demissão realizou-se!...\\
Resta a perseguição, que de ânimo tranquilo aguardo.

S. Paulo, 26 de Novembro de 1869.

LUIZ GAMA.

\textbf{12. PELA ÚLTIMA VEZ}\footnote{. In: \emph{Correio Paulistano}
  (SP), A Pedido, 03/12/1869, p. 1.}

\textbf{*didascália*}

\emph{O artigo continua a série que trata da exoneração do cargo de
amanuense, mas, ao contrário do que o título sugere, não foi nem a
última vez que tratou do episódio da demissão publicamente, nem a última
vez que se dirigiu na imprensa ao "ilustre mestre e honrado amigo"
Furtado de Mendonça. No mês seguinte, Gama publicou um artigo a mais
sobre a demissão, replicando o noticiário da imprensa conservadora do
Rio de Janeiro. Além disso, alguns anos mais tarde dedicaria um artigo
laudatório ao amigo Furtado de Mendonça. Contudo, o que lemos nesse
texto é uma de suas raras descrições autobiográficas. E é bastante
reveladora, tanto de sua trajetória pregressa, quanto de seu estado
anímico para as batalhas do futuro. A demissão do emprego pessoal, como
se lê, ganha ares de manifesto abolicionista e republicano. O
compromisso pessoal torna-se "sonho sublime" de uma nova ordem política
e social, isto é, de um Brasil "sem reis e sem escravos". No entanto,
ainda que afirmasse a luta abolicionista e republicana como ideal de
vida, tinha os olhos no presente que se desenrolava à sua frente. Estava
demitido da polícia, é verdade, e precisava de um novo ganha-pão. Mas
também precisava ter muito claro que a reação dos conservadores não
pararia por ali. Afinal, a mensagem do presidente da província
transmitida por Furtado de Mendonça poderia ser lida assim: "Deixa-te de
patrocinares a causa dos infelizes, postos ilegalmente em cativeiro,
porque o governo, protetor do crime e da imoralidade, DEMITIR-TE-Á do
emprego que exerces, e te PERSEGUIRÁ!!!...". Ao que Gama arrematou: "E a
demissão realizou-se!... Resta a perseguição, que de ânimo tranquilo
aguardo".}

***

O meu ilustre mestre e honrado amigo, o exmo. sr. conselheiro Furtado de
Mendonça\footnote{. Francisco Maria de Sousa Furtado de Mendonça
  (1812-1890), nascido em Luanda, Angola, foi subdelegado, delegado,
  chefe de polícia e secretário de polícia da província de São Paulo ao
  longo de quatro décadas. Foi, também, professor catedrático de Direito
  Administrativo da Faculdade de Direito de São Paulo. A relação de Luiz
  Gama com Furtado de Mendonça é bastante complexa, escapando, em muito,
  aos limites dos eventos da demissão de Gama do cargo de amanuense da
  secretaria de polícia, em 1869. Para que se ilustre temporalmente a
  relação, tenhamos em vista que à época do rompimento público, aos
  finais da década de 1860, ambos já se conheciam e trabalhavam juntos
  há coisa de duas décadas; e, mais, Gama não rompeu definitivamente com
  Furtado de Mendonça, como erroneamente indica a historiografia, visto
  que em 1879 publicou o artigo \emph{Aos homens de bem}, defesa moral e
  política explícita do legado de Furtado de Mendonça.}, teve a
infelicidade de ler com prevenção os meus escritos: traduziu mal as
minhas ideias, tomou a nuvem por Juno\footnote{. Ditado antigo e
  proveniente da mitologia greco-romana que expressa a ideia de alguém
  que se confundiu, que se iludiu com as aparências.}, e julgou-me com
inconveniente precipitação.

A prova cabal deste asserto está estampada na sua primeira explicação
que corre impressa \emph{com} \emph{caráter oficial}. Eis o motivo por
que eu tachei de \emph{ingênua e notável} essa publicação. Será isto um
novo doesto\footnote{. Insulto, acusação desonrosa.}?...

Um meu distinto amigo e muito ilustrado colega da redação do
\emph{Radical Paulistano}\footnote{. Refere-se, provavelmente, a Américo
  de Campos, colega de redação do \emph{Radical Paulistano} que publicou
  um artigo crítico da demissão e, por extensão, em defesa da conduta
  profissional de Gama.} escreveu em minha ausência algumas palavras
amargas, mas sinceras, relativamente à minha demissão. S. Excia. teve a
feliz lembrança de amistosamente impor-me a responsabilidade desse
escrito.

Pois bem, satisfaço os desejos do meu nobre amigo e desvelado protetor;
aceito, com orgulho, a responsabilidade que me impõe.

Agora uma última palavra:\\
À ninguém ainda dei o direito de acoimar-me\footnote{. Tachar-me,
  repreender-me.} de ingrato.\\
A minha história encerra o evangelho da lealdade e da franqueza. O
benefício é para mim um penhor sagrado; \emph{letra} que se não resgata,
porque é escrita no coração.\\
Há cerca de vinte anos, o exmo. sr. conselheiro Furtado, por
nímia\footnote{. Excessiva.} indulgência\footnote{. Benevolência,
  bondade.}, acolheu benigno em o seu gabinete, um soldado de pele
negra, que solicitava ansioso os primeiros lampejos da instrução
primária.\\
Hoje, muitos colegas desse soldado têm os punhos cingidos\footnote{.
  Circundados, envoltos.} de galões\footnote{. Espécie de distintivo de
  determinadas patentes militares ornado na farda.} e os peitos de
comendas.\\
Havia ele deixado de pouco os grilhões de indébito cativeiro que sofrera
por 8 anos, e jurado implacável ódio aos \emph{senhores}.\\
Ao entrar desse gabinete consigo levara ignorância e vontade inabalável
de instruir-se.

Seis anos depois, robustecido de austera moral, o ordenança da delegacia
de polícia despia a farda, entrava para uma repartição pública, fazia-se
conhecido na imprensa como extremo democrata, e esmolava, como até hoje,
para remir os cativos.

Não possuía pergaminhos, porque a inteligência repele os diplomas, como
Deus repele a escravidão.

O ex-soldado, hoje tão honesto como pobre, quaker ou taciturno
ebionita,\footnote{. A passagem é complexa porque faz dupla referência:
  tanto evoca o legado quaker, isto é, movimento protestante que nos
  Estados Unidos da América advogou o abolicionismo radical e uma ideia
  de cristianismo original; quanto nuança essa identidade pela
  qualificação de ebionita, em referência ao movimento cristão ebionita,
  de matriz etíope, que argumentava pela necessidade da pureza de
  crenças e convicções religiosas e políticas.} arvorou à porta da sua
cabana humilde o estandarte da emancipação, e declarou guerra de morte
aos salteadores da liberdade.

Tem por si a pobreza virtuosa; combate contra a imoralidade e o poder.

Os homens bons do país, compadecidos dele, chamam-no de louco; os
infelizes amam-no; o governo persegue-o.

Surgiu-lhe na mente, inapagável, um sonho sublime, que o preocupa: O
Brasil americano e as terras do Cruzeiro sem rei e sem escravos!

Eis o estado a que chegou o discípulo obscuro do exmo. sr. conselheiro
Furtado de Mendonça.

Enquanto os sábios e os aristocratas zombam prazenteiros das misérias do
povo; enquanto os ricos banqueiros capitalizam o sangue e o suor do
escravo; enquanto os sacerdotes de Cristo santificam o roubo em nome do
Calvário\footnote{. Calvário, ou Gólgota, é a colina na qual Jesus foi
  crucificado.}; enquanto a venalidade togada mercadeja impune sobre as
aras da justiça, este filho dileto da desgraça escreve o magnífico poema
da agonia imperial. Aguarda o dia solene da regeneração nacional, que há
de vir; e, se já não viver o velho mestre, espera depô-lo com os louros
da liberdade sobre o túmulo que encerrar as suas cinzas, como testemunho
de eterna gratidão.

S. Paulo, 2 de Dezembro de 1869.

LUIZ GAMA.

\textbf{PORTEIRAS DO VELHO OESTE}

\textbf{*didascália*}

\emph{Recém-saído da Secretaria de Polícia da capital, Gama começaria a
exercer a advocacia com a provisão, espécie de habilitação prática, na
mão. No entanto, mostrá-la teria seus riscos. Era melhor continuar a
sustentar direitos pelos diversos canais de representação disponíveis
numa malha porosa como eram as estruturas policiais, administrativas e
judiciárias da São Paulo da época. Nesta seção, veremos dois casos, que
são relacionados com três cidades do interior paulista, a saber, Amparo,
Campinas e Jundiaí. As localidades informam um fenômeno que se
verificaria ao longo da década: a interiorização da ação jurídica de
Gama. Não seria fácil. Ao contrário, as barreiras que se levantariam
seriam até muitas vezes intransponíveis. O primeiro texto é bastante
revelador das dificuldades que comumente encontraria. A começar que se
tratava de uma provocação. Sim, Gama era chamado à baila, diferentemente
do que normalmente ocorria na capital, onde estava habituado a propor o
debate. Um tal "HOMEM LIVRE" ironizava suas intenções, sugerindo que os
interesses que moviam sua ação enérgica em defesa da libertação dos
escravizados não seriam lá genuínos sentimentos humanitários. No
entanto, por trás da discussão retórica moral estava uma causa de
liberdade explosiva: o inventário de um certo comendador, que era
proprietário de mais de uma centena de escravizados. No caso, o
comendador Ferreira Netto tinha a propriedade legalizada de duzentos e
dezessete negros e negras escravizadas, divididos, em sua maior parte,
por fazendas de Amparo, Campinas e Jundiaí. O autor, que se ocultava sob
o pseudônimo "HOMEM LIVRE", sabia que Gama estava muito bem informado
sobre a causa. À época dos fatos, aquela simples provocação poderia
representar muito mais do que o palavrório moralista parecia indicar. E
a resposta de Gama, que se lê em seguida, bem dá prova que intimidação
repelida é intimidação rechaçada. A resposta de Gama, muito além da
sagacidade que certamente possui, não poderia ser mais enfática. A
segunda causa que compõe essa seção se passa na delegacia de polícia de
Jundiaí. Um ex-colega de farda, i.e., um agente da Força Pública,
instituição a que Gama foi vinculado por seis anos, entre 1848 e 1854,
foi preso em condições ultrajantes. Gama tomou conhecimento do fato e,
com a energia e a tenacidade de costume, requereu soltura e pronto
restabelecimento dos direitos do agente João Francisco de Oliveira. A
prisão ilegal deixou Gama possesso."Tal procedimento manifestamente
ilegal e atentatório da liberdade individual é digno da mais acurada
reflexão", coisa que ele deixou patente ao público. Em passagens prenhes
da verve que todos nós reverenciamos, podemos ler ele próprio
arrematando dois pássaros com a mesma canetada. Por um lado, ponderava,
"assim como é possível que eu desvairado pela liberdade tenha perdido o
bom senso", poderia, por outro, "porém, afirmar com ousadia, que o bom
senso não será encontrado nos gabinetes dos assessores de Jundiaí".
Levantava a bandeira ao mesmo tempo em que caía o juiz.}

\textbf{13. O SR. LUIZ GAMA}\footnote{. In: \emph{Correio Paulistano}
  (SP), A Pedido, 09/02/1870, p. 3.}

\textbf{*didascália*}

\emph{Esse é um dos mais duros ataques que Gama enfrentaria no início de
sua advocacia. Embora pela assinatura não se pudesse saber com exatidão
o seu autor e o lugar de onde escrevia, a indicação da causa como sendo
a dos "escravos que foram de Manoel Joaquim Ferreira Netto e que por
testamento são livres", não deixaria dúvidas sobre os potenciais
interessados em vilipendiar sua imagem e bloquear sua ação jurídica. A
menção ao local onde viviam os escravizados -- "nas fazendas sitas nos
termos de Amparo e Campinas'' -- reforçava a ideia de que os agressores
fossem ou tivessem íntima e familiar ligação com uma dessas cidades. A
informação era importante, especialmente porque a causa dos libertos em
razão do testamento do comendador Ferreira Netto não se resumia só a
Amparo ou Campinas, abarcando também outras cidades como Jundiaí, Santos
e Rio de Janeiro. Assim, saber que o ataque vinha de Amparo e/ou
Campinas se constituía como uma peça-chave até mesmo para a réplica que
Gama, sem demora, já na edição seguinte, daria a conhecimento público. O
escárnio e a malícia da carta aberta -- da primeira frase até a escolha
da assinatura -- são dois dos ingredientes que compõem o ataque. Este,
se por um lado visava influir em uma causa específica, por outro lado
alarmava os escravocratas daquela Roma, a província de S. Paulo, para
aquela ameaça "gaulesa" que tinha nome e sobrenome (e que estampava o
título da publicação): Luiz Gama.}

\emph{***}

Qual a razão por que, sendo, como és, ardente propugnador da
emancipação, deixas que fiquem, nas fazendas sitas nos termos do
Amparo\footnote{. Cidade paulista que dista 140 km da capital.} e de
Campinas, na escravidão, os escravos que foram de Manoel Joaquim
Ferreira Netto e que por seu testamento são livres? Tendo disputado ao
cativeiro um por um todos os que têm direito à liberdade, como
desaproveitas esta grande ninhada?!\footnote{. A expressão pejorativa é
  a primeira amostra das reais intenções do articulista.} Dar-se-á,
acaso, que já esmorecesse o teu santo zelo? Até hoje tem sido o teu
coração um templo, sempre aberto à liberdade; nele sempre acharam as
vítimas do cativeiro refúgio modesto, mas seguro. Mas, se já não é
assim, se outro é o teu propósito, convém torná-lo público para
desengano dos infelizes que pretenderem procurar o teu amparo. É preciso
que a imprensa, como sentinela fiel, ou como outrora em Roma os gansos
do Capitólio,\footnote{. Referência aos lendários gansos capitolinos,
  que alardearam a invasão dos gauleses (390 a. C.), prevenindo os
  romanos do ataque noturno que os estrangeiros planejavam. A metáfora
  explora a ideia de que a imprensa se voltava contra Gama, sugerindo
  que ele fosse uma espécie de impostor que estaria a ludibriar os
  desejos de liberdade dos desvalidos. Gama, portanto, seria
  \emph{persona non grata} na Roma que seria a província de São Paulo.}
diga à liberdade, quando ela, seguindo o costumado caminho, procurar o
teu amparo: Vestal, não entreis naquele templo, está às escuras, o fogo
sagrado já não arde, podeis tropeçar nas piras.\footnote{. Vestal,
  antiga sacerdotisa do culto à Vesta, era a divindade do fogo para os
  antigos romanos. Ao dizer que o fogo sagrado, aqui tomado por símbolo
  da liberdade, não ardia e iluminava o seu próprio templo, a metáfora
  sugere que a verdade não existiria no recinto. O leitor deverá ter
  notado que, no início do texto, o ofensor de Gama apontava que
  "{[}a{]}té hoje tem sido o teu coração um templo". Pelo desfecho,
  contudo, não resta dúvida que a ironia posta acima apenas serviria
  como mote para aumentar o teor da ofensa.}

O HOMEM LIVRE.

\textbf{14. DISTINTO REDATOR {[}réplica{]}}\footnote{. In: \emph{Correio
  Paulistano} (SP), A Pedido, 11/02/1870, p. 3.}

\textbf{*didascália*}

\emph{A réplica de Gama ao artigo do "Homem Livre" é sóbria, defendendo
a um só tempo a sua imagem e o propósito de sua ação abolicionista, e
muitíssimo sagaz, haja vista como responde sobre o processo relacionado
ao testamento do comendador Ferreira Netto. Gama sugere ter tido
conhecimento pela imprensa "que os indivíduos libertados pelo comendador
Ferreira Netto" achavam-se "em cativeiro indébito". E aproveita o que
seria uma informação recém-descoberta para contra-atacar, lamentando que
"o distinto republicano, autor do escrito (...), não tivesse
imediatamente transmitido os preciosos documentos relativos à essa
manumissão". A invertida é fantástica. O tal "HOMEM LIVRE", na tentativa
de desferir um golpe, acabava por confessar estar ciente de um crime --
e nada fazer --, afinal, ele próprio dizia que os escravizados eram
declarados livres por testamento. Para além da discussão pública,
moviam-se placas tectônicas nas bases daquele litígio. Gama já tinha
conhecimento das ações relacionadas ao testamento do comendador Ferreira
Netto e parecia esperar apenas uma oportunidade para fazer algo. Apenas
três meses desse artigo de resposta, Gama seria oficialmente nomeado o
representante das mais de duas centenas de pessoas escravizadas em ação
decorrente do testamento do comendador Ferreira Netto! Mais do que
informação, ele queria mesmo era produzir provas, peticionar e contestar
no processo! Assim, quando dizia "vou meter ombros", falava muito a
sério. Meteria não só ombros, mas braços, tronco e cabeça, tudo o que
tivesse direito, em prol da causa de liberdade dos negros -- legal ou
ilegalmente, pouco lhe importava -- escravizados pelos brancos da
heroica província de S. Paulo. }

\emph{***}

O vosso jornal de hoje deparou-me um artigo, com endereço a mim, inserto
entre as publicações pedidas, subscritas pelo pseudônimo O HOMEM
LIVRE.\footnote{. Cf. reprodução abaixo.}

Peço-vos permissão para responder-lhe.

A ninguém ainda conferi o direito de, por qualquer motivo, pôr em dúvida
a sinceridade e o aferro\footnote{. Afinco, obstinação.} com que
sustento as causas de liberdade que me hão sido confiadas, sendo certo
que o tenho feito espontânea e gratuitamente.

Agora, pelo artigo que acabo de ler, sei que os indivíduos libertados
pelo comendador Ferreira Netto acham-se em cativeiro indébito\footnote{.
  Conforme revelam as movimentações processuais da referida causa de
  liberdade, Gama não só conhecia a situação, como argumentava
  juridicamente que aqueles indivíduos estavam ilegalmente escravizados.}.
Vou promover, como me cumpre, a manumissão\footnote{. Alforria, demanda
  de liberdade.} desses infelizes.

Lamento, entretanto, que o distinto republicano, autor do escrito que
respondo, me não tivesse imediatamente transmitido os precisos
documentos relativos à essa manumissão. Se o tivesse feito, mais pronto
seria eu em promover a ação judicial.

Ao distinto HOMEM LIVRE, pois, rogo o obséquio de prestar-me, por carta
ou verbalmente, os esclarecimentos que tenha obtido, relativos à esta
magna questão a que vou meter ombros\footnote{. Atirar-me ao trabalho,
  trabalhar com afinco.}.

S. Paulo, 10 de Fevereiro de 1870.

\emph{LUIZ GAMA}.

\textbf{15. FORO DE JUNDIAÍ} -- \textbf{(DELEGACIA DE
POLÍCIA)}\footnote{. In: \emph{Correio Paulistano} (SP), A Pedido,
  07/07/1870, p. 2. Jundiaí, município paulista que fica 50 km distante
  de São Paulo (SP), era a principal cidade ao limite norte da capital.}

\textbf{*didascália*}

\emph{Um agente da Força Pública -- categoria que Gama bem conhecia,
afinal, havia servido nela por longos seis anos -- encontrava-se preso
na delegacia de polícia de Jundiaí. De posse de muitíssimos detalhes,
Gama verificou que a prisão, que não fora em flagrante, também não havia
sido determinada por escrito e nem ordenada por autoridade competente.
Assim, a prisão se dava em "transgressão manifesta" do texto normativo
disposto no Código de Processo Criminal. Gama foi atrás de elementos
desse fato juridicamente escabroso e notou que, após corpo de delito e
de uma ordem do juiz para que o agente fosse liberado, uma "pessoa de
perniciosa influência'' teve força bastante para fazer com que o
carcereiro não cumprisse a ordem de soltura a favor de Oliveira. O que
Gama continuaria a narrar seria uma espécie de flagrante forjado, com
direito a invenção de um crime -- inafiançável! -- nunca ocorrido. O
"improvisado crime de tentativa de homicídio" era um atentado contra a
liberdade individual de Oliveira. Gama identifica violações e
ilegalidades, formula um argumento baseado em "bons princípios de
jurisprudência criminal" e peticiona, por três vezes, para que os
direitos de Oliveira fossem restaurados e ele posto em liberdade. Teve
duas das petições indeferidas e a terceira delas repousava, muito
provavelmente, no fundo da gaveta da escrivaninha do juiz. Gama, então,
passa a expor aquele "atentado jurídico", visando "obter o julgamento da
opinião pública, para demonstrar cabalmente a injustiça bárbara de que
está sendo vítima João Francisco de Oliveira". E Gama fazia isso batendo
onde doía mais no juiz arbitrário: jogando luz sobre as decisões e
excertos do processo. Desse modo, Gama juntaria ao artigo uma sentença e
um despacho do juiz João Gonçalves dos Santos Camargo; uma petição, de
sua autoria; e uma consulta, também de sua autoria, respondida por dois
jurisconsultos de bastante prestígio, dois dos irmãos Andradas, e
assinada, na sequência, por outros jurisconsultos importantes, quase
todos professores da Faculdade de Direito de São Paulo. Esse texto se
estabelece, portanto, como um exemplar da literatura que o advogado Luiz
Gama firmava, sobretudo nos jornais paulistanos, mas alcançando também
as porteiras e depois o miolo do velho oeste paulista. }

\emph{***}

Ninguém pode ser preso antes de culpa formada, senão: 1º, em flagrante
delito; 2º, \emph{quando indiciado} em crime inafiançável -- art. 179, §
8º da Const{[}ituição{]}, 131, 133 e 175 do Cód{[}igo{]} do
Proc{[}esso{]} Crim{[}inal{]}.

(SENADOR P. BUENO -- Apontamentos sobre o processo criminal
brasileiro\footnote{. José Antonio Pimenta Bueno (1803-1878), o
  \emph{marquês de São Vicente}, nascido em Santos (SP), foi juiz,
  desembargador, ministro do Supremo Tribunal de Justiça, diplomata e
  político de grande prestígio ao longo do século XIX. Foi presidente
  das províncias de Mato Grosso (1836-1838) e São Pedro do Rio Grande do
  Sul (1850), além de ministro da Justiça (1848), Relações Exteriores
  (1870-1871) e senador do Império (1853-1878). A segunda edição,
  "correta e aumentada", de \emph{Apontamentos sobre o processo criminal
  brasileiro} (1857) teve maior repercussão e foi possivelmente ela que
  Gama consultou para o artigo. Ademais, Gama escolhia como epígrafe a
  obra jurídica de um baluarte do Partido Conservador, indicando, entre
  outros sinais políticos, que o caso em vista não era um palanque
  republicano e, sim, uma causa de direito. Para a citação exata, cf:
  \emph{Apontamentos sobre o processo criminal brasileiro}, José Antonio
  Pimenta Bueno, 1857, p. 276.}).

Nesta cidade foi preso por um agente da força pública e recolhido
imediatamente à prisão, à ordem do delegado de polícia, e contra a
expressa disposição do art. 131 do Código do Processo
Criminal,\footnote{. Cf. Art. 131. "Qualquer pessoa do povo pode, e os
  oficiais de justiça são obrigados a prender, e levar à presença do
  juiz de paz do distrito, a qualquer que for encontrado cometendo algum
  delito, ou enquanto foge perseguido pelo clamor público. Os que assim
  forem presos entender-se-ão presos em flagrante delito".} o indivíduo
aqui residente, de nome João Francisco de Oliveira, sob pretexto de
haver ferido a Jacyntho Francisco de Paula; cumprindo ainda notar que a
prisão não realizou-se em flagrante delito, mas algum tempo depois de
ocorrido o fato, por solicitações do ofendido, e sem que Oliveira fosse
perseguido pelo clamor público. Disto necessariamente conclui-se que a
detenção verificou-se sem determinação, por escrito, da autoridade
competente e, portanto, com transgressão manifesta do que acha-se
disposto nos arts. 132, 133 e 175 do mencionado Código. \footnote{.
  Respectivamente, art. 132. "Logo que um criminoso preso em flagrante
  for à presença do juiz, será interrogado sobre as arguições que lhe
  fazem o condutor e as testemunhas que o acompanharem; do que se
  lavrará termo por todos assinado". Art. 133. "Resultando do
  interrogatório suspeita contra o conduzido, o juiz o mandará pôr em
  custódia em qualquer lugar seguro, que para isso designar; exceto o
  caso de se poder livrar solto, ou admitir fiança, e ele a der; e
  procederá na formação da culpa, observando o que está disposto a este
  respeito no capítulo seguinte". Art. 175. "Poderão também ser presos,
  sem culpa formada, os que forem indiciados em crimes em que não tem
  lugar a fiança; porém nestes, e em todos os mais casos, à exceção dos
  de flagrante delito, a prisão não pode ser executada, senão por ordem
  escrita da autoridade legítima".}

Feito o corpo de delito no ofendido, declarado leve o ferimento, e
julgado o auto procedente, ordenou o digno juiz que fosse o custodiado
posto em liberdade; houve, porém, pessoa de perniciosa influência que
teve força bastante para impedir, por meios clandestinos e para fins
inconfessáveis, que o carcereiro não cumprisse a ordem de soltura
passada em favor de Oliveira; e isto fez-se com calculado artifício, e
no propósito de dar tempo que o ofendido pudesse preparar e apresentar
queixa contra seu agressor, pelo improvisado crime de tentativa de
homicídio!...

Apresentada a queixa, e \emph{antes que fosse devidamente jurada},
passou-se de pronto contramandado, e continuou Oliveira preso,
\emph{como indiciado em crime inafiançável}, servido de base à ordem de
prisão \emph{a simples petição de queixa do autor!... }

\emph{Indiciação}, conforme o direito romano, diz o dr. Vieira Soares no
seu \emph{Manual Político}\footnote{. Referência provável a João Pereira
  Batista Vieira Soares (?-?), advogado e juiz português, bem como a sua
  obra \emph{Manual da religião cristã e legislação criminal portuguesa}
  (1813), um guia com instruções éticas, morais e legais voltado para a
  educação da juventude.}, é a convicção do juiz, resultante de
\emph{prova} ou \emph{veementes} \emph{indícios}, que constituam alguém
suspeito de autoria de crime ou delito.

Desenvolvendo este asserto\footnote{. Embora no original esteja grafado
  com "c", o que não está incorreto, adaptei para a forma como se acha,
  que designa asserção, afirmativa.} acrescenta o mesmo autor: O
arbítrio conferido pelo Código do Processo ao magistrado -- para prender
ou não os indiciados em crime inafiançável, antes de culpa formada --
tem por exclusivo fundamento considerações importantíssimas de ordem
pública, e logicamente repele o capricho estulto\footnote{. Estúpido.},
que pretendem alguns, de poderem as autoridades encarcerar cidadãos por
atos de própria vontade, e sem que para fazerem-no tenham fundamento
razoável.

Baseado nestes bons princípios de jurisprudência criminal requeri, POR
TRÊS VEZES, ordem de soltura em favor do detido. Obtive por duas
indeferimento, e pela terceira ficaram os autos em conclusão.

Tal procedimento manifestamente ilegal e atentatório da liberdade
individual é digno da mais acurada reflexão; visto como por ele
poder-se-á judiciosamente julgar do modo pelo qual são cumpridos e
guardados os preceitos legais neste portentoso império do Brasil.

Não tenho em mente, com este meu escrito, magoar o respeitável sr. João
Gonçalves dos Santos Camargo, a quem muito venero e acato, e cuja
honradez proverbial\footnote{. Notória, amplamente conhecida.} jamais
foi posta em dúvida; quero apenas analisar os atos do delegado de
polícia 1º suplente desta importante cidade, situada a duas horas de
viagem da capital, onde existe uma faculdade de direito e jurisconsultos
eminentes.

Meu intento é tirar à luz meridiana um atentado jurídico, constituído
pela detenção indébita e afrontosa de um homem cujos direitos são
impunemente conculcados\footnote{. Pisoteados, espezinhados, tratados
  com desprezo.}, ainda quando {[}tenham{]} advogados com energia e
tenacidade.

Para realizar este intento e obter o julgamento da opinião pública, para
demonstrar cabalmente a injustiça bárbara de que está sendo vítima João
Francisco de Oliveira, e quanto vale os manejos indecorosos dos
conciliábulos\footnote{. Reunião secreta e, por extensão de sentido
  aplicada ao caso, conspiração, trama.} de camarinha\footnote{. Quarto
  pequeno, podendo ser entendido como refúgio, esconderijo.}, ainda
quando o cauto juiz abroquela-se\footnote{. Defende-se, protege-se.} com
a probidade e com a prudência, basta-me transcrever a sentença que
julgou o corpo de delito; a petição solicitando a retardada soltura do
preso; o despacho negativo do meritíssimo juiz e os pareceres dos
circunspectos jurisconsultos ouvidos sobre a questão.

É possível que os sábios estejam em erro manifesto; assim como é
possível que eu desvairado pela liberdade tenha perdido o bom senso;
posso, porém, afirmar com ousadia, que o bom senso não será encontrado
nos gabinetes dos assessores de Jundiaí.

***

"Julgo procedente o corpo de delito de fls. 12 \emph{usque}\footnote{.
  Até a.} fls. 13, e sendo declarado o ferimento leve, e não sendo o
delinquente preso em flagrante, em vista do decreto nº 1.090 de 1º de
Setembro de 1860,\footnote{. Curiosamente, o decreto citado versava
  "sobre o processo nos crimes de furto de gado ad vacuum", i.e., no
  vácuo, sem dono aparente. O fundamento normativo da sentença,
  portanto, buscava amparo numa lei inteiramente estranha ao caso para,
  como se vê, satisfazer uma vontade particular que não só não possuía
  base legal razoável, como também era contrária às disposições
  expressas do Código de Processo Criminal.} mando que o mesmo indiciado
João Francisco de Oliveira seja relaxado da prisão em que se acha e
posto \emph{incontinenti}\footnote{. Imediatamente, sem demora.}
\emph{em liberdade}, passando-se mandado para esse fim, pagas as custas
de fl. 1 até 9 pelo cofre da municipalidade, de fls. 10 em diante pelo
dito João Francisco de Oliveira.

Jundiaí, 1º de Julho de 1870.

SANTOS CAMARGO.

(Passou-se o mandado, que foi apresentado ao carcereiro às 6 horas da
tarde; e por acordo entre o carcereiro e \emph{mais duas pessoas de
Jundiaí} não foi executado.\\
A queixa foi dada no dia 2, às 8 horas da manhã, e jurado no dia 4 à 1
{[}uma{]} hora da tarde; e a ordem de soltura passada a 1º não se
cumpriu!...).

***

"Ilmo. sr. delegado de polícia.

João Francisco de Oliveira, preso na cadeia desta cidade, por crime de
ferimento simples em Jacyntho Francisco de Paula, a despeito da ordem de
soltura em seu favor passada, vem respeitosamente perante V. S. requerer
o pronto cumprimento da citada ordem.

Contra o suplicante foi dada queixa pelo ofendido, que teve a poética
lembrança de qualificar o fato como tentativa de morte, no calculado
intuito de obter, como indebitamente obteve, a injusta detenção do
suplicante; e sendo certo que para estabelecer indiciação legal sejam
precisos fatos que autorizem a convicção do juiz e não baste, para isso,
a simples alegação do queixoso, o suplicante, em nome da lei.

P{[}ede{]} à V. S. e espera benigno deferimento.

Jundiaí, 3 de Julho de 1870.

Pelo suplicante,

Luiz Gama.

(Despacho)

"Não tem lugar o que requer o suplicante.

Jundiaí, 3 de Julho de 1870.

Santos Camargo".

***

"Pedro fora \emph{ferido levemente} por João; contra o seu ofensor Pedro
deu queixa por \emph{tentativa de morte}, pedindo a detenção
incontinente\footnote{. O mesmo que incontinenti, imediatamente.} do
acusado, fato que verificou-se.

Pergunta-se:

Sendo João residente e morador do foro do delito, e não tendo o autor
provado por modo algum a indiciação criminosa, é regular a detenção do
acusado?

Para determiná-la seria bastante a simples alegação do queixoso?

Resposta ao 1º quesito:

A prisão só pode ter lugar nos casos de flagrante delito e indiciamento
em crimes inafiançáveis -- Código de Processo Criminal, arts. 131 e 175.

Nesta última hipótese é necessário, como condição legal, a ordem escrita
da autoridade competente.

O arbítrio conferido pela lei ao juiz -- para prender ou deixar de
prender --, nos casos de inafiançabilidade do delito, não pode ser
entendido de modo a autorizar a prisão sem motivo algum que, \emph{pelo
menos}, faça presumir a existência jurídica do delito.

A simples alegação ou petição do queixoso não pode por si só ser motivo
suficiente para a ordem de prisão, sob pretexto de haver alguém cometido
crime inafiançável. Está visto que se o juiz tiver \emph{fundamentos
legais} para ordenar a prisão pode fazê-lo na forma da lei --
\emph{ex-officio}.

O segundo quesito está respondido com a resposta do primeiro.

É este o nosso parecer.\footnote{. O parecer é escrito, como se vê, por
  José Bonifácio e seu irmão Antonio Carlos Ribeiro de Andrada Machado.
  A formulação da consulta, a que o parecer se vincula, é de autoria de
  Gama.}

S. Paulo, 3 de Julho de 1870.

\emph{Dr. Antonio Carlos R. de A. Machado e Silva}\footnote{. Antonio
  Carlos Ribeiro de Andrada Machado e Silva (1830-1902) nasceu em Santos
  (SP) e pertence à segunda geração dos Andradas, sendo sobrinho de José
  Bonifácio, "O Patriarca", e filho de pai homônimo. Foi político,
  advogado, professor de Direito Comercial na Faculdade de Direito de
  São Paulo, e sócio de Luiz Gama, por aproximadamente uma década, em um
  escritório de advocacia.}.\\
\emph{José Bonifácio}\footnote{. José Bonifácio de Andrade e Silva, o
  Moço (1827-1886), nasceu em Bordeaux, França, e viveu grande parte da
  vida em São Paulo, onde se graduou e foi professor de Direito. Poeta,
  literato, foi na política que alcançou maior notoriedade, como
  deputado, ministro e senador em sucessivos mandatos desde o início da
  década de 1860.}.

Concordo.

\emph{Dr. Francisco Justino Gonçalves de Andrade}\footnote{. Francisco
  Justino Gonçalves de Andrade (1821-1902), nascido na Ilha da Madeira,
  Portugal, formou-se e fez carreira jurídica em São Paulo. Foi
  professor de Direito Natural e Direito Civil, alcançando notoriedade
  nesse último campo como autor de diversos livros doutrinários.}.

Concordo.

\emph{Crispiniano}\footnote{. José Crispiniano Soares (1809-1876),
  nascido em Guarulhos (SP), foi político advogado e professor de
  Direito Romano da Faculdade de Direito de São Paulo. Figura de
  destaque na política, foi presidente de quatro províncias do Império,
  respectivamente: Mato Grosso (1847-1848), Minas Gerais (1863-1864),
  Rio de Janeiro (1864) e São Paulo (1864-1865).}.

Concordo.

\emph{J. S. Carrão}\footnote{. João da Silva Carrão (1810-1888), o
  conselheiro Carrão, nasceu em Curitiba (PR) e foi advogado e político.
  Presidiu as províncias do Pará (1857-1858) e de São Paulo (1865-1866),
  foi deputado sucessivas vezes, ministro da Fazenda (1866) e senador do
  Império (1880-1888)}.

Concordo.

\emph{Falcão Filho}\footnote{. Clemente Falcão de Souza Filho
  (1834-1887) foi um advogado, empresário e professor catedrático de
  Direito Civil da Faculdade de Direito de São Paulo.}.

Concordo.

\emph{Dr. J. J. de Almeida Reis}\footnote{. José Joaquim de Almeida Reis
  (?-1874) foi professor substituto da Faculdade de Direito de São
  Paulo.}.

Jundiaí, 5 de Julho de 1870.

LUIZ GAMA

\textbf{O VELHO OESTE -- E O VELHO VALE! -- MANDA LEMBRANÇAS}

\textbf{*didascália*}

\emph{Não é segredo que, mesmo após a demissão do cargo de amanuense da
Secretaria de Polícia da capital, Gama dobrava sua aposta nas causas de
liberdade no judiciário, o que, por sua vez, se constituía como
expressão de excelência de seu abolicionismo. É bom relembrarmos,
contudo, que um pouco antes da demissão o seu amigo e ex-chefe de
polícia, Furtado de Mendonça, lhe avisara de que os escravocratas
graúdos não se contentariam apenas em vê-lo longe da repartição
policial, mas também o perseguiriam. Isso tudo fazia muito pouco tempo.
Foi entre novembro e dezembro de 1869. A demissão realizou-se. Restava a
perseguição que, diria Gama àquela altura dos acontecimentos, esperaria
"de ânimo tranquilo". E ela chegou. Três dos cinco textos que compõem
essa seção mencionam, no todo ou em parte, a perseguição que ele sofria.
Em carta privada ao seu filho de onze anos de idade, Benedicto Gama, o
velho Gama confessou ao filho que escrevia aquelas palavras "em momento
supremo, sob a ameaça de assassinato". Para o amigo José Carlos
Rodrigues, Gama comenta que a perseguição chegou a tal ponto que, para
garantir-lhe a vida, ele teve "a casa rondada e guardada pela gentalha",
i.e., a plebe, a quem ele sem dúvida era muito grato. Uma carta aberta
deixava a ameaça de morte anunciada ao público. "Pessoa de subida
distinção desta cidade possui documento, que foi-me manifestado", dizia
Gama, "de que os meus gratuitos inimigos do município de ***, estão
resolvidos a enviar-me para a eternidade". Se o município fica oculto
por trás dos asteriscos, o leitor poderá recorrer às "porteiras do velho
oeste", segunda seção desse volume, para ver que Gama estava envolvido
em lutas judiciais contra alguns "figurões da terra" de Amparo, Campinas
e Jundiaí. Outro artigo que compõe essa seção é o intitulado "Jacareí".
Nele, Gama não é o ameaçado, mas, o que nos é sugestivo, presta
solidariedade e denuncia um possível atentado contra um amigo e, assim
como ele, advogado abolicionista. Embora não fosse o titular da ação,
Gama tomou parte na causa do pardo Benedicto, que corria no juízo
municipal de Jacareí. Assim, é igualmente provável que Jacareí fosse o
município oculto sob os tais asteriscos. Seja como for, Amparo ou
Campinas, no velho oeste paulista, ou Jacareí, no vale do Paraíba
paulista, todas elas poderiam ser o local de onde surgia o plano de
atentado contra Gama. Mesmo Jundiaí, no limite norte da capital, poderia
ser o enigmático domicílio dos "gratuitos inimigos" de Gama. Mas um dos
cinco artigos da seção, intitulado "Comarca de Campinas", demonstra que
Gama, na prática, tocava o barco em frente. A despeito de tamanhas
contrariedades, e no fervor da perseguição, Gama produzia sua literatura
normativo-pragmática, o que, em última instância, significava que ele,
"sem temer os arrojos de alguns salteadores depravados", continuaria na
luta. }

\textbf{16. JACAREÍ}\footnote{. In: \emph{Correio Paulistano} (SP), A
  Pedido, 11/09/1870, p. 3.}

\textbf{*didascália*}

\emph{Gama desagrava o seu "particular amigo", Henrique Marques de
Carvalho, advogado que cuidava de ações judiciais abolicionistas em
comarcas do vale do Paraíba. Carvalho havia aberto uma demanda de
liberdade em favor do pardo Benedicto no juízo municipal de Jacareí.
Porém, "alguns mandões" tramavam uma represália contra o advogado
daquela "torturada questão de liberdade". Mas não eram quaisquer
mandões. Eram da laia de escravocratas "para os quais o espancamento, a
perseguição e o assassinato não passam de uma diversão prazenteira".
Assim, Gama tanto reclamava para que se salvaguardasse a liberdade de
Benedicto, tirando-o do cativeiro ilegal, quanto se zelasse pela
segurança e pela vida do "pobre e honesto" Henrique Marques de Carvalho.
Para Gama, era evidente que, por defender Benedicto, Carvalho estava na
linha de tiro dos figurões do vale do Paraíba. Com o conhecimento de
causa que a vida lhe dava, Gama afirmava, ao melhor estilo de sua verve
literária e leitura de realidade: "eu sei que há mistérios extravagantes
no concerto e perpetração de certos crimes, mistérios que o critério e a
razão repelem como quimeras vãs, mas que a realidade sinistra incumbe-se
de explicar à beira dos túmulos". O alerta de Gama era inequívoco (e
valia para ele também): se nada parasse os escravocratas de Jacareí,
matariam o advogado Carvalho; se nada temessem os escravocratas de São
Paulo, matariam a ele próprio. Mais adiante Gama escreveria: "Façam o
que entenderem. Eu estou no meu posto de honra".}

\emph{***}

O distinto sr. coronel Joaquim Antonio de Paula Machado\footnote{.
  Joaquim Antonio de Paula Machado (1824-1884) foi tenente-coronel e
  juiz municipal em Jacareí (SP).} teve a extrema bondade de responder
aos dois artigos insertos no \emph{Correio Paulistano} de 12 e 17 do
pretérito,\footnote{. Cf. \emph{Correio Paulistano} (SP), A Pedido,
  Jacareí, 12/08/1870, p. 2; \emph{Correio Paulistano} (SP), A Pedido,
  Atentado contra a liberdade, 17/08/1870, pp. 2-3.} sob a firma do meu
particular amigo, o sr. dr Henrique Marques de Carvalho\footnote{. Das
  raras referências que se encontram desse advogado, sabe-se que colou
  grau na Faculdade de Direito de São Paulo (1866) e advogou em
  Piracicaba (SP) e Jacareí (SP). Ao tempo desse artigo, Carvalho estava
  envolvido com demandas de liberdade no vale do Paraíba, interior
  paulista.}, relativos à torturada questão de liberdade, proposta no
juízo municipal de Jacareí, em favor do pardo Benedicto.\footnote{. A
  descrição do caso e qualificações individuais de Benedicto podem ser
  lidos no primeiro artigo de Carvalho. Vejamos um trecho do que Gama
  definia como uma "torturada questão de liberdade" e que, já à primeira
  vista, contém traços criminosos cometidos por autoridades judiciárias
  de Jacareí: "Benedicto, filho de Alexandrina, outrora escravo do
  vigário Fabiano Martins de Siqueira, tendo ciência de que era liberto
  e o mantinham em cativeiro injusto há mais de trinta anos, requereu
  seu direito. Então, logo contra sua pretensão, levantou-se terrível
  celeuma e, para arredá-lo de seu intento (...) o coronel Joaquim
  Antonio de Paula Machado, parente dos interessados, contra a vítima de
  tão iníqua perseguição, chamou o caso a si (...) sem permitir-se que
  Benedicto fosse ouvido por si ou {[}por{]} alguém para sustentar seu
  direito". Cf. \emph{Correio Paulistano} (SP), A Pedido, Jacareí,
  12/08/1870, p. 2}

Lamento sinceramente que nessa meditada resposta, publicada no
\emph{Diário} de 2 do corrente,\footnote{. Cf. \emph{Diário de S. Paulo}
  (SP), Publicações Pedidas, Ao público, 02/09/1870, p. 2.} o sisudo sr.
coronel Joaquim Antonio de Paula Machado, tratando-se de uma questão de
tanta magnitude, transpusesse as lindes\footnote{. Raias, limites.} da
seriedade que o caracteriza e descesse à facécias\footnote{. Chacotas,
  pilhérias.} de mau gosto, impróprias da sua posição social, da sua
madureza de idade e do seu reconhecido critério. S. S. deixou de ser
conveniente e grave, como fora para desejar, à força de querer
representar de jogral, papel pouco digno de um juiz que justifica-se
perante os seus concidadãos.

Não é a pessoa do pobre e honesto sr. dr Henrique Marques de Carvalho
que está em discussão, senão os sagrados direitos de um homem infeliz,
que diz-se indebitamente escravizado, que reclama ansioso a proteção das
leis e o império da justiça em seu auxílio, e que apenas tem encontrado
o apoio do advogado.

Não são os defeitos ou imperfeições físicas do ilustre advogado que
correm perigo diante da funesta prepotência de alguns mandões
impudicos\footnote{. Imorais, sem-vergonha.}, para os quais o
espancamento, a perseguição e o assassinato não passam de uma diversão
prazenteira; são a segurança e a vida de um homem que a nós, como à S.
S., na qualidade de juiz, cumpre amparar para que não seja ele vítima do
bacamarte\footnote{. Antiga arma de fogo de cano curto e largo.}, como
hão sido muitos outros cidadãos pacíficos nessa memorável localidade.

Lembrar-se-á o distinto sr. coronel Joaquim Antonio de Paula Machado, e
desculpar-me-á que eu lhe torture a memória com fatos tão cruciantes
que, não há muito tempo, o seu honrado parente e íntimo amigo, o sr.
tenente-coronel Claudio Machado, que não era mais airoso\footnote{.
  Elegante.} de porte, melhor disposto de membros, nem mais robusto de
ânimo do que o sr. dr. Marques de Carvalho, sofreu às portas da cidade
de Jacareí uma ousada tentativa de assassinato; e lembrar-se-á mais S.
S., que \emph{quatro} foram os formidáveis capangas, adrede\footnote{.
  Previamente, antecipadamente.} escolhidos e assalariados, que, em alto
dia, com afronta inaudita\footnote{. Sem precedentes.} das autoridades,
realizaram o tenebroso atentado.

Não há, pois, motivos para admirar-se o distinto coronel que o digno
advogado, em sua \emph{mórbida imaginação},\footnote{. Gama citava uma
  expressão do coronel-juiz Paula Machado, que dizia ser a narrativa de
  Carvalho, na parte em que denunciava a violência iminente que se
  tramava contra ele e Benedicto, "efeito de \emph{imaginação mórbida}".
  Cf. \emph{Diário de S. Paulo} (SP), Publicações Pedidas, Ao público,
  02/09/1870, p. 2. Grifos originais.} criasse a fantástica suspeita de
estar sendo espreitado por \emph{dois} facínoras.

Não espantou-me, entretanto, força é confessá-lo, a incredulidade
ingênua que manifesta em seu escrito o respeitável sr. coronel Joaquim
Antonio de Paula Machado; eu sei que há mistérios extravagantes no
concerto e perpetração de certos crimes, mistérios que o critério e a
razão repelem como quimeras vãs, mas que a realidade sinistra incumbe-se
de explicar à beira dos túmulos.

Não quero, por enquanto, discutir a intrincada questão do célebre exame
de falsidade do assentamento de batismo do pardo Benedicto, feito
perante o sr. coronel Joaquim Antonio, como juiz municipal, exame que
verificou-se com exclusão inexplicável do advogado de Benedicto;
oportunamente tratarei desse fato.\footnote{. Não localizei tal escrito
  que, ao invés das colunas da imprensa, pode ter recebido outro
  formato, a exemplo de uma carta cerrada ao presidente da província ou
  uma petição ao juiz municipal de Jacareí.}

É certo, porém, que essa diligência camarária\footnote{. Expressão
  jurídica que aponta para uma questão conduzida sem observância estrita
  das expressas formalidades processuais ou, ainda, para decisões
  inadmissivelmente tomadas a portas fechadas.}, praticada com exclusão
da parte e do seu advogado, encerra uma monstruosidade jurídica, que não
comportam a civilização atual e o decoro dos tribunais.

Não discutirei também a responsabilidade do sr. coronel, a idoneidade
dos peritos, e a sua competência como juiz. Para tudo há tempo, e eu não
costumo afirmar quando não posso provar.

Vim à imprensa apenas para defender a reputação do ilustre sr. dr.
Marques de Carvalho, que tão severamente há sabido cumprir o seu árduo
ministério de advogado, e renovar ao respeitável sr. coronel Joaquim
Antonio de Paula Machado a súplica que fiz-lhe em particular:

Nós não pretendemos, nem contamos com os favores da justiça, porque não
acreditamos que os seus ministros possam fazê-los; exigimos simplesmente
a restrita observância da lei e a manutenção integral do
direito.\footnote{. Gama afirma ter tratado com o juiz municipal de
  Jacareí, o coronel Paula Machado, em privado. Isso dá uma dimensão
  interessante para sua atuação na demanda de liberdade de Benedicto.
  Ainda que não tivesse peticionado naquela "torturada questão de
  liberdade", havia participado dela -- e é de se supor que a
  participação tivesse sido decisiva para a estratégia levada a cabo em
  Jacareí -- e não parecia ser uma simples coadjuvação de alguém que
  tomara pé da situação pelas colunas da imprensa.}

Queremos somente que os juízes cumpram o seu dever.

S. Paulo, 8 de Setembro de 1870.

LUIZ GAMA.

\textbf{17. CARTA AO FILHO BENEDICTO GRACCHO PINTO DA GAMA}\footnote{.
  In: MENUCCI, Sud. \emph{O precursor do abolicionismo no Brasil (Luiz
  Gama)}, p. 145. Escrita em 23/09/1870, o conteúdo da carta confere com
  a denúncia pública que fez na imprensa, no dia 24/09/1870, sobre a
  possibilidade de um atentado fatal contra ele. É sugestivo, porém, que
  mesmo Mennucci, não conhecendo o teor da segunda carta, que se lê na
  sequência dessa, soubesse, provavelmente por fontes orais, de
  circunstâncias que apenas na segunda carta revela. Vejamos como
  Mennucci contextualizou a ameaça de morte e a carta ao filho: "Das
  ameaças, ficou-nos um documento insuspeito. É a carta que escreveu ao
  filho, a 23 de setembro de 1870. Dizem que foi traçada pouco antes de
  seguir para o interior do Estado, onde ia defender um réu escravo.
  Embora difícil de averiguar, parece que a atmosfera formada em torno
  desse julgamento, pelos interessados na condenação do negro,
  autorizava a supor que a vida de Gama corria perigo e que sua cabeça
  estava a prêmio. Não me foi possível apurar o caso, documentalmente. A
  carta, entretanto, não deixa dúvida em que Gama atravessava um dos
  momentos mais críticos de sua vida e que tinha certeza de que
  pretendiam eliminá-lo. É o que se vai verificar, lendo-a". Cf. Sud
  Mennucci, \emph{O precursor} \emph{do abolicionismo no Brasil (Luiz
  Gama)}, 1938, especialmente pp. 144-145.}

\textbf{*didascália*}

\emph{A carta de Luiz Gama ao seu filho único, Benedicto Gama, é de suma
importância para uma história que, mais que pessoal e familiar, é a
história de um povo e de um país. A carta possui características que
fazem dela um verdadeiro testamento moral. Benedicto tinha apenas onze
anos de idade quando recebeu essa carta. Se abriu ou não, é uma boa
questão, haja vista que o pai poderia ter ordenado que a abrisse apenas
e estritamente se ele lhe faltasse. Seja como for, quando a carta foi
aberta lá estavam aquelas que seriam as possíveis últimas palavras
escritas pelo velho Gama. A carga de emoção em cada linha e a sobriedade
da forma não deixam indiferente aquele que lê. Que dirá o menino e filho
Benedicto. As ordens e orientações que o pai dá ao filho serviriam de
guia para o menino de onze anos, caso acometido pela tragédia que o
remetente e pai vislumbrava como iminente. A carta, em síntese, reflete
o estado anímico de seu autor e retrata a gravidade do momento. Pedia ao
filho que não se atemorizasse da "extrema pobreza" que o pai lhe legava
e que lembrasse das circunstâncias daquela missiva: "Lembra-te que
escrevi estas linhas em momento supremo, sob a ameaça de assassinato.
Tem compaixão de teus inimigos, como eu compadeço-me da sorte dos
meus".}

\emph{***}

Meu filho.

Dize a tua mãe que a ela cabe o rigoroso dever de conservar-se honesta e
honrada; que não se atemorize da extrema pobreza que lego-lhe, porque a
miséria é o mais brilhante apanágio\footnote{. Atributo, privilégio,
  espécie de recompensa.} da virtude.

Tu evita a amizade e as relações dos grandes homens; eles são como o
oceano que aproxima-se das costas para corroer os penedos\footnote{.
  Rochedo, grande pedra.}.

Sê republicano, como o foi o Homem-Cristo. Faze-te artista; crê, porém,
que o estudo é o melhor entretenimento, e o livro o melhor amigo.

Faze-te apóstolo do ensino, desde já. Combate com ardor o trono, a
indigência e a ignorância. Trabalha por ti e com esforço inquebrantável
para que este país em que nascemos, sem rei e sem escravos, se chame
Estados Unidos do Brasil.

Sê cristão e filósofo; crê unicamente na autoridade da razão e não te
alies jamais a seita alguma religiosa. Deus revela-se tão somente na
razão do homem, não existe em Igreja alguma do mundo.

Há dois livros cuja leitura recomendo-te: a \emph{Bíblia Sagrada} e a
\emph{Vida de Jesus} \footnote{. Obra seminal lançada em 1863 e
  reimpressa centenas de vezes ao longo dos séculos XIX e XX, \emph{Vida
  de Jesus} formou gerações de pensadores racionalistas e humanistas na
  França e no exterior, argumentando a existência de um Jesus histórico
  e mundano. Sobre a importância de Renan na formação de Gama, cf: Lígia
  Fonseca Ferreira, \emph{Luiz Gama: um abolicionista leitor de Renan},
  Estudos Avançados, 2007, vol. 21, nº 60, pp. 271-288.}por Ernesto
Renan\footnote{. Joseph Ernest Renan (1823-1892) foi um escritor,
  filósofo, filólogo e historiador francês. Pelo contexto da citação,
  Gama se revelava um admirador e leitor dedicado da obra de Renan,
  especialmente do humanismo e da história do cristianismo tal qual
  interpretada por Renan. Mantenho o aportuguesamento do prenome de
  Renan conforme o original.}.

Trabalha e sê perseverante.

Lembra-te que escrevi estas linhas em momento supremo, sob a ameaça de
assassinato. Tem compaixão de teus inimigos, como eu compadeço-me da
sorte dos meus.

Teu pai

\emph{Luiz Gama}.

\textbf{18. AO PÚBLICO}\footnote{. In: \emph{Correio Paulistano} (SP),
  24/09/1870, p. 2.}

\textbf{*didascália*}

\emph{Escrita no mesmo dia da carta ao seu filho Benedicto, Gama
advertia ao público que tomara conhecimento de uma séria ameaça dirigida
contra ele. Ele até pontua que aquela não era nem a primeira e nem a
segunda -- "Mais de uma vez..." -- vez que amigos "residentes no
interior da província" alertaram-no de "planos de atentados sérios"
tramados contra ele. Gama revela que tomou precauções diante de tais
ameaças, muito embora afortunadamente elas não viessem a se concretizar.
"Hoje, porém, o caso é mais sério". Alguém de bastante prestígio em São
Paulo, provavelmente uma autoridade de alto escalão, o procurou com um
documento em mãos que não deixava dúvidas de que pretendiam matá-lo. As
três frases que seguem a revelação de que a ameaça de morte não se
resumia a qualquer figura retórica são cabais: "Façam o que entenderem.
Eu estou no meu posto de honra. Tenho amigos em toda a parte". Ou seja,
não demonstrava o mínimo receio, mantinha a bandeira em riste e
convocava os "amigos em toda a parte". Além das duas cartas -- uma
privada, ao filho, outra pública, para a sociedade paulista --, Gama
também mobilizava seus companheiros espalhados em cada recôndito da
província. }

\emph{***}

Mais de uma vez amigos íntimos e importantes, residentes no interior da
província, hão-me dado aviso para acautelar-me, com segurança, contra
planos de atentados sérios, projetados contra minha humilde pessoa.

Entendi dever prevenir-me e nisto fiz consistir o meu plano de
represália.

Hoje, porém, o caso é mais sério.

Pessoa de subida distinção desta cidade possui documento, que foi-me
manifestado, de que os meus gratuitos inimigos do município de ***,
estão resolvidos a \emph{enviar-me para a eternidade}.

Façam o que entenderem.

Eu estou no meu posto de honra.

Tenho amigos em toda a parte. E se os que almejam o meu assassinato,
pessoas que eu bem conheço, estão vivos, devem-no a minha
nímia\footnote{. Excessiva.} prudência.

Podem, entretanto, satisfazer o seu magno e louvável intento.

Eu continuarei na empresa encetada\footnote{. Iniciada, que está em
  desenvolvimento.}, sem temer os arrojos de alguns salteadores
depravados.

São Paulo, 23 de Setembro de 1870.

LUIZ GAMA.

\textbf{19. COMARCA DE CAMPINAS}\footnote{. In: \emph{Correio
  Paulistano} (SP), A Pedido, 15/10/1870, pp. 2-3.}

\textbf{*didascália*}

\emph{Em plena luta política com inimigos que, como disse no artigo
precedente, pretendiam enviar-lhe "para a eternidade", Gama acha tempo
para seguir a tarefa que cada vez mais tomava para si de debater
questões de direito, comentar sentenças de juízes e produzir
conhecimento normativo. Como se a marcha dos acontecimentos caminhasse
suave, Gama resolvia escrever um texto dirigido tanto aos cidadãos
campineiros, quanto aos interessados, sobretudo do alto escalão, no
mundo do direito na província. "Trata-se de assunto grave, discute-se um
ponto importante de direito criminal", apresentava Gama. Depois passaria
a uma síntese do caso e suas conclusões sobre as evidências reunidas e
resumidas ao leitor. No entanto, já no primeiro parágrafo, o autor
adiantava que, além do mérito da causa, discutiria "a notável sentença
firmada pelo eminente jurisconsulto" Vicente Ferreira da Silva Bueno.
Para que o leitor relembre, Silva Bueno foi aquele mesmo que, em
novembro de 1869, como chefe de polícia interino, assinou a portaria de
exoneração de Gama do cargo de amanuense da Secretaria de Polícia da
capital. Gama qualificou o ato de Silva Bueno como um ato criminoso,
baseado numa mentira, levado a cabo para satisfazer o gosto de uma
administração corrupta. Portanto, onze meses após aquele fatídico
evento, esse seria o reencontro entre Silva Bueno e Gama. Que o leitor
espere -- porque terá -- o melhor de Gama. Era o caso de uma acusação de
estelionato contra o caixeiro-viajante João Baptista das Chagas. Gama,
muito provavelmente com o processo em mãos, destaca o depoimento de sete
testemunhas, de onde deduzia nove considerações pacíficas e consensuais,
no todo ou em parte, entre todos os testemunhos. O juiz municipal de
Jundiaí acatara as provas testemunhais, essas mesmas que Gama trazia a
público, e pronunciava o caixeiro-viajante Chagas como incurso no crime
de estelionato. O réu apelou da sentença para o juiz de direito da
comarca e finalmente a causa chegava à escrivaninha do juiz Vicente
Ferreira da Silva Bueno. A crítica de Gama é fulminante. Ele deixaria
que a sentença de Silva Bueno falasse por si. Evitaria adjetivações e
explanações sobre a sua forma e conteúdo. "A sentença que passo a
transcrever fielmente", dizia Gama, "e com a própria ortografia
original, é prova cabal do que afirmo". Após a "penosa transcrição", o
arremate viria em tom satírico, trazendo uma anedota burlesca -- e ao
mesmo tempo uma reprimenda de uma autoridade acadêmica -- sobre a
sentença de Silva Bueno. Gama dava o troco. E não pararia por aí. }

\emph{***}

A praxe e estilo de julgar, e decisão dos arestos\footnote{. Acórdão,
  decisão de tribunal que serve de paradigma para solucionar casos
  semelhantes.} seguida universalmente dos doutores do reino, é o melhor
intérprete da lei.

Ass. 23 de Março de 1786\footnote{. Assento de 23 de Março de 1786, Casa
  da Suplicação de Lisboa. Cf: Manuel Borges Carneiro, \emph{Direito
  Civil de Portugal: Das Pessoas}, tomo I, p. 49, 1826; Antonio Delgado
  da Silva, \emph{Colleção da Legislação Portugueza}, p. 401, 1828.}

Se dignos são de conceito, em jurisprudência pátria, e devem ser
rigorosamente observados no foro do Império os assentos\footnote{. Há
  dúvida na transcrição, se "asserto" ou "assento". Embora a primeira
  seja mais legível, opto pela segunda, pois Gama relaciona essa palavra
  à epígrafe, parecendo mais adequado, portanto, transcrevê-la dessa
  forma.} legais que servem de epígrafe a este meu despretensioso
escrito, ouso humildemente invocar a ilustrada consideração dos doutos,
e das pessoas graduadas em direito, não só para a questão que passo a
expor com a maior fidelidade, como principalmente para a notável
sentença firmada pelo eminente jurisconsulto, o sr. juiz de direito da
comarca de Campinas, dr. Vicente Ferreira da Silva Bueno.\footnote{.
  Vicente Ferreira da Silva Bueno (1815-1873) teve longa carreira
  administrativo-judiciária, exercendo cargos de delegado de polícia,
  juiz municipal, juiz dos órfãos, juiz de direito e desembargador em
  diversas províncias, como Bahia, Paraná, São Paulo e Rio de Janeiro.
  Em 1869, era chefe de polícia interino da província de São Paulo,
  cabendo a ele papel de algoz no espetáculo da demissão de Luiz Gama do
  cargo de amanuense da Secretaria de Polícia.}

Trata-se de assunto grave, discute-se um ponto importante de direito
criminal, e a veneranda sentença do ilustre e provecto\footnote{.
  Experiente.} magistrado está para esta séria questão como o ponto de
apoio, fora da terra, para a celebérrima\footnote{. Superlativo de
  célebre, algo como muitíssimo célebre.} doutrina da alavanca\footnote{.
  Em breve síntese, a alavanca se apoia em um ponto fixo adequado
  (fulcro) para daí multiplicar força mecânica aplicada a um outro
  objeto. Ocorre que a analogia de Gama ironiza a sentença do juiz, na
  medida em que situa "a veneranda sentença" como um "ponto de apoio,
  fora da terra", isto é, algo, de fato, sem base fixa onde se possa
  apoiar e, por consequência, gerar efeitos.} do imortal
Arquimedes\footnote{. Arquimedes de Siracusa (287 a.C-212 a.C.) foi um
  matemático, astrônomo e inventor grego de influência determinante para
  o desenvolvimento da ciência na Antiguidade.}.

***

Os srs. Oliveira Cruz \& Silva, quando negociantes em Jundiaí\footnote{.
  Jundiaí, município paulista que fica 50 km distante de São Paulo (SP),
  era a principal cidade ao limite norte da capital.}, deram queixa, por
crime de estelionato, definido no art. 264, § 4º, do Código
Criminal,\footnote{. Previsão normativa para crimes de estelionato,
  sendo a hipótese do § 4º assim definida: "Em geral, todo e qualquer
  artifício fraudulento pelo qual se obtenha de outrem toda a sua
  fortuna, ou parte dela, ou quaisquer títulos".} contra João Baptista
das Chagas, e alegaram haver Chagas comprado no armazém dos autores
gêneros a crédito, dando-se, para conseguir a transação, como sócio de
João Antonio de Moraes, vulgarmente conhecido pelo nome de -- João
Rufino -- de quem aliás era simples caixeiro, e usando falsamente, para
realizar tal negócio, da firma -- João Baptista \& Rufino -- por ele
astuciosamente improvisada; isto no intuito de iludir, como de fato
iludiu, a boa fé dos queixosos.

Esta fundada alegação ficou evidentemente provada do seguinte modo.

TESTEMUNHAS:

\emph{Gabriel Fernandes da Costa Rego} -- Disse que, por ouvir ao
próprio João Baptista das Chagas, sabe que ele \emph{se dava como sócio}
de João Antonio de Moraes, vulgarmente conhecido por - João Rufino -;
sendo certo, entretanto, que Chagas \emph{era apenas empregado} da casa
de Moraes; assim como sabe mais, por ouvir ao referido Moraes, \emph{que
Chagas nunca foi seu sócio}. - Disse mais, que é certo, e sabe-se
vulgarmente, que Chagas comprava gêneros a diversos, a crédito, dando-se
como sócio de Moraes, e em nome da suposta firma -- João Baptista \&
Rufino.

\emph{Manoel dos Santos Teixeira do Amaral} -- Disse que sabe, por ter
ouvido ao próprio acusado, que ele comprara a Oliveira Cruz \& Silva,
negociantes estabelecidos nesta cidade, vários gêneros \emph{para a
suposta firma ou casa} de -- João Baptista \& Rufino -- e que sabe mais
ainda, por ouvir ao mesmo réu, que tais gêneros não foram pagos,
alegando o mesmo réu, que não efetuava o exigido pagamento por haverem
os gêneros sido comprados, não para ele réu, \emph{mas para a mencionada
firma e casa} -- João Baptista \& Rufino. Disse mais, que tem
conhecimento, por manter relações comerciais com João Antonio de Moraes,
conhecido por -- João Rufino -- \emph{que o réu presente nunca fora dele
sócio}, \emph{mas simples caixeiro}.

\emph{João Baptista de Sampaio} -- Disse que sabe, por ter visto, como
caixeiro que é dos suplicantes, haver o acusado presente comprado vários
gêneros a seus patrões (Oliveira Cruz \& Silva), na qualidade de membro
da firma social -- João Baptista \& Rufino --, gêneros que até hoje não
foram pagos; e que sabe que essa firma era fantástica, que não existia;
e que sabe que o acusado apenas era caixeiro de dita casa, e nunca
sócio.

\emph{Domingos Loureiro da Cruz} -- Disse que, por ouvir aos autores,
sabe que o réu comprara gêneros nesta cidade com a firma de -- João
Baptista \& Rufino -- e que também sabe que não existe aquela firma,
\emph{isto por ter lhe contado o próprio João Antonio de Moraes}, que
era o dono da casa em que o acusado presente era o simples caixeiro; e
que João Antonio de Moraes lhe dissera mais - \emph{ser ele o único
proprietário}, e responsável da sua casa de negócio; e isto deu-se em
ocasião em que ele depoente apresentou ao dito Moraes uma conta de
gêneros comprados pelo acusado \emph{em nome daquela firma social}.

Disse mais, que mais tarde comprando o depoente, nessa casa, um barril
de aguardente, nesse barril tinha as iniciais -- J.B. \& R. -- Disse
mais, que os gêneros que vendeu em sua casa de negócio foram para a de
João Antonio de Moraes, conhecido por João Rufino -- onde era empregado
o acusado presente; \emph{e que a tal casa de} - João Baptista \& Rufino
-- \emph{não existiu}; que ouviu dizer que o acusado comprara gêneros
para essa casa, mas não sabe se tais gêneros eram levados para a casa de
João Antonio de Moraes. Que sabe que João Antonio de Moraes recusou
pagar gêneros em Santos\footnote{. Cidade de Santos, no litoral
  paulista.}, porque tal firma não existia, e que não tinha dado ordem
para semelhante compra: e que as marcas existentes nos barris nada
querem dizer, PORQUE JOÃO RUFINO NÃO SABE LER NEM ESCREVER, e que, por
essa razão, não sabe qual a marca existente nos barris.

\emph{Manoel da Silva} -- Disse que ouviu falar que o acusado presente
era caixeiro de João Antonio de Moraes, conhecido por -- João Rufino --,
que realmente viu o réu presente comprar gêneros para a casa de João
Rufino.

\emph{Antonio José da Costa} -- Disse que ignora que houvesse nesta
cidade casa alguma comercial que girasse com a firma - João Baptista \&
Rufino - assim como ignora que o acusado presente usasse em tempo algum
de tal firma, e para qualquer fim comercial; sabendo entretanto que o
acusado presente era caixeiro de João Antonio de Moraes, conhecido por -
João Rufino - que o acusado presente comprara aos autores diversos
gêneros, como vinho, etc. para a casa comercial de seu amo João Antonio
de Moraes, \emph{mas que disto sabe por ter ouvido ao próprio réu},
\emph{e por ter visto os gêneros na casa de Moraes}. - Disse mais que,
em Santos, abonara , com a sua palavra, o acusado presente, para comprar
em casa de - Eugênio \& Lima -, ficando responsável, na falta do
pagamento, ele testemunha, \emph{isto por causa do crédito de que goza a
casa de Rufino}, para a qual deveriam ser comprados os gêneros, tendo
ele, testemunha, certeza de que Rufino os pagaria.

\emph{João Antonio de Moraes} -- Esta testemunha confirma o{[}s{]}
depoimento{[}s{]} das precedentes e, conseguintemente, a alegação dos
queixosos; e mais acrescenta -- "que o réu, quando seu caixeiro era
autorizado a pôr e dispor da sua casa, menos a fazer compras sem
especial autorização; \emph{sendo certo que não foi autorizado a comprar
gêneros na casa dos autores} ".

***

Deduz-se logicamente destes depoimentos:

1º: Que João Baptista das Chagas era caixeiro de João Antonio de Moraes,
e que dele jamais fora sócio;

2º: Que, por ser o amo analfabeto, era Chagas encarregado da gerência da
casa de negócio, e da respectiva correspondência;

3º: Que, conquanto tivesse Chagas a seu cargo a gerência da casa de
negócio, de exclusiva propriedade de João Antonio de Moraes, não podia
fazer compras sem expressa autorização de seu amo;

4º: Que nenhuma autorização teve Chagas para comprar gêneros, como
confessa ter comprado, aos negociantes Oliveira Cruz \& Silva;

5º: Que a firma social - João Baptista \& Rufino - fora ardilosamente
criada por Chagas para ilícitos fins;

6º: Que abusando Chagas da ignorância de seu amo, e no determinado
intuito de iludir o público, marcava os gêneros da casa com as iniciais
- J.B. \& R. - para calculadamente tornar crível a existência da suposta
firma social - João Baptista \& Rufino;

7º: Que por meio desse estudado e bem combinado embuste conseguiu Chagas
iludir a sincera credulidade dos autores, e dela houve, a crédito,
gêneros no valor de mais de 100\$000;

8º: Que este meio empregado por Chagas para obter os gêneros, -
\emph{simulando um fato comum e acreditável}, \emph{relativamente à sua
pessoa e posição}, fato que, porém, não era real e antes astuciosamente
por ele inventado, de má fé, constitui \emph{artifício fraudulento};

9º: Que João Antonio de Moraes, logo que teve conhecimento deste ardil,
ao qual, naturalmente por ignorância, não deu o devido valor,
\emph{declarou a diversas pessoas com quem tinha relações comerciais},
que Chagas era seu caixeiro, e não sócio, e negou-se a pagar dívidas
contraídas pelo seu dito caixeiro, sem autorização sua, sob a inventada
firma João Baptista \& Rufino.

Conclui-se, portanto, que este procedimento de João Baptista das Chagas,
em face da seguinte disposição do Código Criminal, o constitui
irremediavelmente na posição dificílima de réu indefeso.

"Art. 264 - Julgar-se-á crime de estelionato:\\
{[}(...){]}\\
§ 4º: Em geral, todo e qualquer artifício fraudulento pelo qual se
obtenha a sua

fortuna, ou parte dela, ou quaisquer títulos". \footnote{. Transcrição
  praticamente idêntica ao texto normativo. Embora falte apenas uma
  palavra, não há alteração de sentido.}

Assim entendendo o distinto sr. delegado de polícia e juiz municipal do
termo de Jundiaí, proferiu a seguinte sentença de pronúncia:

"Vistos e examinados estes autos, etc., julgo procedente a queixa dada
pelos autores Oliveira Cruz \& Silva, contra o réu preso João Baptista
das Chagas, pelo crime de estelionato, porquanto acha-se provado nos
autos:

1º: Que o réu João Baptista das Chagas dizia-se sócio de João Antonio de
Moraes, vulgo João Rufino, em uma casa mercantil, nesta cidade, sob a
firma de -- João Baptista \& Rufino. (Testemunhas 1ª, 2ª e 3ª);

2º: Que a firma de - João Baptista \& Rufino -- nunca existiu, era uma
firma fantástica. (Testemunhas 1ª, 2ª e 3ª e outras);

3º: Que o réu era apenas caixeiro de João Antonio de Moraes.
(Testemunhas 1ª, 2ª e 3ª e outras);

4º: Finalmente, que o réu comprou a crédito, a algumas pessoas, entre
outras, aos autores, gêneros para a firma de -- João Baptista \& Rufino
--, alguns dos quais ainda não foram pagos. (Testemunhas 1ª, 2ª e 3ª).

Sendo assim, é claro que, com semelhante procedimento, o réu iludia
àqueles a quem comprava gêneros, porque estes, depositando confiança na
firma -- João Baptista \& Rufino --, vendiam-lhe os seus gêneros à
crédito: usou, portanto, o réu, na expressão do Código Criminal, de
artifício fraudulento para obter de outrem parte de sua fortuna. - Por
estas considerações, pois, pronuncio o réu João Baptista das Chagas
incurso no art. 264, § 4º, do Código Criminal, e sujeito à prisão e
livramento, etc.

(Assinado.)

\emph{Estevam José de Siqueira. }

***

Com esta justa sentença, que outra cousa não é senão o resumo fiel da
prova aduzida no sumário pelos autores, não concordou o laborioso réu, o
que aliás parece-me natural; e crente de que as suas industriais
aspirações outro e melhor prêmio mereciam, apelou de ânimo robusto para
o sr. dr. juiz de direito da comarca, de quem a idade, o saber, a
prática de julgar, e a proverbial\footnote{. Notória, amplamente
  conhecida.} prudência auguravam-lhe\footnote{. Prometiam-lhe.} mais
sábios e profícuos resultados; no que não enganou-se.

Aqui vem a ponto dizer, com um distinto escritor: "O gênio luta
manietado\footnote{. Amarrado, de mãos atadas.} nos cárceres, envolto na
sombria indiferença enquanto não lhe estende protetora mão a
munificente\footnote{. Generosa, magnânima.} sabedoria."

João Baptista das Chagas é o gênio; e o ilustrado sr. juiz de direito de
Campinas, como verdadeiro Júpiter da jurisprudência, bradou-lhe:
Alevanta-te, e caminha!..........

A sentença que passo a transcrever fielmente, e com a própria ortografia
original, é prova cabal do que afirmo:

"Visto e examinado o presente recurço, dou provimento ao mesmo para o
efeito de reformar, como reformo a \emph{pronuncia} recorrida de folhas,
que \emph{pronunciou} ao recorrente como incurço no art. 264, § 4º, do
Código Criminal, porquanto provado como está e consta do depoimento de
testemunhas que o recorrente era caixeiro da casa de negócio de João
Antonio de Moraes, conhecido em Jundiaí por João Rufino, e nessa
qualidade estava encarregado da gerência do negócio, não constando que
João Antonio de Moraes tivesse uma outra pessoa encarregada de
transações ou escrituração de sua casa, visto não saber ler nem
escrever, é claro que depositava no recorrente plena confiança e que lhe
deixava plena faculdade para praticar todos os atos a bem dos interesses
da casa, e disto naturalmente se deduz a faculdade de comprar e vender,
pagar e receber quantias, fazendo as transações necessárias a bem da
Casa.

Assim pouco importa saber se o recorrente inculcava-se ou não sócio da
casa, se usava ou não da firma social -- João Baptista \& Rufino --
(circunstância esta que não está suficientemente provada), se comprava e
vendia sob uma tal firma; porque se é verdade que ele assim
\emph{obrava}, se isto se dizia em Jundiaí e se era falço isto, se o
recorrente usava deste artifício fraudulento etc., é também verdade que
João Antonio de Moraes, morando em Jundiaí, nesse mesmo lugar onde o
recorrente praticava tais atos nunca os coibiu, nunca reclamou contra a
existência dessa sociedade, nunca praticou o menor ato \emph{por onde}
manifestasse ao público comercial de Jundiaí que não tinha semelhante
sócio etc.

Ora, a isto acresce que os objetos comprados pelo recorrente aos
recorridos foram aplicados a benefício da Casa de negócio de João
Antonio de Moraes, portanto -- sócio ou caixeiro o recorrente não
comprou para si, não converteu em seu proveito particular, e sim em
proveito da casa de que era sócio ou Caixeiro, e é muito para notar-se e
digno de reparo que João Antonio de Moraes morando em Jundiaí recebendo
gêneros de Santos com a marca J.B. \& R. (\emph{Note-se que o próprio
juiz já declarou que Moraes é analfabeto, no intuito de conferir ao réu
plenos poderes para dirigir a casa de seu amo ou sócio, palavras
sinônimas nessa memorável sentença)}\footnote{. Comentário original de
  Luiz Gama.} havendo compras de gêneros nas casas comerciais de
Jundiaí, como dizem os recorridos sob tal firma, etc. não \emph{tivesse}
João Antonio de Moraes ou João Rufino uma pessoa, um amigo, que lhe
advertisse da existência daquela firma, não tivesse quem lhe advertisse
que o recorrente seu simples caixeiro - se intitulava sócio, inventava e
usava de uma firma social que o podia comprometer e só depois que vendeu
o negócio ao Caixeiro sócio, e que nessa venda incluiu parte ou restos
daqueles gêneros comprados aos recorridos e com aquela firma e a
quaisquer outros, depois de João Antonio de Moraes se responsabilizar
pelas dívidas da casa é que vem os recorridos denunciar a falcidade
daquela firma, é que vem João Antonio de Moraes jurar que nunca existiu
semelhante firma, que ele nunca autorizou o recorrente a fazer compras,
que só se responsabilizou pelas dívidas por ele feitas, etc., etc.
confessando, porém, que o recorrente lhe merecia confiança e era quem
regia sua casa de negócio! donde se conclui que ao menos tacitamente
consentia nos atos por ele praticados.

Portanto, à vista do exposto e do mais dos autos onde existem à certos
respeitos testemunhas contraproducentes a{[}à{]} intenção dos
recorridos, não me convencendo da existência do artifício fraudulento da
parte do recorrente, reformo como disse a pronúncia de fls. e julgo
improcedente a queixa.

\emph{Dêsse} baixa na culpa ao recorrente e risque-se seu nome do rol de
culpados, e passe-se o alvará de soltura etc. etc.

VICENTE FERREIRA DA SILVA BUENO".

***

Ao terminar a penosa transcrição desta veneranda sentença, lembro-me de
referir aos benévolos leitores uma interessante anedota.

A um dos mais distintos lentes da faculdade jurídica desta cidade,
correligionário político do meritíssimo sr. dr. juiz de direito de
Campinas, mostrei uma cópia deste monumental prodígio de jurisprudência.
O homem leu-a com profunda atenção, e ao terminar a leitura, sorriu, e
disse com ênfase:

-- É um Nero\footnote{. Nero (37-68) foi imperador de Roma e passou à
  história como símbolo de tirania, truculência e violência.} este
Vicente!

Depois dobrou o papel e meteu na algibeira\footnote{. Pequeno bolso
  interno de uma peça de roupa, ou pequena bolsa, sacola.},
acrescentando: "Esta vai para o meu álbum de preciosidades..."

Felizes dos magistrados que são dignos de tão elevado conceito!...

S. Paulo, 10 de Outubro de 1870.

LUIZ GAMA.

\textbf{20. CARTA A JOSÉ CARLOS RODRIGUES}\footnote{. In: Biblioteca
  Nacional, Carta a José Carlos Rodrigues, Documento textual,
  Manuscritos - I-03-02,074, São Paulo, 26/11/1870.}

\textbf{*didascália*}

\emph{Gama endereça uma carta para Nova Iorque. O destinatário era, como
se verá, um amigo querido, tanto dele quanto de sua esposa, Claudina, do
filho Benedicto e do que parece ser uma agregada, ou parente de
Claudina, de nome Leopoldina. É interessante notar que, mais do que uma
correspondência remetida a um correligionário ou colega de ofício, Gama
se refere a José Carlos Rodrigues como um amigo íntimo seu e de toda sua
família, para quem faz questão de mandar lembranças. Claudina e
Leopoldina, diria Gama, "enviam-te muitas saudades". O relance da cena
familiar é um entre os tantos que recheiam a carta. Essa é sem dúvida
uma qualidade ímpar dessa missiva: revela pequenos detalhes, utensílios,
ambientes, acontecimentos e memórias, sobretudo memórias!, dignas de
encher os olhos do leitor. "Quantas recordações saudosas não despertam
estes objetos?...", diz Gama após minuciosa e preciosíssima descrição do
interior de sua casa, onde, que beleza!, "toma-se o saboroso café pelas
mesmas canecas que me deste". Ao se mudar em definitivo para os Estados
Unidos da América, onde se estabeleceria como jornalista, Rodrigues doou
um conjunto de utensílios para Gama e sua família. Gama, como fez
questão de frisar, ficou bastante grato pelo gesto amigo. Na carta, Gama
também trata de assuntos políticos, dá notícias de São Paulo,
sublinhando a recente fundação da Loja América e do Club Republicano,
entidades às quais se vincularia; conta ter sido demitido da polícia;
menciona estar em contato com "ministros presbiterianos de Nova York"
para ter notícias dele, Rodrigues; e segue por outros temas tanto da
política, quanto do que qualificava como "revolução moral" em curso na
província. Mas é no íntimo da casa -- e das memórias -- que a carta
ganha maior relevo. É uma carta escrita, diz Gama, com "as forças
d'alma". Nesse sentido, aliás, nessa força, Gama traria ao papel até
mesmo uma valiosíssima recordação dos tempos de criança. Notem bem: "Eu
ainda hoje, ao cabo de trinta anos, vejo algumas ruas da Bahia, as casas
demolidas pelo incêndio de 37, e os lugares em que brinquei com as
crianças da minha idade". A riqueza da visão da Bahia de 1837 é tanta
que abre uma janela para sua enigmática infância. A carta, como se lerá
adiante, é uma preciosidade sui generis dentre os seus escritos. E cabe
destacar, por fim, que tamanha beleza -- "ao traçar estas linhas nossas
almas se abraçam e entoam epinícios à amizade!" -- foi escrita em meio
ao fogo cruzado dos últimos meses, quando a cabeça de Gama esteve em
novo e sério risco. Ele faz menção a isso de maneira que nos é muito
útil para pensar sobre o período. "Sou detestado pelos figurões da
terra, que já puseram-me a vida em risco, mas sou estimado e muito pela
plebe. Quando fui ameaçado pelos grandes (...) tive a casa rondada e
guardada pela gentalha. A verdade é que a malvadeza recuou vencida". A
maldade que recuasse. Gama não tinha tempo a perder.}

\emph{***}

S. Paulo, 26 de Novembro de 1870.

José Carlos.

A leitura do \emph{Novo Mundo} veio despertar em mim a não cumprida
obrigação de escrever-te, que sobremodo pesava-me; e digo "despertar"
não porque estivesse eu adormecido, mas porque por ela avivaram-se-me as
forças d'alma.

Boas novas de ti tive-as sempre pelos ministros presbiterianos que de
Nova York vinham a esta cidade, e o fato de sabê-las eu de ti
dispensava-me de referi-las de mim, isto não sei se por egoísmo ou por
incúria\footnote{. Desleixo ou falta de iniciativa.}.

Os poucos e verdadeiros democratas desta cidade, onde já existem um
Clube\footnote{. Refere-se ao Club Radical Paulistano.} e uma loja
maçônica\footnote{. Refere-se à Loja América.} que trabalham pelas
ideias republicanas (escuso dizer-te que sou membro de ambos),
tomaram-se de sincero entusiasmo pelo \emph{Novo Mundo}, plaustro de
importantes e úteis conhecimentos da melhor porção da América, que é e
há de ser o farol da democracia universal.

O \emph{Correio Paulistano,} de propriedade do nosso Amigo Joaquim
Roberto, e hoje redigido pelo distinto dr. Américo Brazilio de
Campos\footnote{. Américo Brazilio de Campos (1835-1900), nascido em
  Bragança Paulista (SP), foi advogado, promotor público, jornalista e
  diplomata. Entre diversas colaborações na imprensa, foi redator
  d'\emph{O Cabrião}, diretor do \emph{Correio Paulistano} e fundador
  d'\emph{A Província de São Paulo}. Desde os seus tempos de estudante
  na Faculdade de Direito de São Paulo, na turma que se formou em 1860,
  até a ruptura pública dos finais de 1880, Américo de Campos foi um dos
  parceiros mais próximos de Luiz Gama, podendo ser encontrado em
  diversas fontes atuando ao lado de Gama na imprensa, na política ou na
  tribuna.}, ambos republicanos, vai transcrever a maior parte dos
artigos do \emph{Novo Mundo}.

Não te espantes deste meu republicanismo, que pode afigurar-se ao teu
espírito, afeito\footnote{. Habituado, acostumado. Importante notar que
  a expressão não carrega, necessariamente, estima ou afeição.} ao
servilismo político do Brasil, como sinais de {monomania}\footnote{.
  Espécie de insanidade mental em que uma ideia fixa predomina na
  consciência de um indivíduo.} {arrasadora} da minha parte; asseguro-te
que o Partido Republicano, graças à divina inépcia do sr. D. Pedro II,
organiza-se seriamente em todo império; e os pantafaçudos\footnote{.
  Grosseiros, ridículos.} politicões gangorreiros\footnote{. Não é
  possível cravar o sentido preciso do adjetivo, mas talvez faça alusão
  à gangorra, brinquedo que faz movimentos alternados de baixo para
  cima, e vice-versa, como metáfora da alternância de poder entre os
  dois únicos partidos -- conservador e liberal -- que se sucediam
  mutuamente ao longo do império. Gangorreiros, portanto, pode ser uma
  referência a políticos que oscilavam de um lado a outro da gangorra.}
já declaram-se impotentes para a irrisória obra das ardilosas
cerziduras\footnote{. Ação ou efeito de cerzir, de costurar, de
  remendar. No sentido figurado que se aplica ao contexto, cerzidura --
  grafada à época como "sirgidura" -- significa a costura de diversos
  agrupamentos políticos ("tecidos") em uma mesma bandeira.} do g{rande
estandarte liberal}, que desfaz-se em bandeirolas democráticas, roto
pelos anos de indiferentismo popular e pela enérgica pujança de alguns
caracteres sisudos.

A despeito das tricas\footnote{. Trapaças, sutilezas.} imoralíssimas
postas em prática pelos astuciosos adeptos do corrupto imperialismo, e
das prédicas\footnote{. Pregações, discursos.} calculadas dos
arquisectários da {Infalibilidade}, erguem-se vagarosamente as escolas
gratuitas para alumiamento do povo e organizam-se as associações
particulares para emancipação dos escravos\footnote{. Ilustração,
  esclarecimento.}.

Por outro lado, as seitas protestantes, com as doutrinas evangélicas que
difundem, vão proclamando a liberdade de consciência, base e fundamento
da melhor organização social.

Ainda mais um importante fato tenho que dizer-te.

Tudo isto marcha vagaroso como o caminhar da reflexão; é uma obra
secular na qual o {supremo artista} gasta os dias a somar os segundos e
os minutos; e a província de São Paulo, ocupando a vanguarda, vai
ensinando às suas Irmãs a trilha impérvia\footnote{. Aqui no sentido de
  impenetrável, inacessível.} que ela própria meditando explora. É uma
vasta revolução moral dirigida pela prudência.

\_\_\_\_\_\_

Meu caro José.

É plano inclinado este caminho da política; deixá-lo-ei para tratar de
outros fatos menos importantes e mais íntimos.

Casei-me. Escuso dizer-te com quem. O {Dito}\footnote{. Refere-se ao
  filho, Benedicto, e aqui confessa o apelido carinhoso pelo qual o
  chamava.} já fala, traduz e escreve o alemão como um filho da
Germânia. Isto é dito pelo professor que todos os meses empolga 51.000
réis. Estuda ele mais desenho, francês, inglês e geografia.

Ele, a Claudina e a Leopoldina, que ainda conserva o mesmo nariz de
{[}ilegível{]} narigado, enviam-te muitas saudades.

Fui demitido do lugar de amanuense da Repartição de Polícia, por
sustentar demandas em favor de gente livre posta em cativeiro
indébito!...

Fiz-me rábula e atirei-me à tribuna criminal. Tal é hoje a minha
profissão.

Moro à margem do Rio Tamanduateí, em uma nova e excelente casa de campo.

Sou detestado pelos figurões da terra, que já puseram-me a vida em
risco, mas sou estimado e muito pela plebe. Quando fui ameaçado pelos
grandes, que hoje encaram-me com respeito, e admiram a minha tenacidade,
tive a casa rondada e guardada pela gentalha.

A verdade é que a malvadeza recuou vencida.

Em nossa casa, sempre pobre, mas festejada de contínuo pela alegria,
ainda toma-se o saboroso café pelas mesmas canecas que me deste; os
lampiões são os mesmos que pertenceram-te; as cortinas das janelas foram
tuas. Sobre o velador\footnote{. Utensílio formado de uma haste de
  madeira que, assentada sobre uma base, tem na parte superior um disco
  circular onde usualmente se colocava um lampião ou velas.} de mármore,
que foi teu, está o álbum que deste-me com o teu retrato, com os de
outros amigos, e uma bíblia que foi do finado Macedo.

Quantas recordações saudosas não despertam estes objetos?... E como ao
ler estas linhas tão singelas como os meus sentimentos de pobre, não se
dilatará o teu espírito em demanda destes lugares que outrora
percorreste, durante a tua vida acadêmica, e com que avidez não buscará
ele a realidade destes meus assertos?!

Eu ainda hoje, ao cabo de trinta anos, vejo algumas ruas da Bahia, as
casas demolidas pelo incêndio de 37, e os lugares em que brinquei com as
crianças da minha idade. Por isso, pelo meu, julgo do teu espírito neste
momento.

Eu chego a persuadir-me que ao traçar estas linhas nossas almas se
abraçam e entoam epinícios\footnote{. Cântico feito para comemorar uma
  vitória ou o regozijo por um feliz acontecimento.} à amizade!....

A Deus José.

Sei que o Joaquim Roberto vai escrever-te, e remeter-te os jornais.

Sou como sempre

Teu Amigo obrigadíssimo

Luiz Gama.

\textbf{O HOMEM QUE MAMOU O LEITE DO LIBERALISMO}

\textbf{*didascália*}

\emph{Raphael Tobias de Aguiar, filho do brigadeiro Tobias de Aguiar,
ex-presidente da província de São Paulo, e de Domitila de Castro, a
famosa marquesa de Santos, ficou marcado, na criativa veia literária de
Gama, como o homem que mamou o leite do liberalismo. Tão criativa quanto
implacável, a sátira de Gama não deixou pedra sobre pedra na estrondosa
causa de liberdade do pardo Narciso. Certo dia, às seis horas da manhã,
Raphael Tobias mandou buscar o pardo Narciso para, em seguida, mandar
torturá-lo. Na fina ironia de Gama, Raphael castigava Narciso para
"curá-lo da mania emancipadora de que estava acometido!" A mania
emancipadora, ou sede de justiça, era tanto de Narciso quanto de Gama.
No que dependesse da dupla, a tortura não passaria impune -- nem no
plano retórico, nem no plano normativo. Gama escreve, então, uma série
riquíssima, que pode ser intitulada como "Coisas admiráveis". Admirável
seria um pretenso proprietário tomar posse daquilo que, por força
normativa, não era seu. Gama explica, passo a passo, porque Raphael
Tobias não tinha domínio algum sobre o ex-escravizado Narciso. Assim, em
não havendo domínio, não haveria por que haver posse; e, não havendo
posse, jamais poderia haver castigo. Gama constrói um raciocínio
juridicamente irretocável e que provavelmente figurou em comarcas da
província como doutrina exemplar para casos de alforrias testamentárias,
isto é, demandas de liberdade baseadas em testamento. Contudo, parte da
estratégia de liberdade do caso Narciso passava por ridicularizar
Raphael Tobias, a um só tempo o pretenso proprietário e o torturador de
pessoa livre. Na réplica às primeiras "Coisas admiráveis", Raphael
Tobias expunha suas razões em proceder daquele modo e se defendia das
acusações dizendo que era um liberal de berço, afinal, era um entre
"aqueles que com o leite materno beberam ideias liberais". A frase não
passaria despercebida e logo se converteria em mote para reforçar a
estratégia de liberdade de Gama. Raphael Tobias, "como ele próprio o
afirma (...), mamou com leite os princípios liberais que o distinguem''.
Gama usaria a metáfora de variadas maneiras, sempre demarcando
distinções morais entre ele e o carrasco. "Eu nunca mamei liberdade com
leite", diria Gama, arrematando com a visceral: "Eu não sou fidalgo; não
tenho instintos de carrascos; não mamei liberdade com leite". A tortura,
a escravidão e o liberalismo de Tobias de Aguiar seriam lados
equiláteros de um mesmo triângulo. "O sr. dr. Raphael Tobias sabe que em
nossa pátria o poder dos régulos é superior ao império da lei".
Dificilmente ele responderia pela tentativa de reescravização e pelo
castigo brutal em Narciso. No fim das contas, Raphael Tobias estava
"habituado a beber com leite princípios liberais, e a dar surras nos
seus escravos". Pode-se dizer, lendo as tais "coisas admiráveis", que a
mistura indigesta de sangue com leite constituía o peculiar liberalismo
escravocrata brasileiro do século XIX. "Essa é" -- a escravidão! --
"naturalmente a teta em que S. S. mama liberdade..."}

\textbf{21. TEM O ESCRAVO ESCOLHA DE FORO PARA A PROPOSITURA DE AÇÃO
MANUMISSÓRIA?}\footnote{. In: \emph{A Província de S. Paulo} (SP), Seção
  Judiciária, Tribunal da Relação, 02/08/1877, p. 2. O \emph{Correio
  Paulistano} repercute a publicação desse estudo chamando-o de "uma
  análise jurídica do advogado sr. Luiz Gama (...)." A ação manumissória
  era uma das formas processuais pelas quais se demandava a liberdade.}

\textbf{*didascália*}

\emph{Gama disseca os fundamentos jurídicos de um acórdão do Tribunal da
Relação de São Paulo que decidiu pela limitação do direito do
escravizado em propor uma causa de liberdade. Ao negar um recurso de uma
pessoa escravizada, os desembargadores do tribunal paulista fixavam uma
doutrina que aniquilaria a possibilidade de alguém escravizado demandar
sua liberdade fora do domicílio em que vivia. Diziam os desembargadores
que não caberia exceção -- mesmo em matérias de liberdade -- ao
"princípio geral que estabelece a competência do juiz do domicílio do
réu". Se esta parece uma questão menor, basta pensar em uma pessoa
escravizada em fuga, ou seja, alguém distante do domicílio do réu (nesse
caso, contra quem se demandava). O tribunal decidia, portanto, que o
escravizado, se quisesse lutar por sua liberdade, deveria voltar ao
local de onde havia fugido. Para bom entendedor, o que era certamente o
caso de Gama, a decisão do tribunal significaria antes a morte brutal do
que a possibilidade de um julgamento razoável. Cabia, nesse sentido,
desmontar os pressupostos do acórdão e construir uma resposta normativa
que favorecesse, na prática, demandas de liberdade em qualquer
jurisdição. }

\emph{***}

"Acórdão\footnote{. Decisão de tribunal que serve de paradigma para
  solucionar casos semelhantes.} em Relação, etc.

Negam provimento ao agravo\footnote{. Recurso a uma instância superior
  interposto a fim de se reformar ou modificar decisão interlocutória de
  juiz ou membro de tribunal inferior.} interposto do despacho de fls.
3,\footnote{. Por estar no plural, refere-se à frente e ao verso da
  folha.} por quanto, vistos os autos, foi o mesmo proferido de
conformidade com o direito. O princípio geral que estabelece a
competência do juiz do domicílio do réu, para conhecer das ações contra
este intentadas, não acha exceção na espécie de que se trata.

Ainda nas causas de liberdade, movidas de conformidade com a Lei de 28
de Setembro de 1871\footnote{. Refere-se à conhecida Lei do Ventre
  Livre, que declarava livres os filhos da mulher escravizada nascidos a
  partir da promulgação daquela lei. A lei também regulava outras
  matérias, a exemplo do processamento e julgamento de causas de
  liberdade.}, e seu regulamento, prevalece o princípio de só deverem
elas ser intentadas no foro do domicílio do réu. O privilégio de escolha
de juiz, invocado pelo agravante, é insustentável no regime judiciário,
que vigora. A Ord. Liv 3º, Tit. 5º,\footnote{. A ordenação trata "dos
  que podem trazer seus contendores à Corte por razão dos seus
  privilégios". Embora possuísse força normativa, haja vista o agravante
  tê-la invocado, o título 5º confrontava o disposto na Constituição --
  nomeadamente, o art. 179, § 16 --, que aboliu privilégios que não
  tivessem "utilidade pública". Assim como os desembargadores
  consideravam essa ordenação "insustentável no regime judiciário",
  Cândido Mendes compreendia, no mesmo sentido, que a "prática tem dado
  como revogada essa Ord.". Cf. Cândido Mendes de Almeida,
  \emph{Ordenação e leis do Reino de Portugal}, Terceiro Livro, 1870, p.
  10.} em que se funda o agravante\footnote{. Quem interpõe o recurso de
  agravo.}, nenhuma aplicação tem ao caso e, quando tivesse, é sempre
certo que na prática se tem dado como revogada a mesma Ord., em face do
disposto no art. 179, § 16, da Const{[}ituição{]} do Império.\footnote{.
  Art. 179. ``A inviolabilidade dos direitos civis e políticos dos
  cidadãos brasileiros, que tem por base a liberdade, a segurança
  individual e a propriedade, é garantida pela Constituição do Império,
  pela maneira seguinte: § 16º. Ficam abolidos todos os privilégios que
  não forem essenciais e inteiramente ligados aos cargos por utilidade
  pública''. A construção da frase não deixa dúvida de que os
  desembargadores consultavam a edição de Candido Mendes para formular o
  acórdão. Cf. Cândido Mendes de Almeida, \emph{Ordenação e leis do
  Reino de Portugal}, Terceiro Livro, 1870, p. 10.}

Os favores que a Legislação atual tem outorgado à liberdade não importam
o desconhecimento dos direitos do senhor. Tão garantido é pela lei o
direito de propriedade como o de liberdade. A doutrina sustentada pelo
agravante tornaria desigual a posição dos litigantes, e iria de encontro
ao preceito legal. E assim mandam que para os devidos efeitos subsista o
despacho de que se agrava, pagar as custas \emph{ex-causa}.\footnote{.
  Pela causa.}

S. Paulo, 20 de Março de 1874".

ALENCAR ARARIPE.\footnote{. Tristão de Alencar Araripe (1821-1908),
  nascido em Icó (Ceará), foi político, magistrado e escritor. Ocupou
  diversos cargos no Judiciário, sendo juiz municipal e de direito,
  desembargador e presidente dos tribunais da relação da Bahia e de São
  Paulo, além de ministro do Supremo Tribunal Federal, onde se
  aposentou. Foi chefe de polícia das províncias do Espírito Santo,
  Pernambuco e Ceará, presidente das províncias do Pará e de São Pedro
  do Rio Grande do Sul e ministro da Justiça.}

AQUINO E CASTRO.\footnote{. Olegário Herculano de Aquino e Castro
  (1828-1906), nascido em São Paulo (SP), foi promotor público, juiz,
  desembargador, presidente do Tribunal da Relação de São Paulo,
  ministro e presidente do Supremo Tribunal Federal. Ocupou, também,
  cargos no Legislativo, como deputado (1867-1870 e 1878-1881), e no
  Executivo, como chefe polícia das províncias de Goiás, Rio de Janeiro
  e São Paulo, além de presidente da província de Minas Gerais
  (1884-1885).}

J. N. DOS SANTOS\footnote{. José Norberto do Santos (?-?) foi político e
  magistrado. Presidiu a província do Rio de Janeiro e foi desembargador
  nos tribunais do Maranhão, Bahia, Rio de Janeiro e São Paulo, onde
  também foi presidente desse tribunal (1874-1875).}\\
A. L. DA GAMA.\footnote{. Agostinho Luiz da Gama (?-1880), nascido na
  província do Mato Grosso, foi político e magistrado. Exerceu os cargos
  de juiz municipal, juiz de direito e desembargador do Tribunal da
  Relação de São Paulo. Foi chefe de polícia das províncias da Bahia,
  Pernambuco e na Corte (Rio de Janeiro), além de presidir a província
  de Alagoas.}

*

Nas discussões, em geral, como ainda na de que ora nos ocupamos, para
que bem se possa argumentar, e melhor concluir, preciso é bem assinalar,
e com critério distinguir, os pontos cardeais da questão.

Como se vê do venerando acórdão, que deixamos transcritos, o escravo não
tem escolha do foro, para propositura de ação manumissória; e não tem
tal escolha, pelas seguintes razões, que vamos reproduzir,
enumerando-as, com escrupulosa fidelidade:

1º: Porque o princípio geral, em que estabelece a competência do juiz do
domicílio do réu, para conhecer das ações contra ele intentadas, não
acha exceção na espécie de que se trata;

2º: Porque ainda nas causas de liberdade, movidas de conformidade com a
Lei de 28 de Setembro de 1871, e seu regulamento, prevalece o princípio
de só deverem ser elas intentadas no foro do domicílio do réu;

3º: Porque o privilégio de escolha de juiz, invocado na vertente
hipótese, é insustentável no regime judiciário, que vigora; visto como,

4º: A Ord. Liv. 3º, Tit. 5º, invocada, nenhuma aplicação tem ao caso; e,

5º: Quando tivesse aplicação ao caso, é sempre certo que, na prática, se
tem dado como revogada a mesma Ord., em face do disposto no art. 179, §
16, da Constituição do Império;

6º: Porque os favores que a Legislação atual tem outorgado à liberdade
não importam o desconhecimento dos direitos do senhor; e tanto que,

7º: Tão garantido é, pela lei, o direito de propriedade, como o de
liberdade;

8º: Porque a doutrina contrária tornaria desigual a posição dos
litigantes, e iria de encontro ao preceito legal.

Estes fundamentos, porém, não procedem; porque, além de carecerem de
razão jurídica, não se esteiam em disposição legal; e antes são
evidentemente contrários ao direito escrito, e atacam, de modo
inconveniente, se não desastroso, a própria moral judiciária.

E não procedem estes fundamentos:

O primeiro -- Porque o princípio geral, que estabelece a competência de
juiz do domicílio do réu, para conhecer das ações contra ele intentadas,
tem limites na lei; tais limites encerram exceções à regra geral; e as
exceções acham razão e fundamento na moral, no direito, e no público
interesse; o limite está na Ord. do Liv. 3º, Tit. 5º, § 3º,\footnote{.
  Gama busca na mesma ordenação citada no acórdão, muito embora em outro
  parágrafo, o 3°, fundamento para seu argumento. Nessa passagem da
  ordenação, Gama encontra fulcro para sustentar que o escravizado
  possuía, sim, o favor de escolher o local da propositura da ação
  manumissória. Afinal, conforme tal parágrafo, "o órfão varão menor de
  catorze anos e a fêmea menor de doze, e a viúva honesta, e pessoas
  miseráveis, ainda que sejam autores, têm privilégio de escolher por
  seu juiz os corregedores da corte, ou juiz de ações novas (...)". A
  continuidade do raciocínio, ao que passaria a explicar, tratava de
  equiparar o escravizado à pessoa miserável, o que Gama fazia, por
  outra parte, com igualmente sólido repertório doutrinário.} que aos
MISERÁVEIS outorga o favor de trazerem aos seus contendores à corte;
isto é ao foro da capital, foco de maior civilização, onde está situado
o colendo Tribunal, fora da perniciosa influência de localidade, onde
predomina indebitamente a rude vontade do grosseiro potentado; esta
Ordenação é lei vigente do Império, pela de 20 de Outubro de
1823\footnote{. Aprovada no bojo do processo constituinte de 1823, esta
  lei declarava em vigor uma série de normas portuguesas que possuíam
  inquestionável força normativa no Brasil até abril de 1821. O art. 1º
  da lei fazia explícita menção às Ordenações como um desses conjuntos
  normativos que voltavam oficialmente a ter vigência no Brasil.}, que
explicitamente admitiu-a; são pessoas miseráveis, na frase da lei, -- as
viúvas, os órfãos, \emph{os escravos}, \emph{que litigam pela sua
alforria}, e outras que, por certas circunstâncias, a estas possam ser
comparadas (Ago. Barbos. \emph{appelativ}. 152, nº 5; Novar. \emph{de
privileg}. \emph{miserabil}. \emph{person}. \emph{prelud}. 8, nº 6;
etc.\footnote{. Mantenho excepcionalmente a referência abreviada, haja
  vista a dificuldade, até o momento, em cravar qual a citação exata. De
  todo modo, é bastante provável que Gama se reporte ao jurista
  português Agostinho Barbosa (1589-1649) e uma de suas obras
  civilísticas.} -- Repert. Ord. \emph{verb}. - \emph{Mis}. pág. 543 e
not. (a)\footnote{. Provavelmente, refere-se ao verbete "miserável" do
  \emph{Repertório das Ordenações e Leis do Reino de Portugal} (1795).};
a causa manumissória é considerada, em direito, -- \emph{causa
pia}\footnote{. Caridosa, piedosa.}; porque, no dizer dos jurisconsultos
-- "é o cativo uma pessoa miserável de condição, que necessita, pelo seu
estado lamentável, da eficaz proteção da lei, para fazer valer os seus
direitos naturais, dos quais foi casualmente privado pela lei civil"; --
e tão sagrado era considerado este rigoroso preceito da lei civil, pelo
qual foi conferida ao escravo a faculdade de escolher juiz, para
propositura de ação manumissória, que, em Portugal, no ano de 1615,
movendo-se dúvida, porque um escravo propusera ação contra um
fidalgo-cavalheiro, com vencimento de moradia, que também tinha
privilégio, para escolher juiz, perante o qual fosse demandado (Ord.
Liv. 3º, Tit. 61, § 1º)\footnote{. Por provável erro tipográfico, a
  referência não corresponde ao teor do argumento.} -- julgou-se, que
nas causas sobre liberdade \emph{tinha o escravo maior privilégio}, e
podia escolher o juiz que lhe parecesse, sem permissão de declinatória
da parte do demandado senhor, ainda mesmo quando fidalgo-cavalheiro
fosse, com privilégio de moradia; e assim julgavam os sábios juízes do
absolutismo, que, nas árduas interpretações do direito político,
desatendiam, com critério, os privilégios emanados de concessões régias,
para observar, com restrição e civismo, os ditames piedosos da reta
moral e sã consciência;

O segundo -- Porque ainda mesmo nas causas de liberdade, movidas de
conformidade com a Lei de 28 de Setembro de 1871, e seu Regulamento nº
5.135 de 13 de Novembro de 1872\footnote{. Com mais de 100 artigos, esse
  regulamento, aprovado pelo decreto de número indicado no corpo do
  texto, regulava e modulava os efeitos da Lei do Ventre Livre (Lei nº
  2.040 de 28/09/1871).}, não há limitação alguma ao princípio
sabiamente estatuído na citada Ordenação do Liv. 3º, Tit. 5º, § 3º,
adotada pela Lei de 20 de Outubro de 1823; os casos manumissórios
estabelecidos na Lei de 28 de Setembro de 1871, que respeitou e manteve
os preceitos da legislação anterior, \emph{são especiais}; ainda quando,
portanto, tal limitação houvesse, por ela se não derrogava o
\emph{princípio geral}, como bem determinam os Assentos de 16 de
Novembro de 1700, e 3º de 9 de Abril de 1772\footnote{. Os julgados
  mencionados, provenientes das Casas da Suplicação e do Cível, em
  Lisboa, não parecem ter relação direta com o argumento que Gama
  constrói no parágrafo. Ambos tratam de temas distintos e alheios à
  matérias que levassem a um "princípio geral" derrogado, razão pela
  qual não apresentarei adiante as ementas dos assentos. Ao que me
  parece, salvo melhor juízo, Gama trouxe os assentos como expediente
  retórico ornamental para intrincar o argumento e quiçá confundir
  potenciais replicantes. Os assentos podem ser consultados no excelente
  repositório digital:
  \textless{}http://www.governodosoutros.ics.ul.pt\textgreater{}.}; o
egrégio Tribunal da Relação da Corte, que, como os demais do Império, é
porção da grande Babel judiciária do país, mesmo depois da promulgação
da Lei de 1871, e ainda em o ano precedente, em mais de um acórdão,
reconheceu e confirmou, em benefício do escravo, o direito de escolha de
juiz, fora do conhecido domicílio do senhor; o Decreto nº 4.835 de 1º de
Dezembro de 1871\footnote{. Para execução do art. 8º da Lei do Ventre
  Livre, o decreto definia o regulamento para a matrícula especial dos
  escravizados e dos filhos da mulher escravizada.}, para o caso da
matrícula especial do escravo, e para o efeito da manumissão por conta
do Estado, concede ao escravo, em certa condição, \emph{residência
especial}, e distinta da do senhor; a Lei nº 2.040 de 1871, outorga ao
escravo, para manumissão, por meio de pecúlio\footnote{. Patrimônio,
  quantia em dinheiro que, por lei (1871), foi permitido ao escravizado
  constituir a partir de doações, legados, heranças e diárias
  eventualmente remuneradas.}, \emph{direito de petição}; o Decreto de
12 de Abril de 1832\footnote{. O decreto regulava a execução da Lei de 7
  de Novembro de 1831. Gama, por sua vez, fazia referência indireta ao
  art. 10 do decreto que reconhecia de modo bastante enfático a
  capacidade jurídica do preto (sublinhe-se, não escravizado) requerer
  sua liberdade com base no tráfico ilegal. Gama equipara categorias
  jurídicas que sabia bastante distintas -- "preto" e "escravo" -- para
  reforçar seu argumento, isto é, a formação e extensão de um direito de
  ação ao escravizado, assim como discutir a questão nos termos da
  lógica senhorial a um só tempo escravista e racista. Dada a força
  normativa do artigo, que Gama exploraria outras vezes, vejamos seu
  conteúdo na íntegra desde já. Art. 10. "Em qualquer tempo, em que o
  preto requerer a qualquer juiz, de paz ou criminal, que veio para o
  Brasil depois da extinção do tráfico, o juiz o interrogará sobre todas
  as circunstâncias que possam esclarecer o fato, e oficialmente
  procederá a todas as diligências necessárias para certificar-se dele,
  obrigando o senhor a desfazer todas as dúvidas que se suscitarem a tal
  respeito. Havendo presunções veementes de ser o preto livre, o mandará
  depositar e proceder nos mais termos da lei."}, expedido para execução
da Lei de 7 de Novembro de 1831\footnote{. Considerada uma lei vazia de
  força normativa, recebendo até o apelido de "lei para inglês ver", a
  conhecida "Lei de 1831" previa penas para traficantes de escravizados
  e, de maneira não tão assertiva como a historiografia crava, declarava
  livres os escravizados que chegassem ao Brasil após a vigência da lei.},
respeitando o disposto na Ordenação do Liv. 3º, Tit, 5º, citada, e na
Lei de 10 de Março de 1682\footnote{. O alvará regulava a liberdade e a
  escravidão de negros aprendidos na guerra dos Palmares, na antiga
  capitania de Pernambuco. Conhecido da historiografia sobretudo pela
  regulação da prescrição do cativeiro após cinco anos de posse da
  liberdade, nesse texto Gama se reporta a outro comando normativo do
  alvará -- possivelmente o quinto parágrafo --, em que o rei de
  Portugal outorgava que os cativos poderiam demandar e requerer
  liberdade, ainda que contra o interesse de seus senhores.},
expressamente declara o escravo hábil para requerer a sua manumissão,
perante qualquer juiz de paz ou criminal, que lhe convenha; e,
consequentemente, obriga o senhor a vir responder perante o juiz
escolhido pelo escravo; e o mesmo princípio foi repetido na Lei de 4 de
Setembro\footnote{. A conhecida Lei Eusébio de Queiroz -- Lei de 4 de
  Setembro de 1850 -- estabelecia medidas, ritos e punições para
  reprimir o tráfico atlântico de escravizados.}, e no Decreto de 14 de
Outubro de 1850\footnote{. Regulava a execução da Lei Eusébio de
  Queiroz, definindo como se daria a repressão, processamento e
  julgamento dos contrabandistas.}; é claro, pois, e claro até à
evidência, que o segundo fundamento do venerando acórdão não passa de
mera invenção poética, e de todo ponto contrária ao direito e à
jurisprudência dos Tribunais;

O terceiro -- Porque o privilégio -- de escolha do juiz --, invocado na
vertente hipótese, é incontestável, no regime judiciário que vigora;
porque, estatuído em lei, como se acha, e fica plenamente demonstrado,
só poderá desaparecer por disposição positiva, de nova lei, que
precisamente o revogue;

O quarto -- Porque a Ordenação do Liv 3º, Tit. 5º, invocada, é lei
brasileira, feita pelo Poder Legislativo, e está em pleno vigor; e para
que não tenha aplicação ao vertente caso, do que se infere - \emph{que
terá para outros}, como afirma-se arbitrariamente, no venerando acórdão,
indispensável é que se demonstre que a miséria tornou-se indigna do
favor público, ou que os preceitos de piedade incompatibilizaram-se com
os bons sentimentos, e tornaram-se alheios às regras e princípios de
direito, e normas de sociabilidade, ou que a lei é contraditória, ou que
o manumitente não é pessoa miserável!...;

O quinto -- Porque é certo, e fora de contestação, que, se na prática,
não tem sido observada esta Ordenação, o que aliás não é exato porque a
Relação da Corte, como já o dissemos, há pensado inversamente, será
antes por incúria\footnote{. Negligência, desleixo ou falta de
  iniciativa.} ou por inadvertência dos julgadores, do que por exceção,
como equivocadamente pretende-se no venerando acórdão; nem tampouco
porque tenha sido revogada pelo art. 179, § 16 da Constituição do
Império; a carta constitucional, abolindo resolutamente os privilégios,
fez claríssima exceção dos que fossem \emph{julgados essenciais} e
inteiramente ligados aos cargos, por utilidade pública; e evidente é,
que, em tais termos, referiu-se precisamente o poder, com a imposta
abolição, às concessões honoríficas, graciosas e pessoais, e às regalias
de ordem privada; e não interessou às de ordem pública, que foram, do
modo o mais escrupuloso ressalvados; e muito menos as especialíssimas,
consagradas no direito civil, por princípios benéficos de piedade, para
apoio e justa proteção da miséria; à menos que os modernos juristas, com
entono\footnote{. Sentimento de amor-próprio, que pode ser entendido
  como orgulho, vaidade.} pindárico\footnote{. Por sentido figurado,
  suntuoso, magnífico.}, se bem que baldos\footnote{. Desprovidos,
  carentes.} de senso jurídico, não pretendam, de um só jato, que a
carta\footnote{. Isto é, a Carta de 1824, que Gama habilmente se
  esquivava em chamar de Constituição.} eliminando aqueles privilégios,
abolisse também a miséria, e com ela, no radiado golpe
capitolino\footnote{. No sentido de imponente, triunfal.}, a viuvez, a
orfandade e o cativeiro! -- quanto à mantença\footnote{. Manutenção,
  custeio.} dos privilégios de ordem pública, quer interessem
diretamente aos serventuários do Estado, quer particularmente aos
indivíduos, que, por sua condição excepcional, necessitam do auxílio
peculiar da autoridade, para a defesa regular da sua causa, é fato
inconcusso\footnote{. Inquestionável, indiscutível.}, que avulta em a
nossa legislação; (Deixamos de citar, para não alongarmos inutilmente
este escrito, as disposições que concedem privilégio de foro aos
militares, aos legisladores, aos presidentes de Província, aos
ministros, aos bispos, etc.; com relação ao mandato -- aos príncipes,
arcebispos e bispos, aos duques, marqueses, condes, doutores, militares,
etc.; com relação aos miseráveis -- aos ofendidos, etc.);

O sexto -- Porque os favores que a Legislação atual, que é a mesma em
que nos esteamos, tem outorgado {[}que{]} aos manumitentes, não importam
negação dos direitos dominicais\footnote{. Senhoriais.}; e apenas, por
eles, cuidou o legislador de coibir inveterados\footnote{. Bastante
  antigos, arraigados.} abusos;

O sétimo -- Porque \emph{em direito} é desconhecida a propriedade do
homem sobre o homem; o cativeiro é um fato anormal, transitoriamente
mantido pelos governos, porém repelido formalmente pelo direito; a
liberdade é de direito natural (\emph{Lei 30 de Julho de
1609})\footnote{. Embora se trate de lei relativa à proibição do
  cativeiro de índios no Brasil do início do século XVII, Gama cita-a
  para reforçar seu argumento sobre o direito natural à liberdade. O
  motivo para escolhê-la como um dentre os fundamentos normativos do
  direito à liberdade devia-se mais ao efeito persuasivo de coligir uma
  lei que já contava com quase três séculos de existência, do que ao seu
  conteúdo normativo ambíguo que vacilava sobre as razões de se manter
  ou não o cativeiro no Brasil.}; nas causas que sobre ela versarem,
\emph{pode o juiz dispensar na lei}, para mantê-la (Ord. Liv. 4º, Tit.
11, § 4º)\footnote{. O longo parágrafo quarto começa com a célebre
  sentença que se leria em muitas ações de liberdade no Brasil do século
  XIX: "E porque em favor da liberdade são muitas cousas outorgadas
  contra as regras gerais".}; \emph{porque o cativeiro é contra a
natureza} (cit. Ord. Tit. 42)\footnote{. A ordenação citada cuida de
  assunto diverso -- da não obrigação da pessoa morar em local onde não
  queira ficar. No entanto, em rápido relance, se admite que o cativeiro
  "é contra {[}a{]} razão natural".}; no Brasil não há lei alguma que
instituísse o cativeiro; o suposto direito dominical é uma ficção
odiosa, ilegalmente mantida, por circunstâncias imperiosas, que os
poderes do Estado, compelidos pela vontade pública, tratam com
afano\footnote{. O mesmo que afã, empenho.} de remover.

O oitavo, finalmente -- Porque, sendo essencial a igualdade de posições
dos litigantes, em juízo para a regular propositura e desenvolvimento
dos pleitos, são indispensáveis os favores da lei, em prol dos
miseráveis, que, na ausência de tais favores, serão vítimas da
prepotência dos grandes, que tudo dominam; o que o venerando acórdão
denominou -- \emph{desigualdade --}, é, pelo contrário, o que o
legislador, com muita sabedoria, instituiu, para equilíbrio das
posições, em juízo.

*

Julgamos ter discutido e demonstrado, à face da lei, que a resolução
jurídica, a resolução legal, a resolução que não ataca os verdadeiros
fundamentos do direito, nem os preceitos de moral, nem os puros
sentimentos de piedade, que tanto enobrecem o elevado caráter dos
legisladores e dos juízes dos povos cultos, em questões gravíssimas, de
máximo interesse social, como a de que nos ocupamos, não é certamente a
que, com mais paixão do que civismo, adotou o venerando Acórdão de 20 de
Março de 1874; se não a que, talvez por equívoco, em menos
desafortunados pleitos, seguiu o colendo Tribunal da Relação da Corte; a
que, por muitas vezes, com admirável isenção, e menosprezo de
favoneados\footnote{. Protegidos.} preconceitos, firmaram os doutíssimos
juízes, e os severos Tribunais de Portugal, que sabiam, em nome da
razão, e em homenagem aos direitos naturais do homem, sem infração da
lei civil, antepor o justo interesse do escravo, causa nobilíssima da
redenção, ao orgulho exulado\footnote{. Exilado.} de hiperbólicos
senhores, aos privados interesses dos dominadores do Estado, às graças
do Rei, que era bastante poderoso para criar nobrezas, para instituir
privilégios, para decretar e revogar as leis; porém somenos\footnote{.
  Inferiores, irrelevantes.} para dominar altivas e retas consciências,
e para impedir os ditames da justiça: eram juízes e tribunais livres,
que, à semelhança do Sol, erguiam-se mais alto do que as cúpulas dos
tronos.

S. Paulo, 25 de Junho de 1877.

L. GAMA.

\textbf{A PRISÃO DO FOTÓGRAFO}

\textbf{*didascália*}

\emph{A prisão do fotógrafo Victor Telles e de mais cinco artistas mexeu
com a cidade de São Paulo, aliás, nos dizeres de Gama, com "todo o
país". Fosse apenas figura retórica ou não, o suposto crime alcançou, de
fato, uma proporção fora do comum. A polícia armou um aparato de guerra
para invadir o modesto estúdio de fotografia da rua Direita, centro de
São Paulo, onde Telles trabalhava. A partir da denúncia de uma só
testemunha, Telles e seus companheiros se viram alvo de uma batida
policial que os tomava como suspeitos de um crime gravíssimo contra o
Tesouro Nacional: eram acusados sumariamente pelo crime de falsificação
de papel moeda. O pequeno estúdio do fotógrafo, portanto, abrigaria
máquinas e mais máquinas voltadas para fabricação de dinheiro falso.
Gama assume a defesa dos artistas no tribunal, requerendo ordem de
habeas-corpus, e também na imprensa, através de dois artigos que se lê a
seguir. }

\textbf{22. MOEDA FALSA -- FATOS E BOATOS}\footnote{. In: \emph{A
  Província de S. Paulo} (SP), Seção Livre, 01/02/1878, p. 2.}

\textbf{*didascália*}

\emph{Já na primeira frase tem-se a dimensão da repercussão pública que
a causa havia alcançado na imprensa e nas ruas de diversas cidades do
Brasil. A descrição sucinta do fato de que se discutia a criminalidade é
lapidar: "Victor Telles e mais cinco artistas foram presos como
suspeitos de fabrico e introdução de moeda-papel falsa na circulação
monetária do império". O fotógrafo Victor Telles e os demais artistas
estavam presos há aproximadamente um mês. Gama, por sua vez, contava o
caso com sua habitual maestria narrativa. A "misteriosa reclusão de seis
homens, que, há quase um mês, esperam por formação de culpa!...",
ganhava foros de luta épica, bem ao gosto do poeta, advogado e literato.
Num inquérito viciado e amparado num testemunho contaminado, argumentava
Gama, "Victor Telles tinha adquirido proporções de herói de romance; era
o novo Samuel Gelb, mesmo sem licença do velho Dumas!" O inquérito
policial, contudo, apontava a materialidade do crime e a autoria dos
mesmos artistas como falsários: o simples fotógrafo era apontado como
mentor intelectual de um crime ousado. O promotor público ordenou mais
diligências, entre elas, um exame nas máquinas que seriam destinadas à
fabricação de papel-moeda falso. Este "elemento de prova criminal"
tornou-se peça-chave da defesa de Gama, que passa a discutir alguns dos
quesitos mais importantes neste artigo. De maneira estratégica,
certamente visando a decisão do Tribunal da Relação de São Paulo, que
pautaria o caso na semana seguinte, Gama conclui o texto convencido de
que "é evidente a não existência do delito" de falsificação de papel
moeda. Se houve algo falsificado, foi a lei. O protesto de Gama, afinal,
era "contra o arbítrio que é a falsificação criminosa da lei", ocorrida
pelo "equívoco e a ilusão do juiz", que, "violando o direito, tortura
sem motivo ao cidadão, em nome da segurança comum". }

\emph{***}

Sabe todo o país que o sr. Victor Telles\footnote{. Victor Telles de
  Rebello e Vasconcellos, brasileiro naturalizado, viveu em Montevidéu,
  Uruguai, e em Pelotas (RS), e morava em São Paulo, onde tinha um
  estúdio de fotografia estabelecido na rua Direita.} e mais cinco
artistas foram presos como suspeitos de fabrico e introdução de
moeda-papel\footnote{. Dinheiro e/ou título de crédito conversível em
  ouro ou moeda.} falsa na circulação monetária do império; e que, em
razão de tal suspeita, estão presos há perto de um mês, sem formação de
culpa!...

O sr. dr. Henrique Antonio Barnabé Vincent,\footnote{. Embora não tenha
  informações pessoais de Barnabé Vincent, sabe-se que ele assinou,
  junto com Gama, ainda em 1878, um desagravo público ao juiz Gama e
  Mello. Cf. \emph{A Província de S. Paulo} (SP), Seção Livre,
  23/10/1877, p. 2.} promotor público da comarca, não se satisfazendo
com o resultado das diligências policiais, requereu novos exames, do
modo seguinte:

"O promotor público interino, porque seja necessário, para marchar com
passo seguro, e completar a prova de moedeiros-falsos dos presos Victor
Telles, e outros, necessita que se faça exame em diversos objetos em que
os exames anteriores não foram completos e em outros em que se não fez
exame, como nas duas máquinas de numerar o mal examinado rolo de papel
de linho encontrados na casa de Victor, e chapas metálicas encontradas
na casa de Victor e de Esprik de Verny,\footnote{. Esprik de Verny, ou
  João Esprek de Verny, era alemão e morava na ladeira de Piques, São
  Paulo.} por ser este exame de grande alcance para a denúncia dos
mesmos.

Requer, por isso, que se faça o exame por pessoas profissionais, não de
fotografia, e com urgência.

QUESITOS:

1º: Se o papel de linho apresentado é da mesma natureza ou idêntico ou
imita o papel das notas de papel-moeda do tesouro nacional;

2º: Se o dito papel serve, ou preparado poderá servir para estampar, sem
fazer diferença alguma, notas do tesouro nacional, de cem mil réis, de
cinquenta, de vinte, de dez, de cinco, de dois, de mil ou de quinhentos
réis;

3º: Se as máquinas de numerar servem para numerar notas do tesouro
nacional, se os algarismos estampados por qualquer das duas máquinas são
idênticos em forma aos algarismos dos números das notas do tesouro
nacional;

4º: Se acharam ou existem recibos da casa de Victor Telles numerados
pelas ditas máquinas;

5º: Qual a largura, comprimento e grossura das chapas metálicas
encontradas nas casas de Victor Telles e Esprik de Verny.

6º: Se as chapas têm tamanho suficiente para abrir-se uma forma de
qualquer nota do Tesouro Nacional.

RESPOSTAS

\emph{Ao 1}º \emph{quesito:}

Que pelo exame feito, e conforme os dados ao seu alcance, respondem que
o papel de linho de que se trata, parecendo da mesma natureza do papel
de algumas notas do Tesouro Nacional, não é, contudo, idêntico;

\emph{Ao 2}º \emph{quesito:}

Que o dito papel, mesmo preparado, não pode servir para serem nele
estampadas notas do Tesouro Nacional, de qualquer valor, sem haver
diferenças;

\emph{Ao 3}º \emph{quesito:}

Que as duas máquinas de numerar não servem para as notas do Tesouro
Nacional, cujos algarismos não são idênticos aos estampados por qualquer
das referidas máquinas;

\emph{Ao 4}º \emph{quesito:}

Não respondem por não terem conhecimento do objeto de que aí se trata;

\emph{Ao 5}º \emph{quesito:}

Que entre as chapas apresentadas a exame, existem três com as seguintes
dimensões:

Uma com 188 milímetros de comprimento e 83 ditos de largura;

Outra, com 183 milímetros de comprimento e 74 ditos de largura;

E a terceira com 192 milímetros de comprimento e 75 ditos de largura;

\emph{Ao 6}º \emph{quesito:}

Finalmente, que essas três chapas são as únicas, das apresentadas, que
têm tamanho e espessura suficientes para abrir-se uma forma de qualquer
nota do Tesouro, de 5\$000 réis, 2\$000 réis, 1\$000 e 500 réis
americanas.

(Assinados)

ANTONIO D. DA. C. BUENO - (juiz)

F. H. TRIGO DE LOUREIRO - (perito)

JOÃO R. DA. F. ROSA - (idem)

H. A. B. VINCENT - (promotor)

J. MOREIRA LYRIO - (testemunha)

M. C. QUIRINO CHAVES - (idem)

E. DE OLIVEIRA MACHADO - (escrivão)

\_\_\_\_\_\_\_\_\_

Este exame, que deve ser considerado da maior importância, como elemento
de prova criminal, e que, entretanto, muito favorece a causa dos
supostos fabricantes de moeda falsa, embora obscuro em diversos pontos,
no que concerne à defesa dos acusados, efetuou-se em ausência destes,
cujos direitos não foram devidamente acatados.

Todos conhecem esta lamentável ocorrência, se não calculado embuste, com
que foi surpreendida até a perspicácia da autoridade, e que deu em
resultado a misteriosa reclusão de seis homens, que, há quase um mês,
esperam por formação de culpa!...\footnote{. Fase do processo em que se
  apura os indícios mínimos da existência, natureza e circunstâncias do
  crime e de seus potenciais agentes.}

Todos conhecem, por a leitura dos periódicos e do relatório firmado pelo
exmo. sr. dr. chefe de polícia,\footnote{. Embora não nominado
  expressamente, o chefe de polícia era o próprio Furtado de Mendonça.
  Francisco Maria de Sousa Furtado de Mendonça (1812-1890), nascido em
  Luanda, Angola, foi subdelegado, delegado, chefe de polícia e
  secretário de polícia da província de São Paulo ao longo de quatro
  décadas. Foi, também, professor catedrático de Direito Administrativo
  da Faculdade de Direito de São Paulo. A relação de Luiz Gama com
  Furtado de Mendonça é bastante complexa, escapando, em muito, aos
  limites dos eventos da demissão de Gama do cargo de amanuense da
  secretaria de polícia, em 1869. Para que se ilustre temporalmente a
  relação, tenhamos em vista que à época do rompimento público, aos
  finais da década de 1860, ambos já se conheciam e trabalhavam juntos
  há coisa de duas décadas; e, mais, Gama não rompeu definitivamente com
  Furtado de Mendonça, como erroneamente indica a historiografia, visto
  o presente artigo, \emph{Aos homens de bem}, que é uma espécie de
  defesa moral e política da carreira de Furtado de Mendonça.} os
indícios fundados em presunções, e as presunções destruídas pelas
próprias testemunhas da acusação e pelos exames policiais, que serviram
de base à ilegal detenção de seis cidadãos, com flagrante violação da
lei!...

Há em todo vasto inquérito organizado pela polícia \emph{um só
depoimento} que faz carga aos acusados; e é tal depoimento prestado pelo
sr. Joaquim Fernandes da Cunha, negociante da cidade de Santos; mas este
sr. Fernandes da Cunha, na considerada opinião dos distintos senhores
tenentes Gaspar e Dias Baptista (\emph{está escrita nos autos!}) É
INDIGNO DE FÉ; porque, pelo seu caráter e irregular procedimento, tem má
reputação; era íntimo amigo de Victor Telles, e seu hospedeiro em Santos
veio a S. Paulo, de propósito, para denunciar à polícia Victor Telles e
os seus companheiros; deu como causa da denúncia o fato de não querer
Telles pagar-lhe a quantia de 300\$000 réis!

E a autoridade, seguramente por inadvertência, em vez de mandar que a
denúncia fosse tomada por termo, no sôfrego intuito de arranjar prova,
invertendo as posições, converteu o \emph{denunciante} em
\emph{testemunha}!!...

Neste memorável inquérito tudo tem corrido ao sabor da autoridade; à
mercê dos boatos; ao som das inventivas\footnote{. Alegações inventadas,
  invencionices, fantasias.} as mais extravagantes; e das calúnias
desaforadas: a moeda falsa, as chapas, as gravuras, as máquinas, a
química, e até a sublimada alquimia avultaram na encantada fotografia da
rua Direita! Victor Telles tinha adquirido proporções de herói de
romance; era o novo Samuel Gelb, mesmo sem licença do velho
Dumas!\footnote{. Alexandre Dumas (1802-1870), o pai, nascido em
  Villers-Cotterêts, França, foi jornalista, dramaturgo e romancista de
  grande sucesso. Autor de obras consagradas como \emph{Os três
  mosqueteiros} (1844) e \emph{O Conde de Monte Cristo} (1844-1846),
  também escreveu \emph{Dieu Dispose} (1851), publicada como folhetim no
  \emph{Jornal do Commercio} (1851-1852) sob o título de \emph{Deus
  Dispõe}, e que tem Samuel Gelb como protagonista. Para ler outro
  artigo em que Gama cita um protagonista de um romance de Dumas, cf.
  \emph{Resposta à redação do Diário de S. Paulo}, 29/01/1867.} Para
complemento do quadro dava-se o edifício como minado; e todo o
quarteirão prestes a ir pelos ares!!...

Tudo isto se disse; afirmou-se; a polícia ouviu e não contestou; e a
imprensa repetiu sobressaltada!...

Tudo, porém, tem o seu tempo; depois dos boatos, os fatos.

O sr. Joaquim Fernandes da Cunha, que é o protagonista deste drama, já
representou os seus papéis; fez de \emph{testemunha denunciante},
entidade nova no direito criminal; todos devem dar-se por divertidos; é
tempo de baixar o pano, para que as vítimas do embuste possam voltar aos
lares; e, sem culpas e sem penas, cuidar do trabalho e da família.

Guardamos silêncio enquanto a polícia, tomada de sincero civismo, embora
errando, procurava os vestígios de um crime gravíssimo; de um atentado
contra a fortuna pública e particular; contra a propriedade nacional;
hoje, porém, que é clara, que é evidente a não existência do delito;
hoje que o equívoco e a ilusão do juiz, por sua indesculpável
insistência, violando o direito, tortura sem motivo ao cidadão, em nome
da segurança comum, protestamos contra o arbítrio que é a falsificação
criminosa da lei.

S. Paulo, 31 de Janeiro de 1878.

O advogado, LUIZ GAMA.

\textbf{23. TRIBUNAL DA RELAÇÃO}\footnote{. In: \emph{A Província de S.
  Paulo} (SP), Seção Livre, 10/02/1878, p. 3.}

\textbf{*didascália*}

\emph{Gama rebate a redação da Tribuna Liberal, que havia criticado a
decisão do Tribunal da Relação de São Paulo em conceder ordem de
habeas-corpus em favor de Victor Telles e os outros cinco artistas
presos sem formação de culpa. O artigo tem passagens que detalham dos
bastidores da causa e, também, alguns detalhes da sessão no Tribunal da
Relação de São Paulo. Revela, também, como Gama se constitui em advogado
dos clientes aprisionados, agindo, conforme conta, "por inspiração
própria, e não por conselhos ou sugestões de outrem". Após 33 dias
presos, Telles e os demais artistas conseguem, por intermédio de Gama, a
tão desejada soltura. }

\emph{***}

A notícia relativa à concessão de \emph{habeas-corpus} em favor de
Victor Telles e outros, dada pela \emph{Tribuna Liberal} de hoje, é
inexata em grande parte.

Fui eu quem requereu \emph{habeas-corpus} em prol dos pacientes; e o fiz
em meu nome; por inspiração própria, e não por conselhos ou sugestões de
outrem.

Serviram de fundamento à petição as ilegalidades incontestáveis de que
foram vítimas os custodiados.

É verdade que o paciente Carvalho Amarante foi advertido, quando estava
sendo interrogado pelo exmo. sr. conselheiro presidente do Tribunal, por
se haver encostado na balaustrada\footnote{. Nesse caso, fileira de
  pequenas colunas que divide o espaço do tribunal ocupado por
  advogados, promotores, juízes, testemunhas, réus, serventuários, do
  público do auditório.}; assim como é verdade haver o mesmo Carvalho
Amarante procurado o sr. dr. Aquilino para seu advogado; mas é
igualmente certo que o sr. dr. Aquilino recusara a causa e aconselhou ao
paciente de procurar outro advogado, incluindo o meu humilde nome entre
os considerados que declarou.

Não é também exato que o exmo. sr. conselheiro Gama\footnote{. Agostinho
  Luiz da Gama (?-1880), nascido na província do Mato Grosso, foi
  político e magistrado. Exerceu os cargos de juiz municipal, juiz de
  direito e desembargador do Tribunal da Relação de São Paulo. Foi chefe
  de polícia das províncias da Bahia, Pernambuco e na Corte (Rio de
  Janeiro), além de presidir a província de Alagoas.} insinuasse a
qualquer dos acusados o recurso de \emph{habeas-corpus}. Carvalho
Amarante, sabendo que a polícia o queria prender, por ignorância das
leis do processo, e antes de tomar advogado, foi à casa do sr.
conselheiro Gama, em procura do dr. Aquilino. Encontrou o dono da casa e
narrou-lhe o caso. A resposta do sr. conselheiro Gama foi esta:

"\emph{Vá se apresentar à autoridade, ou espere que o prendam.}"

Foi isto o que narrou perante o Tribunal o sr. Carvalho Amarante, e não
o que lhe é atribuído pela \emph{Tribuna}.

O voto contrário do exmo. sr. conselheiro Gama, aliás improcedente, não
tem a origem que a \emph{Tribuna} lhe empresta; S. Excia. votou para que
fosse de novo ouvido o dr. juiz de direito, por ser deficiente e pouco
clara a informação prestada.

Sou reconhecido como acérrimo\footnote{. Obstinado, inflexível.} inimigo
de arbitrariedades; não dispenso favores nem aos meus próprios amigos;
porque acima da amizade está a lei, a verdade e o público interesse; mas
não posso, por isso mesmo, autorizar, com o meu silêncio, censuras
injustas esteadas em inexatidões.

S. Paulo, 9 de fevereiro de 1878.

O advogado,

LUIZ GAMA.

\textbf{ESTRATÉGIAS DE LIBERDADE}

\textbf{*didascália*}

\emph{Escrito entre outubro de 1880 e janeiro de 1881, esse conjunto de
oito artigos tem o direito de liberdade em tempos de escravidão como
eixo estruturante. É claro que, especialmente naquele período, a
articulação entre fontes do direito e movimento abolicionista ganhava
uma textura singular na escrita de Gama, de modo que muito do que fazia
passava ou se apoiava nesse eixo. A literatura normativo-pragmática que
propõe e elabora, de maneira que se verá original, se concentra em
assuntos do direito civil, principalmente, se estendendo também por
temas da história do direito e do direito público internacional. De uma
carta de Taubaté, discute o fundo de libertação de escravizados e a
regulamentação da Lei de 1871. Protesta sobre a possibilidade de
revogação de uma concessão de alforria no juízo de Itatiba. Repudia o
ato ilegal do juiz de órfãos de Jaú por prender africanos livres em
decorrência de uma ação de inventário. Ataca o procurador da Coroa e sua
ambição desmedida em vender escravos fugidos como se fossem bens do
Estado. Por motivo similar, contesta o juiz de órfãos do Rio de Janeiro,
que colocaria à venda em hasta pública um africano livre que jamais
poderia ser legalmente vendido. Defende o instituto do depósito e as
garantias da curatela para "Elisa, mulher branca, escrava", que corria
risco de morte se voltasse às mãos de seu proprietário, o juiz Camilo
Gavião Peixoto. Por fim, mas não menos notável, defende o africano livre
Caetano, escravizado em Campinas que fugira para São Paulo, com a tese
que marcaria uma das frentes da teoria do direito de Gama: a vigência
dos efeitos manumissórios da Lei de 1818. Escritos num espaço-tempo
relativamente curto, os oito artigos concentram uma estratégia de
liberdade alicerçada nas fontes do direito, tendo ciência, contudo, da
tomada de corpo que a luta abolicionista vinha conquistando na esfera
pública da imprensa.}

\textbf{24. QUESTÃO FORENSE}\footnote{. In: \emph{A Província de S.
  Paulo} (SP), Seção Jurídica, 14/10/1880, pp. 1-2.}

\textbf{*didascália*}

\emph{"Por que escrevo este artigo?" Gama perguntava ao fim de uma
sólida reflexão sobre o direito civil. Respondia, contextualizando seus
leitores: numa audiência do Tribunal da Relação de São Paulo, em que
advogava para seis escravizados, a parte contrária -- o "desembargador
Faria, muito digno procurador da Coroa" -- disse que os argumentos de
Gama não tinham base jurídica razoável. Pelo andamento da sessão, Gama
não teve direito de resposta. Imediatamente, talvez sob o peso da
derrota, Gama tratou de escrever a base jurídica de seu argumento. Para
tanto, formulou uma espécie de genealogia legal sobre o tema da venda do
escravizado fugido como bem do evento. Três dias depois da sessão,
estava pronta a tréplica antes censurada no auditório do tribunal.
Dividido em sete tópicos -- mais prólogo e epílogo --, o comentário
normativo-pragmático sustentava que o escravizado fugido e
posteriormente preso não poderia ser vendido em hasta pública, ou mesmo
retornar ao estado de escravidão. Essa era, precisamente, a pretensão do
procurador da Coroa, vender escravizados presos como bens do Estado. O
argumento de Gama conceitua o que poderia ser chamado juridicamente de
escravizado fugido e escravizado abandonado. A operação minuciosa
visava, ao fim, declarar livre quem de fato, pela fuga, por exemplo, já
fruía da liberdade. Fruía, gozava, possuía até, mas não tinha a prova, o
título, o domínio. Era aí que entrava o papel do Estado, por meio das
autoridades policiais e judiciárias. Outorgar o papel de liberdade. Isto
é, chancelar que o estado de liberdade do escravo fugido -- ou
abandonado, no jargão que habilmente inseria -- deveria ser
oficializado, através de documentos públicos. O Estado, ao contrário,
queria capitalizar; queria manter o estatuto jurídico de escravo e
vender o escravizado em hasta pública, recolhendo muito dinheiro por
isso. Gama sabia bem contra quais interesses sua tese ia. Sabia,
contudo, que a liberdade não tinha preço e quais as fontes do direito
que poderiam servir de fundamento normativo para conquistar os papeis da
liberdade.}

\emph{***}

\emph{Podem ser vendidos como bens do evento}\footnote{. Bens que, por
  não se saber quem era o senhor, proprietário ou herdeiro, deveriam ser
  entregues ao Estado. Como se verá, o argumento de Gama mirava uma
  categoria do direito civil que remetia ao tempo das Ordenações para
  fundamentar pedidos de liberdade no processo criminal. O raciocínio é
  valioso, entre outros motivos, pela construção de uma interpretação
  jurídica original para criar direitos individuais.} \emph{os escravos
fugidos, cujos donos se não conheçam, depois das diligências legais para
descobri-los?}

Não podem ser vendidos como bens do evento os escravos fugidos, cujos
donos se não conheçam, depois das diligências legais para descobri-los;
porque tais escravos devem ser declarados livres.

Cumpre, porém, que o asserto seja demonstrado; porque o asserto é um
fato; o fato tem sua causa; e esta causa é o direito.

I

Como o homem, porque é a sua razão, o direito nasceu; presidiu à
constituição da sociedade; animou o seu desenvolvimento; e sagrou-a sua
estabilidade; sua gênesis é a do homem; e, como o deste, o seu
crescimento é de intussuscepção\footnote{. Por transformação e
  incorporação de elementos formadores.}.

O direito é a vida; repele por sua índole as soluções de continuidade;
como a verdade, é sempre o mesmo; como o progresso, é a evolução
perpétua; como a luz, é uma força regeneradora; e como a liberdade,
eterno e inquebrantável.

Difere da lei, porque é o princípio; e esta, uma modalidade.

Toda a lei que contraria o direito em seus fundamentos é uma violência;
toda a violência é um atentado; o legislador que o decreta é um tirano;
o juiz que o executa, um algoz; o povo que o suporta, uma horda de
escravos.

A lei só é legítima quando promulgada pelo povo; o povo que legisla é um
conjunto de homens livres; a lei é a soberana vontade social; a causa, o
direito natural.

II

O escravo fugido, cujo senhor se ignora, como a cousa perdida, em
análogas circunstâncias reputa-se abandonado.

O \emph{abandono}, considerado como fenômeno jurídico, é relativo e
consiste na desistência de um direito ou de um dever; pelo que é
essencialmente formal.

O \emph{abandono} é \emph{voluntário} ou \emph{presuntivo}; no primeiro
caso, é direto e individual; no segundo, dispositivo e conjectural; e,
quer na primeira, quer na segunda hipótese, é expresso e legal.

Com aplicação a fatos manumissórios, e esta é a questão vertente, o
\emph{abandono voluntário} conserva a nomenclatura técnica; o
conjectural toma ordinariamente o nome de \emph{prescrição aquisitiva};
e, por isso, torna-se condicional.

No Brasil, o abandono voluntário com imediata aplicação à espécie que se
debate está definido no artigo 76 do Regulamento nº 5.135 de 13 de
Novembro de 1871,\footnote{. Art. 76. ``Considera-se abandonado o
  escravo cujo senhor, residindo no lugar, e sendo conhecido, não o
  mantém em sujeição e não manifesta querer mantê-lo sob sua
  autoridade''.} e dá-se "quando o senhor, residindo no mesmo lugar, e
sendo conhecido, não procura por o escravo, não o mantém em sujeição nem
manifesta vontade de conservá-lo sob sua autoridade."

O \emph{abandono conjectural}, ou prescrição, pelo contrário, mediante
condições preestabelecidas na lei, dá-se independente da vontade
dominical, por preterições reais ou presumidas, por considerações de
estado ou de ordem pública.

Exemplo:

"Estando de \emph{fato} livre o que por direito deva ser escravo, poderá
ser demandado pelo senhor por cinco anos somente, no fim do qual tempo
se entende \emph{prescrito} o direito de acionar" (Alvará de 10 de Março
de 1682, nº 5).\footnote{. Embora adaptada, a transcrição preserva o
  teor normativo da parte citada do alvará.}

III

Aplicação feita dos princípios de direito, das disposições da lei e das
regras de jurisprudência, que ficam expostos ao caso emergente; e
considerada a espécie indivíduo preso como escravo fugido, que
espontaneamente confessa a sua condição, cujo senhor não é conhecido, ou
sendo não o reclama, em face da Ordenação do Livro 3º, Título 94, §§ 1º,
2º, 3º e 4º; Portaria de 24 de Dezembro de 1824; instruções anexas à
Portaria 2ª de 4 de Novembro de 1825, §§ 11 e 12; Avisos de 28 de
Janeiro de 1828, 1º de 13 de Abril, o 3º de 5 de Março {[}Maio{]} de
1831, e de 12 de Agosto de 1834; Decretos de 9 de Março {[}Maio{]} de
1842, artigo 44; e nº 1.896 de 14 de Fevereiro de 1857, artigos {[}de{]}
1 a 6; Leis da Assembleia Legislativa desta província sob nº 2 de 21 de
Março de 1860 e nº 33 de 7 de Junho de 1869; Regulamento nº 2.433 de 15
de Junho de 1859; Lei nº 2.040 de 20 de Setembro de 1871, artigo 6º;
Regulamento nº 5.135 de 13 de Novembro de 1872, artigos 75, 76, 77 e 78;
Avisos nº 318 de 10 de Setembro do mesmo ano e nº 639 de 21 de Setembro
de 1878; opinião do respeitável sr. dr. Teixeira de Freitas\footnote{.
  Augusto Teixeira de Freitas (1816-1883), natural de Cachoeira (BA),
  foi juiz, advogado e presidente do Instituto dos Advogados do Brasil
  (IAB). Autor de diversas obras jurídicas, sobretudo no campo do
  Direito Civil, ganhou notoriedade como redator contratado do projeto
  de Código Civil que, todavia, não chegou a ser concluído no século
  XIX.} na \emph{Consolidação das Leis Civis}\footnote{. Lançado quando
  o autor estava à frente do Instituto dos Advogados do Brasil (IAB),
  \emph{Consolidação das Leis Civis} (1857) representou um marco dos
  debates legislativos sobre a codificação civil no Brasil. A edição que
  Gama comenta, todavia, é a 3ª edição, de 1876.}, nota 33, ao artigo
58, páginas 63 e seguintes da 3ª edição; parecer do exmo. conselheiro
d{[}esembargador{]} F. B. da Silveira, \footnote{. Francisco Balthazar
  da Silveira (1807-1887) foi político, procurador da Coroa,
  desembargador de tribunais da Relação e ministro do Supremo Tribunal
  da Justiça (1875-1886).}procurador da coroa e soberania nacional na
Relação da Corte\footnote{. Tribunal de segunda instância com jurisdição
  na Corte.}, publicado no \emph{Direito}, ano 1º, 1873, página 249 --,
resulta de modo evidente, racional, inconfutável\footnote{. Irrefutável.}:
que o escravo preso como fugido, quer seja conhecido o senhor, quer não,
só por inqualificável absurdo, com inversão flagrante dos bons
princípios e violação manifesta, proposital, dos preceitos da lei, por
guia inconsiderado ou inconsciente, poderá ser vendido em hasta
pública\footnote{. O mesmo que leilão judicial público.} como cousa
achada à guisa de besta ou gado, \emph{como propriedade do
vento!...}\footnote{. Evidente arremate sarcástico em que, por
  metonímia, Gama substitui "evento" por "vento", de modo a assinalar o
  absurdo da alegação contrária em intentar se apropriar de algo já há
  muito do estado de natureza.} \footnote{. O conhecimento normativo que
  Gama demonstra nesse excerto ilustra bem o domínio a um só tempo
  enciclopédico e técnico que tinha das fontes do direito. Para
  introduzir o argumento que viria a apresentar na sequência, invoca
  como base jurídica trinta textos jurídicos diferentes, que pertenciam
  a um repertório de dezoito normatividades e doutrinas de variados
  níveis e temporalidades. Vejamos, a começar pelas Ordenações citadas,
  os extratos aduzidos. Respectivamente, tít. 94: "Como se hão de
  arrecadar e arrematar as cousas achadas ao vento", § 1º, especialmente
  a estipulação de prazo: "E em cada cidade e vila haverá um lugar
  assinado conveninente para isto, que seja perto da vila, para a ele
  trazerem as bestas e gados do vento; e serão aí trazidos por o mordomo
  ou rendeiro, à terça-feira de cada uma semana, até se acabarem quatro
  meses, contados do dia que forem assentados no livro (...). § 2º. E se
  dentro dos ditos quatro meses vier o dono da cousa que for achada de
  vento, e fizer certo que é sua, ser-lhe-á entregue e pagará ao
  mordomo, ou rendeiro, as custas que fez em manter e guardar, se dela
  não se serviu. § 3º. E passados os quatro meses, não lhe saindo dono,
  o julgador, a que o conhecimento pertencer, sendo requerido, e vendo
  os autos feitos na forma sobredita, julgará ao mordomo ou a quem o
  direito do vento pertencer, os ditos gados ou bestas que assim andarem
  de vento. E tanto que lhe forem julgadas, as poderá vender e arrematar
  a quem lhe aprouver, e fará delas como de cousa sua. E posto que
  depois de lhe serem julgadas, venham seus donos a demandá-las, não
  serão ouvidos nem recebidos à tal demanda''. § 4º, especialmente o
  primeiro trecho: "E antes do gado ou bestas serem julgadas na maneira
  sobredita, o mordomo, ou rendeiro, ou cujo for o direito do vento, não
  poderão vender, matar, nem amealhar por maneira alguma, nem esconder,
  nem levar para outra parte as cousas que assim trouxerem de vento. Mas
  todo o tempo dos quatro meses as trarão no termo da cidade, ou vila,
  onde forem achadas, e em lugar que as possam ver e saber onde andam, e
  o que o contrário fizer, seja preso e haja a pena que haveria se as
  furtasse (...)". Assinada na véspera de Natal, a portaria de 1824
  regulava a apreensão de escravizados fugidos e destruição de
  quilombos. Cf., especialmente, "(...) os senhores, no ato de receberem
  seus escravos, pagarão as despesas feitas com a apreensão dos mesmos,
  as quais, todavia, será {[}sic{]} conveniente que não excedam a 4\$000
  por cada um, para ficarem mais suaves aos ditos senhores dos escravos
  e à Polícia, de quem recebem o benefício de os haverem quando os
  julgavam perdidos". Instruções anexas à citada portaria de 1825, § 11:
  "Os escravos que forem presos por fugidos ou em quilombos (que os
  comissários procurarão destruir, quando lhes for possível), serão
  imediatamente remetidos à esta intendência, com a respectiva parte e
  conta da despesa, para lhes ser logo paga com gratificação para os
  apreensores. O mesmo se praticará relativamente aos ladrões e
  salteadores, na conformidade do edital de 3 de Janeiro deste ano, que
  também executarão, no que for aplicável aos seus distritos, e não
  estiver posteriormente ordenado o contrário. § 12: "Obrigarão aos
  capitães do mato a que apresentarem seus títulos para os visarem e
  inscreverem os seus nomes em uma lista, de que remeterão cópia à esta
  intendência; ordenando que os ditos capitães lhes participem cada uma
  apreensão de escravos fugidos, para se evitarem extorsões aos
  senhores, e que os escravos se conservem por muito tempo em troncos ou
  em cárceres privados. Os comissários terão a maior vigilância neste
  objeto, participando logo às autoridades os abusos sobre que convier
  dar providências". Aviso nº 18, de 28/01/1828, em que se declarava o
  destino que deviam ter os escravizados retidos em prisão e o depósito
  que deveriam ter quando abandonados por seus donos, cf. o parecer
  anexo e que subsidia o aviso, remetendo-os da prisão ou depósito "para
  a Marinha, onde os escravos servirão no dique, ou em outros trabalhos,
  onde possam ser vistos a toda a hora do dia, e recebendo o juízo que
  apreender, do Tesouro, as despesas necessárias". Sem dúvidas, Gama
  refere-se ao aviso nº 86, de 05/03/1831, que mandava que a polícia da
  Corte entregasse ao juízo dos cativos da cidade "todos aqueles
  escravos que se acharem policialmente presos no calabouço ou em
  qualquer outra prisão (...), e de cujos donos não haja notícia, a fim
  de serem arrematados, conforme a lei". O aviso nº 274, de 12/08/1834,
  determinava que os escravizados que, "dentro de seis meses da
  apreensão e detenção no calabouço não forem reclamados pelos
  senhores", fossem "remetidos ao juiz de órfãos como bem de ausentes".
  Sobre o dec. nº 160, de 09/05/1842, que também regulava a arrecadação
  de bens do evento, ver a transcrição parcial do art. 44 no corpo do
  texto. O citado decreto de 1857 dava providências sobre "escravos
  demorados na Casa de Correção da Corte". Cf. Art. 1º.~``Logo que for
  apreendido e recolhido à Casa de Correção algum escravo fugido, ficará
  imediatamente à disposição do juízo da provedoria, que procederá a
  respeito dele, como dispõe os artigos 46, 47 e 48 do Regulamento de
  11/05/1842; para esse fim, a autoridade policial e o diretor da dita
  Casa farão sem demora as devidas participações''. Art. 2º.~``Os
  mencionados escravos, durante o tempo em que estiverem na Casa de
  Correção, são sujeitos somente às seguintes despesas: § 1º. De
  apreeensão e condução; § 2º. De custas judiciais para os anúncios e
  arrematações; § 3º. De vestuário''. Art. 3º.~``As despesas de sustento
  e curativo são devidas somente por aqueles que não trabalharem''. Art.
  4º.~``Se o escravo for recolhido à Casa de Correção por ordem de seu
  senhor, no recibo se declarará o prazo pelo qual fica ele aí
  depositado, sob a pena de ser havido como abandonado; este prazo pode
  ser prorogado por justos motivos''. Art. 5º.~``Findo o prazo declarado
  no recibo, se procederá a respeito destes escravos como se determina
  nos artigos antecedentes a respeito dos escravos fugidos''. Art.
  6º.~``As disposições dos artigos 2º e 3º são aplicáveis aos escravos
  que se acharem demorados na Casa de Correção por embargo ou depósito
  da Justiça". Da lei provincial paulista de 1860, que tratava do tema
  dos escravizados fugidos presos, destacam-se dois artigos, certamente
  presentes no repertório de Gama ao invocar essa norma. Cf. Art. 4.º
  ``Durante dois meses, contados do recebimento do escravo pelo chefe de
  polícia, se farão repetidos anúncios com as declarações do art. 2 °, e
  outras que acrescerem, e comparecendo o senhor dentro deste prazo,
  mostrando satisfatoriamente o seu domínio, ser-lhe-á entregue o
  escravo pelo chefe de polícia''. Art. 5.° ``Findo o prazo do artigo,
  será o escravo entregue a jurisdição do juízo da provedoria para
  proceder a respeito, como prescrevem as leis em vigor sobre a
  arrecadação dos bens do evento; continuando, entretanto, o escravo nos
  trabalhos públicos até que seja recebido por seu senhor, ou
  arrematado''. Sobre a segunda lei provincial citada, cf.
  especialmente: art. 3º. "90 dias depois da publicação do edital na
  capital, no caso de não ter sido reclamado, será o escravo entregue à
  jurisdição do juízo da provedoria, para proceder a respeito como
  prescrevem as leis em vigor sobre a arrecadação dos bens do evento''.
  Art. 4º. ``Durante o prazo estabelecido no art. antecedente se farão
  repetidos anúncios com as declarações (...) e, comparecendo o senhor,
  dentro deste prazo, ser-lhe-á entregue o escravo, desde que justificar
  o seu domínio, ou o direito que tem à posse dele". Ao regulamento de
  1859, cf. especialmente: art. 85. "São bens do evento os escravos,
  gado ou bestas, achados, sem se saber do senhor ou dono a quem
  pertençam; o seu produto líquido deve ser recolhido à recebedoria do
  município da Corte''. Art. 88. ``Logo que forem apresentados os
  escravos, gado e bestas achadas, e pelas diligências e averiguações a
  que se proceder se não conseguir saber a quem pertencem, se fará
  imediatamente a avaliação (...)". Art. 90. ``Feita a avaliação, se
  passarão logo editais por que {[}pela qual{]} se chamem as pessoas que
  tiverem direito aos escravos, bestas e gado achados do evento, sendo
  30 dias para os escravos e 3 para o gado ou bestas (...)". Da Lei do
  Ventre Livre, Gama cita o art. 6º, que definia quem seria liberto por
  força de lei, fazendo referência às hipóteses dos parágrafos
  seguintes, i. e., § 1º: "os escravos pertencentes à Nação (...); § 2º:
  os escravos dados em usufruto da Coroa; § 3º: os escravos das heranças
  vagas; § 4º: os escravos abandonados por seus senhores (...)". Quanto
  ao regulamento de 1872, que dava execução à Lei do Ventre Livre, ver
  especialmente o cap. VI, dos libertos pela lei. Cf. Art. 75. "São
  declarados libertos: I. Os escravos pertencentes à Nação (...); III.
  Os escravos das heranças vagas; IV. Os escravos abandonados por seus
  senhores"; e o § 1º: "Os escravos pertencentes à Nação receberão as
  suas cartas de alforria em conformidade do dec. nº 4.815 de
  11/11/1871, e terão o destino determinado no mesmo decreto". O aviso
  nº 318, do Ministério da Justiça, interpretava restritivamente parte
  da Lei do Ventre Livre e não considerava os escravizados apreendidos
  como bens do evento como libertos pela lei, estipulando uma distinção
  entre escravizados abandonados e escravizados do evento. Cf. "(...) os
  escravos contemplados na classe dos bens do evento não são os que seus
  senhores abandonam e a que se refere o art. 6º, § 4º, da citada lei,
  mas os achados sem se saber do senhor ou dono à quem pertençam,
  conforme o art. 85 do regulamento de 15/06/1859". Quanto ao aviso nº
  639, de 21/09/1878, cf. a então definição do Ministério da Justiça:
  "Considera-se bem do evento o escravo a respeito do qual não há
  reclamação nem se sabe qual o seu verdadeiro senhor". A citação à
  \emph{Consolidação das leis civis} (3ª ed., 1876), do jurisconsulto
  Teixeira de Freitas, confere exatamente, assim como o parecer
  publicado na revista jurídica \emph{O Direito} (1873), pp.249- 253.
  Sobre este último, ver especialmente o comentário sobre o dec.
  14/02/1857.}

IV

\emph{Bens do evento}, como define o art. 44 do Decreto de 9 de Março de
1842, são "os escravos, gado ou bestas, achados \emph{sem se saber o
senhor ou dono a quem pertençam}."

Desta claríssima disposição, em sentido direto, inevitavelmente resulta
que, se o \emph{senhor} do escravo é \emph{conhecido}, o escravo não
pode pertencer ao evento; e se, tendo aviso da sua prisão, o não
procura, depois de notificado por os meios, e por a autoridade
competente, o tem voluntariamente, formalmente, de modo direto,
abandonado, de conformidade com as disposições combinadas dos artigos 4º
do Decreto nº 1.896 de 14 de Fevereiro de 1857, e 76 do de nº 5.135 de
13 de Novembro de 1872; pelo que deve ser declarado livre, como estatui
a Lei nº 2.040 de 28 de Setembro de 1871, art. 6º, § 4º.\footnote{.
  Respectivamente: Art. 4º.~``Se o escravo for recolhido à Casa de
  Correção por ordem de seu senhor, no recibo se declarará o prazo pelo
  qual fica ele aí depositado, sob a pena de ser havido como abandonado;
  este prazo pode ser prorrogado por justos motivos''. Art. 76.
  ``Considera-se abandonado o escravo cujo senhor, residindo no lugar, e
  sendo conhecido, não o mantém em sujeição e não manifesta querer
  mantê-lo sob sua autoridade''. Art. 6º. ``Serão declarados libertos: §
  4º Os escravos abandonados por seus senhores (...)''.}

Assim também, se o senhor não é conhecido, ou porque não seja
encontrado, por mudança ou por ausência, ou porque o escravo, com ardil,
oculta o seu próprio nome, ou o do seu senhor, ou o do lugar do seu
domicílio, \emph{considera-se abandonado}, para o mesmo efeito de
alforriado ser, nos rigorosos termos da lei citada; e isto assim deve
ser, não só porque verifica-se o caso do \emph{abandono indireto} ou
conjectural, como porque não pode o escravo ficar indefinidamente em
prisão, sem causa justificativa, e contra as disposições em vigor; nem,
principalmente, por a impossibilidade inobstável\footnote{. Que não se
  pode obstar.} da sua venda.

V

O art. 8º da memorável Lei nº 2.040 de 28 de Setembro de 1871\footnote{.
  Art. 8º. ``O governo mandará proceder a matrícula especial de todos os
  escravos existentes do Império, com declaração do nome, sexo, estado,
  aptidão para o trabalho e filiação de cada um, se for conhecida''.},
com previdência muito judiciosa, e para cimeira\footnote{. Em alto
  nível.} acautelar corruptelas judiciárias, estabeleceu a matrícula
especial de todos os escravos existentes no império, e decretou a
manumissão imediata dos que não fossem matriculados.

E, no Regulamento de 1º de Dezembro de 1871, promulgado por o Decreto nº
4.835, da mesma data\footnote{. Para execução do art. 8º da Lei do
  Ventre Livre, o decreto definia o regulamento para a matrícula
  especial dos escravizados e dos filhos da mulher escravizada.}, para
estrita execução daquela mencionada parte da Lei de 28 de Setembro,
imperativamente está determinado, arts. 35 e 45:

"1º {[}art. 35{]}: A pessoa que celebrar qualquer contrato dos
mencionados no art. 45, \emph{sem exibir as relações} ou \emph{certidões
das respectivas matrículas}; a que aceitar as estipulações dos ditos
contratos, sem exigir a apresentação de algum desses documentos; a que
não comunicar à estação competente a mudança de residência para fora do
município, transferência de domínio, ou o falecimento de escravos, ou de
menores livres nascidos de mulher escrava, conforme prescreve este
regulamento; o oficial público que lavrar termo, auto ou escritura de
\emph{transferência de domínio}, ou de penhor, de hipoteca ou de
serviços de escravos, sem as formalidades prescritas no citado art. 45;
o que der passaporte\footnote{. Autorização policial ou judiciária para
  o escravizado transitar pelas ruas, de um ou mais distritos ou
  municípios, na ausência do senhor ou de quem o representasse.} a
escravos sem exigir a apresentação das relações ou certidões de
matrículas; e o que não participar aos funcionários incumbidos da
matrícula as manumissões que houver lançado nas suas notas, incorrerão
na multa de 10\$000 a 50\$000.

2º {[}art. 45{]}: Depois do dia 30 de Setembro de 1872 não se lavrará
escritura de contrato de alienação, transmissão, penhor, hipoteca ou
serviço de escravos, sem que ao \emph{oficial público} que tiver de
lavrar a escritura sejam presentes as relações das matrículas, ou
certidão delas, \emph{devendo ser incluídos no instrumento os números de
ordem dos matriculados}, a data e o município em que se fez a matrícula,
assim como os nomes e mais declarações dos filhos livres de mulheres
escravas, que as acompanharem, nos termos do art. 1º, §§ 5º e 7º da Lei
nº 2.040 de 28 de Setembro do corrente ano.

Também se não dará passaporte\footnote{. Autorização senhorial e/ou
  policial para controle do ir e vir de escravizados por ruas e
  estradas.} a escravos sem que sejam presentes à autoridade que o
houver de dar o documento da matrícula, cujos números de ordens, data e
lugar em que foi feita serão mencionados no passaporte; e se forem
acompanhados por seus filhos livres, devem os passaportes conter os
nomes e mais declarações relativas a estes.

Assim também nenhum inventário ou partilha entre herdeiros ou sócios,
que compreender escravos, e nenhum litígio que versar sobre o
\emph{domínio} ou a \emph{posse} de escravos será admitido em juízo, se
não for, \emph{desde logo}, exibido o documento de matrícula."

Como, portanto, à vista destas disposições inconcussas\footnote{.
  Irrefutáveis, incontestáveis.}, há de o juiz provedor,\footnote{. O
  juiz da Provedoria de Capelas e Resíduos.} improvisado, por
extravagante arbítrio dos poderes judiciários, descurado dos seus
deveres, e do Executivo, por inveterado\footnote{. Arraigado,
  acostumado.} desplante mercador, sem carta\footnote{. Licença,
  documento.}, de \emph{escravos furtados}, expô-los à venda, sem
possuir e sem apresentar relações ou as certidões da matrícula especial?

Como lavrará o escrivão, corréu convencido do crime, escritura ou termo
de arrematação menosprezando a sanção legal e dispensando-se de cumprir
os preceitos imprescindíveis dos artigos 35 e 45 do decreto nº 4.835?

E quem será o comprador culposo desta venda fraudulenta?

Como cumprirá ele a disposição do artigo 21 que o obriga,\footnote{.
  Art. 21. ``Os encarregados da matrícula averbarão no livro desta as
  manumissões, mudanças de residência para fora do município,
  transferências de domínio e óbitos dos escravos matriculados no
  município, à vista das declarações, em duplicata, que, dentro de três
  meses subsequentes à ocorrência desses fatos, são obrigadas a fazer as
  pessoas designadas no art. 3º''.} para a necessária
averbação\footnote{. Anotação à margem da escritura ou termo de um
  registro que incida no documento original.}, a dar conhecimento da
transferência de domínio à repartição fiscal?

Haverá, por privilégio do evento, matrículas por suposição?

Podem os juízes, ou o governo, revogar a lei ao seu talante\footnote{.
  Arbítrio, capricho.}?

Já foi eliminado das disposições vigentes o § 8º do artigo 15 da Carta
Constitucional?\footnote{. Art. 15. ``É da atribuição da Assembleia
  Geral {[}Câmara dos Deputados e Senado{]}: § 8º. Fazer leis,
  interpretá-las, suspendê-las e revogá-las''.}

VI

A questão não é nova; e já foi, com madureza, resolvida.

A 12 de Março de 1874, a recebedoria\footnote{. Repartição pública onde
  se recebiam impostos, taxas, etc.} do município da Corte deu
categórica e proveitosa lição de direito ao douto juiz da provedoria; e
fê-lo de modo louvável, recusando, com ríspido civismo, o recebimento de
imposto de transmissão de propriedade de escravos irregularmente
arrematados como bens de evento, por não constar, da respectiva
\emph{guia}, \emph{a exibição da matrícula especial}, no ato da
arrematação, segundo as prescrições legais em vigor; e o governo,
entaliscado\footnote{. Apertado, entalado.}, entre o direito e o
monstruoso erro, resolveu, com exemplar sabedoria, por Aviso nº 3 de 12
de Novembro de 1875, "que aos escravos recolhidos em casa de detenção, e
arrematados como bens do evento, aproveita a disposição do artigo 19 do
Regulamento de 1º de Dezembro de 1871, \emph{devendo ser considerados
livres}, sem prejuízo dos direitos dos senhores, reclamados por ação
ordinária no juízo competente."\footnote{. Aviso nº 509, de 12/11/1875,
  do Ministério da Agricultura, Comércio e Obras Públicas, declarava que
  os escravizados presos e posteriormente arrematados deveriam ser
  considerados livres, "sem prejuízo dos direitos dos senhores". Embora
  adaptada, a transcrição preserva o teor normativo do aviso. Ademais,
  Gama toma o caso da recebedoria da corte e comenta a "proveitosa lição
  de direito ao douto juiz da provedoria" a partir do exposto na
  introdução ao aviso.}

Isto, sim, é jurisprudência; tem fundamento jurídico e foi externado com
critério.

VII

O legislador de 1871 estabeleceu praticamente, como princípio
abolicionista, e necessário, que seriam declarados livres:

-- Os escravos pertencentes à Nação;

-- Os escravos dados em usufruto à Coroa;

-- Os escravos das heranças vagas;

-- Os escravos abandonados por seus senhores.

Esta medida altamente humanitária, que assinala uma vitória da
civilização, e um grande progresso social, no Brasil, é na expressão de
um exímio filósofo, essencialmente moral e política; e tanto mais
inatacável, na razão da sua existência, quanto é certo que o legislador
não só decretou a libertação, no tempo presente, sem \emph{restrições
onerosas}, dos escravos existentes, sem remuneração alguma para os
cofres do Estado, como calculadamente estendeu-a, prevendo, como devia,
sucessos futuros \emph{aos escravos da Nação, aos das heranças vagas e
aos abandonados pelos senhores}.

Como, pois, mantida cientificamente a economia da lei, supor isentos do
benefício os escravos fugidos cujos donos não sejam sabidos {[}e{]},
como tais, devolvidos ao evento, vendidos pela provedoria, em proveito
dos cofres da Nação?!

Que! O legislador diretamente decreta a manumissão dos escravos das
heranças vagas, dos pertencentes à Nação e dos abandonados pelos
senhores, e, por meios indiretos, às ocultas, com solapado\footnote{.
  Por sentido figurado, dissimulado, disfarçado.} sentimento, procura
locupletar-se\footnote{. Abarrotar-se.} com as migalhas salpicadas por
os acasos do evento?!

E será isto sério?

Será filosófico e moral?

Em que compêndio se encontram estes estólidos\footnote{. Estúpido,
  desprovido de discernimento.} princípios de tão exótica hermenêutica?

Qual é a base ontológica dessa doutrina original?

O direito é um corpo; tem a sua anatomia peculiar; tem as suas cavidades
esplâncnicas\footnote{. Viscerais.}; e estas contêm vísceras delicadas,
que devem ser observadas por peritos e tratadas profissionalmente.

Se a Lei de 18 de Agosto de 1769\footnote{. Conhecida como "Lei da Boa
  Razão", tal norma marcou época em Portugal e estabeleceu balizas
  fundamentais ao desenvolvimento do direito português (e brasileiro),
  ao ordenar, por exemplo, as fontes do direito e a prevalência de
  normas legisladas sobre outros tipos de normas.} está em vigor, os
Palínuros\footnote{. Referência da mitologia romana para a figura de um
  navegador, guia, dirigente.} da escravidão, por honra sua, devem
exigir um mausoléu\footnote{. Imponente monumento funerário.} ao
ministério do exmo. sr. conselheiro Lafayette\footnote{. Lafayette
  Rodrigues Pereira (1834-1917) foi político, fazendeiro, presidente das
  províncias do Ceará (1864-1865) e do Maranhão (1865-1866), ministro da
  Justiça (1878-1880) e presidente do Conselho dos Ministros
  (1883-1884).} e comemorar, com funerais, o monumental Aviso nº 639 de
21 de Setembro de 1878.

\_\_\_\_\_

Por que escrevo este artigo?

Na sessão judiciária do Tribunal da Relação, do dia 8 do
corrente\footnote{. Cf. \emph{A Província de S. Paulo} (SP), Noticiário,
  09/10/1880, p. 2. Destaco esse trecho que bem apresenta, por outro
  ângulo, o contexto da ação: "Durou três horas a discussão da ordem de
  \emph{habeas-corpus} requerida pelo cidadão Luiz Gama em favor de seis
  indivíduos presos, como escravos fugidos, na casa de correção.

  A discussão, além de longa, foi grave; e era notável, contra o
  costume, a concorrência de espectadores no tribunal. (...)

  Por tudo que ali ouvimos, quer do impetrante, quer dos juízes, se nos
  afigura que as questões tendentes ao elemento servil atingem a
  melindroso período, e que os abolicionistas tomam séria atitude
  perante os poderes do Estado.

  O tribunal, por votação unânime, considerou livres e pôs em liberdade
  três dos pacientes e mandou que os outros três continuassem à
  disposição do dr. juiz da provedoria, para proceder como de direito."},
perante numeroso auditório, quando se discutia a ordem de
\emph{habeas-corpus} por mim impetrada em favor de seis infelizes, e
quando já me não era permitido falar, o exmo. sr. desembargador
Faria\footnote{. José Francisco de Faria (1825-1902), natural do Rio de
  Janeiro (RJ), foi político e magistrado. Foi chefe de polícia da Corte
  (Rio de Janeiro), juiz de direito, desembargador dos tribunais da
  Relação de Ouro Preto e de São Paulo, procurador da Coroa, Soberania e
  Fazenda Nacional e ministro do Supremo Tribunal de Justiça. Teve
  muitos embates com Luiz Gama na parte contrária, sendo o mais célebre
  aquele em que Gama advogou \emph{habeas-corpus} para o africano congo
  Caetano (1880).}, muito digno procurador da Coroa, porque eu, na
exposição que fiz, disse acidentalmente\footnote{. Sem que discutisse o
  mérito.} "que o evento estava extinto quanto aos escravos fugidos,
cujos donos eram ignorados", baseando-me na insuspeita opinião do exmo.
sr. conselheiro d. F. B. da Silveira, declarou, para resguardo de sua
opinião:

"Que o evento existe para os escravos fugidos cujos donos são ignorados;
que tais escravos devem ser vendidos pela Provedoria, e o seu produto
recolhido aos cofres do Estado, na forma da lei, como decidiram os
Avisos nº 318 de 10 de Setembro de 1872 e nº 639 de 21 de Setembro de
1878!"

Estas palavras, tão valiosas pela autoridade do cargo, proferidas em
plena sessão do egrégio Tribunal, por magistrado distinto, tanto pelo
seu caráter como pela sua ilustração, em um debate importante,
constituem duplo e gravíssimo perigo: autorizam o curso forçado de um
erro jurídico (tal é a minha humilde opinião), e cavam abismos aos
manumitentes\footnote{. Relativo aos que demandam liberdade.}, já
sobejamente\footnote{. Demasiadamente.} premados\footnote{. Oprimidos,
  violentados.} por a prepotência dos senhores e pela má vontade de
muitos juízes interessados.

Sou abolicionista, sem reservas; sou cidadão; creio ter cumprido o meu
dever.

S. Paulo, 11 de Outubro de 1880.

L. GAMA.

\textbf{25. FATO GRAVE \emph{--} JAÚ}\footnote{. In: \emph{A Província
  de S. Paulo} (SP), Seção Livre, Foro da Capital, 20/10/1880, p. 2.}

\textbf{*didascália*}

\emph{Muito bem informado sobre uma ação de inventário que corria no
juízo dos órfãos da distante Jaú, G., o autor do escrito, denunciava que
o juiz local havia mandado prender oito africanos -- e/ou seus
descendentes -- que seriam livres em virtude da força normativa da Lei
de 26 de Janeiro de 1818, lei que proibia o comércio transatlântico de
escravizados. Gama desenvolveria o mesmo argumento, de modo doutrinário,
menos de dois meses depois, no célebre estudo "Questão Jurídica".
Somando-se ao argumento original as marcas estilísticas e a forma
descritiva da denúncia, nota-se que Gama seguia abrindo caminhos na
imprensa para discutir a ilegalidade da escravidão. }

\emph{***}

Corre no juízo dos órfãos do termo do Jaú um inventário no qual estão
arrolados como escravos \emph{oito pessoas livres}: são africanos,
importados depois da proibição do tráfico e descendentes seus.

Um dos coerdeiros, homem de sã consciência, e de probidade fundida por a
têmpera antiga, teve a virtude, raríssima nestes tempos, de prevenir o
juízo deste grave sucesso.

Parece que o aviso não foi bem recebido!... Pois que, de uma carta
daquela vila, sei que os escravos inventariados foram postos em
prisão!...

Sei também que os demais coerdeiros não levaram a bem o procedimento
franco e leal do seu digno companheiro!...

Não conheço o sr. dr. juiz dos órfãos do termo do Jaú; e tanto basta
para não julgá-lo mal; certo é, porém, que, se ele, em face do § 1º da
Lei de 26 de Janeiro de 1818, julgou necessário \emph{para segurança dos
manumitentes} metê-los em prisão, procedeu com hebraísmo notável.

10 -- 1\footnote{. Essa publicação seria replicada, a partir de
  21/10/1880, em outras dez edições (uma a mais, inclusive, do que o
  sinalizado na numeração).}

G.

S. Paulo, 19 de Outubro de 1880.

\textbf{26. ARESTO NOTÁVEL}\footnote{. In. \emph{Gazeta da Tarde} (RJ),
  17/11/1880, p. 2. A redação da \emph{Gazeta da Tarde} introduz o
  artigo dessa forma: "Luiz Gama, o muito conhecido e notável cidadão,
  aquele à quem tanto devem as ideias democráticas no Brasil, escreveu
  ontem, em S. Paulo, o artigo que vai em seguida. Acometido há quatro
  dias de grave enfermidade, mesmo assim acudiu em prol de homens que
  contra a lei pretende-se escravizar. O artigo, demais, é erudito.
  Oferecemo-lo à atenção dos tribunais brasileiros".}

\textbf{*didascália*}

\emph{Crônica forense baseada em um excerto do noticiário judiciário da
Corte. Gama acrescenta linhas gerais do argumento que vinha
desenvolvendo em diversos escritos da época e que no mês seguinte
ganharia a forma final do estudo normativo-pragmático "Questão
Jurídica". Gama comentava a "extravagante doutrina" do juiz dos órfãos
da Corte, que estipulava uma linha divisória para direitos de liberdade
relacionada à idade dos africanos escravizados; se menores de 49 anos,
seguramente nascidos após a lei proibitiva do tráfico, de 1831, deveriam
ter suas demandas resolvidas no juízo contencioso. A bizarra decisão do
juiz simplesmente barrava demandas de liberdade baseadas na
multinormatividade do contrabando. E o fazia por uma ficção duvidosa. "O
criminoso contrabandista vê nos tribunais", sugeria Gama, um território
facilmente dominado, "onde os sacerdotes discutem teologia, enquanto a
pátria corre perigo..." De modo original, Gama formula uma crítica
jurídica que retira o suposto pioneirismo da Lei de 7 de Novembro de
1831, colocando um novo marco temporal legal para a matéria da proibição
do comércio e escravização de africanos: a Lei de 26 de Janeiro de 1818.
Se o argumento ganhasse força normativa, Gama alcançaria, pela via
legal, a extinção imediata do cativeiro para todos os africanos e seus
descendentes que entraram no Brasil desde janeiro de 1818. Conseguem
imaginar o impacto da tese? O objetivo era, pelas armas do direito, pôr
fim à escravidão de um milhão de vítimas do contrabando ilegal -- e de
Estado. }

\emph{***}

Refere o \emph{Jornal do Commercio} em sua gazetilha\footnote{. Seção
  noticiosa, literária e/ou humorística de um jornal.} de anteontem:

"ARREMATAÇÃO DE ESCRAVOS

Ontem por ocasião de serem abertas as propostas para arrematação dos
escravos pertencentes as menores filhas de José Manoel Coelho da Rocha,
declarou o sr. dr. Justiniano Madureira, juiz da 1ª Vara de Órfãos, que
as propostas relativas aos escravos africanos menores de 49
anos\footnote{. A menção da idade é referência de que se trata de uma
  discussão sobre a legalidade da entrada de africanos escravizados após
  a Lei de 1831.} ficavam adiadas até que seja resolvida no juízo
contencioso a questão que se levantou a respeito dos mesmo escravos."

Sem ofensa da incontestável ilustração deste emérito juiz, que não tenho
a honra de conhecer, declaro-vos que não compreendo esta extravagante
doutrina; e menos ainda esta esquipática\footnote{. Estapafúrdia,
  bizarra, o que não é coerente.} decisão!

Há dúvidas sobre a condição ou sobre o estado dos africanos menores de
49 anos de idade, existentes no país?

Será causa de tais dúvidas a Lei de 7 de Novembro de 1831?\footnote{.
  Considerada uma lei vazia de força normativa, recebendo até o apelido
  de "lei para inglês ver", a conhecida "Lei de 1831" previa penas para
  traficantes de escravizados e, de maneira não tão assertiva como a
  historiografia crava, declarava livres os escravizados que chegassem
  ao Brasil após a vigência da lei.}

Compete a solução de tais dúvidas ao \emph{juízo contencioso}?

Que juízo é esse? Por que lei foi estabelecido?

Pois está revogado o Decreto de 12 de Abril de 1832?\footnote{. O
  decreto regulava a execução da Lei de 7 de Novembro de 1831. Gama, por
  sua vez, fazia referência indireta ao art. 10 do decreto, que
  estabelecia de modo bastante nítido a responsabilidade de qualquer
  juiz frente a uma demanda dessa natureza jurídica. Logo, não havia
  razão alguma no fundamento do juiz que declinava de decidir uma
  questão que, por força de lei, deveria decidir. O art. 10 do decreto
  de 1832 -- este mesmo que Gama indignado perguntava se estava revogado
  -- é taxativo e não deixa espaço para dúvida ou adiamento até que
  juízo contencioso algum resolvesse a questão. Cf. Art. 10. "Em
  qualquer tempo, em que o preto requerer a qualquer juiz, de paz ou
  criminal, que veio para o Brasil depois da extinção do tráfico, o juiz
  o interrogará sobre todas as circunstâncias que possam esclarecer o
  fato, e oficialmente procederá a todas as diligências necessárias para
  certificar-se dele, obrigando o senhor a desfazer todas as dúvidas que
  se suscitarem a tal respeito. Havendo presunções veementes de ser o
  preto livre, o mandará depositar e proceder nos mais termos da lei".}

Deixou-se de considerar especial o processo administrativo,
adrede\footnote{. Previamente.} estabelecido, para esta hipótese
extraordinária?

Por que motivo?

***

É uma curiosa novidade, que, de contínuo, soa-me aos ouvidos, no juízo e
nos tribunais, que a importação de africanos, no Brasil, foi proibida
por Lei de 7 de Novembro de 1831!

Não é a única heresia (\emph{hipocrisia}, talvez que por semelhança de
rima) ia-me caindo dos bicos de pena!

Hoje, nos juízos, e nos tribunais, quando um africano livre, para evitar
criminoso cativeiro, promove alguma demanda, exigem os sábios
magistrados que ele prove - \emph{qual o navio em que veio; qual o nome
do respectivo capitão}.

Negros boçais, atirados a rodo, como irracionais, no porão de um navio;
como carga, como porcos, desconhecedores até da língua dos seus
condutores, obrigados a provar - \emph{a qualidade, e o nome do navio em
que vieram; e o nomeado respectivo capitão}!!

Isto é justiça para negros; e se os negros se reunissem em tribunal,
para honra de tais juízes, não fariam obra pior.

Estes juízes se parecem com o divino Jesus!\\
Este fazia falar os mudos; e aos cegos abrir os olhos!

***

O seu a seu dono.

A glória da proibição do abominável tráfico de africanos, no Brasil,
pertence a nação portuguesa; foi decretado pelo absoluto d. João
VI;\footnote{. João VI de Portugal (1767-1826), nascido em Lisboa,
  Portugal, foi rei de Portugal, Brasil e Algarves.} está na memorável
Lei de 26 de Janeiro de 1818; conta 62 ANOS DE EXISTÊNCIA, e \emph{não}
49; foi promulgada para inteira execução do Tratado de 22 de Janeiro de
1815, e da Convenção Adicional de 28 de Julho de 1817.\footnote{. Gama
  expõe brevemente elementos de sua cronologia normativa do contrabando.
  No mês seguinte, no artigo \emph{Questão Jurídica,} ele daria a
  público de modo magistral o desenvolvimento desse raciocínio
  doutrinário. Vejamos a síntese das normatividades aduzidas. O alvará
  de 26/01/1818 possuía força de lei e reforçava a proibição do comércio
  de escravizados em portos da costa da África acima da linha do
  Equador. Além da punição aos traficantes, previa igualmente condições
  para a liberdade dos escravizados apreendidos. O alvará, ademais,
  executava o tratado bilateral entre Portugal e Grã-Bretanha, de
  22/01/1815, e a Convenção adicional de 28/07/1817, diplomas que
  pactuavam termos e responsabilidades entre ambos países para a
  supressão do comércio transatlântico de escravizados em suas
  respectivas jurisdições.}

Por Aviso de 14 de Julho de 1821 declarou o governo que essa lei estava
em seu inteiro vigor.

Por outro aviso, de 28 de Agosto, do mesmo ano, o governo deu instruções
à comissão mista, prescrevendo normas para o processo de apreensão dos
navios e dos escravos.

Estas instruções foram reproduzidas, e novamente recomendadas por Aviso
de 3 de Dezembro do referido ano.

Por Portaria de 21 de Maio de 1831 o ministério da Justiça recomendava,
\emph{para estrita observância das leis}, a mais rigorosa atividade na
apreensão dos \emph{pretos novos}, que fossem criminosamente importados
no império procedendo a pesquisas e a rigoroso inquérito.\footnote{. A
  portaria nº 111, de 21/05/1831, do ministério da Justiça, recomendava
  vigilância policial para "evitar a introdução de escravos por
  contrabando". Em caso de localizarem africanos desembarcados no Brasil
  por contrabando, as autoridades policiais e judiciárias deveriam
  apreendê-los, lavrar corpo de delito, proceder nos termos de direito
  e, ao fim, restituir as liberdades escravizadas nas malhas do
  contrabando, punindo os "usurpadores dela".}

E o Poder Legislativo, já por a Lei de 20 de Outubro de 1823,\footnote{.
  Aprovada no bojo do processo constituinte de 1823, esta lei declarava
  em vigor uma série de normas portuguesas que possuíam inquestionável
  força normativa no Brasil até abril de 1821. O art. 1º da lei fazia
  explícita menção às Ordenações como um desses conjuntos normativos que
  voltavam oficialmente a ter vigência no Brasil. Com a citação de lei
  nacional, Gama articulava distintas normatividades anti-tráfico.}
tinha explicitamente admitido aquela de 26 de Janeiro de 1818.

Parece que interesses inconfessáveis criam anacronismos nos
tribunais!... Do gládio\footnote{. Espada.} de Themis\footnote{.
  Divindade da Antiga Grécia que personificava as ideias de justiça
  divina, direito natural e bom conselho. Era comumente representada com
  uma balança em uma mão e uma espada na outra mão.} fez-se algema para
escravos...

O criminoso contrabandista vê nos tribunais uma nova Constantinopla,
\footnote{. A metáfora é riquíssima em significados. Pode ser lida,
  entre outras formas, como referência ao expansionismo desordenado
  associado à assunção de Constantinopla como capital e símbolo do
  Império Romano (330-395).}onde os sacerdotes discutem teologia,
enquanto a pátria corre perigo...

A lei é um anemoscópio\footnote{. Cata-vento, instrumento que indica as
  variações e mudanças do tempo. Por extensão de sentido, sugere uma
  coisa que gira ao sabor do vento.}; É o-lo o Deus da situação.

Diante destes desastres judiciários, que se reproduzem todos os dias,
parece que nós, os aventureiros da emancipação, estamos, em nome da lei,
impondo preceito\footnote{. Regra, norma.} ao dislate!...\footnote{.
  Despautério, estupidez.}

Vosso amigo,

L. Gama.

\textbf{27. 2ª VARA CÍVEL}\footnote{. In: \emph{A Província de S. Paulo}
  (SP), Seção Livre, Foro da Capital, 28/11/1880, pp. 1-2. A
  \emph{Gazeta da Tarde} (RJ)\emph{,} edição de 30/11/1880, replica
  partes do artigo.}

\textbf{*didascália*}

\emph{No curso da causa de liberdade de "Elisa, mulher branca, escrava",
Luiz Gama teve um embate forte com dois juízes, entre eles, o "juiz
proprietário" Camilo Gavião Peixoto. Sim, o juiz proprietário de Elisa
tomou parte no próprio processo. O artigo é mais uma aula de direito da
lavra do advogado negro. Atento ao quadro político, Gama afastava dos
abolicionistas a pecha de agitadores que aterrorizavam o país. Ao
contrário: os conservadores da ordem escravocrata, os "arautos do
terror", é que iam "acastelando-se nos tribunais (...), influindo nas
suas decisões e pondo em perigo a providência da lei e a dignidade da
nação". Gama, portanto, chamava para si e para a causa que defendia
valores como o respeito à lei e à dignidade da nação. Claro que estamos
diante do melhor da retórica abolicionista própria do novo momento da
luta política no Império. No entanto, a conjuntura política passava a
matizar ações no juízo local, haja vista a conclusão do artigo de Gama
ser dedicada à distinção de um partido abolicionista e outro
escravocrata. Gama, porém, tinha uma causa concreta para solucionar.
Tinha uma cliente que corria risco de vida. Tinha um "juiz-proprietário"
querendo matá-la. O "senhor estava tomado de ódio violento, queria a
escrava para picá-la a chicote", indignava-se Gama, pedindo que outro
juiz, responsável interino pela jurisdição, acolhesse sua demanda e a
mantivesse em depósito, sobretudo diante do perigo de vida iminente. O
juiz Mello negou a pretensão do advogado Gama. Ao fazer isso, deixou
evidente seu entendimento grosseiro do processamento e julgamento de
causas de liberdade. Estupefato como despacho do juiz, Gama perguntava:
"Isto é direito? Este direito tem fundamento filosófico? Este fundamento
comporta os princípios de lógica?" A conclusão seria mais um monumento à
liberdade, mais uma página memorável de sua literatura
normativo-pragmática em tempos de escravidão.}

\emph{***}

Juiz -- o exmo. sr. dr. Bellarmino Peregrino da Gama e Mello.\footnote{.
  Bellarmino Peregrino da Gama e Mello (?-?) foi advogado, juiz de
  direito, chefe de polícia e desembargador dos tribunais da Relação de
  Ouro Preto e de São Paulo.}

Hoje no Brasil, para muitos poderosos, como outrora em Roma, ao levantar
do império, por entre ondas de sangue, a liberdade é um perigo.

Pretendê-las é despertar cautelas de segurança; auxiliá-la, é dar prova
de falta de patriotismo; promovê-la é atentar contra o direito de
propriedade, abalar a fortuna pública, prejudicar a particular, cavar a
ruína do estado: tal é o terrível boato, sinistramente propalado em
todos os pontos do país, pelos arautos do terror, pelos salteadores da
lei, em prejuízo de um milhão e quinhentas mil vítimas do mais
abominável crime.

O terror vai, infelizmente, pouco e pouco, invadindo os auditórios,
acastelando-se nos tribunais, perturbando a calma e a imparcialidade de
alguns juízes ilustrados e respeitáveis, influindo nas suas decisões e
pondo em perigo a providência da lei e a dignidade da nação.

Eis um exemplo:

Requereu Elisa, mulher branca, escrava do exmo. sr. dr. Camilo Gavião
Peixoto,\footnote{. Camilo Gavião Peixoto (1830-1883) foi banqueiro,
  delegado de polícia e deputado.} a sua alforria, mediante a
indenização do seu justo valor; e apresentou pecúlio\footnote{.
  Patrimônio, quantia em dinheiro que, por lei (1871), foi permitido ao
  escravizado constituir a partir de doações, legados, heranças e
  diárias eventualmente remuneradas.} legalmente constituído em moeda,
no valor de réis 800\$000.

Funcionava então na Segunda Vara Cível o exmo. sr. dr. Rocha Vieira,
substituto, com jurisdição plena em ausência do juiz-proprietário, que
ocupava interinamente uma cadeira no egrégio Tribunal da
Relação.\footnote{. Isto é, Camilo Gavião Peixoto, a um só tempo juiz e
  proprietário, fora chamado a ocupar temporariamente o lugar de
  desembargador do Tribunal da Relação de S. Paulo.}

Foi aceito o requerimento; o emérito juiz, para garantia dos direitos da
libertanda e dos dominicais\footnote{. Senhoriais.}, mandou-a depositar,
bem como o pecúlio, em mão de pessoa idônea, nomeou-lhe
curador\footnote{. Aquele que está, em virtude de lei ou por ordem de
  juiz, incumbido de cuidar dos interesses e bens de quem se acha
  judicialmente incapacitado de fazê-lo.} e ordenou a audição do senhor
relativamente à pretensão; tudo nos termos de direito.

Reassumindo a jurisdição, o muito digno juiz proprietário, o exmo. sr.
dr. Camillo Gavião requereu, sem fazer oposição à pretensão de alforria,
que lhe fosse a escrava entregue para continuar em seu poder. E porque
isto me chegasse ao conhecimento, enderecei de pronto uma petição ao
ilustrado juiz, opondo-me, em nome da moral e da humanidade, àquela
simulada e perversa pretensão referindo - {[}"{]} que o senhor estava
tomado de ódio violento, queria a escrava para \emph{picá-la a chicote,
pois que prometia realizar esta tortura ainda quando a escrava obtivesse
alforria!} {[}"{]} - e concluí apelando para o direito, para a equidade
e para a honra do Benemérito juiz.

Ontem à tarde fui intimado deste venerando despacho:

"Nas causas de arbitramento, para liberdade, \emph{não concede a lei}
(!!!), ao escravo que, por tal meio, pretende libertar-se, o direito de
ser retirado da casa de seu senhor, e depositado. {[}!!!{]}

Mando, pois, que se relaxe o depósito da escrava Elisa, para ser
entregue ao seu senhor, que se obrigará, por termo, nos autos, a não
dispor nem retirar desta cidade a dita escrava, enquanto não se decidir
a presente causa, sob as penas da lei.

Os (sic) peticionário de fl. 10 (L. Gama), acerca da matéria de sua
petição, deve dirigir-se, \emph{querendo}, à polícia, que é a autoridade
competente para prevenir qualquer ato de rigor, punido pela lei, do
senhor contra a escrava.

S. Paulo, 25 de novembro de 1880.

GAMA E MELLO."

\_\_\_\_\_\_\_\_\_\_

Vou agora sujeitar esta admirável norma de jurisprudência de borracha,
depois de cautelosamente besuntada de óleo de nafta,\footnote{. Líquido
  combustível, inflamável.} modificador químico por excelência desta
bamboleante matéria, às lindes\footnote{. Raias, limites.} inalteráveis
da lei.

Afirma o respeitável juiz neste seu calculado despacho:

-- ~1º: que nas causas de liberdade, por arbitramento, é inadmissível o
depósito do libertando, em mão particular, por contrário à lei;\\
-- 2º: que o remédio legal, concedido ao manumitente\footnote{.
  Alforriando, que demanda a liberdade.}, neste caso, para garantia do
seu direito, é assinar, o senhor, um termo, nos autos, pelo qual se
obrigará a não dispor, nem retirar do lugar da ação o libertando,
enquanto não for o pleito decidido;\\
-- 3º: que se o libertando, ou alguém por ele, arrecear-se de violências
físicas ou morais, prejudiciais ao seu direito, deve recorrer à polícia.

Isto, porém, é arbitrário e viola flagrantemente o direito.

Aos pleitos de liberdade, \emph{sem exceção}, para garantia dos
escravos, e segurança dos direitos que possam ter os senhores, precede o
depósito daqueles, em poder de pessoa idônea (Av. 3 de novembro 1783; -
B. Carn. - Dir. Civ. I, I tit. 3º § 32 not. A; Alv. 10 março 1682; -
Ramalho - Prax. Bras. § 100 nº 5 e not.; - Cod. Com. art. 204 e 212; -
revista 12 fevereiro 1873 - Ga{[}z{]} jur. vol 1º pags. 83 e 338; -
argum. do Decreto nº 5135 - 13 novembro 1872, art. 81, § 2º\footnote{.
  Vejamos as referências na ordem em que são citadas. O aviso de 1783 se
  lê na indicação exata dada por Gama, a saber, em Borges Carneiro,
  \emph{Direito Civil}, Livro 1º, Título 3º, § 32, nota A. Segundo
  Borges Carneiro, tal aviso "declarou que as Pretas que se achavam
  presas em cadeia pública, enquanto se litigava sobre sua liberdade,
  fossem, por esta ser mui favorável transferidas para depósitos
  particulares, onde seus contendores se sustentassem durante o
  litígio". É de se notar, igualmente, que o § 32 tratava do "favor da
  liberdade" e se constituía de cinco ideias centrais, sendo quatro
  delas bastante caras ao conhecimento normativo que Gama colocava em
  prática em São Paulo. Descontadas citações internas e referências
  externas, são elas: 1º. "Todo o homem se presume livre; a quem requer
  contra a liberdade incumbe a necessidade de provar"; 2º.
  \emph{"}Quando se questiona se alguém é livre ou escravo, esta ação ou
  exceção goza de muitos privilégios concedidos em favor da liberdade".
  3º. "A favor do pretendido escravo não só pode requerer ele mesmo, mas
  qualquer pessoa (\emph{assertor}), ainda repugnando ele. 4º. "A causa
  da liberdade não admite estimação, por ser ela de valor inestimável
  (...)". O alvará de 1682 regulava a liberdade e a escravidão de negros
  aprendidos na guerra dos Palmares, na antiga capitania de Pernambuco.
  Conhecido da historiografia sobretudo pela regulação da prescrição do
  cativeiro após cinco anos de posse da liberdade, nesse texto Gama se
  reporta a outro comando normativo do alvará -- o quinto parágrafo, que
  regulava pleitos de liberdade --, em que o rei de Portugal outorgava
  que os cativos poderiam demandar e requerer liberdade, ainda que
  contra o interesse de seus senhores. Art. 204. ``Se o comprador sem
  justa causa recusar receber a cousa vendida, ou deixar de a receber no
  tempo ajustado, terá o vendedor ação para rescindir o contrato, ou
  demandar o comprador pelo preço com os juros legais da mora; devendo,
  no segundo caso, requerer depósito judicial dos objetos vendidos''.
  Art. 212. ``Se o comprador reenvia a cousa comprada ao vendedor, e
  este a aceita (art. 76), ou, sendo-lhe entregue contra sua vontade, a
  não faz depositar judicialmente por conta de quem pertencer, com
  intimação do depósito ao comprador, presume-se que consentiu na
  rescisão da venda''. A Revista de 12 Fevereiro de 1873, que aliás
  serviu não só como reforço ao argumento, mas, antes, como base para
  articular o Alvará de 1682 com o Código Comercial (1850), pode ser
  lida em: Candido Mendes de Almeida e Fernando Mendes de Almeida,
  \emph{Arestos do Supremo Tribunal de Justiça}, 1883, p. 770. As
  páginas da \emph{Gazeta Jurídica, Vol. 1} (1873), citadas no corpo do
  texto não conferem. Possivelmente, deu-se algum erro tipográfico,
  motivo pelo qual não replicarei extratos delas aqui. Art. 81. O
  processo sumário é o indicado no art. 65 do dec. nº 4.824 de
  22/11/1871. § 2º. ``Os manutenidos em sua liberdade deverão contratar
  seus serviços durante o litígio, constituindo-se o locatário, ante o
  juiz da causa, bom e fiel depositário dos salários, em benefício de
  qualquer das partes que vencer o pleito. Se o não fizerem, serão
  forçados a trabalhar em estabelecimentos públicos, requerendo-o ao
  juiz o pretendido senhor''.}).

No regulamento promulgado pelo Decreto nº 5.135 de 13 de Novembro de
1872, para execução da Lei de 28 de Setembro de 1871,\footnote{.
  Refere-se à conhecida Lei do Ventre Livre, que declarava livre os
  filhos da mulher escravizada nascidos a partir da promulgação daquela
  lei. A lei também regulava outras matérias, a exemplo do processamento
  e julgamento de causas de liberdade.} no capítulo VII, em que se
estabelece as formas dos processos, lê-se o artigo 80, que se inscreve:

DAS CAUSAS EM FAVOR DA LIBERDADE

E, no artigo 84, está determinado o seguinte:

"Para alforria, \emph{por indenização do valor}, e para a remissão é
suficiente uma petição, na qual exposta a intenção etc., etc."

No art. 83, estatui:

"Nos casos para que este regulamento não designa forma de processo, o
juiz procederá administrativamente."

Está, pois, demonstrado, com evidência incontestável, que a demanda
manumissória\footnote{. Processo em que se demanda a liberdade.} precede
o depósito do manumitente como preliminar necessário dela;

Que o depósito, como a lei recomenda, por ser mais favorável à
liberdade, deve realizar-se em poder de pessoa particular;

Que os direitos do senhor sobre os salários do escravo estão garantidos
por lei;

Que o processo, para aquisição de alforria por indenização do valor, é
\emph{judiciário}, e de forma sumária;\footnote{. Simplificada, célere.}

Que é \emph{judiciário por ter a forma estabelecida na lei}
especialmente.\\
Isto posto, a negação do depósito do libertando é uma violação inegável
da lei.\\
O exmo. sr. dr. Gama e Mello, juiz ilustrado e íntegro, porém de todo
ponto suspeito nestas graves questões de alforria; porque, embora
liberal, como os da sua escola, só admite liberdade \emph{de si para
cima}; e, prevenido por sentimentos políticos, vê em cada libertanda um
desastre para os divinos fazendeiros, e pródromos\footnote{.
  Precursores.} lôbregos\footnote{. Por sentido figurado, tétricos,
  tenebrosos.} das finanças da nação, cuja riqueza para S. Excia. e para
os seus desorientados consectários\footnote{. Efeitos, resultados.}, tem
por base exclusiva o braço dos escravos, cevadores modernos das moréias
de Polião\footnote{. A alegoria é complexa, indicando, em uma possível
  interpertação, que os escravos eram devorados pelo sistema econômico
  vigente. Afinal, Públio Védio Polião (?- a.C.), militar e político
  romano, passou à história pelo trato cruel e homicida que dispensava
  aos seus escravos, lançando-os em tanques d'água repleto de moréias,
  para que fossem por elas devorados.}; esmiuçou curioso as coleções de
arestos\footnote{. Acórdão, decisão de tribunal que serve de paradigma
  para solucionar casos semelhantes.} judiciários e nelas encontrou o
absurdo Acórdão de 26 de junho de 1874, proferido pelos hircos\footnote{.
  O mesmo que bode.} gibosos\footnote{. Corcunda.} da Relação do Ouro
Preto, que se lê no 5º volume do \emph{Direito}, a páginas 66 e
67.\footnote{. Os desembargadores de Ouro Preto decidiam, em síntese,
  pela ilegalidade do depósito judicial da libertanda no curso de uma
  ação de liberdade mediante pagamento. Cf. \emph{O Direito}, 5º Vol.,
  1874, pp. 66-68. É de se notar que, além da denegação de depósito no
  curso da ação de liberdade, o precedente invocado pelos
  desembargadores de São Paulo possuía outro tenebroso paralelo com o
  julgado de Ouro Preto. Assim como a libertanda Elisa era escravizada
  por um juiz em S. Paulo, Umbellina, a libertanda que os
  desembargadores de Ouro Preto insistiram em manter em cativeiro, era
  escravizada por um magistrado, a saber, o desembargador João de Souza
  Nunes Lima.}

S. Excia., juiz sobremodo esclarecido, se o não dominasse o vezo
partidário, com a simples leitura comparada dos artigos 80, 84, 85 do
regulamento citado, nº 5.135 de 1872, veria o monstruoso atentado
grosseiramente cometido pelos corcundas do empório\footnote{. Mercado,
  bazar, centro comercial. Pela sequência na exposição do argumento -- e
  considerando que o desembargador Nunes Lima era parte interessadíssima
  no julgamento de seus pares de Ouro Preto --, pode-se inferir que o
  acórdão lavrado foi como uma peça de empório, i. e., vendido a preço
  de mercado. Possivelmente, esse foi um dos sentidos do emprego da
  palavra "empório" imediatamente após a citação do acórdão.} do Ouro
Preto; e não teria a infelicidade de os imitar, procedimento que
sinceramente deploro.\footnote{. Cf., respectivamente, art. 80: ``Nas
  causas em favor da liberdade: § 1º. O processo será sumário; § 2º.
  Haverá apelações \emph{ex-officio} quando as decisões forem contrárias
  à liberdade. (...)''. Art. 81. ``O processo sumário é o indicado no
  art. 65 do dec. nº 4.824 de 22/11/1871. § 1º. As causas de liberdade
  não dependem de conciliação. § 2º. Os manutenidos em sua liberdade
  deverão contratar seus serviços durante o litígio, constituindo-se o
  locatário, ante o juiz da causa, bom e fiel depositário dos salários,
  em benefício de qualquer das partes que vencer o pleito. Se o não
  fizerem, serão forçados a trabalhar em estabelecimentos públicos,
  requerendo-o ao juiz o pretendido senhor. § 3º. Estes processos serão
  isentos de custas''. Art. 82. ``O processo para verificar os fatos do
  art. 18 deste regulamento é o dos parágrafos do art. 63 do dec. nº
  4.824 de 22/11/1871. § Único. Essa mesma forma de processo servirá
  para verificação do abandono conforme os arts. 76, 77 e 78 deste
  regulamento''. Art. 83. ``No caso de infração do contrato de prestação
  de serviços, a forma do processo é a da lei de 11/10/1837; e o juiz
  competente é o de orfãos nas comarcas gerais, e o de direito nas
  comarcas especiais, onde não houver juiz privativo de orfãos. § Único.
  Havendo perigo de fuga, ou no caso de fuga, pode ser ordenada a prisão
  do liberto contratado, como medida preventiva, não podendo, porém,
  exceder de trinta dias''. Art. 84. ``Para a alforria por indenização
  do valor, para a remissão, é suficiente uma petição, na qual, exposta
  a intenção do peticionário, será solicitada a vênia para a citação do
  senhor do escravo ou do possuidor do liberto. Antes da citação o juiz
  convidará o senhor para um acordo, e só em falta deste proseguirá nos
  termos ulteriores. (...). § 1º. Se houver necessidade de curador,
  precederá à citação nomeação do mesmo curador, em conformidade das
  disposições deste regulamento. § 2º. Feita a citação, as partes serão
  admitidas a louvarem-se em arbitradores, se houver necessidade de
  arbitramento; e o juiz prosseguirá nos termos dos arts. 39, 40 e 58
  deste regulamento, decretando ao final o valor ou o preço da
  indenização, e, paga esta, expedirá a carta de alforria ou o título de
  remissão. § 3º. Se a alforria for adquirida por contrato de serviços,
  esta circunstância será mencionada na carta; e, no caso de ulterior
  remissão, não se passará título especial, mas bastará averbá-la na
  mesma carta''. Art. 85. ``Nos casos para que este regulamento não
  designa forma de processo, o juiz procederá administrativamente''.}

Naquela aludida decisão resolveram aqueles juízes, sem saber o que
faziam, que:

-- "O depósito preliminar do escravo não tem lugar \emph{nos processos
administrativos para arbitramento}, porque esse meio é só admitido
\emph{na ação de liberdade}, \emph{ou de escravidão}. A prática em
contrário não se apoia nem na Lei de 28 de Setembro, nem no regulamento
(!!!), e importa antecipadamente privar aos senhores da posse dos seus
escravos (que solícitos procuradores!), tanto mais que nem se pode
apadrinhar com o perigo de sevícias\footnote{. Crueldades, torturas.}
(que abstrusa\footnote{. Intrincada, obscura.} sabedoria!) pois que, na
forma da lei, pela insuficiência do valor exibido, \emph{podem} os
escravos voltar ao poder dos senhores!..."

Processo administrativo de arbitramento!...

Isto, tristíssimo é de dizê-lo: se não é fruto da mais supina\footnote{.
  Excessiva, demasiada.} ignorância, é uma preterição voluntária do
dever.

Considerarei, para terminar, as imposições finais do venerando despacho
do exmo. sr. dr. Bellarmino.

O escravo não pode ser depositado porque nesta hipótese a lei proíbe o
depósito; logo, a lei neste caso veda expressamente a limitação da posse
dominical,\footnote{. Senhorial.} como porém manda o meritíssimo juiz
que o senhor, por termo assinado nos autos, se obrigue \emph{a não
dispor} \emph{nem retirar o escravo do lugar do pleito? }

Isto é disposição de lei?\\
Que lei é essa?\\
Isto é direito?\\
Este direito tem fundamento filosófico?

Este fundamento comporta os princípios de lógica?\\
O libertando, para garantia do seu direito, pode recorrer à polícia?\\
Pois a polícia já tem alçada ou interferência nas causas cíveis?\\
Os magistrados procedem em tal caso de mão comum com ela?\\
Não, isto não é jurisprudência, não é fruto da inteligência, do estudo e
da ilustração de um magistrado sisudo e respeitável; é uma precipitada
evasão.\footnote{. Evasiva, manobra, desculpa ardilosa.}

O país divide-se atualmente em dois partidos, um filantrópico,
destemido, arvorando o lábaro\footnote{. Estandarte, bandeira.} da
justiça, proclama a liberdade de um milhão e quinhentas mil vítimas; o
outro, imagem viva do Atlântico, intumescido\footnote{. Inchado,
  engrossado.} de cóleras, pretende impedir o curso impetuoso do
Amazonas.

Neste, onde ergueu-se a bandeira negra da escravidão, está o exmo. sr.
dr. Gama e Mello.

S. Paulo, 26 de novembro de 1880.

L. GAMA.

\textbf{28. QUESTÃO JURÍDICA -- Subsistem os efeitos manumissórios da
lei de 26 de Janeiro de 1818 depois das de 7 de Novembro de 1831 e 4 de
Outubro de 1850?}\footnote{. In: \emph{A Província de S. Paulo} (SP),
  Seção Livre, 18/12/1880, p. 5. Aqui houve um pequeno equívoco no
  título original, já que a lei não é de outubro, mas sim de 4 de
  Setembro de 1850. Ao longo do texto, Gama sempre se reporta à data
  correta. Outra observação pertinente: Mennucci, acompanhado por
  Ferreira, ocultou a interrogação final, substituindo-a por um ponto
  final simples, como se o autor não partisse de uma pergunta para
  estruturar seu raciocínio jurídico.}

\textbf{*didascália*}

\emph{Sendo o mais conhecido estudo jurídico de Luiz Gama, haja vista
ser republicado desde 1937, "Questão Jurídica" é um página definitiva na
história do direito e da Abolição no Brasil. Dividido em sete seções, o
artigo estabelece o ano de 1818 como marco temporal da proibição do
comércio de escravizados da África com o Brasil. Ato contínuo, o texto
correlaciona e sustenta que esse marco legislativo geraria direitos de
liberdade até o presente em que era escrito, o ano de 1880. A tese e sua
fundamentação nas fontes do direito, combinadas com a função pragmática
que exerciam, eram inéditas e originais. Se tal tese ganhasse força
normativa, significaria simplesmente o fim da escravidão no Brasil pela
mediação do judiciário. Não foi essa a história, como sabemos. A tese
caiu vencida no Tribunal da Relação. Os desembargadores mandaram o
"preto Caetano, africano livre", voltar ao cativeiro do violento
comendador Polycarpo Aranha. Caetano tinha aproximados sessenta anos de
idade. Se Gama usasse a Lei de 1831 como marco para a proibição da
entrada de africanos escravizados no país, os desembargadores
rejeitariam sumariamente o argumento de Gama de que Caetano era um
africano livre. Ciente disso, Gama elaborou uma genealogia da Lei de
1831 e, com veio de historiador do direito, concluiu que o marco de 1831
estava condicionado ao de 1818. Assim, fincando o ano de 1818 como base,
ampliaria a razão do argumento para Caetano e para a quase totalidade
dos africanos no Brasil de 1880. Era uma estratégia de liberdade ousada.
Mas havia forças sombrias que governavam o Império do Brasil. E o
Partido Liberal, denunciava Gama, estava de corpo e alma comprometido
com tais forças sombrias da política da escravidão. Um parecer do
Conselho de Estado, sob domínio liberal, e um Aviso Confidencial,
escrito por ninguém menos que José Thomaz Nabuco de Araújo, pai de
Joaquim Nabuco, são invocados por Gama para arrematar a sua magistral
aula de direito. Ambos os documentos, fulminava Gama, "foram escritos
com penas de uma só asa, são formas de um só pensamento, representam um
só interesse: sua origem é o terror, seus meios a violência, seu fim a
negação direito. Os fatos têm a sua lógica infalível". }

\emph{***}

Na sessão do colendo Tribunal da Relação, celebrada a 26 do precedente,
quando discutia-se a concessão da ordem de \emph{habeas-corpus}, que
obtive, impetrada a favor do preto Caetano, africano livre, havido como
escravo do sr. comendador Joaquim Polycarpo Aranha\footnote{. Joaquim
  Polycarpo Aranha (1809-1902), natural de Ponta Grossa (PR), foi
  fazendeiro e político estabelecido em Campinas (SP).}, fazendeiro do
município de Campinas, o exmo. sr. desembargador Faria\footnote{. José
  Francisco de Faria (1825-1902), natural do Rio de Janeiro (RJ), foi
  político e magistrado. Foi promotor público, chefe de polícia da Corte
  (Rio de Janeiro), juiz de direito, desembargador dos tribunais da
  Relação de Ouro Preto e de São Paulo, procurador da Coroa, Soberania e
  Fazenda Nacional e ministro do Supremo Tribunal de Justiça. Teve
  muitos embates com Luiz Gama na parte contrária, sendo este, em que
  Gama advogou \emph{habeas-corpus} para o africano congo Caetano, o
  mais célebre.}, digno procurador da Coroa, em enérgico discurso,
apoiando-se nas opiniões dos exmos. deputado Souza Lima, externada na
Câmara temporária, e conselheiro Nabuco de Araújo, \footnote{. José
  Thomaz Nabuco de Araújo Filho (1813-1878), baiano de Salvador, foi
  advogado, juiz de direito e político de expressão nacional. Foi
  deputado, presidente da província de São Paulo (1851-1852), ministro
  da Justiça (1853-1857) e senador do Império (1857-1878).}manifestada
em um parecer do Conselho de Estado\footnote{. Órgão consultivo ao
  imperador, organizado em seções, formado por uma seleção de ministros
  de Estado e outras figuras-chave do direito e da política nacional.
  Para o Segundo Reinado, suas atribuições estão marcadas na Lei nº 234
  de 23 de Novembro de 1841.}, afirmou, por entre aplausos dos exmos.
desembargador Gomes Nogueira\footnote{. Antônio Barbosa Gomes Nogueira
  (1823-1885), nascido em Sabará (MG), foi juiz, desembargador dos
  tribunais da relação de Minas Gerais e de São Paulo e também político,
  presidindo a província do Paraná (1861-1863).} e juízes de direito
drs. Gama e Mello\footnote{. Bellarmino Peregrino da Gama e Mello (?-?)
  foi advogado, juiz de direito, chefe de polícia e desembargador dos
  Tribunal da Relação de Ouro Preto.} e Gonçalves Gomide, que a Lei de
26 de Janeiro de 1818\footnote{. Ementa: Estabelece penas para os que
  fizerem comércio proibido de escravos.} fora implicitamente revogada
por a de 7 de Novembro de 1831\footnote{. Considerada uma lei vazia de
  força normativa, recebendo até o apelido de "lei para inglês ver", a
  conhecida "Lei de 1831" previa punição para traficantes escravizadores
  e, de maneira não tão assertiva como a historiografia crava, declarava
  livres os escravizados que chegassem ao Brasil após a vigência da lei.};
que este fato, aliás de máxima importância, estava no espírito
esclarecido de todo país e dos poderes do Estado, que cogitavam, com
muito patriotismo e critério, dos meios de resolver o tormentoso
problema do elemento servil; e que, se, pelo contrário, essa lei
continuasse em vigor, todos esses homens ilustradíssimos, deputados e
senadores do Império, estadistas notáveis, estariam em grave erro: só o
Poder Judiciário seria bastante para resolver a questão!

Este perigoso discurso, este enviesado parecer do respeitável
magistrado, obrigou-me a escrever este artigo.

\_\_\_\_\_\_\_\_\_\_\_\_\_\_\_\_\_\_

Não sei se é um compromisso; não afirmo que seja um dever, mas, para
mim, é fora de contestação que o honrado sr. procurador da Coroa, por
virtude ou por temor, põe ombros\footnote{. Se dedica, trabalha com
  afinco.} ao carrego\footnote{. Fardo, encargo, carga pesada e onerosa.}
do maquiavelismo\footnote{. Expressão que remete às ideias formuladas
  por Nicolau Maquiavel (1469-1527), pensador político florentino que se
  destacou, em parte, por relativizar a moralidade em prol da eficácia
  das decisões. Nesse contexto, pode-se ler o termo por seu sentido
  figurado, entendendo a ideologia governamental não como um sistema de
  direitos subordinado aos princípios constitucionais e liberais
  marcados na Carta outorgada de 1824, mas enquanto um sistema político
  perverso que se organiza a partir do cálculo interesseiro dos donos do
  poder.} governamental neste melindrado cometimento da abolição da
escravatura.

Essa manifestação tremenda, repleta de inconsequências jurídicas, que
acabo de referir, com cuidada fidelidade, tem duas partes distintas; uma
é a repetição nua dos sofismas políticos do governo chinês, de que fala
o clássico Jeremias Bentham\footnote{. Jeremy Bentham (1748-1832) foi um
  filósofo e jurista inglês que exerceu grande influência entre os
  intelectuais de seu tempo. Embora não reste claro a qual texto de
  Bentham Gama se reportava, é possível conjecturar que fosse o
  \emph{Traité des Sophismes Politiques et des Sophismes Anarchiques}
  {[}Tratado dos Sofismas Políticos e dos Sofismas Anárquicos{]}, edição
  póstuma, de 1840, que reunia manuscritos de Bentham. A referência de
  Gama, contudo, poderia ser outro livro de Bentham, uma vez que
  circulava em português o seu \emph{Sofismas anárchicos - Exame critico
  de diversas Declarações dos Direitos do Homem e do Cidadão} (1823).
  Ambos os livros possuíam ideias enfáticas sobre os limites do poder do
  governante, formas de organização social e crítica do direito. Sobre o
  aportuguesamento do prenome de Bentham conforme se lê no corpo do
  parágrafo, optei em grafá-lo conforme a escrita de Gama.}; a outra é
uma duríssima verdade, uma confissão espantosa, feita voluntariamente à
luz do século e perante a razão universal: a magistratura antiga,
enfeudada\footnote{. Submissa, avassalada, submetida.} aos criminosos
mercadores de africanos, envolta em ignomínia\footnote{. Humilhação,
  desonra, infâmia.}, sepultou-se nas trevas do passado; a moderna,
inconsciente, amedrontada, recua espavorida\footnote{. Apavorada,
  aterrorizada.} diante da lei; encara, com súplice\footnote{. Que
  suplica, que implora.} humildade, o Poder Executivo; e, sem fé no
direito, sem segurança na sociedade, e esquivando-se ao seu dever,
declara-se impossibilitada de administrar justiça a um milhão de
desgraçados!

Onde impera o delito a iniquidade\footnote{. Injustiça.} é lei.

Examinemos a questão de direito.

O rei de Portugal, para estrita execução, nos estados do seu domínio, do
solene tratado celebrado com o governo da Grã-Bretanha a 22 de Janeiro
de 1815,\footnote{. O tratado bilateral proibia que navios carregados de
  pessoas escravizadas, oriundos de portos da costa africana situados ao
  norte da linha do Equador, aportassem em território brasileiro.} e da
Convenção Adicional de 28 de Julho de 1817\footnote{. A convenção
  estipulava condições para efetivar o tratado de 1815 e assegurar a
  proibição do tráfico de escravizados nas jurisdições portuguesas ao
  norte da linha do Equador.}, promulgou o memorável Alvará de 26 de
Janeiro de 1818, cujo primeiro parágrafo assim determina:

"Todas as pessoas, de qualquer qualidade e condição que sejam, que
fizerem armar e preparar navios para o resgate e compra de escravos, em
quaisquer dos portos da Costa d'África, situados ao norte do Equador,
incorrerão na pena de perdimento dos escravos, os quais 'imediatamente
ficarão libertos', para terem o destino abaixo declarado...

Na mesma pena de perdimento dos escravos, para ficarem libertos, e terem
o destino abaixo declarado, incorrerão todas as pessoas, de qualquer
qualidade e condição, que os conduzirem a qualquer dos portos do Brasil,
em navios com bandeira que não seja portuguesa."

\_\_\_\_\_\_\_\_\_\_\_\_

Sem embargo da interessada desídia\footnote{. Negligência,
  irresponsabilidade.} dos juízes e notória venalidade dos funcionários,
que escandalosamente auxiliavam, sem o mínimo rebuço, a transgressão
desta lei, foi ela, de contínuo, mandada observar, tanto em Portugal
como no Brasil.

Aqui, por Aviso de 14 de Julho de 1821, recomendou o governo que as
autoridades pusessem o mais escrupuloso cuidado na sua fiel observância.

Para complemento desta importante providência, por outro Aviso expedido
a 28 de Agosto do mesmo ano, deu instruções à Comissão Mista para
regularidade do serviço de apreensão dos escravos e dos navios
negreiros.

E, por outro, de 3 de Dezembro, novas recomendações foram feitas para
maior solicitude à mesma Comissão.

Em 1823, por a Lei de 20 de Outubro, foi explicitamente adotada sem
limitação alguma a de 1818.\footnote{. Aprovada no bojo do processo
  constituinte de 1823, esta lei declarava em vigor uma série de normas
  portuguesas que possuíam inquestionável força normativa no Brasil até
  abril de 1821. Gama construía o seu argumento, portanto, de modo que
  nem mesmo a nascente legislação nacional escapasse ao repertório
  normativo que concorria para a abolição do tráfico de escravizados e
  liberdades dela decorrentes.}

A 21 de Maio de 1831, o ministro da Justiça expedia a seguinte portaria:

"Constando ao governo de Sua Majestade Imperial que alguns negociantes,
assim nacionais como estrangeiros, especulam, com desonra da humanidade,
o vergonhoso contrabando de introduzir escravos da Costa d'África nos
portos do Brasil, em despeito da extinção de 'semelhante comércio':
manda a Regência provisória, em nome do Imperador\footnote{. Como essa
  portaria é datada de maio de 1831, mês seguinte da Abdicação de Pedro
  I, o imperador em questão, representado pela "Regência provisória", é
  Pedro II, que então contava cinco anos de idade.}, pela Secretaria de
Estado dos Negócios da Justiça, que a Câmara Municipal desta cidade faça
expedir uma circular a todos os juízes de paz das freguesias do seu
território, recomendando-lhes toda vigilância policial ao dito respeito;
e que no caso de serem introduzidos por contrabando alguns escravos
novos, no território de cada uma das ditas freguesias, procedam
imediatamente ao respectivo corpo de delito, e constando por este, que
tal ou tal escravo boçal foi introduzido aí por contrabando, façam dele
sequestro, e o remetam com o mesmo corpo de delito ao juiz criminal do
território, para ele proceder nos termos de direito em ordem a lhe ser
restituída a sua liberdade e punidos os usurpadores dela, segundo o art.
179 do novo Código\footnote{. Isto é, o Código Criminal (1830). Cf. Art.
  179. ``Reduzir à escravidão a pessoa livre que se achar em posse da
  sua liberdade. Penas -- de prisão por três a nove anos e de multa
  correspondente à terça parte do tempo; nunca, porém, o tempo de prisão
  será menor que o do cativeiro injusto e mais uma terça parte''.},
dando de tudo conta imediatamente à mesma Secretaria.

Palácio do Rio de Janeiro, 21 de Maio de 1831.

MANOEL JOSÉ DE SOUZA FRANÇA"\footnote{. A transcrição da portaria
  confere com o original. Como se lê, a portaria nº 111, de 21/05/1831,
  do ministério da Justiça, recomendava vigilância policial para "evitar
  a introdução de escravos por contrabando". Em caso de localizarem
  africanos desembarcados no Brasil por contrabando, as autoridades
  policiais e judiciárias deveriam apreendê-los, lavrar corpo de delito,
  proceder nos termos de direito e, ao fim, restituir as liberdades
  escravizadas nas malhas do contrabando, punindo os "usurpadores dela".}

"N. B. Nesta conformidade se expediram avisos a todas as câmaras
municipais e aos presidentes das províncias, para estes expedirem aos
juízes de paz das mesmas províncias."

A 7 de Novembro deste ano, porque reconhecesse o governo que a lei
vigente, por deficiência manifesta, não atingia ao elevado fim de sua
decretação, e no intuito não só de vedar a continuação do tráfico, "como
de restituir à liberdade os africanos criminosamente importados",
promulgou nova lei:

"Art. 1º: 'Todos os escravos' que entrarem no território ou portos do
Brasil, 'vindo de fora', ficam livres.

Art. 2º: Os importadores de escravos no Brasil incorrerão na pena
corporal do art. 179 do Código Criminal\footnote{. Isto é, pena de
  prisão pelo tempo de cativeiro injusto e ilegalmente imposto a
  terceiros, sendo, ainda, o tempo da punição acrescido em um terço do
  montante total.}, imposta 'aos que reduzem à escravidão pessoas
livres'"...

- "Incorrem na mesma pena os que cientemente comprarem como escravos os
que são declarados livres no art. 1º desta lei."\footnote{. Trata-se de
  uma adaptação autoral do § 4º do art. 3º da Lei de 7 de Novembro de
  1831. A releitura, contudo, preserva o teor normativo do texto. Esse
  parágrafo definia quem seriam considerados como importadores de
  escravizados e quais suas respectivas responsabilidades.}

Para execução desta lei, confeccionou o governo imperial o Decreto de 12
de Abril de 1832\footnote{. O decreto regulava a execução da Lei de 7 de
  Novembro de 1831.}, firmado pelo venerando paulista, senador Diogo
Antonio Feijó\footnote{. Diogo Antonio Feijó (1784-1843) foi um
  sacerdote católico e estadista do Império. Teve destacada atuação na
  burocracia do estado, ocupando posições como deputado, ministro,
  presidente do Senado. Como ministro da Justiça, assinou a Lei que
  marcou seu nome na história legislativa brasileira, proibindo o
  tráfico de escravos para o Brasil (1831).}, ministro e secretário de
estado dos Negócios da Justiça, decreto que contém estas
importantíssimas e salutares disposições:

"Art. 9º: Constando ao intendente geral da polícia, ou a qualquer juiz
de paz ou criminal, que alguém comprou ou vendeu preto boçal\footnote{.
  O negro recém-chegado da África, que ainda não falava o português.}, o
mandará vir a sua presença e examinará se entende a língua brasileira:
'se está no Brasil antes de ter cessado o tráfico da escravatura',
procurando, por meio de intérprete, certificar-se de quando veio
d'África, em que barco, onde desembarcou, por que lugares passou, em
poder de quantas pessoas tem estado, etc. Verificando-se ter vindo
depois da cessação do tráfico, o fará depositar, procederá na forma da
lei, e em todos os casos serão ouvidas, sem delongas supérfluas,
sumariamente, as partes interessadas.

Art. 10º: Em qualquer tempo em que o preto requerer a qualquer juiz, de
paz ou criminal, que veio para o Brasil 'depois da extinção do tráfico',
o juiz o interrogará sobre todas as circunstâncias que possam esclarecer
o fato, 'e oficialmente procederá' a todas as diligências necessárias
para certificar-se dele, obrigando o senhor a desfazer todas as dúvidas
que se suscitarem a tal respeito. Havendo presunções veementes de ser o
preto livre, o mandará depositar e proceder nos mais termos da
lei."\footnote{. A transcrição de ambos os artigos do decreto de
  12/04/1832 confere como original. Uma única ressalva, contudo, que de
  modo algum mitiga o texto normativo, se dá no art. 9º, na alteração de
  ordem na expressão "sem delongas supérfluas, sumariamente". No texto
  legal, se lê "sumariamente, sem delongas supérfluas".}

O mal, porém, não estava só na insuficiência das medidas legislativas,
senão principalmente da máxima corrupção administrativa e judiciária que
lavrava no país.

Ministros da coroa, conselheiros de Estado, senadores, deputados,
desembargadores, juízes de todas as categorias, autoridades policiais,
militares, agentes, professores de institutos científicos, eram
associados, auxiliares ou compradores de africanos livres.

Os carregamentos eram desembarcados publicamente, em pontos escolhidos
das costas do Brasil, diante das fortalezas, à vista da polícia, sem
recato nem mistério; eram os africanos sem embaraço algum levados pelas
estradas, vendidos nas povoações, nas fazendas, e batizados como
escravos pelos reverendos, pelos escrupulosos párocos!...

O exmo. senador Feijó, prevalecendo-se do seu grande prestígio,
sacerdote virtuoso e muito conceituado, levantou enérgica propaganda
entre os seus colegas, nesta província.

Advertiu aos vigários para que não batizassem mais africanos livres como
escravos, porque semelhante procedimento, sobre ser uma inqualificável
imoralidade, era um crime.

Os vigários deram prova de emenda; mostraram-se virtuosos: de então em
diante batizaram sem fazer assentamento de batismo! A religião, como o
vestuário, amolda-se às formas do abdômen de quem o enverga: os ingênuos
vigários também tinham seus escravos...

Os contrabandistas conseguiram tal importância política no Império,
tinham interferência tão valiosa nos atos do governo, que iam ao ponto
de dissolver ministérios, como publicamente, sem réplica nem contestação
asseverou na imprensa o exmo. sr. conselheiro Campos Mello\footnote{.
  Antonio Manuel de Campos Mello (1809-1878) foi político e presidiu as
  províncias de Alagoas (1845-1847) e do Maranhão (1862-1863).}!

Antes disto, transbordando de cólera e patriotismo, exclamara em pleno
parlamento o imortal conselheiro Antonio Carlos\footnote{. Antonio
  Carlos Ribeiro de Andrada Machado e Silva (1773-1845), nascido em
  Santos (SP), foi juiz, desembargador no Tribunal da Relação da Bahia e
  político de grande expressão nacional, destacando-se como um dos
  deputados integrantes da Comissão de Constituição na Assembleia
  Constituinte (1823). É de se notar que o conselheiro Antonio Carlos
  era pai do seu homônimo Antonio Carlos Ribeiro de Andrada Machado e
  Silva (1830-1902), sócio de Luiz Gama por aproximadamente uma década.}:

"O abominável tráfico de africanos terá fim quando as esquadras
britânicas, com os morrões\footnote{. Mechas que se acendiam para atear
  fogo à pólvora dos canhões.} acesos, invadirem os nossos portos."

Aí estão os conceituosos escritos do dr. Tavares Bastos\footnote{.
  Refere-se a Aureliano Tavares Bastos (1839-1875), natural da antiga
  cidade de Alagoas, hoje município de Marechal Deodoro (AL), que foi
  jornalista, escritor e político. A citação aos "conceituosos escritos"
  remete, provavelmente, à edição das \emph{Cartas do Solitário} (1862),
  conjunto de artigos na imprensa que discutia diversas questões
  políticas, entre elas a abolição do tráfico de escravos.}: o vaticínio
cumpriu-se. Eis a Lei de 4 de Setembro de 1850\footnote{. A conhecida
  Lei Eusébio de Queiroz -- Lei de 4 de Setembro de 1850 -- estabelecia
  medidas, ritos e punições para reprimir o tráfico atlântico de
  escravizados.}, cuja estrita execução deve-se à ilustração,
inquebrantável energia, amplitude de vista e altos sentimentos liberais
do conselheiro Eusébio de Queiroz:\footnote{. Eusébio de Queiroz
  Coutinho Mattoso da Camara (1812-1868), nascido em São Paulo de
  Luanda, foi chefe de polícia, deputado, ministro, senador e
  conselheiro do Imperador. Como ministro da Justiça (1848-1852), foi o
  responsável pela Lei de 4 de Setembro de 1850, conhecida como Lei
  Eusébio de Queiroz, que proibiu o tráfico negreiro em caráter
  terminante.}

"Art. 1º: As embarcações brasileiras encontradas em qualquer parte, e as
estrangeiras encontradas nos portos, enseadas, ancoradouros ou mares
territoriais do Brasil, tendo a seu bordo escravos, cuja importação é
proibida pela Lei de 7 de Novembro de 1831, ou havendo-os desembarcado,
serão apreendidas pelas autoridades ou pelos navios de guerra
brasileiros e consideradas importadoras de escravos.

Aquelas que não tiverem escravos a bordo, nem os houverem proximamente
desembarcados, porém que se encontrarem com os sinais de se empregarem
no tráfico de escravos, serão igualmente apreendidas e consideradas em
tentativa de importação de escravos."\footnote{. A transcrição do art.
  1º é literal.}

Para execução desta lei, por Decreto de 14 de Outubro, do mesmo
ano\footnote{. O decreto nº 708 regulava a execução da Lei Eusébio de
  Queiroz, definindo como se dariam a repressão, o processamento e o
  julgamento dos contrabandistas.}, publicou o governo um restrito
regulamento.

\_\_\_\_\_\_\_\_\_\_\_\_\_\_\_

Reproduzi, no próprio contexto, os fundamentos da Lei de 26 de Janeiro
de 1818, da Portaria de 21 de Maio e da Lei de 7 de Novembro de 1831, do
Decreto de 12 de Abril de 1832, da Lei de 4 de Setembro de 1850; e expus
minuciosamente, guardando em tudo a verdade, aliás provada, por fatos
irrecusáveis, os atos sucessivos, atos oficiais, governamentais, dos
quais evidencia-se que a primeira das leis citadas bem como as
subsequentes, estão em seu inteiro vigor.

É princípio invariável de direito, é regra impreterível de hermenêutica,
que as "leis novas", quando são consecutivas e curam de fatos
anteriormente previstos, interpretam-se doutrinalmente por disposições
semelhantes consagradas nas "antigas".

O direito nasceu com o homem, tem a sua história, conta um passado,
revive no presente, e é essencialmente progressivo.

Na relatividade jurídica não se dão soluções de continuidade.

É da harmonia dos princípios e da indeclinável necessidade da sua
aplicação que se deduzem as relações e as formalidades do direito.

A Lei de 26 de Janeiro de 1818 estabeleceu a proibição do tráfico, a
libertação dos africanos, as penas para os importadores e outras medidas
para rigorosa observância destas, "mas, referiu-se aos africanos
provenientes das possessões portuguesas situadas ao norte do Equador."

O legislador de 1831, sem revogar aquela lei, até então propositalmente
mantida, porque não a podia revogar, e não a podia revogar porque a lei
foi decretada para execução dos Tratados de 1815 e 1817, "vigentes; e os
tratados, enquanto vigoram, por tácita convenção, constituem leis para o
mundo civilizado; estatuiu, ampliando as disposições primitivas que
foram expressamente mantidas, que ficariam livres 'todos os escravos
importados no Brasil, vindos de fora, qualquer que fosse a sua
procedência'; criou novas medidas repressivas; aumentou a penalidade; e
procurou pôr termo ao tráfico, que, na realidade, não podia ser
completamente evitado com os meios da legislação anterior, e manteve o
direito à liberdade dos escravos importados contra a proibição legal.

A unidade de vistas na propositura das medidas sociais, a filiação
lógica dos assuntos que formam a sua causa, a singularidade do objeto
ainda que sob manifestações múltiplas e a homogeneidade da consecução
dos fins, fazem com que estas duas leis -- de 1818 e 1831 --, embora
separadas pelas épocas, estejam calculadamente, para a inevitável
abolição do tráfico, na relação mecânica das duas asas, com o corpo do
condor que libra-se\footnote{. Equilibra-se, sustenta-se.} altivo nas
cumeadas\footnote{. Sucessão de cumes montanhosos.} dos Andes.

A Lei de 1831 é complementar da de 1818; a de 1850, pela mesma razão,
prende-se intimamente às anteriores; sem exclusão da primeira, refere-se
expressamente à segunda que é a causa imediata da sua existência; é,
para dizê-lo em uma só expressão técnica, relativamente às duas
anteriores: uma lei regulamentar.

\_\_\_\_\_\_\_\_\_\_\_\_\_\_\_\_\_\_

Em que artificioso direito esteiam as suas esdrúxulas opiniões, os
avaros\footnote{. O mesmo que avarentos.} defensores da bandeira negra,
para afirmar que estas leis estão revogadas?

Na revogação literal?

Dá-se esta por expressa determinação em contrário do que já foi
estatuído em lei análoga anterior.

Se alguma existe, indiquem-na.

Na revogação tácita?

Esta funda-se na falta de objeto, pois que, cessando a razão da lei,
cessa a sua disposição.

Não há no Brasil mais africanos a quem se deva restituir a liberdade?

Afirmá-lo fora insânia.

Na prepotência dos fazendeiros que dominam o eleitorado? Na do
eleitorado que seduz aos magistrados políticos? Na dos magistrados que
julgam parcialmente as causas dos correligionários e amigos? No dos
conselheiros de estado, dos senadores e deputados, que dispõe{[}m{]} da
liberdade de milhões de negros, como administradores de fazendas?

Mas isto é o cerceamento geral do Direito, é um atentado nacional, é a
precipitada escavação de um abismo, é um crime inaudito\footnote{. Sem
  precedentes.}, que só a nação poderia julgar, convertida em tribunal!

Em 1837, no Senado, teve origem um projeto de lei abolicionista,
rigoroso, no qual jeitosamente o partido da lavoura encartou esta
disposição:

"Art. 13º: Nenhuma ação poderá ser intentada em virtude da Lei de 7 de
Novembro de 1831, que fica revogada, e bem assim todas as outras em
contrário."\footnote{. A citação reforça mais uma aspecto da erudição de
  Gama: a leitura de anais parlamentares. Como essa disposição não se
  tornou texto legal, restando apenas como projeto e debate legislativo,
  Gama certamente acessou os anais empoeirados da Câmara e do Senado.
  Original e "jeitosamente" lançado em 1837, o projeto em questão saiu
  da gaveta onze anos depois, em 1848, para nova discussão no
  parlamento. É notável que Gama tenha chamado a atenção para esse ponto
  do projeto legislativo, o art. 13, que, mais de um século depois, a
  historiografia também destacaria como expressão da política da
  escravidão. Sobre o projeto, cf. Annaes do Parlamento Brasileiro,
  Camara dos Srs. Deputados, sessão de 1848, vol. 2, p. 325.}

É, portanto, evidente não só que as leis de 1818 e 1831 consideravam-se
em vigor, como que "só por disposição expressa" podiam ser alteradas ou
revogadas.

O governo inglês protestou energicamente contra a adoção deste projeto
de lei, como atentatório dos tratados existentes, e o projeto adormeceu
no Senado...\footnote{. Para a historiografia sobre o projeto de 1837,
  cf. Tâmis Parron, \emph{A política da escravidão no Império do
  Brasil}, 2011, pp. 230-236; Sidney Chalhoub, \emph{A força da
  escravidão}, 2012, pp.110-126. Waldomiro Silva Júnior, \emph{Entre a
  escrita e a prática: direito e escravidão no Brasil e em Cuba, c.
  1760-1871}, 2015, pp. 175-176.}

Em 1848, O GOVERNO LIBERAL, mais no intuito de proteger aos donos de
escravos do que de favorecer a emancipação, enviou o projeto ao Conselho
de Estado, onde habilmente o lardearam\footnote{. Perfuraram, crivaram.}
de emendas e, assim, recheado, foi entregue ao célebre orador paulista e
deputado, dr. Gabriel José Rodrigues dos Santos, que o apresentou na
Câmara temporária e, sem colher vantagem, o sustentou com o seu
peregrino\footnote{. Especial, raro.} talento.

Novos protestos da Inglaterra surgiram. A maioria que apoiava o governo,
dividiu-se. A oposição conservadora, dirigida pelo deputado Eusébio de
Queiroz, deu auxílio à fração que impugnava esse monstruoso artigo do
projeto. As discussões tomaram caráter gravíssimo e o governo, vendo a
sua causa em perigo, em perspectiva seu exício\footnote{. Estrago,
  prejuízo, ruína. Nas páginas de \emph{O Abolicionista}, de 01/04/1881,
  todavia, parte da frase foi publicada de modo equivocado, como sendo
  "em perspectiva seu exílio", ao invés de "em perspectiva seu exício",
  conforme se lê no original.}, e iminente um grande desastre político,
adiou a votação do projeto!...

Aqui, para a glória do imortal estadista conselheiro Eusébio de Queiroz,
reproduzo as palavras por ele escritas em um parecer relativamente a
esse absurdo artigo do inconsiderado projeto:

"Esse projeto foi ao ponto de extinguir todas as ações cíveis e crimes
da Lei de 7 de Novembro.

Legitimou a escravidão dos homens que essa lei proclamara
livres!!"\footnote{. Trata-se do discurso de Eusébio de Queiroz na
  Câmara dos Deputados em sessão de 16/07/1852. O parecer mencionado por
  Gama constitui um pequeno trecho do célebre discurso de 1852. A certa
  altura do acalorado debate legislativo, Queiroz lê o excerto do
  parecer a que Gama faz referência, dizendo aos seus colegas de
  parlamento que o projeto de 1837 "proclamou diretamente o que só por
  meios indiretos devera tentar, isto é, extinguiu todas as ações cíveis
  e crime da lei de 7 de Novembro (...) {[}e{]} legitimou a escravidão
  dos homens que essa lei proclamara livres!". Como se lê, a transcrição
  de Gama é próxima do literal. Cf. \emph{A escravidão no Brasil: ensaio
  histórico-jurídico-social} , parte 3, Rio de Janeiro, 1867, pp. 38-73,
  do jurisconsulto e político Agostinho Marques Perdigão Malheiro. É
  possível que Gama tenha lido o trecho citado na obra de Malheiro ou,
  como é igualmente provável, nos próprios anais da Câmara dos
  Deputados.}

À escassez dos fundamentos científicos suprem os atilados\footnote{.
  Espertos, sagazes.} defensores da criminosa escravatura com astúcia.

Estão revogadas as leis de 1818 e de 1831, exclamam eles!

São palavras do eminente jurisconsulto e máximo estadista, o exmo. sr.
conselheiro Nabuco de Araújo, externadas em um parecer do Conselho de
Estado. Foi um apreciado espírito liberal que as ditou!

Sim, senhores, venham essas prodigiosas palavras; a questão é de
princípios, é de ideias, é de direito, não é de nomes próprios; se bem
que eu aceito-a, sem receios, neste mesmo plano inclinado em que foi
posta, tenho homem por mim. Além de que a luminosa Minerva\footnote{.
  Divindade romana das artes e da sabedoria.} não é deusa tão esquiva de
quem eu não possa obter alguns raios de luz, por piedosa graça.

O nome do exmo. sr. conselheiro Nabuco, pelos altos foros conquistados
nas letras e na política, que com justiça o puseram por príncipe dos
jurisconsultos pátrios, é, no seio dos mares da jurisprudência, sempre
agitados por tormentas infinitas, tremendo e invencível escolho. Eu,
porém, honrando o nome daquele atrevido navegante\footnote{. Refere-se a
  Vasco da Gama (1469-1524), navegador português que descobriu a rota
  marítima da Europa até a Índia, conectando por mar novas vias
  comerciais entre Ocidente e Oriente.}, imortalizado pelo infeliz
poeta, e mais celebrado talvez pela coragem e ousadia, do que pela
prudência e sabedoria manifestadas em seus atos, mostrarei ao terminar
esta polêmica de máximo interesse público, e perante a ciência, que o
imenso "promontório do Conselho de Estado", onde S. Excia. fazia de
Adamastor\footnote{. Figura mitológica representada na literatura
  portuguesa como um monstro marítimo com poderes para afundar
  embarcações. Em \emph{Os Lusíadas}, Luís de Camões retratou Adamastor
  como um gigante furioso que se opôs às navegações portuguesas. Gama
  havia recorrido à figura de Adamastor em dois outros textos
  precedentes. Cf. \emph{Carta ao sr. dr. Diogo de Mendonça Pinto},
  18/08/1866; e \emph{O grande curador do mal das vinhas} (1859).}, não
é mais difícil de vencer que o dos empolados mares da Boa Esperança.

Começarei neste ponto importantíssimo da questão, por uma
retesia\footnote{. Contenda, disputa.} necessária e formal: à palavra
autorizada do exmo. sr. conselheiro Nabuco, oponho, sem o mínimo receio,
a incontestável do exmo. sr. conselheiro Eusébio de Queiroz.

Senador por senador, jurista por jurista, ilustração por ilustração,
estadista por estadista, patriota por patriota, liberal por... Neste
ponto a vantagem é minha: nos conselhos da coroa ainda não se assentou
um ministro tão altivo, tão independente e tão liberal como o
africano\footnote{. Como anotado acima, Eusébio de Queiroz nasceu em
  Luanda, Angola, no ano de 1812. O leitor certamente percebeu que não é
  sem perspicácia retórica que Gama articula o local de nascimento de
  Queiroz com as ideias de altivez, independência e liberalismo.}
Eusébio de Queiroz.

Quando o exmo. sr. conselheiro Eusébio de Queiroz confeccionou o projeto
de lei de 4 de Setembro de 1850, escreveu, para instrução dos seus
dignos colegas do ministério, uma exposição de motivos\footnote{.
  Conjunto de justificativas e diretrizes de um projeto de lei
  direcionado ao convencimento e esclarecimento dos demais ministros do
  gabinete e, também, do imperador.} que mais tarde leu na Câmara dos
srs. deputados.

Nessa exposição, S. Excia. não só condenava com muito critério o erro
imperdoável do "governo liberal" em 1848, "pretendendo escravizar
africanos livres", o que já demonstrei, como explicava com lealdade
invejável e elevada isenção de ânimo a economia da citada Lei de 1850.

Eis as suas palavras:

"Uma tal providência (alude à pretendida revogação das leis de 1818 e
1831),\footnote{. O comentário é de Gama.} que contraria de frente os
princípios de direito e justiça universal e que 'excede os limites
naturais do Poder Legislativo', não podia deixar de elevar por um lado
os escrúpulos de muitos, e por outro, provocar enérgicas reclamações do
governo inglês, que podia acreditar ou bem aparentar a crença de que
assim o Brasil iria legitimando o tráfico, não obstante a promessa de o
proibir como pirataria. Entendo, pois, que tal doutrina é insustentável
por mais de uma razão.

"Um único meio assim resta para reprimir o tráfico sem faltar às duas
condições acima declaradas (impedir a importação e manumitir-se os
importados),\footnote{. O comentário é de Gama.} e é deixar que a
respeito do passado continue, 'sem a menor alteração, a legislação
existente, que ela' continue igualmente a respeito dos pretos
introduzidos para o futuro, mas que só se apreenderem depois de
internados pelo país e de não pertencerem mais aos introdutores. Assim,
consegue-se o fim, se não perfeitamente, ao menos quanto é possível."

....................

"Os filantropos não terão que dizer, vendo que para as novas introduções
se apresentam alterações eficazmente repressivas, e que, 'para o
passado', não se fazem favores, 'e apenas continua o que está.'

....................

"Por isso entreguei não só a formação da culpa, como todo processo, ao
juízo especial dos auditores da Marinha (juízes de direito), com recurso
para a Relação. 'Bem entendido, só nos casos de apreensão no ato de
introduzir ou sobre o mar.'"\footnote{. A transcrição é literal,
  ressalvados os comentários internos de Gama. Por sua vez, a exposição
  de motivos integra o célebre discurso de Eusébio de Queiroz na sessão
  da Câmara dos Deputados de 16/07/1852. Cf. Agostinho Marques Perdigão
  Malheiro, \emph{A escravidão no Brasil: ensaio
  histórico-jurídico-social}, parte 3, Rio de Janeiro, 1867, pp. 58-59.
  É possível que Gama tenha lido o trecho citado na obra de Malheiro ou,
  como é igualmente provável, nos próprios anais da Câmara dos
  Deputados.}

A Lei de 1850 confirma perfeitamente esta exposição!

\_\_\_\_\_\_\_\_\_\_\_

Qual é, porém, o pensamento do Conselho de Estado a este respeito,
pensamento "libérrimo"\footnote{. Superlativo de livre, algo como
  muitíssimo livre, muitíssimo liberal.}, sustentado pelo exmo. sr.
conselheiro Nabuco de Araújo em um parecer, e por eméritos deputados e
senadores da atual maioria parlamentar?

Ei-lo em suas conclusões:

1º: A Auditoria de Marinha é a autoridade competente para conhecer dos
fatos relativos à importação ilegal de escravos no Brasil; nessa
jurisdição "excepcional" estão compreendidos "todos os escravos
provenientes do tráfico"!...

2º: "Não há outra jurisdição" para julgar a liberdade dos escravos
provenientes do tráfico senão a Auditoria de Marinha!...

3º: É preciso constar o "desembarque, verificar a importância e tráfico"
para que os escravos provenientes sejam havidos por livres!...

4º: E como à Auditoria compete a verificação do tráfico, à ela compete o
julgamento da liberdade dos escravos importados por esse meio!...

É inexato, injurídico, impolítico e improcedente e político o primeiro
ponto das conclusões:

-- É inexato porque não tem base objetiva nos fatos constitutivos da
materialidade da lei, e contraria, de plano, na parte subjetiva, a sua
claríssima disposição;

-- É injurídico porque, contando a lei, além do princípio geral, "uma
exceção", foi esta exceção, com exclusão prejudicial do princípio geral,
elevada à categoria de regra;

-- É impolítico porque, sendo a autoridade e a competência, em assunto
de atribuições, instituídas por lei, e por prevista utilidade pública,
impossível é admitir a existência da primeira sem limitação, nem a da
segunda sem prescrições expressas;

-- É improcedente porque em sentido diametralmente oposto estatui a lei:

Todos os apresamentos de embarcações de que tratam os arts. 1º e 2º,
assim como a liberdade dos escravos "apreendidos no alto mar ou na
costa, antes do desembarque, no ato dele, ou imediatamente depois, em
armazéns e depósitos sitos nas costas e portos, serão processados e
julgados em 1ª instância pela Auditoria da Marinha, e em 2ª
{[}instância{]} pelo Conselho de Estado.

Trata aqui a lei das apreensões realizadas no alto mar, nas costas,
antes dos desembarques, no ato deles, ou imediatamente depois, em
armazéns, depósitos sitos nas costas e portos; não se refere de maneira
alguma aos escravos que, escapando às vistas e à vigilância da Auditoria
da Marinha, se internarem no país, e menos ainda aos vindos
anteriormente; tanto a uns como a outros, "são aplicáveis", como afirmou
o exmo. sr. conselheiro Euzebio, "as disposições da legislação
anterior." A Lei de 1850 cura "exclusivamente dos casos de importação."

É inexato o segundo artigo das conclusões do Parecer do Conselho de
Estado: nem os auditores da Marinha têm competência, fora das hipóteses,
"por exceção", previstas na Lei de 1850, nem a legislação anterior foi
revogada.

Para essas hipóteses especiais rege a Lei de 1850; para as gerais,
quanto aos princípios, as leis de 1818 e 1831; e, quanto às competências
e forma do processo, o Decreto de 12 de Abril de 1832, artigo{[}s{]} 9º
e 10º.

É inexato o terceiro artigo, é despido de conceito jurídico, e até
absurdo; para refutá-lo basta um fato; o fato não constitui uma
maravilha, nem é novo.

-- Dá-se um desembarque de africanos em um dos pontos da costa.

O capitão do navio, pressentindo o movimento seguro, perigoso, iminente,
da autoridade, foge com todos os seus comparsas e abandona os negros em
terra, sem deixar vestígio que o malsine\footnote{. Denuncie.}.

A autoridade apreende os negros, mas não consegue descobrir quem os
conduziu, quando, nem em que navio!

O que faz dos pretos? Vende-os?

Leva-os para si?

Supõem-os caídos do céu por descuido?

Ou manda "constatar" que eles emergiram do solo como tanajuras em verão?

É finalmente inexato o quarto artigo das conclusões.

A decretação de alforria, em regra, compete aos juízes do cível; por
exceção, por desclassificação, estatuída por utilidade pública,
tratando-se de africanos importados depois da proibição do tráfico,
incumbe aos juízes do cível ou aos criminais, "mediante processo
administrativo."

Quando o exmo. sr. conselheiro Nabuco de Araújo era presidente da
heroica província de S. Paulo e avultava entre os chefes prestigiosos do
Partido Conservador, tinha ideias liberalíssimas relativamente aos
africanos escravizados de modo ilícito.

Os agentes policiais\footnote{. Até 1854, e isso nos parece revelador
  para o exemplo que ele passará a narrar, Gama serviu como oficial da
  Força Pública, espécie de agente policial subordinado, em última
  instância, ao presidente da província.}, no município desta cidade,
por diversas vezes apreenderam como escravos fugidos pretos que depois
se verificou serem africanos boçais.

O exmo. sr. conselheiro Furtado de Mendonça, jurisconsulto muito
esclarecido, que exemplarmente exercia a Delegacia de Polícia da
capital, depois das diligências legais, os declarou livres. Estes atos
foram aprovados com louvor pelo exmo. sr. conselheiro Nabuco de Araújo.

Mais tarde, quando S. Excia. era ministro da Justiça\footnote{. Isto é,
  José Thomaz Nabuco de Araújo Filho (1813-1878).}, e mais amadurecido
tinha os frutos da sua numerosa ilustração, acercando de todos os
"andorinhões" políticos e dos "zangões"\footnote{. A expressão, popular
  à época, indicava um indivíduo que vive às custas de outra pessoa,
  explorando de forma constante benefícios e favores alheios.} da
lavoura, que o aturdiam de contínuo, deu-se o seguinte curioso fato, que
bem prova a influência, o predomínio dos "senhores" na política e
governação do Estado.

Foi em 1853 ou 1854, o que não posso agora precisar, por estrago de
notas.

Aconteceu que, em um daqueles anos, viesse à capital certo fazendeiro do
interior, cujo nome devo ocultar, trazendo cartas valiosas de
prestigiosos chefes políticos e, perante as autoridades superiores,
envidasse\footnote{. Empenhasse, empregasse.} esforços para reaver dois
escravos africanos, boçais, que lhe haviam fugido, e que, aprendidos por
um inspetor de quarteirão do bairro suburbano da Água Branca, tinham
sido declarados livres e, como tais, com outros, postos ao serviço do
Jardim Botânico por ordem da Presidência.

Nada aqui podendo conseguir, armou-se de novas recomendações e foi-se
caminho da Corte.

Mês e meio depois, o presidente da província recebeu um "Aviso
confidencial", firmado pelo ministro da Justiça, no qual lia-se o
seguinte:

"Os pretos F... e F..., postos ao serviço do Jardim Público dessa
cidade, escravos fugitivos do fazendeiro B***, residente em A***, foram
muito bem apreendidos e declarados livres pelo delegado de polícia como
africanos ilegalmente importados no império.

Cumpre, porém, considerar que esse fato, nas atuais circunstâncias do
país, é de grande perigo e gravidade; põe em sobressalto os lavradores,
pode acarretar o abalo dos seus créditos e vir a ser causa, pela sua
reprodução, de incalculáveis prejuízos e abalo da ordem pública.

A lei foi estritamente cumprida; há, porém, grandes interesses de ordem
superior que não podem ser olvidados e que devem de preferência ser
considerados.

Se esses pretos desaparecessem do estabelecimento em que se acham, sem o
menor prejuízo do bom conceito das autoridades e sem a sua
responsabilidade, que mal daí resultará?"

.....................

Quinze dias depois, o sr. diretor do Jardim participou à Presidência o
desaparecimento dos dois africanos.

A Presidência imediatamente ordenou ao chefe de polícia as diligências
precisas para descobrimento dos "fugitivos". Foram inquiridos outros
africanos: disseram que à noite entraram soldados na senzala do Jardim,
prenderam, amarraram e levaram os dois pretos.

Não foram descobertos os soldados nem os pretos: e neste ponto ficou o
mistério.

Aquele invocado "Parecer" do Conselho de Estado, como claramente vê-se,
e o "Aviso-confidencial" que acabo de referir, foram escritos com penas
de uma só asa, são formas de um só pensamento, representam um só
interesse: sua origem é o terror, seus meios a violência, seu fim a
negação direito. Os fatos têm a sua lógica infalível.

É a prova inconcussa\footnote{. Inabalável, irrefutável.} de um mau
estado, é uma evolução lúgubre\footnote{. Sinistra, macabra.} da nossa
sociedade, uma das faces mórbidas da sinistra política do medo que a
sobrepuja; é a mancha negra que desde 1837 assinala indelével a bandeira
do Partido Liberal.

O exmo. sr. conselheiro Nabuco, que soube ser homem do seu tempo,
consagrou-se inteiramente às exigências do seu partido; morreu na
firmeza de suas crenças; têm ambos a mesma história. E o futuro, quando
julgá-lo, sobre a lápide do seu túmulo, fazendo justiça ao seu caráter,
perante a imagem da pátria, há de sagrá-lo herói.

S. Paulo, 7 de dezembro de 1880.

LUIZ GAMA.

\textbf{29. LIBERTAÇÃO DE ESCRAVOS PELO FUNDO DE EMANCIPAÇÃO}\footnote{.
  In: \emph{Gazeta do Povo} (SP), Publicações Pedidas, 21/12/1880, p. 2.}

\textbf{*didascália*}

\emph{Gama recebeu uma carta procedente de Taubaté, vale do Paraíba,
interior de São Paulo. Consultavam-no sobre "a classificação dos
escravos que devem ser alforriados pelo fundo de emancipação". Há,
portanto, uma pergunta. O que Gama formula, por sua vez, é um parecer
jurídico. Este, ganhando as páginas dos jornais, tornava-se muito mais
do que uma opinião doutrinária direcionada a um caso ou jurisdição
particular -- Taubaté --, uma vez que assumia a dimensão de uma lição de
direito digna de um catedrático da matéria. Apresenta-se na tribuna da
imprensa, como de costume, "parcialmente prevenido por a grita dos
senhores" e igualmente ciente "para quanto presta o patronato nas
povoações do interior, principalmente quando o despertam a política, as
relações de amizade, e os interesses de família". Gama ponderava as
condições e circunstâncias políticas -- relações de família, amizades e
interesses -- entranhadas num processo dessa natureza jurídica. Feitas
tais ressalvas, Gama argumentava sobre as condições, modos e ordens de
preferência na libertação de escravizados pelo mecanismo do fundo de
emancipação. O raciocínio interpreta a legislação então vigente e a
articula com o conhecimento normativo proveniente da civilística
portuguesa e alemã. Certo de seu objetivo pragmático de solucionar uma
demanda concreta, Gama fazia lembrar que aquela peça se destinava aos
poucos abolicionistas que ocupavam os tribunais, além mesmo dos algozes
-- senhores e juízes --, que embargavam o passo da emancipação. Era,
então, uma peça normativo-pragmática inserida no novo patamar da
campanha abolicionista da década de 1880. Em suas palavras: "Dou estas
linhas humildes como aviso e minguado auxílio a alguns filantropos,
protetores espontâneos de infelizes, que lutam contra embaraços, entre
os quais desgraçadamente avulta a má vontade de certos juízes, propensos
à tortura e à escravidão".}

\emph{***}

Escrevem-me de Taubaté\footnote{. Cidade localizada no Vale do Paraíba
  (SP), distante 130 km de São Paulo, foi a primeira comarca da região.}
que a classificação dos escravos que devem ser alforriados pelo fundo de
emancipação\footnote{. Foi um mecanismo de captação de recursos
  instituído pela Lei de 1871 para promoção de alforrias de escravos,
  através do recolhimento de multas, impostos, cotas e outras verbas
  orçamentárias.} está sendo feita ali de modo irregularíssimo,
arbitrariamente, com ofensa manifesta da disposição da lei, e grave
prejuízo dos direitos dos libertandos.

Não sei quem são os dignos membros da junta classificadora; não os
conheço, não tenho para com eles ódio nem prevenções.

Escrevo estas linhas para evitar preterições, desgostos, e quem sabe se
até desastres.

Eu sei para quanto presta o patronato nas povoações do interior,
principalmente quando o despertam a política, as relações de amizade, e
os interesses de família.

Meu fim é chamar para os fatos que começam de produzir clamores as
atenções do governo, se bem que já parcialmente prevenido por a grita
dos senhores.

A classificação legal é esta:

I. Famílias;

II. Indivíduos.

§ 1º: Na libertação por famílias preferirão:

I. Os cônjuges que forem escravos de diferentes senhores;\\
II. O cônjuge que for casado com livre (Av{[}iso{]} nº 4 {[}sic{]} de 19
de Setembro de 1873);\\
III. Os cônjuges que tiverem filhos nascidos livres em virtude da lei, e
menores de 8 anos;\\
IV. Os cônjuges que tiverem filhos livres menores de 21 anos;\\
V. Os cônjuges com filhos menores escravos;\\
VI. As mães com filhos menores escravos;\\
VII. Os cônjuges sem filhos menores.

§ 2º: Na libertação por indivíduos preferirão:

I. A mãe ou pai com filhos livres;\\
II. Os de 12 a 50 anos de idade, começando pelos mais moços no sexo
feminino, e pelos mais velhos no sexo masculino.

Na ordem da emancipação das famílias e dos indivíduos serão preferidos:

1º: Os que por si ou por outrem entrarem com certa quota para sua
libertação.

2º: Os mais morigerados\footnote{. Bem-comportados.}, a juízo dos
senhores. Em igualdade de condições a sorte decidirá (Decreto
Regulamentar nº 5.135 de 13 de Novembro de 1872, art. 27).\footnote{.
  Gama transcreveu o art. 27 do decreto de 13/11/1872, porém, habilmente
  enxertou uma disposição normativa diversa que não compunha o texto do
  decreto, mas constituía uma das hipóteses de preferência do fundo de
  emancipação. Cf. Aviso nº 335, de 19/09/1873, do Ministério da
  Agricultura, Comércio e Obras Públicas, em que declarava que, sendo um
  dos cônjuges escravizado, "deve este ser classificado de preferência
  na ordem das famílias e não de indivíduos".}

Esta classificação deve compreender somente os escravos que possam ser
libertados pela quota do fundo de emancipação distribuída ao município;
e não a todos, como se fazia, em observância da disposição do art. 41 do
Regulamento nº 5.135, de 13 de Novembro de 1872 (Vid. Decreto nº 6.341,
de 20 de Setembro de 1876, art. 2º).\footnote{. Respectivamente, art.
  41. ``A verificação do valor dos escravos, por algum dos meios
  precedentes, deverá estar concluída até 31 de Dezembro de cada ano e
  compreenderá tantos escravos classificados quantos possam ser
  libertados pela importância do fundo de emancipação''. O art. 2º do
  decreto de 20/09/1876, por sua vez, disciplinava que: ``A
  classificação para as alforrias compreenderá somente aqueles escravos
  que possam ser libertados com a importância da quota distribuída ao
  município''.}

Se, na classificação, houver deficiência nas declarações quanto à ordem
das preferências, dela cabe recurso para o juízo dos órfãos (Regulamento
nº 5.135 de 13 de Novembro de 1872, art. 34; Avisos de 4 de Março e 8 de
Julho de 1876).\footnote{. Respectivamente, art. 34. ``Perante o juiz de
  órfãos deverão os interessados apresentar suas reclamações dentro do
  prazo de um mês, depois de concluídos os trabalhos da junta. As
  reclamações versarão somente sobre a ordem de preferência ou
  preterição na classificação. § Único: Se houver reclamações, o juiz de
  órfãos as decidirá dentro do prazo de 15 dias''. O aviso nº 108, de
  04/03/1876, do Ministério da Agricultura, Comércio e Obras Públicas,
  instruía como proceder na emancipação de escravizados, em especial,
  quanto à liberdade daqueles relacionados nas listas organizadas pelas
  juntas classificadoras dos fundos de emancipação. O aviso nº 393, de
  08/07/1876, também do Ministério da Agricultura, Comércio e Obras
  Públicas, estabelecia regras para a classificação e ordem de
  preferências entre escravizados a serem emancipados via fundo de
  emancipação.}

Este recurso deve ser intentado dentro do prazo de um mês, depois de
concluídos os trabalhos da junta respectiva, antes do ato complementar
do arbitramento (Regulamento citado, arts. 34, 35 e 37).\footnote{.
  Sobre o art. 34, ver nota acima. Art. 35. ``Não havendo reclamações,
  ou decididas estas pelo juiz de órfãos, considerar-se-á concluída a
  classificação''. Art. 37. ``Concluída a classificação do modo acima
  prescrito, o coletor, ou o empregado fiscal de que fala o art. 28,
  promoverá -- nas comarcas gerais, ante o juízo municipal, salva a
  alçada para o julgamento final, e nas comarcas especiais, ante o juízo
  de direito -- o arbitramento da indenização, se esta não houver sido
  declarada pelo senhor, ou, se declarada, não houver sido julgada
  razoável pelo mesmo agente fiscal; ou se~não houver avaliação judicial
  que o dispense''.}

É certo, portanto, que, sem provocação de parte, não pode o juiz,
\emph{ex-officio}, alterar a ordem da preferência. Pode, porém, depois
de esgotado aquele prazo, por exceção, admitir reclamação, nos casos de
força maior ou justo impedimento, que o nosso direito admite, uma vez
que o recurso ou reclamação, embora fora do prazo, seja interposto antes
de começado o processo administrativo (Decreto nº 4.835, de 1º de
Dezembro de 1871, art. 19; cons. cons. est. - secc. just. - 26 de Julho
de 1876, \emph{in fine}; Aviso de 14 de Novembro dito).\footnote{. Para
  execução do art. 8º da Lei do Ventre Livre, o decreto definia o
  regulamento para a matrícula especial dos escravizados e dos filhos da
  mulher escravizada. O art. 19, a seu turno, determinava que os
  "escravos que, por culpa ou omissão dos interessados não forem dados à
  matrícula até o dia 30 de Setembro de 1873, serão por este fato
  considerados libertos, salvo aos mesmos interessados o meio de
  provarem, em ação ordinária, com citação e audiência dos libertos e de
  seus curadores".}

Este recurso ou reclamação, segundo os preceitos legais e regras de
direito, pode ser intentado por o escravo pessoalmente, ou por qualquer
pessoa (\emph{assertor}), \emph{ainda repugnando ele} (Vid.
Hein{[}eccius{]}, § 155 e seguintes; Borges Carneiro, Direito Civil,
Livro I, Título III, § 32, números 1, 2 e 3).\footnote{. Em Borges
  Carneiro, \emph{Direito Civil}, Livro 1º, Título 3º, § 32, se encontra
  uma base doutrinária que se tornou bastante relevante para o
  conhecimento normativo de Gama. Esta seção cuida do tema do "favor da
  liberdade" e se constitui de cinco ideias centrais, sendo três delas
  citadas expressamente por Gama nesse parágrafo. Descontadas citações
  internas e referências externas, são elas: 1º. "Todo o homem se
  presume livre; a quem requer contra a liberdade incumbe a necessidade
  de provar"; 2º. \emph{"}Quando se questiona se alguém é livre ou
  escravo, esta ação ou exceção goza de muitos privilégios concedidos em
  favor da liberdade". 3º. "A favor do pretendido escravo não só pode
  requerer ele mesmo, mas qualquer pessoa (\emph{assertor}), ainda
  repugnando ele". Cf. Borges Carneiro, \emph{Direito Civil}, Livro 1º,
  Título 3º, § 32, pp. 96-97.}

Dou estas linhas humildes como aviso e minguado auxílio a alguns
filantropos, protetores espontâneos de infelizes, que lutam contra
embaraços, entre os quais desgraçadamente avulta a má vontade de certos
juízes, propensos à tortura e à escravidão.

S. Paulo, 20 de Dezembro de 1880.

LUIZ GAMA.

\textbf{30. CONTRAPROTESTO}\footnote{. In: \emph{A Província de S.
  Paulo} (SP), Seção Particular, Itatiba, 04/01/1881, p. 2.}

\textbf{*didascália*}

\emph{Gama veio a público protestar sobre a potencial reescravização
judicial de alforriados no juízo de Itatiba, interior de São Paulo.
Replicado algumas vezes na imprensa, o texto visava, sem dúvidas,
alertar o advogado do senhor de escravizados, assim como o juiz local,
de que ele não só estava muito bem informado dos interesses de
reescravização daquele que já tinham recebido alforria, como
protestaria, contestaria e agiria nos termos da lei para "fazer com que
as alforrias sejam mantidas; porque são regulares e irrevogáveis". }

\emph{***}

O emérito advogado, sr. dr. Pinheiro Lima, meu particular e distinto
amigo, levantou protesto contra as alforrias concedidas pelo sr. João
Elias de Godoy Moreira a escravos da sua exclusiva propriedade, sob o
motivo, aliás de todo o ponto improcedente, de tais alforrias serem
concedidas \emph{em} \emph{fraude de credores}.

É meu intuito, em face do direito e da jurisprudência, fazer com que as
alforrias sejam mantidas; porque são regulares e irrevogáveis: nós temos
lei.

Na hipótese emergente não se dá prejuízo, nem fraude contra credores;
nem cabimento algum têm, contra as liberdades bem adquiridas, as ficções
e sutilezas do direito romano, não menos bárbaro, que mal invocado entre
nós.

3-1\footnote{. Além do indicado, o texto foi replicado pelo menos mais
  quatro vez no mês de janeiro de 1881.}

LUIZ GAMA.

\textbf{31. QUESTÃO JURÍDICA -- O escravo que requer e é admitido a
manumitir-se, por indenização do seu valor, se o preço arbitrado
judicialmente excede ao pecúlio, continua cativo, por deficiência
deste?}\footnote{. In: \emph{Gazeta de S. Paulo} (SP), Ineditoriais,
  13/01/1881, p. 3. Através de uma republicação sem título desse artigo,
  Fonseca resolveu por conta própria dar outro título a esse texto.
  Chamando-se \emph{Questão Jurídica}, sendo publicado tão somente
  algumas semanas depois do homônimo famoso, é possível compreendê-lo
  como \emph{Questão Jurídica -- parte II}. Não por outro motivo, ao que
  parece, \emph{O Abolicionista} (RJ) de 1º de Julho de 1881 o
  republicou imediatamente na sequência do primeiro \emph{Questão
  jurídica}. Ambas as partes, por exemplo, respondem a uma pergunta
  técnica e desenvolvem uma linha de raciocínio jurídico amparada na
  multinormatividade da matéria em questão.}

\textbf{*didascália*}

\emph{A resposta à pergunta que intitula o artigo estava na ponta da
língua do jurista: "Não. Deve o magistrado decretar a sua alforria, nos
termos de direito". Na ponta língua certamente porque pensada, estudada,
refletida e amadurecida pela original experiência que Gama possuía na
produção normativa de liberdade em tempos de escravidão. Nesse estudo
doutrinário, Gama conceitua a formação do pecúlio e discute o direito de
o escravizado demandar liberdade. Por uma interpretação que conectava
diferentes normatividades e temporalidades, haja vista as citações ao
direito romano, português e brasileiro, "conclui-se filosoficamente, com
as regras de boa hermenêutica", qual o modo dos juízes decretarem a
liberdade, ainda que o pecúlio estivesse abaixo do valor arbitrado. E
arrematava a solução normativa adequada: "Deve o juiz decretar a
liberdade do escravo, obrigando a completar o preço em moeda pelos meios
regulares, ou ao pagamento, em serviços, por contrato, lavrado no juízo
dos órfãos na forma da lei; porque 'no conflito de um interesse
pecuniário e da liberdade, prevalece esta'". }

\emph{***}

RESPONDO:

Não. Deve o magistrado decretar a sua alforria, nos termos de direito.

\_\_\_\_\_

Ao escravo é permitida a formação de pecúlio\footnote{. Patrimônio,
  quantia em dinheiro que, por lei (1871), foi permitido ao escravizado
  constituir a partir de doações, legados, heranças e diárias
  eventualmente remuneradas.}, que se poderá constituir por meio de
doações, legados, heranças, e do próprio trabalho e economias, com
permissão do senhor, só neste último caso (Lei nº 2.040 de 28 de
Setembro de 1871, art. 4º; Decre{[}to{]} Reg{[}ulamentar{]} nº 5.135 de
13 de Novembro de 1872, art. 48).\footnote{. As redações do art. 4º da
  Lei de 1871 e do art. 48 do decreto regulamentar são idênticas. Cf.
  ``É permitido ao escravo a formação de um pecúlio com o que lhe
  provier de doações, legados e heranças, e com o que, por consentimento
  do senhor, obtiver do seu trabalho e economias. O Governo
  providenciará nos regulamentos sobre a colocação e segurança do mesmo
  pecúlio''.}

Se o senhor convencionar com o escravo, "ainda que pertença à
condôminos\footnote{. Indivíduo que com outro ou outros exerce o direito
  de propriedade sobre um bem não dividido.}", a concessão de alforria,
ficando {[}fixando{]}, desde logo, o preço, poderá ir recebendo o
pecúlio, em prestações, à proporção que for sendo adquirido, com o juro
de seis por cento, como pagamento parcial (Decr{[}eto{]}
Reg{[}ulamentar{]}, cit{[}ado{]}, art. 49, § ún{[}ico{]}, nº
1).\footnote{. A ressalva destacada entre aspas condiz expressamente com
  a segunda parte do parágrafo único do art. 49. A construção interna do
  parágrafo, todavia, oferece uma interpretação autoral de Gama, na
  medida em que discute o direito à formação do pecúlio como uma
  convenção entre senhor e escravizado -- verbo ou ação de que a lei não
  fazia menção. A simples operação, por sua vez, implicaria no
  reconhecimento de uma relação jurídica com sujeitos capazes de
  contratar ou convencionar. Logo, senhores e escravizados equiparados
  numa relação bilateral. Embora reforçar a capacidade jurídica de seus
  representados possa afigurar-se como um expediente lateral, isso fazia
  parte da estratégia jurídica de Gama e, em muitas causas de liberdade,
  levou a um desfecho vitorioso para suas demandas.}

Este pecúlio, "em quanto inferior seja ao valor razoável do escravo",
dada transferência de domínio, passará às mãos do novo senhor, ou terá
qualquer dos destinos mencionados no art. 49 (Decr. Reg. cit. art.
51).\footnote{. A transcrição do art. 51 do decreto de 13/11/1872
  confere com o original, exceto pela expressão destacada em aspas que
  Gama inseriu no corpo do texto.}

Havendo impossibilidade de arrecadar-se o pecúlio do poder do senhor, "o
escravo tem o direito à alforria", mediante indenização do resto do seu
valor, em dinheiro ou "em serviços", por prazo que não exceda de 7 anos;
"o preço" poderá ser fixado por arbitramento, se não existir avaliação
judicial, que deverá prevalecer (Decr. Reg. cit. art. 52).\footnote{. A
  interpretação está alinhada com o teor -- e mesmo com expressões
  exatas -- do texto normativo.}

O escravo que, por meio do seu pecúlio, puder indenizar o seu valor,
"tem direito à alforria" (Lei nº 2040 cit., art. 4, § 2º; Decr. Reg.
cit. art. 56).\footnote{. A interpretação está alinhada com o teor -- e
  mesmo com expressões exatas -- do texto normativo. Cf. Art. 56. ``O
  escravo que, por meio de seu pecúlio, puder indenizar o seu valor, tem
  direito à alforria. § 1º. Em quaisquer autos judiciais, existindo
  avaliação e correspondendo a esta a soma do pecúlio, será a mesma
  avaliação o preço da indenização (...) para ser decretada
  \emph{ex-officio} a alforria. § 2º. Em falta de avaliação judicial ou
  de acordo sobre o preço, será este fixado por arbitramento''.}

O "direito à liberdade", uma vez adquirido, nos termos da lei,
exercita-se, por petição do escravo, no juízo comum competente,
acompanhada de exibição de "pecúlio suficiente à juízo do Magistrado"
(Decr. Reg. cit., arts. 56, 57, 84 e 86).\footnote{. Após ter destacado,
  nos dois parágrafos precedentes, a ideia de que "o escravo tem direito
  à alforria", realce que habilmente fazia sem enfatizar as condições e
  os termos restritivos da letra da lei, Gama destacava agora, sem
  expressão legal nessa direção, a ideia de "direito à liberdade". A
  construção do argumento, como se nota, não era apressada. Gama
  passava, portanto, de um "direito à alforria", emparedado por uma
  série de condicionantes, para um "direito à liberdade" que valeria já
  com a "convenção" entre senhor e libertando. Para os textos normativos
  citados, cf. respectivamente. Art. 56, ver nota acima. Art. 57. ``Não
  poderá requerer arbitramento, para execução do art. 4º, § 2º da lei, o
  escravo que não exibir, no mesmo ato em juízo, dinheiro ou títulos de
  pecúlio, cuja soma equivalha ao seu preço razoável. § 1º. Não é
  permitida a liberalidade de terceiro para a alforria, exceto como
  elemento para a constituição do pecúlio; e só por meio deste e por
  iniciativa do escravo será admitido o exercício do direito à alforria,
  nos termos do art. 4º, § 2º da lei. § 2º. Prevalecem na libertação,
  por meio do pecúlio, as regras estatuídas no § único do art. 44,
  quanto à entrega do preço do escravo alforriado''. Art. 84. ``Para a
  alforria por indenização do valor, para a remissão, é suficiente uma
  petição, na qual, exposta a intenção do peticionário, será solicitada
  a vênia para a citação do senhor do escravo ou do possuidor do
  liberto. Antes da citação o juiz convidará o senhor para um acordo, e
  só em falta deste prosseguirá nos termos ulteriores. § 1º. Se houver
  necessidade de curador, precederá à citação nomeação do mesmo curador,
  em conformidade das disposições deste regulamento. § 2º. Feita a
  citação, as partes serão admitidas a louvarem-se em arbitradores, se
  houver necessidade de arbitramento; e o juiz prosseguirá nos termos
  dos arts. 39, 40 e 58 deste regulamento, decretando ao final o valor
  ou o preço da indenização e, paga esta, expedirá a carta de alforria
  ou o título de remissão. § 3º. Se a alforria for adquirida por
  contrato de serviços, esta circunstância será mencionada na carta; e,
  no caso de ulterior remissão, não se passará título especial, mas
  bastará averbá-la na mesma carta''. Art. 86. ``O valor da indenização
  para alforria, ou para a remissão, regulará a competência para o
  simples preparo ou para o preparo e julgamento, em conformidade da lei
  nº 2.033 de 20 de Setembro de 1871. Assim, o valor do escravo no caso
  de abandono''.}

ASSIM:

-- Considerando a ilegitimidade da escravidão, "que é contrária à
natureza (L. 4, § 1º, Dig. Stat. Hom.; - Instit. Justi., § 2º, de jur.
person; Ord. Liv 4º Tit. 42, V; visto como, por direito natural, todos
nascem livres, todos são iguais" Inst. Just. pr. de libertin. I, 5; -
Ulp. L. 4 Dig. de Just. A jur. I, 1.; - Alv. 30 Julho 1699; que nada é
mais digno de favor do que a liberdade (Gayo L. 122 Dig. de reg. jur. L.
17); pelo que, em benefício dela, muitas cousas se determinam "contra o
rigor do direito" (L. 24, § 10, Dig. de fideic. libertat.; - Inst., §
4º, de donat; - Ord. Liv. 4, Tit. 11, § 4º); e que são mais fortes, e de
maior consideração as razões que concorrem a seu favor, do que as que
podem fazer justo o cativeiro (Lei de 1º de Abril de 1680).

-- Considerando que o favor da liberdade, por a razão de direito,
exprime a ideia mais benigna (L. 32, § fin. Dig. ad. Leg. Falcid; que,
no que for obscuro, se deve favorecer a liberdade (Paul. L. 179 Dig.); e
que, no caso de dúvida, e de interpretação, deve decidir-se a favor da
liberdade (Pompon. L. 20 de reg. jur).

Acrescentadas às disposições da legislação pátria, que ficam citadas, as
do Decreto e Regulamento nº 5.135 de 13 de Novembro de 1872, arts. 61 e
62, e harmonizadas todas com os princípios aceitos e
inconcussos\footnote{. Estabelecidos, firmados.} do direito
manumissório, conclui-se filosoficamente, com as regras de boa
hermenêutica, que:\footnote{. Após subsidiar seu argumento com textos
  normativos da mais recente e reconhecida produção legislativa, como a
  Lei de 1871 e seu respectivo decreto regulamentar, Gama buscou outro
  corpo textual, muitos deles oriundos do direito romano recepcionado
  pela civilística portuguesa. Contudo, antes de concluir, Gama amarra o
  conhecimento normativo que embasa seu argumento tornando às "leis
  pátrias" do "direito manumissório". Cf. Art. 61. ``É permitido ao
  escravo, em favor de sua liberdade, contratar com terceiro a prestação
  de futuros serviços, por tempo que não exceda de sete anos, mediante o
  consentimento do senhor e aprovação do juiz de órfãos''. Art. 62. ``O
  escravo que pertencer a condôminos, e for libertado por um destes,
  terá direito à sua alforria, indenizando os outros senhores da quota
  do valor que lhes pertencer. Esta indenização poderá ser paga em
  serviços prestados por prazo não maior de sete anos, em conformidade
  do artigo antecedente''.}

-- ~Dada a hipótese de um escravo requerer alforria, mediante
indenização, por pecúlio; de admitido ser, no juízo, por equivaler o
pecúlio "razoavelmente", ao seu valor; de não existir avaliação
judicial; de não querer aceitar o senhor o preço exibido, e, por isso,
ser caso de arbitramento; de, verificado o arbitramento, tornar-se o
pecúlio insuficiente por excedê-lo o valor arbitrado; sendo certo que "o
direito à liberdade", uma vez adquirido, torna-se perpétuo (Perdig. Mal.
Secc. 4, § 127, nº 10, not. 714 e 715, Vol.I);\footnote{. A citação
  confere com a primeira edição de \emph{A escravidão no Brasil: ensaio
  histórico-jurídico social} (1866-1867), obra de Agostinho Marques
  Perdigão Malheiro (1824-1881), natural de Campanha (MG), historiador,
  deputado e advogado que presidiu o Instituto da Ordem dos Advogados
  Brasileiros (1861-1866).}\\
-- ~Deve o juiz decretar a liberdade do escravo, obrigando a completar o
preço em moeda pelos meios regulares, ou ao pagamento, em serviços, por
contrato, lavrado no juízo dos órfãos na forma da lei; porque "no
conflito de um interesse pecuniário e da liberdade, prevalece esta"
(Inst. Just. § 1º de eo fui libertat. caus. III. 12 - sciant commodo
pecuniario praferendum esse libertatis causam).

\begin{itemize}
\item
  S. Paulo, 12 de Junho\footnote{. Por evidente erro tipográfico, uma
    vez que a publicação é de 13/01/1881, o mês indicado não corresponde
    ao mês da escrita. É possível cravar que a data da escrita seja o
    dia 12/01/1881.} de 1881.
\item
\item
  LUIZ GAMA.
\item
\end{itemize}

\textbf{O COCHEIRO E O CÔNSUL}

\textbf{*didascália*}

\emph{Nesses três textos, Gama defende extrajudicialmente dois clientes:
o cocheiro José Lopes de Lima e o vice-cônsul de Portugal em São Paulo,
Félix de Abreu Coutinho. Se a relação advogado-cliente com Lima é
constituída nos autos, não se pode dizer o mesmo com o agente consular.
De todo modo, Gama defende com veemência tanto o cocheiro Lima quanto o
cônsul Coutinho, ambos ofendidos e achincalhados largamente na imprensa.
A defesa de Lima é memorável. Entre declarações de peso, como a de que
nunca possuiu escravizados, Gama defendia o cocheiro negro pela moral e
pelo direito: "Tenho consciência de mim. Sei quando defendo um criminoso
e quando proclamo a inocência dos inculpados. José Lopes de Lima é
vítima da inexplicável odiosidade popular, armada por alguns
especuladores impudicos". Armação odiosa também era a razão pela qual
atacavam o vice-cônsul. Gama chama a responsabilidade para si -- "não
quero que a outrem se atribuam atos que são exclusivamente meus" -- de
algo que os acusadores atribuíam a Coutinho. Os dois casos, processados
simultaneamente, evidenciam que Gama tinha o relógio do direito no
pulso. Ou seja, sabia a hora exata de entrar ou encerrar uma discussão
jurídica na imprensa; sabia a hora de subir ou baixar a temperatura do
litígio. Seja num como em outro caso, preferiu não debater a causa nos
jornais. O fez pontualmente. Numa causa, havia ganhado; noutra, tinha o
andamento nas mãos. Quando disse "sou o eu o autor da demora legal",
sinalizava que o relógio do processo andaria conforme ele quisesse.
Coisa de quem manejava o tempo com maestria. }

\textbf{32. PROCESSO VIRA-MUNDO}\footnote{. In: \emph{Gazeta do Povo}
  (SP), Publicações Pedidas, 23/04/1881, p. 2.}

\textbf{*didascália*}

\emph{Luiz Gama conseguiu no Tribunal do Júri de São Paulo a absolvição
de seu cliente José Lopes de Lima. O resultado, contudo, despertou uma
reação em parte da imprensa que Gama qualificou de leviana, insana e
inspirada pelo ódio. O artigo dirige-se, portanto, a dois desses jornais
que, "sem estudo, sem conhecimento dos fatos, sem critério, sem base
moral, sob o domínio do despeito", atacaram de forma covarde o seu
cliente. E dirige-se também ao presidente do Tribunal do Júri que,
estimulado pelo jogo cínico da imprensa e não pelo mérito da causa,
recorreu da sentença dos jurados. Gama fulminava ambos, jornalistas e
magistrado, igualados como abutres de um "mísero proletário". É de se
imaginar que Gama escrevia com a verve que usou na tribuna do júri.
"José Lopes de Lima é um desgraçado cocheiro, negro, sem fortuna; não
admira, pois, que o espicacem bravejantes os exasperados abutres da
miséria". Não defendia um cliente, apenas. Era um irmão de infortúnios,
para remeter aos termos do compromisso que assumiu no célebre artigo
"Questão de Liberdade", ainda no início de sua carreira. O cocheiro Lima
tinha nome; não era o "Vira-Mundo", apelido que a imprensa insistia em
tachar. Era um cidadão, proletário e negro, absolvido pelo Tribunal do
Júri de São Paulo e no pleno exercício de sua liberdade. }

\emph{\textbf{***}}

É assim que se julga -- levianamente --, sem estudo, sem conhecimento
dos fatos, sem critério, sem base, sem moral, sob o domínio do despeito,
com as inspirações do ódio, pelo assomo de prevenções hiperbólicas, com
os atropelos da cólera e com a barbaridade nativa dos
atrabiliários!...\footnote{. Furiosos, raivosos.}

Acabo de ler duas biliosas parlandas\footnote{. Falatório, palavreado,
  discussão acalorada.}: uma da \emph{GAZETA DE S. PAULO} e outra da
\emph{COMÉDIA}; em ambas censuram-se, com desabrida\footnote{.
  Desagradável.} acrimônia\footnote{. Aspereza, indelicadeza.}, o
augusto Tribunal do Júri, o mais colendo do país\footnote{. Gama
  qualifica o colegiado de juízes de fato, populares e sem remuneração,
  como o mais respeitável do país.}, por a justa, devida e indeclinável
absolvição do réu José Lopes de Lima.

É uma insânia, mas está provada por dois artigos editoriais da ilustrada
imprensa da cidade de São Paulo.

A mim não surpreendeu este caprichoso e desabrido procedimento, de
sobejo\footnote{. De sobra, demasiado.} explicado por a baixa condição
do infeliz que ocupava o banco dos réus.

Tenho consciência de mim. Sei quando defendo um criminoso e quando
proclamo a inocência dos inculpados.

José Lopes de Lima é vítima da inexplicável odiosidade popular, armada
por alguns especuladores impudicos\footnote{. Imorais, sem-vergonha.}.

Lamento que os ilustrados redatores da \emph{GAZETA} e da \emph{COMÉDIA}
se tenham prestado, como publicistas, a servir de zarabatanas\footnote{.
  Instrumento com o qual, pelo sopro em um tubo comprido, se pode atirar
  pedras, grãos, pequenos objetos. Por sentido figurado, indica que os
  redatores serviram-se de meio de ataque.} à fatuidade\footnote{.
  Vaidade, presunção.} e à calúnia.

José Lopes de Lima é um desgraçado cocheiro, negro, sem fortuna; não
admira, pois, que o espicacem\footnote{. Picassem, furassem.}
bravejantes os exasperados abutres da miséria.

Dos distintos redatores da \emph{COMÉDIA}, cujos talentos venero, nada
de particular direi. São muito moços; muito têm que aprender para que
bem conheçam a sociedade e os homens. Sem ofensa do seu caráter posso
dizer-lhes: por enquanto, é duvidosa a sua capacidade para que possam
ter valiosa opinião perante os tribunais e o país.

Quanto ao meu digno e respeitável amigo, redator da \emph{GAZETA}, direi
apenas que tenho razoável fundamento para não aceitar as suas lições de
moral pública; e que me não convencem os seus simulados conselhos de
prudência, desmentidos à luz do dia, em lugar público, de modo
incontestável, \emph{pelos seus próprios atos}.

Não invejo a nobreza de sentimentos de pessoa alguma. Nunca possuí
escravos. Estou habituado a medir os homens por um só nível:
distinguo-os pelas ações. Se eu fosse juiz teria votado pela absolvição
do réu.

O exmo. sr. dr. Rocha Vieira, digno presidente do Tribunal do Júri é,
hoje, alvo de encômios\footnote{. Elogios.}, por ter apelado da decisão
dos juízes de fato\footnote{. Isto é, os jurados do Tribunal do Júri.}.
Estes elogios, porém, devem pesar, sombriamente, na consciência do
emérito juiz, porque ele deve ter a segurança de que os não mereceria,
se o réu, em vez de mísero proletário, fosse algum
\emph{régulo}\footnote{. Chefe de pouca importância, porém tirânico.}
\emph{político}, dos que, com a sua influência e o seu dinheiro dominam
sobranceiros\footnote{. Orgulhosos, arrogantes.} as populações e alinham
as conveniências judiciárias.

Não é meu desejo discutir esta questão pela imprensa.

Se o processo tornar a novo julgamento, eu, no tribunal, responderei
condignamente aos acusadores interessados e extrajudiciais do meu
cliente.

S. Paulo, 23 de Abril de 1881.

LUIZ GAMA.

\textbf{33. {[}RESPOSTA AO SR. F...{]}} \footnote{. Lusus, in: \emph{A
  Província de S. Paulo} (SP), Seção Livre, O vice-consulado de Portugal
  em S. Paulo, 17/03/1881, p. 1.}

\textbf{*didascália*}

\emph{Nota de Luiz Gama extraída de uma comunicação entre partes de um
tumultuoso processo de inventário. Esse rápido excerto sugere que Gama
advogava para o vice-cônsul de Portugal em São Paulo, o comendador Félix
de Abreu, e estava no centro dessa disputa que tomou conta dos jornais
da época.}

***

O sr. comendador Felix de Abreu nada me disse relativamente a inventário
ou espólio de Gomes do Paço; eu, sim, falei ao mesmo senhor e lhe disse
que estava coligindo documentos para requerer inventário.

\emph{Luiz Gama}.\footnote{. Antes de publicar a réplica de Gama, o
  articulista d'\emph{A} \emph{Província} insere uma carta da qual
  oculta a autoria, mantendo-se apenas a possível inicial do autor.
  Assim, \emph{Lusus} introduz a carta: "(...) Leia o público o seguinte
  documento que obtivemos do distinto advogado sr. Luiz Gama:

  ``'Ilmo. Sr. Luiz Gama.

  Digne-se V. S. responder-me ao pé desta o seguinte, autorizando-me a
  fazer de sua resposta o uso que me convier: se o vice-cônsul português
  Felix de Abreu Pereira Coutinho falou alguma vez com V. S. a respeito
  de se requerer inventário dos bens deixados pelo falecido português
  João Gomes do Paço.

  (...)

  F'".}

\textbf{34. A COLÔNIA PORTUGUESA EM S. PAULO -- O SEU A SEU
DONO}\footnote{. In: \emph{Gazeta do Povo} (SP), Publicações Pedidas,
  21/05/1881, p. 2.}

\textbf{*didascália*}

\emph{O artigo de Gama situa-se no curso de um intenso debate na
imprensa sobre o inventário de um barbeiro de provável nacionalidade
portuguesa. Detratores e críticos do vice-cônsul português em São Paulo,
Félix de Abreu, acusavam-no de agir com malícia ou desídia na
representação dos interesses dos portugueses envolvidos nesse caso. Gama
surge na acalorada discussão com o propósito de discernir as
responsabilidades que ele e o agente consular tinham no processo, por
isso o título "O seu a seu dono". Para Gama, o vice-cônsul era alvo de
uma "celeuma calculada, ou arregimentada propaganda" odiosa, em que
pretendiam culpá-lo por fato que não estava ao seu alcance. Sem arrodeio
algum, Gama pôs termo à questão. Como um advogado que dominava o tempo
do direito e do processo e não receava a gritaria na imprensa,
finalizava: "tirando licitamente partido das circunstâncias, e de fatos
que não criei, nem o exmo. sr. vice-cônsul, o autor da demora legal do
inventário de João Gomes do Paço sou eu". }

\emph{***}

Nestes últimos tempos, como de todos é notório, tem-se levantado, nesta
cidade, certa celeuma calculada, ou arregimentada propaganda, contra o
exmo. sr. comendador Felix de Abreu Pereira Coutinho, digno vice-cônsul
de sua majestade fidelíssima.

Esta celeuma ou propaganda, pelo menos segundo o que tenho observado, é
animada por alguns dignos membros da colônia portuguesa, nesta cidade,
que não ocultam sua animadversão\footnote{. Aversão intensa, ódio.}
àquele respeitável agente consular.

E a notável veemência da linguagem, um tanto \emph{realista}, dos
artigos que avultam em nossa imprensa, dá prova inequívoca do ardor e da
paixão partidária, de que estão possuídos os seus entusiásticos autores.

Saiba-se, entretanto, uma vez por todas, que eu nada tenho que ver com
as causas eficientes, nem com as paixões exuladas dos acusadores do
exmo. sr. vice-cônsul; o meu fim, vindo à imprensa, por esta única vez,
nestes encapelados\footnote{. Agitados.} debates, é explicar fatos que
estão propositalmente baralhados\footnote{. O mesmo que embaralhados,
  misturados.}: não quero que a outrem se atribuam atos que são
exclusivamente meus.

Por o falecimento do barbeiro João Gomes do Paço, cuja nacionalidade,
até hoje, ainda não foi regularmente verificada, deram-se preterições
legais, que, com justiça, não podem ser lealmente postas à culpa do
vice-consulado português; e quanto à suposta existência e arrecadação
dos bens desse estrangeiro, se o é, por sua proverbial\footnote{.
  Notória, amplamente conhecida.} prudência, diante de certas delicadas
dificuldades, o exmo. sr. vice-cônsul tem se achado em sérios embaraços.

Sou procurador da senhora viúva de João Gomes do Paço; e desde o
falecimento deste, ainda não requeri o respectivo inventário; e assim
tenho procedido, de caso pensado, para não dar causa a prejuízos da
minha infeliz cliente.

E é certo que, em público, os agressores do exmo. sr. vice-cônsul não me
podem pedir esclarecimentos deste meu reservado procedimento.

A verdade, pois, até hoje, relativamente às espinhosas circunstâncias
que ladeiam esta melindrosa ocorrência, está em que as autoridades do
país tiveram justificáveis escrúpulos e contiveram-se pensadamente
diante da lei; que o exmo. sr. vice-cônsul, não menos avisado, de modo
indireto, procurou apoiar-se em alheio procedimento; e que eu, como
procurador, aguardo oportunidade para proporcionar à minha constituinte
o mais vantajoso resultado, sem atropelo de fórmulas, sem prejuízos de
interesses, sem infrações da lei.

Em conclusão: tirando licitamente partido das circunstâncias, e de fatos
que não criei, nem o exmo. sr. vice-cônsul, o autor da demora legal do
inventário de João Gomes do Paço sou eu.

S. Paulo, 21 de Maio de 1881.

LUIZ GAMA.

\textbf{O ÁS DA ABOLIÇÃO: A CARTA DE LUIZ GAMA PARA FERREIRA DE MENEZES
}

\textbf{*didascália*}

\emph{O título despretensioso -- "Trechos de uma carta" ou simplesmente
"Carta a Ferreira de Menezes" -- oculta a grandeza do projeto literário
abolicionista que ganhou forma na imprensa do Rio de Janeiro da última
década da escravidão. A série de onze artigos, todos publicados entre
dezembro de 1880 e fevereiro de 1881, expressa de modo cristalino a
consistência ideológica e política da visão estratégica de Luiz Gama no
front da luta pela Abolição. É, sem dúvida, um dos mais impactantes e
significativos textos da campanha abolicionista. Na condição de líder do
pujante movimento social abolicionista de São Paulo, Gama dirige-se aos
amigos e correligionários do movimento no Rio de Janeiro, expondo e
sustentando argumentos pela Abolição imediata, ampla, geral, irrestrita.
Em onze partes de uma mesma carta -- afinal, tinha destinatário,
estrutura e finalidades comuns --, Gama elabora uma espécie de programa
não para o futuro distante ou para a diletância intelectual de algum
escanteado do processo em curso, mas um roteiro de luta para o calor da
hora da política brasileira. Assim, começa a série por uma resposta a um
discurso do deputado paulista Moreira de Barros e sua interpretação
sobre a multinormatividade do contrabando. É de se notar como Gama
contesta de pronto a leitura enviesada do deputado e, ao replicá-lo,
propositadamente subsidiava de argumentos seus companheiros de causa
abolicionista. De algum modo, essa ideia perpassa toda a série:
informar, subsidiar e instruir os mais jovens, sobretudo, dos caminhos e
desafios da luta abolicionista. Para isso, Gama traçava quais eram suas
linhas programáticas, seus argumentos-chave e táticas imediatas, além de
categorizar quais eram os adversários do movimento, partidos,
instituições e associações constituídas "dos homens ricos, dos
milionários, da gente que tem o que perder". A carta, portanto, dava
instrumentos para se formar uma nova geração de militantes versados na
semântica do abolicionismo. No entanto, não se formaria "uma falange,
uma legião de cabeças, mas com um só pensamento, animados de uma só
ideia: a exterminação do cativeiro''. Seria necessário muito mais. Seria
necessário assentar bases para um compromisso moral contra a escravidão.
Seria fundamental constituir uma sensibilidade e uma estética
intransigentes com a escravidão. Assim, Gama exigia coerência de seus
companheiros -- "os abolicionistas pobres, inteligentes, que nunca
tiveram escravos (...) que não se corrompem, nem se vendem". Ao mesmo
tempo, estigmatizava os senhores de escravizados, individualmente, como
figuras desumanas, cruéis e assassinas; e, coletivamente, como uma
"classe social" inescrupulosa e repugnante. Gama não capitulou. A carta,
que se revelava um ás na mesa de jogo amarrado, ou que desvelava a
autoria de um ás da literatura, tem uma semântica original, um argumento
poderoso e uma narrativa arrebatadora. Tinha os olhos no presente.
Destinada estava, desde as primeiras linhas, a ganhar a posteridade como
uma das páginas mais instrutivas da história da Abolição no Brasil e nas
Américas. É a história do abolicionismo brasileiro contada desde onde
realmente começou. }

\textbf{35. TRECHOS DE UMA CARTA}\footnote{. In. \emph{Gazeta da Tarde}
  (RJ), Noticiário, 01/12/1880, p. 1.}

\textbf{*didascália*}

\emph{No primeiro dos "trechos de uma carta", Gama inicia com uma
análise da multinormatividade do contrabando, isto é, do conjunto de
textos legais relacionados à proibição do comércio transatlântico de
escravizados da África com o Brasil. Mas não inicia a esmo. Gama estava
muito atento aos debates parlamentares referentes às possibilidades de
extinção da escravidão no Brasil. Tanto estava atento que leu com lupa
um discurso escravocrata e, ato contínuo, preparou uma réplica.
Tratava-se de um pronunciamento profundamente atrelado aos interesses da
escravidão proferido pelo deputado liberal, também de São Paulo, Moreira
de Barros. O ponto central do argumento liberal-escravocrata era um no
qual Gama era o maior dos experts: a vigência da Lei de 1831 e de outras
normatividades do contrabando. Para Moreira de Barros, desde que o
comércio transatlântico de escravizados cessou, as leis perderam sua
eficácia. "Será isto verdade? Se é, o orador estava louco". Gama, como
se vê pelo estilo, seria mais uma vez intransigente. Ciente das disputas
partidárias no parlamento, optou por trazer à baila a opinião de um
antigo baluarte conservador, o ex-ministro Eusébio de Queirós. Não
visava exatamente a simpatia do Partido Conservador, pretendia, muito
antes disso, rotular liberais e conservadores como formas de uma mesma
política, a política da escravidão. }

\emph{***}

Não li o discurso que me dizem ter proferido nestes últimos dias o dr.
Moreira de Barros,\footnote{. Antonio Moreira de Barros (1841-1896),
  paulista de Taubaté, foi deputado, ministro e presidente da província
  de Alagoas.} no qual dissera que a Lei de 1831, bem como toda a
legislação anterior ou posterior à essa época, promulgada para evitar o
tráfego, ficara sem efeito algum, desde que esse cessou completamente.

Será isto verdade? Se é, o orador estava louco.

Tu sabes que a lei mais terrível, e que vedou o tráfico, é a de 4 de
Setembro de 1850, regulamentada por decreto de 14 de Outubro do mesmo
ano, proposta pelo grande conselheiro Euzébio,\footnote{. Euzébio de
  Queiroz Coutinho Mattoso da Camara (1812-1868), nascido em São Paulo
  de Luanda, foi chefe de polícia, deputado, ministro, senador e
  conselheiro do Imperador. Como ministro da Justiça (1848-1852), foi o
  responsável pela Lei de 4 de Setembro de 1850, conhecida como Lei
  Euzébio de Queiroz, que proibiu o tráfico negreiro em caráter
  terminante.} "o espírito mais justo e liberal, até hoje conhecido no
Brasil".

Esse eminente estadista quando teve de dar explicações relativas à
execução da Lei, em sessão de 16 de Julho de 1852, LEU uma exposição de
motivos, na qual se acha o seguinte:

"Um único meio assim resta para reprimir o tráfico... é deixar que a
respeito do passado continue a Legislação existente; que ela continue
igualmente a respeito dos pretos introduzidos para o futuro, mas que só
se apreenderam depois de internados pelo país e não pertencerem mais aos
introdutores... os filantropos não terão que dizer, vendo que, para as
novas introduções, se apresentam alterações eficazmente repressivas, e
que, para o passado, não se fazem favores, e apenas continua o que
está...

Por isso entreguei não só a formação da culpa, como todo o processo ao
juiz especial dos auditores de Marinha, com recurso para a Relação, bem
entendido só nos casos de apreensão, no ato de introduzir, ou sobre o
mar."

Depois desta leitura, continuando o ministro em seu discurso, disse
mais:

"Vê pois a Câmara, que eu havia comunicado aos meus colegas, que os
grandes pensamentos da Lei de 4 de Novembro de 1850, eram pensamentos
nossos em 1849. Nós já então reparávamos a questão das presas, dos
julgamentos do réu; já então mantínhamos a Lei de 7 de Novembro de 1831,
reservando-a, porém, \emph{somente para o passado}, ou para os escravos
depois de internados e compreendidos com os outros; já então
distinguíamos os introdutores dos compradores..."

Veja agora o que diz o Decreto de 12 de Abril de 1832 publicado para a
Lei de 1831, no art. 10:

"Em qualquer tempo em que o preto requerer, a qualquer juiz de paz ou
criminal, que veio para o Brasil depois da extinção do tráfico, o juiz o
interrogará sobre todas as circunstâncias que possam esclarecer o fato,
e oficialmente procederá a todas as diligências necessárias, para
certificar-se delas; obrigando o senhor a desfazer as dúvidas que
suscitarem-se a respeito. Havendo presunções veementes de ser o preto
livre, o mandará depositar e procederá nos mais termos da lei."

Se é verdade que o dr. Moreira de Barros disse o que ouvi se lhe
atribuir, aí está na Lei, e nas palavras do grande Euzébio a mais cabal
resposta.

LUIZ GAMA.

\textbf{36. TRECHOS DE UMA CARTA}\footnote{. In. \emph{Gazeta da Tarde}
  (RJ), 12/12/1880, p. 3.}

\textbf{*didascália*}

\emph{Gama escreve aos amigos e correligionários do Rio de Janeiro,
sobretudo, para falar dos adversários do movimento, "dos homens ricos,
dos milionários, da gente que tem o que perder". Como militante
veterano, Gama traça um estereótipo de um tipo de adversário a que os
abolicionistas precisariam estar vigilantes: uma espécie de
"abolicionista" que acha um meio de dar "uma adversativa" e refugar a
causa. A julgar pela referência que faz ao seu "Questão Jurídica",
pode-se dizer que Gama tinha em mente os republicanos e liberais
paulistas. Estes, num primeiro momento, até pareceriam simpáticos ao
movimento, mas, entre uma adversativa e outra, logo clamavam que o
movimento "é perigoso, atinge a violência, provoca uma catástrofe, deve
ser reprimido!" Gama alertava que essa qualidade de liberal e
republicano quereria uma abolição apenas para o próximo século. Para
eles, dizia Gama, "as alforrias devem provar-se por certidões de óbito".
E dizia mais: a liberdade que eles, os adversários do movimento
abolicionista, defendiam, era "uma liberdade métrica, bacalhocrática,
ponderada, refletida, triturada, peneirada, dinamizada, apropriada a
corpos dessangrados, higiênica e esculápica para os moribundos e
funerária para os mortos." Gama não tinha dúvidas de quem eram. "Da
minha parte, ouço-os: sei o que eles são, e o que querem; sei o que
faço, e prossigo na minha tarefa". A luta pela Abolição vinha de muito
tempo. Gama ensinava o caminho da liberdade. }

\emph{***}

São Paulo, 8 de Dezembro.

Preciso é que tu saibas o que por aqui se diz relativamente à nobre
cruzada emancipadora.

Não me refiro aos nossos amigos, aos nossos correligionários, aos nossos
companheiros de luta; constituímos uma falange, uma legião de cabeças,
mas com um só pensamento, animados de uma só ideia: a exterminação do
cativeiro e breve.

Tratarei dos nossos adversários, dos homens ricos, dos milionários,
\emph{da gente que tem o que perder}.

Provocados, em particular ou em público, nas palestras, nas conversações
familiares, ou nas reuniões, sobre a Lei de 1818, ou de 1831, ou sobre o
projeto de 1837 ou sobre a negra tentativa de 1848, mostram-se lhanos,
dóceis, contritos\footnote{. Arrependidos, pesarosos.}, curvam-se à
razão, reconhecem o direito, confessam a verdade, rendem culto à
liberdade, dão-nos ganho de causa, aplaudem-nos, e proclamam com
entusiasmo a santidade da causa que defendemos, mas... dá-se uma
adversativa; reclamam para si um lugar na propaganda; eles também são
abolicionistas; porque também são brasileiros.

O que nós pretendemos é grande, eles o proclamam; é belo, é invejável; o
que, porém, estamos fazendo, é perigoso, atinge a violência, provoca uma
catástrofe, deve ser reprimido!...

Os filósofos (eles também os têm, e até positivistas) querem que se não
altere o compassado movimento do tempo, a morna calma do taciturno
vanzear\footnote{. Como o balançar vagaroso de uma embarcação em mar sem
  ondas.} da nau do Estado. Para extinguir o cativeiro é preciso criar,
com supina\footnote{. Elevada, notável.} reflexão, o seu condigno
substituto: a sociedade não pode prescindir do servilismo.

A emancipação, a liberdade, hão-de vir com o vagar providencial das
criações geológicas, para evitar-se indigestões morais, não menos
perigosas que as físicas, mormente\footnote{. Sobretudo, principalmente.}
em aspérrimos\footnote{. Muito áspero.} alváreos de implacável
catadura\footnote{. Aspecto, aparência, expressão do semblante.}
africana.

Eles têm a experiência do mundo, sabem que Adão foi feito de barro, e
que a celeridade é um instrumento destruidor. Querem uma abolição
secular; as alforrias devem provar-se por certidões de óbito; uma
liberdade métrica, bacalhocrática\footnote{. Referência a bacalhau que,
  em sentido figurado próprio à época, remetia ao chicote, chibata.},
ponderada, refletida, triturada, peneirada, dinamizada, apropriada a
corpos dessangrados\footnote{. Debilitado, que perdeu muito sangue.},
higiênica e esculápica\footnote{. No sentido de medicada.} para os
moribundos e funerária para os mortos.

Tomaram atitude de bonzo\footnote{. No sentido de indivíduo fingido,
  sonso, preguiçoso.}, constituíram-se imagens ambulantes da
pachorra\footnote{. Lentidão, apatia.}, trazem, em punho, por
ornato\footnote{. Ornamento, adereço.}, a cartilha salvadora da
resignação.

Nós provocamos reações perigosas, por virtude; estrangulamos nos puros
corações o sentimento generoso da piedade; estamos, por indesculpável
imprudência, retardando uma reparação nacional; ferimos de morte o
patriotismo, e menosprezamos, diante do estrangeiro, o pudor da
aristocracia brasileira.

Acrescentam, e isto é notável, que a lavoura tem dois inimigos: nós, e
os próprios correligionários deles, que estão sacrificando a sua causa
no Parlamento!...

Da minha parte, ouço-os: sei o que eles são, e o que querem; sei o que
faço, e prossigo na minha tarefa.

Hoje entreguei na redação da \emph{Província} um trabalho humilde, feito
às pressas, mas de alguma utilidade sobre a tese seguinte: "Subsistem os
efeitos manumissórios da Lei de 26 de Janeiro de 1818, depois da
promulgação das de 7 de Novembro de 1831 e 4 de Setembro e
1850?"\footnote{. Refere-se ao artigo \emph{Questão jurídica,} que se lê
  nesse volume.}

Escrevi-o a propósito de umas asseverações do exmo. sr. desembargador
Faria, procurador da Coroa, feitas no Tribunal da Relação quando se
discutia uma ordem de \emph{habeas-corpus} impetrada por mim em favor de
uma africano livre posto em cativeiro.

LUIZ GAMA.

\textbf{37. CARTA AO DR. FERREIRA DE MENEZES}\footnote{. In.
  \emph{Gazeta da Tarde} (RJ), 16/12/1880. Também publicado em
  \emph{Gazeta do Povo} (SP), 14/12/1880; e \emph{A Província de S.
  Paulo} (SP), 18/12/1880.}

\textbf{*didascália*}

\emph{Gama toma como mote da carta uma notícia recém-publicada no
jornal. A notícia de Itu, interior paulista, o indigna de tal maneira
que o convalescente Luiz Gama se levanta e pega a pena para escrever um
dos mais veementes protestos contra a crueldade da escravidão no Brasil.
Quatro escravizados assassinaram o filho de um influente fazendeiro
escravocrata. Após o cometimento do crime, os escravizados não fugiram,
antes buscaram proteção das autoridades policiais. Revoltados,
"trezentos cidadãos" vão em marcha até o cárcere onde os "quatro
Espártacos" estavam presos e, armados "à faca, à pau, à enxada, à
machado", invadem a repartição policial e "matam valentemente a quatro
homens; menos ainda, a quatro negros; ou, ainda menos, a quatro escravos
manietados em uma prisão!" Dessa "hecatombe" Gama tira uma conclusão
filosófica que sintetizaria sua visão de mundo e de direito: "o escravo
que mata o senhor, que cumpre uma prescrição inevitável de direito
natural, e o povo indigno, que assassina heróis, jamais se confundirão".
A reflexão se opunha diametralmente a de um professor da Faculdade de
Direito de São Paulo, Leite Moraes, que havia se posicionado e
justificado o linchamento de trezentos contra quatro. Gama não pararia
ali. A revolta com o povo indigno da "heróica, a fidelíssima, a
jesuítica cidade de Itu estendia-se para outras praças da província de
São Paulo. É o caso do "auto de fé agrário" ocorrido em Limeira, também
interior paulista, onde um "rico e distinto fazendeiro" e certamente
branco matou um homem negro com os mais violentos requintes de
crueldade. "O escravo foi amarrado, foi despido, foi conduzido ao seio
do cafezal", contava Gama, com a loquacidade de uma testemunha ocular.
"Fizeram-no deitar: e cortaram-no a chicote, por todas as partes do
corpo: o negro transformou-se em Lázaro; o que era preto se tornou
vermelho. Envolveram-no em trapos... Irrigaram-no a querosene:
deitaram-lhe fogo..." Gama, adoentado em sua casa, deve ter chorado,
porque aos outros seguramente o fez. }

\emph{***}

\emph{S. Paulo, 13 de Dezembro de 1880. }

Meu caro Menezes,

Estou em a nossa pitoresca choupana do Brás\footnote{. Gama escreve do
  seu endereço até o fim vida, a "casa de campo" do Brás, muito
  provavelmente o número 25 da Rua do Brás (hoje denominada Rangel
  Pestana), nas cercanias da antiga Estação Norte (atualmente estação
  Pedro II da linha vermelha do metrô paulistano).}, sob as ramas
verdejantes de frondosas figueiras, vergadas ao peso de vistosos frutos,
cercado de flores olorosas, no mesmo lugar onde, no começo deste ano,
como árabes felizes, passamos horas festivas, entre sorrisos inocentes,
para desculpar ou esquecer humanas impurezas.

Daqui, a despeito das melhoras que experimento, ainda pouco saio às
tardes, para não contrariar as prescrições do meu escrupuloso médico e
excelente amigo, dr. Jayme Serva\footnote{. Jayme Soares Serva
  (1843-1901), baiano de Salvador, onde se formou em medicina em 1867.
  Foi voluntário da pátria durante os combates na Guerra do Paraguai e
  de lá voltou com a patente de major médico. Fez carreira médica em São
  Paulo.}.

Descanso dos labores e das elocubrações da manhã, e preparo o meu
espírito para as lutas do dia seguinte.

Este mundo é uma mitologia perpétua; o homem é o eterno Sísifo\footnote{.
  Na mitologia grega, Sísifo era o mais astucioso dos mortais e, por
  abusar da sua esperteza e malícia, foi condenado por toda a eternidade
  a empurrar montanha acima uma enorme pedra redonda de mármore e,
  quando já chegando ao cume da montanha, soltá-la montanha abaixo,
  tornando a carregá-la acima e empurrá-la abaixo num movimento
  incessante e contínuo. Numa bonita passagem, Gama reflete e exclama
  sobre a natureza humana e seus dias de lutas na imprensa e no foro.}.

Acabo de ler, na \emph{Gazeta do Povo}, o martirológio\footnote{. Lista
  dos que morreram ou sofreram por uma causa.} sublime dos quatro
Espártacos\footnote{. Espártacos (109 a.C-71 a.C) foi um
  gladiador-general, estrategista e líder popular que escapou da
  escravidão a que era submetido e, num levante de grandes proporções,
  organizou um exército que enfrentou o poder central de Roma na
  Terceira Guerra Servil (73 a.C-71 a.C). São diversas as citações de
  Gama a Espártacos, grafado de variadas maneiras, a exemplo de
  Spartacus, o que revela sua admiração e até mesmo veneração pela
  história do mártir que venceu o cativeiro e lutou pelo fim da
  escravidão.}, que mataram o infeliz filho do fazendeiro Valeriano José
do Valle.

É uma imitação de maior vulto da tremenda hecatombe que aqui presenciou
a heróica, a fidelíssima, a jesuítica cidade de Itu, e que foi
justificada pela eloquente palavra do exmo. sr. dr. Leite
Moraes\footnote{. Joaquim de Almeida Leite Moraes (1834-1895), paulista
  de Tietê, foi professor de direito, vereador, deputado e presidente da
  província de Goiás.}, deputado provincial, e professor considerado da
nossa faculdade jurídica.

Há cenas de tanta grandeza, ou de tanta miséria, que, por completas, em
seu gênero, se não descrevem: o mundo e o átomo por si mesmo se definem;
o crime e a virtude guardam a mesma proporção; assim, o escravo que mata
o senhor, que cumpre uma prescrição inevitável de direito natural, e o
povo indigno, que assassina heróis, jamais se confundirão.

Eu, que invejo com profundo sentimento esses quatro apóstolos do dever,
morreria de nojo, de vergonha, se tivesse a desgraça de, por torpeza,
achar-me entre essa horda inqualificável de assassinos.

Sim! Milhões de homens livres, nascidos como feras ou como anjos, nas
fúlgidas areias da África, roubados, escravizados, azorragados\footnote{.
  Açoitados, chicoteados.}, mutilados, arrastados, neste país clássico
da sagrada liberdade, assassinados impunemente, sem direitos, sem
família, sem pátria, sem religião, vendidos como bestas, espoliados em
seu trabalho, transformados em máquinas, condenados à luta de todas as
horas e de todos os dias, de todos os momentos, em proveito de
especuladores cínicos, de ladrões impudicos\footnote{. Imorais,
  sem-vergonha.}, de salteadores sem nome, que tudo isto sofreram e
sofrem, em face de uma sociedade opulenta, do mais sábio dos
monarcas\footnote{. Referência tão explícita quanto irônica à figura de
  Pedro II.}, à luz divina da santa religião católica e apostólica
romana, diante do mais generoso e do mais desinteressado dos povos; que
recebiam uma carabina envolvida em uma carta de alforria, com a
obrigação de se fazerem matar à fome, à sede e à bala nos
esteiros\footnote{. Terreno baixo, alagadiço e pantanoso.}
paraguaios\footnote{. Referência à Guerra do Paraguai (1865-1870), maior
  conflito militar do Império e da América do Sul no século XIX.}; e
que, nos leitos dos hospitais, morriam, volvendo os olhos ao território
brasileiro, ou que, nos campos de batalha, caiam, saudando risonhos o
glorioso pavilhão da terra de seus filhos; estas vítimas, que, com o seu
sangue, com o seu trabalho, com a sua jactura\footnote{. No sentido de
  orgulho.}, com a sua própria miséria, constituíram a grandeza desta
nação, jamais encontraram quem, dirigindo um movimento espontâneo,
desinteressado, supremo, lhes quebrasse os grilhões do cativeiro.

Quando, porém, por uma força invencível, por um ímpeto indomável, por um
movimento soberano no instinto revoltado, levantam-se, como a razão, e
matam o senhor, como Lusbel\footnote{. Lúcifer.} mataria a Deus, são
metidos no cárcere; e, aí, a virtude exaspera-se, a piedade contrai-se,
a liberdade confrange-se, a indignação referve, o patriotismo arma-se,
\emph{trezentos concidadãos} congregam-se, ajustam-se, marcham direitos
ao cárcere e aí, (oh! é preciso que o mundo inteiro aplauda) à faca, à
pau, à enxada, à machado, matam valentemente a \emph{quatro homens};
menos ainda, a quatro negros; ou, ainda menos, a quatro escravos
manietados\footnote{. Amarrados, de mãos atadas.} em uma prisão!...

Não! Nunca! Sublimaram, pelo martírio, em uma só apoteose, quatro
entidades imortais!

Quê! Horrorizam-se os assassinos de que quatro escravos matassem seu
senhor? Tremem porque eles, depois da lutuosa cena, se fossem apresentar
à autoridade?

Miseráveis! Ignoram que mais glorioso é morrer livre em uma forca, ou
dilacerado pelos cães na praça pública, do que banquetear-se com os
Neros na escravidão.

Sim! Já que a quadra é dos grandes acontecimentos; já que as \emph{cenas
de horror} estão em moda; e que os nobilíssimos corações estão em boa
maré de exemplares vinditas\footnote{. O mesmo que vingança, desforra.},
leiam mais esta:

Foi no município da Limeira\footnote{. Município do interior paulista,
  distante 140 km da capital.}; o fato deu-se há dois anos.

Um rico e distinto fazendeiro tinha um crioulo, do norte, esbelto, moço,
bem parecido, forte, ativo, que nutria o vício de detestar o cativeiro:
em três meses fez dez fugidas!

Em cada volta sofria um rigoroso castigo, incentivo para nova fuga.\\
A mania era péssima; o vício contagioso e perigosíssima a imitação.\\
Era indeclinável um pronto e edificante castigo.\\
Era a décima fugida; e dez são também os mandamentos da lei de Deus, um
dos quais, o mais filosófico e mais salutar é -- \emph{castigar os que
erram.\\
} O escravo foi amarrado, foi despido, foi conduzido ao seio do cafezal,
entre o bando mudo, escuro, taciturno dos aterrados parceiros: um Cristo
negro, que se ia sacrificar pelos irmãos de todas as cores.

Fizeram-no deitar: e \emph{cortaram-no} a chicote, por todas as partes
do corpo: o negro transformou-se em Lázaro\footnote{. Provável
  referência a Lázaro de Betânia, personagem bíblico descrito no
  Evangelho de João (11:41-44), que, quatro dias depois de morto, teria
  sido ressuscitado por milagre de Jesus. O contexto que invoca o tema
  do sacrifício reforça essa leitura. No entanto, a referência também
  pode ser a Lázaro, mendigo e leproso que protagoniza a conhecida
  parábola \emph{O Rico e Lázaro}, narrada no Evangelho de Lucas
  (16:19-31).}; o que era preto se tornou vermelho.

Envolveram-no em trapos...\\
Irrigaram-no a \emph{querosene}: deitaram-lhe fogo... Auto de fé
agrário!...\\
Foi o restabelecimento da inquisição; foi o renovamento\footnote{. O
  mesmo que renovação.} do \emph{touro de Fálaris}\footnote{. O touro de
  bronze, que leva o nome do déspota Fálaris, foi uma máquina de tortura
  e execução, símbolo máximo da crueldade na Antiguidade. Espécie de
  esfinge taurina onde o executado era confinado e queimado, tendo seus
  gritos de suplício canalizados até a boca da esfinge, que parecia
  urrar com a tortura.}, com dispensa do simulacro de bronze; foi uma
figura das candeias\footnote{. Espécie de tocha que se acende com a
  queima de um pavio embebido em óleo.} vivas dos jardins romanos;
davam-se, porém, aqui, duas diferenças: a iluminação fazia-se em pleno
dia; o combustor\footnote{. O que queima, arde.} não estava de pé
empalado; estava decúbito\footnote{. Posição do corpo de quem está
  deitado, de barriga para baixo ou de costas.}; tinha por leito o chão,
de que saíra, e para o qual ia volver em cinzas.

Isto tudo consta de um auto, de um processo formal, está arquivado em
cartório, enquanto o seu autor, rico, livre, poderoso, respeitado, entre
sinceras homenagens, passeia ufano, por entre os seus iguais.

Dirão que é justiça de salteadores?

Eu limito-me a dizer que é digna dos nobres ituanos, dos limeirenses e
dos habitantes de Entre-Rios.

Estes quatro negros espicaçados pelo povo, ou por uma aluvião de
abutres, não eram quatro homens, eram quatro ideias, quatro luzes,
quatro astros: em uma convulsão sidérea desfizeram-se, pulverizaram-se,
formaram uma nebulosa.

Nas épocas por vir, os sábios astrônomos, os Aragos\footnote{. Dominique
  Francois Jean Arago (1786-1853) foi um astrônomo, deputado e ministro
  francês.} do futuro, hão de notá-los entre os planetas: os sóis
produzem mundos.

Teu

LUIZ GAMA.

\textbf{38. TRECHOS DE UMA CARTA}\footnote{. In. \emph{Gazeta da Tarde}
  (RJ), 28/12/1880.}

\textbf{*didascália*}

\emph{Gama abre esse trecho da carta com uma interpretação interessante
sobre a "lei áurea", não a de 1888, obviamente, porque não a
testemunhou, mas a de 1871, a conhecida lei do ventre livre. Com a
habilidade retórica de praxe, afirma que a libertação do ventre
escravizado foi "imposta ao governo e arrancada ao parlamento" pela
vontade nacional, mas esse mesmo governo e seus magistrados a violariam
escandalosamente, com sofismas, preterições, prevaricações, vícios,
caprichos entre outras condutas não menos criminosas. A conclusão que
Gama tira é fatal: "É que os homens do governo, os juízes e os
funcionários têm famílias, têm amigos, têm interesses, têm escravos!"
Famílias, amigos, interesses e escravidão -- haja união! -- faziam dos
senhores uma classe coesa e organizada. Era contra ela que Gama se
voltava. "Os senhores dominam pela corrupção, têm ao seu serviço
ministros, juízes, legisladores, encaram-nos com soberba, reputam-se
invencíveis". Urgia contra-atacar essa classe. Encarando de volta os
senhores, Gama explicita a marcação racial da escravidão. Afirma que o
negro "é a causa da grandeza do Brasil", que, "chamado escravo, na
expressão legal, este homem sem alma, este cristão sem fé, este
indivíduo sem pátria, sem direitos, sem autonomia, sem razão, é
considerado abaixo do cavalo, é um racional toupeira, sob o domínio de
feras humanas -- os senhores". Gama deflagra então a prova de seu
argumento, isto é, de que os senhores, "cônscios da impunidade, que os
distinguem", agiam como feras humanas. Assim, passa a descrever "um
fato, entre muitos semelhantes, de deslumbradora eloquência". O abandono
do "inocente mulatinho" à porta da casa de seu amigo pessoal, "maior de
sessenta anos" e "de cor preta", Porphirio Pires Carneiro, é de
escandalizar os leitores -- da época e de hoje. "Isto é torpeza de
branco, exclamava ele {[}Carneiro{]} enfurecido, enfiando os dedos
pretos pelos bastos cabelos brancos!" A fúria de Gama e de Carneiro está
no texto de modo arrebatador. Da fúria, contudo, Gama mirava o futuro: a
Abolição. Chegaria o dia, Gama imaginava, em que os senhores "hão de
apertar a mão do liberto, nivelados pelo trabalho, pela honra, pela
dignidade, pelo direito, pela liberdade". }

\emph{***}

S. Paulo, Dezembro de 1880.

Meu caro Menezes.

A lei áurea de 28 de Setembro de 1871, imposta ao governo e arrancada ao
parlamento, por a vontade nacional, em circunstâncias
climatéricas\footnote{. Críticas, perigosas.}, desde o começo
grosseiramente sofismada, senão criminosamente preterida em sua
execução, e que, hoje, muito longe está de satisfazer as aspirações à
civilização e os progressos do país, ainda assim, continua a ser
flagrantemente violada pelo governo, pela Magistratura, pela
monocracia\footnote{. O mesmo que monarquia, autocracia, regime em que o
  governante detém a soberania política, isto é, a palavra final sobre
  assuntos civis.} e pelos donos de escravos.

Dão-se as violações escandalosas contra os manumitentes\footnote{.
  Alforriandos, que demandam liberdade.}, contra os pecúlios\footnote{.
  Patrimônio, quantia em dinheiro que, por lei (1871), foi permitido ao
  escravizado constituir a partir de doações, legados, heranças e
  diárias eventualmente remuneradas.} públicos ou particulares, contra
as arrecadações, contra as avaliações ou arbitramentos, e somente a
favor dos \emph{senhores}!... É que os homens do governo, os juízes e os
funcionários têm famílias, têm amigos, têm interesses, têm escravos!...

As alforrias pelo fundo de emancipação constituem, geralmente falando, a
mais sórdida prevaricação\footnote{.
  Faltar~ao~cumprimento~do~dever~por~interesse~ou~má-fé.}. As
classificações são viciosas; na escolha dos libertandos\footnote{.
  Alforriandos, manumitentes, aqueles que demandam liberdade.} domina o
capricho; os arbitramentos são de excessivo valor.

Pode-se afirmar, salvando raríssimas exceções, que o serviço não tem
desempenho regular, é feito por uma horda de prevaricadores\footnote{.
  Corruptos, aqueles que
  faltam~ao~cumprimento~do~dever~por~interesse~ou~má-fé.}.

Os \emph{senhores} procedem com ostensivo despudor. Tratam os cavalos de
estrebaria como seus próprios irmãos: até aí nada vejo de repreensível;
porque o sábio conde de Chesterfield,\footnote{. Provável referência ao
  4º conde de Chesterfield, Philip Stanhope (1694-1773), aristocrata e
  diplomata. A hipótese é sugerida devido à presença do conde como
  personagem literário de dois livros publicados em meados do século
  XIX: \emph{Barnaby Rudge: A Tale of the Riots of Eighty} (1841), de
  Charles Dickens, e \emph{The Virginians} (1857), de William Makepeace
  Thackeray.} que tinha razões de sobra, dizia \emph{que certos fidalgos
eram menos nobres que os seus cavalos}.

Cobrem-nos (aos cavalos) de lã e de sedas durante o inverno:
envidraçam-lhe as estrebarias, alcatifam\footnote{. Cobrir com tapete.}
o assoalho de escolhida palha e até mandam vir da Europa a sua
alimentação. Durante o verão dão-lhe pastos especiais, fazem-nos mudar
de clima, mandam banhá-los uma e duas vezes por dia.

O homem, porém, a \emph{imagem de Deus}, a máquina viva e ambulante do
trabalho, o negro, o escravo, come do mesmo alimento, no mesmo vasilhame
dos porcos; dorme no chão, quando feliz, sobre uma esteira; é presa dos
vermes e dos insetos; vive semi-nu; exposto aos rigores da chuva, do
frio e do sol; unidos, por destinação, ao cabo de uma enxada, de um
machado, de uma foice; tem como despertador o relho do feitor, as surras
do administrador, o tronco, o viramundo\footnote{. Pesado grilhão de
  ferro.}, o grilhão\footnote{. Cadeia grossa de argolas de ferro.}, as
algemas, o gancho ao pescoço, a fornalha do engenho, os \emph{banhos de
querosene}, as fogueiras do cafezal, o suplício, o assassinato pela fome
e pela sede!... E tudo isto santamente amenizado por devotas orações ao
crepúsculo da tarde e ao alvorecer do dia seguinte.

O negro, disse o meu estimável amigo, o exmo. sr. dr. Belfort
Duarte,\footnote{. Francisco de Paula Belfort Duarte (1844-?),
  maranhense, jornalista, advogado e deputado. Graduado pela Faculdade
  de Direito do Largo de São Francisco (1864).} é a causa da grandeza do
Brasil: pois bem, este miserável grande, fator da opulência daquele
grande miserável, este animal maravilhoso, chamado escravo, na expressão
legal, este homem sem alma, este cristão sem fé, este indivíduo sem
pátria, sem direitos, sem autonomia, sem razão, é considerado abaixo do
cavalo, é um racional topeira, sob o domínio de feras humanas --
\emph{os senhores}.

Por as Leis de 1º de Outubro de 1828, art. 59, e nº 16 de 12 de Agosto
de 1834, art. 1º, foram as câmaras municipais, por motivos de ordem
pública, incumbidas de promover os meios de bom tratamento dos escravos,
e de evitar as crueldades para com eles, mediante comunicações e
propostas às assembleias provinciais.

Qual foi, entretanto, a câmara ou assembleia que já cuidou, ao menos por
mera formalidade, do desempenho deste sagrado e piedoso dever?

Os vereadores e os deputados, ainda os mais ilustres, nunca leram esta
lei.

Outras acertadas providências, no mesmo sentido, para segurança dos
míseros escravos, restrita observância da disposição da lei, defraudada
por \emph{senhores} ferozes, foram dadas pelo Governo Imperial, em
Avisos 4º e 8º de 11 de Novembro de 1831 (Vid{[}e{]} Legisl{[}ação{]}
Brasil{[}eira{]} - Col{[}eção{]} Nabuco).

Por a Lei de 20 de Outubro de 1823\footnote{. Aprovada no bojo do
  processo constituinte de 1823, esta lei declarava em vigor uma série
  de normas portuguesas que possuíam inquestionável força normativa no
  Brasil até abril de 1821. O art. 1º da lei fazia explícita menção às
  Ordenações como um desses conjuntos normativos que voltavam
  oficialmente a ter vigência no Brasil. Não é evidente, contudo, a qual
  norma, recepcionada pela mencionada lei, Gama fazia referência
  indireta.} foi conferido aos presidentes de províncias o encargo tão
importante quão melindroso e humanitário, de cuidar e promover o bom
tratamento dos escravos. Até hoje, porém, as altas administrações
provinciais, que se ocupam de tudo, inclusive as posturas concernentes
ao lixo e nomeações de oficiais da Guarda Nacional, não desceram às
senzalas, senão para assistir a surras!...

Os \emph{senhores}, cônscios da impunidade, que os distingue, procedem
com desplante e com desbrio.

Eis um fato, \emph{entre muitos semelhantes}, de deslumbradora
eloquência.

Há dias, à rua Vinte e Cinco de Março, no bairro da Figueira, margem do
rio Tamanduateí, nesta cidade, arrabalde frequentado por porcos, bestas
soltas e cães vadios, à noite, foi exposto um menino recém-nascido, de
cor parda, à porta do sr. Porphirio Pires Carneiro.\footnote{. Embora
  existam poucos rastros da biografia de Porphirio Pires Carneiro,
  sabe-se, por esse fragmento, que se tratava de um amigo pessoal de
  Luiz Gama. Os poucos registros assinalam que foi morador da freguesia
  da Sé, comerciante em Santos (SP) e funcionário público em São Paulo.
  Contarei mais da biografia de P. P. Carneiro, como também era
  conhecido, na minha tese de doutorado.}

Este homem, que é maior de 60 anos, e paupérrimo, e que a si tomou a
criação do menor arrancado à morte pelo braço do acaso, é de cor preta,
é afilhado do defunto conselheiro Martim Francisco,\footnote{. Martim
  Francisco Ribeiro de Andrada (1775-1844), natural de Santos (SP), foi
  uma figura proeminente da política brasileira da primeira metade do
  século XIX. Um dos Andradas protagonistas da Independência em 1822,
  foi constituinte (1823), deputado por sucessivos mandatos, presidente
  da Câmara dos Deputados e ministro da Fazenda. É conhecido, também,
  por ser pai de José Bonifácio, o Moço (1827-1886), e Antonio Carlos de
  Andrada (1830-1902), personagens que mantiveram estreita relação
  política e profissional com Luiz Gama.} que o criou em seu lar, que o
educou, entre seus filhos, e que à sua custa fê-lo viajar pela Europa;
tem no porte, e no ânimo a nobre altivez e a inflexibilidade nativa dos
Andradas.\footnote{. Referência a, entre outros Andradas, José Bonifácio
  de Andrada e Silva (1763-1838). Nascido em Santos (SP), José Bonifácio
  passou para a crônica político-histórica como o Patriarca da
  Independência do Brasil. Foi um célebre político, naturalista e poeta
  que exerceu diversos postos-chave na política da primeira metade do
  século XIX, dentre eles o de deputado constituinte em 1823.}

O indigno abandono do menor, criminosamente feito, à sua porta, foi-lhe
causa de insônias; revoltou-o.

-- \emph{Isto é torpeza de branco}, exclamava ele enfurecido, enfiando
os dedos pretos pelos bastos cabelos brancos!

Passou uma semana percorrendo os subúrbios; varejou as vendas, auscultou
pelas quitandas, até que um dia deu com a ponta do fio de Ariadna!...

O enjeitado\footnote{. Diz-se, juridicamente, da criança que foi
  abandonada ao nascer ou em tenra idade.}, aquele inocente mulatinho,
atirado aos cães, é um ingênuo\footnote{. Aqui no sentido de filho/a de
  escravizado/a que nasceu livre.}, filho de uma escrava pertencente a
um negociante rico, que, brutalmente, sem defesa possível, obrigou à
mísera mãe a depô-lo à margem de um rio, exposto às intempéries, às
bestas, às feras, embora mais compassivas do que ele!...

Isto devia ser registrado, comentado pelos meus respeitáveis amigos,
pregadores de política positiva, solertes\footnote{. No sentido de
  espertalhões, ardilosos.} redatores da \emph{Província de S. Paulo};
isto deve ser combatido, com tédio, por todos os honestos altruístas:
isto é o detestável \emph{positivismo} dos abutres, que devoram, por
perversidade, míseros recém-nascidos; isto é a \emph{divinização do
crime}, que tanto repugna à probidade imaculada dos castíssimos
redatores do \emph{Correio Paulistano}; isto seria uma infâmia se não
fora um mau hábito inveterado\footnote{. Arraigado, acostumado.} dos
senhores; é o calo das suas pervertidas consciências, que o
\emph{positivismo} não quer ver, não quer extrair, não quer ponderar,
não quer perceber, não quer discutir; e não considera, e não examina e
não discute, porque este peculiar \emph{positivismo negreiro} é um
sistema exótico de esdrúxula filosofia, foi descoberto entre os hebreus
hodiernos,\footnote{. Modernos.} é uma espécie de \emph{Cyrineu}
moderno; sua moral é singularíssima, sua piedade esquipática\footnote{.
  Estapafúrdia, fútil, e/ou que não é coerente.}: está da parte dos
desgraçados, auxilia com brandura e, com amor, exorta\footnote{. Que dá
  estímulo, incentiva.} os pacientes; ajuda-os a carregar a cruz; rende
preitos\footnote{. Homenagem, tributo.} à Lei; pega as fímbrias da
samarra\footnote{. Nesse caso, barra da veste de um condenado à pena de
  morte.}; abraça o algoz; justifica o suplício; subscreve a condenação;
faz mesuras\footnote{. Reverência, cumprimento cerimonioso.} ao
patíbulo\footnote{. Lugar, geralmente um palanque montado a céu aberto,
  onde se erguia o instrumento de tortura (forca, garrote ou guilhotina)
  para a execução dos condenados à pena capital.}; dá um sorriso a César
e uma lágrima ao penitente.

É um \emph{positivismo cortesão, previdente, que calcula quanto escreve,
que lima quanto diz, porque não fira, que procura agradar a todo mundo,
que, cauteloso, não quer comprometer-se}: enfim, \emph{positivismo de
Convento}.

De tudo quanto vejo e observo, meu caro amigo, não me espanto: o mundo é
uma esfera e a vida o movimento.

Os \emph{senhores} dominam pela corrupção, têm ao seu serviço ministros,
juízes, legisladores, encaram-nos com soberba, reputam-se invencíveis.

A luta promete ser renhida, mas \emph{eles} hão de cair. Hão de cair,
sim, e o dia da queda se aproxima.

A corrupção é como pólvora, gasta-se e não reproduz-se.

Hão de cair, porque a Nação inteira se alevanta; e no dia em que todos
estivermos de pé, os ministros, os juízes, os legisladores estarão do
nosso lado: em Sedan, foram os generais os prisioneiros que se
entregaram, não foram os soldados que desonraram Metz.

Os próprios \emph{senhores} -- na granja, na tenda, na taverna ou no
Senado -- onde, entre anciãos venerandos, tem, infelizmente, entrado
alguns prevaricadores vilões, hão de apertar a mão do liberto, nivelados
pelo trabalho, pela honra, pela dignidade, pelo direito, pela liberdade,
dirão, com o imortal filósofo:

-- "Se fosse possível saber o dia em que se fez o primeiro escravo, ele
deveria ser de luto para a humanidade."

LUIZ GAMA.

\textbf{39. TRECHOS DE UMA CARTA}\footnote{. In. \emph{Gazeta da Tarde}
  (RJ), Noticiário, 01/01/1881, p. 1.}

\textbf{*didascália*}

\emph{Em continuidade ao trecho anterior, Gama segue caracterizando os
senhores como uma espécie de classe social que agia unificada, através
de seus clubes, assembleias, associações secretas e representações
políticas. Pela metáfora, associava a consciência de um fazendeiro ao
"porão de um navio negreiro". Pela análise política, definia "os ricaços
da grande lavoura" como "os legítimos possuidores de Africanos livres,
os consócios da pirataria, os fidalgos do art. 179 do Código Criminal",
a saber, aquele que proibia, em tese, a reescravização de alguém. Para
assegurar seus interesses, ameaçados pela agenda abolicionista, os
senhores andavam se reunindo periodicamente. Bem informado sobre a pauta
que os reunia, mas, sobretudo, sobre a moralidade que os distinguia,
Gama descreve, com a prosa que o consagrou como mestre da narrativa, um
típico encontro senhorial. Da reunião política, em que "conspiram
virtuosamente para manutenção do proveitoso crime", voltavam aos seus
aposentos, dormiam bem, comiam melhor ainda, jogavam seus jogos de
cartas e, ao fim, "mandam surrar os negros, e, quando é preciso, para
disciplina e exemplo, até matá-los..." Isso tudo sem desprezar as
questões de fé. "Os senhores ouvem missa, confessam-se, comungam, limpam
a consciência, vivem na mais estreita intimidade com os padres, com os
juízes e com Deus". É evidente que o sarcasmo funcionava como arma
retórica. Gama transitava por muitos espectros da crítica social e
muitas entradas discursivas. Mestre da narrativa que transita por formas
e discursos. Das questões de fé e consciência, tão vivamente expostas,
retoma o fio da política. Vejamos a crítica acachapante: "Os senhores,
entretanto, habituados a ver somente a cor negra dos seus escravos e a
calcular sobre as arrobas de café, veem, no país inteiro, uma vasta
fazenda; estranham a insubordinação abolicionista e exclamam: 'É preciso
impor silêncio, qualquer que seja o meio, a esta horda de desordeiros; é
preciso acabrunhar, pelo terror, a escravatura, para que não veja com
esperança a propaganda; o bacalhau manterá o respeito e a obediência, a
nossa propriedade será garantida pela força pública, auxiliada pelo
capanga". O trecho é denso. Tem a vantagem, contudo, de sintetizar o
pensamento escravista e suas formas de coerção pelo terror moral e
corporal. }

\emph{***}

S. Paulo, Dezembro de 1880.

Meu caro Menezes.

Escuro, carregado de nuvens plúmbeas, triste, melancólico, indefinido
está o firmamento paulistano.

Os horizontes estreitam-se, sem luzes, o zênite, em trevas, mais se
abate, assemelha-se o espaço à consciência de um velho
pantafaçudo\footnote{. Grosseiro, ridículo.} fazendeiro, espécie de
alcatroado\footnote{. Coberto com alcatrão. Gama indica, por metáfora,
  que a consciência de um senhor de escravizados é uma espécie de
  substância escura que impregna o espaço com mau cheiro e impede a
  passagem de luz solar.} porão de navio negreiro.

Congregam-se taciturnos, em povoações diversas, os ricaços da grande
lavoura, os legítimos possuidores de Africanos livres, os consócios da
pirataria, os fidalgos do art. 179 do Código Criminal; sucedem-se as
assembleias, criam-se os clubes, forjam-se representações e, à sombra da
lei, sem estorvo das autoridades, organizam-se secretamente as juntas de
resistência...

Vociferam contra a \emph{loucura} e a \emph{liberdade}, condenam a
imprudência dos emancipadores e conspiram virtuosamente para manutenção
do proveitoso crime!

Para eles, a lei é um escárnio, um obstáculo negro: uma espécie de
escravo, que se modifica ou que se remove a dinheiro.

Contam com a sábia política dos divinos Bonzos\footnote{. Aqui no
  sentido de indivíduo preguiçoso, medíocre, ignorante.} do Conselho de
Estado, com a eloquência servil de alguns senadores, com a ambição de
certos deputados, com a dependência de eleitores, com a venalidade de
votantes!

Terminadas as reuniões, levantam-se, rezam o credo, dão graças à Divina
Providência, e exclamam, em coro: "Ditoso país, invejáveis instituições,
sapientíssimo governo, abençoado povo!".

Tornam prazenteiros para os seus aposentos, dormem à larga, comem com
satisfação, bebem melhor, jogam o \emph{solo}\footnote{. Antigo jogo
  carteado parecido com o atual truco mineiro.}, o
\emph{pacau}\footnote{. Espécie de jogo de baralho comumente jogado na
  fronteira gaúcha.}, o \emph{lasquinet}\footnote{. Jogo de cartas
  semelhante ao vinte e um.}, mandam surrar os negros, e, quando é
preciso, para disciplina e exemplo, até matá-los...

Depois, o negro, que do burro apenas difere na forma, tem por obra de
misericórdia uma sepultura silvestre no cafezal...

Os senhores ouvem missa, confessam-se, comungam, limpam a consciência,
vivem na mais estreita intimidade com os padres, com os juízes e com
Deus.

Há, porém, em tudo isto um erro de cálculo, uma opinião falsa, uma
imprevisão fatal, que conduz a um abismo inevitável.

Há legisladores sinceros que detestam o enorme crime da escravidão; há,
no país, a grande maioria dos homens livres, cuja vontade é lei
inquebrantável; há uma potência invencível -- a opinião pública -- que,
de há muito, decretou a emancipação; há um ódio latente, misterioso,
indomável, por toda parte, que repele os especuladores de carne humana;
há os abolicionistas pobres, inteligentes, que nunca tiveram escravos,
que amam o trabalho, que tranquilos encaram o sacrifício, que não se
corrompem, nem se vendem.

Os \emph{senhores}, entretanto, habituados a ver somente a cor negra dos
seus escravos e a calcular sobre as arrobas de café, veem, no país
inteiro, uma vasta fazenda; estranham a \emph{insubordinação
abolicionista} e exclamam: "É preciso impor silêncio, qualquer que seja
o meio, a esta horda de desordeiros; é preciso acabrunhar, pelo terror,
a escravatura\footnote{. Aqui não é no sentido usualmente dado de
  sistema, comércio ou tráfico, mas no sentido da condição de
  escravizado.}, para que não veja com esperança a propaganda; o
\emph{bacalhau}\footnote{. Chicote, chibata usada para tortura.} manterá
o respeito e a obediência, a nossa propriedade será garantida pela força
pública, auxiliada pelo capanga."

É, porém, certo que a farda do soldado e o ponche do capanga são duas
causas repugnantes entre si; quem arrisca a vida pela liberdade detesta
a escravidão, a espada cinge o braço leal, o trabuco é o símbolo da
traição. Acredito que os \emph{senhores} nos acometam com os capangas,
mas estou certo que os soldados vencerão com o povo.

Com referência a escritos meus, concernentes à propaganda abolicionista,
insertos em um suplemento da \emph{Província} do dia 18, a
russa\footnote{. Pode ser referência tanto a ``complicada'' quanto a
  ``envelhecida''.} redação do \emph{Correio Paulistano} baixou a
terreiro, cumprimentou a da \emph{Província}, por a desafeição com que
trata aos abolicionistas, censurou-a, porém, \emph{por admitir artigos}
emancipadores, e concluiu com este \emph{trecho de ouro}, escrito com
ponta de vergasta\footnote{. Chicote, vara fina usada para açoitar,
  torturar.} embebida em sangue escravo:

"Aderindo a estas tão sensatas observações, inclusive aquelas em que o
ilustrado colega se refere à grande responsabilidade da imprensa pela
circulação dos tais excessos de verbosidade, viramos a primeira folha do
órgão republicano e, num suplemento, achamos a \emph{divinização do
crime}, esta cousa detestável, como diz o colega, esse excesso de
verbosidade que o colega não duvidou pôr em circulação, apesar da grave
responsabilidade..."

O exmo. dr. presidente da província horrorizou-se ao ler a bárbara
cremação do escravo vivo, de que trato na minha carta precedente.

Aqui transcrevo o resultado das indagações a que S. Excia. mandou
proceder; foi hoje publicado na \emph{Tribuna} \emph{Liberal}; é digno
de nota:

"Sabemos que havendo S. Excia. o sr. presidente da província exigido
informações do juiz de direito da comarca da Limeira\footnote{.
  Município do interior paulista, distante 140 km da capital.} acerca do
fato denunciado em uma carta publicada na \emph{Gazeta do Povo} de 14 do
corrente, respondeu o mesmo juiz que, feitas as necessárias indagações
verbais e por busca em cartórios, estava autorizado a afirmar que
naquele município não se dera o fato descrito na publicação, e nem
passeia ali livremente o responsável por esse ou fato semelhante.
Parece, portanto, ter havido equívoco.

É conhecido um filho do município da Limeira, a quem se atribui o crime
da ordem e gravidade de que se acusou a aludida carta, mas esse
indivíduo mora há muitos anos em outro município, que se diz fora teatro
do crime, e pronunciado, como se acha, por outro delito, tem podido
escapar às diligências para a sua prisão."

Estes e outros fatos que irei relatando servirão de prova irrecusável do
estado de barbaria a que tem atingido o Brasil, corrompido, sem moral, e
sem costumes, pela instituição servil.

Não admira, entretanto, que a escravidão conte esforçados apologistas,
porque o cinismo, com ser torpe, na grande pátria dos imortais helenos,
teve escolas e notáveis cultores.

Há quem louve, com entusiasmo, a extrema bondade de alguns
\emph{senhores} e, por isso, a \emph{felicidade invejável dos seus
escravos}; para mim, os bons senhores são como os túmulos de mármore e a
escravidão é como o raio, que semeia ruínas em sua passagem\footnote{.
  Por extrema precariedade do material consultado, pode-se ter leituras
  diferentes dessa mesma frase. A historiadora Ana Flávia Magalhães
  Pinto leu-a como "a escravidão é como o rato, que semeia ruínas em
  suas passagens". Cf. \emph{Escritos de Liberdade: Literatos negros,
  racismo e cidadania no Brasil oitocentista}, 2018, p. 103. No entanto,
  mesmo admitindo ser essa uma solução possível, não me parece a exata,
  uma vez que a concordância e o sentido da oração, combinados com a
  análise tipográfica da letra "t" e "i" e com o uso singular de
  metáforas naturais próprias do repertório do autor, recomendam que a
  sentença deva ficar grafada como se encontra reproduzida na
  transcrição desse volume.}.

LUIZ GAMA.

\textbf{40. CARTA AO DR. FERREIRA DE MENEZES}\footnote{. In.
  \emph{Gazeta da Tarde} (RJ), 04/01/1881.}

\textbf{*didascália*}

\emph{No trecho precedente, Gama elaborou uma síntese do pensamento
escravista, ilustrando-a com as palavras de um hipotético senhor de
escravizados, que dizia que "a nossa propriedade será garantida pela
força pública". Pois bem. Gama retoma a ideia de direito de propriedade
e avança por aspectos jurídicos da escravidão, disciplina que dominava
como poucos. Vejamos primeiro o que chamava de direito de propriedade de
escravizados. A síntese é lapidar: "Os atuais donos de escravos, que
tamanho alarde fazem do - seu direito de propriedade - são portadores
convictos de documentos falsos, são incapazes de exibir títulos
regulares de domínio. Comprados ou herdados, esses escravos foram
criminosamente constituídos, foram clandestinamente transferidos, são
mantidos em cativeiro por culposo favor, por conivência repreensível de
corrompidos juízes". Portanto, num efeito cascata, a manutenção da
propriedade escravizada se dava por meio de uma espécie de crime
continuado. A escravidão, "essa monstruosidade social", diz Gama noutra
definição avassaladora, "originou-se no roubo, é obra de salteadores, e
para a sua nefasta existência concorreram ministros, senadores,
deputados, conselheiros de estado, magistrados, militares, funcionários
de todas as classes, por interesse próprio, pela desídia, pela
corrupção, pela venalidade". Para justificar seu argumento, o que faz de
maneira sólida, Gama utiliza-se de uma miscelânea de exemplos. Com
admirável visão de conjunto, traz o caso de um inventário com oito
africanos livres que corria no juízo local de Jaú. De Guaratinguetá,
outra ação de inventário, dessa vez com treze africanos livres
ilegalmente escravizados. De Mogi das Cruzes, mais um caso de uma
alforria não reconhecida pelo juiz. Finalmente, de São Paulo, "uma
família de pardos" em que todos nasceram livres e que, sob pretexto
legal fútil, fora posta em escravidão. Por término, o fio da meada da
carta: o contra-ataque aos senhores, ou ainda, à classe senhorial.
Funcionava como um diagnóstico e um aviso. Vejamos: "Desde que uma
classe social, infringindo todos os preceitos de equilíbrio moral,
violando as leis do decoro e as fundamentais do estado, dominando as
forças vivas do país, fez da fraude, da violência, do crime, um meio de
poderio, de vida e de adquirir riquezas, implantou, contra si, os
gérmens de uma revolução tremenda, inevitável, que, lentamente
desenvolvida, aproxima-se ao grave período de perigosa explosão". Se a
explosão não veio, a análise das condições estruturais para a desejada
explosão estava lá. E ler as condições e as circunstâncias, tarefa de
primeira ordem para um líder do calibre de Gama, era um passo
fundamental para a marcha exitosa do movimento social.}

\emph{***}

\emph{S. Paulo, Dezembro de 1880}

Meu caro Menezes.

Os míseros escravos, o milhão de penadas vítimas, atualmente postas na
mais cruciante tortura, para o descanso e ventura de alguns milhares de
empertigados\footnote{. Soberbos, vaidosos.} bípedes, ruflados\footnote{.
  Agitados.} zangões\footnote{. A expressão, popular à época, indicava
  um indivíduo que vive às custas de outra pessoa, explorando de forma
  constante benefícios e favores alheios.}, são incontestavelmente
africanos livres, ou descendentes seus, criminosamente importados no
império, posteriormente à promulgação das leis proibitivas do tráfico.

O que os novos, os sábios, os empelicados\footnote{. Cobertos de
  pelicas, de luxos.} altruístas, os \emph{evangelizadores da evolução
política negreira} chamam, de estufadas bochechas -- \emph{elemento
servil} -- é despido de fundamento jurídico, não tem o mínimo apoio na
lei civil do estado, é um escândalo inaudito\footnote{. Sem precedentes.},
da desídia\footnote{. Negligência, irresponsabilidade. Sem esquecer que
  o autor utilizava termos jurídicos como esse para imputar a culpa
  objetiva na autoria de um crime.}, é o imundo parto do suborno, da
perfídia e da mais hedionda prevaricação\footnote{. Corrupção,
  perversão.}.

Os atuais donos de escravos, que tamanho alarde fazem do -- \emph{seu
direito de propriedade --} são portadores convictos de documentos
falsos, são incapazes de exibir títulos regulares de domínio. Comprados
ou herdados, esses escravos foram criminosamente constituídos, foram
clandestinamente transferidos, são mantidos em cativeiro por culposo
favor, por conivência repreensível de corrompidos juízes.

A inobservância da lei, os desmandos, o crime, já constituem estado
normal. Examiná-los, acusá-los, profligá-los\footnote{. Criticá-los,
  atacá-los.} é excentricidade tão original como o aparecimento de
estrelas ao meio dia.

Quando lavra a imoralidade, quando reina a depravação, quando, com os
bons costumes, a justiça vai caminho da proscrição\footnote{. Extinção.},
dizia Miguel Angelo\footnote{. Provável referência a Michelangelo
  Buonarroti (1475-1564), escultor, pintor, arquiteto e poeta italiano,
  protagonista do Renascimento Italiano e um dos maiores artistas da
  história mundial.}, que os homens honestos devem quedar-se, à margem
das correntes do destino, à semelhança dos marcos de pedra --
\emph{ímobile saxum}\footnote{. Rocha inamovível!}!

Eu, porém, digo que, em tal conjuntura, o silêncio é a coparticipação no
delito e que a revolução é a consciência do dever; os povos adormecidos
e os escravos são como Lázaro: precisam que os ressuscitem.\footnote{.
  Referência a Lázaro de Betânia, personagem bíblico descrito no
  Evangelho de João (11:41-44), que, quatro dias depois de morto, teria
  sido ressuscitado por milagre de Jesus.}

Na Vila do Jaú, e creio que no juízo municipal, corre um inventário, no
qual figuram como escravos oito africanos livres\footnote{. Cf., nesse
  volume, \emph{Fato Grave} \emph{-- Jaú}. Visto pelo conhecimento
  interno do processo, este é mais um indício consistente de que o
  pseudônimo "G.", que assinou o mencionado artigo, fosse de fato Gama.}.

\emph{Um dos coerdeiros} denunciou o fato e, porque não fosse atendido,
delatou-o, pela imprensa da capital, implorou com energia admirável
providências contra esse monstruoso escândalo.

Depois de toda esta celeuma o digno juiz mandou, \emph{prudentemente},
pôr os escravos em custódia, para proceder às necessárias
averiguações!...

A suspeita, a denúncia, o indício, a revelação de que um homem sofre
indevido cativeiro, de que é livre, de que o torturam, é motivo para que
seja suspeitado e, de pronto, posto em segura prisão! Novo modo de
proteger, de garantir o direito!...

A liberdade é um crime, é um atentado de ordem pública, é um descalabro
eminente das instituições pátrias; em falta dos pelourinhos\footnote{.
  Coluna de madeira ou de pedra em lugar público onde criminosos e
  escravos eram expostos e torturados.}, das devassas\footnote{.
  Processo inquisitorial sumário sem direito de defesa e meios de
  contestação.}, do baraço\footnote{. Corda feita de fios de estopa ou
  vergas torcidas, usada para açoitar presos e enforcar réus condenados
  à pena de morte.}, do cutelo\footnote{. Instrumento cortante que
  compreende uma lâmina semicircular e um cabo de madeira, usado
  antigamente em execuções por decapitação.}, do "morra por
êlo\footnote{. A expressão remete às punições elencadas na Ordenações
  Filipinas. Pode ser traduzido como "morra por isso", incluindo desde a
  "morte civil", com banimento e degredo, até a "morte física" por
  enforcamento, decapitação ou incineração.}" para detê-la, para
segurá-la, para comprimi-la, inventaram o \emph{positivismo farisaico},
o cárcere judiciário, a evolução retrógrada, a piedade do servilismo, o
lenitivo do açoite!... \footnote{. Ainda uma palavra sobre o uso da
  expressão "morra por êlo". Aplicá-la aqui demonstra tanto o
  conhecimento da matéria criminal do Antigo Regime quanto o grau da
  crítica do autor ao sistema punitivo do século XIX que,
  paradoxalmente, evoluía retroagindo, é dizer, atualizava mecanismos de
  tortura e castigo que se supunham ultrapassados em pleno século das
  luzes.}

A natureza tem as suas leis, é fatal a sua lógica: os que são indignos
da liberdade desejam a escravidão da humanidade. É a inevitável
conclusão do absurdo, é a filosofia do crime, é a razão da
rapina\footnote{. Roubo.} desde que ela tornou-se potência social e
ascendeu o posto governamental.

Em Guaratinguetá\footnote{. Cidade localizada no vale do Paraíba,
  interior paulista.}, certo fazendeiro declarou, por ato espontâneo em
o seu testamento solene, regularmente disposto, e havido como perfeito,
que comprara e mantinha como escravos seus treze africanos livres e
declinou os seus nomes, para que fossem restituídos à liberdade.

O ilustrado dr. juiz de direito da comarca, em sentença judicial,
declarou que tal verba testamentária era insuficiente; e, por isso,
julgou escravos os africanos livres!...

Isto é digno das páginas da história, isto é incontestavelmente o mais
atrevido altruísmo, o mais esplendoroso exemplo de \emph{justiça à moda
positiva}!...

Isto pareceria inacreditável se a magistratura não fosse o \emph{braço
de ferro} dos \emph{senhores.}

A moral, o direito, a lei, a justiça, estão entregues ao capricho, às
conveniências individuais e inconfessáveis, mutradas\footnote{. Seladas,
  carimbadas.} pela ignomínia\footnote{. Desonra, infâmia.}, ao
arbítrio, à má vontade de juízes, que se incompatibilizaram, de há
muito, com a boa razão.

Isto é pungente para quem o sente, é um vexame para a consciência de
quem pensa, é vergonhoso de proferir-se, mas seria um crime ocultá-lo; é
preciso que todos o leiam, é indispensável que todos ouçam-no, porque a
verdade, como o fel, é o néctar do Calvário\footnote{. Calvário, ou
  Gólgota, é a colina na qual Jesus foi crucificado.}.

Em Mogi das Cruzes\footnote{. Município paulista que hoje pertence à
  Região Metropolitana de São Paulo.}, certo cidadão propôs ação
manumissória\footnote{. Processo em que se demanda a liberdade.} em
favor de um indivíduo, que fora, pelo próprio senhor, alforriado
verbalmente. Falecera o libertador sem que reduzisse a escrito a
concessão. Trata-se, portanto, de prová-la por as fórmulas de um
processo judicial; o juiz indeferiu a pretensão, declarando-a
infringente do direito e contrária às normas de jurisprudência!...

Proibir a propositura de ação!\\
Prejulgar do fundamento da causa!\\
Cogitar do valor de provas antes de aduzi-las!

-- É isto da Beócia, d'outra liça.

Onde os perros\footnote{. Cachorros.} se atrelam com linguiça\footnote{.
  Não encontrei registro de autoria desse verso, citado no meio de uma
  argumentação, como era próprio do estilo do autor. Assim, pode-se
  conjecturar que ele o tenha lançado originalmente, sem com isso
  descartar que outro autor ou mesmo que o domínio público de alguma
  região o tenha em conta.}.

Tudo isto são frutos envenenados da perniciosa influência
dominical\footnote{. Religiosa, católica.}, são consequências de grandes
crimes passados, inultos\footnote{. Impune, não vingado.}, que
sinistramente invadem e infeccionam a sociedade hodierna.\footnote{.
  Moderna.}

Queres um exemplo do que foram os traficantes da carne humana?

Eles não se limitavam à revenda de africanos livres, de \emph{negros}
vindos de outra parte do mundo; escravizavam brasileiros, nascidos neste
mesmo solo!

Há, nesta cidade, uma família de pardos, nascidos na vila de Santa
Branca -- um deles é um artista distintíssimo, é um cidadão considerado,
é um homem de bem --, aos meus labores judiciários devem eles o gozo da
sua liberdade; dela faltava-me apenas uma rapariga, cujo senhor acabo de
descobrir no interior da província, pela mediação de um lidador
dedicado.

Essa família, composta de pessoas QUE NASCERAM LIVRES, foi conquistada a
pretexto de cobrança de dívida; e, logo depois, alienada, por um certo
comendador que houve em Jacareí, contrabandista de fama, muito rico,
poderoso, grande proprietário, temido, mais do que respeitado, nunca
vencido, e sempre em tudo vencedor.

Ainda uma recordação do passado e uma referência para terminar.

Cedo a palavra a um velho estadista, de elevada probidade; é a
transcrição de um trecho de uma carta sua:

"Rio, 1º de Maio de 1879\footnote{. A carta é, como ele corrige à
  frente, do ano de 1869.}

Sr. Comendador José Vergueiro\footnote{. Nicolau José de Campos
  Vergueiro (1824-1903), o filho, natural de Piracicaba (SP), foi um
  grande fazendeiro estabelecido em Limeira (SP) que teve o
  protagonismo, entre os cafeicultores paulistas, de propor a
  substituição da mão de obra escravizada pela mão de obra livre e
  estimular a imigração europeia para o Brasil, já na década de 1860.}.

Como o sr. conselheiro Nabuco, na carta que me dirigiu, e que lhe envio,
menciona um ato do meu ministério, em 1848, parece-me conveniente
dizer-lhe algumas palavras, que o expliquem; e o faço com tanto maior
prazer quanto é certo que os acontecimentos que lhe sobrevieram servem
de contraprova a esse feliz sucesso da ilha de Reunião, e plenamente
confirma a asserção de que -- quando a corrente dos acontecimentos não é
dirigida com cautela e prudência, nunca deixa de ser fatal à ordem
pública e à economia social.

Em maio de 1848, ocupando eu a pasta da justiça, procurei, \emph{por
meios persuasivos}, fazer compreender aos principais contrabandistas de
africanos, \emph{que era chegado o momento} (!!!) de tomar-se
providências para cessações {[}sic{]} do tráfico, \emph{que, então, se
fazia publicamente} (!!!). A resposta foi UM RISO DE ESCÁRNIO. Estavam
eles no auge da influência, e, cegos pelo interesse, não viam o abismo
que se lhes abria debaixo dos pés.

Um dia, estando eu na Câmara dos Deputados, entrava pela barra deste
porto um vapor com africanos.

Era demais. Dali mesmo escrevi ao presidente da província do Rio de
Janeiro, o visconde de Barbacena, que os mandasse apreender. A ordem foi
imediatamente cumprida. Não se pode hoje fazer ideia da tempestade que
produziu esse primeiro ato de repressão.

Unidos aos conservadores, os contrabandistas deram batalha ao governo
nas tormentosas eleições de setembro deste ano e, tão forte se tornou a
oposição, principalmente nas altas regiões, entre as personagens daquela
época, que o ministério baqueou a 29 desse mesmo mês, apesar da imensa
maioria que o sustentava na Câmara, que foi dissolvida\footnote{.
  Refere-se ao Poder Moderador. Por disposição constitucional, facultava
  ao monarca dissolver a Câmara dos Deputados quando bem lhe conviesse.}.
Os contrabandistas e seus aliados bateram palmas de contentes: seu
triunfo era completo, mas, infelizmente para eles, e felizmente para o
país, não foi de longa duração. Aquilo que não quiseram fazer por bem,
foram obrigados a fazer por mal. Todos nós recordamos, com verdadeira
mágoa, do modo porque os vasos\footnote{. Navios.} de guerra de sua
majestade britânica procederam em Campos, Cabo Frio, na barra mesmo
deste porto, Paranaguá, etc., etc., e das deportações que o ministério
que nos sucedeu foi obrigado a fazer dos seus aliados da véspera; e dos
processos que mandou instaurar \emph{contra alguns dos nossos principais
fazendeiros}, precedidos de buscas, varejos à mão armada, prisões, etc.,
etc.

A humilhação que então sofremos foi e será eternamente lamentável. Por
culpa de quem?

É, pois, evidente que tais excessos teriam sido evitados se aquelas
medidas de prevenção, tratadas oportunamente, fossem sustentadas pelo
povo e pelos próprios que, até então, se tinham envolvido no tráfico.

Não teríamos sido humilhados, nem eles deportados. Não se teria por essa
causa escoado do império imenso cabedal: \emph{e a obra inevitável,
civilizadora, e cristã da emancipação} estaria presentemente muito
adiantada, se não quase concluída.

Não tememos novos ultrajes daquela natureza, eu o creio: \emph{mas se
não começarmos já} (em 1869) essa obra de regeneração social, podemos
estar certíssimos de que seremos, \emph{em breve, forçados, por qualquer
modo, que desconhecemos, a fazer aquilo que é do nosso rigoroso dever
não retardarmos por mais tempo}".

CAMPOS MELLO\footnote{. Antonio Manuel de Campos Mello (1809-1878) foi
  político e presidiu as províncias de Alagoas (1845-1847) e do Maranhão
  (1862-1863).}

Aqui tens, meu distinto amigo, em sucinto quadro, uma vista das
desgraças do passado, ornado com as cores violáceas das misérias do
presente, aqui verás as causas do desespero nacional; os elementos de
uma reforma ambicionada, inevitável, pronta, criteriosa, profunda, ou,
se o quiserem, os motivos, a justificação da desconfiança, as cóleras
exuladas, e até a revolução, que é sempre feitura dos maus governos.

A escravatura, essa monstruosidade social, não tem aqui uma causa
política que a justifique; originou-se no roubo, é obra de salteadores,
e para a sua nefasta existência concorreram ministros, senadores,
deputados, conselheiros de estado, magistrados, militares, funcionários
de todas as classes, por interesse próprio, pela desídia, pela
corrupção, pela venalidade.

O presente é a reprodução tristíssima do passado, com algumas
modificações intrínsecas.

Pretendem alguns especuladores que o futuro seja a dedução rigorosa ou a
soma destas duas épocas; enganam-se: o futuro será uma nova era, o
resultado de uma memorável convenção ou de uma grande catástrofe; os
sucessos resultam das circunstâncias, estas têm a sua origem nas
variedades do tempo.

Como os barões da Idade Média, hão de cair os landlords\footnote{.
  Senhores de terras e escravos.}.

Desde que uma classe social, infringindo todos os preceitos de
equilíbrio moral, violando as leis do decoro e as fundamentais do
estado, dominando as forças vivas do país, fez da fraude, da violência,
do crime, um meio de poderio, de vida e de adquirir riquezas, implantou,
contra si, os gérmens\footnote{. Estágio inicial do desenvolvimento de
  um organismo.} de uma revolução tremenda, inevitável, que, lentamente
desenvolvida, aproxima-se ao grave período de perigosa explosão.

As leis sociológicas não estão sujeitas às especulações humanas; como as
leis físicas têm períodos de ociosidade, de desenvolvimento e de
substituição: como o Sol tem o seu ocaso; e o Sol quando o atinge, "vai,
por entre nuvens atrás, envolvido em manto de púrpuras..."

Teu\\
LUIZ GAMA

\textbf{41. {[}CARTA AO DR. FERREIRA DE MENEZES{]}}\footnote{. In.
  \emph{Gazeta da Tarde} (RJ), Gazeta da Tarde {[}editorial{]},
  07/01/1881, p. 1.}

\textbf{*didascália*}

\emph{Fazendo referência a um trecho de uma carta anterior, Gama volta
ao tema do debate e da aprovação da Lei de 1871, reafirmando que tal lei
"não satisfaz as justas aspirações abolicionistas do país". Como de
costume, trazia um fato, processo e/ou documento para discutir e dar
suporte ao seu argumento. Nesse caso, torna a falar da agenda política
dos senhores de escravizados. Porém, estrategicamente escapando do
noticiário do calor da hora, volta ao paradigmático ano de 1869, no
contexto das discussões sobre a Abolição ainda durante o gabinete
Zacarias. Os fazendeiros de Limeira (SP), naquela oportunidade,
começaram a discutir condições e o tempo para o fim da escravidão. Para
representar seus interesses, fundaram uma associação e pretendiam
influenciar os debates sobre o tema na Câmara dos Deputados. Nesse
sentido, a associação redigiu um "projeto de lei para emancipação do
elemento servil", em que havia a previsão de que, antes da emancipação
geral, os escravizados deveriam ser matriculados, isto é, possuir uma
espécie de cadastro para que se pudesse fiscalizar a legalidade da
propriedade escravizada. No entanto, o estatuto vinculava essa matrícula
a uma verificação do domínio, ou seja, um dispositivo que atestasse se a
escravização era regular de direito. Ocorre que, com o debate da matéria
na Câmara dos Deputados, tal vinculação caiu perdida e não entrou no
texto legal. Gama interpretou as razões disso ter acontecido. "Esta
salutar verificação", dizia Gama, "se não fosse maliciosamente alterada
pelo Poder Legislativo e pelo governo, daria causa à manumissão de todas
as pessoas ilegalmente escravizadas: evitaram-na; armaram um laço, uma
emboscada, por a qual a fraude está, de contínuo, cometendo impune os
mais horrendos crimes!" Os autores da emboscada e da fraude tinham, mais
uma vez, culpa no cartório (e no parlamento): eram os "senhores, dos
réus de crime de roubo".}

\emph{***}

S. Paulo, Janeiro de 1881.

Meu caro Menezes.

Em uma carta precedente, que tiveste a bondade de estampar na
conceituada \emph{Gazeta da Tarde}, eu disse que a Lei de 28 de Setembro
de 1871 já não satisfaz as justas aspirações abolicionistas do país;
pretendo agora, se me o permitires, justificar este meu asserto,
mediante exibição de prova irrefutável.

O povo, ativo, inteligente, nobre, refletido, altivo, ordeiro,
oberado\footnote{. Onerado, carregado.} de labores, continuamente a
braços com as necessidades múltiplas que o aturdem, vencedor em todas as
dificuldades, irônico diante das desgraças próprias, compassivo, piedoso
para com as alheias, magnânimo para com os governos violentos,
compressivos\footnote{. Opressivos.} e desorganizadores dos seus
direitos; o povo, Atlas\footnote{. Referência a Atlas, o titã da
  mitologia grega condenado por Zeus a sustentar os céus em seus ombros.}
dos tempos modernos, sem fábula, sem figura, entidade que assombra, e
faz estremecer tiranos, tem poucos ócios para dispensá-los a leituras
detidas, estudadas, profundas, ou meditadas: foi por isto que, para ele,
criou-se uma leitura especial, fácil, cômoda, mais deleitável que
instrutiva, mais agradável que trabalhosa -- a dos \emph{periódicos}, a
do \emph{jornal}.

Do que se pensou, do que se disse, do que se escreveu, do que se ouviu,
do que se leu há 10 anos, poucos se lembram já, poucas memórias o
registraram, e menos ainda o conservam.

Este grande livro in-fólio\footnote{.
  Folha~de~impressão~dobrada~ao~meio~de~que~resultam~cadernos~com~quatro~páginas~(no
  contexto, páginas de 22 cm x 32 cm).}, em folhas esparsas, meditado,
composto, escrito, impresso e publicado da noite para o dia, que sobra
em todas as casas, que falta em todas as estantes, e que desaparece com
a mesma rapidez, do dia para a noite, tem ainda outra vantagem de
incontestável proveito, repete-se, reproduz-se, sem aumento de preço,
para os consumidores, sem prejuízo de tempo, de trabalho e de atenções.

Em 1869, depois daquelas memoráveis palavras que o conselheiro Zacharias
sabiamente inseriu na \emph{Fala do Trono}, e da consulta feita ao
Conselho de Estado, que, sem prudência, repeliu-a, agitou-se o país,
pronunciou-se validamente a opinião pública, fez-se a luz relativamente
à emancipação da escravatura.

Os agricultores, notadamente os desta heroica província, viram manchas
no horizonte; ergueram-se, pensaram; e, embora eivados de preconceitos,
aliás, destrutíveis, elevaram-se à altura da grande ideia. Reuniram-se,
por iniciativa própria, entenderam-se, discutiram, constituíram a
importante -- ASSOCIAÇÃO DEMOCRÁTICA CONSTITUCIONAL LIMEIRENSE\footnote{.
  Organização sediada em Limeira, município do interior paulista,
  distante 140 km da capital.} --, isenta do vírus partidário, e do seu
sexo, informe\footnote{. Disforme, grosseiro, grotesco.}, irregular,
defeituoso saiu o importante projeto da Lei nº 2.040 de 28 de Setembro
de 1871. É preciso que o povo o releia; que o confronte criteriosamente
com algumas disposições da lei, e que note, que admire as fraudes
cometidas no parlamento pelos legisladores.

PROJETO PARA A EXTINÇÃO DO ELEMENTO SERVIL NO IMPÉRIO DO BRASIL

Art. 1º. Do dia 1º de Janeiro de 1880 em diante o ventre escravo será
declarado livre em todo o Império do Brasil.\\
Art. 2º. Do dia 1º de Janeiro de 1904 em diante será proclamada a
liberdade geral dos escravos no Império.

Art. 3º. Os poderes competentes farão baixar as leis e regulamentos
necessários para a realização desta emancipação sob as seguintes
bases:\\
§ 1º. O governo mandará desde já abrir em todos os municípios a
matrícula dos escravos existentes com a declaração do nome, sexo, idade,
estado, ofício, cor e sob que título de domínio é possuído cada um. Esta
matrícula se repetirá todos os anos na mesma época.

§ 2º. A lista municipal das matrículas será remetida aos juízes de
direito das respectivas comarcas, que formarão, em resumo, um mapa
estatístico, e enviarão ao presidente da província.\\
§ 3º. Aberta a referida matrícula nos municípios, cada proprietário é
obrigado a exibir uma relação de seus escravos com as declarações do §
1º.

§ 4º. O escravo que não for dado a matrícula, por culpa ou malícia do
seu proprietário, \emph{ipso facto}\footnote{. Necessariamente, pelo
  próprio fato.}, será declarado livre.\\
§ 5º. O proprietário, no ato da entrega da relação dos seus escravos
para a matrícula, receberá em troca um conhecimento ou nota declarativa
do nome, idade, sexo, naturalidade, estado, cor, ofício, e sob que
título são possuídos. Este conhecimento será rubricado pelo agente e
escrivão da repartição municipal encarregada da matrícula e servirá de
título legal de propriedade dali em diante.

Art. 4º. O governo criará estabelecimentos agrícolas e industriais para
receber o fruto do ventre livre.

§ 1º. Os nascidos depois de 1879 serão criados e alimentados pelos
proprietários até a idade de 8 anos, idade esta em que serão recolhidos
para os ditos estabelecimentos, recebendo em troca uma apólice do
governo do valor de quinhentos mil réis, de seis por cento ao ano, e os
nascidos de 1893 em diante devem ser recolhidos em 1901 a
estabelecimentos de caridade mediante a indenização proporcional.

§ 2º. As crianças recolhidas para estes estabelecimentos serão aí
conservadas na aprendizagem e nos labores próprios de sua idade até
completarem 13 anos, e então seus serviços contratados por conta dos
mesmos estabelecimentos, e assim servirão até perfazerem a idade de 21
anos, idade em que poderão trabalhar no que lhes convier como homens
livres que são.

§ 3º. Os escravos que não forem apresentados à matrícula na forma do
art. 3º, embora considerados livres pela força do § 4º do mesmo artigo,
serão apreendidos e recolhidos aos mencionados estabelecimentos, e aí
trabalharão sob contrato até o dia 1º de Janeiro de 1901, época em que
seguirão a carreira que lhes convier.

Art. 5º. Encerrada a matrícula, toda e qualquer transferência de domínio
de escravos será nula, desde que se não faça acompanhar de prova
autêntica de matrícula ou do conhecimento de que fala o § 5º do art.
3º.\\
Art. 6º. Todos os proprietários de escravos são obrigados a participar
dentro em 30 dias à agência municipal da matrícula o óbito e o
nascimento dos seus escravos.

§ 1º. Os que incorrerem em falta perderão o direito de propriedade sobre
o escravo nascido, e a indenização de que trata o § 1º do art. 4º, se
for recolhido aos estabelecimentos do governo, mesmo os de caridade.

No caso de morte não fazendo a participação de que trata o artigo
precedente será o proprietário responsabilizado perante os tribunais do
país.

§ 2º. O proprietário que, dando parte do emancipamento\footnote{. O
  mesmo que emancipação.} de um escravo, mostrar que o libertou na pia
batismal, poderá gozar de seus serviços até a idade de 15 anos, sendo,
porém, obrigado a mandar-lhe ensinar, escrever e contar.\\
Art. 7º. No dia 1º de Janeiro de 1901 todos os proprietários levarão às
repartições respectivas o conhecimento legal que prove a existência de
escravos que ainda possuem, e pelos seus valores obterão uma indenização
proporcional.

§ 1º. Para esta indenização se procederá a uma avaliação em que seja
representado o interesse particular por um louvado\footnote{. Avaliador,
  perito, especialista nomeado ou escolhido pelo juiz para dar parecer
  técnico.} de sua escolha, e o da fazenda pelo seu respectivo fiscal,
ou seus delegados, com recurso aos chefes das tesourarias, ou seus
agentes.\\
§ 2º. Servirá {[}sic{]} de base para as ditas avaliações, a idade e o
sexo, e atendendo-se ao valor atual, para conhecimento do que o governo
mandará formar uma tabela do termo médio pelo qual foram vendidos no ano
de 1868.

§ 3º. Para criação de fundos para esta indenização será levantado, desde
já, um imposto anual de 3\$000 por cabeça de escravo.

A soma arrecadada será recolhida para bancos territoriais, os quais se
encarregarão da referida indenização, e só poderão fazer empréstimos à
lavoura diretamente.

§ 4º. O governo por seus regulamentos garantirá e resguardará o
interesse desses bancos, estatuindo sobre o modo e condições do
empréstimo, e favorecendo as necessidades da lavoura.\\
Art. 8º. Será promulgada uma lei sobre o trabalho livre com juízes
especiais, processo verbal e sumaríssimo, grátis, onde fiquem claras e
definidas as obrigações do locador e locatário, derrogando-se as duas
leis de 1830 e 1837, que por obscuras e não interpretadas têm tornado da
sua execução um caos para as partes que litigam, e um labirinto para os
jurisconsultos que as compulsam\footnote{. Estudam, examinam.}.

§ 1º. Abrir-se-á uma matrícula em a qual se inscreverão todos os
trabalhadores livres, sem propriedade, com declaração do nome, sexo,
idade, estado, cor, nacionalidade e emprego que têm. Na ocasião da
matrícula receberão uma papeleta, sendo obrigados a vir declarar à
matrícula qualquer mudança de estado e de emprego.

§ 2º. Os que incorrem em falta serão multados em \$ ou coagidos a pagar
esta multa pelo valor do trabalho em obras públicas.\\
§ 3º. Na mesma repartição desta matrícula haverá um livro de registro
onde serão registrados todos os contratos dos trabalhadores livres. Sem
estes registros de contratos serão nulos.

§ 4º. Os juízes especiais do trabalho livre julgarão sem demora, dando a
sua decisão na mesma audiência do processo. Não haverão embargos nestas
causas, nem mesmo os à execução. Haverá apelação para os juízes de
direito que também decidirão em termo breve.\\
§ 5º. De seis em seis meses se reunirá um júri em cada município,
composto de dois cidadãos chãos, e abonados do lugar, e o juiz especial
do trabalho livre, onde poderão ser apresentados os contratos de
trabalho livre a fim de serem examinados aqueles a respeito dos quais
alguma das partes se julgue lesada. O júri fará com que os contratos
lesivos sejam corrigidos e emendados na forma da lei. Os dois cidadãos
membros do júri darão o seu voto a respeito, e o juiz especial,
presidente do júri, terá o seu voto de qualidade. O presidente lançará
nos contratos o seu -- visto --, que será rubricado pelos três membros
do júri. Desta decisão não haverá recurso algum.

Salva a redação.

Limeira, sala das sessões da Sociedade Democrática Constitucional
Limeirense, em 1º de Janeiro de 1869.

JOSÉ VERGUEIRO\footnote{. Veja nesse volume a carta que Gama endereçou a
  Vergueiro, justamente no ano de 1869, na qual também discute o projeto
  de abolição da Sociedade Democrática Constitucional Limeirense.
  Nicolau José de Campos Vergueiro (1824-1903), o filho, natural de
  Piracicaba (SP), foi um grande fazendeiro estabelecido em Limeira (SP)
  que teve o protagonismo, entre os cafeicultores paulistas, de propor a
  substituição da mão de obra escravizada pela mão de obra livre e
  estimular a imigração europeia para o Brasil, já na década de 1860.}.

Antes de analisar as disposições de uma lei manda a boa filosofia
estudar as causas essenciais ou imediatas da sua promulgação; porque uma
lei é um monumento social, é uma página de história, uma lição de
etnografia, uma razão de estado.

Tais causas podem ser a consagração dos interesses gerais do país;
atendendo-os o legislador a lei é uma satisfação devida a justas
reclamações nacionais; mas se, pelo contrário, ela é adotada por
imposições egoísticas de uma classe, de um partido, de uma facção, para
lisonjear as suas ambições privadas, constitui um atentado latente,
encerra o gérmen\footnote{. Estágio inicial do desenvolvimento de um
  organismo.} de futuros desequilíbrios políticos, a causa de
protestações veementes e de vindictas perigosas.

Uma lei semelhante é mais do que um erro de governação; é uma inépcia
indesculpável; é um canhão assestado\footnote{. Apontado, direcionado.}
contra a soberania popular.

Tal é a Lei de 28 de Setembro de 1871.

Os legisladores acharam-se entre o patriotismo e as conveniências
transitórias; entre o dever e os seus interesses políticos; entre o
direito e o crime: iludiram ambas as partes!...

O projeto da ASSOCIAÇÃO LIMEIRENSE, cuja honestidade não pode ser posta
em dúvida, foi confeccionado com habilidade notável; serviu de
conselheiro o sobressalto, escreveram-no entre a prudência e o calculado
patriotismo, à sombra da piedade, para acautelamento de futuros e
complicados interesses...

Certo é, porém, que nesse projeto, a SOCIEDADE LIMEIRENSE, no artigo 3º,
§ 1º, estabelecendo a matrícula especial dos escravos, incluiu uma
medida administrativa do mais elevado alcance político; EXIGIU A
VERIFICAÇÃO DA CAUSA DO DOMÍNIO. Esta salutar \emph{verificação}, se não
fosse maliciosamente alterada pelo Poder Legislativo e pelo governo,
daria causa à manumissão de todas as pessoas ilegalmente escravizadas:
evitaram-na; armaram um laço, uma emboscada, por a qual a fraude está,
de contínuo, cometendo impune os mais horrendos crimes!...

O Decreto nº 4.835 de 1º de Dezembro de 1871, capítulo 1º, foi
propositalmente escrito para ressalva do crime.\footnote{. Para execução
  do art. 8º da Lei do Ventre Livre, o decreto definia o regulamento
  para a matrícula especial dos escravizados e dos filhos da mulher
  escravizada. Por sua vez, o capítulo 1º do decreto determinava o modo
  como se daria a matrícula, a exemplo da obrigatoriedade de registro de
  nome, sexo, cor, idade e profissão do escravizado matriculado. No
  entanto, como bem verifica Gama, "propositalmente" não se lê no
  decreto qualquer obrigação de verificação da causa do domínio.}

Os fazendeiros paulistas aconselharam a \emph{retificação dos títulos}:
os poderes do Estado, que tratavam de abolir a escravatura,
proibiram-na!...

O crime protegido pela lei; os salteadores autorizados a fazer
matrículas, sem títulos; as vítimas do delito sacrificadas pelos
legisladores!...

E quando examinamos estes fatos, quando esmiuçamos estes dolos, quando
averiguamos destas simulações, quando condenamos estes
dislates\footnote{. Bobagem, estupidez.}, quando, em nome da lei
violada, pedimos, reclamamos a manumissão dos desgraçados, surgem
vestidos de gala os divinos positivistas aconselhando-nos prudência,
advertindo-nos, em nome dos interesses do Estado, pregando a submissão
dos aflitos, e desculpando, e justificando, e santificando as culpas dos
\emph{senhores, dos réus de crime de roubo}, que têm direito ao fruto da
sua rapina\footnote{. Roubo.}; porque a escravidão deve ser abolida
suave, branda e docemente, ao som delicioso da vergasta\footnote{.
  Chicote, vara fina usada para açoitar.}, por efeito benéfico do
\emph{bacalhau}\footnote{. Chicote, chibata usada para tortura.}, e com
o lento desenvolvimento das leis sociológicas!...

Ah! meu caro amigo, isto seria a triste manifestação da filosofia da
miséria, se não revelasse, tão às claras, as misérias filosóficas dos
positivistas.

Teu\\
LUIZ GAMA.

\textbf{42. CARTA AO DR. FERREIRA DE MENEZES}\footnote{. In.
  \emph{Gazeta da Tarde} (RJ), 22/01/1881.}

\textbf{*didascália*}

\emph{"Lê, examina, admira!" Luiz Gama convoca os leitores a sorrirem
piedosos com ele. É uma tarefa e tanto nos transportamos um tiquinho que
seja ao sentimento que o abolicionista negro manifestou quando leu uma
provocação barata na imprensa. Da provocação cínica, contudo, ele sacou
da manga mais uma réplica aos ofensores que seria histórica. Seria mais
uma página memorável da luta pelo direito e pela liberdade. Como fio
condutor, a crueldade humana, manifestada com notável requinte entre os
poderosos da terra. A narrativa que constrói é de transtornar o leitor.
As referências, o suspense, o estilo, as conclusões... Gama crescia a
cada carta e mostrava o domínio da escrita para muito além da técnica
jurídica, da redação jornalística, da poesia satírica ou da crítica de
costumes, entre outros gêneros textuais a que se dedicou na imprensa.
Numa trama muito bem arranjada, Gama conecta histórias aparentemente
improváveis, como a de um médico-legista em Paris, uma jovem fidalga
paulistana, um padre e senador cearense, um homem escravizado em Minas
Gerais e dois nobres da linha sucessória da monarquia inglesa. Dos casos
que traçou, uma conclusão em comum: "o caráter, a posição do autor
determinam a razão do fato". Não era a lei, o tipo criminal ou as provas
de um processo. Era a autoria. A "posição do autor" determinaria a
coerção legal ou extralegal. Só um dentre os rapidamente mencionados
personagens era negro. Ao se ver a coerção que recebeu, sem recurso,
apelação ou súplica que intervisse na situação, Gama exclamava: "o seu
autor, porém, é um negro!", o que significava dizer que jamais se
poderia esperar resposta semelhante se a cor do autor fosse branca.
Certamente muito tribunal do júri e muita leitura sociológica o fizeram
chegar a essa espécie de realismo jurídico original já nos anos 1880.
Tudo isso, igualmente certo, organizado numa prosa literária de primeira
grandeza. "Fica de pé uma entidade", finalizaria Gama, "é o assassino do
senhor; é a imagem da miséria; é a Sephora dos tempos modernos; é o
leproso social: é o escravo homicida. E arremata: "Tem uma escola -- a
senzala; tem um descanso -- o eito; tem um consolo -- a vergasta; tem um
futuro -- o túmulo". }

\emph{***}

S. Paulo, 17 de Janeiro de 1881.

Meu caro Menezes.

Bem longe estava eu, hoje, de escrever-te estas linhas.

Precisava de algum repouso, e próximo julgava-me de aproveitá-lo, quando
um amigo indignado despertou-me a atenção, mostrando-me os disparates
originalíssimos, que passo a transcrever da \emph{Província} do dia 13.

Sorri-me piedoso quando os li; mas confesso que não é sem tédio que as
translado.

Lê, examina, admira!

É MAIS UM MOTE PARA UMA CARTA.

Li nesta folha uma notícia, tirada do \emph{Colombo}, jornal onde peleja
o Briareu da democracia brasileira -- Lucio de Mendonça.

Notícia muito simples e natural.

É nada mais e nada menos do que um escravo, pondo em exercício o seu
direito sagrado de defesa, matando muito simples e naturalmente um
inocentezinho, \emph{seu amigo}, \emph{para assim recuperar a sua
liberdade}, que também é sagrada é um direito absoluto e muito mais do
que o é a propriedade, que é também o nosso suor, o nosso sangue, e
nossa vida...

Não é assim mesmo, senhores abolicionistas humanos, filantropos,
cristãos, e até católicos \emph{tutti quanti}\footnote{. Do italiano,
  "todos eles".}!

.............................................................................................

Sim! Sim! É mais um futuro Espártaco\footnote{. Espártaco (109 a.C-71
  a.C.) foi um gladiador-general, estrategista e líder popular que
  escapou da escravidão a que era submetido e, num levante de grandes
  proporções, organizou um exército que enfrentou o poder central de
  Roma na Terceira Guerra Servil (73 a.C-71 a.C.). São diversas as
  citações de Gama a Espártaco, grafado de variadas maneiras, a exemplo
  de Spartacus, o que revela sua admiração e até mesmo veneração pela
  história do mártir que venceu o cativeiro e lutou pelo fim da
  escravidão.}, que segue caminho a sublimar-se pelo martírio, em
apoteose.

Que! Horrorizaram-se os homens de que um escravo mate um inocentezinho
seu amigo, somente por ser este parente do seu senhor!

\emph{Qui non temperet a lacrimis}.\\
E...\\
Mãos à obra, meus senhores. O porvir é vosso.\\
É mais uma carta -- para um novo atentado.

\emph{Proudhon}\footnote{. O leitor mais detalhista pode ter reparado,
  com razão, que esse \emph{Proudhon} não tinha nada a ver com José do
  Patrocínio, notório subscritor desse pseudônimo que remete ao
  conhecido ideólogo anarquista francês. Mais: o leitor pode ter
  reparado que \emph{Proudhon} respondia indiretamente ao artigo de Gama
  publicado na \emph{Gazeta da Tarde} em 16/12/1880, texto em que Gama
  defende que "quatro Espártacos (...) sublimaram pelo martírio, numa só
  apoteose". Nesse mesmo texto, Gama questionou: "Quê! Horrorizam-se os
  assassinos de que quatro escravos matassem seu senhor?"
  \emph{Proudhon,} como se vê, deixou pistas explícitas de que leu e que
  respondia Gama. Sobre o ideólogo francês: Pierre-Joseph Proudhon
  (1809-1865), nascido em Besançon, França, foi tipógrafo, escritor,
  político e filósofo anarquista. Foi membro do parlamento francês e
  publicou obras sobre teoria política, propriedade e autogoverno.}.

***

Que felicidade na reprodução dos fatos!\\
Que perspicácia no exame, que critério na escolha!\\
Que filosofia nas observações!\\
Que semelhança, que confronto, e que conclusões!\\
O mundo comparado a um espeto, o raio ao espírito, a tempestade à tosse!

O filósofo é um arroteador\footnote{. Aquele que arroteia, que lavra
  terra inculta, que desmata terreno para nele semear.}; a lógica um
alvião\footnote{. Tipo de enxada, instrumento com uma lâmina de ferro
  usado para cavar terra dura e arrancar pedras.}!\\
Ao que vem a calculada reprodução desta ocorrência?\\
Pretendem, por ela, embargar o passo à propaganda abolicionista?

Quererão, com este fato, justificar e perpetuar a escravidão?

\emph{Proudhon} ou não entende o que lê, ou não sabe o que escreve, ou
ignora o que pretende, ou, como aquele memorável fragmento de antiga
estátua de gladiador acantonado\footnote{. Estabelecido.} em uma praça
de Roma, no ângulo do palácio dos Orsini, insoante {[}sic{]}\footnote{.
  Pelo contexto, possivelmente seria a palavra insinuante, grafada à
  época como "insinoante".}, inconsciente, serve de muro novo às
diatribes\footnote{. Crítica mordaz, discussão exaltada.}
insulsas\footnote{. Insípidas, enfadonhas, que não têm graça alguma.}
dos despeitados salteadores da liberdade.

Uma vez, porém, que nos pretendam dar lições, que apelam para os
exemplos, que servem-se dos escândalos, das misérias, dos desastres, das
aberrações, sejam coerentes, aceitem as retaliações, sofram as
retesias\footnote{. Contenda, disputa.}.

***

Em Paris, afirma Foublaque, conceituado médico legista, Lavarde, um
excelente velho de 70 anos, trêmulo, de contentamento, acolhe risonho,
em seus braços, um lindo neto recém-nascido: mira-o, afaga-o com
extremos, beija-o. Dá-se um momento de silêncio. O velho, de repente
pega do inocente pelas pernas; maneja-o rápido pelo ar; bate-o com o
crânio de encontro ao tronco de um carvalho!...

Este velho era inteligente, ilustrado, nobre, de apurada educação, de
excelentes costumes, de elevado conceito, de provado merecimento!...

Por que cometeria ele este atroz delito, que, por a sua enormidade,
excede à coerção de todos os códigos?!...

A fisiologia explica-o; por que, nos fenômenos da vida humana, para a
ciência, não há mistérios.

Este homem faleceu dois meses depois deste horrível sucesso,
repentinamente, vítima de uma lesão cardíaca.

***

Na cidade de S. Paulo (esta mesma em que escrevo), em o ano de 1831...
uma jovem formosíssima enfermou.

Era conhecida a moléstia? Era grave? Era natural? Seria fruto de um
crime?

Sei, apenas, que ela era fidalga, de família altiva e preponderante.

Com alguns parentes, foi para um subúrbio tomar ares. Nenhum médico a
viu; nenhum médico acompanhou-a.

Uma negra velha, escrava fiel, confidente extremosa, era sua companheira
inseparável; e velava por ela dia e noite: os negros, os indignos, os
miseráveis, os escravos, quando a infâmia acomete os senhores, às vezes,
servem para alguma cousa.

Em certa noite, na solitária habitação, quando dormiam todos, ou fingiam
dormir, entre a jovem senhora e a velha companheira negra apareceu um
novo ente, sem que se desse encanto, nem mistério, nem assombrosa
aparição.

A negra, cauta e cuidadosa, envolveu o \emph{novo ente} em \emph{um xale
de casimira azul}, acalentou-o, abafou-lhe os vagidos\footnote{. Choro,
  gemido de recém-nascido.}; diante do crime desaparecem as condições;
os criminosos são iguais.

A orgulhosa beldade, de pé, alumiada por uma candeia\footnote{. Nesse
  contexto, pequeno aparelho de iluminação abastecido com óleo ou gás
  inflamável.} que pendia da parede escura, disse à escrava:

\begin{itemize}
\item
  --~Leva-o; a esta hora todos dormem; ninguém te verá; \emph{vai à
  ponte da Tabatinguera}: atira-o no rio!...\\
  --~Misericórdia, \emph{Sinhora}!...\\
  -- Vai! Faz o que te digo!...
\item
\end{itemize}

E a negra velha partiu; tomou pela rua do Brás; desapareceu por entre a
noite, densa de sombras, e mais negra do que ela. Levava aos braços o
mesclado filho de um negro. Atravessou as desertas ruas; venceu os
perigos; e, na roda dos expostos\footnote{. A roda dos expostos, também
  conhecida à época como roda dos enjeitados, era um lugar em que se
  abandonava bebês e recém-nascidos. Espécie de tambor com uma
  portinhola giratória, embutido em uma parede, era construído de modo
  que quem abandonava, de um lado, não via quem recebia, do outro lado
  da parede. Pela descrição geográfica que Gama anotou, tomando a rua do
  Brás como ponto de localização, o percurso da "negra velha" deve ter
  dado na roda dos expostos do Convento do Carmo.}, como em tábua de
salvação, depôs benigna, aquele náufrago do pudor...

\begin{itemize}
\item
  Quando ela voltou perguntou-lhe tranquilamente a senhora:
\item
  -- Está feito!
\end{itemize}

-- Sim, senhora, respondeu-lhe a negra confidente... Cheguei à ponte...
E duas lágrimas brotaram-lhe dos olhos, e deslizaram tardias, pelas
faces escuras, como de dois círios\footnote{. Grande vela.} acesos dois
fios de cera sobre um túmulo. Sua voz tornou-se rouca, e ela recomeçou:
cheguei à ponte... não havia ali ninguém...levantei-o nos
braços...atirei-o...ele chorou... o rio deu um grito... e... acabou-se
tudo! Sim; acabou-se; acrescentou a senhora: \emph{é como se faz às
ninhadas de gatos, e cães inúteis}...

Conheci a negra, e a senhora. O filho foi alfaiate; trabalhou em uma
oficina à rua do Rosário; foi soldado; e morreu, crivado\footnote{. Todo
  furado, perfurado.} de úlceras, na enfermaria do Quartel de
Linha\footnote{. O antigo Quartel da Legião dos Voluntários Reais,
  construído em 1790, passou a ser conhecido como Quartel de Linha,
  abrigando um efetivo de aproximadamente mil homens da Guarda Nacional
  em São Paulo em meados do século XIX. Luiz Gama serviu nesse quartel
  em diversas patentes, até chegar a cabo-de-esquadra graduado, em 1854.
  Ao tratar do caso em detalhes, até mesmo sobre especificidades
  internas ao quartel, Gama demonstra indiretamente que continuava a ter
  trânsito dentro da caserna.}.

***

No município de Mogi-Mirim\footnote{. Localizado no interior paulista,
  distante 150 km da capital, Mogi-Mirim teve grande concentração de
  trabalho escravo nas plantações de café.}, na fazenda do doutor..., em
o ano de 186... achava-se hospedado o exmo. desembargador..., nascido em
uma das províncias do norte, e que, aqui, desempenhou altos cargos de
administração.

Estava à janela; atravessou o largo terreiro um mulato; e o hóspede
exclamou:

-- O que veio aqui fazer o padre Pompeu?!\\
-- O padre Pompeu?! redarguiu o proprietário.\\
-- ~Sim: ele mesmo; o senador!...

\begin{itemize}
\item
\item
  Momentos depois o opulento hospedeiro mostrava ao seu amigo um vistoso
  cavalo de raça; tirava-o pelas bridas\footnote{. Rédeas.} o mesmo
  mulato que, momento antes, havia atravessado o terreiro.\\
  O distinto hóspede encarou-o curioso, examinou-o atento, e prosseguiu:
\end{itemize}

-- É perfeito! Tem apenas menos idade; em tudo mais é a imagem do padre
esculpida!

E interrogou:

\begin{itemize}
\item
  -- D'onde és tu?\\
  -- Sou do Ceará.\\
  -- Do Ceará?! Pior...\\
  -- De quem foste escravo?\\
  -- Do senador Pompeu.\\
  -- Onde nasceste?\\
  -- Na casa dele.\\
  -- E quem te vendeu?\\
  -- Ele mesmo.
\item
\item
  Entre os dois amigos, em silêncio, trocaram-se olhares significativos,
  inteligentes. E o hóspede murmurou -- que bárbaro!...
\item
  Respondeu ao júri; foi condenado a açoites; o tribunal, não por
  justiça, se não por esquálida\footnote{. Imunda, torpe.} parcialidade,
  para restituí-lo ao senhor, julgou -- que ele matara em sua defesa
  própria.\\
  Sofreu a ignominiosa\footnote{. Desonrosa.}, a terrível pena; mas
  continuou a fugir! O sentimento da liberdade é a concentração da
  sensibilidade moral; zomba das torturas físicas.\\
  O atilado\footnote{. Perspicaz, sagaz.} senhor descobriu, porém, um
  meio de \emph{domesticá-lo}: casou-o; foi-lhe amarra o matrimônio, e a
  âncora a mulher.\\
  Hoje deve ser um péssimo homem, um ente abjeto, desprezível, infame:
  tornou-se \emph{bom escravo}; merece os gabos\footnote{. Elogios.} do
  senhor.
\end{itemize}

***

\begin{itemize}
\item
  Na província de Minas Gerais, em uma das suas povoações, um negro
  nascido neste libérrimo\footnote{. Superlativo de livre, algo como
    muitíssimo livre, muitíssimo liberal.} país, um miserável escravo,
  ininteligente, inculto, estúpido, bruto, sem costumes, sem caráter,
  sem bons sentimentos, sem pudor, criado como cousa, para adquirir sua
  liberdade, para fazer-se homem, pegou de um \emph{senhor moço},
  menino, inocente, inofensivo, inconsciente, \emph{seu amigo}... e
  matou-o!...\\
  Matar um futuro senhor?... Aniquilar o domínio em gérmen\footnote{.
    Estágio inicial do desenvolvimento de um organismo.}?... Desfazer a
  tirania em miniatura?... Em projeto?... Sob a forma ridícula de pueril
  criança, para evitar o cativeiro, no futuro?!...
\end{itemize}

Este acontecimento espantoso atesta a existência de uma ideia fixa,
perigosa: acusa uma obliteração mental; o seu autor, porém, é um
negro!...

***

Na culta Inglaterra, em o ano de 1483, o duque de Glocester mandou
encerrar na torre de Londres e assassinar, pelo sicário\footnote{.
  Assassino contratado, facínora.} Tirrel, dois indefessos meninos, dois
sobrinhos seus, filhos do seu irmão Eduardo IV, seus tutelados, para
usurpar-lhes o trono e a riqueza.

Este príncipe, com as mãos tintas de sangue, foi sagrado, perante Deus,
à face da Igreja; foi elevado ao poder; teve cultos e adorações; reinou
sobre o povo, com as luzes do clero, e com o auxílio dos sábios!...

Foi um assassino? Foi um ladrão?...

***

Aquele fato, de que foi teatro Paris; aquele crime cometido pelo velho
Lavard: um homem branco, fidalgo, ilustrado, bem procedido, bem
conceituado, compreende-se, explica-se, está no domínio da ciência, tem
uma razão de ser.

Aquela jovem nobilíssima, paulista distinta, rica, importante, poderosa,
que furtivamente, em erma\footnote{. Deserta, despovoada.} habitação,
dava à luz o filho de um escravo; que, de concerto com a sua ilustre
família, abusando, com ignomínia,\footnote{. Desonra, infâmia.} da
fraqueza, da senilidade de uma mulher escrava, à noite, mandava sepultar
vivo, nas águas do Tamanduateí, o fruto pardo das suas relações negras,
foi vítima de uma fraqueza inevitável: tem plena justificação nas leis
da fisiologia; tem direito à absolvição da sociedade; não é uma ré; é
uma vítima.

Aquele duque de Glocester, aquele tio, aquele tutor, que assassina dois
meninos para roubar-lhes o trono e os cabedais\footnote{. Recursos
  financeiros, riquezas materiais.}, era um príncipe, foi um rei, não
foi um ladrão, foi um conquistador.

A governação é uma ciência: é a realização da política: esteia-se em
princípios morais, distende-se\footnote{. Estende-se.} protegendo a
felicidade humana, visa a consecução do bem social.

O crime, a imoralidade são qualificações transitórias de erros comuns
que não atingem os atos dos poderosos do estado; o caráter, a posição do
autor determinam a razão do fato; o crime é tão grosseiro e vulgar como
os criminosos.

***

Deixo em silêncio duas personagens: aquele \emph{Reverendo
Legislador}\footnote{. O padre e senador Pompeu.}, que vendeu o filho; e
\emph{Proudhon}: o tresloucado autor do escrito, que deu causa à carta
que escrevo. O primeiro descende em linha reta do imortal
Judas\footnote{. Judas Iscariot foi um dos doze primeiros discípulos de
  Jesus. De acordo com os Evangelhos, Judas traiu e entregou Jesus para
  seus captores em troca de trinta moedas de prata.}, e pertenceu à
mesma seita; o segundo é um fugitivo da casa dos orates\footnote{.
  Equivale, no contexto, a hospício.}; que agora iniciou-se nos
mistérios das \emph{evoluções positivas do cativeiro}.

Fica de pé uma entidade; é o \emph{assassino do senhor}; é a imagem da
miséria; é a Sephora dos tempos modernos; é o leproso social: é o
escravo homicida. Tem uma escola -- a senzala; tem um descanso -- o
eito\footnote{. Trabalho degradante. Expressão própria para trabalho
  escravo em área rural.}; tem um consolo -- a vergasta\footnote{.
  Chicote, vara fina usada para açoitar, torturar.}; tem um futuro -- o
túmulo. E a escravidão também terá um monumento sagrado, que há de
perpetuá-la, além dos séculos, construído com as pedras amontoadas, na
praça pública, pelos covardes, pelos malvados, pelos assassinos impunes.

Teu\\
LUIZ GAMA.

\textbf{43. {[}CARTA AO DR. FERREIRA DE MENEZES{]}}\footnote{. In.
  \emph{Gazeta da Tarde} (RJ), Gazeta da Tarde {[}editorial{]},
  23/01/1881.}

\textbf{*didascália*}

\emph{Na linha da carta publicada em 07/01/1881, Gama segue direcionando
sua pena para as reuniões políticas senhoriais, compostas de
"fazendeiros abastados, de negociantes e de capitalistas". Nesse trecho,
Gama comenta uma "reunião importantíssima", na qual certamente ele era
persona non grata, que, nada tratando da defesa da direitos senhoriais,
"exclusivamente ocupou-se de suprimentos monetários e aquisição de
colonos para a lavoura". Havia uma sensível mudança de pauta. Os
fazendeiros e empresários não passaram horas discutindo formas de
manutenção do trabalho escravizado. Ao contrário, em evidente sinal de
que a Abolição surgia no horizonte de expectativas do Brasil, aquela
associação debatia o crédito e a imigração de colonos brancos para São
Paulo. "Substituir o trabalho servil; dar dinheiro barato, e comodamente
aos lavradores são as teses que preocupam-no", isto é, ao Club da
Lavoura. Mais: "há vício radical invencível", apontaria Gama, nas teses
do tal Club. A Abolição estava pendente. Não se poderia discutir
introdução de colonos (brancos) e alocá-los nas lavouras sem resolver o
paradoxo do trabalhador escravizado (negro) que já estava nas fazendas.
Gama concluía, em notório jogo de palavras: "O escravo no trabalho da
lavoura é insubstituível". Isto é, no contexto, não o escravo, mas o
negro; insubstituível, não o sistema de trabalho, mas os trabalhadores
já assentados na colocação. Em resumo: não haveria espaço para trazer
colonos brancos quando os trabalhadores negros já estavam no lugar. Ao
comentar a reunião do Club da Lavoura, Gama demonstra estar a par dos
passos da política da escravidão e sua agenda reacionária. Política e
agenda bastante nuançadas, haja vista a sagaz definição que Gama deu dos
componentes daquela associação. "É um conjunto de liberais,
conservadores e republicanos; embora na atualidade, sob o ponto de vista
prático, fora do palavreado costumeiro, os qualificativos políticos
careçam de realidade; porque corre o tempo de muricy, em que cada qual
cuida de si".}

\emph{***}

S. Paulo, 18 de Janeiro de 1881.

Meu caro Menezes.

Venho da \emph{Gazeta de S. Paulo}, onde deixei para ser impressa sua
carta, que endereço-te, relativa a um disparatado escrito, firmado com o
pseudônimo -- \emph{Proudhon}\footnote{. Pierre-Joseph Proudhon
  (1809-1865), nascido em Besançon, França, foi tipógrafo, escritor,
  político e filósofo anarquista. Foi membro do parlamento francês e
  publicou obras sobre teoria política, propriedade e autogoverno.} --,
que notavelmente encerra grosseira injúria a este nome, que designa um
dos maiores gênios que tem abrilhantado o mundo.

O artigo a que aludo foi impresso na \emph{Província} do dia 13; atira
algumas pedradas ao Lúcio de Mendonça, o que não admira, porque o Lúcio
é um astro, e o articulista um abissínio\footnote{. Relativo à
  Abissínia, na região da atual Etiópia. Expressão evidentemente
  pejorativa que contrasta com os usos de referenciais africanos no
  próprio discurso de Gama. Por destoar frontalmente com o autor que
  assinou "Getulino", cantou as "musas de Guiné" e enalteceu a "Líbia
  adusta", todas elas imagens elogiosas à África, pode-se aventar que o
  emprego de "abissínio", nesse contexto, quereria mexer com os brios do
  oponente a todo custo.}; e distribui-me algumas pachuchadas\footnote{.
  Bobagens, asneiras.} idióticas, dignas de piedoso sorriso.

Cada dia que se finda encerra uma data memorável na senda
impérvia\footnote{. Intransitável, impraticável.} que se desbrava aos
passos dos lidadores da emancipação.

No dia 16 deu-se nesta cidade uma reunião importantíssima, e de caráter
grave, constituída de fazendeiros abastados, de negociantes e de
capitalistas. É uma espécie de \emph{Club da Lavoura e
Comércio}\footnote{. Organização social dos interesses de fazendeiros e
  comerciantes.} e o mais digno de atenção de quantos se hão
constituído.

Como sempre acontece entre nós, a julgar pelos fatos, à semelhante
reunião não precedeu estudo e acordo, mas, sem embargo disto, ela
existe.

É um conjunto de liberais, conservadores e republicanos; embora na
atualidade, sob o ponto de vista prático, fora do palavreado costumeiro,
os qualificativos políticos careçam de realidade; porque corre o tempo
de muricy, em que cada qual cuida de si; e as agregações partidárias não
passem de \emph{monções}\footnote{. No sentido de ajuntamento ocasional.}
\emph{de romeiros}, com destino ao poder, que se ajustam, para com maior
segurança, atravessarem desertos cabedelos\footnote{. Dunas, elevações
  de areia.}; há, contudo, aparências delicadas, dignidades calculadas,
e formalidades melindrosas, que não podem ser preteridas. As
exterioridades políticas são como o dogma; todo o seu valor provém do
mistério; mas o venera, mas o idolatra quem menos o entende.

Neste conjunto de respeitáveis personagens, além das distinções, de
exterioridades políticas, há outras cujas gravidades se não pode
dissimular. Existem, ali, abolicionistas; e sem dificuldade, desde já,
indico dois nomes, prestigiosos, socialmente considerados: são os exmos.
srs. dr. Antonio Prado e Lopes de Oliveira.

O Club ao contrário de quantos se hão reunido -- \emph{não tratou, de
modo algum, da defesa dos direitos dominicais}!\footnote{. O mesmo que
  senhoriais.} -- Pura e exclusivamente ocupou-se de \emph{suprimentos
monetários} e \emph{aquisição de colonos} para a lavoura.

Substituir o trabalho servil; dar dinheiro barato, e comodamente aos
lavradores são as teses que preocupam-no.

No meio, pelo qual se pretende obter colonos, há vício radical
invencível; a obtenção de dinheiro depende da de colonos: uma e outra
coisa constituem dois impossíveis.

A abolição do trabalho servil é uma questão pendente; qualquer que seja
a dificuldade superveniente há de realizar-se em curto prazo; a falta de
critério, da parte do governo, daria a conflagração. O escravo no
trabalho da lavoura é \emph{insubstituível}.

As doutrinas econômicas liberais, com aplicação ao estabelecimento de
bancos locais, hipotecários, isolados, se não encerram um impossível, ao
tempo presente conduzem ao desastre no tempo futuro; porque o benefício
ou o mal não estão nas doutrinas; consistem na aplicação; na falta de
oportunidade: os preceitos econômicos sob o ponto de vista prático são
relativos, e não absolutos; são aproveitáveis, não impositivos; salvo
quando, por organizada especulação, procura-se, à sombra dos princípios,
com o auxílio do poder, adquirir riquezas, à custa da desgraça alheia,
ou quando as circunstâncias determinam o contrário.

Foi este um dos assuntos da reunião; manifestaram-se opiniões neste
sentido. E só este ponto, de per si, é bastante para lançar o pomo de
Páris\footnote{. O mesmo que pomo da discórdia. Além das narrativas
  mitológicas relacionadas à Guerra de Tróia, a expressão indica alguma
  coisa que instigue as pessoas a brigarem entre si.} no seio do Club.

O mal comum, uma necessidade iminente, inevitável, determinou a reunião
do Club; a pluralidade das ideias; as desarmonias essenciais hão de
leva-lo ao seu fim.

Não agouro mal a reunião do Club; felicito, com sinceridade, os seus
dignos autores; manifesto, apenas, com alguma antecipação, uma conclusão
lógica.

A época em que atravessamos encerra uma fermentação de filosofias,
conduz uma revolução moral; caminha para o assinalamento de uma época
natural.

***

Acaba de exibir-se, na Assembleia Provincial, um projeto de lei de
extraordinária importância: contém nada menos do que a inamovibilidade
do elemento servil nesta província.

Esta lei, como a que foi promulgada pela Assembleia Provincial do Rio de
Janeiro, é uma espécie de poliedro\footnote{. O que tem muitas faces.}
governamental, um gládio\footnote{. Espada.} de dois gumes, que vai ser
posto na mão do Poder, que, por a mediação da sua monocracia\footnote{.
  Aqui no sentido de exercício da autocracia, regime em que o governante
  detém a soberania política, isto é, a palavra final sobre assuntos
  civis.}, ou dos \emph{seus} magistrados, favoneando\footnote{.
  Favorecendo, protegendo.}, talvez, pretensões escuras, queira
esgrimir\footnote{. Lutar, travar combate.}, nas trevas, com os próprios
abolicionistas.

Cumpre, pois, que estejamos atentos; que observemos, com cuidado, os
passos do governo, e que, de olhos abertos, sejamos como os
Cyrocrothes\footnote{. Figura mitológica, espécie de besta que não
  fechava os olhos e não tinha divisão de dentes.}.

Sempre

Teu

LUIZ GAMA.

\textbf{44. CARTA AO DR. FERREIRA DE MENEZES}\footnote{. In.
  \emph{Gazeta da Tarde} (RJ), 29/01/1881, pp. 2-3.}

\textbf{*didascália*}

\emph{Gama recebeu uma carta anônima. "Não é a primeira. Nestes últimos
tempos tem sido esse o meio escolhido por pessoas desprezíveis, que não
conheço, nem desejo, para insultar-me!... Insensatos". No entanto, a
carta o surpreendeu. Não era mais uma das ameaças do terrorismo
senhorial de algum reacionário da política da escravidão, que, como
sugere Gama, estavam especialmente raivosos nos últimos tempos. Ao
contrário. A carta era escrita por uma "heroína da liberdade" que, pelos
traços que se pode alcançar da leitura, era uma mulher branca de família
influente do Rio de Janeiro ou de Minas Gerais. O entusiasmo de Gama com
a carta foi gigantesco. "Esta carta será, quando gravada na história da
humanidade, a página de ouro da evolução abolicionista no Brasil",
afirmava, realçando que surgia no cenário nacional uma autora do quilate
da romancista norte-americana Harriet Stowe, autora do best-seller "A
Cabana do Pai Tomás". A carta possui incontestável força literária e
integra o panteão do que de melhor produziu a literatura abolicionista
brasileira. É importante ressaltar que a carta só veio a público porque
Gama deu publicidade a ela, sabendo, certamente, do impacto político que
ela poderia exercer, seja pela autoria incomum ou por seu conteúdo de
tirar o fôlego. A Harriet Stowe brasileira, ou "A Neta de Zambo",
contava "uma cena revoltante, horrorosa, cruel e infame (...), praticada
por um homem educado no foro da civilização, na Europa". O estilo da
narrativa era envolvente. "A Neta de Zambo", em viagem ao interior de
Minas Gerais, hospedada num hotel, serve de confidente a uma outra
mulher, que ouviu de seu marido, testemunha ocular, o bárbaro crime que
passaria a contar. A história envolvia mais duas mulheres que, de algum
modo, protagonizavam o enredo. Eram a filha e a mulher de um homem negro
torturado pelo tal fazendeiro educado na Europa. O "negro chamado P."
fora brutalmente torturado por rechaçar que sua filha dormisse com o
senhor de escravizados. Em retaliação física e moral ainda maior, o
fazendeiro resolveu estuprar mãe e filha. "A Neta de Zambo" dava
detalhes da cena macabra e acrescentava à narrativa outro caso
igualmente perturbador. Denunciava, assim, que "o senhor, ou antes, o
assassino, protegido pela lei", era um criminoso a quem a autoridade
pública dava total apoio e cobertura. }

\emph{***}

S. Paulo, 22 de Janeiro de 1881.

Meu caro Menezes.

Difunde-se a luz da liberdade por todos os antros do império.\\
A grande causa diviniza-se, tem por altares os corações de
acros\footnote{. No sentido de pouco flexíveis.} e invencíveis
patriotas.

O evangelho social ressurge verberante e luminoso do seio das sombras; e
altaneiro, supremo, invencível, à semelhança das chamas dos relâmpagos,
propaga-se por todo o Brasil, como a religião do Cristo, outrora, a
rebentar das catacumbas de Roma.

Há três dias, de uma importante povoação do interior desta província
(segundo malsina\footnote{. Denuncia.} o carimbo do correio), recebi
\emph{uma carta anônima}.

Não é a primeira.

Nestes últimos tempos tem sido esse o meio escolhido por pessoas
desprezíveis, que não conheço, nem desejo, para insultar-me!...

Insensatos.

Guardo inalterável silêncio sobre o que me dizem; e continuo na minha
tarefa, com energia e segurança, levando de vencida os bárbaros
lucífugos\footnote{. Que foge da luz, que evita a claridade.}.

A guerra não se faz com palavras; nem com injúrias; nem com ameaças; nem
com lutadores anônimos: os nossos canhões estão assestados\footnote{.
  Apontados, direcionados.} a descoberto; os nossos gladiadores estão de
pé: nós combatemos sem artifício: nossa armadura é o direito.

Desta vez, porém, o anônimo é a investidura da modéstia; a carta é
escrita por uma senhora tão inteligente quão delicada: tu a lerás, algum
dia, no próprio original.

Entende-me: não é uma \emph{senhora de escravos}: é uma personificação
de virtudes: uma senhora de brios: uma brasileira benemérita: uma
heroína da liberdade.

Esta carta será, quando gravada na história da humanidade, a página de
ouro da evolução abolicionista no Brasil.

Não tem data; e tem por assinatura um nome suposto.

À semelhança dos astros, não se sabe de onde veio; ignora-se a data do
seu nascimento; rebrilha no firmamento; a ciência lhe sagrará um nome
eterno.

Se o estilo é um retrato moral, eu lôbrego\footnote{. Diz-se do lugar
  sombrio, escuro, em que quase não há claridade. Pelo contexto, Gama
  sugere que entrevê, que enxerga a autora "através das sombras do
  mistério."} através das sombras do mistério, as lindas feições da
distinta -- \emph{Neta de Zambo}.

Lembro-me de vê-la cavalgando airoso\footnote{. Elegante, gracioso.}
ginete\footnote{. Cavalo de boa procedência, adestrado.}, a correr
ousada pelos páramos\footnote{. Planalto.} de Piratinim\footnote{.
  Provável grafia diferente para Piratininga, nome que designava a
  região da cidade de São Paulo, antes da colonização portuguesa.},
peregrina como as rosas de Erimantho, e formosa como as pérolas de
Golconda\footnote{. Referência provável aos diamantes extraídos das
  minas de Golconda, Índia, considerados como os maiores e mais belos do
  mundo, sendo muitos desses diamantes, hoje, parte da fortuna de reinos
  e estados europeus.}.

Creio ter já conversado, discutido, venerado, e, docemente vencido, pelo
sopro benigno da gratidão, osculando\footnote{. Beijando.} a
destra\footnote{. Mão direita.} veneranda, que, em hora ditosa, traçou
este maravilhoso documento.

Envio-te a carta, por cópia. Deve ser lida por ti, e pelos nossos dignos
companheiros e amigos.

Peço-te que, dela, publiques alguns trechos, dignos da imprensa
ilustrada, dignos da causa nobilíssima que defendemos e da posteridade.

Apelando, porém, para o teu cavalheirismo, exijo que guardes profundo
silêncio relativamente aos tópicos que vão traçados com tinta carmesim.

Quanto a uns, porque são excessivamente encomiásticos\footnote{.
  Elogiosos.}, e concernentes a pessoa a quem muito prezas, e que, pelo
seu caráter, impõem-me este dever. Os outros, como verás, são graves, em
razão de circunstâncias peculiares, e, de todo ponto, confidenciais.

Termino enviando-te um fraternal aperto de mão; dirigindo
epinícios\footnote{. Cântico feito para comemorar uma vitória ou o
  regozijo por um feliz acontecimento.} à nossa esplêndida heroína; e
dando, com efusão, um sincero abraço no seu respeitável consorte.

Enfim: podemos exclamar, com os nossos irmãos dos Estados Unidos da
América do Norte:

-- Surge radiante a aurora da liberdade; e, no seu ninho de luzes, a
nova HARRIET STOWE.

Teu\\
LUIZ GAMA.

\textbf{À LUIZ GAMA}

Senhor:

......................

Sim, Luiz Gama, em último caso, digamos como Condorcet\footnote{.
  Nicolas de Condercet (1743-1794), o marquês de Condorcet, nascido em
  Ribemont, França, foi matemático, filósofo e político. Possivelmente,
  a citação seja retirada do manifesto \emph{Reflections on Negro
  Slavery} (1781), obra considerada abolicionista e de grande impacto
  nas discussões sobre o fim da escravidão nas colônias francesas.}:
"Prefiro as procelas\footnote{. Tempestades, agitações.} da liberdade à
segurança da escravidão."

Alguns jornais da Corte têm ultimamente sido pródigos em ameaças,
insultos e calúnias.

Lamento do fundo d'alma de ver que a descoberta de Gutemberg\footnote{.
  Refere-se, em termos gerais, à imprensa, através do inventor da prensa
  móvel e desenvolvedor do processo gráfico usado para imprimir jornais,
  o gráfico alemão Johannes Gutenberg (1400-1468).}, denominada por este
século de -- astro luminoso, espanca-trevas, percursora do progresso,
propagadora da liberdade, etc., etc., sirva agora por egoísmo e vil
interesse a defender o cancro vergonhoso que rói o desventurado Império
do Brasil, a escravidão!

Os homens que defendem semelhante causa ou odeiam a humanidade, ou então
nunca estiveram em alguns estabelecimentos agrícolas, nos quais os
proprietários são os piores tiranos; e os empregados, algozes que
ultrapassam em crueldade os da extinta inquisição!

Para corroborar o que acabo de expor, a respeito desses odiosos
senhores, citarei e provarei, se necessário for, uma cena revoltante,
horrorosa, cruel e infame: e, cousa singular, praticada por um homem
educado no foro da civilização, na Europa; o qual (talvez) em companhia
dos defensores da opressão e da infâmia, saboreassem o famoso
\emph{champagne}, à sombra de um caramanchão\footnote{. Construção
  simples geralmente feita em jardins e parques para descanso, abrigo ou
  recreação.} acompanhado de cantigas obscenas e báquicas\footnote{.
  Relativo a Baco. Aqui a autora emprega como adjetivo de depravação.}
cantadas por meretrizes. Porém, lancemos um véu sobre este quadro
aviltante e repugnante, e comecemos a narração do crime horrendo,
perpetrado com todo o cinismo e perversidade por este frequentador dos
botequins da rua do Ouvidor.

Há de haver aproximadamente três anos, viajava eu no interior desta
província, em companhia dos meus tios; e ao passarmos na vila de B. meu
tio resolveu pernoitar nela, a fim de resolver alguns negócios. O
proprietário do hotel onde nos hospedamos era casado com uma virtuosa e
sensível mulher.

À noite, depois de termos esgotado tudo quanto tínhamos a dizer e
contar, despedi-me da boa mulher, para ir-me deitar; porém, ela olhando
em redor de si para verificar se ninguém a ouvia, disse-me em tom
confidencial: "espere, quero-lhe contar uma história que lhe há de
entristecer muito e ao mesmo tempo interessar; mas, desde já, peço-lhe o
mais absoluto segredo". Pois não, respondi-lhe eu, ansiosa por saber a
tal história: esteja sossegada, e desde já estou pronta para lhe ouvir.
A dona da casa foi fechar as portas; voltando, assentou-se bem perto de
mim, e assim começou:

"Meu marido, há oito dias, indo para a fazenda de F... tratar de alguns
negócios que o obrigavam a estar na dita fazenda alguns dias, presenciou
o seguinte: ao chegar encontrou o proprietário do sítio, que ordenava o
feitor que amarrasse ao cepo\footnote{. Pedaço ou tronco de árvore
  cortado transversalmente.} da casa do tronco o negro chamado P. pelo
pescoço, cintura e pés.

Depois de executada esta ordem, o dito fazendeiro chegou-se perto do
mísero escravo, e, em tom de mofa, disse-lhe 'Então meu capadócio.....
ontem querias opor-te que a tua filha partilhasse meu leito.... cão, não
sabes que o escravo pertence em corpo e alma ao senhor?...'

O paciente, com os olhos cheios de lágrimas, pediu ao seu algoz pelo
amor de Deus, para que poupasse sua filha, a quem amava extremosamente.
Mas, o odioso e inflexível senhor, respondeu-lhe cínica e
impudentemente\footnote{. Desavergonhadamente.} o seguinte: "Não só tua
filha como tua mulher participarão hoje do meu leito."

.... E, com os olhos injetados de sangue pelo ódio, acrescentou com um
riso sardônico: "E entretanto, logo de noite tomarás duzentos açoites, e
passarás assim a noite; e amanhã, quando fores desamarrado, mandar-te-ei
colocar dois ferros; um no pescoço e outro no pé, para que não possas
\emph{passear muito}...".

O infeliz ao ouvir estas iniquidades, fechara os olhos; seus dentes
rangiam; do peito saía um ruído surdo semelhante àquele que se ouve no
Vesúvio\footnote{. Refere-se ao vulcão Vesúvio, localizado na cidade de
  Nápoles, Itália.} quando ameaça erupção.

Às 10 horas da noite, dois negros robustos, cada um munido de um
azorrague\footnote{. Chicote, chibata formada por várias correias
  entrelaçadas presas num cabo de pau. Instrumento de tortura.},
postaram-se, um à esquerda e outro à direita do infeliz.

O suplício começou. A parte castigada do mártir estava retalhada, e dela
jorrava o sangue em abundância.

O desditoso\footnote{. Infeliz.}, desde o princípio até o fim do
suplício, não soltara um só gemido, um suspiro!...

Entretanto, o senhor, ou antes, o assassino, protegido pela lei, tinha
por meio de ameaças satisfeito seu apetite brutal; e completamente
ébrio\footnote{. Embriagado, bêbado.}; exclamava cambaleando diante de
seus satélites silenciosos:

"\emph{Consummatum est}...".\\
No dia seguinte quando os sicários\footnote{. Assassinos contratados,
  facínoras.} levavam os ferros para algemar o desgraçado, encontraram o
corpo da vítima, feio, gelado, hirto\footnote{. Imóvel, duro.}; enfim um
cadáver!...\\
Deus compadecera-se do infeliz, chamara-o para a região dos
bem-aventurados. O médico fora chamado, e atestava pela fé do seu grau,
que o negro falecera de \emph{apoplexia fulminante}!...

O \emph{tempora}!... O \emph{mores}\footnote{. "Ó, tempos!... Ó,
  costumes!...". Exclamações originalmente de Cícero (106 a.C-43 a.C.),
  denunciando as corrupções e as perversidades da Roma em que viveu.}!...

Terminada a história, a sensível e virtuosa mulher chorava e pedia-me
que rezasse um Padre Nosso e uma Ave Maria, por alma do pobre escravo. E
antes de retirar-se, recomendou-me pela centésima vez que guardasse
segredo.

Mas, como ela e o marido não têm mais que recear a vingança do
assassino, pois que já são falecidos; por isso animei-me a narrar-vos
essa história terrível e tenebrosa.

Termino, pois, citando-vos mais um fato recente que se deu no município
da Limeira\footnote{. Lembrem-se que Gama foi contestado do conteúdo da
  terceira carta, justamente por habitantes de Limeira, município do
  interior paulista, que alegavam que o" grau de civilização e
  sentimentos humanitários da sociedade Limeirense" não permitiria
  crueldades senhoriais. Naquela ocasião, Gama publicou uma "reparação
  devida", retificando a localidade do crime, mas sem recuar da conexão
  do autor do crime com a cidade. Para essa nova acusação que ele deu
  vazão, não foi encontrada réplica ou reparação -- agora devida pelos
  cidadãos limeirenses.}, e para o qual chamo a vossa atenção; é o
seguinte:

Um pobre negro, de comportamento exemplar, pensou um dia libertar-se;
para esse fim ia depositando o produto de algumas economias nas mãos de
algumas pessoas, que lhe tinham sido designadas como habilitadas e
capazes de fazerem valer o seu incontestável direito, em ocasião
oportuna. Mas, ó fatalidade! Um belo dia, o senhor chama o escravo; e
pergunta-lhe para que queria ele o dinheiro, que tinha depositado nas
mãos das ditas pessoas. O coitado do negro expõe-lhe com franqueza o seu
intento. Uma torrente de injúrias sai dos lábios imundos do cruel
senhor. Enfurecido, manda chamar o cruel e sanguinário feitor; e
ordena-lhe que ponha sem mais demora o negro a ferros, e em seguida,
passar-lhe cem chicotadas.

É deste modo que se trata na Terra de Santa Cruz um escravo que aspira,
por meios lícitos, tornar-se um homem livre.

O prêmio que o escravo obteve, por ter tido uma tão nobre aspiração, foi
o seguinte: ferro no pé e no pescoço, cem açoites e sem o dinheiro que
tanto lhe custara a ganhar. E tudo isto passou-se e passa-se no ano de
Nosso Senhor Jesus Cristo de 1881, nas barbas do século dezenove,
denominado por excelência -- das luzes!!!... Escárnio! Ironia
sangrenta!!...

Nobre e generoso sr. Gama, vós conheceis melhor do que eu os males
terríveis que os bárbaros senhores fazem sofrer aos infelizes escravos:
por isso abster-me-ei de vos incomodar com a narração de tantas
atrocidades; não obstante, peço-vos desde já, vênia\footnote{. Licença,
  permissão.} para participar-vos de vez em quando as injustiça e abusos
dos quais são vítimas eternas os desprotegidos da lei dos homens! Se,
entretanto, os poderes competentes não melhorarem a sorte destes
infelizes, ensinai-lhe o meio indicado por vós no artigo -- Resposta ao
pé da letra; isto é, o caminho do desespero!

"Contra o despotismo, a insurreição é o mais sagrado e mais santo dos
deveres" -- Declaração da imortal Convenção\footnote{. Referência
  provável à proclamação da abolição da monarquia francesa, em 1792,
  firmada durante o regime político denominado Convenção Nacional, que
  vigorou entre 1792 e 1795, fundando a Primeira República Francesa.}.

Pois bem, se Convenção aconselhava aos povos livres a insurreição contra
o despotismo, por que no Brasil não se aconselhará aos escravos a
rebelião contra a odiosa e cruel opressão de seus execráveis senhores?!

Sim, todos aqueles que tiverem patriotismo, dignidade e pudor, não podem
deixar de exclamar como Voltaire\footnote{. Fraçois-Marie Arouet
  (1694-1778), mais conhecido pelo pseudônimo Voltaire, nascido em
  Paris, França, foi escritor, historiador e filósofo iluminista de
  grande importância para a história das ideias e da política dos
  séculos XVIII e XIX.}: "ESMAGUEMOS A INFÂMIA!"

Sou de V. S. admiradora e criada.

\emph{Uma neta de Zambo}.

\textbf{45. CARTA AO DR. FERREIRA DE MENEZES}\footnote{. In.
  \emph{Gazeta da Tarde} (RJ), 01/02/1881.}

\textbf{*didascália*}

\emph{O último trecho da carta ao amigo Ferreira de Menezes, e ao
público em geral, tem um quê de comicidade, tão ao gosto da veia
satírica de Gama. O modo como descreve São Paulo é um dos indicativos.
Em uma reunião pública abolicionista, os presentes resolveram fundar uma
associação recreativa e teatral. Para patronos do "Recreio Dramático
Abolicionista", indicaram dois nomes da alta sociedade: um funcionário
público de alto escalão e um médico da cidade. Decididos os
homenageados, não se sabe se por reverência ou pilhéria, publicaram na
imprensa os nomes dos eleitos. Tão logo a publicação ganhou os jornais,
os ditos homenageados vieram a público declinar da posição, alegando não
terem sido previamente nem convidados nem consultados para tomar parte
da assembleia fundadora. Um dos patronos, contudo, avançou em suas
justificativas para não aceitar o posto. Disse o médico João Pedro:
"estou longe de aderir ao movimento abolicionista, que, em meu entender,
considero precipitado, pouco refletido, e inoportuno". Era a deixa que
Gama utilizaria como tema para desfecho da série de cartas ao amigo
Ferreira de Menezes. Revelando fatos pouco conhecidos da trajetória do
médico, Gama caracterizou-o como um refinado hipócrita. "O exmo. doutor
é digno membro da confraria do Frei Thomaz; do que prega nada faz",
dizia Gama, partindo da rima popular para a crítica política, social e
racial. Vejamos o argumento que, de um fôlego só, dirigia ao médico: "Um
homem que veste-se regularmente, que alimenta-se bem, que goza das
melhores comodidades, em uma sociedade opulenta, que sabe evitar o frio
e o calor, que frequenta divertimentos e calça luvas de pelica, deve
{[}no verbo mora o sarcasmo{]}, com sobeja honradez e abundante
filosofia, aconselhar aos seus irmãos negros, aos cativos, mas que
nasceram tão livres, como ele, que são vítimas de um crime horroroso,
seminus, expostos ao sol, ao frio, e à chuva, vestidos de trapos,
sacudidos à bacalhau, que têm por lenitivo a tortura, e por luvas os
calos levantados pela palmatória e pelo cabo da enxada, que sejam
prudentes, que suportem o flagício, que se habituem com os castigos,
tenham paciência, porque mais sofreu Jesus Cristo, e dos desgraçados é o
reino do céu!". A exclamação fulminava o contendor que recusava tão
ínfima participação no movimento abolicionista e, pior, se metia a
analisar o mérito de algo que não conhecia. Gama deu-lhe a resposta ao
pé da letra.}

\emph{***}

S. Paulo, 28 de Janeiro de 1881.

Meu caro Menezes.

É parêmia\footnote{. Alegoria breve, expressão proverbial.} já de
sobejo\footnote{. Excessivo, demasiado.} repetida, mas que, por muito
aguda, sempre vem de molde:

"Este mundo é um vastíssimo teatro onde todos se fazem de cômicos; os
mais hábeis, e não são poucos, representam à custa dos outros; recebem
as espórtulas\footnote{. Aqui no sentido de gorjeta, gratificação em
  dinheiro.}, e riem-se deles!"

Escrevo-te estas linhas entre sorrisos, entre ironias,
lancinantes\footnote{. Que atormenta, aflige.}, ou entre sarcasmos, se o
quiseres, meu nobre e distinto amigo; e d'este meu estado é {[}são{]}
causa{[}s{]} as duas cenas cômicas (pois que trata-se de assunto
teatral) que passo a transcrever das colunas da judiciosa \emph{Gazeta
do Povo}, para regalo teu e dos áticos leitores da tua preciosa
\emph{Gazeta}.

Deu-se, aqui nesta estação de férias e ócios forenses, que aliás não é
fértil de jocosos divertimentos, e isto há poucos dias, uma jovial
reunião (digo jovial por ser composta de jovens) de empregados públicos,
de negociantes, de artistas, e de pensadores (gente que tem o que
perder, como eu, mesmo sem nada possuir).

Esta assembleia de voluntários, constituída sem mandato previamente
conhecido, toda soberana e poderosa, reunida ao sopro sublime do
patriotismo dos seus membros, à guisa das nossas câmaras legislativas;
depois de formalmente constituída, com admirável sabedoria, sem eleições
diretas ou indiretas; sem Saraivas, sem Sinimbús, sem Gaspares, e até
sem Pelotas; e congregando espontaneamente, sectários de todas as seitas
religiosas, e cidadãos de todas as classes e condições; deliberou e
dignou, para seus governadores, dois conspícuos\footnote{. Notáveis,
  respeitáveis.} patriotas, membros distintíssimos da porção mais
elevada da culta sociedade paulistada\footnote{. Pode ser apenas um erro
  tipográfico, onde deveria apenas constar paulista ou paulistana ao
  invés de paulistada. Mas, considerando o contexto do artigo, de
  crítica afiada às práticas sociais da sociedade local, Gama poderia
  ter reforçado o sentido pejorativo para o ajuntamento de paulistas.}:
os exmos. srs. comendador Domingos de Mello Rodrigues Loureiro, cunhado
do exmo. senador marquês de S. Vicente, inspetor aposentado da
tesouraria de Fazenda, e chefe da Caixa Econômica Monte do Socorro, e o
dr. Joaquim Pedro da Silva, de borla e capelo\footnote{. Nesse contexto,
  a expressão, que remete às vestes solenes de um doutor, indica mais do
  que o figurino da pequena capa sobre os ombros (capelo) e o barrete
  adornado (borla) nas mãos; sugere soberba e empáfia.}, e conceituado
médico e operador desta afamada cidade.

Todos os periódicos da capital noticiaram de tropel\footnote{. Com
  grande repercussão.}, e com certa ênfase, que a mim se afigurou
maliciosa, e que a outros pareceu entusiástica, a organização da
sociedade, sob a conspiradora denominação de -- \emph{Recreio Dramático
Abolicionista}.

Os dois anciões venerandos, seletamente designados, pela forma eletiva e
regularíssima, rejeitaram modestamente a oficiosa graça, e recusaram-se
jeitosos de meter ombros\footnote{. Atirar-se ao trabalho, com afinco.}
ao fatal carrego\footnote{. Fardo.}.

Aí vão as duas recusas:

SOCIEDADE RECREIO DRAMÁTICO ABOLICIONISTA

"Foi com bastante surpresa que li, na \emph{Gazeta do Povo} de 2 do
corrente mês, a notícia de ter sido eleito vice-presidente da associação
\emph{Recreio Dramático Abolicionista} isto porque, nem fui convidado
para a reunião, nem consultado a semelhante respeito.

Assim, pois, não posso aceitar o mandato que me foi confiado;
entretanto, agradeço a lembrança honrosa dos dignos fundadores dessa
associação.

S. Paulo, 26 de Janeiro de 1881.

Domingos de M. R. Loureiro."

\_\_\_\_\_\_\_\_

"Ilmo. Sr. redator.

Foi com bastante surpresa que li em sua conceituada folha, de ontem, a
notícia de ter sido eu designado, ou eleito, para o \emph{conselho
abolicionista} da nova associação \emph{Recreio Dramático
Abolicionista}; porquanto nem fui convidado para a reunião em que foi
criada ela, nem consultado para aceitar tal cargo.

Agradecendo, pois, a lembrança honrosa dos dignos fundadores, e visto só
ter disso conhecimento pela imprensa, venho, por meio dela, declarar que
resigno esse mandato; não só pelo modo pouco regular porque me foi
conferido, \emph{como porque confesso francamente que -- partidário das
libertações individuais e bem cabidas -- estou longe de aderir ao
movimento abolicionista, que, em meu entender, considero precipitado,
pouco refletido, e inoportuno...} \footnote{. Realce em itálico
  provavelmente feito por Gama, não por Joaquim Pedro.}

S. Paulo, 25 de Janeiro de 1881.

Dr. Joaquim Pedro."

Sou afeiçoado ao imortal Epaminondas\footnote{. Epaminondas (418
  a.C.-362 a.C.), nascido em Tebas, Grécia, foi general e estadista de
  sua cidade natal, conduzindo Tebas ao patamar de nova potência
  hegemônica da Grécia Antiga.}; e não posso, sem visível
constrangimento, ocultar à clara luz da verdade certos fatos preciosos,
que ela me está de contínuo a sugerir.

Aí vai, portanto, relativamente à esta \emph{mista Associação}, a minha
desprevenida opinião.\\
Esta sociedade \emph{Dramática} e \emph{Abolicionista} é
nimiamente\footnote{. Demasiadamente.} revolucionária, perigosíssima, e
atentatória; quer se a considere em face da estética, quer perante os
códigos.

Perante a poesia dramática é o exício\footnote{. Ruína, perda total.}
inevitável dos \emph{artistas hábeis}; perante o código é uma
candidatura à grilheta\footnote{. No sentido de prisão.}.

Pôr sobre a fronte do divino Sófocles\footnote{. Sófocles (497/6
  a.C-406/5 a.C.), autor de \emph{Antígona} e \emph{Édipo Rei}, é
  considerado um dos maiores dramaturgos da história.} a
pancárpia\footnote{. Coroa de flores.} da manumissão?!

Os exmos. srs. comendador Loureiro, e doutor Joaquim Pedro, como
apreciadores de teatro cumpriram nobremente o seu dever.

Uma sociedade que se propõe a estragar, por beócios\footnote{. Aqui no
  sentido de incultos, ignorantes.}, dramas e comédias, compreende-se,
anima-se, louva-se; mas quando ao estropiamento literário reúne leilões
de prendas, conferências, e outros atos, em benefício de alforrias, em
prol de escravos, dá prova irrecusável de que é composta por
hilotas\footnote{. Miseráveis de extrema ignorância.}, semelhantes
àqueles por quem intercedem.

Os exmos. srs. Loureiro e Joaquim Pedro são dois conselheiros distintos,
dois perfeitos híbleos\footnote{. Que é nativo de Hybla, antiga cidade
  siciliana. Não se sabe quais características quis o autor implicar.},
que ficariam, principalmente o segundo, tocados de indelével
hiposfagma\footnote{. Derrame ocular, sangramento abaixo da conjuntiva
  do olho. Gama sugere que Joaquim Pedro, mais do que chorar, choraria
  sangue.}, se aceitassem os cargos que com tanto acerto recusaram.

O exmo. sr. dr. Joaquim Pedro, que exibe-se com ademanes\footnote{.
  Trejeitos, gestos feitos com as mãos.} estudados, de cabeleireiro de
Paris, a sacudir atilado\footnote{. Cuidadoso.} poeira de arroz aos
olhos do freguês, quando este conta isonte o troco recebido, aproveitou
a oportunidade, para revelar-se consumado estadista. Deu aos
abolicionistas parvos\footnote{. Idiotas, imbecis.}, tolos e estouvados,
famosa lição de mestre: rasgou-lhes o estandarte e arrebatou-lhes o
gorro\footnote{. Um dos mais importantes distintivos da Revolução
  Francesa (1789), o gorro (ou barrete) frígio tornou-se símbolo das
  ideias republicanas.} em plena praça!

-- "A emancipação há de ser feita lenta, individualmente, com muito
critério, com muita prudência!"

A lição é digna de proveito; porque, na expressão dos clássicos, o digno
doutor -- sabe armar no barbeito à perdiz curvado ichó; e pode, sem
competidor, ensinar aos perfumeiros como da barrilha fazem-se delicados
frascos, para finíssima pomada.

E toda esta magra filosofia da paciência o exímio mestre aprendeu no
exílio; depois que \emph{imprudentemente} pretendeu um lugar de
professor, de preparatórios, no curso anexo à Faculdade de Direito, que
\emph{atrevidamente} concorreu a esse cargo, que \emph{estouvadamente} o
aprovaram, que \emph{inopinadamente} o nomearam, e que
\emph{prudentemente} S. M. o Imperador mandou cassar o decreto de
nomeação.

Dá-se agora uma grave anomalia, digna do mais sério reparo: pois o exmo.
sr. dr. Joaquim Pedro, tão pródigo e oficioso em dar lições de
prudência, quando viu cassado o decreto de sua nomeação, não teve
prudência para suportar, em silêncio, este ato de violência; correu à
imprensa; e, nas terríveis contorções de tremenda eclampsia\footnote{.
  Grave convulsão que ataca gestantes e parturientes. Gama usa da
  própria linguagem médica do campo de especialização do oponente para
  lhe deitar o ataque.} atirou ao chefe da nação, ao deus do seu
partido, os mais ferinos baldões\footnote{. Impropérios, injúrias.}, os
mais pesados apodos\footnote{. Ditos depreciativos para ridicularizar
  alguém.}.

O exmo. doutor é digno membro da confraria do Frei Thomaz; do que prega
nada faz.

E, deveras!

Um homem que veste-se regularmente, que alimenta-se bem, que goza das
melhores comodidades, em uma sociedade opulenta, que sabe evitar o frio
e o calor, que frequenta divertimentos e calça luvas de pelica, deve,
com sobeja\footnote{. Excessiva, demasiada.} honradez e abundante
filosofia, aconselhar aos seus irmãos negros, aos cativos, mas que
nasceram tão livres, como ele, que são vítimas de um crime horroroso,
seminus, expostos ao sol, ao frio, e à chuva, vestidos de trapos,
sacudidos à \emph{bacalhau}\footnote{. Chicote, chibata usada para
  tortura.}, que têm por lenitivo a tortura, e por luvas os calos
levantados pela palmatória e pelo cabo da enxada, que sejam prudentes,
que suportem o flagício, que se habituem com os castigos, tenham
paciência, porque mais sofreu Jesus Cristo, e dos desgraçados é o reino
do céu!..................

Isto, meu nobre amigo, é hipocrisia feita de arminhos\footnote{. Coisa
  macia, delicada.}, embrulhada em pergaminho, e vendida por pomada: não
gasto desta \emph{droga}; e tenho por suspeitos estes mercadores
piedosos, estes Bicitos do sentimentalismo.

Não pensem que eu seja desafeto ao exmo. dr. Joaquim Pedro, e que me
esteja aproveitando desta circunstância para combater as suas ideias.
Sou seu amigo, e disto lhe tenho dado provas.

As nossas ideias políticas são opostas; os nossos sentimentos
irreconciliáveis: somos duas entidades distintas: eu amo as revoluções;
e julgo ser um ato sublime dar a vida pelas ideias.

Ele detesta a revolução; mas, se a fizerem, fora de perigo, apanharia os
frutos.

Eu sou um louco; ele, um homem de critério.\\
Dou-lhe os meus sinceros parabéns.

Se algum dia o Brasil produzir um Alighieri, se este escrever uma nova
\emph{Divina Comédia}, e todos tivermos de figurar nesse poema, o Exmo.
Dr. será transformado em \emph{ponte}; por ela passarão todos, bons e
maus: a ponte é a materialização da \emph{imparcialidade}.\footnote{.
  Refere-se a Dante Alighieri (1265-1321), poeta, escritor e político
  florentino. Como se vê, o autor da obra-prima \emph{A Divina Comédia}
  servia de inspiração para Gama refletir sobre o Brasil. Essa, contudo,
  não era a primeira vez que Gama o citava. Cf. \emph{Carta ao exmo. sr.
  deputado dr. Tito de Mattos} {[}III{]}, 13/04/1868.}

Sempre teu

LUIZ GAMA.

\textbf{A EMANCIPAÇÃO AO PÉ DA LETRA}

\textbf{*didascália*}

\emph{Uma das vias da encruzilhada de 1º de dezembro de 1880 foi o
artigo "Emancipação". Depois dele, outros três artigos lhe dão
sequência, tendo a defesa de José do Patrocínio, em particular, e do
movimento abolicionista, de modo geral, como mote do discurso. Ao todo,
portanto, são quatro artigos com abordagem, temática, aliados e
contendores semelhantes. Assim, todos os artigos podem ser lidos como
partes de uma mesma série, haja vista que versam, principalmente, sobre
a defesa de Patrocínio. E não foi qualquer defesa. Gama fez história uma
vez mais ao construir um argumento inédito para a imprensa paulista e
quiçá brasileira da época. Um homem negro defendia um outro homem negro
-- na imprensa dominada por homens brancos -- e ligava a chave da raça
como argumento síntese para se compreender e destruir a escravidão no
Brasil. A mediação da categoria raça, todavia, é complexa. Aos homens
brancos escravocratas, para quem a pele negra era um defeito, um vício e
um estigma, Gama trazia a severa lembrança de que "esta cor é a origem
da riqueza de milhares de salteadores" e que "esta cor convencional da
escravidão, como supõem os especuladores, à semelhança da terra, através
da escura superfície, encerra vulcões, onde arde o fogo sagrado da
liberdade". Aos estrangeiros, certamente em sua maioria homens brancos,
favoráveis e apoiadores do movimento abolicionista, Gama trazia outro
juízo. Os "bondosos estrangeiros, que convivem neste país, sem temor da
negridão da nossa pele" mereceriam crédito distinto. A cor estava em
disputa. Líder dos abolicionistas de São Paulo, Gama visava formar uma
consciência coletiva favorável à grande causa nacional e isso passava
por agregar uma massa de gente de diferentes origens e posições sociais.
Para isso, os expedientes retóricos pareciam oscilar entre ter e não ter
cor. Os ataques ao caráter de Patrocínio, aspecto central do desagravo
de Gama, demonstram isso. Era um homem negro e a defesa racial estava
absolutamente explícita desde o primeiro ponto do argumento. Num dado
momento, porém, o que estava em debate eram tão só atributos morais e
cívicos -- inteligência, brio, patriotismo, nobreza de caráter,
honradez. Atributos esses "que não têm cores", grifava Gama.}

\textbf{46. EMANCIPAÇÃO {[}I{]}}\footnote{. In: \emph{Gazeta do Povo}
  (SP), Publicações Pedidas, 01/12/1880, p. 2.}

\textbf{*didascália*}

\emph{A defesa que Gama faz de José do Patrocínio é uma das páginas mais
brilhantes da história do abolicionismo. Nesse texto, Gama crava o
racismo como ordenador da mentalidade antiabolicionista. A retórica
elegante e incisiva diante da abjeta agressão de que Patrocínio fora
alvo, com a chancela da redação da Província de S. Paulo, demonstra como
Gama modulava o estilo de seu discurso pelas nuances do debate que
travava. A veemência do argumento é fora de questão. Com tamanha
eloquência, podemos até ver o advogado negro na tribuna do júri a falar
para o conselho de sentença. Porque onde fala de Patrocínio fala
certamente de um irmão. "Em nós até a cor é um defeito, um vício
imperdoável de origem, o estigma de um crime", crescia na tribuna o
tribuno negro com a habilidade retórica de quem graduava os predicados
-- "defeito", "vício", "estigma" -- para enfatizar o horror do racismo.
Invocava o "fogo sagrado da liberdade" e dava traços fundamentais do seu
abolicionismo, que era também o abolicionismo negro e radical de
Patrocínio. Falar, escrever e esmolar, verbos caros ao seu estilo de
ação política. Por arremate, Gama fincava a pena na defesa de
Patrocínio, "porque nós, os abolicionistas, animados de uma só crença,
dirigidos por uma só ideia, formamos uma só família, visamos um
sacrifício único, cumprimos um só dever". Crença, ideia, família,
sacrifício, dever. Gama escrevia, portanto, palavras-chave do
abolicionismo que praticava desde o final da década de 1860. }

\emph{***}

Ilustrado redator.

Acabo de ler, sem espanto, mas com pesar, o contristador\footnote{.
  Desolador, triste.} escrito, publicado na SEÇÃO LIVRE da
\emph{Província} de hoje, contra o distinto cidadão José do Patrocínio.

Em nós até a cor é um defeito, um vício imperdoável de origem, o estigma
de um crime; e vão ao ponto de esquecer que esta cor é a origem da
riqueza de milhares de salteadores, que nos insultam; que esta cor
convencional da escravidão, como supõem os especuladores, à semelhança
da terra, através da escura superfície, encerra vulcões, onde arde o
fogo sagrado da liberdade.

O irrefletido brasileiro, que, sob a inscrição supra, teve a
infelicidade de escrever e publicar aquele vergonhoso artigo, a que
aludo, é de espírito mais humilde que os míseros escravos, cujas
manumissões\footnote{. Alforrias, demandas de liberdade.} advogamos.

Nós que falando, escrevendo, e esmolando, de porta em porta, somos
acolhidos com piedoso sorriso, pelos bondosos estrangeiros, que convivem
neste país, sem temor da negridão da nossa pele, que nos franqueiam a
sua bolsa, e nos prodigalizam\footnote{. Doam generosamente.} o seu
óbolo,\footnote{. Esmola, donativo de pouca monta.} para remissão dos
\emph{elefantes negros da lavoura}, temos, por certo, sobejo\footnote{.
  Demasiado, de sobra.} motivo para enojarmo-nos dessa
parolagem\footnote{. Tagarelice, falatório.} sáfia,\footnote{.
  Grosseira, inculta.} indigna da imprensa de um país culto.

Vim ao encontro do gratuito ofensor do cidadão José do Patrocínio,
porque nós, os abolicionistas, animados de uma só crença, dirigidos por
uma só ideia, formamos uma só família, visamos um sacrifício único,
cumprimos um só dever.

José do Patrocínio, por sua elevada inteligência, pelos seus brios, pelo
seu patriotismo, pela nobreza do seu caráter, pela sua honradez,
\emph{que não tem cores}, tornou-se credor da estima, e é digno dos
louvores dos homens de bem.

Ele não precisa desta inculta lição, de bárbaro abissínio,\footnote{.
  Relativo à Abissínia, na região da atual Etiópia. Expressão
  evidentemente pejorativa que contrasta com os usos de referenciais
  africanos no próprio discurso de Gama. Por destoar frontalmente com o
  autor que assinou "Getulino", cantou as "musas de Guiné" e enalteceu a
  "Líbia adusta", todas elas imagens elogiosas à África, pode-se aventar
  que o emprego de "abissínio", nesse contexto, quereria mexer com os
  brios do oponente a todo custo.} para saber que o Sol, quando dardeja
raios da mais alta esfera sobre a lama, desta desprendem-se
miasmas.\footnote{. Fedentina, exalação pútrida que emana de matéria
  orgânica em decomposição. A metáfora é rica em significados, sendo
  provável que Gama estivesse comparando o ofensor de Patrocínio com a
  lama e seu comentário racista com a fedentina que ela exala sob
  intenso calor.}

S. Paulo, 1º de Dezembro de 1880.

L. GAMA.

\textbf{47. {[}EMANCIPAÇÃO - II{]} À REDAÇÃO DA "PROVÍNCIA"}\footnote{.
  In: \emph{A Província de S. Paulo} (SP), Seção Livre, 04/12/1880, p.
  1.}

\textbf{*didascália*}

\emph{Essa breve réplica, Gama defende os termos de seu artigo anterior,
"Emancipação", sobretudo no que diz respeito ao desagravo que fez em
relação a José do Patrocínio, aqui referenciado como "um dos mais
distintos patriotas". A frase que encerra o texto -- Eu também fui
jornalista; sei que um periódico não é uma Vestal, é uma Bíblia." -- tem
não só eloquência, como demarca sua posição profissional e política no
início da década de 1880. A advocacia e a militância republicana eram,
afinal, suas tribunas de luta.}

\emph{***}

MEUS HONRADOS AMIGOS.

A declaração que fizestes em o NOTICIÁRIO da vossa folha de hoje
obriga-me a uma explicação.

Lamentei que nas páginas ilustradas da conceituada \emph{Província}
fosse inserida aquela descompreendida parlanda,\footnote{. Falatório,
  palavreado, discussão acalorada.} ofensiva da dignidade de um dos mais
distintos patriotas; mas não fiz, nem com isto podia fazer, censura à
briosa redação; e menos ainda desconsiderarei a liberdade de imprensa,
que constitui um direito sagrado.

Eu também já fui jornalista; sei que um periódico não é uma
\emph{Vestal},\footnote{. Essa é uma daquelas frases para ser lida e
  relida. Vestal, antiga sacerdotisa do culto à Vesta, divindade do fogo
  para os antigos romanos, assim como termo utilizado à época para
  designar a imagem de uma mulher casta e virtuosa, é aqui recuperado
  por Gama em seu sentido sagrado, ainda que, ou justamente, em
  contraposição à Bíblia.} é uma \emph{Bíblia}.

LUIZ GAMA.

\textbf{48. EMANCIPAÇÃO {[}III{]}}\footnote{. In: \emph{A Província de
  S. Paulo} (SP), Seção Livre, 15/12/1880, p. 1.}

\textbf{*}\textbf{didascália*}

\emph{Gama segue a linha do artigo de igual nome publicado em
01/12/1880, onde defendeu José do Patrocínio dos ataques da redação da
Província de S. Paulo. Atento ao noticiário e aos colunistas do jornal,
Gama pinça uma opinião antiabolicionista que reivindicava o boicote da
Gazeta da Tarde -- "folha ostensivamente abolicionista" -- como protesto
contra a agenda política e econômica relacionada à Abolição da
escravidão. }

\emph{***}

Ilustrados redatores.

Há poucos dias, alguém, que se dá como agricultor, convidou aos seus
colegas para rejeitarem as assinaturas da \emph{Gazeta da Tarde}, folha
ostensivamente abolicionista que se publica na Corte.

Eu, porém, no intuito de prestar valioso serviço aos verdadeiros
agricultores, amigos do país, valho-me da SEÇÃO LIVRE do vosso
conceituado jornal não só para transcrever os luminosos trabalhos da
grande propaganda nacional, como para difundir as puríssimas ideias
econômicas e políticas pregadas magistralmente pela digna redação
daquela apreciada folha.

Vosso, L. GAMA.

\textbf{49. A EMANCIPAÇÃO} -- \textbf{AO PÉ DA LETRA {[}IV{]}}\footnote{.
  In: \emph{Gazeta do Povo} (SP), Publicações Pedidas, 18/12/1880, p. 2.}

\textbf{*didascália*}

\emph{No contexto da polêmica entre Gama e a redação da Província,
acirrada desde o início de dezembro de 1880, esse artigo sobe o tom da
disputa e representa uma espécie de ruptura entre antigos aliados
políticos. Embora não mencione o nome de Gama, o editorial da Província
daquele mesmo dia, citado abaixo, tinha um sujeito a quem aquelas linhas
faziam referência. O "exaltamento e fervor na defesa da ideia"
abolicionista, os "excessos", "os ímpetos de um entusiasmo", os "ânimos
exaltados" falavam de uma pessoa em particular. Gama, que
performaticamente se apresentava como "fabricante de sátiras, em forma
de carapuças", vestiria, conforme disse, "as gorras que me cabem, e que
se acham pendentes do editorial a que aludo". A réplica seria histórica.
Chamava o editorial da Província de "conselhos evangelizadores, escritos
por ateus", que "cogitavam, de barriga para o ar nos meios de esperar a
queda pacífica e voluntária da monarquia desoladora, por milagre das
evoluções calmas, da portentosa sociologia positivista". Era uma
contestação moral, sociológica e política sem qualquer concessão
retórica. Com o sarcasmo que lhe era próprio, Gama deixava explícito que
se dependesse de seus "distintos correligionários, adoradores prediletos
da deusa PREGUIÇA", a Abolição poderia ser adiada até, quem sabe, nunca
acontecer. Gama tinha a urgência da liberdade. Em expediente retórico
arrebatador dizia que até aceitaria as ideias da Província se a ele essa
opção existisse. Em suas palavras: "eu de bom grado aceitaria se me não
achasse ao lado de homens livres, criminosamente escravizados e
pleiteando contra os salteadores do mar, os piratas da costa da África".
A resposta definitivamente ia ao pé da letra. }

\emph{***}

Os meus ilustres e honrados amigos da redação da \emph{Província de São
Paulo} deram, hoje, a lume, escrito sobre gelo, um curioso e memorável
editorial, relativamente aos propagandistas da abolição da escravatura,
que assim começa:

"A propaganda abolicionista está sendo dirigida inconvenientemente por
alguns cidadãos, cujo exaltamento e fervor na defesa da ideia não dão
lugar à calma para poderem medir os efeitos de seus discursos e
escritos.

A agitação que se notava nos espíritos, lá na Corte, vai se estendendo
às províncias e, portanto, tornando-se mais perigosa e talvez menos
eficaz em seus resultados.

Não podemos acompanhar os excessos nem louvar os ímpetos de um
entusiasmo embora sincero, mas incontestavelmente contrário à execução
de uma reforma que não devia ser agitada fora do terreno científico,
segundo a medida do critério positivo.

Pregar a emancipação, invocando o \emph{bom Deus}, pondo em contribuição
os princípios absolutos da justiça divina, da liberdade como dom sagrado
que nos foi conferido pela Providência, inverter a ordem dos fatores do
progresso social, querendo que a minoria tenha o direito de impor à
maioria, pela força, a solução pronta de um problema complexo, cujo
estudo se deve fazer no meio mesmo em que se apresenta cheio de
dificuldades aos ânimos exaltados, não nos parece de boa política.

Os fenômenos sociais não dependem exclusivamente do talento daqueles que
mais se dedicam a uma causa e que a manejam provocando as massas
inconscientes, procurando arrastá-las pelo brilho da eloquência. Eles se
operam por leis naturais e aparecem quando as circunstâncias lhes
proporcionam a oportunidade. Daí vem que as melhores reformas são
aquelas que nascem do convencimento real do povo; são estas as que
consultam as necessidades da época e exprimem o ato positivo da
soberania nacional."

..................................................................................................

Estas palavras, estes conselhos evangelizadores, escritos por ateus, e
por pena republicana, se bem que antirrevolucionária, não me causaram
admiração, e menos ainda abalaram-me o espírito; pois que eu sei, de há
muito, que esses meus distintos correligionários, adoradores prediletos
da deusa PREGUIÇA, deitados sob o \emph{gitai da paciência}, cogitam, de
barrigas para o ar, nos meios de \emph{esperar a queda pacífica e
voluntária} da monarquia desoladora, por milagre das evoluções calmas,
da portentosa sociologia positivista; e, nesta cômoda posição, esperam
que o fruto amadurecido, por exclusiva ação do tempo, lhes caía de manso
à flor dos lábios, a fim de que eles peçam ao primeiro transeunte a
graça de lho empurrar, com jeito, para dentro da boca.

Não é uma censura que faço aos meus respeitáveis amigos; estas humildes
considerações são antes um preito\footnote{. Tributo, manifesto.} de
homenagem rendido, com sinceridade, ao seu elevado talento, pela
maravilhosa compreensão dos áureos princípios e práticas salutares da
\emph{salvadora política positivista}, que eu de bom grado aceitaria, se
me não achasse ao lado de homens livres, criminosamente escravizados e
pleiteando contra os salteadores do mar, os piratas da costa da África.

Ao positivismo da macia escravidão eu anteponho o das revoluções da
liberdade; quero ser louco como John Brown\footnote{. John Brown
  (1800-1859) foi um abolicionista radical que liderou insurreições
  armadas contra a escravidão. Foi condenado à pena de morte e passou à
  história como mártir da Abolição nos Estados Unidos da América.}, como
Espártacos\footnote{. Espártacos (109 a.C-71 a.C.) foi um
  gladiador-general, estrategista e líder popular que escapou da
  escravidão a que era submetido e, num levante de grandes proporções,
  organizou um exército que enfrentou o poder central de Roma na
  Terceira Guerra Servil (73 a.C-71 a.C.). São diversas as citações de
  Gama a Espártacos, grafado de variadas maneiras, a exemplo de
  Spartacus, o que revela sua admiração e até mesmo veneração pela
  história do mártir que venceu o cativeiro e lutou pelo fim da
  escravidão.}, como Lincoln\footnote{. Abraham Lincoln (1809-1865) foi
  um advogado e estadista que presidiu os Estados Unidos da América
  entre 1861-1865, período em que o país atravessou uma Guerra de
  Secessão e pôs fim ao regime escravista.}, como Jesus; detesto, porém,
a calma farisaica de Pilatos.\footnote{. Pôncio Pilatos foi governador
  da Judeia (26-36 a.C.) e presidiu o julgamento que sentenciou a
  crucificação de Jesus. A menção, nesse caso, aponta para uma espécie
  de sentimento fingido que dominava o carrasco do mártir da
  Cristandade.}

Fui, em outros tempos, quando ponteava ritmas, fabricante de sátiras, em
forma de \emph{carapuças}; e, ainda hoje, tenho o vezo da arte.

Dada esta boa razão, indispensável, em face das complicações emergentes,
declaro que aceito, sem escrúpulo, as \emph{gorras}\footnote{. Toucas,
  carapuças.} que me cabem, e que se acham pendentes do editorial a que
aludo.

Peço vênia,\footnote{. Licença, permissão.} porém, para replicar.

Eu, assim como sou republicano sem o concurso dos meus valiosos
correligionários, faço a propaganda abolicionista, se bem que de modo
perigoso, principalmente para mim, de minha própria conta.

Estou no começo: quando a justiça fechar as portas dos tribunais, quando
a \emph{prudência} apoderar-se do país, quando os nossos adversários
ascenderem ao poder, quando da imprensa quebrarem-se os prelos, eu
saberei ensinar aos desgraçados a vereda do desespero.

Basta de \emph{sermões}; acabemos com os idílios.
\includegraphics[width=\textwidth,height=0.17917in]{media/image1.png}

Lembrem-se os evangelizadores do positivismo que nós NÃO ATACAMOS
DIREITOS; PERSEGUIMOS O CRIME, por amor da salvação de infelizes; e
recordem-se, na doce paz dos seus calmos gabinetes, que as alegrias do
escravo são como a nuvem negra: no auge transformam-se em lágrimas.

1880 -- 18 de Dezembro.

LUIZ GAMA.

\textbf{A DEFESA DA CARTA A FERREIRA DE MENEZES}

\textbf{*didascália*}

\emph{Os quatro textos a seguir -- dois de Luiz Gama e dois outros
escritos em resposta a Gama -- giram em torno de assuntos tratados na
Carta a Ferreira de Menezes. Um comendador corre à imprensa para dizer
que não era um dos envolvidos em um crime bárbaro; e outro colunista
tratou logo de pôr em dúvida uma informação que Gama divulgava. Ambas as
cartas tiveram suas respostas. A interlocução pública, por sua vez,
expressa que a Carta a Ferreira de Menezes estava fazendo barulho. E
esse era, afinal, um dos objetivos daquela peça histórica, que pode ser
lida hoje como um dos textos mais importantes da história do
abolicionismo.}

\textbf{50. LIMEIRA -- AO SR. LUIZ GAMA}\footnote{. In: \emph{A
  Província de S. Paulo} (SP), Seção Livre, 21/12/1880, p. 2.}

\textbf{*didascália*}

\emph{Embora carregue apenas três asteriscos como assinatura, de modo a
ocultar o nome civil de seu autor, a carta oriunda de Limeira, interior
de São Paulo, certamente tinha sido escrita por alguém interessado em
fragilizar a narrativa da Carta a Ferreira de Menezes. Evidente sinal,
pode-se constatar logo de saída, de que a carta alcançava uma grande
repercussão. O limeirense dizia que Gama divulgava uma informação
inexata, o que tornaria sua denúncia, em alguma medida, duvidosa.
Aparentemente polida e cordial, a carta "Ao sr. Luiz Gama" buscava, no
fundo, tirar a credibilidade das denúncias e ideias que a Carta a
Ferreira de Menezes trazia ao público. }

\emph{***}

Lemos com surpresa a carta do sr. Luiz Gama dirigida ao dr. Ferreira de
Menezes, inserta na \emph{Gazeta do Povo} de 14 do corrente, no que se
refere ao município da Limeira\footnote{. Cidade do interior paulista,
  distante 140 km da capital.}.

Lamentamos que uma falsa informação em negócio de tanta gravidade
tivesse concorrido para que S. S. formasse do povo Limeirense um juízo
tão desfavorável.

Acostumados a respeitar o caráter de S. S., os nobres e generosos
sentimentos que externa em sua carta, e partilhando a justa indignação
de que se acha possuído, em face de fatos tão revoltant{[}es{]},
pesa-nos a pecha de assassinos e salteadores, por um fato que
desconhecemos.

Asseveramos, pois, ao sr. Luiz Gama, que o fato narrado em sua carta e
lançado em conta ao município da Limeira é inexato e filho ou do
equívoco, ou de despeitado pouco cavalheiro, que propositalmente procura
nodoar\footnote{. Desonrar, macular.} a nossa vida social com a
imputação de um crime nefando, de uma selvageria sem nome.

Temos notícia de um fato análogo, cuja reprovação pública deve ter
assinalado o seu autor, porém esse fato, como poderá informar-se o sr.
Luiz Gama, corre por conta do outro município, e ninguém com verdade
poderá afirmar que as autoridades da Limeira se tenham tornado
moralmente cúmplices de um tal atentado, tolerando que se refugie em seu
seio um celerado\footnote{. Criminoso cruel, facínora.} de tal quilate.

Repugna-nos o papel de delatores; a ninguém denunciamos, defendemo-nos
de uma grave acusação que depõe contra o grau de civilização e
sentimentos humanitários da sociedade Limeirense.

Concluindo, diremos que as autoridades da Limeira, de 1878 para cá,
timbram em respeitar os direitos individuais, sem fazer exclusão da
qualidade, posição e condição de quem quer que seja, de que têm dado
sobejas\footnote{. Demasiadas.} provas, e assim procedem de fronte
altiva, porque costumam sempre pautar seus atos pela consciência pura do
cumprimento do dever e obediência às leis do país, disto podem dar
irrecusável testemunho as pessoas insuspeitas e de conceito não só da
localidade, como da capital.

Esperamos, confiados no critério e circunspecção do sr. Luiz Gama, que
diante de que vimos de expor, reformará seu juízo a nosso respeito e nos
fará a devida justiça.

***\footnote{. Sem assinatura além desses três asteriscos.}

\textbf{51. REPARAÇÃO DEVIDA}\footnote{. In. \emph{Gazeta do Povo} (SP),
  Publicações pedidas, Município da Limeira, 21/12/1880, p. 2.}

\textbf{*didascália*}

\emph{Como o título indica, Gama retificou uma imprecisão da Carta a
Ferreira de Menezes. No entanto, ao fazê-lo, Gama demonstrou como sua
rede de informantes agia rápido e habilmente. Para que não deixasse a
contestação sem resposta, Gama apresentou um fragmento de uma carta
privada que dava lastro à sua versão. Esse tipo de evidência servia de
prova no debate público e ninguém ousaria duvidar da validade que ela
possuía. Sem embargo, o mérito da denúncia seguia de pé: "O crime existe
impune" e o auto processual foi visto, quiçá manuseado e diligenciado
pela testemunha que informava Gama. Diante da gravidade do caso, e com a
evidência reforçada nesse artigo, a discussão se o autor era ou não da
cidade tornava-se uma questão menor. }

\emph{***}

Ao respeitável cavalheiro, que não tenho a honra de conhecer, e que, com
tanta e imerecida urbanidade, a mim se dirige, pela \emph{Província} de
hoje, e reclama contra a atribuição de um crime, que fiz, à pessoa
daquela cidade, dou-me pressa em responder com o trecho seguinte,
extraído de uma carta:

.......................................................

"Acabo de ler a \emph{Província}; se quiseres responder ao articulista
da Limeira\footnote{. Município do interior paulista, distante 140 km da
  capital.}, dou-te a explicação do fato, que fora-te por mim narrado.

O crime existe impune; o que afirmei é a pura verdade; e o criminoso, se
não é da Limeira, lá residiu; e acha-se atualmente em...

Vi o processo; o crime foi cometido não na Limeira, mas na comarca
de......... \footnote{. Gama ocultou o nome da comarca.}

A pessoa que da Limeira escreveu o artigo tem conhecimento do fato
\emph{e o afirma com reserva louvável}. As circunstâncias são
atrocíssimas, muito mais carregadas do que as da tua carta ao dr. F. de
Menezes."

.......................................................

S. Paulo, 21 de Dezembro de 1880.

LUIZ GAMA.

\textbf{52. O EXMO. SR. COMENDADOR J. A. PAULA MACHADO}\footnote{. In:
  \emph{A Província de S. Paulo} (SP), Seção Livre, 12/01/1881, p. 2. A
  réplica do comendador Joaquim Antonio de Paula Machado pode ser lida
  em \emph{A Província de S. Paulo} (SP), Seção Livre, 11/01/1881, p. 2.}

\textbf{*}\textbf{didascália*}

\emph{Mais um elemento que reforça a alta repercussão que a Carta a
Ferreira de Menezes alcançou. O comendador Paula Machado teve de vir a
público, uma vez que havia a suspeição de que fosse ele o comendador
mencionado na denúncia de Gama. Não era. Gama, por sua vez, isentou
Paula Machado de relação com o fato que denunciou e ainda acrescentou um
elemento sugestivo: o "fato gravíssimo a que aludi está arquivado em
cartórios e o seu autor tem o nome registrado nos autos". }

\emph{***}

O \emph{comendador} a quem referi-me, na carta que enderecei ao meu
nobre amigo dr. José Ferreira de Menezes, e que vem inserta na
\emph{Gazeta da Tarde}, não é o exmo. sr. comendador Joaquim Antonio de
Paula Machado.

O fato gravíssimo a que aludi está arquivado em cartórios e o seu autor
tem o nome registrado nos autos.

S. Paulo, 11 de Janeiro de 1881.

LUIZ GAMA.

\textbf{53. RETIFICAÇÃO NECESSÁRIA}\footnote{. In: \emph{Gazeta de S.
  Paulo} (SP), Ineditoriais, 14/01/1881, p. 3.}

\textbf{*didascália*}

\emph{Gama comenta uma notícia dúbia que dava a entender coisa diversa
da que de fato ocorreu. Antonio e Raymundo, escravizados em Campinas, na
fazenda de Polycarpo Souza Aranha, fugiram do domínio senhorial e
procuram o advogado abolicionista na capital. "No cumprimento do meu
dever", defendeu-se Gama, "e para a manutenção da lei, fiz
apresentá-los, com uma petição, assinada por mim, à autoridade
competente". Souza Aranha era bastante conhecido de Gama. Basta dizer
que ambos haviam acabado de sair de um sério litígio no Tribunal da
Relação de São Paulo em razão do pedido de habeas-corpus do africano
Caetano. O fato de Gama peticionar em favor de Antonio e Raymundo
reforça que Gama não se deu por vencido na célebre causa de Caetano. Ao
contrário: voltaria à carga contra o mesmo fazendeiro escravocrata que
era o terror dos trabalhadores escravizados da região de Campinas. }

\emph{***}

A digna redação da \emph{Gazeta do Povo}, em o seu número de hoje,
publicou o seguinte:

"À estação urbana do Brás\footnote{. Também era conhecida como Estação
  do Norte. No mesmo local, atualmente, está a estação ferroviária do
  Brás.} foram recolhidos os pretos Antonio e Raymundo, escravos de
Joaquim Polycarpo de Souza Aranha\footnote{. Joaquim Polycarpo Aranha
  (1809-1902), natural de Ponta Grossa (PR), foi fazendeiro e político
  estabelecido em Campinas (SP).}, fazendeiro de Campinas, por andarem
fugidos.\\
\emph{Foram ali apresentados pelo cidadão Luiz Gama}."

É inexata essa notícia; não apresentei escravo algum à \emph{Estação do
Brás}; surpreende-me a asserção.

Os indivíduos de que trata-se procuraram-me, para que eu os tirasse de
\emph{violências bárbaras} de que eram vítimas.

No cumprimento do meu dever, e para manutenção da lei, fiz
apresentá-los, \emph{com uma petição}, \emph{assinada por mim}, \emph{à}
\emph{autoridade competente}, que está procedendo a{[}s{]} diligências.

Esta é a verdade; e o fato bem diverso do que, seguramente por mal
informada, foi referido por aquela ilustre redação.

S. Paulo, 12 de Janeiro de 1881.

LUIZ GAMA.

\textbf{AGONIZA, MAS NÃO MORRE}

\textbf{*didascália*}

\emph{"Só agora me permite o tempo e a saúde responder", dizia Gama, nos
últimos meses de sua vida. A saúde andava frágil; e o tempo, raro. Nessa
seleção final de artigos, uma pequena e rápida miscelânea que trata de
direito, escravidão, liberdade e propaganda republicana. Gama escreve ao
imperador, abre o baú de seus papeis e de lá retira uma antiga poesia de
José Bonifácio, o Moço, assim como escreve um sólido parecer jurídico
sobre a ilegalidade da escravização por parte de corporações religiosas
da Igreja Católica. O conjunto dos seis artigos finais é como as notas
de um samba triste, que agoniza, mas não morre. Os artigos testemunham a
verve e as letras da "venerável ruína" -- para lembrar a célebre
expressão de Raul Pompéia -- que se tornara Luiz Gama nos meses finais
de sua vida. Ruína, é verdade, porque mesmo tudo que é sólido se
desmancha no ar.}

\textbf{54. MEU NOBRE AMIGO}\footnote{. In: \emph{Gazeta da Tarde} (RJ),
  Expediente, Luiz Gama, 21/07/1881, p. 2. A redação da \emph{Gazeta da
  Tarde} assim apresenta a carta de Gama: "A carta que esse ilustre
  democrata dirigiu à \emph{Gazeta do Povo}, de S. Paulo, recusando,
  como ontem noticiamos, o retrato a óleo que alguns amigos queriam
  mandar tirar para lhe oferecer, é a seguinte".}

\textbf{*}\textbf{didascália*}

\emph{A carta aberta lançada na Gazeta do Povo (SP) e reproduzida na
Gazeta da Tarde (RJ) é apresentada aos leitores como a voz de um líder
que, recusando a promoção pessoal, orienta que se empenhe todas as
forças na grande causa, isto é, na Abolição da escravidão. Assim, a
simples oferta de um presente foi recusada de antemão, sugerindo que
aqueles recursos fossem empregados "na libertação de um escravo". }

***

Sei que V. e mais alguns distintos cidadãos, constituídos em comissão,
tratam de angariar donativos para ofertarem-me o meu retrato.

Penhora-me sobremodo tão elevada quão imerecida prova de apreço. Devo,
porém, declarar a V. com a rude franqueza que é me própria, que esta
prova de estima e consideração contrária é desagradavelmente à nativa
modéstia de meus sentimentos.

Digne-se V., portanto, e os seus respeitáveis amigos, de aceitar um
conselho não pedido, acompanhado de uma humilde e sincera rogativa.

Empregarem o dinheiro colhido, com algum auxílio, se precisão houver, na
libertação de um escravo, que indicarei. Assim prestaremos todos à
humanidade um relevantíssimo serviço, merecedor de melhor apreço, do que
a tela, na qual pretendem imortalizar-me a óleo.

Tenho que as sociedades são vítimas de três calamidades indistintas: a
religião, o rei, e a escravidão.

Trabalhar por extingui-las é um dever imprescindível do cidadão:
cumpramo-lo.

Sou, com muita consideração, de V. criado, obrigado e amigo,

\emph{Luiz Gama}.

\textbf{55. {[}REPRESENTAÇÃO AO IMPERADOR D. PEDRO II{]}}\footnote{. In:
  \emph{Gazeta do Povo} (SP), Publicações Pedidas, 10/09/1881, p. 2. O
  artigo foi republicado na edição seguinte da mesma \emph{Gazeta}.}

\textbf{*didascália*}

\emph{Nesse texto, os autores -- Gama à frente, secundado pelo estudante
Brasil Silvado e o médico Clímaco Barbosa -- endereçam uma representação
pública ao imperador Pedro II. O trio manifestava solidariedade aos
abolicionistas do Ceará e pedia que o imperador intervisse na então
província, impedindo que o seu presidente continuasse a fazer "por mera
perversão, o mal que lhe proíbe a lei". Que os leitores de hoje não se
percam por adjetivações elogiosas e uma tônica que guarda aparente
cordialidade. A dado momento, os autores simplesmente rompem a típica
estrutura e formalidade de uma representação à mais alta autoridade do
país para adverti-lo de que se a luta popular avançasse, a cabeça de
Pedro II estaria em perigo. Vejamos a "rude linguagem. Nas palavras de
Gama e seus companheiros: "Os povos são como os coveiros; quando
arrebatam as diademas, já as cabeças dos reis estão extintas". Sim! Em
comunicação aberta na imprensa -- e que possivelmente tenha ganhado a
forma solene de petição --, Gama e seus camaradas haviam incluído tal
expressão que significava, por um lado, o desejo revolucionário
republicano de decepar a monarquia pela guilhotina e, por outro lado,
sinalizava ao movimento republicano e abolicionista brasileiro que
deveriam todos unificar a luta política popular. Aliás, por falar em
camaradas e luta política, nesse texto encontra-se um breve e eloquente
inventário de levantes populares no Brasil -- da Insurreição Praieira
(1848-1850) à Revolta do Vintém (1880), ambas postas à luz da
Inconfidência e do martírio de Tiradentes (1792). No entanto, em
combinação explosiva, as lutas nacionais passavam a ser associadas à
luta internacional dos trabalhadores. Para comunista nenhum botar
defeito, Gama e seus camaradas escreviam que, a exemplo dos
trabalhadores ingleses, deveriam os poucos brasileiros radicalmente
abolicionistas e republicanos "levantar o espírito dos operários contra
o domínio opressor dos proprietários". Incomum, talvez inédita para a
imprensa da época, a expressão simboliza a união de bandeiras e uma
direção política que tinha visão de longo alcance. Não bastaria só
acabar com a escravidão e a perversa relação senhor / escravizado; era
preciso também projetar o futuro pós-Abolição e preparar um movimento
social que arregimentasse camponeses e operários para uma nova etapa da
luta política, agora rivalizada entre proprietários e trabalhadores. }

\emph{***}

Senhor!

S. Paulo, 7 de Setembro de 1881.

Perante as puras consciências, a verdade, qualquer que seja o modo de
sua manifestação, jamais foi uma irreverência e, menos ainda um
apodo.\footnote{. Dito depreciativo, irônico ou ultrajante.}

Para Vossa Majestade, que tem a virtude por hábito, pela elevação nativa
do seu augusto caráter, a falta de verdade e o servilismo, maiormente em
concorrências políticas ou administrativas, deve constituir imperdoável
defecção.\footnote{. Abandono de uma obrigação ou compromisso.}

Embora em rude linguagem, porque não temo-la aprimorada, favorável ao
objeto, e digna da majestade, aqui daremos em tudo a verdade, porque em
tudo a devemos.

São poucos, Senhor, os que assinam este papel; Vossa Majestade, porém,
sabe que o direito, o civismo, a dignidade, o patriotismo, a razão, se
não avaliam pelo peso, nem pelo número se medem.

Sete são os ministros de Vossa Majestade, que governam este vastíssimo
Império; um só homem, na culta Inglaterra (Joseph Arch),\footnote{.
  Joseph Arch (1826-1919) foi um líder sindical e político inglês. Em
  1872, ajudou a fundar e foi eleito o presidente do Sindicato Nacional
  dos Trabalhadores da Agricultura, movimento social que reunia mais de
  oitenta mil trabalhadores do campo e reivindicava salário digno e
  melhores condições de trabalho para toda a categoria. A referência
  chama a atenção para a leitura que Gama, Silvado e Barbosa faziam da
  luta política dos trabalhadores da Inglaterra e, em especial, da
  importância de uma liderança orgânica para "levantar o espírito dos
  operários contra o domínio opressor dos proprietários".} bastou para
levantar o espírito dos operários contra o domínio opressor dos
proprietários; único também é o Sol, que o mundo inteiro ilumina: o
número nem sempre vem ao caso.

Afirma-se em todo o país que Vossa Majestade Imperial é o centro de
harmonia, de paz e de felicidade social; é, porém, certo que, não poucas
vezes, os elementos de ordem, de tranquilidade pública, são gravemente
conturbados, comprometidos, nas províncias, e na própria capital do
Império, pelos prestigiosos delegados de Vossa Majestade.

A esta hora, Senhor, ao norte do Brasil, na heroica província do Ceará,
em face da lei, e tal é o assunto desta humilde representação, cidadãos
conspícuos, beneméritos, respeitabilíssimos, funcionários conceituados,
honestos servidores do povo, honrados pais de família, tão gloriosos
como Vossa Majestade, estão sendo acintosa, caprichosamente demitidos,
privados de trabalho, de meios de subsistência; e, destarte, perseguidos
com desumanidade!... E tudo isto se faz sob o fútil pretexto de que os
cidadãos demissionários ameaçam a riqueza, a segurança individual, a
propriedade, as instituições; porque são emancipadores de escravos;
opõem-se ao hediondo comércio de carne humana; abominam as surras;
detestam a tortura e a ignomínia;\footnote{. Humilhação, desonra,
  infâmia.} e fazem sacrifícios para a proscrição\footnote{. Extinção.}
do flagício,\footnote{. Sofrimento atroz.} da degradação, das torpezas
inauditas\footnote{. Extraordinárias, sem precedentes.} da escravidão!

E esta perseguição odienta, estas violências inqualificáveis, estas
misérias, estas monstruosidades administrativas, fazem-se perante o
mundo, que nos observa, em plena luz meridiana, sem rebuço, sem
reflexão, sem rubor, sob a égide imperial do sagrado nome de Vossa
Majestade!...

Que Vossa Majestade é grande filósofo, um sábio, não há {[}como /quem
possa{]} negá-lo; é também incontestável que somos nós um povo de
camelos; as provas superabundam por toda a parte; mas os escolhidos
delegados de Vossa Majestade perigosamente se esquecem de que os sábios
são nimiamente\footnote{. Demasiadamente, excessivamente.} justos e
rigorosos; e que os camelos têm, por índole, o mau vezo de {[}se{]}
atirarem com o carrego,\footnote{. Fardo.} quando excessivo ou por mau
posto...

Não acusaremos, não discutiremos, não entraremos em liça\footnote{.
  Disputa.} com o poderoso presidente do Ceará;\footnote{. Refere-se a
  Pedro Leão Veloso (1828-1902), jornalista e político que presidiu a
  província do Ceára entre os meses de abril e dezembro de 1881. A
  julgar pelo tempo entre a publicação desse artigo -- que certamente se
  somava a uma campanha pública na imprensa local e nacional -- e a
  exoneração de Veloso, vê-se que a pressão na imprensa pode ter surtido
  efeito na administração da província.} vem sabemos que uma autoridade,
que tem por si a força pública, o prestígio oficial do poder, milhares
de agentes, florestas de baionetas, e as verbas secretas da polícia à
sua disposição, nunca deixa de, por si, ter a razão: perante ele
damo-nos por vencidos.

Dirigimo-nos calculadamente a Vossa Majestade Imperial, cuja
cordura\footnote{. Sensatez, prudência.} nos atrai, cuja
longanimidade\footnote{. Virtude de suportar com paciência e resignação
  as contrariedades, vexames e insultos.} nos seduz, cuja graça nos
penhora; a Vossa Majestade, que reina, governa e administra, com
deslumbramento dos soberanos do universo; que exerce prudente arbítrio
sobre o ilustrado parlamento, sobre a grave magistratura, sobre os
ministros, sobre as academias, sobre os presidentes; que os pode \emph{e
deve cautamente advertir} de que o memorável dia 21 de de Abril de 1792,
em que a leal cidade do Rio de Janeiro cobriu-se de gala, e
esplendidamente iluminou-se, pela morte de Tiradentes, não mais voltará;
que o dia 1º de Janeiro de 1880 marca uma era de luto para a pátria, de
opróbrio\footnote{. Grande vergonha.} para o governo, e de tristeza para
Vossa Majestade; que, se os ministros têm à destra\footnote{. À direita.}
bravos Enéas, a quem prodigalizam \emph{merecimentos}, à custa do sangue
dos mártires, derramado nas barricadas da rua Uruguaiana, o povo tem as
matas, terá Pedro Ivo\footnote{. Pedro Ivo Veloso da Silveira
  (1811-1852), pernambucano de Olinda, foi um militar e líder político
  que teve destaque na Insurreição Praieira (1848-1850), luta armada
  ocorrida em Pernambuco que tinha por bandeiras, entre outras, o fim do
  Poder Moderador -- peça-chave da monarquia brasileira --, o voto
  livre, a liberdade de imprensa e a convocação de uma Assembleia
  Nacional Constituinte.} e Nunes Machado\footnote{. Joaquim Nunes
  Machado (1809-1849), pernambucano de Goiana, foi juiz de direito,
  desembargador e deputado. Foi um dos líderes da Insurreição Praieira
  (1848-1850) e é considerado um dos seus mártires.} para sagrá-los com
a imortalidade; que a província de S. Paulo está unida à de Minas
Gerais; que, em uma e na outra, dos gemidos dos escravos se poderá
compor um cântico à liberdade; e que este mesmo povo, em faustosa
restauração, assinalando um novo 7 de Abril, poderá responder, com
prudência, às arbitrárias demissões de hoje com o sinistro banimento do
monarca.

Há uma máxima do célebre chefe da dinastia dos Arsam{[}es{]}, na Pérsia,
que os reis não devem esquecer: "Ai dos príncipes cujos governos são
mais temidos que estimados".

Os espinhos das coroas, Senhor, não provém da soberania popular; nascem
das silveiras, que vêm dos governos impolutos.\footnote{. Virtuosos,
  honestos.} Os povos são como os coveiros; quando arrebatam as
diademas, já as cabeças dos reis estão extintas.

Os tronos, como as árvores seculares cobertas de parasitas, caem sem
raízes.

Digne-se Vossa Majestade Imperial de lançar benignas vistas sobre a
briosa província do Ceará, que, por inúmeros títulos, bem o merece; e de
impedir, com clemência ou com justiça, que o seu delegado governe os
infelizes habitantes daquela importante porção do Império à guisa dos
paxás da Turquia; e que o coíba de fazer, por mera perversão, o mal que
lhe proíbe a lei; uma vez que, por má vontade, por inépcia ou por
desídia, não faz o bem que deve aos seus administrados.

Somos, com muito respeito e consideração, Senhor, de Vossa Majestade
Imperial, concidadãos e veneradores.

LUIZ GONZAGA PINTO DA GAMA

BRASIL SILVADO\footnote{. João Brasil Silvado (1854-1911), nascido no
  Rio de Janeiro (RJ), foi advogado, educador, chefe de polícia do
  Distrito Federal durante parte do governo Campos Sales (1898-1902) e
  escritor. Durante o tempo de estudante na Faculdade de Direito de São
  Paulo, colaborou com ações de Gama na imprensa, assim como em eventos
  e associações abolicionistas. Em julho de 1882, meses após a
  publicação desse artigo, Silvado integraria a direção da Caixa
  Emancipadora Luiz Gama, movimento social organizado de auxílio mútuo
  para conquista de alforrias e direitos.}

DR. CLÍMACO BARBOSA\footnote{. Clímaco Barbosa (1839-1912), natural de
  Salvador (BA), foi médico, político e jornalista, sendo
  redator-proprietário da \emph{Gazeta do Povo (SP)} no início da década
  de 1880. Na qualidade de perito e avaliador, colaborou com Luiz Gama
  em diversas ações judiciais. Além da colaboração no foro, Barbosa foi
  também um dos médicos particulares que trataram da saúde de Gama. Cf.
  \emph{Carta pública a seus médicos}, 27/02/1878.}

\textbf{56. O DIGNO SR. DR. GUILHERME CAETANO DA SILVA}\footnote{. In:
  \emph{O Correio Paulistano} (SP), Seção Livre, 04/11/1881, p. 2.
  Guilherme Caetano da Silva, em 1877, foi juiz municipal e de órfãos de
  Jaboticabal. Cf. \emph{A Província de S. Paulo} (SP), Seção Livre,
  15/12/1877, p. 2.}

\textbf{*}\textbf{didascália*}

\emph{Comunicação pública que, embora endereçada a um único
escravizador, certamente possuía a classe senhorial como destinatária da
mensagem.}

\emph{***}

Benedicta, ex-escrava de S. S. foi regularmente alforriada. O ato é
irrevogável, legítimo, está no meu poder.

S. Paulo, 3 de Novembro de 1881.

LUIZ GAMA.

\textbf{57. ACAUTELEM-SE OS COMPRADORES}\footnote{. In: \emph{Gazeta do
  Povo} (SP), Publicações Pedidas, 25/11/1881, p. 3. A nota foi
  republicada diversas vezes em edições seguintes da mesma
  \emph{Gazeta}. Embora não tenha a assinatura de Gama ao final do breve
  texto, trata-se, muito provavelmente, de um escrito de sua autoria,
  não só pelo fato de anteriormente ter reclamado a legalidade do estado
  de liberdade de Benedicta, mas também em razão do acesso a informações
  detalhadas sobre as agruras pelas quais passava Benedicta, às portas
  da reescravização ilegal em município distante do que vivia.}

\textbf{*didascália*}

\emph{Em vinte dias, muita coisa mudou. Na nota anterior, Gama dizia que
Benedicta havia sido "regularmente alforriada" e que ela estava em seu
poder, isto é, aos seus cuidados. Na presente nota, contudo, a
reviravolta: Benedicta havia sido levada -- Em quais condições? À força?
-- para Campinas, a fim de ser reescravizada de papel passado.}

\textbf{***}

Benedicta, que o sr. Romão Leomil levou para Campinas, e que trata de
vender como suposta escrava do digno sr. dr. Guilherme Caetano da
Silva,\footnote{. Guilherme Caetano da Silva foi juiz municipal e de
  órfãos de Jaboticabal. Cf. \emph{A Província de S. Paulo} (SP), Seção
  Livre, 15/12/1877, p. 2.} é forra.\footnote{. Nesse contexto significa
  alforriada, liberta, ou que vive em situação de liberdade de fato.}

\textbf{58. À FORCA O CRISTO DA MULTIDÃO}\footnote{. In:
  \emph{Tiradentes} (RJ), {[}editorial{]}, 21/04/1882, pp. 1-2.}

\textbf{*didascália*}

\emph{Artigo de propaganda republicana. Único artigo de Gama publicado
no jornal Tiradentes (RJ), ele versa justamente sobre a figura do mais
conhecido líder da Inconfidência Mineira, Joaquim José da Silva Xavier.
Gama evoca imagens de sua complexa teologia política, indo de Terâmenes,
estrategista ateniense, a Washington, estrategista militar e presidente
estadunidense; de "Pedro, vacilante" na Jerusalém do cristianismo, ao
"Pedro Primeiro, o esquecido", do Brasil recém-independente. Há muitos
outros textos nos quais Gama explora e sobrepõe temporalidades políticas
distintas. Assim, a comparação improvável do Rio de Janeiro com
Jerusalém -- ou da Paris revolucionária com a Ouro Preto inconfidente --
não deve estranhar o leitor familiarizado com a retórica e o repertório
de metáforas políticas de que Gama lançou mão na imprensa. Em seus
escritos, contam-se dezenas de menções ao Calvário, ao Gólgota, ao
patíbulo, ao cadafalso, à forca, à cruz, assim como diversas
aproximações entre os valores republicanos e a mensagem humanista de
Jesus. Gama, afinal, estava interessado em escrever um "martirológio"
que unisse "os brasileiros e o povo hebreu" numa mesma tradição, onde,
nas lógicas da "musa da história", a Redenção fosse a senda da
"misteriosa evolução" da humanidade. }

\emph{***}

Por entre as sombras e as convulsões agitadas da noite imensa dos
séculos ergueu-se, ao Norte da América, um grupo de Gigantes.

À frente deles, Washington\footnote{. George Washington (1732-1799) foi
  um comandante militar e líder político que foi eleito o primeiro
  presidente da República dos Estados Unidos da América (1789-1797).},
pensativo como Arquimedes\footnote{. Arquimedes de Siracusa (287 a.C-212
  a.C.) foi um matemático, astrônomo e inventor grego de influência
  determinante para o desenvolvimento da ciência na Antiguidade. A
  imagem remete à ideia de um inventor prestes a obter grande revelação.},
com a ponta do gládio\footnote{. Espada.} sagrado embebida no sangue das
batalhas, inscreve no mapa das Nações -- os Estados Unidos; e
Franklin\footnote{. Benjamin Franklin (1706-1790), nascido em Boston,
  Estados Unidos da América, foi escritor, cientista, diplomata,
  político e estadista, sendo um dos "pais fundadores" da República
  norte-americana.}, o moderno Terâmenes\footnote{. Terâmenes (?-404
  a.C.) nasceu em Estíria, atual Áustria, foi estrategista militar e
  estadista ateniense de destacada ação política durante a Guerra do
  Peloponeso (431 a.C.-404 a.C.).}, arrebatando um raio ao Sol, com
lúcidas estrelas, grava no infinito a eterna legenda da Liberdade.

Uma misteriosa evolução faz o fatal clarão repercutir ao Sul;
despertaram os filhos do Brasil: em ruínas organizou-se a
\emph{Inconfidência}.

Esta associação revolucionária constituía um Apostolado completo.

Havia um Cristo naquele conjunto de regeneradores; um Pedro\footnote{.
  Pedro, o Apóstolo, foi um dos doze primeiros discípulos de Jesus e
  fundador da Igreja Católica Romana, no ano 30 da Era Cristã.},
vacilante; um Judas\footnote{. Judas Iscariot foi um dos doze primeiros
  discípulos de Jesus. De acordo com os Evangelhos, Judas traiu e
  entregou Jesus para seus captores em troca de trinta moedas de prata.},
inexcedível; a Ordem foi salva pela fé; a fé consolidou-se pelo martírio
do Mestre.

O dia 21 de Abril de 1792 designa o fatal acontecimento, o mais
memorável que registra a história da América Meridional.\footnote{.
  Outra denominação da América do Sul.}

As ruas que conduziam ao Calvário\footnote{. Calvário, ou Gólgota, é a
  colina na qual Jesus foi crucificado.} regurgitavam\footnote{.
  Transbordavam.} de magnificência; assemelhavam-se às festas da Páscoa
na Judéia.

Era imenso o concurso, um bulício\footnote{. Alvoroço, agitação.} de
cabeças como as ondas inquietas do oceano.

A tropa impotente, unida, compacta, atestava com soberba exuberância o
luxo do poderio, do mando, a fátua\footnote{. Presunçosa.} vaidade do
despotismo deslumbrado.

Nas janelas dos preparados edifícios ostentava-se, com opulência, o sexo
gentil; rebrilhavam as sedas, o ouro e os diamantes: os primores d'arte
desafiavam as obras-primas da natureza.

A Religião, com estudada humildade, dava-se em piedosa farsada; nos
templos reboavam\footnote{. Ecoavam, retumbavam.} festivos cânticos.

Sobre o patíbulo\footnote{. Lugar, geralmente um palanque montado a céu
  aberto, onde se erguia o instrumento de tortura (forca, garrote ou
  guilhotina) para a execução dos condenados à pena capital.}, à guisa
de uma sombra, estava um frade, de pé; com um braço elevado indicava a
eternidade. Acurvou-se um pouco, abraçou o penitente, beijou-lhe a corda
que, à feição de colar, adornava-lhe o pescoço, orvalhou-a de lágrimas.
Com a mão direita, que tinha pelas costas, apertou a do algoz: ambas
eram amigas velhas, costumavam ter destes encontros, estavam tintas de
sangue...

O sacerdote perorou\footnote{. Discursou com pedantismo, falsidade.} por
meia hora. Foi uma estrangulação moral de trinta minutos, lenta como um
capricho de inquisidor. Quando a vítima foi entregue ao carrasco,
restava apenas a morte física.

-- "Tu, contra o teu rei, nem os olhos levantarás."

Foram estas as palavras preambulares do pregador!

Teu rei?!

E o que é o rei senão a feitura do povo?

Quê?! Valerá mais o tarro\footnote{. O mesmo que vaso.} que o
oleiro?\footnote{. Aquele que faz cerâmica, que trabalha em olaria.}

Nos confrontos da teologia com o direito, são vulgares estes santos
absurdos da ortodoxia.

A soberania popular, excetuando-se o NOVENTA E TRÊS, é uma miséria
política, sob a régia forma de um escárnio sacramental.

...................................................................................................................................

À meia hora do dia, como hoje, há 90 anos, expirou aquele que neste
país, primeiro propusera a libertação dos escravos e proclamação da
República. Foi julgado réu de lesa-majestade, mataram-no, mas Tiradentes
morto, como o Sol no ocaso, mostra-se ao universo, tão grande como em
sua aurora.

...................................................................................................................................

A musa da história tem a sua lógica invariável e seu modo peculiar de
traduzir e registrar os acontecimentos.

O altar, as aras sacrossantas do martírio, aquele monumento mandado
levantar pelo vice-rei, pelos magistrados -- pelos fiéis servos da
rainha --, foi substituído por um patíbulo imperial, modelado em bronze;
em vez da forca, há uma estátua. Desapareceu Joaquim José da Silva
Xavier,\footnote{. Joaquim José da Silva Xavier (1746-1792), nascido na
  região de São João del Rei (MG), foi dentista, militar e o
  revolucionário que passou à história como o personagem-símbolo da
  Inconfidência Mineira (1789) e um dos mártires da luta pela
  independência do Brasil.} para ser mais lembrado; surgiu Pedro
Primeiro, o esquecido.\footnote{. Pedro I do Brasil, ou Pedro IV de
  Portugal (1798-1834), nascido em Queluz, Portugal, foi rei de Portugal
  e Algarves e imperador do Brasil.}

Mudaram-se os tempos.

A tragédia perdeu a sua época, a comédia entrou em voga, o lugar do
mártir está ocupado pela figura do cômico, é um arlequim sobre um
túmulo, é um escárnio, é uma indecência, é uma solenidade chinesa do
Paço de S. Cristóvão!...\footnote{. O paço de São Cristóvão foi uma das
  residências da família real portuguesa quando da transferência da
  Corte, de Lisboa para o Rio de Janeiro. Após a Independência do Brasil
  até a Proclamação da República, o paço foi a residência da família
  imperial, local, inclusive, onde nasceu o segundo e último monarca do
  Império, Pedro II. Atualmente, o lugar abriga o Museu Nacional de
  Arqueologia e Antropologia, na Quinta da Boa Vista.}

O êneo\footnote{. Relativo a bronze. Por sentido figurado, firme, tenaz.}
corcel, ousado como seu amo, atira brutalmente as patas por sobre as
cabeças dos miseráveis grandes, dos grandes miseráveis, e dos
miseráveis, que ainda existem sem qualificação.

Os brasileiros e o povo hebreu tiveram dois inspirados precursores da
sua regeneração.

O Rio de Janeiro, como Jerusalém, teve o seu Gólgota\footnote{. Gólgota,
  ou Calvário, é a colina na qual Jesus foi crucificado.}; dois grandes
pedestais, levantados por a natureza, para dois Redentores.

Dois Cristos exigiam dois mundos.

Um divinizou a cruz, o outro, a forca.

A cruz é o emblema da Cristandade, a forca o será da Liberdade.

O martirológio\footnote{. Lista dos que morreram ou sofreram por uma
  causa.} mostra dois pontos culminantes: o Calvário e o Largo do
Rocio.\footnote{. Refere-se à atual praça Tiradentes, no centro do Rio
  de Janeiro (RJ), que antigamente se chamava Largo Rocio, em homenagem
  ao largo homônimo da cidade de Lisboa, Portugal.}

Concidadãos: descubramo-nos, ajoelhamo-nos.

O altar é a pátria; a pátria está no cadafalso\footnote{. O mesmo que
  patíbulo.}.

Rendamos cultos a Tiradentes.

S. Paulo, 21 de Março de 1882.

LUIZ GAMA.

\textbf{59. {[}CARTA A HYPPOLITO DE CARVALHO{]}} \footnote{. In: \emph{A
  Província de S. Paulo} (SP), Seção Livre, Casa Branca, 26/04/1881, p.
  2.}

\textbf{*didascália*}

\emph{Gama autorizou o destinatário da carta "fazer o uso que quiser" da
mesma, o que resultou na publicação desta carta particular na imprensa
de São Paulo. Pela estrutura da comunicação -- e da réplica que Gama
escreve --, a carta, ainda que privada, voltava-se a outros leitores,
sobretudo aos cidadãos de Casa Branca, interior paulista. As cartas
revelam, por sua vez, informações bastante úteis para se compreender a
relação cliente-advogado no Brasil da época.}

\emph{***}

Ilmo. Sr. José Hyppolito.

S. Paulo, 24 de Abril de 1882.

Só agora me permite o tempo e a saúde responder a carta retro.\footnote{.
  A carta que Hyppolito de Carvalho enviou a Gama também foi reproduzida
  nesta mesma edição d'\emph{A} \emph{Província}. Leia:

  "Ilmo. Sr. Luiz Gama.

  Pela franqueza e ornamento de seu caráter, peço-lhe o seguinte favor:

  1º: Quem redigiu e escreveu a representação que dei contra o bacharel
  Fernando A. de Barros, juiz municipal desta {[}cidade; Casa Branca{]},
  ao exmo. ministro da justiça, em data de 9 do corrente?

  2º: Se, além de minhas informações, todas verbais, e os documentos que
  V. S. juntou como provas, alguém mais teve parte, por palavra, por
  escrito, ou intervenção por mais pequena que fosse?

  Peço à V. S. licença para fazer o uso que me convier de sua resposta.

  Sou de V. S. venerador, obrigado criado.

  \emph{José Hyppolito de Carvalho}.

  Casa Branca, 31 de Março de 1882''.}

Sobre documentos, que apresentou V. S., e com explicações, que
prestou-me, redigi a representação que, em seu nome, contra o juiz
municipal dessa cidade, o dr. Fernando Antonio de Barros, foi endereçada
a S. Excia. o ministro e secretário de estado dos negócios da justiça.
Este fato deu-se entre nós ambos exclusivamente, sem intervenção de
terceiro.

Desta minha resposta pode fazer o uso que quiser.

Sou, com o devido respeito e consideração, de V. S., atento criado.

\emph{Luiz Gama}.

\textbf{60. CARTA AO DR. CERQUEIRA CÉSAR}\footnote{. In: Coleção Emanuel
  Araújo (Acervo particular), 17/06/1882.}

\textbf{*didascália*}

\emph{A carta particular entre Gama e Cerqueira César parece pertencer a
um fluxo de correspondência que ambos possivelmente mantinham. "Pensei
na questão", como Gama abre a carta, por exemplo, indica que César havia
consultado Gama previamente sobre o tema que passaria a expor, a saber,
a ilegalidade da escravidão através de doações obtidas por corporações
religiosas. O tom relativamente informal, sobretudo para o teor do
assunto, reforça a ideia de que poderia haver troca de cartas entre os
dois, em que discutiam, como se lerá nessa, doutrinas e conhecimento
normativo e religioso. }

\emph{***}

1882 -- Junho 17

Meu caro dr. Cerqueira César,

Pensei na questão; tenho, para mim, que são livres os escravos
ilegalmente doados às corporações religiosas e de mão-morta\footnote{.
  Condição legal de inalienabilidade de bens.}, que os não podem
adquirir.

Aos fiéis católicos e apostólicos romanos foi expressamente proibido ter
escravos; tornou-se-lhes de todo ponto {defesa}\footnote{. Proibida.} {a
propriedade servil} (Conf{[}erir{]} S{[}ão{]} Math{[}eus{]},
Cap{[}ítulo{]} 7º, v{[}ersículo{]} 12; S{[}ão{]} Gregorio Mag{[}no{]},
Epístol{[}a{]} IV, 12; Bispo de Orleans -- "Carta ao clero de sua
diocese"; E as Bulas\footnote{. Decreto ou documento eclesiástico com
  ordens e instruções determinadas em nome do papa.} de Alexandre, em
1200; de Pio 2º, em 1482; de Paulo 3º, em 1557; de Urbano 8º, em 1639;
de Benedicto 14º, em 1741; Const{[}ituição{]} Pol{[}ítica{]} do
Imp{[}ério{]}, art. 5º).

Sei que a despeito de tão expressa proibição essas Ordens têm possuído
escravos; futil é, porém, tal argumento, tirado da Lei contra a Lei;
pode-se justificar a transgressão; é, porém, absurdo, por ela derrogar o
preceito e desconhecer a moral.

Julgo inabalável esta minha doutrina: a manumissão, entre nós, esteia-se
nas Decretais\footnote{. Carta ou decreto do papa em resposta a alguma
  consulta sobre matéria moral ou jurídica.}.

Seu, como sempre, am{[}igo{]} muito grato.

Luiz.

\textbf{61. {[}PETIÇÃO AO IMPERADOR D. PEDRO II{]}}\footnote{. In:
  \emph{Gazeta da Tarde} (RJ), {[}editorial{]}, Luiz Gama, 08/08/1882,
  p. 1. Os redatores da folha abolicionista, entre eles José do
  Patrocínio, incluíram a seguinte apresentação da petição de Gama ao
  Imperador: "Luiz Gama {[}--{]} Deve estar a esta hora em mão do
  governo uma representação do grande chefe abolicionista, que é o
  símbolo da evangélica resignação no sacrifício em prol da causa, que,
  ferindo os interesses da preguiça nacional, se converte em martírio
  para os seus sustentadores.

  A representação visa a liberdade de homens ilegalmente retidos na
  escravidão e nos dispensa de acrescentar-lhe comentários.

  A singeleza da exposição dá ao leitor conhecimento do assunto e
  critério para o seu juízo.

  Eis a representação". Optei em chamar de petição haja vista a força
  normativa da peça que, até onde se sabe, não possui capacidade de
  representação, uma vez que, constrangida pela ausência de meios,
  limita-se a implorar, pedir, suplicar.}

\textbf{*didascália*}

\emph{Trata-se de uma petição ao imperador Pedro II em que se demanda a
liberdade de setenta e oito libertos, quase todos mantidos ilegalmente
escravizados em Mar de Espanha, Minas Gerais, pelo ilegítimo possuidor,
Leite Brandão. Do imenso grupo de mais de setenta pessoas, dez fugiram
de Mar de Espanha. Desses, cinco chegaram em Lorena, interior paulista,
onde foram presos e depois libertados, e os outros cinco chegaram até
Luiz Gama, na capital paulista. Foram esses últimos que informaram Gama
do caso. Isso demonstra a rede de comunicação que alimentava a advocacia
combativa de Gama. Portanto, cinco libertos, embora ilegalmente
escravizados, empreenderam fuga até São Paulo e contaram a Gama a
tragédia que se passava na fazenda Babilônia, em Mar de Espanha (MG).
Nas palavras de Gama: "são eles {[}os cinco libertos escravizados{]} os
referentes destas graves ocorrências, destas monstruosas transgressões
do direito, destes crimes extraordinários cometidos à face da autoridade
pública". }

Senhor,

Luiz Gonzaga Pinto da Gama, residente na cidade de S. Paulo, vem perante
Vossa Majestade Imperial implorar providências administrativas, a fim de
que não continuem na privação de sua liberdade os libertos constantes da
relação inclusa.

A 3 de maio deste ano, a Ordem Carmelita\footnote{. A Ordem dos
  Carmelitas, ou Ordem do Carmo, é uma instituição religiosa católica de
  800 anos, que tem presença no Brasil desde os finais do século XVI.}
concedeu alforria aos setenta e oito indivíduos mencionados na referida
relação, indivíduos que residem em Mar de Espanha\footnote{. Município
  localizado na região sul de Minas Gerais.}, na fazenda denominada
Babilônia, província de Minas Gerais.

Concedidas estas alforrias e invocadas providências que foram concedidas
pelo Ministério do Império, não tiveram estas execuções. E os libertos
continuam como escravos sob o domínio irregular e ilegal do dr. Joaquim
Eduardo Leite Brandão\footnote{. Joaquim Eduardo Leite Brandão
  (1820-1899), mineiro de São João del Rei, foi fazendeiro e
  proprietário da fazenda Babilônia, que possuía grande concentração de
  escravos no sul de Minas Gerais.}. Dez dos libertos retiraram-se do
poder de Brandão; este, porém, pediu providências à polícia para conter
escravos insubordinados; obteve força; e recolheu-os à prisão! Interveio
a Promotoria Pública, conseguiu a soltura dos detidos; mas os outros,
que se acham na mencionada fazenda, lá continuam no cativeiro!

Cinco destes libertos conseguiram chegar a esta cidade de S. Paulo; são
eles os referentes\footnote{. No sentido de informantes, referenciais.}
destas graves ocorrências, destas monstruosas transgressões do direito,
destes crimes extraordinários cometidos à face da autoridade pública,
com menoscabo\footnote{. Menosprezo, descrédito.} da lei e desprezo da
moral.

Segundo as declarações destes, outros cinco conseguiram deixar a pressão
em que viviam e ficaram na Cachoeira, distrito de Lorena\footnote{. A
  fazenda Cachoeira, de Lorena (SP), é uma velha conhecida de Luiz Gama,
  o que só aumenta a singularidade que essa história possui. Seu antigo
  proprietário, o alferes Antonio Pereira Cardozo, foi o senhor de
  escravos que comprou Luiz Gama, quando ele ainda tinha apenas dez anos
  de idade. Foi na fazenda Cachoeira, também, o cenário do suicídio do
  alferes Cardozo, crime que marcou a história do município e a memória
  de Gama, conforme ele conta na \emph{Carta a Lucio de Mendonça}.},
trabalhando a fim de adquirirem meio de transporte para esta cidade.

É nestas circunstâncias que o impetrante vem implorar a Vossa Majestade
Imperial providências que tirem os libertos do ilegal domínio em que se
acham.

É justiça.

São Paulo, 2 de Agosto de 1882.

LUIZ G. P. DA GAMA.

*****
