\chapterspecial[O livro rejeitado por 40 editores}{}{José Roberto O’Shea]

\section{Sobre o autor}

\textsc{O ano} de 1914 foi, sem dúvida, crucial na carreira literária de James
Joyce. Tendo fixado residência em Zurique, que com o advento da Primeira Guerra
Mundial se torna uma espécie de santuário para exilados, o escritor irlandês
conclui \textit{Um retrato do artista quando jovem}, encaminha \textit{Exilados} para o
prelo, começa a escrever \textit{Ulisses} e consegue, finalmente, publicar
\textit{Dublinenses}, coletânea de contos ambientados em Dublin, na Irlanda. O
livro é a primeira obra em prosa publicada por Joyce.

As dificuldades do escritor com editores, censores e até mesmo com tipógrafos
são célebres. Com muita propriedade, o joyceano Harry Levin assinala que
praticamente todos os escritos do autor só foram publicados depois de grande
polêmica. Quando os editores aceitavam um manuscrito, os tipógrafos
recusavam"-se a compô"-lo; quando a obra era finalmente publicada, os censores a
destruíam; quando a acusação não era de obscenidade, era de blasfêmia; quando
não era de blasfêmia, era de traição.  Uma vez proibida na Irlanda, a obra era
publicada na Inglaterra; uma vez proibida na Inglaterra, era publicada nos
Estados Unidos; e, ao final, era também proscrita na América.\footnote{ Levin, p.~13.}

%\section{A obra originária}

A história da publicação de \textit{Dublinenses} não foge à regra, sendo
particularmente angustiante. Em dezembro de 1905, Joyce envia o manuscrito da
coletânea de contos ao editor Grant Richards. Este, embora surpreso ao receber
de Trieste um livro intitulado \textit{Dublinenses}, simpatiza com a obra e
assina um contrato para sua publicação em março de 1906. Durante um mês, tudo
parece correr bem.

Então, em fevereiro, Joyce encaminha a Richards um conto complementar --- “Dois
galãs” --- que estava fadado a provocar grande rebuliço. Richards, sem ter lido o
conto, envia"-o para o tipógrafo que, levantando objeções quanto à linguagem
“indecente” e quanto à própria situação encerrada na história, aproveita a
oportunidade para estender sua objeção a trechos em outros contos. Nesse
ínterim, Joyce escrevera “Uma pequena nuvem” e estava prestes a enviar mais
esse conto a Richards quando, em abril, o editor informa"-lhe que algumas
alterações seriam necessárias, em virtude de objeções apresentadas pelo
tipógrafo.\footnote{  Ellmann, p.~227. Há que se esclarecer que, de acordo com a
lei inglesa, não apenas o editor como também o tipógrafo de material obsceno
ficam responsabilizados legalmente e estão sujeitos a processo judicial
(Ellmann, p.~228).}

Indignado, Joyce escreve a Richards, afirmando que “em nenhum outro país europeu
civilizado é permitido ao tipógrafo abrir a boca” (citado por Ellmann, p.~228),
mas acaba concordando em fazer algumas das alterações exigidas. Em maio de
1906, respondendo a Richards, ele escreve: 

\begin{quote}
Os pontos nos quais me recuso a ceder constituem simplesmente o arcabouço do
livro. Se elimino tais elementos, como há de ficar o capítulo da história moral
do meu país? Luto pela manutenção desses elementos, pois creio que, ao escrever
meu capítulo da história moral do meu país exatamente da forma como o fiz, dei
o primeiro passo para a libertação espiritual da minha pátria.\footnote{  Citado por
Ellmann, p.~230. As citações feitas nesta introdução são traduzidas
pelo presente tradutor.}
\end{quote}

As principais objeções baseavam"-se em supostas obscenidades, blasfêmias e
traição ao sentimento e aos interesses irlandeses.  No já citado “Dois galãs”,
por exemplo, o ato de Corley receber de uma jovem uma moeda de ouro e exibi"-la
a Lenehan foi considerado indecente. Em “Duplicatas” a menção ao interesse de
Farrington por uma jovem no \textit{pub}, bem como o modo como ele lhe sorria,
a maneira como a jovem olhava para Farrington, e os atos de ela cruzar e
descruzar as pernas a todo momento, de esbarrar na cadeira de Farrington e
falar"-lhe ao sair do \textit{pub} foram, igualmente, tachados de obscenos.

Em junho, Joyce volta a recorrer a Richards: 

\begin{quote}
Acredito, sinceramente, que você está atrasando o avanço da civilização na Irlanda ao 
impedir que o povo irlandês possa contemplar"-se nesse meu belo espelho reluzente.\footnote{ Citado por
Ellmann, p.~230.} 
\end{quote}

Richards concorda em incluir “Dois galãs” e, em julho, Joyce
devolve o manuscrito, na íntegra, ainda que com algumas modificações. “As
irmãs” passara por uma revisão, “Uma pequena nuvem” tinha sido incluído, a
palavra \textit{bloody} fora expurgada em seis situações e mantida em apenas
uma, e um incidente em “Duplicatas” fora reescrito.\footnote{ Ellmann, p.~231.}

Tendo cedido o máximo que sua consciência artística lhe permitia, Joyce expressa
em cartas ao irmão, Stanislaus, a confiança de que o livro seria publicado
imediatamente. Mas estava enganado. No final de setembro, Richards informa a
Joyce que “por ora” não poderá publicar \textit{Dublinenses}, “talvez num futuro
próximo”.\footnote{ Citado por Ellmann, p.~239.} Novamente indignado, Joyce procura o
cônsul britânico em Trieste e pede que este lhe indique um advogado, a ser
consultado a respeito da quebra do contrato por parte de Richards.\footnote{ Ellmann, p.~240.} 
Em desespero, Joyce concorda em retirar da coletânea “Dois galãs” e “Uma
pequena nuvem” e em modificar dois trechos de “Duplicatas” e “Graça” --- mas
Richards recusa mais uma vez o manuscrito. Joyce, então, oferece o livro a
outro editor, John Long, que também o rejeita.

Em 1912, Joyce visita Dublin pela última vez, numa tentativa frustrada de
publicar o livro na Irlanda. O episódio é famoso tanto na biografia de James
Joyce quanto na história de textos proscritos. Maunsel, editor “oficial” da
chamada Renascença irlandesa, quebra o contrato que tinha sido firmado com
Joyce e incinera as provas de \textit{Dublinenses}; Joyce toma a decisão de
jamais retornar à Irlanda.\footnote{ Levin, p.~14.}

O litígio perdura nove anos até que a coletânea é finalmente publicada, em
Londres, pelo próprio Grant Richards, em 1914. Em carta ao agente literário J.B.~Pinker,
escrita em julho de 1917, Joyce resume a agonia da publicação de
\textit{Dublinenses}: 

\begin{quote}
O livro custou"-me em despesas com processos judiciais,
passagens de trem e tarifas postais cerca de três mil francos; custou também
nove anos da minha vida.  Correspondi"-me com sete advogados, cento e vinte
jornais e inúmeros escritores, nenhum dos quais, à exceção de Ezra Pound,
ajudou"-me.\footnote{  Citado por Ellmann, p.~429. Saudando a publicação da
coletânea com uma resenha encomiástica publicada em \textit{The
Egoist} de 14 de janeiro de 1914, Pound resume as tribulações do
manuscrito de \textit{Dublinenses}.} 
\end{quote}

Na mesma carta, Joyce prossegue: 

\begin{quote}
As chapas da edição inglesa (1906) foram destruídas. A segunda edição (Dublin, 1910) foi
incinerada quase na minha presença. A terceira edição (Londres, 1914) encerra o
texto tal e qual por mim escrito e conforme obriguei o editor a publicá"-lo
depois de nove anos.\footnote{ \textit{Ibid.}}
\end{quote}

Concluindo o desabafo, Joyce revela que
“\textit{Dubliners} foi rejeitado por quarenta editores”.\footnote{ \textit{Ibid.}}

\section{Sobre a obra}

Em termos de estrutura, a coletânea de quinze contos pode ser organizada de
acordo com quatro aspectos principais --- infância, adolescência, maturidade
e vida pública ---, na seguinte ordem: infância: “As irmãs”, “Um encontro”,
“Araby”; adolescência: “Eveline”, “Depois da corrida”, “Dois galãs”, “A
pensão”; maturidade: “Uma pequena nuvem”, “Duplicatas”, “Barro”, “Um caso
triste”; vida pública: “Dia de hera na sede do comitê”, “Mãe”, “Graça”, “Os
mortos”. Dentro dessa estrutura, destacam"-se alguns temas essenciais, e.g.,
“paralisia”, “vida e morte”, “epifania”.  Joyce escolhe Dublin como local dos
contos porque, a seu ver, a cidade constitui “o centro da paralisia” da
civilização irlandesa.\footnote{ Levin, p.~30.} Em “As irmãs”, por exemplo, ainda que
jamais explicitado, a paralisia física do padre Flynn é sintoma da paralisia
generalizada que, segundo Joyce, abatera"-se sobre a Irlanda.

É certo que essa imobilidade gera frustração e em diversos contos temos a
impressão de estar lendo o que alguns joyceanos chamam de “crônicas da
frustração”: um padre torna"-se inofensivamente louco; um adolescente
decepciona"-se consigo mesmo e com o amor; a intenção de fuga de uma jovem não
se concretiza; um homem casado sente"-se numa “prisão perpétua”; um pai de
família é incapaz de encarar o patrão; um asceta é incapaz de corresponder ao
amor; um marido egocêntrico percebe que não foi a grande paixão da esposa.

Paralelamente à questão da imobilidade e da frustração, temos a inter"-relação
entre vida e morte, vivos e mortos. Trata"-se, logicamente, do tema central do
primeiro conto e do último, além de ser também o tema de “Um caso triste” e,
indiretamente, de “Dia de hera na sede do comitê”, visto que, neste último, o
conflito gira em torno de um personagem ausente, na verdade falecido, o herói
nacionalista irlandês, Charles Stewart Parnell (1846--1891).

Vale a pena determo"-nos um momento na temática da inter"-relação entre vivos e
mortos para esclarecer o contexto da criação do conto final, e mais extenso da
coleção, “Os mortos”. Ao escrevê"-lo, Joyce foi motivado pela intenção
de apresentar ao mundo uma visão mais indulgente da Irlanda. Tal intenção
torna"-se patente nas cartas do escritor. Joyce escreve a Stanislaus, em
setembro de 1906: 

\begin{quote}
Às vezes, quando me lembro da Irlanda, tenho a impressão de
que fui severo demais. Não reproduzi (pelo menos em \textit{Dublinenses}) nenhum
dos atrativos da cidade\ldots{} Não reproduzi sua charmosa insularidade, nem sua
hospitalidade\ldots{} Não fiz jus à sua beleza.\footnote{ Citado por Ellmann, p.~239.}
\end{quote}

A temática da hospitalidade é abordada indiretamente em “Barro”, mas em “Os
mortos” o tema é tratado de maneira explícita. Na verdade, talvez com o intuito
de reparar o pecadilho confessado a Stanislaus, Joyce abre o conto “Os mortos”
--- que aliás não constava do primeiro manuscrito de \textit{Dublinenses} e
foi concluído após a referida carta de Joyce a Stanislaus --- com uma festa.
Em seu discurso da ceia de Natal, Gabriel Conroy rasga elogios à Irlanda e aos
irlandeses, sobretudo por sua hospitalidade. Se por um lado, a morte
representa, do ponto de vista físico, um momento extremo, por outro, em festas
como as que são anualmente organizadas pelas irmãs Morkan, a morte torna"-se
objeto do encômio de Conroy, sendo lembrada --- e por que não dizer, celebrada ---
em meio à dança, à mobilidade, à vida e aos vivos.

E para dispor de breves \textit{insights} quanto aos significados da vida e da
morte, os personagens joyceanos são objetos de epifanias. Em termos teológicos,
a epifania trata da manifestação de Cristo aos Reis Magos.  A epifania é uma
manifestação espiritual, uma relação transcendental entre o universo interior e
o exterior. Em termos literários, o escritor moderno, capitulando diante da
impossibilidade de compreender o caos que o cerca, busca indícios externos que
o levem a significados internos. Podemos deduzir a partir da visão de mundo
oferecida pelos contos coligidos em \textit{Dublinenses} que há momentos
epifânicos ao alcance de cada um de nós --- basta buscá"-los.

Assim, a “saída” para o caos da modernidade possui uma origem espiritual, ao
menos em nível de inspiração para o escritor. Sabemos que Joyce toma a noção de
epifania emprestada dos ritos da Igreja Católica. Em carta a Stanislaus, o
escritor afirma: 

\begin{quote}
Você não acha que existe uma certa semelhança entre o mistério da missa e o que
estou tentando fazer? Ou seja, estou tentando [\ldots{}] dar às pessoas algum tipo
de \textit{prazer intelectual ou satisfação espiritual}, transformando o pão da
vida cotidiana em algo que tem uma vida artística própria e permanente [\ldots{}]
com o intuito de promover uma elevação mental, moral e
espiritual.\footnote{ Citado por Ellmann, pp.~103--104 (minha ênfase).}
\end{quote}

\section{Sobre o gênero}

Para Joyce, a tarefa do homem de letras é registrar esses estados de
espírito sutis e evanescentes, tornando"-se um “colecionador de epifanias”. E,
embora tal doutrina informe toda a obra de Joyce, \textit{Dublinenses} encerra a
mais contundente coleção de epifanias.\footnote{ Levin, p.~29.} O leitor deve buscar
momentos epifânicos principalmente nas conclusões de “O encontro”, “Araby”,
“Eveline”, “Um caso trágico” e “Os mortos”.

Tipicamente, o escritor moderno, tanto na prosa quanto na poesia, coloca"-se fora
da ação, aguardando um “encontro casual”, ou um “pedaço de conversa”, que
enseje o conto ou o poema. O escritor moderno não visa precipuamente à aventura
romântica, nem ao incidente dramático. Visa representar/transformar a rotina da
vida cotidiana e explorar os mecanismos psicológicos do comportamento humano.
Expressando exatamente esse tipo de intenção, Joyce escreve a Stanislaus,
comentando a respeito de \textit{Dublinenses}: 

\begin{quote}
A ideia que tenho do significado
das coisas triviais é o que desejo passar aos dois ou três infelizes que
porventura venham a ler a minha obra.\footnote{ Citado por Ellmann, p.~231.}
\end{quote}

Felizmente, Joyce enganou"-se quanto à previsão do número de seus leitores. As
epifanias literárias criadas pela imaginação, pela sensibilidade e pelo
trabalho de James Joyce “civilizaram” não somente a Irlanda mas toda a
humanidade.

Apesar dos percalços e pesares, uma vez publicado na íntegra, “o capítulo da
história moral” da Irlanda vingou. A recepção crítica da obra foi logo
favorável.  Em 18 de junho de 1914, o \textit{Times Literary Supplement}
afirma:

\begin{quote}
O autor, Sr.~James Joyce, não focaliza todas as classes de dublinenses, apenas
(quase exclusivamente) aqueles que afundariam se a maré de dificuldades
materiais subisse um pouco. [\ldots{}] A escrita é admirável. Joyce evita exageros.
Deixa"-nos convictos de que seus personagens são exatamente conforme ele os
descreve.
\end{quote}

E em 27 de junho do mesmo ano, o \textit{New Statesman} arrisca:

\begin{quote}
É fácil dizer que Górki é um gênio. Dizer o mesmo de James Joyce requer coragem,
pois seu nome é pouco conhecido; porém, gênio é, exatamente, o que ele é. [\ldots{}]
A maturidade, a marca pessoal e a força desses contos chegam a causar espanto.
[\ldots{}] Tudo é relatado com sobriedade, bastante sobriedade, e muita competência.
Jamais nos entediamos. [\ldots{}] Joyce insiste em aspectos da vida que normalmente
não são mencionados. [\ldots{}] não achamos que seja afetação: simplesmente, o não
mencionável é para ele uma preocupação constante.
\end{quote}

Na verdade, hoje em dia, o fato de Grant Richards ter recebido de Trieste uma
obra de James Joyce intitulada \textit{Dublinenses} já não surpreende como
surpreendeu o editor inglês. Hoje sabemos que o autor de \textit{Chamber music}
(1907), \textit{A portrait of the artist as a young man} (1916),
\textit{Exiles} (1918), \textit{Ulysses} (1922) e \textit{Finnegans wake}
(1939) residiu e trabalhou em várias cidades europeias, sendo quase todas
poliglotas e cosmopolitas. Sabemos, também, que praticamente toda a obra
literária de Joyce foi produzida no exílio, em continente europeu (ironicamente,
à exceção de \textit{Dublinenses}). Contudo, ainda que dolorosamente
\textit{déraciné}, o escritor mantinha suas raízes profundamente cravadas no
solo irlandês. Com efeito, Ellmann esclarece que “em Trieste e em Roma, Joyce
aprende o que tinha desaprendido em Dublin: a ser um dublinense”.\footnote{ Ellmann, p.~263.}


\section{Sobre a tradução}

Tendo publicado uma tradução de \textit{Dublinenses} vinte anos atrás, o
tradutor, primeiramente, gostaria de ressaltar o fato de ser a presente versão
mais que uma tradução revista.  Na realidade, o texto aqui publicado configura
uma nova tradução.  Em segundo lugar, vale dizer que as preocupações com a
observação do estilo marcante da prosa de James Joyce que balizaram a primeira
tradução permanecem rigorosamente as mesmas neste novo trabalho.  No entanto,
passados vinte anos em sua prática de tradutor literário, e contando agora com
fontes de referência inexistentes à época da primeira versão, o tradutor pôde
levar a termo centenas de retoques, ajustes e importantes correções que
resultaram em uma nova tradução.

Qualquer pessoa que teorize a respeito ou trabalhe com tradução reconhece que a
atividade põe em jogo um verdadeiro conflito de lealdade, de vez que a decisão
de privilegiar o texto original ou o texto traduzido continua a ser o grande
problema na teoria e na prática da tradução. O célebre conceito de
\textit{paraphrase}, proposto por John Dryden no prefácio de \textit{Ovid’s
Epistles Translated by Several Hands} (1680), e que configura uma \textit{via
media} na qual o tradutor “mantém em vista” o autor do original, dando maior
ênfase ao sentido do que à literalidade das palavras, permanece a prescrição
mais sensata, ainda que difícil de ser praticada de forma consistente.

No século \textsc{xx}, Peter Newmark, entre outros, aborda a questão do
“conflito de lealdades”. Possivelmente, a principal contribuição de Newmark à
teoria geral da tradução vem a ser os conceitos de tradução comunicativa e
semântica, que tratam exatamente dessa problemática. Para Newmark, a tradução
comunicativa pretende, acima de tudo, assegurar ao máximo a compreensão do
texto pelos leitores do idioma alvo. Nesse tipo de tradução, as estruturas
linguísticas devem ser simples e a linguagem “clara” e “direta”, favorecendo o
emprego de termos “mais genéricos nos trechos de maior complexidade”.\footnote{ Newmark,
1988, p.~39.}

Em contrapartida, Newmark afirma que “a tradução semântica, dentro das
limitações das estruturas semânticas e sintáticas do idioma alvo, procura
reproduzir o exato significado contextual do original”. O estudioso esclarece
que a tradução semântica procura recriar o “sabor” e a “tonalidade” do
original, que as palavras selecionadas pelo autor são “sagradas”, não por serem
mais importantes do que o conteúdo, mas porque forma e conteúdo são
inseparáveis. Esclarece, também, que “a tradução semântica objetiva a
preservação do idioleto do autor”, ou seja, de suas formas pessoais de
expressão.\footnote{ Newmark, 1988, pp.~39---47.} Por conseguinte, a tradução semântica
tende a ser mais literal, mais detalhada, e até mesmo, em determinados trechos,
a soar de forma “estranha”.\footnote{ Newmark, 1988, p.~39.} Newmark conclui que textos literários
em geral devem ser objetos de tradução semântica, ao passo que textos não
literários se prestam à tradução comunicativa.\footnote{ Newmark, 1988, p.~45.}

Contudo, vale ressaltar, visto que tanto a tradução comunicativa quanto a
semântica compartilham preocupações semânticas, a classificação de Newmark
teria mais acuidade tipológica caso estabelecesse distinção entre tradução
comunicativa e \textit{estética}.  A rigor, o próprio Newmark argumenta que
toda tradução é, até certo ponto, comunicativa e semântica. Voltamos, portanto,
a uma questão de ênfase.

Em \textit{Dublinenses}, a abordagem é predominantemente semântica, i.e.,
estética. Assim, procura"-se respeitar e transferir, na medida do possível,
tanto a realidade contextual de \textit{Dublinenses} quanto os traços marcantes
do estilo literário pessoal de Joyce, principalmente porque muitos leitores da
tradução têm consciência de ambos os fatores e certamente contam com a presença
deles na obra traduzida.

Na tentativa de preservar"-se o “sabor” e a “tonalidade” do texto original,
evitou"-se a tradução ou a adaptação de nomes próprios, de modo geral, sobretudo
quando nitidamente relacionados à cultura fonte. Assim, não foram traduzidos os
nomes dos personagens, de acidentes geográficos, de ruas e parques, de unidades
e frações monetárias, de certas bebidas típicas, de associações, de danças
folclóricas, nem os títulos de periódicos e de obras inéditas em língua
portuguesa. Mesmo os pronomes de tratamento foram mantidos, não apenas para
preservar a “realidade da ficção” original mas, também, por uma questão de
eufonia, a fim de evitar verdadeiras paródias sonoras como, “seu Farrington”,
“seu Power”, “dona Conroy” etc.

Não cabe aqui esmiuçar o estilo literário extremamente complexo e sofisticado de
James Joyce. É preciso, no entanto, ressaltar que a tradução procura observar o
elemento mais vital desse estilo --- o domínio total da polifonia, especialmente
evidenciado na riqueza das inflexões presentes nas falas dos personagens.
Assim, na tentativa de transferir os efeitos estéticos do coloquialismo
presente nos diálogos dos dublinenses, nesta tradução, os desvios das
prescrições gramaticais (bem como as eventuais variações ortográficas)
observadas no discurso de certos personagens são absolutamente propositais.
Conforme se observa no original, tais desvios, além dos efeitos estéticos
inerentes à \textit{heteroglossia}, marcam a própria caracterização
sociocultural dos falantes, e.g., Eliza Flynn, Mrs. Kernan, Lenehan e Corley,
Polly Mooney, Lily, Ignatius Gallaher, Freddy Malins e outros.

Mais especificamente, a tradução dos diálogos, em especial nos contos “Depois da
corrida”, “Dois galãs”, “Uma pequena nuvem”, “Dia de hera na sede do comitê”,
“Graça” e “Os mortos”, ao seguir coloquialismos presentes no original, visa a
uma aproximação do discurso oral, em termos da liberalidade do emprego dos
pronomes pessoais oblíquos, de certas regências verbais e nominais, bem como de
um “realismo fonológico” sugerido através de formas como “pra”, “pro”, “cadê” e
de determinadas interjeições.

Ainda no que tange a aspectos de estilo, cabe uma observação sobre pontuação
gráfica e repetição de palavras e sons. Quanto à pontuação, foi preservado o
emprego idiossincrático de dois pontos que, às vezes, aparecem sucessivamente
dentro de uma mesma oração. A intolerância de Joyce com relação a vírgulas é
notória. Ao revisar as provas da primeira edição de \textit{Dublinenses}, por
exemplo, o escritor retirou centenas de vírgulas, inseridas pelo tipógrafo a
título de correção. Nesta tradução, procura"-se reproduzir a pontuação (ou mesmo
a falta de pontuação) do texto original. Apenas quando a ausência da pontuação
no texto traduzido compromete o sentido, enquanto a mesma ausência, no
original, não provoca ambiguidade comparável, o tradutor regulariza o texto por
meio de vírgula ou ponto e vírgula.  E a repetição de palavras (mesmo no caso
dos sucessivos “e”, em lugar das vírgulas desprezadas) não reflete
necessariamente a pobreza vocabular do tradutor; na verdade, tenta"-se
reproduzir o efeito da genialidade de Joyce no que diz respeito à cadência e à
musicalidade de sua prosa. O mesmo se aplica às aliterações e às assonâncias.

Concluindo, o tradutor agradece o convite e a oportunidade de
realizar esta nova tradução.  E, conforme já fora o caso quando da primeira
tradução, o tradutor agradece a valiosa colaboração do Dr. Weldon Thornton,
joyceano de renome internacional e professor de literatura inglesa da
Universidade da Carolina do Norte, em Chapel Hill, nos \textsc{eua}, com quem teve o
privilégio de estudar a obra de Joyce.  Agradece, também, o apoio dos colegas
do Departamento de Língua e Literatura Estrangeira da Universidade Federal de
Santa Catarina que, em Colegiado, concederam"-lhe carga horária para a pesquisa
e a realização deste trabalho.

\begin{bibliohedra}
\tit{Ellmann}, Richard. \textit{James Joyce.} New York: Oxford \textsc{up},
1981.

\tit{Dryden}, John. “Metaphrase, Paraphrase and Imitation”, in
\textit{Readings in Translation Theory}. Ed. Andrew Chesterman.  Finland: Oy
Finn Lectura Ab, 1989, pp.~7--12.

\tit{Joyce}, James. \textit{Letters of James Joyce}. Ed. Stuart Gilbert. New
York: Viking, 1957.

\tit{Joyce}, Stanislaus. \textit{My Brother’s Keeper}. Ed. Richard Ellmann.
New York: Viking, 1958.

\tit{Levin}, Harry. \textit{James Joyce: a critical introduction}. Revised
and Enlarged Edition. New York: James Laughlin, 1960.

\tit{Newmark}, Peter. \textit{Approaches to Translation}. Oxford: Pergamon
Press, 1981/1988.

\titidem. \textit{A Textbook of Translation}. Hemel Hempstead: Prentice Hall,
1988b.

\end{bibliohedra}

