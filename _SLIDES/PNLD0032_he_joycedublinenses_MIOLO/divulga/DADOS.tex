\titulo{Stephen Herói} % Em minúsculas!!!
\autor{James Joyce}  % Apenas sobrenome, se for o caso. Verificar capa.% 
\organizador{Tradução e prefácio}{José Roberto O'Shea} 
%Conferir se é apenas {Organização}; {Organização e tradução} ou apenas {Tradução}%
\isbn{978-85-7715-096-0}
\preco{38}   % Ex.: 14. Não usar ",00"%
\pag{214}   % Número de páginas

\release{\textbf{Stephen Herói} não é apenas a versão inicial e embrionária de um dos romances 
mais influentes do século~\textsc{xx}. A dramática transfiguração sofrida pelo manuscrito original --- que contava com mais de mil páginas e das quais restou “apenas” um terço --- até sua publicação como Um retrato do artista quando jovem, em 1914, nos permite entrever, a um só tempo, tanto o método e o desenvolvimento da técnica narrativa de Joyce como um vívido panorama biográfico. Em fevereiro de 1904, Joyce, que havia acabado de completar 22 anos, começava a escrever \textit{Stephen Herói}, movido pela recusa do periódico Dana em publicar seu manifesto “Um retrato do artista”. Tinha início assim o périplo que o levaria a escrever cerca de mil páginas manuscritas sobre a emergência da consciência artística em Stephen Dedalus --- personagem autobiográfico que retomará no \textit{Retrato} e no \textit{Ulisses} ---, material que o autor refundiria e cristalizaria ao longo de uma década, em um dos romances mais importantes do século~\textsc{xx}: \textit{Um retrato do artista quando jovem.} Em conflito com a ortodoxia da igreja, da pátria e da família, munido de “silêncio, exílio e astúcia”, o jovem Dedalus representado nesta brilhante coleção de “epifanias” é dos retratos mais fascinantes e ricamente detalhados do desenvolvimento intelectual de um jovem artista.\\ \textit{Inédito em língua portuguesa.} 

\noindent\textbf{José Roberto O'Shea} é professor titular de Literatura Inglesa da 
Universidade Federal de Santa Catarina (\textsc{ufsc}). É mestre em Literatura 
pela American University, em Washington, e doutor em Literatura Inglesa e 
Norte-americana pela Universidade da Carolina do Norte, em Chapel Hill, com pós-doutorados 
na Universidade de Birmingham (Shakespeare Institute) e na Universidade de Exeter, ambas na 
Inglaterra, e pesquisador convidado da Folger Shakespeare Library, em Washington, (2010). 
Publicou diversos artigos em periódicos especializados, além de cerca de quarenta traduções, abrangendo as áreas de história, teoria da literatura, biografia, poesia, ficção em prosa e teatro.
}


