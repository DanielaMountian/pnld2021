\SVN $Id: INTRO.tex 10142 2011-11-08 15:13:35Z bruno $
\chapter[Introdução, por José Roberto O'Shea]{introdução}
\hedramarkboth{introdução}{josé roberto o'shea}

\textsc{O ano} de 1914 foi, sem dúvida, crucial na carreira literária de James
Joyce. Tendo fixado residência em Zurique, que com o advento da
primeira guerra mundial se torna uma espécie de santuário para
exilados, o escritor irlandês termina \textit{A portrait of the artist as a
young man} (\textit{Um retrato do artista quando jovem}), encaminha
\textit{Exiles} (\textit{Exílios}) para o prelo, começa a escrever
\textit{Ulysses} e consegue, finalmente, publicar \textit{Dubliners},
coletânea de contos localizados em
Dublin, na Irlanda. O livro é a primeira obra em prosa publicada por
Joyce.

As dificuldades do escritor com editores, censores e até mesmo com
tipógrafos são célebres. Com muita propriedade, o joyceano Harry Levin
assinala que praticamente todos os escritos do autor só foram
publicados depois de grande polêmica. Quando os editores aceitavam um
manuscrito, os tipógrafos recusavam-se a compô-lo; quando a obra era
finalmente publicada, os censores a destruíam; quando a acusação não
era obscenidade, era blasfêmia; quando não era blasfêmia, era traição.
Uma vez proibida na Irlanda, a obra era publicada na Inglaterra; uma
vez proibida na Inglaterra, era publicada nos Estados Unidos; e, ao
final, era também proscrita na América (Levin, p. 13).

\section{\textls{Dubliners}, a obra}

A história da publicação de \textit{Dubliners} não foge à regra, sendo
particularmente angustiante. Em dezembro de 1905, Joyce envia o
manuscrito da coletânea de contos ao editor Grant Richards.

Este, embora surpreso ao receber de Trieste um livro intitulado
\textit{Dubliners}, simpatiza com a obra e assina um contrato para publicação
da mesma em março de 1906. Durante um mês, tudo parece correr bem.

Então, em fevereiro, Joyce encaminha a Richards um conto complementar
--- ``Dois galãs'' --- que estava fadado a provocar um grande
rebuliço. Richards, sem ter lido o conto, envia-o para o tipógrafo
que, levantando objeções quanto à linguagem `indecente' e quanto
à própria situação encerrada na história, aproveita a oportunidade
para estender sua objeção a trechos em outros contos. Nesse ínterim,
Joyce escrevera ``Uma pequena nuvem'' e preparava-se para enviar
mais esse conto a Richards quando, em abril, o editor informa-lhe que
algumas alterações seriam necessárias, em virtude de objeções
apresentadas pelo tipógrafo (Ellman, p. 227).\footnote{ Há que se
esclarecer que, de acordo com a lei inglesa, não apenas o editor como
também o tipógrafo de material obsceno ficam responsabilizados
legalmente e estão sujeitos a processo judicial (Ellmann, p. 228) [Todas as notas são do tradutor, exceto quando indicadas].}

Joyce concorda em fazer somente algumas das alterações exigidas. Em
maio de 1906, respondendo a Richards, ele escreve: ``Os pontos nos
quais me recuso a ceder constituem simplesmente o arcabouço do livro.
Se elimino tais elementos, como há de ficar o capítulo da história
moral do meu país? Luto pela manutenção desses elementos, pois creio
que, ao escrever o meu capítulo da história moral do meu país
exatamente da forma como o fiz, dei o primeiro passo para a liberação
espiritual da minha pátria''.\footnote{ As citações feitas nesta
introdução são traduzidas pelo presente tradutor.}

Em junho, Joyce volta a recorrer a Richards: ``Acredito,
sinceramente, que você está atrasando o progresso da civilização na
Irlanda ao impedir que o povo irlandês possa contemplar-se nesse meu
espelho reluzente'' (\textit{Letters}, pp. 63---64). Richards concorda em
incluir ``Dois galãs'' e, em julho, Joyce devolve o manuscrito, na
íntegra, ainda que com algumas modificações. ``As irmãs'' passara
por uma revisão, ``Uma pequena nuvem'' tinha sido incluído, a
palavra \textit{bloody} fora expurgada em seis situações e mantida em apenas
uma, e um incidente em ``Cópias'' fora reescrito (Ellmann, p.
231).

Tendo cedido o máximo que a consciência lhe permitia, Joyce expressa
em cartas ao irmão, Stanislaus, sua confiança de que o livro seria
publicado imediatamente. Mas estava enganado. No final de setembro,
Richards informa a Joyce que `por ora' não poderá publicar
\textit{Dubliners}, `talvez num futuro próximo' (Ellmann, p. 239).
Indignado, Joyce procura o cônsul britânico em Trieste e pede que
este lhe indique um advogado, a ser consultado a respeito da quebra do
contrato por parte de Richards (Ellmann, p. 240). Em desespero, Joyce
concorda em retirar da coletânea ``Dois galãs'' e ``Uma pequena
nuvem'' e em modificar dois trechos de ``Cópias'' e
``Graça'' --- mas Richards recusa mais uma vez o manuscrito.
Joyce, então, oferece o livro a outro editor, John Long, que também o
rejeita.

Em 1912, Joyce visita Dublin pela última vez, numa tentativa frustrada
de publicar o livro na Irlanda. O episódio é famoso tanto na biografia
de James Joyce quanto na história de textos proscritos. Maunsel,
editor `oficial' da chamada Renascença irlandesa, quebra o contrato
que tinha sido firmado com Joyce e incinera as provas de \textit{Dubliners};
Joyce toma a decisão de jamais retornar à irlanda (Levin, p.l4).

O litígio perdura nove anos até que a coletânea é finalmente publicada
em Londres, em 1914, pelo próprio Grant Richards. Em carta ao agente
literário J. B. Pinker, escrita em julho de 1917, Joyce resume a
agonia da publicação de \textit{Dubliners}: ``O livro custou-me em
despesas com processos judiciais, passagens de trem e tarifas postais
cerca de três mil francos; custou também nove anos da minha vida.
Correspondi-me com sete advogados, cento e vinte jornais e inúmeros
escritores, nenhum dos quais, à exceção de Ezra Pound,
ajudou-me.''\footnote{ Saudando a publicação da coletânea com uma
resenha encomiástica publicada em \textit{The Egoist} de 14 de janeiro de 1914,
Pound resume as tribulações do manuscrito de \textit{Dubliners}.} Na
mesma carta, Joyce prossegue: ``As chapas da edição inglesa (1906)
foram destruídas. A segunda edição (Dublin, 1910) foi incinerada quase
na minha presença. A terceira edição (Londres, 1914) encerra o texto
tal e qual por mim escrito e conforme obriguei o editor a publicá-lo
depois de nove anos''. Concluindo o desabafo, Joyce revela que
``\textit{Dubliners} foi rejeitado por quarenta editores'' (citado em Ellmann,
p. 429).

Uma vez publicado na íntegra, `o capítulo da história moral' da
Irlanda vingou. Na verdade, hoje em dia, o fato de Grant Richards ter
recebido de Trieste urna obra de James Joyce intitulada \textit{Dubliners} já
não surpreende como surpreendeu o editor inglês. Hoje sabemos que o
autor de \textit{Chamber music} (1907), \textit{A portrait of the
artist as a young man} (1916), \textit{Exiles} (1918),
\textit{Ulysses} (1922) e \textit{Finnegan's wake} (1939)
residiu e trabalhou em várias cidades, sendo quase todas poliglotas e
cosmopolitas. Sabemos, também, que praticamente toda a obra literária
de Joyce foi produzida em exílio, no continente europeu (ironicamente
a exceção de \textit{Dubliners}). Contudo, ainda que dolorosamente \textit{déraciné}, o
escritor mantinha suas raízes profundamente cravadas no solo irlandês
Com efeito, Ellmann esclarece que ``em Trieste e em Roma, Joyce
aprende o que tinha desaprendido em Dublin: a ser um dublinense''
(Ellmann, p. 263).

Em termos de estrutura temática, a coletânea pode ser organizada de
acordo com quatro aspectos principais: infância, adolescência,
maturidade e morte. Dentro dessa estrutura, destacam-se alguns temas
essenciais, e.g., `paralisia', `vida e morte', `epifania'.
Joyce escolhe Dublin como local dos contos porque, a seu ver, a cidade
constitui `o centro da paralisia' da civilização irlandesa (Levin,
p. 30). Em ``As irmãs'', por exemplo, ainda que jamais
explicitado, a paralisia do Padre Flynn é um sintoma da paralisia
generalizada que, segundo Joyce, abatera-se sobre a Irlanda.

Com efeito, essa imobilidade gera a frustração e em diversos contos
temos a impressão de estar lendo o que alguns joyceanos denominam
\textit{annals of frustration}: um padre torna-se inofensivamente louco; um
adolescente decepciona-se consigo mesmo e com o amor; a intenção de
fuga de uma jovem não se concretiza; um homem casado sente-se numa
`prisão perpétua'; um pai de família é incapaz de encarar o patrão;
um asceta é incapaz de corresponder ao amor; um marido egocêntrico
percebe que não foi a grande paixão da esposa etc.

Paralelamente à questão da imobilidade e da frustração, temos a
inter-relação entre vida e morte, vivos e mortos. Trata-se,
logicamente, do tema central do primeiro conto e do último, além de ser
também o tema de ``Um caso trágico'' e, indiretamente, de ``Dia
da hera na sede do comitê'', visto que, neste último, o conflito
central gira em torno de um personagem ausente, na verdade falecido, o
herói nacionalista irlandês, Charles Stewart Parneli.

Vale a pena determo-nos um momento na temática da inter- relação entre
vivos e mortos para esclarecer o contexto da criação do conto final, e
mais extenso da coleção, ``Os mortos''. Ao escrever ``The
dead'', Joyce foi motivado pela intenção de apresentar ao mundo uma
visão mais indulgente da Irlanda. Tal intenção torna-se extremamente
clara nas cartas do escritor. Joyce escreve a Stanislaus, em setembro
de 1906: ``Às vezes, quando me lembro da Irlanda, tenho a impressão
de que fui severo demais. Não reproduzi (pelo menos em
\textit{Dubliners}) nenhum dos atrativos da cidade... Não reproduzi sua charmosa
insularidade, nem sua hospitalidade... Não fiz jus à sua beleza''
(Ellmann, p. 239).

``Argila'' trata, indiretamente, da questão da hospitalidade, mas
``Os mortos'' aborda o tema de maneira bastante explícita. Na
verdade, talvez com o intuito de reparar o pecadilho confessado a
Stanislaus, Joyce abre o conto ``Os mortos'', que não constava do
primeiro manuscrito de \textit{Dubliners}, e que foi concluído após a referida
carta, com uma festa. Em seu discurso da ceia de Natal, Gabriel Conroy
rasga elogios à Irlanda e aos irlandeses, especialmente por sua
hospitalidade. Se, por um lado, a morte representa do ponto de vista
físico o êxtase máximo, por outro, em festas como as que são
anualmente organizadas pelas irmãs Morkan, a morte torna-se objeto do
encômio de Conroy, sendo lembrada --- e por que não dizer, celebrada
--- em meio à dança, ao movimento, à vida e aos vivos.

E para dispor de um breve \textit{insight} quanto ao significado da vida (e da
morte?), os personagens joyceanos são objetos de epifanias. Em termos
teológicos, a epifania trata da manifestação de Cristo aos Reis Magos.
Como sabemos, a epifania é uma manifestação espiritual, uma relação
transcendental entre o universo interior e o exterior. Em termos
literários, o escritor moderno, capitulando diante da
impossibilidade de compreender o caos que o cerca, busca indícios
externos que o levem a significados

internos. Podemos deduzir a partir da visão de mundo oferecida pelos
contos coligidos em \textit{Dubliners} que há momentos epifânicos ao alcance de
cada um de nós, basta buscá-los.

Assim, essa `saída' para o caos da modernidade possui uma origem
espiritual, ao menos em nível de inspiração para o escritor. Sabemos
que Joyce toma a noção de epifania emprestada dos ritos da Igreja
católica. Em carta a Stanislaus, o escritor afirma: ``Você não acha
que existe uma certa semelhança entre o mistério da missa e o que eu
estou tentando fazer? Ou seja, estou tentando (...) dar às pessoas
algum tipo de \textit{prazer intelectual ou satisfação espiritual},
transformando o pão da vida cotidiana em algo que tem uma vida
artística própria e permanente (...) com o intuito de promover [junto
às pessoas] uma elevação mental, moral e espiritual'' (Stanislaus,
pp. 103-104) (minha ênfase).

Em suma, para Joyce, a tarefa do homem de letras é registrar esses
estados de espírito sutis e evanescentes, tornando-se um
`colecionador de epifanias'. E, embora a doutrina informe toda a
obra de Joyce, \textit{Dubliners} encerra a mais contundente coleção de
epifanias (Levin, p. 29). O leitor deve buscar momentos epifânicos
principalmente nas conclusões de ``O encontro'', ``Araby'',
``Eveune'', ``Um caso trágico'', e ``Os mortos''.

Tipicamente, o escritor moderno, tanto na prosa quanto na poesia,
coloca-se fora da ação, aguardando um `encontro casual', ou um
`pedaço de conversa', que enseje o conto ou o poema. O escritor
moderno não visa precipuamente à aventura romântica, nem ao incidente
dramático. Visa a representar/transformar a rotina da vida hodierna e
explorar os mecanismos psicológicos do comportamento humano.
Expressando exatamente esse tipo de intenção, Joyce escreve a
Stanislaus, comentando a respeito de \textit{Dubliners}: ``A idéia que tenho
do significado das coisas corriqueiras é o que desejo passar aos
dois ou três infelizes que porventura venham a ler a minha obra''.
Felizmente, Joyce enganou-se quanto à previsão do número de seus
leitores. As epifanias literárias criadas pela imaginação, pela
sensibilidade e pelo trabalho de James Joyce `civilizaram' não
somente a Irlanda mas toda a humanidade.

\section{\textls{Dublinenses}, a tradução}

Qualquer pessoa que teorize a respeito ou trabalhe com tradução
reconhece que a atividade põe em jogo um verdadeiro conflito de
lealdade; a decisão de privilegiar o texto original ou o texto
traduzido continua a ser o grande problema na teoria e na prática da
tradução. O célebre conceito de \textit{paraphrase}, proposto por John Dryden
no prefácio de sua tradução de \textit{Ovid's epistles} (1680), e que configura
uma \textit{via media} na qual o tradutor `mantém em vista' o autor do
original, dando maior ênfase ao sentido do que à literalidade das
palavras, permanece a prescrição mais sensata, ainda que difícil de ser
praticada de forma consistente.

No século \versal{XX}, Peter Newmark, entre outros, aborda a questão do
`conflito de lealdades'. Na verdade, a principal contribuição de
Newmark à teoria geral da tradução vem a ser os conceitos de tradução
comunicativa e semântica, que tratam exatamente dessa problemática.
Para Newmark, a tradução comunicativa pretende, acima de qualquer
outra intenção, assegurar ao máximo a compreensão do texto pelos
leitores do idioma alvo. Nesse tipo de tradução, as estruturas
linguísticas devem ser simples e a linguagem `clara' e
`direta', favorecendo o emprego de termos ``mais genéricos nos
trechos de maior complexidade'' (Newmark, p. 39).

Em contrapartida, Newmark afirma que ``a tradução semântica,
dentro das limitações das estruturas semânticas e sintáticas do idioma
alvo, procura reproduzir o exato significado contextual do
original''. O estudioso esclarece que a tradução semântica procura
recriar o `sabor' e a `tonalidade' do original, que as palavras
selecionadas pelo autor são `sagradas', não por serem mais
importantes do que o conteúdo, mas porque forma e conteúdo são
inseparáveis. Esclarece, também, que ``a tradução semântica
objetiva a preservação do idioleto do autor'', ou seja, de suas
formas pessoais de expressão (Newmark, pp. 39- 47). Por conseguinte, a
tradução semântica tende a ser mais literal, mais detalhada, e até
mesmo, em determinados trechos, a soar deforma `estranha' (p. 39).
Newmark conclui que textos literários em geral devem ser objetos de
tradução semântica, ao passo que textos não literários se prestam à
tradução comunicativa (p. 45).

Vale ressaltar que, uma vez que tanto a tradução comunicativa quanto a
semântica possuem, obviamente, uma preocupação semântica, a
classificação de Newmark teria mais acuidade tipológica caso
estabelecesse uma diferença entre tradução comunicativa e \textit{estética}.
Com efeito, o próprio Newmark argumenta que toda tradução é, até
certo ponto, comunicativa e semântica. Voltamos, portanto, a uma
questão de ênfase.

Em \textit{Dublinenses}, a abordagem é predominantemente semântica, i.e.,
estética. Assim procura-se respeitar e transferir, na medida do
possível, tanto a realidade contextual de \textit{Dubliners} quanto os traços
marcantes do estilo literário pessoal de Joyce, principalmente
porque grande parte dos leitores da tradução tem plena consciência de
ambos os fatores e certamente conta com a presença dos mesmos na
obra traduzida.

Na tentativa de preservar-se o `mundo' do texto original, evitou-se
a tradução ou a adaptação de nomes próprios de um modo geral,
especialmente quando nitidamente relacionados à cultura fonte. Assim,
não foram traduzidos os nomes dos personagens, de acidentes
geográficos, de ruas e parques, de unidades e frações monetárias, de
certas bebidas típicas, de associações, de danças folclóricas, nem os
títulos de periódicos e de obras não traduzidas para a língua
portuguesa. Mesmo os pronomes de tratamento foram mantidos, não apenas
por fidelidade à `realidade da ficção' original mas, também, por
uma questão de eufonia, para evitar verdadeiras paródias sonoras como,
`seu Farrington', `seu Power', `dona Conroy', `srta.
Morkan' etc.

Não cabe aqui discutir o estilo literário extremamente complexo e
sofisticado de James Joyce. Basta ressaltar que a tradução procura
observar o elemento mais vital desse estilo --- o domínio total da
polifonia, especialmente evidenciado na riqueza das inflexões
presentes nas falas dos personagens. Assim, na tentativa de transferir
os efeitos estéticos do coloquialismo presente nos diálogos dos
dublinenses, nesta tradução, os desvios das prescrições gramaticais
(bem como as variações ortográficas) observadas no discurso de certos
personagens são absolutamente propositais. Conforme observa-se no
original, tais desvios, além dos efeitos estéticos inerentes à
\textit{heteroglossia}, marcam a própria caracterização sociocultural dos
falantes e.g., Eliza Flynn, Mrs. Kernan, Lenehan e Corley, Polley
Mooney, Lily, Ignatius Gallaher, Freddy Malinse e outros.

Mais especificamente, a tradução dos diálogos, principalmente nos
contos ``Depois da corrida'', ``Dois galãs'', ``Uma pequena
nuvem'', ``Dia de hera na sede do comitê'', ``Graça'' e
``Os mortos'' ao seguir desvios gramaticais e coloquialismos
presentes no original, visa a uma aproximação do discurso oral, em
termos da liberalidade do emprego dos pronomes pessoais oblíquos, de
certas regências verbais e nominais, bem como de um `realismo
fonológico', sugerido através de formas como `pra', `pro',
`tá', `tava', `cadê' e de determinadas interjeições.

Finalmente, ainda no que tange a aspectos de estilo, cabe uma
observação sobre pontuação gráfica e repetição de palavras e sons. A
intolerância de Joyce com relação a vírgulas é notória. Ao revisar as
provas da primeira edição de \textit{Dubliners}, por exemplo, o escritor
retirou centenas de vírgulas, inseridas pelo tipógrafo a título de
correção. Nesta tradução, procura-se reproduzir a pontuação (ou
mesmo a falta de pontuação) do texto original. Apenas quando a
ausência da pontuação no texto traduzido compromete o sentido,
enquanto a mesma ausência, no original, não provoca uma ambiguidade
comparável, o tradutor regulariza o texto por meio de vírgula ou ponto
e vírgula. Seguindo a mesma orientação, alguns empregos de dois pontos
foram substituídos por ponto e vírgula.

A repetição de palavras (mesmo no caso dos sucessivos `e', em lugar
das vírgulas desprezadas) não reflete necessariamente a pobreza
vocabular do tradutor; na verdade, tenta-se reproduzir o efeito da
genialidade de Joyce no que toca a cadência e a musicalidade de sua
prosa. O mesmo aplica-se às aliterações e às assonâncias.

Concluindo, o tradutor agradece a valiosa colaboração do Dr.Weldon
Thornton, joyceano de renome internacional e professor de literatura
inglesa da University of North Carolina, em Chapel Hill, nos EUA.
Agradece, também, o apoio dos colegas do Departamento de Língua e
Literatura Estrangeira da Universidade Federal de Santa Catarina
que, em Colegiado, concederam-lhe
 vinte horas semanais em seu Plano Individual de Trabalho, para
pesquisa e realização da tradução. E agradece, ainda, a decisiva
contribuição de sua revisora, companheira de todas as horas.

\begin{bibliohedra}

\tit{Ellmann}, Richard. \textit{James Joyce (1959)}. Nova York: Oxford
\versal{UP}, 1981.

\tit{Joyce}, James. \textit{Letters of James Joyce}. Ed.~Stuart
Gilbert. New York: Viking, 1957.

\tit{Joyce}, Stanislaus. \textit{My brotlier's keeper}. Ed.~Richard Ellmann.
New York: Viking, 1958.

\tit{Levin}, Harry. \textit{James Joyce: a critical introduction}. Edição revista e
ampliada. New York: James Laughlin, 1960.

\tit{Newmark}, Peter. \textit{Approaches to translation}. Oxford: Pergamon Press,
1981.

\end{bibliohedra}


