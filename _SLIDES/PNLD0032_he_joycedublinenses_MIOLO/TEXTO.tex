\part{Dublinenses}


\chapter{As irmãs}
\hedramarkboth{As irmãs}{James Joyce}

%\openany

\textsc{Não havia esperança} para ele desta vez: era o terceiro derrame.  Noite
após noite eu passara pela frente da casa (era período de férias) e observara o
quadrado iluminado da janela: e noite após noite eu o encontrara iluminado do
mesmo modo, uma luz fraca e uniforme.  Se ele estivesse morto, pensava, eu
veria o reflexo das velas nas persianas abaixadas, pois sabia que duas velas
tinham de ser colocadas à cabeceira do defunto.  Ele sempre me dizia:
\textit{Não vou durar muito neste mundo}, e eu achava que fosse conversa fiada.
Agora sabia que era verdade.  Toda noite ao erguer os olhos para a janela eu
repetia em voz baixa a palavra \textit{paralisia}.  Sempre soara estranha aos
meus ouvidos, como a palavra \textit{gnômon} em Euclides e a palavra
\textit{simonia} no catecismo.  Mas agora soava como o nome de algum ser
maleficente e pecaminoso.  Enchia"-me de medo, e ainda assim eu ansiava por me
aproximar dela e contemplar sua força mortífera.

O velho Cotter estava sentado diante do fogo, fumando, quando desci para
jantar.  Enquanto minha tia servia o mingau ele disse, como se estivesse
retomando algum comentário anterior:

--- Não, não vou dizer que ele fosse exatamente\ldots{} mas havia algo
estranho\ldots{} havia algo de misterioso nele.  Na minha opinião\ldots{}

Deu umas baforadas no cachimbo, decerto elaborando na mente a tal opinião.
Velho chato e bobo! Quando o conhecemos era um sujeito bem cativante, que
falava de bagaço e de serpentinas; mas logo cansei"-me dele e de suas histórias
intermináveis a respeito do alambique.

--- Eu tenho a minha teoria sobre o assunto --- ele disse.  --- Acho que era um
daqueles\ldots{} casos estranhos\ldots{} Mas é difícil saber ao certo\ldots{}

Recomeçou a dar baforadas no cachimbo sem nos expor sua teoria.  Meu tio
percebeu que eu estava prestando atenção e disse, dirigindo"-se a mim:

--- Pois é, seu velho amigo se foi, sinto informar"-lhe.

--- Quem? --- disse eu.

--- O padre Flynn.

--- Ele morreu?

--- Mr.~Cotter acaba de nos dar a notícia.  Passou em frente da casa há pouco.

Eu sabia que estava sendo observado portanto continuei a jantar como se a
notícia não tivesse me interessado.  Meu tio explicou ao velho Cotter:

--- O garoto e ele eram grandes amigos.  O velho ensinou"-lhe muitas coisas,
fique você sabendo; e dizem que queria muito bem a ele.

--- Que Deus tenha piedade de sua alma --- disse minha tia com devoção.

O velho Cotter olhou para mim durante alguns instantes.  Senti que seus olhos
negros, pequenos e redondos como duas contas, me examinavam mas eu não
satisfaria sua curiosidade e mantive os olhos no meu prato.  Ele se voltou para
o cachimbo e finalmente deu uma cusparada nojenta na lareira.

--- Eu é que não deixaria meus filhos --- ele disse --- terem muita conversa
com um sujeito daqueles.

--- O que o senhor está querendo dizer, Mr.~Cotter? --- perguntou minha tia.

--- O que eu estou querendo dizer --- respondeu Cotter --- é que é ruim
para as crianças.  Eu acho o seguinte: um menino deve correr por aí e brincar
com meninos da idade dele e não\ldots{} Não estou certo, Jack?

--- Tenho o mesmo princípio --- disse meu tio.  --- Os meninos têm de aprender
a se defender.  Eu sempre digo pra esse rosa"-cruz aqui: faça exercícios.  Saiba
que quando eu era moleque todas as manhãs tomava banho frio, no inverno e no
verão.  E é isso que me dá forças agora.  A educação é importante e tudo o
mais\ldots{} Mr.~Cotter vai aceitar um pedacinho dessa perna de carneiro ---
ele acrescentou, dirigindo"-se a minha tia.

--- Não, não, não pra mim --- disse o velho Cotter.

Minha tia retirou a travessa do guarda"-comida e colocou"-a sobre a mesa.

--- Mas por que o senhor acha que não é bom para crianças, Mr.~Cotter?  ---
ela perguntou.

--- É ruim para as crianças --- disse o velho Cotter --- porque elas se
impressionam com facilidade.  Quando as crianças veem coisas assim, a senhora
sabe, isso tem consequências\ldots{}

Enchi a boca de mingau com medo de deixar escapar minha raiva.  Velho chato e
imbecil, nariz de bêbado!

Já era tarde quando peguei no sono.  Embora zangado com o velho Cotter por me
tratar como criança, fiquei quebrando a cabeça para entender o significado de
suas frases incompletas.  Na escuridão do quarto eu achei que estivesse vendo
novamente o rosto grave e cinzento do paralítico.  Cobri a cabeça com as
cobertas e tentei pensar no Natal.  Mas o rosto cinzento ainda me seguia.
Murmurava; e entendi que desejava confessar algo.  Senti minha alma escapando
para uma região agradável e maldosa; e ali encontrei o rosto novamente
esperando por mim.  Começou a confessar"-se com uma voz murmurante e perguntei a
mim mesmo por que ele não parava de sorrir e por que tinha os lábios tão úmidos
de saliva.  Então lembrei"-me de que ele morrera de paralisia e senti que eu
também exibia um leve sorriso, como que para absolver o simoníaco de seu
pecado.

Na manhã seguinte depois do café fui ver a casinha que ficava na Great Britain
Street.  Era uma loja modesta registrada sob o termo genérico \textit{Modas}.
O estabelecimento vendia principalmente botinhas para crianças e guarda"-chuvas;
e em dias normais havia um cartaz pendurado do lado de dentro da vitrine,
dizendo: \textit{Troca"-se forro de guarda"-chuva}.  Naquele dia o cartaz não
estava visível, pois as persianas estavam abaixadas.  Um buquê de flores de
crepe estava amarrado com uma fita à argola da porta.  Duas mulheres pobres e
um menino estafeta estavam lendo o cartão preso ao buquê.  Eu também me
aproximei e li:

\begin{quote}\centering
1º de julho de 1895\smallskip

Reverendo James Flynn (outrora da Saint Catherine’s Church,\\
em Meath Street), aos sessenta e cinco anos.

\scshape r.i.p.
\end{quote}

A leitura do cartão convenceu"-me de que ele estava morto e fiquei abalado
diante da minha própria hesitação.  Se ele não estivesse morto eu iria até a
saleta escura nos fundos da loja onde o encontraria sentado em sua poltrona
diante da lareira, quase sufocado dentro do sobretudo.  Minha tia teria talvez
mandado para ele por meu intermédio um pacote de rapé High Toast e o presente o
despertaria de seu cochilo entorpecido.  Era sempre eu quem esvaziava o pacote
dentro da tabaqueira preta, pois suas mãos tremiam tanto que o impossibilitavam
de realizar tal tarefa sem derramar no chão a metade do rapé.  Mesmo quando
elevava a manzorra trêmula até o nariz pequenas nuvens de fumaça escapavam"-lhe
por entre os dedos caindo pelo casaco.  Talvez fossem aquelas chuvas constantes
de rapé que lhe atribuíssem às velhas vestes sacerdotais um tom verde
desbotado, pois o lenço vermelho, encardido como sempre, com manchas de rapé
acumuladas durante toda a semana, e com o qual ele limpava os grãos que caíam,
era bastante ineficaz.

Eu queria entrar para vê"-lo mas não tinha coragem de bater à porta.  Afastei"-me
lentamente andando pelo lado ensolarado da rua, lendo pelo caminho os cartazes
de teatro nas vitrines das lojas.  Achei estranho que nem eu nem o dia
estivéssemos de luto e fiquei até aborrecido ao descobrir em mim uma sensação
de liberdade como se tivesse sido libertado de algo em consequência da morte
dele.  A sensação causou"-me espanto pois, como dissera meu tio na noite
anterior, ele havia me ensinado muitas coisas.  Tinha cursado o Pontifício
Colégio Irlandês em Roma e me ensinara a pronunciar latim corretamente.
Contara"-me histórias sobre as catacumbas e sobre Napoleão Bonaparte, e me
explicara o significado dos diversos ritos da missa e dos paramentos usados
pelo sacerdote.  Às vezes divertia"-se colocando"-me questões difíceis,
perguntando como devemos proceder em determinadas circunstâncias ou se tais e
tais pecados eram mortais ou veniais ou apenas imperfeições.  Suas perguntas
fizeram"-me ver como são complexos e misteriosos certos institutos da Igreja que
anteriormente tinham me parecido extremamente simples.  Os deveres do padre em
relação à eucaristia e ao sigilo do confessionário pareceram"-me tão graves que
me perguntava como era possível alguém encontrar em si mesmo a coragem de
assumi"-los; e não fiquei surpreso quando ele me disse que os padres da Igreja
haviam escrito livros da grossura de um catálogo de endereços e com um tipo tão
pequeno quanto o que é utilizado nos anúncios de proclamas no jornal,
elucidando todas essas questões intrincadas.  Muitas vezes, quando pensava
nessas questões eu não era capaz de respondê"-las ou apresentava alguma resposta
tola e hesitante, diante da qual ele costumava sorrir e menear a cabeça duas ou
três vezes.  Às vezes me arguia sobre as réplicas da missa, que me havia
obrigado a decorar; e, quando eu gaguejava, ele sorria com ar pensativo e
meneava a cabeça, de quando em vez enfiando nas narinas, alternadamente,
grandes bolotas de rapé.  Quando sorria seus dentes grandes e desbotados
apareciam e a língua repousava sobre o lábio inferior --- hábito que me
incomodava no início de nossa amizade quando eu ainda não o conhecia bem.

Enquanto caminhava ao sol lembrei"-me das palavras do velho Cotter e tentei
lembrar a sequência do sonho.  Recordei"-me de ter visto umas cortinas compridas
de veludo e um lustre antigo pendurado que balançava.  Minha sensação era de
ter ido muito longe, a alguma terra distante onde os costumes eram estranhos
--- à Pérsia, pensei\ldots{}  Mas não conseguia lembrar o final do sonho.

À noitinha minha tia levou"-me à casa enlutada.  O sol já se pusera; mas as
janelas das casas viradas para o oeste refletiam o dourado marrom de um grande
bloco de nuvens.  Nannie nos recebeu no \textit{hall}; e, como seria inadequado
gritar"-lhe aos ouvidos, minha tia limitou"-se a apertar"-lhe a mão em nome de
todos nós.  A velhinha apontou para cima com ar de interrogação e, diante do
sinal de assentimento esboçado por minha tia, adiantou"-se a nós e subiu com
dificuldade a escada estreita, com a cabeça curvada pouco acima do nível do
corrimão.  No primeiro patamar ela parou e fez um gesto para que nos
aproximássemos da porta aberta da câmara mortuária.  Minha tia entrou e a
velhinha, vendo que eu hesitava, pôs"-se novamente a me chamar fazendo sinal com
a mão.

Entrei na ponta dos pés.  O quarto através da barra de renda da cortina estava
banhado por uma luz fosca e dourada na qual as velas pareciam chamas pálidas e
finas.  Ele já estava no caixão.  Nannie fez um sinal e nós três ajoelhamo"-nos
ao pé da cama.  Eu fingia rezar mas não conseguia me concentrar porque os
murmúrios da velha me distraíam.  Reparei"-lhe o desleixo da saia alfinetada na
parte de trás e os saltos das botas totalmente gastos nas laterais.  Imaginei o
velho padre sorrindo deitado dentro do caixão.

Nada disso.  Quando nos levantamos e nos aproximamos da cabeceira da cama vi
que ele não estava sorrindo.  Jazia ali, solene e corpulento, paramentado como
se fosse rezar a missa, as manzorras frouxas segurando um cálice.  Seu rosto
estava feroz, cinzento e maciço, com narinas negras e cavernosas delineadas por
uma leve penugem branca.  Havia no quarto um forte odor --- as flores.

Fizemos o sinal da cruz e nos retiramos.  Na saleta do andar de baixo
encontramos Eliza dignamente sentada na poltrona que pertencera a ele.  Tateei
o caminho até chegar à minha cadeira de costume, no canto, ao mesmo tempo que
Nannie se dirigiu até o guarda"-louças e trouxe uma garrafa de \textit{sherry} e
algumas taças de vinho.  Colocou"-as sobre a mesa e convidou"-nos a beber uma
tacinha de vinho.  Então, obedecendo a uma indicação da irmã, serviu o
\textit{sherry} e passou"-nos as taças.  Insistiu para que eu aceitasse também
uns \textit{cream crackers}, mas recusei por achar que faria muito barulho ao
comê"-las.  Ela pareceu bastante decepcionada diante de minha recusa e
dirigiu"-se em silêncio até o sofá, onde sentou"-se atrás da irmã.  Ninguém
falava: todos olhávamos para a lareira vazia.

Minha tia esperou até que Eliza suspirasse e então disse:

--- Ah, bem, ele foi para um mundo melhor.

Eliza voltou a suspirar e inclinou a cabeça em sinal de assentimento.
Minha tia dedilhou a haste da taça de vinho antes de tomar um pequeno gole.

--- Ele\ldots{} em paz? --- perguntou.

--- Ah, em muita paz, minha senhora --- disse Eliza.  --- Não deu nem pra
perceber o momento em que ele deu o último suspiro.  Foi uma morte linda, Deus
seja louvado.

--- E tudo\ldots{}?

--- O padre O’Rourke esteve com ele na terça"-feira e deu a extrema"-unção e
preparou ele direitinho.

--- Então ele sabia?

--- Estava bastante conformado.

--- Ele parece bastante conformado --- disse minha tia.

--- Foi isso que a mulher que chamamos pra banhá"-lo disse.  E disse que ele
parecia estar dormindo, de tão tranquilo e conformado que estava.  Quem poderia
supor que ele seria um defunto tão bonito.

--- É mesmo --- disse minha tia.

Tomou mais um pequeno gole e disse:

--- Bem, Miss Flynn, em todo caso deve ser um grande conforto para as
senhoras saberem que fizeram tudo por ele.  As senhoras foram muito boas para
ele, não resta dúvida.

Eliza esticou o vestido para encobrir os joelhos.

--- Ah, coitado do James! --- ela disse.  --- Deus sabe que fizemos tudo o que
pudemos, mesmo pobres do jeito que somos\ldots{} não deixamos que nada faltasse
enquanto ele esteve por aqui.

Nannie reclinara a cabeça sobre a almofada do sofá e estava prestes a pegar no
sono.

--- Vejam só a pobre da Nannie --- disse Eliza olhando para ela ---, está
esgotada.  Que trabalheira tivemos, eu e ela, chamando a mulher pra banhá"-lo e
depois tendo que vesti"-lo e depois o caixão e depois marcando a missa na
capela.  Se não fosse o padre O’Rourke nem sei o que seria de nós.  Foi ele
quem trouxe essas flores e as duas velas da capela e escreveu o obituário pro
\textit{Freeman’s General} e cuidou da papelada do cemitério e do seguro do
pobre do James.

--- Que bondade da parte dele! --- disse minha tia.

Eliza fechou os olhos e sacudiu levemente a cabeça.

--- Ah, nada como a gente ter velhos amigos --- ela disse.  --- No fim das
contas\ldots{} amigos nos quais se possa confiar.

--- É verdade mesmo --- disse minha tia.  --- E tenho certeza de que agora
que alcançou o descanso eterno ele não se esquecerá das senhoras e de toda a
bondade que dispensaram a ele.

--- Ah, coitado do James! --- disse Eliza.  --- Não nos deu o menor trabalho.
A gente não ouvia ele dentro de casa mais do que ouvimos agora.  Pois é, eu sei
que ele foi pra\ldots{}

--- Quando tudo isso acabar é que as senhoras vão sentir falta dele ---
disse minha tia.

--- Eu sei --- disse Eliza.  --- Não vou mais levar pra ele o caldo de carne,
nem a senhora, madame, vai poder mandar o rapé de presente pra ele.  Ah,
coitado do James!

Deteve"-se, como se estivesse refletindo sobre o passado, e então disse com um
ar sagaz:

--- A senhora sabe, eu notei uma coisa estranha nele ultimamente.  Sempre
que eu trazia o caldo encontrava o breviário caído no chão e ele recostado na
poltrona, de boca aberta.

Ela encostou um dedo na ponta do nariz e franziu a testa: então continuou:

--- Mesmo imóvel ele repetia que antes que o verão acabasse ele sairia pra dar
uma volta num dia bonito só pra rever a velha casa onde a gente nasceu em
Irishtown e que levaria a Nannie e eu com ele.  Ah, se a gente pudesse alugar,
ao menos por um dia, baratinho, ele dizia, uma dessas carruagens modernas de
que o padre O’Rourke tinha falado, dessas que não fazem barulho\ldots{} dessas
que têm rodas ``reumáticas''\ldots{} no Johnny Rush aqui perto e sair pra dar uma
volta nós três juntos numa tarde de domingo.  Ele estava mesmo decidido a fazer
o tal passeio\ldots{} Coitado do James!

--- Que o Senhor tenha misericórdia de sua alma! --- disse minha tia.

Eliza pegou um lenço e enxugou os olhos.  Em seguida tornou a guardá"-lo no
bolso e olhou absorta e calada para a lareira vazia.

--- Ele era escrupuloso demais --- ela disse.  --- As obrigações do sacerdócio
eram pesadas demais pra ele.  E aí, a senhora sabe, teve uma vida frustrada.

--- É --- disse minha tia.  --- Era mesmo um homem desiludido.  Dava pra
notar.

O silêncio tomou conta da saleta e, aproveitando"-me do silêncio, aproximei"-me
da mesa e provei o \textit{sherry} e voltei calado para minha cadeira no canto.
Eliza parecia ter entrado num profundo estado onírico.  Aguardamos
respeitosamente até que ela própria quebrasse o silêncio: e após uma longa
pausa ela falou, lentamente:

--- Foi aquele cálice que ele quebrou\ldots{} Foi ali que tudo começou.
Disseram que não tinha problema, é claro, quero dizer, que o cálice estava
vazio.  Mas mesmo assim\ldots{} Disseram que foi culpa do coroinha.  Mas o
pobre James ficou tão abalado, Deus tenha piedade dele!

--- E então foi isso? --- disse minha tia.  --- Eu ouvi qualquer
coisa\ldots{}

Eliza assentiu com a cabeça.

--- Afetou a mente dele --- ela disse.  --- Depois que aquilo aconteceu ele
deu pra falar sozinho, não queria conversa com ninguém e ficava perambulando
pelos cantos.  Então uma noite ele foi chamado pra atender uma pessoa e eles
não conseguiram descobrir onde ele tinha se metido.  Procuraram de cima a baixo
na casa e não encontraram o menor sinal dele.  Então o sacristão disse que era
pra eles procurarem na capela.  Então eles pegaram as chaves e abriram a capela
e o sacristão e o padre O’Rourke e um outro padre procuraram por ele com uma
lanterna\ldots{} E imagine a senhora que lá estava ele, sentado sozinho no
escuro dentro do confessionário, bem acordado e com um leve sorriso!

Ela parou subitamente como se quisesse escutar algo.  Eu também me pus a
escutar; mas não havia o menor ruído na casa: e eu sabia que o velho padre
jazia imóvel dentro do caixão conforme o havíamos visto, solene e feroz na
morte, um cálice vazio sobre o peito.

Eliza voltou a falar:

--- Bem acordado e com um leve sorriso\ldots{} Então, é claro, quando viram
aquilo acharam que havia algo errado com ele\ldots{}


\chapter{Um encontro}
\hedramarkboth{Um encontro}{James Joyce}

\textsc{Foi Joe Dillon} quem nos despertou para o Velho Oeste.  Ele tinha uma
pequena coleção de antigos exemplares das revistas \textit{The Union
Jack}, \textit{Pluck} e \textit{The Halfpenny Marvel}.  Toda tarde
depois da escola nos reuníamos no quintal da casa dele e brincávamos de guerra
contra os índios.  Ele e o irmão caçula e gorducho, Leo o preguiçoso,
aquartelavam"-se no paiol de feno em cima do estábulo e nós atacávamos tentando
expulsá"-los de lá; ou travávamos uma batalha ferrenha no gramado.  No entanto,
por melhor que lutássemos jamais conseguíamos fechar um cerco nem vencer uma
única batalha e todos os nossos ataques resultavam na dança da vitória
realizada por Joe Dillon.  Os pais dele assistiam diariamente à missa das oito
da manhã na Gardiner Street e o aroma sereno de Mrs.~Dillon enchia o
\textit{hall} da casa.  Mas ele era intenso demais nas brincadeiras e nós que
éramos mais jovens ficávamos um pouco intimidados.  Parecia um índio quando
irrompia pelo jardim, com um velho abafador de bule na cabeça, batendo com o
punho numa lata e gritando:

--- Ia! iaka, iaka, iaka!

Ninguém acreditou muito quando correu a notícia de que ele tinha vocação para o
sacerdócio.  Mas era verdade.

O espírito da travessura instalou"-se em nós e, sob tal influência, as
diferenças culturais e físicas foram esquecidas.  Formamos nosso bando, alguns
com ousadia, alguns de brincadeira e alguns com certo temor: e entre estes
últimos, entre os índios relutantes que tinham receio de parecer estudiosos ou
fracotes, eu estava incluído.  As aventuras narradas na literatura sobre o
Velho Oeste não me diziam muito mas, ao menos, abriam"-me portas por onde fugir.
Eu preferia os contos policiais americanos povoados de garotas descabeladas e
destemidas e lindas.  Embora não houvesse nada de impróprio naqueles contos e
ainda que sua intenção fosse às vezes literária eles circulavam
clandestinamente na escola.  Um dia quando o padre Butler ouvia a leitura de
quatro páginas sobre história romana, o desastrado do Leo Dillon foi flagrado
com um exemplar de \textit{The Halfpenny Marvel}.

--- Esta página ou esta aqui?  Esta?  Dillon, você, levante"-se!  \textit{O dia
apenas}\ldots{} Vamos, continue!  Que dia?  \textit{O dia apenas
amanhecera}\ldots{}  Você estudou a lição?  O que é isso aí no seu bolso?

Nossos corações dispararam quando Leo Dillon entregou o folhetim e todos
assumimos um ar inocente.  Padre Butler folheou as páginas, com o cenho
fechado.

--- Que porcaria é essa? --- ele disse.  --- \textit{O cacique apache!}  É isto
o que o senhor lê em vez de estudar História Romana?  Que eu não encontre mais
essa coisa desprezível neste colégio.  O sujeito que escreveu isto, suponho, só
pode ser algum escritorzinho desprezível que faz esse tipo de coisa para ganhar
dinheiro e comprar bebida.  Muito me admiro que meninos como vocês, educados,
leiam essas coisas.  Eu até entenderia se vocês fossem\ldots{} alunos de escola
pública.  Agora, Dillon, siga meu conselho, comece a estudar ou\ldots{}

A bronca em plena aula abalou no meu íntimo a glória do Velho Oeste, e a
expressão confusa no rosto inchado de Leo Dillon afetou minha consciência.  Mas
quando o efeito repressivo da escola ficou para trás voltei a ansiar por
sensações intensas, pela fuga que somente aquelas histórias de arruaças
pareciam me proporcionar.  A brincadeira de guerra do final da tarde passou a
ser para mim tão maçante quanto a rotina matinal da escola pois eu queria me
envolver em aventuras reais.  Mas aventuras reais, refleti, não acontecem a
pessoas que ficam dentro de casa: devem ser buscadas mundo afora.

As férias de verão estavam próximas quando decidi interromper o tédio da vida
escolar ao menos por um dia.  Com Leo Dillon e um menino chamado Mahony
planejei matar aula.  Cada um de nós economizou seis \textit{pence}.  Marcamos
o encontro às dez horas da manhã em Canal Bridge.  A irmã mais velha de Mahony
escreveria um bilhete justificando"-lhe a ausência e Leo Dillon pediria ao irmão
para dizer que ele estava doente.  Combinamos seguir pela Wharf Road até chegar
à marina, então atravessar de balsa e caminhar até a Pigeon House.  Leo Dillon
temia que pudéssemos encontrar o padre Butler ou alguém do colégio; mas Mahony
perguntou, com muita sensatez, o que o padre Butler estaria fazendo na Pigeon
House.  Sentimo"-nos mais confiantes: e eu concluí a primeira parte do plano
recolhendo seis \textit{pence} dos dois, ao mesmo tempo exibindo"-lhes meus
seis.  Quando fizemos os últimos preparativos na véspera estávamos um tanto
nervosos.  Trocamos apertos de mão, rindo, e Mahony disse:

--- Até amanhã, companheiros!

Naquela noite dormi mal.  De manhã fui o primeiro a chegar à ponte, pois era o
que morava mais perto.  Escondi meus livros numa moita num local isolado nos
fundos do quintal perto do buraco onde as cinzas eram despejadas e segui
apressado pela beira do canal.  Era uma manhã de sol ameno na primeira semana
de junho.  Sentei"-me na mureta da ponte admirando meus frágeis sapatos de lona
que na noite anterior eu havia branqueado com argila e contemplando os dóceis
cavalos que subiam a ladeira puxando um bonde cheio de gente a caminho do
trabalho.  Os galhos das grandes árvores que margeavam a avenida estavam
sorridentes com pequenas folhas verde"-claro através das quais a luz do sol
penetrava obliquamente e refletia na água.  As pedras de granito da ponte
começavam a aquecer e pus"-me a bater com a mão na mureta acompanhando o ritmo
de uma melodia que me veio à mente.  Sentia"-me muito feliz.

Cinco ou dez minutos após estar ali sentado vi o uniforme cinzento de Mahony
aproximando"-se.  Ele subiu a ladeira, sorrindo, e escalou a mureta da ponte
para sentar"-se ao meu lado.  Enquanto esperávamos ele retirou o estilingue que
fazia estufar o bolso interno do paletó e explicou os aperfeiçoamentos
introduzidos.  Perguntei"-lhe por que havia trazido o estilingue e ele respondeu
que queria se esbaldar com os pombos.  Mahony falava muita gíria e chamava o
padre Butler de Velho Bunsen.\footnote{ Trocadilho referente ao
bico de Bunsen. [Todas as notas são do tradutor, exceto quando indicadas.]}  
Esperamos mais quinze minutos e nem sinal de Leo
Dillon.  Finalmente, Mahony pulou da mureta e disse:

--- Vamos embora.  Eu sabia que o balofo ia amarelar.

--- E os seis \textit{pence} dele?\ldots{} --- eu disse.

--- Confiscados --- disse Mahony.  --- Melhor pra nós: um \textit{shilling} e
meio em vez de um \textit{shilling} só.

Caminhamos pela North Strand Road até chegar em Vitriol Works e então dobramos
à direita na Wharf Road.  Mahony começou a brincar de índio assim que saímos do
campo de visão dos transeuntes.  Correu atrás de um grupo de meninas pobres,
empunhando o estilingue desarmado e, quando dois meninos pobres, por
cavalheirismo, começaram a atirar pedras em nós, ele propôs um contra"-ataque.
Argumentei que os meninos eram menores do que nós e, então seguimos nosso
caminho, e o bando de crianças esfarrapadas ficou lá gritando:
\textit{Protestantes! Protestantes!} pensando que éramos protestantes porque
Mahony, que tinha a pele morena, trazia no boné o escudo prateado de um clube
de críquete.  Ao chegarmos a Smoothing Iron tentamos fechar o cerco; mas
fracassamos pois para tal precisávamos no mínimo de três pessoas.  Vingamo"-nos
de Leo Dillon chamando"-o de amarelo e apostando quantas varadas Mr.~Ryan daria
nele às três da tarde.

Chegamos então ao rio.  Passamos um bom tempo andando pelas ruas barulhentas
sustentadas por paredões de pedra, observando o trabalho dos guindastes e
máquinas e, por causa da nossa imobilidade, levando bronca dos motoristas das
carroças rangentes.  Era meio"-dia quando chegamos ao cais e, como os
trabalhadores estavam almoçando, compramos duas broas de trigo e sentamo"-nos
para comer em cima de uns canos de metal à beira do rio.  Contemplamos o
espetáculo do comércio de Dublin --- as barcaças visíveis ao longe devido aos
rolos de fumaça branca, a frota pesqueira marrom acima de Ringsend, o grande
veleiro branco sendo descarregado no cais na margem oposta.  Mahony disse que
seria legal sair pelo mar num daqueles veleiros e até eu, vislumbrando aqueles
mastros, via ou imaginava a pequena dose de geografia que me fora ministrada na
escola ganhando consistência aos poucos diante dos meus olhos.  Escola e lar
pareciam afastar"-se cada vez mais de nós e sua influência desaparecia.

Atravessamos o Liffey na balsa, pagando a passagem para sermos transportados em
companhia de dois operários e um judeu baixinho que segurava uma bolsa.
Permanecemos sérios, quase solenes, mas num dado momento durante a travessia
trocamos um olhar e rimos.  Quando desembarcamos ficamos observando o belo
veleiro de três mastros ser descarregado, aquele que tínhamos visto do outro
cais.  Um curioso disse que se tratava de um barco norueguês.  Caminhei até a
altura da popa e tentei decifrar a inscrição mas, sem conseguir fazê"-lo, voltei
e examinei os marujos estrangeiros para ver se algum deles tinha olhos verdes
pois eu tinha uma ideia meio estranha\ldots{} Os olhos dos marinheiros eram
azuis e acinzentados e até negros.  O único marinheiro cujos olhos poderiam ser
considerados verdes era um sujeito alto que divertia as pessoas que estavam no
cais gritando jocosamente todas as vezes que caía um caixote:

--- Calma!  Calma!

Quando nos cansamos daquela cena seguimos lentamente em direção a Ringsend.  O dia
havia se tornado quente e abafado, e nas vitrines das mercearias biscoitos
bolorentos ficavam descorados.  Compramos biscoitos e chocolate e saímos
comendo diligentemente pelas ruas esquálidas onde vivem os pescadores com suas
famílias.  Como não encontramos nenhuma leiteria compramos duas garrafas de
limonada com framboesa num armazém.  Depois de matar a sede, Mahony entrou num
beco perseguindo uma gata mas o bicho fugiu para um terreno baldio.  Estávamos
os dois bastante cansados e quando chegamos ao terreno baldio fomos diretamente
até um barranco de cima do qual podíamos ver o Dodder.

Já era tarde e estávamos cansados demais para prosseguir com o projeto de
visitar a Pigeon House.  Precisávamos estar de volta em casa antes das quatro
ou nossa aventura seria descoberta.  Mahony olhou pesaroso para o estilingue e
só se animou quando sugeri que voltássemos para casa de trem.  O sol se
escondeu atrás das nuvens e deixou"-nos com nossos pensamentos exaustos e com as
migalhas de nosso farnel.

Não havia ninguém além de nós no terreno baldio.  Estávamos deitados no
barranco já há algum tempo em silêncio quando vi um homem aproximando"-se vindo
do lado oposto do terreno.  Observei o sujeito despreocupadamente enquanto
mascava um daqueles talinhos verdes que as garotas usam para adivinhar o
futuro.  Ele caminhava devagar ao longo do barranco.  Trazia uma das mãos
apoiada no quadril e com a outra segurava um pedaço de pau com o qual batia de
leve na relva.  Estava mal vestido com um terno preto meio esverdeado e usava
um chapéu barato de copa alta.  Parecia idoso, pois o bigode era grisalho.
Quando passou perto de nós olhou"-nos de relance e seguiu seu caminho.  Seguimos
o sujeito com o olhar e vimos que quando tinha dado mais ou menos cinquenta
passos voltou"-se e veio em nossa direção.  Vinha bem devagar, sempre batendo
com o pedaço de pau no chão, tão devagar que pensei que estivesse procurando
algo na relva.

Parou diante de nós e cumprimentou"-nos.  Respondemos e ele sentou"-se ao nosso
lado no barranco, devagar e com movimentos cautelosos.  Começou a falar do
tempo, dizendo que o verão prometia ser muito quente e acrescentando que as
estações haviam mudado bastante desde sua época de menino --- há muito tempo.
Disse que o período mais feliz da vida de uma pessoa era sem dúvida os anos que
passava na escola e que daria tudo para voltar a ser jovem.  Enquanto ele
expressava esses sentimentos, que nos entediavam um pouco, permanecemos
calados.  Começou então a falar de escola e de livros.  Perguntou se havíamos
lido a poesia de Thomas Moore ou a obra de Sir Walter Scott e de Lord Lytton.
Fingi ter lido todos os livros por ele mencionados de maneira que ao final ele
disse:

--- Ah, estou vendo que você é um rato de biblioteca como eu.  Agora ---
acrescentou, apontando para Mahony que nos fitava de olhos arregalados --- este
aí é diferente.  Gosta de jogos.

Ele disse que tinha em casa toda a obra de Sir Walter Scott e Lord Lytton e que
nunca se cansava de ler esses escritores.  Obviamente, acrescentou, havia
certas obras de Lord Lytton impróprias para meninos.  Mahony perguntou por que
eram impróprias para meninos --- pergunta que me irritou e constrangeu porque
achei que o sujeito fosse pensar que eu era tão idiota quanto Mahony.  O
sujeito, no entanto, apenas sorriu.  Notei que entre seus dentes amarelados
havia inúmeras falhas.  Então perguntou qual de nós dois tinha mais namoradas.
Mahony disse em tom de descaso que tinha três gatonas.  O sujeito perguntou
quantas eu tinha.  Respondi que não tinha nenhuma.  Ele não acreditou em mim e
disse que tinha certeza de que eu tinha uma namorada.  Fiquei calado.

--- Conta pra gente --- disse Mahony ao homem com um ar de atrevimento ---
quantas garotas tem o senhor?

O sujeito voltou a sorrir e disse que quando era da nossa idade tivera várias
namoradas.

--- Todo menino --- ele disse --- tem uma namorada.

Sua atitude diante da questão pareceu"-me demasiadamente liberal para um homem
da idade dele.  No fundo eu achava que o que ele dissera a respeito de meninos
e namoradas era razoável.  Mas não gostava das palavras em sua boca e fiquei
cismado com o fato de ele ter tremido uma ou duas vezes como se estivesse com
medo de alguma coisa ou sentisse um arrepio.  Percebi também que falava
corretamente.  Pôs"-se a falar"-nos de garotas, de seus cabelos macios e bonitos,
de suas mãos macias e de como nenhuma garota era tão séria quanto aparentava.
A coisa que mais gostava, prosseguiu, era olhar para uma menina bonita, com
mãos belas e alvas e cabelos bonitos e macios.  Dava"-me a impressão de estar
repetindo algo que sabia de cor ou que, hipnotizada pelo seu próprio discurso,
sua mente girava e girava lentamente numa mesma órbita.  Às vezes falava como
se estivesse se referindo a algum fato notório, e às vezes baixava o tom da voz
e falava misteriosamente, como se estivesse nos contando um segredo que ninguém
pudesse saber.  Repetia as mesmas frases inúmeras vezes, com pequenas variações
e envolvendo"-as com sua voz monótona.  Continuei a olhar para a base do
barranco, ouvindo o que ele dizia.

Depois de muito tempo o monólogo cessou.  Ele se levantou lentamente, dizendo
que precisava nos deixar por um minuto, por alguns minutos, e, sem desviar o
olhar, vi quando ele se afastou a passos lentos em direção ao outro lado do
terreno.  Permanecemos calados depois que ele se foi.  Após alguns minutos em
silêncio ouvi Mahony exclamar:

--- Vou te contar!  Olha o que ele está fazendo!

Como não respondi e nem ergui os olhos Mahony exclamou novamente:

--- Vou te contar\ldots{} Que velho safado!

--- Se ele perguntar nossos nomes --- eu disse --- você é Murphy e eu sou
Smith.

Nada mais dissemos um ao outro.  Eu estava pensando se deveria ir embora ou não
quando o sujeito voltou e sentou"-se novamente ao nosso lado.  Ele mal acabara
de sentar quando Mahony, avistando a gata que lhe havia escapado, deu um pulo e
partiu atrás do bicho correndo terreno afora.  O sujeito e eu assistimos à
caça.  A gata escapuliu mais uma vez e Mahony começou a atirar pedras no muro
que ela havia escalado.  Desistindo da brincadeira, pôs"-se a perambular pelo
fundo do terreno.

Após algum tempo o sujeito dirigiu"-se a mim.  Disse que meu amigo era um menino
um tanto arruaceiro e perguntou se levava muitas chibatadas na escola.  Tive o
ímpeto de responder com indignação que não éramos alunos de escola pública para
levar \textit{chibatadas} como ele dizia; mas fiquei calado.  Começou a falar
sobre castigo corporal em meninos.  Sua mente, como que mais uma vez
hipnotizada por suas próprias palavras, parecia girar e girar lentamente em
torno do novo centro.  Disse que meninos daquele jeito tinham de levar
chibatadas e chibatadas das boas.  Quando um menino era arruaceiro e
indisciplinado o melhor remédio era uma bela surra.  Palmatória ou cachação não
bastavam: o moleque precisava mesmo é do calor de umas gostosas chibatadas.
Fiquei surpreso com tais opiniões e sem querer ergui os olhos e o encarei.  Ao
fazê"-lo deparei"-me com um par de olhos verde"-garrafa olhando para mim por baixo
de uma testa trêmula.  Desviei o olhar.

O sujeito prosseguiu com o monólogo.  Parecia haver esquecido o liberalismo que
há pouco demonstrara.  Disse que se encontrasse um menino conversando com
garotas ou querendo namorar iria chicoteá"-lo e chicoteá"-lo: isso o ensinaria a
não falar com garotas.  E se um menino tivesse namorada e mentisse a respeito
do assunto ele lhe daria uma surra da qual o pirralho jamais se esqueceria.
Disse que nada no mundo lhe daria tanto prazer.  Descreveu para mim a tal surra
como se estivesse revelando um mistério profundo.  Teria mais prazer naquilo,
ele disse, do que em qualquer outra coisa no mundo; e seu tom de voz, à medida
que me introduzia nesse mistério, tornava"-se afetuoso e parecia pedir que eu o
compreendesse.

Esperei até a próxima pausa do monólogo.  Então levantei"-me bruscamente.  Para
não demonstrar nervosismo demorei"-me um instante fingindo amarrar o sapato e
então, dizendo que precisava ir embora, despedi"-me.  Subi o barranco devagar
mas meu coração batia descompassado com medo que ele me agarrasse pelos
tornozelos.  Quando alcancei o topo do barranco dei meia"-volta e, sem olhar
para ele, gritei para o outro lado do terreno:

--- Murphy!

Minha voz saiu com um tom forçado de valentia, e senti vergonha da estratégia
mesquinha.  Fui obrigado a chamar novamente para que Mahony ouvisse e me
respondesse.  Como meu coração palpitou quando ele atravessou o terreno
correndo em minha direção!  Corria como se viesse me socorrer.  Senti remorso,
pois no fundo sempre sentira por ele um certo desprezo.


\chapter{Araby}
\hedramarkboth{Araby}{James Joyce}

\textsc{A North Richmond Street,} por ser uma rua sem saída, era tranquila
exceto no horário em que a Christian Brother’s School dispensava os meninos.
No fundo do beco ficava uma casa de dois andares desabitada, construída no meio
de um terreno quadrado e separada das casas vizinhas.  As outras casas da rua,
ciosas das vidas decentes que abrigavam, olhavam"-se com suas imperturbáveis
fisionomias amarronzadas.

O antigo inquilino da casa em que morávamos, um sacerdote, morrera na sala dos
fundos.  O ar, pesado devido à pouca ventilação, pairava pelos cômodos, e o
depósito que havia atrás da cozinha estava entulhado de papel velho.  Entre a
papelada encontrei alguns livros de bolso, cujas páginas estavam enrugadas e
úmidas: \textit{The Abbot}, de Sir Walter Scott, \textit{The Devout
Communicant} e \textit{The Memoirs of Vidocq}.  Eu gostava mais deste último
porque tinha as folhas amareladas.  No centro do quintal abandonado que havia
atrás da casa via"-se uma macieira e alguns arbustos esparsos, sob um dos quais
encontrei, enferrujada, a bomba de encher pneu de bicicleta que pertencia ao
antigo inquilino.  Tinha sido um padre bastante caridoso; em seu testamento
deixara todo seu dinheiro para instituições de caridade e a mobília da casa
para a irmã.

Ao chegarem os curtos dias de inverno, anoitecia bem antes de acabarmos de
jantar.  Quando nos reuníamos na rua as casas tinham se tornado sombrias.  O
pedaço do céu acima de nós assumia um tom sempre mutante de violeta e em
direção ao céu os lampiões da rua erguiam suas luzes fracas.  O ar frio ardia e
brincávamos até brilhar de suor.  Nossos gritos ecoavam na rua deserta.  Nossas
brincadeiras levavam"-nos às ruelas escuras e lamacentas atrás das casas, onde
brincávamos de corredor polonês, correndo dos chalés até os portões dos fundos
dos quintais escuros e encharcados onde recendia o cheiro dos buracos usados
para despejar cinzas, e de lá até os estábulos mal iluminados e fedorentos onde
às vezes um cocheiro escovava um cavalo ou extraía música das ferragens das
rédeas.  Quando voltávamos para nossa rua, a luz das janelas das cozinhas
iluminava o exterior.  Caso meu tio fosse visto dobrando a esquina
escondíamo"-nos na sombra até termos certeza de que ele havia entrado em casa.
Ou se a irmã de Mangan viesse ao batente da porta chamar o irmão para tomar o
chá, de nosso esconderijo na sombra, ficávamos observando enquanto ela olhava a
rua de cima a baixo.  Esperávamos para ver se ela ficaria ali fora ou se
voltaria para dentro de casa e, caso ficasse, saíamos do esconderijo e nos
encaminhávamos resignadamente até os degraus da porta da casa de Mangan.  Ela
esperava por nós, com sua silhueta recortada na luz da porta entreaberta.  O
irmão sempre fazia um pouco de pirraça antes de obedecê"-la, e eu ficava
encostado ao corrimão olhando para ela.  O vestido balançava quando ela mexia o
corpo e os cabelos macios balançavam de um lado para o outro.

Todas as manhãs eu me deitava no chão da sala da frente para vigiar a porta da
casa dela.  Eu deixava apenas um dedo da persiana aberta para não ser visto.
Quando ela aparecia no batente meu coração dava um salto.  Eu corria até o
\textit{hall}, agarrava os livros e partia atrás dela.  Não tirava os olhos de
sua silhueta marrom e, quando nos aproximávamos do ponto em que nossos caminhos
se separavam, eu apertava o passo e passava por ela.  Isso se repetia manhã
após manhã.  Nunca falei com ela, a não ser algumas palavras casuais, e mesmo
assim seu nome era um apelo irresistível ao meu sangue tolo.

Sua imagem me acompanhava até em locais mais avessos ao romance.  Nas tardes de
sábado quando minha tia ia ao mercado eu era convocado a acompanhá"-la para
ajudar a carregar as sacolas.  Caminhávamos pelas ruas fulgurantes,
acotovelando"-nos com bêbados e mulheres que pechinchavam, em meio aos
xingamentos dos operários, aos gritos agudos dos meninos que anunciavam
mercadorias e ofertas enquanto vigiavam barricas de carne de porco, e à voz
nasalada dos cantores de rua, que interpretavam canções populares sobre
O’Donovan Rossa ou baladas acerca dos problemas da nossa terra.  Aqueles sons
convergiam para mim numa única sensação de vida: eu imaginava que conseguia
carregar meu cálice a salvo através de uma multidão de inimigos.  O nome dela
vinha"-me aos lábios em dados momentos formando preces e louvores estranhos que
nem eu mesmo compreenderia.  Meus olhos enchiam"-se de lágrimas (eu não sabia por
quê) e às vezes uma torrente parecia transbordar do meu coração e desaguar no
peito.  Pouco me preocupava o futuro.  Não sabia sequer se um dia conversaria
com ela, ou se o fizesse, de que modo poderia expressar minha confusa adoração.
Mas meu corpo era como uma harpa e os gestos e palavras dela eram como dedos
correndo pelas cordas.

Certa noite fui até a sala dos fundos na qual o padre tinha falecido.  Era uma
noite chuvosa e escura e na casa reinava absoluto silêncio.  Através de uma
vidraça quebrada eu ouvia a chuva violando a terra, as incessantes agulhas de
água divertindo"-se nos canteiros encharcados.  Lá embaixo brilhava ao longe um
lampião ou uma janela iluminada.  Felizmente eu não enxergava muito bem.  Era
como se meus sentidos desejassem estar velados e, sentindo"-me prestes a
perdê"-los, apertei com força uma palma da mão contra a outra até ficarem
trêmulas, murmurando \textit{Amor!  Amor!} inúmeras vezes.

Finalmente falou comigo.  Diante das primeiras palavras que ela me dirigiu
fiquei tão confuso que não soube o que responder.  Perguntou"-me se eu pretendia
ir ao \textit{Araby}.  Não me recordo se respondi sim ou não.  Vai ser uma
feira e tanto, ela disse; e acrescentou que adoraria ir.

--- E por que não vai? --- perguntei.

Enquanto falava ela girava e girava uma pulseira de prata que usava no
pulso.  Não podia ir, respondeu, porque seu colégio faria retiro naquela
semana.  O irmão dela e dois outros meninos estavam naquele momento brigando
por causa de seus bonés e eu era o único ali ao lado do corrimão.  Ela segurava
uma das barras, inclinando a cabeça em minha direção.  A luz do lampião do
outro lado da rua revelava a curva nívea do pescoço, iluminava os cabelos que
ali repousavam e, descendo, iluminava os dedos agarrados ao corrimão.
Escorregava pelo lado do vestido e revelava a ponta branca da anágua, visível
naquele momento de descontração.

--- Você é que tem sorte --- ela disse.

--- Se eu for --- eu disse --- trago alguma coisa pra você.

Incontáveis fantasias tomaram conta dos meus pensamentos depois daquela noite,
estivesse eu acordado ou dormindo!  Meu desejo era aniquilar os dias
entediantes que me separavam da ida à feira.  Os deveres escolares
irritavam"-me.  À noite em meu quarto e de dia na sala de aula a imagem dela
interpunha"-se entre meus olhos e a página que me esforçava em ler.  No silêncio
em que minha alma se deleitava as sílabas da palavra \textit{Araby} ecoavam e
produziam um encantamento oriental.  Pedi permissão para ir à feira no sábado à
noite.  Minha tia ficou surpresa e disse que esperava não se tratar de uma
reunião da maçonaria.  Passei a responder a poucas perguntas durante a aula.
Notei que a atitude de meu professor passara da amabilidade à preocupação; ele
disse que fazia votos de que eu não estivesse ficando preguiçoso.  Eu não
conseguia concatenar meus pensamentos.  Não tinha mais paciência com as tarefas
da vida cotidiana que, intrometendo"-se entre mim e meu desejo, pareciam
brincadeiras de criança, daquelas bem monótonas, bem chatas.

Na manhã de sábado lembrei meu tio que desejava ir à feira à noite.  Ele estava
todo nervoso diante do cabideiro do \textit{hall}, procurando a escova de
chapéus e respondeu rispidamente:

--- Já sei, menino, já sei.

Como ele se encontrava no \textit{hall} não pude me deitar em frente à janela
da sala da frente.  Saí de casa de mau humor e segui sem a menor pressa em
direção à escola.  O ar estava cortante de tão frio e meu coração já se enchia
de dúvidas.

Quando voltei para casa na hora do jantar meu tio não havia chegado.  Ainda era
cedo.  Sentei"-me e fiquei olhando para o relógio durante algum tempo e, quando
o tique"-taque começou a me irritar, saí da sala.  Subi as escadas e percorri o
andar de cima da casa.  Os quartos vazios e frios e soturnos provocaram em mim
uma sensação de liberdade e fui de quarto em quarto cantando.  Da janela da
frente vi meus companheiros brincando lá embaixo na rua.  Seus gritos chegavam
abafados e irreconhecíveis aos meus ouvidos e, encostando a testa no vidro
gelado, olhei para a casa escura onde ela morava.  Devo ter ficado de pé ali
durante uma hora, vendo tão"-somente a figura vestida de marrom criada pela
minha imaginação, delineada suavemente pela luz do lampião na curva do pescoço,
nos dedos presos ao corrimão e na barra do vestido.

Quando voltei ao andar de baixo encontrei Mrs.~Mercer sentada diante da
lareira.  Era uma mulher idosa e faladeira, viúva de um agiota, que colecionava
selos para colaborar com uma obra de caridade.  Fui obrigado a ouvir os
mexericos à mesa do chá.  O lanche prolongou"-se por mais de uma hora e nada de
meu tio chegar.  Mrs.~Mercer levantou"-se para partir: disse que sentia muito
não poder ficar mais, mas já passava das oito e ela não gostava de estar fora
de casa tarde da noite, pois o sereno fazia"-lhe mal.  Quando ela se foi comecei
a caminhar de um lado ao outro da sala, com os punhos cerrados.  Minha tia
disse:

--- Acho que você vai precisar adiar sua ida à feira.

Às nove horas ouvi o ruído da chave de meu tio na porta da rua.  Ouvi"-o
resmungar algumas palavras e o cabideiro balançar sob o peso de seu sobretudo.
Eu sabia muito bem interpretar esses sinais.  Quando ele estava no meio do
jantar pedi"-lhe dinheiro para ir à feira.  Ele havia se esquecido do assunto.

--- A essa hora já está todo mundo na cama, no segundo sono --- ele disse.

Não esbocei o menor sorriso.  Minha tia disse"-lhe com firmeza:

--- Será que você não pode dar logo o dinheiro e deixar o menino ir?  Ele se
atrasou por sua culpa.

Meu tio pediu desculpas pelo esquecimento.  Disse que acreditava no velho
ditado que dizia que quem só trabalha e nunca se diverte torna"-se um chato.
Perguntou"-me aonde ia e, quando repeti a informação que acabara de dar,
perguntou"-me se conhecia “O adeus do árabe ao corcel”.  Quando saí da cozinha
ele estava prestes a recitar para minha tia os primeiros versos do poema.

Segurei firme o florim que ganhara, enquanto descia pela Buckingham Street rumo
à estação.  A visão das ruas reluzentes a gás e repletas de pessoas fazendo
compras me fez lembrar o propósito da minha escapada.  Tomei assento num vagão
de terceira classe num trem vazio.  Após uma demora intolerável o trem partiu
lentamente da estação.  Arrastou"-se entre casas caindo aos pedaços e atravessou
o rio cintilante.  Na estação de Westland Row uma multidão comprimiu"-se contra
as portas do vagão; mas os cobradores empurraram"-nas para trás, dizendo que
aquele era um trem especial com destino à feira.  Continuei sozinho no vagão.
Poucos minutos depois o trem estacionou ao lado de uma plataforma de madeira,
improvisada.  Desembarquei e vi no mostrador iluminado de um relógio que
faltavam dez minutos para as dez.  Diante de mim estava o imponente pavilhão
exibindo a palavra mágica.

Não consegui encontrar nenhum guichê onde pudesse comprar entrada por seis
\textit{pence} e, com medo de que a feira fechasse, passei rapidamente pela
catraca, entregando um \textit{shilling} a um homem aparentando cansaço.  Vi"-me
num grande saguão circundado a meia altura por uma galeria.  Quase todos os
estandes estavam fechados e a maior parte do recinto estava às escuras.
Identifiquei ali um silêncio semelhante ao que reina numa igreja após a missa.
Caminhei timidamente até o centro da feira.  Algumas pessoas estavam diante dos
poucos estandes que ainda permaneciam abertos.  Em frente a uma cortina, sobre
a qual as palavras \textit{Café Chantant} apareciam em letras luminosas e
coloridas, dois sujeitos contavam dinheiro numa salva.  Ouvi o ruído das
moedas.

Com dificuldade de lembrar por que tinha vindo, fui até um dos estandes e
examinei uns vasos de porcelana e uns aparelhos de chá ornados de flores.  À
frente do estande uma jovem conversava e ria ao lado de dois rapazes.  Notei
que tinham sotaque inglês e pus"-me a escutar vagamente a conversa.

--- Ah, eu nunca disse uma coisa dessas!

--- Ah, disse, sim!

--- Ah, não disse, não!

--- Ela não disse?

--- Disse, sim.  Eu ouvi.

--- Ah, que\ldots{} lorota!

Percebendo a minha presença a moça aproximou"-se e perguntou"-me se desejava
comprar alguma coisa.  O tom de voz dela não era dos mais convidativos: parecia
ter falado comigo por obrigação.  Olhei humildemente para dois grandes jarros
perfilados como sentinelas orientais ao lado da entrada escura do estande e
murmurei:

--- Não, obrigado.

A jovem mudou a posição de um dos jarros e voltou para a companhia dos dois
rapazes.  Continuaram a falar do mesmo assunto.  Uma ou duas vezes a moça olhou
para mim por cima do ombro.

Embora soubesse que era inútil ficar ali, permaneci por uns momentos em frente
ao estande, para fingir que estava realmente interessado nas mercadorias.
Então dei meia"-volta lentamente e caminhei pelo centro do pavilhão.  Os dois
\textit{pennies} tilintavam dentro do meu bolso ao se chocarem contra a moeda
de seis \textit{pence}.  Ouvi uma voz gritar do fundo da galeria que as luzes
seriam apagadas.  A parte superior do pavilhão estava agora completamente às
escuras.

Olhando para a escuridão lá em cima vi a mim mesmo como uma criatura comandada
e ludibriada pela vaidade; meus olhos queimavam de angústia e raiva.


\chapter{Eveline}
\hedramarkboth{Eveline}{James Joyce}

\textsc{sentou"-se} à janela vendo a noite invadir a avenida.  Encostou a cabeça
na cortina e o odor de cretone empoeirado encheu"-lhe as narinas.  Sentia"-se
cansada.

Poucas pessoas passavam por ali.  O sujeito que morava no final da rua passou a
caminho de casa; ela ouviu seus passos estalando na calçada de concreto e
depois rangendo no caminho coberto com cascalho em frente às casas novas e
vermelhas.  Tempos atrás havia ali um terreno baldio onde eles brincavam toda
noite com os filhos dos vizinhos.  Então um sujeito de Belfast comprara o
terreno e construíra casas --- não eram casas pequenas e escuras como as deles,
mas casas vistosas de tijolo aparente e com telhados luzidios.  As crianças da
avenida costumavam reunir"-se para brincar naquele terreno --- crianças das
famílias Devine, Water, Dunns, o pequeno Keogh, que era manco, ela e seus
irmãos e irmãs.  Ernest, no entanto, nunca brincava: já estava crescido.  O pai
dela muitas vezes enxotava"-os do terreno com sua bengala de madeira preta; mas
geralmente o pequeno Keogh ficava de olho e dava o alarme quando avistava o pai
dela se aproximando.  Apesar de tudo consideravam"-se bem felizes naquela época.
O pai dela ainda não estava tão mal; e além disso, a mãe ainda estava viva.
Isso tudo acontecera havia muito tempo; ela e seus irmãos e irmãs tinham
crescido; a mãe estava morta.  Tizzie Dunn também morrera, e a família Water
retornara à Inglaterra.  Tudo se modifica.  Agora ela iria embora, como os
outros, ia sair de casa.

Casa!  Correu os olhos pela sala, revendo todos os objetos conhecidos que ela
espanava uma vez por semana havia tantos anos, e se perguntou de onde viria
tanta poeira.  Talvez jamais tornasse a ver aqueles objetos conhecidos dos
quais jamais imaginara separar"-se um dia.  Contudo, durante todos aqueles anos
ela jamais soubera o nome do padre cuja fotografia amarelada pendia da parede
acima da pianola quebrada ao lado da gravura colorida que louvava a beata
Margarida Maria Alacoque.  Ele tinha sido colega de escola do pai dela.  Sempre
que mostrava a foto a uma visita seu pai repetia casualmente a mesma frase:

--- Ele está em Melbourne agora.

Ela havia concordado em partir, em deixar a própria casa.  Teria sido uma
decisão sensata?  Tentou avaliar cada lado da questão.  Em casa ao menos tinha
um teto e comida; estava cercada de pessoas que conhecia desde criança.  É
claro que a rotina era pesada, tanto em casa quanto no emprego.  O que diriam
na loja quando descobrissem que ela fugira de casa com um sujeito qualquer?
Que era uma idiota, talvez; e sua vaga seria preenchida através de um anúncio
no jornal.  Miss Gavan ficaria satisfeita.  Sempre implicara com ela,
especialmente quando havia gente em volta.

--- Miss Hill, não está vendo estas senhoras esperando?

--- Mexa"-se, Miss Hill, por favor!

Ela não derramaria muitas lágrimas por deixar a loja.

Em seu novo lar, num país distante e desconhecido, não seria assim.  Estaria
casada --- ela, Eveline.  As pessoas a tratariam com respeito.  Não seria
tratada como a mãe o fora.  Mesmo agora, que estava com mais de dezenove anos,
sentia"-se às vezes ameaçada pela violência do pai.  Sabia que tinha sido isso a
causa daquelas palpitações.  Quando eram crianças ele nunca havia batido nela,
conforme batia em Harry e em Ernest, porque ela era menina; mas ultimamente
passara a ameaçá"-la e a dizer o que faria com ela não fosse a lembrança da mãe
falecida.  E agora não havia mais quem a protegesse.  Ernest estava morto e
Harry, que trabalhava com decoração de igrejas, estava quase sempre ausente
viajando pelo país.  Além do mais, o inevitável bate"-boca sobre dinheiro todo
sábado à noite começara a deixá"-la exausta.  Ela sempre entregava o salário
inteiro --- sete \textit{shillings} --- e Harry sempre enviava o que podia, mas
o problema era conseguir arrancar dinheiro do pai.  Ele dizia que ela
desperdiçava dinheiro, que não tinha juízo, que não lhe daria seu dinheiro
suado para ser jogado fora, e dizia muito mais, pois geralmente ficava em
péssimo estado nas noites de sábado.  Afinal, ele acabava dando"-lhe o dinheiro
e perguntava"-lhe se pretendia comprar as provisões para o jantar de domingo.
Então ela era obrigada a sair correndo para o mercado, segurando firme a bolsa
preta de couro enquanto abria caminho na multidão com os cotovelos, e voltava
para casa tarde carregada de pacotes.  Trabalhava pesado para manter a casa em
ordem e garantir às duas crianças que haviam ficado sob seus cuidados a
oportunidade de frequentar a escola devidamente alimentadas.  O trabalho era
pesado --- uma vida difícil --- mas agora que estava prestes a deixar tudo para
trás não considerava a vida que levava de todo indesejável.

Estava prestes a explorar uma outra vida ao lado de Frank.  Frank era muito
gentil, másculo, sincero.  Concordara em fugir com ele na barca noturna para
tornar"-se sua esposa e viver ao seu lado em Buenos Aires onde uma casa a
esperava.  Com que nitidez se recordava da primeira vez em que o vira; ele
alugava um quarto numa casa na avenida que ela costumava frequentar.  Tudo
parecia ter acontecido há apenas algumas semanas.  Ele estava parado no portão,
com o boné no alto da cabeça e o cabelo despenteado caído sobre o rosto
bronzeado.  Então começaram a se conhecer.  Ele costumava esperá"-la todas as
noites à porta da loja para acompanhá"-la até em casa.  Levou"-a para assistir
\textit{A jovem boêmia} e ela ficou radiante por sentar"-se ao lado dele num
setor do teatro onde não costumava ficar.  Ele adorava música e tinha uma voz
razoável.  As pessoas sabiam que os dois estavam namorando e, quando ele
cantava a canção sobre a jovem que amava um marinheiro, ela sempre sentia um
agradável acanhamento.  Ele costumava chamá"-la de \textit{Poppens}, a título
de brincadeira.  A princípio a ideia de ter um namorado não passara de uma
empolgação e então ela começou a gostar dele.  Frank contava histórias de
países distantes.  Começara a vida como taifeiro ganhando uma libra por mês a
bordo de um navio da Allan Line com destino ao Canadá.  Disse"-lhe também os
nomes de todos os navios em que viajara bem como de diversas companhias de
navegação.  Velejara pelo estreito de Magalhães e contara"-lhe histórias a
respeito dos terríveis habitantes da Patagônia.  Estabelecera"-se em Buenos
Aires, dizia ele, e voltara à terra natal apenas para passar férias.  O pai
dela, é claro, descobrira o namoro e a proibira de sequer dirigir"-lhe a
palavra.

--- Conheço bem esses marinheiros --- ele disse.

Um dia o pai discutira com Frank e a partir de então ela fora obrigada a
encontrar"-se com o namorado às escondidas.

A noite aprofundava"-se na avenida.  O branco de duas cartas que tinha ao colo
se tornou indistinto.  Uma era para Harry; a outra era para o pai.  Ernest era
seu irmão preferido mas também gostava de Harry.  O pai estava ficando velho,
ela notara; ele sentiria a falta dela.  Às vezes ele sabia ser agradável.
Pouco tempo antes, quando ficara acamada um dia inteiro, ele tinha lido para
ela um conto de terror e lhe preparado torradas.  Em outra ocasião, quando a
mãe ainda estava viva, fizeram juntos um piquenique em Hill of Howth.
Lembrava"-se do pai colocando o chapéu da mulher para divertir as crianças.

Estava chegando a hora, mas ela continuava sentada à janela, com a cabeça
encostada na cortina, aspirando o cheiro de cretone empoeirado.  Lá embaixo na
avenida ela ouvia um realejo tocando.  Conhecia a canção.  Estranho que o
realejo surgisse ali naquela noite para lembrá"-la da promessa que fizera à mãe,
da promessa de manter o lar unido enquanto pudesse.  Lembrou"-se da noite em que
a mãe morrera; era como se estivesse novamente no quarto fechado e escuro do
outro lado do \textit{hall} e lá fora ouvisse a melancólica canção italiana.
Tinham dado seis \textit{pence} ao tocador de realejo e pedido que ele fosse
embora.  Lembrou"-se do pai voltando ao quarto da enferma com um andar emproado
e dizendo:

--- Malditos italianos!  O que eles querem aqui?

Enquanto divagava a visão deplorável da vida que a mãe levara tocou"-a no fundo
do ser --- uma vida de sacrifícios banais culminando em loucura.  Estremeceu
quando voltou a ouvir a voz da mãe repetindo com desvairada insistência:

--- Derevaun Seraun! Derevaun Seraun!

Levantou"-se num sobressalto de pavor.  Fugir!  Precisava fugir!  Frank a
salvaria.  Daria uma vida a ela, talvez, até amor.  E ela queria viver.  Por
que haveria de ser infeliz?  Tinha direito à felicidade.  Frank a tomaria nos
braços, a envolveria em seus braços.  Ele a salvaria.

\smallskip

\noindent\dotfill

\smallskip

Lá estava ela no meio da multidão ondulante na estação de embarque de North
Wall.  Ele segurou"-lhe a mão e ela sabia que ele lhe dirigia a palavra,
repetindo alguma coisa a respeito das passagens.  A estação estava repleta de
soldados carregando malas marrons.  Através dos largos portões do embarcadouro
ela podia ver o vulto negro do navio, atracado ao longo do cais, com as vigias
iluminadas.  Ela nada respondeu.  Sentia o rosto pálido e frio e, num labirinto
de aflição, rezou pedindo a Deus que a guiasse, que lhe mostrasse qual era seu
dever.  O navio lançou dentro da névoa um silvo longo e triste.  Se partisse,
amanhã estaria no mar ao lado de Frank, navegando em direção a Buenos Aires.
As passagens dos dois já estavam compradas.  Seria possível voltar atrás depois
de tudo o que ele fizera por ela?  A aflição provocava"-lhe náuseas e ela
continuava a mover os lábios rezando fervorosamente em silêncio.

Um sino repicou em seu coração.  Deu"-se conta de que ele lhe agarrara a mão:

--- Vem!

Todos os mares do mundo agitavam"-se dentro de seu coração.  Ele a estava
levando para esses mares: ele a afogaria.  Ela se agarrou com ambas as mãos às
grades de ferro.

--- Vem!

Não! Não! Não!  Era impossível.  Suas mãos agarravam"-se ao ferro em desespero.
No meio dos mares ela deu um grito de angústia!

--- Eveline!  Evvy!

Ele correu para o outro lado do cordão de isolamento e chamou"-a para que o
seguisse.  Gritaram para que fosse em frente mas ele continuou a chamá"-la.  Ela
o encarou com o rosto pálido, passivo, como um animal indefeso.  Seus olhos não
demonstraram qualquer sinal de amor ou adeus ou reconhecimento.


\chapter{Depois da corrida}
\hedramarkboth{Depois da corrida}{James Joyce}

\textsc{Os carros deslizavam} em direção a Dublin, numa velocidade constante
como bolas numa canaleta pela Naas Road.  No topo da colina em Inchicore
espectadores comprimiam"-se para ver os carros passarem disparados de volta ao
ponto de partida e por aquele canal de miséria e inércia fluíam a riqueza e o
fruto do trabalho do Continente.  De vez em quando os espectadores vibravam
como vibram os gratos oprimidos.  No entanto, simpatizavam mesmo era com os
carros azuis --- os carros de seus amigos, os franceses.

Os franceses, afinal, tinham sido os vencedores morais.  A equipe conseguira um
ótimo resultado; obtivera o segundo e o terceiro lugares e o piloto do carro
alemão, vencedor da corrida, era supostamente belga.  Cada carro azul, por
conseguinte, recebia uma ovação especial ao alcançar o topo da colina e as
saudações eram agradecidas com sorrisos e meneios de cabeça pelos que estavam
no interior dos veículos.  Dentro de um dos carros de linhas arrojadas viajava
uma equipe de quatro rapazes cujo estado de espírito estava naquele momento
mais animado do que o que se costuma observar nos gauleses vitoriosos; de fato,
os quatro rapazes estavam quase eufóricos.  Eram Charles Ségouin, proprietário
do carro; André Rivière, jovem canadense de nascimento e técnico em
eletricidade; Villona, um húngaro gigantesco e um jovem muito elegante chamado
Doyle.  Ségouin estava bem humorado porque recebera inesperadamente algumas
encomendas adiantadas (estava prestes a abrir uma agência de automóveis em
Paris) e Rivière estava de bom humor porque seria nomeado gerente do
estabelecimento; os dois rapazes (que eram primos) estavam bem humorados também
devido ao sucesso dos carros franceses.  Villona estava bem humorado porque
tinha almoçado muito bem; e além disso era otimista por natureza.  O quarto
integrante do grupo, no entanto, estava nervoso demais para sentir"-se realmente
feliz.

Tinha cerca de vinte e seis anos de idade, um bigode fino castanho"-claro e
olhos acinzentados e inocentes.  Seu pai, que no início da vida fora um
nacionalista radical, logo mudara o modo de pensar.  Ganhara dinheiro como
açougueiro em Kingstown e depois que abriu lojas em Dublin e nos subúrbios fez
fortuna.  Tivera também a sorte de firmar contratos com a polícia e finalmente
se tornara tão rico que os jornais de Dublin a ele se referiam como um príncipe
mercador.  Mandara o filho à Inglaterra para estudar num renomado colégio
católico e mais tarde encaminhou"-o à Dublin University para estudar Direito.
Jimmy não levou os estudos muito a sério e por algum tempo seguiu caminhos
errados.  Tinha dinheiro e popularidade; e dividia seu tempo entre atividades
musicais e automobilísticas.  Foi então mandado para Cambridge por um semestre
para ampliar um pouco seus horizontes.  O pai, queixando"-se, mas no fundo
orgulhoso dos excessos do filho, pagara as contas e o trouxera de volta para
casa.  Em Cambridge ele conhecera Ségouin.  Eram apenas conhecidos mas Jimmy
gostava muito da companhia daquele indivíduo tão viajado e que supostamente era
dono de alguns dos maiores hotéis da França.  Com um sujeito desses (seu pai
era da mesma opinião) valia a pena fazer amizade, mesmo que ele não fosse um
companheiro tão glamoroso.  Villona era divertido também --- pianista brilhante
--- mas, infelizmente, muito pobre.

O carro seguia faceiro com sua carga de juventude eufórica.  Os dois primos
sentados no banco da frente; Jimmy e o amigo húngaro viajavam atrás.  Villona
estava mesmo de ótimo humor; por quilômetros e quilômetros viera entoando uma
melodia à meia"-voz.  Os franceses lançavam por cima dos ombros suas risadas e
palavras jocosas e frequentemente Jimmy era obrigado a se esticar para a frente
a fim de entender as palavras pronunciadas com rapidez.  Isso não era nada
agradável para ele, pois era quase sempre obrigado a adivinhar o significado
das palavras e a gritar contra o vento uma resposta coerente.  Além disso, o
cantarolar de Villona e o barulho do carro aumentavam a confusão.

Deslocamento rápido no espaço estimula a pessoa; o mesmo se dá com a fama; o
mesmo se dá com o acúmulo de dinheiro.  Eram três bons motivos para o
nervosismo de Jimmy.  Naquele dia fora visto por muitos amigos em companhia
desses sujeitos oriundos do Continente.  No posto de controle Ségouin o
apresentara a um dos competidores franceses e, em resposta ao murmúrio confuso
por ele oferecido como cumprimento, a cara bronzeada do piloto revelara uma
fileira de dentes brancos e brilhantes.  Foi um alívio após tal honra regressar
ao mundo profano dos espectadores em meio a cutucadas e olhares expressivos.  E
quanto a dinheiro --- ele realmente dispunha de uma quantia substanciosa.
Ségouin, talvez, não achasse a quantia tão alta assim mas Jimmy que, apesar dos
deslizes, havia herdado fortes instintos, sabia muito bem com que dificuldade o
capital havia sido acumulado.  Em um passado recente, a consciência desse fato
o fizera manter as despesas dentro de um limite razoável de irresponsabilidade
e, se ele valorizava o dinheiro mesmo na época em que satisfazia os caprichos
de sua vontade, quanto mais agora que estava prestes a comprometer a maior
parte do seu capital!  Para ele o assunto era sério.

Sem dúvida, o investimento era seguro e Ségouin dera a impressão de que era por
favor e amizade que o miserável dinheiro irlandês seria incluído na
capitalização do negócio.  Jimmy admirava a sagacidade empresarial do pai e
nesse caso fora o pai quem primeiro sugerira o investimento; trabalhar com
automóveis dava dinheiro, dinheiro grosso.  Além disso, Ségouin ostentava o ar
inconfundível da riqueza.  Jimmy pôs"-se a calcular em termos de dias de
trabalho o valor do belo carro em que ora viajava.  Como era macio!  Com que
classe tinham deslizado pelas estradas do interior!  Como num passe de mágica a
jornada despertara uma autêntica vontade de viver e galantemente o sistema
nervoso esforçava"-se para acompanhar a corrida sacolejante do ágil animal azul.

Desceram a Dame Street.  A rua estava movimentada e o trânsito mais intenso do
que de hábito, barulhento com as buzinas dos motoristas e as sinetas dos
impacientes condutores de bondes.  Ségouin estacionou próximo ao banco e Jimmy
e o amigo desceram.  Um pequeno grupo de pessoas reuniu"-se na calçada para
prestar homenagem ao motor que roncava.  Naquela noite os amigos jantariam no
hotel de Ségouin, e, nesse ínterim, Jimmy e o amigo que se hospedava com ele
iriam até em casa trocar de roupa.  O carro manobrou lentamente para seguir em
direção à Grafton Street enquanto os dois jovens abriam caminho entre os
curiosos.  Caminharam rumo ao norte com um inexplicável sentimento de decepção,
enquanto a cidade fazia pender seus pálidos globos de luz sobre eles em meio a
uma bruma de noite de verão.

Na casa de Jimmy a tal ceia estava sendo considerada uma ocasião especial.
Havia da parte dos pais dele uma mescla de orgulho e receio, e também uma certa
preocupação no sentido de não gaguejar ao mencionar nomes de grandes cidades
mundo afora.  Jimmy causou ótima impressão depois que trocou de roupa e, no
momento em que equilibrava as pontas da gravata borboleta, deve ter deixado o
pai satisfeito, mesmo em termos comerciais, por haver conseguido para o filho
qualidades geralmente impossíveis de serem compradas.  Por tudo isso, o pai
fora mais amável com Villona do que de costume e suas maneiras revelavam
sincero respeito por talentos estrangeiros; mas as gentilezas do anfitrião
provavelmente passaram despercebidas pelo húngaro, que começava a pensar apenas
no jantar.

O jantar foi excelente, requintado.  Jimmy convenceu"-se de que Ségouin tinha de
fato um gosto extremamente refinado.  O grupo recebeu a adesão de um jovem
inglês chamado Routh que Jimmy encontrara em Cambridge acompanhado de Ségouin.
Os jovens cearam numa sala aconchegante iluminada por lâmpadas elétricas em
formato de velas.  Conversaram animada e abertamente.  Jimmy, que começava a
dar asas à imaginação, visualizava a alegre jovialidade dos franceses
elegantemente entrelaçada na moldura firme dos modos do inglês.  Uma bela
imagem, pensou, e bem traçada.  Admirava a habilidade com que o anfitrião
conduzia a conversa.  Os cinco rapazes tinham gostos diferentes e àquela altura
estavam com a língua solta.  Villona, com todo respeito, passou a discorrer
para o jovem inglês, um tanto perplexo, sobre a beleza dos madrigais ingleses,
lamentando o desaparecimento dos instrumentos antigos.  Rivière, com uma certa
falta de tato, incumbiu"-se de explicar a Jimmy a razão do sucesso dos mecânicos
franceses.  A voz ressonante do húngaro estava prestes a ecoar ridicularizando
os espúrios alaúdes pintados pelos artistas românticos quando Ségouin conduziu
a conversa para a política.  Tratava"-se de um tópico do agrado de todos.
Jimmy, sob o efeito do álcool, sentiu nascer dentro de si o fervor adormecido
que herdara do pai: acabou conseguindo mexer com o apático Routh.  O ambiente
foi esquentando e a tarefa de Ségouin foi ficando cada vez mais difícil: havia
até o perigo de alguém levar algo a mal.  O atento anfitrião, na primeira
oportunidade, ergueu a taça e brindou à Humanidade e, quando todos tinham
bebido, escancarou a janela num gesto cheio de significado.

Naquela noite a cidade usava a máscara de uma capital.  Os cinco rapazes
perambularam por Stephen’s Green envoltos numa nuvem de fumaça aromática.
Falavam alto e alegremente e suas capas pendiam dos ombros.  As pessoas
abriam"-lhes caminho.  Na esquina da Grafton Street um homem baixo e gordo
ajudava duas belas damas a entrar em um carro dirigido por outro sujeito gordo.
O carro partiu e o homem baixo e gordo avistou o grupo de rapazes.

--- André!

--- É o Farley!

Seguiu"-se uma torrente de palavras.  Farley era americano.  Ninguém sabia dizer
ao certo em torno do que girava a conversa.  Villona e Rivière eram os mais
efusivos, mas todos estavam animados.  Entraram num carro, espremendo"-se, às
gargalhadas.  O veículo corria ao lado da multidão, agora diluída em cores
suaves, em meio à música de sinos jubilosos.  Tomaram o trem em Westland Row e
em questão de segundos, assim pareceu a Jimmy, já estavam saindo de Kingstown
Station.  O trocador cumprimentou Jimmy; era um senhor idoso:

--- Bela noite, senhor!

Era uma noite serena de verão; o porto parecia um espelho escurecido aos pés
dos rapazes.  Caminharam em direção a ele, de braços dados, cantando em coro
\textit{Cadet Rousselle}, e batendo os pés no chão a cada estribilho:

--- \textit{Ho! Ho! Hohé, vraiment!}

Tomaram um bote a remo no píer e rumaram para o iate do americano.  A bordo
haveria comida, música, carteado.  Villona disse com entusiasmo:

--- Que delícia!

Na cabine havia um pequeno piano.  Villona tocou uma valsa para Farley e
Rivière, Farley como cavalheiro e Rivière fazendo"-se de dama.  Então
improvisaram uma quadrilha, criando evoluções originais.  Que farra!  Jimmy
divertiu"-se a valer; aquilo sim era viver!  Finalmente, Farley perdeu o fôlego
e gritou \textit{Chega!}  Um criado serviu uma ceia leve, e os rapazes
sentaram"-se para comer, por uma questão de educação.  Mas beberam, também: era
uma noitada de boêmia!  Brindaram à Irlanda, à Inglaterra, à França, à Hungria
e aos Estados Unidos.  Jimmy fez um discurso, um discurso longo, Villona
dizendo \textit{Muito bem! Muito bem!} a cada pausa.  Houve muitos aplausos
quando ele se sentou.  O discurso deve ter sido dos bons.  Farley bateu"-lhe nas
costas e deu uma gargalhada.  Que turma animada!  Que companhia agradável!

Cartas! Cartas!  A mesa foi arrumada.  Villona voltou para o piano e dedilhou
umas improvisações.  Os outros jogavam partida após partida, atirando"-se
audazmente à aventura.  Beberam à saúde da dama de copas e da dama de ouros.
Jimmy lamentou não haver uma plateia: a espirituosidade corria à solta.  As
apostas subiram e o dinheiro começou a rolar.  Jimmy não sabia exatamente quem
estava ganhando mas sabia que estava perdendo.  A culpa era dele mesmo que
frequentemente se confundia em relação às suas próprias cartas e os parceiros
tinham de calcular seus débitos.  Eles estavam endiabrados mas Jimmy queria que
parassem: já era tarde.  Alguém brindou ao iate \textit{The Belle of
Newport}, e em seguida alguém propôs uma grande partida final.

O piano silenciara; Villona provavelmente subira ao tombadilho.  Foi uma
partida terrível.  Interromperam"-na bem próximo do desfecho para brindar à
sorte.  Jimmy deu"-se conta de que a disputa ficara entre Routh e Ségouin.  Que
emoção!  Jimmy sentia"-se bastante agitado; perderia, é claro.  Quanto teria
assinado em débitos?  Os rapazes levantaram"-se para completar os últimos
lances, falando e gesticulando.  Routh venceu.  A cabine do iate balançou com a
balbúrdia dos jovens e as cartas foram amontoadas.  Começaram então a conferir
os ganhos.  Farley e Jimmy foram os que mais perderam.

Ele sabia que no dia seguinte haveria de se arrepender, mas por ora estava
satisfeito com o fato de terem parado, satisfeito com o estupor sombrio que
encobriria sua leviandade.  Colocou os cotovelos sobre a mesa e apoiou a cabeça
nas mãos, contando a pulsação nas têmporas.  A porta da cabine abriu"-se e ele
viu o húngaro de pé dentro de um feixe de luz cinzenta:

--- Amanhece, senhores!


\chapter{Dois galãs}
\hedramarkboth{Dois galãs}{James Joyce}

\textsc{A noite quente} e cinzenta de agosto descera sobre a cidade e uma
cálida brisa, lembrança do verão, circulava pelas ruas.  As ruas, com as lojas
fechadas para o repouso dominical, fervilhavam com uma multidão alegre e
colorida.  Como pérolas iluminadas os lampiões brilhavam do topo dos postes
altos sobre aquele quadro vivo que, mudando constantemente de forma e cor,
fazia soar em meio ao ar da noite quente e cinzenta um burburinho imutável,
incessante.

Dois rapazes desciam a ladeira que vinha de Rutland Square.  Um deles estava
concluindo um longo monólogo.  O outro, que caminhava à beira da calçada e às
vezes era obrigado a pisar na rua, devido à indelicadeza do companheiro, exibia
um semblante entretido e atento.  Era troncudo e tinha o rosto corado.  Usava
no alto da cabeça um quepe de iatista e a narrativa à qual dava ouvidos
provocava constantes ondas de expressão que se quebravam sobre seu rosto,
partindo dos cantos do nariz e dos olhos e da boca.  Pequenos jatos de um riso
chiado irrompiam"-lhe do corpo trêmulo.  Seus olhos, brilhando com um prazer
velhaco, a cada momento voltavam"-se para o rosto do companheiro.  De vez em
quando ele ajeitava a capa de chuva que trazia pendurada ao ombro à moda dos
toureiros.  Os culotes, os sapatos brancos de sola de borracha e a capa
pendurada displicentemente exprimiam juventude.  Mas sua silhueta tornava"-se
rotunda à altura da cintura, os cabelos eram ralos e grisalhos e o rosto, uma
vez passadas as ondas de expressão, tinha uma aparência gasta.

Quando estava certo de que o relato havia terminado riu discretamente durante
meio minuto.  Então disse:

--- É!\ldots{}  Essa ganha o primeiro prêmio!

Sua voz parecia desprovida de ânimo; e para reforçar as palavras ele
acrescentou com humor:

--- Essa ganha sozinha, disparada e, se você me permite, o prêmio
\textit{recherché!}

Depois de dizer isso ficou sério e calou"-se.  A língua estava cansada pois ele
havia passado a tarde inteira falando num bar em Dorset Street.  A maioria das
pessoas considerava Lenehan um parasita mas, apesar da sua reputação, a
sagacidade e a eloquência de Lenehan sempre impediam que os amigos
estabelecessem certas práticas contra ele.  Tinha um jeito desinibido de
aproximar"-se de um grupo de amigos num bar e manter"-se habilmente na periferia
até ser incluído na roda.  Era um vagabundo bem"-humorado, provido de um grande
repertório de histórias, trovas e anedotas.  Era imune a todo e qualquer tipo
de ofensa.  Ninguém sabia como ele se desincumbia da difícil tarefa de
sobreviver, mas seu nome estava vagamente associado ao turfe.

--- E onde você a encontrou, Corley? --- ele perguntou.

Corley passou rapidamente a língua pelo lábio superior.

--- Uma noite dessas, meu caro --- ele disse --- eu estava descendo a Dame
Street e vi uma bela vadia parada embaixo do relógio de Waterhouse e eu disse
boa"-noite, você sabe como é.  Então saímos para dar uma volta pelo canal e ela
me disse que era empregadinha numa casa em Baggot Street.  Naquela noite eu a
agarrei pela cintura e dei"-lhe uns apertos.  Então, no domingo seguinte,
camarada, marcamos um encontro.  Fomos até Donnybrook e eu a levei pra um
terreno baldio que tem lá.  Ela me disse que andava com um leiteiro\ldots{} Foi
muito bom, camarada.  Ela me trazia cigarro toda noite e pagava o bonde na ida
e na volta.  E uma noite me trouxe dois charutos dos bons\ldots{} é, coisa
fina, você sabe, da marca preferida do velhão\ldots{} Olha, fiquei com medo que
ela engravidasse.  Mas ela não é boba, não.

--- Vai ver que ela está achando que você vai casar com ela --- Lenehan disse.

--- Eu falei pra ela que estou desempregado --- Corley disse.  --- Contei
que tinha trabalhado na loja Pim.  Ela não sabe meu nome.  Sou malandro demais
pra cair na besteira de revelar meu nome pra ela. Ela acha que sou importante,
entendeu?

Lenehan voltou a rir, mais uma vez, discretamente.

--- De todas as boas que ouvi --- ele disse --- essa ganha o prêmio, disparada.

Os passos decididos de Corley agradeceram o elogio.  O movimento de seu corpo
musculoso obrigava o amigo a executar uns pequenos saltos, da calçada à rua e
de novo à calçada.  Corley era filho de um investigador de polícia e herdara o
porte e o andar do pai.  Caminhava com os braços estendidos ao lado do corpo,
com o tronco reto e balançando a cabeça de um lado para o outro.  Era uma
cabeça grande, esférica e oleosa, que suava em qualquer tipo de clima; e um
chapéu grande e redondo, assentado de lado, parecia um bulbo que havia brotado
em cima de outro.  Olhava sempre em frente como se estivesse numa parada e,
quando desejava olhar para alguém na rua, precisava girar o corpo na altura dos
quadris.  Naquela ocasião estava desempregado.  Sempre que surgia uma vaga de
trabalho algum amigo o informava.  Era visto frequentemente em companhia de
policiais à paisana, conversando com ar sério.  Estava a par dos detalhes de
todos os casos do momento e se aprazia em emitir julgamentos.  Falava sem ouvir
o discurso de seus interlocutores.  Sua conversa versava principalmente sobre
ele próprio: o que dissera a fulano e o que fulano lhe dissera e o que ele
respondera para encerrar a questão.  Quando relatava esses diálogos, aspirava a
primeira letra de seu nome à maneira dos florentinos.

Lenehan ofereceu um cigarro ao amigo.  Enquanto os dois caminhavam em meio à
multidão Corley às vezes virava"-se e sorria para as garotas que passavam mas
Lenehan mantinha os olhos fixos na lua cheia e pálida circundada por dois
halos.  Observava com atenção a película cinzenta do crepúsculo passar pela
cara da lua.  Finalmente, disse:

--- Escute\ldots{} conta pra mim, Corley, você vai conseguir o que pretende,
não vai?

A resposta de Corley foi piscar o olho expressivamente.

--- Ela vai cair na conversa? --- perguntou Lenehan, incrédulo.  --- Com mulher
a gente nunca sabe.

--- Vai, sim --- disse Corley.  --- Sei lidar com ela, camarada.  Ela está
caidinha por mim.

--- Você é mesmo um Lotário, um sedutor --- disse Lenehan.  --- E dos bons!

Uma ponta de deboche contrabalançava seus modos obsequiosos.  Por precaução,
costumava deixar que seus elogios pudessem ser interpretados como gozação.  Mas
Corley não tinha uma mente sutil.

--- Não há nada como uma empregadinha --- afirmou.  --- Pode acreditar no que
estou dizendo.

--- Vindo de alguém que já experimentou todas elas\ldots{} --- disse Lenehan.

--- Sabe, antigamente eu costumava sair com garotas de família, você sabe
como é --- disse Corley, numa espécie de desabafo ---, garotas que viajavam na
linha South Circular.  Saía por aí com elas, camarada, de bonde e pagava a
passagem ou levava as garotas a um concerto ou ao teatro ou comprava chocolate
e balas, coisas assim.  Gastava um bocado de dinheiro com elas --- acrescentou,
em tom persuasivo, como se estivesse certo de que não estava sendo levado a
sério.

Mas Lenehan estava disposto a acreditar; assentiu com um meneio de cabeça.

--- Conheço esse joguinho --- ele disse.  --- É joguinho de otário.

--- E nunca consegui porcaria nenhuma com isso --- disse Corley.

--- Nem eu --- disse Lenehan.

--- A não ser com uma delas --- disse Corley.

Umedeceu com a língua o lábio superior.  A recordação fez seus olhos brilharem.
Ele também contemplou o halo pálido da lua, agora praticamente oculta, e
parecia meditar.

--- Ela até que era\ldots{} legal --- disse em tom de remorso.

Calou"-se um instante.  Então acrescentou:

--- Ela agora está na vida fácil.  Uma noite dessas eu a vi descendo a Earl
Street num carro com dois sujeitos.

--- E você tem alguma culpa no cartório, suponho --- disse Lenehan.

--- Ela teve outros antes de mim --- retrucou Corley assumindo um ar
filosófico.

Dessa vez Lenehan estava inclinado a não acreditar.  Balançou a cabeça para um
lado e para o outro e sorriu.

--- Você sabe que não me engana, Corley --- ele disse.

--- Juro por Deus! --- disse Corley.  --- Ela mesma não me contou?

Lenehan fez um gesto trágico.

--- Que descarada!--- ele disse.

Ao passarem diante da amurada do Trinity College, Lenehan desviou os passos até
o meio da rua e olhou o relógio.

--- Vinte minutos --- ele disse.

--- Tem tempo de sobra --- disse Corley.  --- Ela vai esperar.  Eu sempre a
deixo esperando um pouco.

Lenehan riu baixinho.

--- Deus do céu, Corley, você sabe mesmo lidar com elas! --- ele disse.

--- Conheço todas as manhas delas --- Corley confessou.

--- Mas, escuta --- disse Lenehan novamente ---, você tem certeza de que vai
conseguir tudo o que pretende?  É um assunto meio delicado, você sabe.  Elas
são muito fechadas nesse assunto.  Hein?\ldots{} O que você me diz?

Seus olhos pequenos e brilhantes buscavam segurança no rosto do companheiro.
Corley sacudiu a cabeça para frente e para trás como que para espantar um
inseto teimoso, e franziu o cenho.

--- Vai dar tudo certo --- ele disse.  --- Deixa comigo, está bem?

Lenehan calou"-se.  Não queria abalar o humor do amigo, nem ser mandado para o
inferno e ouvir que seus conselhos eram dispensáveis.  Era preciso um pouco de
tato.  Mas o cenho de Corley logo voltou a descontrair"-se.  Seu pensamento
seguia em outra direção.

--- Uma vadia, e das boas --- ele disse, em tom de elogio ---, é isso mesmo o
que ela é.

Percorreram a Nassau Street e depois dobraram em Kildare Street.  Próximo à
entrada do clube um sujeito tocava harpa no meio da rua, cercado de uma pequena
roda de ouvintes.  Dedilhava as cordas distraidamente, às vezes olhando de
relance para cada novo ouvinte e às vezes, entediado, olhando para o céu.  E a
harpa, igualmente distraída, parecia também entediada com os olhares dos
estranhos e as mãos de seu amo.  Uma das mãos tocava nas cordas graves a canção
\textit{Silent, o Moyle}, enquanto a outra percorria as cordas agudas ao final
de cada sequência de notas.  As notas da melodia ressoavam graves e profundas.

Os dois rapazes subiam a rua em silêncio, seguidos pela música melancólica.
Quando chegaram ao Stephen’s Green atravessaram a rua.  Ali o barulho dos
bondes, as luzes e a multidão resgataram"-nos do silêncio.

--- Olha lá ela! --- disse Corley.

Na esquina da Hume Street havia uma jovem parada.  Usava vestido azul e um
chapeuzinho branco de marinheiro.  Estava no meio"-fio, balançando uma
sombrinha.  Lenehan animou"-se.

--- Vamos dar uma olhada de perto, Corley --- ele disse.

Corley olhou enviesado para o amigo e um sorriso sonso apareceu"-lhe nos lábios.

--- Está querendo me passar pra trás? --- ele perguntou.

--- Que nada! --- disse Lenehan com veemência.  --- Não estou pedindo pra você
me apresentar à garota.  Só quero dar uma olhada.  Não vou engolir ela, não.

--- Ah\ldots{} Uma olhada? --- disse Corley com mais amabilidade.  --- Tudo
bem\ldots{} Vamos fazer o seguinte.  Eu vou até lá para falar com ela e você
passa pela gente.

Combinado! --- disse Lenehan.

Corley já havia passado uma perna por cima da corrente que protegia a esquina
quando Lenehan gritou:

--- E depois?  Onde a gente vai se encontrar?

--- Dez e meia --- respondeu Corley passando a outra perna.

--- Onde?

--- Na esquina da Merrion Street.  A essa hora a gente estará voltando.

--- Faça tudo direitinho --- disse Lenehan ao despedir"-se.

Corley não respondeu.  Atravessou a rua faceiro balançando a cabeça para um
lado e para o outro.  Seu porte, o andar displicente e o som firme das botas
conferiam"-lhe um quê de conquistador.  Aproximou"-se da jovem e, sem
cumprimentá"-la, começou imediatamente a conversar.  Ela balançava a sombrinha
com mais rapidez e executava meias"-voltas na ponta dos saltos dos sapatos.  Uma
ou duas vezes quando ele lhe falou baixinho ela riu e inclinou a cabeça.

Lenehan observou"-os durante alguns minutos.  Então caminhou a passos rápidos ao
longo da corrente e atravessou a rua na diagonal.  Ao alcançar a esquina da
Hume Street sentiu no ar um perfume forte e com um olhar nervoso e furtivo
examinou a mulher.  Ela usava roupa domingueira.  A saia de sarja azul estava
presa à cintura por um cinto preto de couro.  O fivelão prateado parecia
afundar"-lhe o centro do corpo, prendendo o tecido leve da blusa branca como um
clipe.  Usava também uma jaquetinha preta com botões de madrepérola e um boá
preto surrado.  As bordas da gola de tule tinham sido cuidadosamente
desarrumadas e um grande ramo de flores vermelhas pendia"-lhe do peito, com as
hastes para cima.  Os olhos de Lenehan aprovaram"-lhe o corpinho robusto e
firme.  Uma vitalidade honesta e agreste brilhava no rosto dela, nas bochechas
gordas e coradas e nos ousados olhos azuis.  Os traços faciais eram grosseiros.
Tinha narinas largas, uma boca escancarada continuamente aberta num alegre
sorriso, e dois dentes protuberantes.  Quando passou por eles, Lenehan tirou o
boné e, dali a uns dez segundos, Corley devolveu a saudação.  O gesto envolveu
um leve aceno com a mão e um toque reflexivo que alterou o ângulo da posição do
chapéu.

Lenehan andou até o Hotel Shelbourne e ali ficou esperando.  Depois de breve
espera viu que eles se aproximavam e, quando dobraram à direita, seguiu"-os,
pisando macio com seus sapatos brancos, descendo por um lado da Merrion Square.
Enquanto caminhava lentamente, fazendo seus passos coincidirem com os deles,
observou a cabeça de Corley que a todo momento virava em direção ao rosto da
jovem como uma grande bola girando num pino.  Manteve o casal ao alcance da
vista até vê"-los subindo a escada de acesso ao bonde que ia para Donnybrook;
então deu meia"-volta e retornou pelo mesmo caminho.

Agora que ele estava sozinho seu rosto parecia mais velho.  A satisfação
parecia tê"-lo abandonado e, ao caminhar diante das grades do Duke’s Lawn,
correu a mão pelas barras de ferro.  A canção que o harpista executara começou
a cadenciar"-lhe os movimentos.  As solas de seus sapatos macios marcavam a
melodia enquanto seus dedos executavam variações deslizando displicentemente
pelas barras de ferro após cada sequência de notas.

Caminhou a esmo por Stephen’s Green e desceu a Grafton Street.  Embora seus
olhos captassem uma série de detalhes da multidão a sua volta, a observação era
feita de uma maneira morosa.  Achava banal tudo o que supostamente deveria
agradá"-lo e não correspondia aos olhares convidativos que o cercavam.  Sabia
que teria de falar muito, inventar histórias e fazer graça, mas tanto o cérebro
quanto a garganta estavam secos demais para realizar tal proeza.  A questão de
como passar o tempo até chegar a hora de encontrar"-se com Corley perturbava"-o
um pouco.  Não via outra coisa a fazer, a não ser continuar caminhando.  Dobrou
à esquerda quando chegou à esquina da Rutland Square e sentiu"-se mais à vontade
na rua escura e quieta, cujo aspecto soturno calhava com seu estado de
espírito.  Finalmente deteve"-se em frente à vitrine de um estabelecimento
barato onde havia um letreiro com a palavra \textit{Refrigerantes} pintada com
letras brancas.  No vidro da vitrine via"-se pintado: \textit{Ginger Beer} e
\textit{Ginger Ale}.  Um pernil fatiado estava exposto sobre uma grande
travessa azul e, ao lado, num prato, havia um pedaço de pudim de ameixa.  Ele
contemplou com ar grave a comida durante algum tempo e então, após correr os
olhos com atenção pela rua de ponta a ponta, entrou rapidamente no recinto.

Estava com fome pois, à exceção de algumas bolachas que dois garçons lhe
trouxeram de má vontade, nada comera desde o café da manhã.  Sentou"-se a uma
mesa de madeira sem toalha onde já se encontravam duas operárias e um mecânico.
Uma jovem de aspecto desmazelado veio servi"-lo.

--- Quanto custa um prato de ervilha? --- ele perguntou.

--- Três \textit{halfpence}, moço --- disse a jovem.

--- Me traz aí um prato de ervilhas --- ele disse --- e uma garrafa de
\textit{ginger beer}.

Falou asperamente para dissimular qualquer ar de refinamento pois sua entrada
havia provocado uma interrupção na conversa.  Sentia o rosto quente.  Fingindo
agir com naturalidade empurrou o boné mais para o topo da cabeça e apoiou os
cotovelos na mesa.  O mecânico e as duas operárias examinaram"-no de cima a
baixo antes de voltarem a conversar a meia voz.  A garota trouxe o prato de
ervilhas quentes, temperadas com pimenta e vinagre, um garfo e a garrafa de
\textit{ginger beer}.  Ele comeu com avidez e achou o prato tão gostoso que fez
questão de guardar o nome da lanchonete.  Quando terminou as ervilhas bebeu a
\textit{ginger beer} e ficou ali sentado pensando na aventura de Corley.
Imaginava o casal de pombinhos andando por alguma rua escura; ouvia a voz de
Corley fazendo galanteios efusivos e via novamente o sorriso malicioso da
mulher.  A visão o fez sentir na carne sua própria miséria de bolso e de
espírito.  Estava cansado de dar cabeçadas, de fazer das tripas coração, de
expedientes e intrigas.  Completaria trinta e um anos de idade em novembro.
Será que nunca conseguiria um bom emprego?  Será que nunca teria um lar?  Como
seria bom poder sentar"-se diante de uma lareira e de um jantar apetitoso.  Já
perambulara demais pelas ruas com amigos e mulheres.  Bem sabia quanto valiam
aqueles amigos: e quanto valiam as mulheres, também.  Certas experiências
haviam tornado seu coração amargo diante do mundo.  Mas não fora totalmente
abandonado pela esperança.  Sentiu"-se melhor depois de comer, menos saturado
com a própria vida, menos derrotado espiritualmente.  Talvez ainda conseguisse
se fixar em algum canto aconchegante e ser feliz, se encontrasse uma garota
bobinha com um pouco de dinheiro.

Pagou dois \textit{pence} à jovem desmazelada e saiu da lanchonete para
reiniciar a ronda.  Dobrou em Capel Street e seguiu em direção à prefeitura.
Então dobrou em Dame Street.  Na esquina da George’s Street encontrou dois
amigos e parou para falar com eles.  Agradava"-lhe a ideia de poder descansar um
pouco da caminhada.  Os amigos perguntaram"-lhe se tinha visto Corley e quais
eram as últimas novidades.  Respondeu que passara o dia com Corley.  Os amigos
falavam pouco.  Olhavam vagamente para algumas figuras na multidão e às vezes
faziam um comentário crítico.  Um dos dois disse que uma hora antes vira o Mac
em Westmoreland Street.  Ao que Lenehan comentou que estivera com o Mac na
noite anterior no Egan’s.  Os rapazes que haviam visto Mac na Westmoreland
Street perguntaram se era verdade que Mac ganhara um bom dinheirinho numa
partida de bilhar.  Lenehan não sabia: disse que Holohan pagara"-lhes os
drinques no Egan’s.

Ele deixou a companhia dos amigos às quinze para as dez e subiu a George’s
Street.  Dobrou à esquerda no mercado municipal e dirigiu"-se à Grafton Street.
A multidão de rapazes e moças havia se dispersado um pouco e ao subir a rua ele
ouviu grupos de amigos e casais despedindo"-se.  Caminhou até chegar ao relógio
do prédio do College of Surgeons: ia começar a bater dez horas.  Seguiu às
pressas ao longo do lado norte do Green, temendo que Corley chegasse antes da
hora combinada.  Quando alcançou a esquina da Merrion Street posicionou"-se à
sombra de um lampião e acendeu um dos cigarros que tinha guardado.  Encostou"-se
no poste do lampião e fixou os olhos no ponto de onde esperava ver Corley e a
mulher surgirem.

Sua mente voltou a trabalhar.  Perguntava"-se se Corley teria conseguido.
Perguntava"-se se já tinha feito o pedido, ou se deixaria para o último
instante.  Sentia todas as angústias e emoções pertinentes à situação do amigo,
além das que se relacionavam a sua própria situação.  Mas a lembrança da cabeça
de Corley girando de um lado para o outro acalmou"-o bastante: tinha certeza de
que Corley alcançaria seus intentos.  De repente assaltou"-o a ideia de que
Corley talvez tivesse levado a garota em casa por outro caminho, deixando"-o
para trás.  Vasculhou a rua com os olhos: nenhum sinal deles.  E com certeza
meia hora já se passara desde que consultara o relógio do College of Surgeons.
Será que Corley faria uma coisa dessas?  Acendeu o último cigarro e pôs"-se a
fumar bastante nervoso.  Olhava com muita atenção toda vez que parava um bonde
na esquina oposta da praça.  Vai ver que foram mesmo para casa por algum
caminho diferente.  O papel do cigarro rompeu"-se e ele atirou a guimba na rua
com um palavrão.

Súbito, avistou"-os caminhando em sua direção.  Levou um pequeno susto, cheio de
satisfação, e, mantendo"-se colado ao poste, tentou adivinhar o desfecho do caso
pela maneira como os dois caminhavam.  Andavam depressa, a garota com passos
curtos e rápidos, enquanto Corley seguia ao lado com sua passada larga.
Aparentemente, não estavam conversando.  Um presságio do desfecho espetava"-o
como a ponta de um instrumento afiado.  Sabia que Corley fracassaria; sabia que
daria tudo errado.

Dobraram em Baggot Street e ele os seguiu de perto, pela outra calçada.  Quando
paravam ele também parava.  Trocaram algumas palavras e então a garota desceu
uns degraus que davam acesso a uma casa.  Corley permaneceu no meio"-fio, a
certa distância dos degraus.  Passaram"-se alguns minutos.  Então a porta se
abriu lenta e cautelosamente.  Uma mulher desceu os degraus correndo e tossiu.
Corley virou"-se e foi ao encontro dela.  O corpo volumoso de Corley escondeu a
mulher durante alguns segundos e então ela reapareceu correndo degraus acima.
A porta fechou"-se atrás dela e Corley saiu caminhando rapidamente em direção a
Stephen’s Green.

Lenehan apertou o passo na mesma direção.  Caíam uns pingos de chuva.  Ele os
interpretou como um aviso e, virando a cabeça para trás para ver a casa em que
a garota havia entrado e para certificar"-se de que não estava sendo observado,
atravessou a rua em disparada.  O nervosismo e a corrida fizeram"-no ofegar.
Gritou:

--- Ei, Corley!

Corley virou a cabeça para ver quem o chamara, e então continuou no mesmo
passo.  Lenehan correu atrás dele, arrumando a capa sobre os ombros com uma das
mãos.

--- Ei, Corley! --- gritou novamente.

Emparelhou com o amigo e olhou atentamente para o seu rosto.  Nada havia ali de
revelador.

--- E aí? --- perguntou.  --- Conseguiu?

Haviam chegado à esquina de Ely Place.  Ainda sem responder, Corley virou à
esquerda e subiu uma ruela.  A expressão do seu rosto era serena.  Lenehan
acompanhava os passos do amigo, respirando com dificuldade.  Estava atordoado,
e um tom de ameaça transparecia"-lhe na voz.

--- Você não vai me contar? --- perguntou.  --- Conseguiu?

Corley deteve"-se embaixo do primeiro lampião e olhou para frente, carrancudo.
Então com um gesto solene estendeu o braço em direção à luz e, sorrindo, abriu
a mão lentamente diante do olhar assombrado do discípulo.  Ali brilhava uma
pequena moeda de ouro.


\chapter{A pensão}
\hedramarkboth{A pensão}{James Joyce}

\textsc{Mrs.~Mooney} era filha de açougueiro.  Era uma mulher plenamente capaz
de resolver sozinha seus próprios problemas: uma mulher decidida.  Casara"-se
com o empregado de confiança do pai e abrira um açougue perto de Spring
Gardens.  Mas logo após a morte do sogro Mr.~Mooney caiu em degradação.
Começou a beber, a dar desfalques e a afundar"-se em dívidas.  Era inútil
fazê"-lo prometer parar de beber: voltava às velhas práticas poucos dias depois.
De tanto brigar com a mulher na presença dos fregueses e de tanto comprar carne
de má qualidade arruinou o negócio.  Certa noite partiu para cima da esposa com
o cutelo e ela foi obrigada a dormir na casa de um vizinho.

Depois disso deixaram de viver sob o mesmo teto.  Ela procurou um padre e
obteve a separação e a custódia dos filhos.  Não lhe daria nem dinheiro nem
comida nem mesmo um quarto na casa; de maneira que ele foi obrigado a se
alistar como ajudante do delegado local.  Era um alcoólatra insignificante,
meio corcunda, com a cara branca e o bigode branco e as sobrancelhas brancas,
delineadas sobre dois olhinhos sempre vermelhos e congestionados; e passava o
dia inteiro sentado na sala destinada aos intendentes, esperando ser chamado
para algum serviço.  Mrs.~Mooney, que fechara o açougue e com o dinheiro que
restara abrira uma pensão em Hardwicke Street, era uma mulher grande e
imponente.  Sua casa abrigava uma população flutuante composta de turistas de
Liverpool e de Isle of Man e, ocasionalmente, artistas de teatro de variedades.
Os residentes fixos trabalhavam em escritórios na cidade.  Ela comandava a
pensão com astúcia e firmeza, sabendo quando conceder crédito, quando ser
severa e quando fazer vista grossa.  Os rapazes que moravam na pensão
referiam"-se a ela como \textit{A Madame}.

Os rapazes de Mrs.~Mooney pagavam quinze \textit{shillings} por semana, pelo
quarto e refeições (a cerveja não estava incluída).  Tinham interesses e
ocupações comuns e portanto formavam um grupo bastante coeso.  Debatiam entre
si as possibilidades tanto de amigos quanto de estranhos.  Jack Mooney, filho
da Madame, empregado de um escritório em Fleet Street, tinha a reputação de ser
um caso difícil.  Era dado a dizer obscenidades: geralmente chegava em casa às
altas horas.  Quando encontrava os amigos tinha sempre uma boa piada para lhes
contar e estava sempre na pista de alguma grande jogada, ou seja, algum cavalo
ou alguma artista.  Era também hábil com os punhos e gostava de cantar canções
picantes.  Nas noites de domingo costumava haver uma reunião na sala de visitas
de Mrs.~Mooney.  As artistas do teatro de variedades davam um pequeno show; e
Sheridan tocava valsas e polcas e improvisava os acompanhamentos.  Polly
Mooney, filha da Madame, também cantava:

\begin{verse}\itshape
Sou uma\ldots{} garota sapeca.\\
Por que então, esconder:\\
Todos já devem saber.
\end{verse}

Polly era esbelta e tinha dezenove anos; seus cabelos eram claros e macios e
sua boca era pequena e de lábios carnudos.  Os olhos, cinzentos e levemente
esverdeados, tinham o hábito de virar para cima quando ela falava com alguém, o
que lhe dava um ar de madona perversa.  Mrs.~Mooney encaminhara a filha para
ser datilógrafa no escritório de um comerciante de cereais mas, como um
desclassificado intendente de delegacia, dia sim dia não, ali comparecia
pedindo para dar uma palavrinha com a filha, ela retirara a moça do emprego e a
destinara ao trabalho doméstico.  Sendo Polly uma jovem bastante vivaz a ideia
era torná"-la responsável pelos rapazes.  Além disso, rapazes gostam de sentir
que há uma mulher jovem por perto.  Polly, é claro, flertava com eles mas 
Mrs.~Mooney, perita em avaliações, sabia que os rapazes queriam apenas um
passatempo: nenhum deles assumiria qualquer compromisso.  As coisas continuaram
nesses termos durante bastante tempo e Mrs.~Mooney começava a achar que deveria
mandar Polly de volta à datilografia quando percebeu alguma coisa entre a filha
e um dos rapazes.  Passou a observar o casal com discrição.

Polly sabia que estava sendo observada, mas o silêncio persistente da mãe não
podia ser mal"-interpretado.  Não havia qualquer cumplicidade explícita entre
mãe e filha, qualquer entendimento explícito, e, embora os moradores da pensão
começassem a falar a respeito do caso, Mrs.~Mooney não intervinha.  Polly
adotara um comportamento estranho e o rapaz ficou visivelmente perturbado.
Finalmente, quando avaliou ter chegado o momento certo, Mrs.~Mooney interveio.
Lidava com questões morais assim como um açougueiro lida com a carne: e nesse
caso sua decisão já estava tomada.

Era uma ensolarada manhã de domingo no início do verão, prometendo calor, mas
com uma brisa agradável.  Todas as janelas da pensão estavam abertas e as
cortinas de renda ondulavam do lado de fora por baixo das persianas levantadas.
O campanário da George’s Church repicava insistentemente, e os fiéis, sozinhos
ou em grupos, atravessavam o pequeno círculo em frente à igreja, indicando o
destino de seus passos tanto por sua atitude circunspecta quanto pelos volumes
que levavam nas mãos enluvadas.  O café da manhã terminara na pensão e a mesa
da sala estava coberta de pratos com manchas amarelas de ovo e restos de bacon.
Mrs.~Mooney sentou"-se na poltrona de vime e ficou observando Mary, a empregada,
tirar a mesa.  Mandou que Mary recolhesse as migalhas e os pedaços de pão para
serem utilizados no preparo do pudim de pão servido toda terça"-feira.  Quando a
mesa ficou limpa, os pedaços de pão recolhidos, o açúcar e a manteiga a salvo e
trancafiados, ela pôs"-se a reconstituir a conversa que tivera com Polly na
noite anterior.  Suas suspeitas tinham fundamento: ela fora direta nas
perguntas e Polly fora direta nas respostas.  Ambas tinham ficado
constrangidas, é claro.  Mrs.~Mooney ficara constrangida por não querer receber
a notícia com a naturalidade que desejava e por não querer parecer conivente, e
Polly ficara constrangida não apenas porque esse tipo de assunto era sempre
constrangedor mas também porque não desejava que as pessoas pensassem que em
sua sábia inocência houvesse tramado tudo à revelia da mãe.

Mrs.~Mooney olhou instintivamente para o pequeno relógio dourado que havia no
console da lareira assim que percebeu, em meio a seus devaneios, que os sinos
da George’s Church tinham parado de tocar.  Onze e dezessete: havia tempo de
sobra para resolver a questão com Mr.~Doran e ainda assistir à missa do
meio"-dia em Marlborough Street.  Tinha certeza de que sairia vitoriosa.  Para
começar tinha a seu favor a pressão social: era a mãe indignada.  Permitira que
ele morasse sob seu teto, supondo tratar"-se de um homem honrado, e ele
simplesmente abusara da sua hospitalidade.  Ele estava com trinta e quatro ou
trinta e cinco anos de idade, e portanto não poderia alegar falta de
maturidade; tampouco poderia dizer que não sabia do que ela estava falando pois
era um homem vivido.  Ele tinha mesmo se aproveitado da inocência e da
imaturidade de Polly: isto era evidente.  A questão era a seguinte: que tipo de
retratação estaria ele disposto a fazer?

Espera"-se sempre uma retratação nesses casos.  Para o homem é simples: desfruta
do momento de prazer e segue seu caminho como se nada tivesse acontecido, mas a
moça sofre as consequências.  Algumas mães dariam um caso desses por encerrado
aceitando uma quantia em dinheiro; ela conhecia alguns casos assim.  Mas não
faria isso.  Para ela somente um tipo de retratação poderia compensar a perda
da honra da filha: o casamento.

Avaliou mais uma vez todos os trunfos antes de mandar Mary ao quarto de Mr.~Doran 
para avisá"-lo que ela desejava lhe falar.  Estava certa de que sairia
vitoriosa.  Era um rapaz sério, não era libertino nem falava alto como os
outros.  Caso se tratasse de Mr.~Sheridan ou Mr.~Beade ou Bantam Lyons sua
tarefa seria muito mais árdua.  Estava certa de que ele evitaria um escândalo.
Todos os moradores da pensão estavam a par do caso; alguns chegaram a inventar
detalhes.  Além disso, ele trabalhava há treze anos numa grande firma
pertencente a um grande comerciante de vinhos, muito católico, e um escândalo
talvez acarretasse a perda do emprego.  Ao passo que se concordasse tudo
poderia acabar bem.  Ela sabia que ele tinha uma boa renda e suspeitava que
tivesse um bom dinheiro guardado.

Quase onze e meia!  Levantou"-se para se olhar no espelho.  A expressão decidida
estampada no rosto corado a satisfez e ela se lembrou de algumas mães,
conhecidas suas, que não conseguiam se livrar das filhas.

Mr.~Doran estava deveras ansioso naquela manhã de domingo.  Duas vezes tentara
barbear"-se mas a mão estava tão trêmula que fora obrigado a desistir.  Uma
barba ruiva de três dias emoldurava"-lhe os maxilares e a cada dois ou três
minutos as lentes de seus óculos ficavam embaçadas, de maneira que era preciso
tirá"-los e limpá"-los com o lenço.  A lembrança da confissão da noite anterior
causava"-lhe uma dor aguda: o padre arrancara dele os detalhes mais ridículos do
caso e ao final exagerara de tal forma a gravidade do pecado que ele quase se
sentiu grato por merecer uma saída através de retratação.  O mal já estava
feito.  O que mais poderia fazer agora a não ser casar ou fugir?  Não teria
forças para aguentar a pressão.  Com toda a certeza o caso seria comentado e
chegaria aos ouvidos do patrão.  Dublin é uma cidade pequena: todo mundo sabe
da vida de todo mundo.  Ele sentia um nó na garganta ao ouvir através de sua
imaginação agitada o velho Mr.~Leonard gritar com sua voz rouca: \textit{Mande
Mr.~Doran aqui, por favor}.

Todos aqueles anos de serviço para nada!  Todo o seu trabalho e dedicação
jogados fora!  Quando jovem ele fizera das suas, é claro; gabara"-se de ter
ideias próprias e nos bares negara aos companheiros a existência de Deus.  Mas
isso eram águas passadas\ldots{} ou quase.  Ainda comprava toda semana um
exemplar do \textit{Reynold’s Newspaper} mas cumpria seus deveres religiosos e
durante nove décimos do ano levava uma vida normal.  Tinha dinheiro suficiente
para se casar; isso não era problema.  Mas sua família olharia para ela com
certo desprezo.  Em primeiro lugar havia a figura do pai mal"-afamado e ainda
por cima a pensão dirigida pela mãe começava a pegar uma fama dúbia.
Desconfiava que estava caindo numa armadilha.  Podia ver os amigos conversando
sobre o caso às gargalhadas.  Ela \textit{era} mesmo um pouco vulgar; às vezes
falava \textit{se eu ver} e \textit{se eu sabia}.  Mas que importância teria a
gramática se de fato a amasse?  Não conseguia decidir se devia amá"-la ou
desprezá"-la pelo que ela fizera.  Evidentemente, ele tivera participação no
caso.  Seus instintos instavam"-no a continuar descompromissado, a não se casar.
Casando, não há mais saída, diziam"-lhe os instintos.

Ele estava sentado na beira da cama, desanimado e sem paletó, quando ela bateu
de leve na porta e entrou.  Contou"-lhe tudo, que se abrira com a mãe e que esta
falaria com ele naquela manhã.  Chorou e agarrou"-se ao pescoço dele, dizendo:

--- Ah, Bob! Bob!  O que vou fazer?  O que será de mim?

Daria cabo da vida, afirmou.

Ele a consolou com moderação, dizendo"-lhe que não chorasse, que tudo acabaria
bem, que não tivesse medo.  Sentia o peito dela arfando junto ao seu.

Não era o único culpado do ocorrido.  Recordava"-se muito bem, com a memória
paciente e peculiar dos celibatários, das primeiras carícias acidentais que o
vestido, o hálito e os dedos da jovem lhe fizeram.  Então certa noite, quando
já era tarde e ele se despia para dormir, ela batera timidamente à porta do
quarto.  Queria acender a vela na dele, pois que havia se apagado devido a um
golpe de ar.  Era a noite em que ela tomava banho de banheira.  Usava uma
camisola larga de flanela estampada.  O peito de seus pés brancos aparecia
pelas aberturas dos chinelos felpudos e o sangue corria cálido em sua pele
perfumada.  Um perfume suave emanava também das mãos e dos punhos enquanto ela
acendia e firmava a vela.

Nas noites em que ele chegava tarde era ela quem esquentava seu jantar.  Mal
sabia o que estava comendo, sentindo a presença dela a sós, à noite, na casa
adormecida.  E como era atenciosa!  Se a noite estava fria ou úmida ou se
ventava havia sempre uma bebida quente à espera dele.  Quem sabe seriam felizes
juntos\ldots{}

Costumavam subir juntos a escada na ponta dos pés, cada um segurando uma vela,
e no terceiro patamar diziam um boa"-noite relutante.  Beijavam"-se.  Lembrava"-se
bem dos olhos dela, do toque de sua mão e de como ele delirava\ldots{}

Mas o delírio passa.  Repetiu a frase dita por ela, dirigindo"-a a si mesmo:
\textit{O que vou fazer?}  O instinto de celibatário exortava"-o a resistir.
Mas o pecado tinha sido cometido; mesmo por uma questão de honra ele achava que
era preciso se redimir de tal pecado.

Estava sentado na cama ao lado dela quando Mary chegou à porta e disse que a
Madame queria vê"-lo na sala de visitas.  Levantou"-se para vestir o colete e o
paletó, mais desanimado do que nunca.  Quando estava pronto aproximou"-se dela
para consolá"-la.  Tudo acabaria bem, não havia nada a temer.  Quando deixou o
quarto ela ficara na cama chorando e dizendo entre leves gemidos: \textit{Ah,
meu Deus!}

Ao descer as escadas os óculos ficaram tão embaçados que ele precisou tirá"-los
para limpar.  Tinha vontade de subir pelo telhado e sair voando em direção a
algum país onde estaria livre daquela enrascada, mas uma força o impelia escada
abaixo, degrau por degrau.  As fisionomias implacáveis do patrão e da Madame
contemplavam seu embaraço.  Ao descer o último lance da escada passou por Jack
Mooney que vinha da copa com duas garrafas de cerveja embaixo do braço.
Cumprimentaram"-se friamente; e os olhos do amante fitaram durante um ou dois
segundos uma cara larga de buldogue e dois braços grossos e curtos.  Quando
chegou ao pé da escada olhou para cima e viu Jack olhando para ele da porta do
quarto de depósito.

Subitamente lembrou"-se da noite em que um dos artistas do teatro de variedades,
um londrino baixinho e louro, fizera uma referência indecorosa a Polly.  A
noite quase fora estragada em virtude da violenta reação de Jack.  Todos
tentaram acalmá"-lo.  O artista, mais pálido do que nunca, sorrira e afirmara
que não tivera intenção de ofender: mas Jack não parava de berrar que se algum
sujeito desrespeitasse sua irmã faria o fulano engolir os dentes, faria mesmo.

\smallskip

\noindent\dotfill

\smallskip

Polly continuou algum tempo sentada na beira da cama, chorando.  Então enxugou
as lágrimas e foi até o espelho.  Molhou a ponta da toalha na jarra de água e
refrescou os olhos.  Olhou"-se de perfil e arrumou um grampo acima da orelha.
Então voltou até o pé da cama e sentou"-se.  Contemplou os travesseiros durante
um bom tempo e estes despertaram"-lhe na mente lembranças íntimas e agradáveis.
Recostou a nuca na cabeceira da cama, fria e de ferro, e pôs"-se a sonhar
acordada.  Já não havia em seu rosto o menor sinal de perturbação.

Esperou pacientemente, quase feliz, calma, suas lembranças aos poucos
transformando"-se em esperanças e visões do futuro.  Tais esperanças e visões
eram de tal forma detalhadas que ela já não enxergava os travesseiros brancos
diante dos olhos nem lembrava que estava aguardando algo.

Finalmente ouviu a mãe chamá"-la.  Levantou"-se com um salto e correu até a
balaustrada da escada.

--- Polly! Polly!

--- O que é, mamãe?

--- Desça, querida.  Mr.~Doran quer falar com você.

Lembrou"-se então do que estava esperando.


\chapter{Uma pequena nuvem}
\hedramarkboth{Uma pequena nuvem}{James Joyce}

\textsc{Oito anos} antes ele acompanhara o amigo até a estação de North Wall e
ali desejara"-lhe boa viagem.  Gallaher progredira na vida.  Bastava constatar
seus ares de homem viajado, seu terno de \textit{tweed} bem cortado e o sotaque
arrojado.  Poucas pessoas têm talento como ele e menos ainda são as que
permanecem imunes ao sucesso.  Gallaher tinha a cabeça no lugar e merecia
vencer.  Como era bom tê"-lo como amigo.

Desde a hora do almoço Little Chandler só pensava no encontro com Gallaher, no
convite que Gallaher lhe fizera e na grande cidade de Londres onde Gallaher
vivia.  Era chamado de Little Chandler porque, embora de estatura pouco
inferior à média, dava a impressão de ser baixo.  Suas mãos eram pequenas e
brancas, o corpo era franzino, a voz era mansa e os modos eram refinados.
Tinha o maior cuidado com os cabelos claros e sedosos e com o bigode, e usava
no lenço um perfume discreto.  As meias"-luas em suas unhas eram perfeitas e
quando ele sorria revelava uma fileira de dentes brancos e infantis.

Sentado à sua escrivaninha na King’s Inns pensava nas mudanças ocorridas
naqueles oito anos.  O amigo que conhecera outrora maltrapilho e pobre havia se
tornado uma figura proeminente da imprensa londrina.  A cada momento desviava
os olhos de seus escritos entediantes e olhava através da janela do escritório.
O brilho de um pôr do sol de final de outono cobria os gramados e as calçadas.
Lançava uma suave névoa de poeira dourada sobre acompanhantes mal vestidas e
velhos decrépitos que cochilavam nos bancos; reluzia em todas as coisas móveis
--- nas crianças que corriam gritando pelos caminhos cobertos com cascalho e em
qualquer pessoa que passasse pelos jardins.  Ele observou a cena e pensou na
vida; e (como ocorria sempre que pensava na vida) entristeceu"-se.  Uma
melancolia afável tomou conta do seu ser.  Percebeu como era inútil lutar
contra a sorte, constatação que resultava do peso da sabedoria que o tempo lhe
trouxera.

Lembrou"-se dos livros de poesia que tinha nas estantes em casa.  Foram
comprados no seu tempo de solteiro e muitas noites, sentado na pequena sala ao
lado do \textit{hall}, sentira"-se tentado a retirar um deles da prateleira e
ler para a esposa.  Mas a timidez sempre o impedia de fazê"-lo; e então os
livros permaneciam nas prateleiras.  Às vezes recitava versos para si mesmo e
isso trazia"-lhe algum consolo.

Quando chegou a hora de encerrar o expediente levantou"-se e despediu"-se
meticulosamente da escrivaninha e dos colegas.  Emergiu dos arcos feudais dos
prédios da King’s Inns, uma figura decente e modesta, e desceu a passos rápidos
a Henrietta Street.  O crepúsculo dourado estava desaparecendo e o ar
tornara"-se cortante.  Uma horda de crianças maltrapilhas povoava a rua.
Brincavam no meio da rua ou escalavam os degraus diante das portas escancaradas
ou encolhiam"-se nas soleiras como ratos.  Little Chandler não lhes prestou
atenção.  Abriu caminho habilmente em meio àquela vida ínfima que fazia lembrar
vermes e à sombra das mansões desoladas e fantasmagóricas nas quais a velha
nobreza de Dublin outrora se divertira.  Nenhuma lembrança do passado o
atingia, pois sua mente estava tomada de uma alegria presente.

Nunca tinha estado no Corless’s mas conhecia a fama do estabelecimento.  Sabia
que era aonde as pessoas se dirigiam ao sair do teatro para comer ostra e tomar
licor; e ouvira dizer que lá os garçons falavam francês e alemão.  Passando a
pé à noite em frente ao local ele tinha visto veículos parados à porta e
mulheres muito bem"-vestidas, escoltadas por cavalheiros, desembarcando dos
carros e entrando às pressas no recinto.  Usavam vestidos que farfalhavam e
vários xales.  Tinham o rosto empoado e suspendiam a barra do vestido no
momento em que encostavam o pé na terra, como Atalantas assustadas.  Sempre que
passava por ali evitava olhar.  Tinha o hábito de andar depressa pela rua mesmo
de dia, e sempre que estava na cidade tarde da noite apressava ainda mais o
passo, nervoso e apreensivo.  Às vezes, contudo, alimentava o próprio medo.
Buscava as ruas mais escuras e estreitas e, enquanto avançava corajosamente, o
silêncio que pairava em volta de seus passos perturbava"-o, as figuras que
caminhavam em silêncio perturbavam"-no; e às vezes um riso distante e furtivo
fazia"-o tremer como vara verde.

Dobrou à direita em direção à Capel Street.  Ignatius Gallaher na imprensa
londrina!  Quem poderia prever uma coisa dessas há oito anos?  Pois é, mas
agora que reexaminava o passado, Little Chandler era capaz de recordar"-se de
inúmeros indícios do potencial do amigo.  Diziam que Ignatius Gallaher era meio
louco.  A bem da verdade, naquela época ele andava em companhia de uns sujeitos
devassos, bebia muito e tomava dinheiro emprestado de todo mundo.  Chegou mesmo
a se envolver num negócio escuso, algo relacionado a dinheiro: ao menos, era
essa uma das versões da fuga.  Mas ninguém lhe negava o talento.  Havia sempre
algo\ldots{} alguma coisa em Ignatius Gallaher que inevitavelmente
impressionava as pessoas.  Mesmo quando estava em dificuldades e sem saber o
que mais fazer para conseguir dinheiro mantinha"-se altivo.  Little Chandler
lembrou"-se (e a lembrança o fez enrubescer de orgulho) de uma das expressões de
Ignatius Gallaher sempre que este se encontrava em situação difícil:

--- Me deem um tempo, rapazes --- ele dizia em tom jocoso.  --- Onde está meu
chapéu de pensador?

Assim era Ignatius Gallaher; e, vamos e venhamos, como era possível deixar de
admirá"-lo?

Little Chandler apressou o passo.  Pela primeira vez na vida sentia"-se superior
às pessoas pelas quais passava.  Pela primeira vez sua alma revoltou"-se contra
a deselegância sombria de Capel Street.  Não havia dúvida: para vencer na vida
era mesmo preciso ir embora.  Em Dublin nada se podia fazer.  Ao cruzar a
Grattan Bridge, olhou rio abaixo em direção às docas e apiedou"-se dos casebres
ali apinhados.  Pareciam um bando de vagabundos, amontoados ao longo das
margens do rio, os paletós velhos cobertos de poeira e fuligem, estupefatos
diante do crepúsculo e esperando que a primeira brisa fria da noite ordenasse
que se levantassem, batessem a poeira dos ombros e partissem.  Perguntou a si
mesmo se não seria capaz de escrever um poema que expressasse tal ideia.
Talvez Gallaher conseguisse publicar o poema em algum jornal de Londres.  Seria
capaz de escrever algo original?  Não sabia ao certo que ideia desejava
expressar, mas a sensação de ter sido tocado por um momento poético nasceu
dentro dele como esperança de criança.  Seguiu adiante com destemor.

Cada passo aproximava"-o de Londres, distanciando"-o cada vez mais da vida sóbria
e destituída de arte que levava.  Uma luz começou a tremular no interior de sua
mente.  Não era tão velho --- tinha trinta e dois anos.  Era possível
afirmar que seu temperamento acabara de alcançar a maturidade.  Quantos
momentos e quantas impressões desejava expressar em verso!  Sentia tudo isso no
fundo do seu ser.  Dispusera"-se a avaliar a própria alma para verificar se era
alma de poeta.  A seu ver, melancolia era a característica predominante em seu
temperamento, mas era uma melancolia mesclada de recaídas de fé e resignação e
uma alegria singela.  Se conseguisse expressar tudo isso num livro de poesias
talvez as pessoas o escutassem.  Jamais seria popular: tinha consciência disso.
Jamais levantaria as massas mas talvez pudesse tocar um pequeno grupo de mentes
irmãs.  Os críticos ingleses, talvez, o classificassem como integrante da
escola celta devido ao tom melancólico de seus poemas; ademais, sua poesia
seria dada a alusões.  Começou a imaginar as sentenças das resenhas que seriam
escritas a respeito do livro: \textit{Chandler tem o dom do verso
suave e gracioso\ldots{}}  \textit{Tristeza e melancolia informam os
poemas}\ldots{ A marca celta}.  Era uma pena seu nome não ser mais irlandês.
Talvez fosse melhor adotar também o sobrenome materno: Thomas Malone Chandler,
ou melhor ainda: T.~Malone Chandler.  Falaria com Gallaher a respeito.

Estava tão absorto em seus devaneios que passou pela rua e foi obrigado a
voltar.  Ao se aproximar do Corless’s o velho nervosismo começou a dominá"-lo e
ele parou em frente à entrada indeciso.  Finalmente abriu a porta e entrou.

A luz e o barulho do bar detiveram"-no por alguns instantes no batente da porta.
Correu os olhos ao seu redor, mas a vista ficou turva devido ao brilho de um
grande número de taças de vinho em tons vermelho e verde.  O bar parecia estar
repleto e ele achou que as pessoas o observavam com curiosidade.  Olhou de
relance para a esquerda e para a direita (franzindo a testa para dar a
impressão de que viera ali com algum propósito sério), mas assim que a vista
clareou um pouco se deu conta de que ninguém tinha sequer virado o rosto para
olhá"-lo: e, conforme combinado, lá estava Ignatius Gallaher encostado no balcão
com as pernas ligeiramente abertas e bem plantadas no chão.

--- Olá, Tommy, velho de guerra, que bom que você veio!  Como é, o que vai
beber? Eu estou bebendo uísque: está melhor do que aquele que a gente toma do
outro lado do Canal.  Soda?  Tônica?  Nem mineral?  Eu também não.  Estraga o
sabor\ldots{} Ei!  Garçom, traga"-nos duas doses de uísque maltado, no
capricho\ldots{}  Pois é, e como você tem se virado desde a última vez em que
nos vimos?  Deus do céu, você está ficando velho!  Está vendo em mim algum
sinal de velhice, está?  Um pouco grisalho e careca\ldots{} é isso?

Ignatius Gallaher tirou o chapéu e exibiu a cabeça coberta por cabelos cortados
à reco.  O rosto era maciço, pálido e bem barbeado.  Os olhos, de um
azul acinzentado, contrabalançavam"-lhe a palidez doentia e brilhavam acima da
gravata alaranjada que ele usava.  Entre esses traços díspares os lábios
pareciam longos e amorfos e descorados.  Ele baixou a cabeça e com dois dedos
complacentes tocou os cabelos escassos no topo do crânio.  Little Chandler
sacudiu a cabeça em sinal de discordância.  Ignatius Gallaher recolocou o
chapéu.

--- Acaba com a gente --- ele disse --- essa vida de jornal.  Sempre aquele
corre"-corre, procurando matéria e às vezes nada encontrando: e ainda por cima,
sempre precisando descobrir alguma novidade.  Ao diabo com editores e
revisores, agora posso dizer, ao menos por uns dias.  Estou feliz pra diabo,
isso eu garanto, de estar de volta à terrinha.  Fazem bem à gente, umas
feriazinhas.  Sinto"-me mil vezes melhor desde que desembarquei nesta Dublin
querida e suja\ldots{}  Aqui está, Tommy.  Água?  Diga quanto basta.

Little Chandler deixou que seu uísque ficasse bem diluído.

--- Você não sabe o que está perdendo, rapaz --- disse Ignatius Gallaher.  ---
Tomo o meu puro.

--- Costumo beber muito pouco --- disse Little Chandler modestamente.  --- Meia
dose ou um pouco mais quando encontro algum velho amigo: não passo disso.

--- Muito bem --- disse Ignatius Gallaher, com uma voz alegre ---, à nossa
saúde, aos velhos tempos e velhas amizades.

Brindaram e beberam.

--- Encontrei alguns velhos amigos hoje --- disse Ignatius Gallaher.  ---
O’Hara parece estar em maus lençóis.  Com o que ele está trabalhando?

--- Não está trabalhando --- disse Little Chandler.  --- Está na pior.

--- Mas o Hogan tem um bom emprego, não tem?

--- Tem; trabalha na Secretaria do Interior.

--- Encontrei"-o em Londres uma noite e ele me pareceu muito bem de vida\ldots{}
Coitado do O’Hara!  Bebida, não é?

--- Além de outras coisas --- disse Little Chandler sem querer esticar o
assunto.

Ignatius Gallaher riu.

--- Tommy --- ele disse ---, estou vendo que você não mudou absolutamente nada.
Continua o mesmo sujeito sério que me fazia sermão nas manhãs de domingo quando
me doía a cabeça e a minha língua parecia uma lixa.  Você precisa dar umas
voltas pelo mundo.  Nunca foi a lugar nenhum, nunca viajou?

--- Já estive em Isle of Man --- disse Little Chandler.

Ignatius Gallaher riu.

--- Isle of Man! --- ele disse.  --- Vá a Londres ou Paris: Paris seria ainda
melhor.  Uma viagem assim lhe faria bem.

--- Você conhece Paris?

--- Mas é claro!  Já estive muitas vezes por lá.

--- E é mesmo tão bonita quanto dizem? --- perguntou Little Chandler.

Little Chandler tomou um pequeno gole do seu drinque enquanto Ignatius Gallaher
esvaziou o próprio copo de uma só vez.

--- Bonita? --- disse Ignatius Gallaher, detendo"-se na palavra e no sabor da
bebida.  --- Não se trata disso, sabe.  É claro, é bonita\ldots{}  Mas é a vida
parisiense; isso é o que vale.  Ah, não existe cidade como Paris em termos de
animação, diversão, agitação\ldots{}

Little Chandler terminou de beber seu uísque e, após várias tentativas,
conseguiu atrair a atenção do \textit{barman}.  Pediu mais uma dose.

--- Estive no Moulin Rouge --- Ignatius Gallaher prosseguiu depois que o
\textit{barman} recolheu os copos --- e visitei todos os cafés boêmios.
Ambiente pesado!  Não é lugar pra um rapaz carola como você, Tommy.

Little Chandler ficou calado até o \textit{barman} retornar com as bebidas:
então bateu delicadamente seu copo no do amigo e retribuiu"-lhe o brinde.
Começava a sentir certa decepção.  O sotaque e o modo de falar de Gallaher
desagradavam"-no.  Havia no amigo algo de vulgar que ele anteriormente nunca
notara.  Mas talvez fosse apenas uma consequência da vida que levava em Londres
em meio ao tumulto e à competição do ambiente jornalístico.  O velho charme
ainda era visível por trás daqueles novos modos espalhafatosos.  E, afinal,
Gallaher estava de fato mais vivido, tinha viajado mundo afora.  Little
Chandler olhava o amigo com inveja.

--- Tudo em Paris é alegria --- disse Ignatius Gallaher.  --- Lá as pessoas
aproveitam a vida\ldots{} e você não acha que estão certas?  Se quiser
divertir"-se pra valer, vá a Paris.  E, fique sabendo, elas têm um carinho todo
especial pelos irlandeses.  Quando descobriram que eu era irlandês só faltaram
me carregar no colo, meu camarada\ldots{}

Little Chandler deu quatro ou cinco goles.

--- Me conta uma coisa --- ele disse ---, é verdade que Paris é tão\ldots{}
depravada quanto dizem?

Ignatius Gallaher fez um gesto católico com o braço direito.

--- Todo lugar é depravado --- ele disse.  --- É claro que em Paris a gente
encontra muita safadeza.  E só ir a algum baile promovido pelos estudantes, por
exemplo.  A gente se diverte um bocado, sabe, quando as \textit{cocottes}
começam a se soltar.  Imagino que você saiba o que elas são, não é?

--- Já ouvi falar --- disse Little Chandler.

Ignatius Gallaher esvaziou o copo e sacudiu a cabeça.

--- Ah! --- exclamou.  --- Digam o que quiserem.  Não existe mulher como a
parisiense: em termos de classe, de pique\ldots{}

--- Então é mesmo uma cidade depravada --- insistiu Little Chandler,
timidamente.  --- Quero dizer, quando comparada a Londres ou Dublin?

--- Londres! --- disse Ignatius Gallaher.  --- Tanto faz seis ou meia dúzia.
Pergunte ao Hogan, meu jovem.  Eu mostrei um pouco de Londres a ele quando
apareceu por lá.  Ele vai abrir os seus olhos\ldots{}  Tommy, não deixe esse
uísque virar refresco; bebe logo.

--- Não, calma\ldots{}

--- Ah, vamos lá, só mais uma dose não vai fazer mal nenhum.  Vai ser o quê? A
mesma coisa, certo?

--- Certo.  Tudo bem.

--- François, mais uma rodada\ldots{} Quer fumar, Tommy?

Ignatius Gallaher tirou a cigarreira do bolso do paletó.  Os dois amigos
acenderam charutos e deram baforadas em silêncio aguardando que os drinques
fossem servidos.

--- Vou lhe dizer o que eu acho --- disse Ignatius Gallaher, emergindo das
nuvens de fumaça nas quais se refugiara.  --- Este mundo está muito louco.
Falar em depravação!  Eu já ouvi cada caso\ldots{} que é isso, o que estou
dizendo?\ldots{} eu já constatei cada caso de\ldots{} depravação\ldots{}

Pensativo, Ignatius Gallaher deu umas baforadas no charuto e em seguida, com um
tom de voz calmo, qual um historiador, passou a descrever ao amigo alguns
quadros da imoralidade que campeava no exterior.  Fez um resumo dos vícios de
diversas capitais e declarou"-se inclinado a conceder o primeiro prêmio a
Berlim.  Alguns fatos ele não podia garantir (os amigos lhe contaram), mas
quanto a outros tivera experiência própria.  Não deixou nada nem ninguém de
fora.  Revelou segredos de instituições religiosas do Continente e discorreu
sobre algumas das práticas em voga na alta sociedade: e concluiu contando, com
detalhes, uma história a respeito de uma duquesa inglesa, história essa por ele
tida como verídica.  Little Chandler ficou atônito.

--- Ah, pois é --- disse Ignatius Gallaher ---, aqui estamos nesta Dublin velha
de guerra onde não se ouve falar dessas coisas.

--- Como isso aqui deve ser maçante pra você --- disse Little Chandler ---
depois de conhecer tantos lugares!

--- Você sabe --- disse Ignatius Gallaher ---, pra mim, estar aqui é um
descanso.  E, afinal, é a terrinha, como a gente diz, não é?  A gente não
consegue deixar de sentir alguma coisa por ela.  É da natureza humana\ldots{}
Mas conta pra mim alguma coisa sobre você.  O Hogan me falou que você agora
desfruta\ldots{} das alegrias do matrimônio.  Faz dois anos, não é?

Little Chandler enrubesceu e sorriu.

--- É --- ele disse.  --- Em maio fez doze meses que me casei.

--- Espero que não seja tarde demais pra lhe desejar felicidades --- disse
Ignatius Gallaher.  --- Se eu soubesse seu endereço teria expressado meus votos
na ocasião.

Estendeu a mão e Little Chandler aceitou o cumprimento.

--- Tommy --- ele disse ---, desejo a você e aos seus toda a alegria do mundo,
velho amigo, e rios de dinheiro, e que você fique pra semente.  E são os votos
de um amigo sincero, um velho amigo.  Você sabe disso, não é?

--- Sei --- disse Little Chandler.

--- Tem filhos? --- perguntou Ignatius Gallaher.

Little Chandler voltou a enrubescer.

--- Temos um --- ele disse.

--- Menino ou menina?

--- Um garoto.

Ignatius Gallaher deu um tapa sonoro nas costas do amigo.

--- Parabéns --- ele disse ---, sabia que você não faria feio, Tommy.

Little Chandler sorriu, olhou para o copo um tanto aturdido e mordeu o lábio
inferior com três dentes brancos, infantis.

--- Gostaria que você desse uma chegada lá em casa uma noite dessas --- ele
disse --- antes de voltar pra Londres.  Minha esposa vai gostar muito de
conhecê"-lo.  A gente pode ouvir um pouco de música e\ldots{}

--- Muito obrigado, meu velho amigo --- disse Ignatius Gallaher ---, foi pena a
gente não ter se encontrado antes.  É que eu preciso ir embora amanhã à noite.

--- Quem sabe ainda esta noite\ldots{}?

--- Sinto muito, amigo.  Eu estou aqui com um conhecido meu, um sujeito
esperto, e nós combinamos de ir a um carteado.  Se não fosse isso\ldots{}

--- Ah, nesse caso\ldots{}

--- Mas quem sabe? --- disse Ignatius Gallaher respeitosamente.  --- Ano que
vem talvez eu dê um pulo aqui, agora que já quebrei o gelo.  Fica pra próxima.

--- Muito bem --- disse Little Chandler ---, a próxima vez que você vier a
gente vai se reunir.  Combinado?

--- Sim, combinado --- disse Ignatius Gallaher.  --- Ano que vem, se eu vier,
\textit{parole d’honneur}.

--- E pra selar o trato --- disse Little Chandler --- vamos beber mais um.

Ignatius Gallaher tirou do bolso um grande relógio de ouro.

--- Vai ser o último? --- perguntou.  --- Porque, você sabe, eu tenho um
compromisso.

--- Ah, sim, com certeza --- disse Little Chandler.

--- Então vamos --- disse Ignatius Gallaher ---, vamos tomar mais um, um
\textit{deoc an doruis\ldots{}} é assim que se diz pequena dose de uísque na
língua vernácula, creio eu.

Little Chandler fez o pedido.  O rubor que lhe subira à face momentos antes
agora ali se fixara permanentemente.  Qualquer coisa o fazia corar: e agora
sentia"-se encalorado e agitado.  Três pequenas doses de uísque tinham"-lhe
subido à cabeça e o charuto forte de Gallaher o deixara tonto, visto que tinha
constituição física delicada e raramente bebia.  A aventura de encontrar
Gallaher depois de oito anos, de estar ao lado de Gallaher no Corless’s cercado
de luzes e barulho, de escutar as histórias de Gallaher e de compartilhar ainda
que por pouco tempo da vida nômade e triunfal de Gallaher, havia alterado o
equilíbrio de sua natureza sensível.  Sentia de maneira marcante o contraste
entre sua vida e a vida do amigo e isso lhe parecia injusto.  Gallaher era
inferior a ele tanto em termos de berço quanto em questão de educação.  Tinha
certeza de que seria capaz de superar o amigo, de fazer algo melhor do que tudo
o que o amigo fizera ou viria a fazer, algo mais grandioso do que um jornalismo
de segunda categoria, se apenas tivesse a oportunidade.  O que o impedia?
Aquela timidez infeliz!  Queria vingar"-se de algum modo, afirmar sua
virilidade.  Percebera a motivação da recusa de Gallaher ao seu convite.  A
amabilidade de Gallaher era apenas uma atitude condescendente para com ele
assim como a visita à Irlanda era um ato de condescendência para com o país.

O \textit{barman} trouxe"-lhes as bebidas.  Little Chandler empurrou um copo em
direção ao amigo e pegou o outro avidamente.

--- Quem sabe? --- ele disse, enquanto erguiam os copos.  --- Quando você vier
no ano que vem eu não terei o prazer de desejar felicidades a Mr.~e Mrs.~Ignatius Gallaher.

Ignatius Gallaher ao mesmo tempo em que bebia piscou o olho por cima da borda
do copo com expressividade.  Quando acabou de beber estalou os beiços com ar de
decisão, pôs o copo de volta sobre a bancada e disse:

--- Nem sonhe, meu jovem.  Vou me divertir bastante e correr o mundo antes de
me amarrar; se é que um dia vou me amarrar.

--- Vai, um dia você vai se amarrar --- disse Little Chandler com serenidade.

Ignatius Gallaher posicionou a gravata alaranjada e os olhos azul acinzentados
bem diante do amigo.

--- Você acha? --- perguntou.

--- Você vai se amarrar --- repetiu Little Chandler com firmeza.  --- Como todo
mundo se encontrar a garota certa.

Tinha falado com um pouco mais de ênfase do que o necessário e percebeu que
tinha se entregado; contudo, embora o rubor houvesse se espalhado pela sua
face, ele não desviou os olhos do olhar do amigo.  Ignatius Gallaher observou"-o
durante alguns instantes e então disse:

--- Se um dia isso acontecer, você pode ter certeza de que não vou perder tempo
com romantismo.  Vou dar o golpe do baú.  Se a mulher não tiver uma bela conta
bancária não vai servir pra mim.

Little Chandler sacudiu a cabeça.

--- Está duvidando, homem de Deus? --- disse Ignatius Gallaher com veemência.
--- Sabe lá o que é isso?  Basta eu abrir a boca e amanhã mesmo posso ter
mulher e dinheiro.  Você não acredita?  Eu que o diga.  Há centenas\ldots{} o
que estou dizendo?\ldots{} milhares de alemãs e judias ricas, podres de ricas,
que me agradeceriam\ldots{} Você vai ver só, meu jovem.  Vou jogar as cartas
certas.  Quando me meto numa coisa é pra valer.  Você que me aguarde.

Levou o copo à boca com um gesto brusco, esvaziou o conteúdo e deu uma
gargalhada.  Então desviou o olhar pensativamente e disse num tom mais calmo:

--- Mas não estou com pressa.  Elas podem esperar.  Não consigo me imaginar
preso a uma mulher, entende?

Imitou com a boca um gesto de quem prova comida e fez uma cara esquisita.

--- Acaba ficando um pouco rançoso, acho eu --- ele disse.

\smallskip

\noindent\dotfill

\smallskip

Little Chandler sentou"-se numa saleta ao lado do \textit{hall}, tendo ao colo
uma criança.  Por economia não tinham empregada, mas a irmã caçula de Annie,
Monica, vinha todas as manhãs e à noite por uma hora para ajudar.  Mas Monica
fora embora havia muito tempo.  Eram quinze para as nove.  Little Chandler
chegara em casa tarde e, além disso, esquecera"-se de trazer do Bewley’s o
pacote de café que prometera a Annie.  É claro que ela estava de mau humor e
lhe respondia monossilabicamente.  Disse que não queria nada mas quando se
aproximou a hora em que a loja da esquina fecharia ela decidiu sair para
comprar duzentos gramas de chá e um quilo de açúcar.  Colocou a criança
adormecida no colo do marido e disse:

--- Tome.  Não o acorde.

Sobre a mesa havia uma pequena luminária com copa de porcelana branca cuja luz
caía sobre uma foto de um porta"-retratos com moldura de osso.  A foto era de
Annie.  Little Chandler olhou para a foto, fixando o olhar nos lábios finos e
apertados.  Ela usava a blusa azul"-claro de tecido leve que ele lhe trouxera de
presente num sábado qualquer.  Havia custado dez \textit{shillings} e onze
centavos, além de muita agonia e nervosismo! Como havia sofrido naquele dia,
esperando próximo à porta da loja até que o recinto estivesse vazio, ficando
ali em frente ao balcão, tentando aparentar tranquilidade enquanto a vendedora
empilhava blusas femininas diante dele, pagando e esquecendo"-se de pegar o
troco, sendo chamado de volta pela moça da caixa e, finalmente, examinando o
pacote para ver se estava bem embrulhado, no momento em que saía da loja, na
tentativa de disfarçar o rubor que lhe tomava o rosto.  Quando ele chegou em
casa com a blusa Annie beijou"-o e disse que a peça era muito bonita e que
estava na moda; mas ao tomar conhecimento do preço atirou a blusa sobre a mesa
e disse que era um verdadeiro roubo cobrar dez \textit{shillings} e onze
centavos por aquilo.  A princípio ela queria devolver a blusa mas quando a
experimentou ficou maravilhada, especialmente com o feitio das mangas, e
deu"-lhe um beijo e disse que era muita gentileza dele se lembrar dela.

--- Hum!\ldots{}

Olhou friamente dentro dos olhos da fotografia e estes reagiram também
friamente.  Decerto eram belos e o rosto era igualmente belo.  Mas para ele
havia naquele rosto certa perversidade.  Por que era tão indiferente e altivo?
A expressão calma dos olhos o irritava.  Os olhos repeliam"-no e desafiavam"-no:
neles não havia paixão, nem enlevo.  Pensou no que Gallaher dissera acerca de
judias ricas.  Ah, como são cheios de paixão, de volúpia, pensava ele, aqueles
olhos escuros orientais!\ldots{} Por que se casara com os olhos daquela
fotografia?

A pergunta assustou"-o e ele interrompeu o devaneio e correu um olhar nervoso
pela saleta.  Havia algo de medíocre na bela mobília por ele comprada a
crédito.  A própria Annie havia escolhido os móveis e estes faziam"-no lembrar
da esposa.  Eram convencionais e bonitinhos.  Um sombrio ressentimento contra
sua própria vida cresceu dentro dele.  Ele não poderia escapar daquela casinha?
Seria tarde demais para tentar a vida audaciosa de Gallaher?  Poderia ir para
Londres?  Ainda não havia terminado de pagar a mobília.  Se conseguisse
escrever um livro que fosse publicado, quem sabe as portas se abririam para
ele.

Tinha a sua frente sobre a mesa um livro de poemas de Byron.  Abriu"-o com a mão
esquerda com todo cuidado para não acordar a criança e começou a ler o primeiro
poema ali impresso:

\begin{verse}\itshape
Calados estão os ventos e quieta a sombra da noite,\\
Nenhum zéfiro sequer caminha pelo pomar,\\
No momento em que volto ao túmulo de Margaret\\
E espalho flores sobre o pó que tanto amo.
\end{verse}

Interrompeu a leitura.  Sentia o ritmo dos versos a sua volta na sala.  Como
eram melancólicos!  Seria capaz de escrever daquela maneira, de expressar em
versos a melancolia que trazia na alma?  Quanta coisa gostaria de descrever: a
sensação que tivera algumas horas antes em Grattan Bridge, por exemplo.  Se
conseguisse resgatar aquele estado de espírito\ldots{}

A criança acordou e começou a chorar.  Ele deixou o livro de lado e tentou
fazer com que se calasse: mas a criança não parava de chorar.  Pôs"-se a
embalá"-la nos braços mas os berros eram cada vez mais agudos.  Embalou"-a com
mais velocidade enquanto tentava ler a segunda estrofe:

\begin{verse}\itshape
No interior desta cela estreita jaz seu corpo,\\
O corpo que certa vez\ldots{}
\end{verse}

Era inútil.  Não conseguia ler.  Não conseguia fazer nada.  Os berros da
criança arrebentavam"-lhe os tímpanos.  Era inútil, inútil!  Estava numa prisão
perpétua.  Seus braços tremiam de raiva e subitamente curvando"-se sobre o rosto
da criança gritou:

--- Pare!

A criança calou"-se por uma fração de segundo, num espasmo de medo, e recomeçou
a berrar.  Ele deu um salto da cadeira e pôs"-se a atravessar a sala a passos
largos com a criança no colo.  O pequeno soluçava violentamente, perdendo o
fôlego por quatro ou cinco segundos e então voltando a urrar.  Os berros
ecoavam nas paredes finas da sala.  Ele tentou acalmá"-la mas a criança soluçava
convulsivamente.  Olhou para o rosto contraído e trêmulo do pequeno e começou a
ficar assustado.  Contou sete soluços sem uma única pausa e abraçou
aterrorizado a criança.  Se o pequeno morresse!\ldots{}

A porta se abriu com um estrondo e uma jovem entrou correndo, ofegante.

--- O que aconteceu? O que aconteceu? --- ela gritou.

A criança, ouvindo a voz da mãe, parecia que ia estourar de tanto chorar.

--- Não é nada, Annie\ldots{} não é nada\ldots{} Ele começou a chorar\ldots{}

Ela atirou no chão o pacote com as compras e arrebatou"-lhe a criança dos
braços.

--- O que você fez com ele? --- gritou, fitando"-o com os olhos arregalados.

Little Chandler encarou o olhar da mulher por um momento e sentiu o coração
gelar ao constatar o ódio ali estampado.  Gaguejou:

--- Não foi nada\ldots{} Ele\ldots{} ele começou a chorar\ldots{} Não
consegui\ldots{} Não fiz nada\ldots{} O quê?

Sem dar"-lhe a menor atenção ela começou a andar de um lado ao outro da sala,
agarrada à criança e murmurando:

--- Meu homenzinho!  Meu menino!  Está com medo, querido?\ldots{} Pronto,
querido!  Pronto, pronto!\ldots{} Gatinho da mamãe!\ldots{} Pronto, calma!

Little Chandler sentiu o rosto arder de vergonha e afastou"-se da luz do abajur.
Ouviu os soluços da criança cada vez menos frequentes: e lágrimas de remorso
irromperam"-lhe nos olhos.


\chapter{Duplicatas}
\hedramarkboth{Duplicatas}{James Joyce}

\textsc{A campainha tocou} furiosamente e, quando Miss Parker chegou ao canal
de voz, ouviu"-se um brado furioso com um lancinante sotaque da Irlanda do
Norte:

--- Mande o Farrington vir aqui!

Miss Parker voltou para sua máquina, dizendo a um homem que estava sentado à
mesa escrevendo:

--- Mr.~Alleyne quer falar com você lá em cima.

O homem resmungou \textit{Que vá pro diabo!} por entre os dentes e empurrou a
cadeira para trás a fim de se levantar.  Quando se levantou via"-se que era alto
e corpulento.  Tinha o rosto flácido, cor de vinho tinto, com sobrancelhas e
bigode alourados: os olhos saltavam um pouco das órbitas e estavam remelentos.
Levantou o tampo do balcão e, passando pelos clientes, deixou o escritório com
passos arrastados.

Arrastou"-se escada acima até alcançar o segundo patamar, onde uma porta exibia
uma placa de metal com a inscrição: \textit{Mr.~Alleyne}.  Ali deteve"-se,
bufando de cansaço e consternação, e bateu à porta.  A voz estridente gritou:

--- Entre!

O homem entrou na sala de Mr.~Alleyne.  No mesmo instante Mr.~Alleyne, homem de
baixa estatura usando óculos de aro de ouro numa cara bem"-escanhoada, ergueu a
cabeça por detrás de uma pilha de documentos.  A cabeça era tão rosada e calva
que parecia um grande ovo esquecido em cima da papelada.  Mr.~Alleyne não
perdeu um segundo:

--- Farrington?  O que está acontecendo?  Por que é que sempre tenho de
reclamar de você?  Pode me dizer por que ainda não preparou a cópia do contrato
firmado entre Bodley e Kirwan?  Já disse que tem de estar pronto até as quatro
horas.

--- Mas o Mr.~Shelley disse, senhor\ldots{}

--- \textit{Mr.~Shelley disse, senhor}\ldots{}  Por obséquio siga as minhas
instruções e não o que \textit{Mr.~Shelley disse, senhor}.  Você sempre arruma
desculpa pra se livrar de trabalho.  Saiba que se a cópia desse contrato não
estiver pronta até o final da tarde levarei o caso ao Mr.~Crosbie\ldots{}
Entendeu bem?

--- Sim, senhor.

--- Entendeu bem?\ldots{} É, e mais uma coisinha!  Falar com você é como falar
com as paredes.  Entenda de uma vez por todas que você tem apenas meia hora
para o almoço e não uma hora e meia.  Quantos pratos você come?  Eu até
gostaria de saber\ldots{} Estamos entendidos?

--- Sim, senhor.

Mr.~Alleyne voltou a curvar a cabeça sobre a pilha de papéis.  O homem fitou o
crânio lustroso que dirigia os negócios da Crosbie \& Alleyne, avaliando sua
fragilidade.  Um espasmo de ódio apertou"-lhe a garganta por um momento e então
passou, deixando uma aguda sensação de sede.  O homem reconheceu a sensação e
pensou que naquela noite precisaria de uns bons tragos.  Era a segunda quinzena
do mês e, se conseguisse concluir a cópia a tempo, talvez Mr.~Alleyne lhe
autorizasse um vale.  Permaneceu ali, com o olhar fixo naquela cabeça acima da
pilha de papéis.  Subitamente, Mr.~Alleyne começou a remexer os papéis,
procurando algo.  Então, como se não tivesse percebido a presença do outro até
aquele momento, voltou a erguer a cabeça, dizendo:

--- Ei!  Vai ficar aí o dia todo?  Francamente, Farrington, você é mesmo
folgado!

--- Eu estava esperando pra ver\ldots{}

--- Muito bem, não precisa esperar pra ver nada.  Pode descer e voltar pro seu
trabalho.

O homem arrastou"-se até a porta e, ao sair da sala, ouviu Mr.~Alleyne gritar
que se o contrato não estivesse copiado até o final da tarde, o assunto seria
levado ao conhecimento de Mr.~Crosbie.

Ele retornou a sua mesa no andar inferior e contou as folhas que faltavam ser
copiadas.  Pegou a pena e mergulhou"-a na tinta mas continuava a olhar com ar
estúpido para as últimas palavras que havia escrito: \textit{Em hipótese
alguma poderá o citado Bernard Bodley ser}\ldots{}  A tarde chegava ao
seu final e dentro de poucos minutos seriam acesas as lamparinas a gás: então
ele poderia escrever.  Sentia necessidade de molhar a garganta seca.  Pôs"-se de
pé e, levantando mais uma vez o tampo do balcão, saiu do escritório.  Ao vê"-lo
passar, o chefe do escritório olhou"-o com um ar interrogativo.

--- Não se preocupe, Mr.~Shelley --- disse o homem, apontando com o dedo o
local para onde se dirigia.

O chefe do escritório olhou para o porta"-chapéus e, vendo que ali não faltava
nenhum deles, não fez qualquer comentário.  Assim que chegou ao patamar o homem
tirou do bolso um boné de lã xadrez, colocou"-o na cabeça e desceu correndo a
escada estreita.  Ao sair caminhou furtivamente junto às paredes das casas até
a esquina e desapareceu por uma porta.  Agora estava a salvo no aconchego e na
penumbra do estabelecimento dirigido por O’Neill, e, ocupando com a cara
avermelhada, cor de vinho ou de carne escura, todo o espaço da janelinha que
dava para o bar, chamou:

--- Ei, Pat, me dá aí uma cerveja, meu velho.

O \textit{barman} trouxe"-lhe um copo de cerveja preta.  O homem bebeu a cerveja
de um só gole e pediu uma bala de menta.  Colocou sobre o balcão uma moeda, e
enquanto o \textit{barman} tateava na penumbra para encontrá"-la, retirou"-se do
aconchego do local tão furtivamente quanto havia entrado.

A escuridão, acompanhada de uma neblina espessa, vencia o crepúsculo daquele
fevereiro e os lampiões em Eustace Street tinham sido acesos.  O homem caminhou
rente às paredes até chegar à porta do escritório, perguntando a si mesmo se
conseguiria terminar a cópia a tempo.  Na escada um perfume forte e úmido
saudou"-lhe as narinas: evidentemente Miss Delacour chegara enquanto ele estava
no O’Neill.  Enfiou o boné de volta no bolso e entrou no escritório, assumindo
um ar distraído.

--- Mr.~Alleyne está procurando por você --- disse o chefe do escritório com um
tom severo.  --- Onde você se meteu?

O homem olhou de relance para os dois clientes diante do balcão insinuando que
a presença dos mesmos o impedia de responder.  Como se tratava de dois homens,
o chefe deu uma gargalhada.

--- Conheço esse truque --- ele disse.  --- Cinco vezes por dia é um
pouco\ldots{} Olhe, é melhor você se apurar e levar pro Mr.~Alleyne as cópias
da nossa correspondência sobre o caso Delacour.

Tais palavras na presença de clientes, a corrida escada acima e a cerveja
tomada num só gole desnortearam o homem e, ao sentar"-se a sua mesa para juntar
os documentos solicitados, ele constatou que era inútil tentar terminar a cópia
do contrato antes das cinco e meia.  A noite escura e úmida aproximava"-se e ele
desejava passá"-la nos bares, bebendo com os amigos em meio ao brilho das
luminárias a gás e ao tilintar dos copos.  Reuniu a correspondência do caso
Delacour e saiu do escritório.  Torcia para que Mr.~Alleyne não desse pela
falta das duas últimas cartas.

O perfume forte e úmido recendia pela escada até a sala de Mr.~Alleyne.  Miss
Delacour era uma mulher de meia"-idade com aparência de judia.  Dizia"-se que Mr.~Alleyne 
gostava tanto da pessoa dela quanto de seu dinheiro.  Ela vinha com
frequência ao escritório e suas visitas eram demoradas.  Estava sentada ao lado
da escrivaninha dele, envolta numa nuvem de perfume, afagando o cabo da
sombrinha e agitando a longa pena preta espetada no chapéu.  Mr.~Alleyne fizera
girar a cadeira para olhá"-la de frente e apoiava o pé direito lepidamente sobre
o joelho esquerdo.  O homem depositou a correspondência sobre a escrivaninha e
inclinou"-se em sinal de respeito mas nem Mr.~Alleyne nem Miss Delacour notaram
a reverência.  Mr.~Alleyne bateu com o dedo sobre a pasta da correspondência e
em seguida fez"-lhe um sinal, querendo dizer: \textit{Pronto; agora pode
ir}.

O homem retornou ao escritório no andar inferior e sentou"-se novamente a sua
mesa.  Pregou os olhos na frase incompleta: \textit{Em hipótese alguma poderá o
citado Bernard Bodley ser}\ldots{} e achou estranho que nome e sobrenome
iniciassem com a mesma letra.  O chefe do escritório apressava Miss Parker,
dizendo que ela jamais aprontaria as cartas a tempo de seguirem por correio no
mesmo dia.  O homem ficou escutando durante alguns minutos as batidas das
teclas e então retomou o trabalho a fim de concluir a cópia.  Mas sua cabeça
estava confusa e sua mente divagava em direção ao brilho e à algazarra dos
bares.  A noite estava perfeita para bebidas quentes.  Continuou a copiar com
muito esforço, mas quando o relógio bateu cinco horas ainda faltavam catorze
páginas.  Diabo!  Não conseguiria terminar a tempo.  Tinha vontade de xingar em
voz alta, de dar um murro em alguma coisa.  Estava tão enfurecido que escreveu
\textit{Bernard Bernard} em lugar de \textit{Bernard Bodley} e teve de
refazer a folha.

Sentia"-se capaz de destruir sozinho todo o escritório.  Seu corpo ansiava por
ação, queria correr para a rua e perpetrar violência.  Todas as afrontas que
sofrera na vida enfureciam"-no\ldots{}  Será que poderia pedir um vale
confidencialmente ao caixa?  Não, o caixa não resolveria nada, não resolveria
droga nenhuma: não daria vale nenhum\ldots{} Sabia onde encontrar os
companheiros: Leonard e O’Halloran e Nosey Flynn.  O barômetro de sua natureza
emocional indicava tempestade.

Estava tão entregue a devaneios que foi preciso que chamassem seu nome duas
vezes antes que respondesse.  Mr.~Alleyne e Miss Delacour estavam parados do
outro lado do balcão e todos os funcionários tinham se voltado na expectativa
de algum incidente.  O homem levantou"-se.  Mr.~Alleyne começou a esbravejar,
afirmando que faltavam duas cartas.  O homem respondeu que nada sabia a
respeito das cartas, que fizera cópias fiéis.  A espinafração continuou: era
tão mordaz e violenta que o homem mal podia controlar o impulso de descer o
punho fechado sobre a cabeça do boneco ali parado diante dele.

--- Não sei nada a respeito dessas duas cartas --- disse com cara de idiota.

--- \textit{Não"-sabe"-nada}.  É claro que não sabe nada --- disse Mr.~Alleyne.
--- Diga"-me --- acrescentou, buscando com os olhos a aprovação da mulher que
estava ao seu lado ---, você acha que sou imbecil?  Você acha que sou algum
grande imbecil?

O homem olhou para o rosto da mulher, para a cabecinha em formato de ovo e de
volta para a mulher; e, sem que ele próprio percebesse, sua língua encontrou
uma saída feliz:

--- Não creio, meu senhor, que caiba a mim resolver essa questão.

Os colegas de trabalho prenderam a respiração.  Todos ficaram atônitos (o autor
da sutileza não menos que os outros) e Miss Delacour, pessoa gorda e bonachona,
exibiu um grande sorriso.  Mr.~Alleyne ficou vermelho como pimentão e sua boca
estremeceu numa expressão de cólera.  Brandiu o punho cerrado próximo ao rosto
do homem até o braço vibrar como a haste de uma máquina elétrica:

--- Seu insolente!  Seu insolente!  Eu acabo com você!  Você vai ver!  Peça
desculpas por sua insolência ou considere"-se na rua!  Vai pra rua, estou lhe
dizendo, se não me pedir desculpas!

\smallskip

\noindent\dotfill

\smallskip

Esperou embaixo da soleira de uma porta em frente ao escritório, do outro lado
da rua, para ver se o caixa sairia sozinho.  Todos os funcionários saíram e
finalmente o caixa apareceu acompanhado do chefe do escritório.  Era inútil
tentar abordar o assunto quando ele estava acompanhado do chefe do escritório.
O homem percebeu que estava mesmo numa situação difícil.  Fora obrigado a
humilhar"-se e pedir desculpas a Mr.~Alleyne pela insolência e sabia que
doravante o escritório seria para ele uma verdadeira casa de marimbondo.
Lembrava"-se da maneira como Mr.~Alleyne perseguira o pequeno Peake, levando"-o a
se demitir e assim abrir uma vaga para o sobrinho.  Sentia"-se violento e
sedento e vingativo, irritado consigo mesmo e com todos a sua volta.  
Mr.~Alleyne não lhe daria um minuto de sossego; sua vida seria um inferno.  Dessa
vez ele tinha de fato agido como um tolo.  Será que não era capaz de controlar
a língua?  Mas nunca tinham se entendido, ele e Mr.~Alleyne, desde o dia em que
este o surpreendera imitando seu sotaque de nativo da Irlanda do Norte para
divertir Higgins e Miss Parker: naquele instante tudo começara.  Poderia ter
pedido dinheiro a Higgins, mas Higgins nunca tinha um tostão.  Um sujeito que
tem de manter duas casas, obviamente não podia\ldots{}

Sentiu novamente o corpo volumoso ansiar pelo consolo de um bar.  A neblina
começava a fazer com que sentisse frio e ele perguntou a si mesmo se não
poderia pedir um trocado ao Pat, no O’Neill’s.  Não conseguiria arrancar dele
mais do que um \textit{shilling} --- e um \textit{shilling} de nada
adiantaria.  Contudo, precisava arrumar dinheiro onde quer que fosse: gastara
seu último centavo naquela cerveja e dentro de pouco tempo seria tarde demais.
De repente, enquanto manuseava a corrente do relógio, lembrou"-se da casa de
penhores de Terry Kelly, em Fleet Street.  Que grande ideia!  Por que não
pensara nisso antes?

Desceu rapidamente pelo beco de Temple Bar, resmungando que todos podiam ir
para o inferno pois ele teria uma bela noitada.  O empregado de Terry Kelly
disse \textit{uma coroa!} mas o depositário ofereceu apenas seis
\textit{shillings}; ficou mesmo por seis \textit{shillings}.  Ele saiu radiante
da casa de penhores, formando com as moedas um pequeno cilindro preso entre o
polegar e os outros dedos da mão.  Em Westmoreland Street as calçadas estavam
repletas de rapazes e moças que voltavam do trabalho e moleques maltrapilhos
corriam por toda parte gritando as manchetes das edições vespertinas.  O homem
seguia no meio da multidão, observando o espetáculo com certa satisfação e
lançando olhares altivos às secretárias.  Em sua cabeça ecoavam as sinetas dos
bondes e o barulho dos ônibus e em suas narinas já recendia o aroma do álcool.
Enquanto avançava pensava nos termos em que relataria o incidente aos amigos:

--- Então, eu olhei pra ele\ldots{} friamente, vocês sabem como é, e olhei pra
ela.  Então voltei a olhar pra ele\ldots{} com toda calma, vocês sabem.
\textit{Não creio que caiba a mim resolver essa questão}, eu disse.

Nosey Flynn estava sentado no mesmo canto de sempre no Davy Bryne’s e, quando
ouviu a história, pagou para Farrington uma meia dose, afirmando que a resposta
era uma das mais astutas que já ouvira.  Farrington, por sua vez, retribuiu o
drinque.  Logo depois O’Halloran e Paddy Leonard chegaram e a história foi
repetida.  O’Halloran pagou uma rodada de cerveja preta quente e contou a
história da resposta que dera ao chefe de escritório quando trabalhava na firma
Callan’s, em Fownes’s Street; mas, como a resposta seguia a tradição das
tiradas ingênuas dos pastores nas éclogas, ele foi obrigado a admitir que a
resposta não era tão astuta quanto a de Farrington.  Diante disso Farrington
convidou a turma a esvaziar seus copos e a se preparar para nova rodada.

No momento em que pediam o veneno quem haveria de chegar se não Higgins!
Evidentemente, vinha juntar"-se aos companheiros.  A turma pediu"-lhe que desse a
sua versão da história, e ele o fez com muita vivacidade, visto que cinco meias
doses de uísque sobre o balcão constituíam forte estimulante.  Rolaram de rir
quando ele imitou Mr.~Alleyne brandindo o punho cerrado bem próximo à cara de
Farrington.  Então imitou Farrington, dizendo, \textit{E assim foi, meus
ilustres senhores, com toda calma do mundo}, enquanto Farrington olhava os
companheiros com seus olhos caídos e remelentos, sorrindo e de vez em quando
recolhendo do bigode com o lábio inferior gotas perdidas da bebida.

Quando a rodada terminou houve uma pausa.  O’Halloran tinha dinheiro mas os
outros dois pareciam desprovidos; de modo que saíram todos do bar um tanto
desconcertados.  Na esquina de Duke Street, Higgins e Nosey Flynn seguiram à
esquerda enquanto os outros três continuaram em direção ao centro.  Uma chuva
fina caía nas ruas frias e, quando chegaram ao Ballast Office, Farrington
sugeriu que fossem ao Scotch House.  O bar estava repleto de homens e vibrava
com a algazarra provocada pelas línguas e pelos copos.  Os três companheiros
abriram caminho afastando da porta os lamurientos vendedores de fósforos e
posicionaram"-se numa extremidade do balcão.  Começaram a contar histórias.
Leonard apresentou"-os a um jovem chamado Weathers que estava se apresentando no
Tívoli como acrobata e palhaço.  Farrington pagou uma rodada.  Weathers disse
que aceitaria meia dose de uísque irlandês com água mineral.  Farrington, que
tinha boas noções de etiqueta, perguntou aos companheiros se também queriam
seus drinques com água; mas estes pediram a Tim que preparasse suas doses
puras.  O assunto mudou para teatro.  O’Halloran pagou uma rodada e Farrington
pagou outra; Weathers protestou, afirmando que a hospitalidade estava sendo só
da parte da Irlanda.  Prometeu levá"-los aos bastidores a apresentá"-los a belas
garotas.  O’Halloran disse que ele e Leonard iriam, mas que Farrington não
podia porque era um homem casado; e Farrington olhou de soslaio para os
companheiros com seus olhos caídos e remelentos, demonstrando que sabia que
estavam apenas brincando.  Weathers fez o grupo aceitar mais uma rodadinha,
paga por ele, e prometeu encontrá"-los mais tarde no Mulligan’s, em Poolberg
Street.

Quando o Scotch House fechou eles foram até o Mulligan’s.  Alojaram"-se numa
saleta no fundo do bar e O’Halloran pagou uma rodada de uísque puro, especial,
em meia dose.  Já estavam todos um tanto alegres.  Farrington estava prestes a
pagar outra rodada quando Weathers chegou.  Para alívio de Farrington, desta
vez o jovem pediu apenas um copo de cerveja.  O dinheiro estava acabando mas
ainda dispunham do suficiente para continuar bebendo.  Naquele momento duas
jovens usando uns chapelões, acompanhadas de um rapaz de terno xadrez, entraram
e sentaram"-se a uma mesa próxima à deles.  Weathers cumprimentou"-os e disse aos
companheiros que o trio acabara de sair do Tívoli.  Os olhos de Farrington a
todo momento desviavam"-se em direção a uma das garotas.  Havia na aparência da
jovem algo sensacional.  Trazia uma echarpe enorme de musselina azul"-pavão
envolta no chapéu e amarrada sob o queixo com um grande laço; e usava luvas de
um amarelo vivo, que iam até os cotovelos.  Farrington olhou embevecido para o
braço roliço que ela movia a todo instante e com muita graça; e quando, após
algum tempo, ela retribuiu"-lhe o olhar ele encantou"-se com seus grandes olhos
castanhos, cuja expressão firme ainda que enviesada o fascinava.  Ela olhou
para ele de relance uma ou duas vezes e, ao retirar"-se do recinto, roçou em sua
cadeira e disse \textit{Oh, pardon!} com um sotaque londrino.  Ele acompanhou"-a
com o olhar na expectativa de que ela se voltasse, mas decepcionou"-se.
Amaldiçoou a falta de dinheiro e amaldiçoou as rodadas que havia pago,
sobretudo os uísques com água mineral que pagara para Weathers.  Se havia uma
coisa que ele detestava era gente parasita.  Estava de tal modo irritado que
perdeu o fio da conversa dos amigos.

Quando Paddy Leonard o cutucou ele se deu conta de que conversavam sobre
demonstrações de força.  Weathers exibia o bíceps e vangloriava"-se tanto que os
outros dois resolveram convidar Farrington para defender a honra nacional.
Farrington arregaçou a manga da camisa e exibiu o bíceps.  Os dois braços foram
examinados e comparados e chegou"-se à conclusão de que seria necessária uma
prova de força.  Limparam a mesa e os dois homens nela apoiaram os cotovelos,
entrelaçando as mãos.  Quando Paddy Leonard dissesse \textit{Já!} cada um
deveria tentar forçar a mão do outro para baixo até tocar a mesa.  Farrington
mostrava"-se compenetrado e decidido.

A prova começou.  Ao cabo de uns trinta segundos Weathers conseguiu aos poucos
abaixar a mão do oponente até encostá"-la à mesa.  O rosto cor de vinho de
Farrington ficou ainda mais escuro de ódio e humilhação por ter sido derrotado
por um fedelho daqueles.

--- Não vale colocar o peso do corpo.  Jogue limpo --- ele disse.

--- Quem é que não jogou limpo? --- perguntou o outro.

--- Vamos de novo.  Vai ser a melhor de três.

A prova recomeçou.  As veias saltaram na testa de Farrington, e a palidez de
Weathers transformou"-se num tom rubro.  Mãos e braços tremiam com o esforço.
Após uma luta ferrenha Weathers conseguiu novamente, aos poucos, encostar a mão
do oponente no tampo da mesa.  Houve um ensaio de aplauso por parte dos
espectadores.  O \textit{barman}, de pé ao lado da mesa, fez com a cara rosada
um sinal em direção ao vencedor e disse com um tom grosseiro e abusado:

--- É, quem pode, pode!

--- E você lá entende disso? --- esbravejou Farrington, voltando"-se contra o
sujeito.  --- Pra que você tem de meter o bedelho?

--- Psst! Psst! --- disse O’Halloran, percebendo a violência estampada na cara
de Farrington.  --- Vamos pagar a conta, rapazes.  Tomemos o último drinque e
depois vamos embora.

\smallskip

\noindent\dotfill

\smallskip

O homem de pé na esquina de O’Connell Bridge aguardando o bonde de Sandymount
que o levaria para casa estava bastante mal"-humorado.  Transbordava com uma
raiva reprimida e um desejo de vingança.  Sentia"-se humilhado e insatisfeito;
já nem sentia o efeito do álcool; e tinha no bolso apenas dois \textit{pence}.
Amaldiçoava tudo.  Envolvera"-se numa encrenca no escritório, empenhara o
relógio, gastara todo o dinheiro que tinha; e nem sequer se embriagara.  Voltou
a sentir sede e desejou retornar ao bar quente e malcheiroso.  Perdera a fama
de fortão, deixando"-se derrotar duas vezes por um rapazola.  Seu coração
transbordava de ódio e, quando se lembrava da mulher com o chapelão que roçara
nele e dissera \textit{Oh, pardon!} o ódio quase o sufocava.

O bonde deixou"-o em Shelbourne Road e ele arrastou o corpanzil à sombra
projetada pelo muro do quartel.  Detestava voltar para casa.  Quando entrou
pela porta lateral encontrou a cozinha vazia e o fogo quase apagado.  Gritou ao
pé da escada:

--- Ada!  Ada!

Sua esposa era uma mulherzinha de cara comprida que oprimia o marido quando
este estava sóbrio e por ele era oprimida quando estava bêbado.  Tinham cinco
filhos.  Um garotinho desceu a escada correndo.

--- Quem está aí? --- perguntou o homem, espreitando na escuridão.

--- Eu, papai.

--- Eu quem?  Charlie?

--- Não, papai.  Tom.

--- Cadê a sua mãe?

--- Está lá na igreja.

--- Ah, é\ldots{} Ela lembrou de deixar jantar pra mim?

--- Lembrou, papai.  Eu\ldots{}

--- Acenda a lamparina.  Por que essa escuridão?  As outras crianças estão
dormindo?

O homem desmoronou sobre uma cadeira enquanto o menino acendia a lamparina.
Começou a imitar a voz desafinada do garoto, dizendo consigo mesmo: \textit{Lá
na igreja.  Lá na igreja, sim, senhor!} Quando a lamparina já estava acesa ele
deu um soco na mesa e gritou:

--- Cadê o meu jantar?

--- Já vou\ldots{} preparar, papai --- disse o garoto.

O homem ficou furioso, deu um salto da cadeira e apontou para o fogo.

--- Com este fogo!  Você deixou o fogo apagar!  Por Deus, eu vou te ensinar a
não fazer mais isso!

Deu um passo e pegou a bengala que estava atrás da porta.

--- Vou te ensinar a não deixar o fogo apagar! --- ele disse, arregaçando a
manga para deixar o braço livre.

O garoto gritou, \textit{Ah, pai!} e correu em volta da mesa choramingando, mas
o homem seguiu"-o e agarrou"-o pelo casaco.  O garoto olhou ao redor em desespero
e, sem ver um meio de escapar, caiu de joelhos.

--- Deixa o fogo apagar da próxima vez, deixa! --- disse o homem espancando o
menino vigorosamente com a bengala.  --- Tome isto, seu fedelho!

O menino deu um urro de dor quando a bengala o atingiu na coxa.  Pôs as mãos
para cima entrelaçando os dedos e sua voz tremia de pavor.

--- Ah, pai! --- ele gritou.  --- Não bate em mim, pai!  Eu\ldots{} eu rezo uma
Ave Maria pro senhor\ldots{} Eu rezo uma Ave Maria pro senhor pai, se o senhor
não bater em mim\ldots{} Eu rezo uma Ave Maria\ldots{}


\chapter{Barro}
\hedramarkboth{Barro}{James Joyce}

\textsc{A chefe tinha} permitido que ela saísse tão logo as mulheres
terminassem o lanche e Maria aguardava ansiosamente pela noite de folga.  A
cozinha estava estalando: a cozinheira dissera que os caldeirões de cobre
pareciam espelhos.  O fogo estava forte e brilhante e sobre uma das mesinhas
laterais havia quatro bolos de levedo dos grandes.  Os bolos davam a impressão
de estar inteiros; mas se a pessoa olhasse de perto veria que tinham sido
cortados em fatias longas e grossas, estando prontos para serem servidos
durante o lanche.  A própria Maria os havia cortado.

Maria era uma mulher de estatura baixa, bem baixa, mas tinha um nariz bastante
comprido e um queixo bastante comprido.  Sua voz era um pouco nasalada, e
falava sempre com amabilidade: \textit{Sim, querida} e \textit{Não, querida}.
Sempre era chamada a intervir quando as mulheres discutiam por causa das tinas
de lavar e sempre conseguia apaziguá"-las.  Um dia a chefe dissera"-lhe:

--- Maria, você é mesmo uma apaziguadora!

A subchefe e duas senhoras do Comitê tinham escutado o elogio.  E Ginger Mooney
vivia dizendo que não saberia o que fazer com a muda que cuidava dos ferros de
engomar se não fosse Maria.  Todos queriam muito bem a Maria.

As mulheres tomariam o chá às seis horas da tarde e ela poderia sair antes das
sete.  De Ballsbridge a Pillar, vinte minutos; de Pillar a Drumcondra, vinte
minutos; e vinte minutos para fazer as compras.  Chegaria lá antes das oito.
Pegou a bolsa com fecho de prata e releu as palavras ali gravadas:
\textit{Lembrança de Belfast}.  Gostava muito daquela bolsa porque havia sido
comprada por Joe há cinco anos quando ele e Alphy estiveram em Belfast durante
o feriado de Pentecostes.  A bolsa continha duas moedas de meia coroa e alguns
níqueis.  Sobrariam ainda cinco \textit{shillings} depois que pagasse a
passagem de bonde.  Como a noite seria agradável, com a criançada cantando!  Só
esperava que Joe não chegasse bêbado.  Ele ficava tão alterado quando bebia.

Muitas vezes convidara"-a para morar com eles; mas ela achava que atrapalharia
(embora a esposa de Joe fosse muito gentil com ela) e, além do mais, já se
acostumara à rotina da lavanderia.  Joe era um bom sujeito.  Ela cuidara dele e
de Alphy também; e Joe costumava dizer:

--- Mamãe é mamãe, mas minha mãe de verdade é a Maria.

Depois da desavença que houve em casa os rapazes conseguiram para ela aquele
emprego na lavanderia Dublin by Lamplight, e ela gostava do trabalho.  Sempre
tivera má impressão dos protestantes mas agora achava que eram pessoas boas, um
tanto caladas e sérias, mas mesmo assim pessoas muito boas de se conviver.
Além disso, tinha suas plantas na estufa e gostava de cuidar delas.  Possuía
lindas samambaias e cactos e, sempre que alguém vinha visitá"-la, ela dava à
pessoa um ou dois brotinhos tirados da estufa.  Só não gostava de uma coisa: os
folhetos religiosos protestantes pregados nas paredes; mas a chefe era uma
pessoa tão boa, tão fina.

Quando a cozinheira avisou que tudo estava pronto ela foi até o refeitório das
mulheres e tocou o sino.  Dentro de poucos minutos as mulheres começaram a
entrar, duas em duas, e três em três, enxugando nas saias as mãos fumegantes e
desenrolando as mangas das blusas para cobrir os braços vermelhos igualmente
fumegantes.  Sentaram"-se diante das canecas enormes que a cozinheira e a muda
enchiam com chá quente, já misturado com leite e açúcar, servido em grandes
jarras de metal.  Maria supervisionava a distribuição do bolo, certificando"-se
de que cada mulher receberia sua cota de quatro fatias.  Houve muito riso e
muita brincadeira durante o lanche.  Lizzie Fleming disse que a aliança
certamente seria encontrada por Maria e, embora Fleming já tivesse dito isso em
várias outras vésperas de Todos os Santos, Maria riu e disse que não queria
aliança nenhuma nem homem nenhum; e quando ela ria os olhos verde"-acinzentados
brilhavam com uma timidez tristonha e a ponta do nariz quase encostava na ponta
do queixo.  Então Ginger Mooney ergueu a caneca de chá e brindou à saúde de
Maria, enquanto as outras mulheres batiam com as canecas na mesa, e disse que
era uma pena não ter ali uma cerveja preta para misturar ao chá.  E Maria riu
novamente até que a ponta do nariz quase encostou na ponta do queixo e até que
seu corpo minúsculo quase se desconjuntasse pois sabia que Mooney não falava
por mal embora, é claro, falasse como uma mulher vulgar.

Mas como Maria ficou contente quando as mulheres terminaram o chá e a
cozinheira e a muda começaram a tirar a mesa!  Foi até seu pequeno quarto e,
lembrando"-se de que na manhã seguinte era dia de missa, mudou o ponteiro do
despertador de sete para seis horas.  Então tirou o avental e as botas que
usava em casa e estendeu sobre a cama sua melhor saia e colocou ao pé da cama
as minúsculas botas de sair.  Trocou de blusa também e, diante do espelho,
lembrou"-se da maneira como se vestia para ir à missa nas manhãs de domingo
quando era menina; e contemplou com certa satisfação aquele corpinho por ela
tantas vezes enfeitado.  Apesar dos anos, julgava seu corpinho ainda em forma.

Lá fora as ruas cintilavam com a chuva e ela deu graças por sua velha capa de
chuva marrom.  O bonde estava cheio e ela foi obrigada a sentar"-se no banquinho
que havia na parte posterior do veículo, ficando de frente para todos os
passageiros, com os pés mal alcançando o piso.  Esquematizou na mente tudo o
que faria e pensou como era bom ser independente e ter no bolso o próprio
dinheiro.  Esperava passar uma noite agradável.  Tinha certeza de que seria
agradável mas não podia deixar de lamentar que Alphy e Joe não estivessem se
falando.  Agora viviam se desentendendo mas quando meninos eram excelentes
amigos: mas a vida é assim mesmo.

Desceu do bonde em Pillar e abriu caminho com pressa em meio à multidão.
Entrou na confeitaria Downes’s mas ali havia tanta gente que teve de esperar um
bom tempo até ser atendida.  Comprou uma dúzia de doces sortidos, e finalmente
saiu da loja carregando um grande pacote.  Então pensou no que mais deveria
comprar: queria comprar algo bem gostoso.  Com certeza já teriam comprado maçãs
e nozes.  Era difícil decidir o que comprar e a única ideia que lhe vinha à
mente era bolo.  Resolveu comprar um pedaço de bolo de ameixa, mas a cobertura
de amêndoa do bolo de ameixa da Downes’s era sempre rala e ela foi até outra
confeitaria em Henry Street.  Ali escolheu com toda calma e a jovem elegante
que atendia, visivelmente aborrecida com a demora, perguntou"-lhe se era um bolo
de casamento que ela desejava adquirir.  A pergunta fez Maria enrubescer e
sorrir para a jovem; mas esta levava tudo muito a sério e finalmente cortou uma
larga fatia de bolo de ameixa, fez o embrulho e disse:

--- Dois \textit{shillings} e quatro centavos, por favor.

Chegou a pensar que teria de viajar de pé no bonde que ia para Drumcondra pois
nenhum rapaz parecia notá"-la mas um cavalheiro idoso cedeu"-lhe o lugar.  Era um
cavalheiro troncudo e usava chapéu marrom de aba dura; tinha uma cara quadrada
e vermelha e bigode grisalho.  Maria achou que era galante como um coronel e
muito mais educado do que os rapazes que ficavam simplesmente parados olhando
para frente.  O cavalheiro começou a conversar com ela a respeito do feriado de
Todos os Santos e do tempo chuvoso.  Disse que supunha que o pacote estivesse
cheio de guloseimas para os pequenos e acrescentou que os jovens tinham todo o
direito de se divertir.  Maria concordou e olhou para ele com simpatia,
emitindo monossílabos inaudíveis e balançando a cabeça discretamente.  Ele foi
muito gentil e, ao descer em Canal Bridge, ela agradeceu e fez uma leve
reverência com a cabeça, e ele retribuiu a reverência e tirou o chapéu,
sorrindo amavelmente; e enquanto subia a rua, inclinando a cabecinha para baixo
por causa da chuva, ela pensava como era fácil reconhecer um cavalheiro, mesmo
que o cavalheiro tivesse tomado umas e outras.

Todos disseram: \textit{A Maria chegou!} quando ela entrou na casa de Joe.  Joe
já havia chegado do trabalho, e a criançada trajava roupas domingueiras.  Duas
filhas da vizinha, já bastante crescidas, estavam presentes e havia muita
brincadeira.  Maria entregou o pacote de doces ao menino mais velho, Alphy,
para que fizesse a divisão e Mrs.~Donnelly disse que era muita bondade dela
trazer um pacote tão grande e fez com que as crianças dissessem:

--- Obrigado, Maria.

Mas Maria disse que havia trazido algo especial para o papai e para a mamãe,
uma coisa que eles certamente apreciariam, e procurou o pedaço de bolo de
ameixa.  Procurou na sacola que recebera na Downes’s e nos bolsos da capa de
chuva e no cabideiro do \textit{hall} mas nada encontrou.  Então perguntou às
crianças se alguma delas teria --- por engano, é claro --- comido o bolo
mas as crianças responderam que não e deram a entender que não queriam bolo
nenhum caso fossem acusadas de roubo.  Cada um tinha uma solução para o
mistério e Mrs.~Donnelly disse que era óbvio que Maria esquecera o bolo no
bonde.  Maria, lembrando"-se de como o cavalheiro de bigode grisalho a deixara
aturdida, enrubesceu de vergonha e desgosto e decepção.  Ao pensar no fracasso
de sua singela surpresa e nos dois \textit{shillings} e quatro centavos jogados
fora, por pouco não caiu no choro.

Mas Joe disse que não tinha importância e a trouxe para sentar"-se perto da
lareira.  Foi muito gentil com ela.  Contou"-lhe as últimas novidades do
escritório, bem como uma resposta ferina que dera ao gerente.  Maria não
entendeu por que Joe ria tanto de sua própria resposta mas disse que o gerente
deveria ser um sujeito bastante prepotente.  Joe disse que ele até que não era
tão difícil se a pessoa soubesse lidar com ele, que era um homem decente desde
que ninguém lhe pisasse os calos.  Mrs.~Donnelly tocou piano para as crianças
e estas dançaram e cantaram.  Então as duas meninas da vizinha serviram as
nozes.  Ninguém conseguiu encontrar o quebra"-nozes e Joe, que já estava ficando
irritado, perguntou como era que eles achavam que Maria comeria nozes sem um
quebra"-nozes.  Mas Maria disse que não gostava de nozes e pediu que não se
preocupassem com ela.  Então Joe perguntou"-lhe se aceitaria uma cerveja preta e
Mrs.~Donnelly disse que tinham em casa uma garrafa de vinho do Porto se ela
preferisse.  Maria disse que não desejava coisa alguma: mas Joe insistiu.

Maria então deixou que a vontade dele prevalecesse e sentaram"-se ao pé da
lareira, conversando sobre os velhos tempos e Maria achou por bem aproveitar a
oportunidade para interceder a favor de Alphy.  Joe porém exclamou que Deus
podia fulminá"-lo com um raio se ele voltasse a dirigir a palavra ao irmão e
Maria disse que lamentava ter tocado no assunto.  Mrs.~Donnelly disse ao marido
que era lamentável vê"-lo falar daquela maneira sobre alguém que tinha o mesmo
sangue dele mas Joe disse que Alphy não era irmão dele e o assunto quase causou
uma briga.  Mas Joe disse que não perderia a calma numa véspera de dia santo e
pediu à esposa que abrisse mais uma cerveja.  As duas filhas da vizinha
organizaram umas brincadeiras típicas da véspera do Dia de Todos os Santos e
logo o ambiente voltou a ficar alegre.  Maria sentia"-se radiante ao ver as
crianças tão felizes e Joe e a esposa de bom humor.  As meninas da vizinha
colocaram alguns pires sobre a mesa e posicionaram as crianças diante da mesa,
com os olhos vendados.  Uma pegou o breviário e as outras três pegaram a água;
e quando uma das meninas da vizinha apanhou a aliança Mrs.~Donnelly apontou"-lhe
o dedo, como se estivesse dizendo: \textit{Ah, eu conheço esse truque!}  A
menina enrubesceu.  Então insistiram em vendar os olhos de Maria e posicioná"-la
diante da mesa para ver que sorte tiraria; e, enquanto colocavam"-lhe a venda,
Maria ria tanto, tanto, que a ponta do nariz quase tocava a ponta do queixo.

Trouxeram"-na até a mesa em meio a risadas e piadas e ela estendeu a mão
para o ar conforme as orientações que recebia.  Ergueu e depois abaixou a mão e
tocou um dos pires.  Sentiu nas pontas dos dedos uma substância mole e úmida e
ficou surpresa porque ninguém dizia nada e porque não lhe removiam a venda.
Houve uma pausa de alguns segundos; e então um empurra"-empurra e uma série de
cochichos.  Alguém disse algo a respeito do jardim, e por fim Mrs.~Donnelly
falou asperamente com uma das meninas da vizinha e disse"-lhe que jogasse a
coisa fora imediatamente: aquilo não tinha a menor graça.  Maria entendeu que a
brincadeira não tinha dado certo e por isso teria de repeti"-la: e na segunda
tentativa conseguiu pegar o breviário.

Depois disso Mrs.~Donnelly tocou a canção folclórica \textit{Miss McCloud} para
as crianças e Joe obrigou Maria a aceitar uma taça de vinho.  Em pouco tempo
estavam todos novamente alegres e Mrs.~Donnelly disse que Maria entraria para o
convento antes do final do ano pois na brincadeira pegara o breviário.  Maria
nunca vira Joe tão gentil com ela como naquela noite, tão cheio de palavras
meigas e recordações.  Ela declarou que eram todos bons demais para com ela.

Finalmente as crianças ficaram cansadas e sonolentas e Joe pediu a Maria que
antes de ir embora cantasse uma daquelas canções antigas.  Mrs.~Donnelly
disse, \textit{Cante, Maria, por favor!}  E Maria teve de levantar"-se e se
colocar ao lado do piano.  Mrs.~Donnelly mandou as crianças ficarem quietas e
ouvirem a canção de Maria.  Então tocou a introdução e disse, \textit{Agora,
Maria!} e Maria, com a face tomada de rubor, começou a cantar com uma vozinha
pequena e trêmula.  Cantou \textit{Sonhei que morava}, e quando chegou à
segunda estrofe repetiu a primeira:

\begin{verse}\itshape
Entre muros de mármore, sonhei que morava,\\
Servos e vassalos de mim se acercavam\\
E que era eu a esperança e o orgulho\\
De todos que ali habitavam.\\
Minhas riquezas, não podia contar,\\
De um nome ilustre podia me gabar;\\
Mas também sonhei, e foi do que mais gostei,\\
Que tu sempre irias me amar.
\end{verse}

Mas ninguém se dispôs a mostrar"-lhe o engano, e quando ela terminou a canção
Joe estava bastante emocionado.  Afirmou que nada se comparava aos velhos
tempos e que música alguma se comparava à do velho Balfe, não importava o que
dissessem; e seus olhos ficaram tão cheios de lágrimas que ele não foi capaz de
encontrar o que procurava e viu"-se obrigado a perguntar à esposa onde estava o
saca"-rolhas.


\chapter{Um caso triste}
\hedramarkboth{Um caso triste}{James Joyce}

\textsc{Mr.~James Duffy} residia em Chapelizod porque queria viver o mais longe
possível de sua cidade e porque achava os outros subúrbios de Dublin ruins,
modernos e pretensiosos.  Morava numa casa velha e sombria e das janelas
avistava o alambique desativado ou mais adiante o rio de pouca profundidade em
cujas margens foi construída Dublin.  As paredes altas do seu quarto sem
forração de carpete eram desprovidas de quadros.  Ele próprio comprara cada
peça do mobiliário: uma cabeceira de cama de ferro pintada de preto, um
lavatório de ferro, quatro cadeiras de cana"-da"-índia, um cabideiro, um balde de
metal para guardar carvão, um aparador e atiçadores para a lareira e uma
escrivaninha quadrada, de tampo duplo.  Uma estante com prateleiras de madeira
branca tinha sido improvisada numa reentrância da parede.  O leito era arrumado
com lençóis brancos e uma colcha preta e vermelha cobria o pé da cama.  Um
pequeno espelho pendia acima do lavatório e durante o dia uma lamparina de copa
branca era o único enfeite sobre o console da lareira.  Os livros na estante de
madeira branca estavam organizados de baixo para cima de acordo com o tamanho.
Um volume com as obras completas de Wordsworth ficava no canto da prateleira
mais baixa e um exemplar do \textit{Maynooth Catechism}, inserido na capa de
tecido de um livro de anotações, ficava no canto oposto da prateleira mais
alta.  A escrivaninha estava sempre guarnecida de apetrechos para escrever.
Dentro da escrivaninha era guardado o manuscrito de uma tradução de
\textit{Michael Kramer}, de Hauptmann, com as marcações de cena escritas em
tinta vermelha, bem como um pequeno maço de papéis presos por um pegador de
bronze.  Nessas folhas, de vez em quando, era registrada uma frase e, num
momento de humor, a chamada de um anúncio de \textit{Bile Beans} tinha sido
colada na primeira página.  Quando se erguia o tampo da escrivaninha um aroma
suave exalava --- aroma de lápis de cedro ou de goma arábica ou de alguma
maçã um pouco passada ali esquecida.

Mr.~Duffy detestava qualquer coisa que indicasse desordem física ou mental.  Um
médico medieval o classificaria como saturnino.  Seu rosto, que encerrava a
história completa de seus anos, era marrom como as ruas de Dublin.  Em sua
cabeça comprida e avantajada cresciam cabelos negros ressecados e um bigode
castanho não chegava a encobrir"-lhe a boca pouco amistosa.  As maçãs do rosto
acentuavam seu aspecto severo; mas não havia severidade nos olhos que,
contemplando o mundo por baixo das sobrancelhas castanhas, davam a impressão de
ser ele um homem sempre pronto a descobrir nos outros qualidades redentoras
ainda que frequentemente se decepcionasse.  Vivia a certa distância do próprio
corpo, observando os próprios atos com olhares furtivos e duvidosos.  Tinha um
estranho hábito autobiográfico que o fazia, de vez em quando, redigir
mentalmente uma curta sentença sobre si mesmo contendo o sujeito na terceira
pessoa e o predicado no pretérito.  Jamais dava esmola a pedintes e caminhava
com passo firme, tendo à mão uma bengala de madeira de lei.

Havia muitos anos era caixa de um banco particular em Baggot Street.  Todas as
manhãs vinha de Chapelizod de bonde.  Ao meio"-dia almoçava no Dan Burke’s
--- uma garrafa de cerveja e meia porção de biscoitos de araruta.  Às quatro
da tarde ficava livre.  Jantava num pequeno restaurante em George’s Street onde
sentia"-se a salvo da juventude dourada de Dublin e onde o cardápio era simples
e honesto.  À noite costumava sentar"-se ao piano da senhoria ou caminhava pelos
arredores da cidade.  O gosto pela música de Mozart às vezes o levava à ópera
ou a um concerto: as únicas extravagâncias em sua vida.

Não tinha conhecidos nem amigos, nem igreja, nem credo.  Vivia espiritualmente
sem qualquer comunhão com terceiros, visitando parentes por ocasião do Natal e
os acompanhando ao cemitério quando morriam.  Incumbia"-se dessas duas
obrigações sociais por uma questão de dignidade mas não fazia qualquer outra
concessão às convenções que regem a vida civil.  Chegava a pensar que em
determinadas circunstâncias seria capaz de roubar o banco onde trabalhava, mas,
como tais circunstâncias nunca se apresentavam, sua vida se desenrolava
serenamente --- uma história sem aventuras.

Certa noite viu"-se sentado ao lado de duas mulheres numa sala de concertos.  A
sala, quase vazia e silenciosa, era um triste prenúncio de fracasso.  A senhora
que estava ao seu lado correu o olhar uma ou duas vezes pela sala deserta e
disse:

--- Que pena a casa estar tão vazia esta noite!  É tão difícil cantar para
cadeiras vazias!

Ele interpretou a observação como um convite à conversa.  Surpreendia"-o que ela
parecesse estar tão à vontade.  Enquanto conversavam ele se esforçou no sentido
de gravá"-la permanentemente na memória.  Quando foi informado de que a jovem
sentada ao lado dela era sua filha pensou que a mulher seria cerca de um ano
mais nova do que ele.  Seu rosto, que deve ter sido belo, conservava uma
expressão inteligente.  Era um rosto oval com traços bem marcados.  Os olhos
eram azul"-escuro e resolutos.  A princípio fitavam a pessoa com um ar
desafiador mas titubeavam com um movimento que parecia um desmaio da pupila
dentro da íris, indicando momentaneamente um temperamento de grande
sensibilidade.  A pupila logo se recompunha, o temperamento recém"-exposto
voltava a ser controlado pela prudência, e a jaqueta de astracã, modelando o
busto um tanto volumoso, voltava a acentuar o tom de desafio.

Ele reencontrou a mulher algumas semanas depois num concerto em Earlsfort
Terrace e aproveitou os momentos em que a filha não estava prestando atenção
para adquirir um pouco de intimidade.  Ela se referiu uma ou duas vezes ao
marido mas com um tom de voz que não denotava uma advertência.  Chamava"-se 
Mrs.~Sinico.  O tataravô do marido nascera em Livorno.  O marido era capitão da
marinha mercante e fazia a linha entre Dublin e a Holanda; tinham apenas uma
filha.

Ao se deparar com ela pela terceira vez tomou coragem e marcou um encontro.
Ela compareceu.  Foi o primeiro de uma série; encontravam"-se sempre à noite e
buscavam os locais mais sossegados para suas caminhadas.  Mr.~Duffy,
entretanto, não era dado a procedimentos escusos e, constatando que eram
obrigados a se encontrar às escondidas, forçou"-a a convidá"-lo a frequentar a
casa dela.  O Capitão Sinico incentivava as visitas dele, pois pensava que a
mão da filha estava em jogo.  Dispensara a esposa de sua galeria de prazeres
com tamanha presteza que sequer imaginava que alguém pudesse por ela se
interessar.  Com Mr.~Sinico sempre viajando e a filha fora da casa dando aulas
de música, Mr.~Duffy tinha muitas oportunidades para desfrutar da companhia da
mulher.  Nenhum dos dois tivera anteriormente uma aventura daquelas e não viam
em seu relacionamento qualquer incongruência.  Aos poucos os pensamentos dele
se entrelaçaram com os dela.  Ele lhe emprestava livros, oferecia"-lhe ideias,
compartilhava com ela seu intelecto.  Ela era toda ouvidos.

Às vezes em troca das teorias por ele expostas ela revelava algum fato de sua
vida particular.  Com uma atenção quase maternal instava"-o a extravasar sua
natureza: tornou"-se sua confessora.  Ele lhe contou que durante algum tempo
frequentara reuniões do Partido Socialista Irlandês onde se sentira uma figura
singular em meio a um grupo de operários circunspectos reunidos num sótão mal
iluminado por uma lamparina a óleo.  Quando o partido dividiu"-se em três
facções, cada qual com seu próprio líder e seu próprio sótão, ele deixou de
comparecer.  Os debates dos trabalhadores, segundo ele, eram por demais
acalorados; o interesse que apresentavam na questão dos salários era exagerado.
Ele os considerava realistas extremados que careciam de uma exatidão que era
fruto de um ócio ao qual jamais teriam acesso.  Nenhuma revolução social, ele
afirmava, abalaria Dublin nos próximos séculos.

Ela lhe perguntou por que não colocava suas ideias no papel.  Para quê?  ele
perguntou, com um desdém estudado.  Para competir com frasistas, homens
incapazes de raciocinar em termos de causa e efeito durante sessenta segundos?
Para submeter"-se à crítica de uma classe média obtusa que confia sua moralidade
à polícia e as belas"-artes aos donos de galerias?

Ele visitava com frequência um bangalô de propriedade dela nos arredores de
Dublin; ali costumavam ficar a sós à noite.  Aos poucos, à medida que seus
pensamentos se entrelaçavam, passaram a falar de assuntos menos genéricos.  A
companhia dela era como solo tépido para uma planta exótica.  Muitas vezes ela
deixava que a noite caísse sobre os dois, não acendendo a lamparina.  A sala
discreta, na penumbra, o isolamento dos dois e a música que ainda vibrava em
seus ouvidos os unia.  Essa união o enaltecia, polia as arestas de seu
temperamento, trazia emoção à sua vida mental.  Às vezes surpreendia"-se ouvindo
o som de sua própria voz.  Ele achava que aos olhos dela ascenderia a uma
dimensão angelical; e, à medida que se apegava à natureza ardente da amiga,
ouvia uma voz estranha e impessoal, que reconhecia como sendo dele próprio,
insistindo a respeito da incurável solidão da alma.  Não podemos nos dar, dizia
a voz: pertencemos a nós mesmos.  As conversas terminaram quando certa noite,
demonstrando um ardor fora do comum, Mrs.~Sinico tomou"-lhe a mão
apaixonadamente e apertou"-a contra o rosto.

Mr.~Duffy ficou muito surpreso.  A interpretação que a mulher dera às suas
palavras decepcionou"-o.  Não a visitou durante uma semana; então escreveu"-lhe
convidando"-a para um encontro.  Como ele não queria que essa última conversa
fosse perturbada por confissões encontraram"-se numa pequena confeitaria próxima
ao portão do parque.  Era um dia frio de outono mas apesar do frio vaguearam
pelas trilhas do parque durante quase três horas.  Concordaram em romper o
relacionamento: todo e qualquer vínculo, ele dissera, é um vínculo com o
sofrimento.  Quando saíram do parque caminharam em silêncio em direção ao
bonde; mas ali ela começou a tremer tão violentamente que, com receio de que
ela tivesse mais uma crise, ele se despediu às pressas e deixou"-a.  Alguns dias
mais tarde ele recebeu um pacote contendo seus livros e partituras.

Quatro anos se passaram.  Mr.~Duffy voltara à vida pacata de sempre.  Seu
quarto continuava sendo testemunho da ordem que reinava em sua mente.  Algumas
partituras novas tinham sido acrescentadas ao atril na sala do andar inferior e
na estante havia agora duas obras de Nietzsche: \textit{Assim falou
Zaratustra} e \textit{A gaia ciência}.  Raramente escrevia nas folhas
de papel que guardava na escrivaninha.  Uma frase, escrita dois meses após a
última conversa com Mrs.~Sinico, dizia: “Amor entre homem e homem é impossível
porque não deve haver relacionamento sexual, e amizade entre homem e mulher é
impossível porque pressupõe relacionamento sexual”.  Manteve"-se longe dos
recitais com receio de encontrá"-la.  Perdeu o pai; o sócio minoritário do banco
aposentou"-se; e ele continuava indo de bonde ao centro da cidade toda manhã e
no final da tarde voltava para casa a pé depois de jantar moderadamente em
George’s Street e ler o jornal vespertino como sobremesa.

Certa noite, quando estava prestes a enfiar na boca uma garfada de carne cozida
e repolho sua mão se deteve.  Fixou os olhos numa coluna da edição
vespertina apoiada na garrafa de água.  Devolveu a garfada ao prato e leu a
coluna atentamente.  Então bebeu um copo de água, empurrou o prato para o lado,
dobrou o jornal sobre a mesa entre os cotovelos e releu a coluna inúmeras
vezes.  O repolho começou a soltar no prato uma gordura fria e esbranquiçada.
A garçonete veio perguntar"-lhe se a comida não estava bem cozida.  Ele disse
que estava a contento e deu mais umas garfadas, com dificuldade.  Então pagou a
conta e saiu.

Caminhou rapidamente em meio ao crepúsculo de novembro, batendo com a bengala
na calçada num ritmo marcado, e com a ponta amarelada do jornal aparecendo no
bolso lateral do sobretudo.  Na viela sossegada que vai do portão do parque até
Chapelizod ele desacelerou o passo.  A bengala batia na calçada com pancadas
menos enfáticas e sua respiração ofegante, quase emitindo um silvo suspirante,
condensava"-se no ar frio de inverno.  Quando chegou em casa subiu direto para o
quarto e, retirando o jornal do bolso, releu a coluna sob a luz fraca que
entrava pela janela.  Leu em silêncio, mas movendo os lábios como um padre ao
ler as orações do breviário.  Eis a coluna:


\begin{quote}
{\centering \textsc{mulher morre na estação de sydney}\\
\textsc{um caso triste}\par}

\medskip

\textit{Hoje no Hospital Municipal de Dublin o legista de plantão (na
ausência do Dr.~Leverett) fez a autópsia do corpo de Mrs.~Emily
Sinico, de quarenta e três anos de idade, morta ontem à noite
na Estação de Sydney.  A autópsia revelou que a mulher, ao tentar
atravessar a via férrea, foi atropelada pela locomotiva que
partiu de Kingstown às 22h, sofrendo traumatismos na
cabeça e ferimentos no lado direito do corpo que lhe causaram a morte.}

\textit{James Lennon, condutor da locomotiva, declarou que trabalha na
estrada de ferro há quinze anos.  Ao escutar o apito tocado pelo
funcionário da estação, deu partida no trem e alguns
instantes depois freou bruscamente ao ouvir uma gritaria.  O
trem seguia devagar.}

\textit{P.~Dunne, carregador de bagagem, declarou que quando o trem
estava prestes a partir, viu uma mulher tentando atravessar os
trilhos.  Correu em direção a ela e gritou, mas, antes de alcançá"-la,
ela fora apanhada pelo para"-choque da locomotiva e atirada ao chão.}

\textit{Membro do júri: “O senhor viu a mulher cair?”}

\textit{Testemunha: “Vi”.}

\textit{Em seu depoimento o Sargento Croly, da polícia civil, declarou que ao
chegar à estação encontrou a mulher estirada na plataforma,
aparentemente morta.  Mandou o corpo ser transportado para o saguão
de embarque para aguardar a chegada da ambulância.}

\textit{O policial nº57 confirmou o depoimento.}

\textit{Dr.~Halpin, cirurgião"-assistente do Hospital Municipal de Dublin,
constatou que a mulher tivera duas costelas fraturadas e sofrera
sérias contusões no ombro direito.  O lado direito da cabeça
também fora atingido na queda.  No entanto, os ferimentos não eram
suficientemente graves para causar a morte de uma pessoa.  Na
opinião do médico, a morte tinha sido causada por parada cardíaca.}

\textit{Mr.~H.B.~Patterson Finlay, em nome da estrada de ferro,
expressou grande pesar pelo acidente.  A companhia sempre tomou todas
as precauções possíveis para evitar que pessoas atravessassem os
trilhos fora das passarelas, colocando avisos em cada estação e
utilizando portões com mola, de modelo patenteado, nas
passagens de nível.  A vítima tinha o hábito de atravessar os
trilhos tarde da noite ao deslocar"-se de uma plataforma à outra e,
considerando outras circunstâncias do caso, ele não acreditava
que a companhia tivesse qualquer responsabilidade.}

\textit{Capitão Sinico, de Leovilhe, subúrbio da estação de Sydney e
marido da vítima, também prestou depoimento.  Declarou que a vítima era sua
esposa.  Não se encontrava em Dublin na hora do acidente pois
chegara de Roterdã naquela manhã.  Estavam casados havia vinte
e dois anos e viveram felizes até cerca de dois anos atrás, quando a
esposa começou a apresentar um comportamento um tanto
desregrado.}

\textit{Miss Mary Sinico disse que ultimamente a mãe começara a sair à
noite para comprar bebida alcoólica.  Ela, a depoente, havia inúmeras
vezes tentado aconselhá"-la e a induzira a filiar"-se à liga
contra o alcoolismo.  Só chegou em casa uma hora depois de o acidente
ter ocorrido.}

\textit{O júri pronunciou a sentença de acordo com o laudo médico e
isentou Lennon de toda e qualquer responsabilidade.}

\textit{O legista de plantão declarou que se tratava de um caso
bastante triste, e apresentou suas condolências ao Capitão Sinico e à
filha.  Exortou a estrada de ferro a tomar medidas enérgicas
para evitar que acidentes semelhantes àquele voltassem a ocorrer.
Ninguém foi responsabilizado.}
\end{quote}

\smallskip

\noindent\dotfill

\smallskip

Mr.~Duffy ergueu os olhos do jornal e olhou pela janela contemplando a
melancólica paisagem noturna.  O rio corria quieto ao longo do alambique
desativado e de vez em quando uma luz aparecia numa casa em Lucan Road.  Que
fim!  A narrativa da morte da mulher deixou"-o revoltado e deixou"-o igualmente
revoltado o fato de haver revelado a ela coisas que para ele eram sagradas.  As
frases gastas, as expressões vazias de condolência, as palavras cuidadosamente
escolhidas por um repórter induzido a esconder os detalhes de uma morte banal e
vulgar revolviam"-lhe o estômago.  Ela não tinha apenas degradado a si mesma;
ela o degradara também.  Ele era capaz de visualizar a trilha esquálida de seu
vício desprezível e malcheiroso.  Sua alma gêmea!  Pensou nas infelizes que
encontrara trôpegas pela rua, carregando canecas e garrafas para serem enchidas
por um \textit{barman}.  Deus do céu, que fim!  Evidentemente, não estava
preparada para a vida, não tinha a mínima força de vontade, presa fácil dos
vícios, uma dessas ruínas sobre as quais tem sido edificada a civilização.
Como pôde ela descer tanto!  Será possível que ele tinha se enganado tanto a
seu respeito?  Lembrou"-se do rompante que ela tivera naquela noite e
interpretou a reação da mulher de um modo mais implacável do que nunca.  Não
tinha agora a menor dificuldade em aprovar as medidas por ele tomadas.

Quando a luz se extinguiu e suas lembranças começavam a se dissipar, ele teve a
impressão de que a mão dela tocara a sua.  O impacto que a princípio
afetara"-lhe o estômago agora lhe afetava os nervos.  Vestiu às pressas o
sobretudo e o chapéu e saiu porta afora.  O ar frio esperava"-o na soleira e
penetrou"-lhe pelas mangas do casaco.  Quando chegou ao bar em Chapelizod Bridge
entrou e pediu uma bebida forte.

O proprietário serviu"-o obsequiosamente mas não puxou assunto.  No bar havia
cinco ou seis operários conversando sobre o valor das terras de um senhor em
County Kildare.  Bebericavam em seus canecões de cerveja e fumavam, cuspindo no
assoalho e de vez em quando raspando com a ponta das botinas um pouco de pó de
serragem para encobrir as cusparadas.  Mr.~Duffy sentou"-se numa banqueta e
ficou olhando em direção ao grupo, sem enxergá"-los nem ouvi"-los.  Logo depois
eles foram embora e ele pediu mais uma bebida.  Levou um bom tempo para beber
essa dose.  O bar estava bastante quieto.  O proprietário esparramava"-se sobre
o balcão lendo o \textit{Herald} e bocejando.  De quando em vez ouvia"-se um
bonde sacolejando na rua deserta.

Sentado ali, revivendo os momentos que passara ao lado dela e evocando
alternadamente as duas imagens em que então a concebia, deu"-se conta de que ela
estava morta, tinha deixado de existir, tinha se transformado numa lembrança.
Começou a se sentir angustiado.  Perguntou a si mesmo se não poderia ter feito
algo mais.  Não poderia ter proposto a ela uma farsa; não poderia ter vivido
com ela ostensivamente.  Tinha agido da forma que considerava mais correta.
Que culpa poderia ter?  Agora que ela se fora ele compreendeu a solidão em que
ela vivia, sozinha naquela sala noite após noite.  A vida dele seria igualmente
marcada pela solidão, até que ele, também, morresse, fosse transformado numa
lembrança --- se é que alguém dele se lembraria.

Já passava das nove quando saiu do bar.  A noite estava fria e soturna.  Entrou
no parque pelo portão principal e caminhou por baixo das árvores esquálidas.
Percorreu as trilhas desertas por onde juntos tinham caminhado quatro anos
antes.  Parecia que ela caminhava ao seu lado na escuridão.  Às vezes tinha a
impressão de sentir a voz dela roçar"-lhe o ouvido, de sentir a mão dela roçar a
sua.  Deteve"-se para escutar melhor.  Por que lhe negara a vida?  Por que a
condenara à morte?  Sentia"-se moralmente despedaçado.

Ao chegar ao ponto mais elevado de Magazine Hill ele parou e olhou rio abaixo
em direção a Dublin, cujas luzes cintilavam vermelhas e hospitaleiras na noite
fria.  Olhou pela encosta da colina e, ao pé da ladeira, à sombra do muro do
parque, viu figuras humanas deitadas.  Aquelas cenas de amor furtivas e venais
levaram"-no ao desespero.  Desprezou a retidão de sua própria vida; sentiu que
tinha ficado fora da festa da vida.  Um ser humano talvez o tivesse amado e ele
negara"-lhe vida e felicidade: condenara essa pessoa à ignomínia, a uma morte
vergonhosa.  Sabia que as criaturas prostradas à sombra do muro olhavam para
ele e desejavam que fosse embora.  Ninguém o queria; estava fora da festa da
vida.  Desviou o olhar para o rio cinzento e reluzente, que se retorcia em
direção a Dublin.  Do outro lado do rio, avistou um trem de carga retorcendo"-se
ao sair da Estação de Kingsbridge, como um verme com a cabeça flamejante
retorcendo"-se na escuridão, obstinado, prosseguindo a duras penas.  O trem saiu
lentamente de seu campo de visão; mas ele ainda ouvia o ruído penoso da
locomotiva repetindo as sílabas do nome dela.

Retornou pelo mesmo caminho que viera, com o ritmo da locomotiva martelando"-lhe
os ouvidos.  Começou a duvidar do que lhe dizia a memória.  Parou embaixo de
uma árvore e esperou que o ritmo cessasse.  Já não sentia a presença dela na
escuridão, nem a voz dela roçando"-lhe o ouvido.  Permaneceu ali alguns minutos,
escutando.  Não ouvia nada: na noite reinava um silêncio total.  Continuou
escutando: silêncio total.  Sentiu"-se sozinho.


\chapter{Dia de hera na sede do comitê}
\hedramarkboth{Dia de hera na sede do comitê}{James Joyce}

\textsc{O velho Jack} juntou as cinzas com um pedaço de papelão e espalhou"-as
com todo cuidado sobre o monte de carvão queimado.  Quando o carvão ficou
ligeiramente encoberto a escuridão abateu"-se sobre seu rosto mas, assim que ele
voltou a abanar o fogo, sua sombra arqueada escalou a parede oposta e seu rosto
lentamente ressurgiu na luz.  Era o rosto de um velho, ossudo e hirsuto.  Os
olhos úmidos e azuis piscavam diante do fogo e a boca úmida se abria de quando
em vez, como se mascasse algo mecanicamente ao se fechar.  Quando o carvão
estava em brasa ele encostou o pedaço de papelão na parede, suspirou e disse:

--- Está melhor agora, Mr.~O’Connor.

Mr.~O’Connor, jovem de cabelos grisalhos, cujo rosto era desfigurado por
manchas e espinhas, acabara de enrolar um cigarro, num cilindro perfeito, mas
ao perceber que alguém lhe dirigia a palavra desfez pensativamente o cilindro.
Então, ainda pensativo, recomeçou a enrolar o cigarro e após alguns instantes
de reflexão decidiu selar o papel com saliva.

--- Mr.~Tierney disse a que horas estaria de volta? --- perguntou com uma voz
rouca e esganiçada.

--- Não disse, não.

Mr.~O’Connor prendeu o cigarro nos lábios e começou a apalpar os bolsos.
Retirou um maço de cartões de cartolina.

--- Vou apanhar o fósforo pro senhor --- disse o velho.

--- Não precisa, isto aqui serve --- disse Mr.~O’Connor.

Escolheu um dos cartões e leu:

\clearpage

\begin{quote}\centering
\textsc{eleições municipais}\smallskip

\textit{Distrito da Bolsa de Valores}\\
\textit{Mr.~Richard J.~Tierney respeitosamente solicita o favor do
seu voto e do seu apoio nas próximas eleições para o Distrito da
Bolsa de Valores.}
\end{quote}

Mr.~O’Connor tinha sido contratado por um partidário de Mr.~Tierney com o
propósito de angariar votos em um dos setores do distrito mas, como o clima
estava inclemente e suas botas não eram impermeáveis, ele passara a maior parte
do dia sentado diante do fogo na sede do comitê em Wicklow Street, na companhia
de Jack, o velho zelador.  Os dois estavam sentados na mesma posição desde que
aquele curto dia de inverno anoitecera.  Era seis de outubro; lá fora tudo era
frio e desolação.

Mr.~O’Connor rasgou um pedaço do cartão e queimou"-o para acender o cigarro.
Enquanto ele assim procedia a chama iluminou uma folha de hera, escura e
lustrosa, presa à lapela do seu paletó.  O velho observava"-o atentamente e,
pegando outra vez o pedaço de papelão, recomeçou a abanar o fogo lentamente
enquanto o companheiro fumava.

--- É mesmo --- disse ele, retomando a conversa ---, é difícil saber criar um
filho.  Quem diria que ele acabaria nisso!  Mandei ele estudar no Christian
Brothers e fiz o que pude por ele, e agora anda por aí enchendo a cara.  Eu bem
que tentei colocar ele na linha.

Voltou a guardar o pedaço de papelão com um gesto cansado.

--- Se eu não fosse agora um velho, eu endireitava ele.  Eu descia a bengala
nas costas dele e arrebentava ele\ldots{} como cansei de fazer.  A mãe, o
senhor sabe, fica mimando ele com isso e aquilo\ldots{}

--- É isso que estraga os filhos --- disse Mr.~O’Connor.

--- É mesmo --- disse o velho.  --- E em troca a gente recebe pouca
gratidão\ldots{} é só desaforo.  Toda vez que vê que tomei um trago ele me
desacata.  Onde é que esse mundo vai parar, com os filhos tratando os pais
assim?

--- Quantos anos ele tem? --- perguntou Mr.~O’Connor.

--- Dezenove --- disse o velho.

--- Por que você não manda ele trabalhar?

--- Já mandei!  Não parei de sustentar a bebida dele desde que deixou a escola?
\textit{Não vou te sustentar}, eu disse pra ele.  \textit{Trate de arrumar
emprego}.  Mas, é claro, é pior quando ele arruma emprego: gasta tudo o que
ganha em bebida.

Mr.~O’Connor sacudiu a cabeça dando a entender que compreendia a situação, e o
velho se calou, olhando para o fogo.  Alguém abriu a porta da sala e gritou:

--- Ei, é aqui a reunião da maçonaria?

--- Quem está aí? --- disse o velho.

--- O que vocês estão fazendo no escuro? --- retrucou a voz.

--- É você, Hynes? --- perguntou Mr.~O’Connor.

--- Eu mesmo.  O que vocês estão fazendo no escuro? --- disse Mr.~Hynes,
aproximando"-se da luz da lareira.

Era um rapaz alto e magro com bigode castanho"-claro.  Gotas de chuva
pendiam"-lhe da aba do chapéu e a gola do casaco estava levantada.

--- Então, Mat --- disse ele dirigindo"-se a Mr.~O’Connor ---, como vão as
coisas?

Mr.~O’Connor sacudiu a cabeça.  O velho afastou"-se da lareira e, após andar aos
tropeços pela sala retornou com duas velas, as quais enfiou no fogo, uma de
cada vez, e depois levou até a mesa.  Surgiu uma sala desnuda e o fogo perdeu
sua coloração alegre.  As paredes estavam nuas, à exceção de um cartaz contendo
um discurso de eleição.  No meio da sala havia uma mesinha com pilhas de
papéis.

Mr.~Hynes apoiou"-se no console da lareira e disse:

--- Ele já pagou vocês?

--- Ainda não --- disse Mr.~O’Connor.  --- Queira Deus que ele não nos deixe na
mão logo nesta noite.

Mr.~Hynes riu.

--- Ah, ele vai pagar.  Não precisa ter medo --- ele disse.

--- Espero que pague logo, se é que está agindo com seriedade --- disse Mr.~O’Connor.

--- E você, Jack, o que acha? --- disse Mr.~Hynes, dirigindo"-se ao velho em tom
irônico.

O velho retornou a seu lugar próximo ao fogo, dizendo:

--- Dinheiro, ele tem.  Não é como aquele lambão.

--- Que lambão? --- disse Mr.~Hynes.

--- Colgan --- disse o velho com desprezo.

--- Só porque Colgan é operário você diz isso?  Qual é a diferença entre um
pedreiro honesto e eficiente e dono de bar, hein?  Por acaso um operário não
tem o mesmo direito de entrar pro Conselho Municipal que qualquer outra
pessoa\ldots{} é, tem até mais direito que aqueles vira"-casacas que estão
sempre de chapéu na mão na frente de qualquer aristocrata!  Não é verdade, Mat?
--- disse Mr.~Hynes, dirigindo"-se a Mr.~O’Connor.

--- Acho que você está certo --- disse Mr.~O’Connor.

--- Um deles é um homem simples e honesto que não foge da raia.  Vai
representar a classe operária.  Esse sujeito que contratou vocês só está
querendo um emprego.

--- É claro que a classe operária deve ser representada --- disse o velho.

--- O operário --- disse Mr.~Hynes --- só leva pontapé e não ganha nada.  Mas é
o trabalhador que produz tudo.  O operário não está atrás de grandes empregos
para os filhos, os sobrinhos e os primos.  O operário não vai jogar na lama a
honra de Dublin para agradar a um monarca alemão.

--- Como assim? --- disse o velho.

--- Você não sabe que eles pretendem fazer um discurso de boas"-vindas ao rei
Eduardo, se ele vier à Irlanda no ano que vem?  Por que temos de nos prostrar
diante de um rei estrangeiro?

--- Nosso candidato não vai votar a favor do discurso --- disse Mr.~O’Connor.
--- Vai se candidatar pela facção nacionalista.

--- Não vai? --- disse Mr.~Hynes.  --- Esperem só pra ver se ele vai ou não
vai.  Eu conheço ele.  Não é o chamado Tierney o Trapaceiro?

--- Por Deus!  Talvez você tenha razão, Joe --- disse Mr.~O’Connor.  --- Em
todo caso, espero que ele apareça com o dinheiro.

Os três homens calaram"-se.  O velho começou a juntar mais cinzas.  Mr.~Hynes
tirou e sacudiu o chapéu e desvirou a gola do casaco, revelando, com o gesto,
uma folha de hera na lapela.

--- Se este aqui estivesse vivo --- ele disse, apontando para a hera na lapela
---, discurso de boas"-vindas nem entrava em discussão.

--- Isso é verdade --- disse Mr.~O’Connor.

--- É mesmo.  Deus estava com a gente naquela época! --- disse o velho.  ---
Naquele tempo a coisa era animada.

A sala ficou novamente em silêncio.  Então um homem agitado e de baixa
estatura, fungando e com as orelhas geladas, empurrou a porta.  Aproximou"-se
rapidamente do fogo, esfregando as mãos como se pretendesse com elas produzir
uma fagulha.

--- Nada de dinheiro, pessoal --- ele disse.

--- Pode sentar aqui, Mr.~Henchy --- disse o velho, oferecendo"-lhe a cadeira.

--- Ah, não se levante, Jack, não se levante --- disse Mr.~Henchy.

Cumprimentou Mr.~Hynes secamente com um meneio de cabeça e sentou"-se na cadeira
que o velho desocupara.

--- Você cobriu a Aungier Street? --- ele perguntou a Mr.~O’Connor.

--- Cobri, sim --- disse Mr.~O’Connor, apalpando os bolsos à procura do bloco
de notas.

--- Foi à casa do Grimes?

--- Fui.

--- Então?  Qual é a posição dele?

--- Não prometeu nada.  Disse: \textit{Não vou revelar meu voto a
ninguém}.  Mas acho que podemos contar com ele.

--- Por quê?

--- Perguntou o nome dos candidatos, e eu disse.  Mencionei o nome do padre
Burke.  Acho que podemos contar com ele.

Mr.~Henchy começou a fungar e a esfregar violentamente as mãos sobre o fogo.
Então disse:

--- Pelo amor de Deus, Jack, traga um pouco de carvão.  Deve ter sobrado algum.

O velho saiu da sala.

--- Nada feito --- disse Mr.~Henchy, sacudindo a cabeça.  --- Perguntei àquele
engraxatezinho, mas ele disse: \textit{Que é isso, Mr.~Henchy!
Quando eu vir que o trabalho está correndo bem não vou me esquecer do
senhor; o senhor pode ter certeza disso}.  Miserável!  Mas, o que mais
poderia ele ser?

--- Eu não te falei, Mat? --- disse Mr.~Hynes.  --- Tierney, o Trapaceiro.

--- Ah, mais trapaceiro que ele não existe --- disse Mr.~Henchy.  --- Não é à
toa que ele tem aqueles olhinhos de suíno.  Maldito seja!  Por que não paga
logo como um homem, em vez de dizer: \textit{É, Mr.~Henchy, eu vou
precisar falar primeiro com o Mr.~Fanning\ldots{} tenho gasto muito
dinheiro!} Engraxatezinho miserável!  Acho que já esqueceu o tempo em que o pai
dele tinha aquela loja de roupas usadas em Mary’s Lane.

--- Isso é mesmo verdade? --- perguntou Mr.~O’Connor.

--- Meu Deus, mas é claro! --- disse Mr.~Henchy.  --- Você nunca ouviu falar?
E os clientes costumavam ir até a loja domingo de manhã antes dos bares abrirem
pra comprar um paletó ou uma calça\ldots{} isso mesmo!  E o pai do Tierney, o
Trapaceiro, tinha sempre uma garrafinha escondida no canto.  Já pensou?  Pois
é.  Foi ali que ele veio ao mundo.

O velho voltou com uns pedaços de carvão e espalhou"-os no fogo.

--- É um começo e tanto --- disse Mr.~O’Connor.  --- Como ele espera que a
gente trabalhe pra ele se ele não paga?

--- Não posso fazer nada --- disse Mr.~Henchy.  --- Com certeza vou encontrar
os intendentes no \textit{hall} quando voltar pra casa.

Mr.~Hynes riu e, descolando os ombros do console da lareira, preparou"-se para
ir embora.

--- Vai ser ótimo quando o rei Eduardo vier --- ele disse.  --- Pois bem,
rapazes, já vou.  Até mais tarde.

Retirou"-se da sala sem demonstrar pressa.  Mr.~Henchy e o velho ficaram calados
mas, no momento em que a porta se fechava, Mr.~O’Connor, que estivera
contemplando o fogo com um ar sonhador, gritou subitamente:

--- Até mais, Joe!

Mr.~Henchy aguardou alguns instantes e então fez um gesto com a cabeça em
direção à porta.

--- Eu gostaria de saber --- ele disse olhando para o fogo --- a razão da
presença do nosso amigo aqui.  O que ele queria?

--- É mesmo, coitado do Joe! --- disse Mr.~O’Connor, atirando ao fogo a ponta
do cigarro.  --- Ele está tão quebrado quanto qualquer um de nós.

Mr.~Henchy fungou vigorosamente e deu uma cusparada volumosa, quase apagando o
fogo, que emitiu um silvo em sinal de protesto.

--- Aqui entre nós, e pra ser sincero --- disse Mr.~Henchy ---, acho que ele
está do outro lado.  É espião do Colgan, se quiserem saber minha opinião:
\textit{Vá até lá e tente descobrir como eles estão se saindo.  Não
vão suspeitar de você}.  Perceberam?

--- Ah, o coitado do Joe é um sujeito decente --- disse Mr.~O’Connor.

--- O pai dele era um homem decente --- admitiu Mr.~Henchy ---, o velho Larry
Hynes! Quanta coisa boa ele fez no tempo dele!  Mas receio que nosso amigo aí
não seja do mesmo quilate.  Diabo, sei que não é fácil estar quebrado, mas não
sei como alguém pode ser parasita.  Será que ele não podia ter um pouco de
hombridade?

--- De mim é que ele não recebe nenhuma demonstração de apreço quando aparece
aqui --- disse o velho.  --- Ele que trabalhe pro lado de lá e que não venha
espionar a gente.

--- Não sei, não --- disse Mr.~O’Connor, descrente, tirando do bolso papel de
cigarro e tabaco.  --- Acho que o Joe Hynes é um sujeito honesto.  E é
habilidoso, também, com uma pena na mão.  Lembram daquilo que ele
escreveu\ldots{}?

--- Pra mim, esses fenianos radicais são muito sabidinhos --- disse Mr.~Henchy.  
--- Aqui entre nós, e pra ser sincero, sabem o que eu acho desses
velhacos?  Acho que a metade deles consta da folha de pagamento do governo
inglês.

--- Nunca se sabe --- disse o velho.

--- Ah, mas eu sei --- disse Mr.~Henchy.  --- São contratados pelo governo
inglês\ldots{}  Não me refiro ao Hynes\ldots{}  Não, assim também, não; acho
que ele está um pouco acima disso\ldots{} Mas tem um certo aristocratazinho,
vesgo de um olho\ldots{} vocês sabem a que patriota me refiro?

Mr.~O’Connor assentiu com a cabeça.

--- Aquele sim é descendente direto do Major Sirr!  Grande patriota!  Aquele
sujeito venderia o país por quatro centavos\textit{\ldots{}} pois é\ldots{} e
ainda cairia de joelhos pra agradecer ao Todo"-poderoso o fato de ter um país
pra vender.

Ouviu"-se uma batida na porta.

--- Entre! --- disse Mr.~Henchy.

Um indivíduo que parecia um clérigo empobrecido ou um ator empobrecido apareceu
à porta.  A roupa preta que vestia estava um tanto apertada, espremendo"-lhe o
corpo de baixa estatura, e era impossível discernir se usava um colarinho de
clérigo ou de leigo, pois a gola da sobrecasaca surrada, cujos botões sem forro
refletiam a luz das velas, estava virada para cima.  Usava um chapéu
arredondado de feltro duro e preto.  Seu rosto, cintilando com gotas de chuva,
fazia lembrar um queijo amarelo e úmido exceto onde duas maçãs rosadas
indicavam os ossos da face.  Abriu subitamente a bocarra para expressar
decepção e ao mesmo tempo arregalou os olhos azuis brilhantes para expressar
alegria e surpresa.

--- Padre Keon! --- disse Mr.~Henchy, dando um pulo da cadeira.  --- É o
senhor? Entre!

--- Ah, não, não, não! --- disse o padre Keon rapidamente, fazendo um bico como
se estivesse falando com uma criança.

--- O senhor não quer entrar e sentar um pouco?

--- Não, não, não! --- disse o padre Keon, falando com uma voz controlada,
meiga, aveludada.  --- Não quero atrapalhar vocês agora!  Estou apenas
procurando por Mr.~Fanning\ldots{}

--- Ele está lá no Black Eagle --- disse Mr.~Henchy.  --- Mas o senhor não quer
entrar e sentar um minuto?

--- Não, não, obrigado.  É um pequeno assunto de trabalho --- disse padre Keon.
--- Muito obrigado.

Ele se afastou da soleira e Mr.~Henchy, pegando uma das velas, foi até a porta
para iluminar a escada.

--- Ah, não se preocupe, por favor!

--- Não, a escada está muito escura.

--- Não, não, dá pra enxergar\ldots{} Muito obrigado.

--- O senhor está enxergando agora?

--- Muito bem, obrigado\ldots{} Obrigado.

Mr.~Henchy trouxe a vela de volta e recolocou"-a sobre a mesa.  Voltou a
sentar"-se próximo ao fogo.  Fez"-se um breve silêncio.

--- Diga"-me uma coisa, John --- disse Mr.~O’Connor, acendendo mais um cigarro
com outro cartão.

--- Hein?

--- Qual é na verdade a função dele?

--- Faz uma pergunta mais fácil --- disse Mr.~Henchy.

--- O Fanning e ele parecem unha e carne.  Estão sempre juntos no Kavanagh’s.
Ele é padre mesmo?

--- É, acho que é\ldots{} Acho que pode ser chamado de ovelha negra.  Não temos
muitos do tipo dele, graças a Deus!  Mas existem alguns\ldots{} De qualquer
forma, é um infeliz\ldots{}

--- E como é que ganha o pão de cada dia? --- perguntou Mr.~O’Connor.

--- Isso é outro mistério.

--- Ele está ligado a alguma capela ou paróquia ou instituição ou\ldots{}?

--- Não --- disse Mr.~Henchy.  --- Acho que vive por conta própria\ldots{} Deus
me perdoe --- acrescentou ---, mas achei que ele fosse a dúzia de cervejas.

--- Será que não tem por aqui alguma coisa pra beber? --- perguntou Mr.~O’Connor.

--- Também estou com a garganta seca --- disse o velho.

--- Perguntei àquele engraxatezinho três vezes --- disse Mr.~Henchy --- se
ele não podia mandar subir uma dúzia de garrafas de cerveja.  Ia falar com ele
outra vez agora há pouco mas ele estava debruçado no balcão, em mangas de
camisa, numa conversa séria com o Cowley, aquele vereador.

--- E por que não falou? --- perguntou Mr.~O’Connor.

--- Ah, eu não tinha como me aproximar enquanto ele estava falando com o
Cowley.  Fiquei esperando até que ele me visse, e então falei: \textit{Com
relação àquele assunto que eu falei com o senhor}\ldots{} E ele
respondeu: \textit{Está tudo certo, Mr.~Henchy}.  É, mas com certeza
aquele anãozinho esqueceu a promessa.

--- Naquele mato tem coelho --- disse Mr.~O’Connor pensativamente.  --- Vi os
três confabulando ontem na esquina da Suffolk Street.

--- Acho que sei qual é o esqueminha deles --- disse Mr.~Henchy.  --- Hoje em
dia se um sujeito quiser ser nomeado prefeito tem que dever dinheiro aos
vereadores.  Aí eles colocam o sujeito na prefeitura.  Por Deus!  Estou
pensando seriamente em me tornar vereador.  O que vocês acham?  Eu daria pra
esse tipo de trabalho?

Mr.~O’Connor riu.

--- Pelo menos no que se refere a dívidas\ldots{}

--- Eu, saindo do Palácio Municipal --- disse Mr.~Henchy --- todo empetecado,
com o Jack perfilado atrás de mim, com peruca branca, hein?

--- E eu como seu secretário particular, John.

--- Isso mesmo.  E o padre Keon seria meu capelão.  Assim, fica tudo em
família.

--- Falando sério, Mr.~Henchy --- disse o velho ---, o senhor faria melhor
figura que muitos deles.  Outro dia eu estava conversando com o Keegan, o
porteiro.  \textit{O que você acha do novo patrão, Pat?} perguntei pra ele.
\textit{Você continua na boa vida?} eu disse.  \textit{Boa vida!} disse ele.
\textit{O homem parece que vive de brisa!} E sabem o que mais ele me
disse? Por Deus, não dá pra acreditar.

--- O quê? --- disseram Mr.~Henchy e Mr.~O’Connor.

--- Ele me disse: \textit{O que você acha de um prefeito de Dublin que
manda comprar meio quilo de costeleta pro jantar?  Você acha que isso
é boa vida?} disse ele.  \textit{Tsc, tsc, tsc!} disse eu.  \textit{Meio quilo
de costeleta}, disse ele, \textit{entrando pelo portão do Palácio
Municipal.  Tsc!} disse eu, \textit{que espécie de gente é essa que nos
governa?}

Naquele momento ouviu"-se uma batida à porta, e um menino esticou o pescoço para
dentro da sala.

--- O que é? --- perguntou o velho.

--- Mandaram do Black Eagle --- disse o menino, caminhando de lado e
depositando no chão uma cesta, com um tilintar de garrafas.

O velho ajudou o menino a transferir as garrafas para cima da mesa e conferiu a
entrega.  Em seguida, o menino colocou a cesta no braço e perguntou:

--- Tem algum casco aí?

--- Que cascos? --- disse o velho.

--- Você não vai deixar a gente beber primeiro? --- disse Mr.~Henchy.

--- Mandaram eu pedir os cascos.

--- Volte amanhã --- disse o velho.

--- Venha cá, menino! --- disse Mr.~Henchy.  --- Dê uma corrida até o
O’Farrell’s e peça um saca"-rolhas emprestado\ldots{} diga que é pro Mr.~Henchy.  
Diga que a gente devolve logo.  Pode deixar a cesta aqui.

O menino retirou"-se e Mr.~Henchy começou a esfregar as mãos, dizendo
alegremente:

--- Pois é, afinal de contas ele não é tão mau assim.  Pelo menos mantém a
palavra.

--- Não tem copo --- disse o velho.

--- Ah, isso não é problema, Jack --- disse Mr.~Henchy.  --- Muita gente boa já
bebeu no gargalo.

--- Em todo caso, é melhor do que nada --- disse Mr.~O’Connor.

--- Ele não é um mau sujeito --- disse Mr.~Henchy, --- é que está nas mãos do
Fanning.  Até que é bem"-intencionado, vocês sabem, à moda dele.

O menino voltou com o saca"-rolhas.  O velho abriu três garrafas e estava
devolvendo o saca"-rolhas quando Mr.~Henchy perguntou ao menino:

--- Quer um pouco, menino?

--- Se o senhor quiser me dar, eu agradeço --- disse o menino.

O velho abriu outra garrafa, de má vontade, e entregou"-a ao menino.

--- Quantos anos você tem? --- ele perguntou.

--- Dezessete --- respondeu o menino.

Como o velho não disse mais nada o menino pegou a garrafa e disse: \textit{Com
os meus respeitos}, \textit{senhor}, dirigindo"-se a Mr.~Henchy, bebeu a
cerveja, colocou a garrafa sobre a mesa e limpou a boca com a manga do paletó.
Então pegou o saca"-rolhas e retirou"-se da sala, caminhando de lado e murmurando
uma despedida.

--- É assim que começa --- disse o velho.

--- É o primeiro passo --- disse Mr.~Henchy.

O velho distribuiu as três garrafas que acabara de abrir e beberam todos ao
mesmo tempo.  Depois de beber, colocaram as garrafas sobre o console da
lareira, ao alcance da mão, e respiraram fundo, satisfeitos.

--- É, eu trabalhei bastante hoje --- disse Mr.~Henchy após uma pausa.

--- Foi mesmo, John?

--- Foi, sim.  Garanti um ou dois votos em Dawson Street, o Crofton e eu.  Aqui
entre nós, o Crofton é um bom sujeito, é claro, mas como cabo eleitoral é um
fracasso.  Fica mudo, olhando pra cara das pessoas enquanto eu falo.

Nesse momento dois homens entraram na sala.  Um, bastante gordo, usava um traje
de sarja azul que parecia prestes a escorregar de sua figura arredondada.
Tinha uma cara grande cuja expressão fazia lembrar um bezerro, olhos azuis
penetrantes e bigode grisalho.  O outro, bem mais jovem e de aspecto mais
frágil, tinha a cara magra e bem"-escanhoada.  Usava paletó de colarinho alto e
chapéu coco de aba larga.

--- Olá, Crofton! --- disse Mr.~Henchy ao sujeito gordo.  --- Falando no
diabo\ldots{}

--- De onde veio a bebida? --- perguntou o jovem.  --- A vaca deu leite?

--- Ah, como sempre, a primeira coisa que o Lyons vê, é a bebida! --- disse Mr.~O’Connor rindo.

--- É assim que vocês fazem campanha? --- disse Mr.~Lyons.  --- E o Crofton e
eu lá fora na chuva gelada angariando votos!

--- Ora, vão pro diabo! --- disse Mr.~Henchy.  --- Eu consigo mais votos em
cinco minutos do que vocês dois numa semana.

--- Abra duas garrafas, Jack --- disse Mr.~O’Connor.

--- Como vou abrir --- disse o velho --- sem o saca"-rolhas?

--- Espere, espere só um minuto! --- disse Mr.~Henchy, levantando"-se
bruscamente. --- Vocês conhecem esse truque?

Pegou duas garrafas em cima da mesa e, levando"-as até o fogo, colocou"-as sobre
a grelha.  Então voltou a sentar"-se diante do fogo e tomou um gole no gargalo.
Mr.~Lyons sentou"-se na ponta da mesa, empurrou o chapéu para o topo da cabeça e
começou a balançar as pernas.

--- Qual é a minha garrafa? --- ele perguntou.

--- Esta aqui, meu jovem --- disse Mr.~Henchy.

Mr.~Crofton sentara"-se sobre um caixote e olhava fixamente para a outra
garrafa.  Estava calado por dois motivos.  O primeiro motivo, mais do que
suficiente, era porque nada tinha a dizer; o segundo motivo era porque
considerava os companheiros inferiores a ele.  Tinha sido cabo eleitoral de
Wilkins, da facção conservadora, mas quando os conservadores retiraram seu
candidato e, escolhendo entre os males o menor, deram seu apoio ao candidato
nacionalista, ele foi contratado para trabalhar para Mr.~Tierney.

Em questão de minutos ouviu"-se um \textit{Pok!}\ldots{} e a rolha da garrafa de
Mr.~Lyons voou longe.  Mr.~Lyons pulou da mesa, foi até a lareira, pegou a
garrafa e voltou para a mesa.

--- Eu estava acabando de contar pra eles, Crofton --- disse Mr.~Henchy ---,
que conseguimos alguns votos importantes hoje.

--- Votos de quem? --- perguntou Mr.~Lyons.

--- Bem, consegui o voto do Parkes, do Atkinson e garanti o do Ward em Dawson
Street.  Aquele é dos bons: grã"-fino, velho conservador!  \textit{Mas o seu
candidato não é nacionalista?} disse ele.  \textit{É um homem
decente}, disse eu.  \textit{Será favorável a qualquer coisa que venha a
beneficiar este país.  É um contribuinte de peso}, eu disse.  \textit{Possui
inúmeros imóveis na cidade e três estabelecimentos comerciais
e não seria vantajoso para ele manter os impostos baixos?  É
um cidadão influente e respeitado}, disse eu, \textit{cumpridor da lei e não
pertence a qualquer partido\ldots{} bom, ruim ou indiferente}.
É assim que se deve falar com esse tipo de gente.

--- E a respeito do discurso de boas"-vindas ao rei? --- perguntou Mr.~Lyons,
depois de dar um gole e estalar a língua.

--- Escute aqui --- disse Mr.~Henchy.  --- O que precisamos neste país,
conforme eu disse ao velho Ward, é de capital.  A vinda do rei significa
entrada de divisas no país.  Os cidadãos de Dublin vão se beneficiar.  Vejam
todas aquelas fábricas ociosas lá perto das docas!  Pensem no dinheiro que pode
ser gerado se as velhas indústrias, usinas, moinhos e estaleiros forem
reativados.  É de capital que precisamos.

--- Escute aqui, John --- disse Mr.~O’Connor.  --- Por que haveremos de dar
boas"-vindas ao rei da Inglaterra? O próprio Parnell não\ldots{}

--- Parnell --- disse Henchy --- está morto.  --- Agora, eu acho o seguinte: o
sujeito é impedido pela maldita mãe de ser coroado e só consegue sentar no
trono quando já está grisalho.  É um homem sensível, e deseja o nosso bem.  A
meu ver é um ótimo sujeito e não é dado a frescuras.  Ele deve pensar:
\textit{A velha nunca se deu ao trabalho de visitar esses malditos irlandeses.
Meu Deus, eu vou até lá ver como eles são}.  E nós vamos insultar o
sujeito no momento em que vem nos fazer uma visita de cortesia?  Hein?  Não é,
Crofton?

Mr.~Crofton assentiu com a cabeça.

--- Mas, afinal de contas --- disse Mr.~Lyons em tom de discordância ---, você
sabe, a vida do rei Eduardo não é lá\ldots{}

--- O que passou, passou --- disse Mr.~Henchy.  --- Eu o admiro.  É um sujeito
comum, como você e eu.  Gosta de um trago e é um tanto mulherengo, talvez, mas
tem espírito esportivo.  Diabo, será que nós irlandeses não conseguimos jogar
limpo?

--- Está bem --- disse Mr.~Lyons.  --- Mas vamos analisar o caso de Parnell.

--- Pelo amor de Deus --- disse Mr.~Henchy ---, o que um caso tem a ver com o
outro?

--- Estou me referindo --- disse Mr.~Lyons --- aos nossos ideais.  Por que
haveríamos de dar boas"-vindas a um sujeito desses?  Você acha que depois do que
Parnell fez poderia continuar sendo nosso líder?  E por que, então, haveríamos
de homenagear Eduardo~\textsc{vii}?

Hoje é o aniversário de Parnell --- disse Mr.~O’Connor --- e não vamos mexer em
feridas que já cicatrizaram.  O homem está morto e debaixo da terra e todos o
respeitam, até os conservadores --- acrescentou virando"-se para Mr.~Crofton.

\textit{Pok!} A rolha retardatária voou do gargalo da garrafa de Mr.~Crofton.
Este levantou"-se do caixote onde estava sentado e foi até a lareira.  Ao
retornar ao seu lugar, trazendo na mão a presa capturada, disse com uma voz
profunda:

--- O nosso partido o respeita porque ele foi um cavalheiro.

--- Exatamente, Crofton! --- disse Mr.~Henchy com euforia.  --- Era o único
homem capaz de lidar com aquele saco de gatos.  \textit{Fora, seus cachorros!
Pra fora, seus vira"-latas!} Foi assim que ele os tratou.  Entre, Joe!  Entre!
--- ele gritou, vendo Mr.~Hynes à porta.

Mr.~Hynes entrou, caminhando lentamente.

--- Abra mais uma garrafa de cerveja, Jack --- disse Mr.~Henchy --- Ah, esqueci
de que não temos saca"-rolha! Me passe aí uma garrafa, que eu ponho aqui perto
do fogo.

O velho entregou"-lhe outra garrafa e ele a depositou sobre a grelha.

--- Sente"-se, Joe --- disse Mr.~O’Connor ---, estamos justamente falando a
respeito do Chefe.

--- Pois é, pois é! --- disse Mr.~Henchy.

Mr.~Hynes sentou"-se na parte lateral da mesa ao lado de Mr.~Lyons e ficou
calado.

--- Em todo caso, eis aí um deles --- disse Mr.~Henchy --- que não o renegou.
Por Deus, temos que reconhecer, Joe!  Não, por Deus, você ficou do lado dele,
como um homem!

--- Ah, Joe --- disse Mr.~O’Connor subitamente.  --- Como é mesmo aquilo que
você escreveu\ldots{} lembra"-se?  Está aí com você?

--- Isso mesmo! --- disse Mr.~Henchy.  --- Recite pra nós.  Você já ouviu,
Crofton? Ouça agora: é esplêndido.

--- Vamos lá --- disse Mr.~O’Connor.  --- Vá em frente, Joe.

A princípio Mr.~Hynes parecia não se recordar do escrito ao qual se referiam
mas após um momento de reflexão, disse:

--- Ah, já sei\ldots{} Claro, mas aquilo já passou.

--- Vamos, homem! --- disse Mr.~O’Connor.

--- Ssshh, ssshh --- disse Mr.~Henchy.  --- Agora, Joe!

Mr.~Hynes hesitou um pouco mais.  Então, em meio ao silêncio tirou o chapéu,
colocou"-o sobre a mesa e pôs"-se de pé.  Dava a impressão de estar ensaiando o
poema mentalmente.  Após uma longa pausa anunciou:

\begin{quote}\centering
\textsc{a morte de parnell}

\textit{6 de outubro de 1891}
\end{quote}

Pigarreou uma ou duas vezes e então começou a recitar:

\begin{verse}\itshape
Está morto.  Nosso Rei sem coroa está morto.\\
Ah, Erin, chore de dor e tristeza,\\
Pois está morto aquele que o bando\\
De modernos hipócritas abateu com frieza.

Mataram"-lhe os cães covardes\\
Por ele próprio glorificados;\\
E as esperanças e os sonhos de Erin\\
Perecem com o rei mortificado.

Em palácios e choupanas\\
O coração irlandês, em desatino,\\
Chora de tristeza --- ele se foi,\\
Aquele que nos traçaria o destino.

Com ele Erin alcançaria a fama,\\
E a bandeira verde gloriosa,\\
Seus estadistas, poetas e guerreiros,\\
Seriam reconhecidos de maneira honrosa.

Ele sonhava (tudo não passou de um sonho!)\\
Com a Liberdade: mas eis que enquanto lutava\\
Para conquistar seu ideal, a traição\\
Separou"-o daquilo que tanto amava.

Malditas as mãos traidoras e covardes\\
Que com um beijo derrubaram seu Senhor,\\
Que o traíram, entregando"-o à multidão\\
De padres aduladores, inimigos do amor.

Que a vergonha eterna perturbe\\
A consciência do que tentava\\
Macular, desonrar o grande nome\\
Daquele que tanto o repudiava.

Ele tombou como tombam os bravos,\\
Altivo, destemido e honrado,\\
E na morte ora encontra"-se unido\\
Aos heróis irlandeses do passado.

Que ruído nenhum lhe perturbe o sono!\\
Descanse em paz; já não precisa ter\\
Ambições humanas de outrora,\\
Nem as benesses da glória alcançar.

Conseguiram, conseguiram abatê"-lo.\\
Mas, Erin, ouça, seu espírito poderia\\
Ressurgir das cinzas, tal qual a Fênix,\\
No alvorecer de qualquer dia,

Do dia em que nos trará o reino da Liberdade.\\
E nesse dia poderá a Irlanda\\
Na taça que erguer à Felicidade\\
Brindar a uma tristeza: à memória de Parnell.
\end{verse}

Mr.~Hynes voltou a sentar"-se na beirada da mesa.  Quando terminou a recitação
houve um momento de silêncio e então uma salva de palmas: até Mr.~Lyons
aplaudiu.  O aplauso prosseguiu durante alguns instantes.  Quando cessou, os
ouvintes beberam em silêncio nos gargalos das garrafas.

\textit{Pok!} A rolha voou da garrafa de Mr.~Hynes, mas Mr.~Hynes permaneceu
sentado sobre a mesa, enrubescido e com a cabeça descoberta.  Parecia não ter
ouvido o chamado.

--- Parabéns, Joe! --- disse Mr.~O’Connor, tirando do bolso os papelotes e o
tabaco para melhor disfarçar a emoção.

--- O que você achou do poema, Crofton? --- disse Mr.~Henchy enfaticamente.
--- É muito bom?  Não é?

Mr.~Crofton disse que se tratava de um escrito excelente.


\chapter{Mãe}
\hedramarkboth{Mãe}{James Joyce}

\textsc{Mr.~Holohan,} vice"-secretário da Associação Eire Abu, percorria toda
Dublin havia quase um mês, com as mãos e os bolsos cheios de pedacinhos de
papel sujo, providenciando uma série de recitais.  Tinha uma perna mais curta
que a outra e por isso os amigos chamavam"-no de Hoppy Holohan.\footnote{ Em
inglês, o verbo \textit{to hop}, entre outros significados, quer dizer ``mancar''.
O apelido sugere, também, o adjetivo \textit{happy}, “feliz”.}
Percorreu toda a cidade, passou horas nas esquinas discutindo e fazendo
anotações; mas no final das contas foi Mrs.~Kearney quem providenciou tudo.

Miss Devlin tornara"-se Mrs.~Kearney por pirraça.  Fora educada num excelente
colégio de freiras onde aprendera francês e música.  Por ser pálida de natureza
e por ter atitudes rígidas fizera poucas amizades na escola.  Quando atingira a
idade de se casar fizeram"-na frequentar diversas residências, onde seus
talentos musicais e suas maneiras refinadas tornaram"-se objeto de admiração.
Sentava"-se no centro do círculo glacial de seus dotes, aguardando algum
proponente que ousasse penetrá"-lo para oferecer"-lhe uma vida resplandecente.
Mas os rapazes que encontrava eram tipos comuns e ela em nada os incentivava e
procurava consolar seus desejos românticos devorando secretamente grandes
quantidades de bombons turcos.  No entanto, quando se aproximou a idade limite
e as línguas dos amigos começaram a ficar afiadas, ela a todos calou casando"-se
com Mr.~Kearney, fabricante de botas que residia em Ormond Quay.

Era bem mais velho do que ela.  Sua prosa, sempre séria, irrompia de quando em
vez em meio a uma espessa barba castanha.  No final do primeiro ano do
casamento Mrs.~Kearney percebeu que um homem do tipo de seu marido tinha mais
utilidade do que um indivíduo romântico, mas ela jamais chegou a abandonar suas
ideias românticas.  Ele era abstêmio, econômico e religioso: comungava toda
primeira sexta"-feira do mês, às vezes acompanhado dela, mais frequentemente
sozinho.  Mas ela nunca abandonara a religião e era para ele uma boa esposa.
Quando iam a uma festa numa casa estranha, bastava ela erguer levemente a
sobrancelha para que ele se levantasse e se despedisse e, quando a tosse o
incomodava, ela cobria"-lhe os pés com um edredom e preparava"-lhe uma bebida
quente.  Ele, por sua vez, era um pai exemplar.  Fazendo depósitos semanais
junto a determinada instituição ele garantia que cada uma de suas duas filhas
ao completar a idade de vinte e quatro anos dispusesse de um dote no valor de
cem libras.  Encaminhou a filha mais velha, Kathleen, a um bom colégio de
freiras, onde ela aprendeu francês e música, e mais tarde custeou"-lhe os
estudos no Conservatório.  Todos os anos em julho Mrs.~Kearney tinha a
oportunidade de dizer a uma de suas amigas:

--- Meu querido marido vai nos propiciar algumas semanas em Skerries.

Se não fosse Skerries era Howth ou Creystones.

Quando a Renascença Irlandesa começou a ser reconhecida, Mrs.~Kearney decidiu
tomar partido do nome da filha e contratou um professor particular de irlandês.
Kathleen e a irmã passaram a trocar com as amigas cartões"-postais de cenas
irlandesas.  Em domingos de festa, quando Mr.~Kearney levava a família à
pró"-catedral, um pequeno grupo de pessoas costumava reunir"-se após a missa na
esquina da Cathedral Street.  Eram amigos dos Kearney --- simpatizantes de
música ou do Partido Nacionalista; e, quando os mexericos esgotavam"-se,
trocavam apertos de mão, rindo ao ver tantas mãos emaranhadas, e despediam"-se
falando irlandês.  Em pouco tempo o nome de Kathleen Kearney corria de boca em
boca.  Dizia"-se que era dotada de talento musical, além de ser ótima moça e
que, ademais, era defensora do movimento em prol do idioma nacional.  
Mrs.~Kearney sentia"-se envaidecida com tudo isso.  Por conseguinte não se
surpreendeu quando um dia Mr.~Holohan a procurou para convidar Kathleen para
acompanhar ao piano uma série de quatro grandes recitais a serem promovidos no
Antient Concert Rooms pela associação à qual ele pertencia.  Ela o conduziu à
sala de visitas, convidou"-o a sentar"-se e serviu licor e biscoitos numa salva
de prata.  Envolveu"-se de corpo e alma no projeto, oferecendo e recusando
ideias: finalmente assinaram um contrato que contemplava Kathleen com oito
guinéus para acompanhar ao piano os quatro grandes recitais.

Visto que Mr.~Holohan era inexperiente em questões tão delicadas como redação
de material publicitário e elaboração de programas, Mrs.~Kearney orientou"-o.
Era uma mulher de tato.  Sabia quais artistas deveriam ter os nomes impressos
em maiúsculas bem como os nomes que deveriam aparecer em minúsculas.  Sabia que
o primeiro tenor não gostaria de entrar em cena em seguida ao número cômico
apresentado por Mr.~Mead.  Com o intuito de prender a atenção da plateia ela
intercalou no programa as peças mais fracas e os clássicos de sucesso.  Mr.~Holohan 
visitava"-a diariamente para pedir"-lhe a opinião sobre diversos
aspectos.  Ela era sempre amável e acessível --- na verdade, acolhedora.
Colocava a garrafa de licor diante dele, dizendo:

--- Sirva"-se, Mr.~Holohan!

E enquanto ele se servia ela dizia:

--- Não faça cerimônia!  Não faça cerimônia!

Tudo corria muito bem.  Mrs.~Kearney comprou na loja Brown Thomas uma linda
peça de cetim cor"-de"-rosa para reformar a parte da frente do vestido de
Kathleen.  Custou um bom dinheiro; mas há ocasiões em que uma pequena despesa é
justificável.  Comprou uma dúzia de ingressos a dois \textit{shillings} cada
para o último recital e enviou"-os àqueles amigos que no caso de não receberem o
ingresso provavelmente não compareceriam.  De nada se esqueceu e, graças a ela,
todas as providências cabíveis foram devidamente tomadas.

Os recitais estavam marcados para quarta, quinta, sexta e sábado.  Quando 
Mrs.~Kearney chegou ao Antient Concert Rooms na noite de quarta"-feira acompanhada da
filha, não gostou do que viu.  Alguns rapazes, com tarjas azuis nos paletós,
estavam parados no vestíbulo: nenhum deles usava traje a rigor.  Entrou
acompanhada da filha e um olhar de relance pela porta aberta da sala de
concertos esclareceu"-lhe o motivo da ociosidade dos recepcionistas.  Por um
momento chegou a achar que tinha se enganado quanto ao horário.  Mas não,
faltavam vinte minutos para as oito.

No camarim detrás do palco foi apresentada ao secretário da Associação, Mr.~Fitzpatrick.  
Ela sorriu e apertou"-lhe a mão.  Era um homenzinho de rosto
branco e inexpressivo.  Reparou que ele usava um chapéu marrom displicentemente
caído de lado e que seu sotaque era monocórdio.  Enquanto conversava com ela,
mascou a ponta do programa que trazia na mão, reduzindo"-a a uma papa.  Não
parecia ser do tipo que se deixava abalar por decepções.  Mr.~Holohan entrava
seguidamente no camarim trazendo notícias da bilheteria.  Músicos e cantores
conversavam um pouco nervosos, olhavam"-se a todo momento no espelho, e
enrolavam e desenrolavam suas partituras.  Quando já eram quase oito e meia, as
poucas pessoas presentes na sala começaram a demonstrar impaciência.  
Mr.~Fitzpatrick apareceu, sorriu inexpressivamente para a plateia, e disse:

--- Muito bem, senhoras e senhores, creio que devemos iniciar o espetáculo.

Mrs.~Kearney retribuiu"-lhe com um olhar de desprezo a frase, cuja última sílaba
fora pronunciada sem a menor expressão, e disse à filha, para animá"-la:

--- Está pronta, querida?

Na primeira oportunidade chamou Mr.~Holohan de lado e pediu"-lhe que explicasse
o que estava acontecendo.  Mr.~Holohan não tinha qualquer explicação.  Disse
que o comitê enganara"-se ao programar quatro recitais, quatro era um número
excessivo.

--- E os artistas! --- disse Mrs.~Kearney.  --- A gente vê que estão se
esforçando, mas na verdade não são bons.

Mr.~Holohan admitiu que os artistas não eram bons mas o comitê, disse ele,
tinha decidido deixar um pouco de lado os três primeiros recitais e reunir os
grandes talentos na noite de sábado.  Mrs.~Kearney nada retrucou, mas, à medida
que os números medíocres se sucediam no palco e que o pequeno número de pessoas
na plateia diminuía, começou a se arrepender por ter gasto dinheiro com um
recital daqueles.  Havia algo que a desagradava em tudo aquilo e o sorriso
amarelo de Mr.~Fitzpatrick irritava"-a profundamente.  Mesmo assim, resolveu
permanecer calada e aguardar o desfecho da situação.  O recital terminou pouco
antes das dez horas e todos se retiraram rapidamente.

O recital da noite de quinta"-feira atraiu mais público mas Mrs.~Kearney notou
imediatamente que a maioria das pessoas entrara de graça.  A plateia
comportou"-se de maneira indecorosa, como se a apresentação não passasse de um
ensaio geral.  Mr.~Fitzpatrick parecia estar se divertindo; não percebia que
Mrs.~Kearney o observava indignada.  Ele havia se posicionado num canto do
palco, e a todo momento esticava a cabeça por trás da cortina e trocava
risadinhas com dois amigos que se encontravam no balcão.  No decorrer da noite
Mrs.~Kearney ficou sabendo que o recital de sexta"-feira seria cancelado e que o
comitê moveria céus e terras para garantir casa cheia na noite de sábado.  Ao
saber disso ela procurou Mr.~Holohan.  Interpelou"-o no momento em que ele
corria mancando para levar a uma moça um copo de limonada e perguntou"-lhe se
aquilo que escutara tinha fundamento.  Sim, tinha fundamento.

--- Mas, é claro, isso não altera o contrato --- ela disse.  --- O contrato
prevê quatro recitais.

Mr.~Holohan parecia estar com pressa: pediu"-lhe que falasse com Mr.~Fitzpatrick.  
Mrs.~Kearney estava começando a ficar preocupada.  Chamou Mr.~Fitzpatrick de lado 
e disse a ele que a filha tinha sido contratada para quatro
recitais e que, decerto, nos termos do contrato, ela deveria receber o valor
estipulado, independentemente do número de recitais apresentados pela
associação.  Mr.~Fitzpatrick, que custou a compreender a que ela se referia,
declarou"-se incapaz de resolver a questão e disse que levaria o caso ao comitê.
A raiva que Mrs.~Kearney sentia começou a transparecer nas maçãs do rosto e foi
com muito esforço que se controlou para não perguntar:

--- E quem compõe esse tal de \textit{comitê}, por gentileza?

Mas ela sabia que tal pergunta não conviria a uma senhora de classe: portanto,
calou"-se.

Meninos invadiram as principais ruas de Dublin na manhã de sexta"-feira com
pacotes de folhetos publicitários.  Chamadas especiais apareceram nos jornais
de edição vespertina lembrando aos amantes da música o grande espetáculo que
lhes aguardava na noite seguinte.  Mrs.~Kearney sentia"-se mais confiante mas
achou por bem demonstrar ao marido a sua preocupação.  Ele ouviu atentamente e
disse que talvez fosse melhor acompanhá"-la na noite de sábado.  Ela concordou.
Respeitava o marido assim como respeitava a Empresa de Correios e Telégrafos,
algo grande, seguro e confiável; e embora tivesse conhecimento dos seus poucos
talentos, reconhecia"-lhe o valor abstrato enquanto ser humano do sexo
masculino.  Sentia"-se satisfeita com o fato de ele ter tido a ideia de
acompanhá"-la.  Recapitulou mentalmente os planos.

Chegou então a noite do grande recital.  Mrs.~Kearney, acompanhada do marido e
da filha, chegou ao Antient Concert Rooms quarenta e cinco minutos antes da
hora marcada para o início do espetáculo.  Por azar a noite estava chuvosa.
Mrs.~Kearney deixou a roupa e as partituras da filha sob os cuidados do marido
e percorreu o teatro inteiro à procura de Mr.~Holohan e Mr.~Fitzpatrick.  Não
conseguiu encontrá"-los.  Perguntou aos recepcionistas se algum membro do comitê
encontrava"-se na sala de concertos e, com muito custo, um recepcionista trouxe
uma mulherzinha chamada Miss Beirne, a quem Mrs.~Kearney explicou que desejava
falar com um dos secretários.  Miss Beirne disse que estes deveriam chegar a
qualquer momento e perguntou em que poderia ser útil.  Mrs.~Kearney olhou
inquisitivamente para aquele rosto envelhecido e petrificado numa expressão de
confiança e entusiasmo, e respondeu:

--- Não é nada, obrigada!

A mulherzinha disse que esperava ver a sala repleta.  Ficou olhando a chuva até
que a tristeza da rua molhada apagou"-lhe do rosto retorcido toda a confiança e
todo o entusiasmo.  Então suspirou e disse:

--- Pois é, fizemos todo o possível, a senhora sabe.

Mrs.~Kearney precisava retornar ao camarim.

Os artistas estavam chegando.  O baixo e o segundo tenor já estavam lá.  O
baixo, Mr.~Duggan, era um jovem esbelto com um bigode preto e ralo.  Era filho
do porteiro de um edifício no centro da cidade e, quando menino, costumava
cantar notas graves e prolongadas que ressoavam no saguão do prédio.  De origem
humilde, conseguira subir na vida e tornar"-se um cantor de primeira linha.  Já
havia atuado em grandes óperas.  Certa noite, quando o cantor principal
adoeceu, ele fez o papel do rei na ópera \textit{Maritana}, no Queen’s Theatre.
Cantou com grande sentimento e volume de voz e foi calorosamente recebido pelas
galerias; mas, infelizmente, não causou maior impressão porque sem querer
limpou o nariz uma ou duas vezes com a mão enluvada.  Era despretensioso e
falava pouco.  Falava tão suavemente que os erros de gramática passavam
despercebidos e por causa da voz nunca bebia nada além de leite.  Mr.~Bell, o
segundo tenor, era um sujeito alourado e de baixa estatura que todo ano
participava dos concursos no Feis Ceoil.  Na quarta tentativa recebera uma
medalha de bronze.  Estava extremamente nervoso e sentia"-se extremamente
ameaçado pelos outros tenores e disfarçava o nervosismo através de uma
amabilidade exagerada.  Tinha prazer em revelar às pessoas a provação que um
recital representava para ele.  Por isso, ao avistar Mr.~Duggan dirigiu"-se a
ele e perguntou:

--- Também vai participar?

--- Vou --- respondeu Mr.~Duggan.

Mr.~Bell sorriu para o companheiro de sofrimento, estendeu a mão e disse:

--- Aperte aqui!

Mrs.~Kearney passou pelos dois jovens e foi até a extremidade da cortina para
espiar a sala.  Os lugares estavam sendo ocupados rapidamente e um burburinho
agradável circulava no auditório.  Ela voltou e falou em particular com o
marido.  A conversa era obviamente a respeito de Kathleen pois ambos olhavam
seguidamente para ela, que conversava com uma amiga, Miss Healy, a contralto,
filiada ao Partido Nacionalista.  Uma mulher desconhecida e desacompanhada, de
rosto pálido, atravessou a sala.  As mulheres seguiam com olhos penetrantes o
vestido azul desbotado que se esticava para lhe encobrir o corpo magro.  Alguém
disse que se tratava de Madame Glynn, a soprano.

--- De onde será que desenterraram essa? --- disse Kathleen a Miss Healy.  ---
Nunca ouvi falar nela.

Miss Healy foi obrigada a sorrir.  Mr.~Holohan entrou mancando no camarim
naquele momento e as duas jovens perguntaram"-lhe quem era a desconhecida.  
Mr.~Holohan disse que se tratava de Madame Glynn e que viera de Londres.  Madame
Glynn colocou"-se no canto da sala, empunhando firmemente uma partitura enrolada
e de vez em quando mudando a direção do seu olhar assustado.  A sombra
escondeu"-lhe o vestido desbotado mas por vingança salientou"-lhe as reentrâncias
da clavícula.  O burburinho da plateia estava cada vez mais forte.  O primeiro
tenor e o barítono chegaram juntos.  Estavam ambos elegantemente vestidos,
robustos e bonachões, e emprestaram à companhia um certo \textit{glamour}.

Mrs.~Kearney trouxe a filha até o local onde os dois se encontravam, e
dirigiu"-lhes amavelmente a palavra.  Pretendia entrosar"-se bem com a dupla mas,
enquanto se esforçava em ser gentil, seus olhos seguiam Mr.~Holohan que,
puxando da perna, fazia suas incursões ardilosas.  Na primeira oportunidade,
desculpou"-se e partiu atrás dele.

--- Mr.~Holohan, eu queria ter uma palavrinha com o senhor --- ela disse.

Encaminharam"-se para um canto discreto do corredor.  Mrs.~Kearney perguntou"-lhe
quando a filha seria paga.  Mr.~Holohan disse que tais questões ficavam por
conta de Mr.~Fitzpatrick.  Mrs.~Kearney retrucou que não tinha nada a tratar
com Mr.~Fitzpatrick.  A filha assinara um contrato para receber oito guinéus e
que haveria de ser paga.  Mr.~Holohan disse que o assunto não era de sua
alçada.

--- Como não é de sua alçada? --- perguntou Mrs.~Kearney.  --- Não foi o senhor
mesmo que trouxe o contrato?  Seja lá como for, se não é de sua alçada, é da
minha alçada e pretendo cuidar direitinho do assunto.

--- É melhor a senhora falar com Mr.~Fitzpatrick --- disse Mr.~Holohan com
frieza.

Não tenho nada a tratar com Mr.~Fitzpatrick --- repetiu Mrs.~Kearney.  ---
Tenho o contrato e pretendo fazer valer o que foi acertado.

Quando retornou ao camarim suas faces estavam levemente coradas.  O ambiente
estava alegre.  Dois sujeitos de sobretudo obstruíam a frente da lareira e
conversavam informalmente com Miss Healy e o barítono.  Um era repórter do
\textit{Freeman} e o outro era Mr.~O’Madden Burke.  O repórter do
\textit{Freeman} viera informar que não poderia assistir ao recital pois
precisava cobrir a conferência de um padre americano na Mansion House.
Instruiu"-os a deixar um artigo em nome dele na sede do \textit{Freeman} e ele
se encarregaria de incluí"-lo na edição.  Tinha a cabeça grisalha, uma voz
agradável e bons modos.  Segurava nos dedos um charuto apagado e o cheiro do
tabaco pairava a sua volta.  Não tinha intenção de se demorar ali porque
recitais, músicos e cantores deixavam"-no extremamente entediado, mas permaneceu
algum tempo encostado ao console da lareira.  Miss Healy estava bem em frente a
ele, falando e rindo.  Era suficientemente maduro em termos de idade para
compreender a razão de toda aquela amabilidade e suficientemente jovem em
termos de espírito para tirar proveito do momento.  O calor, o aroma e a
coloração do corpo da mulher estimulavam"-lhe os sentidos.  Tinha a agradável
consciência de que era por sua causa que aquele colo arfava, de que o riso, o
perfume e os olhares sugestivos eram um tributo a ele.  Quando já não podia
mais ficar, despediu"-se dela pesaroso.

--- O’Madden Burke vai escrever o artigo --- ele explicou a Mr.~Holohan --- e
eu me encarrego da publicação.

--- Muito obrigado, Mr.~Hendrick --- disse Mr.~Holohan.  --- Eu sei que o
senhor se encarregará da publicação.  Mas, não vai aceitar nada antes de ir
embora?

--- Não seria mal --- disse Mr.~Hendrick.

Os dois seguiram por umas passagens sinuosas e subiram uma escada escura e
chegaram a uma saleta isolada onde um dos recepcionistas abria garrafas para um
pequeno grupo de homens.  Um desses cavalheiros era Mr.~O’Madden Burke, que por
instinto descobrira a saleta.  Era um idoso afável que, quando parado,
equilibrava o corpo volumoso com o auxílio de um grande guarda"-chuva de seda.
Seu nome grandiloquente era o guarda"-chuva moral no qual equilibrava a delicada
situação de suas finanças.  Era muito respeitado.

Enquanto Mr.~Holohan recepcionava o repórter do \textit{Freeman}, Mrs.~Kearney
falava tão efusivamente com o marido que este foi obrigado a lhe pedir que
abaixasse o tom de voz.  A conversa entre as outras pessoas presentes no
camarim havia se tornado um tanto tensa.  Mr.~Bell, o primeiro a se
apresentar, estava a postos, com a partitura na mão, mas a pianista não dava
sinal de vida.  Era evidente que algo estava errado.  Mr.~Kearney mantinha o
olhar parado e cofiava a barba, enquanto Mrs.~Kearney cochichava algo no ouvido
de Kathleen com uma exaltação contida.  Da sala de concertos ouvia"-se o público
impaciente, batendo palmas e pés.  O primeiro tenor e o barítono e Miss Healy
estavam lado a lado, aguardando tranquilamente, mas Mr.~Bell estava com os
nervos bastante abalados por recear que o público achasse que o atraso era
culpa sua.

Mr.~Holohan e Mr.~O’Madden Burke entraram no camarim.  Imediatamente 
Mr.~Holohan deu"-se conta da tensão que pairava no ar.  Aproximou"-se de 
Mrs.~Kearney e falou"-lhe energicamente.  Enquanto falavam, o barulho na sala
aumentou.  Mr.~Holohan ficou vermelho e nervoso.  Falava pelos cotovelos, mas
Mrs.~Kearney repetia secamente:

--- Ela não vai entrar em cena.  Vai ter de receber os oito guinéus.

Mr.~Holohan apontou desesperadamente para a sala, onde a plateia aplaudia e
batia com os pés no chão.  Apelou para Mr.~Kearney e para Kathleen.  
Mas Mr.~Kearney continuava a cofiar a barba e Kathleen baixou o olhar, mexendo o bico
do sapato novo: disse que não era culpa dela.  Mrs.~Kearney repetiu:

--- Ela não entra em cena sem receber o dinheiro.

Após um breve bate"-boca Mr.~Holohan saiu às pressas, mancando.  O silêncio
abateu"-se sobre o camarim.  Quando a tensão provocada pelo silêncio ficou
insuportável Miss Healy disse ao barítono:

--- Você esteve com Mrs.~Pat Campbell nesta semana?

O barítono respondeu que não mas que soubera que ela passava bem.  A conversa
parou por ali.  O primeiro tenor inclinou a cabeça e começou a contar os elos
da corrente dourada que trazia à cintura, sorrindo e cantarolando notas
aleatórias para checar o efeito da voz na cavidade frontal.  De vez em quando
alguém olhava para Mrs.~Kearney.

O barulho do auditório estava ensurdecedor no momento em que Mr.~Fitzpatrick
irrompeu no camarim, seguido por Mr.~Holohan, ofegante.  As palmas e o ruído
dos pés agora eram acompanhados de assobios.  Mr.~Fitzpatrick trazia na mão
algum dinheiro.  Contou quatro notas, entregou"-as a Mrs.~Kearney e disse que
receberia a outra metade no intervalo.  Mrs.~Kearney disse:

--- Faltam quatro \textit{shillings}.

Mas Kathleen levantou a barra do vestido e disse: \textit{Vamos, Mr.~Bell}, a
primeira atração, que tremia como vara verde.  O cantor e a pianista entraram
juntos.  O barulho na sala de concertos abrandou.  Seguiu"-se uma pausa de
alguns segundos: e então ouviu"-se o piano.

A primeira parte do recital foi um sucesso à exceção do número apresentado por
Madame Glynn.  A pobre mulher cantou \textit{Killarney} com uma voz fraca e
ofegante, com todos aqueles floreios ultrapassados de entonação e pronúncia que
a seu ver atribuíam elegância a sua interpretação.  Dava a impressão de ter
ressuscitado de um velho guarda"-roupa de teatro e as galerias zombaram de seus
trinados agudos e melosos.  O primeiro tenor e a contralto, em contrapartida,
fizeram a casa vir abaixo.  Kathleen executou uma seleção de peças irlandesas e
foi generosamente aplaudida.  A primeira parte do programa foi encerrada com um
veemente poema patriótico declamado por uma jovem que organizava apresentações
de grupo de teatro amador.  Os versos receberam os aplausos merecidos; e, ao
final, os homens saíram para o intervalo, satisfeitos.

Enquanto isso o ambiente no camarim parecia eletrizado.  Em um canto estavam
reunidos Mr.~Holohan, Mr.~Fitzpatrick, Miss Beirne, dois recepcionistas, o
barítono, o baixo e Mr.~O’Madden Burke.  Mr.~O’Madden Burke afirmava ter sido
aquela a situação mais infame que presenciara em toda a sua vida.  A carreira
de Miss Kathleen Kearney estava daquele dia em diante liquidada em Dublin, ele
dizia.  Perguntaram ao barítono o que pensava da conduta de Mrs.~Kearney.  Ele
preferia não se expressar a respeito do assunto.  Tinha recebido a sua parte e
não queria criar caso com ninguém.  Disse, contudo, que Mrs.~Kearney poderia
ter tido um pouco de consideração pelos demais artistas.  Recepcionistas e
secretários discutiam acirradamente sobre o que deveria ser feito na hora do
intervalo.

--- Concordo com Miss Beirne --- disse Mr.~O’Madden Burke.  --- Não devemos
pagar mais nada.

No outro canto do camarim encontravam"-se Mrs.~Kearney e o marido, Mr.~Bell,
Miss Healy e a jovem que recitara o poema patriótico.  Mrs.~Kearney dizia que
o comitê a tratara de modo vexatório.  Não poupara despesas nem esforços e era
recompensada daquela maneira.  

Pensavam que estavam lidando apenas com uma menina e que, portanto, podiam
fazer o que bem entendessem.  Mas ela mostraria que estavam enganados.  Jamais
a teriam tratado daquela maneira se ela fosse homem.  Mas os direitos de sua
filha seriam respeitados: ninguém a passaria para trás.  Se não pagassem até o
último centavo ela faria um escândalo em Dublin.  É claro que se preocupava com
os demais artistas.  Mas o que mais ela poderia fazer?  Pediu a opinião do
segundo tenor e este respondeu que a seu ver ela não tinha sido tratada
corretamente.  Em seguida pediu a opinião de Miss Healy.  No fundo Miss Healy
queria juntar"-se ao outro grupo mas não tinha coragem de fazê"-lo porque era
grande amiga de Kathleen e frequentadora da casa dos Kearney.

Assim que terminou a primeira parte Mr.~Fitzpatrick e Mr.~Holohan dirigiram"-se
a Mrs.~Kearney para informá"-la de que os outros quatro guinéus seriam pagos
após a reunião do comitê, a ser realizada na terça"-feira da semana seguinte, e
que, caso a filha não tocasse na segunda parte do recital, o comitê
consideraria o contrato rescindido e não pagaria mais nada.

--- Ainda não vi comitê algum --- disse Mrs.~Kearney com indignação.  --- Minha
filha tem um contrato.  Ou recebe as oito guinéus ou não põe o pé naquele
palco.

--- A senhora me causa espanto, Mrs.~Kearney --- disse Mr.~Holohan.  --- Nunca
pensei que nos trataria dessa maneira.

--- E a maneira que os senhores me trataram? --- perguntou Mrs.~Kearney.

De tanta raiva seu rosto estava ruborizado e ela parecia prestes a agredir
alguém.

--- Exijo meus direitos --- ela disse.

--- A senhora deveria agir com um pouco de decência --- disse Mr.~Holohan.

--- Ah, deveria mesmo?\ldots{}  E por que não recebo uma resposta decente ao
perguntar quando minha filha será paga?

Atirou a cabeça para trás e falou com um tom de voz desafiador:

--- Procurem os secretários.  Não tenho nada a ver com isso.  Por tudo o que
fiz eu deveria exigir até mais.

--- Pensei que a senhora fosse uma dama --- disse Mr.~Holohan, afastando"-se
bruscamente.

Dali em diante, a conduta de Mrs.~Kearney foi criticada por todos: todos
concordaram com a atitude do comitê.  Ela ficou parada próximo à porta, pálida
de ódio, gesticulando e discutindo com o marido e a filha.  Esperou até chegar
o momento do início da segunda parte na esperança de que os secretários a
procurassem.  Mas Miss Healy gentilmente concordara em acompanhar uma ou duas
peças.  Mrs.~Kearney foi obrigada a dar passagem ao barítono e à nova pianista
que se encaminhavam para o palco.  Ficou imóvel por um instante como um irado
ícone de pedra e, quando as primeiras notas da canção lhe feriram os ouvidos,
agarrou o xale da filha e disse ao marido:

--- Chame um coche!

Ele partiu imediatamente.  Mrs.~Kearney envolveu a filha com o xale e o seguiu.
Ao passar pela porta deteve"-se e olhou firme nos olhos de Mr.~Holohan.

--- O senhor ainda vai se ver comigo --- ela disse.

--- Mas eu não quero mais nem ver a senhora --- disse Mr.~Holohan.

Kathleen seguiu timidamente a mãe.  Mr.~Holohan pôs"-se a andar de um lado ao
outro no camarim, para ver se conseguia refrescar"-se pois sentia a pele arder
em chamas.

--- Que senhora de classe! --- ele dizia.  --- Ah, que senhora de classe!

--- Você agiu bem, Holohan --- disse Mr.~O’Madden Burke, apoiando"-se no
guarda"-chuva em sinal de aprovação.


\chapter{Graça}
\hedramarkboth{Graça}{James Joyce}

\textsc{Dois cavalheiros} que estavam no banheiro naquele momento tentaram
levantá"-lo: mas ele estava inerte.  Jazia encolhido ao pé da escada na qual
havia tombado.  Conseguiram virá"-lo de barriga para cima.  O chapéu tinha
rolado a alguns metros de distância e sua roupa estava lambuzada com a imundice
do chão onde caíra de cara.  Estava de olhos fechados e ao respirar emitia um
ruído como um ronco.  Um filete de sangue escorria"-lhe do canto da boca.

Os dois cavalheiros e um \textit{barman} carregaram"-no escada acima e
deitaram"-no novamente no chão do bar.  Em poucos minutos ele estava cercado por
uma roda de homens.  O gerente do bar perguntou em voz alta quem ele era e quem
o acompanhava.  Ninguém sabia quem ele era mas um \textit{barman} disse que
tinha lhe servido uma pequena dose de rum.

--- Ele estava sozinho? --- perguntou o gerente.

--- Não, senhor.  Tinha dois cavalheiros com ele.

--- E onde eles estão?

Ninguém sabia; uma voz disse:

--- Deixem ele respirar.  Está desmaiado.

A roda de curiosos abriu"-se e voltou a se fechar como um elástico.  Uma escura
medalha de sangue tinha se formado no mosaico do assoalho próximo à cabeça do
homem.  O gerente, assustado com a palidez cinzenta estampada no rosto do
sujeito, mandou chamar a polícia.

Desabotoaram"-lhe o colarinho e desfizeram"-lhe o nó da gravata.  Ele abriu os
olhos por um instante, suspirou e voltou a cerrá"-los.  Um dos cavalheiros que o
haviam carregado escada acima trazia na mão um chapéu de seda amassado.  O
gerente continuava a perguntar se alguém conhecia o ferido ou sabia aonde
tinham ido os amigos que o acompanhavam.  A porta do bar se abriu e um policial
gigantesco entrou.  A pequena multidão que o seguira pelo beco detivera"-se do
lado de fora da porta, acotovelando"-se para olhar através do vidro.

O gerente começou imediatamente a relatar o que sabia.  O policial, um jovem
com traços faciais marcados e rígidos, ouviu o relato.  Movia a cabeça
lentamente para a direita e para a esquerda, e em direção ao gerente e ao
indivíduo estendido no assoalho, como se temesse estar sendo vítima de alguma
alucinação.  Em seguida, tirou as luvas, pegou uma caderneta que trazia presa à
cintura, molhou a ponta do lápis na língua e preparou"-se para fazer o
interrogatório.  Perguntou com um sotaque desconfiado, de provinciano:

--- Quem é este homem?  Qual o seu nome e endereço?

Um jovem vestido com traje de ciclista abriu caminho na roda de espectadores.
Ajoelhou"-se ao lado do ferido e pediu água.  O policial também ajoelhou"-se,
para ajudar.  O jovem limpou o sangue que escorria da boca do ferido e pediu um
pouco de \textit{brandy}.  O policial repetiu o pedido com um tom de voz
autoritário até que um \textit{barman} chegou correndo com um copo.  Enfiaram o
\textit{brandy} pela goela do homem.  Dentro de poucos segundos ele abriu os
olhos e olhou em volta de si mesmo.  Contemplou a roda de caras e então,
compreendendo o que havia se passado, tentou ficar de pé.

--- O senhor está melhor agora? --- perguntou o jovem com roupa de ciclista.

--- Não \ldots{} oi \ldots{} ada --- disse o ferido, tentando levantar"-se.

Ajudaram"-no a se levantar.  O gerente disse alguma coisa a respeito de hospital
e alguns circunstantes deram palpites.  Colocaram na cabeça do homem o chapéu
de seda amarfanhado.  O policial perguntou:

--- Onde o senhor mora?

O homem, sem responder, começou a enroscar a ponta do bigode.  Tentou minimizar
o acidente.  Aquilo fora coisa à toa, ele disse: apenas um pequeno acidente.
Falava com uma voz engrolada.

--- Onde o senhor mora? --- repetiu o policial.

O homem pediu que chamassem uma charrete.  Enquanto o pedido estava em
discussão, um cavalheiro de pele clara, alto e ágil, vestindo um sobretudo
amarelo, aproximou"-se, vindo do fundo do bar.  Ao ver a cena, gritou:

--- Ei, Tom!  Qual foi o problema, meu velho?

--- Não \ldots{} oi \ldots{} ada --- disse o homem.

O recém"-chegado examinou a figura deplorável que tinha diante de si e então
virou"-se para o policial, dizendo:

--- Pode deixar, seu guarda.  Eu o levo pra casa.

O policial elevou a mão ao capacete e respondeu:

--- Perfeitamente, Mr.~Power!

--- Vamos, Tom --- disse Mr.~Power, segurando o amigo pelo braço.  --- Não
quebrou nada.  Hein?  Pode andar?

O jovem ciclista segurou o outro braço do homem e o círculo de pessoas se
abriu.

--- Em que enrascada você se meteu? --- perguntou Mr.~Power.

--- O cavalheiro rolou escada abaixo --- disse o jovem.

--- sssou"-lhe \ldots{} uito \ldots{} gradecido, \ldots{} me"-meu \ldots{} ovem ---
disse o inválido.

--- Não há de quê.

--- \ldots{} amos \ldots{} omar um \ldots{} raguinho?

--- Agora não.  Agora não.

Os três homens saíram do bar e a pequena multidão escoou pela porta, seguindo
pelo beco.  O gerente levou o policial até a escada para este examinar a cena
do acidente.  Concluíram que o cavalheiro tinha pisado em falso.  Os presentes
voltaram para o balcão e um \textit{barman} começou a limpar os vestígios de
sangue no assoalho.

Quando chegaram em Grafton Street Mr.~Power assobiou para chamar uma charrete.
O inválido voltou a dizer da melhor maneira possível:

--- ssou"-lhe \ldots{} uito \ldots{} gradecido, \ldots{} mmeu \ldots{} ovem.
Gos\ldots{} aria de vol\ldots{} ar a \ldots{} ê"-lo.  Mmmeu \ldots{} ome é Kernan.

O susto e a dor fizeram com que ficasse um pouco sóbrio.

--- Não foi nada --- disse o jovem.

Trocaram um aperto de mão.  Kernan foi alçado para dentro da charrete e,
enquanto Mr.~Power dava instruções ao motorista, ele tornou a expressar sua
gratidão ao jovem e disse que lamentava não poderem tomar um trago juntos.

--- Fica pra outra vez --- disse o jovem.

A charrete partiu em direção a Westmoreland Street.  Quando passaram pelo
Ballast Office o relógio marcava nove e meia.  Um vento leste cortante soprava,
proveniente do estuário do rio.  Kernan estava todo encolhido de frio.  O amigo
perguntou"-lhe como tinha ocorrido o acidente.

--- \ldots{} ão \ldots{} osso \ldots{} alar, a\ldots{} igo --- ele respondeu.  ---
Ma\ldots{} uquei a \ldots{} íngua.

--- Deixe"-me ver.

O amigo esticou a cabeça por cima do banco da frente para olhar a boca de
Kernan mas nada pôde ver.  Acendeu um fósforo e, protegendo a chama com as mãos
em formato de concha, voltou a espiar a boca aberta por Kernan obedientemente.
O balanço da charrete fazia a chama aproximar"-se e distanciar"-se da boca
aberta.  Os dentes e a gengiva inferior estavam cobertos de sangue coagulado e
um minúsculo pedaço da língua tinha sido decepado.  O fósforo se apagou.

--- Está feio --- disse Mr.~Power.

--- \ldots{} ão \ldots{} oi \ldots{} ada --- disse Kernan, fechando a boca e
suspendendo a gola do paletó imundo para proteger a garganta.

Mr.~Kernan era caixeiro"-viajante da velha guarda e acreditava na dignidade da
profissão.  Jamais era visto na cidade sem um chapéu de seda decente e
polainas.  Graças a esses dois itens do vestuário, dizia ele, um indivíduo é
capaz de passar por qualquer inspeção.  Cultuava a memória de Napoleão, o
grande Blackwhite, cuja tradição evocava, ora imitando"-o, ora contando lendas a
seu respeito.  Quanto às práticas comerciais modernas, a única que observou foi
a instalação de um pequeno escritório em Crowe Street, em cuja janela se via o
nome e o endereço de sua firma --- London, \textsc{e.c.  }No console da lareira
do pequeno escritório via"-se um pequeno batalhão de latinhas de chumbo e sobre
a mesa diante da janela quatro ou cinco tigelas de louça que geralmente
continham um líquido negro.  Nessas tigelas Mr.~Kernan tomava chá.  Bebia um
gole, degustava e em seguida cuspia o chá na grelha da lareira.  Então punha"-se
a elucubrar.

Mr.~Power, bem mais jovem, trabalhava no Royal Irish Constabulary Office, a
polícia irlandesa, sediada em Dublin Castle.  A curva de sua ascensão social
interceptava a curva do declínio do amigo, mas o declínio de Mr.~Kernan era
atenuado pelo fato de que alguns dos amigos que o conheceram no auge do sucesso
ainda o estimavam.  Mr.~Power era um desses amigos.  Suas dívidas misteriosas
tinham se tornado célebres em seu círculo de amizades; era um jovem simpático.

A charrete parou em frente a uma casa modesta em Glasnevin Road e Mr.~Kernan
foi amparado até a porta.  A esposa colocou"-o na cama enquanto Mr.~Power, que
permanecera no andar de baixo, na cozinha, perguntava às crianças o nome da
escola que frequentavam e em que cartilha estavam.  As crianças, duas meninas e
um menino, cientes da prostração do pai e da ausência da mãe, começaram a pular
em volta dele.  Surpreso com os modos e o sotaque das crianças, Mr.~Power
franziu o cenho com ar pensativo.  Após algum tempo Mrs.~Kernan entrou na
cozinha, exclamando:

--- Que cena!  Ah, um dia desses ele vai se acabar, por Deus!  Está bebendo
desde sexta"-feira.

Mr.~Power teve o cuidado de deixar claro que não tinha nada a ver com aquilo,
que se deparara com a cena por mera coincidência.  Mrs.~Kernan, lembrando"-se
dos serviços de Mr.~Power na ocasião de desentendimentos domésticos, e dos
muitos empréstimos oportunos, disse:

--- Ah, não precisa dizer nada, Mr.~Power.  Sei que o senhor é amigo dele; não
é como aqueles outros que andam com ele e que não valem nada.  Só prestam mesmo
pra ficar do lado dele enquanto ele tem dinheiro no bolso e pra mantê"-lo longe
da mulher e da família.  Belos amigos!  Quem estava com ele hoje, o senhor pode
me dizer?

Mr.~Power sacudiu a cabeça e permaneceu calado.

--- Desculpe --- ela continuou ---, mas eu não tenho nada em casa pra
oferecer pro senhor.  Mas se o senhor esperar um minutinho eu mando buscar
qualquer coisa no Fogarty’s ali na esquina.

Mr.~Power levantou"-se.

--- A gente estava esperando que ele chegasse em casa com dinheiro.  Parece que
ele nunca se lembra que tem casa.

--- Ah, que é isso, Mrs.~Kernan! --- disse Mr.~Power.  --- Vamos fazer com que
ele comece uma vida nova.  Vou falar com o Martin.  Ele vai ajudar.  Qualquer
noite dessas passaremos por aqui para conversar.

Ela o acompanhou até a porta.  O charreteiro andava de um lado para o outro na
calçada, sacudindo os braços para manter"-se aquecido.

--- Foi muita bondade do senhor trazer ele em casa --- ela disse.

--- Não foi nada --- disse Mr.~Power.

Mr.~Power entrou na charrete.  Quando o veículo deu a partida ele tirou o
chapéu e cumprimentou"-a com um ar de jovialidade.

--- Vamos fazer dele um novo homem --- ele disse.  --- Boa"-noite, Mrs.~Kernan.

\smallskip

\noindent\dotfill

\smallskip

Os olhos perplexos de Mrs.~Kernan observaram a charrete até que esta
desapareceu.  Então desviou o olhar, entrou em casa e esvaziou os bolsos do
marido.

Era uma mulher de meia"-idade, ativa e prática.  Não fazia muito tinha celebrado
suas bodas de prata e se reaproximado do marido, dançando com ele na ocasião
uma valsa tocada por Mr.~Power.  Nos tempos de namoro achara que Mr.~Kernan
tinha lá o seu charme: e ainda hoje ela corria até a porta de uma capela sempre
que havia casamento e, ao ver o casal de nubentes, lembrava"-se com prazer do
dia em que saíra da Star of the Sea Church, em Sandymount, apoiada no braço de
um homem alegre e robusto, elegantemente vestido de sobrecasaca e calça lilás e
equilibrando charmosamente no outro braço um chapéu de seda.  Depois de três
anos já achava a vida de esposa algo cansativa e, mais tarde, quando passou a
achá"-la insuportável, tornara"-se mãe.  O papel de mãe não lhe trouxe
dificuldades insuperáveis e durante vinte e cinco anos cuidou do lar para o
esposo com eficiência.  Os dois filhos mais velhos estavam bem encaminhados.
Um trabalhava numa loja de tecidos em Glasgow e o outro era empregado de um
comerciante de chá em Belfast.  Eram bons filhos; escreviam com frequência e às
vezes mandavam dinheiro para casa.  Os outros filhos ainda estavam em idade
escolar.

No dia seguinte Mr.~Kernan mandou uma mensagem para o escritório e ficou
acamado.  A mulher preparou"-lhe um caldo de carne e repreendeu"-o severamente.
Tolerava os pileques frequentes do marido como quem aceita variações
climáticas, cuidava dele zelosamente nessas ocasiões e sempre insistia para que
tomasse um bom café da manhã.  Havia maridos piores.  Depois que os meninos
cresceram ele jamais fora violento, e ela sabia que ele era capaz de subir e
descer toda a Thomas Street para concluir uma venda, por menor que fosse.

Duas noites depois, os amigos vieram visitá"-lo.  Ela os conduziu escada acima
até o quarto, cujo ar estava impregnado de cheiro de suor, e fez com que se
sentassem perto da lareira.  A língua de Mr.~Kernan, que durante o dia
deixava"-o irritado devido a umas fisgadas, tornara"-se menos afiada.  Estava
sentado na cama recostado nos travesseiros e o leve rubor visível em suas
bochechas inchadas fazia com que estas parecessem brasas cobertas de cinzas.
Desculpou"-se com os amigos pela desordem do quarto, mas ao mesmo tempo olhava
para eles com certo orgulho, orgulho de veterano.

Não suspeitava estar sendo vítima de uma conspiração que os amigos, 
Mr.~Cunningham, Mr.~M’Coy e Mr.~Power, tinham revelado a Mrs.~Kernan na sala de
visitas.  A ideia partira de Mr.~Power mas o planejamento fora confiado a 
Mr.~Cunningham.  Mr.~Kernan era de origem protestante e, embora tivesse se
convertido ao catolicismo por ocasião de seu casamento, fazia vinte anos que
não pisava numa igreja.  Além disso, gostava de dar umas alfinetadas no
catolicismo.

Mr.~Cunningham era o homem perfeito para esse tipo de caso.  Era colega de 
Mr.~Power, embora mais velho.  Sua vida doméstica não era das mais felizes.  As
pessoas sentiam por ele grande simpatia, pois sabiam que havia se casado com
uma mulher desclassificada e alcoólatra incurável.  Montara casa para ela seis
vezes; e nas seis vezes ela empenhara a mobília.

Todo mundo respeitava o pobre Martin Cunningham.  Era um homem por demais
sensato, influente e perspicaz.  Seu conhecimento da natureza humana,
decorrente de uma astúcia inata reforçada através de longos anos no trato de
casos policiais, tinha sido enriquecido por breves imersões nas águas da
filosofia geral.  Era bem"-informado.  Os amigos acatavam"-lhe as opiniões e
achavam seu rosto parecido com o de Shakespeare.

Quando o plano lhe foi revelado, Mrs.~Kernan disse:

--- Entrego tudo nas mãos do senhor, Mr.~Cunningham.

Após um quarto de século de vida conjugal restavam a ela poucas ilusões.
Religião para ela era uma questão de hábito, e achava que um homem da idade de
seu marido não haveria de se modificar muito antes da morte.  Sentia"-se
inclinada a ver no acidente sofrido por ele certo merecimento e, se não
receasse parecer sanguinária, teria dito àqueles cavalheiros que a língua do
marido nada perdera por ter sido encurtada.  Contudo, Mr.~Cunningham sabia o
que estava fazendo; e religião é religião.  O esquema talvez surtisse efeito e,
ao menos, mal não faria.  Sua fé nada tinha de extravagante.  Acreditava
firmemente no Sagrado Coração como a devoção católica de maior utilidade e
aceitava os sacramentos.  Sua fé não transcendia os limites da cozinha e,
dependendo da necessidade, podia crer tanto na Fada da Morte quanto no Espírito
Santo.

Os cavalheiros começaram a conversar a respeito do acidente.  Mr.~Cunningham
disse que tinha conhecimento de um caso semelhante.  Um senhor de setenta anos
de idade mordera um pedaço da língua durante um ataque epiléptico e a língua
voltara a crescer, de modo que a mordida não deixara o menor vestígio.

--- Sei, mas, eu não tenho setenta anos --- disse o inválido.

--- Deus o livre --- disse Mr.~Cunningham.

--- Não está doendo agora? --- perguntou Mr.~M’Coy.

Mr.~M’Coy tinha sido um tenor de relativa fama.  A esposa, que fora soprano,
ainda dava aulas de piano para crianças a preços módicos.  Sua existência não
tinha sido traçada como uma linha reta entre dois pontos e em alguns momentos
fora levado a viver de expedientes.  Tinha sido funcionário da Midland Railway,
corretor de classificados para o \textit{Irish Times} e para o
\textit{Freeman’s Journal}, caixeiro"-viajante contratado sob comissão por uma
mineradora de carvão, detetive particular, escrivão no gabinete do subdelegado
e recentemente tinha sido nomeado secretário do médico legista municipal.  Essa
nova função fez despertar um interesse profissional no caso de Mr.~Kernan.

--- Dor?  Não muita --- respondeu Mr.~Kernan.  --- Mas estou sentindo enjoo.
Tenho vontade de vomitar.

--- É efeito da bebida --- disse Mr.~Cunningham com firmeza.

--- Não --- disse Mr.~Kernan.  --- Acho que peguei um resfriado na charrete.
Tem alguma coisa que agarra na minha garganta, catarro ou\ldots{}

--- Muco --- disse Mr.~M’Coy.

--- Fica vindo do fundo da garganta, coisa nojenta.

--- Pois é, pois é --- disse Mr.~M’Coy ---, vem do tórax.

Ele olhou ao mesmo tempo para Mr.~Cunningham e Mr.~Power com um ar de desafio.
Mr.~Cunningham fez com a cabeça um rápido sinal de consentimento e Mr.~Power
disse:

--- Ah, tudo está bem quando bem acaba.

--- Sou muito grato a você, meu amigo --- disse o inválido.  

Mr.~Power fez um gesto com a mão.

--- Aqueles dois que estavam comigo\ldots{}

--- Quem estava com você? --- perguntou Mr.~Cunningham.

--- Um sujeito.  Não sei o nome dele.  Que diabo!  Como é mesmo o nome dele?
Sujeito baixinho de cabelo ruivo\ldots{}

--- E quem mais?

--- Harford.

--- Humm --- fez Mr.~Cunningham.

Quando Mr.~Cunningham fez esse comentário, os demais ficaram em silêncio.
Sabia"-se que ele dispunha de fontes secretas de informação.  Nesse caso o
monossílabo tinha um sentido moral.  Mr.~Harford às vezes integrava um pequeno
grupo que saía da cidade aos domingos logo depois do meio"-dia com o propósito
de chegar o mais rápido possível a algum bar da periferia onde eram confundidos
com viajantes.  Mas os companheiros de viagem jamais esqueciam a sua origem.
Começara a vida como um agiota desconhecido que emprestava a operários pequenas
somas a juros extorsivos.  Mais tarde tornara"-se sócio de um sujeito baixo e
gordo, Mr.~Goldberg, do Liffey Loan Bank.  Embora se limitasse a seguir o
código de ética judaico, seus confrades católicos, sempre que sofriam direta ou
indiretamente a pressão de suas cobranças, referiam"-se a ele amargamente
chamando"-o de judeu irlandês e de analfabeto, e interpretavam como desaprovação
divina à usura a manifestação de debilidade mental em seu filho.  Em outras
ocasiões lembravam"-se do que nele havia de bom.

--- Onde será que ele foi parar? --- disse Mr.~Kernan.

Ele queria que os detalhes do acidente permanecessem vagos.  Queria que os
amigos pensassem que tinha ocorrido algum mal"-entendido, que Mr.~Harford e ele
tinham se desencontrado.  Os amigos, que conheciam de perto os modos de 
Mr.~Harford quando bebia, ficaram calados.  Mr.~Power voltou a dizer:

--- Tudo está bem quando bem acaba.

Mr.~Kernan mudou de assunto imediatamente.

--- Que rapaz decente, aquele estudante de medicina --- ele disse.  --- Se não
fosse ele\ldots{}

--- Ah, senão fosse ele --- disse Mr.~Power --- o caso poderia ter acabado em
sete dias de cadeia sem direito a fiança.

--- É mesmo --- disse Mr.~Kernan, tentando lembrar"-se dos fatos.  --- Lembro"-me
que tinha um policial.  Rapaz decente, pelo menos parecia.  Como foi que a
coisa aconteceu?

--- O que aconteceu foi que você estava de cara cheia, Tom --- disse 
Mr.~Cunningham com ar circunspecto.

--- Afirmação verdadeira --- disse Mr.~Kernan, igualmente circunspecto.

--- Suponho que você tenha colocado o guarda no seu devido lugar, Jack ---
disse Mr.~M’Coy.

Mr.~Power não gostava de ser chamado pelo primeiro nome.  Não era mesquinho,
mas não podia se esquecer de que Mr.~M’Coy recentemente empreendera uma cruzada
em busca de valises e baús para que Mrs.~M’Coy pudesse honrar compromissos
imaginários no interior.  Mais do que do fato de ter sido ludibriado,
ressentia"-se do jogo baixo.  Portanto, respondeu a pergunta como se fosse 
Mr.~Kernan quem a formulara.

O relato deixou Mr.~Kernan indignado.  Era muito cioso de sua cidadania,
pretendia ter com a cidade um relacionamento mutuamente honrado e não admitia
afrontas feitas por indivíduos que a seu ver não passavam de caipiras.

--- É pra isso que pagamos impostos? --- perguntou.  --- Pra alimentar e vestir
esses ignorantes\ldots{} e eles não passam mesmo disso.

Mr.~Cunningham riu.  Era funcionário do governo apenas no horário do
expediente.

--- O que mais haveriam de ser, Tom? --- ele disse.

Imitando um forte sotaque do interior, disse em tom de comando:

--- Número 65, pegue o seu repolho!

Todos riram.  Mr.~M’Coy, que desejava a todo custo participar da conversa,
fingiu desconhecer a história.  Mr.~Cunningham disse:

--- Segundo dizem, o incidente se passou no posto de treinamento de recrutas
para onde levam aqueles brutamontes que vêm da roça, aqueles bobocas, entendem?
O sargento deu ordem pra eles perfilarem contra a parede e levantarem seus
pratos.  

Ele encenava a história com gestos grotescos.

--- Na hora do jantar, entendem?  Então ele tinha diante de si em cima da mesa
uma bacia de repolho cozido e uma colher enorme que mais parecia uma pá.  Com
uma colherzona ele pegava dentro da bacia um um monte de repolho e atirava o
repolho para o outro lado da sala e os infelizes tinham de aparar o repolho com
seus pratos: \textit{Número 65, pegue o seu repolho!}

Todos riram novamente: mas Mr.~Kernan continuava bastante indignado.  Falava em
escrever uma carta para os jornais.

--- Esses selvagens vêm pra cá --- disse ele --- e acham que podem mandar em
todo mundo.  Eu nem preciso te dizer, Martin, que tipo de gente é.

Em silêncio, Mr.~Cunningham indicou que concordava.

--- É como tudo nesse mundo --- ele disse.  --- Tem gente boa e gente má.

--- Ah, isso é verdade, tem gente boa, sim, admito --- disse Mr.~Kernan,
satisfeito.

--- A melhor coisa é não se meter com eles --- disse Mr.~M’Coy. --- É o que eu
acho!

Mrs.~Kernan entrou no quarto e, colocando uma bandeja sobre a mesa, disse:

--- Sirvam"-se, cavalheiros.

Mr.~Power levantou"-se para servir, cedendo seu lugar à mulher.  Ela recusou,
dizendo que estava passando roupa no andar de baixo, e, após fazer pelas costas
de Mr.~Power um sinal para Mr.~Cunningham, encaminhou"-se para a porta do
quarto.  O marido chamou"-a:

--- E pra mim, você não traz nada, querida?

--- Pra você!  Um tapa na cara! --- disse Mrs.~Kernan com aspereza.

O marido voltou a chamá"-la:

--- Nada pro seu pobre maridinho?

Ele fez uma cara tão cômica e falou com uma voz tão ridícula que a distribuição
de garrafas de cerveja ocorreu em meio a muito riso.

Os cavalheiros beberam, colocaram os copos novamente na mesa e fizeram uma
pausa.  Então Mr.~Cunningham virou"-se para Mr.~Power e disse com naturalidade:

--- Quinta"-feira à noite, foi isso que você disse, Jack?

--- Quinta"-feira, isso mesmo --- disse Mr.~Power.

--- Combinado! --- disse Mr.~Cunningham prontamente.

--- Podemos nos encontrar no M’Auley’s --- disse Mr.~M’Coy.  --- Acho que é o
local mais conveniente.

--- Mas não podemos nos atrasar --- disse Mr.~Power em tom sério --- porque com
certeza vai estar superlotado.

--- Podemos nos encontrar às sete e meia --- disse Mr.~M’Coy.

--- Combinado! --- disse Mr.~Cunningham.

--- Certo, às sete e meia no M’Auley’s!

Houve um breve de silêncio.  Mr.~Kernan aguardou para ver se os amigos lhe
confiariam o assunto.  Então, perguntou:

--- Tem alguma coisa no ar?

--- Ah, não é nada --- disse Mr.~Cunningham.  --- É só um negócio que a gente
está combinando pra quinta"-feira.

--- Vão à ópera, não é? --- disse Mr.~Kernan.

--- Não, não --- disse Mr.~Cunningham em tom de evasiva ---, é só um\ldots{}
assunto espiritual.

--- Ah --- disse Mr.~Kernan.

Houve outro momento de silêncio.  Então Mr.~Power disse, sem rodeios:

--- Pra ser sincero, Tom, nós vamos fazer um retiro.

--- É isso mesmo --- disse Mr.~Cunningham --- Jack e eu e o M’Coy\ldots{} vamos
lavar a alma.

Pronunciou a metáfora com um vigor simplório e, incentivado por sua própria
voz, prosseguiu:

--- Sabe, temos que admitir que somos um bando de salafrários, todos nós.  Isso
mesmo, todos nós --- acrescentou com uma generosidade um tanto rude e
dirigindo"-se a Mr.~Power, disse:

--- Confesse!

--- Eu confesso --- disse Mr.~Power.

--- E eu confesso --- disse Mr.~M’Coy.

--- De modo que vamos juntos lavar a alma --- disse Mr.~Cunningham.

Uma ideia ocorreu"-lhe.  Virou"-se para o inválido e disse:

--- Sabe que acabei de ter uma ideia, Tom?  Se você for conosco, fecharemos
duas parcerias.

--- Boa ideia --- disse Mr.~Power.  --- Nós quatro juntos.  

Mr.~Kernan manteve"-se calado.  A proposta tinha para ele pouco sentido mas,
percebendo que algumas medidas de caráter espiritual estavam prestes a ser
tomadas em seu nome, achou que por uma questão de dignidade deveria resistir ao
convite.  Permaneceu fora da conversa durante um bom tempo, apenas ouvindo, com
um sereno ar de oposição, enquanto os amigos falavam dos jesuítas.

--- A visão que tenho dos jesuítas não é das mais negativas --- ele disse,
intervindo finalmente.  --- Têm uma boa educação.  E acho que são
bem"-intencionados.

--- É a maior ordem da Igreja, Tom --- disse Mr.~Cunningham, com
entusiasmo.  --- O general que comanda a Sociedade dos Jesuítas vem logo depois
do papa.

--- Não resta a menor dúvida --- disse Mr.~M’Coy.  --- Se a pessoa quiser
eficiência sem perda de tempo deve procurar um jesuíta.  Eles têm muita
influência.  Vou contar um caso pra vocês\ldots{}

--- Os jesuítas são homens excelentes --- disse Mr.~Power.

--- Há um fato curioso --- disse Mr.~Cunningham --- com relação à ordem
jesuíta.  Todas as outras ordens da Igreja passaram por algum tipo de reforma
mas a ordem jesuíta nunca precisou de reforma.  Nunca entrou em decadência.

--- É mesmo? --- perguntou Mr.~M’Coy.

--- É um fato --- disse Mr.~Cunningham.  --- É História.

--- Vejam os seus templos --- disse Mr.~Power.  --- Vejam as congregações que
eles têm.

--- Os jesuítas atendem principalmente às elites --- disse Mr.~M’Coy.

--- É claro --- disse Mr.~Power.

--- É mesmo --- disse Mr.~Kernan.  --- É por isso que tenho simpatia por
eles.  São aqueles mais pra leigos que pra padres, broncos, autoritários
que\ldots{}

--- São todos bons sujeitos --- disse Mr.~Cunningham ---, cada um à sua
maneira.  O clero irlandês é respeitado no mundo inteiro.

--- Ah, é mesmo --- disse Mr.~Power.

--- Ao contrário do que se observa em certos cleros do continente europeu ---
disse Mr.~M’Coy ---, indignos do nome.

--- Talvez você tenha razão --- disse Mr.~Kernan, capitulando.

--- É claro que tenho razão --- disse Mr.~Cunningham.  --- Já vivi o suficiente
neste mundo pra poder avaliar o caráter de alguém.

Os cavalheiros voltaram a beber, cada um seguindo o exemplo do outro.  
Mr.~Kernan parecia estar analisando mentalmente a situação.  Ficara impressionado.
Reconhecia a capacidade que Mr.~Cunningham tinha de avaliar o caráter das
pessoas bem como de interpretar fisionomias.  Pediu maiores detalhes.

--- Ah, é só um retiro --- disse Mr.~Cunningham.  --- O padre Purdon é quem vai
liderar.  É pra homens de negócio, você entende.

--- Ele não vai ser muito duro com a gente, Tom --- disse Mr.~Power, tentando
persuadi"-lo.

--- Padre Purdon?  Padre Purdon? --- indagou o inválido.

--- Ah, decerto você o conhece, Tom --- disse Mr.~Cunningham, com firmeza.  ---
É um sujeito muito bem"-humorado!  Tem os pés na terra como nós.

--- Ah\ldots{} sim.  Acho que sei quem é.  Rosto vermelho; alto.

--- É esse mesmo.

--- E Martin, diga"-me uma coisa\ldots{} Ele é um bom pregador?

--- Bem\ldots{} Não se trata propriamente de pregação, você sabe.  É apenas uma
conversa franca entre amigos, um apelo ao bom senso, entende?

Mr.~Kernan deliberava.  Mr.~M’Coy disse:

--- O padre Tom Burke, aquele, sim!

--- Ah, o padre Tom Burke --- disse Mr.~Cunningham ---, aquele era um orador
nato: Você chegou a ouvi"-lo, Tom?

--- Se cheguei a ouvi"-lo! --- disse o inválido, reagindo.  --- Claro que ouvi!
Foi\ldots{}

--- E há quem diga que ele não era um grande teólogo --- disse Mr.~Cunningham.

--- É mesmo? --- disse Mr.~M’Coy.

--- Ah, nada de grave, é claro.  Dizem apenas que às vezes as pregações dele
não eram das mais ortodoxas.

--- Ah!\ldots{} era um homem extraordinário --- disse Mr.~M’Coy.

--- Certa ocasião ouvi ele falar --- prosseguiu Mr.~Kernan.  --- Não me recordo
agora o assunto do sermão.  Crofton e eu estávamos sentados no fundo da\ldots{}
assembleia\ldots{} vocês sabem\ldots{} da\ldots{}

--- Nave --- disse Mr.~Cunningham.

--- Isso mesmo, no fundo, perto da porta.  Não me recordo agora o\ldots{} Ah,
sim, era sobre o papa, o papa que morreu.  Agora me lembro.  Palavra de honra,
foi magnífico, o estilo de oratória.  E a voz!  Meu Deus!  Que voz ele tinha!
“Prisioneiro do Vaticano”, assim ele se referiu ao papa.  Lembro"-me do que
Crofton me disse quando saímos\ldots{}

--- Mas o Crofton é protestante, não é? --- disse Mr.~Power.

--- Claro que é --- disse Mr.~Kernan --- protestante e dos bons.  Entramos no
Butler’s em Moore Street\ldots{} juro, eu fiquei muito comovido, a verdade tem
de ser dita\ldots{} e lembro"-me bem das palavras dele: \textit{Kernan}, ele
disse, \textit{rezamos em altares diferentes, mas nossa fé é a mesma}.
E o disse muito bem.

--- Tem muita verdade nessas palavras --- disse Mr.~Power.  --- A capela ficava
lotada de protestantes quando o padre Tom fazia suas pregações.

--- Entre nós não há tantas diferenças --- disse Mr.~M’Coy.  --- Todos
acreditamos no\ldots{}

Hesitou um momento.

--- \ldots{} no Redentor.  A diferença é que eles não acreditam no papa nem na
Mãe de Deus.

--- Mas, é claro --- disse Mr.~Cunningham, com serenidade e convicção --- que a
nossa religião é \textit{a} religião, a fé antiga, a fé original.

--- Quanto a isso não resta dúvida --- disse Mr.~Kernan, gentilmente.

Mrs.~Kernan chegou à porta do quarto e anunciou:

--- Tem visita pra você!

--- Quem é?

--- Mr.~Fogarty.

--- Ah, entre!  Entre!

Um rosto pálido e ovalado avançou em direção à luz.  O arco formado pelo bigode
claro era duplicado pelas sobrancelhas claras, arqueadas sobre um olhar
surpreso e agradável.  Mr.~Fogarty era um modesto dono de mercearia.  Tinha ido
à falência com um bar na cidade porque restrições financeiras obrigaram"-no a
trabalhar apenas com cervejarias e destilarias de segunda classe.  Abrira então
uma pequena mercearia em Glasnevin Road onde, segundo ele próprio se gabava,
suas boas maneiras conquistariam a simpatia das donas de casa do bairro.  Tinha
uma certa elegância, costumava cumprimentar as crianças e falava corretamente.
Não era totalmente desprovido de cultura.

Mr.~Fogarty trouxera um presente, uma garrafinha de um uísque especial.
Perguntou educadamente a respeito da saúde de Mr.~Kernan, colocou a lembrança
sobre a mesa e sentou"-se entre os pares.  Mr.~Kernan gostou muito do presente
sobretudo porque sabia que sua conta no armazém de Mr.~Fogarty não estava
zerada.  Ele disse:

--- Nunca duvidei de você, amigo!  Quer abrir a garrafa, Jack, por favor?

Mr.~Power mais uma vez preparou"-se para servir.  Os copos foram lavados e cinco
pequenas doses de uísque foram servidas.  A conversa ficou ainda mais animada.
Mr.~Fogarty, sentado na ponta da cadeira, parecia bastante interessado.

--- O papa Leão \textsc{xiii} --- disse Mr.~Cunningham --- foi um dos luminares da era.
Seu grande ideal, como vocês sabem, era a união das Igrejas romana e grega.
Foi o objetivo de sua vida.

--- Sempre ouvi dizer que era um dos maiores intelectuais da Europa --- disse
Mr.~Power.  --- Isto é, além de ser papa.

--- E era mesmo --- disse Mr.~Cunningham ---, e talvez fosse \textit{o maior}.
Seu lema, como vocês sabem, enquanto papa, era \textit{Lux super Lux}, Luz
sobre Luz.

--- Não, não --- disse Mr.~Fogarty um tanto afoito.  --- Acho que você está
enganado.  Era \textit{Lux in Tenebris}, creio eu, Luz nas Trevas.

--- Ah, é --- disse Mr.~M’Coy ---, \textit{Tenebrae}.

--- Protesto --- disse Mr.~Cunningham com firmeza ---, era \textit{Lux super
Lux}.  E o lema de Pio \textsc{ix}, seu predecessor, era \textit{Crux super Crux}, ou
seja, Cruz sobre Cruz, para diferenciar os dois pontificados.

O protesto foi concedido.  Mr.~Cunningham continuou.

--- Leão \textsc{xiii}, como vocês sabem, era um grande erudito e poeta.

--- Tinha um rosto enérgico --- disse Mr.~Kernan.

--- É verdade --- prosseguiu Mr.~Cunningham.  --- Escrevia poesia em latim.

--- É mesmo? --- disse Mr.~Fogarty.

Mr.~M’Coy saboreou o uísque com satisfação e balançou a cabeça, expressando
dupla intenção, dizendo:

--- Não é piada, não.  Acreditem.

--- Não nos ensinaram nada disso na escola pública, Tom --- disse Mr.~Power,
seguindo o exemplo de Mr.~M’Coy.

--- Muito homem de valor frequentou escola pública levando debaixo do braço a
sua cota de carvão --- disse Mr.~Kernan com ar solene.  --- O sistema antigo
era bem melhor: educação simples e honesta.  Nada dessas baboseiras
modernas\ldots{}

--- Exatamente --- disse Mr.~Power.

--- Nada de superficialidades --- disse Mr.~Fogarty.

Enunciou a palavra e tomou um gole da bebida com ar sério.

--- Li em algum lugar --- disse Mr.~Cunningham --- que Leão \textsc{xiii} escreveu um
poema sobre a invenção da fotografia\ldots{} em latim, é claro.

--- Sobre a fotografia! --- exclamou Mr.~Kernan.

--- Isso mesmo --- disse Mr.~Cunningham.

E também tomou seu gole.

--- Bem, convenhamos --- disse Mr.~M’Coy ---, pensando bem, a fotografia não é
algo fantástico?

--- Ah, é claro --- disse Mr.~Power ---, as grandes mentes enxergam melhor.

--- Como dizia o poeta: \textit{As grandes mentes estão próximas da
loucura} --- disse Mr.~Fogarty.

A mente de Mr.~Kernan parecia estar confusa.  Esforçou"-se para se lembrar de
algum ponto controverso levantado pela teologia protestante e por fim
dirigiu"-se a Mr.~Cunningham.

--- Diga"-me uma coisa, Martin --- ele disse.  --- Não é verdade que alguns
papas\ldots{} não o atual, é claro, nem o antecessor, mas alguns dos
antigos\ldots{} não eram lá\ldots{} você sabe\ldots{} flor que se cheirasse?

Houve um momento de silêncio.  Mr.~Cunningham disse:

--- Ah, é claro, alguns foram um tanto problemáticos\ldots{} Mas a coisa mais
impressionante é a seguinte: nenhum deles, nem o mais alcoólatra, nem o
mais\ldots{} descarado mulherengo, nenhum mesmo, jamais proferiu \textit{ex
cathedra} uma só palavra de falsa doutrina.  Agora, isso não é impressionante?

--- É mesmo --- disse Mr.~Kernan.

--- Sim, porque quando fala \textit{ex cathedra} --- Mr.~Fogarty explicou ---
o papa é infalível.

--- Sim --- disse Mr.~Cunningham.

--- Ah, eu sei o que é a infalibilidade do papa.  Ainda me lembro.  Eu era bem
jovem quando\ldots{}  Foi isso que\ldots{}?

Mr.~Fogarty interrompeu"-o.  Pegou a garrafa e serviu mais uma rodada.  
Mr.~M’Coy, vendo que não haveria o suficiente para todos, disse que ainda não
terminara sua primeira dose.  Os outros aceitaram sob protesto.  A melodia
alegre do uísque caindo dentro dos copos constituiu um agradável interlúdio.

--- Do que você estava falando, Tom? --- perguntou Mr.~M’Coy.

--- Infalibilidade papal --- disse Mr.~Cunningham --- foi a maior questão de
toda a história da Igreja.

--- Qual foi a questão, Martin? --- perguntou Mr.~Power.

Mr.~Cunningham elevou dois dedos gordos.

--- No sacro colégio de bispos, arcebispos e cardeais, dois indivíduos se
opuseram à questão, enquanto todos os outros se declararam a favor.  À exceção
desses dois o conclave era unânime.  Mas esses dois não se rendiam!

--- Ah! --- disse Mr.~M’Coy.

--- E um deles era um cardeal alemão chamado Dolling\ldots{} ou Dowhing\ldots{}
ou\ldots{}

--- Se era Dowhing não era alemão, disso eu tenho certeza --- disse Mr.~Power,
rindo.

--- Bem, esse grande cardeal alemão, seja lá qual for o seu nome, era um dos
dois; e o outro foi John MacHale.

--- Quem? --- gritou Mr.~Kernan.  --- John de Tuam?

--- Você tem certeza? --- perguntou Mr.~Fogarty, duvidando.

--- Pensei que se tratava de algum italiano ou americano.

--- John de Tuam --- repetiu Mr.~Cunningham ---, o próprio.

Tomou novo gole e os outros o acompanharam.  Então recomeçou:

--- Lá estavam eles, cardeais, bispos e arcebispos, provenientes dos quatro
cantos da terra, e os dois lutando com unhas e dentes até que por fim o próprio
papa levantou"-se e \textit{ex cathedra} declarou a infalibilidade como dogma da
Igreja.  No mesmo instante, John MacHale, que estivera combatendo
incansavelmente a ideia, levantou"-se e gritou com uma voz de leão:
\textit{Credo!}

--- \textit{Eu creio!} --- disse Mr.~Fogarty.

--- \textit{Credo!} --- disse Mr.~Cunningham.  --- Foi uma demonstração de fé.
No instante em que o papa falou, ele se submeteu.

--- E o tal de Dowhing? --- perguntou Mr.~M’Coy.

--- O cardeal alemão não se submeteu.  Abandonou a Igreja.

As palavras de Mr.~Cunningham construíram na imaginação dos ouvintes uma imagem
imponente da Igreja.  Sua voz rouca e profunda emocionou"-os ao pronunciar a
palavra de fé e submissão.  Quando Mrs.~Kernan retornou ao quarto, secando as
mãos, encontrou o grupo sério.  Ela não perturbou o silêncio, apenas
debruçou"-se por cima do pé da cama.

--- Eu vi John MacHale --- disse Mr.~Kernan --- uma vez, e enquanto vida eu
tiver, jamais me esquecerei.

Dirigiu"-se à esposa em busca de confirmação.

--- Eu já te contei várias vezes, né?

Mrs.~Kernan consentiu.

--- Foi na inauguração da estátua de Sir John Gray.  Edmund Dwyer Gray estava
discursando, um monte de besteiras, e lá estava aquele senhor idoso, com cara
de velho ranzinza, fitando"-o por baixo de umas sobrancelhas peludas.

Mr.~Kernan franziu a testa e, abaixando a cabeça como um touro bravo, arregalou
os olhos para a esposa.

--- Deus do céu! --- ele exclamou, voltando a descontrair a testa.  --- Nunca
vi um ser humano com um olhar daqueles.  Era como se estivesse dizendo:
\textit{Você não me engana, meu rapaz}.  Tinha olhos de gavião.

--- Nenhum dos Gray prestava --- disse Mr.~Power.  

Houve nova pausa.  Mr.~Power virou"-se para Mrs.~Kernan e disse com uma
jovialidade abrupta:

--- Mrs.~Kernan, vamos fazer do seu esposo um católico devoto, piedoso, fiel e
temente a Deus.

Fez um gesto com o braço, indicando todo o grupo.

--- Vamos fazer um retiro juntos e confessar nossos pecados\ldots{} e só Deus
sabe o quanto precisamos de uma confissão.

--- Não me oponho --- disse Mr.~Kernan, sorrindo com um pouco de nervosismo.

Mrs.~Kernan achou conveniente disfarçar a satisfação que sentia.  Portanto,
disse:

--- Tenho pena do pobre padre que vai ter de ouvir as histórias de vocês.

A expressão facial de Mr.~Kernan mudou.

--- Se ele não gostar --- disse com aspereza ---, então que\ldots{} já sabe,
né?  Vou contar pra ele apenas algumas mazelas.  Não sou tão ruim assim\ldots{}

Mr.~Cunningham interveio prontamente.

--- Vamos todos renunciar a Satanás --- ele disse --- juntos, sem nos
esquecermos de suas artimanhas e suas pompas.

--- Arreda, Satanás! --- disse Mr.~Fogarty, rindo e olhando para os outros.

Mr.~Power manteve"-se calado.  Tinha plena consciência de que não era o líder do
grupo.  Mas um ar de satisfação cruzou"-lhe momentaneamente o rosto.

--- Tudo o que teremos de fazer --- disse Mr.~Cunningham --- é segurar velas
acesas e renovar nossas promessas do batismo.

--- Ah, não se esqueça da vela, Tom --- disse Mr.~M’Coy ---, de jeito nenhum.

--- O quê? --- disse Mr.~Kernan.  --- Vou ter de segurar vela?

--- Ah, vai --- disse Mr.~Cunningham.

--- Ora, que diabo! --- disse Mr.~Kernan, com sensatez.  --- Assim já é demais.
Faço qualquer negócio.  Faço o tal retiro e a confissão, e\ldots{} tudo o mais.
Mas\ldots{} nada de vela!  Ora, que diabo!  Vela não é comigo!

Sacudia a cabeça com uma expressão grave.

--- Vejam só! --- disse a esposa.

--- Vela não é comigo --- disse Mr.~Kernan, ciente do efeito que produzira nos
ouvintes e continuando a sacudir a cabeça em sinal de recusa.

--- Esse negócio de lanterna mágica não é comigo.  

Todos riram a valer.

--- Mas que católico de meia tigela vocês arrumaram! --- disse a esposa.

--- Nada de vela! --- repetiu Mr.~Kernan, obstinado.  --- Está dito!

\smallskip

\noindent\dotfill

\smallskip

O transepto da igreja jesuíta da Gardiner Street estava quase lotado; e ainda
assim homens continuavam a entrar pela porta lateral e, seguindo as indicações
de um irmão leigo, caminhavam na ponta dos pés pela nave até encontrarem lugar
para sentar.  Estavam todos bem vestidos e circunspectos.  A luz dos
candelabros da igreja refletia sobre uma assembleia de paletós negros e
colarinhos brancos, suavizada aqui e acolá pela presença de um \textit{tweed},
sobre as pilastras escuras de mármore verde e sobre os lúgubres reposteiros.
Os homens sentavam"-se nos bancos, franzindo ligeiramente as calças na altura
dos joelhos e mantinham os chapéus em local seguro.  Permaneciam com uma
postura ereta e olhavam com ar formal para o ponto de luz vermelha suspenso
diante do altar"-mor.

Em um dos bancos próximos ao púlpito estavam Mr.~Cunningham e Mr.~Kernan.  No
banco de trás sentava"-se Mr.~M’Coy: e logo atrás Mr.~Power e Mr.~Fogarty.  
Mr.~M’Coy tentara em vão conseguir um lugar no banco junto aos amigos e, quando o
grupo tinha se acomodado numa disposição quincuncial, ele tentou em vão fazer
piadinhas.  Visto que estas não foram bem recebidas, desistiu.  Até mesmo ele
se deu conta da atmosfera circunspecta e até mesmo ele passou a corresponder ao
espírito de religiosidade.  Com um sussurro Mr.~Cunningham indicou a Mr.~Kernan
a presença de Mr.~Harford, o agiota, sentado do outro lado, e de Mr.~Fanning,
oficial de cartório e grande benfeitor da cidade, sentado bem em frente ao
púlpito ao lado de um dos vereadores recém"-eleitos para a câmara.  À direita
deles estava o velho Michael Grimes, proprietário de três casas de penhor, e o
sobrinho de Dan Hogan, candidato a um emprego na Câmara de Vereadores.  Alguns
bancos à frente estava Mr.~Hendrick, redator"-chefe do \textit{Freeman’s
Journal}, e o pobre O’Carroll, velho amigo de Mr.~Kernan e, no passado,
importante figura do comércio local.  Aos poucos, à medida que identificava
rostos conhecidos, Mr.~Kernan começou a sentir"-se mais à vontade.  O chapéu,
reabilitado pela esposa, repousava em seus joelhos.  Uma ou duas vezes puxou os
punhos do paletó com uma das mãos, enquanto com a outra segurava de leve, mas
com firmeza, a aba do chapéu.

Uma figura imponente, cuja parte superior do corpo estava encoberta com uma
sobrepeliz, foi vista subindo com dificuldade a escada do púlpito.  Naquele
momento a congregação agitou"-se, pegando lenços e sobre eles ajoelhando com
todo cuidado.  Mr.~Kernan seguiu o exemplo.  Via"-se agora a figura ereta do
padre no interior do púlpito, com dois terços do volumoso corpo coroado por uma
face maciça e vermelha surgindo por cima da balaustrada.

Padre Purdon ajoelhou"-se, voltou"-se em direção ao ponto de luz vermelha e,
encobrindo o rosto com as mãos, pôs"-se a orar.  Passado algum tempo descobriu a
face e levantou"-se.  A congregação também se levantou e sentou"-se novamente nos
bancos.  Mr.~Kernan recolocou o chapéu sobre os joelhos e apresentou ao
pregador um rosto atento.  O pregador enrolou as mangas largas da sobrepeliz
com gestos estudados e correu os olhos lentamente pela multidão de rostos.
Então, disse:

\begin{quote}\itshape
\ldots{} Pois os filhos deste mundo são mais prudentes em sua
geração do que os filhos da luz.  E eu vos digo: fazei amigos com o
Mamon da iniquidade a fim de que, no dia em que morrerdes, eles vos
recebam nos tabernáculos eternos.
\end{quote}

Padre Purdon interpretou o texto com uma segurança que ressoou por toda a
igreja.  Era um dos trechos das Escrituras, disse ele, mais difíceis de serem
interpretados com correção.  Era um texto que, ao leitor desavisado, poderia
parecer ir de encontro à moralidade sublime pregada por Jesus Cristo em outras
passagens.  Mas, disse ele aos ouvintes, o trecho pareceu"-lhe especialmente
adequado para nortear aqueles cuja função é conduzir os negócios do mundo e que
não desejam fazê"-lo à maneira mundana.  Era o trecho apropriado para homens de
negócio e profissionais liberais.  Jesus Cristo, com seu conhecimento divino de
todas as facetas de nossa natureza humana, sabe que nem todos os homens recebem
o chamado para a vida religiosa, que a grande maioria é obrigada a viver no
mundo e, até certo ponto, para o mundo: nesse trecho, Ele pretende dar"-lhes uma
palavra de conselho, colocando diante delas como exemplos de vida religiosa os
próprios adoradores de Mamon, que entre todos os homens eram os menos convictos
em matéria de religião.

Disse aos ouvintes que não estava ali naquela noite com o propósito de
atemorizar ninguém, nada extravagante; mas como um homem do mundo falando a
seus pares.  Viera falar a homens de negócio e falaria à maneira empresarial.
Se lhe permitissem uma metáfora, disse, ele se apresentaria como o contador
espiritual do grupo; e desejava que cada um dos ouvintes abrisse seu
livro"-razão, o livro"-razão de sua vida espiritual, e verificasse se os
registros conferiam com as contas da consciência.

Jesus Cristo não é um chefe muito exigente.  Ele compreende nossos fracassos,
compreende as fraquezas de nossa pobre natureza pecadora, compreende as
tentações desta vida.  Todos nós, de vez em quando, sofremos tentações: todos
nós temos nossos fracassos.  Mas apenas uma coisa, ele disse, pediria aos
ouvintes.  Era o seguinte: sejam honestos e viris perante Deus.  Se as contas
conferirem, digam:

\textit{Pois bem, verifiquei minhas contas.  Tudo está certo}.  

Mas se, conforme costuma ocorrer, houver algumas discrepâncias, admitam a
verdade, sejam francos e digam como homens:

\textit{Pois bem, examinei minhas contas.  Encontrei este e aquele erro.
Mas, com a graça de Deus, corrigirei isto e aquilo; acertarei minhas
contas}.


\chapter{Os mortos}
\hedramarkboth{Os mortos}{James Joyce}

\textsc{Lily,} a filha do zelador, estava literalmente exausta.  Mal
acompanhara um cavalheiro até a despensa atrás do escritório no primeiro andar
e o ajudou a tirar o sobretudo quando o som sibilante da campainha da porta
ecoou novamente e ela teve de correr pelo corredor vazio para receber mais um
convidado.  Ainda bem que não tinha de recepcionar as senhoras também.  Mas
Miss Kate e Miss Julia tinham pensado nisso e convertido o banheiro do segundo
andar num toalete feminino.  Miss Kate e Miss Julia estavam lá, alvoroçadas e
fofocando e rindo, andando uma atrás da outra até o topo da escada, olhando por
cima da balaustrada e chamando Lily para perguntar quem havia chegado.

Era sempre um grande acontecimento, a festa anual das irmãs Morkan.  Todo mundo
que as conhecia vinha à festa, parentes, velhos amigos da família, integrantes
do coral em que Julia cantava, alunos de Kate com idade suficiente, e até mesmo
alguns dos alunos de Mary Jane.  A festa jamais fracassara.  Ano após ano tudo
sempre transcorrera em grande estilo até onde as pessoas podiam se lembrar;
desde que Kate e Julia, depois da morte do irmão, Pat, mudaram"-se da casa em
Stoney Batter e levaram Mary Jane, única sobrinha, para morar com elas em
Usher’s Island numa casa escura e lúgubre, cuja parte superior fora alugada de
um tal Mr.~Fulham, um negociante de cereais que morava no andar térreo.  Isso
fora há exatamente trinta anos.  Mary Jane, que à época era uma garotinha,
atualmente era o esteio da casa, pois tocava órgão em Haddington Road.
Completara o curso no Conservatório e todo ano apresentava um concerto com seus
alunos na sala do segundo andar do Antient Concert Rooms.  Muitos de seus
alunos provinham de boas famílias residentes na linha de Kingstown e Dalkey.
Apesar de idosas, as tias faziam a sua parte.  Julia, embora já bastante
grisalha, cantava como primeiro soprano na igreja de Adam and Eve e Kate, fraca
demais para sair muito à rua, dava lições de música para principiantes no velho
piano de armário no quarto dos fundos.  Lily, a filha do zelador, realizava
para elas serviços domésticos.  Embora levassem uma vida modesta, gostavam de
comer bem; tudo do bom e do melhor: alcatra da boa, chá do mais caro e sempre a
melhor cerveja em garrafa.  Lily raramente cometia deslizes ao encomendar
mantimentos, portanto tinha uma boa relação com as três patroas.  Eram
exigentes, só isso.  Mas a única coisa que não toleravam era contestação.

É claro, tinham toda razão de serem exigentes naquela noite.  E também já
passava das dez horas e nem sinal de Gabriel com a mulher.  Além disso, estavam
mortas de medo que Freddy Malins aparecesse embriagado.  Não queriam por nada
deste mundo que algum aluno de Mary Jane o visse em tal estado: e quando estava
alto às vezes era muito difícil controlá"-lo.  Freddy Malins sempre chegava
tarde mas elas se perguntavam o que poderia estar detendo Gabriel: e por isso a
cada dois minutos elas vinham até a balaustrada para perguntar a Lily se
Gabriel ou Freddy tinham chegado.

--- Ah, Mr.~Conroy --- disse Lily a Gabriel ao abrir"-lhe a porta ---, Miss Kate
e Miss Julia estavam pensando que o senhor não chegaria nunca.  Boa"-noite, Mrs.~Conroy.

--- Suponho que estivessem mesmo --- disse Gabriel ---, mas elas se esquecem
que minha mulher leva três horas intermináveis pra se arrumar.

Parou um instante sobre o capacho, raspando a neve das galochas, enquanto Lily
conduzia a esposa até o pé da escada e gritava:

--- Miss Kate, Mrs.~Conroy está aqui.

Kate e Julia desceram a escada escura um tanto trôpegas.  Ambas beijaram a
mulher de Gabriel, disseram que ela devia estar morta de cansada e perguntaram
se Gabriel viera com ela.

--- Aqui estou, pontual como o Correio, tia Kate!  Podem subir.  Subo já ---
disse Gabriel, do \textit{hall} escuro.

Ele continuou a limpar vigorosamente os pés enquanto as três mulheres subiam a
escada, rindo, em direção ao banheiro feminino.  Uma leve camada de neve se
formara como uma capa nas ombreiras de seu sobretudo e como biqueiras nas
pontas das galochas; e, no momento em que os botões do sobretudo roçaram os
frisos da roupa enrijecidos pelo frio, produzindo um leve rangido, um aroma
gélido de ar livre exalou das pregas e das dobras.

--- Está nevando de novo, Mr.~Conroy? --- perguntou Lily.

Ela o precedera até a pequena sala para ajudá"-lo a tirar o sobretudo.  Gabriel
sorriu ao ouvir a jovem pronunciar seu sobrenome com três sílabas e olhou para
ela.  Era uma garota esbelta, na flor da idade, e tinha a tez pálida e os
cabelos cor de feno.  A luz da lamparina a gás que havia na despensa tornava"-a
mais pálida ainda.  Gabriel a conhecia desde a época em que ela era criança e
costumava sentar"-se no primeiro degrau da escada embalando uma boneca de pano.

--- Está, Lily --- ele respondeu ---, e acho que vai continuar a noite toda.

Ele olhou para o teto da despensa, que tremia com o arrastar e o bater de pés no
andar de cima, escutou por um instante o som do piano e voltou a olhar para a
moça, que dobrava o sobretudo cuidadosamente, colocando"-o no canto de uma
prateleira.

--- Diga"-me, Lily --- ele disse, com um tom de voz amável ---, você ainda está
na escola?

--- Ah, não senhor --- ela respondeu.  --- Acabei meus estudos este ano e já
chega.

--- Ah, nesse caso --- disse Gabriel, brincando ---, suponho que qualquer dia
seremos convidados para o seu casório, hein?

A moça olhou para ele por cima do ombro e disse com amargura:

--- Os homens de hoje só querem saber de conversa fiada e de se aproveitar da
gente.

Gabriel enrubesceu, como se percebesse que cometera uma gafe e, desviando o
olhar, sacudiu os pés para livrar"-se das galochas e com o cachecol lustrou os
sapatos de couro.

Era um jovem forte, de boa estatura.  O tom corado de suas faces subia até
quase a testa, onde se dispersava em manchas amorfas e avermelhadas; e em seu
rosto escanhoado cintilavam as lentes cristalinas e os aros dourados dos óculos
que lhe protegiam os olhos suaves e inquietos.  Seu cabelo preto e brilhante
era repartido ao meio e penteado numa longa curva por trás das orelhas, onde
revelava uma pequena onda formada pela marca do chapéu.

Depois de lustrar os sapatos levantou"-se e puxou a sobrecasaca para ajustá"-la
ao corpo rechonchudo.  Então retirou do bolso mais do que depressa uma moeda.

--- Ah, Lily --- ele disse, enfiando a moeda na mão da moça ---, estamos na
época do Natal, não é?  É só\ldots{} uma pequena\ldots{}

E acelerou o passo em direção à porta.

--- Que é isso! --- exclamou a moça, saindo atrás dele.  --- Não posso, senhor,
não posso aceitar.

--- Natal!  Natal! --- disse Gabriel, quase correndo em direção à escada e
fazendo com a mão um gesto de protesto.

A moça, vendo que ele chegara o topo da escada, gritou:

--- Então, obrigada, senhor!

Ele aguardou do outro lado da porta do salão até que a valsa terminasse,
ouvindo o roçar das saias na porta e o arrastar dos pés.  Ainda estava abalado
devido à resposta inesperada e amarga da moça.  A reposta tinha provocado uma
melancolia que ele agora tentava afastar arrumando os punhos da camisa e o laço
da gravata.  Então retirou do bolso do colete um pedaço de papel e passou os
olhos sobre os tópicos a serem abordados no discurso.  Estava hesitante em
relação aos versos de Robert Browning, pois receava que não estivessem ao
alcance dos ouvintes.  Alguma citação que eles pudessem reconhecer de
Shakespeare ou das \textit{Melodies} seria mais apropriada.  O rude estalar de
saltos e o arrastar das solas dos sapatos dos homens faziam"-no lembrar que o
nível cultural dos presentes diferia do seu.  Faria papel ridículo ao citar
versos que os presentes não pudessem compreender.  Pensariam que estava
exibindo sua educação universitária.  Cometeria um deslize com eles assim como
cometera uma gafe com a moça na despensa.  Adotara o tom errado.  O discurso era um
equívoco do começo ao fim, um fracasso total.

Naquele momento as tias e a esposa saíram do banheiro feminino.  As tias eram
duas velhotas pequeninas, vestidas com discrição.  Tia Julia era alguns
centímetros mais alta.  Seus cabelos, presos e encobrindo ligeiramente as
orelhas, eram grisalhos; e igualmente acinzentado, com sombreados escuros, era
seu rosto grande e flácido.  Embora tivesse postura firme e ereta, o olhar vago
e a boca entreaberta davam"-lhe a aparência de uma mulher que não sabia onde
estava nem para onde ia.  Tia Kate era mais animada.  Seu rosto, mais saudável
que o da irmã, era só rugas e sulcos, como uma maçã vermelha e murcha, mas seus
cabelos, trançados à moda antiga, conservavam o tom de nozes maduras.

As duas beijaram Gabriel sem a menor cerimônia.  Era o sobrinho predileto,
filho de Ellen, a irmã mais velha já falecida, que se casara com T.J.~Conroy,
funcionário do cais do porto.

--- A Gretta me disse que vocês não vão tomar um coche de volta a Monkstown no
final da noite, Gabriel --- disse tia Kate.

--- Não --- disse Gabriel, virando"-se para a mulher.  --- Basta o que nos
aconteceu no ano passado, não é?  A senhora está lembrada, tia Kate, do
resfriado que a Gretta pegou?  As janelas do coche chacoalhando e o vento leste
soprando depois que passamos por Merrion.  Uma beleza.  A Gretta pegou um belo
resfriado.

Tia Kate franziu a testa e meneou a cabeça a cada palavra.

--- Está certo, Gabriel, está certo --- ela disse.  --- Todo cuidado é pouco.

--- Mas quanto à Gretta --- disse Gabriel ---, ela voltaria pra casa caminhando
na neve, se eu deixasse.

Mrs.~Conroy riu.

--- Não dê ouvidos a ele, tia Kate --- ela disse.  --- É mesmo um chato, com o
abajur verde pra proteger a vista do Tom à noite e obrigando o menino a fazer
halteres e forçando a Eva a tomar mingau.  Pobre criança!  E ela não consegue
nem olhar pro mingau!\ldots{}  Ah, mas vocês nem imaginam o que ele me faz usar
agora!

Deu uma gargalhada e olhou de relance para o marido, cujo olhar cheio de
admiração e felicidade perambulava do vestido para o rosto e em seguida para o
cabelo da mulher.  As duas tias também riram a valer, pois o excesso de
preocupação por parte de Gabriel era para elas motivo de constantes
brincadeiras.

--- Galochas! --- disse Mrs.~Conroy.  --- Essa foi a última.  Sempre que está
molhado lá fora eu tenho de usar minhas galochas.  Até numa noite como a de
hoje ele queria que eu usasse galochas mas recusei"-me.  A próxima coisa que vai
comprar pra mim será um escafandro.

Gabriel exibiu um sorriso nervoso e endireitou o nó da gravata, tentando
aparentar segurança, enquanto tia Kate quase rolava de rir, de tanto que gostou
da piada.  O sorriso logo desapareceu do rosto de tia Julia e seus olhos
sisudos dirigiram"-se ao rosto do sobrinho.  Depois de uma pausa ela perguntou:

--- E o que são galochas, Gabriel?

--- Galochas, Julia! --- exclamou a irmã.  --- Deus do céu, você não sabe o que
são galochas?  A gente as calça por cima dos\ldots{} por cima dos sapatos, não
é, Gretta?

--- Sim --- disse Mrs.~Conroy.  --- São umas coisas de látex.  Cada um de nós
dois tem agora um par.  Gabriel diz que todo mundo usa galocha no continente.

--- Ah, no continente --- murmurou tia Julia, meneando lentamente a cabeça.

Gabriel franziu a testa e disse, como se estivesse um pouco zangado:

--- Não é nada do outro mundo, mas a Gretta acha muita graça porque diz que a
palavra a faz lembrar dos Christy Minstrels.

--- Mas diga"-me, Gabriel --- disse tia Kate, com muito tato.  --- Decerto, você
providenciou o quarto.  A Gretta estava dizendo\ldots{}

--- Ah, o quarto está garantido --- replicou Gabriel.  --- Fiz reservas no
Gresham.

--- Com certeza --- disse tia Kate ---, foi a melhor providência a ser tomada.
E as crianças, Gretta, você não fica preocupada?

--- Ah, é só uma noite --- disse Mrs.~Conroy.  --- Além disso, a Bessie vai
cuidar delas.

--- Com certeza --- disse tia Kate novamente.  --- Que sossego é ter uma moça
como aquela, em quem se pode confiar!  Vejam essa Lily, não sei o que está
acontecendo com ela ultimamente.  Já não é mais a mesma.

Gabriel estava prestes a fazer algumas perguntas a respeito desse assunto, mas
ela desviara subitamente o olhar em direção à irmã, que começara a descer a
escada e agora se curvava por cima da balaustrada.

--- Essa agora, por favor! --- ela disse quase irritada --- Aonde vai a Julia?
Julia!  Julia!  Aonde você vai?

Julia, que alcançara a metade da escada, voltou e anunciou calmamente:

--- Freddy chegou.

Naquele momento uma salva de palmas e os acordes finais da pianista indicavam
que a valsa tinha terminado.  A porta do salão foi aberta pelo lado de dentro e
alguns pares saíram.  Tia Kate puxou Gabriel subitamente para um canto e
murmurou"-lhe ao ouvido:

--- Dê um pulo lá embaixo, Gabriel, por favor, e veja se ele está em condições,
e não permita que suba se estiver muito grogue.  Tenho certeza de que está grogue.  Tenho
certeza.

Gabriel foi até a escada e pôs"-se a escutar.  Ouvia duas pessoas falando na
despensa.  Então reconheceu a risada de Freddy Malins.  Desceu a escada pisando
duro.

--- Que alívio --- disse tia Kate a Mrs.~Conroy --- Gabriel estar aqui.  Sempre
fico mais tranquila quando ele está aqui\ldots{}  Julia, Miss Daly e Miss Power
vão aceitar um refresco.  Obrigada pela bela valsa, Miss Daly.  Foi lindo.

Um sujeito alto, de rosto murcho, bigode rijo e grisalho e pele morena, que
saía da sala ao lado de sua parceira, disse:

--- E nós podemos beber alguma coisa, também, Miss Morkan?

--- Julia --- disse tia Kate sumariamente ---, aqui estão Mr.~Browne e Miss
Furlong.  Leve"-os para dentro, Julia, com Miss Daly e Miss Power.

--- Estou sempre bem acompanhado --- disse Mr.~Browne, apertando os lábios até
eriçar o bigode e sorrindo com todas as rugas.  --- A senhora sabe, Miss
Morkan, o motivo para que gostem tanto de mim é que\ldots{}

Não terminou a frase mas, percebendo que tia Kate tinha se afastado e que não o
escutaria, conduziu imediatamente as três jovens à sala dos fundos.  No meio da
sala havia duas mesas quadradas, lado a lado, sobre as quais tia Julia e o
zelador estavam estendendo uma grande toalha.  No aparador havia travessas e
pratos e copos e montes de facas e garfos e colheres.  A tampa do piano de
armário também servia de aparador para salgados e doces.  Ao lado de um
aparador menor num canto dois rapazes bebiam refresco.

Mr.~Browne dirigiu suas protegidas para lá e convidou"-as, em tom de galhofa, a
beber do ponche especialmente preparado para as senhoras, quentinho, forte e
doce.  De vez que elas disseram que não bebiam nada forte, ele abriu três
garrafas de limonada.  Então pediu licença a um dos rapazes e, apoderando"-se da
garrafa, serviu"-se de uma generosa dose de uísque.  Os rapazes olharam"-no com
respeito enquanto ele experimentava a bebida.

--- Que Deus me ajude --- ele disse, sorrindo ---, é prescrição médica.

Em seu rosto murcho abriu"-se um sorriso largo, e as três jovens responderam ao
gracejo com sonoras risadinhas que lhes sacudiram o corpo, fazendo"-lhes tremer
os ombros num gesto nervoso.  A mais afoita disse:

--- Ora, Mr.~Browne, tenho certeza de que o médico nunca lhe prescreveu esse
tipo de remédio.

Mr.~Browne tomou mais um gole do uísque e disse, fazendo uma imitação
desajeitada:

---Sabem de uma coisa, eu sou como a célebre Mrs.~Cassidy, que supostamente
dizia: \textit{Olha, Mary Grimes, se eu não estiver bebendo,
obrigue"-me a fazê"-lo, pois sei que quero beber}.

Seu rosto aquecido tinha se aproximado com excessiva intimidade e a sua voz
descambara para o sotaque rude das classes mais baixas de Dublin, de modo que
as moças, instintivamente, calaram"-se após o gracejo.  Miss Furlong, que era
aluna de Mary Jane, perguntou a Miss Daly o nome da bela valsa que acabara de
executar; e Mr.~Browne, vendo"-se ignorado, virou"-se prontamente para os dois
rapazes, que se mostraram mais receptivos.

Uma jovem de faces rosadas e vestido violeta entrou agitada na sala, batendo
palmas e gritando:

--- \textit{Quadrilles!  Quadrilles!}

Nos calcanhares dela veio tia Kate, gritando:

--- Dois cavalheiros e três damas, Mary Jane!

--- Ah, aqui estão Mr.~Bergin e Mr.~Kerrigan --- disse Mary Jane.  --- 
Mr.~Kerrigan, o senhor dançaria com Miss Power?  Miss Furlong, por favor, a senhora
fica com Mr.~Bergin.  Ah, agora está tudo certo.

--- Três damas, Mary Jane --- disse tia Kate.

Os dois rapazes perguntaram às damas se estas lhes dariam o prazer de dançar, e
Mary Jane voltou"-se para Miss Daly.

--- Ah, Miss Daly!  Você é mesmo muito gentil, depois de tocar duas valsas, mas
estamos com poucas damas esta noite.

--- Não tem o menor problema, Miss Morkan.

--- Mas eu tenho um par encantador para você, Mr.~Bartell D’Arcy, o tenor.  Vou
fazer com que ele cante mais tarde.  Toda Dublin está empolgada com ele.

--- Voz adorável, voz adorável! --- disse tia Kate.

No momento em que o piano repetiu a introdução da primeira quadrilha Mary Jane
apressou"-se em conduzir seu grupo para fora do salão.  Mal haviam saído quando
tia Julia entrou, olhando para trás.

--- O que houve, Julia? --- perguntou tia Kate preocupada.  --- Quem é?

Julia, que carregava uma pilha de guardanapos, virou"-se para a irmã e disse,
calmamente, como se a pergunta lhe causasse espécie:

--- É o Freddy, Kate, e o Gabriel está com ele.

De fato, logo atrás dela Gabriel podia ser visto guiando os passos de Freddy
Malins no patamar da escada.  Este, homem de seus quarenta anos de idade, tinha
a altura e o porte de Gabriel, e os ombros um tanto caídos.  Seu rosto era
gordo e pálido e os únicos vestígios de cor estavam localizados nos lóbulos
carnudos das orelhas e nas narinas largas.  Tinha traços grosseiros, nariz
chato, a fronte convexa e lábios inchados e protuberantes.  As pálpebras
pesadas e o cabelo ralo e emaranhado conferiam"-lhe um aspecto sonolento.
Gargalhava estridentemente de uma história que contara a Gabriel enquanto
subiam a escada e ao mesmo tempo esfregava o olho esquerdo com as costas do
punho esquerdo.

--- Boa"-noite, Freddy --- disse tia Julia.

Freddy Malins deu boa"-noite às irmãs Morkan de um modo que parecia displicente
em virtude da voz sempre um pouco engrolada e, vendo que Mr.~Browne sorria para
ele ao lado do aparador, atravessou um tanto cambaleante a sala e começou a
repetir à meia"-voz a história que acabara de contar a Gabriel.

--- Ele não está tão ruim assim, não é? --- disse tia Kate a Gabriel.

O cenho de Gabriel estava carregado mas ele ergueu as sobrancelhas prontamente
e respondeu:

--- Não, não!  Mal dá pra se notar.

--- Mas que sujeitinho terrível! --- ela disse.  --- Pensar que a pobre mãe na
véspera do Ano"-Novo o fez jurar abstinência.  Mas, vamos, Gabriel, para o
salão.

Antes de sair da sala de jantar ao lado de Gabriel, ela fez um sinal de
advertência com o dedo em riste para Mr.~Browne.  Mr.~Browne balançou a cabeça
indicando que compreendera o aviso e, assim que ela se foi, disse a Freddy
Malins:

--- Escute aqui, Freddy, vou lhe servir um bom copo de limonada pra você se
sentir melhor.

Freddy Malins, que se aproximava do clímax da história, recusou a oferta com um
gesto impaciente mas Mr.~Browne, depois de chamar a atenção de Freddy Malins
para o estado lastimável de seus trajes, serviu"-lhe um copo cheio de limonada.
A mão esquerda de Freddy Malins aceitou mecanicamente o copo, já que a mão
direita, também mecanicamente, ocupava"-se em ajeitar a roupa.  Mr.~Browne, cujo
rosto tornara a se enrugar num sorriso, serviu"-se de uísque enquanto Freddy
Malins, antes mesmo de chegar ao clímax da história, explodia num ataque de
riso estridente misturado a um acesso de tosse e, colocando na mesa o copo
intato e transbordante, esfregava as costas da mão esquerda no olho esquerdo,
tentando repetir a última frase, à medida que o acesso de riso lhe permitisse.

\smallskip

\noindent\dotfill

\smallskip

Gabriel não conseguia prestar atenção à peça clássica, cheia de floreios e
trechos difíceis, com a qual Mary Jane se graduara no Conservatório e que ora
executava diante do salão silencioso.  Gostava de música mas, para ele, a peça
que ela tocava era desprovida de melodia e ele duvidava que fosse melódica para
os outros ouvintes, embora estes tivessem insistido para que Mary Jane tocasse
algo.  Quatro rapazes, que ao ouvirem os primeiros acordes do piano tinham
vindo da sala de jantar até a porta do salão, em questão de minutos
retiraram"-se discretamente, dois de cada vez.  As únicas pessoas que pareciam
acompanhar a música eram a própria Mary Jane, cujas mãos deslizavam pelo
teclado e, nas pausas, dele afastavam"-se tal e qual uma sacerdotisa durante uma
imprecação, e tia Kate de pé a seu lado virando as páginas da partitura.

Os olhos de Gabriel, ofuscados pelo assoalho encerado, que brilhava embaixo do
grande candelabro, desviaram"-se para a parede que ficava atrás do piano.  Nela
havia um quadro da cena da sacada de \textit{Romeu e Julieta} e ao lado um
outro quadro tecido por tia Julia quando menina, em lã vermelha, azul e marrom,
com os dois principezinhos assassinados na Torre.  Com certeza, aquele tipo de
trabalho manual era ensinado na escola que elas frequentaram quando meninas,
pois, como presente de aniversário, sua mãe bordara para ele umas cabecinhas de
raposa num colete de seda lilás forrado de cetim marrom e com botões redondos
em formato de amora.  Era estranho que sua mãe não tivesse nenhum talento
musical, mas tia Kate costumava referir"-se a ela como o cérebro da família
Morkan.  Tanto Kate quanto Julia sempre deixavam transparecer certo orgulho da
irmã sisuda e conservadora.  Havia um retrato dela em frente ao espelho do
aparador.  Ela trazia ao colo um livro aberto onde apontava alguma coisa para
Constantino, sentado a seus pés e vestido de marinheiro.  Os nomes dos filhos
tinham sido escolhas dela, visto que ela era muito ciosa da importância da vida
em família.  Graças a ela, Constantino era hoje pároco de Balbriggan e, graças
a ela, Gabriel graduara"-se pela Royal University.  Uma sombra abateu"-se em seu
rosto quando se lembrou da objeção carrancuda que ela fizera a seu casamento.
Ainda sofria ao se lembrar de certas frases sarcásticas; certa vez, a mãe
chamara Gretta de roceira esperta e isso, absolutamente, não era verdade.  Foi
Gretta quem tratou dela durante o longo período terminal de sua doença na casa
de Monkstown.

Ele sabia que Mary Jane se aproximava do final da peça pois ela agora tocava
novamente a melodia do prelúdio, executando uma série de floreios após cada
acorde e enquanto aguardava o desfecho seu ressentimento esmoreceu.  A peça
encerrou com um arpejo de oitavas agudas e uma forte oitava final no grave.
Uma calorosa salva de palmas felicitou Mary Jane que, enrubescendo e enrolando
às pressas a partitura, fugiu do salão.  Os aplausos mais vigorosos partiram
dos quatro rapazes de pé próximos a porta, que tinham se retirado para a sala
de jantar no início da apresentação e que voltaram assim que o piano silenciou.

Organizou"-se uma dança típica, lanceiros.  Gabriel fez par com Miss Ivors.  Ela
era uma jovem franca e falante, com rosto sardento e grandes olhos castanhos.
Não usava decote e o grande broche preso à gola de sua blusa estampava o
emblema irlandês.

Quando tomaram seus lugares ela disse bruscamente:

--- Tenho uma conta a acertar com você.

--- Comigo? --- disse Gabriel.

Ela meneou a cabeça com ar sério.

--- De que se trata? --- perguntou Gabriel, sorrindo diante do ar solene da
jovem.

--- Quem é G.C.? --- respondeu Miss Ivors, olhando firme para ele.

Gabriel enrubesceu e começava a franzir a testa, como se não compreendesse,
quando ela disse asperamente:

--- Ah, como ele é inocente!  Descobri que você escreve para o \textit{Daily
Express}.  Não se envergonha?

--- Por que haveria de me envergonhar? --- perguntou Gabriel, piscando os olhos
e tentando sorrir.

--- Pois eu sinto vergonha por você --- disse Miss Ivors com franqueza.  ---
Escrever pra um jornal como aquele.  Não imaginava que você fosse anglófilo.

Um ar de perplexidade estampou"-se no rosto de Gabriel.  Era verdade que
escrevia uma coluna literária todas as quartas"-feiras para o \textit{Daily
Express}, pelo que recebia quinze \textit{shillings}.  Mas isso certamente não
queria dizer que fosse anglófilo.  Os livros que ele recebia para resenhar eram
quase mais gratificantes que o mísero cheque.  Sentia imenso prazer em manusear
capas e virar páginas de livros que acabaram de sair da tipografia.
Quase todos os dias quando terminava de dar suas aulas costumava caminhar pelo
cais até os sebos, o Hickey’s em Bachelor’s Walk, o Webb’s ou o Massey’s, em
Aston’s Quay, ou o O’Clohissey’s, que ficava numa travessa.  Não sabia como se
defender da acusação.  Queria dizer que a literatura estava acima da política.
Mas eram amigos havia muitos anos e suas carreiras tinham transcorrido
paralelamente, primeiro na universidade e depois como professores: não ousaria
responder"-lhe com uma frase pedante.  Continuou a piscar os olhos, tentando
sorrir, e murmurou um tanto desajeitado que não via nada de político no ato de
escrever resenhas literárias.

Quando chegou a vez de os dois atravessarem o salão ele ainda estava perplexo e
avoado.  Miss Ivors tomou"-lhe calidamente a mão e disse com um tom de voz suave
e amistoso:

--- É claro que eu estava só brincando.  Vamos, é nossa vez de atravessar o
salão.

Quando voltaram a ficar lado a lado ela começou a falar a respeito da
universidade e Gabriel sentiu"-se mais à vontade.  Uma amiga tinha"-lhe mostrado
a crítica que ele fizera dos poemas de Browning.  Assim descobrira o segredo:
mas ela gostara imensamente da resenha.  Então disse, subitamente:

--- Ah, Conroy, você faria uma excursão às Ilhas Aran no próximo verão?  Vamos
ficar por lá um mês inteiro.  Vai ser maravilhoso, em pleno Atlântico.  Você
deveria ir.  Mr.~Clancy vai, e também Mr.~Kilkelly e Kathleen Kearney.  Seria
maravilhoso para a Gretta, também, se ela quisesse ir.  Ela é de Connacht, não
é?

--- A família dela --- disse Gabriel secamente.

--- Mas você vai, não vai? --- disse Miss Ivors, tocando o braço dele com sua
mão cálida e certa ansiedade.

--- A questão é --- disse Gabriel --- que já fiz planos para ir\ldots{}

--- Ir pra onde? --- perguntou Miss Ivors.

--- Bem, é que todo ano eu faço uma viagem de bicicleta com uns amigos e
assim\ldots{}

--- Mas onde? --- perguntou Miss Ivors.

--- Bem, costumamos ir à França ou à Bélgica ou talvez à Alemanha --- disse
Gabriel, com certo embaraço.

--- E por que você vai à França e à Bélgica --- disse Miss Ivors --- em vez de
conhecer o seu país?

--- Bem --- disse Gabriel ---, por um lado pra praticar os idiomas e por outro
pra mudar de ares.

--- E você não precisa praticar o seu próprio idioma, o irlandês?  ---
perguntou Miss Ivors.

--- Bem --- disse Gabriel ---, se a questão é essa, você sabe, irlandês não é o
meu idioma.

Os pares mais próximos tinham se voltado e prestavam atenção ao interrogatório.
Nervoso, Gabriel olhava para a direita e para a esquerda e tentava manter o bom
humor em meio àquela situação constrangedora que começava a causar uma
vermelhidão em sua fronte.

--- E por que não visita a sua terra --- prosseguiu Miss Ivors ---, a respeito
da qual é tão ignorante, a sua gente, o seu país?

--- Ah, pra dizer a verdade --- retrucou Gabriel subitamente --- estou farto do
meu país, farto!

--- Por quê? --- perguntou Miss Ivors.

Gabriel calou"-se pois a resposta o deixara exaltado.

--- Por quê? --- repetiu Miss Ivors.

Chegara o momento da dança em que tinham de fazer a visita a outros pares e,
como ele nada respondera, Miss Ivors disse com amabilidade:

--- Está claro que você não tem o que responder.

Gabriel tentou disfarçar o nervosismo dançando com grande energia.  Evitava
olhá"-la nos olhos pois vira em seu rosto um ar de rancor.  Mas quando voltaram
a se encontrar na roda ficou surpreso ao sentir que ela lhe apertava a mão com
firmeza.  Ela fitou"-o inquisitivamente até fazê"-lo sorrir.  Então, no momento
em que a roda ia recomeçar a girar, ela ficou na ponta dos pés e sussurrou"-lhe
ao ouvido:

--- Anglófilo!

Quando a dança terminou Gabriel dirigiu"-se a um dos cantos do salão, onde a mãe
de Freddy Malins estava sentada.  Era uma velhota corpulenta e doente, de
cabeça branca.  Sua voz falhava como a do filho e ela era ligeiramente gaga.
Tinha sido informada de que Freddy chegara e que estava razoavelmente sóbrio.
Gabriel perguntou"-lhe se tinha feito uma boa travessia.  Ela morava em Glasgow
com a filha casada e visitava Dublin uma vez por ano.  Ela respondeu
placidamente que fizera uma travessia excelente e que o capitão tinha sido
muito atencioso com ela.  Falou também a respeito da linda casa da filha em
Glasgow e do ótimo círculo de amizades que tinha.  Enquanto ela tagarelava,
Gabriel procurava tirar da cabeça o incidente desagradável entre ele e Miss
Ivors.  É claro que a moça ou mulher, ou seja la o que fosse, era bastante
patriota, mas tudo tem sua hora.  Talvez ele não devesse ter respondido daquela
maneira.  Mas ela não tinha o direito de chamá"-lo de anglófilo em público, nem
de brincadeira.  Tentara ridicularizá"-lo em público, fazendo"-lhe perguntas
embaraçosas e fitando"-o com aqueles olhos de coelha.

Viu a esposa caminhando em sua direção em meio aos pares que valsavam.  Quando
chegou ao seu lado, ela lhe disse ao ouvido:

--- Gabriel, tia Kate quer saber se você vai trinchar o ganso, como de costume.
Miss Daly vai cortar o pernil e eu vou servir o suflê.

--- Está bem --- disse Gabriel.

--- Ela vai mandar as crianças comerem lá dentro assim que terminar esta valsa
para que possamos ter a mesa só para nós.

--- Você estava dançando? --- perguntou Gabriel.

--- Claro que estava.  Você não me viu?  Que discussão foi aquela com Molly
Ivors?

--- Nada.  Por quê?  Ela comentou alguma coisa com você?

--- Por alto.  Estou tentando convencer esse tal Mr.~D’Arcy a cantar para nós.
Ele é muito pretensioso, eu acho.

--- Não foi discussão --- disse Gabriel, mal"-humorado ---, ela queria que eu
fizesse uma viagem ao oeste da Irlanda e eu disse que não iria.

A esposa juntou as mãos com alegria e deu um pequeno salto.

--- Ah, vamos, Gabriel! --- ela disse.  --- Eu adoraria rever Galway.

--- Você pode ir, se quiser --- disse Gabriel com frieza.

Ela fitou"-o por um instante, então virou"-se para Mrs.~Malins e disse:

--- Eis um bom marido, Mrs.~Malins.

Enquanto ela se afastava pelo meio do salão, Mrs.~Malins, sem se abalar com a
interrupção, começou a falar a Gabriel dos lindos recantos e das lindas
paisagens da Escócia.  O genro todo ano levava a família aos lagos onde
costumavam pescar.  O genro era um pescador de mão cheia.  Certa vez pescara um
peixe, um peixão lindo, que o cozinheiro do hotel preparou para o jantar.

Gabriel mal escutava o que ela dizia.  Agora que se aproximava a hora da
ceia ele voltara a pensar no discurso e na citação.  Ao ver Freddy Malins
atravessando o salão para falar com a mãe, Gabriel cedeu"-lhe a cadeira e
retirou"-se para o vão da janela.  O salão já estava vazio e da sala de jantar
ouvia"-se o tilintar de pratos e talheres.  As pessoas que permaneciam no salão
pareciam cansadas de dançar e conversavam à meia"-voz em pequenos grupos.
Gabriel tamborilou os dedos trêmulos e cálidos na vidraça fria da janela.  Como
devia estar frio lá fora!  Como seria agradável sair caminhando sozinho ao
longo do rio e pelo parque!  A neve estaria encobrindo os galhos das árvores e
formando um chapéu lustroso no topo do monumento em homenagem a Wellington.
Como seria mais agradável ficar lá fora do que se sentar àquela mesa!

Examinou os tópicos do discurso: hospitalidade irlandesa, tristes recordações,
as Três Graças, Páris, a citação de Browning.  Lembrou"-se de uma frase que
escrevera na resenha: \textit{Tem"-se a sensação de estar ouvindo uma música
atormentada pelo pensamento}.  Miss Ivors elogiara a resenha.  Teria sido
sincera?  Teria ela vida própria por trás de todo aquele propagandismo?  Nunca
houvera nenhum ressentimento entre os dois até aquela noite.  Sentia"-se abatido
diante da ideia de que ela estaria à mesa do jantar, observando"-o com aquele
olhar crítico e inquisitivo enquanto ele discursasse.  Talvez nem se importasse
em vê"-lo fracassar no discurso.  Veio"-lhe à mente uma ideia encorajadora.  Ele
diria, em alusão a tia Kate e tia Julia: \textit{Senhoras e Senhores, a geração
que ora declina entre nós pode ter tido seus defeitos mas a
meu ver possui certas qualidades, como hospitalidade, senso de humor,
simpatia, solidariedade, as quais a nova geração séria e
intelectualizada que ora cresce entre nós parece carecer.} Ótimo:
palavras perfeitas para Miss Ivors.  O que importava se suas tias não passavam
de duas velhas ignorantes?

Um burburinho na sala atraiu"-lhe a atenção.  Mr.~Browne aproximava"-se pela
porta, escoltando galantemente tia Julia, que se apoiava em seu braço, sorrindo
e inclinando a cabeça para o lado.  Uma salva de palmas um tanto fraca
acompanhou"-a até o piano e então, quando Mary Jane sentou"-se na banqueta, e tia
Julia, que já não sorria, voltou"-se de modo a projetar a voz na sala, as palmas
gradualmente se estancaram.  Gabriel reconheceu o prelúdio.  Era uma antiga
canção de tia Julia: \textit{Arrayed for the Bridal}.\footnote{ Isto é,
“vestida para as núpcias”.}  Sua voz, forte e límpida, dominava os
trinados que embelezavam a melodia e embora cantasse numa cadência acelerada
não atropelava um floreio sequer.  Acompanhar aquela voz, sem olhar para o
rosto da cantora, era deixar"-se levar num voo veloz e seguro.  Gabriel e os que
estavam a sua volta aplaudiram vigorosamente quando a canção terminou,
unindo"-se aos aplausos calorosos que vieram da sala de jantar.  A ovação
parecia tão autêntica que um pingo de rubor surgiu nas faces de tia Julia no
momento em que se curvava para recolocar sobre o atril o velho livro de
partituras encadernado em couro, com suas iniciais gravadas na capa.  Freddy
Malins, que inclinara a cabeça para ouvir melhor, ainda aplaudia depois que
todos tinham parado e tagarelava com a mãe, que meneava a cabeça com um ar
grave em sinal de aprovação.  Quando finalmente parou de aplaudir, ele se
levantou com um salto e atravessou a sala a passos largos em direção à tia
Julia, cuja mão ele espremeu entre as suas, agitando"-a pois as palavras lhe
faltavam ou a rouquidão o impedia de falar.

--- Eu estava acabando de dizer pra mamãe --- ele disse --- que nunca ouvi a
senhora cantar tão bem, nunca.  Nunca ouvi a voz da senhora tão bela quanto
esta noite.  Ora!  Acredita em mim?  É a pura verdade.  Palavra de honra que é
verdade.  Nunca ouvi a sua voz tão límpida e tão\ldots{} tão clara e límpida,
nunca.

Tia Julia abriu um grande sorriso e murmurou alguma coisa a respeito de elogios
enquanto se desvencilhava das mãos dele.  Mr.~Browne estendeu"-lhe a mão e disse
aos que estavam perto dele como se fosse um mestre de cerimônias apresentando
uma estrela à plateia:

--- Miss Julia Morkan, minha descoberta mais recente!

Ele ria a valer de suas próprias palavras, quando Freddy Malins virou"-se para
ele e disse:

--- Olhe, Browne, não sei se você está falando sério, mas uma descoberta melhor
do que essa você não faria.  O que eu posso dizer é que nunca a ouvi cantar tão
bem assim desde que aqui compareço.  É a pura verdade.

--- Nem eu --- disse Mr.~Browne.  --- Acho que a voz dela progrediu muito.

Tia Julia deu de ombros e disse com um orgulho contido:

--- Trinta anos atrás até que a minha voz não era das piores.

--- Eu sempre disse pra Julia --- disse tia Kate enfaticamente --- que ela
estava perdendo tempo naquele coral.  Mas ela nunca ouve meus conselhos.

Ela se virou como se desejasse apelar para o bom senso dos circunstantes no
sentido de coibir uma criança teimosa enquanto tia Julia olhava absorta para um
ponto a sua frente, com um sorriso vago e reminiscente estampado no rosto.

--- Não --- continuou tia Kate ---, ela nunca ouviu nem obedeceu a ninguém, se
matando dia e noite naquele coral, noite e dia.  Seis horas da manhã, em pleno
dia de Natal!  E tudo isso pra quê?

--- Não seria para honrar a Deus, tia Kate? --- perguntou Mary Jane, girando na
banqueta do piano e sorrindo.

Tia Kate voltou"-se indignada para a sobrinha e disse:

--- Eu sei muito bem que é pra honrar a Deus, Mary Jane, mas não vejo nada de
honrado quando o papa manda botar pra fora do coral senhoras que se mataram a
vida toda e recruta uns fedelhos pra ocuparem seus lugares.  Suponho que seja
pelo bem da Igreja, já que a medida partiu do papa.  Mas não é justo, Mary
Jane, e não está certo.

Ela se deixara levar pela emoção e teria prosseguido em defesa da irmã, visto
que o assunto era para ela algo não resolvido, mas Mary Jane, notando que os
pares tinham voltado, interveio em tom apaziguador:

--- Calma, tia Kate, a senhora está fazendo um escândalo na frente de 
Mr.~Browne, que pertence a outra fé.

Tia Kate virou"-se para Mr.~Browne, que sorria em virtude da referência feita a
sua religião, e apressou"-se em dizer:

--- Ah, eu não duvido que o papa esteja certo.  Não passo de uma velha
ignorante e não me atreveria a uma coisa dessas.  Mas existe algo que se chama
cortesia e gratidão.  E se eu estivesse no lugar da Julia, eu teria dito isso
ao próprio padre Healy\ldots{}

--- E além do mais, tia Kate --- disse Mary Jane ---, estamos todos com fome e
quando estamos com fome somos todos uns briguentos.

--- E quando estamos com sede também somos briguentos --- aduziu Mr.~Browne.

--- Então é melhor começarmos a cear --- disse Mary Jane --- e acabar essa
discussão mais tarde.

No patamar da escada, em frente à porta do salão, Gabriel encontrou a esposa e
Mary Jane tentando convencer Miss Ivors a ficar para a ceia.  Mas Miss Ivors,
que já estava de chapéu e agora abotoava a capa, recusava"-se a permanecer.  Não
tinha o menor apetite e já passava da hora de se retirar.

--- Só mais dez minutos, Molly --- disse Mrs.~Conroy.  --- Dez minutos não vão
atrasá"-la tanto assim.

--- Só pra dar uma beliscada --- disse Mary Jane ---, depois de tanto que você
dançou.

--- Realmente, não posso --- disse Miss Ivors.

--- Estou achando que você não se divertiu nada --- disse Mary Jane
decepcionada.

--- Claro que me diverti, pode ter certeza --- disse Miss Ivors.  --- Mas agora
tenho mesmo de ir embora.

--- Mas como você vai pra casa? --- perguntou Mrs.~Conroy.

--- Ah, daqui até o cais é um pulinho.

Gabriel hesitou um instante e disse:

--- Se me permite, Miss Ivors, acompanho a senhorita até a sua casa, já que
precisa mesmo ir embora.

Miss Ivors desvencilhou"-se deles.

--- Nem me fale nisso! --- ela exclamou.  --- Pelo amor de Deus, vão jantar e
não se preocupem comigo.  Sei muito bem cuidar de mim mesma.

--- Você é de fato um número, Molly --- disse Mrs.~Conroy com franqueza.

--- \textit{Beannacht libh!}\footnote{ Isto é, literalmente, “bênçãos para
vocês”, ou seja, “adeus”.} --- gritou Miss Ivors, em meio a uma
risada, enquanto descia a escada correndo.

Mary Jane assistiu à cena, com uma expressão de perplexidade estampada no
rosto, enquanto Mrs.~Conroy debruçou"-se sobre a balaustrada para ouviu bater a
porta do \textit{hall}.  Gabriel perguntou a si mesmo se não seria ele o
responsável por aquela saída repentina.  Mas ela não parecia estar
mal"-humorada: fora embora rindo.  Ele ficou olhando pensativo para a escada.

Naquele momento tia Kate surgiu com seus passos trôpegos, vindo da sala de
jantar, torcendo as mãos em sinal de desespero.

--- Onde está o Gabriel? --- ela gritou.  --- Onde se meteu o Gabriel?  Estão
todos esperando lá dentro, e ninguém para trinchar o ganso!

--- Aqui estou, tia Kate! --- exclamou Gabriel, tomado de súbita animação.  ---
Pronto para trinchar um bando de gansos, se for preciso.

Um ganso gordo e bem assado encontrava"-se numa das extremidades da mesa e na
outra, em cima de uma camada de papel amassado e enfeitado com raminhos de
salsa, havia um grande pernil sem pele, polvilhado com farinha de rosca, o osso
meticulosamente envolto com papel cortado em franjas e ao lado viam"-se rodelas
de carne assada.  Entre os pratos rivais, posicionados nas extremidades da
mesa, havia fileiras de acompanhamentos: dois pequenos potes com gelatina,
vermelha e amarela; uma travessa rasa cheia de cubos de manjar branco com calda
vermelha; uma grande travessa verde, em formato de folha e com o cabo imitando
um talo, contendo rubros cachos de passas e amêndoas descascadas; duas
travessas formando par, contendo figos de Smirna arrumados na forma de
retângulos; um pudim com pedacinhos de nozes na cobertura; um pequeno pote com
bombons de chocolate e confeitos embrulhados em papel dourado e prateado; e uma
jarra de cristal contendo longos talos de aipo.  No centro da mesa, como
sentinelas guardando uma fruteira que exibia uma pirâmide de laranjas e maçãs
americanas, havia duas garrafas bojudas de vidro lapidado, uma com vinho do
Porto e a outra com \textit{sherry}.  Sobre o piano um pudim aguardava numa
enorme travessa amarela e atrás do pudim havia três esquadrões de garrafas de
cerveja e água mineral, perfilados de acordo com as cores dos uniformes: as
garrafas dos dois primeiros esquadrões eram escuras, com rótulos nas cores
marrom e vermelho, e as do terceiro eram menores e brancas, com faixas verdes
enviesadas.

Gabriel ocupou seu posto resolutamente à cabeceira da mesa e, após examinar o
fio da faca de trinchar, enterrou com firmeza o garfo no ganso.  Sentia"-se
confiante, pois trinchava uma ave como ninguém e gostava imensamente de ocupar
a cabeceira de uma mesa farta.

--- Miss Furlong, o que a senhora deseja? --- ele perguntou.  --- Uma asa ou
uma fatia do peito?

--- Só uma fatiazinha do peito.

--- E a senhorita, Miss Higgins?

--- Ah, como queira, Mr.~Conroy.

Enquanto Gabriel e Miss Daly serviam ganso, pernil ou carne assada, Lily levava
de convidado a convidado uma travessa de batatas empanadas quentes cobertas por
um guardanapo branco.  Isso tinha sido ideia de Mary Jane que sugerira também
purê de maçã para acompanhar o ganso, mas tia Kate dissera que ganso assado
simples, sem purê de maçã, sempre fora mais do que suficiente.  Mary Jane
servia os alunos e escolhia para eles os melhores pedaços, e tia Kate e tia
Julia abriam e traziam de cima do piano garrafas de cerveja para os cavalheiros
e garrafas de água mineral para as damas.  Havia uma certa balbúrdia, muito
riso e ruído, ruído de vozes fazendo e corrigindo pedidos, de facas e garfos,
de rolhas saltando.  Gabriel começou a servir novas porções assim que terminou
a primeira rodada antes mesmo de ter se servido.  Houve protestos generalizados
e veementes e ele aquiesceu, tomando um grande gole de cerveja, pois a função
de trinchar o ganso o deixara encalorado.  Mary Jane sentou"-se discretamente
para jantar mas tia Kate e tia Julia continuavam a cambalear à volta da mesa,
uma atrás da outra, uma atrapalhando a outra, uma dando à outra ordens que não
eram ouvidas.  Mr.~Browne exortou"-as a se sentarem para jantar e Gabriel fez o
mesmo mas elas afirmavam que mais tarde comeriam, e finalmente Freddy Malins
levantou"-se e, agarrando tia Kate, achatou"-a na cadeira em meio à risada geral.

Quando todos estavam bem servidos, Gabriel disse, sorrindo:

--- Agora, se alguém mais quiser um pouco do que as pessoas vulgares chamam de
boia que se manifeste.

Um clamor de vozes insistiu para que ele começasse a cear e Lily aproximou"-se,
trazendo"-lhe três batatas por ela especialmente guardadas.

--- Muito bem --- disse Gabriel, educadamente, enquanto tomava mais um gole de
cerveja ---, por favor, esqueçam que eu existo, senhoras e senhores, por alguns
minutos.

Começou a comer e se manteve fora da conversa que transcorria na mesa enquanto
Lily retirava os pratos.  O assunto era a companhia de ópera que ora se
apresentava no Theatre Royal.  Mr.~Bartell D’Arcy, o tenor, jovem moreno com um
belo bigode, rasgou elogios à primeira contralto da companhia mas Miss Furlong
disse que ela tivera uma performance um tanto vulgar.  Freddy Malins disse que
um líder negro estava cantando na segunda parte do show de variedades do Gaiety
e que o sujeito era dono de uma das mais belas vozes de tenor que ele ouvira na
sua vida.

--- O senhor já teve a oportunidade de ouvi"-lo? --- ele perguntou a 
Mr.~Bartell D’Arcy sentado do outro lado da mesa.

--- Não --- respondeu Mr.~Bartell D’Arcy com descaso.

--- Pois é --- Freddy Malins explicou ---, eu gostaria de saber o que o senhor
acha dele.  Eu acho que ele tem uma grande voz.

--- Só mesmo o Teddy,\footnote{ Mr.~Browne se confunde ao mencionar o primeiro
nome de Malins.} pra descobrir as coisas que valem a pena --- disse
Mr.~Browne para que todos ouvissem.

--- E por que ele não teria uma boa voz? --- perguntou Freddy Malins com
aspereza.  --- Só porque é negro?

Ninguém respondeu à pergunta e Mary Jane conduziu a conversa de volta à ópera
legítima.  Um de seus alunos dera"-lhe um ingresso para assistir à uma
apresentação de \textit{Mignon}.  É claro que a produção tinha sido muito boa,
disse ela, mas não fora capaz de tirar da cabeça a pobre Georgina Burns.  
Mr.~Browne lembrava"-se de uma época ainda mais remota, quando as antigas companhias
italianas costumavam vir a Dublin --- Tietjens, Ilma de Murzka, Campanini, a
grande Trebelli, Guiglini, Ravelli, Aramburo.  Naquele tempo, disse ele,
ouvia"-se em Dublin canto de verdade.  Contou também como a galeria superior do
velho Theatre Royal costumava lotar noite após noite, e como certa noite um
tenor italiano bisou cinco vezes o “Deixe"-me tombar como soldado”, sem omitir o
dó de peito uma única vez, e ainda como, em seu entusiasmo, os rapazes que
frequentavam as galerias às vezes desatrelavam os cavalos do coche de uma
\textit{prima donna} e puxavam eles mesmos o veículo até o hotel em que ela se
hospedava.  Por que não encenavam mais as grandes óperas, ele perguntou,
\textit{Dinorah, Lucrezia Borgia}?  Porque não encontravam vozes capazes: eis o
porquê.

--- Ora --- disse Mr.~Bartell D’Arcy ---, suponho que hoje em dia existam
cantores tão bons como os de antigamente.

--- Onde estão eles? --- perguntou Mr.~Browne em tom de desafio.

--- Em Londres, Paris, Milão --- disse Mr.~Bartell D’Arcy com amabilidade.  ---
Suponho que Caruso, por exemplo, seja muito bom, se não for melhor que qualquer
um dos homens que o senhor mencionou.

--- Pode ser --- disse Mr.~Browne.  --- Mas confesso que duvido muito.

--- Ah, eu daria tudo pra ouvir Caruso cantar --- disse Mary Jane.

--- Pra mim --- disse tia Kate, que estivera às voltas com um pedaço de osso
--- só existiu um tenor.  Do meu gosto, quero dizer.  Mas imagino que nenhum de
vocês ouviu falar dele.

--- Quem, Miss Morkan? --- perguntou Mr.~Bartell D’Arcy educadamente.

--- Chamava"-se --- disse tia Kate --- Parkinson.  Ouvi"-o cantar quando estava
no auge e acho que ele tinha a voz de tenor mais límpida que já se ouviu de um
ser humano.

--- Estranho --- disse Mr.~Bartell D’Arcy.  --- Realmente, nunca ouvi falar
dele.

--- Sim, sim, Miss Morkan tem razão --- disse Mr.~Browne.  --- Já ouvi falar do
velho Parkinson, mas não foi do meu tempo.

--- Foi um tenor inglês, bonito, com uma voz límpida, doce, aveludada --- disse
tia Kate com entusiasmo.

Assim que Gabriel terminou, o enorme pudim foi transferido para a mesa.  O
tilintar de garfos e colheres recomeçou.  A mulher de Gabriel servia as fatias
de pudim e passava os pratos pela mesa.  No meio do caminho eles eram detidos
por Mary Jane, que acrescentava gelatina de framboesa ou laranja ou ainda
manjar e calda.  O pudim tinha sido feito por tia Julia e ela recebeu elogios
de todos os cantos da mesa.  Ela entretanto disse que o mesmo não ficara bem
dourado.

--- Espero, Miss Morkan --- disse Mr.~Browne ---, que a senhora ache que eu
esteja bem dourado, pois sou “bronzeado” da cabeça aos pés.\footnote{ As
palavras de Mr.~Browne, referindo"-se ao seu próprio nome, encerram um
trocadilho intraduzível, em torno do verbo \textit{to brown}, dourar, tostar, e
do adjetivo \textit{brown}, tostado, dourado (marrom).}

Todos os cavalheiros, à exceção de Gabriel, experimentaram o pudim em
deferência à tia Julia.  De vez que Gabriel jamais comia doce, o aipo fora
deixado especialmente para ele.  Freddy Malins também comeu um talo de aipo
acompanhando o pudim.  Tinha sido informado de que aipo era excelente para o
sangue e estava sob cuidados médicos à época.  Mrs.~Malins, que se mantivera
calada durante toda a ceia, disse que o filho iria para Mount Melleray dentro
de uma ou duas semanas.  Os presentes então falaram sobre Mount Melleray, de
como o ar era puro naquela região, de como eram hospitaleiros os monges e que       \enlargethispage{.4em}
jamais pediam um único centavo aos hóspedes.

--- Vocês estão dizendo --- perguntou Mr.~Browne incrédulo --- que um sujeito
pode se hospedar lá, como se fosse um hotel, comer e beber e sair sem pagar um
centavo?

--- Ah, a maioria das pessoas na hora de ir embora faz uma doação ao monastério
--- disse Mary Jane.

--- Seria bom se tivéssemos uma instituição dessas na nossa igreja --- disse
Mr.~Browne com franqueza.

Ficou atônito ao ser informado que os monges faziam voto de silêncio,
levantavam"-se às duas horas da manhã e dormiam em seus próprios caixões.
Perguntou a razão de tal comportamento.

--- São as regras da congregação --- afirmou tia Kate com firmeza.

--- Sim, mas por que isso? --- perguntou Mr.~Browne.

Tia Kate repetiu que eram regras, nada mais.  Mr.~Browne continuava sem
entender.  Freddy Malins explicou, da melhor maneira que pôde, que os monges
procuravam reparar os pecados cometidos por todos os pecadores do mundo.  A
explicação não fez muito sentido pois Mr.~Browne sorriu e disse:

--- A ideia até que me agrada mas um confortável colchão de molas não seria
melhor que um caixão?

--- O caixão --- disse Mary Jane --- é pra lembrar"-lhes da morte.

O assunto tornara"-se lúgubre e fora enterrado no silêncio da mesa, durante o
qual ouvia"-se Mrs.~Malins dizer, à meia"-voz, à pessoa que estava a seu lado na
mesa:

--- São homens muito bons, os monges, muito devotos.

Passas e amêndoas e figos e maçãs e laranjas e bombons de chocolate e confeitos
percorriam a mesa e tia Julia pedia aos convidados que se servissem de vinho do
Porto ou de \textit{sherry}.  A princípio Mr.~Bartell D’Arcy nada aceitou mas foi
cutucado por uma das pessoas que estavam a seu lado, que lhe sussurrou algo, e
resolveu aceitar.  Aos poucos, à medida que as últimas taças eram servidas, a
conversa foi esmorecendo.  Fez"-se uma pausa, interrompida apenas pelo ruído do
vinho e pelo ranger de cadeiras.  As irmãs Morkan, as três, fixaram o olhar na
toalha da mesa.  Alguém tossiu uma ou duas vezes e então alguns cavalheiros
começaram a dar umas pancadinhas no tampo da mesa pedindo silêncio.  Fez"-se
silêncio e Gabriel empurrou a cadeira para trás e se levantou.

As batidas na mesa tornaram"-se imediatamente mais audíveis, como incentivo, e
finalmente cessaram por completo.  Gabriel apoiou os dez dedos trêmulos sobre a
toalha e sorriu para os presentes sem conseguir esconder o nervosismo.  Diante
das fileiras de rostos que o encaravam ele ergueu os olhos para o lustre.  O
piano tocava uma valsa e ele ouvia o farfalhar das saias roçando na porta do
salão.  Talvez houvesse pessoas no cais lá fora, na neve, olhando para as
janelas iluminadas e ouvindo a valsa.  O ar lá fora era puro.  Ao longe
estendia"-se o parque onde os galhos das árvores pesavam com a neve.  O
monumento a Wellington usava um cintilante chapéu de neve que reluzia no
sentido oeste por cima da brancura que cobria o parque Fifteen Acres.

Ele começou:

--- Senhoras e senhores.  Coube a mim esta noite, conforme sucedeu nos anos
anteriores, desempenhar uma tarefa das mais agradáveis, uma tarefa para a qual,
todavia, receio serem inadequadas minhas pobres qualidades de orador.

--- Não, não! --- disse Mr.~Browne.

--- Mas, seja como for, peço apenas que levem em conta a minha intenção e que
me ouçam por alguns momentos enquanto eu tento traduzir em palavras tudo o que
sinto esta noite.

--- Senhoras e senhores, não é a primeira vez que nos reunimos sob este teto
hospitaleiro, em torno desta mesa farta.  Não é a primeira vez que somos alvo
ou\ldots{} talvez, melhor dizendo\ldots{} vítimas da hospitalidade de certas
damas tão amáveis.

Traçou com o braço um círculo no ar e parou.  Todos riram e olharam para tia
Kate e tia Julia e Mary Jane, que enrubesceram de satisfação.  Gabriel
prosseguiu com mais segurança:

--- A cada ano que passa aumenta minha certeza de que nenhuma tradição honra
tanto o nosso país e deve ser defendida com mais fervor do que nossa
hospitalidade.  A meu ver, trata"-se de uma tradição ímpar (e tenho viajado
bastante por esse mundo afora) entre as nações modernas.  Alguns podem dizer
que, no nosso caso, essa hospitalidade é mais um defeito que uma qualidade da
qual devemos nos orgulhar.  Mesmo que isso seja verdade, trata"-se, creio eu, de
um defeito magnífico, um defeito que espero seja sempre cultivado entre nós.
De uma coisa, ao menos, estou certo.  Enquanto este teto abrigar as referidas e
amáveis senhoras --- e desejo de coração que o faça por muitos e muitos anos
--- a tradição da autêntica, calorosa e cordial hospitalidade irlandesa, a nós
legada por nossos antepassados e que por nós deve ser legada a nossos
descendentes, há de permanecer viva.

Um burburinho de entusiástica aprovação percorreu a mesa.  Passou pela mente de
Gabriel a lembrança de que Miss Ivors não se achava mais ali e que se retirara
com uma atitude pouco gentil: ele disse com autoconfiança:

--- Senhoras e senhores.  Uma nova geração cresce em nosso meio, uma geração
impulsionada por novas ideias e novos princípios.  É uma geração séria e
entusiasmada com essas novas ideias, e tal entusiasmo, mesmo quando mal
direcionado, é, a meu ver, na maioria dos casos autêntico.  Mas vivemos numa
época de ceticismo e, permitam"-me dizer, numa época atormentada pelo
pensamento: e às vezes receio que essa nova geração, intelectualizada ou
superintelectualizada, carecerá das qualidades humanas, como a hospitalidade e
o bom humor que pertencem a um tempo passado.  Ouvindo há pouco os nomes de
todos aqueles grandes cantores do passado, confesso que me pareceu estarmos
vivendo numa época menos grandiosa.  Aqueles dias, sem exagero, podem ser
chamados de grandiosos: e se ora estão além de nossa lembrança esperemos, ao
menos, que em congraçamentos como este ainda possamos recordá"-los com orgulho e
carinho, que ainda possamos trazer em nossos corações a memória daqueles mortos
ilustres cuja glória o mundo não há de deixar perecer.

--- Muito bem! --- disse Mr.~Browne em voz alta.

--- No entanto --- continuou Gabriel, a voz assumindo uma inflexão mais suave
---, em congraçamentos como este sempre nos ocorrem tristes recordações:
recordações do passado, da juventude, de como as coisas mudaram, de rostos
ausentes aqui esta noite.  Nossa caminhada pela vida é marcada por essas
lembranças tristes: e se nos permitíssemos ficar remoendo tais lembranças não
teríamos a coragem de seguir em frente com destemor na nossa luta em meio aos
vivos.  Todos temos deveres vivos e afetos vivos que requerem, e têm o direito
de requerer, a nossa incansável dedicação.

--- Por conseguinte, não me deterei no passado.  Não permitirei que reflexões
melancólicas e moralistas se intrometam em nosso meio esta noite.  Estamos aqui
reunidos para uma breve pausa em meio ao corre"-corre da nossa vida cotidiana.
Estamos aqui reunidos como amigos, com espírito de amizade, como companheiros,
e por que não dizer, até com espírito de \textit{camaraderie}, e ainda como
convidados das --- como haverei de chamá"-las? --- das Três Graças do mundo
musical de Dublin.

A mesa irrompeu em aplausos e risos diante dessa tirada.  Tia Julia,
envaidecida, perguntou às pessoas que estavam ao seu lado na mesa o que fora
mesmo que Gabriel dissera.

--- Disse que somos as Três Graças, tia Julia --- disse Mary Jane.

Tia Julia não compreendeu, mas ergueu os olhos, sorrindo para Gabriel, que
prosseguiu no mesmo tom:

--- Senhoras e senhores.  Não me atreverei a proceder aqui como procedeu Páris
em outra ocasião.  Não ousarei fazer uma escolha entre elas.  A tarefa seria ao
mesmo tempo antipática e acima de minha pobre capacidade.  Pois quando penso em
cada uma delas, seja a nossa principal anfitriã, cujo bom coração, bom até
demais, é objeto de elogio por todos que a conhecem; seja sua irmã, que parece
dotada de eterna juventude e cujo canto foi para todos nós esta noite uma
surpresa e uma revelação; ou, finalmente, quando penso em nossa mais jovem
anfitriã, talentosa, alegre, esforçada, a melhor sobrinha do mundo, confesso,
senhoras e senhores, que não sei a qual das três concederia o prêmio.

Gabriel olhou de relance para as tias e, vendo o largo sorriso estampado no
rosto de tia Julia e as lágrimas nos olhos de tia Kate, apressou"-se em concluir
o discurso.  Ergueu garbosamente a taça de vinho do Porto, enquanto todos os
presentes tomavam nas mãos as suas respectivas taças, em expectativa, e disse
com sonoridade:

--- Brindemos às três.  Brindemos à sua saúde, prosperidade, longevidade e
felicidade, e que continuem por muito tempo a ocupar a invejável posição que,
pelo seu próprio esforço, galgaram em suas profissões, bem como o lugar de
honra e carinho que ora ocupam no coração de cada um de nós.

Todos os convidados levantaram"-se, de taça na mão e, virando"-se para as três
senhoras, cantaram em uníssono, sob o comando de Mr.~Browne:

\begin{verse}\itshape
As três são boas companheiras,\\
As três são boas companheiras,\\
As três são boas companheiras,\\
Ninguém pode negar.
\end{verse}

Tia Kate recorria abertamente ao lenço e até tia Julia parecia comovida.
Freddy Malins marcava o ritmo com o garfo do pudim e os cantores
entreolhavam"-se, como numa conversa melodiosa, enquanto concluíam
enfaticamente:

\begin{verse}\itshape
Mentiras não podem pregar,\\
Mentiras não podem pregar.
\end{verse}

Então, voltando"-se novamente para as anfitriãs, repetiram:

\begin{verse}\itshape
As três são boas companheiras,\\
As três são boas companheiras,\\\clearpage
As três são boas companheiras,\\
Ninguém pode negar.
\end{verse}

A aclamação encontrou eco além da porta da sala de jantar, por parte de vários
outros convidados, e Freddy Malins, de garfo em punho, continuava a reger.

\smallskip

\noindent\dotfill

\smallskip

O ar cortante da madrugada entrou no \textit{hall} em que eles estavam e tia
Kate disse:

--- Fechem a porta, por favor.  Mrs.~Malins vai pegar um baita resfriado.

--- O Browne está lá fora, tia Kate --- disse Mary Jane.

--- O Browne está em toda parte --- disse tia Kate, baixando o volume da voz.

Mary Jane riu do tom de voz da tia.

--- Ora, tia Kate --- ela disse com certa malícia.  --- Ele até que é bastante
atencioso.

--- Grudou aqui como esparadrapo --- disse tia Kate no mesmo tom --- durante
todo o Natal.

Riu de suas próprias palavras, dessa vez num tom mais brando, e então
acrescentou rapidamente:

--- Mas peça a ele que entre, Mary Jane, e feche a porta.  Queira Deus que ele
não me tenha ouvido.

Naquele momento a porta do \textit{hall} se abriu e Mr.~Browne entrou,
dobrando"-se de rir.  Vestia um sobretudo verde comprido, com punhos e gola
imitando astracã, e trazia à cabeça um gorro de pele ovalado.  Apontou em
direção ao cais coberto de neve de onde vinha um assobio longo e sibilante.

--- Teddy vai atrair todos os coches de Dublin --- ele disse.

Gabriel surgiu da despensa, vestindo com dificuldade o sobretudo e, olhando a
sua volta no \textit{hall}, disse:

--- A Gretta ainda não desceu?

--- Foi apanhar as coisas dela, Gabriel --- disse tia Kate.

--- Quem está tocando lá em cima? --- perguntou Gabriel.

--- Ninguém.  Todos já se foram.

--- Não, não, tia Kate --- disse Mary Jane.  --- Bartell D’Arcy e Miss
O’Callaghan ainda não se foram.

--- Em todo caso, alguém está brincando ao piano --- disse Gabriel.

Mary Jane olhou para Gabriel e para Mr.~Browne e disse sentindo um arrepio:

--- Só de olhar pra vocês dois encapotados assim já sinto frio.  Eu não
gostaria de estar no lugar de vocês e ter de ir pra casa numa hora dessas.

--- Pois nada me agradaria tanto neste momento --- disse Mr.~Browne
intrepidamente --- quanto uma estimulante caminhada no campo ou um passeio numa
charrete em alta velocidade puxada por um animal fogoso.

--- Nós tínhamos um ótimo cavalo e uma charrete --- disse tia Julia com ar
melancólico.

--- O inesquecível Johnny --- disse Mary Jane, rindo.

Tia Kate e Gabriel também riram.

--- Ora!  O que havia de tão extraordinário nesse Johnny? --- perguntou 
Mr.~Browne.

--- O falecido e saudoso Patrick Morkan, ou seja, nosso avô --- explicou
Gabriel ---, conhecido no fim da vida como o velho cavalheiro, era fabricante
de cola.

--- Um momento, Gabriel --- disse tia Kate, rindo"---, ele fabricava goma de
amido.

--- Seja lá cola ou goma --- disse Gabriel ---, o velho tinha um cavalo chamado
Johnny.  E Johnny trabalhava pro velho, dando volta atrás de volta pra ativar o
moinho.  Até aí, tudo bem; mas agora vem o lado trágico da história do Johnny.
Um belo dia o velho cavalheiro resolveu acompanhar a fina flor local para
assistir a uma parada militar no parque.

--- Que Deus tenha piedade da alma dele --- disse tia Kate com misericórdia.

--- Amém --- disse Gabriel.  --- Então, como eu dizia, o velho cavalheiro
arreou Johnny e pôs a melhor cartola e o melhor colarinho e partiu em grande
estilo, saindo, da mansão de seus antepassados, perto de Back Lane, creio eu.

Todos riram, até Mrs.~Malins, da maneira como Gabriel contava a história e tia
Kate disse:

--- Um momento, Gabriel, na verdade, ele não morava em Back Lane.  O moinho é
que ficava lá.

--- E da mansão dos antepassados --- continuou Gabriel --- partiu ele, montado
em Johnny.  E tudo correu muito bem até que Johnny avistou a estátua do Rei
William: e se ele se apaixonou pelo cavalo montado pelo rei ou se achou que
estava de volta ao moinho, ninguém sabe, mas o fato é que começou a dar voltas
em torno da estátua.

Já tendo calçado as galochas, Gabriel deu uma volta pelo \textit{hall} em meio
ao riso geral.

--- Uma volta, e outra volta, lá ia ele --- disse Gabriel --- e o velho
cavalheiro, que era todo metido a elegante, ficou indignado: \textit{Vamos,
animal!  O que significa isso?  Johnny! Johnny!  Que modos são esses!  Vai entender esse cavalo!}

As gargalhadas que se seguiram à encenação que Gabriel fez do incidente foram
interrompidas por sonoras batidas à porta da rua.  Mary Jane correu e fez
entrar Freddy Malins.  Freddy Malins, com o chapéu no alto da cabeça e os
ombros encolhidos de frio, bufava em virtude do esforço despendido.

--- Só consegui um coche --- disse ele.

--- Ah, nós pegamos um mais abaixo no cais --- disse Gabriel.

--- Isso mesmo --- disse tia Kate.  --- É melhor não deixar Mrs.~Malins exposta
a correntes de ar.

Mrs.~Malins desceu os degraus da entrada da casa amparada pelo filho e por 
Mr.~Browne e, com muita dificuldade, foi içada para dentro de um coche.  Freddy
Malins também entrou no coche com dificuldade e levou um bom tempo para
acomodá"-la no assento, tudo sob a orientação de Mr.~Browne.  Finalmente ela
foi acomodada e Freddy Malins convidou Mr.~Browne a subir no coche.  Após um
longo falatório, Mr.~Browne entrou no veículo.  O cocheiro ajeitou a manta
sobre os joelhos, e curvou"-se para ouvir o endereço.  O falatório aumentou e o
homem recebeu instruções conflitantes de Freddy Malins e de Mr.~Browne, cada
qual com a cabeça do lado de fora de uma janela do coche.  O problema era
decidir em que ponto do caminho deixariam Mr.~Browne, e tia Kate, tia Julia e
Mary Jane, de pé na porta da casa, davam seus palpites em meio a instruções
truncadas, contradições e muito riso.  Quanto a Freddy Malins, já não podia
falar de tanto rir.  Esticava a cabeça pela janela e voltava a se sentar, a
cada instante, arriscando perder o chapéu, e informava à mãe sobre o progresso
das negociações, até que finalmente Mr.~Browne deu um berro, assustando o
cocheiro e abafando as gargalhadas:

--- Sabe onde fica o Trinity College?

--- Sei, sim, senhor --- disse o cocheiro.

--- Então, vá direto em direção aos portões do Trinity College --- disse 
Mr.~Browne --- e de lá nós lhe diremos como prosseguir.  Entendeu?

--- Sim, senhor --- disse o cocheiro.

--- Então, voando para o Trinity College!

--- Certo, senhor --- disse o cocheiro.

O cavalo levou uma chibatada e o coche partiu sacolejando pelo cais em meio a
um coro de risadas e despedidas.

Gabriel não fora até a porta juntar"-se aos outros.  Ficara numa parte escura do
\textit{hall} olhando escada acima.  Uma mulher estava de pé próxima ao
primeiro patamar, igualmente na penumbra.  Ele não conseguia ver o rosto dela
mas os babados da saia em tons terra e salmão eram visíveis e na sombra
pareciam preto e branco.  Era sua esposa.  Estava encostada na balaustrada
ouvindo alguma coisa.  Gabriel surpreendeu"-se ao vê"-la tão imóvel e prestou
atenção para ouvir também.  Mas pouco ouvia a não ser o rumor das risadas e da
conversa vindo da porta da rua, alguns acordes de piano e algumas notas
cantadas por uma voz masculina.

Deteve"-se ali na penumbra do \textit{hall}, tentando identificar a canção
que a voz entoava e com o olhar fixo na mulher.  Havia em sua atitude graça e
mistério como se ela fosse símbolo de algo.  Perguntou a si mesmo o que poderia
simbolizar uma mulher na penumbra, no topo de uma escada, ouvindo música ao
longe.  Se fosse pintor ele a retrataria naquela pose.  O chapéu de feltro azul
ressaltaria o bronze dos cabelos contrastando com o fundo escuro e os babados
escuros da saia contrastariam com os mais claros.  \textit{Música ao longe},
ele intitularia o quadro se fosse pintor.

A porta foi fechada; e tia Kate, tia Julia e Mary Jane entraram pelo
\textit{hall}, ainda rindo.

--- Mas o Freddy não é impossível? --- disse Mary Jane.  --- É mesmo
impossível.

Gabriel nada disse mas apontou para a parte superior da escada onde se via sua
esposa.  Agora que a a porta da rua estava fechada a voz e o piano eram mais
audíveis.  Gabriel fez um sinal com a mão para que elas ficassem caladas.  A
canção parecia composta numa tonalidade irlandesa arcaica e o cantor estava
inseguro tanto em relação à letra quanto à voz.  A voz, que parecia dorida em
virtude da distância em que se encontrava o cantor bem como de uma forte
rouquidão, tentava em vão ressaltar a cadência da melodia com palavras que
exprimiam sofrimento:

\begin{verse}\itshape
Ah, a chuva cai nos meus espessos cachos\\
E o orvalho umedece a minha pele,\\
Meu filho jaz enregelado\ldots{}
\end{verse}

--- Ah! --- exclamou Mary Jane.  É Bartell D’Arcy cantando, e a noite inteira
ele tinha se recusado a cantar.  Ah, mas vou obrigá"-lo a cantar uma canção
antes de ir embora.

--- Ah, isso mesmo, Mary Jane --- disse tia Kate.

Mary Jane adiantou"-se rapidamente e correu até a escada, mas antes mesmo de
alcançar o primeiro degrau o canto foi interrompido e o teclado do piano foi
fechado bruscamente.

--- Ah, que pena! --- ela exclamou.  --- Ele está descendo, Gretta?

Gabriel ouviu a mulher responder que sim e a viu descer a escada em sua
direção.  Logo atrás vinham Mr.~Bartell D’Arcy e Miss O’Callaghan.

--- Ah, Mr.~D’Arcy! --- exclamou Mary Jane.  --- Que maldade sua parar de
cantar logo agora que nós estávamos entusiasmados ouvindo o senhor.

--- Insisti com ele a noite toda --- disse Miss O’Callaghan ---, Mrs.~Conroy e
eu, e ele nos disse que estava com um resfriado terrível e que não podia
cantar.

--- Ah, Mr.~D’Arcy --- disse tia Kate ---, mas que balela!

--- Vocês não perceberam que estou rouco como uma arara? --- disse Mr.~D’Arcy
com aspereza.

Ele entrou na despensa às pressas e vestiu o sobretudo.

Os demais, desconcertados diante da resposta grosseira, não sabiam o que dizer.
Tia Kate franziu a testa e fez sinal para os outros indicando que deixassem o
assunto de lado.  Mr.~D’Arcy, com o semblante carregado, agasalhava
cuidadosamente o pescoço com o cachecol.

--- É esse tempo --- disse tia Julia, depois de uma pausa.

--- Pois é, todo mundo anda resfriado --- disse tia Kate prontamente ---, todo
mundo.

--- Estão dizendo --- acrescentou Mary Jane --- que faz trinta anos que não
neva desse jeito; e eu li esta manhã no jornal que está nevando em toda a
Irlanda.

--- Adoro ver neve --- disse tia Julia com um ar triste.

--- Eu também --- disse Miss O’Callaghan.  --- Acho que Natal só é Natal com
neve na rua.

--- Mas o pobre Mr.~D’Arcy não gosta de neve --- disse tia Kate, sorrindo.

Mr.~D’Arcy surgiu da despensa, todo enrolado e abotoado, e em tom de
arrependimento explicou"-lhes como apanhara o resfriado.  Todos deram"-lhe
conselhos e disseram que era uma pena vê"-lo assim e que deveria proteger bem a
garganta do sereno.  Gabriel observava a esposa, que não entrara na conversa.
Ela estava de pé exatamente embaixo da claraboia empoeirada e a chama da
lamparina iluminava"-lhe o bronze dos cabelos, cabelos que poucos dias atrás ele
a vira secar diante da lareira.  Ela permanecia imóvel e parecia não se dar
conta da conversa a sua volta.  Finalmente voltou"-se para eles e Gabriel notou
que nas faces dela havia um rubor e que seus olhos brilhavam.  Uma súbita onda
de alegria invadiu"-lhe o coração.

--- Mr.~D’Arcy --- ela disse ---, qual é o nome dessa canção que o senhor
estava cantando?

--- \textit{The Lass of Aughrim}\footnote{ Isto é, “A jovem de Aughrim”.} 
--- disse Mr.~D’Arcy ---, mas não consegui me lembrar da letra direito.  Por
quê? A senhora conhece a canção?

--- \textit{The Lass of Aughrim} --- ela repetiu.  --- Não conseguia me lembrar
do nome.

--- É uma bela canção --- disse Mary Jane.  --- É uma pena o senhor estar sem
voz hoje.

--- Ora, Mary Jane --- disse tia Kate ---, não amole Mr.~D’Arcy.  Não admito
que o amolem.

Vendo que estavam prontos para sair, ela os conduziu até a porta, onde se
despediram:

--- Então, boa"-noite, tia Kate, e obrigado, foi uma noite tão agradável.

--- Boa"-noite, Gabriel.  Boa"-noite, Gretta!

--- Boa"-noite, tia Kate, e muito obrigada.  Boa"-noite, tia Julia.

--- Ah, boa"-noite, Gretta; eu não tinha visto você.

--- Boa"-noite, Mr.~D’Arcy.  Boa"-noite, Miss O’Callaghan.

--- Boa"-noite, Miss Morkan.

--- Mais uma vez, boa"-noite.

--- Boa"-noite a todos.  Bom regresso.

--- Boa"-noite.  Boa"-noite.

A madrugada ainda estava escura.  Uma luz pálida e amarelada pairava sobre as
casas e o rio; e o céu parecia estar despencando.  Neve derretida misturava"-se
com lama sob os pés, e havia apenas filetes e tufos de neve nos telhados, na
mureta do cais e nas grades.  Os lampiões ainda ardiam refletindo sua luz
avermelhada no ar brumoso e, do outro lado do rio, o palácio Four Courts
erguia"-se imponente desafiando o céu carregado.

Ela caminhava à frente dele ao lado de Mr.~Bartell D’Arcy, trazendo os sapatos
debaixo do braço dentro de um embrulho marrom e suspendendo a barra da saia com
as mãos para não sujá"-la.  Ela já não exibia uma postura elegante, mas os olhos
de Gabriel ainda brilhavam de felicidade.  O sangue latejava"-lhe nas veias; e
os pensamentos precipitavam"-se em sua mente, orgulhosos, felizes, carinhosos,
intrépidos.

Ela caminhava à frente dele tão leve e tão ereta que ele desejava alcançá"-la na
surdina, agarrá"-la pelos ombros e sussurrar"-lhe ao ouvido algo tolo e
apaixonado.  Parecia"-lhe tão frágil que tinha ímpetos de defendê"-la de um
perigo qualquer e então ficar a sós com ela.  Momentos da vida íntima dos dois
irromperam"-lhe na memória como estrelas.  Um envelope lilás ao lado da xícara
de café e ele afagando o papel com a mão.  Passarinhos cantando na hera e o
reflexo ensolarado da cortina tremeluzindo no assoalho: na ocasião sentira"-se
tão feliz que não conseguira comer.  Os dois numa plataforma repleta de gente e
ele pressionando o tíquete da passagem contra a palma de sua mão cálida e
enluvada.  Os dois lado a lado no frio, espiando através de uma janela gradeada
um homem fazendo garrafas de vidro diante de um forno ardente.  Fazia muito
frio.  A face dela, fragrante no ar frio, estava bem próxima à dele; e
subitamente ele se dirigiu ao sujeito que estava diante do forno:

--- O fogo está quente, moço?

Mas o homem não pôde ouvi"-lo devido ao ruído do forno.  Foi melhor assim.  Ele
poderia ter dado uma resposta grosseira.

Uma onda de felicidade terna e ainda mais intensa emergiu do coração dele e
percorreu"-lhe as artérias numa cálida torrente.  Como o brilho terno das
estrelas, momentos da vida deles juntos, dos quais ninguém tinha e jamais teria
conhecimento, precipitavam"-se e iluminavam"-lhe a memória.  Ele ansiava por
fazê"-la lembrar daqueles momentos, fazê"-la esquecer os anos de sua insípida
vida conjugal e lembrar somente dos momentos de êxtase.  Pois os anos, a seu
ver, não tinham aniquilado suas almas.  Os filhos, os escritos dele, os
afazeres domésticos dela não tinham apagado a terna chama que traziam na alma.
Em uma carta que escrevera a ela àquela época ele dissera: \textit{Por
que será que palavras como estas parecem"-me tão insípidas e
frias? Será porque não existe palavra terna o bastante para expressar
o seu nome?}

Como música ao longe essas palavras por ele escritas anos antes chegavam até
ele vindas do passado.  Ansiava por ficar a sós com ela.  Quando todos tivessem
ido embora, quando estivessem no quarto do hotel, então estariam a sós.  Ele
pronunciaria seu nome à meia"-voz:

--- Gretta!

Talvez ela não escutasse na primeira vez: estaria se despindo.  Então algo na
voz dele chamaria sua atenção.  Ela se voltaria para olhá"-lo\ldots{}

Na esquina de Winetavern Street encontraram um coche.  O ruído do coche
sacolejante proporcionou"-lhe alívio pois tinha uma desculpa para não puxar
conversa.  Ela olhava pela janela e parecia cansada.  Os demais falavam pouco,
apontando algum edifício ou alguma rua.  O cavalo galopava exausto sob o céu
sombrio da madrugada, puxando o coche sacolejante, e Gabriel sentiu"-se
novamente ao lado dela num coche, galopando para embarcar no navio, galopando
para a lua de mel.

Quando o coche passou pela O’Connell Bridge, Miss O’Callaghan disse:

--- Dizem que sempre que a gente atravessa a O’Connell Bridge vê um cavalo
branco.

--- Desta vez estou vendo um homem branco --- disse Gabriel.

--- Onde? --- perguntou Mr.~Bartell D’Arcy.

Gabriel apontou para a estátua, parcialmente coberta pela neve.  Então
cumprimentou a estátua com um meneio de cabeça e um aceno.

--- Boa"-noite, Dan --- ele disse em tom jocoso.

Quando o coche parou à entrada do hotel Gabriel desceu rapidamente e, apesar
dos protestos de Mr.~Bartell D’Arcy, pagou a corrida.  Deu ao cocheiro um
\textit{shilling} de gorjeta.  O homem agradeceu e disse:

--- Um próspero Ano Novo pro senhor, doutor.

--- Pra você, também --- disse Gabriel cordialmente.

Ao descer do coche ela se apoiou um instante no braço dele e assim permaneceu
enquanto se despedia dos outros na calçada.  Apoiava"-se levemente no braço
dele, com a mesma leveza com que o fizera quando dançaram juntos algumas horas
antes.  Naquele momento ele tinha se sentido orgulhoso e feliz, feliz por tê"-la
para si, orgulhoso de sua postura graciosa e digna.  Mas agora, após tantas
memórias agradáveis, o primeiro toque do corpo dela, musical e exótico e
perfumado, despertou nele um desejo ardente.  Protegido pelo silêncio dela
pressionou"-lhe o braço contra seu corpo; e, em frente à porta do hotel, ele
teve a sensação de que haviam escapado de suas próprias vidas e de suas
obrigações, escapado do lar e dos amigos e fugido juntos com corações
exultantes para uma nova aventura.

Um velho cochilava numa enorme poltrona no saguão de entrada.  Ele acendeu uma
vela e guiou"-os até a escada.  Os dois seguiram"-no em silêncio, com passos
suaves sobre o espesso tapete que cobria os degraus.  Ela subia logo atrás do
porteiro, de cabeça baixa, com os ombros delicados caídos como se suportassem
um peso, e com a saia firmemente segura.  Ele desejava abraçá"-la na altura dos
quadris e suspendê"-la no ar, pois seus braços tremiam de desejo de tocá"-la e
somente cravando as unhas na palma da mão pôde ele conter o impulso de
arrebatá"-la.  O porteiro parou no meio da escada para endireitar a vela
gotejante.  Eles também pararam degraus abaixo.  No silêncio Gabriel podia
ouvir a cera derretida pingando no pratinho e o coração batendo"-lhe no peito.

O porteiro conduziu"-os por um corredor e abriu uma porta.  Então colocou sobre
a mesa de cabeceira a vela instável e perguntou a que horas gostariam de ser
despertados.

--- Às oito --- disse Gabriel.

O porteiro apontou para o interruptor de luz elétrica e começou a balbuciar uma
desculpa mas Gabriel interrompeu"-o.

--- Não queremos luz nenhuma.  A luz que vem da rua já basta.  E digo mais ---
acrescentou, apontando para a vela ---, o senhor pode levar essa beleza de
vela, por gentileza.

O porteiro pegou a vela, ainda que lentamente, pois ficou surpreso com a ordem
inusitada.  Em seguida murmurou ``boa"-noite'' e retirou"-se.  Gabriel passou o
trinco na porta.

Uma luz espectral do lampião da rua formava um longo feixe que ia da janela até
a porta.  Gabriel atirou sobre o sofá o sobretudo e o chapéu e atravessou o
quarto em direção à janela.  Ficou olhando para a rua no intuito de abrandar um
pouco a emoção que sentia.  Então virou"-se e encostou"-se na cômoda de costas
para a luz.  Ela retirara o chapéu e o casaco e estava de pé diante de um
grande espelho giratório, abrindo os colchetes do cós da saia.  Gabriel
deteve"-se um momento, observando"-a, e então disse:

--- Gretta!

Ela se afastou lentamente do espelho e caminhou em direção a ele dentro do
feixe de luz.  Tinha o rosto tão grave e cansado que Gabriel não conseguiu
pronunciar uma palavra sequer.  Não, ainda não era o momento.

--- Parece cansada --- ele disse.

--- Um pouco --- ela respondeu.

--- Sentindo alguma coisa?

--- Não, cansaço: só isso.

Ela foi até a janela e ali se posicionou, olhando a rua.  Gabriel esperou um
pouco mais e então, temeroso de que a timidez o dominasse, disse bruscamente:

--- A propósito, Gretta!

--- Sim?

--- Sabe, o coitado do Malins --- ele disse rapidamente.

--- Sim. O que tem ele?

--- Pois é, coitado, até que é um sujeito decente --- prosseguiu Gabriel num
tom de voz forçado.  --- Pagou aquele dinheiro que me devia, e sinceramente eu
nem esperava que fosse pagar.  É uma pena ele não sair do lado do Browne,
porque no fundo ele não é mau sujeito.

Sentia"-se trêmulo de contrariedade.  Por que se mostrava ela tão distante?  Não
sabia como começar.  Será que ela também estava contrariada por algum motivo?
Se ao menos se voltasse para ele ou viesse ao seu encontro de livre e
espontânea vontade!  Arrebatá"-la naquele estado seria brutal.  Não, aguardaria
até ver um pouco de ardor nos olhos dela.  Ansiava por desvendar o motivo do
estranho estado de espírito em que ela se encontrava.

--- Quando foi que você lhe emprestou dinheiro? --- ela perguntou, após uma
pausa.

Gabriel esforçou"-se para não explodir num linguajar grosseiro a respeito do
bêbado do Malins e do dinheiro.  Queria gritar"-lhe do fundo da alma, apertar o
corpo dela contra o seu, subjugá"-la.  Mas disse:

--- Ah, no Natal, quando ele abriu aquela lojinha de cartões de boas"-festas em
Henry Street.

Estava em tal estado de exaltação e desejo que não ouviu quando ela se
aproximou, vindo da janela.  Ela parou um instante diante dele, fitando"-o com
um olhar estranho.  Então, colocando"-se subitamente na ponta dos pés e
repousando as mãos suavemente nos ombros dele, ela o beijou.

--- Você é uma pessoa muito generosa, Gabriel --- ela disse.

Gabriel, trêmulo de prazer diante do beijo inesperado e das palavras
inusitadas, passou levemente as mãos nos cabelos dela, quase sem tocá"-los.  Os
cabelos estavam lavados e viçosos.  O coração dele transbordava de felicidade.
Exatamente quando mais a desejava ela viera por vontade própria.  Talvez os
pensamentos dela estivessem correndo paralelamente aos dele.  Talvez ela
tivesse sentido o desejo impetuoso que o consumia, e então resolvera ceder.
Agora que ela se rendera tão facilmente, ele se perguntava por que tinha sido
tão tímido.

Segurou o rosto dela entre as mãos.  Então, envolvendo"-a com um dos braços e
trazendo"-a para junto de si, disse à meia"-voz:

--- Gretta, querida, em que você está pensando?

Ela não respondeu e tampouco cedeu inteiramente ao abraço.  Ele repetiu, à meia"-voz:

--- Conta pra mim, Gretta.  Acho que já sei do que se trata.  Será que sei?

Ela não respondeu de imediato.  Então disse em meio a uma explosão de lágrimas:

--- Ah, estou pensando naquela canção, \textit{The Lass of Aughrim}.  

Desvencilhou"-se dos braços dele e correu para a cama e, agarrando"-se à
cabeceira, escondeu o rosto.  Gabriel ficou atônito durante alguns instantes e
então foi ao encontro dela.  Quando passou na frente do espelho giratório viu
sua própria figura, em corpo inteiro, o tórax largo e robusto, o rosto cuja
expressão sempre o intrigava quando diante de um espelho e os óculos dourados,
brilhantes.  Deteve"-se a alguns passos dela e disse:

--- O que tem a ver a canção?  Por que ela te faz chorar?

Ela ergueu a cabeça que estava apoiada nos braços e enxugou os olhos com as
costas da mão como uma criança.  A voz dele assumiu um tom mais benévolo do que
ele pretendia.

--- Por que, Gretta? --- ele perguntou.

--- Estou pensando numa pessoa que muito tempo atrás costumava cantar aquela
canção.

--- E que pessoa foi essa muito tempo atrás? --- perguntou Gabriel, sorrindo.

--- Foi uma pessoa que conheci em Galway quando eu morava com minha avó --- ela
disse.

O sorriso desapareceu do rosto de Gabriel.  Uma raiva maçante voltou a se
instalar em sua mente e o calor maçante do desejo voltou a esquentar"-lhe as
veias.

--- Alguém por quem você esteve apaixonada? --- ele perguntou ironicamente.

--- Foi um rapaz que eu conheci --- ela respondeu ---, chamado Michael Furey.
Ele costumava cantar essa canção, \textit{The Lass of Aughrim}.  Ele era muito
sensível.

Gabriel ficou calado.  Não queria que ela pensasse que ele estava interessado
no tal rapaz sensível.

--- Lembro"-me tão bem dele --- ela disse após um momento.  --- Que olhos ele
tinha: olhos grandes, negros!  E que expressão, que expressão!

--- Ah, então você está apaixonada por ele? --- disse Gabriel.

--- Eu costumava sair pra caminhar com ele --- ela disse --- na época em que
morava em Galway.

Um pensamento passou pela mente de Gabriel.

--- Então era por isso que você queria ir a Galway com Molly Ivors? --- ele
disse com frieza.

Ela olhou para ele e perguntou espantada:

--- Para quê?

O olhar dela o desconcertou.  Ele deu de ombros e disse:

--- Sei lá eu?  Pra ver o tal rapaz, talvez.

Ela desviou o olhar em direção ao feixe de luz e à janela e permaneceu em
silêncio.

--- Ele está morto --- ela disse finalmente.  --- Morreu aos dezessete anos de
idade.  Não é terrível morrer tão jovem assim?

--- O que ele fazia na vida? --- perguntou Gabriel, ainda com ironia.

--- Trabalhava no gasômetro --- ela disse.

Gabriel sentiu"-se diminuído pelo fracasso de sua ironia e pela evocação da
figura do morto, um garoto que trabalhava no gasômetro.  Enquanto ele revivia
as lembranças da vida íntima do casal, cheio de ternura e alegria e desejo, ela
o comparava mentalmente com um outro homem.  Uma grande sensação de insegurança
o assaltou.  Via"-se como uma figura ridícula, um menino fazendo gracinhas para
as tias, um sentimental nervoso e ingênuo, discursando para plebeus e
idealizando seus próprios desejos ridículos, o sujeito presunçoso que vira
refletido no espelho.  Instintivamente deu as costas para a luz com receio de
que ela visse a vergonha que lhe queimava a fronte.

Procurou manter o tom frio de interrogatório mas quando voltou a falar a voz
soou humilde e inócua.

--- Imagino que você esteve apaixonada por esse Michael Furey, Gretta --- ele
disse.

--- Fui feliz ao lado dele naquela época --- ela disse.

Tinha a voz velada e triste.  Gabriel, dando"-se conta de que seria inútil
tentar levá"-la na direção em que pretendera, acariciou a mão dela e disse,
igualmente triste:

--- E ele morreu de quê, Gretta, tão jovem?  Foi tuberculose?

--- Acho que morreu por mim --- ela respondeu.

Ao ouvir a resposta um vago terror apossou"-se de Gabriel como se, no momento em
que esperava triunfar, algum ser intangível e vingativo o atacasse, reunindo em
seu mundo obscuro forças para atacá"-lo.  Mas, procurando agir racionalmente,
livrou"-se da sensação e continuou a acariciar a mão dela.  Não fez mais
perguntas, pois achava que tudo se revelaria espontaneamente.  A mão dela
estava cálida e úmida: não respondia ao seu toque, mas ele continuou a
acariciá"-la assim como tinha acariciado a primeira carta dela naquela manhã de
primavera.

--- Foi no inverno --- ela disse ---, no início do inverno, quando eu estava
prestes a deixar a casa de minha avó pra vir estudar aqui no colégio de
freiras.  Ele estava adoentado na pensão em Galway e não o deixavam sair, e
escreveram pra  família dele, em Oughterard.  Estava definhando, foi o que
disseram, ou algo assim.  Eu nunca soube ao certo.

Fez uma pausa e suspirou.

--- Pobre rapaz --- ela prosseguiu.  --- Gostava muito de mim e como era meigo.
Costumávamos sair juntos, caminhando, você sabe, como se faz no interior.  Ele
ia estudar canto por motivo de saúde.  Tinha uma ótima voz, pobre Michael
Furey.

--- Sim, e daí? --- perguntou Gabriel.

--- E daí quando chegou o momento de eu ir embora de Galway e vir pro colégio
interno ele piorou muito e não me deixaram vê"-lo, então escrevi uma carta
dizendo que ia pra Dublin e que voltaria no verão e que quando voltasse
esperava vê"-lo bem melhor.

Fez uma pausa para controlar a voz e então prosseguiu:

--- Então, na véspera da partida, eu estava na casa da minha avó em Nun’s
Island, fazendo as malas, quando ouvi uma pedrinha bater na vidraça.  A janela
estava tão úmida que eu não conseguia ver nada lá fora, então do jeito que eu
estava, desci a escada correndo e saí pela porta dos fundos e lá estava o pobre
coitado, no fundo do quintal, tiritando.

--- E você não disse a ele que voltasse pra casa? --- perguntou Gabriel.

--- Eu implorei que voltasse imediatamente pra casa e disse que a chuva ia
acabar com ele.  Mas ele disse que não queria viver.  Lembro"-me muito bem dos
olhos dele, muito bem!  Lá estava ele no canto do muro ao lado de uma árvore.

--- E ele foi pra casa? --- perguntou Gabriel.

--- Sim, foi pra casa.  E quando eu completei uma semana no colégio interno ele
morreu e foi enterrado em Oughterard, que é a terra da família dele.  Ah, o dia
em que recebi a notícia que\ldots{} que ele estava morto!

Ela parou, sufocada pelos soluços e, prostrada pela emoção, atirou"-se de bruços
sobre a colcha, soluçando.  Gabriel ainda lhe segurou a mão um pouco, indeciso,
e então, constrangido por imiscuir"-se na dor da mulher, largou gentilmente a
mão e caminhou em silêncio até a janela.

\smallskip

\noindent\dotfill

\smallskip

Ela adormecera.

Gabriel, apoiado no cotovelo, olhou um instante sem ressentimento para os
cabelos emaranhados da mulher e para a boca entreaberta, e ouviu sua respiração
profunda.  Então ela vivenciara aquele romance: um homem morrera por sua causa.
Pouco lhe doía agora o papel sem importância que ele, o marido, desempenhara na
vida dela.  Olhava para ela adormecida, como se os dois jamais tivessem vivido
juntos como marido e mulher.  Seus olhos curiosos fitaram longamente o rosto e
os cabelos dela: e, ao imaginar como ela fora na época em que era dotada de uma
beleza infantil, um estranho sentimento de compaixão invadiu"-lhe a alma.  Não
ousava dizer nem para si mesmo que o rosto dela já não era belo, mas sabia que
já não era o rosto pelo qual Michael Furey enfrentara a morte.

Talvez ela não tivesse lhe contado a história inteira.  Desviou o olhar para a
cadeira onde ela atirara algumas peças de roupa.  Uma alça de combinação pendia
sobre o chão.  Uma bota estava de pé, com o cano caído: a outra estava tombada
ao lado.  Lembrou"-se do turbilhão de emoções que sentira uma hora atrás.  De
onde surgira tudo aquilo?  Da festa na casa das tias, do discurso idiota, do
vinho e da dança, das despedidas alegres no \textit{hall}, do prazer da
caminhada na neve ao longo do rio.  Pobre tia Julia!  Ela também em breve seria
um espectro ao lado do espectro de Patrick Morkan e seu cavalo.  Ele bem que
notara o olhar abatido da senhora quando cantou \textit{Arrayed for the
Bridal}.  Talvez em breve ele estaria sentado naquele mesmo salão, de luto, com
o chapéu apoiado sobre os joelhos.  As cortinas estariam fechadas e tia Kate
estaria sentada ao seu lado, chorando e assoando o nariz e contando como Julia
morrera.  Ele procuraria palavras de consolo, e encontraria somente frases
banais e inúteis.  Sim, sim: isso aconteceria em breve.

O ar dentro do quarto gelou seus ombros.  Ele se esticou cuidadosamente embaixo
dos lençóis e ficou deitado ao lado da esposa.  Um por um estavam todos
transformando"-se em espectros.  Seria preferível passar para o outro mundo de
maneira corajosa, na glória de uma paixão, que murchar e secar lentamente na
velhice.  Ele pensou no fato de que aquela que estava deitada ao seu lado
ocultara no coração durante tantos anos aquela imagem dos olhos do amado
dizendo a ela que não queria viver.

Lágrimas abundantes encheram"-lhe os olhos.  Ele próprio jamais tivera esse tipo
de sentimento em relação a uma mulher mas sabia que aquilo era amor.  Mais
lágrimas vieram"-lhe aos olhos e na penumbra ele imaginou ver a figura de um
rapaz parado embaixo de uma árvore pingando.  Havia outras figuras em volta.  A
alma dele se acercara da região habitada pela vasta legião dos mortos.  Ele
pressentia a existência errática e perambulante dos mortos, embora fosse
incapaz de apreendê"-la.  Sua própria identidade desaparecia num mundo cinzento
e incorpóreo: o mundo sólido, antes construído e habitado por esses mortos,
dissolvia"-se e se esvaía.

Leves batidas na vidraça fizeram"-no virar"-se para a janela.  Recomeçava a
nevar.  Sonolento, ele observou os flocos prateados e escuros precipitando"-se
obliquamente contra a luz do lampião.  Chegara o momento de iniciar a viagem
para o oeste.  Sim, os jornais tinham acertado: a neve cobria toda a Irlanda.
Precipitava"-se por toda a sombria planície central, nas montanhas sem árvores,
precipitava"-se suavemente sobre o Bog de Allen e, mais para o oeste, suavemente
se precipitava sobre as ondas escuras e traiçoeiras do Shannon.  Precipitava"-se
também no cemitério solitário da colina onde jazia Michael Furey.  Acumulava"-se
sobre as cruzes inclinadas e sobre as lápides, sobre as pontas do gradil do
pequeno portão, sobre os espinhos toscos.  Sua alma desfalecia lentamente
enquanto ele ouvia a neve precipitando"-se placidamente no universo e
placidamente se precipitando, descendo como a hora final sobre todos os vivos e
os mortos.

\openright

