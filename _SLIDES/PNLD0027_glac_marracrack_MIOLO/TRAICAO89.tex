\chapterspecial{Traição}{}{}
 

Na época \versal{GS} tinha quatorze anos e vivia com a família. O~pai,
caminhoneiro, era alcoólatra e um homem ausente que, ao chegar em casa,
fazia cenas inesquecíveis: investia sobre a mãe e os irmãos de \versal{GS}.
Estupidamente autoritário e arrogante.

\versal{GS} chegou a uma conclusão: dar um basta no pai.

Um dia chamou a mãe de lado e declarou que ela não apanharia mais.
Enchendo"-se de coragem, pediu à mãe que se retirasse. Enfrentou a fera
embriagada. Que rugia palavras ferinas e caçava a mãe anunciando
porradas.

\versal{GS} proclamou a independência da mãe. Ingênuo e ousado, ele se fez
crítico rigoroso dos hábitos etílicos paternos com teimosia de menino
turrão. Depois deu as costas. Esqueceu"-se, porém, que o pai andava
armado. E~recebeu três facadas nas~costas.

Foi socorrido por um irmão. Passou um mês no hospital entre a vida e a
morte. O~pai não tomou conhecimento da sorte do filho e continuou sua
vida miserável.

\versal{GS} virou menino de rua por cinco anos. Planejou a morte do pai na forma
de um homicídio premeditado que, no entanto, descartou na prática.

Iniciou no caminho das pedras. Não demorou para que se tornasse
dependente químico. Mas a fumaça do crack não lhe trazia prazer ou
euforia. Trazia"-lhe certa ``normalidade psíquica'' sempre que ele era
tomado por uma profunda inquietação. E~ao descer aos abismos de si mesmo
fumava pedra e ficava ``normal''.

\asterisc{}

Há um mês o pai de \versal{GS} morreu. De câncer. Depois de quinze anos o filho
havia retornado para casa. Encontrou o pai no leito de morte. A~cena foi
clássica, dessas que aparecem em cinema, em novela.

O velho, agônico, pediu luz: trouxeram um clarão oblíquo que iluminou o
rosto do pai e do filho. \versal{GS}, perturbadíssimo, ficou confuso, tremeu e
achou que ia ter um desmaio. De repente ajoelhou e pediu perdão. O~pai
expirou quando murmurava alguma mensagem para o filho.

\asterisc{}

Depois do enterro do velho tudo para \versal{GS} continuou sendo culpa, culpa e
mais culpa.

\versal{GS} começou a pensar muito na morte. Virou um obcecado pela morte. Pegava
cada vez mais droga na biqueira e fechava"-se no seu quarto para usar.
Imaginava"-se dentro de um túmulo. Mas ao fumar não tinha a impressão de
que enlouquecia: pelo contrário, achava que estaria voltando ao
``normal''.

Drogava"-se cada vez mais compulsivamente. A~culpa reaparecia e o consumo
de crack aumentava. \versal{GS} ficava endividado com os traficantes.

Certo dia \versal{GS} resolveu incitar seu próprio extermínio esticando sem
limites os prazos das dívidas. Mas um traficante que era um pouco seu
amigo reagiu e parou de lhe vender droga. Proclamou, indignado, que
costumava mandar matar ``noias'' sem valor, mas não queria eliminar um pai
de família. \versal{GS} foi comprar em outra biqueira de gente desconhecida.
Acionou de novo sua roleta-russa.

Nesta altura eu decido voltar atrás na história de \versal{GS} e ``cair de pau''
em cima desse pedido de perdão ao pai. Reajo indignado. Digo que
considero sublime o perdão, até de forma cristã. Pode"-se perdoar ao
inimigo, pode"-se perdoar ao ato torpe, até ao gesto homicida. Mas o
perdão cabe quando não existe covardia. Não se pode pedir perdão a
outrem se é esse outro o criminoso.

\versal{GS} sabe disso e ao mesmo tempo não sabe.

Pois reparem vocês: vejam, como o cego Tirésias, a luz da escuridão.

Após as três facadas \versal{GS} arquitetou por muito tempo o parricídio. No
entanto, a dor na consciência pelo ato hediondo seria insuportável. Ali
havia o temor de uma maldição, como se esta viesse de alguma esfinge
enfurecida. Porque na figura de um pai, mesmo sendo ele um criminoso,
existe também a figura sagrada que nos provoca respeito e medo.

Perante essa figura \versal{GS} sucumbiu feito uma vítima que, submissa ao
extremo perante o algoz, ergue"-se da punição e beija o malfeitor. \versal{GS} fez
uma transmutação maluca da culpa e assumiu uma dívida alheia. E~se lhe
brotou uma consciência dessa troca de dívida, essa consciência estava
contaminada pela fúria das Erínias.

\versal{GS} percebeu que carregava o pai dentro de si e queria matá"-lo.
Atormentado, cedeu à força do ódio, mas antes fez um julgamento de si
próprio.

Foi juiz, promotor, advogado de defesa e júri. Como houvesse dois lados
e os dois lados empatassem, \versal{GS} deu a si mesmo o voto de Minerva.

O veredicto: culpado.

\asterisc{}

Quando \versal{GS} sai a caminho das pedras, ele não carrega tão somente uma faca
espetada nas costas. Ele é o autor das facadas. Ele é também um grande
bode expiatório que desce aos próprios infernos.

Ele é uma espécie de anti"-Orfeu e não lhe resta nenhum amor a nenhuma
Eurídice. Resta a antipedra alquímica, a pedra da qual sai a fumaça
tóxica da loucura. Fumaça que é um estranho e absurdo ``remédio'' para
lhe trazer não euforia, não alívio apenas da fissura adoidada, e nem
mesmo a pura viagem da ``noia'', mas para lhe trazer uma paz maldita.

Enquanto \versal{GS} tenta inútil e loucamente ``curar"-se'' a si próprio com a
cocaína inflamável da pedra, ele continua a encenar a roleta-russa de
sua existência. E é sempre o pai na berlinda: na hora da morte e em
qualquer dia e naquele dia.

\versal{GS}, pensativo, fixa os olhos no infinito. Depois desabafa, intensamente
dramático, e dirige suas lembranças aos seus quatorze anos.

Ah, é sempre aquela conversa medonha, é sempre aquele momento exato e
inocente de um pequeno guerreiro distraído --- em que ele, \versal{GS}, moleque
vacilão, mano, virou as costas e… Ele também, a partir daquele
dia, foi morto a traição.

\begin{center}\asterisc{}\end{center}
%\begingroup\small

\emph{Foi um texto escrito há muitos anos, antes que eu viesse a
trabalhar com adictos, e surgiu a partir de um dos meus primeiros
contatos com dependência severa de crack. O~caso me impressionou muito
pela radicalidade, despojamento mórbido e implicações psicanalíticas.}

\emph{Se, antes de mais nada, o que vem à tona é a grande e nebulosa
questão do pai, eu começo por fazer o seguinte preâmbulo: na periferia
onde exerço meu trabalho há tantos anos, o pai é o grande ausente e é
figura amada e também odiada. Mas o vácuo paterno está sempre criando
adaptações nos deserdados do pai.}

\emph{Para deixar este assunto mais claro vou citar as inúmeras
``confrarias dos manos'' onipresentes na periferia e que viram
sociedades móveis e expansivas, sociedades algo anárquicas, sem estatuto
e mantidas na ``raça'', algumas decantadas no} hip-hop \emph{e no} funk.

\emph{E se tantos dessas confrarias se chamam tanto mutuamente de
``manos'', certamente é para preencher algum vácuo do pai. E~se eles o
fazem de maneira conservadora, podem fazê"-lo também de maneira mais
ousada e radical, senão até criminosa. Mas, no fundo, trata"-se do grande
dilema do pai.}

\emph{É o que está escrito lá, brilhantemente, na Teogonia, de
Hesíodo.}

\emph{Mas, enfim, este \versal{GS} é um membro atípico e radical dessas
confrarias de deserdados, e um membro tão atípico que nem chegou a
frequentá"-las.}

\emph{A~respeito deste assunto eu digo mais. Vejo todas essas confrarias
exercendo um papel aglutinador, sendo uma espécie de ode vulgar à
ausência do pai simbólico e concreto.}

\emph{Seguindo ainda um pouco mais longe, eu imagino que até Freud
explicaria tamanho dilema porque na carência de tantos manos existe lá
no fundo da ``alma'' de todos o pai ausente como pedra no meio do
caminho; um pai cujo sacrifício porcamente totêmico é cercado de
baixarias e crueldades domésticas até assassinas. Coisa, aliás, bem
humana, porém coisa nada elevada. Coisa bem diferente de quando se fala
na morte sacrificial do pai sob um manto de civilização e permitindo a
consciência adequada de uma morte simbólica.}

\emph{Sem falar que se trata de uma situação percebida com alguma
clareza no caso de quem, por exemplo, faz boa terapia ou de quem, em
jargão psicanalítico, cresce vencendo o que se tem como um complexo de
Édipo ou algo equivalente.}

\emph{Mas eu conheço muitos casos no ambiente periférico do chamado
``povão'' em que a ``morte do pai'' se desloca do simbólico para a
brutalidade concreta, de tal maneira que, no burburinho da vida pequena
a ``barra'' pode ficar pesadíssima.}

\emph{É nesse meio que conflitos bárbaros e mesquinhos trazem, na
rabeira da miséria, retaliações chegando a uma aniquilação curta e
grossa do filho em ambientes onde irrompem saturnos devoradores sem
majestade, sem limites e quase sempre bêbados.}

\emph{É um território onde o cacete baixa na rotina do dia a dia e onde
as sutilezas simbólicas de qualquer discurso são carregadas na enxurrada
comum das desavenças alimentadas pela cachaça diária; desavenças
vinculadas a abandonos e esquecimentos, ainda mais partindo do paizão
machista meio rude que cada vez mais deixa de ser o chefe porque uma
grande parte das famílias é comandada pela mãe.}

\emph{Em muitos ambientes domésticos desestruturados onde existe a
presença incerta e movediça de pai e mãe, a violência corriqueira não
costuma dar muitas chances ao diálogo. De repente, tudo pode ser
interrompido com a rapidez seja de um golpe de mão, de chutes deitando
no chão as tralhas domésticas, da rudeza do palavrão, ou, nos casos
radicais, da truculência brutal de um golpe de faca.}

\emph{\versal{GS}, moleque sensível e estranhamente sofisticado por sua miséria,
permitiu"-se comunicar para mim, depois de adulto, seu anti"-heroísmo de
vingador dúbio contra o crime paterno.}

\emph{Se havia aí algum ocultamento da parte dele, não estou certo. Mas
eu me lembro de que no seu discurso havia transparência, despojamento e
necessidade de encontrar desesperadamente um ouvinte.}

\emph{O~que me chamou muito a atenção no seu sofrimento foi a maneira
como ele introjetou o desejo parricida e transmutou a culpa do pai,
passando a ser ele mesmo o autor das facadas: eis aí uma chave de
interpretação costurada no baú da culpa e das especificidades de um caso
particular e notável.}

\emph{Reconheço que com este \versal{GS} eu me perdi em diagnósticos, ou nem me
interessei por diagnósticos. Se estes podem ser encontrados, vão além da
medicina e começam por ser elaborados na bruteza da existência ou no que
decorre como sequela da facada nas costas alimentando um rol de
sofrimentos subsequentes.}

\emph{O~que eu sei muito bem é que \versal{GS} era um adicto grave que tinha
ideação suicida e fez uma roleta-russa cutucando a ira de um traficante
que o protegeu e lhe deu uma excêntrica lição de moral.}

\emph{Eis aí, aliás, uma horrível comédia de humor negro, caro leitor,
uma estranha comédia que se inicia com a seguinte pergunta: não é o caso
de dizer que \versal{GS}, levado pela enxurrada dos acontecimentos, apenas seguiu
muito além do que seria o limite habitual das neuroses pequeno-burguesas?}

\emph{Mas também admito que os referenciais deste caso possam ser
outros, e que todo este drama me lembra um outro fato: o de que muitos
contraventores, e alguns traficantes, costumam ser ``caretas'' quando
aliam truculência a moralismo.}

\emph{Este é precisamente o mundo em que \versal{GS} vive, um mundo onde o crime
é conservador, capitalista e não suporta tipos que nem \versal{GS}. Estes tipos,
por vezes, espirram fora da contravenção pelo êxito letal da execução
quando ``pisam muito na bola''.}

\emph{Isso acontece muito na periferia onde quem tem menos chora mais,
onde os códigos da lei não estão escritos e onde o exorcismo banal feito
em pacotes nas igrejas evangélicas faz um pífio papel que caberia aos
terapeutas. Toda essa realidade, porém, não costuma vir à tona na grande
mídia.}

\emph{Sem falar que o dependente de crack, na periferia distante, acaba
sofrendo preconceitos iguais aos que surgem da sociedade mais abonada ou
dita ``certinha''. Esse dependente de crack sofre preconceitos de
gente do crime, vejam só!}

\emph{Mas \versal{GS} tem ao menos uma certa grandeza inconsciente. Ele se parece
com alguns personagens agônicos do teatro grego perseguidos pela fúria
das Erínias. Ele me lembra personagens (degradados porém) de um Ésquilo
por exemplo.}

\emph{\versal{GS} é o herói das sombras, é o vingador a mando dos deuses (que ele
ignora), é aquele que sabe de seu papel de herói pela via do
inconsciente. Na superfície da consciência \versal{GS} se vê e se sabe como um
caso gravíssimo de dependência de crack e de ódio ao pai.}

\emph{Nas entrelinhas de seu discurso \versal{GS} deve horrorizar"-se por estar
carregando alguns segredos internos para cuja ação ele se vê impotente
enquanto trai a si mesmo. ~~}
%\endgroup