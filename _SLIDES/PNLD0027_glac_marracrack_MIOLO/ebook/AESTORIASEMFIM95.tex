\chapterspecial{A história sem fim~}{}{}
 

\section{Relatos de «viagem»}

Entro numa academia onde as argolas são movidas pela força do
pensamento. Participo de um campeonato. Sou raptado e levado para um
cativeiro onde tem uma organização, a máfia japonesa, a Yakusa. Do outro
lado tem a máfia mexicana que age de maneira parecida com a Yakusa.

Sou destinado a me casar com certa mulher. É~uma loira muito bonita. A~mulher está sendo enganada pelos que são da ``faculdade de rituais'',
que é a própria Yakusa.

Deus me ajuda e, quando fico preso nas garras da Yakusa, Deus me deixa
preso nas esfinges. As esfinges são um grupo de estátuas. Uma delas é a
Medusa. Mas a Medusa é do bem, e é minha aliada.

Tem um tal de Salomão que não é o da bíblia mas um gênio a meu favor.
Faço um ``intermédio'' com Salomão. Por vezes tenho de salvar Salomão
das garras do mal porque, do contrário, a história pode ser interrompida.

Eu também tenho que salvar a Medusa de mágicas feitas com o objetivo de
vencer a Medusa. Para isso eu passo onde ela mora com o cavalo alado, o
Pégaso.

Aparece um tal de Leozinho, um rapaz que faz parte da ``faculdade dos
rituais''. Leozinho é da turma do mal. Imita a minha voz com o objetivo
de me enganar.

Fico sabendo que na máfia mexicana tem uma mulher médica que come carne
de cachorro e carne de pessoas.

Quem sou eu, afinal de contas? Eu sou a razão e o princípio de tudo. Mas
tenho que pensar em todos os obstáculos para salvar a mulher. Não
consigo porque me prendem muito, e porque os do mal me surpreendem a
cada momento.

Mesmo tendo aliados, os aliados que acreditam em mim deixam de acreditar
porque param de lutar para o bem e vão lutar para o mal.

Muitas vezes eu acabo quase morto, feito em pedaços. Mas eu me
reconstituo, e só não morro mesmo porque sou salvo pelas esfinges e pela
Medusa.

Deus intervém em tudo e faz parar a Máfia de vez em quando. Deus pára
todos os que são do mal e também os que são do bem. Então, de repente,
todos ficam prestando atenção em Deus.

Duas vezes eu tenho a chance de salvar a mulher loira num lugar onde tem
um monte de espadas e gente do mal que tenta pegar as espadas. O~monte
de espadas foi arrumado por Salomão, que fez as espadas para os monges
seus discípulos.

Em algumas noites eu tenho o descanso do guerreiro e ajudo a preparar as
espadas para Salomão. A~gente do mal não consegue fazer com que as
espadas virem espadas de verdade. Somente eu, Salomão e Deus.

Peço um cavalo emprestado ao Príncipe da Morte, que está do lado do bem.
O~Príncipe da Morte pede que eu salve uma mulher da influência maléfica
vinda de um palácio próximo. Não consigo porque o palácio próximo está
tomado.

Sou o mais perseguido porque ajudei muita gente. Em cada lugar tem o mal
e tem o bem. Os do mal sempre se juntam pra me prender, me raptam e me
levam para um cativeiro.

Nesta jornada também tem sexo. O~pessoal da Yakusa me traz umas
mulheres. Eles fazem de tudo para me deixarem impuro para que eu traia
as forças do bem. Mesmo não querendo, sou tentado com sexo. Por vezes
sou torturado com sexo. Mas estou destinado a me casar com a mulher
loira.

Ela sempre espera por mim.

\section{A~entrevista com ele}

A ``viagem sem fim'' continuava, capítulo por capitulo, depois que ele
ia dormir. Mas a viagem era interrompida quando ele acordava, e retomava
no outro dia de noite.

Os medos e pressentimentos só melhoravam quando ele dormia ou quando
surgia uma nova história dentro da história sem fim.

Agora, porém, não há mais sonhos. Sobra apenas um vazio.

Eu pergunto o que ele acha de toda a história sem fim, e se ele acha que
aconteceu. Ele até acha que aconteceu, mas pensa também que foi sonho.
Daí fica na dúvida porque tudo parecia muito real.

Ele não sabe dizer se a história pode continuar. Ainda acredita que o
objetivo final da história era ele casar-se com a mulher loira, quando
então a ``história sem fim'' deveria ter um fim.

Ou ele acredita que a história possa existir em outro tempo. Ou acredita
que a história possa existir em outro lugar.

Na verdade, ele tem vontade de entrar de novo na história porque ela traz
a paz e um certo prazer.

No entanto, ele se lembra agora de um outro final.

Certo dia ele foi se aproximando da mãe logo depois que tinha acordado.

Ele ainda estava meio confuso.

De repente ele sentiu que a mãe era um obstáculo em sua jornada rumo ao
seu casamento com a mulher loira.

Tudo aconteceu numa época em que ele tinha unhas longas e cabelos
longos. Era um rapaz assustador, grande e forte.

Ele segurou o pescoço da mãe.

Sua irmã, policial militar, chegava naquele momento em casa de manhã.

Sua irmã pensou que ele estivesse querendo matar a mãe por causa das
drogas.

Sua irmã lhe deu um tiro que acertou seu pescoço.

O tiro interrompeu o sonho porque depois do tiro os capítulos da
``história sem fim'' foram encerrados.

Ele ficou internado muito tempo. Quase morreu.

Aquelas imagens do sonho nunca mais voltaram, e hoje ele não vê como
retomar a história a não ser, talvez, sonhando o sonho.

Mas ele ainda se sente um herói na lembrança do sonho, e eu pergunto se
a lembrança do sonho o assusta. Ele responde que não. O~que o assusta é
a realidade.

Ele confessa que gostaria de retomar a ``viagem sem fim'', que adoraria
entrar de novo na história, capítulo por capítulo, viajando a cada noite
depois de cada crise diurna de agonia, crise que era aliviada como se
cada capítulo fosse um longo e prolongado mesclado.

\begin{center}\asterisc{}\end{center}
\begingroup\small

\emph{É um discurso nitidamente psicótico. Mas há uma organização sutil
neste discurso que lembra uma saga heroica bizarra, tal como a vemos em
muitos filmes infantojuvenis.}

\emph{Existem aqui elementos que fariam o gosto de roteiristas. O~que
destoa são cacos psicóticos produzidos por um viés torto de associação
de pensamentos. Na verdade, não se trata de uma história sem fim mas de
uma história com fim sucessivamente anunciado, porém protelado por
inúmeros percalços estranhos.}

\emph{O~enredo é simples. Os personagens são do imaginário popular
tradicional. Há um conflito, há um desejante e um desejado, há um objeto
alvo preferido, há um protagonista e vários antagonistas se
metamorfoseando uns nos outros. Há uma luta no palco da vida pequena
envolvendo elementos que oscilam como se viessem do acaso ou pela
intromissão espúria de forças estranhas, com invocações divinas e
mágicas.}

\emph{Eu diria que existe aí quase um caleidoscópio psicótico produzido
por um rapaz muito problemático. Então --- eu me pergunto --- como é
esse rapaz? É um tipo alto e forte, de semblante pesado, adicto em
múltiplas drogas, principalmente em mesclado, e que me contou uma novela
íntima e particular.}

\emph{Existe nessa novela estranha uma lógica na forma de um enredo que
tem, apesar da distorção psicótica, semelhanças a modelos arquetípicos
do mito do herói.}

\emph{O~material original era mais prolixo ainda, e tive que resumir um
pouco o discurso do rapaz. Preservei, no entanto, a maneira como as
frases dele se fundiam, descarrilhavam e tangenciavam mesclando-se numa
fuzarca por vezes bíblica, ou por vezes mitológica de transformações
mágicas, e sendo tudo costurado num quadro algo onírico.}

\emph{Sem falar que o protagonista desse bizarro discurso tem alianças
suspeitas e sofre sua ``paixão'', cumpre sua saga, e é alvo de um
sofrimento imposto por algum Destino que o leva a uma redenção profana,
mas com um anunciado prêmio final.}

\emph{Essa psicose, possivelmente induzida por substância psicoativa,
tende a reproduzir uma busca incessante e repetitiva de prazer. Mas o
que é interessante e dramático é o caminho estabelecido para chegar
nesse prazer.}

\emph{Porque me parece que na angústia de protelar indefinidamente a
recompensa final estaria o prazer sempre anunciado, o pote de ouro no
final do arco íris, a ``brisa'' ilusória embutida na ação de um mesclado
diário consumido antes de dormir e transposto para agir toxicamente no
universo onírico.}

\emph{Mas todo esse discurso parcialmente fragmentado se faz literatura,
quer pela irreverência insana e sofisticada do rapaz, quer pelo choque
brutal de realidade na forma estúpida de um tiro varando o seu pescoço
depois de uma cena doméstica assustadora rompendo a manhã.}

\emph{De uma certa maneira, é como se o rapaz assistisse a uma novela.
Ele sempre esperaria ingenuamente um próximo capítulo. O~final trágico
foi uma amarga ironia do Destino.}

\emph{Lembro-me de que não se tratava de um rapaz violento. Ele apenas
assustava pelo tamanho. A~reação da irmã policial militar deu-se no
susto, e a tragédia total só não se consumou pela intervenção dos
deuses.}

\emph{Ah, eu não me esqueço que, numa das entrevistas comigo, já estando
bem orientado, o rapaz mostrou uma atitude ambígua colocada a meio
caminho de uma possessão e de~uma experiência escolhida. Eu fiquei sem
saber até que ponto teria havido da parte dele uma apropriação} a
posteriori \emph{das imagens oníricas e/ou também psicóticas na viagem
``química'' da droga.}

\emph{Era como se o território do sonho fosse uma passagem a meio
caminho entre lúcida e louca para o inconsciente. Tanto assim que o
rapaz foi irônico consigo mesmo ao ficar em dúvida sobre a realidade da
saga heroica como se ele fosse um crítico de sua curiosa novela
particular.}

\emph{E falando em novela particular, faço agora uma comparação breve
com a crônica Ivan, o terrível, cujo personagem tem seu ``cinema''
íntimo. Mas na história sem fim, se há também um ``cinema'' íntimo há um
cinismo pueril ao extremo, há um cinismo infantil e regredido permitindo
que depois o horror de repente emerja do choque concreto via uma bala de
revolver.}

\emph{Mostrando que, muitas vezes, a violência do mundo das drogas não é
diretamente manifestação da violência do drogado, mas é o que resulta de
uma cascata de fatores convergindo para uma ação francamente
destrutiva.}

\emph{Nesta história poderia até se dizer que a violência mais estúpida é
o custo amargo de misturar cacos edípicos mal resolvidos com objetos do
desejo suspeitos e ainda por cima transitar loucamente entre ``máfias''
contemporâneas numa salada mista de breguice mitológica, sob as bênçãos
e a cumplicidade de Deus -- o que evidencia a meu ver algum delírio
místico de poder, ou seja, o que evidencia um verdadeiro coquetel
molotov psíquico quando esse tipo de delírio vem atrelado a núcleos
libidinosos.}

\emph{Quando o rapaz agarrou o pescoço da mãe, ele estaria ainda num
estado crepuscular. E~por falar em ``estado crepuscular'', me lembro
agora de uma entrevista que vi com Oliver Sacks na qual ele argumenta
que todo mundo pode ter, de vez em quando alucinações, se considerarmos
aquele intervalo crepuscular no final do sonho e antes do acordar de
fato, quando temos a sensação de ``ver'' ao vivo certas imagens dos
sonhos.}

\emph{No entanto, não se pode, no caso deste rapaz, se aventurar a dizer
muita coisa sobre o que existe debaixo do} iceberg \emph{da psicose tóxica de
um indivíduo inserido no seu mundo de miséria e drogas.}

\emph{É claro que não se trata de coisa boa. Mas, de qualquer maneira,
eu até compreendo~que o ato do rapaz agarrar o pescoço da mãe tenha sido
horripilante aos olhos da irmã, que já tinha uma imagem formada do irmão
como um ``noia'' perigoso.}

\emph{Então para mim esta crônica ilustra bem como os dependentes
químicos costumam ser facilmente demonizados, e como uma droga como o
crack ou o mesclado pode ser eleita como bode expiatório para outros
problemas que incluem situação familiar, miséria social e econômica
etc.}

\emph{Posso tranquilamente afirmar que, se essa tragédia familiar
tivesse tido repercussão midiática, ela reforçaria o chavão
preconceituoso de que o uso de drogas ilícitas, mormente o crack, produz
nos usuários ``instintos homicidas''.}

\emph{Mas --- pergunto --- o que se vê na realidade? O que se vê na
realidade não é a manifestação de um ``instinto homicida''. É~uma pífia
saga de mais um zé ninguém vítima de sua busca fantasiosa por um efêmero
e traiçoeiro prazer.}

\emph{E se a conjunção infeliz dos fatos interrompeu a história sem fim
com um fim nada terapêutico, ou então um fim estupidamente terapêutico,
essa interrupção foi apenas um brutal antídoto para aplacar uma
possessão.}

\emph{Como se o último capítulo da história sem fim estivesse predito
pelas conjunções mágicas das ``faculdades dos rituais'' explodindo numa
manhã desmancha-prazeres.}

\emph{De resto, eu ainda creio que, para este rapaz, toda sua aventura
parecia coisa de criança. Com uma cínica diferença daquilo que, em
outras circunstâncias, seria infantil: havia droga pesada que a criança
pode não alcançar (pelo menos ainda não pode), e sim o adulto pode
fazê-lo ao se deixar levar pela sua criança interna.}

\emph{Em resumo, tudo isso é novamente uma ciranda cinematográfica
íntima das ilusões perdidas produzindo um circo de horrores. Horrores
periféricos. Horrores paulistanos e universais.}
\endgroup