\chapterspecial{Máscara do corpo em dois tempos}{}{}
 

Eu não o via há mais de ano e levo algum tempo para reconhecê"-lo. É~o
\versal{JR}, o carioca simpático de Vila Isabel, com aquele jeito malandro, cheio
dos expedientes, cujo irmão era amigo do \emph{rapper} \versal{MV} Bill.

Eu então me lembro de ter conhecido \versal{JR}, todo solto e bem à vontade,
quando ele participou de uma leitura dramática da minha peça no papel do
traficante Marcelão.

Nos intervalos da leitura surgiu a oportunidade de conversarmos sobre
muitos assuntos. Ele me disse gostar de aventuras, de ouvir histórias.
Era dado a prazeres banais ou até mesmo fúteis. Gabava"-se de ser um
sedutor. Tinha várias moradas. Topava qualquer parada ou negócio que lhe
desse vantagem.

Ele é um tipo de adicto que a gente costuma chamar de ``buscador de
novidades''. E~parece ter sido um menino hiperativo, sempre tomado por
inquietações e por um descaramento irreverente até aceitável conforme um
jeito carioca e meio malandro de ser.

Ele tem facilidade para diversas funções. É~pau pra toda obra. Chegou a
fazer oportunamente o papel de monitor aqui na Comunidade Terapêutica,
dando palpites em recuperações, e em tratamentos.

Certa vez assumiu temporariamente um ``bico'' de segurança quando
protagonizou um episódio quase explosivo cuidadosamente abafado.

Tinha ocorrido uma discussão pesada em que um rapaz recém-admitido na
Comunidade descontrolou"-se, ficou enfurecido e precisou ser contido. \versal{JR}
--- tranquilo e com maestria de brigador de rua --- imobilizou o rapaz
com um golpe efetivo, rapidamente deitou"-o no chão e a fúria acabou.
Juro que me pareceu coisa de profissional.

\asterisc{}

Faz um ano que o vi pela última vez.

Agora estou diante de quem me parece ainda ser um rapaz, apesar de seus
trinta e poucos anos. Mas ele é uma sombra do que havia sido. Apesar de
conservar um jeito inalterado de eterno adolescente, ele tem, dessa vez,
a face chupada e os olhos encovados. Está magérrimo e todo descuidado.

Me abraça, encosta a cabeça demoradamente no meu peito para logo se
abrir e se derramar sobre suas desventuras.

Confessa ter entrado na maldita onda do crack. Abaixa a cabeça com
humildade e entrega, quieta"-se um pouco e depois fala bastante.

Pela primeira vez, ouço, num jorro contínuo, sua história de vida.

Ele é filho caçula de uma família suburbana de Vila Isabel em que
praticamente todo mundo é gente ``certinha'', religiosa, crente e
sofrida. Só ele de ovelha negra.

Nos seus descaminhos desde tenra juventude ele buscava sempre estar em
trânsito pelo mundo, e perdera bem cedo o contato com sua família.

Já rapaz, voltara a ver o pai após alguns anos de ausência e viu o pai se
acabando todo inchado sem conseguir sair de uma cama.

O pai, que sempre fora rígido e autoritário, o abraçou e morreu pouco
tempo depois. E~se a cena lembraria cinema romântico, seguiu"-se um
clímax antirromântico.

Após o velório \versal{JR} foi tomado por um profundo desencanto. Foi a uma
biqueira, comprou pedras de crack e atirou a sorte na valeta da
existência. Daí não parou mais de usar.

Não conseguiu se reconciliar com a família, e os familiares não lhe
pouparam a acusação de que teria sido o responsável pela morte do velho.
Para tanto arrolaram decepções ao longo dos anos em seu comportamento
desviante das normas da família.

Sempre insistiam na culpa dele, apesar do pai ter partido desta vida aos
setenta e poucos anos, inchado e cheio de moléstias crônicas. Mas alguém
precisava ser o bode expiatório.

\asterisc{}

Começo a examinar \versal{JR}, que tira a roupa e fica de cueca. Vejo um corpo
escasso de massa muscular, as pernas finas, os braços finos. Um perfil
caricato que mal lembra o rapaz forte que conheci há um ano.

Outras revelações me chegam ao conhecimento. Vejo uma grande tatuagem
agressiva nas suas costas. Desconfio ser tatuagem de cadeia. \versal{JR} me diz
calmamente que ficou preso a partir dos dezenove anos por oito anos em
regime fechado. O~motivo é 157, assalto a mão armada.

\versal{JR} detalha um grande assalto numa cidade do interior, muito comentado na
época nos jornais e em que uma pessoa morreu baleada, e me diz que quem
atirou foi outro.

É claro que não tenho como saber a verdade. Mas estou acostumado com
essas conversas. Quase todos dizem a mesma coisa. É~sempre um outro que
atira. Velha história.

Fico sabendo que a pena foi de dezesseis anos, mas \versal{JR} cumpriu oito anos
porque foi beneficiado com leis que contemplam o abreviamento do tempo
de prisão.

\asterisc{}

Faço uma anamnese mais detalhada a respeito do envolvimento de \versal{JR} com o
crack. Ele escancara tudo sobre a maldita pedra enquanto me faz
confissões apontando ``onde o bicho pega''.

Revela também que usou várias drogas desde a adolescência, desde quando
era o garoto rebelde em conflito constante com sua família suburbana
conservadora e religiosa.

Cedo meteu"-se na ``coisa errada'', seguindo no curso de oportunidades
tortas e crescendo desde a delinquência menor para a contravenção maior.
Se bem que até então nada de crack, mesmo porque, quando ele iniciou sua
vida no crime, essa droga seria incompatível com sua posição.

O crack desabou sobre ele no final e no embalo das desilusões domésticas
e da morte do pai. Ainda assim o crack acabou sendo a busca daquela
``brisa'' radical inalcançável, ou virou uma fútil esperança vã na crise
instalada e sob o peso das acusações da família.

Foi naquela época que \versal{JR} perambulou pelas ruas feito um zumbi, e seguiu,
como tantos ``noias'' que ele desprezava antes, a \emph{via crucis}
costumeira. Não se envergonhou em me dizer que furtou, pediu nos faróis,
chafurdou em lixo.

Eu agora noto que o tom confessional de \versal{JR} busca ir além dos descaminhos
com as pedras. Na verdade, ele traz uma demanda de explicação da minha
parte para confrontar uma demanda íntima dele.

Eu intuo do que se trata e arrisco dizer que seu maior fardo é acreditar
ter sido ele quem matou o pai de desgosto. Segue"-se a confirmação disso
quando \versal{JR} me entrega seu sentimento de culpa ao titubear de tensão, sem
me dar a resposta, e ao ficar um pouco em silêncio, encolhido e de
cabeça baixa.

Enfim, de todos os cacos restantes do que \versal{JR} tem como imagem de família,
é a morte do pai o que mais lhe pesa. Não é à toa que agora uma lágrima
forte e decisiva escorre no seu rosto. Uma lágrima que seria testemunha
não apenas de sentimento, mas de algum caráter de sua parte.

Antes de mandar ele se vestir peço a ele que, ainda de cueca --- exposto
naquela decadência magérrima de quem teve um porte atlético elegante e
de quem se mostrou ágil, malandro e sedutor --- peço a ele que se olhe
demoradamente no espelho e busque algumas reflexões. Ah, quem sabe ele o
faça de maneira a ``cair a ficha'' para ter alguma tomada de consciência
e para que alguma mudança efetiva possa ocorrer.

Novamente vestido, \versal{JR} não parece assumir mais aquela ginga malandra
carioca de Vila Isabel. Ele parece trazer outra máscara extensa como
marca sobre todo o corpo.

E permanece meio indiferente na sala, de cabeça baixa, e com um olhar
fundo e melancólico. O~olhar é reflexivo e imerso na dor das revelações
daquilo que ele já sabia a respeito de si mesmo, mas não reconhecia
plenamente. E~se agora reconhece é como se todas as pedras de crack
tivessem se espatifado simbolicamente sobre todas as suas ilusões e
desilusões.

\asterisc{}

Naquele dia encontra"-se de plantão um segurança que eu conheço bem. É~um
homem bastante prático e dotado de personalidade forte. Comentou
rapidamente, antes de \versal{JR} entrar, que esse rapaz seria um ``rato de
clínica'', conforme uma expressão que se usa muito. Ele seria um desses
casos que ficam ``enxugando gelo'', perdidos e perambulando de porta em
porta nas comunidades terapêuticas.

Este segurança é um homem maduro e massudo. Já trabalhou na polícia
civil. É~um pouco pau pra toda obra. De vez em quando faz um trabalho de
monitor. Gosta de dar palpites sobre adictos. Por vezes acerta ao fazer
incursões leigas com certa psicologia intuitiva e popular que não se
aprende nos livros e sim na escola da vida.

Ele é um homem meio rígido e com tendência um pouco moralista. Não é má
pessoa. Tem uma certa postura autoritária e paternalista e insiste em
dizer que não gosta de malandragem e de bandidagem, ainda que se
tratando de adictos.

No entanto, fica admirado ao perceber que \versal{JR} me conhece e repara bem nas
saudações do encontro e quando \versal{JR} encosta a cabeça demoradamente no meu
peito parecendo uma criança.

Eu chamo esse segurança para conversar. Digo a ele, com muita franqueza,
que seria melhor ele revisar seus conceitos sobre drogados. Porque
praticamente todos esses dependentes são complicados e ambivalentes e há
casos e casos, histórias e histórias. Com ou sem malandragem, o ser humano
é complexo e difícil de julgar.

Aproveito a chance para contar aquele episódio elegante em que \versal{JR} ---
agindo em oportuno momento --- imobilizou com maestria o rapaz
enfurecido. O~segurança massudo e forte aprecia o feito, dá aprovação à
atitude do outro e me conta algumas de suas proezas.

Ficamos o segurança e eu bastante tempo de prosa. Durante a conversa
relembro outra ocasião em que eu estava falando para um grupo de pessoas
na Comunidade.

Foi cerca de um ano atrás.

Era um pequeno encontro e também uma palestra e uma aula informal. O~assunto? Um dos meus preferidos: a relação entre os mitos e o
comportamento das pessoas, principalmente o comportamento dos adictos.

Na pequena plateia interessada estava aquele rapaz carioca com jeito
ágil e malandro, o \versal{JR}. Era quem mais se ligava e mais prestava atenção.
Seus olhos brilhavam inquietos de busca. Principalmente quando eu
sintonizei o assunto que interessava: os mitos gregos de Hermes e de
Dionísio, que, lá no fundo, são configurações arquetípicas e estão muito
presentes nessa população marginal atolada nos becos e nas sombras.

O arquétipo de Hermes predomina entre os jovens delinquentes, entre
tipos ``lisos'' quase sempre irrequietos buscadores de novidade. Que são
ambíguos, funcionam como mensageiros, e estão prontos para desafios e
tarefas que, muitas vezes, contrariam as normas estabelecidas ou se
encontram além do bem e do mal.

Também o arquétipo de Dionísio está bastante presente, e Dionísio faz
uma composição, ou uma ``dobradinha'', com Hermes.

Lembro"-me de que, depois da palestra, \versal{JR} havia me procurado com brilho
nos olhos, fez muitas perguntas, ficou todo ``ligado''.

E \versal{JR} é um tipo que eu diria Hermes/Dionísio, numa complementaridade
entre a mobilidade ágil e malandra de Hermes e o intenso jogo sedutor de
Dionísio.

Não é à toa que, sem conhecer mitologia grega, ele identificou"-se na
dupla Hermes/Dionísio. Eu disse a ele o quanto Hermes é o malandro, é o
leva e traz, é o deus dos comerciantes e também o deus dos ladrões. Mas
Hermes é também um portador de mensagens com uma espécie de dupla chave
que, quando ligada, funciona negativa ou positivamente. Tanto assim que
Hermes também é o resgate daquilo que pode ser recuperação, tratamento,
ou, até mesmo, cura.

Já Dionísio simboliza a embriaguez, o êxtase e o entusiasmo no
sentido mais amplo e simbólico. É~o patrono dos atores, mas seria
também, e por que não, o patrono dos adictos.

\versal{JR}, por sua vez, é um adicto que está sempre em trânsito, e em movimento
à maneira de Hermes. Tem uma instabilidade amorosa bastante dionisíaca.
Tanto assim, que são incontáveis as aventuras com seu séquito particular
do que seriam mulheres brasileiras ``bacantes'' --- uma penca de garotas
fáceis ou até difíceis passando pela sua cama.

No entanto, \versal{JR} é carente de um amor verdadeiro, e sempre carregou uma
ansiedade a vida toda, além de uma inquietação doentia e de uma angústia
quase permanente.

Por fim, ele admite que sempre esteve na busca do reencontro com o
feminino na saudade de sua mãe sofrida, e também sempre esteve na busca
do afeto masculino e do reconhecimento pelo pai.

\asterisc{}

O segurança corpulento fica bastante ``ligado'' em todo este assunto.
Está também bem motivado e inquieto. Percebo que seria sua vez de me
dizer algumas verdades sobre sua vida. Ele então me conta peripécias e
descaminhos, acentuando o tom de sua voz grossa, pesada, sem qualquer
leveza adolescente que lembre o estilo de \versal{JR}.

Confessa que foi também usuário de muitas drogas, fez muitas loucuras,
mas sempre procurou manter"-se dentro da lei assumindo e até financiando
seu próprio ``vício''.

Eu então percebo um detalhe muito significativo. Não existe nele nada de
Hermes/Dionísio, absolutamente nada. Muito pelo contrário: sendo ele um
homem massudo e forte, é um homem impositivo que sempre quis ser um
mandachuva, embora essa tendência tenha sido contrariada na prática
quando ele acabou numa posição subalterna, mas que conflita com o fato
dele vir de uma família autoritária, com pai militar.

Puxa vida, de repente tenho a clara percepção do quanto ele é um tipo
Zeus, e daí tenho uma ideia da hora, como se diz popularmente. Conto ao
``segurança Zeus'' a famosa história de como Hermes, criança, roubou o
rebanho de Apolo.

Aquilo foi a primeira façanha do pequeno Hermes. Foi um precoce golpe de
mestre. A~ousadia, porém, foi flagrada por Apolo que, sendo um deus e
seu irmão mais velho, logo descobriu a tramoia. Para aplacar a fúria de
Apolo, Hermes não perdeu o rebolado. Inventou a lira e a deu de presente
ao irmão, que ficou seduzido de contente e virou o deus da música dentre
outras prerrogativas. E~o paizão Zeus contemporizou os conflitos entre
os irmãos distanciado no seu papel dominante. Vislumbrou, desde sua
olímpica distância controladora e autoritária, a interessante
malandragem de Hermes, deu uma gargalhada e o fez mensageiro do Olimpo
para que tudo seguisse, afinal de contas, seu curso devido.

Neste momento o segurança corpulento está muito contente. Está impactado
com a história e quer aproximar"-se de \versal{JR}. Sem muita sutileza, sugere que
\versal{JR} faça alguns serviços, ou alguns ``bicos'' ali na Comunidade, que ele
poderia orientar, chefiar.

\asterisc{}

No entanto, caro leitor, a vida não é uma aventura olímpica. A~vida tem
arestas que nenhuma mitologia apara.

Eu acabo de omitir para este segurança um segredo: naquele assalto
praticado por \versal{JR} o morto foi outro segurança, e aquele tombou com
certeza para indignação de quem clama por justiça.

Apesar dos pesares, creio que \versal{JR} tenha seus atenuantes. Mesmo que se
trate de quem tenha caráter suspeito, seja considerado ou julgado
bandido e venda imagens diferentes em tempos diferentes de máscaras que
são papéis verdadeiros ou falsos, porém complementares no bem e no mal.

Mas tudo isso é típico de Hermes com um pouco de Dionísio.

Zeus, lá em cima, na chefia, costuma saber o que faz para colocar ordem
no mundo.

\begin{center}\asterisc{}\end{center}
%\begingroup\small

\emph{Certos depoimentos trazem conteúdos emocionais e pessoais de maior
dimensão. Esta confissão de \versal{JR} seria um deles. Tal como na história da
lágrima atrás do gorro ninja, esta história (real, bem real) mostra para
mim claramente que quem viveu o mundo do crime e tem sentimentos (ou
mesmo caráter) é bem diferente de outros que vivem a lógica fria e
calculista da contravenção e não se abrem à humildade da culpa, do
arrependimento, ou da mudança de vida.}

\emph{A~história de \versal{JR} é mais uma história de uma criatura com marcante
personalidade a transitar pelo mundo das sombras desde quando era um
garoto perdido, literalmente perdido.}

\emph{Ele é um adicto que eu classifico como do tipo ``buscador de
novidades'', ele é mais um dependente iludido por perspectivas loucas de
falsa liberdade, sempre atrás de ``brisas'' fantásticas e coisa e tal,
como tantos outros.}

\emph{Ele até seria um malandro ``clássico'', apesar do estereótipo
banal do malandro tornar minha observação suspeita. E~ainda faço questão
de dizer que alguns tipos malandros feito \versal{JR} costumam aparecer bastante
à minha frente, e podem vir, conforme reza o folclore, até do morro,
feito aqueles que são do samba e companhia, e cujos perfis misturam
chavões com realidade.}

\emph{O~que mais me fascinou para a feitura desta crônica foi a ponte
escancarada com a mitologia grega, que é meu guia frequente em
atendimentos, análises e avaliações de dependentes. E~eu já comentei,
aliás, que nesse mundo dito por aí imprecisamente como sendo
``marginal'', e entre jovens contraventores, predomina o arquétipo de
Hermes.}

\emph{Dentre os meninos do crime e da ``malandragem'' Hermes ganha
longe. Aqueles que têm sentimento aflorado e um pouco mais ou um pouco
menos de caráter, mesmo os que não sejam ``flor que se cheire'',
transitam entre as polaridades de Hermes. Que é o deus mensageiro e é um
deus bom e mau, por vezes o curador e também o malandro liso e até mesmo
o rapina e o safado.}

\emph{Existe, porém, aqui, uma ``dobradinha'' curiosa. Na dualidade
volúvel de Hermes entra um pouco de Dionísio e talvez Dionísio ocupe o
segundo lugar na tipologia desses meninos.}

\emph{Se Dionísio é o ``patrono'' dos atores e está na raiz do teatro
ocidental, ele seria recebido com entusiasmo caso fosse o patrono dos
adictos. Pois muitos adictos contumazes são dionisíacos sem que o
saibam, e ainda atuam na inconsciência de atuarem explicitamente.}

\emph{Quando essa droga é o crack o envolvimento toma proporções muito
sombrias devido à terrível e ensandecida fissura, sem falar nas atuações
rudemente teatrais que podem descambar para atos francamente
contraventuais tidos como repugnantes.}

\emph{Mas volto à questão da busca por um diagnóstico. \versal{JR} talvez tenha
sido uma criança portadora do que se pode chamar de transtorno
disruptivo de infância.}

\emph{O~transtorno disruptivo tem um espectro genérico e engloba os
transtornos de conduta, como também o de oposição e as várias
modalidades do chamado transtorno de déficit de atenção e
hiperatividade. Sinceramente não sei se é o caso de \versal{JR}. Até poderia ser,
mas concluir não é fácil, e ainda pode não haver transtorno algum afora
as vicissitudes da vida e mais a adicção.}

\emph{Enfim, dentro de um quadro impreciso talvez seja melhor dizer que
\versal{JR} foi um menino atípico que mal conviveu no meio familiar e acabou
destoando no ambiente conservador e provinciano de uma rígida família
evangélica de um subúrbio do Rio.}

\emph{No entanto, e mesmo já tendo sido réu confesso, ele conseguiu
manter algumas qualidades indiscutíveis, sem falar que ele ---
anti"-herói das sombras --- desceu aos infernos de sua ``noia'' e teve
vislumbres de lucidez e até de ``iluminação'' (eu gosto bastante desta
palavra iluminação no sentido budista, oriental). Acho que ouvi essa
palavra quando ele me contou sua história com bastante sentimento e
apesar de estar ``detonado'' pelo crack.}

\emph{Eu também não posso esquecer que a história dele foi pontuada por
uma lágrima, uma lágrima sincera remetendo à questão do pai. Digo a velha
e universal questão do pai. Questão dolorida expressa através dessa
lágrima especial tão sinalizadora de uma carência geral dos ``manos''.}

\emph{Por isso \versal{JR} talvez ainda estivesse preservado como pessoa e muito
lúcido em sua trajetória com relação ao que se entende como o perfil de
um ``noia''. Mas ele é rebelde desde muito jovem, ele é um desses
adictos que vão irresponsavelmente atrás de expedientes excitantes para
quebrar a rotina, ele é do tipo aventureiro e pau pra toda obra.}

\emph{Coisa, aliás, típica de Hermes. Por isso mesmo, \versal{JR} é um desses
para quem vale a pena um esforço terapêutico de resgate porque ele tem o
que dar. Mas com ele há que se agir com bastante cautela e com alguma
delicadeza. Há que se agir sem instrumentalização da culpa, sem reforçar
castigos ou maldições; pelo contrário, deve"-se agir usando"-se as
ferramentas simbólicas da cultura universal.}

\emph{\versal{JR}, mesmo sendo semelhante a tantos outros dependentes de crack,
conseguiu sintonizar seu instante de ``iluminação'' com algum
autoconhecimento e teve oportunidade de se ver espelhado no que é
universal.}

\emph{Por isso eu afirmo que a referência universal à cultura é tão
necessária no tratamento quanto é necessário não descuidar do trivial
básico da medicina: escuta e foco no humano com corpo e ``alma'' unidos
num todo.}

\emph{Sem falar que, muitas vezes, é preciso relativizar ou relevar o
que a vida traz de cruel e injusto em suas cotidianas misérias para se
acreditar nas melhoras do ser humano.}

\emph{Mas \versal{JR}, como todo Hermes misturado a Dionísio, é uma caixa de
surpresas. Como muitos dos seus semelhantes algo malandros, todos estão
em trânsito pelos quatro cantos do mundo. Seu resgate final da adicção é
uma incógnita, é uma incerteza. É~outro x da questão.}

\emph{Sobra na passagem deles pelo palco da vida o brilho hermético e o
charme dionisíaco naquilo que pode virar literatura e o início de uma
crônica.~}
%\endgroup