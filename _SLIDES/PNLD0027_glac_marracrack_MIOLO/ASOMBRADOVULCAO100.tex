\chapterspecial{A sombra do vulcão}{}{}
 

O caso dele é pouco transparente, embora não existam casos totalmente
transparentes. Trata"-se de uma questão médica e também psiquiátrica,
porém em termos e relativamente.

O caso dele é também enigmático, e eu me reporto agora às epilepsias
atípicas que cursam com mudanças bruscas e episódicas de comportamento e
podem ser assustadoras.

Por outro lado, ele tem sempre mostrado uma aparente calma. Sendo
contumaz usuário de cocaína, já experimentou o crack, e de ambos se diz
abstinente há alguns meses.

Ele toma medicação direitinho. Ultimamente não apresentou queixas
importantes. No entanto, hoje surgem dele revelações que me surpreendem,
mesmo que ele, na entrevista anterior, tenha mencionado umas crises
convulsivas e tenha me mostrado um eletroencefalograma normal.

Mas eu ainda tenho dúvida da existência de alguma epilepsia, ou então,
chego a supor que a crise convulsiva seja uma manifestação isolada por
conta do excesso da droga no passado.

Verifico, todavia, que a história dele não se encaixa em certos modelos
típicos, e não poderia ser explicada apenas como decorrência de crises
convulsivas.

Ocorre que o homem confessa pensamentos de fazer maldade com as pessoas,
e que não permanecem apenas como pensamentos. Viram atos. Viram coisas
horrendas das quais ele mal se dá conta.

Porque, de repente, ele sente o perigo se aproximando, e então tudo pode
``explodir'' em fúria explícita.

Ele, todavia, não admite ser portador de uma maldade da alma. E~se ele
nega ter matado alguém, quase chegou a fazer isso.

Recentemente arrancou com os dentes um pedaço de orelha de um desafeto e
o guardou junto com um pedaço de camisa ensanguentado.

O desentendimento começara por motivo banal, com uma desavença entre
vizinhos por causa de uma mísera porção de telhado quebrado que o
vizinho se recusara a consertar.

Daí ele ficou possuído de um ódio profundo contra o vizinho e virou uma
fúria. Foram necessárias seis viaturas da \versal{PM} para contê"-lo.

Ele admite que sente normalmente arrependimento após um ato de agressão,
mas tudo se passa de maneira estranha, pois o arrependimento vem com
certo prazer pelo ato consumado, e isso o perturba. E~se ele procede no
ataque com pura impulsividade e se fica fora de si, também existe
premeditação porque ele é capaz de ficar dias maquinando em silêncio uma
vingança.

Ele declara que sua vontade de retaliação não vem do nada. Há sempre um
motivo por detrás das explosões, como, por exemplo, uma pequena ofensa
ou um desaforo. Mas, apesar de suas várias crises epilépticas, jamais
ele ``parte do zero'' contra alguém, jamais age gratuitamente na pura
maldade na escolha de uma vítima.

Por vezes ele chega até a comprar briga alheia, como numa vez em que
quis fazer justiça com as próprias mãos contra um rapaz que foi
desrespeitoso a um velhinho.

\asterisc{}

Os pais dele se separaram quando ele tinha dez anos. O~pai é um militar
extremamente rígido, mas nunca batia nele, usava apenas da força
intimidadora das palavras. Não foi assim a mãe, que tem problemas
mentais, sofre de convulsões, recebeu eletrochoques e já foi internada
no antigo Juqueri.

Constrangido, ele confessa ter apanhado muito da mãe, numa frequência
absurda de agressões desde tenra infância. Os motivos eram banais, sejam
traquinagens de criança ou, mais tarde, traquinagens de moleque ou de
rapaz.

A mãe perdia qualquer controle. Usava ferramentas no ataque. Certa vez o
golpeou com uma picareta. Chegava a lhe esfregar cocô na cara, e tanto o
fazia que, quando ele bateu recentemente no vizinho por causa do
telhado, esfregou cocô na cara do desafeto, além de lhe ter quebrado
toda a casa.

Ele confessa muitas brigas ao longo de sua vida, como tantas vezes
ocorreu no trânsito na época em que ainda trabalhava como motorista de
ônibus. Sem falar nas brigas no tempo de menino, na escola, quando ele
perdia o controle por qualquer besteira.

Ele vai se recordando de que tinha os tais ataques desde menino. Não
sabe dizer exatamente quando começaram, mas tinha sido antes da sua
iniciação nas drogas aos treze anos. Foi logo antes de quando
experimentou lança-perfume e solventes, e depois maconha e álcool. Mais
tarde ficou dependente de cocaína, de ``farinha branca''. Conheceu o
crack, mas tinha preferência pela cocaína.

Eu faço um encaminhamento dele para o psiquiatra e também me pergunto se
este seria mais um caso de epilepsia temporal, ou então, se seria um
caso de transtorno de impulsividade.

Procuro responder para mim mesmo que este poderia ser, na verdade, um
caso limítrofe, situado num terreno intermediário entre um transtorno de
personalidade e alguma atividade elétrica do cérebro de natureza
epileptiforme.

Enquanto reflito, de repente o homem faz um comentário delicado. Ele me
confidencia que ``teme ser um psicopata''.

Ele me diz que leu alguma coisa a respeito deste assunto. Viu uma
matéria na televisão. Na verdade, ele tem crítica suficiente para
confiar desconfiando, para perceber que a mídia banaliza as psicopatias
e que pessoas leigas ficam assustadas com explicações simplórias para
assunto tão complexo.

Ele confessa estar muito preocupado com o temor de alguma doença
estranha vinda através de sua família como herança da carne. Confessa um
medo de agredir o filho por conta dessa herança maldita. Lembra"-se de
que uma vez quase arrebentou o filho de pancadas por causa de uma
teimosia do menino. Chega a ter vontade de fazer isso por motivos banais
quando o filho faz alguma traquinagem.

\asterisc{}

Eu penso comigo mesmo, mais uma vez, que se por um lado tudo no caso
dele é uma grande sombra, por outro lado não existe muita novidade
debaixo do sol.

Há, no entanto, algumas certezas relativas no meio das dúvidas. Com
certeza, além das possibilidades epileptiformes, este homem produz um
relato consistente de sua criação. Ele fala, com precisão de narrador
seguro, sobre sua mãe insana e possuída que lhe golpeava picaretas na
cabeça, que lhe fazia escorrer sangue da face a ponto de lhe deixar
marcas na testa até hoje. Ele admite que, lá no fundo, deve sentir ódio
da mãe e que, devido a esse sentimento, evita se encontrar com ela para
que os demônios familiares não venham à tona.

Ao final da entrevista, e ao me mostrar as marcas na cabeça, ele faz uma
expressão melancólica, tendo os olhos postos no infinito.

\asterisc{}

Ele me parece agora um homem muito calmo. Súbito faz uma reflexão breve,
porém de repente me devolve um olhar estranho que parece incluir uma
aura sinistra.

Ele se contém após breves instantes de inquietação. Ou então, eu noto
que ele pode estar farmacologicamente contido, e assim vai retomando
suas boas maneiras, vai adquirindo sua aparente calma, edificando sua
tranquilidade como se essa tranquilidade fosse uma
\emph{persona}~peculiar, ou como se fosse uma tranquilidade
``vulcânica'' potencialmente eruptiva e apenas falsa e superficialmente
sincera.

%\begingroup\small
\begin{center}\asterisc{}\end{center}

\emph{É um caso que me impressionou por se tratar de uma pessoa que não
dava sinal de que podia ter crises absurdas de violência. Por outro
lado, o caso talvez exemplifique o papel de certas epilepsias atípicas
em transtornos de personalidade e/ou de impulsividade.}

\emph{Mas isso é apenas hipótese, porque epilepsia é o que este homem
alegava ter, e ele de fato tinha sido medicado para esse transtorno. E~se o exame (eletroencefalograma) era normal, isso não revelava muito,
porque existem várias epilepsias com este exame normal.}

\emph{De qualquer maneira eu parto da hipótese do quão significativo é o
fato deste homem ter incorporado essa epilepsia --- verdadeira ou falsa
--- em sua vida. Mas a minha intenção não é transformar, de forma
reducionista, a queixa deste homem num caso apenas médico.}

\emph{Sem dizer que a questão em si da epilepsia é complexa. A~começar
que nem se trata propriamente de uma doença, é sim de uma manifestação
sindrômica.}

\emph{Uma crise convulsiva é, em resumo, uma espécie de tempestade
elétrica no cérebro com um padrão repetitivo quanto à forma, ritmo e
duração.}

\emph{Toda a rede neural é um imenso sistema de conexões eletroquímicas
onde uma quantidade imensa de neurônios conta com uma quantidade ainda
muito maior de ligações (sinapses), configurando uma rede, digamos
assim, assustadora de comunicação. Todas as atividades cerebrais podem
ser, de alguma forma, medidas por ativações eletroquímicas de acordo com
inúmeras combinações extremamente complexas.}

\emph{A~epilepsia tem história. Já foi a doença sagrada na época dos
gregos antigos, já foi possessão satânica na Idade Média. Depois
civilizou"-se, porém civilizou"-se em termos.}

\emph{Em época moderna inventou"-se até um ``caráter'' ou uma
``personalidade epiléptica'', algo que não tem mais reconhecimento hoje.
Gente famosa, como Dostoiévski e Machado de Assis, teria sido
epiléptica.}

\emph{O~assunto ainda está em aberto e admite um território ambíguo
entre o que seria uma crise verdadeira ou crise orgânica, e outras
manifestações da mente e do cérebro.}

\emph{O~assunto é também complexo, e justamente por isso vou criar
provocações para o leitor.}

\emph{Do ponto de vista da rede neural e dessa transmissão
eletroquímica, admite"-se a hipótese (e apenas hipótese) de que exista um
determinismo nas ações que emanam de comandos do cérebro. Melhor
dizendo: há quem admita que a nossa capacidade livre de escolha seja, lá
no fundo, uma ilusão.}

\emph{Conforme essa hipótese, teríamos que deixar um pouco de lado a
suposição da existência de um ``eu'' totalmente autônomo por detrás da
nossa identidade íntima, e também teríamos que deixar um pouco de lado a
suposição da existência de um ``eu'' relacionado à antiga ideia de
``alma''.}

\emph{Sabemos bem que a esse ``eu'' tem sido atribuída uma capacidade
intrínseca decisória que pode ser chamada (de um ponto de vista não
apenas religioso) de livre"-arbítrio. Mas, lá no fundo, pergunto: seria
essa capacidade intrínseca decisória uma ficção explanatória?}

\emph{Reflita você, caro leitor, e de resto, calma aí, não quero
concluir ou ir longe no mérito dessas questões. Apenas provoco. Estou
ciente de que há os que respondem sim ou não a essa pergunta. E~ainda
digo que existe uma interface meio indefinida entre essas duas posturas,
embora o ser humano seja apenas previsível e determinado teoricamente.
Na prática sempre contamos com a presença de uma capacidade decisória.}

\emph{Tanto é verdade que, no desenrolar da vida real, ninguém alcança o
fundo de si mesmo e ninguém é transparente nem a si e nem aos outros.}

\emph{Se por acaso achamos que somos o que pensamos, não sabemos tudo o
que pensamos e não sabemos bem porque somos e nem porque decidimos.}

\emph{No entanto, existe um conhecimento objetivo a respeito da mente e
do cérebro.}

\emph{E bem a propósito volto agora à epilepsia.}

\emph{Hoje se sabe da existência de certas crises epilépticas,
principalmente as de lobo temporal, que se expressam através de mudanças
de conduta, eventualmente com manifestações antissociais agressivas e
até alterações bizarras de comportamento.}

\emph{Deve haver por aí muitos indivíduos até ``monstruosos'', vítimas de
tempestades elétricas com padrão e ritmo. Mas na prática dos casos que
vemos no dia a dia não é fácil provar a favor ou contra a origem de
certos distúrbios de comportamento como sendo tempestades elétricas do
cérebro.}

\emph{Daí a pergunta"-chave: tais pessoas decidem o que fazem ou elas são
levadas a uma ação por um mecanismo eletroquímico determinista, seja no
caso dessa ``tempestade'' episódica que é, sob alguns aspectos, o que se
tem como uma convulsão peculiar?}

\emph{Eis uma boa questão, e uma questão complicada. E~se não é o caso
de muito filosofar aqui, digo que a região temporal do cérebro é crítica
pela proximidade com áreas do sistema límbico que regulam a alta
complexidade do mecanismo de nossas emoções, como, por exemplo, uma
região chamada amígdala.}

\emph{Mas, se tudo isso ainda traz à tona problemas filosóficos e até
jurídicos, bem a propósito cito o caso do direito penal. Se levarmos em
conta uma abordagem rigorosamente científica, como ficaria o livre
arbítrio para muitos atos criminais? Qual seria a base eletroquímica ou
neuroquímica da culpa ou do dolo? E como fica a questão do mal e da
delinquência como escolhas da pessoa acusada de um crime?}

\emph{Juro humildemente, caro leitor, que não tenho respostas finais
para essas difíceis questões. E~não estou simplesmente dizendo que
ninguém escolhe nada. Longe disso.}

\emph{Vou logo confessar o seguinte. Torço humildemente pela existência
de alguma etérea ``alma''. Afinal de contas, não há como a gente
dispensar a crença na existência de um intangível ``eu'' profundo.}

\emph{Ou então, trocando o assunto desta crônica em miúdos: eu busco
entender e identificar alguma ``alma'' neste homem, e não tenho certeza
nem de sua epilepsia supostamente de lobo temporal.}

\emph{Sei apenas que este homem é um dependente grave de drogas e que,
embora tenha experimentado o crack, ficou adicto à cocaína,
``farinha''.}

\emph{De resto, julgo importante diferenciar seu transtorno de conduta
violento de um eventual e possível transtorno de personalidade, que, se
for o caso, é uma marca profunda de como estaria estruturado o seu
caráter (ou melhor, a falta de caráter).}

\emph{Mas este homem, sendo epiléptico ou não, me parece vítima da sua
fúria intempestiva e vítima das misérias e absurdidades de sua criação.
Lá no fundo ele poderia ter apenas um transtorno de impulsividade que o
faz explosivo frente a provocações ou mesmo injustiças, e que o faz
egocentricamente explosivo frente a deslizes alheios.}

\emph{Tenho visto inúmeros casos assim entre os adictos, que, por sinal,
são pessoas, em geral, de ``pavio'' curto. Mas o ``pavio'' curto não é
apenas um problema de adictos. O~ser humano em geral tem tolerância (ou
intolerância) variada à frustração. E~ademais essas variações de
intolerância só existem em função da velha busca por objetos do
desejo.}

\emph{Eis aí uma coisa que a psicanálise descobriu em cima do que
deveria estar evidente desde antes do teatro grego. Portanto, de alguma
forma, tudo o que existe no alcance dos nossos sentidos é desejo.}

\emph{Para Buda tudo é dor. Tudo à nossa volta deve ser uma composição
de dor e desejo --- eis o mundinho de todos nós. Então as pessoas reagem
de maneira diferente à dor das frustrações presentes no dia a dia. E~os
adictos que o digam porque carregam no cérebro e na mente uma
intolerância basal à frustração bem acima da média.}

\emph{Os adictos anseiam obcecados por gratificação contínua e acabam
sendo ingenuamente repetitivos. Eu até diria que alguns deles parecem
``convulsionar'' existencialmente. Quase todos são um pouco} puer\emph{, são
regredidos emocionalmente. Dentre eles há também os histéricos, os
exibicionistas, os narcisistas, e há os que são organicamente detonados
e têm crises orgânicas, a exemplo de uma epilepsia temporal e de outras
marcas na fragilidade do cérebro.}

\emph{Mas no caso deste homem, se é interessante buscar o auxílio
explicativo de áreas diversas da ciência, eu admito que este mesmo homem
também não esteja livre de uma herança genética a indicar a
possibilidade de uma transmissão de genes defeituosos geração após
geração. Ao menos a partir da mãe supostamente psicótica que foi
internada no Juqueri e era cruel ao exercer uma alegada ``maldade''
perante o filho quando menino.}

\emph{Complicado, né?! Sei lá eu, pode até ser!}

\emph{Fiquei convencido dessas mazelas familiares quando ouvi dele um
relato brilhante sobre família, embora deste relato eu não assine
embaixo dizendo ser mesmo verdade tudo o que ele me disse. Mesmo que,
recentemente, eu tenha visto este paciente bem melhor, controlado com
medicação, muito lúcido.}

\emph{Eu vi nele uma certa sinceridade. Ele parecia fazer esforços para
controlar sua fúria, e disso parecia ter boa crítica. E~a menos que esse
indivíduo seja um ótimo ator ou fingidor de primeira, ele não me sugere
ser um ``monstro''.}

\emph{Restaria a ele sim um caráter e até lhe caberia uma condição
relativa de vítima sem lhe tirar uma responsabilidade sobre sua vida.}

\emph{Digo mais: mesmo sem saber quem ele é de verdade, especulo e penso
agora como ele é bem diferente de um Mersault do romance} O Estrangeiro \emph{conforme a crônica ``Sem choro nem vela''.}

\emph{Sim, porque ele não tem a frieza existencialista ateia ou no
mínimo agnóstica de um Mersault. Ele é crédulo e sugestionável. Suas
emoções extravasam. Ele é ``quente'', e a melhor imagem que tenho dele é
do vulcão feito montanha.}

\emph{Quieto, imóvel, soltando um pouquinho de fumaça e criando uma
grande sombra quando o sol bate do lado oposto. ``A sombra do vulcão'',
portanto, é um título que veio tardio e a calhar.}

\emph{A~sombra do vulcão é um contraponto a mostrar como todos nós somos
frágeis por sermos feitos de carne e osso e dessa massa de neurônios a
se comunicar eletroquimicamente.}

\emph{A~menos --- repito --- da existência de uma ``alma''
plurimetafórica que muitos de nós acreditamos de maneiras diversas, e a
respeito da qual alguém ainda alegaria a seguinte pergunta: quem prova o
contrário da existência dessa ``alma''?!}

\emph{Mas com ou sem uma ``alma'' íntima, somos desejo e somos
impulsividade. Ao bobearmos na ciranda das tentações do mundo, viramos
escravos do desejo, viramos adictos com ou sem epilepsias temporais ou
coisa e tal.}

\emph{Nossa pífia autonomia pode desaparecer com ou sem droga química.
Eis aí a dependência propriamente dita, que conta ainda com o desejo,
força bruta a crescer e a chegar por vezes na fúria má dos homens bons,
como talvez (digo mesmo talvez) seja o caso deste homem, que é um tema
suculento em discussões filosóficas sobre determinismo} versus \emph{livre
arbítrio.}

\emph{No depoimento deste homem profundamente perturbado abrem"-se
dúvidas cruéis caso se leve em conta a suposição de que ele não seja tão
somente vítima de uma doença neuropsíquica.}

\emph{Então eu me pergunto: seria ele antes um produto de sua história?
Seria ele todo construído a partir de contradições vivas inscritas
através de sua criação e sentidas na carne, na porrada e na fúria?
Talvez.}

\emph{Eu o vejo bastante como uma contradição humana viva e aberta que
para mim vira literatura. Apesar da sua agonia de vida, sua história
pode ter um final menos infeliz do que o temido trágico, porque vulcões
podem ficar extintos ou permanecer quietos por quase uma eternidade. Ou,
de repente, não.}
%\endgroup