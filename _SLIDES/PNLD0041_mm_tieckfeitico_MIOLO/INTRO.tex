\chapterspecial{Ludwig Tieck e a poesia popular}{}{Maria Aparecida Barbosa}

\section{Sobre o autor}

\noindent{}Johann Ludwig Tieck (Berlim, 1773--\textit{id.}, 1853), 
foi um dos mais importantes fundadores do movimento romântico alemão. 
Estudou História, Filologia e Literatura Moderna em Halle, 
Göttingen e Erlangen. Entre suas primeiras narrativas estão os romances 
\textit{Die Geschichte des Herrn William Lovell} (A história de William Lovell, 
1795--96) e \textit{Franz Sternbalds Wanderungen} 
(Peregrinações de Franz Sternbald, 1798). 
Ao publicar esse último, um romance de artista, conhece o filósofo 
Friedrich von Schlegel.

Dois anos mais tarde, passa a residir em Jena, 
onde efervesciam as  discussões do ``Primeiro Romantismo'' ou ``Romantismo 
Inicial'', e nessa  cidade conviveu com os escritores August e Friedrich von Schlegel,  Novalis, Brentano, e os filósofos Fichte e Schelling. Além de poeta e escritor, destacou"-se como editor e tradutor. Como tradutor, verteu o \textit{Don Quixote}, de Cervantes, e boa parte da obra de Shakespeare, dando seguimento às traduções iniciadas por A.W.~von Schlegel, 
em parceria com sua filha Dorothea, e o marido desta,
Graf von Baudissin.

Como editor, publicou obras de Kleist e Novalis.
Os contos e as peças teatrais da coletânea \textit{Phantasus} conquistaram 
sucesso junto ao público, bem como suas novelas tardias. Em 1834, consagrou 
a Camões a novela histórica \textit{Tod des Dichters} (Morte do poeta), onde 
reconta o destino do poeta maior da língua portuguesa. Em 1842, a convite 
de Frederico Guilherme \textsc{iv}, rei da Prússia, estabelece"-se em Berlim, 
participando ativamente da sociedade literária de então.


Desde 1792 quando estudava inglês em Göttingen, Tieck se dedicava ao
estudo dos dramas de Shakespeare. Em 1796, é publicada \textit{Der Sturm}, 
sua tradução de \textit{The Tempest}, acompanhada do ensaio
``Shakespeares Behandlung des Wunderbaren'' (O tratamento shakespeariano
do maravilhoso). Referindo"-se ao ambiente mágico em torno dos
personagens Próspero e Talibã, ele enalteceu no escritor inglês a
``prodigiosa fecundidade da fantasia'' (\textit{die wunderbare Schnelligkeit der
Phantasie}) e a ``incompreensível instantaneidade e flexibilidade da
imaginação'' (\textit{die unbegreiflich schnelle Beweglichkeit der Imagination}).
Não obstante a tentativa de Tieck no sentido de corresponder à
preciosidade do colorido mágico, em resenha publicada em 1798 na
\textit{Athenaeum}, revista de ensaios que instituiu o movimento
romântico alemão, August von Schlegel lamentou a dissolução dos
arrojados versos do drama \mbox{inglês} em prosa, pois atribuía grande 
valor a fundamentos sensíveis como rima e métrica.

%A crítica de seu futuro interlocutor repercutiria de modo construtivo no próximo trabalho tradutório: 

\section{Sobre a obra}

\textit{Feitiço de amor e outros contos} reúne seis das narrativas macabras ou fantásticas de \textit{Phantasus}, coletânea de Ludwig Tieck que compõe"-se de ``contos maravilhosos artísticos'', que são composições literárias com características populares, mas estilizadas por um escritor. 
\textit{Phantasus} é introduzida pelo texto ``À guisa de prefácio, 
a W.~Schlegel'', cujo primeiro parágrafo eu aponho a seguir, a título de
ilustrar o intercâmbio bastante profícuo entre dois intelectuais do período:

\begin{hedraquote}
Foi uma bela época de minha vida, quando primeiramente o conheci e
também seu irmão Friedrich; mais bela ainda, quando nós e Novalis
vivemos juntos pela arte e pela ciência e nos encontrávamos em diversos
empreendimentos. Agora, o destino nos separou há muitos anos. Senti sua
falta em Roma e, da mesma forma, mais tarde, em Viena e Munique. A
saúde precária impediu"-me de procurá"-lo no local onde você residia; só
em espírito e na lembrança eu podia viver com você.
\end{hedraquote}
  
A coletânea foi concebida em três partes, cada uma com sete textos,
mas assim como tantas outras obras dos escritores românticos, permanece
inacabada em forma de fragmento: consistiu no final das contas em sete
contos maravilhosos artísticos e seis dramas. Os textos mais antigos e a ideia de
integrá"-los numa composição intermediada por diálogos à maneira da
série de contos \textit{Decameron}, do escritor italiano Boccaccio,		
remontam à última década do século \textsc{xviii}, ou seja, 
ao início do período romântico. A complementação, através de 
intervenções e de textos extras aconteceu por volta de 1810. 

Sete homens, alguns deles perdidos durante um passeio, e quatro mulheres
reúnem"-se acidentalmente numa casa de campo. As condições românticas e
ideais do convívio entre esses amigos favorecem tanto as conversas
sobre educação e desilusões amorosas, como discussões sobre arte em
geral: poesia, jardinagem e teatro, abordando questões pertinentes à
época. Os diálogos entre as narrativas deixam margens a deduções e
identificações dos personagens como figuras representantes da sociedade
intelectual daquela época. 

Num poema apresentado ainda no começo do livro, Fântaso, deus grego dos
sonhos com seres inanimados, empresta nome à série de contos e peças
teatrais de Tieck, pois é ele o guia que inicia o poeta pelas
manifestações assustadoras da natureza no universo: o medo, a tolice, o
gracejo, o amor.  Ao despertar febril e inspirado pela vidência, o
poeta passou a contar aos amigos a primeira história: ``O Loiro Eckbert''.

Tieck convocou o personagem Ernst para referir"-se à forma de narrativa
exemplar na qual deveriam se pautar os amigos"-autores: 

\begin{hedraquote}
O bem e o mal perfazem a aparição dupla, que a criança mais facilmente
compreende em toda história, que em toda composição mais de uma vez nos
sensibiliza, que nos agrada em seus mistérios de formas diversas por si
mesmos se desvelando, se revelando à compreensão. Há uma maneira de ver
a vida \mbox{ordinária} como um conto maravilhoso, assim é possível se deparar
com o maravilhoso como se fosse o mero cotidiano.
\end{hedraquote}

A seleção dos seis contos que perfazem esta publicação, inédita em português,
privilegia o horror nas narrativas de 
\textit{Phantasus}. E esse recorte encontra"-se
representado por excelência e de maneira mais explícita no conto
``Feitiço de amor''. Inclusive na atmosfera perturbadora de expectativa que
se instaura no conto ``O cálice'' e evolui para a celebração final		
através da solução dos desenlaces, o macabro predomina e permeia toda a
narrativa. Mas nesse caso, todavia, fica evidente que, assim como nos
``contos populares'', a trama nos ``contos artísticos'' de Tieck nem sempre
conflui ao final macabro. 

As traduções que o leitor apreciará a seguir representam o primeiro esforço para 
apresentar um autor chave do Romantismo alemão ao público brasileiro, 
autor sobre o qual o crítico Otto Maria Carpeaux escreveu: 

\begin{hedraquote}
Nos vinte e dois volumes das Obras Completas de Tieck 
muita coisa boa e bela está enterrada e esquecida. Durante trinta anos, 
Tieck rivalizou com Goethe: muitos consideravam"-no o verdadeiro centro da
literatura alemã.\footnote{ Otto Maria Carpeaux. \textit{A literatura alemã}. 
São Paulo: Nova Alexandria, 1994.}
\end{hedraquote}

\section{Sobre o gênero}

\epigraph{A Poesia é una, amálgama se compondo desde tempos imemoriais
ao futuro longínquo, com as obras hoje acessíveis, as perdidas passíveis
de complementação por nossa fantasia e as do porvir deixando"-se pressentir. 
Incessantemente ela quer se revelar, de tempos em tempos surgindo rejuvenescida 
e em novas formatações.}{\textsc{tieck}, prefácio às \textit{Canções de trovadores}}

A tradução dos contos macabros do poeta romântico Ludwig Tieck dois
séculos após a publicação da coletânea \textit{Phantasus}, na qual estão
inseridos, pressupõe que o espírito romântico e poético que ele nos 
legou possa se revestir de um pontual ressurgimento da
Poesia Una, ora numa configuração em língua portuguesa. 

A pesquisa sobre as diversas raízes e heranças dessa literatura fornece
um emaranhado espesso, com derivações de canções de trovadores
medievais e da literatura inglesa que o escritor admirava. Priorizando
a dimensão histórica, forneço a seguir algumas informações sobre o
contexto ao qual Tieck se filia na literatura alemã, numa
tentativa de perceber em que medida o escritor recorria aos modelos da
literatura inglesa e dos trovadores alemães medievais na composição de
sua obra. Pois é da afinidade com o ``maravilhoso'' e o horror
proveniente dessas fontes que nascem seus contos majestosos, cuja
amostra nós organizamos e traduzimos do alemão com a intenção de
apresentá"-la ao leitor brasileiro. 

Tieck traduziu Cervantes e, juntamente com a filha, Dorothea (1799--1841), 
von Baudissin (1789--1878) e Wilhelm von Schlegel (1767--1845), 
verteu e editou dramas de Shakespeare. Foi um persistente leitor e
divulgador do escritor inglês e a predileção deixou marcas indeléveis
em sua produção literária. Nos contos ``A Montanha das Runas'' e ``Eckart
Fiel e Tannenhäuser'', por exemplo, a influência de \textit{Ossian},
criações do escocês James Macpherson baseadas em baladas gálicas, é
evidente no conteúdo, no qual percebemos semelhante fecundidade da
fantasia e o mundo de heróis em infindáveis batalhas (Eckart). Por
outro lado, atribuindo grande valor às características da poesia
medieval dos trovadores, Tieck resgatou a forma predileta desses poetas,
os versos heptassilábicos em redondilha maior, modelo poético que
muitas vezes adotou. Sobretudo no conto ``Eckart'', tentei traduzir 
os versos nessa forma ao português, inclusive
empregando \textit{cursus velox}, com tônicas na 2ª e 7ª sílabas.

Diz"-se que as trupes de teatro inglês que circularam pela Alemanha no
século \textsc{xvii} já haviam encenado versões anônimas e descaracterizadas de
Shakespeare. Mas a recepção do dramaturgo inglês nos territórios de
língua alemã é inaugurada concretamente em 1761, ainda no auge do
Iluminismo, quando o escritor Christoph Wieland (1733--1813) levou ao
palco de sua terra natal, a cidade de Biberach, a primeira encenação de um
drama shakespeariano na Alemanha: sua tradução de \textit{The Tempest}. 

No contexto do movimento literário ``Sturm und Drang'' (Tempestade e
Ímpeto), que de certa maneira prenuncia o romantismo alemão, a pesquisa
de Johann Gottfried Herder (1744--1803) se pautava predominantemente em
canções, baladas e contos populares. Num livro que publicou em
1773, ele conjugava seu próprio texto ``Trecho de uma correspondência
sobre Ossian e as baladas dos povos antigos'', ``Sobre a arquitetura
alemã'', de Goethe, bem como ``A história alemã'', de Justus Möser. O
título \textit{Von deutscher Art und Kunst} (Sobre o modo e a arte
alemães) pode proporcionar uma ideia do bombástico efeito que a
reivindicação do grupo surtiu no ambiente intelectual da época,
dividido entre a tradição do barroco e as forças iluministas. 

Foi nesse mesmo ensaio que Herder pela primeira vez empregou o conceito de
\textit{Volkslieder} (canções populares); a menção consta da passagem onde
descreve seus inúmeros contatos em regiões provincianas com canções
populares, canções de província (\textit{Provinziallieder}), canções de
camponeses (\textit{Bauerlieder}), as quais em vitalidade, ritmo, inocência e
potência na linguagem nada deixariam a desejar. Mas acrescentou: ``No
entanto, quem se ocupa de canções do povo? das ruas, becos e mercados
de peixe! das rodas populares de canto entre camponeses?''.

Dando corpo às teorias, em 1778/9 vem à luz a coletânea \textit{Volkslieder} na
qual o linguista integrou cantigas de camponeses, cantos indianos, bem
como poemas de Goethe, Matthias Claudius e Shakespeare. Justificando a
mixórdia de textos, Herder procurou demonstrar sua convergência na
origem que comungavam, todos provenientes de culturas efervescentes 
de vitalidade:

\begin{hedraquote}
Quanto mais selvagem, ou mais cheio de vida, quanto mais inconsequente
um povo é [\ldots], mais selvagem, mais vital, livre, sensual e lírica
serão necessariamente suas canções, caso tenha canções! Quanto mais
distanciado do pensamento artificial e científico e das letras é um
povo, menos apropriadas serão suas canções para o papel e os versos
mortos: dos aspectos líricos, vitais e ao mesmo tempo rítmicos do
canto, da atualidade viva das imagens, do contexto e ao mesmo tempo da
contundência dos motivos, da sensibilidade e simetria das palavras,
sílabas, em alguns casos inclusive das letras, a evolução da melodia e
centenas de outros pormenores que dizem respeito ao mundo vivo e à
ideia de canções de provérbios e de poesia nacional (Spruch- und
Nationallieder); disso, e somente disso depende a essência, a força
prodigiosa que essas canções possuem. Nisso consiste o encanto, a mola
propulsora da eterna herança e prazer do canto do povo!
\end{hedraquote}


Ao expressar"-se nesses termos, Herder buscava reabilitar o espírito
literário vital e vigoroso, mas também o ``nacional'', o que gerou
interpretações controversas. A polêmica em torno do conceito ``contos
populares'' (\textit{Volksmärchen}) desencadeia uma cisão dentro 
do romantismo alemão. Na medida em que era dotada de vigor poderoso 
e peculiar, porém, a tradição cultural de origem popular passava a 
representar uma alternativa aos territórios dominados em sua luta 
pela autonomia cultural e política, como era o caso da Alemanha, 
sob influências francesas e, mais recentemente, greco"-romanas.

O grupo da cidade de Heidelberg, composto entre outros por Ludwig Achim
von Arnim (1781--1831) e Clemens Brentano (1778--1842) tentava fundar,
dentro do movimento romântico, um segmento marcado pela autorreferência
e nacionalismo, excluindo tudo que é estrangeiro.\footnote{ ``alles
ausländische\ldots{} \textit{ausgeschlossen}'', segundo Segeberg: ``Phasen der
Romantik''. In: \textit{Romantik"-Handbuch}, p. 52)} Identificando"-se
como herdeiros da tradição iniciada por Herder, esses autores publicam
em 1805/6 e 1808 a obra \textit{Des Knaben Wunderhorn} (A cornucópia
do menino), uma coletânea de contos populares. As interpretações da
herança de Herder foram conduzidas de modo diferente, respectivamente
pelos escritores românticos das cidades de Jena e os de Heidelberg, e
essas tendências nortearam, portanto, duas vertentes de pesquisa. 

Célebres se tornaram as publicações dos irmãos Grimm, Jacob (1785--1863)
e Wilhelm (1786--1859) contendo \textit{Kinder- und Hausmärchen}
(Contos maravilhosos para o lar e as crianças), que
representam uma interpretação universal. As primeiras edições surgiram
com os volumes de 1812 e 1815, com contos extraídos da tradição oral, e
com o volume de 1822, com versões diferentes e esclarecimentos — uma
contribuição para a pesquisa dos contos populares.

Numa correspondência ao escritor Ludwig Achim von Arnim, com quem
colaborara nas pesquisas para \textit{A cornucópia do menino}, Jacob tentou
delimitar o autêntico sentido do ``conto popular'' (\textit{Volksmärchen}), para
ele a mais elevada acepção e a forma original da poesia, do ``conto maravilhoso
artístico'' (\textit{Kunstmärchen}): ao primeiro caso, ele se referia como um ato
de criação coletiva da alma popular. O segundo caso constituía a poesia
artística (\textit{Kunstpoesie}), resultado de uma preparação subjetiva e pessoal. 

Outros românticos, como Ludwig Tieck e Novalis (1772--1801),
transformaram"-lhe o conteúdo popular num significado individual, de
acordo com um plano de criação artística bem consciente. O poeta
Novalis empregou parte do acervo, por exemplo, em seu romance
fragmentário \textit{Heinrich von Ofterdingen}. Por sua vez, Tieck
recorria ao conto popular a fim de atingir seu programa
político"-literário, que consistia em denunciar a banalidade da
literatura de entretenimento através de estilizações de contos de
Charles Perrault (\textit{O Gato de Botas, Barba Azul}). Coerente, sempre
perseguiu o objetivo de conferir traços de romantização ao \mbox{ordinário}.  

Como tradutor, Tieck também mergulhou na tradição popular, traduzindo
canções de trovadores medievais (da
Suábia, no sul da Alemanha) ao alemão de sua época, de 1803. A experiência
adquirida do contato com as formas métricas rigorosas durante essa
tradução redundou em embasamento propício às próprias criações, pois
consistiu em exercício de escansão e versificação dentro dos moldes
daquela poesia de arquitetura primorosa. No prefácio à adaptação da
poesia medieval, Tieck justifica a fidelidade às rimas peculiares às
canções originais, devida segundo ele menos ao impulso de afetação que
à valorização do tom e do som, da sensibilidade a palavras e sons
semelhantes, pois inerente em parentescos repletos de mistérios subjaz
necessariamente a busca comum pela poesia através da musicalidade.
Resguardando o caráter das canções de trovadores medievais, desenvolveu
a habilidade dos setissilábicos em redondilha maior que mais tarde
empregou nos contos ``A Montanha das Runas'' e ``Eckart''. Com a
finalidade de ilustrar, aponho abaixo, respectivamente, uma amostra da
adaptação das trovas medievais (Tieck), em seguida uma estrofe do
``Eckart'' em alemão e minha tradução:

\begin{verse}
In rechter Schöne ein Morgensterne\\
Ist meine Frau der ich gerne\\
Diene und immer dienen will\\
Wie klein sie mir Freude mehre\ldots{}

Der Fels spring voneinander,\\
Ein bunt Gewimmel drein,\\
Man sieht Gestalten wandern\\
Im wunderlichen Schein.

A rocha ao meio se cinde,\\
E roja de dentro um tropel.\\
Se vê figuras surgindo,\\
Ao clarão misterioso do céu.
\end{verse}

Se em alguns dos contos mais antigos, encontramos a similaridade da
forma com o trovadorismo medieval que Tieck adaptava talvez
concomitantemente, no desenvolvimento das narrativas fica visível o
esforço por prolongar os interstícios que introduzem, enfim, 
o ``maravilhoso'', efeito de postergação que admirou e elogiou em \textit{The Tempest} e
\textit{A Midsummer Night's Dream}, peças que considerava suaves e amenas em
contraste com as gigantescas figuras de Macbeth ou Othello.
Diferentemente da noite velada assustando os mortais em \textit{Hamlet}, o
reino da noite naqueles dois casos seria clareado por branda luz de
luar.  Mas ao invés do ``maravilhoso cômico'' das duas comédias
shakespearianas, tom que ele privilegiou em sua peça teatral ``O Gato de
Botas'' por exemplo, nos desfechos dos contos maravilhosos artísticos desta seleção
prevalece o tom do gênero maravilhoso aliado ao horror.  

\begin{bibliohedra}

\tit{arnim}, L. A. von. e \textsc{brentano}, C. \textit{Des Knaben Wunderhorn}.
Disponível em 15/8/2009: 
<http://gutenberg.spiegel.de/?id=5\&xid=3246\&\break kapitel=1\#gb\_found>

\tit{frank}, Manfred (org.). \textit{Ludwig Tieck Phantasus }(volume
6). In: \textit{Ludwig Tieck Schriften in zwölf Bänden}.  Frankfurt am
Main: Deutscher Klassiker Verlag, 1985. 

\tit{herder}, Johann Gottfried. \textit{Von deutscher Art und Kunst}.
Disponível em 15.08.2009: <http://www.zeno.org/Literatur/M/Herder,+Johann\break+Gottfried/Theoretische+Schriften/Von+deutscher+Art+und+\break Kunst/1.+Auszug+aus+einem+Briefwechsel?hl=zusammenhange\break +und+gleichsam+notdrange>

\tit{tieck}, Ludwig. “Shakespeares Behandlung des Wunderbaren”. In:
\textit{Der Sturm} [The Tempest], trad.~Ludwig Tieck. Disponível em 15/8/2009: 
<http://www.gutenberg.ca/ebooks/tieck"-sturm/tieck"-sturm-00-h.html>

\end{bibliohedra}


















