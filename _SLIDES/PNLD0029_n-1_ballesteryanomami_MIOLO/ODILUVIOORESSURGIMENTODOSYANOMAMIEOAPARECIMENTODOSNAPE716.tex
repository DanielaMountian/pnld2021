\chapterspecial{O dilúvio}{O ressurgimento dos yanomami e o aparecimento dos napë}{}
% Resolver subtitulo

\letra{D}{epois da morte} do irmão e do sobrinho, Yoahiwë e Omawë fugiram rio
abaixo. Havia somente um rio, o rio Tanape. Eles encontraram, no
percurso, outro sobrinho, filho de Manakayariyoma, cuja mãe se
considerava a irmã dos dois, por ter o mesmo nome que a irmã deles. 

Como se deu esse encontro? Enquanto os dois estavam no meio do rio, eles
escutaram um chamado vindo de cima. 

To! To! Escutaram um som descendo na direção deles. Hɨ̃ tuuuuu! Fazia o
bebê descendo na direção deles. O sobrinho desceu na sua direção com
sede. Hɨ̃ɨɨ tëɨ! A criança estava amarrada em uma haste de palmeira.

--- Ũa, ũa, ũa! Tio! Tio!

Acabava de descer ao chão, sentado no meio daquela haste de palmeira:

--- Sede! Sede! Tio! Sede!

Eles o agarraram. Não foi gente que deu à luz esse bebê. Ele não
tinha pai gerador, mas apenas apareceu nessa haste de palmeira. 

Levaram"-no, muito sedento. Ninguém o gerou! Levaram"-no, pois era o
sobrinho. 

Foi lá que os dois encontraram o filho de Manakarariyoma. Por causa
desse bebê encontraram o grande rio fechado. Os humanos foram
exterminados por causa daquele bebê sedento que surgiu do nada. 

Como? Omawë não abriu sem razão essa água na qual se afogaram os
Yanomami; foi por causa do sobrinho cujo fôlego se apagava. Ele morria de
sede. 

Como se deu esse evento?

--- Ẽ, ẽ, ẽ! --- assim ele respirava. 

Vendo a criança arfar, com a moleira se esvaziando, o tio chorava por
causa da sede do bebê. 

--- Não vou deixar meu pequeno sobrinho morrer! Não vou ficar feliz,
não! --- dizia chorando e recuando. 

Os olhos do bebê estavam virando e Omawë dava a espuma de sua baba para
ele beber. O bebê chupava, mas chupava em vão. Sua boca se enganava com a
urina do tio. Ao final, o irmão mais novo refletiu e decidiu conseguir água de qualquer forma. 

Ele previu que a água estava guardada embaixo de uma pedra. Tuku, tuku, tuku! A água batendo debaixo da pedra fazia esse som. 

A pedra era muito dura, estava bem fincada; mesmo assim ele conseguiu
tirar um pouco, não de vez. Ele suspendeu a pedra só um pouco,
inclinando-a para o lado, não por baixo. Assim que ele a empurrou um pouco, a
água jorrou. O bebê, que já estava morto, ressuscitou por causa dessa
água.

Tuuuuuuuu! Fez a água. 

A água logo jorrou longe. O jorro caiu lá, bem em baixo. A água se
curvou e escondeu o céu. A água jorrou durante duas noites e o rio
encheu rapidamente. Por causa da sede daquele bebê, a gente daquela
época desapareceu. 

Foi depois desse evento que nossos antepassados surgiram; a água levou
os mortos lá pra baixo, onde os dois preparavam redes. Omawë, Yoasiwë e o
sobrinho fizeram uma cerca de paxiúba dura como ferro fincado na água,
para reter os mortos levados pelo rio. No mesmo lugar onde Omawë e
Yoahiwë se localizavam, eles fizeram outro jirau bem forte para moquear
os mortos e fazer aparecer os \emph{napë.} 

Feito isso, a água já trazia outras pessoas sofrendo, os afogados. O rio ficava bem estreito no meio da terra dos \emph{napë}. Pegaram os
mortos lá bem perto da terra dos \emph{napë}, lá em baixo. Chegavam
pelas águas os quase mortos, onde estavam os dois. 

O que fizeram Omawë e seu irmão? Eles usaram aquelas redes que haviam
tecido. O irmão mais velho, que era mais esclarecido, disse: 

--- Irmão menor! Faça logo, os meus já estão passando! Meu irmão, os
primeiros já estão passando! 

Prepararam corda de embira. Os dois teceram um tipo de tarrafa transparente.
Eles salvaram as pessoas com tarrafa de embira \emph{omaoma}. Não paravam de
lançar a rede onde chegavam os mortos, pegando um a um. Puxaram a rede
como se fossem peixes, muitos peixes. 

Feito isso, ele e o irmão mais novo os jogaram em cima do jirau, um por
um, quando o fogo grande se erguia; eles os assaram como se fossem
caça. 

Xãaaai! O fogo crescia com a gordura derretida de gente. O irmão
mais novo cortou as folhas novas de sororoca e as deixava no chão. Em
cima dessas folhas, juntava os corpos cremados, enquanto o outro
recuperava os corpos no mesmo instante. 

O sobrinho, que rapidamente cresceu, ajudava seu tio. 

Eles os jogavam em cima das folhas, os corpos cremados. O irmão mais
velho recuperava os corpos e os raspava. 

Ele raspava os corpos cremados com um tipo de colher grande. Xoe! Xoe!
Xoe! A pele dos cremados produzia esse som. 

Eles jogavam os corpos cremados em cima das folhas novas de sororoca,
que estavam no chão uma perto da outra. 

Aqueles que estavam bem raspados, eles separavam, pegando"-os com um
tipo de arpão. Tuuuuuuuu! Eles não erravam. 

--- Aë, aë, aë, aë, aë, tãrai! --- faziam os dois assim. --- Aë, aë, aë,
pei kë oooo! Tãrai! 

Fizeram assim levantar os Yanomami ressuscitados com flechas na mão. 

--- Todos, todos, todos, todos! Tãrai! De pé! --- disse Omawë.

Primeiro ressurgiram os Yanomami e depois apareceram os \emph{napë}. 

Somente os Yanomami Horonamɨ se ergueram com as flechas na mão. 

--- Kia! Kia! Kia! Ha, ha, haaaa! --- diziam eles. --- Tãrai, ha asi
ɨ̃ɨɨ! 

Eram gordos, pintados e enfeitados, com as penas de cauda de arara,
altivos.

--- Ha, ha, ha, ha! 

O irmão mais velho riu sem parar dos que estavam se transformando. Olha
aqui! As mulheres púberes se erguiam elegantes, apesar de terem morrido
afogadas, elas reapareceram como moças novas. 

Depois de ter amontoado a metade dos corpos cremados, eles os jogaram na
água. Os mortos não caíram na água em silêncio. 

O primeiro a dizer: Ĩxima! Ĩxima! se tornou piranha. Ĩxima! Ĩxima! Todos
que disseram isso se tornaram piranhas. 

--- Koooorooouuu! Kuxu, kuxu, kuxu! --- faziam. 

Os demais se transformaram em matrinxã. Apareceu um monte de peixes
flutuando. A superfície da água ficou completamente coberta e sumia de
tanto peixe, a água não se mexia mais de tanto peixe, de cuja carne gostamos
tanto.

Quando dizemos:

--- É um pacu! --- na verdade, comemos a carne de gente que se tornou
peixe, comemos a carne preta de Yanomami. 

Da nossa carne de Yanomami surgiram os \emph{napë}. Os \emph{napë} surgiram a partir dos Yanomami que se transformaram. A partir dos nossos corpos cremados, com a transformação, os \emph{napë} surgiram. 

Antes mesmo da existência dos \emph{napë}, já havia aparecido a moradia
deles, antes do surgimento dos motores, do canto do galo, e antes dos
próprios ancestrais. Embora sejam a origem de tudo isso, antes da
existência dos \emph{napë} e das \emph{napëyoma} falando uma língua
estrangeira, Omawë e Yoasiwë se surpreenderam quando ressuscitaram e
rasparam os outros. 

Os ressuscitados ficavam de pé, elegantes, e atiravam flechas para
provar que estavam sãos e salvos.

--- Irmão menor, aqueles que eu queria fazer surgir, eu os
deixei à deriva na água! Veja o resultado! --- disse o irmão maior,
porque havia ressuscitado todos eles. 

Omawë se transformou e, onde se transformou, a imagem dele se
tornou \emph{napë}. Ele criou os \emph{napë}. Lá a sua imagem se misturou
e ainda está lá, eles não a veem. 

--- Eu sou aquele que ressuscitou vocês! --- ele diz isso? Omawë não diz
isso. Ele existe ainda, ainda está vivo em algum lugar. --- Sou aquele que
ressuscitou vocês! --- Omawë não nos diz isso. 

Não fica perto onde eles estão morando. Os dois que realizaram essa
ressurreição são os que deram origem aos \emph{napë.} Apesar de os dois
ficarem lá longe em seguida, outros \emph{napë} chegaram. 

O que aconteceu? Se vocês forem pra lá, vocês não chegarão. Os dois
foram até os \emph{napë} de pele vermelha. 

O rio abaixo fica dentro da terra. Como qualquer rio, a cabeceira da
água nunca fica baixa, quando as fontes das águas se juntam, elas
descem; assim as águas do dilúvio se juntaram e formaram o mar. O rio
não desce plano, o rio acaba e entra na terra. 

Os dois que fizeram a transformação moram lá, além dessa parte do mundo,
rio abaixo, lá onde emboca a mãe do rio, onde entra um rio só. Eles são
eternos. 

Onde você pode ir? Não há aonde ir. Lá fica o rio, onde não há floresta,
onde não há nada. Daquele lado vivem os dois, que fizeram a grande
transformação. 

Os \emph{napë} se espalharam. Lá, eles se reproduziram, rio abaixo; não
há mais ninguém onde ressuscitaram. Eles se dividiram rio abaixo. Foram
morar em outros rios. Sim. Foi assim, nenhum dos que viviam
antes sobreviveu. 

Quando o rio levou os Yanomami, sobreviveram somente dois xaponos. Os
antepassados renasceram e se desenvolveram. Eles sobreviveram para
sempre. 

As montanhas altíssimas se chamavam Ũaũaiwë, Rapai e Wãima. Os
sobreviventes conseguiram subir até o cume da serra Rapai, se agruparam
como carapanãs, como moscas, e estavam tristes. 

A serra Ũaũaiwë acabou afundando; restou somente o cume da montanha. O céu parecia se apoiar no cume das montanhas Rapai e Wãima. Não deu para
o rio atingir o cume dessas montanhas. 
