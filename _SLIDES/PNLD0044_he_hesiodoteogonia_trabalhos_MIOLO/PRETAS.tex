\textbf{Teogonia} (em grego \textit{theogonia}: \textit{theos} $=$ deus +
\textit{genea} $=$ origem) é um poema de 1022 versos hexâmetros datílicos que
descreve a origem e a genealogia dos deuses. Muito do que sabemos sobre os
antigos mitos gregos é graças a esse poema que, pela narração em primeira
pessoa do próprio poeta, sistematiza e organiza as histórias da criação do
mundo e do nascimento dos deuses, com ênfase especial a Zeus e às suas façanhas
até chegar ao poder. A invocação das Musas, filhas da Memória, pelo aedo
Hesíodo é o que lhe dá o conhecimento das coisas passadas e presentes e a
possibilidade de cantar em celebração da imortalidade dos deuses; e é a partir
daí que são narradas as peripécias que constituem o surgimento do universo e de
seus deuses primordiais.  


\textbf{Trabalhos e dias} (em grego \textit{Erga kai Hamerai}) é um poema épico
de 828 versos em que são contados alguns dos mitos gregos mais conhecidos até
hoje, como o de Prometeu e o de Pandora. Diferente da \textit{Teogonia}, que
apresenta a origem dos deuses, este poema é voltado para a condição dos
mortais, explicitando suas necessidades e limitações, com foco no trabalho
agrícola baseado nas estações do ano. Com a ajuda das Musas, o poeta narra em
primeira pessoa e se dirige a seu irmão Perses, na tentativa de ensinar a ele
verdades divinas a respeito das práticas humanas.


\textbf{Christian Werner} é professor livre-docente de língua e literatura
grega na Faculdade de Letras da Universidade de São Paulo (\versal{USP}).
Publicou, entre outros, \textit{Duas tragédias gregas: Hécuba e Troianas}
(Martins Fontes, 2004).







