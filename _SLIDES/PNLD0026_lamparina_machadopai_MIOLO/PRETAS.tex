

\textbf{Alexandre Rosa}, escritor e
  cientista social formado pela \versal{FFLCH-USP}, é mestre em Literatura
  Brasileira pelo Instituto de Estudos Brasileiros da \versal{USP}. Participou da
  Coleção Ensaios Brasileiros Contemporâneos -- Volume Música (Funarte,
  2017), com o ensaio \emph{Três Raps de São Paulo}, em parceria com
  Guilherme Botelho e Walter Garcia.

\textbf{Flávio Ricardo Vassoler} é doutor em Letras pela \versal{FFLCH-USP}, com estágio doutoral junto à
  Northwestern University (\versal{EUA}). É autor das obras literárias \emph{Tiro
  de Misericórdia} (nVersos, 2014) e \emph{O Evangelho segundo Talião}
  (nVersos, 2013) e organizador do livro de ensaios \emph{Dostoiévski e
  Bergman: o niilismo da modernidade} (Intermeios, 2012).

\textbf{Ieda Lebensztayn} é
  pesquisadora de pós"-doutorado na Biblioteca Brasiliana Mindlin,
  \versal{BBM-USP/FFLCH-USP} (Processo \versal{CNP}q 166032/2015-8), com estudo a
  respeito da recepção literária de Machado de Assis. Doutora em
  Literatura Brasileira pela \versal{FFLCH-USP}, fez pós"-doutorado no \versal{IEB-USP}
  sobre a correspondência de Graciliano Ramos. Autora de
  \emph{Graciliano Ramos e a} Novidade\emph{: o astrônomo do inferno e
  os meninos impossíveis} (Hedra, 2010). Organizou, com Thiago Mio
  Salla, os livros \emph{Cangaços} e \emph{Conversas}, de Graciliano
  Ramos, publicados em 2014 pela Record.





