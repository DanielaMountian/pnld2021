\textbf{Mark Twain}, pseudônimo de Samuel Langhorne Clemens
(1835--1910), foi um dos mais importantes e prolíficos escritores americanos.
Atuou como tipógrafo, repórter, colunista, palestrante, minerador e piloto de
barco a vapor --- no rio Mississippi ---, ocupação que o levou a conhecer a
realidade do sul e do meio"-oeste americanos, inspiração e pano de fundo para
grande parte de seus personagens e romances, entre eles os famosos \textit{The
Adventures of Tom Sawyer} (1876) e \textit{Adventures of Huckleberry Finn}
(1885). Enveredou também pela crônica, e escreveu contos, ensaios e aforismos.
Dono de uma linguagem irônica e cheia de humor, criticou a sociedade
escravocrata da época, cujos valores, fanatismo e hipocrisia foram alguns dos
alvos prediletos de sua sátira.

\textbf{\textit{Diários de Adão e Eva e outras sátiras bíblicas}} reúne os principais
textos da sátira de Twain a alguns livros e personagens bíblicos. Os diários de
Adão e Eva foram escritos separadamente e publicados em conjunto, em formato de
livro, em junho de 1906, pela Harper and Brothers Publishing House. Era, porém,
desejo do autor publicar essas obras em sequência com outros contos de temática
afim, como o “Solilóquio de Adão'', a “Autobiografia de Eva” e “Passagens do
diário de Satã”, aqui coligidos, entre outros. Isto ocorreu apenas recentemente,
em 1995, quando Baetzhold e McCullogh reuniram e publicaram, na obra intitulada
\textit{The Bible According to Mark Twain}, os mais relevantes textos de
temática bíblica encontrados em uma “caixa póstuma”, espécie de caixa"-preta
literária em que Twain guardava tudo o que escrevera e fora censurado pelas
editoras. Inédito em edição brasileira, esse conjunto de textos ilustra a fase
pessimista --- embora satírica --- de Twain e torna acessível ao público uma das
criações literárias mais engraçadas do grande autor americano, ainda pouco
conhecido entre nós.

\textbf{Sergio Romanelli} é professor doutor no departamento de línguas e
literaturas estrangeiras e na pós"-graduação em estudos da tradução da
Universidade Federal de Santa Catarina. Bolsista em produtividade pelo CNPq.
Possui graduação em letras e filosofia (Universidade de Milão), mestrado (2003)
e doutorado (2006) em linguística aplicada pela Universidade Federal da Bahia.
Presidente da \textsc{apcg} (Associação Brasileira dos Pesquisadores em Crítica
Genética). Editor das revistas \textit{Manuscrítica} e \textit{In"-traduções}.
Coordena os grupos de pesquisa Estudos Linguísticos e aquisição/aprendizagem do
italiano como língua estrangeira e Política Editorial e tradução no Brasil
contemporâneo.

\textbf{Hanna Betina Gotz} possui graduação em licenciatura português"-inglês pela
Universidade Federal do Rio Grande do Sul (1986), mestrado em literatura de
expressão africana, Department of Black Studies, Ohio State University (1989),
mestrado em aquisição de língua estrangeira, Department of Education Studies,
Ohio State University (1991), mestrado em estudos da tradução na Universidade
Federal de Santa Catarina (2011) e doutorado em literatura comparada, Department
of Spanish and Portuguese, Ohio State University (1998). Atualmente é professora
da rede \textsc{senai} de ensino da Grande Florianópolis.

