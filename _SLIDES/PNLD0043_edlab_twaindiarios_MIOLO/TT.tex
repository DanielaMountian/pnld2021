\SVN $Id: TEXTO.tex 13001 2014-01-15 21:06:12Z bruno $

\chapter[Fragmentos do diário de Adão]{Fragmentos do\break diário de Adão\footnoteInSection{Optamos aqui pela versão original, 
conforme publicada no livro \textit{The Niagara Book. A Complete Souvenir of Niagara Falls Containing Sketches, Stories and Essays --- 
Descriptive, Humorous, Historical and Scientific by W.D.~Howells, Mark Twain\ldots{} and Others}. Bufallo: Underhill \& Nichols, 1893.}}
\hedramarkboth{fragmentos do diário de Adão}{mark twain}



\dia{A primeira menção autêntica\\ às cataratas do Niágara}


\begin{flushright}
\itshape{traduzido do manuscrito original\\ por Mark Twain}
\end{flushright}



\dia{Segunda-feira}  
Esta nova criatura de cabelos longos está sempre no meu caminho e
me seguindo para cima e para baixo. Não gosto disso, não estou acostumado
a ter companhia. Preferiria que ficasse com os outros animais\ldots{}
Hoje está nublado e o vento sopra do leste; acho que nós vamos ter chuva\ldots{}
\textit{Nós}? De onde tirei essa palavra? Ah, me lembro agora, é a nova criatura que a usa.

\dia{Terça-feira}  Andei examinando as enormes cachoeiras. É o que há de mais bonito
na propriedade, acho. A nova criatura as chama de cataratas do Niágara --- por
quê? Eu é que não sei. Ela diz que se \textit{parecem} com as cataratas do Niágara.
Isso não é explicação que se dê; é puro capricho e imbecilidade. Não tenho 
chance de dar nome a nada. A nova criatura dá nome a tudo que lhe aparece na
frente, antes mesmo de eu poder protestar. E usa sempre o mesmo pretexto de que se
\textit{parece} com a coisa. O dodô, por exemplo. Ela diz que é só olhar para ele
que a gente vê que “ele se parece com um dodô”. Vai ter de ficar com esse
nome, sem dúvida. Me cansa ficar remoendo isso, até porque não me faz bem.
Dodô! Não se parece mais com um dodô do que eu!

\dia{Quarta-feira}    Construí uma cabana para me proteger da chuva, mas não pude ficar
com ela só para mim, em paz. A nova criatura já se enfiou aqui. Quando tentei
expulsá-la, começou a jorrar água por aqueles dois buracos por onde ela
olha, e enxugava com o dorso das patas, fazendo um barulho que mais parecia o de
animais quando estão inquietos. Eu queria que essa coisa não falasse; fica o
tempo todo falando. Parece um golpe baixo contra a pobre criatura, um
insulto, mas não é isso que quero dizer. É que até agora eu nunca tinha ouvido
a voz humana, e qualquer som novo ou estranho se impondo aqui sobre a solene
tranquilidade dessas sonhadoras solidões fere meu ouvido e parece desafinado. E
esse novo som está tão próximo de mim, sobre meu ombro, bem no meu ouvido,
primeiro de um lado, depois do outro, e só estou acostumado a sons que estejam
mais distantes de mim.

\dia{Sexta-feira} Essa mania de dar nome às coisas continua sem parar, independente do
que eu fizer. Eu tinha um ótimo nome para este lugar, era musical e bonito ---
\textsc{jardim do éden}. Pessoalmente, continuo a chamá-lo assim, mas não em
público. A nova criatura diz que é cheio de florestas, rochedos e paisagens,
e portanto não lembra um jardim. Diz que se \textit{parece} mais com um parque e só.  
Por causa disso, sem me consultar, o rebatizou de Parque das cataratas do Niágara. 
Bastante pretensioso, na minha opinião. E já tem até uma placa: 

\begin{center}
\textsc{não pise na grama}
\end{center}


Minha vida agora já não é tão feliz quanto antes.

\dia{Sábado}  Essa nova criatura come frutas demais. Desse jeito nós vamos ficar sem.
“Nós” de novo --- palavra \textit{dela}; minha, agora, também, de tanto ouvir. Bastante
neblina nesta manhã. Não saio sozinho com neblina, ela sim. Sai
independente do tempo e volta com os pés cheios de lama. E fala sem parar!
Costumava ser tão agradável e calmo antes.

\dia{Domingo}  Aguentei firme. Esse dia está ficando cada vez mais difícil. Foi
escolhido, em novembro passado, para ser o dia de descanso. Eu já tinha seis
desses dias por semana. Esta manhã, encontrei a nova criatura tentando apanhar
maçãs da árvore proibida.

\dia{Segunda-feira}   Essa coisa diz que seu nome é Eva. Para mim tanto faz, não tenho
nada contra. Diz que é para eu chamá-la assim, quando quiser que se aproxime.
Eu disse que achava supérfluo. Essa palavra, evidentemente, fez meu prestígio
aumentar; e, de fato, é uma palavra grande, boa, vai pegar logo. Diz que não é
uma coisa, mas uma mulher. Isso é meio duvidoso; para mim dá na mesma;
o que ela é não me importa, desde que fique na dela e não fale mais. 

\dia{Terça-feira}  Ela sujou a propriedade inteira com seus nomes execráveis e
plaquinhas ofensivas:  

\begin{verse}
\raisebox{-.5mm}{\ding{43}}\quad\textsc{por aqui para a piscina de hidro}\\
\raisebox{-.5mm}{\ding{43}}\quad\textsc{por aqui para a ilha das cabras}\\
\raisebox{-.5mm}{\ding{43}}\quad\textsc{caverna dos ventos por aqui}
\end{verse}

Ela diz que esse parque daria um belo resort de verão se tivéssemos esse
costume. Resort de verão --- outra de suas invenções --- somente palavras, sem
significado algum. O que é um resort de verão? Mas é melhor não perguntar, 
ela detesta dar explicações.

\dia{Sexta-feira}  Agora fica implorando para eu não caminhar pelas pedras das
cataratas. Que mal pode haver? Diz que dá calafrios nela. Me pergunto por quê;
sempre fiz isso\ldots{} sempre gostei dessas quedas e dessa sensação gostosa. Achei
que era para isso que as cataratas serviam. Não vejo outra utilidade nelas, e
para alguma coisa devem ter sido feitas. Ela diz que foram criadas apenas para
servir de paisagem\ldots{} como os rinocerontes e o mastodonte. 

Atravessei as cataratas dentro de um barril, isso não foi bom o suficiente para
ela. Fui até o outro lado dentro de uma tina, ainda não estava bom.
Atravessei o redemoinho e as quedas num traje de folhas de figo. Rasgou todo.
Daí em diante, só reclamações chatas sobre minha extravagância. 
Sou importunado demais por aqui. O que eu preciso é de uma mudança de ares.

\dia{Sábado}  Escapei na terça passada à noite e viajei por dois dias; construí outra
cabana para mim em um lugar recluso, apagando minhas pegadas tão bem quanto
pude, mas ela me achou com a ajuda de um bicho que domesticou e chamou de
lobo, e se aproximou fazendo aquele som penoso de novo, com água saindo
daqueles buracos com que ela olha. Fui obrigado a voltar com ela, mas acho que
vou dar no pé outra vez assim que tiver uma chance. Ela se ocupa com um monte de
bobagens; entre outras coisas, estudar por que os animais chamados leões e tigres se
alimentam de grama e flores, quando, como ela diz, seus dentes indicam que
eles foram feitos para comer uns aos outros. Isso é uma tolice, porque fazer
isso significaria matar um ao outro, o que acabaria introduzindo, a meu ver, o
que se chama de “morte”; e a morte, como me foi dito, ainda não entrou no
parque. O que de certo modo é uma pena. 

\dia{Domingo}   Aguentei firme. 

\dia{Segunda-feira}   Acho que agora entendo para que serve a semana: é para ter tempo
de se recuperar do cansaço do domingo. Parece uma boa ideia\ldots{} Ela anda
subindo de novo naquela árvore. Joguei torrões nela até ela descer.
Ela disse que não tinha ninguém olhando. Parece achar que isso é uma justificativa boa
o bastante para ficar se arriscando desse jeito. Eu disse isso a ela. A palavra
``justificativa'' elevou sua admiração por mim  ainda mais, e a inveja também,
pensei. É boa essa palavra.

\dia{Terça-feira}   Ela disse que foi feita a partir de uma costela do meu corpo. Isso
é no mínimo duvidoso, se não coisa pior. Não estou sentindo falta de costela
nenhuma\ldots{} Ela está bem aflita por causa daquele abutre; diz que grama não lhe
cai bem; está com medo que não poder criá-lo porque acha que é da
natureza dele alimentar-se de carne estragada. O abutre tem que se virar da
melhor forma possível com o que tem por aqui. Não podemos subverter a ordem
das coisas só para o bem dele.

\dia{Sábado}  Ontem ela caiu no lago enquanto se admirava. Coisa que ela faz o
tempo todo. Quase se afogou e disse que a sensação não foi nada boa. Isso
fez que começasse a ter pena das criaturas que vivem lá no fundo, que 
decidiu chamar de peixes. Aliás, ela continua pondo nomes nas coisas, inclusive
nas que não precisam de nome, até porque elas nem atendem quando são chamadas.
Mas ela não se importa, é uma tonta mesmo; e assim acabou pegando um monte
desses peixes ontem à noite e os enfiou na cama para ficarem quentinhos. Fiquei
observando-os o dia inteiro, e acho que não estão mais felizes aqui do que
estavam antes, só mais quietos. À noite vou botar todos para fora. Não vou
dormir com eles de novo, são pegajosos, e é nojento ficar
deitado sem roupa ao lado deles.

\dia{Domingo}  Aguentei firme.

\dia{Terça-feira}   Agora ela anda às voltas com uma cobra. Os outros animais estão
felizes, pois ela estava sempre incomodando e fazendo experiências com eles, e
eu fico feliz porque a cobra fala, e isso me dá uma folga.

\dia{Sexta-feira}  Ela disse que foi a cobra quem a instigou a comer da fruta daquela
árvore alegando que o resultado seria uma grande, bela e nobre
aprendizagem. Eu disse que isso teria outras consequências também\ldots{} introduziria
a morte no mundo. Que burrada, teria sido melhor guardar esse comentário só
para mim; acabei dando uma ideia a ela\ldots{} achou que poderia salvar o abutre doente
e também fornecer carne fresca para os leões e os tigres desesperados. Mandei-a
ficar longe daquela árvore. Ela disse que não. Vejo problemas à frente. Vou cair fora outra vez.

\dia{Quarta-feira}   Foi bem  movimentada. Escapei ontem e cavalguei a noite
toda, o mais rápido que meu cavalo pôde, na esperança de conseguir ficar
bem longe do parque e me esconder em outra região, antes que começassem os
problemas; mas não era para ser. Mais ou menos uma hora depois de o sol nascer,
quando eu cavalgava por uma relva florida onde milhares de animais pastavam, dormiam, 
ou brincavam uns com os outros, de acordo com seu desejo,
de repente eles irromperam em grunhidos e a planície foi tomada por uma comoção
frenética, e os animais começaram a atacar uns aos outros. Eu sabia o que era ---
Eva tinha comido da fruta, e a morte adentrava nosso mundo\ldots{} Os tigres
comeram meu cavalo, me ignoraram quando mandei que parassem, e teriam me
devorado também se eu tivesse ficado --- o que não fiz, pois saí correndo\ldots{}
Encontrei esse lugar fora do parque. Foi até confortável por alguns dias, mas
ela me descobriu. Me encontrou e logo chamou o lugar de Tonawanda --- porque,
para variar, achava que \textit{parecia} com Tonawanda. Para falar a verdade, não fiquei
chateado por ela ter aparecido porque aqui tem pouca coisa para colher, e ela
trouxe algumas maçãs. Fui obrigado a comê-las, estava morto de fome. Era
contra os meus princípios, mas quando a gente está com fome os princípios não
contam muito\ldots{} Ela veio coberta de galhos e ramos de folhas, e quando
perguntei o que pretendia com isso e arranquei tudo e joguei fora, ela
ficou meio sem graça e corou. Nunca tinha visto uma pessoa ficar sem
graça e corar, e aquilo me pareceu inadequado e idiota. Ela disse que logo eu
saberia por quê. Ela tinha razão. Faminto como eu estava, larguei a metade da
maçã que estava comendo --- certamente a melhor que eu provara,
considerando que já findava a estação --- e me cobri com os galhos e ramos 
jogados fora, e lhe falei com autoridade, mandei que pegasse mais
alguns e que não fizesse um espetáculo. Ela obedeceu, e depois
disso rastejamos até o lugar onde a guerra dos animais selvagens tinha
acontecido e recolhemos algumas peles, com as quais mandei que ela fizesse roupas
mais apropriadas para ocasiões públicas. São um pouco desconfortáveis, é
verdade, mas têm estilo, e isso é que importa quando se trata de roupas\ldots{} Acho
que ela é uma bela de uma companheira. Percebi que ficaria um tanto
solitário e deprimido sem ela agora que perdi minha propriedade. Outra coisa:
ela disse que nos mandaram trabalhar para o nosso sustento de agora em diante.
Ela vai ser útil. Eu vou supervisionar.

\dia{Dez dias depois}  Ela me acusa de ser a causa do nosso desastre! Diz, com
aparente sinceridade e verdade, que a Serpente garantiu que a fruta
proibida não era maçã e sim castanha. Eu disse que então era inocente, 
pois não tinha comido nenhuma castanha. Ela disse que a Serpente lhe
informara que “castanha” era um termo figurado que significava piada
velha e bolorenta. Fiquei sem graça, pois fiz muitas piadas para passar o
tempo, apesar de achar, honestamente, que eram novas quando pensei nelas. Ela
me perguntou se tinha feito alguma bem no momento da catástrofe. Fui obrigado a
admitir que fiz uma para mim mesmo, mas não em voz alta. Era assim:
estava pensando sobre as Quedas e disse: “Que maravilha é ver um
imenso corpo d'água caindo!”. Então, em um instante, num
pensamento brilhante que me passou pela cabeça, eu o soltei, dizendo: “Seria
muito melhor vê-lo subindo até aqui!”\ldots{} e eu estava quase morrendo de tanto rir
quando a natureza inteira irrompeu em guerra e morte e tive que fugir para
não arriscar minha vida. ``Isso”, disse ela triunfante, “é bem isso; a Serpente
mencionou exatamente essa piada, e chamou-a de `A Primeira Castanha', e disse que
coincidia com a criação.'' Ai de mim, de fato a culpa é toda minha. Quem dera não
fosse tão espirituoso; ah, se não tivesse tido um pensamento tão brilhante!

\dia{No ano seguinte}  Chamamos essa criaturinha de Caim. Eva a pegou enquanto eu estava
fora, armando redes na costa norte do lago Erie; apanhou-a na floresta a uns
três, quatro quilômetros da nossa caverna\ldots{} pode também ter sido a uns seis, sete
quilômetros, ela não sabe bem ao certo. Ele se parece um pouco com a gente, e
pode até ser um parente. Isso é o que Eva acha, mas penso que está enganada. A
diferença em tamanho reforça a conclusão de que se trata de uma nova espécie de
animal\ldots{} um peixe, talvez, apesar de que, quando o joguei na água para ter
certeza, ele afundou, e ela se atirou na água e o pegou antes que desse tempo
de ver se meu experimento confirmava minha suspeita. Continuo achando
que é um peixe, mas ela não se importa com o que seja, e não me deixa pegá-lo
para fazer novas tentativas. Não entendo isso. Com a chegada dessa nova
criatura, parece que a natureza dela mudou completamente, ela ficou irracional
com relação aos experimentos. Pensa nele mais do que em qualquer outro
animal, mas não sabe dizer por quê. Sua mente está confusa, desordenada; tudo
confirma isso. Às vezes, quando esse peixe choraminga e quer ir para a água,
ela fica com ele nos braços quase a noite toda. Nesses momentos, a água vem
de dentro daqueles buracos por onde ela olha, e ela faz carinho nas
costas dele e uns sons bem suaves com a boca para acalmá-lo. Tenta
demonstrar de mil maneiras que o entende e que tem pena dele. Nunca a vi
fazer isso com nenhum outro peixe, e isso me incomoda muito. Ela
costumava carregar os filhotes de tigre do mesmo modo, e brincava com eles,
antes de perdermos a propriedade, mas era só brincadeira; não se preocupava
tanto com eles quando a comida não lhes caía bem. 

\dia{Domingo}  Ela não trabalha aos domingos, fica por aí deitada, fatigada, e
gosta de ver o peixe revolvendo-se por cima dela de um lado para outro. Emite 
uns sons abobalhados para entretê-lo, faz de conta que está
mordiscando suas patinhas para ele rir. Eu nunca tinha visto um peixe que
conseguisse rir. Isso me faz duvidar\ldots{} Comecei a gostar dos domingos também.
Ficar supervisionando a semana inteira deixa o corpo cansado. Deveria ter mais
domingos. Antigamente eles eram quase insuportáveis, mas agora eles vêm a calhar.

\dia{Quarta-feira}   Esse ser não é um peixe. Ainda não descobri exatamente o que é.
Faz uns sons horrorosos quando não está satisfeito, mas quando está, diz
“gugu, dadá”. Não é um dos nossos, já que não sabe caminhar; não é
um pássaro, não sabe voar; não é um sapo, porque não sabe saltar; não é
uma cobra, pois não rasteja. Tenho certeza de que não é um peixe, mas não
encontro chance de descobrir se consegue ou não nadar. Passa o tempo todo
deitado, em geral de costas, de pernas para o ar. Nunca vi
outro animal fazer isso. Eu disse que achava isso um enigma; mas ela
apenas admirou a palavra, sem entender. A meu ver, ou é um enigma, ou é um
tipo de inseto. Se ele morrer, vou desmembrá-lo para ver como é que ele
funciona. Nunca uma coisa me deixou tão perplexo.

\dia{Três meses depois} A perplexidade aumenta em vez de diminuir. Durmo bem
pouco. Ele parou de ficar só deitado, agora também fica esperneando. Mas 
difere dos outros animais de quatro patas porque as da frente são bem mais
curtas, e por causa disso a parte principal do corpo dele fica apontando
desconfortavelmente para cima, e isso não é nada atraente. Foi construído
como nós, mas a sua maneira de se movimentar mostra que não é da nossa espécie.
As pernas da frente curtas e as traseiras compridas indicam que deve ser da família
dos cangurus, mas de uma espécie diferente, pois os verdadeiros cangurus pulam
e esse não. Mesmo assim é de uma variedade bem curiosa e interessante, que ainda
não foi catalogada. Como eu a descobri, me sinto no direito de assegurar que o
crédito da descoberta seja associado ao meu nome. Assim, chamei-o de
\textit{Kangaroorum adamiensis}\ldots{} Deve ter sido jovem quando veio, pois nesse 
meio-tempo cresceu excessivamente. Está umas cinco vezes maior do que quando chegou,
e quando está descontente é capaz de produzir de vinte e duas a trinta e cinco
vezes mais ruídos do que no início. Coerção não muda isso, na verdade tem o efeito
contrário. Por essa razão, desisti de analisar seu mecanismo. Ela o acalma por
meio da persuasão e dando-lhe coisas que antes tinha me proibido de dar. Como
falei, eu não estava em casa quando ele apareceu, mas ela jurou que o encontrou
na floresta. Parece estranho que seja o único, mas deve ser isso mesmo, pois 
cansei de procurar nessas últimas semanas tentando encontrar outro para juntar à
minha coleção e para que ele tenha com quem brincar. Isso certamente o
faria acalmar-se e conseguiríamos domesticá-lo mais facilmente. Mas não encontro
vestígio de outro igual, e o mais estranho é que não há pegadas. Ele
deve viver na superfície, não consegue se cuidar sozinho; mas como
se movimenta sem deixar rastro? Armei dezenas de armadilhas, que não
deram em nada. Peguei tudo que é tipo de animal pequeno, exceto esse; animais
que entram na armadilha por pura curiosidade, para ver por que o leite está lá,
sem tomar nem um gole sequer. 

\dia{Três meses depois}  O canguru continua a crescer, o que é muito estranho e
intrigante. Nunca imaginei que a fase de crescimento dele poderia ser tão longa.
Agora está com pelos na cabeça. É diferente do pelo dos cangurus e exatamente
igual ao nosso cabelo, que é muito mais fino e macio, e em vez de ser preto é
vermelho. Estou pasmo com o desenvolvimento imprevisível dessa
aberração zoológica inclassificável. Ah se eu conseguisse pegar mais um, mas
não tenho esperanças; é uma nova variedade e a única amostra; é isso. Mas
consegui agarrar um legítimo canguru e o trouxe para cá pensando que este,
estando sozinho, preferiria ter aquele como companhia a não ter ninguém de sua
espécie. Ou pelo menos outro animal de que poderia se sentir próximo e que se
compadecesse dele nessa condição de abandono, entre estranhos, que não conhecem
seus jeitos e hábitos e não sabem o que fazer para que se sinta entre amigos.
Mas foi um erro. Ele teve um ataque quando viu o canguru, tanto que me
convenci de que ele realmente nunca tinha visto um. Tenho pena desse pobre e
ruidoso animalzinho, mas não há nada que possa fazer para deixá-lo feliz. Ah se
eu conseguisse domesticá-lo\ldots{} mas isso está fora de questão; quanto mais tento,
pior ele fica. Me parte o coração vê-lo assim, em seus tormentos de tristeza e
paixão. Quero soltá-lo, só que ela não quer nem ouvir falar disso. Parece cruel e
atípico, mas ela pode estar certa. Ele pode estar mais solitário do
que nunca, pois se eu não consigo encontrar nenhum dos seus semelhantes, que dirá ele?

\dia{Cinco meses depois}   Não é um canguru. Não, porque se firma segurando
no dedo dela  e assim consegue dar alguns passos com as pernas traseiras, antes
de cair. Deve ser um tipo de urso; apesar de não ter uma cauda\ldots{} pelo menos não
ainda\ldots{} nem pelo, exceto na cabeça. Ele continua crescendo. E isso é muito curioso, 
pois ursos atingem a fase adulta antes.
Ursos são perigosos desde a catástrofe, e eu não vou ficar satisfeito com
ele fuçando por aqui sem uma focinheira. Eu lhe ofereci um canguru se ela
deixasse ele ir, mas não adiantou. Eva parece disposta a nos colocar em
tudo quanto que é situação de risco. Ela não era assim antes de perder o juízo.

\dia{Duas semanas depois}   Examinei sua boca. Ainda não apresenta perigo: tem apenas
um dente. E ele ainda não tem cauda. Faz mais barulho agora do que antes,
principalmente à noite. Eu me mudei dali. Mas voltarei sempre pelas manhãs, para o
café e para verificar se já tem mais dentes. Se sua boca ficar
cheia de dentes, será a hora de ele ir, com ou sem cauda, pois um urso não
precisa de uma para se tornar perigoso.

\dia{Quatro meses depois}  Me ausentei por um mês para caçar e pescar lá naquela
região que ela chama de Buffalo; não sei bem por quê, a não ser que seja por
não ter búfalos por lá. Nesse meio-tempo, o urso aprendeu a andar por aí
sozinho, somente nas pernas traseiras, e diz “papá” e “mamá”. Certamente
trata-se de uma nova espécie. Essa semelhança no uso de palavras pode ser
puramente acidental, é claro, e pode não ter propósito ou significado nenhum;
mas mesmo nesse caso é extraordinário, pois os ursos não sabem
fazer isso. Essa imitação da fala, a falta geral de pelos e a completa
ausência de cauda são indicação suficiente de que deve ser um novo tipo de urso.
Um estudo mais aprofundado desse animal haveria de ser bem interessante. Enquanto
isso, partirei numa expedição distante para as florestas do norte a fim de fazer
uma pesquisa exaustiva. Deve haver outro em algum lugar, e esse talvez vá se
tornar menos perigoso se estiver na companhia de outros de sua espécie. Vou
sair imediatamente, mas não sem antes colocar uma focinheira nele.

\dia{Três meses depois}  A caçada foi muito, muito cansativa e sem sucesso.
Enquanto estive fora, sem sair da propriedade, ela conseguiu pegar outro! Eu
nunca tive essa sorte. Mesmo que tivesse caçado por essas florestas
durante uns cem anos, nunca toparia com outro desses.

\dia{No dia seguinte}   Comparei o antigo com o mais novo, e  ficou claro que pertencem
à mesma espécie. Queria empalhar um para minha coleção, mas ela é contra por uma razão ou por outra, 
então acabei desistindo da ideia, apesar de achar
que é um erro. Seria uma perda irreparável para a ciência se eles
desaparecessem. O mais velho está mais domesticado do que antes e ri e fala
como um papagaio, tendo aprendido isso, é claro, por ter passado tanto tempo na
companhia de um e por ter a faculdade imitativa bem desenvolvida. Ficarei muito
surpreso se for comprovado que é um novo tipo de papagaio; por outro lado, não
deveria ficar tão surpreso assim, pois ele já foi de tudo quanto era espécie desde os
primeiros dias, quando ainda parecia ser um peixe. O mais novo é tão feio
quanto o primeiro era no começo; a mesma cor de enxofre e carne crua e aquela
cabeça com formato meio estranho, sem nenhum pelo. Ela o chama de Abel.

\dia{Dez anos depois}   Eles são meninos; descobrimos há muito tempo. Era a
forma como eles chegavam, tão pequenos e imaturos, que nos baratinava. Também já
temos umas meninas. Abel é um bom menino, mas se Caim tivesse permanecido um
urso teria se aperfeiçoado mais. Depois de todos esses anos, me dei conta de que
no começo estava errado sobre Eva; é melhor viver fora do Jardim com ela,
do que nele sem ela. No começo achava que ela falava demais; mas hoje ficaria
mal se não falasse ou se saísse da minha vida. Abençoada seja a castanha que
nos aproximou e que me ensinou a ver a bondade do seu coração e a doçura do seu
espírito!

\begin{center}
\textsc{fim}
\end{center}




\chapter{Diário de Eva}
\hedramarkboth{diário de eva}{mark twain}
\medskip

{\small
\hfill\textit{Traduzido do original}
}
\\

\dia{Sábado} Estou com quase um dia de vida agora. Cheguei ontem. Ao menos é o que
parece. E deve ser isso mesmo, pois se houve um dia antes do de ontem
eu não estava lá para ver, ou me lembraria dele. Mas pode ser, claro, que tenha
havido um dia antes de ontem e eu simplesmente não tenha notado. Muito
bem; vou ficar de olho, e se algum dia antes de ontem acontecer, vou tomar
nota. É melhor começar direito e não deixar os registros ficarem confusos,
meu instinto me diz que esses detalhes ainda vão ser importantes para os
historiadores um dia. Me sinto como um experimento, exatamente como um
experimento; seria impossível alguém se sentir mais como um experimento do que
eu. Por isso começo a ficar convencida de que é exatamente isso que \textit{eu sou} --- um
experimento e nada mais.

Se eu for mesmo um experimento, será que sou o único? Não, acho que não. Acho que o
resto faz parte também. Sou a principal, mas acho que o resto também tem a ver
com o assunto. Será que a minha posição está assegurada, ou devo ficar de olho e
tomar cuidado? Prefiro ficar de olho. O instinto me diz que a vigilância
eterna é o preço da supremacia. (Acho essa é uma boa frase para alguém tão
jovem como eu.)

Tudo está mais bonito hoje do que ontem. Na pressa de terminar ontem, as
montanhas ficaram um pouco picotadas, e algumas planícies ficaram tão
abarrotadas de lixo e sobras que o aspecto ficou meio desolador. Peças de arte
nobres e belas não devem ficar sujeitas à pressa; e este majestoso novo mundo é
mesmo um nobre e belo trabalho. E certamente maravilhoso, chegando quase à
perfeição, apesar de ter sido feito em tão pouco tempo. Há estrelas demais em algumas partes
e de menos em outras, mas isso pode ser remediado logo, sem dúvida. A lua se
soltou ontem, e caiu fora do esquema --- uma grande perda; me corta o
coração pensar nisso. Não há nada entre os ornamentos e decorações
comparável a sua beleza e acabamento. Devia ter sido mais bem fixada. Ah, se
conseguíssemos pegá-la de volta.

Mas não dá para saber onde foi parar. Além do mais, quem a pegar vai
escondê-la; sei disso porque faria o mesmo. Acho que poderia ser honesta em
relação a tudo o mais, mas já descobri que a essência e o centro da minha
natureza é o amor pelo belo, uma paixão pelo belo, e que não seria seguro
confiar em mim em posse de uma lua que pertencesse a outra pessoa que não
soubesse que ela estava comigo. Eu poderia abrir mão de uma lua que eu achasse
durante o dia, pois teria medo de que alguém estivesse vendo; mas, se eu a
achasse na escuridão, tenho certeza de que inventaria uma desculpa para não ter
que dizer nada a respeito. Porque eu adoro as luas, elas são tão lindas,
tão românticas. Gostaria que tivéssemos cinco ou seis delas; eu nunca mais
iria dormir; não me cansaria nunca de ficar deitada sobre os musgos, olhando para elas.

As estrelas são bacanas também. Gostaria de ter algumas para pôr no cabelo. Mas
acho que nunca vou conseguir. Você ficaria surpreso de saber como elas estão longe daqui,
apesar de não parecerem tão distantes. Quando elas apareceram
pela primeira vez, ontem à noite, tentei derrubar algumas usando um bambu, mas
não era comprido o bastante, o que me deixou surpresa; então tentei atirar uns torrões
nelas, até cansar, mas não consegui derrubar nenhuma. Deve ser porque sou
canhota e não arremesso bem. Mesmo quando mirei numa outra, para ver se
acertava finalmente a que queria, não consegui. Vi a mancha preta do torrão
passar bem no meio daquele conglomerado dourado; fiz isso umas quarenta ou
cinquenta vezes, mas passava de raspão; se tivesse aguentado um
pouco mais, talvez conseguisse acertar uma em cheio.

Por isso chorei um pouco, o que é natural, suponho, para alguém da minha idade,
e depois de descansar acabei pegando uma cesta e fui caminhando em
direção à borda do círculo, lá onde as estrelas ficam perto do solo,
achei que poderia pegar com as mãos. Isso seria melhor, daria
para reuni-las com cuidado, de modo que não quebrassem. Mas era mais longe do que eu pensava, e
acabei desistindo; estava tão cansada que não conseguia dar mais nem um passo.
Além disso, tinha os pés muito inchados e doloridos.

Não pude voltar para casa; era muito longe, estava esfriando, mas achei uns
tigres e me aninhei entre eles; foi adoravelmente confortável, seu
hálito era doce e agradável, porque só comem moranguinhos. Nunca tinha visto um
tigre antes, mas conheci logo pelas listras. Se conseguisse uma
dessas peles, faria um belo vestido para mim.

Hoje já tenho uma noção melhor das distâncias. Andava tão louca para pegar
tudo que achava bonito, tentava alcançar, ávida, mas quando algumas
dessas coisas estavam fora do meu alcance ou, ao contrário, quando pareciam estar só a
uns dois metros de distância --- mas entre espinhos ---, foi aí que
aprendi uma lição. Fiz até um axioma, de cabeça --- o meu primeiro: \textit{o experimento
arranhado evita o espinho}. Acho que está muito bom para alguém da minha
idade.

Segui o outro experimento por aí, ontem à tarde, a distância, para tentar entender para que
ele tinha sido feito. Mas não consegui descobrir. Acho que é um
homem. Nunca tinha visto um, mas parecia ser, e tenho quase certeza de que
era mesmo. Sinto que tenho mais curiosidade com relação a ele do que
com relação aos outros répteis. Suponho que seja um, pois tem cabelo desgrenhado e
olhos azuis e se parece com um réptil. Não tem ancas, se afina como uma
cenoura, se abre como um guindaste; acho que é um réptil, mas
pode também ser uma obra arquitetônica.

No começo tinha medo dele, e corria cada vez que ele se virava, com medo de
que me perseguisse; aos poucos fui notando que a criatura só queria
fugir; depois, quando minha timidez diminuiu, eu o segui por horas e
horas, sempre uns vinte metros atrás, o que o deixou nervoso e infeliz.
A certa altura, ele estava bastante preocupado e subiu numa árvore. Esperei
um bom tempo, mas acabei desistindo e fui para casa.

Hoje, a mesma coisa. Por minha causa, subiu outra vez na árvore.

\dia{Domingo} Esse ser ainda está lá em cima. Descansando, parece. Mas isso é
um pretexto: domingo não é dia de descanso. O sábado é que foi escolhido
para isso. Ele demonstra estar mais interessado em
descansar do que em qualquer outra coisa. Eu ficaria cansada de tanto
descansar. Me canso só de ficar sentada olhando para a árvore. Me pergunto para
que afinal ele serve. Nunca o vejo fazendo coisa nenhuma.

Eles devolveram a lua ontem à noite, fiquei \textit{tão} contente! Acho que foi
muito honesto da parte deles. Ela escorregou e caiu de novo, mas não me
incomodei; não é preciso se preocupar quando se tem vizinhos
assim; eles vão pegar de volta. Gostaria de poder fazer alguma coisa para
expressar minha gratidão. Gostaria de mandar algumas estrelas para eles,
temos muito mais do que precisamos. Quer dizer, eu, não nós, pois dá para perceber que
aquele réptil não se interessa por nada disso.

Ele não tem gostos refinados, nem mesmo é gentil. Quando fui lá ontem à noite,
na hora do crepúsculo, ele tinha descido e estava tentando pegar aqueles
peixinhos manchados que brincam no lago; tive que afugentá-lo com torrões para
que subisse na árvore outra vez e os deixasse em paz. Fico pensando se é para \textit{isso} que
ele serve. Será que não tem coração? Não tem compaixão por essas pequenas
criaturas? Será que foi projetado para um trabalho tão desumano? Parece que
sim. Um dos torrões o atingiu atrás da orelha, e ele fez uso da linguagem.
Fiquei emocionada, foi a primeira vez que ouvi alguém falar, além de mim.
Não entendi as palavras, mas pareciam bem expressivas.

Quando descobri que sabia falar, fiquei mais interessada nele, pois eu adoro
falar; falo o dia inteiro, até quando estou dormindo. Sou muito interessante,
mas se tivesse alguém com quem conversar poderia ser duplamente interessante,
e não pararia nunca se ele também quisesse conversar.

Se esse réptil for mesmo um homem, não poderá mais ser chamado de coisa,
isso não seria gramatical, seria? Isso seria um \textit{ele}. Pelo menos é o que eu acho.
Nesse caso, teríamos que chamá-lo assim: no nominativo, \textit{ele}; no dativo, \textit{lhe}; no
possessivo, \textit{seu}. Bom, vou considerá-lo um homem e chamá-lo de ele até descobrir
o que é de verdade. É mais prático do que ficar em dúvida.

\dia{Domingo da semana seguinte} Passei a semana na cola dele, para nos conhecermos
melhor. Eu é que tinha que ficar puxando conversa, porque ele era tímido, mas não
me importei. Ele parecia feliz por eu estar próxima, e usei bastante o sociável
“nós”, porque ele parecia gostar de se sentir incluído.

\dia{Quarta-feira} Agora estamos nos acertando muito bem mesmo, e nos conhecendo
cada vez mais. Ele não me evita mais como antes, o que é um bom sinal;
demonstra que gosta de me ter por perto. Isso me agrada, eu me esforço para
ser útil no que puder, de modo que seu apreço por mim aumente. Durante os últimos dois
dias assumi a tarefa de nomear as coisas, o que lhe trouxe muito
alívio, porque ele não leva muito jeito para isso. Evidentemente está muito
grato. Não consegue pensar em nenhum nome racional para desmentir essa
impressão de que lhe falta talento. Mas não deixo que perceba que estou ciente
desse seu defeito. Sempre que aparece uma nova criatura eu já dou um nome,
antes que ele tenha tempo de se expor com aquele silêncio desconcertante. Assim,
já o poupei de muitos embaraços. Eu não tenho um defeito como esse.
No momento em que ponho os olhos num animal, já sei o que é. Não preciso
refletir nem por um segundo; o nome certo vem instantaneamente, como
uma inspiração, e é isso que deve ser, uma inspiração, pois tenho certeza de que
não estava dentro de mim nem meio minuto antes. Parece que só pelo formato da
criatura e pelo jeito como ela age sei que animal é.

Quando o dodô apareceu, ele pensou que fosse um gato selvagem --- eu vi nos seus
olhos. Mas eu o poupei. Fui cuidadosa em fazer isso de maneira que não
ferisse seu orgulho. Simplesmente falei de um jeito bem natural, como uma
surpresa agradável, e não como se estivesse querendo lhe passar uma informação:
“Ora, veja se não é um dodô!”. Então expliquei --- sem que parecesse uma
explicação --- como eu sabia que era um dodô, e, mesmo percebendo que ele estava um
pouco incomodado por eu saber que criatura era aquela, ficou evidente que ele
me admirava. Que sensação prazeroza! E fiquei me lembrando disso, antes de dormir, 
com enorme satisfação. Como algo tão insignificante pode
nos deixar tão felizes, especialmente quando nos sentimos merecedores.

\dia{Quinta-feira} Minha primeira mágoa. Ontem ele me evitou e pareceu querer
que eu não falasse mais com ele. Não pude acreditar, pensei que havia
algum engano, eu adorava ficar com ele, e adorava ouvi-lo falar, por
isso não conseguia entender por que tanta antipatia em relação a mim
quando eu não havia feito nada. Mas essa sensação acabou se confirmando,
então decidi ir embora e me sentei sozinha naquele lugar de onde o avistei
pela primeira vez naquela manhã em que fomos feitos e eu não sabia ainda o que
ele era e fiquei indiferente; mas agora esse lugar estava carregado de tristeza, 
e tudo ali me fazia lembrar dele; meu coração estava sofrendo. Eu não sabia muito
bem por quê, pois era um sentimento novo, que eu jamais havia sentido,
era um mistério que eu não conseguia decifrar.

Quando a noite chegou, não suportei mais a solidão e fui ao novo abrigo
que ele construiu para perguntar o que eu tinha feito de errado e como repará-lo 
e fazer com que fosse amável comigo de novo. Mas ele me pôs
para fora, na chuva; e foi essa a minha primeira mágoa.

\dia{Domingo} Está tudo bem de novo e estou feliz; mas aqueles foram
dias bem difíceis, e evito pensar neles.

Tentei pegar umas daquelas maçãs para ele, mas minha mira não é muito boa.
Fracassei, mas acho que a boa intenção o agradou. Elas são proibidas, e ele diz
que isso vai me prejudicar, mas se vou me prejudicar tentando agradá-lo, por que 
devo me importar com esse tipo de dano?

\dia{Segunda-feira} Esta manhã eu lhe disse meu nome, na esperança de que ficasse
interessado, mas ele não deu a mínima. É estranho. Se me dissesse seu nome,
eu me interessaria. Acho que o som do nome dele seria mais agradável para meus
ouvidos do que qualquer outro som.

Ele fala bem pouco. Talvez seja porque não é muito inteligente e, 
como é muito sensível com relação a isso, tenta disfarçar. 
É uma pena que se sinta assim, porque inteligência não é nada; 
é no coração que estão os valores. Gostaria de poder fazê-lo entender 
que um coração amoroso é riqueza mais que suficiente, e que sem isso 
o intelecto é pobreza.

Embora fale tão pouco, tem um vocabulário considerável. Esta manhã
usou uma palavra surpreendentemente boa. Ele, claro, se deu conta
disso, pois a usou duas vezes depois, tentando aparentar casualidade. 
Não se saiu muito bem, mas ainda assim demonstrou possuir alguma 
percepção. Sem dúvida essa semente pode germinar, se cultivada.

De onde ele tirou aquela palavra? Não me lembro de tê-la usado alguma vez.

Não, ele não mostrou o menor interesse pelo meu nome. Tentei disfarçar minha
decepção, mas acho que não consegui. Saí e me sentei num montículo de
musgos, com os pés na água. É para lá que eu vou quando necessito de
companhia, quando quero ver alguém, conversar um pouco. Aquele lindo
corpo branco pintado na água não é suficiente --- mas é melhor que nada, é
melhor que a solidão absoluta. Ele fala quando eu falo; fica triste quando eu
fico; me conforta com sua compaixão; diz: “Não fique abatida, pobre
menina sem amigos”. Ela é uma boa amiga para mim,
minha única amiga, ela é minha irmã.

Na vou esquecer nunca, nunca mesmo, da primeira vez que ela me abandonou. Meu
coração estava como chumbo dentro do corpo. “Ela era tudo o que eu tinha, e
foi embora!” Em desespero eu disse: “Você parte meu coração; não
consigo mais suportar esta vida!”, e escondi o rosto entre as mãos,
desconsolada. Quando retirei as mãos do rosto, depois de um tempo, ali estava ela
novamente, branca, brilhante e bonita, e eu me joguei em seus braços!

Senti então a felicidade plena; eu já conhecia a felicidade, mas nunca daquele modo, 
aquele êxtase. Nunca mais duvidei dela depois disso. Às
vezes, ela ficava distante --- talvez uma hora, às vezes o dia inteiro, mas eu
esperava e não duvidava; eu dizia: “Ela deve estar ocupada, ou viajando,
mas voltará”. De fato: ela sempre voltava. À noite, se estivesse
escuro, não voltava, pois era uma criaturinha tímida, mas quando a lua
aparecia ela vinha. Não tenho medo da noite, mas ela é mais jovem do que
eu, nasceu depois de mim. Foram muitas as visitas que lhe fiz; ela é meu
consolo e refúgio quando a vida está difícil --- é isso que ela é.

\dia{Terça-feira} Passei a manhã toda trabalhando para melhorar a propriedade, e
fiquei longe dele de propósito, na esperança de que se sentisse sozinho e
viesse ao meu encontro. E nada.

Ao meio-dia dei minhas atividades por encerradas e me entretive saltitando
com as abelhas e as borboletas e me divertindo entre as flores,
essas maravilhosas criaturas que captam o sorriso de Deus e o preservam! Eu as
colhi e trancei em coroas e guirlandas e me vesti com elas enquanto lanchava
--- maçãs, obviamente; depois me sentei à sombra, ansiosa, esperando. Mas ele não
apareceu.

Não faz mal. Não teria dado em nada mesmo, pois ele não gosta de
flores. Ele as chama de lixo, e não consegue distinguir umas das outras, acha
que por ser assim é superior. Não gosta de mim, não gosta de flores, não
gosta da pintura do céu ao anoitecer --- será que há algo de que ele
goste afora construir cabanas onde se enfiar para se proteger da chuva boa
e límpida, apalpar melões e selecionar uvas, para ver como
esses frutos estão se desenvolvendo?

Coloquei um galho seco no chão e tentei fazer um buraco com a ajuda de outro
galho, para testar uma ideia. De repente levei um susto. Uma película
transparente e azulada surgiu do buraco; larguei tudo e corri! Achei que fosse
um espírito, fiquei muito assustada! Olhei para trás, ele não estava me
seguindo; então me escorei numa pedra e descansei, ofegante, e esperei até 
minhas pernas pararem de tremer, aí voltei com cuidado, alerta, olhando tudo,
pronta para correr se fosse o caso; quando cheguei perto, afastei uns galhos
de roseiras e espiei --- desejando que o homem estivesse por ali, pois eu estava
tão linda e atraente ---, mas o espírito tinha desaparecido. Fui até lá;
havia um punhado de um pó delicado, cor-de-rosa, no buraco. Coloquei o dedo nele
para sentir a textura, gritei \textit{ai}! e retirei o dedo. A dor foi cruel.
Pus o dedo na boca e, pisando primeiro num pé, depois noutro, gemendo,
eu logo aliviei a dor; então, cheia de interesse, comecei a examinar
aquilo de novo.

Estava curiosa para descobrir o que era esse pó rosa. De repente o nome me
ocorreu, embora nunca o tivesse ouvido. Era \textit{fogo}! Impossível alguém
estar mais certo sobre qualquer coisa neste mundo do que eu estava naquele momento.
Então, sem hesitar, chamei aquilo de fogo.

Tinha criado algo que não existia antes; acrescentei algo novo aos inúmeros
bens deste mundo; me dei conta disso e fiquei orgulhosa da minha invenção,
queria correr e encontrá-lo para falar dela, talvez subisse em seu conceito
--- mas refleti e acabei desistindo. Não, ele não se interessaria. Ele perguntaria
para que serve, e o que eu poderia responder, já que não era \textit{bom}
para nada, era apenas belo, meramente belo.

E assim suspirei e não fui. Pois não servia para nada; não servia para construir
uma cabana, não ajudaria no desenvolvimento dos melões, não apressaria a
colheita dos frutos; não tinha utilidade, era pura bobagem e vaidade; ele
desprezaria minha invenção e diria palavras rudes. Mas para mim não era algo
desprezível. Eu disse: “Oh, fogo, eu te amo, delicada criatura rosa, por
ser \textit{belo} --- e basta!”, e quase o acolhi em meu peito. Mas me contive. E então
me veio à cabeça outra máxima. Mas, como era tão parecida com a
primeira, tive medo de que não passasse de plágio: “O Experimento queimado evita o
fogo”.

Voltei à labuta, e quando obtive bastante pó de fogo outra
vez, despejei sobre minha mão coberta com grama seca, na intenção de assim
poder carregá-lo até em casa, para tê-lo sempre ao meu lado e poder brincar com ele; 
mas uma brisa atiçou o fogo, espalhando-o em minha direção; larguei tudo e corri.
Quando me voltei, o espírito azul estava se elevando, se esticando e se
espalhando como uma nuvem, e na hora pensei em um nome para ele ---
\textit{fumaça}! ---, e, palavra de honra, nunca tinha ouvido falar em fumaça.

Logo labaredas amarelas e vermelhas brilhantes ascendiam por entre a fumaça, e
imediatamente eu as batizei de \textit{chamas}, e eu estava certa de novo, apesar
de serem as primeiras que surgiram no mundo. Elas subiram pelas árvores e
flamejaram esplendidamente para dentro e para fora daquele vasto e crescente 
volume de fumaça; e só me restava bater palmas, rir e dançar em meu êxtase; era tudo
tão novo e estranho, tão maravilhoso e belo!

Ele veio correndo, parou e olhou, mas não disse uma palavra por vários minutos. Depois
perguntou o que era. Ah, foi lamentável ele ter perguntado de modo
tão direto. Tive de responder, claro, e respondi. Disse a ele que
era fogo. Se ele se incomodou por eu saber e ele ter de perguntar, não foi
culpa minha; eu não tinha a menor intenção de aborrecê-lo. Depois de uma pausa
ele perguntou:

“Como foi que apareceu?”

Outra pergunta direta, mais uma resposta direta.

“Fui eu que inventei.”

O fogo se espalhava cada vez mais. Ele foi até a beira do lugar
que estava em chamas e, olhando para baixo, disse:

“E o que é isso?”

“São pedaços de carvão.”

Pegou um para examinar, mas mudou de ideia e colocou de volta no chão.
Foi embora. Não tem interesse por \textit{nada} mesmo.

Eu, porém, estava interessada. Havia cinzas --- de uma cor meio cinza e suave,
delicadas e bonitas ---, e de imediato eu sabia o que eram. E as brasas; eu
conhecia as brasas também. Achei minhas maçãs e ajuntei-as, estava contente,
porque ainda sou muito jovem e meu apetite é grande. Mas fiquei frustrada, elas
tinham estourado e estavam podres. Aparentemente podres; mas na verdade não,
estavam até melhores do que as cruas. O fogo é lindo; e acho que algum dia ainda
vai ser útil.

\dia{Sexta-feira} Eu o vi novamente, por alguns instantes; na segunda-feira passada, ao
anoitecer, também, mas só por um momento. Esperava que ele me elogiasse por
tentar melhorar a propriedade, minhas intenções eram as melhores e eu
me esforcei bastante. Mas ele não estava satisfeito, deu meia-volta e foi embora.
Ele também estava chateado por outra razão; tentei mais uma vez convencê-lo
a não atravessar mais as cataratas. Isso porque o fogo me revelou uma nova
paixão --- bem nova mesmo, e marcadamente diferente do amor, da tristeza, e
desses outros sentimentos que eu já conhecia --- o \textit{medo}. Era horrível! ---
queria nunca ter descoberto, me traz momentos obscuros, estraga a minha
felicidade, me dá arrepios, tremores. Mas não consegui dissuadi-lo,
ele ainda não descobrira o medo, e assim não tinha como me entender.

\begin{center}
\leafNE
\end{center}


\section{Fragmento do diário de Adão}

{\itshape
Preciso me lembrar que ela é muito jovem, apenas uma menina, e tenho de fazer
concessões. Ela é puro interesse, curiosidade, vivacidade. O mundo para ela é
um charme, um milagre, um mistério, uma alegria. Ela nem consegue falar de
pura alegria quando encontra uma nova flor, tem de passar a mão,
acariciar, cheirar e falar com ela, e despejar nomes carinhosos sobre todas elas. Ela é
louca por cores; pedras marrons, areia amarela, musgos cinzentos, folhagem
verde, céu azul, o perolado da aurora, a sombra púrpura nas montanhas, as
douradas ilhas flutuantes em mares escarlate do poente, a pálida lua
velejando através das camadas de nuvens, as joias estelares
brilhando na imensidão do espaço --- nenhuma delas com qualquer valor prático,
até onde eu pude perceber, mas sim porque têm cores e são majestosas, é o bastante para ela, 
ela é louca por cores. Se conseguisse se acalmar e ficar
quieta por pelo menos dois minutos, seria um espetáculo
repousante. Nesse caso, acho que poderia olhar para ela e apreciá-la, para ser franco, acho
que poderia mesmo, estou começando a me dar conta de que ela é criatura de uma graciosidade fora do comum
de criatura --- flexível, esbelta, curvilínea, bem torneada, ágil e, certa vez,
quando estava parada feito estátua de mármore branco e banhada de sol sobre
um rochedo, com a cabeça inclinada para trás e as mãos a proteger-lhe
os olhos, olhando o voo de um pássaro no céu, percebi como ela era linda.
}\par

\dia{Segunda-feira, meio-dia} {\itshape Se houver alguma coisa no planeta em que ela não
esteja interessada, não consigo lembrar o que possa ser. Há animais aos quais sou
indiferente, mas ela não. Não faz discriminação, ela aceita todos; acha que
todos são tesouros, cada novo animal é bem-vindo.


Quando o gigantesco brontossauro veio caminhando devagar até a
clareira, ela o considerou uma bela aquisição para a propriedade. Eu o considerei
uma calamidade. É um bom exemplo da falta de harmonia em nossa visão
das coisas. Ela acreditava que conseguiria domesticá-lo; eu queria considerá-lo
apenas como um novo bem da propriedade, e me mudar de lá. Ela achava
que poderia amansá-lo com treinamento e fazer dele um animal
doméstico; eu disse que um bichinho de mais de seis metros de altura e um pouco
mais de vinte e cinco metros de comprimento não seria apropriado num
lugar assim; mesmo com a melhor das intenções, e sem querer mal a ninguém,
ele poderia se sentar sobre a casa e esmagá-la, porque só de observar os olhos
dele dá para ver que é meio distraído. Mesmo assim o coração dela estava
inclinado a ficar com aquele monstro, e ela não estava disposta a abrir mão dele. Ela achou
que poderíamos começar um laticínio com ele, queria que eu a ajudasse a tirar
leite; mas eu não ia fazer isso, era muito arriscado. O sexo do animal não parecia ser o
correto, e além do mais não tínhamos uma escada. Então ela quis montá-lo e
apreciar o cenário. Uns dez a doze metros só de cauda arrastando pelo chão,
como uma árvore caída, e ela achava que podia subir nele; estava errada, claro;
quando chegou na parte mais íngreme, que era bem lisa, escorregou, e teria
se machucado não fosse por mim.

Estaria ela satisfeita agora? Não. Nada nunca a satisfaz, exceto provas concludentes;
teorias não testadas estão fora de questão, ela não as aceita. Ela já nasceu imbuída do espírito científico. 
É o certo, reconheço; me atrai, sinto a sua influência, e se ficasse mais com ela
acho que investigaria também. Bom, ela tinha mais uma teoria sobre esse
colosso: achava que se conseguíssemos domesticá-lo e torná-lo dócil
poderíamos colocá-lo no meio do rio e utilizá-lo como ponte. A verdade é que
ele já era bem manso --- pelo menos no que lhe dizia respeito ---, e assim
ela acabou pondo sua teoria em prática, mas fracassou: toda vez que o colocava na posição certa no
rio e ia até a beira para tentar atravessar por cima dele, ele vinha atrás e a
seguia, como se fosse uma montanha de estimação. Como os outros animais. Todos
eles fazem isso.
}

\begin{center}
\leafNE
\end{center}

%\section{Fragmento do diário de Eva}

\dia{Sexta-feira} Terça\ldots{} quarta\ldots{} quinta\ldots{} hoje: todos esses
dias sem vê-lo. É bastante tempo para ficar sozinha; mas é melhor ficar sozinha
do que não ser bem-vinda.

Eu precisava de companhia --- acho que fui feita para isso ---, e fiz 
amizade com os animais. Eles são charmosos, têm boa disposição e
bons modos, nunca estão de cara amarrada, nunca nos fazem sentir como
intrusos, sorriem e balançam a cauda quando têm uma, e estão sempre
prontos para brincar ou fazer um passeio, ou qualquer coisa que a gente
propuser. São verdadeiros cavalheiros. Nos divertimos muito todos esses dias, 
não me senti sozinha nem por um instante. Solitária? Não, não posso
dizer isso. Sempre há um bando deles ao redor --- às vezes, quatro ou
cinco acres cheios ---, não dá para contar; e quando se fica parado numa
rocha, olhando por cima dessa vastidão peluda, tão estampada e alegre e feliz,
com cores e reflexos vivazes e brilhantes de sol, ondulada de listras, dá para
pensar que é um lago, mas sabemos que não é; há tempestades de pássaros
sociáveis, e furacões de asas zunindo, e quando o sol paira sobre aquela
comoção de penas é resplendor de todas as cores imagináveis,
suficiente para cegar uma pessoa.

Fizemos longas excursões, vi uma boa parte da terra, quase tudo;
sou a primeira viajante, e a única. Quando estamos em marcha, o cenário
é grandioso --- não há nada igual em lugar algum. Para o meu conforto, monto
num tigre ou num leopardo, porque são suaves e têm as costas curvadas, e me
encaixo bem ali, e porque são animais bonitos, mas para distâncias longas e paisagens
monto num elefante. Ele me levanta com a tromba. Descer eu consigo
sozinha; quando estamos prontos para acampar, ele se senta, e deslizo por suas costas.

Os animais e pássaros são todos muito simpáticos uns com os outros, não há
disputas. Todos falam, e todos falam comigo, mas deve
ser numa língua estrangeira, pois não consigo entender palavra do
que dizem; mas em geral me entendem quando respondo, particularmente o
cachorro e o elefante. Isso me envergonha. É uma demonstração de que são mais
inteligentes do que eu, portanto meus superiores. Isso me incomoda, quero ser
o Experimento principal --- é isso que pretendo.

Aprendi uma boa quantidade de coisas. Sou instruída, mas no começo não sabia nada.
Era totalmente ignorante. No princípio isso me aborrecia, porque, apesar de
ficar observando por tanto tempo, eu nunca estava por perto quando a água subia de
volta pelas cataratas. Mas agora não me importo mais. Fiz experiências e mais
experiências, e finalmente descobri por que a água não subia. Ela subia,
sim, mas só à noite. Por isso é que o lago nunca seca, o que ocorreria, claro,
se durante a noite a água não subisse. O melhor é sempre testar as coisas por
meio de experimentos; assim é que se aprende; se depender só da
adivinhação, da suposição e da conjectura, você nunca será uma pessoa instruída.

Certas coisas não há como descobrir; mas você nunca saberá que não dá
só pela adivinhação ou suposição; não, você deve ser paciente e continuar
fazendo experiências, até descobrir que não tem como descobrir. Aí é que
está o prazer, isso faz o mundo mais interessante. Se não houvesse nada para
descobrir, seria chato. Mesmo ficar tentando e não descobrir nada é tão
interessante quanto tentar e descobrir, talvez mais. O segredo das águas
era um tesouro, até que o entendi, aí a emoção acabou, a sensação foi de
perda.

Por meio de experimentos, sei que a madeira nada, assim como as folhas secas e as
penas, e muitas outras coisas; desse modo, por meio de evidência cumulativa, se
sabe que pedras também nadam; mas você tem que aceitar simplesmente que sabe,
pois não há maneira de comprovar --- até agora. Quem sabe encontrarei uma --- e
então aquele prazer desaparecerá. Essas coisas me deixam triste, pois aos
poucos, quando tiver descoberto tudo, não haverá mais emoção, e eu adoro
emoção! Uma noite dessas, só de pensar nisso, não consegui dormir.

No começo não conseguia entender para que eu tinha sido feita, mas agora acho
que foi para buscar os segredos deste maravilhoso mundo e ser feliz, e
agradecer ao Criador por ter feito tudo isso. Creio que ainda há muito para
aprender --- assim espero; e devagar, sem me apressar muito, acho que levará 
várias semanas. Assim espero. Quando você joga uma pena no ar, ela flutua,
vai planando e desaparece; aí você joga um torrão de barro --- e ele não
desaparece. Ele cai, toda vez. Testei várias vezes, e foi sempre assim. Me
pergunto por quê. É claro que não cai, mas parece que cai. Imagino que seja
ilusão de óptica. Das duas, uma. Não sei qual. Talvez seja a pena, talvez o
torrão. Não consigo comprovar qual dos dois, só consigo demonstrar que um
dos dois é falso, e deixo a pessoa fazer sua escolha.

Por meio da observação sei que as estrelas não vão durar. Vi algumas das
melhores derreterem e escorregarem do céu. Se uma pode derreter, é possível
que as outras também derretam na mesma noite. Essa tristeza virá --- sei
disso. Decidi que as observarei toda noite até cair de sono;
e vou fixar aqueles campos cintilantes na minha memória para que aos poucos,
quando tiverem sido retiradas, eu possa devolver aquela miríade linda de
estrelas ao céu negro e fazê-las brilhar novamente, e duplicá-las com o 
borrão de minhas lágrimas.

\begin{center}
\leafNE
\end{center}
%\asterisc

\section{Depois da Queda}

Quando olho para trás, o Jardim parece um sonho que eu tive. Era bonito,
insuperavelmente bonito, encantadoramente bonito; e agora está perdido,
nunca mais o verei.

O Jardim está perdido, mas eu o achei, e estou feliz. Ele me ama do jeito dele;
eu o amo com todas as forças da minha natureza passional, e isso, penso, está de
acordo com minha juventude e meu sexo. Se pergunto por que o amo, descubro que
não sei e que realmente não me importo muito com isso; suponho que esse
tipo de amor não seja um produto da razão ou da estatística, como o amor de alguém
por outros répteis e animais. Acho que é para ser assim. Amo alguns pássaros pelo
seu canto; mas não amo Adão por causa do seu canto --- não, não é assim; quanto
mais ele canta, menos eu quero ouvir. Ainda assim, peço que cante, porque
desejo aprender a gostar de tudo que lhe interessa. Tenho certeza de que consigo
aprender, no início era insuportável, mas agora sim. Pode azedar o leite,
não me importa, eu me acostumo a tomá-lo assim mesmo.

Não é por causa da sua inteligência que o amo --- não, não é. Não é ele o culpado
de sua pouca inteligência, não foi ele quem a fez; ele é como Deus o
criou, e isso basta. Havia um sábio propósito nisso, tenho certeza. Aos poucos
a sua inteligência vai se desenvolver, mas não será de uma hora para outra; além do mais, 
não há pressa, ele está muito bem do jeito que é.

Não é por causa do seu modo cortês e atencioso nem por sua delicadeza que eu o
amo. Ele é pobre nesses quesitos, mas é bom o suficiente assim como é, e está
melhorando.

Não é por causa da sua capacidade que o amo --- não, não é. Acho que tem essa
qualidade dentro de si, não sei por que esconde de mim. É minha única
dor. No mais, ele é franco e aberto comigo, agora. Tenho certeza de que não me
esconde nada além disso. Fico magoada quando ele me esconde alguma coisa, às
vezes perco até o sono pensando nisso, mas vou tirar isso da cabeça, não
vai atrapalhar minha felicidade, que no mais é plena e transbordante.

Não é por causa da sua educação que o amo --- não, não é. Ele é autodidata, e de
fato sabe uma série de coisas, mas nem tantas.

Não é por causa do seu cavalheirismo que o amo --- não, não é. Ele me dedurou, mas
não o culpo; acho que é uma peculiaridade do sexo, e ele não escolheu o dele.
É claro que eu não o deduraria, morreria antes, mas é uma peculiaridade
do sexo também, e não posso receber crédito por isso, porque também não
escolhi o meu.

Então, por que é que o amo? Acho que é \textit{só porque ele é masculino.}

No fundo ele é bom, e o amo por isso, mas também o amaria sem isso. Se ele me
batesse e abusasse de mim, continuaria amando. Eu sei. É uma questão de sexo,
acho.

Ele é forte e bonito, e o amo por isso, o admiro e me orgulho dele, mas o amaria
mesmo sem essas qualidades. Se ele fosse simples, eu o amaria do mesmo jeito; se
ele estivesse meio acabado, o amaria também, e trabalharia por ele, me
mataria trabalhando por ele, e rezaria por ele, e o velaria ao pé da cama até
o dia da minha morte.

Sim, eu acho que o amo meramente porque ele é \textit{meu} e é \textit{masculino}. Acho que não há
outra razão. Assim, é como eu disse: esse tipo
de amor não é produto da razão nem da estatística. Simplesmente \textit{aparece} --- não
se sabe de onde --- e não pode ser explicado. Nem precisa.

É o que eu acho. Mas sou apenas uma menina, a primeira que examinou essa
questão, e pode ser que em minha ignorância e inexperiência eu tenha me
enganado.

\begin{center}
\leafNE
\end{center}

\section{Quarenta anos depois}

É minha prece, meu desejo, que possamos passar desta vida juntos --- um desejo
que nunca deverá se extinguir nesta terra, mas que existirá dentro do
coração de cada esposa que ama, até o final dos tempos; e deverá ser chamado pelo
meu nome.

Mas se um de nós tiver de partir antes, rogo que seja eu; pois ele é forte, eu
sou fraca, não sou necessária para ele como ele o é para mim --- a
vida sem ele não seria vida; como eu a suportaria? Esta oração também é
imortal, e não deixará de ser feita enquanto minha raça continuar. Sou a
primeira esposa; e na última das esposas serei perpetuada.

\begin{center}
\leafNE
\end{center}

\section{No túmulo de Eva}
\medskip

``Onde quer que ela estivesse, lá estava o Éden.''
\\

\hfill\textit{Adão}  

\begin{center}
\textsc{fim}
\end{center}




\chapter[Passagens do diário de Satã]{Passagens do\break diário de Satã}
\hedramarkboth{passagens do diário de satã}{mark twain}

\textsc{Muitos anos atrás,} eu estava entre arbustos, perto da Árvore do
Conhecimento, quando o Homem e a Mulher ali chegaram e tiveram uma
conversa. Também desta vez estive presente, quando voltaram,
depois de tantos anos. Eram como antes --- apenas um menino e uma menina;
elegantes, curvilíneos, esbeltos, flexíveis --- brancos
como a neve, levemente enrubescidos pela cor rosa do céu, inocentemente
inconscientes de sua nudez, graciosos de ver; belos,
além das palavras.

Novamente ouvi a conversa. Como da outra vez, eles quebravam a cabeça
para entender aquelas palavras: ``bem'', ``mal'', ``morte'', e
tentavam decifrar seu significado; mas não eram capazes disso,
claro. Adão disse:

“Venha, talvez a gente consiga encontrar Satã. Ele deve entender dessas coisas.”

Então eu me aproximei, e enquanto olhava para Eva, admirando-a, disse:

“Você nunca me viu antes, doce criatura, mas eu já. Conheço todos os
animais; em termos de beleza, ninguém se iguala a você. Seu cabelo,
seus olhos, seu rosto, os tons da sua pele, sua forma, o charme dos seus
brancos membros afilados --- tudo belo, adorável, perfeito.”

Isso a agradou, e ela se olhou, esticando um pé e uma mão para poder
admirá-los; e então, inocentemente disse:

“É um prazer poder ser tão bela. E Adão\ldots{} é assim também.”

Ela o fez dar uma volta para lá e para cá, para
exibi-lo, com um orgulho tão sem malícia nos olhos
azuis; e ele\ldots{} considerou isso tão normal, e estava tão
inocentemente feliz, que disse: ``Quando estou com flores na
cabeça, fico ainda mais bonito''.

Eva confirmou: ``É verdade --- você vai ver'', e
correu para lá e para cá, como uma borboleta, e colheu
flores, e em pouco tempo trançou as flores em uma guirlanda e colocou
sobre a cabeça dele; então, na ponta dos pés, ajeitou aqui e ali, com seus
dedos ágeis, a cada toque ganhando em graça e forma; ninguém sabe como nem
por quê, mas há nisso alguma lei cuja arte e segredo apenas ela domina,
sem que ninguém possa aprender; e, quando finalmente ficou como queria,
ela bateu palmas de satisfação, e esticando-se o beijou
--- uma visão tão linda, no conjunto, eu nunca
tinha visto nada assim.

De volta ao que interessa no momento, o significado dessas palavras\ldots{}
Será que devo dizer?

Certamente ninguém mais do que eu gostaria de revelar, mas como faria isso? Não
conseguia pensar em uma maneira de fazê-la entender, e lhe disse isso, e
então falei:

“Tentarei, mas não sei se vai adiantar. Por exemplo, o que é dor?”

“Dor? Não sei.”

“Claro que não. Como poderia? Não existe dor em seu mundo; a dor é
impossível para você, que nunca experimentou a dor física. Reduzindo a
uma fórmula, a um princípio, o que teremos?”

“O que teremos?”

“Isto: coisas que estão fora do nosso âmbito --- do nosso mundo particular ---
coisas que por nossa constituição e habilidades não
somos capazes de ver, sentir ou vivenciar --- \textit{não podem ser
entendidas por meio de palavras}. Em resumo, é isso. É um
princípio, é axiomático, é uma \textit{lei}. Entende agora?

A delicada criatura parecia confusa, e depois de tudo que eu disse fez esta
vaga observação:

“O que quer dizer ‘axiomático’?”

“Ela perdeu o \textsc{x} da questão. É claro que perderia. Mas seu esforço
significava meu sucesso, pois era a confirmação nítida da verdade do que
eu vinha tentando explicar. ‘Axiomático’ era, naquele momento, uma coisa
fora do âmbito de sua experiência, por isso não tinha significado para ela.
Ignorei a pergunta e continuei:

“O que é o medo?”

“Medo? Não sei.”

``Claro que não. Como saberia? Você nunca
sentiu isso, e não pode sentir se não pertence ao seu mundo. Nem com
centenas de milhares de palavras eu conseguiria fazer você entender o que é o
medo. Como, então, explicar a morte? Você nunca a viu, é
estranha ao seu mundo; fica impossível fazer a palavra significar algo
para você, pelo menos a meu ver. De certa forma, é como um
sono\ldots{}''

``Ah, isso eu sei o que é!''

``Mas é sono só de certo modo, como eu disse. Na verdade é mais
do que um sono.''

``O sono é agradável, o sono é adorável.''

``É, mas a morte é um sono \textit{longo}, muito longo.''

``Melhor ainda. Então, não pode haver nada mais agradável que a morte.''

Pensei comigo: ``Pobrezinha, um dia saberá que
verdade patética acabou de dizer; algum dia você talvez dirá, de coração
partido: `Venha até mim, oh, morte, a
compadecida! Faça-me mergulhar em seu misericordioso
esquecimento, oh refúgio dos tristes, amiga dos desamparados e
desconsolados!'”. Então eu disse em voz alta:
``Mas esse sono é eterno''.

A palavra entrou por um ouvido e saiu pelo outro. Não tinha como ser diferente.

``Eterno? O que quer dizer ‘eterno’?''

``Ah, esse é mais um conceito que não existe no seu mundo,
ainda não. Não há como lhe explicar isso.''

Era um caso sem esperança. Palavras referindo-se a coisas alheias a sua
vivência eram como uma língua estrangeira para ela, sem nenhum significado. 
Ela era como um bebê cuja mãe diz: ``Não coloque o
dedo na chama da vela, você vai se queimar''. “Queimar” é uma
palavra estrangeira para o bebê e não vai deixá-lo com medo até que a
experiência revele o sentido. É perda de tempo a mãe fazer um
comentário desse, o bebê continuará brincando sorridente, e colocará o dedo na
linda chama\ldots{} \textit{uma vez}. Depois dessas
reflexões, que guardei só para mim,  disse outra vez que
achava que não conseguiria fazê-la entender o significado da palavra
“eterno”. Ela ficou em silêncio por alguns momentos, remoendo essas
questões profundas no motorzinho ainda virgem que era sua mente; então desistiu do
quebra-cabeça e mudou de assunto:

``Bom, há ainda estas outras palavras. O que é o bem? E o que
é o mal?''

``Aí está, mais uma dificuldade. Esses conceitos também são alheios ao
seu mundo. Existem apenas no reino da
moralidade. Você não tem moral.''

``O que é moral?''

``É um sistema de leis que distingue entre o certo e o errado;
moral ou imoral. Essas coisas não existem para você. Eu não sei se consigo
explicar, você não entenderia.''

``Mas tente.''

``Bom, obediência à autoridade constituída é uma lei moral.
Suponha que Adão a proibisse de colocar seu filho no rio e deixá-lo lá a
noite toda\ldots{} você o deixaria lá?''

Ela respondeu com uma simplicidade tão amável e
inocente:

``Claro que sim, se quisesse.''

``Bom, como eu disse, você não teria como saber; você não tem
noção de dever, ordens, obediência\ldots{} essas coisas não significam nada para você. No
seu estado atual, você não é responsável por nada do que faz, diz ou
pensa. É impossível que faça algo errado, pois você não tem noção
do que é certo e errado mais do que os outros animais. Você e eles só podem
fazer o que é certo, o que quer que façam será sempre
certo e inocente. Essa é uma condição divina, a mais sublime e pura
condição tanto na terra como no céu. É o dom dos anjos. Os anjos são
completamente puros e sem pecado, pois não distinguem o certo do errado, e
todos os seus atos são irrepreensíveis. Ninguém pode fazer algo errado se
não sabe distinguir \textit{entre} o certo e o errado.''

``E tem alguma vantagem em saber?''

``Vantagem nenhuma! Esse conhecimento seria como tirar dos
anjos tudo o que é mais divino, mais angelical, o que os degradaria de
maneira imensurável.''

``E existem pessoas que sabem a diferença entre o certo e o
errado?''

``Não no\ldots{} bem, não no céu.''

``O que traz esse conhecimento?''

``O senso moral.''

``O que é isso?''

``Bom\ldots{} não importa. Agradeça por não
saber do que se trata.''

``Por quê?''

``Porque é uma degradação, um desastre. Sem ele, \textit{não
dá} para errar; com ele, dá. Por isso, ele só tem uma
função, só uma --- a de ensinar a \textit{errar}.
Não tem nenhuma outra função. Foi ele que inventou o erro; o erro simplesmente não
existe até que o senso moral o faça existir.''

``Como se adquire esse senso moral?''

``Comendo da fruta desta árvore, aqui. Mas por que você quer
saber? Gostaria de ter senso moral?

Ela se voltou, desejosa, para Adão:

``Você gostaria de ter?''

Ele não mostrou nenhum interesse particular, disse apenas:

``Sou indiferente. Não entendi nada dessa conversa, mas se
você quiser podemos comer a fruta, não vejo por que não.''

Pobres criaturas ignorantes, a ordem de se absterem de
comer da fruta não significou nada para eles; eram apenas crianças, e não
podiam entender coisas não provadas e abstrações verbais que representavam
coisas alheias a seu pequeno mundo e a suas parcas experiências. Eva
esticou-se para alcançar uma maçã! Adeus Éden e seus prazeres sem pecado;
bem-vindos pobreza e dor, fome e frio e sofrimento, privação, lágrimas e
vergonha, inveja, disputas, malícia e desonra, idade, cansaço, remorso; e
a seguir, desespero e orações para livrar-se da morte,
sem suspeitar que os portões do inferno se abrem mais
adiante! Ela experimentou --- e a fruta caiu de sua mão.

Foi triste. Ela parecia alguém que acaba de acordar lento e
confuso de um sono profundo. Olhou para mim meio perturbada, sem
entender, então olhou para Adão, segurando aquele seu véu de longos cabelos
dourados para trás com as mãos, até que seu olhar inquisidor caiu sobre
a própria nudez. O sangue rubro subiu-lhe às faces, ela se escondeu atrás
de um arbusto e ficou lá, chorando:

``Oh, minha modéstia se foi, minhas formas inofensivas
Tornaram-se uma vergonha --- minha mente era pura e limpa; pela primeira vez
maculei-a com um pensamento sujo!'' Ela gemia e murmurava em
sua dor, e deixou a cabeça pender, dizendo: ``Estou
degradada, caí tão baixo, nunca conseguirei me reerguer''.

Os olhos de Adão fixos nela mostravam um assombro sonolento; ele não
conseguia entender o que tinha acontecido, já que isso acontecia fora do
seu mundo; as palavras dela não faziam o menor sentido para quem ainda
não tinha senso moral. E esse assombro só aumentou, pois, sem que ela
se desse conta, seus cem anos de vida
se apoderaram dela e fizeram seus olhos divinos e os matizes de
suas carnes frescas desbotarem,
os cabelos ficaram grisalhos, a boca e os olhos ganharam vincos,
sua forma encolheu, e o lustre acetinado de sua pele
desapareceu.

Tudo isso o inocente menino presenciou; então, fiel e corajoso, ele
pegou a maçã e provou; não disse nada.

As mudanças caíram sobre ele também. Ele juntou ramos suficientes
para ambos cobrirem suas vergonhas; deram meia-volta e seguiram seu
caminho, de mãos dadas, curvados pela idade, e assim saíram de cena.


\chapter{Autobiografia de Eva}
\hedramarkboth{autobiografia de eva}{mark twain}

\sectionitem

\begin{center}
Começarei por algumas passagens do meu diário.
\end{center}

\begin{center}
\textit{Diário de Eva}
\end{center}


\dia{Segunda, 8 de janeiro, ano 1} Quem sou eu? O que sou? Onde estou?

\dia{Segunda, mais tarde} Estas perguntas continuam sem resposta. Não tem importância; deixa para lá.

\dia{Duas semanas depois} Que solidão. Que monotonia. Um verdadeiro tédio. Os pássaros e os tigres
e as coisas são companhia agradável, eles me amam e eu os amo, mas ultimamente, aqui, eles parecem
de algum modo insuficientes. Falta algo, não sei o quê.

Se ao menos eles pudessem ver como sou bonita, bem roliça e macia, e como os
meus membros têm formas delicadas. É possível que eles vejam; às vezes penso que sim; mas só
olham, eles não \textit{dizem} --- pelo menos não numa língua que eu entenda.

Começo a ter certeza de que é isso que falta --- ouvir isso \textit{dito}.

\dia{Quarta} Nada se compara à beleza do reflexo do meu corpo branco e esguio no lago,
quando fico ali na beira, me inclino e deixo meus cabelos loiros penderem no ar.
Ontem, o leopardo me encarou como que extasiado; corei de prazer;
meus olhos queimaram num fogo úmido, e eu poderia tê-lo abraçado --- mas ele não \textit{disse}
nada de sua admiração. Estava tão ansiosa por isso!

Então ele se voltou e viu outro leopardo, e seus olhos flamejaram de receptividade, e ele
disse algo que não entendi e então avançou impaciente, e os dois adentraram a floresta
lado a lado, afagando um ao outro. Isso me fez chorar de rancor.
Vou dar uma sova nesse outro leopardo.

\dia{Sexta} Há alguma coisa errada, com certeza --- pelo menos um desvio do usual; pode ser
o destino, pode ser apenas acidental --- seja como for, o fato é que todas as outras
criaturas têm companheiros, eu não. Sou a única sozinha. Há dois leopardos, dois mastodontes,
dois megatérios, dois pterodáctilos, dois tigres, dois leões, dois elefantes, dois de todas as coisas
--- e eu sem ninguém. Isso não pode ser natural; sem dúvida não era para ser assim. O que
foi que eu fiz? Não fiz nada. Não merecia esse desgosto, essa vergonha, esse constrangimento.

\dia{Sábado} Pois isso \textit{é} um constrangimento. Antes nem tanto, mas desde que pensei nisso
ontem me sinto como se todos os olhos estivessem voltados para mim, furtivos, inquisidores,
e isso me faz sentir tímida e infeliz.

Tola também. Pois foi tolice falar \textit{com} as criaturas, em vez de falar comigo mesma, esta manhã,
Dizendo: ``Nossa, eu não sabia que era tão tarde; eu aqui, matando o tempo, e meu companheiro pode chegar
a qualquer momento de sua longa viagem e ficar \textit{muito} desapontado se eu não estiver lá para recebê-lo''.
Era tolice, mas o que eu podia fazer? Queria tanto que eles pensassem que tenho um companheiro,
que sou como os outros animais, e não uma aberração. Mas é claro que eles não me entenderam, meus
esforços foram em vão. Quase me convenci de que um ou dois animais podem ter compreendido
o que eu disse e contado aos outros, mas não foi bem assim. Quando eu me sentava, eles ficavam todos ao
redor, observando e ouvindo, mas não davam sinal de ter entendido. Pouco depois,
uma pequena e gentil forasteira, bonita, de olhar dócil, e totalmente desconhecida para mim --- eu a chamarei
de Gazela ---, aproximou-se e aconchegou a cabeça em meus braços; eu a abracei cheia
de gratidão, acreditando que ela havia entendido, e disse: ``Você \textit{entende}, não entende, minha
querida? Diga que entende e que vai me ajudar a encontrar meu companheiro perdido''.

Mas ela não sabia do que eu estava falando; perdi todas as minhas esperanças, levantei e fui
embora chorando, com algumas criaturas a me seguir por hábito, enquanto outras ficaram onde estavam,
indiferentes.

\dia{Três meses depois, maio} Como as coisas mudaram! Já não são mais casais, são famílias agora.       
Junto a cada casal há os mais adoráveis filhotinhos, e tão ladinos e graciosos, tão bonitinhos e fofinhos!
Brincam comigo o dia todo, e dormem amontoados por cima de mim à noite, afastam minha solidão
e me trazem paz. Todas as criaturas que voam têm ninhos e ninhadas de pequeninos, e os bosques
estão repletos de seus trinados encantadores.

Por isso sou mais feliz que antes. Tento afastar de mim esse pensamento --- o pensamento
de um companheiro ---; durante o dia consigo, fico contente e não sinto a minha dor. Mas
à noite sonho --- e sonho.

\sectionitem

\dia{19 de maio} Agora há um potro, e o casal de cavalos está feliz. O potro é do tamanho de um cãozinho,
e seus pais, do tamanho de um labrador. Eles têm cinco dedos, o que não parece o mais acertado
para um cavalo, mas não parecem se importar com isso.

A princípio, considerei a grande aranha peluda um animal feio e desagradável de ter por perto,
mas a paciência e o bom temperamento dela acabaram me conquistando. As moscas e os besouros
a infernizam o dia todo, zumbindo e esvoaçando em volta enquanto ela tenta dormir; e se atiram
em sua linda teia, debatendo-se e pelejando na tentativa de libertar-se; em vez de ficar furiosa, ela
rompe os grilhões e os liberta, passando então horas a fio a remendar os buracos da teia, sem
nunca demonstrar qualquer ressentimento. Ela não tem tempo para sair em busca de alimento, e me admira
que se mantenha saudável e rechonchuda; ela suga o orvalho, mas eu nunca a vi comer nada. Todas
as outras criaturas comem.

O leão come nabos; o tigre come tomates; a hiena vive de morangos, a raposa come maçãs, a lontra
come cerejas, o abutre come laranjas, a águia come bananas, o falcão vive de peras,
mas se a aranha come alguma coisa eu ainda não descobri o que é.

Todos os animais brincam juntos, reunidos de acordo com o tamanho: os carneiros e os cordeiros e os cavalos e os cervos
e os lobos e os cães, e outros de tamanho semelhante, em um grupo; os camelos e bezerros e vacas e leões e tigres e leopardos e assim por diante em outro;
o elefante, o mastodonte e o megatério, o rinoceronte, a preguiça, grande como o tronco de uma árvore, o sapo gigante, com três metros de altura,
em outro; e os enormes sáurios em outro ainda. E como esses monstros saltam e dão cambalhotas quando estão com vontade de brincar!
Quando apostam corrida o chão treme sob seus pés, e quando eles se chocam ouve-se a pancada à distância de um quilômetro, e ver
os grandalhões lançarem os pequeninos, como o elefante e o hipopótamo, pelo ar com a mesma facilidade com que eu atiraria um punhado de terra,
e com tanto bom humor e se divertindo à beça, é um esporte \textit{tão} divertido que me leva às lágrimas de tanto rir, de modo que
eu sempre corria para observá-los.

E aqueles lagartos imensos, meu Deus! Quando eles balançam a cauda, as árvores menores vão abaixo como se fossem ervas daninhas!
Mas eles gostam mesmo é de lutar. Ontem, um grande ictiossauro, com trinta metros de altura, lutou corpo a corpo com sua parceira,
de vinte e nove metros, e saltavam treze metros, debatendo-se um nos braços do outro, e rugindo como um furacão, foi incrível de ver.

É mesmo uma pena que eu não tenha parceiro para desfrutarmos juntos esses momentos.

\sectionitem

\dia{25 de maio} Tenho companhia. Aquele lindo papagaio do qual estive cuidando fala! Nunca fiquei tão surpresa. Esta manhã
ele se escarranchou pelo galho, daquele seu jeito bizarro, até me alcançar, então inclinou a cabeça de modo tão manhoso,
e disse:

--- Eva boazinha, Eva pobrezinha! Polly quer biscoito!     

Ele falou isso com todas as letras. Eu o apertei contra o coração, abracei-o com alegria, dizendo:

--- Oh, meu querido! Você é meu parceiro?

Mas ele apenas gritou, a voz esganiçada, numa explosão de gargalhadas diabólicas:

--- Biscoito! Biscoito! Coitadinho do Polly! Polly quer biscoito! Rápido, Satã!

Ah, que êxtase ouvir a minha língua natal!

--- Você \textit{tem} de ser o meu parceiro perdido --- diga! --- eu imploro!

Foi um pensamento tolo, mas eu estava enlouquecida de tanta saudade. Ele tinha uma parceira ---
eu já sabia disso. Ela vinha todos os dias, e eles saíam à cata de caquis e rabanetes para os filhotes. Ela apareceu e
o levou com ela, embora ele continuasse a vociferar aquela palavra estranha, ``biscoito'', e estivesse enormemente exasperado e
bastante relutante em sair.

``Eva.'' O que será isso? E Satã\ldots{} e biscoito\ldots{} Polly é um nome, é claro, seu próprio nome. Quem lhe deu esse nome? Nenhum outro animal
tem nome, até onde eu saiba. Vou arranjar um para mim mesma, se eu conseguir pensar em um bem bonito.

\dia{3 de junho} Polly está ficando bastante sociável e divertido agora que ele sabe que eu também sei falar. Ele é alegre e festivo e insolente
e fala e ri e grita o tempo todo. Mas no final das contas é um pouco decepcionante, ele quase não dá valor a conversações mais elevadas,
e o seu leque de assuntos é tão limitado. Outro defeito --- ele se repete demais. Isso é vulgar. E indica uma mentalidade inferior, assim como
uma indiferença ao aperfeiçoamento. Não quero julgar injustamente, mas, para ser franca, sou obrigada a dizer que acredito que ele carece de
espiritualidade. Pode parecer indelicado, mas acho que é verdade; de fato, eu sei que é. Ele não expressa muito bem suas emoções; dificilmente apresenta qualquer
traço de sentimento; seu espírito é mundano; se eu tento conduzi-lo a planos mais elevados de pensamento, ele fica
aborrecido e diz: ``Chega!''; e ontem, quando falei com grande emoção ``Quão majestoso é o universo, quão nobre é a criação, quão amplo,
quão impressionante\ldots{}'', ele explodiu num guincho agudo e áspero, seguido da sua risada odiosa e vociferou uma torrente de palavras estranhas
que pelo que meu instinto me dizia não deviam ser nada agradáveis, e pediu um biscoito.

Fiquei profundamente magoada e mudei de assunto. Eu tinha tantas esperanças com relação a ele --- e agora o sonho acabou.
Ele não pode suprir as necessidades mais elevadas da minha natureza; meus anseios a esse respeito permanecerão insatisfeitos,
meu espírito sedento não será aplacado. Foi uma revelação amarga.

Mas onde foi que ele aprendeu inglês? Com a companheira é que não foi, ela não fala outra língua. Às vezes fico convencida de que
ela deve ter encontrado o meu companheiro --- deve tê-lo visto cara a cara ---, ouvido ele falar! Então fui arrebatada pelo mundo e quase desmaiei com
o adorável êxtase que se seguiu a esse pensamento.

Nesses momentos, eu lhe suplicava que me dissesse; rogava, implorava,
ajoelhava-me diante dele com lágrimas nos olhos, --- inutilmente, pois tudo o
que ele respondia é que queria mais uma bolacha. E eu queria saber o que é uma
bolacha, para poder atender ao seu desejo e dar um tempo desse tema cansativo.
Ele estava habituado às más companhias, incultas, o linguajar demonstrava. Será
que era mesmo meu companheiro? Esse pensamento me é prazeroso, mas também
doloroso. Eu queria que meu companheiro fosse nobre, cortês e refinado; e, oh,
ele é com certeza! Não devo crer no contrário.

\dia{7 de junho} Finalmente me dei conta de que Eva é um nome --- \textit{meu}
nome! Quem me deu esse nome? Polly? Não, essa pobre cabecinha de vento não seria
capaz de criar nada. Seria ele então meu companheiro? Creio que sim e não me
permitirei duvidar disso. Ele sabe de minha existência, disso eu tenho certeza,
ele está a minha procura, aflito, com saudade de mim --- talvez esteja pensando
em mim nesse exato momento --- sussurrando meu nome! Oh, eu poderia desmaiar de
felicidade! Oh, meu Satã, meu querido!

Mas talvez não seja ele. Satã é um nome --- estou convencida disso; mas
ultimamente houve um outro: Adão. Por um momento, pareceu significar uma
espécie, depois parecia significar o indivíduo de uma espécie, mas ultimamente
pareceu às vezes significar o \textit{nome} de um indivíduo --- algo que remete
a esse indivíduo em particular. Não importa, contanto que eu encontre meu
companheiro, o nome não tem importância nenhuma. Oh, mas que eu possa vê-lo ao
menos! E saber de antemão quando, para poder me pentear e aparecer em minha
melhor forma.
\sectionitem


\dia{9 de junho} As mesmas questões intrigantes continuam na minha cabeça boa
parte do tempo. Quem sou eu? De onde vim? E não encontro resposta. Polly não
consegue responder; de qualquer modo, vejo que nem tenta; o assunto não
lhe interessa e ele não fala sobre isso. Sua mente é frívola, ele só se importa com
coisas banais. É patético ver a rapidez com que ele foge de assuntos mais
sérios. Acho que suspeita de suas limitações e tem vergonha delas. Quando
fica ansioso, esquece o decoro e irrompe em uma série de obscenidades. Isso é
outra prova de que teve contato com alguém que fala minha língua e que pertence a
minha espécie, pois não pode ter inventado essas frases --- não seria capaz.
Será que ele conheceu Satã? E Adão? Gostaria que esse tivesse sido
meu privilégio. Satã é um nome bonito. Pergunto-me se ele gostaria de me
encontrar quando estou com uma guirlanda de orquídeas na cabeça. Acho que sim.
Estou usando uma agora.

É muito solitário. Não sempre, mas muitas vezes. Nunca durante o dia, porque estou
sempre ocupada durante o dia, nomeando e classificando as criaturas, plantas,
árvores, flores e coisas, das oito ao meio-dia e das duas às seis. Pois esse é o
meu expediente. E quando me dá na telha, o que acontece quase sempre, tiro uns
dias para brincar com os animais. Também com certa frequência subo no lombo do
elefante e faço longas excursões pelas florestas dos morros ou vales floridos,
acampando quando a noite vem. Sempre acompanhada, pois os animais vêm em
manadas, e os pássaros, em revoadas. Milhares! Nunca se cansam, brincando e
correndo o tempo todo. Bonito de ver. E a vida, e a graça, e a animação, e as
cores cintilantes com o brilho dos raios de sol, cada lombo aveludado e cada asa
brilhante fazendo sua contribuição especial para o espetáculo geral.

Mas à noite, quando as criaturas estão dormindo, muitas vezes me sinto só. Deitada
sobre um campo de flores, com nada para fazer exceto apreciar a lua vagando pelo
céu e cobrindo as flores e as planícies com sua luz cor de prata. E fica tudo tão
profundamente quieto! Nenhum ruído --- pois não há animais notívagos. Deve haver
alguns. Tentei treinar a coruja e a raposa para vigiar à noite, mas eles
não se interessaram, não quiseram assumir esse papel. Não comem nada a não ser
laranjas e abacaxis, e à noite não têm como apanhar essas frutas.

Então deito e fico olhando a lua ou as estrelas, e penso naquele
mistério, naquele problema: quem sou eu? De onde venho? Sou apenas um acidente?
Será que aconteci apenas, ou fui intencional? E onde está Satã? Onde está Adão? Nas
minhas excursões viajei léguas e mais léguas, mas não encontrei sinal deles.
Não cheguei a nenhum limite? Será que o mundo não tem um fim? Sei que viajei mais
de duzentas milhas em linha reta e ainda não cheguei ao limite --- sempre há mais
terra. Essa vastidão é terrível, oprime o meu espírito.

Ainda lembro como se fosse ontem. Eu era uma jovenzinha faceira, cheia de vida, com
aquele esplêndido entusiasmo da juventude; viajar era meu deleite; a cada milha da
minha jornada coisas novas se revelavam e cada novidade me trazia um prazer
especial; todos os dias viajava por um mundo novo, novas maravilhas, novas
combinações de beleza, majestade e sublimidade, e todas as noites eu sonhava com
os encantamentos do dia e os revivia. Essa tinha sido a minha viagem
mais longa; nunca estivera tão longe de casa. Estava escalando
montanhas --- era uma nova experiência.

\dia{diário} Era fácil de perceber que estava ficando mais frio, mas não havia chuva.
Isso era estranho. Antes, chovia quando ficava mais fresco. Não havia explicação
para essa nova coisa. O frio foi aumentando --- crescia de forma estável. Examinei
de perto a terra e as pedras, mas permaneciam como antes --- a mudança de
temperatura não estava sendo influenciada por elas. Eu estava inquieta. Havia algo
de sinistro e assombroso nesse frio sem causa aparente.

Fiquei mais calma quando refleti que essa condição anormal das coisas não duraria muito mais
e que com paciência e perseverança aos poucos alcançaríamos uma proximidade
maior com o sol e esse frio não resistiria aos seus raios tão próximos, e então
não mais tremeríamos de frio e começaríamos a tostar, isso sim. Mas, milagre dos
milagres, o oposto aconteceu. Quanto mais escalávamos em direção à fonte de calor,
mais frio ficava. Essa é a verdade. Um mistério que provavelmente
nunca será explicado.

Antes da noite o frio era de amargar, de arrepiar; e o vento,
Em vez de acariciar meu corpo
desprotegido como de costume, era cruel, cortante. Meu cabelo, ajeitado em
duas longas tranças sobre as costas, absorveu a umidade da atmosfera e ficou
duro, congelado. Minhas bochechas e o nariz doíam tanto que me fizeram chorar.

De repente, chegamos a um precipício bem alto, e de lá avistamos
lá longe, embaixo, um vale encantador, de beleza e glória inimagináveis,
adormecido à luz do sol. Que pradaria, que campos, que bosques, que clareiras, quantas
variedades de flores, que suavidade, que riqueza, que resplendor e brilho, que
rapsódia de cores --- um sonho! E por lá corriam quatro rios brilhantes,
estampados de reflexos, serpenteando de um lado a outro, as suaves distâncias
daquele céu de solidão e paz, que desaparecia gradualmente no remoto lugar
onde a terra e o céu se fundiam.

Minha pobre casinha, que até então parecia tão bonita\ldots{}

Como chegar até aquele alegre vale? Essa era minha preocupação. Eu desceria até lá
e lá viveria para sempre. “Lá certamente encontrarei Satã, lá encontrarei Adão”,
disse, “lá não estarei mais sozinha”.

Mas não encontrei nenhuma passagem naquele precipício. Ansiosa, subi e desci
por pequenas trilhas, procurando, mas não havia nenhum caminho. E durante
esse tempo o sol foi afundando. Finalmente, a escuridão se apossou de tudo, e
aquelas terras do meu desejo desapareceram. Eu não era mais que um esboço de uma
moça; sentei e chorei. Os animais vieram e me consolaram, e tentaram me
dizer que eu tinha amigos, que não sofresse. Então me levantei e com eles fui
procurar um refúgio para passar a noite.

Deitei, e eles se aconchegaram ao meu redor, e seu pelo me aqueceu e seu calor me fez adormecer.
Acordei na aurora e vi que
algo muito estranho estava acontecendo. Um pó branco se espalhava, e
quando caía em meu braço transformava-se em água. Fiquei apavorada. Subi
nas costas do elefante sem me importar para onde estava indo, e assim nos
livramos dessa estranha invasão dos céus.

Batizei isso de neve --- e é exatamente isso. O elefante me levou até o pé da
montanha; por duas semanas ficamos ladeando a base dessas planícies, tentando
achar o lugar onde os rios desembocavam, para conseguir entrar no Vale Alegre;
mas nunca encontramos.

Não fiquei satisfeita por muito tempo. Dia e noite aquela visão me surgia na
ternura do sonho e me atormentava com uma saudade implacável, pedindo para rever
aquelas planícies outra vez. Os arredores da minha morada tinham perdido o
charme; pareciam tão sem graça e pobres; não me davam mais nenhum prazer.

E foi assim que, em setembro, peguei o caminho de novo, com meus gigantes me
seguindo, e mais uma vez procurei pelos rios, dia após dia, mas não achei;
então escalei as montanhas, estava determinada a encarar a neve para
usufruir novamente aquela vista.

E eu vi! Parada na neve e no vento cortante, nas brechas entre neve pesada e
chuva de gelo, entrevi o mágico vale novamente.

Depois disso, persegui o lugar. Acampei na base da montanha e todos os dias
escalava até o cume para me alimentar da visão, sempre procurando um modo de
descer o precipício.

\dia{diário} Hoje, aqui em cima, tive o mais estupendo choque da minha vida. Na neve
fresquinha eu deparei com uma grande pegada de um humano.

Minha cabeça ficou zonza, quase desmaiei. Meus membros inferiores balançaram de
fraqueza --- ou talvez por causa da gratidão, que revigora. Fiquei parada, olhando
atenta para aquela marca de pé em um delicioso delírio de felicidade, depois
me agachei e a beijei. Beijei uma dúzia de vezes; depois, já em cima do elefante, disse: “Siga essas pegadas!”.

Eu me lembro bem. Segui as pegadas umas quinze ou vinte milhas ao longo das curvas do
precipício, então desci por uma trilha quase imperceptível, estreita
demais para o elefante. Deixei-o naquele ponto junto com outros animais e
desci a pé, seguida por tigres e leões, macacos e alces irlandeses, ursos e outras
criaturas menores, centenas, milhares, e muitos, muitos pássaros.

Finalmente cheguei ao vale. A glória, o esplendor, a beleza da realidade superaram
minha visão; e a fragrância era inebriante. Mas as pegadas tinham desaparecido,
bem onde começava um gramado aveludado. Havia ali uma miríade de
criaturas, e todos nos davam as boas-vindas carinhosos; mas meu parceiro não estava entre eles.

Foi um dia triste. Na verdade, o mais agudamente triste de todos que já
vivera até então. Estou velha agora: curvada, rebentada,
debilitada, enviuvada, e minha cabeça está branca de tanto sofrimento. Enfrentei
a dor por uns mil anos, mas esse dia se destaca em minha memória por ter sido
o mais marcante: foi o dia em que senti minha primeira verdadeira tristeza; foi a
primeira vez que fiquei com o coração realmente partido. Este manuscrito apagado
está manchado de lágrimas, e ainda agora, depois de dez séculos, mais lágrimas.
Choro de pena daquela pobre criança, pois agora, tão distante
daquela época, sinto como se não fora eu, mas um filho que perdi --- 
meu filho. Outras mães devem ter sentido algo semelhante ao relembrar não como
eram antes, como foi meu caso, mas aqueles pequenos seres,
filhos e filhas, que cresceram e ficaram adultos, maduros. Algumas vezes,
por momentos, essas pobres mães têm visões de seus filhinhos brincando por aí, e
reconhecem suas vozes e risadas --- sumidas há muito tempo! É então que
sentem uma dor lancinante, sabendo que essas criancinhas se foram para sempre,
mesmo que sua versão adulta esteja presente e ainda seja preciosa. Os amados e
perdidos! \textit{Elas} sabem --- as mães! Elas sabem o que os adultos são e o que
foram um dia --- o que foram e o que são é o mesmo, e ainda assim tão diferente. O
que são permanece, e o que foram partiu da vida dessas mães para nunca mais, a
não ser em visões.

Ainda assim, apesar do desgaste dos séculos, consigo ver aquela criatura sedosa
exatamente como era, a mais bela naquele belo Éden; e o velho coração, apesar
de enrijecido, ainda sente a dor da frustração daquele dia.

\dia{diário} Há semanas perambulo por aqui. Caminhei léguas e mais léguas em todas as
direções sem encontrar o menor vestígio dele. O vale é cruelmente vasto! Mais cedo,
hoje, tive uma boa ideia e então peguei um perdigueiro e subi com ele até onde
encontrei a pegada na neve, esperançosa de que ele o encontraria o mais tardar em uma hora. Mas foi mais uma frustração;
ele não se interessou. Como se aquela pegada não exalasse cheiro nenhum, saiu
fuçando noutra direção. Isso prova que nossa espécie não exala cheiro. O
que pode indicar nossa superioridade --- primazia mesmo. Mas essa questão a gente pode
deixar para mais tarde.

\dia{Ano 2, 6 de janeiro} Meu aniversário. Estou fazendo um ano. E nesse dia, entre
todos, eu o encontrei. Estava deitado na relva, sob uma árvore frondosa, dormindo.
Meu coração quase disparou de felicidade. Não há palavras para descrever
sua beleza: tão jovem, tão cheio de vitalidade, desabrochando como eu. Mas era mais
alto, vigoroso e musculoso, porém não tão largo nos quadris. Tinha cabelo
castanho crespo, que pendia, negligente, sobre os ombros --- oh, tão lindo! E a pele
do rosto, tão branca e rosada, agradável de tocar, como a minha. Acariciei-o à
vontade, brinquei com o cabelo dele, beijei seus olhos e lábios e nunca me
senti tão feliz em todo esse meu primeiro ano de vida. Um menino tão encantador
--- e todo meu! Sussurrei, e murmurei, e sussurrei outra vez, e disse,
Carinhosa: “Satã, querido Satã!”, não para acordá-lo, só para ouvir seu
nome e sentir a vibração.

Finalmente ele acordou, e se sentou, olhando fixo para mim. Não consegui me conter e
joguei meus braços ao redor de seu pescoço, e o apertei contra mim e o beijei
na boca --- uma, duas, três, seis vezes. Isso o assustou; ele me empurrou e se
levantou, e seus olhos ardiam de raiva e espanto, e suas narinas se abriam e
fechavam como as de um cavalo, e seus joelhos tremiam.

Como ele pôde me usar assim? O que tinha feito? Eu não queria o seu mal. Estava
feliz por estar com ele, e só queria expressar isso do único modo que conhecia;
sou jovem, e não tive ninguém que me ensinasse, e, se fiz uma besteira, será que foi
assim tão grave para eu ser tão humilhada? Não pude acreditar que isso tinha
acontecido. Nunca tinha sido tratada desse jeito antes; os animais sempre
retribuíram meu amor e nunca pensaram em machucar meu corpo ou envergonhar meu
orgulho. E era exatamente isso que tinha acontecido.

Me virei e fui embora, com o rosto entre as mãos, soluçando; meu sonho tinha
Desmoronado; ele não era o que eu imaginara; nunca mais queria vê-lo.

Não olhei para trás.

Estou em minha casa outra vez. Está chovendo e está escuro. Queria ter uma mãe. Os
outros têm. Nunca tinha sentido falta de uma antes.

\dia{Um mês depois} Essa ferida ficou comigo, nunca mais me aproximei
daquele lugar. Tinha decidido que ficaria absorta em meu trabalho e que iria
esquecê-lo. Fiz meu trabalho, mas não sentia mais o mesmo gosto que antes.
Fiz um monte de fósseis, mas não eram mais tão bons, pois meu coração não estava
ali. Alguns até passavam, mas a maioria ficou grosseira e nem um pouco artística;
precisava de um toque final. Eu os queria para o Quaternário, mas não eram bonitos
o suficiente para isso, fui obrigada a contragosto a colocá-los de volta na
categoria de Primários, onde obviamente eles nada mais são do que apócrifos e
inúteis, pois não fazem parte, e confundirão a ciência algum dia; mas o que devo
fazer? Jogar fora? Provavelmente não. Como invenções, até que não são tão
ruins. Montei um plantígrado hidrocefálico com traços de répteis e
moluscos e uma disposição evolucionária ao desenvolvimento de penas, que vai     
atrair a atenção em algum lugar mais adiante. Poderá até servir na reconciliação
entre a ciência e as escrituras, mas quanto mais olho para ele mais me convenço
de que terão de fazer concessões consideráveis de ambos os lados antes disso
(antes de chegarem a um meio-termo).

Esses longos e tristes dias --- como duraram, nossa! Apesar de mim, comecei
secretamente a desejar --- não importa o quê. Tive vergonha do que
senti e resisti. Um dia, na areia lisa da beira do lago, encontrei as seguintes
palavras:

“Me desculpa. Estou arrependido. Perdoa!”

Reconheci suas pegadas também. Por um momento senti o sangue pulsando em minhas
veias, e eu\ldots{}

E então lembrei novamente daquele dia, e aquele amargor voltou. Apaguei aquelas
palavras, fui embora e me escondi.

No dia seguinte voltei, implacável, ressentida, para apagá-las mais uma vez. Mas
para minha surpresa não havia mais nenhuma palavra ali para apagar. Fiquei amargamente ofendida.
Foi outra afronta. Disse a mim mesma que ele era uma pessoa sem caráter; para mim deu, não
quero mais saber dele.

No dia seguinte, o mesmo aconteceu --- nada para apagar. Estava envergonhada
por ter vindo, e disse que seria a última vez, nunca mais viria.

Mantive minha palavra. Eu fui, sim, mas não para ver se havia palavras: não me
dei conta de que estava lá até eu estar lá. Então fiquei procurando outras coisas.

Não havia mal nisso, e assim fui até lá todos os dias para procurar outras coisas.
Mas nunca achei nada. Até que um dia, mecanicamente, sem perceber
o que estava fazendo, eu mesma escrevi algumas coisas:

“Volta.”

Quando vi o que tinha feito, fiquei ainda mais envergonhada, e risquei as
palavras. Ficou assim:

“\st{Volta.}”

No outro dia, lá estavam. E nada mais. Nada, só algumas pegadas. E por três
dias lá estavam elas --- e mais nada. Só pegadas. Chorei, mesmo sem motivo --- 
e risquei completamente as palavras. Mais tarde, meio
aérea, escrevi de novo, e de novo meu orgulho se voltou contra mim, e
apaguei parcialmente: “\st{Volta}”.

E lá estavam elas no dia seguinte. E também uma flor, se é que a gente pode chamar
um ramo de capim de flor; era um \textit{dente-de-leão}. Fiquei arrepiada, mas
joguei o ramo fora, pois foi muito impertinente da parte dele fazer isso. À noite,
voltei lá e peguei o capim; não sei bem por quê, mas foi só um impulso --- pois nem
gosto de dente-de-leão.

No dia seguinte as palavras e as pegadas e um \textit{malmequer}. Fiquei com a
flor, apesar de meu impulso ser o de jogar fora. Estava me sentindo péssima e,
incomodada, apaguei as palavras. Então, escrevi de novo, mas
esqueci de riscar.

“Volta.”

No orvalho da manhã seguinte, fui correndo até o lago; meu coração batendo como
nunca; lá estavam elas --- as palavras --- só as palavras, nada mais!
As lágrimas rolaram, e uma fraqueza tomou conta de mim até eu desfalecer\ldots{} em
seus braços!

Foi um dia abençoado. “Vem para casa”, disse ele. “Essas terras de clima tão hostil
não são nada boas para minha Eva”. Então me beijou, e eu disse:

“Qualquer lugar onde Satã estiver é bom para mim.”

Ele riu, deu uma batidinha na minha bochecha e disse:

“Simploriazinha querida!”

“Por quê?” 

E foi assim que ele me disse quem era, e quem era Satã, e nós dois rimos e tagarelamos
sobre isso como as crianças bobas que éramos. Oh, juventude despreocupada! Oh,
juventude dourada. Oh, única coisa preciosa nesta vida tão desgastante. Damos
tão pouco valor enquanto temos, e como choramos ao te perder!

Diariamente caminhávamos pela relva do Éden de braços dados, e
Ele sempre me contava a mesma história, e ela sempre me parecia nova e
estimulante, sempre uma alegria para os meus ouvidos: contou o quanto sofrera,
naquele dia trágico, e que me seguiu a distância, esperando que eu me voltasse; e
foi morar perto de mim, durante todo esse tempo, nunca tendo voltado ao
Éden; e tentou nunca me perder de vista, sempre com medo de que eu o avistasse,
pensando que eu o odiava. E muitas foram as vezes em que o interrompi, pedindo que
repetisse algumas coisas que já tinha dito centenas de vezes:

“E quando foi que descobriu que me amava, meu amor?”

“Quando foi que escreveu aquelas palavras?”

“Está feliz, Adão?”

Ele calou minha boca em resposta, ms não com a mão.

“Mas eu riscava as palavras.”

“É, verdade; é, realmente. Quase dava para notar.”

Parecia uma brincadeira astuta, o modo astuto como ele disse aquilo. E rimos muito.

“Bom, mas então, se sabia ler, por que não voltou?”

“Por que estava tão seguro de si?”

“Tão seguro de mim?”

“Sim”, eu disse. “Ela me quer, mas tenta fazer de conta que está indiferente. Vou
esperar. Ela vai escrever novamente --- e de forma mais direta. E mais de uma vez. E
continuará escrevendo essas palavras até esquecer de riscar. E foi
\textit{assim}, franca, direta e sem perturbação a sua confissão!”     

“Mas como você é safado!”

“Mas bem charmoso.”

“Charmoso! Não para mim. Como você pôde
aguentar toda essa espera?”

``\textit{Foi} duro, mas tive que fazer você confessar. E houve também
circunstâncias modificadoras, que tornaram as coisas mais fáceis para
mim.''

“O que você quer dizer?”

“Ora, tive a sua companhia. O dia inteiro eu a segui e me alimentei da sua visão. Muitas
vezes estive tão perto de você que quase podia tocar. Às vezes cheguei a te tocar.”

“Tocar de verdade?”

Dá para dizer que sim. Até te beijei enquanto você dormia.”

“Ah se eu tivesse como saber! Mas foi melhor assim. Foi muito indelicado.”

“Foi criminoso, inclusive; foi um roubo. Chegue sua boca mais para cá, vou devolver cada um.”

E devolveu.

“Adão, se você me amou tanto, por que me presenteou com aquele dente-de-leão
tão sem graça?”

“Só para te testar, meu amor. Eu pensei: se ela ficar com as flores, vou saber
que também me ama.”

“A prova falhou! Eu joguei fora.”

“É, eu vi.”

“Coitadinho, me sinto mal agora. Como foi que você se sentiu?”

“Mal, muito mal.”

“Bom, você não vai mais precisar sofrer por causa disso. Vou confessar um segredo: eu
voltei lá e ajuntei as flores.”

“É, eu lembro. Eu estava lá.”

Então eu lhe dei uns tapinhas nas orelhas e rimos muito por causa dessas bobagens, 
como se nunca mais esse tipo de dor e sofrimento fosse abalar o mundo.

Amor, paz, conforto, contentamento desmedido --- assim era a vida no Paraíso. Era
um prazer estar vivo. Dor não havia. Nem enfermidades nem marcas físicas que
assinalassem a passagem do tempo; doenças, cuidado, sofrimento podem até ser     
sentidos do lado de lá, mas não no Éden. Lá não havia espaço para isso, e lá
nunca os avistamos. Todos os dias eram assim, todos como um sonho bom.

As distrações eram muitas, pois éramos crianças, e ignorantes; ignorantes numa
concepção diferente da de hoje. Não conhecíamos nada --- nada mesmo. Estávamos
começando do zero --- no começo do começo. Tínhamos que aprender o
\textit{abc} das coisas. Hoje as crianças com quatro anos
sabem de coisas que nós aos trinta ignorávamos. Pois éramos crianças sem
instrutores e babás. Não havia ninguém lá para nos explicar as coisas. Não havia
dicionário, e não tínhamos como saber se estávamos usando as palavras corretamente
ou não. Gostávamos das palavras compridas, e lembro que muitas vezes as
empregávamos porque soavam bem ou pela sua dignidade, mesmo desconhecendo o
significado; e quanto à ortografia, era desregrada. Mas não dávamos a mínima para
essas bobagens, tanto que tínhamos acumulado um vasto e pomposo vocabulário, mas
nem ligávamos para o sentido ou para os métodos.

Mas estudar, aprender e descobrir a causa, a natureza e o propósito de tudo que
cruzava nosso caminho eram nossas paixões, e essa pesquisa preenchia nossos dias
com um interesse vivo e contagiante. Adão era um cientista por natureza e por
inclinação; posso dizer, com acerto, que eu também era, e adorávamos nos chamar
usando esse grande nome. Ambos tínhamos a ambição de derrotar o outro em
descobertas científicas, e isso adicionou um estímulo a nossa amistosa
rivalidade, e efetivamente nos protegeu de cairmos na ociosidade e em ocupações frívolas
e menos frutíferas em nossa busca de prazer.

Nossa primeira descoberta científica memorável, a lei de que a água e outros
fluidos correm para baixo, não para cima. Foi Adão que descobriu isso. Ficou
dias conduzindo experimentos em segredo, não disse nada, pois
queria estar absolutamente certo antes de dizer qualquer coisa. Eu desconfiava que
algo de suma importância estava perturbando seu grande intelecto, pois que seu
sono era agitado, ele se mexia muito. Enfim ele teve certeza,
e só então me falou. Não pude acreditar, parecia tão estranho, tão
impossível. Minha surpresa foi o seu triunfo, sua recompensa. Ele me levou de
riacho em riacho, dezenas deles, dizendo sempre: “Então, pode ver que corre sempre
para baixo, em todos a correnteza vai para baixo, nunca para cima. Minha
teoria está certa, provada, estabelecida, nada pode contradizê-la”. E foi
realmente um grande prazer ver sua exaltação por causa da grande descoberta.

Nos dias de hoje nenhuma criança se maravilha vendo a água descer em vez de subir; mas
na época isso era algo grandioso e tão inacreditável quanto qualquer outro fato
com que deparei. Veja, aquela verdade simples estava debaixo do meu nariz desde o
primeiro dia da minha existência, mas nunca reparei. Levei algum tempo
para aceitar e me ajustar a ela, e por um bom tempo não podia olhar para uma
correnteza sem verificar voluntária ou involuntariamente se a água não subia mesmo
pela margem, meio que esperando ver a lei de Adão ser violada. Mas finalmente
me convenci, e assim foi. Daquele dia em diante eu ficaria muito surpresa e
perplexa se encontrasse uma queda-d’água indo na direção errada. O conhecimento se
adquire com muito trabalho; nada é de graça.

Aquela foi sua primeira grande contribuição à ciência --- e por mais de dois séculos
foi identificada por seu nome --- A lei de precipitação fluida de Adão. Qualquer um
o cativava mencionando casualmente a lei quando estava por perto. Estava um tanto
imodesto --- não vou tentar esconder ---, mas não chegou a ficar insuportável.
Nada chegou a deixá-lo arrogante; era tão querido e correto. Sempre fazia um gesto
com a mão como se sua descoberta não fosse tudo aquilo, e dizia que mais dia menos
dia outro cientista a descobriria. De qualquer modo, se um visitante ouvisse falar
dele e não tivesse tato suficiente para mencionar, mesmo que de leve, o acontecimento, era
certo que o estranho não seria mais convidado. Depois de uns dois séculos, a
descoberta dessa lei entrou em disputa e muito se brigou por ela durante uns cem anos,
e o crédito acabou sendo dado a alguém mais recente. Foi uma pancada
violenta. Adão nunca mais foi o mesmo. Ele carregou aquela mágoa
por mais de seiscentos anos, e acredito que isso foi o que encurtou sua vida.
Evidentemente durante seus dias ele foi preferido pelos reis de toda a raça,
sendo considerado o Primeiro Homem, e recebia as honras que vinham com esse grande
título. Porém, essas distinções não compensaram aquela lamentável violação, pois
foi um verdadeiro cientista, e o primeiro. Ele me segredou mais de uma vez que se
pudesse manter a glória da lei de precipitação de fluidos não teria se
importado de se passar por filho e tornar-se o segundo homem. Fiz o que pude para
confortá-lo. Disse que como Primeiro Homem sua fama estava garantida, e que
chegaria um tempo em que o nome do descobridor embusteiro seria apagado, morto e
esquecido sobre a face da terra. Eu realmente acreditava nisso. Nunca deixei de
acreditar. Esse dia, com certeza, ainda virá.

A segunda grande descoberta científica foi minha: entender como o leite é
introduzido na vaca. Ambos ficamos maravilhados com esse mistério
durante muito tempo. Seguimos as vacas por anos a fio --- quer dizer, durante o dia ---,
mas nunca as pegamos tomando algum tipo de líquido daquela cor. E assim
finalmente admitimos que faziam isso apenas durante a noite. Estabelecemos turnos para
observá-las à noite. O resultado foi o mesmo ---, e o quebra-cabeça seguiu sem
solução. Esse procedimento era esperado em principiantes; percebemos
agora que não eram muito científicos. Com o tempo, a experiência nos ensinou 
métodos mais eficientes. Uma noite, deitada à toa, olhando as estrelas, uma grande
ideia me passou pela cabeça, e encontrei meu caminho! Meu primeiro impulso era o de
acordar Adão e contar tudo, mas resisti e mantive segredo. Não dormi nada
o restante da noite. Ao primeiro sinal de que a madrugada se aproximava,
sorrateira fui até a floresta e, em uma clareira gramada, fiz um pequeno
cercado, bem seguro, dentro do qual prendi uma vaca. Tirei todo o leite dela, e
deixei-a presa lá. Não havia nada ali para beber --- ela devia conseguir
leite por alguma alquimia secreta, ou permaneceria seca.

Passei o dia inteiro inquieta, e não falava coisa com coisa de tão preocupada, mas
Adão estava bem ocupado também, tentando inventar uma tabela de multiplicação,
e nada percebeu. Ao anoitecer, ele tinha chegado a seis vezes nove são vinte e sete, e
embriagado de felicidade com sua façanha, ignorando tudo o que acontecia ao redor.
Fui ver a vaca. Minhas mãos tremiam tanto por medo de fracassar
que por alguns momentos não consegui segurar a teta com firmeza; por fim consegui,
e o leite saiu! Dois galões. Quase dez litros, sem ela ter de onde tirar o
líquido. Aí me veio a explicação: o leite não entrava pela boca, ele era
condensado da atmosfera através dos pelos dela. Corri para contar a Adão, e sua
alegria foi tão grande quanto a minha, seu orgulho de mim era inexprimível.

Ele disse:

“Você sabe que não fez só uma contribuição de peso e ressonância para a
ciência, mas duas.”

Era verdade. Através de uma série de experimentos chegamos à conclusão de que o
ar da atmosfera consistia de água em suspensão invisível, e de que os componentes da
água, hidrogênio e oxigênio, na proporção de dois daquele para um deste,
podiam ser expressados pelo símbolo H$_2$O. Minha descoberta revelou
que havia ainda outro ingrediente --- o leite. Ampliamos a sigla para
H$_2$OL.

\dia{diário} Outra descoberta. Percebi, um dia, que William McKinley não estava bem. Ele
é o leão original. E foi meu bicho de estimação desde o início. Examinei-o para
ver o que tinha de errado com ele e descobri que um repolho, que ele não tinha
mastigado bem, estava preso em sua garganta. Fui incapaz de retirar, por isso, peguei um
cabo de vassoura e empurrei o repolho para baixo. Isso o aliviou. Ao longo
dessas tentativas, fiz que abrisse bem as mandíbulas, e foi então que percebi
algo de peculiar em seus dentes. Submeti os dentes dele a um exame cuidadoso e
científico, e o resultado foi surpreendente: o leão não é vegetariano; ele é
carnívoro, come carne! Pelo menos foi projetado para isso.

Corri ao encontro de Adão e lhe contei, mas é claro que ele zombou de mim:

“Onde é que ele encontraria carne?”

Tive que admitir que não sabia.

“Pois então, podes ver que tua ideia e apócrifa. Carne não era para ser
comida, caso contrário teria sido fornecida. Não tendo sido fornecida,
conclui-se, obrigatoriamente, que carnívoros não foram introduzidos na ordem das
coisas. É uma dedução lógica, não é?”

“É sim.”

“Será que tem um ponto fraco nessa ordem?”

“Não.”

“Neste caso, o que você tem a dizer?”

“Que existe algo melhor que lógica.”

“Verdade? E o que seria?”

”Fatos.”

Chamei um leão e pedi que abrisse a boca.

”Olhe essa mandíbula superior, que enorme”, eu disse. “Esse dente longo curvado
para a frente não é um canino?”

Ele estava surpreso. E disse, impressionado:

“Macacos me mordam, não é que é?”

“E o que são esses quatro, atrás dele?”

“Pré-molares, ou meus dentes do juízo!”

“O que são aqueles dois lá atrás?”

“Molares, quando vejo um, sei se é ou não é. Não tenho mais nada a dizer. As
estatísticas não mentem. Essa besta não é um herbívoro.”

Ele é sempre assim. Nunca é mesquinho, invejoso. Sempre é justo, magnânimo. Se você
prova algo a ele, ele cede imediatamente de forma nobre e graciosa. Fico pensando se
sou digna de um rapaz tão maravilhoso, dessa criatura bela, de espírito tão
generoso.

Isso foi há uma semana. Examinamos um por um os animais da propriedade, e vimos que
está cheia de animais que até agora nem suspeitávamos serem carnívoros. De
certo modo é muito comovente, agora, ver um imponente tigre de Bengala comer
morangos e cebolas, parece tão pouco característico, tão fora do normal, mas nunca
me senti tão tocada.

Hoje, na floresta, ouvimos uma Voz.

Procuramos por ela, mas não a achamos. Adão disse que já a tinha ouvido, mas nunca
visto, apesar de parecer estar próximo dela. Por isso parecia estar certo de que
era algo como o ar, algo que não dava para ver. Pedi que ele me dissesse tudo que
sabia sobre a Voz, mas ele não sabia quase nada. A Voz era o Senhor do Jardim,
disse ele, que lhe tinha dito para decorar e cuidar do Jardim; disse também que
não deveríamos comer da fruta de certa árvore, que se comêssemos
certamente morreríamos. Nossa morte seria certa. Isso era tudo o que Adão sabia. Eu queria
ver essa árvore, e foi assim que fizemos uma bela e longa caminhada até onde ela
estava, numa clareira linda e afastada. Lá nos sentamos e
ficamos olhando para ela com muito interesse, e conversamos. Adão disse que essa
era a Árvore do Conhecimento do Bem e do Mal.

“Do bem e do mal?”

“Sim.”

“E o que é isso?”

“O que é o quê?”

“Ora, essas coisas. O que é o bem?”

“Não sei. Como poderia saber?”

“Tá bom, mas e o mal, o que é?”

“Imagino que seja o nome de algo, mas não sei o quê?”

“Mas Adão, você deve ter alguma ideia do que seja.”

“Por que eu deveria ter alguma ideia? Nunca vi essa coisa antes, como
poderia formar uma concepção do que é? O que você acha que pode ser?”

É claro que eu não tinha noção, era ilógico querer que ele tivesse.
Não havia como um de nós dois saber do que se tratava. Era uma palavra nova, como
a outra; nunca tínhamos ouvido, e nenhuma das duas significava nada para nós. 
Continuei matutando sobre o assunto, e perguntei:

“Adão, há também aquelas outras palavras novas --- “morrer”, e “morte”. O que
\textit{elas} significam?”

“Não tenho a menor ideia.”

“Sim, mas o que você acha que significam?”

“Minha filha, você não consegue ver que é impossível eu tentar adivinhar
coisas sobre as quais sou
completamente ignorante? Uma pessoa não consegue \textit{refletir} sobre uma coisa se
não tem subsídios \textit{com os quais} formular algum pensamento, não
é verdade?”

“É sim, eu sei disso. Mas olha que embaraçoso isso é: justamente por não conseguir
saber, mais \textit{quero} saber.”

Ficamos calados por um tempo, tentando analisar o quebra-cabeça de diversos
ângulos; de repente achei um jeito, e fiquei surpresa por não termos pensado
nisso desde o início, já que era tão simples. Levantei e disse:

“Como somos idiotas! Vamos comer da fruta. Aí morremos e ficamos sabendo o que é e
não teremos mais que esquentar a cabeça com isso.”

Adão achou que era a coisa certa a fazer. Levantou-se imediatamente para colher
uma maçã, quando de repente uma criatura curiosa veio se enleando, era de um
tipo que nunca tínhamos visto; claro, deixamos para lá o assunto sem interesse
científico especial para nos ocupar de algo que tinha.

Seguimos esse gnomo ondeado, que se movia lentamente pelos vales e montanhas, até
que chegamos lá embaixo, no lado sul do vale onde está a grande figueira de
Bengala; foi lá que o alcançamos. Que felicidade, que triunfo: ele é um
pterodáctilo. É um amor, tão feinho! E tem um temperamento e uma voz odiosos.
Chamamos dois tigres para nos levar de volta, mas o agarramos e trouxemos
conosco. Agora tenho-o comigo, e já é tarde, mas não consigo ir me deitar, pois é
um demônio fascinante, uma nobre contribuição para a ciência. Sei que não
conseguirei dormir pensando nele, querendo que o dia amanheça logo, para poder
explorá-lo e examiná-lo em detalhe e descobrir o segredo de sua origem, e
desvendar quanto dele é ave e quanto dele é réptil, e ver se é um dos mais fortes
entre os sobreviventes da espécie, o que sem dúvida deve ser, dá para ver só de
olhar para ele. Oh, ciência, diante de ti todos os outros interesses se esvanecem!

Adão acorda. Pede que eu não esqueça de registrar essas quatro novas palavras. Isso
mostra que ele mesmo já as esqueceu. Mas eu não. Para o bem dele, estou sempre
prestando atenção. Ele é que está montando o dicionário --- é o que ele pensa ---, mas
me dei conta de que o trabalho mesmo quem faz sou eu. Não tem problema, eu gosto
de fazer tudo o que ele quer que eu faça; e, quando se trata do dicionário,
tenho o maior prazer em fazer o trabalho, assim evito que ele seja humilhado,
pobrezinho. Sua ortografia não é muito científica. Ele soletra gato com \textit{q} e
gatimônias com \textit{g}, apesar de as duas palavras terem a mesma raiz.

\dia{Três dias depois} Demos-lhe o nome de Terry (à cobra), forma curta, e,
nossa, ele é um amor! Nesses três dias ficamos completamente absorvidos com ele.
Adão se pergunta como foi que a ciência se virou até aqui sem ele, e eu também
acho isso. O gato se aproveitou da situação sabendo que Terry, a cobra, era um
estranho, mas se arrependeu. Terry deu uma surra nele de cabo a rabo, o que o
deixou num estado duvidoso no que diz respeito ao pelo, e foi assim que se
retirou com ar de quem tinha levado o maior susto, estava decidido a
ir embora e pensar sobre o assunto para entender por que deu tudo errado. Terry
é simplesmente demais --- não há outra criatura igual a ele. Adão o examinou
detalhadamente e tem certeza de que ele é o sobrevivente da espécie. Acho que Thomas
pensa diferente.

\dia{diário, Ano 3} No começo de julho, Adão notou que um peixe no lago estava
desenvolvendo pernas --- um da família das baleias. Apesar de não ser uma
baleia de fato, como se estivesse em um estado menos evoluído. Era um girino.
Observamos com muito interesse, pois se as pernas realmente crescessem e se
tornassem utilizáveis, seria nosso propósito fazer com que outros peixes também as 
desenvolvessem, para poder sair da água e caminhar com mais liberdade.
Muitas vezes nos preocupamos com essas criaturas, sempre molhadas e sem conforto,
sempre restritas à água, enquanto os outros animais podiam sair e brincar entre
as flores e se divertir. Em pouco tempo as pernas se desenvolveram, e a baleia
tornou-se um sapo. Veio para a beira e ficou pulando e cantando, bem feliz,
principalmente à noite, sua gratidão era sem limites. Muitos outros o seguiram, e
em breve tínhamos música em abundância, o que representava uma grande melhoria,
em comparação com o silêncio de antes.

Trouxemos muitos tipos de peixes para a beira e os soltamos na relva, mas foi uma
decepção só: nenhum desenvolveu pernas. Era estranho, não conseguíamos entender.
Em uma semana todos tinham voltado para a água, e não conseguíamos explicar,
pareciam mais satisfeitos lá do que na terra. Tomamos isso como evidência de
que os peixes, em geral, gostam mais da água do que da terra, e de que, com exceção
das baleias, nenhum deles se interessou pela terra. Havia algumas baleias de grande
porte em um lago bastante grande a uns quinhentos quilômetros vale acima, e Adão foi até
lá com a ideia de fazer uma criação e aumentar seu bem-estar.

Quando ele estava fora há uma semana, nasceu Caim. Foi uma surpresa grande para
mim, pois não tinha noção de que algo estava para acontecer. Mas é como Adão sempre
diz: “O inesperado sempre acontece”.

No começo não sabia o que pensar. Achei que fosse um animal. Mas,
examinando mais detalhadamente, dava para ver que não era, já que não tinha nem
dentes nem pelos, e parecia um desses filhotinhos abandonados. Alguns detalhes
pareciam de humanos, mas não o suficiente para justificar que cientificamente o
classificássemos nessa categoria. Assim, inicialmente o consideramos um
\textit{lusus naturae} --- uma aberração ---, e foi necessário deixar por
isso mesmo e esperar para ver como se desenvolvia.

Logo comecei a ficar mais interessada nele, e esse interesse aumentava a cada
dia, até que foi assumindo um tom mais caloroso, tornou-se afeto, depois
amor, depois idolatria, e minha alma toda ficou absorvida na criatura, e fui
consumida por um sentimento de gratidão e alegria. A vida tinha se transformado em
felicidade, arrebatamento, êxtase; eu esperava ansiosa, dia após dia, contava as horas e
os minutos até Adão voltar, para compartilhar essa
felicidade quase insuportável.

\dia{Ano 4-5} Finalmente ele veio, mas não imaginava que fosse uma criança. Estava
bem-intencionado, tratava-o com carinho e amor, mas era antes de mais
nada um cientista, e só em segundo lugar um homem; isso era da sua natureza. Mas não
tirava nenhuma conclusão a menos que fosse cientificamente provada. As
situações alarmantes por que passei durante os doze meses seguintes por causa dos
experimentos desse estudioso são indescritíveis. Ele expôs a criança a toda
sorte de desconforto e inconveniente imaginável a fim de determinar que tipo de
ave, réptil ou quadrúpede ela era e para que servia, e assim tive de segui-lo
dia e noite, receosa e desesperada para apaziguar as dores do pobrezinho e ajudá-lo a
suportá-las da melhor forma possível. Adão acreditava que eu o tinha encontrado na
floresta, e eu estava feliz e grata por ele pensar assim, pois a ideia de
sair de vez em quando para caçar outro da mesma espécie o deixava bem
entusiasmado, o que daria à criança e a mim períodos abençoados de repouso e
tranquilidade. Ninguém pode imaginar o alívio que eu sentia quando ele parava com
seus experimentos aflitivos, reunia armadilhas e iscas e se mandava para a
floresta. Assim que a gente o perdia de vista, eu abraçava meu tesouro e o
sufocava com beijos, e chorava de gratidão. Essa pobre coisinha parecia se dar
conta de que algo ótimo nos tinha acontecido, pois esperneava e gritava de alegria,
abria aquela boca que era pura gengiva, e sorria um sorriso alegre de orelha a
orelha, ou sei lá o que eram aquelas coisas presas na lateral.

\dia{Ano 10} Então veio nosso filho Abel. Acho que tínhamos um ano ou dois quando Caim
nasceu, e três ou três e meio quando Abel se juntou a nós. Foi por aí que Adão
começou a entender. Gradualmente seus experimentos foram ficando mais leves, e
finalmente, um ano depois do nascimento de Gladys e Edwina --- nos anos 5 e 6, ele
parou completamente. Passou a amar as crianças com paixão depois de conseguir
classificá-las cientificamente, e de lá até agora a felicidade no Éden é
perfeita.

Temos nove filhos, a metade meninos, e a outra, meninas.

Caim e Abel estão começando a aprender. Caim já sabe somar tão bem quanto eu, e
consegue multiplicar e subtrair algumas coisas. Abel não é tão rápido quanto o
irmão mentalmente, mas tem persistência e parece compensar a rapidez. Abel aprende
sobre muitas coisas em três horas como Caim, mas Caim fica brincando durante duas
dessas três horas. Assim, como Adão diz, “Abel fica um bom tempo na estrada,
mas também chega no horário.” Adão concluiu que persistência é um talento, e
classificou essa palavra nessa categoria em seu dicionário. Soletrar também é
um talento, estou certa disso. Com toda a inteligência de Caim, ele não conseguiu
aprender a soletrar. O fato de ser como o pai, o mais inteligente
de todos, e não saber ortografia é uma calamidade. Eu sei soletrar, e Abel
também. Esses fatos não provam nada, pois não se pode deduzir um
princípio a partir de alguns exemplos apenas, mas pelo menos indicam que a capacidade
de aprender a escrever é um talento inato, sem deixar de
ser um sinal de inferioridade intelectual. Paridade de raciocínio, a falta dela é    
um sinal de grande poder mental. Às vezes, quando Adão finalmente conseguiu que uma
palavra tão longa quanto “raciocinação” passasse pela sua moenda e está lá enxugando o
suor, poderia adorá-lo por ser tão intelectualmente incrível, surpreendente e
sublime. Ele consegue soletrar “fhísica” de várias formas, e não só daquele jeito mais comum.

Caim e Abel são dois garotos bem queridos, e tomam conta dos irmãozinhos e
irmãzinhas. Os quatro mais velhos da ninhada vão a toda parte, conforme a vontade, e
às vezes, por mais de dois ou três dias, não vemos nem sombra deles. Uma vez
perderam a Gladys e voltaram sem ela. Não conseguiam lembrar
quando e onde a tinham perdido. Era longe, disseram, mas não sabiam quão longe,
era uma região desconhecida para eles, rica em frutinhas
silvestres, daquelas que chamamos de “sombra fatal” --- não sei por quê. Não tem
significado nenhum, mas tem na composição uma daquelas palavras usadas pela Voz, e
adoramos empregar novas palavras toda vez que surge uma oportunidade, assim
tornamos essas palavras funcionais e convenientes. Eles são fãs dessas frutinhas,
e ficaram muito tempo às voltas com elas, comendo, e foi numa dessas ocasiões que, com o passar do
tempo, perceberam que Gladys não estava com eles, e não respondia quando
começaram a chamar.

No dia seguinte ainda não tinha aparecido. Nem no seguinte, nem depois.
Três dias e ainda não tinha voltado. Foi muito estranho. Nada parecido
acontecera até então. Isso despertou nossa curiosidade. Adão achava
que se não aparecesse até o dia seguinte, ou pelo menos até dois dias depois,     
deveríamos mandar Caim e Abel atrás dela.

E assim fizemos. Ficaram fora três dias, mas a encontraram. Passou por algumas
aventuras. No escuro, na primeira noite, ela caiu em um rio e foi levada pela
correnteza até muito longe, não sabia dizer quão longe, mas finalmente foi
jogada num banco de areia. Depois disso, viveu com uma família de cangurus
que a entreteve de forma bem hospitaleira, muito sociável. A mãe canguru
era dócil e materna, tirava os bebês da pochete e saía, descendo e subindo
por vales para coletar frutas e nozes das mais finas e saborosas, e quase todas
as noites eles recebiam a visita de ursos e coelhos e águias e galinhas, raposas e
hienas, touros e outras criaturas --- divertiam-se a valer. Os animais tinham
pena da nossa garotinha porque não tinha pelo; quando dormia, sempre a cobriam
com folhas e barba-de-velho para proteger sua pele delicada, e foi assim que os
meninos a encontraram. Primeiro ela ficou doente de saudade, mas depois
melhorou.

Essa era a expressão dela, “doente de saudade”. Incluímos a expressão no dicionário, e
já estamos chegando a um acordo sobre o que significa. É composta de duas palavras
que já tínhamos, com sentido próprio, mas que juntas aparentemente ainda não
tinham significado nenhum. Compilar um dicionário é muito interessante,
mas difícil, como diz Adão.

\dia{Ano 15} As crianças prometem. Edwina faz bonecas convincentes de cenouras
bifurcadas, com galhinhos atravessados no tórax que servem de braços, e um
rabanete como cabeça. Gladys ajuda o pai a entalhar elefantes e mastodontes em
ossos, e Abel ajuda-o a fazer facas e ponteiras de flecha de pedras para o 
sambaqui. Caim é o mais esperto de todos. Ele é ótimo para fazer os 
fósseis mais simples, e logo vai assumir essa tarefa, eu acho. Ele inventou um 
fóssil sozinho.

\chapter{Solilóquio de Adão}
\hedramarkboth{solilóquio de adão}{mark twain}
\medskip

\noindent (\textit{Inspecionando o dinossauro no Museu de História Natural.})

Estranho\ldots{} muito estranho. Não me lembro dessa criatura. (\textit{Depois de olhar longamente, admirado}).
É, ele é realmente maravilhoso! Um mero esqueleto de dezessete metros de comprimento por
cinco de altura! Até agora, que eu saiba, encontraram apenas um dessa espécie --- sem
dúvida esse é um de tamanho médio; uma pessoa certamente não iria topar com um
desse por acaso ao sair daqui e entrar no parque, como se fosse o maior de todos
os cavalos da América; não, ele toparia com um que pareceria pequeno ao lado do
maior da Normandia. É muito provável que o maior dos dinossauros tivesse uns vinte e sete
metros de comprimento e uns sete de altura. Ele seria cinco vezes mais comprido que
um elefante; um elefante seria para ele o que um terneiro seria para um elefante.
Que volume tem essa criatura! Que peso! Tão comprido como a maior das baleias, mas
tem duas vezes o seu volume. E provavelmente uma carne boa, tipo a suína, carne
suficiente para alimentar uma vila durante um ano inteiro\ldots{} Imagine cem deles
enfileirados, drapeados com uma manta dourada! --- algo majestoso para uma procissão
de coroação. Mas caro, pois certamente come demais, só pessoas da corte e
milionários poderiam bancar um.

Realmente não me lembro dele; nem Eva nem eu ouvimos falar dele antes de ontem.
Falamos com Noé sobre ele; Noé enrubesceu e mudou de conversa. Voltando ao
assunto, e pressionando um pouco ele finalmente confessou que quanto à
estocagem da Arca as estipulações não tinham sido seguidas ao pé da letra em alguns
pequenos detalhes que não eram vitais. Havia algumas irregularidades. Ele disse
que a culpa era dos meninos --- principalmente os filhos, mas parcialmente também
por indulgência dele mesmo. Naquela época eles estavam no auge da inconstância da
juventude, na alegre primavera da vida, seus cem anos quase não se faziam
sentir, e\ldots{} bem, ele também já fora um menino, por isso não teve pulso para ser
tão exigente com eles. E assim\ldots{} bem, eles fizeram o que não deviam,
e ele --- para ser camarada --- fazia de conta que não via. Mas no geral foram
fiéis ao trabalho, considerando a idade. Coletaram e armazenaram um bom
estoque dos animais realmente úteis, e também, quando Noé não estava vendo, uma
multidão de animais menos úteis, como moscas, mosquitos, cobras e assim por
diante, e realmente acabaram deixando na margem uma boa quantidade de animais que
possivelmente tiveram algum valor em uma ou outra época. Estes
eram principalmente os sáurios de trinta metros de comprimento, e monstruosos
mamíferos, como o megatério e similares, e eles tinham realmente uma desculpa para
deixá-los para trás, duas razões na verdade: (1) estava claro que em dado momento
seriam necessários como fósseis para museus; e (2) houve um problema de cálculo. A
Arca ficou menor do que devia, não tinha espaço para todas as
criaturas. Na verdade, só de material fóssil havia o suficiente para fretar vinte e cinco
arcas como a nossa. Quanto ao dinossauro, a consciência de Noé estava tranquila,
ele não constava da lista de carga, e os meninos não sabiam que
tal criatura existia. Disse que não se culparia por não saber da existência dele,
porque o dinossauro era um animal americano, e a América não tinha sido descoberta
ainda.

Noé continuou dizendo que “tinha sim repreendido os meninos por não terem
maximizado o espaço. E por terem descartado alguns animais que não
prestavam e substituído outros, como o mastodonte, que poderia ter sido útil ao
homem na lida mais pesada, fazendo o que os elefantes fazem. Mas eles disseram que
esses animais teriam aumentado o trabalho deles além de suas forças no que diz
respeito ao trato e à alimentação, uma vez que a mão de obra era escassa.
Nisso meus filhos tinham razão. Não tínhamos uma bomba, e havia apenas uma janela;
tínhamos que soltar um balde por ela e depois içá-lo de volta por uns quinze metros
mais ou menos, era muito cansativo; aí ainda tínhamos que carregar a
água até o porão --- mais quinze metros ---, quando a água era para os elefantes e
outros da família, uma vez que os mantínhamos lá embaixo, no porão, para dar mais
lastro à embarcação. Por causa daquelas condições, perdemos muitos animais,
principalmente os mais seletos, que teriam valido uma fortuna em coleções, como os
diferentes tipos de leões, tigres, hienas, lobos e assim por diante, que já não
bebiam mais água depois que ela se misturou com a água salgada. Mas nunca
perdemos um gafanhoto ou um grilo sequer, nem mesmo um gorgulho, um rato, um germe
da cólera, nem ao menos um desses míseros seres. No frigir dos ovos, acho que nos
saímos bem. Éramos pastores e agricultores, nunca tínhamos estado no mar,
éramos ignorantes em assuntos navais, e sei com certeza que há mais
diferença entre agricultura e navegação do que as pessoas podem imaginar. Penso que
estes dois ofícios não devem se misturar. Sem pensa o mesmo, Jafé
também. Quanto ao que Cam pensa, isso não é importante. Cam é tendencioso; 
quero ver você me apontar um presbiteriano que não o seja”.     

Disse isso de forma agressiva; havia ali um quê de desafio. Evitei a
provocação mudando de assunto. Para Noé, a discussão é uma paixão, uma doença, que
está tomando conta dele há mais de trinta e tantos mil anos, se não mais, e o está
tornando impopular, desagradável; muitos dos seus velhos amigos têm medo de
cruzar com ele. Até mesmo estranhos logo começam a evitá-lo, mesmo que
inicialmente fiquem felizes em conhecê-lo e o admirem por conta da sua tão
celebrada aventura. Por um tempo ficam orgulhosos de receber sua atenção, já que
é uma pessoa tão distinta; mas ele os tritura com seus argumentos, e logo eles começam
a desejar --- como o restante --- que algo tivesse acontecido com a Arca.

\sectionitem

\noindent (\textit{Sentado num banco no parque, no meio da tarde, ocioso, observando a passagem
dessas espécies para lá e para cá.})

E pensar que essa multidão não é mais que uma
pequena fração da população da terra! Todos são meus parentes de sangue, cada um
deles! Eva devia ter vindo comigo; isso teria aguçado seu coração afetuoso,
ela nunca soube manter a compostura ao encontrar com algum parente --- sempre saía
beijando um por um, pretos, brancos e os demais.

(\textit{passa um carrinho de bebê.})

Mudou tão pouco --- nada na verdade. Eu me lembro bem do nosso primeiro filho --- deixe-me
ver\ldots{} vai fechar trezentos mil anos agora na terça; e esse aí é como aquele
primeiro. Então, entre aquele primeiro bebê e esse agora, não há nada que os
diferencie. A mesma falta de cabelo, a mesma ausência de dentes, o mesmo corpinho
frágil, esse aparente vazio mental e essa mesma falta de atrativos. E mesmo assim
Eva adorava aquele primeiro, e era bonito vê-la com ele. A mãe desse aí
também adora o seu; vê-se isso no olhar dela, que brilha tal qual o de Eva naquela
época. E pensar que algo tão sutil e intangível quanto um olhar poderia migrar de
rosto em rosto durante trezentos mil anos e permanecer igual, sem um traço de
mudança! E aqui está, iluminando o rosto dessa jovem criatura do mesmo modo como
iluminou o rosto de Eva tanto tempo atrás --- a criatura mais jovem que já vi e a
mais antiga, diante de mim. É claro, o dinossauro, mas esse pertence a outra
espécie.

Ela aproximou o carrinho do banco, sentou-se e começou a empurrá-lo para a frente
e para trás com uma mão enquanto segurava o jornal com a outra, absorta na
leitura. “Minha nossa!”, ela exclamou, o que me surpreendeu e me fez
perguntar, modesta e respeitosamente, o que tinha acontecido. Ela,
educada, me mostrou o jornal e disse, apontando com o dedo:

“Ali, parece um fato, mas não sei se é.”

Foi desconcertante. Tentei disfarçar, e despretensiosamente virei o jornal frente e
verso, mas ela me olhou de um jeito, percebi que não consegui disfarçar. De
repente ela perguntou, meio hesitante:

“O senhor não sabe ler?”

Tive que confessar que não. Ela ficou admiradíssima. Mas isso teve um efeito
agradável --- ela se interessou por mim, e eu fiquei grato, pois já estava me
sentindo sozinho, sem ter com quem conversar. O jovem que fazia as vezes
de guia para mim --- por iniciativa própria, já que não lhe pedi --- acabou
perdendo um compromisso no museu, e eu fiquei triste quando se foi, pois ele tinha    
sido uma boa companhia. Quando disse à jovem que não sabia ler, ela fez outra
pergunta embaraçosa:

“De onde o senhor vem?”

Disfarcei --- para ganhar tempo e me posicionar --- e disse:

“Adivinhe. Vamos ver quão perto consegue chegar.”

Ela se iluminou e exclamou:

“Vou adorar, se o senhor não se importar. E se eu adivinhar
o senhor vai me dizer?”

“Sim.”

“Jura pela sua honra?”

“Pela minha honra? O que isso quer dizer?”

Ela riu prazerosamente e disse:

“Esse é um bom começo. Eu tinha \textit{certeza} de que ia gostar dessa expressão.
Agora, uma coisa eu sei. Eu sei\ldots{}”

“Sabe o quê?”

“Que o senhor não é americano. Não vai me dizer que é, é?”

“Não, você tem razão. Não sou. Pela minha honra, como vocês dizem.”

Ela parecia bem satisfeita consigo mesma, e disse:

“Sei que nem sempre sou esperta, mas \textit{isso} foi muito esperto. Mas também
nem tão esperto assim, pois já estava imaginando que o senhor fosse estrangeiro por
causa de outro detalhezinho.”

“Qual?”

“Seu sotaque.”

Ela era uma observadora perspicaz; eu falo, de fato, com um sotaque, vamos dizer
assim\ldots{} divino, e ela conseguiu notar esse “saborzinho” de um sotaque de fora. E
prosseguiu, cheia de charme, inocentemente satisfeita com seu triunfo:

“No momento em que o senhor disse ‘Vamos ver quão perto consegue chegar’, eu disse a mim mesma:
‘Aposto que é estrangeiro, e aposto o dobro que é inglês’. Essa é sua
nacionalidade, não é?”

Me senti mal por ter de acabar com sua vitória, mas era preciso:

“Hum, você vai ter que tentar de novo.”

“O quê, o senhor não é inglês?”

“Não, pela minha honra.”

Ela me observou de cima a baixo, evidentemente pensando, juntando
uma informação com outra, e então disse:

“É, a bem da verdade, o senhor não parece mesmo inglês.” Então acrescentou: “O fato
é que o senhor não se parece com nenhum estrangeiro --- com ninguém que eu, ao menos,
tenha cruzado antes. Vou tentar adivinhar outra vez”.

Ela foi dizendo o nome de cada país de que conhecia, e aos poucos foi
desanimando. Finalmente disse:

“Você deve ser O Homem sem Pátria. Aquele da história. O senhor parece não ter
nacionalidade. Como veio parar na América? Tem parentes aqui?”

“Sim, muitos.”

“Ah, então veio para vê-los.”

“É, mais ou menos.”

Ela ficou um tempo sentada, pensando, e então disse:

“Não vou desistir ainda. Onde o senhor mora quando está em casa --- na cidade grande ou no
interior?”

“O que você acha?”

Bom, não sei bem. O senhor me parece meio interiorano, se me permite,
mas também tem alguma coisa da cidade, não muito, sei lá, apesar
de não saber ler, o que é muito curioso, e além disso não está acostumado a
jornais. Assim, diria que vive principalmente no interior quando está em casa, e
só de vez em quando fica na cidade. Acertei?”

“Isso mesmo.”

“Ótimo, assim posso tentar de novo.”

Ela acabou ficando exausta de tanto nomear cidades. Sem sucesso. Depois quis que eu a
ajudasse dando algumas dicas, como ela disse, perguntando se minha cidade era
grande. Sim. Muito grande? Sim. E eles têm móveis lá? Não. E luz elétrica? Não. E
estradas de ferro, hospitais, faculdades, polícia? Não.

“Mas então não é um lugar civilizado! Onde será que fica? Seja legal e
diga ao menos uma peculiaridade de lá, para eu poder adivinhar.”

“Muito bem. Só uma: o lugar tem portais feitos de pérolas.”

“Ah, essa não. Isso é em Nova Jerusalém. Essa brincadeira não é justa. Mas pode
deixar, vou adivinhar assim mesmo. Já vai me ocorrer, quando eu menos
esperar. Ah, tive uma ideia. Fale um pouco em sua língua --- isso pode ser uma boa
pista.”

Fiz o obséquio de dizer uma ou duas frases. Ela balançou a cabeça,
desesperançada.

“Não, isso não soa humano”, disse. “Quer dizer, não soa como nenhuma dessas
línguas estrangeiras. É até bonito, bem bonito, mas sei que nunca ouvi
nada nessa língua. E se o senhor pronunciasse o seu nome --- qual é o seu nome,
afinal, será que pode me dizer?”

“Adão.”

“Adão?”

“Sim.”

“Mas Adão do quê?”

“É só isso --- só Adão.”

“Nada mais, só isso?”

“É, nada mais que isso.”

“Nossa, que interessante. Tem tanto Adão por aí; como distinguir uns dos outros?”

“Isso não é problema; lá de onde eu venho, sou o único.”

“Jura? Agora o senhor acabou comigo. Me faz até pensar que é aquele, o
ancestral, o original. Esse era o nome dele também. Ele também não tinha
sobrenome, como o senhor.” E então, maliciosamente ela disse: “Imagino que já ouviu falar
dele.”

“Oh, sim. Você o conhece? Já o viu alguma vez?”

“Se já o vi? Se vi o Adão? Graças a Deus não. Eu morreria de susto.”

“Não vejo por quê.”

“Não?”

“Não.”

“Como o senhor não vê por quê?”

“Porque não há motivo para alguém se assustar com um parente.”

“Parente?”

“Ora, ele não seria um parente seu, distante?”

Ela achou isso muito divertido, e disse que era realmente verdade,
mas que nunca teria sido esperta o suficiente para pensar nisso. Eu estava
achando essa uma sensação nova e muito agradável, a de ser admirado, e estava a
ponto de fazer novas brincadeiras quando um rapaz apareceu. Ele se
plantou ao lado da jovem e fez um comentário chocho sobre o tempo, mas ela o
olhou meio de lado, levantou, pegou o carrinho do bebê e foi embora.


