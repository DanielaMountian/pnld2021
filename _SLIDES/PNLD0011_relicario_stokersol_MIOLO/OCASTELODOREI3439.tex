%!TEX root=LIVRO.tex
\chapter{O castelo do Rei}

Quando contaram ao pobre Poeta que aquela que mais amava jazia
enferma à sombra do perigo, ele ficou à beira da loucura.

Durante semanas estivera sozinho; ela, sua Esposa, fora para longe, para
seu velho lar, a fim de ver um velho ancestral antes de ele morrer.

Por alguns dias, o coração do Poeta estivera oprimido por uma estranha
tristeza. Não sabia a sua causa; só sabia, com a profunda empatia
que é o dom do poeta, que aquela que ele amava estava doente. Esperou
ansiosamente por notícias. Quando as novidades chegaram, o choque,
embora já esperasse uma mensagem triste, fora demais para ele, e
ficou à beira da loucura.

Em sua tristeza e ansiedade, saiu ao jardim que, por longos anos, havia
cultivado para ela. Ali, entre flores resplandecentes, onde as velhas
estátuas suavemente brancas se erguiam, com as cercas de teixo ao fundo,
ele se deitou na grama de verão, alta e não cortada, e chorou com sua
cabeça enterrada no chão.

Pensou em todo o passado -- sobre como havia conquistado sua Esposa
e como eles se amavam; e lhe parecia uma coisa triste e cruel que ela
estivesse longe e em perigo, e que ele não estivesse perto para
confortá"-la ou mesmo compartilhar sua dor.

Muitos, muitos pensamentos lhe voltaram contando a história de anos
fatigantes, cuja melancolia e solidão ele havia esquecido no resplendor
de seu amável lar\ldots{}

De como na juventude eles, o par, haviam se conhecido e, num instante,
amado um ao outro. De como a pobreza dele e a grandeza dela os haviam
mantido separados. De como ele lutara e batalhara na estrada íngreme e
pedregosa rumo à fama e à fortuna.

De como, por todos aqueles anos fatigantes, lutara com a 
ideia única de conseguir um tal lugar na história de seu tempo que o fizesse
capaz de se aproximar e dizer para ela ``eu te amo'', e para seus
parentes orgulhosos ``sou digno, pois também me fiz grande''.

De como, em meio a todo esse sonhar com um tempo feliz que poderia vir,
ele se manteve silente quanto a seu amor. De como nunca a vira ou ouvira
a voz dela, ou mesmo conhecera sua morada, para evitar que, sabendo,
falhasse no propósito de sua vida.

De como o tempo -- tal qual sempre acontece com aqueles que trabalham
com honestidade e obstinação -- coroou as labutas e a paciência em sua
vida.

De como o mundo chegou a conhecer e reverenciar e amar seu nome como o
de alguém que, por seu exemplo, ajudara os fracos e os cansados; de
alguém que purificava os pensamentos de todos os que escutavam suas
palavras; e de alguém que havia varrido a baixeza ante a grandeza e a
simplicidade de seus nobres pensamentos.

De como o sucesso se seguira ao despertar da fama.

De como, enfim, no seu coração timorato pela dúvida do amor, nascera o
pensamento de que finalmente alcançara a grandeza que justificava
sua busca pela mão daquela que amava.

De como havia retornado à sua terra natal, e lá ainda a encontrara
desimpedida.

De como, quando ousara contar a ela sobre seu amor, ela lhe
sussurrara que também havia esperado todos aqueles anos, pois sabia que
no fim ele viria reivindicá"-la.

De como ela havia vindo com ele, como sua noiva, para o lar que ele
estivera construindo para ela por todos esses anos. De como, ali,
viveram felizes; e ousaram olhar atentamente aos longos anos por vir em
busca de alegria e contentamento sem limites.

De como ele pensou que, mesmo um pouco enfraquecido em sua força pelo
incessante trabalho dos anos e pela preocupação da esperança, podia
vislumbrar tempos felizes por vir.

Mas, ah!, que esperança; pois quem sabe o que o futuro pode trazer?
Somente há pouco tempo sua Amada o deixara saudável, partindo por causa
do dever; e, agora, ela estava doente, sem tê"-lo por perto para ajudar.

Todo o sol de sua vida parecia estar desvanecendo. Todos os longos anos
de espera e a paciente persistência na virtude que havia coroado seus
anos com o amor não pareciam mais do que um sonho efêmero, e tudo fora em
vão -- tudo, tudo em vão.

Agora, com a sombra pairando sobre sua Amada, a nuvem parecia estar
acima e em volta deles, e contendo nela, em seus recessos ofuscados, a
perdição de ambos.

``Por quê? Oh, por quê?'', perguntou o pobre Poeta ao ar invisível, ``o
amor veio a nós? Por que paz e alegria e felicidade, se as turvas asas do
perigo ensombrecem o ar em torno dela e me deixam só, a chorar?''

Assim lamentou, e delirou, e chorou; e amargas horas o atravessaram
em sua solidão.

Deitado no jardim com sua face enterrada na grama alta,
vieram a ele e disseram, chorando, que notícias -- tristes, de fato --
haviam chegado.

Enquanto falavam, ele levantou sua pobre cabeça e os fitou; e eles viram
nos olhos grandes, escuros e tenros que agora ficara um tanto
perturbado. Sorriu triste para eles, como se não estivesse entendendo
bem o significado das palavras deles. Tão ternamente quanto puderam,
tentaram dizer"-lhe que aquela que ele mais amava estava morta.

\imagemmedia{}{./img/16.png}

Disseram:

``Ela andou pelo Vale das Sombras'', mas ele pareceu não os compreender.

Sussurraram:

``Ela ouviu a Música das Esferas'', mas ele ainda não os compreendia.

Falaram"-lhe pesarosos, e disseram:

``Ela agora reside no Castelo do Rei.''

Ele lhes dirigiu um olhar ávido, como que para perguntar:

``Que castelo? Que rei?''

Arquearam suas cabeças; e, enquanto se viravam, chorando, murmuraram
suavemente:

``O Castelo do Rei da Morte.''

Ele nada disse; então, viraram de volta suas faces chorosas. Viram que
ele havia se levantado e estava de pé, com um firme propósito em seu
rosto. Então, disse suavemente:

``Vou encontrá"-la; porque lá onde ela mora eu também posso morar.''

Eles lhe disseram:

``Você não pode ir. Ela está além do Portal, no Reino da Morte.''

Um propósito firme cintilou nos olhos sérios e amáveis do Poeta enquanto
ele lhes respondia pela última vez:

``Aonde ela foi, para lá eu vou também. Pelo Vale das Sombras farei meu
caminho. Nestes ouvidos também soará a Música das Esferas. Procurarei e
encontrarei minha Amada nos Salões do Castelo do Rei. Abraçá"-la"-ei firme
-- mesmo diante da face medonha do Rei da Morte.''

Quando ouviram essas palavras, abaixaram suas cabeças novamente,
choraram e disseram:

``Ai! Ai!''

O poeta virou"-se e os deixou; e foi embora. Teriam"-no seguido de bom
grado; mas, com um gesto, pediu ali ficassem. Então, sozinho, em seu
pesar, partiu.

Enquanto andava, virou"-se e balançou sua mão num gesto de adeus.
Então, por alguns instantes, deteve"-se com a mão levantada, e a girou
lentamente em todas as direções.

De repente, sua mão estendida parou e apontou. Seus amigos, olhando"-o,
viram onde, para além do Portal, a imensidão indolente se espraiava.
Ali, no meio da desolação, a névoa dos pântanos se suspendia como um
manto de trevas no horizonte longínquo.

Quando o Poeta apontou para lá, havia um brilho de felicidade -- muito,
muito fraco -- em seus olhos pobres e tristes, enlouquecidos pela
perda, como se, ao longe, ele contemplasse algum sinal ou esperança da
perdida.

Rápida e tristemente, o Poeta viajou pelo dia escaldante.

A Hora de Descanso chegou, mas ele continuou a jornada. Não parou nem
atrás de sombra, nem atrás de descanso. Nunca, nem mesmo por um instante, parou
para esfriar seus lábios ressecados com um gole gelado das fontes
cristalinas.

Os viajantes fatigados, descansando em sombras frescas ao lado das
fontes, levantavam suas cabeças estafadas e olhavam"-no com olhos
modorrentos quando ele passava apressado. Ele não lhes prestava
atenção, e continuava sempre em frente com um propósito firme em seus
olhos, como se alguma faísca de esperança, irrompendo das névoas dos
pântanos distantes, o encorajasse.

Assim viajou por todo o dia escaldante, e por toda a noite silenciosa.
De manhã bem cedo, quando a promessa do sol ainda não nascido despertava
o céu oriental com uma luz pálida, ele se aproximou do Portal. O
horizonte sobressaía sombrio na luz fria da manhã.

Lá, como sempre, estavam os Anjos que mantinham guarda e vigilância, e,
oh!, impressionante!, apesar de invisíveis a olhos humanos, eram
vistos pelo Poeta.

Quando ele se aproximou, eles o fitaram com pena e abriram bastante suas
asas, como que para abrigá"-lo. Ele falou; e, de seu coração atormentado,
as tristes palavras saíram docemente de seus lábios pálidos:

``Dizei, Vós que guardais o Reino, minha Amada passou por aqui em
jornada para o Vale das Sombras, para ouvir a Música das Esferas, e para
habitar no Castelo do Rei?''

Os Anjos no Portal inclinaram suas cabeças em sinal de assentimento.
Eles se viraram e olharam para fora do Reino, para onde, longe, na
vastidão ociosa, as úmidas névoas rastejavam, vindas do coração inerte
do pântano.

Eles bem sabiam que o pobre e solitário Poeta estava em busca de sua
Amada; então, não o impediram, nem o encorajaram a ficar. Tiveram
pena dele, muita, por ele amar demais.

Abriram caminho para que ele pudesse passar pelo Portal sem obstáculos.

Assim, o Poeta seguiu em frente para o deserto desolado a fim de procurar
sua Amada no Castelo do Rei.

Durante algum tempo, passou por jardins cuja beleza era mais perfeita
que a dos jardins do Reino. A doçura de todas as coisas penetrava nos
sentidos como odores das Ilhas dos Abençoados.

A sutileza do Rei da Morte, que reina nos Domínios do Mal, é grande. Ele
ordenou que o caminho além do Portal fosse feito com muitos encantos.
Assim, aqueles que se desviam dos caminhos consagrados ao bem sempre
encontram em torno de si tanta beleza que, em sua alegria, a melancolia
e a crueldade e a culpa do deserto são esquecidas.

Mas, à medida que o Poeta seguia adiante, essa beleza começava a
desvanecer.

Os belos jardins se pareciam com jardins dos quais foi retirada a mão do
cuidado, e cujas ervas em abundância sufocam, à medida que nascem e
crescem, a vida das mais finas flores.

De aleias gélidas sob galhos esparramados, e da relva viçosa que tocava
tão suave quanto veludo os pés dolorosos do Viajante, a estrada se
tornou um caminho pedregoso e árido, completamente aberta aos raios
abrasantes do sol. As flores começaram a perder seu odor e a definhar,
impedidas de crescer. Grandes moitas de cicuta se elevavam por todos os
lados, infectando o ar com seu odor fétido.

Grandes fungos cresciam nos buracos escuros nos quais jaziam poças de
água salobra. Árvores altas, com galhos como esqueletos, erguiam"-se --
árvores que não tinham folhas, e parar sob suas sombras significava
morrer.

Então, rochas enormes barravam o caminho. Só se podia atravessá"-las por
passagens estreitas e tortuosas, em penhascos que pendiam como maciços
e ameaçavam desabar e engolfar o Visitante a qualquer momento.

Aqui, a noite começou a cair, e a névoa turva subindo dos pântanos
longínquos tomou formas estranhas de coisas sombrias. Na solidez
distante das montanhas, os animais selvagens começaram a rugir em suas
tocas cavernosas. O ar se tornara medonho com os sons apavorantes da
estação noturna.

Mas o pobre Poeta não prestava atenção às visões e aos sons maléficos do
temor. Seguia sempre em frente -- não pensava nos terrores da noite.
Para ele, não havia medo da escuridão -- nenhum medo da morte -- nenhum
pensamento no horror. Procurava por sua Amada no Castelo do Rei; e,
nessa ávida busca, todos os terrores naturais eram esquecidos.

Assim ele seguia adiante, através da noite infindável. Andava pelo desfiladeiro
acima. Pelas sombras das grandes rochas, ileso, ele
passava. Os animais selvagens cercavam"-no rugindo furiosamente -- seus
grandes olhos flamejando como estrelas ardentes na escuridão da
noite.

Das rochas altas, grandes jiboias rastejavam e se penduravam para
capturar sua presa. Das fendas do escarpado das montanhas, e de fissuras
cavernosas do caminho rochoso, serpentes venenosas rastejavam e se
erguiam para dar o bote.

Mas, embora chegassem bem perto, todas as coisas nocivas se abstinham de
atacar, pois sabiam que o Visitante solitário estava indo para o Castelo
do Rei.

Ele rumava mais adiante, mais adiante -- incessante -- sem parar pelo
caminho --, mas sempre avançando em sua busca.

Quando por fim a luz do dia irrompeu, o sol se levantou sobre uma visão
desoladora. Ali, lutando para avançar pelo caminho rochoso, o pobre e
solitário Poeta seguia sempre em frente, não dando atenção ao frio ou à
fome ou à dor.

Seus pés estavam descalços, e suas pegadas no caminho polvilhado de
pedras estavam marcadas de sangue. Em volta e atrás dele, longe, mantendo
o mesmo passo nos cumes da cadeia de montanhas, vinham os animais
selvagens que olhavam"-no como a uma presa, mas que se abstinham de
tocá"-lo porque ele buscava o Castelo de seu Rei.

No ar rodopiavam os pássaros repugnantes, que sempre seguem o rastro dos
moribundos e dos perdidos. Pairavam os abutres de pescoços nus, com
olhos ávidos e bicos famintos. Suas grandes asas batiam preguiçosamente
no ar parado enquanto seguiam o rastro do Viajante. Os abutres são um
povo paciente e aguardam sua presa sucumbir.

Saindo dos recessos cavernosos nos vales estreitos da montanha negra,
rastejavam, velozes e em silêncio, as serpentes que ali espreitam. Veio
a jiboia, com suas dobras colossais e caracóis intermináveis, de onde a
pequena cabeça chata observa com olhos perspicazes. Com toda sua
tribo, veio a sucuri, que captura sua presa pela força e a esmaga com a
temível rigidez de seu abraço. Vieram as najas e todas aquelas que com
seu veneno destroem suas presas. Aqui, também, vieram aquelas serpentes,
as mais terríveis de todas para suas presas -- as que fascinam com olhos
de estranha magia e com a graça lenta de sua abordagem.

Aqui, vieram ou ficaram à espera cobras dissimuladas, que tomam a cor da
erva, ou da folha, ou do galho morto, ou da poça viscosa, e que
ficam espreitando nesses lugares para atacar suas presas desatentas.

Havia grandes serpentes trepadeiras, de corpo ágil, dessas que se
penduram em rochas ou galhos. Segurando"-se firme em seu apoio à
distância, dão o bote para baixo com a rapidez da luz, quando,
como chicotes, arremessam de longe seus corpos sobre suas presas.

Assim, apareceram todas essas coisas nocivas para encontrar o Homem
em Busca e para tomá"-lo de assalto. Porém, quando tomavam conhecimento
de que ele estava indo para o terrível Castelo de seu Rei, e viam como
seguia em frente sem medo, elas se abstinham de atacar.

A fatal jiboia e a sucuri, erguendo"-se em dobras colossais, ficaram
passivas e, dessa vez, mantiveram"-se quietas como pedra. As najas
retraíram novamente suas presas venenosas. Os olhos sedutores e
profundamente intensos da cobra encantadora ficaram pálidos e abatidos
quando ela sentiu que seu poder de atração era ineficaz. E, em meio a seu
bote letal, a cobra trepadeira deteve seu curso, e ficou dependurada na
rocha ou no galho como uma linha frouxa.

Muitos seguiam o Viajante no deserto selvagem, aguardando e esperando
uma chance de destruí"-lo.

Muitos outros perigos também se apresentavam ao pobre Viajante no ócio
do deserto. À medida que avançava, o caminho rochoso se tornava mais
íngreme e mais escuro. Fumaças lúridas e névoas gélidas e mortais se
erguiam.

Então, nesse caminho pela vastidão indiscernível, apareciam coisas estranhas
e terríveis.

Mandrágoras -- metade planta, metade homem -- berravam para ele com um
grito desesperador, agudo, quando, sem conseguir fazer o mal, esticavam
seus braços medonhos em vão.

Espinhos gigantes cresciam pelo caminho; eles perfuravam seus pés
sofridos e rasgavam sua carne enquanto ele seguia em frente. O Poeta
sentia a dor, mas não lhe dava atenção.

Durante toda a longa e terrível jornada, só teve um único pensamento
que não o de sua ávida busca pela Amada. Pensou que os filhos dos
homens poderiam aprender muito com a jornada para o Castelo do Rei, que
começara tão bela entre jardins perfumados e sob a sombra fresca das
árvores espraiadas. Em seu coração, o Poeta falou à multidão dos filhos
dos homens; e de seus lábios as palavras fluíram como música, pois
compôs uma canção sobre o Portão Dourado que os Anjos chamam de
\textsc{Verdade}.

\begin{verse}
Não passe o Portal do Reino do Pôr do Sol, não!\\

Pare onde os Anjos em sua vigília estão.\\

Cuidado! Mesmo estando abertos os portões não passe,\\*

Do lado de cá, seguro, relaxe.\\!

Ainda que jardins perfumados e caminhos frescos chamem,\\

Os vales da noite mais sombrios lá jazem.\\

Descanse! Descanse contente. Pare, ainda imaculado,\\*

Não procure os horrores do deserto desolado.\\
\end{verse}

Assim, esmagando todos os obstáculos com seus pés que sangravam, seguiu
sempre em frente o pobre e perturbado Poeta para procurar sua Amada no
Castelo do Rei.

E à medida que seguia adiante, até mesmo a vida animal parecia morrer
atrás de si. Os chacais e os animais selvagens mais covardes se safavam.
Leões e tigres, e ursos, e lobos, e todos os mais corajosos entre os
ferozes animais de caça, que seguiam seu rastro até mesmo depois de os
outros terem parado, agora começavam a hesitar em sua perseguição.

Eles rosnavam baixo e então rugiam alto com as cabeças levantadas; os
pelos eriçados de suas bocas se agitavam irados, e os grandes dentes
brancos rangiam nervosamente em raiva aturdida. Continuavam
um pouco mais, e paravam novamente, rugindo e rosnando como antes.
Então, um a um, pararam, e o pobre Poeta continuou sozinho.

No ar, os abutres rodopiavam e crocitavam, parando e hesitando em seus
voos. Como os animais selvagens, eles também pararam, após um longo
tempo, de seguir no ar o Viajante em seu caminho.

Por mais tempo que todos, seguiam em frente as cobras. Contorcendo"-se
muito e deslizando furtivamente, seguiam bem de perto os passos do
Homem em Busca. Nas marcas de sangue de seus pés sobre as duras rochas
elas encontravam alegria e esperança, e continuavam a segui"-lo sem parar.

Mas chegou a hora em que o aspecto horrível dos lugares pelos quais o
Poeta passava deteve até mesmo o rastejo das serpentes -- os
desfiladeiros sombrios de onde saem ventos venenosos que varrem com
desolação até as tocas dos animais de rapina -- a rigidez estéril que
marcha sobre os vales da desolação. Aqui, as próprias serpentes
sorrateiras pararam seu curso, e também desapareceram gradualmente.
Retornaram deslizando, sorrindo com um rancor mortal, às suas rachaduras
repulsivas.

Então chegaram lugares em que as plantas e o verdor começaram a
desaparecer. As próprias ervas se tornaram mais e mais atrofiadas e
inanes. Mais além, definharam até ficarem com a esterilidade de rochas
inanimadas. Então, as ervas mais nocivas, que cresciam em formas
medonhas de trevas e terror, perderam até mesmo o poder que
normalmente sobrevive depois que elas morrem. Definhadas e atrofiadas
até mesmo do mal, elas condensaram"-se como uma pedra morta. Aqui, até mesmo
a Figueira mortal não conseguia lançar raízes na terra pestífera.

Então chegaram lugares em que, na entrada do Vale das Sombras, até as
coisas sólidas perdiam sua substância, e se derretiam em névoas pútridas
e geladas que flutuavam ao redor.

Enquanto passava, o Poeta ensandecido não conseguia sentir terra firme
sob seus pés ensanguentados. Andava nas sombras, e no meio delas,
seguia em frente, através do Vale das Sombras, a fim de procurar sua Amada no
Castelo do Rei.

O Vale das Sombras parecia ter uma extensão interminável. Circundado por
névoas abundantes, olho algum poderia penetrar onde se erguiam as
grandes montanhas entre as quais o Vale ficava.

No entanto, lá estavam elas -- a Montanha do Desespero de um lado, e a
Colina do Medo do outro.

Até aqui o pobre cérebro perplexo do Poeta não havia percebido todos os
perigos, e horrores, e dores que o circundavam -- exceto, somente, a
lição que lhe ensinavam. Mas agora, perdido como estava no vapor amortalhado
do Vale das Sombras, ele não conseguia pensar em nada mais que os
terrores do caminho. Estava cercado por fantasmas pavorosos, que de vez
em quando se erguiam silenciosos na névoa, e se perdiam novamente antes
que ele pudesse apreender totalmente seu sentido horrível.

Então, lampejou através de sua alma um pensamento terrível.

Seria possível que sua Amada tivesse viajado para lá? Haviam"-na
acometido as dores que faziam tremer em agonia seu próprio estado de
espírito? Era mesmo necessário que ela tivesse sido aterrorizada por
todos aqueles horrores ao redor?

Ao pensar em sua Amada, sofrendo tanta dor e medo, soltou um grito
amargo que soou por toda a solidão -- que partiu o vapor do Vale e ecoou
nas cavernas das montanhas do Desespero e do Medo.

O grito selvagem, prolongado pela agonia na alma do Poeta, soou por todo
o Vale, até que as sombras que o povoavam despertaram temporariamente
para a vida"-na"-morte. Elas voavam rápida e indistintamente, agora
desvanecendo e logo depois se lançando novamente à vida -- até que o
Vale das Sombras ficou todo povoado por fantasmas despertos.

Oh!, naquela hora houve agonia na alma do pobre Poeta ensandecido.

Mas logo em seguida veio uma tranquilidade. Quando o susto de sua
primeira agonia passou, o Poeta se deu conta de que aos Mortos não
chegam os horrores da jornada que ele empreendeu. É somente para os
Vivos que existe o horror da passagem até o Castelo do Rei. Com esse
pensamento, veio"-lhe uma tal paz que até mesmo ali -- no escuro Vale das
Sombras -- insinuou"-se uma música suave, que soou na escuridão do
deserto como a Música das Esferas.

Então o pobre Poeta se lembrou do que lhe haviam dito; que sua Amada
havia percorrido o Vale das Sombras, que conhecia a Música das Esferas,
e que habitava o Castelo do Rei. Então ele pensou que, como estava no
Vale das Sombras, e como podia escutar a Música das Esferas, logo
deveria enxergar o Castelo do Rei, onde sua Amada habitava. Assim,
continuou esperançoso.

Mas, ai!, aquela mesma esperança era uma nova dor, da qual ele antes nada sabia.

Até ali, havia caminhado cegamente, não se importando para onde
ia ou o que se aproximava dele, contanto que seguisse adiante em sua
busca; mas agora a escuridão e o perigo do caminho guardavam novos
terrores, e por isso o Poeta ficou a imaginar como eles poderiam deter
seu curso. Tais pensamentos tornavam o caminho de fato longo, pois os
momentos pareciam durar toda uma era de esperança. Avidamente, procurou
pelo vindouro fim, quando, além do Vale das Sombras pelo qual
viajava, enxergaria erguidas as torres altas do Castelo do Rei.

O desespero parecia crescer nele; e, à medida que crescia, soava, sempre
mais alta, a Música das Esferas.

Adiante, sempre adiante, precipitou"-se com pressa furiosa o pobre e
ensandecido Poeta. As sombras turvas que povoavam a névoa recuavam
quando ele passava, estendendo"-lhe mãos de alerta, com dedos longos e
sombrios de um frio mortal. No silêncio amargo do momento, elas pareciam
dizer:

``Volte! Volte!''

Cada vez mais alto soava agora a Música das Esferas. Cada vez mais
rápido, com pressa furiosa, febril, corria o Poeta, no meio das Sombras
recuantes do vale sombrio. As sombras que ali povoavam, à medida que
desapareciam à sua frente, pareciam lamuriar um alerta pesaroso:

``Volte! Volte!''

Em seus ouvidos ainda soava, incessante, a crescente turbulência da
música.

Cada vez mais rápido, corria adiante; até que, por fim, a extenuada
natureza cedeu, e ele caiu de bruços na terra, desacordado, sangrando, e
sozinho.
\smallskip
Depois de um tempo -- que ele não podia nem mesmo supor quanto foi --,
despertou de seu desmaio.

Por um momento, não conseguiu imaginar onde estava; e seus sentidos
dispersos eram incapazes de ajudá"-lo.

Tudo era sombra e frio e tristeza. Uma solidão reinava ao seu redor,
mais mortal do que qualquer coisa com a qual já tivesse sonhado. Não
havia brisa no ar; nenhum movimento de uma nuvem que passasse. Nenhuma
voz ou barulho de ser vivo da terra, ou da água, ou do ar. Nenhum
farfalhar de folha ou balançar de galho -- tudo estava silencioso,
morto e abandonado. Entre as eternas colinas de sombra ao redor jazia o
vale desprovido de tudo o que vive e cresce.

As névoas flutuantes, com sua multidão de sombras, haviam ficado para
trás. Até mesmo os terrores apavorantes do deserto não estavam lá. O
Poeta, quando mirou ao redor de si, em sua completa solidão, desejou o
ímpeto da tempestade ou o estrondo da avalanche para romper o horror
pavoroso das trevas silenciosas.

Então, o Poeta percebeu que havia chegado ali depois de atravessar o
Vale das Sombras; que, embora assustado e enlouquecido, ouvira a
Música das Esferas. Pensava arduamente nisso ao andar pelo
desolado Reino da Morte.

Olhou em volta de si, temendo não ver em lugar algum o terrível Castelo
do Rei, onde sua Amada habitava; e berrou quando o medo de seu coração
encontrou voz:

``Não aqui! Oh, não aqui, no meio desta horrível solidão.''

Então, em meio ao silêncio circundante, sobre colinas distantes, suas
palavras ecoaram:

``Não aqui! Oh, não aqui!'', até que, com o eco e o reverberar do eco na
rocha, aquele ermo morto foi povoado de vozes.
\smallskip
De repente, as vozes do eco cessaram.

Do céu lúrido acima irrompeu o som terrível do estrondo de um trovão.
Reverberou pelos céus distantes. Bem longe, sobre o anel infinito do
horizonte gris, arrojou"-se -- indo e voltando -- estrepitando --
crescendo -- desaparecendo. Atravessou o éter, murmurando um som
ominoso, como se fizesse ameaças, e em seguida estalou com a voz de uma
pavorosa ordem.

De seu rugido veio um som semelhante a uma palavra:

``Adiante.''

O Poeta caiu de joelhos e recebeu com lágrimas de alegria o som do
trovão, que havia arrebatado, como um Poder de Cima, a desolação
silenciosa do ermo. O trovão lhe disse que, dentro e acima do Vale das
Sombras, propagavam"-se os poderosos sons do comando dos Céus.

Então o Poeta ficou de pé e, com o coração renovado, continuou adiante,
penetrando no ermo.

À medida que caminhava, o ressoar do trovão ia se extinguindo aos
poucos, e, novamente, o silêncio da desolação reinou sozinho.

Assim o tempo passou aos poucos; mas nunca chegava o descanso para os pés
fatigados. Adiante, ainda adiante ele seguia, com uma única memória a
animá"-lo -- o eco em seus ouvidos do trovão cujo estrondo havia
reverberado pelo Vale da Desolação:

``Adiante! Adiante!''

A estrada se tornava menos e menos rochosa conforme ele seguia em
seu caminho. Os grandes penhascos diminuíam e se encolhiam, e a
vegetação do brejo já subia até o sopé da montanha.

Após um longo tempo, as colinas e os desvãos das fortalezas da montanha
desapareceram. O Viajante seguiu seu caminho por entre ruínas
indiscerníveis, nas quais não havia nada exceto o chão
vacilante de pântanos e brejos.

Adiante, adiante ele vagueou, tropeçando cegamente com pés fatigados na
estrada sem fim.

Sobre sua alma pairava cada vez mais próxima a escuridão do desespero.
Enquanto estivera vagando entre as gargantas da montanha, em certa
medida era encorajado pela esperança de que, a qualquer momento,
alguma curva no caminho pudesse lhe mostrar o fim de sua jornada. Alguma
entrada de um desfiladeiro escuro poderia lhe desvelar, agigantando"-se
na distância -- ou mesmo próximo a ele --, o terrível Castelo do Rei.
Mas, agora, com a desolação monótona do pântano silencioso à sua volta,
percebeu que o Castelo não poderia existir sem que ele o visse.

Ficou por um tempo ereto, e se virou lentamente para que a volta
completa do horizonte fosse abarcada por seus olhos ávidos. Ai, ele não
viu coisa alguma! E lá não havia nada exceto a linha escura do
horizonte, onde a terra triste se encontrava com o céu homogêneo. Tudo,
tudo estava condensado em trevas silenciosas.

Cambaleou para mais adiante. Sua respiração ficou rápida e dificultosa.
Seus membros fatigados tremiam ao mantê"-lo fragilmente em pé. Sua força
-- sua vida -- estava diminuindo depressa.

Em frente, em frente, ele corria, sempre em frente, com uma ideia
desesperadamente fixa em sua pobre mente ensandecida: no Castelo do Rei
encontraria sua Amada.

Tropeçou e caiu. Nenhum obstáculo detinha seus pés. Era somente por sua
própria fraqueza que sucumbira.

Rapidamente se levantou e seguiu adiante com pés ligeiros. Temia que, se
caísse, talvez não fosse capaz de se levantar de novo.

Novamente caiu. Novamente se levantou e continuou seu caminho
desesperadamente, com um objetivo cego.

Assim, por um tempo continuou a avançar, tropeçando e caindo, mas se
erguia sempre e não parava no caminho. Continuou a busca por sua
Amada, que morava no Castelo do Rei.

Por fim, ficou tão fraco que, quando desabou, não pôde mais se levantar.

Ficou cada vez mais fraco enquanto jazia de bruços; e sobre seus olhos
ávidos pairou o véu da morte.

Mas mesmo então veio o conforto, pois ele sabia que sua corrida havia
acabado, e que logo encontraria sua Amada nos Salões do Castelo do
Rei.

Ao ermo ele contou seus pensamentos. Sua voz saiu com um som fraco, como
o suave uivo que antecede um vendaval passando por juncos no outono
gris:

``Mais um pouco. Logo a encontrarei nos Salões do Rei, e não nos
separaremos mais. É por isso que vale a pena passar pelo Vale das
Sombras e escutar a Música das Esferas com sua esperança dorida. Qual é
a vantagem, se o Castelo fica longe? Rápidos correm os pés dos mortos.
Ao espírito fugaz, toda distância é somente um átimo. Não temo ver agora
o Castelo do Rei; pois lá, dentro de seu Salão principal, logo
encontrarei minha Amada -- para não mais separar dela.''

Enquanto falava, sentiu que o fim estava próximo.

Do pântano em frente a ele vinha uma névoa imóvel que se espalhava. Ela se
ergueu silenciosamente, mais alto -- mais alto --, envolvendo todo o
vasto ermo ao redor. Tomava matizes mais profundos e mais escuros à
medida que se erguia. Era como se o Espírito das Trevas estivesse
escondido ali dentro, e se tornasse mais potente com o vapor que se
espalhava.

Aos olhos do moribundo Poeta, a névoa que pairava era um castelo
sombrio. Elevavam"-se o torreame e a lúgubre torre principal. O portão de
entrada, com seus recessos cavernosos e suas torres salientes, tinha a
forma de uma caveira. As ameias distantes erguiam"-se altas, penetrando o
ar silencioso. Bem naquele chão sobre o qual o Poeta jazia extenuado,
começava, turva e escura, uma vasta trilha que levava à penumbra dos
portões do Castelo.

O Poeta moribundo ergueu sua cabeça e observou. Seus olhos tão
enfraquecidos, animados pelo amor e pela esperança de seu espírito,
trespassaram os muros negros da fortaleza e os terrores sombrios dos
portões.

\imagemgrande{}{./img/29.png}


Ali, dentro do grande Salão em que o próprio Rei dos Terrores, severo,
reúne sua corte, ele a viu, aquela que procurava. Ela estava nas
fileiras daqueles que esperam pacientemente por seus Amados para seguir
com eles ao Reino da Morte.

O Poeta percebeu que só precisava esperar mais alguns breves instantes,
e ele era paciente -- abatido, no entanto, jazia em meio às Solidões
Eternas.

De longe, além do distante horizonte, veio uma luz fraca, como a da
manhã de um dia vindouro.

À medida que brilhou mais forte, o Castelo se destacou mais e mais
claramente; até que, na manhã desperta, ele se revelou em toda sua
extensão gélida.

O Poeta moribundo soube que o fim estava próximo. Com um último esforço,
pôs"-se de pé, pois que ereto e destemido, como manda a hombridade,
poderia então se encontrar face a face com o severo Rei da Morte diante
dos olhos de sua Amada.

Ao longe, o sol do dia nascente se erguia sobre o contorno do horizonte.

Um raio de luz disparou para cima.

Quando o raio atingiu o cume da torre principal do Castelo, o Espírito
do Poeta, no tempo de um instante, flutuou pela trilha à sua frente.
Flutuou através do portal fantasmagórico do Castelo, e encontrou com
alegria o Espírito gêmeo que amava diante da própria face do Rei da
Morte.

Mais rápido do que o lampejo de um raio, todo o Castelo derreteu no
nada; e o sol do dia vindouro brilhou calmamente sobre as Solidões
Eternas.

No Reino dentro do Portal nasceu o sol do dia vindouro. Brilhou calmo e
vivamente num belo jardim, onde, em meio à grama alta do verão, jazia o
Poeta, mais frio do que as estátuas de mármore à sua volta.
