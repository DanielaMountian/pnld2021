\chapterspecial{O Príncipe da Rosa}{}{}
 

Há muito, muito tempo -- tanto tempo que, se alguém tenta pensar tão no
passado, é ainda há mais tempo --, o Rei Mago reinava o País Sob o Pôr
do Sol.

Ele era um rei velho, e sua barba branca crescera tão longa que quase
tocava o chão. E~ele passou todo seu reinado tentando fazer seu povo
feliz.

Ele tinha um filho de quem gostava muito. Esse filho, o Príncipe Zaphir,
era muito merecedor do afeto de seu pai pois ele era tão bom quanto se
podia ser.

Ele ainda era um garoto, e nunca tinha visto sua mãe, que tivera um belo
e doce semblante e que morrera quando ele era apenas um bebê. Amiúde
ficava muito triste por não ter tido mãe, quando pensava que os outros
garotos tinham mães carinhosas, em cujos joelhos eles aprenderam a
rezar, e que vinham lhes dar beijos à noite em suas camas. Ele sentia
que era estranho muitas das pessoas pobres nos domínios de seu pai terem
mães, enquanto ele, o príncipe, não tinha.

\imagemmedia{}{./img/01.png}

Quando ele pensava assim, tornava"-se muito humilde; pois ele sabia que
nenhum poder, ou riqueza, ou juventude, ou beleza salvaria qualquer
pessoa da danação de todos os mortais, e que a única coisa bela no
mundo, cuja beleza dura para sempre, é uma alma pura e justa. Ele sempre
lembrava, entretanto, que, se ele não tinha mãe, tinha um pai que o
amava muito e, assim, ficava consolado e contente.

Ele costumava refletir muito sobre diversas coisas; e frequentemente,
até mesmo durante a luminosa hora do descanso, quando todas as pessoas
dormiam, ele ia para o bosque perto do palácio e pensava e pensava sobre
tudo o que era belo e verdadeiro, enquanto seu fiel cão Gomus se
agachava a seus pés e, algumas vezes, balançava a cauda, como se para
dizer:

``Aqui estou eu, príncipe. Também não estou dormindo''.

O Príncipe Zaphir era tão bom e gentil que nunca machucou qualquer coisa
viva. Se ele via na sua frente um verme rastejando sobre a estrada,
passaria cuidadosamente acima dele para que não o machucasse. Se ele
visse uma mosca que caíra na água, ele a levantaria com cuidado e a
soltaria, livre, ao ar glorioso: tão bom era ele que todos os animais
que uma vez o tinham visto o reconheciam, e quando ele se sentava em seu
lugar favorito no bosque, ali se elevava um zumbido alegre de todos os
seres vivos. Aqueles insetos brilhantes, cujas cores mudam de hora em
hora, mostravam suas cores mais claras e aqueciam"-se no cintilar da luz
do sol que penetrava oblíqua entre os galhos das árvores. Os insetos
ruidosos colocavam seus abafadores para que não o perturbassem; e os
passarinhos descansando nas árvores abriam seus olhos redondos e
brilhantes, e saíam, e piscavam os olhos à luz, e entoavam canções
jubilosas de boas"-vindas com todas as suas notas mais doces.

Assim ocorre sempre com pessoas afáveis e amáveis. Os seres vivos que
têm vozes tão doces quanto as de um homem ou de uma mulher e que têm
idiomas próprios, apesar de não os podermos entender, todos falam com
eles com notas alegres e desejam boas"-vindas de suas maneiras próprias e
belas.

O Rei Mago tinha orgulho de seu garoto corajoso, bom e belo, e gostava
de vesti"-lo muito bem. E~todas as pessoas adoravam observar seu rosto
límpido e sua vestimenta vistosa. O~Rei mandou os grandes mercadores
procurarem perto e longe até que conseguiram a maior e mais fina pena
que já fora vista. Ele mandou colocar essa pena na frente de um belo
chapéu, da cor de um rubi, e fixou"-a com um broche feito de um diamante
grande. Ele deu esse chapéu a Zaphir em seu aniversário.

Enquanto o Príncipe Zaphir andava entremeando as árvores, as pessoas
viam a distância a grande pluma branca acenando. Todos ficavam alegres
quando a viam, e corriam às janelas e às portas, reverenciando, sorrindo
e agitando as mãos enquanto seu belo príncipe passava. Zaphir sempre
reverenciava e sorria de volta; e ele amava seu povo e glorificava"-se no
amor que eles tinham por ele.

Na Corte do Rei Mago havia uma companheira para Zaphir, quem ele amava
muito. Era a Princesa Bluebell. Ela era filha de outro rei que fora
injustamente privado de seu domínio por um inimigo cruel e traidor, e que
havia procurado o Rei Mago pedindo ajuda e morrera em sua Corte depois
de viver ali por muitos, muitos anos. Mas o Rei Mago havia acolhido sua
filhinha órfã e criou"-a como sua própria filha.

Uma grande vingança havia recaído sobre o usurpador maléfico. Os
Gigantes haviam atacado seus domínios e assassinaram"-no e toda a sua
família, e mataram todas as pessoas do reino, e destruíram até mesmo
todos os animais, exceto os selvagens, que eram como os próprios
Gigantes. Então as casas começaram a vir abaixo devido à velhice e à
deterioração, e os belos jardins tornaram"-se selvagens e abandonados. E~assim, quando depois de muitos longos anos os Gigantes se cansaram e
voltaram para seus lares longínquos, o país que a Princesa Bluebell
tinha era uma desolação tão vasta que ninguém que lá entrasse saberia
que ali já haviam morado pessoas.

A Princesa Bluebell era muito jovem e muito, muito bela. Ela, como o
Príncipe Zaphir, nunca conhecera o amor materno, pois sua mãe, também,
havia morrido quando ela era jovem. Ela amava muito o Rei Mago, mas ela
amava o Príncipe Zaphir mais do que todo o resto do mundo. Eles sempre
haviam sido companheiros, e não havia um pensamento sequer do coração
dele que ela não conhecesse antes de lhe ocorrer. O~Príncipe Zaphir
amava"-a também, mais ternamente do que podem dizer as palavras, e por
ela ele faria qualquer coisa, sem importar o tamanho do perigo. Ele
tinha esperança de que, quando ele fosse um homem e ela uma mulher, ela
se casaria com ele, e ajudariam o Rei Mago a reinar em seu domínio justa
e sabiamente, e tinha esperança de que não haveria dor ou pobreza por
todo o país se pudessem evitar.

O Rei Mago mandou fazer dois pequenos tronos; e quando ele se sentou
cerimoniosamente em seu grande trono, as duas crianças sentaram"-se de
cada lado dele, e aprenderam como ser Rei e Rainha.

A Princesa Bluebell usava um robe de arminho como o de uma Rainha, e um
pequeno cetro e uma pequena coroa, e o Príncipe Zaphir tinha uma espada,
tão brilhante quanto um raio de luz, pendurada em uma bainha dourada.

Atrás do trono do Rei os cortesãos costumavam se reunir. E~muitos deles
eram notáveis e bons, e outros eram somente vaidade e egoísmo.

Havia Phlosbos, o Primeiro Ministro, um homem muito, muito velho com uma
barba longa parecida com seda branca, que carregava um bastão branco com
um anel de ouro nele.

Havia Janisar, o Capitão da Guarda, com bigodes impetuosos e uma
armadura pesada como vestimenta.

E então havia Tufto, um cortesão antigo, um velho tolo que não fazia
nada senão vaguear em torno dos grandes nobres e prestar"-lhes
deferência; e todos, de alto a baixo, desprezavam"-no muito. Ele era
gordo e não tinha cabelos ou pelos no rosto, nem mesmo sobrancelhas; ele
parecia -- oh!, tão engraçado com sua grande cabeça calva bem branca e
macia.

Havia Sartorius, um cortesão jovem e tonto, que pensava que a roupa era
a coisa mais importante do mundo, e que se vestia apropriadamente nas
melhores roupas que conseguia. Mas as pessoas apenas sorriam e até mesmo
riam dele, pois não há honra advinda de roupas bonitas, mas somente do
que está no próprio homem que as veste. Sartorius sempre tentava se
colocar à frente em toda parte a fim de exibir suas belas roupas, e
pensava que, porque os outros cortesãos não tentavam repeli"-lo pelo
mesmo motivo, eles reconheciam seu direito de ser o primeiro. Não era
assim, no entanto; eles apenas o desprezavam e não fariam o que ele
fazia.

Havia também Skarkrou, que era exatamente o oposto de Sartorius, e que
pensava -- ou fingia pensar -- que a falta de asseio era uma coisa boa.
E~ele era tão ou mais orgulhoso de seus trapos do que Sartorius era de
suas belas roupas. Também era desprezado, pois era vaidoso, e sua
vaidade tornava"-o ridículo.

Então havia Gabbleander, que nada mais fazia além de falar desde a manhã
até a noite, e que falaria da noite até a manhã se conseguisse alguém
para ouvi"-lo. Também riam dele pois as pessoas não conseguem falar
sempre com juízo se falam demais. As coisas tolas são lembradas, mas as
sábias são esquecidas. E~assim, os tagarelas vêm a ser considerados
tolos.

Mas ninguém deve pensar que toda a Corte do bom Rei Mago era como essas
pessoas. Não! Havia muitas, muitas pessoas boas, e grandes, e nobres e
corajosos homens. Mas a vida é assim em qualquer país, até mesmo no País
Sob o Pôr do Sol: há tolos tanto quanto sábios, covardes tanto quanto
corajosos e homens maus tanto quanto homens bons.

Crianças que desejam se tornar homens importantes e bons ou mulheres
boas e nobres devem tentar conhecer bem todas as pessoas que encontram.
Assim, elas perceberão que não há ninguém que não tenha um tanto de
bondade; e quando virem grandes tolices, ou um pouco de malvadeza, ou um
pouco de covardia, ou algum erro ou fraqueza em outra pessoa, elas devem
se examinar cuidadosamente. Então verão que, talvez, elas mesmas também
têm alguns defeitos -- apesar de, talvez, não se revelarem da mesma
forma -- e que devem tentar vencer esses defeitos. Assim, elas se
tornarão melhores à medida que crescem; e outros as examinarão, e quando
estas pessoas descobrirem que não têm defeitos, irão amá"-las e
honrá"-las.

Bem, um dia o Rei Mago sentou"-se em seu trono com seu manto e sua coroa,
segurando seu cetro em sua mão.

À sua direita sentou"-se a Princesa Bluebell com seu manto, sua coroa e
seu cetro, tendo ao lado seu cãozinho Smg.

Esse cão era um grande favorito. Primeiramente, fora chamado de Sumog
porque o cão de Zaphir se chamava Gomus, e este era seu nome escrito ao
contrário. Mas então foi nomeado Smg porque era um nome que não se
poderia gritar, mas somente sussurrar. Bluebell não tinha necessidade de
mais do que isso, pois Smg nunca estava longe, e ficava sempre perto de
sua dona e a protegia.

À esquerda do Rei sentou"-se o Príncipe Zaphir, em seu pequeno trono, com
sua espada brilhante e sua imponente pena.

Mago estava escrevendo leis para o bem de seu povo. Em volta dele
estavam reunidos todos os cortesãos e muitas pessoas estavam no salão, e
outras muitas lá fora na rua.

De repente, ouviu"-se um som alto -- o estalar de um chicote e o soar de
uma trombeta -- que se aproximou cada vez mais, e as pessoas na rua
começaram a murmurar. Surgiram gritos altos, o Rei parou para ouvir, e
as pessoas viraram a cabeça para ver quem vinha. A~multidão se abriu, e
um mensageiro, de botas e esporas, e coberto de poeira, correu para o
salão e ajoelhou"-se em um joelho ante ao Rei, estendendo um papel que o
Rei Mago pegou e leu avidamente. O~povo esperou em silêncio para ouvir
as notícias.

O Rei ficou profundamente tocado, mas sabia que seu povo estava ansioso;
então, falou"-lhes, ficando de pé:

``Meu povo, um perigo grave surpreendeu nosso Reino. Soubemos, por este
despacho da província de Sub"-Tegmine, que um terrível Gigante surgiu dos
pântanos além da Terra"-de"-Ninguém e que está devastando o país. Mas não
tema, meu povo, pois hoje muitos soldados se apresentarão com suas
armas, e ao pôr do sol de amanhã o Gigante terá sucumbido,
acreditamos.''

As pessoas curvaram suas cabeças com murmúrios de agradecimento, e foram
todos quietos para suas casas.

Naquela noite, um corpo de soldados selecionados saiu com corações
valentes para lutar contra o Gigante, e as pessoas os saudaram em seu
caminho.

Por todo o dia seguinte e a noite seguinte, as pessoas, bem como o Rei,
estiveram muito ansiosas; e na segunda manhã eles esperaram notícias de
que o Gigante tivesse sido derrotado.

Mas nenhuma notícia veio até o anoitecer; então, um homem cansado,
coberto com poeira e sangue, e ferido mortalmente, veio se arrastando
para a cidade.

As pessoas abriram caminho e ele foi para diante do trono, curvou"-se e
disse:

``Ah! Rei, tenho de lhe dizer que seus soldados foram mortos -- todos
exceto eu. O~Gigante triunfa e avança em direção a cidade''.

Tendo dito isso, a dor de seus ferimentos aumentou tanto que ele gritou
diversas vezes e caiu; e quando o ergueram, ele estava morto.

Diante da triste notícia que ele trouxe, um lamento baixo surgiu do
povo. As viúvas dos soldados mortos deram um pequeno grito, dirigiram"-se
ao trono do Rei e ali se jogaram, levantando as mãos para o alto, e
disseram:

``Oh, Rei! Oh, Rei!'', e não puderam dizer mais nada por causa do choro.

Então o coração do Rei ficou muito, muito magoado, e ele tentou
consolá"-las, mas seu melhor consolo estava em suas lágrimas -- pois as
lágrimas de amigos ajudam a aliviar os problemas. Ele falou ao povo,
dizendo:

``Ah! Nossos soldados eram muito poucos. Hoje à noite enviaremos um
exército, e talvez o Gigante sucumba''.

Naquela noite, um exército aguerrido, com muitas máquinas de guerra, com
bandeiras voando e bandas tocando, partiu contra o Gigante.

No comando do exército cavalgava Janisar, o capitão, com sua armadura de
aço incrustada com ouro reluzindo ao brilho do pôr do sol. Os adornos
escarlates e alvos de seu grande cavalo de guerra negro mostravam"-se
esplêndidos. A~seu lado, a alguma distância, em seu caminho, cavalgava o
Príncipe Zaphir em seu palafrém branco.

O povo todo se reuniu para desejar ao exército sucesso em sua partida; e
muitas pessoas tolas que acreditavam na sorte atiraram sapatos velhos
depois da passagem deles. Um desses sapatos acertou Sartorius, que
estava, como de costume, forçando a dianteira para se exibir, e manchou
seu olho de preto, e o preto do sapato escorreu em sua roupa nova,
estragando"-a. Outro sapato -- um pesado, com saltos de ferro -- acertou
Tufto, que estava falando com Janisar, no topo de sua cabeça calva, e
cortou"-a, e então todos riram.

Imagine o quanto um homem é desprezado quando as pessoas riem quando ele
se machuca. O~velho Tufto caiu e ficou bastante raivoso, e então as
pessoas riram ainda mais;pois nada é mais engraçado do que quando uma
pessoa está tão nervosa que perde todo o autocontrole.

Todas as pessoas aclamavam à medida que o exército passava. Mesmo as
pobres viúvas dos soldados mortos estavam aclamando. E~os homens que
partiam olhavam para elas e decidiam que venceriam ou morreriam, como
bravos soldados em serviço.

A Princesa Bluebell foi com o Rei Mago para o topo da torre do palácio,
e juntos eles assistiram aos soldados enquanto marchavam. O~rei entrou
logo, mas Bluebell continuou lá, observando os capacetes cintilando e
reluzindo ao pôr do sol até que o sol mergulhou no horizonte.

Bem naquele momento, o Príncipe Zaphir, que havia retornado, juntou"-se a
ela. Então, ao crepúsculo no topo da torre, com muitos milhares de
corações ávidos e ansiosos na cidade abaixo deles, e com o belo céu
acima, as duas crianças se ajoelharam e rezaram pelo sucesso do exército
pela manhã.

Na cidade não se dormiu naquela noite.

No dia seguinte, as pessoas estavam repletas de ansiedade. E à medida
que o dia gradualmente avançava e não havia notícias, elas ficaram ainda
mais ansiosas.

Rumo à noite, elas ouviram o som de um grande tumulto ao longe. Sabiam
que a batalha continuava; e, assim, esperaram ainda mais por notícias.

As pessoas não foram de modo algum para a cama naquela noite; mas, por
toda a cidade, fogueiras de vigia foram acesas e todos ficaram acordados
esperando por notícias.

Mas nenhuma notícia veio.

Então o medo se tornou tão grande que os rostos dos homens e das
mulheres ficaram tão brancos e seus corações tão frios quanto a neve.
Por um tempo muito longo ficaram em silêncio, pois homem algum ousava
falar.

Finalmente, uma das viúvas dos soldados mortos levantou"-se e disse:

``Irei me levantar e ir ao campo de batalha para ver o que está
acontecendo lá e trarei notícias para aquietar vossos pobres corações
palpitantes''.

Então muitos homens ergueram"-se e disseram:

``Não! Não deve ser assim. Nós iremos. Seria uma vergonha para nossa
Cidade se uma mulher fosse onde homens não conseguiriam. Nós iremos''.

Mas ela respondeu a eles com um sorriso tristonho:

``Ah! Não tenho medo da morte, já que meu corajoso marido foi morto. Não
desejo viver. Vocês devem defender a cidade, eu irei''.

Imediatamente, ela saiu da cidade na manhã cinzenta e fria em direção ao
campo de batalha. À~medida que se afastava e desaparecia na distância,
ela parecia ao povo ansioso como um fantasma da Esperança desaparecendo
diante deles.

O sol nasceu e brilhou nos céus até que a hora do descanso chegou; mas
os homens não se preocuparam com ela, sempre vigiando e esperando.

Nesse instante, eles viram de longe a silhueta de uma mulher correndo.
Eles se dirigiram para ela e descobriram que era a viúva. Ela colocou"-se
no meio deles e gritou:

``Ai! Ai! Ah! O nosso exército está disperso. Nossos mais fortes estão
sob o domínio do orgulho de sua força. O~Gigante triunfa e temo que tudo
esteja perdido''.

Do povo ouviu"-se um grande lamento, e um silêncio caiu sobre eles, tão
grande era seu medo.

Então o Rei reuniu toda sua Corte e seu povo, e aconselhou"-se quanto ao
melhor a se fazer. Muitos pareciam pensar que um novo exército deveria
partir, formado por todos aqueles que estavam dispostos a morrer, se
necessário, pelo bem do País. Mas havia muita perplexidade.

Enquanto discutiam, o Príncipe Zaphir estava sentado silencioso em seu
trono. E~seus olhos mais de uma vez se encheram de lágrimas diante do
pensamento do sofrimento de seu povo amado. Ele, então, levantou"-se e se
pôs diante do trono.

Houve silêncio até que ele começasse a falar.

Quando o Príncipe se pôs, de chapéu nas mãos, ante ao Rei, havia no
rosto dele um olhar com tanta determinação que aqueles que o viram não
evitaram ter uma nova esperança. O~Príncipe falou:

``Oh, Rei, Pai, antes que decida algo, escute"-me. É~certo que, se há
perigo no Reino, o primeiro a enfrentá"-lo é o Príncipe, em quem o povo
confia. Se há dor a ser sentida, quem deve senti"-la antes dele? Se a
morte vem a qualquer um, certamente deveria recair sobre seu cadáver.
Rei, Pai, espere somente um dia. Deixe"-me partir amanhã para enfrentar o
Gigante. Essa viúva lhe contou que agora ele está dormindo depois do
combate. Amanhã eu o encontrarei em combate. Se eu sucumbir, então será
hora de arriscar a vida de seu povo; e se for ele que sucumba, então
tudo estará bem''.

\imagemmedia{}{./img/02.png}


O Rei Mago sabia que o Príncipe havia falado apropriadamente, e apesar
de afligi"-lo ver seu amado filho indo ao encontro de tal perigo, não
tentou impedi"-lo, e disse:

``Oh, filho, digno de ser rei, você falou apropriadamente! Seja como
você entender''.

Então o povo deixou o Salão, e o Rei Mago e a Princesa Bluebell beijaram
Zaphir. Bluebell disse"-lhe:

``Zaphir, você fez bem'', e olhou para ele, orgulhosa.

Imediatamente, o príncipe retirou"-se à cama para que pudesse dormir, e
assim estar forte para a manhã.

Por toda aquela noite os ferreiros e os armeiros e os ourives
trabalharam duro e rápido. Até a aurora, as fornalhas brilharam e as
bigornas soaram; e todas as mãos habilidosas nessas artes trabalharam
com esforço.

Pela manhã eles levaram ao Salão, e colocaram diante do trono como um
presente ao Príncipe Zaphir, uma armadura tal como antes nunca havia
sido vista.

Era trabalhada em aço e ouro e feita toda com lamelas. Cada lamela era
como uma folha diferente, e era inteira polida e brilhante como o sol.
Entre as folhas havia joias e muitas outras mais estavam presas nelas
como gotas de orvalho. Assim, a armadura cintilava à luz até ofuscar os
olhos de quem a olhasse -- pois os habilidosos armeiros pretendiam que,
quando o Príncipe lutasse, seu inimigo pudesse ser parcialmente cegado
com o brilho e, assim, errar seus golpes.

O capacete era como uma flor;a insígnia do Príncipe fora fundida em cima
dele, e a pena e o grande diamante de seu chapéu foram fixados na
frente.

Quando o príncipe se equipou, mostrava"-se tão nobre e corajoso que o
povo aclamou aos gritos que ele venceria e que tinha grandes esperanças
renovadas.

Então seu pai, o Rei, abençoou"-o, a Princesa Bluebell beijou"-o, verteu
algumas lágrimas e deu"-lhe uma graciosa rosa, a qual ele fixou em seu
capacete.

Entre brados do povo, o Príncipe Zaphir partiu para lutar contra o
Gigante.

Seu cão, Gomus, queria ir, mas ele não podia ser levado. Então Gomus se
aquietou e uivou, pois sabia que seu querido amo estava em perigo e
desejou estar com ele.

Depois que o Príncipe partiu, a Princesa Bluebell subiu ao topo da torre
e observou"-o até que ele estivesse tão longe a ponto de ela não mais
poder ver o lampejo de sua bela armadura à luz do sol. Inicialmente,
quando ela estava se despedindo de Zaphir -- e ela sabia que poderia ser
uma despedida eterna --, não derramou uma lágrima para não causar dor a
seu amado Príncipe, pois ela sabia que ele estava rumando para a batalha
e precisaria de toda sua coragem e de toda sua firmeza. Então o último
olhar que Zaphir viu no rosto de sua Bluebell foi um sorriso amável,
esperançoso e confiante. Assim, ele partiu para a batalha fortalecido
pelo pensamento de que o coração dela o acompanhava e que, apesar de o
corpo dela estar longe, seu espírito estava próximo a ele.

Quando ele partiu, realmente, para bem longe da vista, e ela se colocou
sozinha no topo da torre, Bluebell chorou bastante. E~o grande medo de
seu coração, de que Zaphir poderia ser morto, deixou"-a fatalmente
triste. Ela pensou que poderia ocorrer de ele ser morto pelo maléfico
Gigante, que já havia destruído dois exércitos, e que, então, ela nunca
mais iria vê"-lo -- nunca mais veria o amor nos olhos queridos e
verdadeiros -- nunca mais ouviria os tons de sua voz tenra e doce --
nunca mais sentiria o bater de seu coração grande e generoso.

E então ela chorou, oh!, muito amargamente. Mas, enquanto chorava,
ocorreu"-lhe o pensamento de que a vida não jaz no poder dos homens, ou
mesmo dos gigantes; e, assim, ela enxugou suas lágrimas, ajoelhou"-se e
rezou com coração humilde, ficando confortada, assim como as pessoas
sempre ficam quando rezam com sinceridade.

Então ela desceu ao grande salão, mas o Rei Mago não estava lá. Ela
procurou"-o para consolá"-lo, pois sabia que o coração dele devia estar
sofrendo por seu filho em perigo.

Ela encontrou"-o em seus aposentos e ele, também, estava rezando. Ela se
ajoelhou ao seu lado, e eles colocaram os braços em torno um do outro --
o velho Rei e a criança órfã -- e rezaram juntos. E~assim ambos se
consolaram.

\imagemmedia{}{./img/03.png}


Juntos, esperaram, e esperaram pacientemente, pelo retorno de seu amado.
Toda a cidade esperou também; e nem de dia nem de noite houve sono no
País Sob o Pôr do Sol, pois todos estavam aguardando o retorno do
Príncipe.

Quando Zaphir deixou a cidade, ele rumou sempre em direção ao Gigante
até o sol brilhar alto nos céus, tão brilhante que sua armadura dourada
reluzia como fogo. E~então ele andou sob a proteção das árvores, e não
parou nem mesmo na hora do descanso, mas continuou sempre em frente.

Próximo à noite, ele ouviu e viu coisas estranhas.

Ao longe o chão parecia tremer, e um surgiu estrondo surdo de rochas
sendo destruídas e de florestas sendo derrubadas. Esses eram os sons dos
passos do Gigante à medida que ele se aproximava da cidade. Mas o
Príncipe Zaphir, apesar de os sons serem terríveis, não tinha medo e
avançou bravamente. Então, começou a encontrar muitas coisas vivas, que
passavam por ele muito rapidamente -- pois elas eram as mais velozes de
suas espécies e, assim, haviam fugido do Gigante mais rápido do que as
demais.

Elas vinham, em centenas e milhares, sua quantidade aumentando mais e
mais à medida que o tempo passava, e à medida que o Príncipe e o Gigante
se aproximavam.

Havia todos os animais do campo, e todas as aves do ar, e todos os
insetos que voam e rastejam. Leões e tigres, e cavalos e ovelhas, e
ratos e gatos e camundongos, e galos e galinhas, e raposas e gansos e
perus, todos estavam misturados, grandes e pequenos, e todos estavam tão
atemorizados pelo Gigante que se esqueceram de ter medo uns dos outros.
Assim, fugiam juntos, gatos e ratos, lobos e carneiros, raposas e
gansos; os fracos não tinham medo, nem os mais fortes queriam fazer
algum mal.

Entretanto, à medida que vinham, todas as coisas vivas pareciam saber
que o Príncipe Zaphir era mais corajoso do que elas, e abriam caminho
para ele passar. As coisas mais fracas, e aquelas mais atemorizadas, não
procediam à sua fuga, mas tentavam chegar o mais perto possível do
Príncipe; e muitas preferiam retornar, seguindo"-o em direção ao Gigante,
a não ficar perto dele.

Mais adiante, depois de um tempo, ele encontrou todos os animais velhos
que não podiam ir tão rápido quanto os demais, e todos os pobres seres
vivos feridos, e todos aqueles que eram lentos. Esses, também, não
tentaram ir mais longe, pois sabiam que estariam mais seguros perto de
um homem corajoso do que em uma fuga desamparada.

Então o Príncipe Zaphir viu algo, ainda muito longe, que parecia uma
portentosa montanha.

Estava se movendo em sua direção, e seu coração bateu alto, parte por
pensar na batalha vindoura, parte com esperança.

O Gigante aproximava"-se cada vez mais. Seus passos esmagavam as rochas,
e com sua poderosa clava ele varria as florestas de seu caminho.

As criaturas vivas atrás do Príncipe Zaphir tremeram de medo e
esconderam suas caras na poeira. Alguns animais, como algumas pessoas
tolas, pensaram que se não vissem algo que não desejariam ver, esse algo
deixaria então de existir.

Muito tolo da parte deles.

Então, à medida que o Gigante se aproximou, o Príncipe Zaphir sentiu que
a hora da batalha havia chegado.

\imagemmedia{}{./img/04.png}


Quando ele ficou cara a cara com um inimigo mais poderoso do que
qualquer coisa que já tinha visto, Zaphir sentiu"-se como nunca antes.
Não era que ele estava com medo do Gigante, pois ele se sentia com tanta
coragem que, para o bem de seu povo, poderia alegremente ter morrido da
forma mais dolorosa possível. Era que ele percebeu que coisa pequena ele
era no enorme mundo.

Ele viu mais claramente do que já vira antes que era apenas um ponto --
um mero átomo -- no enorme mundo; e, em um instante, percebeu que, se a
vitória fosse dele, não seria porque seu braço era forte ou seu coração
valente, mas porque tinha a força de vontade dada por Aquele que governa
o universo.

Então, em sua humildade, o Príncipe Zaphir rezou pedindo por forças. Ele
tirou sua esplêndida armadura, que brilhava como um sol na terra, tirou
seu esplêndido capacete e colocou"-o ao lado da rutilante espada; e as
partes da armadura jazeram em um amontoado inanimado ao seu lado.

Era uma bela visão a daquele jovem garoto ajoelhado ao lado da armadura
descartada. O~amontoado brilhante jazia belo, cintilando no claro pôr do
sol com milhões de lampejos coloridos, até que pareceu até mesmo estar
sobre um ser vivo. No entanto, o amontoado era triste, miserável e
desprezível ao lado do garoto. Ali ele se ajoelhou, rezando
humildemente, com seus olhos profundamente sérios acesos pela verdade e
pela confiança que jazia em seu coração limpo e em sua alma pura.

A armadura reluzente parecia o trabalho das mãos do homem -- como o era,
e o trabalho das mãos de homens bons e verdadeiros. Mas o belo garoto,
ajoelhado em confiança e em fé, era o trabalho das mãos de Deus.

Enquanto rezava, o Príncipe Zaphir revira toda sua vida passada, desde o
primeiro dia de que conseguia se lembrar até aquele mesmo momento, face
a face com o Gigante. Não havia um pensamento indigno que ele tivesse
tido, nem uma palavra rude que tivesse dito, nem um olhar colérico que
tivesse provocado dor em outra pessoa, que não voltou à sua mente.
Afligia"-o muito haver tantos, pois se amontoavam tão abundantemente que
ele ficou impressionado somente com a quantidade deles.

É sempre assim, as coisas que fazemos erroneamente -- apesar de elas
parecerem pequenas no momento, e apesar de, por causa da dureza de
nossos corações, passarmos levianamente ao largo delas -- voltam"-nos com
amargura quando o perigo nos faz pensar no quão pouco fizemos para
merecermos ajuda e o quanto fizemos para merecermos punição.

O coração do Príncipe Zaphir foi purificado pela penitência de todas as
coisas erradas feitas no passado, e por elevadas determinações de ser
bom no futuro. E~quando sua humilde reza terminou, ele se levantou e
sentiu em seus braços uma força que ele não conhecia. Ele sabia que não
era a sua própria força, mas que era o instrumento humilde da salvação
de seu povo querido. E~em seu coração ele ficou agradecido.

O Gigante viu imediatamente o brilho da armadura áurea, e soube que
outro inimigo se aproximara dele.

Ele deu um rugido estrondoso de raiva e fúria que soou como o eco de um
trovão. Nas colinas distantes o som ecoou, ribombou através dos vales ao
longe e decaiu a murmúrios e rosnados baixos, como os de animais
selvagens, nas cavernas e nas cavidades das montanhas.

O Gigante começou seu ataque com esse barulho a fim de que pudesse
atemorizar seus inimigos. Mas o coração valente do Príncipe não tremeu
de medo. Ele se tornou mais valente do que nunca quando ouviu o barulho;
sabia, pois, que havia mais necessidade de coragem, para que seu povo, e
até mesmo o Rei, seu pai, e Bluebell não caíssem sob o poder do Gigante.

Enquanto entre as pedras e as florestas as pegadas do Gigante embatiam,
e enquanto subia em volta de seus pés o pó da desolação que ele causava,
o Príncipe Zaphir juntou do riacho alguns seixos arredondados.

Ele encaixou um no estilingue que carregava.

Assim que levantou seu braço para rodopiar o estilingue em volta de sua
cabeça, o Gigante viu"-o, riu e apontou desdenhoso em sua direção com
suas grandes mãos, que eram mais brutas do que as garras de tigres. A~risada que o Gigante trovejou era tão terrível -- tão rude e raivosa e
horrível que as coisas vivas que haviam levantado os tímidos olhos para
observar a luta enterraram novamente as cabeças na terra, e tremeram de
medo.

Mas até mesmo tendo rido com escárnio para seu inimigo, a perdição do
Gigante estava proferida.

Em volta da cabeça do Príncipe Zaphir o estilingue circulou, e o seixo
sibilante voou. Acertou bem na têmpora do Gigante, e mesmo com a risada
de escárnio em seus lábios, e com sua mão estendida apontando com
menosprezo, ele caiu de bruços.

Enquanto caía, emitiu um único grito, mas um grito tão alto que
percorreu as colinas e os vales como o estrondo de um trovão. Ao som, as
coisas vivas novamente se acovardaram e fraquejaram de medo.

Ao longe, as pessoas da cidade ouviram o poderoso som, mas elas não
sabiam o que ele significava.

Quando o grande corpo do Gigante caiu de bruços, a terra tremeu por
muitas milhas ao redor devido ao choque. E~quando seu grande porrete
caiu de sua mão, derrubou muitas árvores altas da floresta.

Então, o Príncipe Zaphir ajoelhou e rezou com gratidão fervorosa por sua
vitória.

Rapidamente se levantou e, porque sabia da amarga ansiedade do Rei e do
povo, nem parou para recolher sua armadura, e dirigiu"-se rápido para a
cidade levando as notícias felizes.

A noite havia caído agora e o caminho estava escuro;mas o Príncipe
Zaphir tinha confiança e seguiu na escuridão com coração valente e
esperançoso.

Logo, os seres vivos que eram nobres circundaram"-no com gratidão, e
todos que puderam seguiram"-no de perto. Havia muitos animais nobres--
leões e tigres e ursos, bem como animais domésticos. E~seus grandes
olhos fogosos pareciam lampiões e ajudaram"-no em seu caminho.

Entretanto, à medida que se aproximaram da Cidade, os animais selvagens
começaram a se retrair, pois, apesar de confiarem em Zaphir, eles temiam
os outros homens. Emitiram um rosnado baixo de pesar e pararam, e o
Príncipe Zaphir continuou sozinho.

Por toda a noite a cidade permanecera acordada. Na corte, o Rei Mago e a
Princesa Bluebell esperavam e observavam juntos, as mãos dadas. O~povo
nas ruas se sentou em volta de suas fogueiras de vigia, e ousavam falar
somente em sussurros.

Assim, a longa noite passou.

Por fim, o céu do oriente começou a empalidecer; e então uma risca de
fogo rubro disparou pelo horizonte e o sol nasceu em sua glória. E~assim
fez"-se dia. O~povo, quando viu a luz e ouviu o cantar revigorado dos
pássaros, teve esperança. E~aguardaram ansiosamente pela vinda do
Príncipe.

Nem o Rei Mago, nem a Princesa Bluebell ousaram subir para o alto da
torre; e esperaram pacientemente no salão. Seus rostos estavam pálidos
como a morte.

As sentinelas da cidade e aqueles que se juntaram a elas observavam a
longa estrada, esperando ver em algum momento a armadura áurea do
Príncipe Zaphir reluzindo à luz esfuziante da manhã e sua grande pluma
branca, que conheciam muito bem, acenando à brisa. Eles sabiam que
poderiam vê"-la de longe e então davam apenas uma olhada de vez em quando
a distância.

De repente, houve brados de todas as pessoas -- e então uma quietude
repentina.

Eles se levantaram, e esperaram todos por notícias.

Pois, oh!, que alegria!, lá, entre eles -- separado de sua armadura
brilhante e de sua pluma que acenava, porém vigoroso -- estava seu amado
Príncipe.

Havia vitória em seu olhar.

Ele sorriu para eles, levantando as mãos como se abençoando, e apontou
para o palácio do Rei, como para dizer:

``Nosso rei! Ele tem o direito de ouvir as mais novas notícias''.

Ele passou, entrando no salão, todas as pessoas seguindo"-o.

Quando o Rei Mago e a Princesa Bluebell ouviram o brado e sentiram a
quietude que se seguiu, seus corações começaram a bater forte e
aguardaram muito apreensivos.

A Princesa Bluebell sentiu um calafrio e chorou um pouco, aproximou"-se
do Rei e apoiou seu rosto em seu peito.

Enquanto escondia seu rosto, apoiando"-o no rei, ela sentiu"-o
sobressaltar. Ela rapidamente levantou os olhos, e ali -- oh!, alegria
das alegrias! -- estava seu amado Zaphir entrando no saguão, com todo o
povo seguindo"-o.

O Rei desceu de seu trono e tomou"-o nos braços, beijando"-o; Bluebell
também colocou seus braços em torno dele e o beijou na boca.

O Príncipe Zaphir falou, dizendo:

``Oh!, Rei, meu Pai, e oh!, Povo! -- Deus foi bom para conosco e Seu
braço deu"-nos a vitória. Veja! O Gigante sucumbiu no orgulho de sua
força!''

Então um tal brado surgiu do povo que o teto tremeu novamente, e o
barulho percorreu toda a Cidade nas asas do vento. A~multidão contente
bradou mais e mais até que o som ondulou por todo o Domínio, e em Sob o
Pôr do Sol naquela hora nada houve senão alegria. O~Rei chamou Zaphir de
seu Filho Valente, e a Princesa Bluebell beijou"-o novamente, chamando"-o
de seu Herói.

Naquele mesmo momento, lá longe na floresta, o Gigante jazia sucumbido
pelo orgulho de sua força -- a coisa mais vil de todo o mundo --, e
sobre seu cadáver corriam raposas e arminhos. As cobras rastejavam em
torno de seu corpo; e ali, também, arrastavam"-se todos os piores seres
vivos que haviam fugido dele quando ele vivia.

Ao longe, agrupavam"-se os abutres para sua presa.

Perto do Gigante morto, brilhando na luz, jazia a armadura áurea. A~grande pluma branca desprendeu"-se do capacete e até mesmo agora acenava
na brisa.

Quando o povo veio ver o Gigante, descobriu que ervas daninhas já haviam
crescido onde seu sangue tinha sido derramado, mas também que, em volta
da armadura que o Príncipe havia despido, um anel de graciosas flores
havia crescido. A~mais bela de todas era uma roseira em flor, pois a
rosa que a Princesa Bluebell tinha dado a ele havia criado raízes e
florescido novamente, formando uma coroa de rosas vivas em volta do
capacete, e reclinava"-se contra a haste da pluma.

Então o povo levou de volta, respeitosamente, a armadura dourada; o
Príncipe Zaphir disse, porém, que não fora tal armadura, mas sim um
coração verdadeiro, a melhor proteção, e que ele não ousaria vesti"-la
novamente.

Então eles a penduraram na Catedral entre as grandes bandeiras antigas e
os capacetes de cavaleiros de outrora como um memorial da vitória sobre
o Gigante.

O Príncipe Zaphir tirou do capacete a pena que o Rei, seu pai, havia
antes lhe dado e usou"-a novamente em seu chapéu. A~rosa que florescera
estava plantada no centro do jardim do palácio, e ela cresceu tanto que
muitas pessoas podiam se sentar sob ela, abrigando"-se do sol pela
abundância de suas flores.

\imagemmedia{}{./img/05.png}

Quando o aniversário do Príncipe Zaphir chegou, o povo fez, em segredo,
grandes preparos.

Quando ele se levantou de manhã para ir à Catedral, todo o povo havia se
reunido e formado uma fila de cada lado do caminho. Toda pessoa, velha e
nova, segurava uma rosa. Aqueles que tinham muitas rosas trouxeram uma
para quem não tinha; e cada pessoa tinha somente uma rosa para que todos
pudessem ser iguais aos olhos do Príncipe que amavam. Eles haviam
removido todos os espinhos dos caules para que não machucassem os pés do
Príncipe. À~medida que ele passava, o povo jogava suas rosas no caminho,
até que toda a longa rua ficou cheia de flores.

Quando o Príncipe passava, as pessoas se inclinavam e recolhiam as rosas
que seus pés haviam tocado, e as pessoas guardaram"-nas com muito
carinho.

Durante toda a vida, a cada aniversário do Príncipe, elas repetiram isso.
Quando Zaphir e Bluebell se casaram, eles espargiram o caminho deles com
rosas da mesma forma, pois o povo os amava muito.

Por muito e muito feliz viveu O Príncipe da Rosa -- pois assim o
chamavam -- e sua bela esposa, a Princesa Bluebell.

Quando na plenitude do tempo o Rei Mago faleceu -- pois todos os homens
falecem --, eles reinaram como Rei e Rainha. Reinaram com justiça e
altruisticamente, sempre se renegando e lutando para fazer as pessoas
boas e felizes.

Eles foram abençoados pela paz.
