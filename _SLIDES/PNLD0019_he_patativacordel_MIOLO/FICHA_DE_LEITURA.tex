\documentclass[11pt]{hedrabook_ficha}
\usepackage[brazilian]{babel}
\usepackage{ucs}
\usepackage[utf8x]{inputenc}
\usepackage[T1]{fontenc}
\usepackage{hedracrop}
\usepackage{hedrabolsolayout}
\usepackage[protrusion=true,expansion]{microtype}
\usepackage{comment,lipsum,footmisc}
\usepackage[minionint,mathlf]{MinionPro}

\makeatletter
  \renewenvironment{frontispiciopage}[2]{
    \thispagestyle{empty}
    \def\@ano{#2}
    \vspace*{\stretch{1}}
    \begin{fancytable}{#1}%
      \putline{\hspace*{36mm}}{}
      \ThirdpageAuthorCathegory               %comentar quando não houver AUTOR
      \putline{T\'itulo}{\textsc{\@title}}}{
      \putline{S\~ao Paulo}{\@ano\vspace{5mm}}
      \bigline{}{}{
        \setlength{\unitlength}{1mm}%
        \hspace{1mm}\begin{picture}(25,25)(0,2.2)%
          %\color{white}
          \put(0, 0){\line(1, 0){2}}
          \put(0, 0){\line(0, 1){2}}
          \put(23, 0){\line(1, 0){2}}
          \put(25, 0){\line(0, 1){2}}
          \put(0, 25){\line(1, 0){2}}
          \put(0, 23){\line(0, 1){2}}
          \put(23, 25){\line(1, 0){2}}
          \put(25, 23){\line(0, 1){2}}
          \put(5.8,13){\makebox(0,0)[l]{\fontencoding{OT1}\fontfamily{ptm}\selectfont\logosize{}hedra}}%
        \end{picture}}
      \bigline{}{%
        \setlength{\unitlength}{\baselineskip}%
        \begin{picture}(0,0)(0,0)%
          \put(0,1){\makebox(0,0)[l]{\interno@traco}}%
          \put(0,2.5){\makebox(0,0)[l]{\interno@traco}}%
          \put(0,3.5){\makebox(0,0)[l]{\interno@traco}}%
          \put(0,5){\makebox(0,0)[l]{\interno@traco}}%
          \put(0,8.7){\makebox(0,0)[l]{\interno@traco}}%
        \end{picture}}{%
        \setlength{\unitlength}{\baselineskip}%
        \hspace{\stretch{1}}\begin{picture}(0,0)(-4,0)%
          \put(-4,1){\makebox(0,0)[l]{\interno@traco}}%
          \put(-4,3){\makebox(0,0)[l]{\interno@traco}}%
          \put(-4,4){\makebox(0,0)[l]{\interno@traco}}%
          \put(-4,10){\makebox(0,0)[l]{\interno@traco}}%
          \put(-4,11){\makebox(0,0)[l]{\interno@traco}}%
          \put(-4,12){\makebox(0,0)[l]{\interno@traco}}%
          \put(-4,16){\makebox(0,0)[l]{\interno@traco}}%
          \put(-4,17){\makebox(0,0)[l]{\interno@traco}}%
          \put(-4,21){\makebox(0,0)[l]{\interno@traco}}%
          \put(-4,23){\makebox(0,0)[l]{\interno@traco}}%
          \put(-4,24){\makebox(0,0)[l]{\interno@traco}}%
        \end{picture}}
    \end{fancytable}}
\makeatother

\AtBeginDocument{%
  \selectlanguage{brazilian}
  \fontsize{11pt}{13.2pt}\selectfont
  \parskip=0pt
  \setlength{\unitlength}{1mm}%
  \setcounter{secnumdepth}{-2}%
  \pagestyle{plain}}


\begin{document}

\author{Patativa do Assaré}
\title{Cordel --- Patativa do Assaré}
\begin{frontispiciopage}{4cm}{2013}
  %comentar/modificar linhas abaixo conforme necessário
  %\putline{Tradução}{}
  %\putline{Introdução}{}
  \putline{Organização e introdução}{Sylvie Debs}
\end{frontispiciopage}

\pagebreak
\section{Sobre o autor}

Antonio Gonçalves da Silva, o Patativa do Assaré, foi um dos mais importantes
poetas brasileiros. Nascido no Cariri, na Serra de Santana, próximo a Assaré,
no Ceará, a 5 de março de 1909, desde menino fazia versos e os apresentava a
quem quisesse ouvir. Só em 1956 seus poemas aparecereiam em livro, com a edição
pela Borsoi, do Rio de Janeiro, do belo \textit{Inspiração Nordestina}. O
sucesso de grande público, contudo, viria pouco depois, em 1964, com a gravação
em disco de ``A triste partida'', poema musicado pelo Rei do Baião, Luiz
Gonzaga. Poeta de genuína inspiração popular, Patativa do Assaré tornou-se
sinônimo de poesia popular no país, tendo lançado, em sua longa vida de quase
um século, uma dezena de livros e discos com seus poemas, além de inúmeros
folhetos avulsos de cordel. De sua obra, destacam-se os livros
\textit{Inspiração nordestina} (1956), \textit{Cante lá que eu canto cá}
(1978), \textit{Ispinho e fulô} (1988), \textit{Balceiro} (1991), \textit{Aqui
tem coisa} (1994) e \textit{Digo e não peço segredo} (2001); e os discos
\textit{Poemas e canções} (1979), \textit{A terra é naturá} (1981) e
\textit{Canto nordestino} (1989). Doutor \textit{honoris causa} em diversas
universidades, objeto de teses, filmes e peças, a voz da ave canora que lhe deu
nome, a patativa, ainda será ouvida por muitos e muitos anos em qualquer canto
do Brasil. Patativa morreu aos 93 anos em sua casa em Assaré.

\section{Síntese dos poemas}

\medskip

\paragraph{“História de Aladim e a lâmpada maravilhosa”} Um feiticeiro contrata
o jovem Aladim para resgatar uma lâmpada em uma gruta. Aladim fica com a
lâmpada, que possuía um gênio que podia realizar os desejos de quem a
possuísse. Apaixondao por Clarice, a filha do sultão de Bagdá, Aladim consegue
casar-se com ela por meio da mágica do gênio. Mas o feiticeiro retorna e
consegue reaver a lâmpada. Com ela, rouba o palácio, os escravos e a esposa de
Aladim. Este então tem que resgatar seus bens e sua esposa. 

\paragraph{“O padre Henrique e o dragão da maldade”} O poema conta a história
de um padre que vivia na cidade do Recife. Por seu engajamento nos movimentos
sociais, de esquerda, e por seus vínculos com o movimento estudantil, o padre
foi assassinado. O poema celebra a biografia do padre, dizendo que foi um homem
que lutou pela melhoria das condições de vida dos mais pobres. É a crônica de
um acontecimento que chocou a população pela violência dirigida a um personagem
caridoso.

\paragraph{“Emigração e as consequências”} O poema defende que a seca no
Nordeste obriga as famílias a abandonarem suas tarras natais e migrarem para o
Sul do país. Ao chegarem ao Sul, as famílias não conseguem empregos, e, quando
conseguem, seus salários são insuficientes. Logo a mãe também vai trabalhar. Os
filhos pequenos, expostos a privações, se envolvem com outros menores,
evetualmente já introduzidos na delinquência, e algumas vezes ingressam na vida
criminosa e comprometem para sempre suas vidas.

\paragraph{“Brosogó, Militão e o diabo”} Brosogó era um simples e decente
vendedor ambulante. Um dia, entra na casa de um coronle chamado Militão. Sem
conseguir vender nada, quer comprar meia dúzia de ovos de Militão. Como o
coronel não tem troco, diz a Brosogó que leve os ovos e volte depois para
pagar. Brosogó prospera e consquista muitas posses. Um ano e sete meses depois,
volta para pagar a dívida e é enganado por Militão, que faz as contas dizendo
que os ovos teriam sido chocados e que as galinhas nascidas poriam novos ovos e
teriam novas crias, e assim por diante. Brosogó fica desiludido, mas é ajudado
pelo Diabo, a quem teria acendido velas em um dia em que não lembrava de mais
nenhum santo a quem agradecer. 

\paragraph{“ABC do Nordeste fragelado”} O poema descreve a transformação
geográfica e social que sofre as regiões do Nordeste por falta de chuva.
Enumera como os animais sofrem e migram, como a vegetação fica seca e como as
pessoas, por falta de água e alimento, perdem o ânimo e a força para continuar
a luta cotidiana. 

\section{Poesia de cordel: oralidade e escuta coletiva} 

A poesia de cordel, dizem os especialistas, é uma poesia escrita para ser lida,
enquanto o repente ou o desafio é a poesia feita oralmente, que mais tarde pode
ser registrada por escrito. Essa divisão é muito esquemática. Por exemplo, o
cordel, mesmo sendo escrito e impresso para ser lido, costumava ser lido em
volz alta e desfrutado por outros ouvintes além do leitor. A poesia popular,
praticada principalmente no Nordeste do Brasil, tem muita influência da
linguagem oral, aproveita muito da língua coloquial praticada nas ruas e na
comunicação cotidiana. 

Naturalmente, portanto, pode-se considerar a poesia narrativa do cordel uma
forma de poesia mais compartilhada e desfrutada coletivamente, o que dá também
uma grande ressonância social. Muitos dos temas do cordel são originários das
tradições populares e eruditas da Europa medieval e moderna. Outros temas são
retirados de tradições orientais, como neste livro de Patativa “A história de
Aladim e a lâmpada maravilhosa”. O personagem Aladim pertence ao Livro das mil
e uma noites, um dos famosos conjuntos de histórias de todos os tempos. Também
encontramos temas retirados das novelas de cavalaria medievais e das narrativas
bíblicas. Ao lado destes temas mais literários, encontram-se os temas locais,
quase sempre narrados na forma de crônicas de coisas realmente acontecidas,
como em “Padre Henrique e o dragão da maldade”. Também há as histórias
fantásticas, que se valem das tradições semirreligiosas, ligadas à experiência
com o mundo espiritual. 

Os grandes poemas de cordel são perfeitamente metrificados e rimados. A métrica
e a rima são recursos que favorecem a memorização e tradicionalmente se costuma
dizer que são resquícios de uma cultura oral, na qual toda a tradição e
sabedoria são sabidas de cor.  


\section{O sertão geográfico e cultural}

O sertão tem mitos culturais próprios. Contemporaneamente, o sertão evoca
principalmente o sofrimento resignado daqueles que padecem a falta de chuva e
de boas safras na lavoura. Evoca a experiência histórica de uma região
empobrecida, embora tenha sido geradora de riquezas, como o cacau e cana de
açúcar, ambos bens muito valiosos. 

O sertão formou também o seu imaginário por meio de grandes personalidades e
uma pujante expressão artística. Além do cordel, o sertão viu nascer ritmos tão
importantes quanto o forró e o baião. Produziu artistas tão expressivos quanto
Luiz Gonzaga, grande cantor da vida do sertanejo em canções como “Asa branca”.
Um escultor como Mestre Vitalino criou toda uma tradição de representação da
vida e dos hábitos sertanejos em miniaturas de barro. A gravura popular, que
sempre acompanha os folhetos de cordel, também floresceu em diversos pontos e
ficou mais famosa em Juazeiro do Norte, no Ceará, e em Caruaru, no estado de
Pernambuco. 

Dentre os grande mitos do sertão, está certamente o do cangaço com seu líder
histórico, mas também mítico, Virgulino Ferreira, o Lampião. Até hoje as
opiniões se dividem: para alguns foi uma grande homem, para outros um bandido
impiedoso. 

Uma figura muito presente na cultura nordestina é o Padre Cícero Romão,
considerado beato pela Igreja Católica. Consta que teria feito milagres e
dedicado sua vida aos pobres. 

\section{Variação linguística}

A línguística moderna usa o termo “idioleto” para marcar grupos distintos no
interior de uma língua. Um idioleto pode ser a fala peculiar de uma região, de
um grupo étnico ou de uma dada profissão. 

Uma das grandes forças da poesia popular do Nordeste se origina em sua forma
muito própria de falar, com um ritmo muito diferente dos falares do sul, e
também muito diferentes entre si, pois percebe-se a diferença entre os falares
de um baiano, um cearense e um pernambucano, por exemplo.

Além desse aspecto rítmico, quase sempre também há palavras peculiares a certas
regiões. 

\section{Sugestões de atividades}
\begin{itemize}


\item Atividade de leitura. Esta atividade tem por objetivo sensibilizar os
  alunos para a escuta de poesia. O professor deve ler um conjunto de estrofes
  para exemplificar uma leitura que se construa com uma pronúncia clara, pausas
  e ênfases adequadas. Após isso, cada aluno deve ler uma estrofe, procurando
  marcar o ritmo e as rimas, bem como as pausas e ênfases expressivas. Para
  enriquecer a experiência com outros recursos, o professor pode mostrar o
  próprio poeta lendo alguns de seus poemas no link a seguir, do site You Tube:
  www.youtube.com/watch?v=RTEfYnMNNpc. Uma atividade como essa pode
  auxiliar no desenvolvimento da percepção da voz e da fala como meio
  indispensáveis à boa convivência social.

\item Propor uma atividade de pesquisa sobre o personagem de Aladim. Os alunos,
  organizados em grupos, devem apresentar o personagem, suas origens, sua
  importância, e também encontrar uma versão da história de Aladim. Comparar as
  versões: o que se acrescenta e o que se exclui? Em seguida, organizar um
  debate livro que procure apresentar hipóteses para as relações entre a poesia
  popular produzida no sertão do Nordeste e as histórias do livro As mil e uma
  noites. Que outros personagens conhecemos deste livro? Como as histórias são
  transmitidas?

\item O poema “Padre Henrique e o dragão da maldade” narra um acontecimento
  factual, ocorrido em 1969 na cidade do Recife. Um padre foi brutalmente
  assassinado pelos agentes da repressão do Governo Federal, que teriam
  procurado com a violência atingir outro integrante histórico da política
  brasileira, o arcebispo Dom Hélder Câmara. O professor poderia trazer mais
  informações sobre esse crime, que exemplifica o período da Ditadura Militar
  no Brasil. Após inteirar os alunos do conteúdo histórico do tema, cada um
  escreveria uma pequena reflexão escrita sobre o papel do poeta no caso. Qual
  a importância de um poeta tomar como tema uma injustiça social? Indicar aos
  alunos que quando dizemos que “a poesia fala de amor”, não devemos nos
  esquecer que a poesia também costuma abordar muitos outros temas. Um exemplo
  contemporâneo de poesia de denúncia social é o rap. Pode encaminhar a
  discussão para uma comparação entre o rap e o cordel.

\item No poema “Emigração e suas conseqûencias” podemos ver um outro aspecto da
  poesia social de Patativa do Assaré. Neste caso, o poeta não tematiza um
  drama particular, centrado em um personagem, mas aborda o próprio nordestino
  em seu destino de migrante, forçado pela escassez que é consequência
  histórica da seca. O tema da seca é central em pelo menos dois grandes
  clássicos da narrativa brasileira, Vidas secas, de Graciliano Ramos, e O
  quinze, de Rachel de Queiroz. Aqui se sugere orientar a reflexão para a
  questão ambiental que se associa a consequências sociais. Qual a causa
  ambiental da seca no Nordeste brasileiro? Como uma mudança climática se
  relaciona às características econômicas e sociais de uma dada região? 

\item O poema “Brosogó, Militão e o diabo” pode ser considerado um “causo”, ou
  seja, uma história protagonizada por personagens típicos e que em geral traz
  uma lição, uma moral. Pedir aos alunos que façam uma descrição dos
  personagens da história e de seu caráter. Dentre os personagens, existe algum
  cujo caráter mais complexo? Qual o ensinamento que o poema encerra?
  Introduzir a noção filosófico-política de maniqueísmo. O poema possui uma
  visão maniqueísta dos atos e das consequências?

\item Leia o trecho a seguir, do poema “ABC do Nordeste flagelado”:
\begin{quote}
D — De manhã, bem de manhã,
vem da montanha um agouro 
de gargalhada e de choro  da feia e triste cauã: 
um bando de ribançã  
pelo espaço a se perder,
pra de fome não morrer,
vai atrás de outro lugar,
e ali só há de voltar,
um dia, quando chover.
\end{quote}

Para ler um poema é necessário saber o siginificado de todas as palavras. Pedir
aos alunos para pesquisar o siginificado das palavras “agouro”, “acauã” e
“ribançã”. Por que a acauã muda de lugar? Qual a relação entre a acauão e o
Nordestino descrito em “Emigração e as consequências”? 

\end{itemize}

\section{Sugestões de leitura para o professor} 

\begin{description}\labelsep0ex\parsep0ex
\newcommand{\tit}[1]{\item[\textnormal{\textsc{\MakeTextLowercase{#1}}}]}
\newcommand{\titidem}{\item[\line(1,0){25}]}

\tit{DIEGUES JÚNIOR}, Daniel. \textit{Literatura popular em verso}. Estudos. Belo Horizonte: Itatiaia, 1986. 

\tit{MARCO}, Haurélio. \textit{Breve história da literatura de cordel}. São Paulo: Claridade, 2010.

\tit{TAVARES}, Braulio. \textit{Contando histórias em versos. Poesia e romanceiro popular no Brasil}. São Paulo: 34, 2005.

\tit{TAVARES}, Braulio. \textit{Os martelos de trupizupe}. Natal: Edições Engenho de Arte, 2004 

\end{description}


\end{document}
