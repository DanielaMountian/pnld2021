\chapter{Vida e obra de Zé Vicente}

\section{Sobre o autor}

Zé Vicente, pseudônimo de Lindolfo Marques de Mesquita, foi um dos
principais cordelistas brasileiros. Nascido ainda no século XIX, em 11 de
janeiro de 1898, em Belém do Pará, Lindolfo foi jornalista, político,
editor e poeta, entretanto sempre separou sua produção poética de sua
vida pública. Até os anos 1920 trabalhou como repórter no jornal \textit{Folha
do Norte}, momento em que inventou seu pseudônimo para assinar algumas
crônicas humorísticas, que mais tarde levaria para \textit{O Estado do Pará}.
Concomitantemente à carreira de jornalista, Lindolfo era funcionário
público, mas perdeu seu emprego e foi demitido do jornal, com a chegada
de Vargas ao poder em 1930, já que era partidário da oposição. Após um
período no Rio de Janeiro volta ao Pará, agora identificado com o
governo, onde exerce diversos cargos públicos chegando a ser prefeito
da cidade de Vigia, durante o Estado Novo, e diretor do Departamento
Estadual de Imprensa e Propaganda (\textsc{deip}), espécie de sucursal estadual
do Departamento de Imprensa e Propaganda (\textsc{dip}). 
As atividade poéticas de Zé Vicente acompanham, portanto, a
trajetória política de Lindolfo, sendo partidário do
``baratismo'' espécie de defesa dos pobres, transformando-as 
em longos poemas políticos sobre o período varguista.

\section{Sobre a obra}

%\subsection{Síntese dos poemas}

\paragraph{``A greve dos bichos''}

Num mundo controlado por animais, cada um deles possui uma função na
máquina governamental, associando as características físicas desses
animais à sua função social. Em determinado momento, o Quati, segundo
o autor, metido a sabichão, propõe uma revolta. A greve é reprimida
pelo governo, mas em uma assembleia decide-se que o Jacaré,
chefe do governo, deve ser deposto. A Onça mata o Jacaré e assume o
governo, mas passa a reprimir os grevistas. 


\paragraph{``O Brasil rompeu com `eles'''}

O autor narra o momento que o estado brasileiro, em 1942, 
após a ratificação pelo governo da chamada Carta do Atlântico, 
rompe com as potências do eixo -- Alemanha, Itália e Japão --, e se une
aos aliados. Caracterizando cada país do eixo, e defendendo que são
povos vis, o poema avisa que a partir dali quem simpatizar com eles
passa a ser um inimigo.

\paragraph{``O azar, a cruz e o diabo. Divertida história do homem
mais azarado do mundo''}

História de um homem que nasce no dia de finados e que desde seu
nascimento só traz desgraça para o mundo. A mãe morre no parto, assim
como praticamente toda a família, em decorrência desse acontecimento.
Aos dez anos, azarado e sem muitas perspectivas, é convidado pelo diabo
a visitar o inferno, onde vê muitas das punições ali existentes. Quando
convidado a ficar, esse garoto tira do bolso uma cruz com a imagem de
Jesus Cristo, causando grande confusão no inferno.

\paragraph{``Peleja de Chico Raimundo e Zé Mulato''}

Típica disputa entre cantadores, modalidade de 
poesia oral muito praticada no Nordeste. No dia da
festa de São João, organizada por Seu Quincas, Chico Raimundo espera a
chegada de Zé Mulato, cantador de grande fama, que anda pelo sertão
destruindo a fama dos cantadores locais. A narrativa apresenta a
disputa entre ambos, e extrapola pela via do poema e da canção, aquilo
que poderia ser uma disputa em armas, já que se trata da defesa do
legado de Chico Raimundo, contra a chegada de um forasteiro.


\paragraph{``Combate e morte de `Lampião'''}

História do surgimento de Lampião, desde seu nascimento, no interior de
Pernambuco, até sua morte, na fazenda Angicos, no inteiro do Sergipe. O
autor narra as atrocidades de Lampião e a disputa pessoal que se cria
entre ele e o tenente Bezerra, seu captor e executor.

\paragraph{``O golpe do seu Gegê ou o choro dos deputados''}

Em 1937, o poema narra o momento em que Getúlio Vargas dissolve a câmara
dando o primeiro passo para a implantação do Estado Novo. Favorável a
Getúlio, o autor apresenta os deputados de vários estados e
principalmente os do Pará, seu estado de origem, como senhores
corruptos e pouco dedicados ao trabalho.

\paragraph{``Peleja de Armando Sales e Zé Américo''}

Retomando a forma de disputa entre cantadores, o autor apresenta uma
disputa entre os dois políticos que seriam candidatos à presidência no
ano de 1938, caso Getúlio Vargas não tivesse constituído o Estado Novo
e se mantido no poder. Sobre a disputa vale ressaltar que o autor opõe
a origem dos políticos, o primeiro membro da oligarquia paulista e o
segundo da oligarquia da Paraíba, assim como retoma a disputa entre
cantadores e homens no sertão. Um dado importante é que Sales seria o
candidato da oposição e Américo o da situação. 


\section{Sobre o gênero}

A poesia de cordel, dizem os especialistas, é uma poesia escrita para
ser lida, enquanto o repente ou o desafio é a poesia feita oralmente,
que mais tarde pode ser registrada por escrito. Essa divisão é muito
esquemática. Por exemplo, o cordel, mesmo sendo escrito e impresso para
ser lido, costumava ser lido em voz alta e desfrutado por outros
ouvintes além do leitor. A poesia popular, praticada principalmente no
Nordeste do Brasil, tem muita influência da linguagem oral, aproveita
muito da língua coloquial praticada nas ruas e na comunicação
cotidiana. 

Naturalmente, portanto, pode-se considerar a poesia narrativa do cordel
uma forma de poesia mais compartilhada e desfrutada coletivamente, o
que lhe dá também uma grande ressonância social. Muitos dos temas do
cordel são originários das tradições populares e eruditas da Europa
medieval e moderna. Encontramos temas retirados das novelas de
cavalaria medievais e das narrativas bíblicas. Como no caso de
``O azar, a cruz e o diabo. Divertida história do
homem mais azarado do mundo''. Ao lado destes temas
mais literários, encontram-se os temas locais, quase sempre narrados na
forma de crônicas de coisas realmente acontecidas, como em
``Peleja de Chico Raimundo e Zé
Mulato''. E as chamadas reportagens jornalísticas que,
no caso de Zé Vicente, misturam-se a poemas de exaltação política do
período varguista, caso de ``O Brasil rompeu com
`eles''' e ``O golpe do seu Gegê ou O choro dos
deputados''. Também há as histórias fantásticas, que
se valem das tradições semirreligiosas, ligadas à experiência com o
mundo espiritual. 

Os grandes poemas de cordel são perfeitamente metrificados e rimados. A
métrica e a rima são recursos que favorecem a memorização e
tradicionalmente se costuma dizer que são resquícios de uma cultura
oral, na qual toda a tradição e sabedoria são sabidas de cor. 


\subsection{O sertão geográfico e cultural}

O sertão tem mitos culturais próprios. Contemporaneamente, o sertão
evoca principalmente o sofrimento resignado daqueles que padecem a
falta de chuva e de boas safras na lavoura. Evoca a experiência
histórica de uma região empobrecida, embora tenha sido geradora de
riquezas, como o cacau e a cana-de-açúcar, ambos bens muito valiosos. 

O sertão formou também o seu imaginário por meio de grandes
personalidades e uma pujante expressão artística. Além do cordel, o
sertão viu nascer ritmos tão importantes quanto o forró e o baião.
Produziu artistas tão expressivos quanto Luiz Gonzaga, grande cantor da
vida do sertanejo em canções como ``Asa
branca''. Um escultor como Mestre Vitalino criou toda
uma tradição de representação da vida e dos hábitos sertanejos em
miniaturas de barro. A gravura popular, que sempre acompanha os
folhetos de cordel, também floresceu em diversos pontos e ficou mais
famosa em Juazeiro do Norte, no Ceará, e em Caruaru, no estado de
Pernambuco. 

Dentre os grande mitos do sertão, está certamente o do cangaço com seu
líder histórico, mas também mítico, Virgulino Ferreira, o Lampião. Até
hoje as opiniões se dividem: para alguns foi um grande homem, para
outros um bandido impiedoso. Neste volume, o leitor vai encontrar 
um cordel dedicado ao mito do cangaço (``Combate e morte de `Lampião'''). 

Uma figura muito presente na cultura nordestina é o Padre Cícero Romão,
considerado beato pela Igreja Católica. Consta que teria feito milagres
e dedicado sua vida aos pobres. 

\subsection{Variação linguística}

A linguística moderna usa o termo
``idioleto'' para marcar grupos
distintos no interior de uma língua. Um idioleto pode ser a fala
peculiar de uma região, de um grupo étnico ou de uma dada profissão. 

Uma das grandes forças da poesia popular do Nordeste se origina em sua
forma muito própria de falar, com um ritmo muito diferente dos falares
do sul, e também muito diferentes entre si, pois percebe-se a diferença
entre os falares de um baiano, um cearense e um pernambucano, por
exemplo.

Além desse aspecto rítmico, quase sempre também há palavras peculiares a
certas regiões. 
%\pagebreak
%\section{Sugestões de atividades}
%\begin{enumerate}
%
%\item \textit{Atividade de leitura}. Esta atividade tem por objetivo sensibilizar os
%alunos para a escuta de poesia. O professor deve ler um conjunto de
%estrofes para exemplificar uma leitura que se construa com uma
%pronúncia clara, ritmo, pausas e ênfases adequadas. Após isso, cada aluno deve
%ler uma estrofe, procurando marcar o ritmo e as rimas, bem como as
%pausas e ênfases expressivas. O trabalho de leitura pode auxiliar o
%professor na realização de um diagnóstico dos alunos, em relação à
%pontuação e ao ritmo do texto, além de possibilitar um desenvolvimento
%da percepção da voz e da fala como meios indispensáveis à boa
%convivência social.
%
%
%\item No poema ``A greve do bichos'', Zé
%Vicente utiliza-se de animais para narrar um acontecimento que só
%poderia se dar entre seres humanos. Esse recurso é muito utilizado
%nas fábulas, em que as características físicas dos animais contribuem
%para a caracterização da função desse personagem na sociedade. Alguns
%desses animais tem traços de caráter muito fecundos para o
%desenvolvimento da narrativa. O professor pode pedir aos alunos que
%pesquisem as tendências de caráter de alguns desse animais e em seguida
%compare-as com a função que eles exercem no texto. Bons exemplos para
%isso são o Quati, o Boi, o Galo, o Jacaré e a Onça. 
%
%
%\item Ainda em ``A greve dos bichos'', o
%autor apresenta uma reflexão sobre esse acontecimento, sugerindo que a
%greve tem pouca ou nenhuma repercussão política. Entretanto, o poema
%apresenta analogias importantes sobre o desenvolvimento dos processos
%políticos, em especial, o momento em que a greve é conclamada, a
%repressão contra ela, a assembleia geral, a derrubada do Jacaré,
%espécie de atentado ou golpe realizado pela Onça e o período de chegada
%do poder da Onça, quando ela acaba por comer todos os grevistas,
%situação análoga de uma ditadura. O professor pode pedir aos alunos que
%separados em grupos caracterizem cada um desses momentos do texto e em
%seguida, realizem uma pesquisa histórica sobre os momentos em que casos
%como esses aconteceram. Como elemento conclusivo da discussão o
%professor pode trabalhar ainda um trecho do livro ou do filme
%``A revolução dos bichos'', de George
%Orwell, e comparar como esse autor utiliza-se dos animais para
%apresentar seus personagens, atentando como essa estratégia narrativa,
%bem como esses acontecimentos políticos, são recorrentes na literatura. 
%
%
%\item Em diversos poemas do livro o autor apresenta temáticas políticas.
%Em ``O golpe do seu Gegê ou o choro dos
%deputados'', Zé Vicente apresenta uma visão de apoio às
%posições tomadas pelo governo. A partir da leitura do texto, o
%professor pode pedir para os alunos expressarem qual é o acontecimento
%que está sendo narrado. Em seguida, realizar uma discussão sobre quais os
%argumentos utilizados pelo autor para justificar a dissolução da
%câmara, realizada por Getúlio Vargas, e por fim realizar uma discussão
%sobre a utilização do vocábulo ``negrada'' como forma pejorativa
%para os deputados, mesmo sabendo que no momento de escrita do poema os
%deputados não eram negros, e ao menos institucionalmente a
%escravidão no Brasil havia acabado quase cinquenta anos antes.
%
%
%\item No poema ``A peleja de Chico Raimundo e Zé
%Mulato'', Vicente narra a disputa entre dois
%cantadores em uma festa de São João. O texto é fecundo em variações
%linguísticas que podem ser analisadas pelos alunos. O professor pode
%pedir para que os alunos releiam as quinze primeiras estrofes do poema e
%apresentem os vocábulos que eles não conhecem, bem como aqueles para 
%os quais conhecem sinônimos. Alguns exemplos
%desses vocábulos são: aluá, macaxeira, banzé, mutuca e jerimum. A
%partir desse levantamento o professor pode mostrar como a variação
%linguística contribui para a formação do poema, principalmente no
%modelo de sextilhas utilizado por Zé Vicente, em que, em geral, há
%sempre uma alternância de versos para a composição das rimas.
%
%
%\item Em ``Combate e morte de
%`Lampião''', a história do cangaceiro Virgulino Ferreira da Silva é narrada pelo
%autor. Entretanto, esse poema é resultado de toda uma atmosfera dos anos
%1930, que coincide com o momento em que Lampião é capturado e morto
%pelas tropas do governo. Sendo assim, o professor pode comparar a visão
%trazida pelo poema com a visão dos alunos sobre Lampião, hoje visto
%por algumas pessoas como uma espécie de Robin Hood do sertão. Para
%contribuir com a análise o professor pode ainda utilizar-se dos dados
%biográficos do autor e de seu alinhamento com o governo de Getúlio
%Vargas, bem como da campanha anticangaço realizada nesse período, que
%teve como ponto alto a morte de Virgulino Ferreira em 1938. 
%
%
%\item O poema ``O Brasil rompeu com
%`eles''' narra o
%momento em que o Brasil em meio à Segunda Guerra, afasta-se das
%potências do eixo -- Alemanha, Itália e Japão. Para isso, o autor
%estabelece uma caracterização de cada um desses povos e investe numa
%campanha contra todos aqueles que os apoiam e, por consequência,
%contra os cidadãos dessas nações que vivem no Brasil. 
%O professor pode pedir aos alunos que definam qual papel o autor
%atribui aos japoneses durante a guerra. Em seguida, discutir a presença, no
%texto, de uma incitação à desconfiança e, a partir disso, refletir com os
%alunos como um elemento externo ao país teve ressonâncias internas,
%associados ao fantasma da Quinta-coluna, também citada no texto. Para
%aprofundar a discussão, o professor pode trazer trechos do livro ou do
%filme \textit{Corações sujos}, escrito
%por Fernando Moraes, que narra perseguição aos japoneses no Brasil,
%durante o período da Segunda Guerra Mundial.
%Como estímulo à reflexão, pode-se indicar os trechos abaixo:
%
%
%\begin{verse}
%
%Japonês foi traiçoeiro\\*
%Contra a América do Norte,\\*
%Mas na sua falsidade\\*
%O Japão não teve sorte.\\*
%Agora, é vivo, ele vai,\\*
%Sentir o frio da morte.
%
%Japonês andou fingindo\\*
%Que era anjo de candura, \\*
%Mas de repente mostrou\\*
%Quanto tem a cara dura,\\*
%Agredindo de emboscada\\*
%Pensando que era bravura.
%
%[\ldots{}]
%
%Brasileiro tem cuidado,\\*
%Sê um firme sentinela.\\*
%Defende nosso Brasil\\*
%Contra a perfídia amarela.\\*
%Não te fies na conversa\\*
%Que te jogam por tabela.
%
%
%\end{verse}
%
%
%\item Nos poemas do livro, Zé Vicente apresenta por duas vezes uma situação
%descrita como peleja, sendo que no primeiro caso essa é um peleja entre
%cantadores e no segundo entre políticos. A partir da leitura dos dois
%poemas, o professor pode pedir aos alunos que apresentem as diferenças
%entre esses dois poemas, no tocante a sua localização geográfica, à
%maneira como os cantadores e os políticos utilizam seus argumentos e
%ao resultado da peleja. Em seguida, o professor pode conduzir a
%discussão para as diversas correlações históricas que o gênero da peleja possui 
%-- tendo início nas disputas entre nobres na Baixa Idade
%Média, sendo transferido para o sertão na disputa entre dois homens, na
%disputa entre cantadores e na disputa entre políticos -- mostrando
%aos alunos como o exercício do cantar no sertão muito se assemelha ao
%exercício do político orador, já que ambos procuram convencer os
%demais de suas qualidades, mesmo que para isso utilize-se tanto da
%exaltação de seus feitos, quanto da desqualificação dos feitos de 
%seus oponentes. 



\begin{bibliohedra}

\tit{DIEGUES JÚNIOR}, Daniel. \textit{Literatura popular em verso}. Estudos. Belo Horizonte: Itatiaia, 1986. 

\tit{MARCO}, Haurélio. \textit{Breve história da literatura de cordel}. São Paulo: Claridade, 2010.

\tit{TAVARES}, Braulio. \textit{Contando histórias em versos. Poesia e romanceiro popular 
no Brasil}. São Paulo: 34, 2005.

\tit{TAVARES}, Braulio. \textit{Os martelos de trupizupe}. Natal: Edições Engenho de Arte, 2004


\end{bibliohedra}