\chapterspecial{O {surgimento} {da} {banana}}{}{}
 

 

\letra{A}{ história da banana"-pacovã.} No início era assim. Nossos antepassados
surgiram e não sabiam plantar bananas. Não fosse por isso, não haveria
essas bananeiras. Não teria aparecido esse tipo de banana. 

Como pensou e agiu aquele que fez surgir a banana, depois de morar e se
estabelecer? Geralmente a gente vai à mata e encontra um lugar como se
alguém tivesse roçado, um lugar queimado e limpo, bem no meio da selva.
A gente chama esse lugar de ``queimado do Fantasma''. Nesse tipo de
lugar se encontra um telhado de palha, como aquele que nós costumamos
tecer. 

Embora ninguém tenha dito ao Fantasma, ``teça as palhas assim!'', ele as
teceu, apesar de ninguém ter ensinado para ele. Depois de Horonamɨ ver o
queimado, ele encontrou o Fantasma, dono do queimado, que morava ali.
Nesse tipo de lugar, erguem"-se os pés de sororoca, que são semelhantes
às bananeiras, mas não dão banana. 

O surgimento das bananeiras, não foi porque o Fantasma cortou, queimou e
roçou a sororoca. Ele não as plantou. Elas simplesmente surgiram no dia
seguinte. 

Proto! \emph{Pauximɨ}! Proto! \emph{Rokomɨ}! Proto! \emph{Monarimɨ}!
Proto! \emph{Pakatarimɨ}!
Proto! \emph{Nakoaximɨ}! \emph{Rokoya}! \emph{Rokoroko}! \emph{Roorewë}! Saíram
somente as bananeiras delas mesmas. Dois dias depois, o Fantasma voltou
ao lugar onde havia queimado as sororocas e viu que haviam nascido
também batatas"-doces. Não foi em outros xapono que ele pegou. Lá onde
Fantasma tinha seus alimentos, onde havia as bananeiras, as sororocas se
transformaram em bananas"-pacovãs e a batata"-doce surgiu. Ali também dava
cará, ária, pimenta e o mamoeiro. Foi o Fantasma que fez aparecer as
bananeiras. Elas vêm do Fantasma. 

Por que ele as fez aparecer? Porque ele tinha um filho, que ele tinha de
alimentar. 

Ao ouvir a voz do filho do Fantasma, Horonamɨ descobriu a sua moradia e
pegou com ele umas mudas de bananeira. 

O Fantasma não tinha outros parentes. Ele mostrou aos Yanomami que é
possível ter somente um filho. Ele fez apenas um filho, apesar de sua
esposa ser moça. Agora, ele não é mais pajé como era em vida. 

Aquele que vinha, Horonamɨ, encontrou as bananeiras e pediu mudas ao
Fantasma. Quando não existiam nem roças, nem Yanomami, depois de
Horonamɨ pegar as bananeiras, ao chegar ao seu xapono, ele deu o nome a
elas, deixando com isso o ensinamento de como plantar as bananeiras. Ele
as pegou para nós as termos. Até hoje existem as bananas de diferentes
variedades: \emph{rokomɨ}, \emph{nakoaximɨ}, \emph{rokoya}, \emph{pauximɨ,
monarimɨ, pakatarimɨ}. Assim foi. 

Nossos antepassados e os antepassados dos \emph{napë} não comeram banana
desde o início. Hoje, tanto os \emph{napë} quanto os Yanomami plantam
bananas, a partir do ensinamento de Horonamɨ.

 

\section{Como os napë descobriram a banana}

 

\emph{Quando a menina yanomami tem sua primeira menstruação, ela fica em
reclusão por um período entre uma semana e dez dias, dentro de um
pequeno cômodo feito de folhas de açaí no xapono. Essa reclusão a
protege do assédio de espíritos num momento em que ela fica em
evidência. Aqui a moça atrai o interesse do rio, que a carrega para fora
do} xapono\emph{ para se casar com ela.}

 \asterisc{}

Como aconteceu a descoberta da banana pelos \emph{napë}? Qual foi o
Yanomami que levou as bananeiras aos \emph{napë}? Ninguém levou as mudas
de bananeira aos \emph{napë}. Uma mulher estava de reclusão. A água saiu
e as roças afundaram. Essa água levou a mulher e por onde a levou, levou
também as bananeiras afundadas, até aonde os \emph{napë} vivem; foi o
rio que levou as bananeiras para que eles, os \emph{napë} as
descobrissem. O rio desejava a mulher menstruada porque ela era bonita.
No que ela se tornou? O rio a levou porque a desejava. Da mulher
menstruada que as águas levaram, sua imagem se espalhou nos rios.
Multiplicou"-se a partir dela mesma. Foi a água que a pegou. O rio
disse: 

--- Meu sogro, quero uma mulher! Me dê a sua filha! 

O rio entrou, perseguindo a mulher. O rio entrou rápido. Olha só a água!
Ela entrava por trás das casas, apesar de a terra ser alta. 

--- Prako! Prako! --- disse o grande rio. 

O pai mandou pintar a filha, nessa hora ele a pintou, seu irmão a
pintou. O pai mandou seu filho pintá"-la. Ele estava com muito medo de se
afogar na água, que vinha ameaçadora, se mexendo como em plena
tempestade. A água se mexia com grandes banzeiros, nos quais foi
jogada a mulher pintada, apesar da sua beleza. Seu pai a fez afundar. O rio levou a sua filha, e não a devolveu. Ela não se afogou, e o rio a
levou como sua esposa. 

--- Eu, apesar de ser água, farei dela a mãe d'água! Eu vou pegá"-la ---
disse o rio. 

Por isso, esta Yanomami se tornará a mãe do rio. O rio se retirou.
Depois de pintarem seu rosto com desenhos bonitos, colocaram penas de
cauda de papagaio nas suas orelhas. Feito isso, as folhas de açaizeiro
da reclusão foram removidas e a água entrou. O xapono dele era como os
nossos. 

--- Mãe! Mãe! Pinte minha irmã! Enfeite"-a! Enfeite"-a depressa! --- disse
o irmão da moça.\footnote{   A moça enfeitada normalmente seria entregue a um marido humano, não a um
marido rio.}

--- Essa ideia dói muito, meu filho, mas não tem jeito, entregue mesmo
tua irmã! 

Apesar de se ser o rio, assim falou o pai. Ele mandou entregar a filha.
Foi assim que ele disse. Existe um canto sobre a mulher levada pelo rio,
há um canto sobre ela: 

\emph{Xiri tõi!} 

\emph{Xiri tõi,} 

\emph{Xiri tõiwë,} 

\emph{Xiri tõi,} 

\emph{Xiri tõi,} 

\emph{Xiri tõi,} 

\emph{Xiri tõiwë!} 

Ela cantou. Quando ela pronunciou o nome de seu marido, o rio
respondeu: 

--- \emph{Tuuuuuuuuuuuu}!

--- \emph{Xiri tõi! Xiri tõi! Xiri tõi!} --- cantou o pai. 

Ele falou assim, cantou assim e, quando parou de cantar, o xapono quase
caiu, levado pelo rio. O irmão a pegou para jogá"-la, apesar de ela estar
chorando. Ela chorava, por causa do seu irmão: 

--- \emph{Ɨ̃ɨaaaɨ̃ɨ}! Meu irmão! Meu irmão! Não fique triste! Meu pai! Meu
pai! Não fique triste! Minha mãe! Minha mãe! Não fique triste! --- disse
ela. 

Enquanto ela chorava assim, o irmão a pegou. 

--- Hɨ̃ɨ! Kopou! --- ele a jogou de cabeça. 

Fazendo assim, a água a pegou e logo a levou. O rio cheio a estava
esperando. Quando o rio se retirou, revelou uma grande extensão de
terra. 

---Puuu! --- disse o rio. 

Foi assim, o rio desceu de uma vez só. 

--- Aëëë! --- ela disse. 

A mulher se tornou boto, aquele que boia na superfície da água, pois a
jogaram na água quando ela estava menstruada; ela estava de reclusão, a
vagina dela estava ainda sangrando. Por isso se tornou a mãe da água. A imagem dela se espalhou e ocupou todos os rios. Aquelas
bananeiras \emph{rokoroko} que a água levou, bem como as pacovas, se
multiplicaram na terra dos \emph{napë}. Assim foi, as bananeiras se
multiplicaram. 
