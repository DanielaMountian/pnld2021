\titulo{trabalhos e dias} % Em minúsculas!!!
\autor{Hesíodo}  % Apenas sobrenome, se for o caso. Verificar capa.% 
\organizador{Tradução e organização}{Christian Werner}   %Conferir se é apenas {Organização}; {Organização e tradução} ou apenas {Tradução}%
\isbn{978-85-7715-320-6}
\preco{22}   % Ex.: 14. Não usar ",00"%
\pag{92}   % Número de páginas
\release{\textit{Trabalhos e dias} (em grego \textit{Erga kai Hamerai}) é um
poema épico de 828 versos em que são contados alguns dos mitos gregos mais
conhecidos até hoje, como o de Prometeu e o de Pandora. Diferente da
\textit{Teogonia}, que apresenta a origem dos deuses, este poema é voltado para
a condição dos mortais, explicitando suas necessidades e limitações, com foco
no trabalho agrícola baseado nas estações do ano. Com a ajuda das Musas, o
poeta narra em primeira pessoa e se dirige a seu irmão Perses, na tentativa de
ensinar a ele verdades divinas a respeito das práticas humanas.
Em nova tradução de Christian Werner, esta edição bilíngue do clássico grego conta com introdução e notas explicativas.
\textbf{Sobre o autor:}
\textsc{Christian Werner} é professor livre-docente de língua e literatura
grega na Faculdade de Letras da Universidade de São Paulo (\textsc{usp}).
Publicou, entre outros, \textit{Duas tragédias gregas: Hécuba e Troianas}
(Martins Fontes, 2004).}
