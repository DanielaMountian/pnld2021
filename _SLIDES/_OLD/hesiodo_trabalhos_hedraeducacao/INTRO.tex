

\chapter[Introdução, por Christian Werner]{Introdução}  
\hedramarkboth{Introdução}{Christian Werner}


\epigraph{Riobaldo, a colheita é comum,
mas o capinar é sozinho\ldots{}}{J.~Guimarães Rosa, \textit{Grande sertão: veredas}}

\epigraph{%
  Nos deuses [confia]: amiúde para longe de males\\
  endireitam varões que jazem na negra terra\\
  e amiúde derrubam até os que estão muito bem\\
  e os deitam de costas; ocorrem depois muitos males,\\
  e vaga carente de sustento e confuso no espírito.
}{Arquíloco (século \textsc{vii} a.~C.)}

\textit{Trabalhos e dias} é o poema grego no qual se mencionam Pandora e
sua caixa, as linhagens ou raças do homem (a “linhagem de ouro”) e uma
bela representação das estações do ano (três) com as atividades
agrícolas associadas. Além de trechos propriamente
mitológicos e de uma figuração ético-poética da vida agrícola, tópicos
morais, políticos e religiosos compõem esse poema que, em certa medida,
utiliza a mesma linguagem das narrativas épicas de Homero. Nele, porém,
não é de super-homens como Aquiles e Odisseu que se fala, mas de outros
tipos de heróis: o poeta, que de tudo sabe; o  bom rei, que zela para
que a justiça se faça presente na sua comunidade; e o agricultor bem
sucedido, que, para produzir riqueza através de sua propriedade, deve
não só trabalhar arduamente, mas atentar a uma série enorme de
regularidades climáticas, morais e religiosas.

Além de \textit{Trabalhos e dias}, somente chegaram inteiros até nós os
poemas \textit{Teogonia} e \textit{Escudo de Héracles} entre aqueles
atribuídos na Antiguidade ao grego Hesíodo, poeta que teria vivido
entre os séculos \textsc{viii}-\textsc{vii} a.~C., ou seja, mais ou menos 
na mesma época que Homero, tido como o autor da \textit{Ilíada} e da
\textit{Odisseia}. Todavia, inúmeros aspectos relacionados à cultura
grega coeva que podem ser reconstruídos hoje com uma margem de erro razoavelmente 
grande \footnote{ Por exemplo, quase nada sabemos sobre o advento, a expansão e, sobretudo, 
os usos da escrita na Grécia na época dos poetas em questão.} fazem muitos pesquisadores 
duvidarem da existência de um poeta histórico chamado Hesíodo e que ele tenha
composto por escrito os poemas associados ao seu nome. Para analisarmos
as condições que propiciaram poemas como os citados, a arqueologia e a
história do Mediterrâneo Oriental em geral e da Grécia em particular
são uma importante ajuda; mas, para entender um poema como
\textit{Trabalhos e dias}, o próprio texto ainda é nossa principal
ferramenta, de sorte que muitas das questões a ele pertinentes
precisarão continuar sem uma resposta categórica. Isso não nos impede de
procedermos à investigação do funcionamento do próprio texto.

\section{voz poética}

A voz poética que enuncia os \textit{Trabalhos} certamente fala de si,
ou seja, de um poeta ao qual está associada. Por uma questão de
economia, ao longo desta introdução, usarei “Hesíodo” como sinônimo
dessa voz poética, embora este nome não seja, ele mesmo, mencionado
nenhuma vez ao longo do poema. Como destacaram Ford (1997), Scodel (2011)
e outros, o que particularmente distingue essa voz poética, diferenciando-a daquela
dos poemas homéricos, é o fato de ela se dirigir a um público específico com quem
o poeta mantém uma relação estreita, seu irmão Perses (e os reis).
Estamos longe do público impessoal pressuposto por um poema épico como
a \textit{Ilíada}, mas bastante próximos de certos tipos de poemas que
podemos chamar, de forma bem genérica, de líricos, os quais, no
universo grego arcaico, podiam ter um ou mais destinatários explícitos
(indivíduos e/ou a população de uma cidade).

Para muitos intérpretes, porém, nada impede que a representação de tudo
aquilo que diz respeito ao poeta dos \textit{Trabalhos} \footnote{ O mesmo vale
para a \textit{Teogonia} e para, pelo menos, uma certa parte da lírica
grega arcaica.} seja fruto não de sua existência histórica, mas intrínseco à tradição 
da qual o poema é dependente, que, por falta de material transmitido, é difícil de ser 
reconstruída.\footnote{ Uma analogia “caseira”: a poesia de Gregório de Matos Guerra.} 
Nesse sentido, Hesíodo seria um mito. Os eventos que perfazem um certo pano de fundo
do poema não são, porém, uma “mentira” ou uma “ficção”, mas sim, como
defende Nagy (1990), uma componente que contribui para a autoridade do
discurso como um todo, provido de eficácia no âmbito das
sociedades em que ele foi apresentado e pelas quais foi assimilado,
fazendo parte de um cânone. Dessa forma, Hesíodo e Perses não
seriam nem realidades históricas nem ficções, mas elementos de uma
tradição mitopoética em uma sociedade tradicional em que o conhecimento dependia 
de formas específicas, ligadas à oralidade, para ser apresentado e transmitido 
de geração em geração. Em última instância, e parafraseando Nagy, dificilmente
saberemos um dia, com segurança, se foi um Hesíodo histórico que deu
vida aos poemas ou se são poemas e tradições poéticas que deram vida à 
figura que passou a ser conhecida como Hesíodo.

De forma alguma esse problema deve ser subestimado ou, o que seria ainda
pior, tomado como impedimento para uma interpretação do poema (muito
pelo contrário: ele é um ótimo ponto de partida). Para o leitor moderno, 
a primeira impressão ao ler o poema pode ser a de que se está diante de 
uma obra carente de qualquer tipo de unidade, uma colcha de retalhos 
composta ao longo de muito tempo no âmbito de uma tradição. O desafio ao qual
somos direcionados pelos nossos hábitos de leitura é procurar entender
se há uma unidade por trás de partes nitidamente distintas, algum
núcleo formal e/ou temático que dê conta do todo que é um poema. 

Como primeira baliza na busca de algum princípio unificador, podemos
invocar o modo como na 1\textsuperscript{a} metade do século \textsc{xx} d.~C.
muitos eruditos entenderam a literatura grega arcaica: na precisa síntese de Versnel 
(2011, p. 214), ela foi descrita como “marcada por uma dicção paratática,
‘aditiva’ ou ‘aglutinadora’, contendo tais qualidades como abundância
(\textit{poikilia}), autonomia e predominância de partes separadas, sua
qualidade funcional, a ligação entre partes disparatadas e não
raramente contraditórias ou incompatíveis e a carência (aparente?) de
um conceito ou tema central unificador ou ligante”. Certamente podemos
concordar que essa descrição estilística se encaixa perfeitamente no
poema em questão. Ressalte-se, porém, a discreta interrogação acerca do
princípio unificador que não é totalmente deixado de lado.

Outra observação preliminar é que não estamos diante de uma narrativa: a
querela entre os dois irmãos da qual fala o poema (versos 27--39) não é
seu fio condutor do mesmo modo que o retorno de Odisseu o é na
\textit{Odisseia}, pois o poema não desenvolve uma sucessão de eventos
no tempo; aliás, ele praticamente nada diz dessa briga. Clay (2003) foi 
feliz ao adotar o termo “monólogo dramático” para dar conta do
tipo de interação comunicativa que está em jogo: de um lado, o sujeito
performador que assimila e apresenta a voz poética; de outro, seu
público, ou seja, nós. Nessa interação, a briga entre o poeta e seu
irmão certamente fornece um enquadramento logo no início do poema, mas
esse evento é minimamente abordado de forma direta, embora seja incorporado
e aflore às vezes mais, às vezes menos explicitamente no fluxo contínuo
composto por  histórias, conselhos e descrições. A briga, portanto, é
um primeiro princípio unificador bastante tênue como elemento de uma
construção lógica que permite ao ouvinte interligar trechos que,
(somente?) para um leitor moderno, talvez pareçam discretos. Mesmo que
considerássemos que o público primeiro do poema conheceria detalhes de
uma suposta briga real entre duas figuras históricas, isso
seria modificado na recepção posterior do poema.

\section{o poeta e zeus no proêmio}

Como em outras composições gregas da época desse poema e que
também usam sua unidade métrica, o hexâmetro datílico, os
\textit{Trabalhos} iniciam com um proêmio (1--10). Esse trecho é uma das
façanhas poéticas mais notáveis que conhecemos desse período (se não
for uma adição bem posterior, como já defendido na Antiguidade). Uma
sucessão de figuras de linguagem comprimidas em uma sequência sonora
quase encantatória, abundante em rimas, externas e internas, e outros
tipos de repetição, tornam o trecho virtualmente intraduzível. A
exibição do domínio desses recursos poéticos reforça a ação de Hesíodo,
a demarcação de sua voz e de sua autoridade em relação a duas outras 
autoridades geralmente presentes na poesia hexamétrica, as Musas e Zeus.

Assim, a densidade da linguagem pode ser o ponto de partida para uma
interpretação do trecho, sobretudo se aceitarmos, como defendido por
Calvert Watkins (1995), que a primeira palavra do proêmio
(\textit{M o\=usai}, “Musas”) como que está inscrita na sua última
(\textit{muth\=esaim\=en}, “poria em um discurso”); tal façanha nos fornece
um elemento fundamental para pensar o proêmio como um todo. Os dois
termos são altamente marcados, o que se evidencia a partir do seguinte
trecho do outro poema atribuído a Hesíodo: 


\settowidth{\versewidth}{“pastores rústicos, infâmias vis, ventres somente,xx}

\begin{verse}[\versewidth]
  Este discurso, primeiríssimo ato,\\> disseram-me as deusas,\\
  as Musas do Olimpo, filhas de Zeus\\> porta-égide:\\
  “pastores rústicos, infâmias vis, ventres somente,\\
  sabemos muita coisa enganosa falar\\> semelhante a genuínas,\\
  e sabemos, quando queremos, verdades\\> proclamar.
\end{verse}

\attrib{\textit{Teogonia}, 24--28}


\textit{Mo\=usai} diz respeito à instância metafísica sobre a qual se
apoia o cantor oral grego para justificar a veracidade, a autoridade do
seu canto. A forma verbal \textit{muth\=esaim\=en} (o verbo é derivado de
\textit{muthos}) refere-se a um discurso cuidadosamente elaborado por
meio do qual o falante quer fazer valer a sua autoridade, de
preferência, em um ambiente público. \footnote{ Essa interpretação do verbo e do
substantivo, no contexto da poesia épica, é de Martin (1989).}

Entre a autoridade das Musas e a sua própria, o poeta assume um discurso
que se aproxima maximamente da atuação de Zeus no mundo dos homens e,
com isso, parcialmente se afasta das Musas, aquelas encarregadas de
preservar as façanhas dos heróis épicos, guerreiros do passado. Os versos 3--8 
celebram Zeus, mas quase que exclusivamente através do modo como 
ele age sobre os homens, o que culmina com sua administração da justiça (7 e 9). 
A contiguidade entre os dois pronomes que abrem o verso 10, “tu” (Zeus) e 
“eu” (poeta), garante ao ouvinte que a eficácia do discurso deriva do conhecimento 
de Hesíodo e apoia-se na atuação de Zeus, a quem o poeta pede ajuda em um momento 
de tensão, marcado pelo imperativo “atende” (9). Ignorar as Musas seria impensável 
para um poeta do período; mas aqui elas parecem ser deslocadas para uma posição
marginal, e sua menção, do modo como é feita, talvez também sirva para
reforçar que neste poema não serão glorificados os heróis do passado.

Os versos seguintes (11--26) desenvolvem um tema tradicional, a oposição
entre as atividades ligadas à guerra e aquelas ligadas à paz, tema esse
que aparece no seguinte trecho da \textit{Odisseia}: 



\begin{verse}[\versewidth]
  Por certo audácia me deram Ares e Atena,\\
  e força rompe-batalhão. Quando escolhesse para tocaia\\
  varões excelentes, males aos inimigos engendrando,\\
  nunca o ânimo orgulhoso pressentiu minha morte,\\
  mas, após bem na frente saltar, com a lança matava,\\
  dos varões inimigos, quem me recuasse com os pés.\\
  Esse era eu na guerra; e o trabalho\\
  não me era caro\\
  e o senso doméstico que cria radiantes crianças,\\
  mas sempre as naus com remos foram-me caras,\\
  guerras, dardos bem polidos e flechas,\\
  funestos, que para os outros horripilantes são.\\
  Mas isso era-me caro, o que o deus no juízo pôs;\\
  cada varão se deleita em trabalhos distintos.
\end{verse}
\attrib{Homero, \textit{Odisseia} 14, 216--28}

\section{o herói agricultor}

Na passagem hesiódica, todo o valor positivo é deslocado para o trabalho
agrícola; para quem persegue a riqueza desse modo, conflitos, em que pese o caráter 
ambíguo, dos versos 25--26 parecem ausentes. O modelo de vida oferecido pelos 
\textit{Trabalhos} não permite empate; ele se pretende o melhor, e seu herói é 
o agricultor que domina uma pletora de diferentes atividades ao longo do ano para 
superar seu vizinho. Terá Hesíodo mencionado poetas que competem com poetas (26) 
para sugerir que seu poema tem essa pretensão, a de mostrar que a vida de seus heróis é
mais digna de ser admirada e emulada que aquela dos heróis homéricos?

Todo aquele que se envolve com atividades infrutíferas, estéreis,
contribui para o fortalecimento da Disputa censurável. O antídoto é um
só: trabalho - que, no poema, é essencialmente o trabalho agrícola, e,
secundariamente, com reservas, a navegação. Não estamos diante de algo
nem mesmo próximo de uma ética valorizadora do trabalho (e da riqueza)
em si. Uma interpretação possível do verso 41 (“quão grande valia há na
malva e no asfódelo”) é a de que os reis que talvez aceitem ser corrompidos
por Perses (38--39) não percebem o valor de uma vida que se satisfaz com
o mínimo (a malva e o asfódelo compõem uma alimentação deficiente; suficiente, 
talvez, apenas para um asceta). Quanto ao trabalho, ele é um mal necessário 
(42--44, 90--92), mas, ao mesmo tempo, aliado da justiça. Quando os homens se 
dedicam a atividades cujo ganho se dá sem trabalho, toda a sociedade perde, 
inclusive os reis que deveriam zelar pelo seu bom funcionamento.\footnote{ “Reis” 
traduz o plural do substantivo grego \textit{basileus}; não se trata de um rei
no sentido estrito, mas de um nobre cujo poder político em uma
comunidade o torna uma figura central na administração da justiça.} O
poema é, em boa parte, um discurso persuasivo; o irmão de Hesíodo deve
ser movido a adotar um modo de vida mais benéfico para si e para a
sociedade da qual ele participa.

Um dos recursos retóricos que Hesíodo manobra ao longo de todo o
poema são os enunciados mais ou menos enigmáticos. Um exemplo é o verso
mencionado no último parágrafo: ele deve ser entendido ironicamente?
Ou literalmente, talvez propondo um saber do conhecimento de poucos, paradoxal 
para a maioria? Ou enigmaticamente? Outras expressões, por não remeterem 
diretamente ao referente e pelo seu provável grau de estranheza (“cinco-galhos” 
por mão, por exemplo, no verso 742), pedem para ser decifradas pelos ouvintes. 
Nada impede que a tradição à qual pertencia o poema já tivesse condicionado aqueles 
com ela familiarizados a ouvirem e interpretarem tais enunciados. No trecho
inicial do poema, por exemplo, o ouvinte é claramente convidado a
estranhar a existência de duas famílias de Disputas. Na sequência, 
mesmo que se trate de um provérbio comum, o ouvinte não pode identificar 
se o verso 41 (“tolos, não sabem quão maior é a metade que o todo”) também 
diz respeito a Perses -- mais valeria para ele respeitar a divisão da herança 
paterna? mais vale a metade de uma propriedade ser bem gerida que um todo mal gerido? --, 
ou somente à sociedade na qual os reis dispensam sentenças. Por fim, nem sempre 
é claro quando certos tipos de construção não têm a finalidade de serem principalmente 
bem humorados.

A figura do irmão, assim, serve sobretudo de intermediária para
Hesíodo fazer o público decifrar seu poema e assimilar seu
raciocínio. Se Perses não escutar o irmão, ele cometerá um erro tão
tolo quanto aquele de Epimeteu, o irmão de Prometeu (47--105). A
história do roubo do fogo pelo Titã e a vingança de Zeus, que criou a
primeira mulher, Pandora, também é mencionada na \textit{Teogonia}, mas
em \textit{Trabalhos} ela é reconfigurada tendo em vista o discurso que
Hesíodo dirige ao irmão.

\section{os mitos: pandora e as linhagens do homem}

Tanto a história de Prometeu e Pandora quanto o mito das linhagens do
homem (106--201) giram em torno da separação entre homens e deuses; mais
especificamente, da necessidade dos homens de trabalhar para obter seu
sustento. Na primeira história, Zeus e os demais deuses não surgem como
figuras simpáticas aos mortais, mas, como notou Martin (2004),
é o desnível de conhecimento no âmbito de duas duplas – Zeus/Prometeu e
Prometeu/Epimeteu – que encaminha a história, no que se reproduz a
moldura discursiva do poema, a sapiência de Hesíodo sobrepondo-se à
tolice de Perses. Vale dizer que Perses não é considerado um caso
perdido, não é desprezado por Hesíodo. Se assim fosse, não haveria
razão para o discurso existir. Hesíodo certamente se apresenta como
sumamente sábio, mas o problema do irmão, assim como o
dos heróis homéricos que costumam ser chamados de \textit{nêpios}
(nesta tradução de \textit{Trabalhos e dias}, sempre “tolos”), é não se
darem conta, na hora correta, de consequências danosas de suas ações,
ou seja, ao modo de Epimeteu.


Voltando à história de Prometeu, em um primeiro nível, o fogo
escondido por Zeus e trazido de volta por Prometeu aponta para as
conquistas técnicas que tornam a agricultura uma atividade mais simples
e produtiva. O fogo, porém, também é análogo à destruição consumidora
consubstanciada na primeira mulher. Tecnicamente impecável e altamente
sedutora, a mulher é um mal que se opõe ao homem trabalhador. Nesse
sentido, ela está do lado de Perses e dos reis que aceitam serem
subornados. Os presentes que a compõem – Pan-dora, “todos os presentes”,
as dádivas que cada deus confere à criatura moldada por Hefesto – são
enganadores, não trazem bem algum para os homens e, assim como os
presentes que ganham os reis “come-presente” (39) e a parte do
patrimônio que Perses tentou (ou tenta) conquistar injustamente, têm
um efeito danoso no futuro. Os males futuros, porém, não se anunciam
mais aos homens; Zeus lhes tirou a voz, o contrário do que determinou
fosse feito com Pandora. Entre os homens, o que resta é dar ouvidos ao
sábio que conhece a origem desses males desconhecidos. Essa é uma das
faces da “esperança”, essa entidade naturalmente ambivalente que também
dá conta da expectativa depositada pelos reis e Perses no lucro que
podem ganhar sem trabalho. Para quem percebe a falácia dessa
expectativa, resta enunciar e defender que Zeus “endireita o torto”
(7).

Se o foco da história de Prometeu é a dificuldade da vida humana (em especial, 
a necessidade do trabalho como garantia do sustento, diferença marcante entre 
a vida humana e a divina), na história seguinte acerca das linhagens do homem, 
a relação principal não é mais vertical, mas horizontal: os modos diversos dos 
homens agirem reciprocamente, ou bem pela justiça (\textit{dik\=e}), ou bem 
pela violência (\textit{hybris}). Esse, pelo menos, é um dos eixos da
influente interpretação de Vernant (2002), para quem ambas as categorias são 
aquelas que dão sentido ao mito.

A polarização mais clara que perpassa a história é, porém, aquela
mesma que separa os irmãos Hesíodo e Perses, o conflito (nos seus modos
diversos) e a sua ausência. A linhagem presente tem diante de si uma
dupla alternativa: ou a autodestruição (como as linhagens de prata, bronze
e a dos heróis), ou uma situação próxima daquela que viveram
os homens de ouro (225--47), no caso, uma cidade onde o máximo de
justiça garantirá o máximo de prosperidade para todo aquele que quiser
trabalhar sua terra. A tese do poema, portanto, torna-se clara: o
trabalho é um mal, mas somente se todos se dedicarem a ele a
humanidade poderá se reaproximar daquilo que perdeu para sempre, uma
vida sem trabalho e outras aflições. Nesse tipo de sociedade, o espaço
para injustiças geradoras de conflitos é mínimo. É tolice buscar vantagens 
na sua força quando se acreditar ser o mais forte em uma dada situação; 
uma série de mecanismos atrás dos quais estão Zeus e Justiça
farão com que o usurpador seja punido. Ou não?

Tanto o proêmio do poema quanto outros de seus trechos, sobretudo
aqueles que mostram que não basta ao homem trabalhar sua terra seguindo
todas as regras, pois o clima depende, em última instância, dos
desígnios de Zeus, indicam que o poder do deus é arbitrário. Esse
talvez seja o mistério – e fascínio – último do poema, entender como
pode ser Zeus arbitrário e justo. Por um lado, é responsável a
ignorância humana: “Sempre cambiante é a ideia de Zeus porta-égide, e
para os homens mortais é difícil apreendê-la” (483--84). Por outro lado,
alguns homens conseguem chegar perto da verdade; um deles é
Hesíodo, e todo aquele que por ele for persuadido seguirá seus
conselhos, mesmo que seu sucesso não seja cem por cento garantido:

\begin{verse}[\versewidth]
  Este o melhor de todos, quem por si tudo apreender\\
  ao refletir no que será melhor, depois e no fim.\\
  Distinto também o persuadido por quem fala bem.\\
  Quem por si não apreender nem, de outro ouvindo,\\
  lançar no ânimo, esse é um varão infrutífero.
\end{verse}

\attrib{\textit{Trabalhos}, 293--97}


Isso ajuda a explicar por que Hesíodo não enuncia claramente a moral da
fábula do falcão e do rouxinol (202--12). Por um lado, ela aponta para o
sucesso do mais forte, algo que todo homem, ao longo de sua vida,
parece testemunhar. Se atentarmos para o fato de que o rouxinol vale como
um poeta e que a voz de Hesíodo se faz notar explicitamente no verso
213 ao se dirigir a Perses, então os versos que seguem, nos quais
sobressai a oposição entre a Cidade Justa e a Cidade Injusta, compõem,
na verdade, a resposta dada pelo rouxinol-poeta ao falcão.
Não vale uma resposta presa a uma só dimensão temporal, assim como só
entendemos como funciona uma das linhagens do homem se a colocarmos em
oposição às outras; nem mesmo o fim da linhagem de ferro garantirá a
vitória inequívoca de um ou outro dos animais da fábula. Para Hesíodo,
porém, \textit{no contexto da linhagem de ferro}, a resposta do rouxinol é
a boa. A Justiça, a filha de Zeus (256--57), é um presente do deus
(279--80) e, como tal, a contrapartida de Pandora. A diferença entre as
duas é que Pandora e o que ela representa irremediavelmente estão entre
os homens; Justiça só fica entre eles se os homens a cultivarem.

A partir do verso 286, Hesíodo paulatinamente abandona o tema da
justiça, que é necessária, coletivamente, para o bem de todos, e passa
a abordar, de forma mais específica, o tema do trabalho, que é, em
primeiro lugar, uma ocupação individual. Nessa dobra do texto, a figura
negativa deixa de ser a criatura violenta, que só confia na própria
força, e passa a ser o inativo e/ou o mendigo; Perses passa a ser
localizado entre aquele que quer tomar a riqueza alheia e o inativo
faminto com vergonha de trabalhar (298--319). Entre o trabalho e a
inação está o roubo, uma das ações moralmente vis para a qual Zeus
garante compensação (320--34) e que, mais adiante, também merecerá um
trocadilho: o homem “sono-diurno” é o ladrão (605). Ao fazer o elogio
do trabalho em oposição à inação, o pessimismo fica de lado quanto à
ordem garantida por Zeus; para uma sociedade na qual o trabalho é a
opção da maioria, Zeus está presente e não vale o que enuncia o falcão
ao rouxinol.

\section{<<almanaque agrícola>>}

O trecho que inicia no verso 286 e vai até o verso 380 e os versos
694--764 apresentam, na forma de um catálogo, uma série de sentenças
morais, preceitos e conselhos, típicos da literatura sapiencial, que
flanqueiam o calendário no qual se desdobram as atividades próprias da
agricultura e da navegação. O “tu” ao qual se dirige o poeta não é mais
necessariamente apenas Perses, mas, em muitas passagens, é um “tu”
genérico (Schmidt, 1986). Não é difícil entender a função do trecho
que vai de 286--380, pois são enumerados uma série de preceitos que, se
forem seguidos, permitirão ao homem uma vida melhor no que diz respeito
aos deuses, sua sociedade e sua propriedade. Entre a justiça, uma
conquista coletiva, e a riqueza, pessoal, há uma série de ações que o
homem deve realizar para o seu sucesso material e moral: 



\begin{verse}[\versewidth]
  Mas diante de Excelência suor puseram os deuses\\
  imortais: longa e íngreme é a via até ela,\\
  e áspera no início; e quando se chega ao topo,\\
  fácil depois ela é, embora sendo difícil.
\end{verse}

\attrib{\textit{Trabalhos} 289--92}


Erro comum na apreciação do poema é achar que, ao iniciar o que
amiúde se denomina “almanaque agrícola” (381--617), o objeto do poema
passa a ser um elenco de atividades agrícolas associadas a certos
fenômenos celestes e climáticos, como se esse longo trecho fosse um
manual para o agricultor inexperiente. Obviamente, esse não devia ser o
uso do poema, assim como não precisamos (de fato, acho que não devemos)
pensar em Hesíodo como um poeta-agricultor, alguém que de dia geria sua
propriedade e de noite ou nas horas vagas, pensava nos seus poemas. 
Mas então para que serve essa passagem propriamente agrícola?

Em primeiro lugar, tanto nela quanto no curioso trecho que é o
último do poema (os “dias”), há temas que reaparecem alhures
(Lardinois, 1998). Segundo Nelson (1996, p. 53), por exemplo,
“não é como ser um agricultor, mas o que o ciclo do ano, com seu
equilíbrio entre bem e mal, lucro e risco, ansiedade e descanso,
implica acerca da vontade de Zeus o que Hesíodo está ensinando”. De
fato, Zeus continua tão presente quando anteriormente; sua atuação
garante uma ordem sem a qual a agricultura seria impossível (414--16).
Se os trabalhos são sazonais, as mudanças entre as estações (565)
devem-se a Zeus, que, não por acaso, é pai delas (repare nos seus
nomes):


\begin{verse}[\versewidth]
  A segunda, [Zeus] fez conduzir a luzidia\\> Norma, mãe das Estações,\\
  Decência, Justiça e a luxuriante Paz,\\
  elas que zelam pelos trabalhos dos homens mortais,\\
  e as Moiras, a quem deu suma honraria o astuto Zeus,\\
  Fiandeira, Sorteadora e Inflexível, que concedem\\
  aos homens mortais terem bens e males.
\end{verse}

\attrib{\textit{Teogonia}, 901--6}

A fartura parece advir da vontade de Zeus (465), mas esta está sujeita a
mudanças bruscas (483), imprevisíveis (488). A luta do homem, porém,
não é contra a vontade de Zeus, mas a partir dela. Por isso a descrição
dos fenômenos climáticos é tão importante: o bom agricultor é aquele
que nunca deixa passar o momento correto de realizar certas atividades
precisas. Além disso, cada estação tem particularidades que ultrapassam
a mera contextualização dos trabalhos do agricultor e que se relacionam
com outros trechos do poema; exemplifiquemos isso através do trecho
sobre o inverno (493--563).

Em primeiro lugar, observe-se que o trecho mais longo dedicado a uma
estação trata da parte do ano durante a qual menos se trabalha.
Trata-se apenas da exibição, por parte do poeta, de seus recursos
descritivos? É clara a oposição com o verão, já que são essas as duas
estações nas quais prepondera a ausência do trabalho, sendo que numa 
o sofrimento é máximo, ao passo que na outra abundam situações prazerosas 
(Manakidou, 2006).

A linguagem enigmática também não está ausente dessa parte do poema.
Parte do episódio invernal é marcada por metáforas sexuais (Bagordo, 2009), 
provavelmente ecoando a ausência  de fertilidade da estação,
a começar pelo vento Bóreas, que, com a sua potência, consegue penetrar
em quase todos os espaços. Reitera-se o frio onipresente através de um
catálogo de espaços e criaturas que a ele se submetem (a infertilidade
da estação por meio do seu contrário), e tanto mais surpreende o
vento, um notório raptor de virgens moças, não conseguir atingir uma
jovem que parece ignorar a dureza do inverno por ter uma série de
luxos à sua disposição (519--23). Figuras femininas, sobretudo mortais,
não costumam ser retratadas de forma positiva no poema, e não é por
acaso que se menciona que a moça é inexperiente sexualmente, sobretudo
se aceitarmos que a expressão “sem-osso” alude ao pênis. As mudanças de
estação não exigem ações distintas apenas na roça, mas também em
relação às criaturas femininas, às quais as figuras masculinas do poema
encontram-se necessariamente ligadas; no episódio invernal, quem não
trabalha ou é um velho alquebrado, um indolente cujo único prazer é a
masturbação, ou um pobre coitado que, devido à solidão, precisa refrear
seu impulso sexual. Quando a mulher estiver no auge da excitação no
verão, os homens, por sua vez, estarão na situação oposta.

Após concluir a parte referente à agricultura, Hesíodo também trata
brevemente da navegação, sobretudo das épocas para se viajar com
menores riscos (618--94). Nesse trecho, quase \textit{pendant}
ao início do poema, a família do poeta volta ao primeiro plano, não
somente através da explicitação de que Perses é irmão do poeta, mas
também que seu pai é um emigrante que adotou a navegação como forma de
tentar fugir da pobreza. O trecho mais notável diz respeito a
uma disputa poética na qual Hesíodo foi vitorioso. A passagem é repleta
de ironia e humor – Hesíodo fala de navegação sem nunca ter navegado,
com exceção de um trecho de menos de cem metros; o conhecimento de que carece
para compor seu canto é dado pelas Musas – e parece ser construída uma
comparação com a poesia épica heroica em relação à qual a tradição
representada por Hesíodo é vitoriosa.

Como acontece alhures na poesia hexamétrica arcaica, o trecho final
desse poema representa como que um anticlímax, ou melhor, ele retoma
temas ou forma explorados anteriormente, só que em uma chave menor,
informando ao público ouvinte que a performance do poeta
oral está chegando ao fim (Kelly, 2007). Os “dias” referem-se a alguns
dias fastos e nefastos do mês, e sua menção reforça para o ouvinte que
o sucesso de um homem não depende apenas do seu próprio trabalho, mas
de uma série de fatores que estão além de sua vontade, entre eles, a
arbitrariedade dos deuses: 

\begin{verse}[\versewidth]
Os outros [sc. dias] são incertos, sem destino, nada trazem.\\
Cada um louva dia distinto, e poucos sabem.\\
Ora é uma sogra, ora uma mãe um dia\\
desses. Venturoso e afortunado quem tudo isso\\
conhece e trabalha, de nada culpados contra os deuses,\\
aves discernindo e transgressões evitando.
\end{verse}

\attrib{\textit{Trabalhos}, 823--8}
%VERIFICAR NUMERO DOS VERSOS

\section{da tradução}

O texto aqui traduzido é, basicamente, aquele proposto por Martin L.
West; algumas soluções de outros filólogos, porém, foram incorporadas,
sobretudo as de Glenn Most, contrário a um excesso de correções ao
texto transmitido.

Algo do que se perdeu na tradução – por exemplo, algumas figuras
etimológicas – é recuperado nas notas.

Na maioria das edições, há dois tipos de marcação que separam certas
unidades do poema \textit{grosso modo} equivalentes a parágrafos e
capítulos. Tais marcações pressupõem um modo de composição e recepção e,
quiçá, de performance. Como as divisões são construções dos intérpretes
modernos, deixei-as de lado para que cada leitor procure ele mesmo sua
forma de estruturar o texto. Isso não significa pressupor que o texto,
ao ser composto e/\,ou apresentado oralmente, tenha sido dado em
fluxo contínuo; reforça-se, porém, que tinha a associação de ideias como
um dos seus elementos estéticos.

A numeração das notas de rodapé em forma de lemas segue o número que 
indica um verso ou um conjunto de versos do poema.


\begin{bibliohedra}
\tit{ARRIGHETTI,} G. (2007) \textit{Esiodo opere}. Introdução, tradução e
comentário. Milano: Mondadori
  
\tit{BAGORDO}, A. (2009) “Zum \textit{anósteos} bei Hesiod (\textit{Erga}
524): Griechische Zoologie, indogermanische Dichtersprache oder etwas
anderes?”. \textit{Glotta} 85: 31--58
  
\tit{BEALL}, E. F. (2001) “Notes on Hesiod’s \textit{Works and days} 383--828”.
\textit{American Journal of Philology} 122: 155--71
  
\titidem (2004) “The plow that broke the plain epic tradition: Hesiod’s
\textit{Works and days} vv. 414--503”. \textit{Classical Antiquity}: 23:
1--31
  
\tit{BLAISE}, F.; \textsc{judet de la combe}, P.; \textsc{rousseau}, P. (1995) (org.) \textit{Le
métier du mythe: lectures d’ Hésiode}. Lille: Presses Universitaires du
Septentrion
  
\tit{BLÜMER}, W. (2001) \textit{Interpretation archaischer Dichtung. Die
mythologischen Partien der Erga} \textit{Hesiods}. 2 vol. Münster:
Aschendorff
  
\tit{BOYS-STONES}, G. R.; \textit{haubold}, J. H. (2010) \textit{Plato and Hesiod}.
Oxford: Oxford University Press
  
\tit{CALAME}, C. (2004) “Succession des âges et pragmatique poétique de la
justice: le récit hésiodique des cinq espèces humaines” \textit{Kernos}
17: 67--102

\titidem (2005) “Prélude à une poésie d’action: le proème des
\textit{Travaux} d’ Hésiode.” In: \textit{Masques d’autorité: Fiction
et pragmatique dans la poétique grecque antique}. Paris: Belles Lettres
  
\tit{CLAY}, J. S. (2003) \textit{Hesiod’s cosmos}. Cambridge: Cambridge
University Press
  
\tit{CURRIE}, Bruno (2007) “Heroes and holy men in early Greece:
Hesiod's \textit{theios aner}”. COPPOLA, Alessandra
(org.) \textit{Eroi, eroismi, eroizzazioni dalla Grecia antica a Padova
e Venezia}. Padova: S.A.R.G.O.N
  
\tit{FORD}, A. (1997) “Epic as genre”. In: \textit{morris}, I.; \textit{powell}, B. (org.)
\textit{A new companion to Homer}. Leiden: Brill
  
\tit{GAGARIN}, M. (1973) “\textit{Dike} in the \textit{Works and days}”.
\textit{Classical Philology} 68: 81--94
  
\tit{HEATH}, M. (1985) “Hesiod’ didactic poetry”. \textit{Classical Quarterly} 36: 245--63 
  
\tit{HESIOD}. (2006) \textit{Theogony, Works and days, testimonia}. Edited and
translated by Glenn W. Most. Cambridge, Mass.: Harvard University Press
  
\tit{HESÍODO}. (1996)\textit{ Os trabalhos e os dias.} Tradução, introdução e
comentários: M. C. Lafer. São Paulo: Iluminuras 
  
\titidem (2011) \textit{Teogonia}. Tradução, introdução e notas: C.
Werner. São Paulo: Hedra
  
\tit{KELLY}, A. (2007) “How to end an orally-derived epic poem?”
\textit{Transactions of the American Philological Association} 137:
371--402
  
\tit{LARDINOIS}, A. (1998) “How the days fit the works in Hesiod’s
\textit{Works and days}”. \textit{American Journal of Philology} 119:
319--36
  
\tit{MANAKIDOU}, Flora P. (2006) “Autour de la structure des \textit{Travaux
et les Jours} d’Hésiode: la ‘dualité’ du cosmos hésiodique”.
\textit{Gaia} 10: 149--67
  
\tit{MARSILIO}, M. S. (1997) “Hesiod’s winter maiden.” \textit{Helios} 24:
101--111

\titidem  (2000) \textit{Farming and poetry in Hesiod’s Works and days}.
Boston: University Press of America
  
\tit{MARTIN}, R. P. (1989) \textit{The language of heroes: speech and
performance in the Iliad}. Ithaca: Cornell University Press

\titidem (1992) “Hesiod’s metanastic poetics”. \textit{Ramus} 21: 11--33
  
\titidem (2004) “Hesiod and the didactic double”. \textit{Synthesis} 11:
31--54
  
\tit{MONTANARI}, F.; \textsc{rengakos}, A.; \textsc{tsagalis}, C. (2009) \textit{Brill’s
Companion to Hesiod}. Leiden: Brill
  
\tit{MORDINE}, M. J. (2006) “Speaking to kings: Hesiods \textit{ainos} and the
rhetoric of allusion in the \textit{Works and Days}.” \textit{Classical
Quarterly}  56: 363--73
  
\tit{MOST}, G. W. (1997) “Hesiod’s myth of the five (or three or four) races”.
\textit{Proceedings of the Cambridge Philological Society} 43: 104--27
  
\tit{NAGY}, G. (1990) “Hesiod and the\textit{} poetics of Pan-Hellenism”. In:
\textit{Greek mythology and poetics}. Ithaca: Cornell Univesity Press
  
\tit{NELSON}, S. A. (1996) “The drama of Hesiod’s farm”. \textit{Classical
Philology} 91: 45--53

\titidem (1998) \textit{God and the land: the metaphysics of farming in
Hesiod and Vergil}. With a translation of \textit{Works and Days} by
David Grene. New York: Oxford University Press 
  
\tit{PUCCI}, P. (1977) \textit{Hesiod and the language of poetry}. Baltimore:
Johns Hopkins University Press
  
\tit{RENEHAN}, R. (1980) “Progress in Hesiod”. \textit{CP} 75: 339--58 (resenha
do comentário de M. West aos \textit{Trabalhos e dias} na forma de
comentários suplementares)
  
\tit{RODRÍGUEZ ADRADOS}, F. (2001) “La composición de los poemas hesiódicos”.
\textit{Emerita} 69: 197--223
  
\tit{ROSEN}, R. M. (1990) “Poetry and sailing in Hesiod’s \textit{Works and
days}”. \textit{Classical Antiquity} 9: 99--113
  
\tit{SCHMIDT}, J.-U. (1986) \textit{Adressat und Paraineseform: zur Intention
von Hesiods ‘Werken und Tagen’}.  Göttingen: Vandenhoeck \& Ruprecht
  
\tit{SCODEL}, R. (2011) “\textit{Works and days} as performance”. In: MINCHIN,
E. (org.) \textit{Orality, literacy and performance in the ancient
world}. Leiden: Brill
  
\tit{VERDENIUS}, W. J. (1985) \textit{A commentary on Hesiod} Work and
Days\textit{ vv.1--382}. Leiden: Brill
  
\tit{VERNANT}, J.-P. (1992) \textit{Mito e sociedade na Grécia antiga}. Rio de
Janeiro: José Olympio

\titidem (2002) \textit{Mito e pensamento entre os gregos}. 2. ed. Rio de
Janeiro: Paz e Terra
  
\tit{VERSNEL}, H. S. (2011) \textit{Coping with the gods: wayward readings in
Greek theology}. Leiden: Brill 
 
\tit{WATKINS}, C. (1995) \textit{How to kill a dragon: aspects of
Indo-European poetics}. New York: Oxford University Press 
  
\tit{WEST}, M. L. (1978) \textit{Hesiod Works \& Days: edited with prolegomena
and commentary}. Oxford: Oxford University Press
\end{bibliohedra}

