\chapterspecial{A {anta} {que} {andava} {nas} {árvores}}{}{}
 

 

\letra{F}{oi Horonamɨ} quem perguntou os nomes dos animais. Horonamɨ encheu a
floresta de animais. 
Horonamɨ encontrou a anta Xamari, que andava como Yanomami. Ela andava
nos galhos baixos, vindo em sua direção. 

Hukru! Hukru! Prãããõ! ela fez ao cair. 

Ela andava nas árvores como os cuatás. Afinal, ele encontrou a anta
andando nas árvores. Felizmente, ele fez com que ela descesse, para que
nós pudéssemos comê"-la. 

É sempre um acontecimento quando matamos uma anta para comê"-la! 

A anta não andava no chão: andava nas árvores de uma espécie nativa de
louro, atravessando os galhos e comendo as frutas maduras. Horonamɨ fez
quebrar o galho para que a anta caísse. Depois de cair, ela se acostumou
a andar no chão. 

A anta chegou ao xapono dos esquilos, mas lá não deu certo, então ela
foi para a mata. Os esquilos se juntaram quando anta ainda era Yanomami,
e a chamaram. Queriam saber quanto ela aguentava comer.

Os esquilos viviam como Yanomami: moravam em um xapono no alto das
árvores e faziam festas como nós, embora eles fossem se tornar animais.
Um dia, eles chamaram as cutias, os caititus, as queixadas, as antas, os
papagaios e as maitacas. Havia muita comida, mas os convidados não
conseguiram comer tudo. Até a anta também desistiu de comer, pois
pressentiam que algo ia acontecer. 

De repente, todos eles se transformaram em animais. 

As queixadas também eram Yanomami. Os cipós se
arrebentaram e elas caíram. Foi lá, na região do xapono dos esquilos
onde não conseguiram comer, pois estavam prestes a se transformar. Não
havia nenhuma queixada antes de eles se transformarem. Nessas regiões,
não havia queixada. Subiram até o alto, subiram, estavam subindo até a
ponta do cipó. Lá, o cipó arrebentou no meio. Queixada! Se isso não
tivesse acontecido, lá naquela floresta, hoje as queixadas andariam nas
árvores. 

A anta foi quem caiu primeiro e passou a andar no chão, tornando"-se um
animal terrestre. Em seguida, o cipó das queixadas arrebentou. Outros
Yanomami, que ficaram na parte superior do cipó se transformaram em
macacos cuatás. Assim foi. 

As queixadas ocuparam toda a floresta. Elas desceram rio abaixo.
Horonamɨ conseguiu assim fazer a anta descer ao chão, e hoje nós as
comemos. Assim que foi. Não havia animais no início, pois eles viviam
espalhados, como os Yanomami, em vários xaponos. 

Yãukuakua! Yãukuakua! Ninguém fazia assim. É assim mesmo. Esse
grande animal que anda no chão, quando estamos famintos de carne, nós a
comemos, ela anda mesmo no chão. Nós a comemos. 

 
