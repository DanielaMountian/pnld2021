\chapterspecial{{Horonam}ɨ e o {tatu}}{O {surgimento} {do} {cipó} e {da} {embira}}{}
 %Subtítulo está estranho

 

 

\letra{O}{ Tatu era Yanomami} e era muito comprido.\footnote{  Era gente, e tinha os hábitos e o corpo semelhantes aos dos Yanomami. Trata-se aqui do tatu-de-rabo-mole-comum (\emph{Cabassous unicinctus}).}  Horonamɨ
encontrou o Tatu. 
Por que Horonamɨ cortou o Tatu bem na cintura? Nós, Yanomami, amarramos
terçados e fazemos as cordas de arco com o cipó"-de"-apuí que se ergue na
mata. Nós o cortamos e descascamos. É com isso que nós amarramos nossas
redes, com as embiras de cipó"-de"-apuí. 

Horonamɨ cortou o Tatu. Antes disso não havia linha de pesca. Nossos
antepassados não tinham corda de rede. Depois de encontrar o Tatu,
depois de esticar suas tripas, depois de destruí"-lo, ele o cortou em
pedaços. 

Foi Tatu quem fez aparecer o machado, pois foi ele quem o fabricou. Ele
percebeu que certo tipo de madeira dura parecia um cabo de machado.
Assim, o Tatu possuía o único machado. Ele ensinou aos \emph{napë} como
fabricar o machado. Então ele não tinha dificuldade em tirar o mel, pois
tinha o machado. Ele fez um cabo comprido, depois de quebrar um pau,
enfiou e amarrou o machado de pedra em um pau, era um machado de pedra;
depois de amarrá"-lo, ele partiu um tronco e tomou mel. Os antepassados
não tomavam mel, não sabiam tomar. Ele ensinou a tomar mel, ele que
existiu primeiro, quando os Yanomami não existiam, quando este inventor
não morava entre eles, ele ensinou a tomar mel. Esse tatu se
chama \emph{moro}. Horonamɨ o encontrou. 

Ku, kõu, kõu, kõu, kõu, kõu!, fazia Tatu, cortando o tronco.
Horonamɨ ouviu esse som pela manhã. 

--- Ho! Quem produz esse som, eu quero ver. Dá para ouvir de longe --- disse
Horonamɨ. 

Ele logo foi em direção ao som. O Tatu estava sozinho; o som fazia
zoada. Horonamɨ estava indo na direção do som e parou. Tatu derramava o
mel \emph{tima},\footnote{  Mel de uma abelha de mesmo nome, que faz sua colméia no oco dos troncos, próximo ao solo.} ele o derramava de uma árvore à qual deu o nome
de \emph{roa}\footnote{  Árvore alta e de madeira dura.} \emph{.} Horonamɨ ficou de pé parado, perto de Tatu, fazendo um som com a boca para chamar sua atenção. Aí fez
outro som com a boca, mas Tatu nem olhava, ele cortava sem parar, com as
pernas abertas. Naquela época, ninguém chamava o outro de ``sogro'',
Horonamɨ nos ensinou então a chamar de ``sogro'':\footnote{  Sogro, ou tio. O uso desse termo indica uma relação de respeito. Horonamɨ quer se aproximar de Tatu. Trata"-se também de uma observação irônica, pois as mulheres ainda não existem no período em que acontecem as histórias de Horonamɨ, e portanto as relações de aliança (sogro/cunhado) não são uma possibilidade.}

--- Hɨ̃ɨɨ, meu sogro! --- disse. --- Meu sogro! --- disse Horonamɨ com
uma voz assustadora. 

Quando disse isso, o Tatu parou. 

--- Ɨ̃ɨ̃! Õ! --- disse assustado. --- Ɨ̃! Õ! De quem é essa voz? --- O Tatu
falava assim. --- De quem é essa voz? --- ele respondeu, com uma voz que
não era normal. Era o seu jeito de falar mesmo. 

Horonamɨ olhou, sorriu.

--- Sogro! O que você está comendo? O que é isso? --- disse Horonamɨ. 

--- Não pergunte quem eu sou! --- ele disse --- Você sabe quem eu sou!
Sou o Tatu! --- disse ele. Dizendo isso, ele perguntou: 

--- Qual é o seu nome? --- ele desafiou Horonamɨ a dizer seu nome. 

--- Ɨ̃ɨ, eu sou Horonamɨ. 

Horonamɨ falava com uma voz bem bonita, pois ele era bonito. 

--- Hɨ̃ɨ, meu filho, eu sou o Tatu. 

O Tatu era esbranquiçado. Ele era branco, como os \emph{napë}. Ele o
chamou logo. 

--- O que você está querendo fazer? O que você está cortando? 

--- Ɨ̃ɨ̃! Estou comendo assim! Estou comendo isto.

--- Eu quero experimentar --- disse Horonamɨ. --- Quero experimentar um
pouco! Posso beber? Que tipo de mel é? 

--- Não pergunte o que é! É o mel \emph{tima} --- disse o Tatu. 

A partir desse momento, nós, Yanomami, aprendemos a chamar esse mel
de \emph{tima}. 

--- Lá tem mel \emph{tima}! --- ao vê"-lo, eu direi assim. 

Foi o Tatu que ensinou o nome. Horonamɨ chegou mais perto daquele que
estava falando. O Tatu maroto chamou Horonamɨ. 

--- Vai! Experimente, meu filho! Experimente, meu filho! O buraco da
colmeia ficou aberto. Pise nesse buraco e entre nela! --- disse. 

Era uma armadilha para fazer Horonamɨ entrar no buraco da árvore. Horonamɨ
aceitou: 

--- Hɨ̃ɨɨ! Será que o buraco tem espaço suficiente? O mel está jorrando,
está gotejando mesmo. O buraco da colmeia está em baixo. A colmeia acaba
aí. Entre lá dentro! Fique mais em cima, pise para baixo! Eu estou
olhando! --- disse o Tatu, malicioso. 

Quando ele disse isso, Horonamɨ cedeu e entrou logo. Foi logo e entrou,
a colmeia fazia barulho, e ele foi até o alto da colmeia. Ficou de pé lá
no alto dela. De pé, onde ele entrou, pelo buraco que Tatu tinha feito.
O Tatu fechou o buraco, e não havia outra saída. O Tatu prendeu Horonamɨ
lá em cima. Horonamɨ gritava lá dentro. Não tinha como sair. Se Horonamɨ
fosse um Yanomami como outro qualquer, ele jamais sairia. Ele gritou e
gritou lá de dentro, sofrendo, gritando e chorando. Chorava como
criança. O Tatu, que o prendeu, fugiu correndo para longe. Aquele que
estava preso por si só fez espocar a árvore. O Tatu já estava longe. 

--- Ele não vai me seguir --- pensou o Tatu, muito seguro de si. 

Horonamɨ, com seu pensamento e seu sopro forte, arrebentou a
árvore \emph{roa}. Ele ficou de pé e olhou ao redor, mas o feioso que o
prendeu não estava mais ali. Horonamɨ ficou sozinho. 

--- Hɨ̃ɨɨ! 

Depois de pular com a explosão, passou pegando a dala e a zarabatana que
estavam penduradas. Colocou nas costas. 

--- Hɨ̃ɨɨɨ! --- gemeu. --- O que tem o nome de Moro, esse feioso,
ele ferrou comigo! --- disse, triste.

Horonamɨ não errou de lugar: ele correu logo para onde o Tatu havia ido,
e foi rápido, ensinando"-nos a correr. Horonamɨ correu na direção do
lugar onde havia muitas pedras saídas da terra; ele correu e correu,
seguindo os rastros do Tatu, como fazem os cachorros. Daí, Horonamɨ
correu dando uma volta, e cortou o caminho do Tatu. Horonamɨ o
encontrou e o Tatu se assustou. Como o Tatu o havia prendido, ele ficou
com medo e com raiva por dentro, e tentou agradá"-lo, mas não conseguiu
suscitar a compaixão de Horonamɨ. 

O Tatu apareceu.

--- Taha! Arrá!--- disse Horonamɨ. 

Era mesmo o Tatu. Ele espreitava, com a mão sobre a testa, à procura de
mel. Olhava passando entre as árvores. Horonamɨ já estava de pé, pegou
um atalho e deu uma volta. O Tatu se confundiu na floresta e acabou
chegando justo onde estava Horonamɨ. Horonamɨ estava de pé, atrás da
árvore, e deu um susto grande nele. Horonamɨ queria cortar aquele que o
havia aterrorizado. Ele decidiu levá"-lo até um tronco, fingindo que ali
havia uma colmeia, para fazê"-lo se abaixar. O Tatu pegou o machado. 

--- Hɨ̃! Meu filho, aqui está! Aqui está! --- disse. --- Hõ, hõ, hõ,
hõ! Meu filho! Hõ, hõ, hõ, hõ! Venha cá ver! Olhe aqui! Meu filho, aqui
está! --- disse Horonamɨ. 

Horonamɨ dizia isso tentando agradar o Tatu, e ia indo atrás dele. 

--- Hɨ̃ɨɨ! Me passa isso que você tem aí no ombro, está afiado
mesmo? --- disse Horonamɨ, astuto. 

A falsa colmeia fazia barulho, e Horonamɨ fez diminuir esse barulho,
para que o Tatu abaixasse a cabeça para ver melhor a colmeia. Enquanto o
Tatu olhava para a colmeia com a cabeça abaixada, enquanto ele estava
nessa posição baixa, ele dizia:

--- Aqui está a entrada da colmeia! 

Quando o Tatu disse isso, o machado já estava na mão de Horonamɨ e,
enquanto o Tatu abaixava a cabeça, Horonamɨ o cortou bem na cintura. 

Krihii, kriihii!, fez Horonamɨ, cortando o Tatu para se vingar, pois
ele tinha sofrido por causa do Tatu. 

--- Ëëëëããaaë! --- gemeu a parte de cima do longo corpo do Tatu. 

Apesar de ser só um pedaço, a parte superior correu embora, sofrendo. Do
lado de cá ficou a parte inferior; as tripas vinham se esticando e a
parte superior ficava rolando. Assim, as tripas foram se esticando até
lá, elas não se arrebentaram. A parte superior daquele que Horonamɨ
havia cortado, e que ele queria que se tornasse o tatu \emph{moro}, foi
lá para cima, até onde estão os espíritos. Foi para lá que fugiu a parte
superior do Tatu. Aqui no chão ficou a parte inferior. 

Só um pedaço do Tatu chegou aos espíritos. Suas tripas não apodreceram;
elas foram até onde se erguem as árvores e subiram nelas. Uma parte das
tripas do Tatu se transformou em cipó"-de"-apuí e outra parte se
transformou na embira \emph{xinakotorema}, com a qual, depois dessa
transformação, os Yanomami começaram a amarrar as cabeças das redes de
cipó. Foi assim.

Apesar de nossos antepassados saberem fazer redes de cipó, eles se
deitavam no chão, pois não havia corda. Eles se deitavam no chão ---
colocavam a rede de cipó no chão para deitar. 

Como foi que eles descobriram a rede de cipó? Eles não sabiam descascar
o cipó"-titica com os dentes, então era assim.\footnote{  O cipó"-titica é usado na fabricação de cestos.}  Até as
moças deitavam no chão. Deitavam uns em cima dos outros, como os
cachorros. Sofriam na escuridão. Eles eram assim. Dormiam passando frio.
Para que nossos antepassados não passassem mais necessidades, as tripas
de Tatu se tornaram cipó"-de"-apuí que amarra as redes. Foi assim. 

Depois da transformação das tripas, eles passaram a usar o cipó para
fazer terçados e machados de pedra, e para amarrar a cabeça das redes,
também feitas de um tipo de cipó. Depois, com o passar do tempo, eles
teceram cestos. No início eles também não sabiam tecer cestos. Assim
foi. Esta história acabou.

 
