\textbf{Artur Azevedo} (São Luís, 1855--Rio de Janeiro, 1908) é um dos
melhores comediógrafos brasileiros e foi nosso primeiro grande homem de
teatro. Além de dramaturgo, foi cronista, contista e poeta. Escreveu
mais de duzentas peças, entre originais, traduções e adaptações. Ficou
conhecido como o maior escritor de revistas de ano brasileiras, mas seu
legado para o teatro posterior compreende, principalmente, as comédias
e burletas, dentre as quais se destacam \textit{A Capital Federal}
(1897) e \textit{O~Mambembe} (1904).

\textbf{A pele do lobo e outras peças} inclui cinco textos curtos de
Artur Azevedo, cuja temática gira em torno de costumes nacionais.
\textit{Amor por anexins} (1870) foi a primeira peça do autor.
\textit{A pele do lobo} (1875) faz uma sátira divertida ao sistema de policiamento do
Império. \textit{\mbox{O Oráculo}} (1907) é um texto que dialoga com a
tradição da comédia.
\textit{Como eu me diverti!} (1893) e \textit{\mbox{O Cordão}} (1908)
tratam do carnaval e são exemplos importantes da conturbada
posição de Artur Azevedo entre os escritores de seu tempo, por ser um
autor eminentemente popular.

\textbf{Larissa de Oliveira Neves} é professora e pesquisadora de teoria
e história do teatro no Instituto de Artes, da Unicamp. Doutora em
Teoria e História Literária pela Unicamp, estuda dramaturgia e é
especialista na obra de Artur Azevedo.


