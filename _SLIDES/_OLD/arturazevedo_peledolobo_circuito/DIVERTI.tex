\chapter[Como eu me diverti!]{Como eu me diverti!\subtitulo{Conto"-comédia}}
\hedramarkboth{Como eu me diverti!}{Artur Azevedo}

\begin{linenumbers}

\castpage

\cast{Jorge}{empregado no comércio}

\cast{O Comendador Andrade}{negociante, sócio principal da firma Andrade,\\ Gomes \& Companhia}

\cast{Um Médico}

\cast{Dona Maria}{excelente senhora de meia"-idade, estabelecida\\ com casa de
alugar cômodos a moços solteiros}

\vfill
A ação passa"-se no Rio de Janeiro, em quarta"-feira de cinzas. 
Atualidade.

\pagebreak 
\newactnamed{Ato único}

\stagedir{A cena representa a sala e a alcova que Jorge ocupa em casa de Dona
Maria. Atirado sobre um velho canapé, um hábito de frade encardido de
suor e sujo de lama. No chão, um par de luvas, igualmente sujas, e um
nariz de papelão quase a desfazer"-se, preso a uns grandes bigodes e a
um par de óculos.}

\newscenenamed{Cena I}

\stagedir{\textsc{Dona Maria, O Médico}}

\repl{O Médico} Que tem ele?

\repl{Dona Maria} Não sei, doutor, não sei. O senhor Jorge tem muito bom
coração, mas tem muito má cabeça: é doido pelo carnaval.

\repl{O Médico} Gabo"-lhe o gosto.

\repl{Dona Maria} Ontem vestiu"-se de frade, pôs aquele nariz postiço e
andou, num carro todo enfeitado de flores, ao lado de uma sujeita que
mora no \textit{Hôtel Ravot}, acompanhando um préstito. Só o vestuário
da pelintra lhe custou perto de oitocentos mil réis!

\repl{O Médico} Quem lhe disse?

\repl{Dona Maria} Os meus hóspedes não têm segredo para mim.

\repl{O Médico} Adiante.

\repl{Dona Maria} Para se não constipar, o pobre moço levou consigo, por
baixo do hábito, uma garrafa de conhaque, e de vez em quando
atiçava"-lhe que era um gosto! Quando o préstito passou pela primeira
vez na rua do Ouvidor (eu estava lá\ldots{}), já ia o frade que não se podia
lamber! Depois, na rua da Constituição, isto sei eu por um amigo dele,
que tudo viu, outro moço, também fantasiado, bifou"-lhe a pelintra, e
isso deu lugar\ldots{}

\repl{O Médico} \ldots{} a um rolo! Pudera!\ldots{}

\repl{Dona Maria} Racharam"-lhe a cabeça!

\repl{O Médico} Naturalmente.

\repl{Dona Maria} E o demônio do rapaz andou toda a noite, de cabeça
rachada, à procura da tal mulher, dos Fenianos para os Tenentes,\footnote{
Fenianos (1867), Democráticos (1869) e Tenentes do Diabo (1855), três grandes
sociedades carnavalescas que desfilavam no centro da cidade, nos dias de
carnaval.} e dos
Tenentes para os Democráticos, bebendo sempre, até cair na rua do Fogo,
às três horas da madrugada!\ldots{}

\repl{O Médico} Com efeito!

\repl{Dona Maria} A polícia levou"-o para a estação da Travessa do Rosário, e
pela manhã uns amigos, que tinham sido avisados, trouxeram"-no para
casa.

\repl{O Médico} Onde está ele?

\repl{Dona Maria} Naquela alcova. Há cinco horas que ali está deitado, sem
dar acordo de si. Por isso, mandei chamá"-lo, doutor.

\repl{O Médico} Fez bem. Vamos vê"-lo. \paren{Entram na alcova.}

\newscenenamed{Cena II}

\stagedir{\textsc{Jorge, O Médico, Dona Maria}. Na alcova, Jorge está de cama, com a cabeça amarrada, os olhos
fechados, os braços caídos. O médico, ao ver o enfermo, tem um
movimento que escapa à Dona Maria.}

\repl{O Médico} \paren{Tomando o pulso ao doente.} Não tem febre.
\paren{Depois de examinar"-lhe a cabeça.} O ferimento nada vale\ldots{} Já
lhe puseram uns pontos falsos; é quanto basta\ldots{} O seu hóspede tem
apenas o que os estudantes chamam uma “ressaca”; precisa de descanso e
mais nada. Quando voltar a si, se quiser tomar alguma coisa, dê"-lhe uma
canja, dois dedos de vinho do Porto misturado com água de
Vichy,\footnote{ Água com propriedades consideradas terapêuticas.} 
um pouco de marmelada, e disse. Se amanhã continuar
incomodado, que tome um laxante.

\newscenenamed{Cena III}

\stagedir{\textsc{O Médico, Dona Maria}. Na sala.}

\repl{O Médico} \paren{Tomando o chapéu.} A senhora não imagina como
estimei por ter sido chamado para ver este senhor Jorge! Foi uma
providência!

\repl{Dona Maria} Por que, doutor?

\repl{O Médico} Conheço"-o, mas não sabia que se tratava dele. É o namorado,
quase noivo de minha afilhada, filha do meu velho amigo Raposo. A
menina gosta dele, e o pai já estava meio inclinado a consentir no
casamento: tinham"-lhe dado boas informações sobre este pândego. Agora,
porém, vou prevenir o compadre, e dissuadir minha afilhada, que é muito
dócil e me ouve acatamento.

\repl{Dona Maria} Valha"-me Deus! E sou eu a culpada de tudo isto!

\repl{O Médico} Culpada, por quê?

\repl{Dona Maria} Por ter mandado chamar o padrinho! Pobre rapaz!\ldots{}

\repl{O Médico} A senhora deve estar, pelo contrário, satisfeita, por ter
indiretamente contribuído para este resultado. \paren{Voltando"-se
para a alcova.} Que grande patife! Namorar uma menina pura como uma
flor, e andar de carro, publicamente, embriagado, em companhia de uma
prostituta!

\repl{Dona Maria} No carnaval tudo se desculpa.

\repl{O Médico} Nada! Eu sou o padrinho, o segundo pai daquele anjo!
\paren{Vai saindo.}

\repl{Dona Maria} \paren{Tomando"-o pelo braço.} Doutor, doutor, não vá
assim zangado com o senhor Jorge\ldots{} não diga nada à família da
menina\ldots{} Ah! se eu soubesse\ldots{} Mas que quer?\ldots{} Vejo que este hóspede
tem segredos para mim\ldots{} \paren{O doutor tenta safar"-se.} Ouça
doutor\ldots{} ele tem um bom emprego\ldots{} é muito estimado pelos patrões\ldots{}

\repl{O Médico} E a minha afilhada tem um dote de cento e cinquenta contos!

\repl{Dona Maria} \paren{Aterrada, largando o braço do Médico.} Cento e
cinquenta contos!

\repl{O Médico} \paren{Saindo.} Fora o que lhe há de caber por morte do
pai! \paren{Chegando à porta, para, volta"-se e diz:} Canja\ldots{} vinho
do Porto\ldots{} água de \textit{Vichy}\ldots{} marmelada\ldots{} e disse!
\paren{Sai.}

\newscenenamed{Cena IV}

\stagedir{\textsc{Dona Maria}, depois \textsc{Andrade}}

\repl{Dona Maria} \paren{Dona Maria fica perplexa, de olhos baixos, na
atitude de Fedra,\footnote{ Personagem mitológica que, ao ser rejeitada pelo enteado Hipólito, por quem se
apaixonara, levanta uma série de calúnias sobre ele, que o levam à morte.
Sentindo"-se tremendamente culpada, ela se mata. Sobre esse mito, o dramaturgo
Jean Racine (1639--1699) escreveu a tragédia \textit{Fedra}, muito encenada no Brasil,
durante do século \textsc{xix}.} quando diz:} \textit{Juste ciel! qu’ai je faite aujourd’hui?}\footnote{
“Ó justo céu! Por que fiz o que fiz?”. Fala da personagem Fedra, na tragédia de
Jean Racine. (Racine, \textit{Fedra} (Trad. Millôr Fernandes.)}
\paren{É despertada bruscamente pelo comendador Andrade, que entra
com grande espalhafato.}

\repl{O Comendador} \paren{Gritando.} Onde está o senhor Jorge?

\repl{Dona Maria} \paren{Consigo.} Um homem zangado! É ele, é o pai da
menina!\ldots{}

\repl{O Comendador} Senhora, pergunto"-lhe pelo senhor Jorge!

\repl{Dona Maria} Está doente\ldots{} naquela alcova\ldots{} dorme\ldots{}

\repl{O Comendador} Já me contaram as façanhas que ele praticou esta noite!
\paren{Apanhando o nariz postiço.} Cá está uma prova!
\paren{Atira"-o longe.}

\repl{Dona Maria} Desculpe"-lhe essa rapaziada, e não lhe negue a mão da
menina.

\repl{O Comendador} A mão da menina! Que menina?

\repl{Dona Maria} Sua filha.

\repl{O Comendador} Minha filha? Qual delas? Pois este mariola ainda por
cima se atreve a erguer os olhos para uma das filhas do seu patrão!

\repl{Dona Maria} Do seu patrão? Ah! então não é o senhor Raposo?

\repl{O Comendador} Que Raposo, nem meio Raposo! Eu sou o Comendador
Andrade, sócio principal da firma Andrade, Gomes \& Companhia! O senhor
Jorge está dormindo, disse a senhora\ldots{}

\repl{Dona Maria} Sim, senhor.

\repl{O Comendador} Pois bem; quando acordar, diga"-lhe que eu aqui estive, e
o ponho no olho da rua! Que apareça para fazermos as contas!

\repl{Dona Maria} Atenda, senhor Comendador!

\repl{O Comendador} A nada atendo! A casa Andrade, Gomes \& Companhia não
pode ter empregados que se embriagam e passam a noite no xadrez! Era o
que faltava! \paren{Sai arrebatadamente.}

\newscenenamed{Cena V}

\stagedir{\textsc{Jorge, Dona Maria}. Na alcova.}

\repl{Jorge} \paren{Abre um olho, depois o outro, olha em volta de si,
certifica"-se de que está em sua casa, dirige a Dona Maria um sorriso de
agradecimento, solta um longo suspiro, e exclama com voz rouca e
sumida} --- Como eu me diverti!

\bigskip
\begin{center}
\textsc{Cai o pano}
\end{center}

\end{linenumbers}
