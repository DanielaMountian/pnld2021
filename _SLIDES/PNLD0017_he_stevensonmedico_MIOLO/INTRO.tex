\chapterspecial{Vida e obra de R.\,L.\,Stevenson}{}{Braulio Tavares}
\hedramarkboth{Vida e obra}{braulio tavares}

%\section{Sobre médicos e monstros}
\section{Sobre o autor}

Robert Louis Stevenson (Edimburgo, 1850---Samoa, 1894) 
descende de uma família de engenheiros que construiu alguns dos 
faróis da costa escocesa. Em 1857, seus pais transferem-se para 
Edimburgo. Aos dezessete anos, Stevenson ingressa no curso de 
engenharia, mas, abandonando-o, passou a fazer direito e se formou em 1875. 
Seus dois primeiros livros, \textit{An Inland Voyage} (1878) e \textit{Travels with a Donkey in the Cevennes} (1879), são descrições de suas viagens. Em 1878, viaja 
à Califórnia ao encontro de Fanny Van de Grift Osbourne, que 
conhecera dois anos antes, e com quem se casaria. Publicou \textit{New 
Arabian Nights} em 1882, reunindo seis narrativas escritas 
entre 1877 e 1880. Por estas narrativas, é considerado um dos 
primeiros cultores ingleses do conto. Seu volume de 1887, 
\textit{The Merry Men and Other Tales and Fables}, traz o conto ``Markheim'', 
que antecipa o tema do duplo que estaria na base de \textit{O estranho caso 
do Dr. Jekyll e Mr. Hyde}. \textit{A Ilha do tesouro} (\textit{The Treasure Island}, 1883), seu maior sucesso, é a primeira de uma série de narrativas 
de aventura, que inclui \textit{The Black Arrow}, do mesmo ano, e \textit{Kidnapped} (1886). Em 1888, vai morar em Samoa, onde permanece os últimos seis anos de sua vida.


No apêndice dessa edição, pode"-se aprofundar na relação de Stevenson com sua obra ao ler os depoimentos de amigos e familiares do autor sobre a composição da história.
Os depoimentos da esposa e do enteado de Stevenson sobre a concepção da
história mostram um dos aspectos mais
fascinantes desta narrativa: o fato de que uma primeira versão foi
destruída por Stevenson após críticas veementes da mulher, Fanny, para
quem se tratava de uma história meramente sensacionalista, à qual
faltava uma dimensão alegórica.  Podemos pensar, portanto, que ao
aceitar a crítica da esposa, cuja opinião tinha em alta conta,
Stevenson teria diminuído, no segundo manuscrito, muitos aspectos crus
e brutais do primeiro, e dado mais destaque às referências morais e
religiosas.  O manuscrito queimado seria mais próximo do pesadelo
original que levou Stevenson a compor a obra, e de certa forma, como tantas vezes acontece na literatura,
um prolongamento dele, uma reconstituição, com o autor já desperto, do
estado mental peculiar que produziu aquelas imagens.  O sonho serve, em
casos assim, como ponto de partida para uma narrativa que mesmo
conscientemente escrita tem o sonho como um diapasão, um parâmetro
intuitivo pelo qual o autor percebe se ao escrever está se aproximando
ou se afastando da visão original.

\section{Sobre a obra}

As três grandes narrativas de terror do século \textsc{xix} surgiram de
pesadelos de seus autores.  Mary Shelley sonhou a história de
\textit{Frankenstein} (1818) após o famoso sarau às margens de um lago
suíço, em que ela, seu marido e outros amigos se propuseram a escrever
histórias de fantasmas.  O filho de Bram Stoker afirmou que a ideia
para \textit{Drácula} (1897) veio ao seu pai num pesadelo após comer
casquinhas de caranguejo.  Quanto ao sonho de Stevenson que deu origem
ao \textit{Estranho caso do Dr.~Jekyll e Mr.~Hyde} (1886), a presente
edição transcreve depoimentos deixando claro que: 1) Stevenson escreveu
a história depois de ter sonhado os seus elementos principais; 2)
quando sua esposa, Fanny Osbourne, criticou essa primeira versão, ele a       
queimou e redigiu uma segunda, que é o texto que todos conhecemos.

No sonho, Stevenson imaginou, para seu próprio horror, as três cenas que
resultaram no \textit{Estranho caso do Dr.~Jekyll e Mr.~Hyde}: o
assassinato visto da janela, um homem que é perseguido, um homem
bebendo uma poção e se transformando em outra pessoa. Quando sonhamos,
somos o dramaturgo que urdiu a história, e somos também a plateia que é
surpreendida por ela.  A dualidade das mentes, o Eu que contém em si um
Outro, já estava presente no momento mesmo da geração das imagens que
deflagraram a narrativa.  Em seu “Um capítulo sobre o sonho”, Stevenson
confessa ficar maravilhado diante dessa possibilidade de que sua mente
inventasse uma história e ele próprio acompanhasse seu desenrolar sem
ter a mínima ideia de como seria o seu fim.


Milhares de páginas já foram escritas sobre \textit{Dr.~Jekyll and Mr.~Hyde}, 
que aparenta ser inesgotável. Um grande livro, curiosamente, é muitas 
vezes um livro imperfeito, contraditório, divergente de si mesmo, uma 
conta que deixa sempre um resto.  Daí essa profusão de tentativas frustradas 
de síntese tentando conciliar os aspectos contraditórios desta noveleta.  
A pressa na composição fez inclusive Stevenson cometer pequenos erros de 
continuidade (nas datas, por exemplo). Os críticos fazem reparos a detalhes 
que vão desde a pouca relevância das personagens femininas até o excesso de 
referências bíblicas.  O poder de fascinação da narrativa, contudo, permanece 
intacto, e a única coisa que pode diminuí"-lo é, ironicamente, o status de clássico 
literário que conquistou, e a sua intensa exploração pela cultura de massas. 
Dificilmente alguém lerá a história de Jekyll e Hyde sem que já saiba do seu desfecho, 
ou, pelo menos, que o “médico” e o “monstro” são uma só pessoa.

Numa crítica na revista \textit{Sur}, em 1941, Jorge Luis Borges
ironizou as versões cinematográficas desta história: “Spencer Tracy
prepara sem medo a versátil beberagem e transforma"-se em Spencer Tracy
com peruca diferente e traços negroides”.  A utilização de atores
famosos nos papéis"-título --- Tracy no filme de Victor Fleming (1941),
Fredric March no de Rouben Mamoulian (1931) --- reforça essa
previsibilidade do desfecho, da identidade entre as personagens
anunciados no título. Borges sugere (para o cinema) uma alternativa
engenhosa, a de escalar dois atores famosos (ele sugere Tracy e George
Raft) para que um se transformasse no outro.  

O impasse de \textit{Jekyll }\textit{e}\textit{ Hyde} é o impasse de
todos os clássicos, histórias tão largamente discutidas que sua fama, e
a natureza dos seus desfechos, costuma preceder sua leitura.  Quantos
leitores do \textit{Grande Sertão: Veredas} penetram nesse livro sem
saber o segredo de Diadorim?  Quantos leitores de \textit{O Assassinato
de Roger Ackroyd} ignoram o desfecho"-surpresa urdido por Agatha
\mbox{Christie?}  Histórias que contêm a revelação de um segredo crucial serão
sempre histórias problemáticas, inclusive para o crítico, obrigado a
revelar o segredo para poder comentá"-lo – o que torna dos mais
espinhosos o ofício de criticar adequadamente um romance de mistério
detetivesco.

Num dos seus sonetos mais finamente burilados, “Dualismo” (em
\textit{Tarde}, 1919), Olavo Bilac escreveu:


\begin{verse}
Capaz de horrores e de ações sublimes,\\
não ficas das virtudes satisfeito\\
nem te arrependes, infeliz, dos crimes;\\
e, no perpétuo ideal que te devora,\\
residem juntamente no teu peito\\
um demônio que ruge e um deus que chora.
\end{verse}

Poderia ser uma descrição do tormento mental do Dr.~Jekyll.  Descrições
superficiais do livro de Stevenson sugerem que a dualidade expressa por
ele se refere ao Bem (o Dr.~Jekyll) e o Mal (Mr.~Hyde).  Como a leitura
atenta mostrará, não é tão simples assim.  Hyde pode ser o Mal em
estado puro, mas Jekyll é um homem dividido.  Stevenson afirmou mais de
uma vez que não via Jekyll como um puro de coração, mas como um
hipócrita que sabe muito bem o que quer e tenta a todo custo reservar
para si o melhor de dois mundos.  Comporta"-se como o viciado que,
incapaz de resistir à tentação da droga, diz para si mesmo, todos os
dias: “É só esta vez, a última de todas, e nunca mais”.  No começo,
Jekyll toma a poção para se transformar em Hyde; no final do livro é
Hyde, como personalidade dominante, quem precisa tomá"-la para
transformar"-se em Jekyll e escapar aos perseguidores. 

A história de Stevenson é contada de diversos pontos de vista, e este é
um dos seus trunfos como narrativa.  É uma história que a cada
capítulo, a cada depoimento, parece recomeçar do zero, ou recomeçar de
um ponto afastado e convergir, como todas as anteriores, para o centro
invisível, o segredo da vida do Dr.~Jekyll.  Embora o fio principal da
história acompanhe Mr. Utterson, o advogado, ele vai recolhendo ao
longo do tempo os testemunhos de pessoas (Mr. Enfield, o Dr.~Lanyon, o
mordomo Poole) que lhe relatam fatos envolvendo Mr. Hyde ou o Dr.~Jekyll 
de maneira inexplicável. Os dois últimos capítulos são dois
longos documentos escritos por Lanyon e Jekyll, em que toda a verdade é
revelada.  Essa aproximação gradual, de diferentes direções, realça a
sensação de uma história que nenhum dos seus participantes entende por
inteiro.  



\section{Sobre o gênero}
%gênero das narrativas duplas 
É como um romance policial que \textit{Jekyll and Hyde} se inicia,
sugerindo ao leitor a possibilidade de que um respeitável médico
londrino esteja sendo vítima de chantagem por parte de um indivíduo
repulsivo e sem escrúpulos.  Não se sabe a origem do aparente poder de
Hyde sobre Jekyll; num romance vitoriano, as explicações mais
plausíveis seriam um caso homossexual entre os dois, ou a possibilidade
de que Hyde fosse um filho bastardo que volta para atormentar o pai que
o renegou.  Histórias de vida dupla (uma fachada respeitável, uma
atividade dissoluta e clandestina) são lugar"-comum na literatura
inglesa da época, sendo que as versões fantásticas mais notórias são
este livro de Stevenson e \textit{O retrato de Dorian
Gray}, de Oscar Wilde (1890).  Steven Marcus analisou essa duplicidade
em \textit{The Other Victorians} (1964), onde mostra a relação de
interdependência inconsciente entre, por exemplo, as obras de Charles
Dickens e \textit{My Secret Life}, o clássico anônimo da pornografia
vitoriana. 

Dupla fachada, dupla porta de entrada, tudo isso foi usado por Stevenson
em sua concepção da casa de Jekyll, uma casa longa que atravessa todo o
quarteirão e deste modo abre portas para duas ruas diferentes.  A
entrada principal é o endereço oficial do Dr. Jekyll, e dá para uma
praça movimentada; a porta traseira, cuja chave pertence a Hyde, dá
para uma rua secundária, comercial, que à noite fica praticamente
deserta.  Os filmes de Rouben Mamoulian e de Victor Fleming exploram
com habilidade essa transição entre o mundo de Jekyll e o mundo de
Hyde.  O espaço mental se transforma num espaço físico. No cinema, isto
proporciona um trajeto de revelações percorrido pela câmera. No livro,
é um quebra"-cabeças montado aos poucos pelos relatos dos amigos de
Jekyll. 

Além desse aspecto do romance policial e das narrativas de vida dupla características do período --- vale relembrar também o clássico ``William Wilson'' (1839) de Edgar Allan Poe, ``A morte amorosa'' (1836), de Théophile Gautier e ``O Horlá'' de Guy de Maupassant (1887) --- o romance de Stevenson se inscreve fortemente no gênero gótico, estilo que floresceu na Inglaterra como uma resposta às ideias
e movimentos decorrentes do Iluminismo oriundo da França no século
\versal{XVIII}, cuja característica principal consistia em afirmar a primazia da
razão sobre as demais formas de pensamento e a religião em geral. O~movimento propôs uma análise da sociedade tendo como ponto de partida a
observação empírica dos costumes, das leis, do comportamento, entre
outros, o que influenciou sobremaneira a literatura na virada do século
\versal{XVIII} ao \versal{XIX}. Assim, à medida que ascendia a instâncias cada vez mais altas nos planos econômico, político e científico"-intelectual, a
burguesia veio a encontrar no romance sua forma de expressão literária
por excelência.

No entanto, alguns autores ingleses desviaram"-se do afã provocado pelo
nascimento das ideias iluministas de esclarecimento intelectual e
progresso. Tais autores, cujos principais representantes são Ann
Radcliffe\footnote{1729"-1807, autora de \emph{O~velho barão
inglês} (1777). É~digno de nota que um grande número de novelas do
gênero gótico foi escrito por mulheres, algo até então incomum.},
Horace Walpole\footnote{1717"-1797, autor de \emph{O~castelo de
Otranto} (1764).}
 e Matthew Gregory
 Lewis\footnote{1775"-1818, autor de \emph{O~monge} (1796).}, 
ao mesmo tempo que aceitavam as mudanças
provocadas pelo pensamento racionalista, colocavam"-no em xeque,
valorizando e explorando dimensões sombrias e sobrenaturais da
experiência que seriam inacessíveis pelas luzes da razão. Os autores
do gênero gótico combinam a modernidade da medicina e dos transportes
com a atmosfera medieval de castelos frios, cheios de salas secretas e
passagens subterrâneas sombrias; o refinamento dos novos costumes com o
barbarismo e a excentricidade; a descrição realista das ações e dos
ambientes com sentimentos de desolação e abandono. Atingindo seu auge na
década de 1790, a literatura gótica influenciou
diversos autores de gerações posteriores que não se dedicaram ao gênero.
Por exemplo, Jane Austen, que em \emph{A~abadia de Northanger} (1818)
conta uma história na qual a protagonista é leitora de romances góticos
(tal como a própria Austen). Centrando sua narrativa na discussão desse
tipo de literatura, a autora busca realçar seus aspectos interessantes e
criticar seus pontos fracos.


Na história de Stevenson, pode"-se notar nitidamente as características desse gênero: o lado racional, iluminista do homem, representado em um homem da ciência e da medicina, o Dr.\,Jekyll, em combate com seu lado sombrio e irracional, recalcado pelas luzes da razão esclarecida, o Mr.\, Hyde. Através de um lugar"-comum que se estabelecia na literatura desde inícios do século \textsc{xix}, a vida dupla das personagens, Stevenson encenou as questões caras ao gótico e se inscreveu, ao lado de narrativas como \textit{Frankenstein} e \textit{Drácula}, como um dos maiores representantes do gótico e da florescente literatura de terror.


Não é demais lembrar ainda que esses dois outros clássicos do terror acima citados
têm composição semelhante.  \textit{Frankenstein}
são três histórias e níveis sucessivos: o começo e o fim do livro são a
narrativa do capitão Walton, cujo navio recolhe no gelo da Sibéria um
homem, o Dr.~Victor Frankenstein, que lhe conta uma longa história; no
meio dessa história, surge uma terceira, a do monstro propriamente
dito, encapsulada dentro da narrativa do doutor.  \textit{Drácula} é
uma montagem de cartas, depoimentos, notícias de jornais e outros
relatos, cuja característica principal é que cada uma das pessoas que
escreve vê apenas em parte a história em que está envolvida.  Em todos
estes livros, existe um mistério central que é parcialmente desvelado
por observadores cujas visões incompletas se superpõem e se iluminam. 
Em \textit{Jekyll e Hyde}, lido hoje, com o
conhecimento que temos sobre o modo como foi composto, essa impressão é
aumentada pela sensação de que uma história terrível aos poucos vai se
revelando aos nossos olhos, mas nunca se revelará por completo: é um
pesadelo censurado, a versão liberada de um original mais tenebroso,
que se perdeu.\linebreak


%paulo: jogar para o início?
\noindent O texto do romance usado para esta tradução foi o de \textit{Strange Case 
of Dr.~Jekyll and Mr.~Hyde and Other Tales}, editado, com introdução 
e notas, por Roger Luckhurst (Oxford, 2006, Oxford World’s Classics). 
Também desse volume foram traduzidos “Um capítulo sobre o sonho” de 
Stevenson, “As desintegrações do Ego” de Henry Maudsley e “A personalidade 
multiplex” de Edward Myers.  Do livro \textit{Essais sur l’art de la fiction}, de 
Stevenson, editado por Michel LeBris (Payot, Paris, 1992) foram 
traduzidos os textos “Esse outro Eu, meu companheiro” de Stevenson, “Quando 
ocorreu o pesadelo de Mr.~Hyde\ldots{}” de Lloyd Osbourne, 
e “Recordações de Mr.~Hyde” de Fanny Van de Grift"-Stevenson.

