\chapter[Viagem em volta do meu quarto]{Viagem em volta\break do meu quarto}  


\section*{Capítulo i}

Como é glorioso começar uma nova carreira e aparecer de repente para o
mundo intelectual com um livro de descobertas na mão, como um cometa
inesperado brilha no espaço!

Não, eu não manterei mais o meu livro \textit{in petto}; aí está ele,
senhores, leiam. Planejei e realizei uma viagem de quarenta e dois dias
em volta do meu quarto. As observações interessantes que fiz, e o
prazer contínuo que experimentei ao longo do caminho, me fizeram querer
torná-la pública; a certeza de ser útil me fez decidir. Meu coração
experimenta uma satisfação inexprimível quando penso no número infinito
de infelizes aos quais ofereço uma fonte segura contra o tédio e um
alívio para os males que suportam. O prazer que se sente ao viajar em
seu quarto está a salvo da inveja inquieta dos homens, e independe da fortuna.

Haverá alguém, realmente, tão infeliz, tão abandonado, que não tenha um
reduto aonde possa se retirar e se esconder de todo mundo? Esses são
todos os preparativos da \mbox{viagem.}

Estou certo de que todo homem sensato adotará meu sistema, qualquer que
possa ser seu caráter e qualquer que seja seu temperamento: seja
avarento ou pródigo, rico ou pobre, jovem ou velho, nascido sob a zona
tórrida ou perto do polo, ele pode viajar como eu; enfim, na imensa
família dos homens que fervilham sobre a face da terra, não há um só
-- não, nem um só (evidentemente entre aqueles que habitam quartos)
que possa, depois de ter lido esse livro, recusar sua aprovação à nova
maneira de viajar que apresento ao mundo.

\section*{Capítulo ii}

 Poderia começar o elogio de minha viagem dizendo que ela não me custou
nada; esse assunto merece atenção. Aí está primeiramente celebrado,
festejado pelas gentes com uma fortuna medíocre; mas há uma outra
classe de homens junto da qual é ainda mais certo um feliz sucesso, por
esta mesma razão de nada custar. -- E então, junto de quem? Ora essa!
Vocês ainda perguntam? Junto dos ricos! Além disso, que recurso não é
para os doentes esta maneira de viajar! Não terão mais que temer a
intempérie do ar e das estações. -- Para os poltrões, que estarão ao
abrigo dos ladrões e não encontrarão nem precipícios nem valetas.
Milhares de pessoas que antes de mim não tinham ainda ousado, outras
que não puderam, outras enfim que não tinham sonhado em viajar, vão se
resolver a partir do meu exemplo. O ser mais indolente hesitaria em se
pôr na estrada comigo para buscar um prazer que não lhe custaria nem
pena nem dinheiro? -- Coragem então, partamos. -- Sigam-me, vocês
todos que uma mortificação de amor, uma negligência de amizade
retiveram em casa, longe da pequenez e da perfídia dos homens. Que
todos os infelizes, os doentes e os entediados do universo me sigam!
-- Que todos os preguiçosos se levantem em massa! E vocês que remoem
em seus espíritos projetos sinistros de reforma ou aposentadoria por
qualquer infidelidade; vocês que, em um budoar, renunciam ao mundo pela
vida; amáveis anacoretas de uma noitada, venham também, deixem,
creiam-me, essas negras ideias; vocês perdem um instante de prazer sem
ganhá-lo para a sabedoria: concedam acompanhar-me em minha viagem.
Andaremos em pequenas jornadas, rindo, o longo caminho dos viajantes
que viram Roma e Paris;\footnote{ O autor refere-se aqui provavelmente
às narrações de viagem, muito em moda no século \textsc{xviii}, e sobretudo à
tradição do \textit{Grand Tour} que conduzia viajantes pela Itália. As
\textit{Cartas persas} (1721) de Montesquieu também participam dessa
tradição.} -- nenhum obstáculo poderá nos fazer parar e
deixando-nos levar pela nossa imaginação nós a seguiremos onde quer que
ela nos conduza.

\section*{Capítulo III}

Há tantas pessoas curiosas no mundo! -- Estou convencido de que querem
saber por que minha viagem em volta do meu quarto durou quarenta e dois
dias no lugar de quarenta e três, ou qualquer outro espaço de tempo;
mas como eu explicaria para o leitor, já que eu mesmo ignoro? Tudo que
posso assegurar é que, se a obra é longa demais para seu gosto não
dependeu de mim fazê-la mais curta; qualquer vaidade de viajante à
parte, eu teria me contentado com um capítulo. Estava, é verdade, no
meu quarto, com todo o prazer e graça possíveis, mas, ai de mim, não
podia sair à vontade. Acredito mesmo que, sem a intervenção de certas
pessoas poderosas que se interessavam por mim, e pelas quais meu
reconhecimento não se extinguiu, eu teria tido todo o tempo de escrever
um in-folio, tanto que os protetores que me faziam viajar em meu quarto
estavam dispostos em meu favor!

E, entretanto, leitor razoável, veja o quanto esses homens estavam
enganados, e veja bem, se podes, a lógica que vou expor.

 Há algo mais natural e justo que bater-se com alguém que pisa no seu pé
por inadvertência, ou que deixa escapar qualquer termo picante em um
momento de desapontamento, cuja causa é a sua imprudência, ou que enfim
tem a infelicidade de agradar a sua amante?

 Vai-se a um prado e lá, como Nicole fazia com o Burguês
Fidalgo,\footnote{ Peça de Molière.} tentamos atacar em quarta
enquanto ele se defende em terça;\footnote{ Movimentos de esgrima.} 
e, para que a vingança seja certa e completa, apresentamos seu
peito descoberto, e corremos o risco de nos fazer matar por nosso
inimigo para nos vingarmos dele. -- Vemos que nada é mais consequente,
e de qualquer modo encontramos pessoas que desaprovam este louvável
costume! Mas o que é também tão consequente quanto todo o resto, é que
essas mesmas pessoas que o desaprovam e que querem que o olhemos como
uma falta grave, tratariam ainda pior aquele que se recusasse a
cometê-lo. Mais de um infeliz, para se adequar a essa posição, perdeu
sua reputação e seu emprego; de sorte que quando temos a infelicidade
de ter o que chamamos de um caso, não faríamos mal em tirar a sorte
para saber se devemos terminá-lo seguindo as leis ou o costume, e como
as leis e os costumes são contraditórios, os juízes poderiam também
jogar suas sentenças nos dados. -- E provavelmente também é a uma
decisão desse tipo que é necessário recorrer para explicar por que e
como minha viagem durou quarenta e dois dias exatamente.

\section*{Capítulo IV}

 Meu quarto está situado no quadragésimo-quinto grau de latitude,
segundo as medida do padre Beccaria:\footnote{ Contemporâneo de Xavier
de Maistre, ensinava física na universidade de Turim.} sua
orientação é entre o levante e o poente; forma um quadrado longo com
trinta e seis passos em toda a volta, beirando a parede bem de perto.
Minha viagem terá, entretanto, mais que isso, porque eu o atravessarei
frequentemente ao longo e ao largo, ou mesmo diagonalmente, sem seguir
nem regra nem método. -- Eu farei até ziguezagues, e percorrerei todas
as linhas possíveis em geometria, se a necessidade o exigir. Não gosto
das pessoas que são tão vigorosamente os mestres de seus passos e de
suas ideias que dizem: ``Hoje, farei três visitas, escreverei quatro
cartas, terminarei essa obra que comecei.''

 Minha alma está tão aberta a toda sorte de ideias, de gostos e
sentimentos, recebe tão avidamente tudo que se apresenta!\ldots\ -- E por
que ela recusaria as alegrias que estão dispersas pelo difícil caminho
da vida? Elas são tão raras, tão esporádicas, que seria preciso ser
louco para não parar, e até voltar no caminho, para colher todas
aquelas que estão ao nosso alcance. Não há nada mais atraente, para
mim, que seguir as ideias pelo rastro, como o caçador persegue a caça,
sem pretender tomar alguma estrada. Assim, quando viajo no meu quarto,
percorro raramente uma linha reta: vou da minha mesa para um quadro
colocado em um canto; de lá parto obliquamente para ir até a porta;
mas, ainda que no momento em que parto minha intenção seja ir para lá,
se encontro minha poltrona no caminho, não me faço de rogado e lá me
arranjo imediatamente. -- Uma poltrona é um móvel excelente, é
sobretudo de máxima utilidade para todo homem meditativo. Nas longas
noites de inverno é por vezes doce, e sempre prudente, estender-se nela
molemente, longe do alarido das reuniões numerosas. -- Um bom fogo,
livros, canetas; quantos recursos contra o tédio! E que prazer ainda
esquecer nossos livros e nossas canetas para atiçar o fogo,
deixando-nos levar por alguma doce meditação, ou arrumar algumas rimas
para distrair os amigos! As horas então correm por nós e caem no
silêncio da eternidade, sem nos fazer sentir sua triste passagem. 

\section*{Capítulo V}

 Depois da minha poltrona, indo na direção norte, descobre-se minha
cama, que está colocada no fundo do quarto, e que forma a mais
agradável perspectiva. Ela está situada da maneira mais adequada: os
primeiros raios do sol vêm brincar nas minhas cortinas. -- Eu os vejo,
nos belos dias de verão, avançar ao longo da parede branca à medida que
o sol se levanta: os olmos que estão em frente a minha janela os
dividem de mil maneiras, e os fazem balançar sobre minha cama,
cor-de-rosa e branca, que espalha para todos os lados, através dos seus
reflexos, um matiz encantador. -- Ouço o murmúrio confuso das
andorinhas que se apossaram do telhado da casa e de outros passarinhos
que moram nos olmos: então, mil ideias risonhas ocupam meu espírito, e,
no universo inteiro, ninguém tem um acordar tão agradável, tão plácido
como o meu.

 Asseguro que adoro gozar desses doces instantes, e que eu prolongo
sempre, tanto quanto possível, o prazer que encontro em meditar no doce
calor do meu leito. -- Seria esse um teatro que empresta mais à
imaginação, que desperta as mais ternas ideias, mais que o móvel em que
me esqueço de vez em quando? -- Leitor discreto, não se assuste --
mas eu não poderia falar da felicidade de um amante que segura, pela
primeira vez, em seus braços, uma esposa virtuosa? Prazer inefável, que
meu mau destino me condena a não experimentar jamais! Não é em um leito
que uma mãe, ébria de alegria pelo nascimento de um filho, esquece suas
dores? É lá que os prazeres fantásticos, frutos da imaginação e da
esperança, vêm nos agitar. -- Enfim, é neste móvel delicioso que nós
esquecemos, durante uma metade da vida, as aflições da outra metade.
Mas que multidão de pensamentos agradáveis e tristes se apressam em meu
cérebro? Mistura desconcertante de situações terríveis e deliciosas!

 Um leito nos vê nascer e nos vê morrer; é o teatro variável onde o
gênero humano representa, alternadamente, dramas interessantes, farsas
risíveis e tragédias assustadoras. -- É um berço guarnecido por
flores; é o trono do amor; é um sepulcro.

\section*{Capítulo VI}

 Este capítulo é apenas para os metafísicos. Ele vai jogar luz sobre a
natureza do homem: é o prisma com o qual poderemos analisar e decompor
as faculdades humanas, separando a força animal dos raios puros da
inteligência. 

 Seria impossível para mim explicar como e por que eu me inflamava nos
primeiros passos que fiz começando minha viagem, sem explicar ao
leitor, nos mínimos detalhes, meu sistema da \textit{alma e da besta}.
-- Esta descoberta metafísica influencia, por outro lado, tanto minhas
ideias e ações, que seria muito difícil compreender este livro se eu
não oferecesse a sua chave desde o começo.

 Percebi, por diversas observações, que o homem é composto de uma alma e
de uma besta. -- Estes dois seres são absolutamente distintos, mas tão
encaixados um no outro, ou um sobre o outro, que é preciso que a alma
tenha uma certa superioridade sobre a besta para estar em situação de
fazer a distinção.

 Guardo de um velho professor (é uma lembrança muito distante) que
Platão chamava de \textit{o outro} à matéria. Está bem; mas preferiria
dar esse nome, por excelência, à besta que está junto a nossa alma. É
realmente esta substância que é \textit{o outro}, e que nos provoca de
uma maneira tão estranha. Percebemos muito superficialmente que o homem
é duplo; mas é, digamos, porque é composto de uma alma e de um corpo; e
acusamos este corpo de não sei quantas coisas, mas sem nenhuma razão
séria seguramente, pois ele é tão incapaz de sentir como de pensar. É a
besta que é preciso incriminar, este ser sensível, perfeitamente
distinto da alma, verdadeiro \textit{indivíduo} que tem sua existência
separada, assim como seus gostos, suas inclinações, sua vontade, e que
só está acima dos outros animais porque é mais bem educado e provido de
órgãos mais perfeitos.

 Senhores e senhoras, fiquem orgulhosos de sua inteligência tanto quanto
queiram; mas desconfiem bastante da\textit{ outra}, sobretudo quando
vocês estiverem juntos!

 Fiz não sei quantas experiências sobre a união dessas duas criaturas
heterogêneas. Reconheci, por exemplo, claramente, que a alma pode se
fazer obedecer pela besta, e que, por um retorno deplorável, esta
frequentemente obriga a alma a agir contra sua vontade. De regra, uma
tem o poder legislativo e a outra o poder executivo; mas esses dois
poderes se contrariam com frequência. -- A grande arte de um homem de
gênio é saber bem educar sua besta, para que ela possa seguir sozinha,
enquanto a alma, livre desta penosa relação, pode elevar-se até o céu. 

 Mas é preciso que se esclareça isso em um exemplo.

 Quando se lê um livro, senhor, e uma ideia mais agradável entra de
repente em sua imaginação, sua alma imediatamente se agarra a ela e
esquece o livro, enquanto seus olhos seguem maquinalmente as palavras e
as linhas; acaba-se a página sem compreendê-la e sem lembrar-se do que
se leu. -- Isso acontece porque sua alma, tendo ordenado que sua
companheira fizesse a leitura, não a advertiu de que iria retirar-se
por um pouco de tempo; de sorte que a \textit{outra} continuava a
leitura que sua alma não escutava mais.

\section*{Capítulo VII}

 Isso não lhe parece claro? Aí está um outro exemplo:

 Um dia do verão passado, caminhava na direção do paço. Tinha pintado
por toda a manhã e minha alma, contente em meditar sobre a pintura,
deixou aos cuidados da besta levar-me ao palácio do rei.

 ``Que arte sublime é a pintura!'', pensava minha alma. Feliz daquele que
foi tocado pelo espetáculo da natureza, que não é obrigado a fazer
quadros para viver, que não pinta unicamente por passatempo, mas que,
tocado pela majestade de uma bela fisionomia, e pelos jogos admiráveis
da luz que se funde em mil tintas sobre o rosto humano, se esforça para
aproximar os efeitos sublimes da natureza em suas obras! Feliz ainda o
pintor que o amor pela paisagem engaja em passeios solitários, que sabe
exprimir sobre a tela o sentimento da tristeza que lhe inspira uma
floresta sombria ou um campo deserto! Suas produções imitam e
reproduzem a natureza; ele cria mares novos e negras cavernas
desconhecidas ao sol: à sua ordem, verdes bosques saem do nada, o azul
do céu se reflete nos quadros; ele conhece a arte de perturbar os ares
e de fazer rugir as tempestades. Em outros momentos ele oferece ao
olhar do espectador encantado os campos deliciosos da antiga Sicília:
vemos ninfas perdidas fugindo da perseguição de um sátiro através dos
juncos; templos de uma arquitetura majestosa elevam sua fronte soberba
acima da floresta sagrada que os envolve, a imaginação se perde nas
estradas silenciosas desse país ideal; os azulados distantes se
confundem com o céu, e a paisagem inteira, repetindo-se nas águas de um
rio tranquilo, forma um espetáculo que nenhuma língua pode descrever.

 Enquanto minha alma fazia essas reflexões, a \textit{outra} fazia seu
caminho, e Deus sabe onde ela ia! -- No lugar de ir à corte, seguindo
a ordem recebida, ela derivou de tal forma lentamente para a esquerda
que no momento em que minha alma a resgatou ela estava à porta de
\textit{madame} de \textit{Hautcastel}, a meia milha do palácio real.

 Deixo para o leitor pensar o que teria acontecido se ela tivesse
entrado sozinha na casa de uma tão bela dama. 

\section*{Capítulo VIII}

 Se é útil e agradável ter uma alma separada da matéria, a ponto de
fazê-la viajar sozinha quando se julga apropriado, esta faculdade tem
também seus inconvenientes. É a ela, por exemplo, que eu devo a
queimadura\footnote{Provável erro do autor na organização dos capítulos. A única menção a alguma queimadura no texto se encontra no final desse mesmo capítulo.}  
de que falei nos capítulos precedentes. -- De hábito dou a                         
minha besta a responsabilidade pelo preparo de meu almoço, é ela quem												
torra o meu pão e o corta em fatias. Ela faz maravilhosamente o café, e
o consome mesmo com frequência sem que minha alma se envolva, a menos
que essa se distraia ao vê-la trabalhar, mas isso é raro e muito
difícil de executar: porque é fácil pensar em outra coisa quando se faz
qualquer operação mecânica, mas é extremamente difícil de se olhar
agindo, por assim dizer -- ou para me explicar seguindo meu sistema,
de aplicar sua alma a examinar a marcha da sua besta, e de vê-la
trabalhar sem tomar parte nisso. -- Aí está o mais espantoso esforço
metafísico que o homem pode executar.

 Tinha deposto minha tenaz sobre a brasa para fazer torrar meu pão e,
algum tempo depois, enquanto minha alma viajava, um cepo em brasa rola
no chão: minha pobre besta levou a mão à tenaz e eu queimei meus dedos.

\section*{Capítulo IX}

 Espero ter desenvolvido suficientemente minhas ideias nos capítulos
precedentes para dar o que pensar ao leitor e para colocá-lo no ponto
de fazer as descobertas nesta brilhante jornada: ele só poderá ficar
satisfeito consigo se um dia chegar a saber fazer sua alma viajar
sozinha; os prazeres que esta faculdade lhe trará compensarão, de
resto, os quiproquós que poderão surgir. Haverá satisfação maior que
estender assim sua existência, ocupar ao mesmo tempo a terra e os céus
e duplicar, por assim dizer, seu ser? -- O desejo eterno e jamais
satisfeito do homem não é aumentar sua força e suas faculdades, querer
estar onde não está, lembrar o passado e viver no futuro? Quer comandar
exércitos, presidir academias, quer ser adorado pelas Belas e, se tem
tudo isso, lamenta então os campos e a tranquilidade e inveja a cabana
dos pastores: seus projetos, suas esperanças fracassam sem cessar
diante das infelicidades reais ligadas à natureza humana, não saberia
encontrar a felicidade. Um quarto de hora de viagem comigo lhe
mostraria o caminho.

 Eh! por que ele não deixa à \textit{outra} estes miseráveis cuidados,
esta ambição que o atormenta? -- Venha, pobre infeliz! Faça um esforço
para romper sua prisão e, do alto do céu aonde vou conduzi-lo, do meio
dos orbes celestes e do empíreo, olhe sua besta, jogada no mundo,
correr sozinha pelo caminho do acaso e das honras; veja com que
gravidade ela anda entre os homens: as pessoas se afastam com respeito
e, creia-me, ninguém perceberá que ela está sozinha; é a menor das
preocupações da multidão saber se tem uma alma ou não, se ela pensa ou
não. -- Mil mulheres sentimentais a amarão furiosamente sem perceber;
ela pode até mesmo alcançar, sem o socorro da sua alma, o mais alto
favor e a maior fortuna. -- Enfim, eu não me espantaria nada se, no
nosso retorno do empíreo, sua alma, voltando para casa, encontrasse na
besta um grande senhor. 

\section*{Capítulo X}

 Que não se pense que ao invés de manter minha palavra, dando a
descrição de minha viagem ao redor do meu quarto, vou para o campo para
fugir dos afazeres: grande engano, porque minha viagem realmente
continua e enquanto minha alma, dobrando-se sobre si mesma, percorria,
no capítulo precedente, os desvios tortuosos da metafísica, estava em
minha poltrona em que tinha me inclinado de forma que seus dois pés
dianteiros ficassem levantados a dois dedos do chão e, balançando-me
para a esquerda e para a direita, ganhando terreno, tinha sem perceber
chegado muito perto da parede. -- É minha forma de viajar quando não
estou apressado. -- Nesta circunstância, minha mão tinha pegado
maquinalmente o retrato de \textit{madame} de \textit{Hautcastel}, e a
outra se distraía tirando a poeira que o cobria. Esta ocupação lhe dava
um prazer tranquilo, e este prazer se fazia sentir em minha alma, ainda
que ela estivesse perdida nas extensões do céu: porque é bom observar
que, quando o espírito viaja assim no espaço, ele se mantém sempre
ligado aos sentidos por não sei que laço secreto, de sorte que, sem se
afastar de suas ocupações, ele pode tomar parte das alegrias aprazíveis
da \textit{outra}; mas, se em certo ponto esse prazer aumenta, ou se
ela é tocada por algum espetáculo inesperado, a alma imediatamente
retoma seu lugar com a velocidade do raio. 

 Foi o que me aconteceu enquanto eu limpava o retrato.

 À medida que o lenço retirava a poeira e fazia aparecer os cachos dos
cabelos louros, e a guirlanda de rosas que os coroa, minha alma, desde
o sol para onde ela se tinha transportado, sentiu um leve tremor de
prazer, e partilhou simpaticamente a alegria de meu coração. Esta
alegria tornou-se menos confusa e mais viva quando o tecido, de um só
golpe, descobriu o rosto radiante desta sedutora fisionomia; minha alma
esteve a ponto de deixar os céus para gozar do espetáculo. Mas
estivesse ela nos Campos Elíseos, estivesse assistindo a um concerto de
querubins, não teria se demorado meio segundo, quando sua companheira,
mantendo sempre mais interesse em sua obra, decidiu pegar uma esponja
molhada que se lhe apresentava e passá-la imediatamente sobre as
sobrancelhas e os olhos, sobre o nariz, sobre as maçãs do rosto, --
sobre esta boca; -- Ah Deus! meu coração bate: -- sobre o queixo,
sobre o seio: aconteceu por um momento, toda a figura pareceu renascer
e sair do nada. -- Minha alma se precipitou do céu como uma estrela
cadente: encontrou a \textit{outra} em um êxtase encantador, e chegou a
aumentá-lo ao dividi-lo. Esta situação singular e imprevista fez
desaparecer o tempo e o espaço para mim. -- Eu existi por um instante
no passado, e rejuvenesci contra a ordem da natureza. -- Sim, aí está,
esta mulher adorada, é ela mesma: eu a vejo sorrindo; ela vai falar
para dizer que me ama. -- Que olhar! Vem que te aperto contra meu
coração, alma da minha vida, minha segunda existência! -- vem
partilhar minha embriaguez e minha felicidade!

 Esse momento foi curto, mas foi encantador: a razão fria retomou logo
seu império, e, no espaço de um piscar de olhos envelheci um ano
inteiro; -- meu coração ficou frio, enregelado, e eu me encontrei no
mesmo nível da multidão dos indiferentes que pesam sobre o globo.

\section*{Capítulo XI}

 Não se deve antecipar os acontecimentos: a pressa em comunicar ao
leitor meu sistema da alma e da besta me fez abandonar a descrição de
minha cama antes do que eu deveria; quando tiver terminado retomarei
minha viagem no ponto em que a interrompi no capítulo precedente. --
Peço somente que relembre que deixamos \textit{a metade de mim}
segurando o retrato de Hautcastel perto da parede, a quatro passos de
minha escrivaninha. Tinha esquecido, falando de minha cama, de
aconselhar, a todo homem que possa, a ter uma cama cor-de-rosa e
branca: é certeza que as cores nos influenciam a ponto de nos alegrar
ou entristecer segundo suas nuances. -- O rosa e o branco são duas
cores consagradas ao prazer e à felicidade. -- A natureza, dando-as à
rosa, lhe deu a coroa do império da Flora; -- e, quando o céu quer
anunciar um belo dia ao mundo, colore as nuvens com esse encantador
matiz ao nascer do sol.

 Um dia subíamos com dificuldade a extensão de um atalho íngreme: a
amável Rosalie estava à frente, sua agilidade lhe dava asas: não
podíamos segui-la. -- De repente, tendo chegado ao pé de um monte, ela
se virou para nós para retomar o fôlego e sorriu de nossa lentidão. --
Talvez nunca as duas cores que elogio tenham triunfado dessa maneira:
suas faces afogueadas, seus lábios de coral, dentes brilhantes, seu
pescoço de alabastro sobre um fundo de verdura, golpearam todos os
olhares. Foi preciso que parássemos para contemplá-la. Nem falo de seus
olhos azuis, nem do olhar que ela pôs sobre nós, porque fugiria do meu
assunto e porque aliás penso nisso o menos que me é possível. Para mim
é suficiente ter dado o mais belo exemplo imaginável da superioridade
dessas duas cores sobre todas as outras e de suas influência sobre a
felicidade dos homens.

 Não irei mais adiante por hoje. De que assunto poderia tratar que não
fosse insípido? Que ideia não se apaga com essa ideia? -- Não sei nem
mesmo quando poderei retomar a obra. -- Se continuo, e se o leitor
deseja ver seu fim, que ele se encomende ao anjo distribuidor de
pensamentos e que lhe peça para não mais misturar a imagem desse monte
em meio à multidão de pensamentos incoerentes que ele me manda a todo
momento. 

 Sem esta precaução acabou-se a minha viagem.

\section*{Capítulo XII}

\noindent\dotfill

\noindent o monte

\noindent\dotfill

\section*{Capítulo XIII}

 Os esforços são em vão; é preciso adiar a tarefa e ficar por aqui
apesar de minha vontade: é uma etapa militar.

\section*{Capítulo XIV}

Disse que gostava, singularmente, de meditar no doce calor de minha
cama, e que sua cor agradável contribui muito para o prazer que ali
encontro.

Para obter esse prazer, meu criado recebeu a ordem de entrar no meu
quarto uma meia hora antes daquela em que tenha resolvido me levantar.
Eu o ouço caminhar com leveza e tatear pelo quarto com discrição; esse
barulho me concede o deleite de me sentir cochilar, prazer delicado e
desconhecido de muitas pessoas.

Somos despertados o suficiente para perceber que não estamos
completamente despertos e para calcular confusamente que a hora dos
afazeres e das preocupações ainda está na ampulheta do tempo.
Insensivelmente meu homem vai ficando mais barulhento; ele é difícil de
conter e além disso ele sabe que a hora fatal se aproxima. -- Olha meu
relógio, faz soar os berloques para me advertir; mas faço ouvidos
moucos e, para alongar ainda mais essa hora encantadora, não há
dificuldade que não imponha obstáculo a esse pobre
infeliz. Tenho cem ordens preliminares a lhe dar para ganhar tempo. Ele
sabe muito bem que essas ordens, que lhe dou com bastante mau humor,
são pretexto para continuar na cama simulando não desejar isso.
Ele faz cara de quem não percebe e eu lhe sou verdadeiramente grato por
isso.

Enfim, quando esgotei todos os meus recursos, ele avança para o meio do
quarto e se planta lá, os braços cruzados, na mais perfeita
imobilidade.

Concordarão comigo que não é possível desaprovar meu pensamento com mais
espírito e discrição: além disso eu nunca resisto a este convite
tácito; estendo os braços para testemunhar que compreendi, e num
instante estou sentado.

Se o leitor reflete sobre a conduta de meu criado, poderá se convencer
de que, em certos afazeres delicados, da mesma natureza desse, a
simplicidade e o bom senso valem infinitamente mais que o espírito mais
destro. Ouso assegurar que o discurso mais elaborado sobre os
inconvenientes da preguiça não me faria decidir sair tão prontamente da
minha cama como a reprovação muda de \textit{Monsieur} Joannetti.

É um perfeito bom homem esse M.~Joannetti, e ao mesmo tempo aquele
dentre todos os homens que mais conviria a um viajante como eu. Ele
está acostumado às frequentes viagens da minha alma, e jamais ri das
inconsequências da \textit{outra}, ele até mesmo a dirige quando ela
está sozinha; de maneira que poderíamos dizer então que ela é conduzida
por duas almas. Quando ela se veste, por exemplo, ele me adverte com um
sinal que ela está a ponto de colocar suas meias ao contrário, ou seu
casaco antes do colete. -- Minha alma muitas vezes se distraiu vendo o
pobre Joannetti correr atrás da louca sob os arcos da cidadela para
adverti-la de que tinha esquecido seu chapéu -- uma outra vez seu
lenço.

Um dia (confessá-lo-ei?) sem este fiel criado que a alcançou no pé da
escada, a atordoada se encaminhava para a corte sem espada, tão
audaciosamente como o grão-mestre de cerimônias carregando a augusta
batuta.

\section*{Capítulo XV}

 -- Vem, Joannetti -- eu lhe disse -- pendura este retrato.

 Ele tinha me ajudado a limpar e sabia tanto sobre o que produziu o
capítulo do retrato quanto sobre o que se passa na lua. Era justamente
ele que me tinha apresentado a esponja molhada e que, por este ato,
aparentemente indiferente, tinha feito minha alma percorrer cem milhões
de lugares em um instante. Ao invés de recolocá-lo no lugar ele parou,
por sua vez, para enxugá-lo. -- Uma dificuldade, um problema para
resolver lhe dava um ar de curiosidade que eu percebi. 

 -- Vejamos -- eu lhe disse -- o que encontras para criticar nesse
retrato?

 -- Oh! nada, \textit{monsieur}. 

-- Mas considera!

Ele o colocou sobre uma das prateleiras da minha escrivaninha; depois,
distanciando-se de alguns passos:

-- Eu queria -- ele disse -- que o senhor me explicasse porque esse
retrato me olha sempre, qualquer que seja o lugar do quarto onde estou.
Pela manhã, quando arrumo a cama, sua figura me olha e ainda me segue
com os olhos enquanto me movo.

-- De forma que, Joannetti -- eu lhe disse --, se o quarto estivesse
cheio de gente, esta bela dama repararia em todo mundo e em todos os
cantos ao mesmo tempo? 

-- Oh, sim, \textit{monsieur}.

-- Ela sorriria para os que vão e vêm como para mim?

Joannetti não respondeu nada. -- Estendi-me em minha poltrona e,
baixando a cabeça, deixei-me levar pelas meditações mais sérias. --
Que iluminação! Pobre amante! Enquanto te entedias longe de tua amante,
junto de quem talvez já tenhas sido substituído; enquanto fixas
avidamente teu olhos sobre seu retrato e imaginas (ao menos em pintura)
ser o único olhado, a pérfida esfinge, tão infiel como a original, leva
seus olhares sobre o que a cerca, e sorri para todo mundo. 

Aí está uma similaridade moral entre certos retratos e seus modelos, que
nenhum filósofo, nenhum pintor, nenhum observador tinha ainda
percebido.

Eu caminho de descobertas em descobertas.

\section*{Capítulo XVI}

 Joannetti estava ainda na mesma posição esperando a explicação que ele
tinha me pedido. Fiz sair minha cabeça das dobras de minha
\textit{roupa de viagem}, onde tinha me afundado para meditar à vontade
e para chegar às tristes reflexões que acabo de fazer.  

-- Não vês, Joannetti --, eu lhe disse depois de um momento de silêncio e
virando a minha poltrona para o seu lado --, não vês que um quadro, sendo
uma superfície plana, os raios de luz que partem de cada ponto desta
superfície\ldots?

Joannetti, com esta explicação, abriu tanto os olhos que deixava ver sua
pupila inteira; além disso tinha a boca entreaberta: esses dois
movimentos da figura humana anunciam, segundo o famoso Le
Brun,\footnote{ Charles Le Brun (1619--1690), pintor e arquiteto francês.
Pintor de grandes motivos históricos e religiosos, domina a pintura do
século \textsc{xvii} francês, tendo se ligado ao reino de Luís \textsc{xiv}. Nomeado
primeiro pintor, foi encarregado da decoração de Versalhes. Escreveu
também um \textit{Traité de physionomie}, a que X.~de Maistre faz
referência.} a última etapa do espanto. Era a minha besta, sem
dúvida, que tinha se colocado em tal conversa; minha alma, aliás, sabia
que Joannetti ignora completamente o que é uma superfície plana, e mais
ainda o que são os raios de luz: a prodigiosa abertura de suas
pálpebras tendo me feito voltar para mim mesmo, reenfiei a cabeça no
colete da minha roupa de viagem e lá a afundei de tal modo que quase a
escondi inteira.

 Resolvi jantar ali mesmo: a manhã já tinha se adiantado bastante, um
passo a mais no meu quarto teria levado meu almoço para a noite.
Escorreguei até a borda da minha poltrona e, pondo os dois pés sobre a
lareira, esperei pacientemente pela refeição. Essa é uma atitude
deliciosa: seria, eu acho, bem difícil de encontrar uma outra que
reunisse tantas vantagens e que fosse tão cômoda para as paradas em uma
viagem. 

 Rosine, minha fiel cachorrinha, nunca deixa então de vir puxar as abas
da minha casaca de viagem para que eu a pegue no colo; ela encontra um
leito pronto e bem cômodo no ângulo que formam as duas partes de meu
corpo: uma consoante V representa à maravilha minha situação. Rosine se
atira sobre mim se eu não a pego logo que pede. Eu a encontro lá com
frequência sem saber como ela veio parar ali. Minhas mãos se ajeitam
sozinhas da maneira mais favorável ao seu bem-estar seja porque há uma
simpatia entre esta amável besta e a minha, seja porque apenas o acaso
o decide -- mas não creio absolutamente no acaso, nesse triste
sistema, nessa palavra que não significa nada. -- Acreditaria antes no
magnetismo\footnote{ Trata-se aqui de Antoine Mesmer (1733--1815),
médico vienense que se estabeleceu em Paris. O ``magnetismo animal''
postulava que os corpos são suscetíveis a serem modificados pelos
``fluxos'' atmosféricos e pelos fluidos elétricos.}; -- acreditaria
antes no martinismo\footnote{ Teoria mística que ganhou o nome de seu
fundador, Martines de Pasqually, figura de identidade controversa morta
em 1778.}. Não, não acreditaria nisso jamais. 

 Há uma tal realidade nas relações que existem entre esses dois animais
que quando ponho os dois pés na lareira, por pura distração, quando a
hora do almoço ainda está distante, e eu ainda nem penso em fazer uma
\textit{parada}, Rosine, entretanto, já responde a esse movimento, trai
o prazer que experimenta mexendo levemente a cauda; a discrição a retém
em seu lugar, e a \textit{outra}, que percebe, lhe é grata: ainda que
incapazes de raciocinar sobre a causa que produz tudo isso, entre elas
se estabelece um diálogo mudo, uma relação de sensação muito agradável
e que não deveria ser atribuída, de todo, ao acaso. 

\section*{Capítulo XVII}

 Que não me reprovem por ser prolixo nos detalhes, é a maneira dos
viajantes. Quando partimos para subir o Mont Blanc, quando vamos
visitar a larga abertura do túmulo de Empédocles,\footnote{ Uma das
versões para a morte do filósofo grego Empédocles afirma que ele se
atirou na cratera do vulcão Etna.} não paramos nunca de descrever
exatamente as menores circunstâncias, o número de pessoas, o de mulas,
a qualidade das provisões, o excelente apetite dos viajantes, tudo
enfim, até os passos em falso das montarias, é registrado
cuidadosamente no diário, para a instrução do universo sedentário. Por
este princípio, resolvi falar de minha querida Rosine, adorável animal
de que gosto com verdadeira afeição, e consagrar-lhe um capítulo
inteiro. 

 Nesses seis anos que vivemos juntos não houve nenhum esfriamento entre
nós; ou, se algumas altercações se colocaram entre ela e eu, asseguro
de boa fé que o maior erro sempre esteve do meu lado e que Rosine
sempre deu os primeiros passos na direção da reconciliação.

 À noite, quando é repreendida, ela se retira tristemente e sem
murmurar: no dia seguinte, às primeiras luzes da manhã ela está ao lado
da minha cama, numa atitude respeitosa e, ao menor movimento de seu
mestre, ao menor sinal de despertar, ela anuncia sua presença pelos
batimentos precipitados da sua cauda na minha mesa de cabeceira.

 E por que eu recusaria minha afeição a este ser carinhoso que jamais
deixou de me amar desde o momento em que começamos a viver juntos?
Minha memória não seria suficiente para enumerar as pessoas que se
interessaram por mim e me esqueceram. Tive alguns amigos, muitas
amantes, uma multidão de ligações, ainda mais conhecidos; e,
entretanto, não sou mais nada para todo este mundo que até meu nome
esqueceu.

 Quantos protestos, quantas ofertas de ajuda! Eu poderia contar com sua
fortuna, sua amizade eterna e sem reservas!

 Minha querida Rosine, que não me ofereceu nenhum serviço, me dá a maior
ajuda que se pode dar à humanidade: ela me amava antes e ainda me ama
hoje. E também, eu não temo dizê-lo, eu a amo com uma porção do mesmo
sentimento que dedico aos meus amigos.

 Que se diga o que se quiser.

\section*{Capítulo XVIII}

 Deixamos Joannetti na atitude de espanto, imóvel na minha frente,
esperando o fim da sublime explicação que eu tinha começado.

 Quando me viu afundar de repente a cabeça em meu roupão e terminar
assim a minha explicação, não duvidou nem um instante que eu fosse
ficar mudo por falta de boas razões, e não me imaginava, em
consequência disso, terrificado pela dificuldade que ele tinha me
proposto.

  Apesar da superioridade que ele ganhava sobre mim, ele não demonstrou
o menor orgulho e absolutamente não procurou se aproveitar de sua
vantagem. -- Depois de um pequeno momento de silêncio, ele pegou o
retrato, recolocou-o no lugar, e se retirou delicadamente, na ponta dos
pés. Sentiu que a sua presença era uma espécie de humilhação para mim,
e sua delicadeza lhe sugeriu que se retirasse sem me deixar perceber.
-- Sua conduta, nesta ocasião, me interessou vivamente, e o colocou
para sempre em um bom lugar no meu coração. Ele terá, sem dúvida, um
lugar no do leitor; e, se ele for alguém sensível o suficiente para
recusá-lo depois de ter lido o capítulo seguinte, o céu lhe deu, sem
dúvida, um coração de mármore.

\section*{Capítulo XIX}

 -- Por Deus! -- eu lhe disse um dia --, é a terceira vez que eu
ordeno que me compres uma escova. Que cabeça! Que animal!

 Ele não disse uma palavra: já não tinha dito nada a um insulto parecido
na véspera. ``Ele é tão consciencioso!'' dizia eu; e não imaginava nada.

 -- Vai procurar um pano para limpar meus sapatos -- eu disse,
encolerizado.

Enquanto ele ia, arrependia-me de ter sido duro com ele. Minha raiva
passou imediatamente quando vi o cuidado com que se encarregava de
tirar a poeira de meus sapatos sem tocar em minhas meias. Apoiava minha
mão sobre ele em sinal de reconciliação. ``Ora! disse então para mim
mesmo, há então homens que limpam os sapatos dos outros por dinheiro?''
A palavra dinheiro foi um traço de luz que me esclareceu. Lembrei-me de
repente que fazia muito tempo que eu não dava nenhum ao meu criado.

-- Joannetti, eu lhe disse retirando meu pé, tens algum dinheiro?

Um meio sorriso de justificação apareceu em seus lábios a essa pergunta.


-- Não, senhor, há oito dias que não tenho um centavo; gastei tudo que
me pertencia para suas compras.

-- E a escova? É por isso, então, que\ldots

Ele sorriu de novo. Ele poderia ter dito a seu mestre: ``Não, não sou um
cabeça vazia, um \textit{animal}, como tivestes a crueldade de dizer a
vosso fiel servidor. Pagai-me 23 libras, dez soldos e quatro \textit{deniers}
que me deveis, e eu comprarei vossa escova.'' Ele se deixou maltratar
injustamente para não expor seu mestre à vergonha de sua própria
cólera.

Que o céu o abençoe! Filósofos, cristãos! Vocês leram isso?

-- Toma, Joannetti -- disse --, corra comprar a escova.

-- Mas, \textit{monsieur}, quereis ficar assim com um sapato branco e
outro preto?

-- Vai, eu te digo, compra a escova; deixa, deixa esta poeira sobre o
meu sapato.

Ele saiu; peguei o pano e limpei deliciosamente meu sapato esquerdo
sobre o qual deixei cair uma lágrima de arrependimento.

\section*{Capítulo XX}

 As paredes de meu quarto estão guarnecidas de estampas e quadros que o
embelezam singularmente. Gostaria de todo meu coração de fazê-los
examinar pelo leitor um a um, para agradá-lo e distraí-lo ao longo do
caminho que ainda devemos percorrer até chegar a minha escrivaninha;
mas é tão impossível explicar claramente um quadro quanto fazer um
retrato semelhante a partir de uma descrição.

 Que emoção não experimentaria, por exemplo, contemplando a primeira
estampa que se apresenta ao olhar! -- Veria ali a infeliz Carlota,
enxugando lentamente, com uma mão trêmula, as pistolas de
Alberto.\footnote{ Personagens do \textit{Werther} de Goëthe. Na cena
descrita do romance, Alberto entrega ao servo de Werther as pistolas
com que este cometerá o suicídio.} Negros pressentimentos e todas
as angústias do amor sem esperança e sem consolação estão impressas em
sua fisionomia; enquanto o frio Alberto, cercado de pastas de processos
e de velhos papéis de toda espécie, se volta friamente para desejar boa
viagem a seu amigo. Quantas vezes já não tentei quebrar o vidro que
cobre esta estampa, para arrancar este Alberto de sua mesa, insultá-lo,
massacrá-lo! Mas ainda sobrarão muitos Albertos nesse mundo. Que homem
sensível não tem o seu, com quem é obrigado a viver e contra quem os
arrebatamentos da alma, as doces emoções do coração e os impulsos da
imaginação se chocam como ondas contra os rochedos? -- Feliz daquele
que encontra um amigo cujo coração e espírito lhe convêm, um amigo que
a ele se une para uma conformidade de gostos, de sentimentos e de
conhecimentos; um amigo que não seja atormentado pela ambição ou pelo
interesse; -- que prefere a sombra de uma árvore à pompa de uma corte!

 Feliz daquele que possui um amigo!

\section*{Capítulo XXI}

 Eu tinha um: a morte arrancou-o de mim; ela o alcançou no começo de sua
carreira, no momento em que sua amizade tinha se tornado uma
necessidade imperativa para meu coração. -- Nós nos sustentávamos
mutuamente nos trabalhos penosos da guerra; só tínhamos um cachimbo
para os dois, bebíamos do mesmo copo, dormíamos sob o mesmo teto e, nas
circunstâncias infelizes em que estamos, o lugar onde vivíamos juntos
era para nós uma nova pátria: eu o vi como alvo de todos os perigos da
guerra, e de uma guerra desastrosa. A morte parecia nos economizar um
para o outro: ela esgotou mil vezes seus tiros em volta dele sem
atingi-lo; mas isso foi para me fazer mais sentida a sua perda. O
tumulto das armas, o entusiasmo que se apodera da alma exposta ao
perigo teriam impedido seus gritos de chegar ao meu coração. Sua morte
teria sido útil a seu país e funesta para os inimigos: eu a teria
sentido menos. Mas a perda no meio das delícias de um acampamento de
inverno! Vê-lo expirar nos meus braços no momento em que ele parecia
transbordar de saúde, no momento em que nossa ligação se estreitava
mais pelo repouso e pela tranquilidade! -- Ah! eu jamais me
consolarei!

 Entretanto sua memória vive apenas no meu coração; não existe mais
entre aqueles que viviam ao seu lado e que o substituíram: essa ideia
me faz ainda mais penoso o sentimento da sua perda. A natureza,
igualmente indiferente à sorte dos indivíduos, veste novamente o seu
brilhante vestido de primavera e se enfeita com toda sua beleza em
torno do cemitério onde ele repousa. As árvores se cobrem de folhas e
entrelaçam seus galhos, os pássaros cantam sob a folhagem; os insetos
zumbem em volta das flores; tudo respira a alegria e a vida no repouso
da morte. -- E à noite, enquanto a lua brilha no céu, e eu medito
perto desse lugar triste, escuto o grilo perseguir alegremente seu
canto infatigável, escondido sob a erva que cobre o túmulo silencioso
do meu amigo. A destruição insensível dos seres, e todas as
infelicidades da humanidade, nada são perto do grande todo. -- A morte
de um homem sensível que expira em meio a seus amigos desolados e a de
uma borboleta que o ar frio da manhã faz perecer no cálice de uma flor
são duas etapas comparáveis no curso da natureza. O homem não é nada
além de um fantasma, uma sombra, um vapor que se dissipa no ar\ldots

 Mas a aurora matinal começa a clarear o céu; as negras ideias que me
agitavam desaparecem com a noite e a esperança renasce no meu coração.
-- Não, aquele que inunda assim o oriente de luz não o fez brilhar
assim aos meus olhos para mergulhá-los em seguida na noite do nada.
Aquele que estende este horizonte incomensurável, aquele  que eleva
essas massas enormes, cujo sol doura os picos gelados, é também aquele
que ordenou ao meu coração que batesse e ao meu espírito que pensasse.

 Não, meu amigo não entrou no nada; qualquer que seja a barreira que nos
separa eu o verei de novo. -- Não é sobre um silogismo que eu fundo a
minha esperança. -- O voo de um inseto que atravessa os ares é
suficiente para me persuadir; e frequentemente a vista do campo, o
perfume dos ares e eu não sei que encanto espalhado a minha volta
elevam de tal maneira meus pensamentos que uma prova invencível da
imortalidade entra com violência em minha alma e a ocupa por inteiro.

\section*{Capítulo XXII}

Há muito tempo o capítulo que acabo de escrever se apresentava a minha
pena e eu o tinha sempre rejeitado. Tinha me prometido deixar ver neste
livro apenas a face risonha de minha alma, mas esse projeto me escapou
como tantos outros: espero que o leitor sensível me perdoe de lhe ter
pedido algumas lágrimas; e se alguém acha que na verdade eu poderia ter
suprimido esse capítulo triste, pode arrancá-lo de seu exemplar, ou
mesmo jogar o livro no fogo. 

Basta que o consideres de acordo com teu coração, minha querida
Jenny,\footnote{ Jeanne-Baptiste de Maistre (1762--1824), uma das irmãs
do autor.} tu, a melhor e a mais amada das mulheres; -- tu, a
melhor e a mais amada das irmãs; é a ti que dedico minha obra: se ela
tem tua aprovação terá a de todos os corações sensíveis e delicados; e
se perdoas as loucuras que algumas vezes, contra a minha vontade, me
escapam, enfrento todos os censores do universo. 

\section*{Capítulo XXIII}

 Direi apenas uma palavra sobre a outra estampa.

 É a família do infeliz Ugolino expirando de fome: em volta dele, um de
seus filhos está estendido sem movimentos a seus pés, os outros lhe
estendem os braços enfraquecidos e lhe pedem pão, enquanto o pai
infeliz, apoiado contra uma coluna da prisão, o olho fixo e selvagem, o
rosto imóvel, na horrível tranquilidade que advém do último período de
desespero, morre por sua vez a sua própria morte e a de seus filhos, e
sofre tudo que a natureza humana pode sofrer.

 Bravo cavaleiro de Assas,\footnote{ Nicolas d’Assas (1733--1760), capitão
durante a Guerra dos sete anos, morreu como herói ao ser surpreendido pelo
inimigo em uma patrulha noturna.} aí estás expirando sob cem
baionetas, por um esforço de coragem, por um heroísmo que não vemos
mais hoje em dia!

 E tu que choras sob estas palmeiras, negra infeliz! tu que um bárbaro,
que sem dúvida não era inglês, traiu e desamparou; -- que posso dizer?
tu que ele teve a crueldade de vender como uma vil escrava apesar do
teu amor e dos teus serviços, apesar do fruto da tua ternura que levas
no ventre, -- não passarei jamais frente a tua imagem sem render a
homenagem devida a tua sensibilidade e teus infortúnios!\footnote{ É
possível que o autor esteja aqui se referindo à história do livro
\textit{Narrative of Joanna}, de John Gabriel Stedman, que conta a
história de uma negra, grávida de seu raptor branco, inglês, que foi
vendida como escrava na América do Sul. Esse livro é na verdade um
excerto das crônicas de viagem intituladas \textit{Narrative of a Five
Years' Expedition Against the Revolted Negroes of
Surinam, in Guiana, on the Wild Coast of South America, from the Year
1772 to 1777}, que foi publicado originalmente em 1796, um ano depois
da primeira publicação de \textit{Voyage autour de ma chambre}.
Histórias como essa, entretanto, datadas do final do século \textsc{xviii} e
início do \textsc{xix}, já retratavam o horror -- humanista para os românticos,
capitalista para os governos -- que a Europa começava a sentir em
relação ao mundo escravocrata que ela própria tinha criado, e que
geraria, no decorrer do século \textsc{xix}, os movimentos antiescravagistas. A
observação de Xavier de Maistre, sobre o bárbaro que ``não era inglês'',
é certamente irônica.}

 Paremos um instante frente a esse outro quadro: uma jovem pastora que
cuida sozinha de seu rebanho no alto dos Alpes:\footnote{ Trata-se,
provavelmente, de um quadro do próprio autor, intitulado \textit{La
Bergère des Alpes}.} ela está sentada sobre um velho tronco de
pinheiro recoberto por grandes folhas de um tufo de cacália cuja flor
lilás se ergue acima da sua cabeça. A lavanda, o tomilho, a anêmona, a
centáurea, flores de todo tipo, que cultivamos com dificuldade em
nossas serras e jardins, e que nascem nos Alpes em toda sua beleza
primitiva, formam o tapete brilhante sobre o qual erram suas ovelhas.
-- Amável pastora, dize-me onde se encontra o canto feliz da terra
onde moras? de que aprisco distante partiste esta manhã ao nascer do
sol? -- Eu poderia morar contigo? -- Mas, pobre de mim! a doce
tranquilidade que gozas não tardará a desaparecer: o demônio da guerra,
não contente em desolar as cidades, vai logo levar a desordem e o medo
até teu retiro solitário. Os soldados já avançam; eu os vejo subir de
montanha em montanha e se aproximar das nuvens. -- O barulho do canhão
se faz ouvir na alta morada da tempestade. -- Foge, pastora, apressa
teu rebanho, esconde-te nas grotas mais distantes e mais selvagens: não
há mais repouso nesta triste terra!

\section*{Capítulo XXIV}

 Não sei como isso me acontece: há algum tempo meus capítulos terminam
sempre com um tom sinistro. Em vão, quando os inicio, fixo meus olhares
em algum objeto agradável, em vão embarco em calmaria -- logo
enfrento uma borrasca que me faz derivar. Para pôr fim a essa agitação
que não me deixa o controle de minhas ideias, e para pacificar os
batimentos do meu coração, que tantas imagens tocantes agitaram demais,
não vejo outro remédio senão uma dissertação.

Sim, quero colocar esse pedaço de vidro em meu coração.

E esta dissertação será sobre a pintura; porque não há meio agora de
dissertar sobre outro assunto. Não posso de qualquer modo descer do
ponto a que subi há pouco: além disso, é a obsessão de meu tio
Tobie.\footnote{ Referência clara ao personagem de \textit{Tristram
Shandy}, de L. Sterne.} 

Queria dizer, \textit{en passant}, algumas palavras sobre a questão da
preeminência entre a sedutora arte da pintura e a da música: sim, quero
colocar algo na balança, ainda que seja um grão de areia, um átomo.

Diz-se em favor do pintor que ele deixa alguma coisa depois dele; seus
quadros sobrevivem a ele e eternizam sua memória. 

Responde-se que os compositores deixam também óperas e concertos; --
mas a música é sujeita à moda e a pintura não o é. Os pedaços de música
que tocaram nossos avós são ridículas para os amantes de hoje e nós os
colocamos entre as operetas bufas, para fazer rir os netos daqueles que
eles outrora faziam chorar.

Os quadros de Rafael encantarão nossa posteridade como fascinaram nossos
ancestrais.

Aí está meu grão de areia.

\section*{Capítulo XXV}

 -- Mas que me importa, me disse um dia \textit{madame} de Hautcastel,
que a música de Cherubini ou de Cimarosa seja diferente daquela de seus
predecessores? Que me importa que a música antiga me faça rir, contanto
que a nova me enterneça deliciosamente? É necessário então para minha
felicidade que meus prazeres se pareçam com os da minha trisavó? O que
você me diz da pintura, uma arte que só é experimentada por uma classe
bem pouco numerosa de pessoas, enquanto que a música encanta tudo que
respira?

 Não sei mais, neste momento, o que poderia ter respondido a essa
observação, na qual eu não pensava ao começar esse capítulo.

 Se eu a tivesse previsto, talvez não tivesse empreendido esta
dissertação. E que não se tome isso de forma alguma como um preciosismo
de músico. -- Palavra de honra que não o sou; -- não, eu não sou
músico: uso o céu por testemunha assim como todos aqueles que me
ouviram tocar violão.

 Mas, supondo o mérito da arte idêntico de um lado e de outro, não seria
necessário se apressar para atribuir mérito de arte ao mérito do
pintor. -- Veem-se crianças tocarem cravo como grandes mestres; jamais
se vê um bom pintor de doze anos. A pintura, além do gosto e do
sentimento, exige uma cabeça pensante, que os músicos podem dispensar.
Veem-se todos os dias homens sem cabeça e sem coração fazerem soar num
violão, numa harpa, sons encantadores. 

 Pode-se educar a raça humana para tocar cravo e, quando ela é treinada
por um bom mestre, a alma pode viajar à vontade enquanto os dedos vão
maquinalmente tirar os sons em que ela absolutamente não se prende. --
Não se poderia, ao contrário, pintar a coisa mais simples do mundo sem
que a alma aí empregasse todas as suas faculdades.

 Se, entretanto, alguém percebesse a distinção entre a música de
composição e a de execução, asseguro que me embaraçaria um pouco. Ai de
mim! Se todos os praticantes de dissertações tivessem boa-fé, seria
assim que todas terminariam. -- Ao começar o exame de uma questão,
posto que já estamos intimamente convencidos, tomamos comumente o tom
dogmático, como eu fiz para a pintura, apesar da minha hipócrita
imparcialidade; mas a discussão levanta objeções -- e tudo termina em
dúvida.

\section*{Capítulo XXVI}

 Agora que estou mais tranquilo vou me esforçar para falar sem emoção
dos dois retratos que seguem o quadro d’ \textit{A pastora dos Alpes}. 

 Rafael! teu retrato só poderia ser pintado por ti. Quem mais ousaria
fazê-lo? -- Teu rosto aberto, sensível, espiritual, anuncia teu
caráter e teu gênio.

 Para agradar a tua sombra, coloquei perto de ti o retrato de tua
amante, a quem todos os homens de todos os séculos pedirão eternamente
contas pelas obras sublimes de que tua morte prematura privou as artes.


 Quando eu examino o retrato de Rafael sinto-me invadido de um respeito
quase religioso por este grande homem que, na flor da idade,
ultrapassou toda a Antiguidade, e cujos quadros geram admiração e
desespero nos artistas modernos. -- Minha alma, admirando-o,
experimenta um movimento de indignação contra esta italiana que
preferiu seu amor a seu amante, e que apagou em seu seio esta chama
celeste, este gênio divino.

 Infeliz! tu não sabias então que Rafael tinha anunciado um quadro ainda
melhor que o da \textit{Transfiguração}? -- Ignoravas que cingias em
teus braços o favorito da natureza, o pai do entusiasmo, um gênio
sublime, um deus?

 Enquanto minha alma faz essas observações, sua \textit{companheira},
fixando um olho atento no rosto encantador desta beleza funesta,
sente-se pronta a perdoar-lhe a morte de Rafael.

 Minha alma em vão lhe reprova sua fraqueza extravagante, ela nem é
ouvida. -- Entre essas duas senhoras se estabelece, em ocasiões desse
tipo, um diálogo singular que frequentemente termina com vantagem para
o \textit{mau princípio}, e de que eu reservo uma amostra para um outro
capítulo.

\section*{Capítulo XXVII}

 As gravuras e os quadros de que acabo de falar empalidecem e
desaparecem ao primeiro olhar que se lance sobre o quadro seguinte: as
obras imortais de Rafael, de Corrège e de toda a escola da Itália não
sustentariam o paralelo. E eu sempre o guardo para a última parte, a
peça de reserva, quando dou a alguns curiosos o prazer de viajar
comigo; e posso garantir que, assim que chamo a atenção para este
quadro sublime dos conhecedores e dos ignorantes, das pessoas do mundo,
dos artesãos, das mulheres e das crianças, mesmo dos animais, sempre vi
quaisquer espectadores darem, cada um a sua maneira, sinais de prazer e
de encantamento: de tal maneira a natureza ali foi admiravelmente
manifesta!

 Ah! que quadro poderia lhes ser apresentado, senhores; que espetáculo
poderia ser posto sob seus olhos, senhoras, mais certo de seu voto, que
a fiel representação de vocês mesmos? O quadro de que falo é um
espelho, e ninguém até hoje se lembrou de criticá-lo; ele é, para todos
que o olham, um quadro perfeito sobre o qual não há mais nada a ser
dito.

 Concordaríamos sem dúvida que ele deve ser considerado como uma das
maravilhas da região onde passeio.

 Passarei em silêncio pelo prazer que experimenta o físico meditando
sobre os estranhos fenômenos da luz que representa todos os objetos da
natureza sobre esta superfície polida. -- O espelho apresenta ao
viajante sedentário mil reflexões interessantes, mil observações que
fazem dele um objeto útil e precioso.

 Vocês, que o Amor teve ou tem ainda sob seu império, aprendam que é
frente a um espelho que ele afia suas flechas e medita suas crueldades;
é lá que ensaia suas manobras, que estuda seus movimentos, que se
prepara por antecedência para a guerra que vai declarar; é lá que ele
se exercita para os olhares doces, os pequenos melindres, os amuos
astuciosos, como um ator ensaia para si mesmo antes de se apresentar
para o público. Sempre imparcial e verdadeiro, um espelho devolve aos
olhos do espectador as rosas da juventude e as rugas da idade, sem
caluniar e sem envaidecer ninguém. -- Único entre todos os conselhos
dos grandes, ele diz constantemente a verdade.

Essa vantagem me tinha feito desejar a invenção de um espelho moral,
onde todos os homens poderiam se ver com seus vícios e suas virtudes.
Sonhava até em propor um prêmio a alguma academia por essa descoberta,
até que reflexões maduras me provaram a inutilidade disso.

Ah, é tão raro que a feiúra se reconheça e quebre o espelho! Em vão os
vidros se multiplicam em torno de nós e refletem com uma exatidão
geométrica a luz e a verdade; no momento em que os raios vão penetrar
em nossos olhos e nos pintam tais como somos, o amor-próprio insinua
seu prisma enganador entre nós e nossa imagem e nos apresenta uma
divindade.

E de todos os prismas que existiram, desde o primeiro que saiu das mãos
do imortal Newton, nenhum possuiu uma força de refração tão forte e
produziu cores tão agradáveis e tão vivas como o prisma do
amor-próprio.

Ora, já que os espelhos comuns anunciam em vão a verdade e que cada um
está contente de sua figura; já que eles não podem fazer o homem
conhecer suas imperfeições físicas, de que serviria meu espelho moral?
Pouca gente colocaria os olhos nele, e ninguém ali se reconheceria,
exceto os filósofos. -- Mesmo deles duvido um pouco.

Tomando o espelho pelo que ele é, espero que ninguém me culpe por tê-lo
colocado acima de todos os quadros da Escola da Itália. As damas, cujo
gosto não saberia fingir, e cuja decisão deve regular tudo, lançam
comumente seu primeiro olhar sobre este quadro quando elas entram em um
aposento. 

 Milhares de vezes vi damas, e mesmo cavalheiros, esquecerem seus
amantes no baile, a dança e todos os prazeres da festa, para
contemplar, com uma evidente indulgência, este quadro encantador -- e
honrarem-lhe ainda com um olhar, de tempos em tempos, no meio da
contradança mais animada.

 Quem poderia então disputar o posto que lhe atribuo entre as
obras-primas da arte de Apeles?\footnote{ Pintor grego do século 4 a.C. [N. da T.]} 

\section*{Capítulo XXVIII}

 Tinha enfim chegado perto de minha escrivaninha; esticando os braços
até já poderia tocar o ângulo mais perto de mim, quando me vi no
momento de destruir todos os meus trabalhos, e de perder a vida. -- Eu
deveria deixar passar em silêncio o acidente que me aconteceu, para não
desencorajar os viajantes; mas é tão difícil de cair da carruagem de
que me sirvo que, convenhamos, é preciso ser infeliz ao extremo --
tão infeliz como eu sou, para correr tal perigo. Encontrei-me estendido
por terra, completamente virado e revirado; e tão rápido, tão
inopinadamente, que teria sido tentado a colocar em dúvida a minha
infelicidade, se pontadas na cabeça e uma dor violenta no ombro
esquerdo não me provassem tão avidamente a sua autenticidade.

 Foi ainda um mal passo da \textit{minha metade}. Assustada pela voz de
um pobre que de repente pedia esmola em minha porta, e pelos latidos de
Rosine, ela fez cair bruscamente a minha poltrona, antes que minha alma
tivesse tempo de lhe advertir que faltava um calço atrás; a impulsão
foi tão violenta que minha carruagem se achou absolutamente fora do seu
centro de gravidade e virou sobre mim.

 Aí está, asseguro, uma das ocasiões onde tive mais de que me queixar de
minha alma; porque, no lugar de estar aborrecida pela falta que acabara
de fazer, e repreender sua companheira por sua precipitação, ela se
descuidou a ponto de partilhar o ressentimento mais \textit{animal}, e
de maltratar com palavras aquele pobre inocente.

 -- Vadio, vá trabalhar! -- ela lhe disse (invectiva execrável,
inventada pela avara e cruel riqueza!).

 -- Senhor -- ele disse para me comover --, eu sou Chambéry\ldots

 -- Azar seu.

 -- Sou Jacques; o senhor me viu no campo; era eu quem levava os
carneiros para o campo.

 -- O que o senhor faz aqui?

 Minha alma começava a se arrepender da brutalidade de minhas primeiras
palavras. -- Acredito mesmo que ela tinha se arrependido um instante
antes de deixá-las escapar. É assim que, quando encontramos
inopinadamente, no nosso caminho, um buraco ou um charco, nós o vemos,
mas não temos mais tempo de evitá-lo.

 Rosine conseguiu me devolver o bom-senso e o arrependimento: ela tinha
reconhecido \textit{Jacques}, que tinha muitas vezes dividido seu pão
com ela, e lhe testemunhava, com carinhos, sua lembrança e seu
reconhecimento.

 Enquanto isso, Joannetti, tendo reunido os restos do meu jantar que
estavam destinados para o seu próprio, deu-os para \textit{Jacques} sem
hesitar.

 Pobre Joannetti!

 É assim que, em minha viagem, vou recebendo lições de filosofia e de
humanidade de meu criado e de meu cachorro.

\section*{Capítulo XXIX}

 Antes de ir mais longe, quero destruir uma dúvida que poderia ter se
introduzido no espírito dos meus leitores. 

 Não gostaria que, por nada nesse mundo, desconfiassem que eu empreendi
essa viagem unicamente por não saber o que fazer, e forçado, de alguma
forma, pelas circunstâncias: asseguro aqui, e juro por tudo que me é
caro, que eu tinha o plano de empreendê-la muito tempo antes do
acontecimento\footnote{ Um duelo do qual o autor tomou parte, quando
era membro da guarnição de Alexandria.}  que me fez perder minha
liberdade durante quarenta e dois dias. Esta licença forçada foi apenas
uma ocasião de me colocar na estrada mais cedo.

 Sei que a declaração gratuita que faço aqui parecerá suspeita para
certas pessoas; -- mas sei também que as pessoas desconfiadas não
lerão esse livro: já têm ocupações suficientes nas suas casas e nas de
seus amigos; têm mesmo outros assuntos: e as boas pessoas acreditarão
em mim.

 Concordo, entretanto, que teria preferido me ocupar desta viagem em um
outro momento e que teria escolhido, para executá-la, mais a quaresma
que o carnaval: de qualquer modo, reflexões filosóficas, que me vieram
do céu, me ajudaram muito a suportar a privação dos prazeres que Turim
apresenta em quantidade nesses momentos de barulho e de agitação. --
Certamente, eu me dizia, as paredes do meu quarto não são tão
magnificamente decoradas quanto as de um salão de baile: o silêncio de
minha \textit{cabine} não vale o agradável barulho da música e da
dança; mas, entre as brilhantes personagens que encontramos nessas
festas, é certo que há algumas mais entediadas que eu. 

 E por que eu teimaria em considerar os que estão em uma situação
agradável, enquanto o mundo formiga de pessoas mais infelizes que eu na
minha? -- No lugar de me transportar pela imaginação para esse soberbo
cassino de diversões, onde tantas belezas são eclipsadas pela jovem
Eugénie, para me pensar feliz tenho que apenas parar um instante ao
longo das ruas que conduzem para lá. -- Um punhado de desafortunados,
deitados seminus sob os pórticos de prédios suntuosos, parecem
próximos de morrer de frio e miséria.

 Que espetáculo! Queria que esta página do meu livro fosse conhecida de
todo o universo; queria que se soubesse que, nesta cidade, onde tudo
respira opulência, durante as noites mais frias de inverno, uma
multidão de infelizes dorme ao relento, a cabeça apoiada sobre uma
pedra ou sobre a soleira de um palácio.

 Aqui, um grupo de crianças apertadas umas contra as outras para não
morrer de frio. -- Lá, uma mulher tremente e sem voz para queixar-se.
-- Os passantes vão e vêm, sem ficarem emocionados com um espetáculo
ao qual estão acostumados. -- O barulho das carruagens, a voz da
intemperança, os sons encantadores da música, se misturam às vezes aos
gritos destes infelizes, e formam uma horrível dissonância.

\section*{Capítulo XXX}

 Aquele que se apressasse a julgar uma cidade a partir do capítulo
precedente se enganaria muito. Falei dos pobres que lá encontramos, de
seus gritos comoventes e da indiferença de algumas pessoas em relação a
eles; mas não disse nada da multidão de homens caridosos que dormem
enquanto os outros se divertem, que se levantam cedo no começo do dia e
vão socorrer o infortúnio, sem testemunho e sem ostentação. -- Não,
não deixarei isso em silêncio: -- quero escrever sobre o reverso da
página \textit{que todo o universo deve ler}.

 Depois de ter assim partilhado sua fortuna com seus irmãos, depois de
ter derramado bálsamo nesses corações despedaçados pela dor, eles vão
para as igrejas, enquanto o vício fatigado dorme sobre o edredom,
oferecer a Deus suas orações e lhe agradecer suas dádivas: a luz da
lâmpada solitária enfrenta ainda no templo a do dia nascente e eles já
estão prostrados ao pé dos altares; -- e o Eterno, irritado pela
duração da avareza dos homens, detém seu raio pronto para punir.

\section*{Capítulo XXXI}

 Quis dizer alguma coisa sobre estes infelizes na minha viagem, porque a
ideia da sua miséria com frequência me desviou do caminho. Às vezes
tocado pela diferença de sua situação e da minha, parava de repente
minha berlinda, e meu quarto me parecia prodigiosamente embelezado. Que
luxo inútil! Seis cadeiras! duas mesas! uma escrivaninha! um espelho!
que ostentação! Minha cama, sobretudo, minha cama, cor de rosa e
branca, e minhas duas cobertas, me pareciam desafiar a magnificência e
a moleza dos monarcas da Ásia. -- Estas reflexões me fizeram
indiferente aos prazeres que me haviam garantido: e, de reflexões em
reflexões, meu acesso de filosofia chegava a tal ponto que eu teria
visto um baile no quarto vizinho, que eu teria ouvido o som de violinos
e clarinetes, sem sair do meu lugar; -- teria ouvido com meus dois
ouvidos a voz melodiosa de Marchesini,\footnote{ Castrato italiano
(1755--1829) de grande sucesso, cujo nome verdadeiro era Louis Marchesi.}
essa voz que me pôs tantas vezes fora de mim -- sim, eu o
teria ouvido sem tremer; -- ainda mais, teria olhado sem a menor
emoção a mais bela mulher de Turim, a própria Eugénie, arrumada dos pés
à cabeça pelas mãos de \textit{Mademoiselle} Rapous.\footnote{ Modista
famosa no momento em que o autor escrevia a obra.} -- Isso, no
entanto, não é bem certo. 

\section*{Capítulo XXXII}

 Mas, permitam-me perguntar, senhores, vocês se divertem tanto quanto
antes no baile e no teatro? -- Para mim, asseguro, há algum tempo
todas as assembleias numerosas me inspiram um certo terror. Nelas me
vejo assolado por um sonho sinistro. -- Em vão faço esforços para
espantá-lo, ele volta sempre, como o de \textit{Athalie}.\footnote{
Personagem que dá nome a uma das tragédias de Racine. Athalie, rainha
de Judá, após ter eliminado sua família para evitar possíveis herdeiros
do trono, tem um sonho recorrente com um menino que ela acaba por
encontrar e que, ela não sabe, é seu neto, salvo ainda bebê do seu
projeto assassino.} -- Talvez porque a alma, inundada hoje de
ideias negras e quadros dolorosos, encontra em todos os lugares razões
para tristeza -- como um estômago viciado converte em veneno os
alimentos mais saudáveis. De qualquer modo, aí está meu sonho: quando
estou em uma dessas festas, no meio dessa multidão de homens amáveis e
calorosos, que dançam, que cantam -- que choram nas tragédias, que só
demonstram alegria, franqueza e cordialidade, digo a mim mesmo: -- Se,
nessa assembleia educada, entrasse de repente um urso branco, um
filósofo, um tigre, ou qualquer outro animal desta espécie, e, subindo
até a orquestra, ele gritasse com uma voz furiosa:

 -- Humanos infelizes! ouçam a verdade que lhes fala pela minha boca:
vocês são oprimidos, tiranizados, vocês são infelizes, vocês se
entediam. Saiam dessa letargia! Vocês, músicos, comecem por quebrar
esses instrumentos sobre as suas cabeças; que cada um se arme de um
punhal; a partir de agora não pensem mais em diversões nem em festas;
subam às coxias, degolem todo mundo; que as mulheres também mergulhem
suas mãos tímidas no sangue! Saiam, vocês estão livres, arranquem seu
rei de seu trono e seu Deus de seu santuário!

 Pois bem! Isso que o tigre disse, quantos desses homens
\textit{encantadores} executariam? -- Quantos talvez pensassem nisso
antes que ele entrasse? Quem o sabe? -- Não se dançava em Paris há
cinco anos atrás?\footnote{ ``Vê-se que esse capítulo foi escrito em
1794; é fácil de se perceber lendo a obra que ele foi abandonado e
retomado.'' [N. do A.] agregada à edição de 1839.}

 -- Joannetti, fecha as portas e as janelas. -- Não quero mais ver a
luz; que nenhum homem entre em meu quarto; -- coloca meu sabre ao
alcance de minha mão -- sai mesmo tu, e não reaparece mais na minha
frente!

\section*{Capítulo XXXIII}

 -- Não, não, fica, Joannetti; fica, pobre menino: e tu também, Rosine;
tu que adivinhas minhas dores e que as ameniza com teus carinhos; vem,
minha Rosine; vem. Consoante V e moradia.\footnote{ Ver capítulo \textsc{xvi}. O
narrador oferece aqui, à cachorrinha Rosine, a consoante \textsc{v} -- a
posição em V em que ele se coloca para lhe dar colo -- e moradia.} 

\section*{Capítulo XXXIV}

 A queda de minha poltrona prestou ao leitor o serviço de reduzir minha
viagem de uma boa dúzia de capítulos, porque levantando eu estava
frente a frente com a minha escrivaninha, e porque não tinha mais tempo
de fazer reflexões sobre o número de estampas e quadros que eu ainda
tinha a percorrer e que teriam podido alongar minhas excursões sobre pintura.

 Deixando então à direita os retratos de Rafael e sua amante, o
cavaleiro d’Assas e a pastora dos Alpes, seguindo à esquerda ao lado da
janela, descobrimos a minha escrivaninha: é o primeiro e mais visível
objeto que se apresenta aos olhos do viajante, seguindo a estrada que
acabo de indicar.

 Acima dela há algumas prateleiras que servem de biblioteca; -- o todo
é coroado por um busto que termina a pirâmide, e este é o objeto que
mais contribui para o embelezamento da região.

 Abrindo a primeira gaveta à direita, encontramos um escrínio, papel de
todo tipo, penas todas apontadas, cera para lacrar. -- Tudo isso daria
vontade de escrever mesmo ao ser mais indolente. -- Estou certo, minha
cara Jenny, que, se por acaso abrisses esta gaveta, responderias à
carta que te escrevi no ano passado. -- Na gaveta oposta jazem
confusamente empilhados os materiais da enternecedora história da
prisioneira de Pignerol, que vocês lerão em seguida, meus queridos
amigos.\footnote{ O autor não manteve a palavra; e se qualquer coisa
apareceu sob esse título, o autor da \textit{Viagem em torno do meu
quarto} declara que ele não tem nada a ver com isso. [N. do A.]} 

 Entre estas duas gavetas há uma depressão onde jogo as cartas à medida
que as recebo: lá se encontram todas aquelas que recebi nos últimos dez
anos; as mais antigas estão arrumadas, segundo suas datas, em muitos
pacotes, as novas estão misturadas; restam-me muitas que datam do
início da minha juventude.

 Que prazer rever nessas cartas as situações interessantes dos nossos
verdes anos, sermos transportados de novo para esses tempos felizes que
não veremos mais!

 Ah! como meu coração está repleto! Como ele se deleita tristemente
enquanto meus olhos percorrem as linhas traçadas por um ser que não
existe mais! Aí está sua grafia, seu coração conduzia sua mão, para mim
ele escrevia esta carta, e esta carta é tudo que me resta dele!

 Quando coloco a mão neste reduto, é raro a tire dali durante todo o
dia. É assim que o viajante atravessa rapidamente algumas províncias da
Itália, fazendo com pressa algumas observações superficiais, para se
fixar em Roma durante meses inteiros. -- É o mais rico veio da mina
que exploro. Que mudança nas minhas ideias e nos meus sentimentos! Que
diferença nos meus amigos! Quando os examino nesse tempo e hoje,
vejo-os mortalmente agitados pelos projetos que hoje não mais lhes
dizem respeito. Olhamos um acontecimento como uma grande infelicidade;
mas falta o fim da carta, e o fim do acontecimento é completamente
esquecido: não posso saber do que se tratava -- mil prejulgamentos
nos assaltavam; o mundo e os homens nos eram totalmente desconhecidos,
mas também que calor havia nas nossas trocas! que ligação íntima! que
confiança sem limites!

 Éramos felizes com nossos erros. -- E agora: -- Ah! não é mais assim;
foi necessário que lêssemos, como os outros, no coração humano; e a
verdade, caindo no meio de nós como uma bomba, destruiu para sempre o
palácio encantado da ilusão.

\section*{Capítulo XXXV}

 Só caberia a mim fazer um capítulo sobre esta rosa seca que aí está, se
o assunto valesse a pena: é uma flor do carnaval do ano passado. Eu
mesmo tinha ido colhê-la nas serras do Valentin\footnote{ Castelo real,
situado fora de Turim, na altura da ponte suspensa, sobre a margem
esquerda do Pó, e que foi transformado em uma fábrica de tabaco.}
e à noite, uma hora antes do baile, cheio de esperança e com uma
agradável emoção, eu iria apresentá-la a \textit{madame} de Hautcastel.
Ela a pegou, colocou-a sobre a penteadeira sem olhá-la e sem olhar
mesmo para mim. -- Mas como ela poderia ter me dado atenção? estava
ocupada olhando para si mesma. Em pé frente a um grande espelho, toda
arrumada, dava um último toque em seus adornos: estava tão preocupada,
sua atenção estava tão completamente absorvida pelas fitas, as gazes e
os pompons de todo tipo amontoados a sua frente, que eu não consegui
nem mesmo um olhar, um sinal. -- Resignei-me: segurava humildemente os
alfinetes prontos, arrumados na minha mão; mas sua almofadinha de
alfinetes estando mais ao seu alcance, ela os pegava de sua
almofadinha -- e se eu estendesse a mão, ela os pegava da minha mão
--, indiferentemente; e para pegá-los ela tateava, sem tirar os olhos
de seu espelho, por medo de se perder de vista.

 Segurei por algum tempo um segundo espelho atrás dela, para lhe
permitir julgar melhor seus adornos; e, sua fisionomia se repetindo de
um espelho em outro, vi então uma perspectiva de coquetes onde nenhuma
me dava atenção. Enfim, deveria reconhecer? fazíamos, minha rosa e eu,
uma muito triste figura.

 Acabei por perder a paciência e, não podendo mais resistir ao despeito
que me devorava, larguei o espelho que segurava e saí com um ar
furioso, e sem pedir licença.

 -- Onde o senhor vai? -- ela me disse virando-se de lado para ver seu
perfil. 

Não respondi nada; mas fiquei algum tempo escutando na porta, para saber
o efeito que produziria minha saída.

 -- Não vês -- ela dizia a sua camareira, depois de um instante de
silêncio --, não vês que esse corselete é muito largo para meu
tamanho, sobretudo em baixo, e que é preciso fazer uma pence com os
alfinetes?

 Como e por que essa rosa seca se encontra lá sobre uma prateleira da
minha escrivaninha é o que eu certamente não direi, porque declarei que
uma rosa seca não merece um capítulo.

 Notem bem, minhas senhoras, que não faço nenhum comentário sobre a
ventura da rosa seca. Não digo que \textit{Madame} de Hautcastel fez
bem ou mal por preferir, a mim os seus adornos, nem que eu tenha o
direito de ser recebido de outra maneira. 

 Evito ainda com mais cuidado tirar conclusões gerais sobre a realidade,
a força e a duração do afeto das damas por seus amigos. Eu me contento
em lançar este capítulo (já que é um), de lançá-lo, digo, no mundo, com
o resto da minha viagem, sem endereçá-lo a ninguém, e sem recomendá-lo
a ninguém.

 Acrescentaria apenas um conselho para vocês, senhores; coloquem bem na
sua cabeça o fato de que em um dia de festa sua amante não lhes
pertence mais.

 No momento em que os adornos começam o amante não é mais que um marido,
e o baile apenas se torna o amante.

 Todo mundo sabe, de resto, o que ganha um marido em querer se fazer
amado à força; recebam pois seu infortúnio com paciência e bom humor.

 E não se iluda, senhor: se o senhor é tratado bem no baile, não é de
forma alguma por sua qualidade como amante, é porque o senhor faz parte
do baile, e é, consequentemente, uma fração de sua nova conquista; o
senhor é um décimo de um amante: ou mesmo, talvez, seja porque o senhor
dança bem, e a fará brilhar. Enfim, o que pode haver aí de mais
lisonjeiro para o senhor na boa acolhida que ela lhe der é que ela
espere que declarando como seu amante um homem de mérito como o senhor
ela incitará o ciúme de suas companheiras; sem esta consideração, ela
nem mesmo o olharia. 

 Aí se vê quem compreende a situação; será necessário resignar-se e
esperar que vosso papel de marido tenha passado. -- Conheço mais de um
que gostaria de se livrar da situação por um preço assim tão baixo.

\section*{Capítulo XXXVI}

 Prometi um diálogo entre minha alma e a \textit{outra}; mas há certos
capítulos que me escapam, ou antes há outros que escorrem da minha
pluma, apesar de mim mesmo, e que derrotam meus projetos: deste número
é o que trata da minha biblioteca, que farei o mais curto possível. --
Os quarenta e dois dias vão terminar e um espaço de tempo igual não
será suficiente para dar conta da rica região onde viajo
agradavelmente.

 Minha biblioteca é, portanto, composta por romances, já que é preciso
lhes dizer -- sim, romances e alguns poetas escolhidos.

 Como se já não tivesse problemas suficientes, partilho ainda
voluntariamente os de mil personagens imaginários, e os sinto tão
vivamente quanto os meus: quantas lágrimas não verti por esta infeliz
Clarissa\footnote{ Personagem do romance \textit{Clarissa, or the
History of a Young Lady} (1748), de Samuel Richardson, traduzido na
França pelo Abbé Prevost.} e pelo amante de Carlota.\footnote{ Personagem 
do \textit{Werther} (1774), de Goëthe.}

 Mas se procuro assim falsas aflições, encontro, ao contrário, neste
mundo imaginário, a virtude, a bondade, o desinteresse, que ainda não
encontrei reunidos no mundo real onde existo. -- Encontro lá uma
mulher como a desejo, sem caprichos, sem leviandade, sem melindre: não
digo nada sobre a beleza; podemos nos fiar em minha imaginação, eu a
faço tão bela que não há nada a dizer. Em seguida, fechando o livro,
que não responde mais a minhas ideias, eu a tomo pela mão e percorremos
juntos uma região mil vezes mais deliciosa que o Éden. Que pintor
poderia representar a paisagem encantada onde coloquei a divindade do
meu coração? e que poeta poderá jamais descrever as sensações vivas e
variadas que experimento nessas regiões encantadas?

 Quantas vezes não maldisse este \textit{Cleveland},\footnote{ \textit{Le
Philosophe Anglais ou Histoire de Monsieur Cleveland} (1739), romance
do Abbé Prevost.} que se engaja a todo instante em novas
infelicidades que ele poderia evitar! -- Não posso suportar este livro
e este encadeamento de calamidades; mas se o abro por distração,
preciso devorá-lo até o fim.

 Como deixar este pobre homem na terra dos Abaquis? o que seria dele com
esses selvagens? Ouso ainda menos abandoná-lo na excursão que faz para
sair do seu cativeiro.

 Enfim, sinto de tal forma suas penas, interesso-me de tal maneira por
ele e sua desafortunada família, que a aparição inesperada dos ferozes
Ruintons me deixam de cabelos arrepiados: um suor frio me cobre
enquanto eu leio essa passagem, e meu medo é tão vivo, tão real como se
fosse eu a ser assado e comido por esta canalha.

 Quando já chorei e amei o suficiente procuro algum poeta e parto de
novo para outro mundo. 

\section*{Capítulo XXXVII}

 Desde a expedição dos Argonautas até a assembleia dos Notáveis; desde o
mais fundo do inferno até a última estrela fixa além da Via Láctea, até
os confins do universo, até as portas do caos, esse é o vasto campo
onde passeio de todas as maneiras  e totalmente à vontade, porque não
me falta mais que o espaço. É para lá que transporto minha existência a
partir de Homero, de Milton, de Virgílio, de Ossian etc.

 Todos os acontecimentos que se sucederam entre essas duas épocas, todas
as regiões, todos os mundos e todos os seres que existiram entre esses
dois termos, tudo é meu, tudo isso me pertence tão bem, tão
legitimamente como os barcos que entravam no Pireu pertenciam a um
certo ateniense.\footnote{ Trasilau, personagem grego retomado por
Camões em uma de suas oitavas, acreditava em sua loucura ser
proprietário de todos os navios que entrassem no porto de Pireu, em Atenas.} 

 Amo sobretudo os poetas que me transportam para a mais alta
Antiguidade: a morte do ambicioso Agamêmnon, os furores de Orestes e
toda a história trágica da família dos Atreus, perseguidos pelo céu, me
inspiram um terror que os acontecimentos modernos não fazem nascer em
mim.

 Vejam a urna fatal que contém as cinzas de Orestes. Quem não tremeria
diante disso? Electra! Infeliz irmã, acalma-te: é o próprio Orestes
quem traz a urna, e estas cinzas são as de seus inimigos!

 Não encontramos mais rios parecidos com Xanto ou Escamandro; -- não se
veem mais campos como os da Hespéria ou da Arcádia. Onde estão hoje as
ilhas de Lemnos e de Creta? Onde está o famoso labirinto? Onde está o
rochedo que Ariane, abandonada, molhava com suas lágrimas? -- Não
vemos mais Teseus, Hércules menos ainda; os homens e mesmo os heróis de
hoje são pigmeus.

 Quando quero me dar em seguida uma cena animada e gozar de todas as
forças de minha imaginação, me agarro corajosamente às dobras da veste
flutuante do divino cego de Albion,\footnote{ Milton, autor de
\textit{Paraíso perdido}. Albion é um nome antigo, e literário, da
própria Inglaterra.} no momento em que ele se lança para o céu, e
em que ousa se aproximar do trono do Eterno. -- Que musa pôde
sustentá-lo, a essa altura, onde nenhum homem antes dele tinha ousado
levar seu olhar? Do ofuscante átrio celeste que o avaro Mamon olhava
com seus olhos invejosos, passo com horror para as vastas cavernas onde
habita Satã; -- assisto ao conselho infernal, misturo-me à multidão de
espíritos rebeldes e escuto seus discursos.

 Mas é preciso que eu confesse aqui uma fraqueza que frequentemente me reprovei. 

 Não posso me impedir de dedicar algum interesse a este pobre Satã (falo
do Satã de Milton) desde que ele foi expulso do céu. Mesmo condenando a
opiniaticidade do espírito rebelde, asseguro que a firmeza que ele
demonstra no extremo do infortúnio e a grandeza de sua coragem me
forçam a admirá-lo apesar de mim mesmo. -- Ainda que eu não ignore as
desgraças derivadas da empresa funesta que o conduziu a empurrar a
porta dos infernos para vir perturbar a relação dos nossos primeiros
pais eu não posso, por mais que eu tente, desejar o momento de vê-lo
perecer no caminho ante a confusão do caos. Até acredito que o ajudaria
voluntariamente sem a vergonha que me detém. Sigo todos os seus
movimentos e sinto tanto prazer em viajar com ele como se estivesse em
boa companhia. Ainda é bom considerar que, apesar de tudo, é um diabo,
que tem como projeto pôr a perder o gênero humano; que é um verdadeiro
democrata, não como aqueles de Atenas, mas como os de Paris: mas tudo
isso não me cura de minha opinião.

 Que vasto projeto! E que audácia na execução!

 Quando as espaçosas e tríplices portas dos infernos se abriram na sua
frente, de batente a batente, e quando o profundo buraco do nada e da
noite apareceu sob seus pés em todo seu horror, ele percorreu com um
olhar intrépido o sombrio império do caos; e, sem hesitar, abrindo suas
vastas asas, que poderiam cobrir um exército inteiro, precipitou-se no
abismo.

 Desafio o mais valente a tentar. E este é, para mim, um dos belos
esforços da imaginação, como uma das mais belas viagens que jamais
foram feitas -- depois da viagem em torno do meu quarto.

\section*{Capítulo XXXVIII}

 Não terminaria mais se quisesse descrever a milésima parte dos eventos
singulares que me acontecem quando viajo perto de minha biblioteca; as
viagens de Cook e as observações de seus companheiros de viagem, os
doutores Banks e Solander, não são nada em comparação a minhas
aventuras neste único lugar: acho também que passaria minha vida nele
numa espécie de encantamento, não fosse o busto de que falei, sobre o
qual meus olhos e pensamentos terminam sempre por se fixar, qualquer
que seja a situação de minha alma; e, quando ela está violentamente
agitada, ou quando se abandona ao desânimo, só posso olhar esse busto
para recolocá-la em estado normal: é o diapasão com o qual afino o
arranjo variável e discordante de sensações e percepções que formam
minha existência. 

 Como é parecido! -- Ali estão os traços que a natureza tinha dado ao
mais virtuoso dos homens. Ah! se o escultor tivesse podido fazer
visível sua grande alma, seu gênio e seu caráter! -- Mas em que me
meti? Por acaso aqui é o lugar de fazer seu elogio? É para os homens a
minha volta que me dirijo? Ah! que lhes importa? 

 Contento-me em prostrar-me frente a tua imagem adorada, oh! o melhor
dos pais! Ai de mim! esta imagem é tudo que me resta de ti e de minha
pátria: deixaste a terra no momento em que o crime ia invadi-la; e tais
são os males com que estes nos abatem que tua própria família é
constrangida a olhar hoje a tua perda como uma benesse. Por quantos
males passarias em uma vida mais longa! Ô meu pai, tu, na morada
da felicidade, conheceste a sorte de tua família numerosa? Sabes que teus
filhos estão exilados desta pátria que serviste durante sessenta anos
com tanto zelo e integridade? Sabes que eles estão proibidos de visitar
teu túmulo? -- Mas a tirania não pôde arrancar-lhes a parte mais
preciosa de tua herança, a lembrança de tuas virtudes e a força de teus
exemplos: no meio da torrente criminosa que levava sua pátria e sua
fortuna ao abismo, eles permaneceram inalteravelmente unidos sobre a
linha que tu lhes tinha traçado; e, quando eles puderem de novo
prostrar-se sobre tuas cinzas veneradas, estas certamente os
reconhecerão. 

\section*{Capítulo XXXIX}

 Prometi um diálogo, mantenho a palavra. -- Era manhã, ao nascer do
dia: os raios de sol douravam ao mesmo tempo o pico do monte Viso e o
das montanhas mais altas da ilha que é nossa antípoda; e \textit{ela}
já estava acordada, seja porque seu despertar prematuro foi o resultado
das visões noturnas que a colocam com frequência em uma agitação tão
fatigante quanto inútil; seja porque o carnaval, que estava então em
seu fim, foi a causa oculta de seu despertar -- esse tempo de prazer e
de loucura tendo uma influência sobre a máquina humana como as fases da
lua e a conjunção de certos planetas. Enfim, \textit{ela} estava
acordada e muito acordada, quando minha alma se livrou dos laços do
sono. 

 Há muito tempo esta partilha confusamente as sensações da
\textit{outra}; mas ela estava ainda embaraçada entre os crepes da
noite e do sono; e estes crepes lhe pareciam transformados em gazes, em
cambraias, em musselinas. -- Minha pobre alma estava então como que
empacotada em todo esse aparato, e o deus do sono, para retê-la mais
seguramente em seu império, ajuntou a esses laços tranças de cabelos
louros em desordem, laços de fitas, colares de pérolas: era digna de
piedade para quem a visse debater-se nessas redes. 

 A agitação da minha parte mais nobre se comunicava com a
\textit{outra}, e esta por sua vez agia fortemente sobre minha alma.
-- Eu tinha chegado a um estado difícil de descrever até que minha
alma, enfim, seja por sagacidade, seja por acaso, encontrou a maneira
de se livrar das gazes que a sufocavam. Não sei se ela encontrou uma
abertura ou se ela pensou simplesmente em levantá-las, o que seria mais
natural; o fato é que ela encontrou a saída do labirinto. As tranças de
cabelos em desordem ainda estavam lá; mas isso não era mais um
\textit{obstáculo}, era antes um \textit{meio}; minha alma a segurou,
como um homem que se afoga se agarra aos capins do rio; mas o colar de
pérolas se rompeu durante a ação, e as pérolas desfiando rolaram sobre
o sofá e de lá sobre o \textit{parquet} de Madame de Hautcastel; porque
minha alma por uma bizarria cuja razão seria difícil de explicar se
imaginava na casa desta dama: um grande buquê de violetas caiu no chão,
e minha alma, acordando por isso, voltou para sua casa, trazendo assim
a razão e a realidade. Como se imagina, ela desaprovou seriamente tudo
que se tinha passado em sua ausência; e é aqui que começa o diálogo que
é o assunto desse capítulo.

 Minha alma nunca tinha sido tão mal recebida. As reprovações que ela
considerou fazer neste momento crítico acabaram por perturbar a
relação: foi uma revolta, uma insurreição formal.

 -- E então! -- disse minha alma --, é assim que, durante minha
ausência, no lugar de recuperar suas forças com um sono agradável, e
tornar-se assim mais pronta para executar minhas ordens, a senhora
pensa insolentemente (o termo sendo um pouco pesado) em se permitir
arroubos que eu não sancionei?

 Pouco acostumada a esse tom de superioridade, a \textit{outra} replicou
irritada:

 -- A senhora acha que lhe cai bem, \textit{Madame} (para distanciar da
discussão toda ideia de familiaridade), que lhe cai bem dar-se ares de
decência e virtude? Ah! não é aos desregramentos de sua imaginação e
suas ideias extravagantes que eu devo tudo que lhe desagrada em mim?
Por que a senhora não estava lá? Por que a senhora teria o direito de
deleitar-se sem mim, nas frequentes viagens que faz sozinha? Alguma vez
desaprovei suas seções no empíreo ou nos Campos Elísios, suas
conversações com os intelectuais, suas profundas especulações (um pouco
de chacota, como se vê), seus castelos na Espanha, seus sistemas
sublimes? E eu não teria o direito, quando a senhora me abandona assim,
de me deleitar com os favores que me concede a natureza e os prazeres
que ela me apresenta?

 Minha alma, surpresa por tanta vivacidade e eloquência não sabia o que
responder. -- Para arrumar a situação, encarregou-se de cobrir com o
véu da indulgência as reprovações a que \textit{ela} tinha se
permitido; e, afim de não parecer ser ela a fazer os primeiros passos
para a reconciliação, considerou usar também o tom de cerimônia.

 -- ``\textit{Madame}'' (Se o leitor achou essa palavra deslocada quando
se dirigia a minha alma, que dirá agora, por pouco que queira
lembrar-se do assunto dessa discussão? Minha alma não percebeu o
extremo ridículo desse modo de falar, tanto a paixão obscureceu a
inteligência!), disse por sua vez com uma cordialidade afetada\ldots\ --
``\textit{Madame} -- disse então --, eu asseguro à senhora que nada me daria
tanto prazer como vê-la deleitar-se com todos os prazeres a que sua
natureza é suscetível, ainda que eu não partilhe deles, se estes
prazeres não lhe fossem nocivos e se eles não alterassem a harmonia que\ldots

Aqui minha alma foi interrompida vivamente: 

-- Não, não, não sou mais a tola da sua suposta indulgência: a estada
forçada que fazemos juntos neste quarto em que viajamos; a ferida que
recebi, que quase me destruiu, e que ainda sangra; -- tudo isso não é
fruto do seu orgulho extravagante e de seus preconceitos bárbaros? Meu
bem-estar e minha própria existência não valem nada quando suas paixões
a dirigem, e a senhora pretende se interessar por mim, e que suas
reprovações vêm de sua amizade?

 Minha alma viu bem que não representava o melhor papel nesta
ocasião:  começava além disso a perceber que o calor da discussão tinha
obliterado sua causa e, aproveitou a circunstância para fazer uma
brincadeira: ``\textit{Faça café}'', ela disse a Joannetti que entrava no
quarto. -- O barulho das xícaras atraíram toda a atenção da
\textit{insurgente},\footnote{ O substantivo ``insurgent'' não é comum em
francês, apenas o verbo. O substantivo já havia sido usado para
denominar um certo grupo militar húngaro e mais tarde os grupos
revoltosos americanos que se levantaram contra os \mbox{ingleses.}} 
naquele momento ela esqueceu todo o resto. É assim que mostrando um
chocalho para crianças nós as fazemos esquecer os frutos malsãos que
elas pedem sapateando.

 Cochilei insensivelmente enquanto a água esquentava. -- Gozava do
prazer encantador com que entretive meus leitores e que experimentamos
quando nos sentimos adormecer. O barulho agradável que fazia Joannetti,
batendo a cafeteira nos cães da lareira, ressoava em meu cérebro e
fazia vibrar todas as minhas fibras sensitivas, como a vibração de uma
corda de harpa faz ressoar as oitavas. -- Enfim, vi como que uma
sombra a minha frente; abri os olhos, era Joannetti. Ah! que perfume!
que agradável surpresa! Café! creme! uma pirâmide de pães tostados!

 Bom leitor, tome o desjejum comigo.

\section*{Capítulo XL}

 Que tesouro rico em delícias a boa natureza entregou aos homens cujo
coração sabe aproveitar! e que variedade de delícias! Quem poderia
contar suas nuances inumeráveis nos diversos indivíduos e nas
diferentes idades da vida? a lembrança confusa das da minha infância
ainda me faz vibrar. Tentarei pintar aquelas que experimenta o homem
jovem cujo coração começa a queimar com todos os fogos do sentimento?
Nesta idade feliz em que ignoramos ainda até mesmo o nome do interesse,
da ambição, do ódio e de todas as paixões vergonhosas que degradam e
atormentam a humanidade; durante este período, ai ai!, muito curto, o
sol brilha com uma luz que não encontramos mais pelo resto da vida. O
ar é mais puro, as fontes são mais límpidas e mais frescas; -- a
natureza tem particularidades, os bosques têm atalhos que não
encontramos mais na idade madura. Deus! que perfumes enviam essas
flores! como esses frutos são deliciosos! com que cores a aurora se
paramenta! Todas as mulheres são amáveis e fiéis; todos os homens são
bons, generosos e sensíveis: em todos os lugares encontramos
cordialidade, franqueza e ações desinteressadas; na natureza só existem
flores, virtudes e prazeres.

 As agitações do amor, a certeza da felicidade não inundam nosso coração
de sensações tão vivas quanto variadas?

 O espetáculo da natureza e sua contemplação como um todo e nos detalhes
abrem para a razão uma imensa carreira de deleites. Logo a imaginação,
planando sobre esse oceano de prazeres, aumenta o seu número e
intensidade; as sensações diversas se unem e se combinam para formar
novas; os sonhos da glória se misturam às palpitações do amor; a
generosidade anda ao lado do amor-próprio que lhe estende a mão; a
melancolia vem de tempos em tempos jogar sobre nós seu véu solene, e
mudar nossas lágrimas em prazeres. Enfim, as percepções do espírito, as
sensações do coração, as próprias lembranças dos sentidos são, para o
homem, fontes inesgotáveis de prazeres e de felicidade.

 Que não nos assustemos, portanto, com o fato de que o barulho que fazia
Joannetti, batendo a cafeteira nos cães da lareira, e o aspecto
inesperado de uma xícara de creme tenham gerado em mim uma impressão
tão viva e tão agradável.

\section*{Capítulo XLI}

 Pus em seguida minha\textit{ roupa de viagem}, depois de tê-la
examinado com um olhar complacente; e foi então que eu resolvi fazer um
capítulo \textit{ad hoc}, para fazer o leitor conhecê-la. Sendo a forma
e a utilidade destas roupas tão conhecidas de todos, eu trataria mais
particularmente de sua influência sobre o espírito dos viajantes. --
Minha roupa de viagem para o inverno é feita do tecido mais quente e
mais macio que me foi possível encontrar; ela me envolve inteiramente
da cabeça aos pés; e, quando estou na minha poltrona, as mãos nos
bolsos e a cabeça afundada na gola da roupa, pareço com a estátua sem
pés e sem mãos de Vishnu, que encontramos nos pagodes das
Índias. 

 Tacharemos, se quisermos, de prejulgamento, a influência que atribuo às
roupas de viagem sobre os viajantes; o que posso dizer de certo quanto
a isso é que me parece tão ridículo seguir um só passo em minha viagem
ao redor do meu quarto vestido com meu uniforme e com a espada ao lado
quanto sair pelo mundo de roupão. -- Se me visse vestido assim,
seguindo todos os rigores da pragmática, não somente não estaria apto a
continuar minha viagem, mas acredito que também não estaria mesmo em
condições de ler o que escrevi até agora sobre ela, e menos ainda de
compreendê-lo.

 Mas isso o espanta? Não vemos todos os dias pessoas que se acreditam
doentes porque têm a barba longa, ou porque alguém lembra de enxergar
neles um ar doente e de dizer-lhes? As roupas têm tanta influência
sobre o espírito dos homens que há valetudinários que se consideram
melhores quando se veem de roupa nova e peruca empoada: enganam assim o
público e a si mesmos com uma toalete bem cuidada; -- morrem de
manhãzinha, bem penteados, e sua morte choca todo mundo.

 Às vezes esquecemos de avisar alguns dias antes o conde de\ldots\ que ele
deveria montar guarda: um caporal iria se levantar tarde, no mesmo dia
em que ele deveria ficar de guarda, e anunciar-lhe esta triste notícia;
mas a ideia de levantar-se em seguida, de colocar suas grevas e sair
assim, sem ter pensado nisso de véspera, o incomodava de tal maneira
que ele preferia mandar dizer que estava doente e que não sairia de
casa. Vestia então seu roupão e mandava o barbeiro retirar-se; isso lhe
dava um ar pálido, doente, que alarmava sua mulher e toda sua família.
Ele era na verdade ele mesmo, \textit{ligeiramente mal arrumado} nesse dia. 

 Contava a todo mundo, um pouco para sustentar o caso, um pouco também
porque ele acreditava estar assim. -- Insensivelmente o roupão operava
sua influência: os caldos que ele tinha tomado, para bem ou para mal,
lhe causaram náuseas; logo os parentes e amigos mandavam saber
notícias: não precisava tanto para colocá-lo decididamente na cama. 

 À noite, o doutor Ranson\footnote{ Médico muito conhecido em Turim
quando este livro foi escrito. [N. do A.]} encontrava seu pulso
\textit{concentrado}, e ordenava a sangria para o dia seguinte. Se o
serviço durasse um mês a mais, o doente estaria perdido.

 Quem poderia duvidar da influência das roupas de viagem sobre os
viajantes, quando consideramos que o pobre conde de\ldots\ pensou mais de
uma vez estar fazendo a viagem para o outro mundo por ter
inoportunamente colocado, neste aqui, seu roupão?

\section*{Capítulo XLII}

 Eu estava sentado perto de minha lareira, depois do jantar, dobrado em
minha \textit{roupa de viagem} e abandonado voluntariamente a toda sua
influência, esperando a hora da partida, quando os vapores da digestão,
enviados ao meu cérebro, obstruíram de tal modo as passagens pelas
quais as ideias se movimentam quando vêm dos sentidos, que toda
comunicação se viu interceptada; da mesma maneira como meus sentidos
não transmitiam mais nenhuma ideia ao meu cérebro, este, por sua vez,
não podia mais enviar o fluido elétrico que os anima e com o qual o
engenhoso doutor Valli\footnote{ Eusebio Valli (1755--1816). Médico
italiano que estudou bioeletricidade.} ressuscita as rãs mortas. 

Compreende-se facilmente, depois de ter lido este preâmbulo, porque
minha cabeça caía sobre meu peito, e como meus músculos do polegar e do
indicador de minha mão direita, não mais ativados por esse fluido, se
relaxaram até o ponto em que um volume das obras do marquês de
Caraccioli,\footnote{ Louis-Antoine Caraccioli (1719--1803). Escritor,
poeta e historiador francês.} que eu segurava entre esses dois
dedos, escapou sem que eu percebesse e caiu na lareira. 

Acabara de receber visitas, e minha conversação com as pessoas que
tinham saído tinha tratado da morte do famoso médico Cigna,\footnote{ Gian 
Francesco Cigna (1734--1791). Médico italiano, fundador da Academia
de Ciências de Turim.} que tinha acabado de morrer, e que era
universalmente lamentado: era sábio, trabalhador, bom médico e famoso
botânico. O mérito deste homem hábil ocupava meu pensamento.
Entretanto, eu me dizia, se me fosse permitido invocar as almas de
todos aqueles que ele pode ter feito passar para o outro mundo: quem
sabe se sua reputação não sofreria algum revés?

Eu me encaminhava insensivelmente para uma dissertação sobre a medicina
e sobre o progresso que ela fez desde Hipócrates. -- Perguntava-me
como seria se as personagens famosas da antiguidade que morreram em seu
leito, como Péricles, Platão, a célebre Aspásia\footnote{ Aspásia de
Mileto (440 a.C.). Ateniense respeitada que participava ativamente de um
círculo intelectual. Foi amante de Péricles, com quem teve um
filho.} e o próprio Hipócrates, mortos como pessoas comuns, de uma
febre pútrida, inflamatória e verminosa, como seria se os tivéssemos
sangrado e enchido de remédios. 

Não será possível dizer por que eu sonhava com essas quatro personagens
e não com outras. -- Quem pode explicar um sonho? -- Tudo que posso
dizer é que foi minha alma que evocou o doutor de Kos,\footnote{ Ilha
onde Hipócrates fundou a sua escola de medicina.} o de Turim e o
famoso homem de estado que cometeu tantas belas coisas e tantas grandes faltas.

Mas, quanto a sua elegante amiga, asseguro humildemente que foi a
\textit{outra} que lhe fez um sinal. -- Entretanto, quando penso
nisso, fico tentado a experimentar um pequeno movimento de orgulho;
porque está claro que, neste sonho, a balança em favor da razão estava
quatro a um. É muito para um militar da minha idade. 

 De qualquer forma, enquanto me entregava a essas reflexões, meus olhos
conseguiram fechar-se e adormeci profundamente; mas, fechando os olhos,
a imagem das personagens nos quais tinha pensado permaneceu pintada
sobre essa tela fina que chamamos \textit{memória} e estas imagens,
misturando-se em meu cérebro com a ideia da evocação dos mortos, me
fizeram chegar à fila de Hipócrates, Platão, Péricles, Aspásia e o
doutor Cigna com sua peruca.

 Vi-os todos sentarem-se nos assentos ainda arrumados em torno do fogo;
apenas Péricles manteve-se de pé para ler os jornais. 

 -- Se as descobertas de que o senhor me fala são verdadeiras -- dizia
Hipócrates ao doutor --, e se elas tivessem sido tão úteis à medicina como
o senhor pretende, teria visto diminuir o número de homens que desce
cada dia ao reino sombrio, cuja lista comum, segundo os registros de
Minos, que eu mesmo verifiquei, é constantemente a mesma de
antigamente. 

 O doutor Cigna virou-se para mim: 

-- O senhor tem sem dúvida ouvido falar destas descobertas? -- ele me
disse --; conhece a de Harvey sobre a circulação do sangue; a do imortal
Spallanzani sobre a digestão, da qual conhecemos agora todo o
mecanismo?

E deu um longo detalhamento de todas as descobertas concernentes à
medicina, e de toda a quantidade de remédios que devemos à química; fez
um discurso acadêmico em favor da medicina moderna.

 -- Eu deveria acreditar -- respondi então --, que estes grandes
homens ignoram tudo que o senhor acaba de lhes dizer e que sua alma,
separada dos entraves da matéria, encontra algo de obscuro em toda a
natureza?

 -- Ah! que erro! -- gritava o \textit{protomédico}\footnote{ Título
muito conhecido na legislação do rei da Sardenha, o que gera aqui uma
brincadeira de caráter puramente local. [N. do A.]} do Peloponeso -- os
mistérios da natureza estão escondidos tanto dos mortos como dos vivos;
aquele que criou e dirige tudo é o único que sabe o grande segredo que
os homens em vão se esforçam para descobrir: aí está o que aprendemos
de seguro nas margens do Styx; e, creia-me -- ele ainda disse
dirigindo a palavra ao doutor --, livre-se deste resto de espírito
corporativo que o senhor trouxe para a morada dos mortais, já que os
trabalhos de mil gerações e todas as descobertas dos homens não puderam
alongar em um só instante sua existência, já que Caronte passa cada dia
em seu barco a mesma quantidade de sombras, não nos fatiguemos mais
defendendo uma arte que, entre os mortos onde estamos, não seria útil
nem mesmo aos médicos.

 Assim falou o famoso Hipócrates, para meu grande espanto.

 O doutor Cigna sorriu; e, como os espíritos não saberiam recusar as
evidências nem calar a verdade, não só ele concordou com Hipócrates mas
afirmou ele próprio, enrubescendo como as grandes inteligências, que
ele sempre tinha duvidado. 

 Péricles, que tinha se aproximado da janela, soltou um grande suspiro
de que adivinhei a causa. Ele lia um número do \textit{Moniteur}, que       %Iuri sugere que criemos nota para "Moniteur". 
anunciava a decadência das artes e das ciências; via sábios ilustres
deixar suas sublimes especulações para inventar novos crimes; e
estremecia ao ouvir uma horda de canibais comparar-se aos heróis da
generosa Grécia, fazendo perecer sobre o cadafalso, sem vergonha e sem
remorsos, idosos veneráveis, mulheres, crianças, e cometendo a
sangue-frio os crimes mais atrozes e mais inúteis.

 Platão, que tinha escutado nossa conversa sem dizer nada, vendo-a
terminada de repente de uma forma inesperada, tomou por sua vez a
palavra.

 -- Eu compreendo -- ele nos disse --, como as descobertas, que seus
grandes homens em todos os ramos das ciências médicas fizeram, são
inúteis para a medicina, que só poderia mudar o curso da natureza às
custas da vida dos homens; mas as pesquisas feitas na política tiveram
resultados diferentes. As descobertas de Locke sobre a natureza do
espírito humano, a invenção da imprensa, as observações feitas e
acumuladas pela história, tantos livros profundos que espalharam a
ciência até entre o povo; -- tantas maravilhas enfim terão sem dúvida
contribuído para tornar os homens melhores e a república feliz e sábia
que eu tinha imaginado, e que o século em que vivia me fez olhar como
um sonho impraticável, existe, sem dúvida, hoje no mundo?

 A esta pergunta, o honesto doutor baixou os olhos e só respondeu com
lágrimas; depois, como as enxugava com seu lenço, fez sem querer sua
peruca virar, de forma que uma parte de seu rosto ficou escondida.

 -- Deuses imortais -- disse Aspásia emitindo um grito cortante --,
que estranha figura! É então uma descoberta de seus grandes homens que
lhes deu a ideia de se arrumar assim com o crânio de uma outra pessoa?

 Aspásia, que as dissertações dos filósofos faziam bocejar, tinha pegado
um jornal de moda que estava sobre a lareira e o folheava há algum
tempo, quando a peruca do médico a fez fazer essa exclamação; e, como o
assento estreito e inseguro sobre o qual ela estava sentada era muito
incômodo para ela, tinha colocado, sem modos, suas duas pernas nuas,
ornadas de faixas, sobre a cadeira de palha que se encontrava entre ela
e eu, apoiando seu cotovelo sobre um dos largos ombros de Platão.

 -- Não é um crânio -- o doutor lhe respondeu pegando sua peruca e
jogando-a no fogo --, é uma peruca, \textit{mademoiselle}, e eu não
sei por que não joguei este ornamento ridículo nas chamas do Tártaro
quando cheguei até vocês: mas os ridículos e os preconceitos são tão
fortemente inerentes a nossa miserável natureza que eles nos seguem
ainda algum tempo além do túmulo.

 Eu sentia um prazer singular ao ver o doutor abjurar assim ao mesmo
tempo a medicina e sua peruca.

 -- Eu lhe asseguro -- lhe disse Aspásia --, que a maior parte dos
penteados que estão apresentados no caderno que folheio mereciam a
mesma sorte que o seu, de tal maneira são extravagantes!

 A bela ateniense se divertia muito percorrendo aquelas ilustrações, e
se espantava com razão com a variedade e a bizarria dos costumes
modernos. Uma figura entre outras a chocou: a de uma jovem dama,
representada com um penteado dos mais elegantes, e que Aspásia
considerou apenas um pouco alto; mas a peça de gaze que cobria a
garganta era de uma amplidão tão extraordinária que se via apenas a
metade do rosto\ldots\ Aspásia, não sabendo que essas formas prodigiosas
eram só obra de goma, não pode se impedir de evidenciar um assombro que
teria sido redobrado no sentido inverso se a gaze fosse transparente.

 -- Mas, explique-nos -- ela disse --, por que as mulheres de hoje
parecem mais ter roupas para esconder-se que para vestir-se: deixam ver
quando muito seu rosto, com o que, apenas, podemos reconhecer seu sexo,
de tal maneira as formas dos seus corpos estão desfiguradas pelas
dobras bizarras dos tecidos! De todas as figuras que estão
representadas nessas folhas, nenhuma deixa a descoberto a garganta, os
braços e as pernas: como seus jovens guerreiros não ficam tentados a
destruir tal costume? Aparentemente -- ela acrescentou --, a virtude
das mulheres de hoje, que se mostra em todas as suas vestimentas,
ultrapassa muita a de minhas contemporâneas?

 Terminando essas palavras, Aspásia me olhou e parecia me pedir uma
resposta. Fingi não perceber; e, para me dar um ar de distração,
empurrava sobre a brasa, com a tenaz, os restos da peruca do doutor,
que tinham escapado do incêndio.  Percebendo em seguida que uma das
faixas que apertavam a sandália de Aspásia estava desamarrada:

-- Permita-me -- eu lhe disse --, formosa dama. -- E, falando
assim, abaixei-me vivamente, levando minhas mãos para a cadeira onde eu
acreditava ver suas duas pernas que em outros tempos fizeram grandes
filósofos delirar.

 Estou persuadido de que, nesse momento, chegava ao verdadeiro
sonambulismo, porque esse movimento de que falo foi muito real; mas
Rosine, que de fato repousava na cadeira, tomou esse movimento como
para si e, saltando agilmente em meus braços, mergulhou novamente nos
infernos as sombras famosas evocadas pela minha roupa de viagem.

 Encantador país da imaginação, tu que o Ser benfeitor por excelência
entregou aos homens para consolá-los da realidade, é preciso que eu te
deixe.

 É hoje que certas pessoas, das quais dependo, pretendem me devolver a
liberdade, como se a tivessem tirado de mim! Como se estivesse em seu
poder usurpá-la de mim um só instante e me impedir de percorrer ao meu
bel-prazer o vasto espaço sempre aberto a minha frente! Impediram-me de
percorrer uma cidade, um ponto; mas me deixaram o universo inteiro: a
imensidade e a eternidade estão a minha disposição.

 É hoje então que fico livre, ou, antes, que vou voltar para os ferros!
O jugo dos afazeres vai de novo pesar sobre mim; não farei mais um
passo que não seja medido pela conveniência e pelo dever. Feliz
ainda se alguma deusa caprichosa não me fizer esquecer um e outro e se
eu escapar deste novo e perigoso cativeiro.

 Ah! por que não me deixariam terminar minha viagem! Seria para me punir
que me confinaram em meu quarto? Nesta região deliciosa, que guarda
todos os bens e todas as riquezas do mundo? Seria o mesmo que exilar um
camundongo em um sótão.  

 Entretanto jamais havia percebido tão claramente que sou
\textit{duplo}. Enquanto lamento meus prazeres imaginários sinto-me
consolado à força: um poder secreto me conduz; ele me diz que necessito
do ar e do céu e que a solidão se parece com a morte. Eis-me
paramentado; minha porta se abre; erro pelos largos pórticos da rua Pó;
mil fantasmas agradáveis esvoaçam diante dos meus olhos. Sim, aí está
este hotel, esta porta,  esta escada; estremeço por \mbox{antecipação.}
\pagebreak

É assim que sentimos antes o gosto ácido quando cortamos um limão para
comer. 

Ó minha besta, minha pobre besta, cuida de ti!
\ \\

\hfil\textsc{fim da viagem em volta do meu quarto}

\chapter[Expedição noturna em volta do meu quarto]{Expedição noturna em\break volta do meu quarto}

\section*{Capítulo I}

Para gerar algum interesse sobre o novo quarto em que faço uma expedição
noturna, devo explicar aos curiosos como ele chegou a minhas mãos.
Continuamente distraído de minhas ocupações na casa barulhenta em que
morava, considerava já há muito procurar na vizinhança um retiro mais
solitário até que um dia, percorrendo uma notícia biográfica sobre M.
de Buffon, li que esse homem célebre tinha escolhido em seus jardins um
pavilhão isolado, que não continha nenhum outro móvel além de uma
poltrona e a escrivaninha na qual ele trabalhava, nem nenhuma outra
obra além do manuscrito em que se empenhava. 

As quimeras de que me ocupo oferecem tantas diferenças com os trabalhos
imortais de M. de Buffon que o pensamento de imitá-lo, mesmo nesse
ponto, não me teria sem dúvida jamais vindo ao espírito sem um
incidente que o determinou. Um criado, tirando a poeira dos móveis,
acreditou ver muita em uma tela pintada em pastel que eu tinha acabado
de terminar, e o enxugou tão bem com um pano que conseguiu, de fato,
desembaraçá-lo de toda a poeira que eu tinha arrumado lá com bastante
cuidado. Depois de me ter zangado bastante com esse homem, que estava
ausente, e não ter lhe dito nada quando voltou, seguindo o meu costume,
coloquei-me em campanha e voltei para casa com a chave de um quartinho
que tinha alugado num quinto andar da rua da Providência. Fiz
transportar para lá no mesmo dia os materiais de minhas ocupações
favoritas e comecei em seguida a passar lá a maior parte do meu tempo,
ao abrigo do frêmito doméstico e dos limpadores de telas. Nesse reduto
isolado as horas passavam por mim como se fossem minutos e mais de uma
vez meus sonhos me fizeram esquecer a hora do jantar. 

Ó doce solidão! Conheci os encantos com que embriaga teus amantes.
Infeliz daquele que não pode estar só um dia de sua vida sem sentir o
tormento do tédio e que prefere, se for preciso, conversar antes com
tolos do que consigo mesmo.

Confessarei, todavia, que amo a solidão nas grandes cidades; mas, a
menos que forçado por uma circunstância grave, como uma viagem em volta
de meu quarto, só quero ser eremita pela manhã; à noite gosto de rever
rostos humanos. Os inconvenientes da vida social e aqueles da solidão
se destroem assim mutuamente, e estes dois modos de existência
embelezam um ao outro.

Entretanto, a inconstância e a fatalidade das coisas deste mundo são
tais que a própria vivacidade dos prazeres que eu gozaria em minha nova
morada deveria ter me feito prever quão curta seria sua duração. A
Revolução Francesa, que transbordava para toda parte, acabava de
escalar os Alpes e se precipitava sobre a Itália. Fui arrastado pela
primeira onda até Bolonha: mantive meu eremitério, para o qual fiz
transportar todos os meus móveis esperando por tempos melhores. Há alguns
anos não tinha pátria; descobri certo dia que estava sem emprego.
Depois de passado um ano inteiro vendo homens e coisas de que não mais
gostava, e desejando coisas e homens que não via mais, voltei a Turim.
Era preciso tomar partido. Saí do albergue  \textit{De La Bonne Femme},
onde tinha desembarcado, na intenção de devolver meu quartinho ao
proprietário e me desfazer de meus móveis.

Voltando para meu eremitério experimentei sensações difíceis de
descrever: tudo tinha conservado sua ordem, o que significa a desordem
na qual eu o tinha deixado. Os móveis amontoados contra as paredes
tinham sido postos ao abrigo da poeira pela altura do apartamento, mas
as plumas ainda estavam no tinteiro ressecado e encontrei sobre a mesa
uma carta começada.

``Ainda estou em minha casa'', digo-me com uma verdadeira satisfação. Cada
objeto me lembrava algum acontecimento de minha vida e meu quarto
estava forrado de lembranças. Em lugar de voltar para o albergue, tomei
a resolução de passar a noite no meio de minhas propriedades: mandei
pegarem minha mala e ao mesmo tempo planejei partir no dia seguinte,
sem saudar ou me aconselhar com ninguém, abandonando-me sem reservas à
Providência.

\section*{Capítulo II}

 Enquanto fazia essas reflexões, glorificando-me por um plano de viagem
bem calculado, o tempo passava e meu criado não voltava. Era um homem
que a necessidade me tinha feito requisitar para meu serviço há algumas
semanas, e sobre cuja fidelidade tinha levantado suspeitas. Apenas se
me apresentou a ideia de que ele pudesse ter me levado a mala e corri
ao albergue: foi na hora. Quando virava a esquina da rua onde se
encontrava o albergue \textit{De La Bonne Femme}, eu o vi sair
precipitadamente pela porta precedido de um camareiro encarregado de
minha mala. Ele próprio tinha se encarregado de meu baú e, em vez de
voltar-se para o meu lado ele se encaminhava para a esquerda numa
direção oposta à que devia tomar. Sua intenção tornava-se manifesta.
Segui-o com cuidado e, sem dizer nada, caminhava já há algum tempo
junto dele antes que percebesse. Se quiséssemos pintar a expressão de
susto e temor com sua maior intensidade no rosto humano ele teria sido
o modelo perfeito quando me viu a seu lado. Tive todo o tempo para
fazer o estudo porque ele estava tão desconcertado pela minha aparição
inesperada e com a seriedade com que eu o olhava que continuou a andar
algum tempo comigo sem proferir uma palavra, como se estivéssemos
passeando juntos. Enfim balbuciou o pretexto de um compromisso na rua
Gran’Doire; mas eu o recoloquei no caminho correto e voltamos ao
apartamento, onde eu o despedi.

 Foi só então que me propus a fazer uma nova viagem pelo meu quarto,
durante a última noite que deveria passar ali, e comecei no mesmo
instante a me ocupar com os preparativos. 

\section*{Capítulo III}

 Há muito tempo eu desejava rever a região que tinha em outro momento
percorrido tão deliciosamente, e cuja descrição não me parecia
completa. Alguns amigos que o tinham experimentado pediam que eu a
continuasse e eu teria me decidido sem dúvida antes se não tivesse sido
separado dos meus companheiros de viagem. Voltava de mau grado para a
estrada. Ai de mim! Voltava sozinho. Iria viajar sem meu querido
Joannetti e sem a amável Rosine. Mesmo o meu primeiro quarto havia
sofrido a mais desastrada revolução, que digo! Ele não existia mais.
Seus muros faziam agora parte de um horrível barraco enegrecido pelo
fogo, e todas as invenções mortíferas da guerra tinham se reunido para
destruí-lo de alto a baixo.\footnote{ Este quarto situava-se na cidade
de Turim, e esta nova viagem foi realizada algum tempo depois da tomada
desse lugar pelos Austro-russos. [N. do A.]  A ocupação ocorreu em 26 de maio
de 1799. [N. da T.]} A parede em que estava pendurado o quadro de \textit{Madame} de
Hautcastel tinha sido aberta por uma bomba. Enfim, se eu não tivesse,
felizmente, viajado antes dessa catástrofe, os sábios dos nossos dias
jamais teriam tomado conhecimento deste quarto notável. É assim que,
sem as observações de Hiparco,\footnote{ Astrônomo grego da escola de
Alexandria, séc. \textsc{ii} a.C.} hoje ignoraríamos que havia uma estrela a
mais entre as Plêiades, que desapareceu depois desse famoso astrônomo. 

 Forçado pelas circunstâncias eu tinha já há algum tempo abandonado meu
quarto e transportado meus pertences para um outro lugar. A
infelicidade não é grande, pode-se dizer. Mas como substituir Joannetti
e Rosine? Ah! isso não é possível. Joannetti tinha se tornado para mim tão
necessário que a sua perda não será jamais reparada. Quem
pode, de resto, esperar viver sempre com as pessoas que lhe são
queridas? Como estes enxames de moscas que vemos rodopiar pelos ares
durante as belas noites de verão, os homens se encontram por acaso e
por bem pouco tempo. Felizes ainda se, no seu movimento rápido, tão
habilmente quanto as moscas, não batem as cabeças umas contra as outras.


 Uma certa noite estava para me deitar. Joannetti me servia com seu zelo
de sempre e parecia mesmo mais atento. Quando trouxe a luz olhei para
ele e vi uma alteração marcando sua fisionomia. Deveria, contudo, ter
desconfiado que o pobre Joannetti me servia pela última vez? Não
manteria o leitor com uma incerteza mais cruel que a verdade. Prefiro
dizer sem meias palavras que Joannetti casou-se naquela mesma noite e
me deixou no dia seguinte.

 Mas que ele não seja acusado de ingratidão por ter deixado seu patrão
tão bruscamente. Eu sabia das suas intenções há muito tempo e seria
errado ter me oposto a isso. Numa manhã especial um oficial veio a
minha casa para me dar a notícia e tive tempo, antes de rever
Joannetti, de me enraivecer e de me acalmar, o que lhe economizou as
reprimendas que esperava. Antes de entrar no meu quarto ele afetou uma
palavra em voz alta para alguém desde o corredor, para me fazer
acreditar que ele não tinha medo e, armando-se de toda a insolência que
poderia penetrar em uma boa alma como a sua, apresentou-se com um ar
determinado. Vi no seu rosto, naquele momento, tudo que se passava em
sua alma e não fiquei aborrecido com ele. Os impertinentes de hoje em
dia de tal maneira assustaram as boas pessoas sobre os perigos do
casamento que um recém-casado parece com frequência um homem que acaba
de sofrer uma queda assustadora sem se machucar e que está ao mesmo
tempo assustado de medo e satisfação, o que lhe dá um ar ridículo. Não
era, portanto, de estranhar que as ações do meu fiel servidor se
ressentissem da bizarria de sua situação. 

 -- Então está casado, meu querido Joannetti? -- disse-lhe rindo.

 Ele só tinha se preparado para a minha cólera de maneira que todas as
suas precauções foram em vão. Voltou repentinamente a sua conduta usual
e mesmo um pouco mais sensível, pois começou a chorar. 

 -- O que o senhor quer! -- ele me disse com uma voz alterada -- Eu
tinha dado a minha palavra!

 -- Ah! Por Deus! Fizeste bem, meu amigo! Que você possa ser feliz com sua
mulher e com você mesmo! Que você possa ter filhos que se pareçam com você! É
preciso então que nos separemos? 

 -- Sim, senhor; contamos nos estabelecer em Asti.

 Aqui Joannetti baixou os olhos com um ar embaraçado, e respondeu com os
dois tons mais baixos:

 -- Minha mulher encontrou um condutor da sua região que está voltando
com a charrete vazia e que parte hoje. Seria uma ótima ocasião; mas\ldots
entretanto\ldots\ será quando aprouver ao senhor\ldots\ ainda que uma ocasião
como essa não se encontre facilmente.

 -- E então! Tão rápido? -- eu lhe disse. 

Um sentimento de saudade e afeição, misturado com uma forte dose de
despeito, me fez guardar um instante de silêncio. 

-- Não, certamente -- eu lhe respondi bem duramente -- eu não vou retê-lo
mais; parte agora mesmo, se assim está bom. 
Joannetti empalideceu. -- Sim, parte, meu amigo, vai encontrar tua mulher; 
seja sempre tão bom, tão honesto como foste comigo.

 Fizemos alguns arranjos, disse-lhe adeus tristemente: ele partiu.

 Esse homem me tinha servido por quinze anos. Um instante nos separou.
Não o revi mais.

 Refletia, passeando pelo meu quarto, sobre esta brusca separação.
Rosine foi com Joannetti sem que ele percebesse. Um quarto de hora mais
tarde a porta se abriu, Rosine entrou. Vi a mão de Joannetti que a
empurrava para dentro do quarto, a porta se fechou de novo e senti meu
coração apertar\ldots\  Ele já não entra em minha casa! Alguns minutos
foram suficientes para fazer estranhos um para o outro dois velhos
companheiros de quinze anos. Oh triste, triste condição da humanidade,
não poder jamais encontrar um objeto estável sobre o qual colocar a
menor das suas afeições!

\section*{Capítulo IV}

 Rosine também então vivia longe de mim. Sabes sem dúvida, com algum
interesse, minha querida Marie,\footnote{ Mais adiante no texto a
``interlocutora'' será Sophie.} que com quinze anos ela era ainda o mais
amável dos animais, e que a mesma inteligência superior, que a
distinguia outrora de toda sua espécie, serviu-lhe igualmente para
suportar o peso de sua velhice. Teria preferido não ter me separado
dela, mas quando se trata do bem estar dos amigos, não devemos
consultar apenas o seu prazer ou interesse? O interesse de Rosine era
deixar a vida ambulante que ela levava comigo e gozar enfim em seus
velhos dias de um repouso que seu mestre não esperava mais. Sua idade
avançada me obrigava a mandar carregá-la. Acreditei que deveria lhe
ceder a reserva.\footnote{ O autor, como militar, brinca aqui com a
``aposentadoria'' da sua cachorrinha.} Uma religiosa benfeitora se
encarregou de cuidar dela pelo resto de seus dias, e eu sei que nesse
retiro ela aproveitou de todas as vantagens que suas boas qualidades,
sua idade e sua reputação a tinham tão justamente feito merecer. 

 E é tal, pois, a natureza dos homens que a felicidade parece não ser
feita para eles. O amigo ofende seu amigo sem querer, e mesmo os
amantes não vivem sem querelas; enfim, já que, desde Licurgo até nossos
dias, todos os legisladores falharam em seus esforços para fazer os
homens felizes, terei ao menos o consolo de ter feito a felicidade de
um cachorro.

\section*{Capítulo V}

 Agora que já dei a conhecer ao leitor os últimos detalhes da história
de Joannetti e de Rosine, só me resta dizer uma palavra sobre a alma e
a besta para estarmos perfeitamente em regra com ele. Essas duas
personagens, sobretudo a última, não representarão mais um papel tão
interessante na minha viagem. Um amável viajante, que seguiu a mesma
carreira que eu,\footnote{ \textit{Segunda viagem em volta de meu
quarto}, por um anônimo; capítulo \textsc{i}. [N. do A.]} considerou que elas devem
estar fatigadas. Ai de mim! Ele tem muita razão. Não é que minha alma
tenha perdido algo de sua atividade, ao menos tanto que ela possa
perceber; mas suas relações com \textit{a outra} mudaram. Esta não tem
mais a mesma vivacidade nas suas réplicas, ela não tem mais\ldots\ como
explicar isso?\ldots\ Eu ia dizer a mesma presença de espírito, como se uma
besta pudesse ter disso!

 De qualquer modo, e sem entrar em uma explicação embaraçosa, diria
somente que, levado pela confiança que me testemunhava a jovem
Alexandrine, eu lhe tinha escrito uma carta bastante terna, quando
recebi uma resposta educada, mas fria, e que terminava, em seus
próprios termos, com: ``Esteja certo, \textit{monsieur}, que conservarei
sempre pelo senhor os sentimentos da mais sincera estima.'' -- ``Deus do
céu!'' -- considerei em seguida --, ``estou perdido.'' Desde esse dia fatal
resolvi não mais propor meu sistema da alma e da besta. 

 Em consequência disso, sem fazer distinção entre esses dois seres e sem
separá-los, eu os passarei adiante juntos, como certos mercadores suas
mercadorias, e viajarei em bloco para evitar qualquer inconveniente. 

\section*{Capítulo VI}

 Seria inútil falar das dimensões do meu novo quarto. Ele se parece
tanto com o primeiro que até poderíamos nos enganar numa primeira olhada,                    %T: No lugar de "poderíamos" havia "teríamos". Alterei pois não fazia sentido. 
se, por uma precaução do arquiteto, o teto não se inclinasse
obliquamente do lado da rua e não deixasse o telhado na direção que
exigem as leis da hidráulica para o escoamento da chuva. Recebe a luz
do dia por apenas uma abertura de dois pés e meio de largura por quatro
pés de altura, elevada em torno de seis ou sete pés acima do chão, e a
que se chega por meio de uma escadinha.

 A elevação de minha janela acima do chão é uma dessas circunstâncias
felizes que podem ser igualmente devidas ao acaso ou ao gênio do
arquiteto. A luz quase perpendicular que ela espalhava em meu reduto
lhe dava um aspecto misterioso. O antigo templo do Panteão recebe o dia
mais ou menos da mesma maneira. Além disso, nenhum objeto exterior
poderia me distrair. Como esses navegadores que, perdidos sobre o vasto
oceano, não veem mais que o céu e o mar, eu só via o céu e meu quarto,
e os objetos exteriores mais próximos sobre os quais meus olhares
poderiam se colocar eram a lua ou a estrela da manhã: o que me colocava
em relação imediata com o céu e dava a meus pensamentos um voo elevado
que eles não teriam jamais feito se eu tivesse escolhido meu alojamento
ao nível do chão.

 A janela de que falei elevava-se telhado acima e formava uma belíssima
lucarna. Sua altura sobre o horizonte era tão grande que quando os
primeiros raios de sol vinham clareá-la a rua ainda estava em sombras.
Assim eu gozava de uma das mais belas vistas que se possa imaginar. Mas
a mais bela vista pode logo nos fatigar quando a vemos sempre; o olho
se habitua e não fazemos mais caso dela. A situação de minha janela me
preservava ainda desse inconveniente, porque jamais via o espetáculo
magnífico do campo de Turim sem subir quatro ou cinco degraus, o que me
proporcionava alegrias sempre vivas, pois eram moderadas. Quando,
cansado, queria me dar uma agradável distração, acabava meu dia subindo
a minha janela.

 No primeiro degrau ainda não via nada além do céu; logo o templo
colossal de La Superga\footnote{ Igreja magnífica construída pelo rei
Vittorio Amedeo \textsc{ii}, em 1706, para o cumprimento de uma promessa que ele
havia feito à Virgem, se os Franceses levantassem o cerco de Turim. La
Superga serve de sepultura aos príncipes da casa de Savóia. [N. do A.]}
começava a aparecer. A colina de Turim, sobre a qual ele repousa,
elevava-se pouco a pouco diante de mim, coberta de florestas e ricos
vinhedos, oferecendo com orgulho ao sol poente seus jardins e seus
palácios, enquanto que casas simples e modestas pareciam meio
escondidas nos vales, para servir de refúgio ao sábio e favorecer suas
meditações.

 Colina encantadora! Muitas vezes me viste procurar teus refúgios
solitários e preferir teus atalhos afastados dos passeios brilhantes da
capital; viste-me com frequência perdido nos teus labirintos de
verdura, atento ao canto da cotovia da manhã, o coração cheio de uma
vaga inquietude e do desejo ardente de fixar-me para sempre nos teus
vales encantados. 

-- Eu te saúdo, colina encantada! Estás pintada em meu coração! Possa o
orvalho celeste fazer, se isso é possível, teus campos mais férteis e
teus pequenos bosques mais densos! Possam teus habitantes gozar a paz e
a felicidade e que tuas sombras lhes sejam favoráveis e salutares!
Possa enfim tua terra feliz ser sempre o doce asilo da verdadeira
filosofia, da modesta ciência, da amizade sincera e hospitaleira que aí
encontrei!

\section*{Capítulo VII}

 Eu comecei minha viagem precisamente às oito horas da noite. O tempo
estava calmo e prometia uma bela noite. Tinha tomado minhas precauções
para não ser incomodado por visitas, que são muito raras nas alturas em
que morava, sobretudo nas circunstâncias em que então me encontrava, e
para ficar sozinho até a meia-noite. Quatro horas seriam bastante
suficientes para a execução da minha empreitada, não desejando fazer
dessa vez mais que uma simples excursão em volta do meu quarto. Se a
primeira viagem durou quarenta e dois dias é porque não pude decidir
fazê-la mais curta. Não quis tampouco sujeitar-me a viajar muito em
carro, como antes, persuadido de que um viajante a pé vê muitas coisas
que escapam àquele que se move muito rápido. Resolvi então ir,
alternadamente e seguindo as circunstâncias, a pé ou a cavalo: novo
método que ainda não dei a conhecer e de que veremos logo a utilidade.
Enfim, propus-me a tomar notas no caminho e descrever minhas
observações à medida que as fizesse, para não esquecer nada.

 A fim de pôr ordem em minha empreitada, e de dar-lhe uma nova chance de
sucesso, achei que era preciso começar por compor uma epístola
dedicatória e escrevê-la em verso para torná-la mais interessante. Mas
duas dificuldades me embaraçaram e me fizeram renunciar, apesar de
todas as vantagens que podia tirar disso. A primeira era saber a quem
endereçaria a epístola, a segunda como iria me virar para fazer versos.
Depois de ter refletido maduramente não tardei a entender que era
razoável fazer primeiro a minha epístola o melhor que pudesse e
procurar em seguida alguém a quem ela pudesse convir. Pus mãos à obra
na mesma hora e trabalhei por mais de uma hora sem conseguir encontrar
uma rima para o primeiro verso que tinha feito e que queria manter,
porque me parecia muito feliz. Lembrei-me então muito a propósito de
ter lido em algum lugar que o célebre Pope jamais compunha nada
interessante sem ser obrigado a declamar por muito tempo em voz alta e
de agitar-se em todas as direções em seu gabinete para excitar sua
verve. Tentei imediatamente imitá-lo\ldots\ Peguei os poemas de Ossian e os
recitei em voz alta, passeando em grandes passos para me dar
entusiasmo. 

 Vi de fato que este método exaltava pouco a pouco minha imaginação e me
dava um sentimento secreto de capacidade poética de que certamente
teria me aproveitado para compor com sucesso minha epístola dedicatória em
verso se, infelizmente, não tivesse esquecido a obliquidade do teto do
meu quarto, cujo rebaixamento rápido impediu minha testa de ir para
frente tanto quanto meus pés na direção que tinha tomado. Bati com
tanta força a cabeça contra essa maldita barreira que o telhado da casa
foi sacudido: os pardais que dormiam sobre as telhas voaram assustados
e o contragolpe me fez recuar três passos para trás.

\section*{Capítulo VIII}

 Enquanto passeava assim para exercitar minha verve, uma jovem e bonita
mulher que morava embaixo de mim, assustada com a barulheira que eu
fazia, e acreditando talvez que eu oferecesse um baile em meu quarto,
enviou seu marido para se informar da causa do barulho. Eu estava ainda
completamente aturdido pela contusão que tinha sofrido quando a porta
se abriu. Um homem de idade, com um rosto melancólico, esticou a cabeça
e passeou o olhar curioso em meu quarto. Quando a surpresa de me
encontrar sozinho lhe permitiu falar\ldots

-- Minha mulher está com enxaqueca, senhor -- disse-me com um ar
descontente. -- Permita-me observar que\ldots

Eu logo o interrompi e meu estilo se ressentiu da altura dos meus
pensamentos. 

-- Respeitável mensageiro de minha bela vizinha -- disse-lhe na
linguagem dos bardos --, por que teus olhos brilham sob tuas espessas
sobrancelhas, como dois meteoros na floresta negra de Cromb? Tua bela
companheira é um raio de luz e eu morreria mil vezes mais cedo antes de
querer incomodar seu repouso; mas teu aspecto, ó respeitável
mensageiro, teu aspecto é sombrio como a abóbada mais escondida da
caverna de Camora, quando as nuvens reunidas da tempestade obscurecem a
face da noite e pesam sobre os campos silenciosos de Morven.\footnote{ Assim 
como Cromb e Camora, referências lendárias escocesas presentes
nos poemas de Ossian.}

O vizinho, que aparentemente jamais tinha lido os poemas de Ossian,
tomou indevidamente o acesso de entusiasmo que me animava por um acesso
de loucura e pareceu muitíssimo embaraçado. Como minha intenção não
era, de modo algum, ofender, ofereci-lhe uma poltrona e pedi que se
sentasse, mas percebi que ele se retirava cuidadosamente e se benzia
dizendo a meia-voz: 

-- \textit{È matto, per Bacco, è matto!}\footnote{ Está louco, por Baco, está
louco!}

\section*{Capítulo IX}

 Eu o deixei sair sem querer aprofundar até que ponto sua observação era
fundamentada, e sentei-me na escrivaninha para tomar nota desses
acontecimentos, como sempre faço; mas apenas abri uma gaveta na qual
esperava encontrar papel e fechei-a bruscamente, perturbado por um
desses sentimentos dos mais desagradáveis que podemos experimentar, o
do amor-próprio humilhado. O tipo de surpresa de que fui tomado nessa
ocasião se parece àquela que experimenta o viajante alterado quando,
aproximando seus lábios de uma fonte límpida, percebe uma rã que o olha
do fundo da água. Aquilo não era, entretanto, nada mais que o mecanismo
e a carcaça de uma pomba mecânica que, a exemplo de Arquitas,\footnote{
Arquitas de Tarento (428 a.C.--347 a.C.), matemático grego, um dos
pitagóricos mais importantes, influenciou Euclides. Reputa-se que
construiu uma pomba mecânica que teria voado 200m.} tinha me
proposto outrora a fazer voar pelos ares. Tinha trabalhado sem descanso
na sua construção durante mais de três meses. Chegado o dia do teste,
coloquei-a na borda de uma mesa, depois de ter cuidadosamente fechado a
porta, a fim de manter a descoberta secreta e causar uma agradável
surpresa aos meus amigos. Um fio mantinha o mecanismo imóvel. Quem
poderia imaginar as palpitações de meu coração e as angústias de meu
amor-próprio quando aproximei a tesoura para cortar o laço fatal?
\ldots\ Zap!\ldots\ O mecanismo da pomba parte e se desenvolve fazendo barulho.
Levanto os olhos para vê-la passar; mas, depois de ter feito algumas
voltas sobre si mesma ela cai e vai se esconder sobre a mesa. Rosine,
que dormia lá, se distancia tristemente. Rosine, que jamais tinha visto
nem mesmo um frango, nem um pombo, nem o menor pássaro, sem atacá-los e
persegui-los, não se dignou nem mesmo a olhar para minha pomba que se
debatia no chão\ldots\ Foi o golpe de misericórdia para meu amor-próprio.

Fui tomar ar nos arredores.

\section*{Capítulo X}

 Tal foi a sina de minha pomba mecânica. Enquanto o gênio da mecânica a
destinava a seguir a águia nos céus, o destino lhe deu as inclinações
de uma toupeira. 

 Passeava tristemente e desencorajado como sempre ficamos depois de uma
esperança perdida, quando, levantando os olhos, distingui um grupo de
grous que passava sobre minha cabeça. Parei para examiná-los. Avançavam
em uma organização triangular, como a coluna inglesa na batalha de
Fontenoy. Eu os via atravessar o céu de nuvem em nuvem. ``Ah! como voam
bem!'', eu dizia baixinho, ``com que segurança parecem deslizar sobre o
invisível caminho que percorrem!''. Confessarei? Ai de mim! Que me
perdoem! O horrível sentimento da inveja uma vez, uma só vez, entrou em
meu coração, e foi por causa dos grous. Persegui-os com meus olhares
ciumentos até os limites do horizonte. Durante muito tempo, imóvel no
meio da multidão que passeava, observei o movimento rápido das
andorinhas, e me espantei por vê-las suspensas no ar, como se jamais
tivesse visto este fenômeno. O sentimento de uma admiração profunda,
desconhecida para mim até então, iluminava minha alma. Acreditava estar
vendo a natureza pela primeira vez. Ouvia com surpresa o zunir das
moscas, o canto dos pássaros, e este barulho misterioso e confuso da
criação viva que celebra involuntariamente seu autor. Concerto
inefável, ao qual apenas o homem tem o sublime poder de juntar
inflexões de reconhecimento! 

-- Quem é o autor deste mecanismo brilhante? -- gritava para mim mesmo
nos transportes que me animavam. -- Quem é aquele que, abrindo sua mão
criadora, deixou escapar a primeira andorinha para os ares? Aquele que
deu a ordem a essas árvores para sair da terra e estender seus ramos
para o céu? E tu, que avanças majestosamente sob a sombra, criatura
encantadora, cujos traços ordenam o respeito e o amor, quem te colocou
sobre a superfície da terra para embelezá-la? Qual é o pensamento que
desenhou tuas formas divinas, que foi forte o suficiente para criar o
olhar e o sorriso da beleza inocente?\ldots\ E eu mesmo, quem sente
palpitar meu coração, qual é a finalidade da minha existência? O que eu
sou, e de onde venho, eu, o autor da pomba mecânica centrípeta?\ldots

 Apenas tinha pronunciado esta palavra bárbara quando, voltando de
repente a mim como um homem adormecido sobre quem se jogou um balde de
água, percebi que muitas pessoas se tinham posto a minha volta para me
examinarem, enquanto meu entusiasmo me fazia falar sozinho. Vi então a
bela Georgina poucos passos à frente. A metade de sua bochecha
esquerda, carregada de \textit{rouge}, que eu entrevia através dos
cachos de sua peruca loura, acabou de me pôr de novo ao corrente dos
assuntos desse mundo, de que acabava de me ausentar brevemente.

\section*{Capítulo XI}

  Desde que me refiz, um pouco, do abalo que me tinha causado a pomba
mecânica, a dor da contusão que tinha recebera se fez sentir vivamente.
Passei a mão sobre a testa e encontrei uma nova protuberância
precisamente na parte da cabeça onde o doutor Gall localizou a
protuberância poética. Depois de me ter recolhido alguns momentos para
fazer um último esforço em favor de minha epístola dedicatória, peguei
um lápis e pus mãos à obra. Qual foi meu espanto!\ldots\ Os versos corriam
de si mesmos sob minha pluma; preenchi duas páginas em menos de uma
hora e concluí desta circunstância que, se o movimento era necessário à
cabeça de Pope para compor versos, não era necessário mais que uma
contusão para tirá-los da minha. Entretanto, não mostrarei ao leitor o
que fiz depois porque a rapidez prodigiosa com que se sucedem as
aventuras da minha viagem me impediu de colocar a mão neles uma última
vez. Apesar dessa reticência não é de se duvidar que se olhe o acidente
que me aconteceu como uma descoberta preciosa, cujo uso pelos poetas
jamais seria demasiado.

 Estou, na verdade, tão convencido da infalibilidade deste novo método
que, no poema em vinte e quatro cantos que compus desde então, e que
será publicado com \textit{A prisioneira de Pignerol},\footnote{ O autor
parece depois ter renunciado a publicar \textit{La Prisionnière de
Pignerol}, por esta obra se aproximar demais do gênero do romance.
[N. do A.]} não considerei necessário até o presente começar os versos; mas
passei a limpo quinhentas páginas de notas que formam, como se sabe,
todo o mérito e o volume da maior parte dos poemas modernos.

 Como sonhava profundamente com minhas descobertas, andando no meu
quarto encontrei minha cama, sobre a qual caí sentado, e minha mão,
achando-se por acaso caída sobre minha touca: tomei então a atitude de
cobrir a cabeça e de me deitar.

\section*{Capítulo XII}

 Estava na cama há um quarto de hora e, ao contrário do meu comum, ainda
não dormia. À ideia da minha epístola dedicatória tinham sucedido
reflexões mais tristes: minha luz, que estava se apagando, não
projetava mais que uma claridade inconstante e lúgubre do fundo da
arandela e meu quarto tinha o ar de um túmulo. Um vento abriu de
repente a janela, apagou a vela e fechou a porta com violência. A tinta
negra de meus pensamentos cresceu na obscuridade.

 Todos os meus prazeres passados, todas as minhas penas presentes vieram
fundir-se em meu coração e o encheram de tristezas e amarguras.

 Ainda que faça esforços contínuos para esquecer meus sofrimentos e
tirá-los de meu pensamento às vezes acontece, quando não tomo cuidado,
de eles voltarem todos juntos à memória, como se lhes tivessem aberto
uma eclusa. Não me resta outra coisa a fazer nessas ocasiões além de me
abandonar à torrente que me leva e minhas ideias tornam-se então tão
negras, todos os objetos me parecem tão lúgubres que acabo sempre rindo
de minha loucura, de forma que o remédio se encontra na própria
violência do mal.

 Estava ainda em meio a toda a força de uma dessas crises melancólicas
quando uma lufada de vento que tinha aberto minha janela e fechado
minha porta enquanto passava, depois de ter dado algumas outras voltas
em meu quarto, folheado os livros e jogado um caderno da minha viagem
no chão, entrou finalmente nas minhas cortinas e veio morrer na minha
bochecha. Senti a frescura doce da noite e, considerando isso como um
convite de sua parte, levantei-me em seguida e fui para minha escada
gozar da calma da natureza. 

 \section*{Capítulo XIII}

 O tempo estava sereno: a Via Láctea, como uma nuvem leve, dividia o
céu; um doce raio partia de cada estrela para vir até mim e quando
examinava uma delas aplicadamente suas companheiras pareciam cintilar
mais vivamente pra chamar minha atenção.

 É um encanto sempre novo para mim este de contemplar o céu estrelado, e
não tenho que me repreender por ter feito uma só viagem, nem mesmo um
simples passeio noturno, sem pagar o tributo de admiração que devo às
maravilhas do firmamento. Ainda que sinta toda a incapacidade de meu
pensamento nestas altas meditações, encontro um prazer inexprimível em
me ocupar delas. Gosto de pensar que não é o acaso que conduz até meus
olhos esta emanação de mundos distanciados, e cada estrela gera com sua
luz um raio de esperança em meu coração. E o quê! Estas maravilhas não
teriam outra relação comigo além dessa de brilhar para meus olhos? Meu
pensamento que se eleva até elas e meu coração que se emociona por sua
aparência lhes seriam estrangeiros?\ldots\ Espectador efêmero de um
espetáculo eterno, o homem levanta os olhos para o céu por um instante
e os fecha para sempre; mas, durante o instante rápido que lhe é dado,
de todos os pontos do céu e desde a beirada do universo, um raio
consolador parte de cada mundo e vem tocar seu olhar para lhe anunciar
que existe uma relação entre a imensidão e ele, e que ele está
associado à eternidade.

\section*{Capítulo XIV}

 Um sentimento ruim, entretanto, perturbou o prazer que eu experimentava
deixando-me levar por essas meditações. Quão poucas pessoas, eu me
dizia, aproveitam agora comigo do espetáculo sublime que o céu
apresenta inutilmente para os homens adormecidos!\ldots\ Ainda que não
contemos com os que dormem; mas que custaria àqueles que passeiam,
àqueles que saem em massa do teatro, olhar um instante e admirar as
constelações reluzentes que brilham de todas as partes sobre suas
cabeças? -- Não, os espectadores atentos de Escapino ou de
Jocrisse\footnote{ Personagens de comédias teatrais, ambas da tradição
da \textit{commedia dell’arte}.} não se dignam a levantar os olhos:
vão voltar brutalmente para suas casas, ou para outro lugar, sem sonhar
que o céu existe. Que bizarria!\ldots\ porque podemos vê-lo sempre e de
graça eles não o querem. Se o firmamento estivesse sempre coberto para
nós, se o espetáculo que ele nos oferece dependesse de um empresário,
os primeiros lugares sobre os telhados seriam caríssimos e as damas de
Turim disputariam violentamente minha lucarna.

 -- Oh! Se fosse soberano de um país, eu gritava, tomado de uma justa
indignação, faria a cada noite soar o toque do sino e obrigaria meus
súditos de todas as idades, de todos os sexos e todas as condições a se
colocarem à janela para olhar as estrelas. Aqui a Razão, que só tem no
meu reino um direito discutível de advertência, foi entretanto mais
feliz que o comum nas representações que me propôs quanto ao édito
leviano que eu queria proclamar em meus Estados. 

-- Senhor, ela me disse, Vossa Majestade não consideraria fazer uma
exceção em favor das noites chuvosas, já que, nesse caso, o céu estando
coberto\ldots\ 

-- Muito bem, muito bem, respondi, não tinha pensado nisso: anotai uma
exceção em favor das noites chuvosas. 

-- Senhor, acrescentou, acho que seria adequado excluir também as
noites serenas, quando o frio é excessivo e a brisa sopra, já que a
execução rigorosa do édito abateria vossos felizes súditos com
entupimentos e catarros. 

Eu começava a ver muitas dificuldades na execução de meu projeto, mas me
custava desistir das decisões. 

-- Será necessário, eu disse, escrever para o Conselho de Medicina e à
Academia de Ciências para fixar o grau centígrado do termômetro em que
meus súditos poderão ser dispensados de se colocar à janela; mas eu
quero, exijo absolutamente que a ordem seja executada com rigor. 

-- E os doentes, senhor?

-- Não precisa nem dizer, que sejam excluídos: humanidade antes de
tudo. 

-- Se não temesse fatigar Vossa Majestade, faria ainda notar que
poderíamos (no caso que se julgasse adequado e que não apresentasse
grandes inconvenientes) acrescentar também uma exceção em favor dos
cegos, já que, sendo privados do órgão da visão\ldots

-- Pois bem, é tudo? -- eu a interrompi com azedume.

-- Perdão, senhor; mas e os apaixonados? O coração indulgente de Vossa
Majestade poderia constrangê-los a olhar assim as estrelas? 

-- Está bem, está bem -- diz o rei --, posterguemos isso: pensaremos nisso
com a cabeça repousada. O senhor me entregará uma memória detalhada sobre
isso.

Bom Deus! Bom Deus! Quanto é preciso refletir antes de emitir um édito
disciplinar!

\section*{Capítulo XV}

 As estrelas que contemplei com mais prazer jamais foram as mais
brilhantes, mas as menores, aquelas que, perdidas em uma distância
incomensurável, pareciam apenas pontos imperceptíveis, sempre foram
minhas estrelas favoritas. A razão disso é simples: conceberemos
facilmente que fazendo minha imaginação fazer o mesmo caminho por um
lado de seus globos que meus olhos fazem deste aqui para chegar até
elas, encontro-me levado sem esforço a uma distância aonde poucos
viajantes chegaram antes de mim e me espanto, chegando lá, de estar
apenas no começo deste vasto universo: porque seria, acredito, ridículo
pensar que não existe uma barreira além da qual o nada começa, como se
o nada fosse mais fácil de compreender que a existência! Depois da
última estrela imagino ainda uma outra que não saberia também ser a
última. Atribuindo limites à criação, por mais que afastados, o
universo não me parece mais que um ponto luminoso, comparado à
imensidade do espaço vazio que o envolve, neste assustador e sombrio
nada, no meio do qual ele estaria suspenso como uma lâmpada solitária.

 Aqui cobri os olhos com minhas duas mãos, para afastar-me de toda
espécie de distração e dar a minhas ideias a profundidade que tal
assunto exige; e, fazendo um esforço mental sobrenatural, compus um
sistema do mundo, o mais completo que até agora surgiu. Aí está com
todos os seus detalhes; o resultado das meditações de toda uma vida.

``Acredito que o espaço, sendo\ldots''

 Mas isso merece um capítulo à parte; e, dada a importância da matéria,
será o único de minha viagem que terá um título.

\section*{Capítulo XVI}

\textsc{sistema do mundo}

 Acredito que o espaço, sendo infinito, a criação também o é, e que Deus
criou na sua eternidade um infinito de mundos na imensidão do espaço.

\section*{Capítulo XVII}

 Confesso, entretanto, de boa fé, que não compreendo mais meu sistema
que quaisquer outros sistemas desenvolvidos até hoje na imaginação dos
antigos e modernos filósofos; mas o meu tem a vantagem preciosa de
estar contido em quatro linhas,\footnote{ Considere-se, obviamente, o
texto em sua publicação original.} sendo enorme como é. O leitor
indulgente quererá também observar que ele foi composto inteiro no topo
de uma escada. Eu o teria, entretanto, embelezado com comentários e
notas se, no momento em que estava mais intensamente preocupado com meu
assunto não tivesse sido distraído por sons encantadores que vieram
tocar agradavelmente meus ouvidos. Uma tal voz melodiosa como jamais
havia ouvido, sem excetuar até mesmo a de Zénéide, uma dessa vozes que
estão sempre em uníssono com as fibras de meu coração, cantava perto de
mim uma romança de que não perdi uma palavra e que não me saiu mais da
memória. Ouvindo com atenção, descobri que a voz partia de uma janela
mais baixa que a minha: infelizmente não podia vê-la pois a extremidade
do telhado, acima do qual se elevava minha lucarna, a escondia dos meus
olhos. Entretanto, o desejo de descobrir a sereia que me enfeitiçava
com seus acordes aumentava com o encanto da romança, cujas
palavras tocantes teriam arrancado lágrimas do ser mais insensível.
Logo em seguida, não podendo mais resistir a minha curiosidade, subi
até o último degrau, coloquei um pé na borda do telhado e, segurando-me
por uma mão na borda da janela, suspendi-me assim sobre a rua,
arriscando cair. 

 Vi então sobre um balcão a minha esquerda, um pouco abaixo de mim, uma
jovem mulher de \textit{négligé} branco: sua mão sustentava sua cabeça
encantadora, inclinada o suficiente para deixar entrever, à luz dos
astros, um perfil muito interessante, e sua atitude parecia imaginada
para apresentar completamente, para um viajante aéreo como eu, um talhe
esbelto e bem proporcionado; um dos seus pés nus, jogado
descuidadamente para trás, estava virado de modo que me era impossível,
apesar da obscuridade, perceber suas felizes dimensões, enquanto um
chinelinho, do qual ele estava separado, os determinava de forma ainda
melhor a meu olho curioso. Eu a deixo a pensar, querida Sophie, qual
era a intensidade de minha situação. Não ousava fazer a menor
exclamação, por medo de intimidar minha vizinha, nem o menor movimento,
por medo de cair na rua.

 Entretanto, um suspiro me escapou apesar de minha vontade, mas consegui
reprimi-lo à metade; o resto foi levado por um zéfiro que passava e
tive todo o tempo para examinar a sonhadora, sustentado nessa posição
perigosa pela esperança de ouvi-la ainda cantar. Mas, ai de mim! Sua
romança tinha acabado e meu mau destino a fez guardar o silêncio mais
teimoso. Enfim, depois de ter esperado por bastante tempo, considerei
poder me arriscar a endereçar-lhe a palavra: tratava-se apenas de
encontrar um cumprimento digno dela e dos sentimentos que ela me tinha
inspirado. Oh! Como me arrependi de não ter terminado minha epístola
dedicatória em versos! Como a teria encaixado bem nessa ocasião! Minha
presença de espírito não me abandonou na necessidade. Inspirado pela
doce influência dos astros e pelo desejo mais forte ainda de ter
sucesso junto a uma bela, depois de ter tossido ligeiramente para
preveni-la e para deixar o som de minha voz mais doce:

 -- O tempo está lindo esta noite --, eu lhe disse com o tom mais
afetuoso que me foi possível.

\section*{Capítulo XVIII}

 Acho que posso ouvir daqui \textit{Madame} de Hautcastel, que não me
deixa esquecer de nada, pedir contas da romança de que falei no
capítulo precedente. Pela primeira vez em minha vida me encontro
perante a dura necessidade de lhe recusar alguma coisa. Se eu inserisse
esses versos na minha viagem não deixariam de me considerar seu autor,
o que atrairia para mim, além da possibilidade de contusões, mais de
um comentário maldoso que quero evitar. Continuarei então o relato de
minha aventura com minha amável vizinha, aventura cuja catástrofe
inesperada, assim como a delicadeza com que a conduzi, são feitas para
interessar todas as classes de leitores. Mas, antes de saber o que ela
me respondeu e como foi recebido o cumprimento engenhoso que eu lhe
tinha endereçado, devo responder previamente a certas pessoas que se
consideram mais eloquentes que eu e que me condenarão sem piedade por
ter começado a conversação de uma maneira tão trivial -- segundo seu
julgamento. Eu lhes provarei que, se tivesse sido espirituoso nessa
ocasião importante, teria infringido completamente as regras da
prudência e do bom gosto. Todo homem que começa uma conversa com uma
bela usando uma palavra bonita ou fazendo um cumprimento, tão
lisonjeador como ele possa ser, deixa entrever as pretensões que não
devem transparecer antes de começarem a ser fundadas. Além disso, se
ele se faz de espirituoso, está evidente que procura brilhar e
consequentemente que pensa menos em sua dama que em si mesmo. Ora, as
damas querem que nos ocupemos delas e, ainda que elas não façam sempre
as mesmas reflexões que acabo de escrever, possuem um sentido refinado
e natural que lhes ensina que uma frase trivial, dita pelo único motivo
de começar uma conversa e aproximar-se delas, vale mil vezes mais que
um rasgo de espírito inspirado pela vaidade, e mais ainda (o que vai
parecer bastante espantoso) que uma epístola dedicatória em versos.
Além disso, ainda sustento (devido ao fato de meu sentimento ser
observado como um paradoxo) que este espírito leve e brilhante da
conversação não é nem mesmo necessário na relação mais duradoura, se
foi verdadeiramente o coração que a formou; e, apesar de tudo que as
pessoas que só amaram pela metade dizem dos longos intervalos que
deixam entre eles os sentimentos vivos de amor e amizade, o dia é
sempre curto quando é passado junto a sua amiga, e o silêncio é tão
interessante quanto qualquer discussão. 

 Seja qual for minha dissertação, é certo que não vi nada de melhor para
dizer, na borda do telhado onde me encontrava, que as tais palavras em
questão. Apenas as tinha pronunciado e minha alma se transportou
inteira ao tímpano de meus ouvidos para captar até a menor nuance de
sons que eu esperava ouvir. A bela levantou a cabeça para me olhar:
seus longos cabelos se soltaram como um véu e serviram de fundo para
seu rosto encantador, que refletia a luz misteriosa das estrelas. Sua
boca estava já entreaberta, suas palavras doces alcançavam seus
lábios\ldots

 Mas, oh céu! Qual foi minha surpresa e meu terror!\ldots\ Um barulho
sinistro se fez ouvir:

 -- Que fazes lá, \textit{madame}, a esta hora? Entra! -- disse uma voz
masculina e sonora do interior do apartamento.

 Fiquei petrificado.

\section*{Capítulo XIX}

 Tal deve ser o barulho que vem assustar os culpados quando abrimos de
repente a sua frente as portas ardentes do Tártaro; ou tal ainda deve
ser aquele que fazem, sob as abóbadas infernais, as sete cataratas do
Styx,\footnote{ Rio do inferno, segundo a mitologia clássica.} de
que os poetas esqueceram de falar.

\section*{Capítulo XX}

 Um fogo-fátuo atravessou o céu nesse momento e desapareceu quase
imediatamente. Meus olhos, que a claridade do meteoro tinha arrebatado
por um instante, voltaram para o balcão, e só viram o chinelinho. Minha
vizinha, na sua retirada precipitada, tinha esquecido de vesti-lo.
Contemplei longamente este chinelinho, digno da tesoura de
Praxíteles,\footnote{ Escultor grego, século \textsc{iv} a.C. Ao lado de Fídias,
um dos maiores escultores gregos.} com uma emoção cuja
intensidade não ousarei confessar; mas, o que poderá parecer bastante
singular, e que não saberei explicar para mim mesmo, é que um encanto
intransponível me impedia de desviar meu olhar, apesar de todos os
esforços que fazia para levá-lo para outros objetos. 

 Conta-se que, quando uma serpente olha para um rouxinol, o infeliz
passarinho, vítima de um encanto irresistível, é forçado a se aproximar
do réptil voraz. Suas asas rápidas só lhe servem para conduzi-lo a sua
perdição e cada esforço que ele faz para distanciar-se o aproxima do
inimigo que o persegue com seu olhar inevitável. 

 Tal era o efeito desse chinelinho sobre mim, sem que, entretanto,
pudesse dizer com certeza quem, o chinelinho ou eu, era a serpente, já
que segundo as leis da física a atração deveria ser recíproca. É certo
que esta influência funesta não era mais que um jogo de minha
imaginação. Estava tão realmente e tão verdadeiramente atraído que
estive por duas vezes a ponto de afrouxar a minha mão e me deixar cair.
Entretanto, como o balcão para o qual queria ir não estava exatamente
sob a minha janela, mas um pouco ao lado, vi muito bem que a força da
gravidade inventada por Newton, combinando-se com a atração oblíqua do
chinelinho, geraria uma queda em diagonal e eu teria caído sobre uma
guarita, que não me parecia, da altura onde me encontrava, ter o
diâmetro maior que o de um ovo, de sorte que meu objetivo não seria
alcançado\ldots

 Assim eu me segurei mais fortemente ainda à janela e, fazendo um
esforço de vontade, consegui levantar meus olhos e olhar o céu. 

\section*{Capítulo XXI}

 Ficaria em dificuldades para explicar e definir exatamente a espécie de
prazer que experimentava nesta circunstância. Tudo que posso afirmar é
que não havia nada em comum com aquilo que tinha me feito sentir,
alguns momentos antes, o aspecto da Via Láctea e do céu estrelado.
Entretanto, como nas situações mais embaraçosas da minha vida sempre
gostei de raciocinar sobre o que se passa em minha alma, quis nessa
ocasião ter uma ideia bem clara do prazer que pode sentir um bom homem
quando contempla o chinelinho de uma dama, comparado ao prazer que lhe
proporciona a contemplação das estrelas. Para isso, escolhi no céu a
constelação mais visível. Era, se não me engano, o trono de
Cassiopeia,\footnote{ A constelação de Cassiopeia, visível a partir do
hemisfério norte, apresenta um formato que lembra a letra W, e já foi
chamada de Trono e de Cadeira (ou ``dama da cadeira'').} que se
encontrava sobre minha cabeça, e eu olhava alternadamente a constelação
e o chinelinho, o chinelinho e a constelação. Vi então que essas duas
sensações eram de natureza muito diferente: uma estava em minha cabeça
enquanto a outra me parecia localizar-se na região do coração. Mas o
que não confessaria sem um pouco de vergonha é que a atração que me
levava para o chinelinho encantado absorvia todas as minhas faculdades.
O entusiasmo que me tinha causado algum tempo antes o aspecto do céu
estrelado só existia fragilmente, e anulou-se em seguida, quando ouvi a
porta do balcão reabrir-se e percebi um pezinho, mais branco que o
alabastro, avançar docemente e apossar-se do chinelinho. Quis falar;
mas, não tendo tido tempo de me preparar como da primeira vez, não
encontrei minha presença de espírito de sempre e ouvi a porta do balcão
se fechar antes de ter imaginado qualquer coisa conveniente a dizer.

\section*{Capítulo XXII}

 Os capítulos precedentes serão suficientes, espero, para responder
vitoriosamente a uma acusação de \textit{Madame} de Hautcastel, que não
hesitou em denegrir minha primeira viagem sob o pretexto de que não
houve lá ocasião para se fazer amor. Ela não poderia fazer a mesma
censura a esta viagem e, ainda que minha aventura com minha amável
vizinha não tenha sido levada longe, posso assegurar que encontrei ali
mais satisfação que em mais de uma outra circunstância onde me tinha
imaginado muito feliz, por falta de comparação. Cada um goza a vida a
sua maneira; mas eu consideraria não dar o que devo à benevolência do
leitor se lhe deixasse ignorar uma descoberta que, mais que qualquer
outra coisa, contribuiu até aqui para minha felicidade (com a condição,
de qualquer modo, de que isso fique entre nós), porque não se trata de
nada menos que um novo método de fazer amor, muito mais vantajoso que o
precedente, sem ter nenhum de seus numerosos inconvenientes. Sendo esta
invenção especialmente destinada às pessoas que quiserem adotar minha
nova maneira de viajar, acho que devo consagrar alguns capítulos a sua
instrução. 

\section*{Capítulo XXIII}

 Tinha observado, no curso de minha vida, que quando estava apaixonado
segundo o método comum minhas sensações jamais correspondiam a minhas
esperanças e minha imaginação se via derrotada em todos os seus planos.
Refletindo com atenção sobre o assunto considerei que, se me fosse
possível estender o sentimento que leva ao amor individual sobre todo o
sexo de que é objeto, conseguiria alegrias novas sem me comprometer de
maneira nenhuma. Que reprovação, de fato, poderíamos fazer a um homem
que se encontrasse de posse de um coração tão enérgico para amar todas
as mulheres amáveis do universo? Sim, \textit{madame}, eu amo a todas,
e não somente aquelas que conheço, ou que espero encontrar, mas todas
aquelas que existem sobre a face da terra. Mais ainda, amo a todas as
mulheres que existiram e aquelas que existirão, sem contar um número
bem maior ainda que minha imaginação tira do nada. Todas as mulheres
possíveis enfim estão compreendidas no vasto círculo de minhas
afeições.

 Por que injusto e bizarro capricho fecharia um coração como o meu nas
fronteiras estreitas de uma sociedade? Que digo? Por que circunscrever
seu voo aos limites de um reino ou mesmo de uma república? 

 Sentada ao pé de um carvalho batido pela tempestade, uma jovem viúva
indiana mistura seus suspiros ao barulho dos ventos furiosos. As armas
do guerreiro que ela amava estão suspensas sobre sua cabeça, e o
barulho lúgubre que elas fazem chocando-se umas contra as outras traz a
seu coração a lembrança da felicidade vivida. Entretanto, o raio cruza
as nuvens e a luz lívida dos relâmpagos se reflete em seus olhos
imóveis.  Enquanto a lenha que deve consumi-la se levanta, só, sem
consolação, no estupor do desespero, ela espera uma morte assustadora
que uma crença cruel a faz preferir à vida. 

 Que doce e melancólica alegria não experimenta um homem sensível
aproximando-se dessa desafortunada para consolá-la. Enquanto estou
sentado sobre a erva, ao lado dela, procuro dissuadi-la do horrível
sacrifício e, misturando meus suspiros aos seus e minhas lágrimas a
suas lágrimas, encarrego-me de distraí-la de suas dores, toda a cidade
acorre à casa de madame d’A***, cujo marido acaba de morrer em um
ataque de apoplexia. Resolvida também a não sobreviver a sua
infelicidade, insensível às lágrimas e aos pedidos de seus amigos, ela
se deixa morrer de fome; e, desde esta manhã, quando imprudentemente
anunciaram-lhe esta notícia, a infeliz só comeu um biscoito e bebeu
apenas um copinho de vinho de Málaga. Só dou a essa mulher desolada a
mera atenção necessária para não infringir as leis de meu sistema
universal, e distancio-me logo de sua casa porque sou naturalmente
ciumento e não quero me comprometer com uma multidão de consoladores,
muito menos com pessoas muito fáceis de consolar.

 As belezas infelizes têm particularmente direitos sobre meu coração e o
tributo de sensibilidade que lhes devo não enfraqueceu o interesse que
dirijo às que são felizes. Esta disposição varia infinitamente meus
prazeres, e me permite passar, cada vez, da melancolia à alegria e de
um repouso sentimental à exaltação.

 Frequentemente formo também intrigas amorosas na história antiga e
apago linhas inteiras dos velhos registros do destino. Quantas vezes
não parei a mão parricida de Virgínio e salvei a filha
desafortunada,\footnote{ Virgínia, filha de Virgínio, importante militar
romano, noiva de um ex-tribuno. Foi assassinada pelo pai que queria
livrá-la de ser entregue como escrava e esposa a Ápio Cláudio,
decênviro romano que a reivindicava -- história narrada por Tito Lívio
em seu \textit{Ab urbe codita libri}.} vítima dos excessos do
crime e da virtude ao mesmo tempo! Este acontecimento me enche de
terror quando volta ao meu pensamento; não me espanta que tenha sido
origem de uma revolução. 

 Espero que as pessoas razoáveis, assim como as almas caridosas,
reconheçam que tratei do caso amigavelmente. E todo homem que conhece
um pouco o mundo julgará como eu que, se tivéssemos acreditado no
decênviro, esse homem apaixonado não teria deixado de fazer justiça à
virtude de Virgínia: os pais não teriam se envolvido, o pai Virgínio,
no fim, teria se acalmado, e o casamento teria se dado em todas as
formas requeridas pela lei.

 Mas o amante infeliz abandonado, em que ele se transformou? Pois bem, o
amante, que ganhou com o assassinato? Mas, já que querem apiedar-se da
sua sorte, eu informo, querida Marie, que, seis meses após a morte de
Virgínia ele estava não apenas consolado mas muito bem casado, e que
depois de ter tido muitos filhos e perdido a sua mulher casou-se
novamente, seis semanas depois, com a viúva de um tribuno do povo.
Estas circunstâncias, ignoradas até nossos dias, foram descobertas e
decifradas a partir de um manuscrito palimpsesto da biblioteca
Ambrosiana, por um sábio arqueólogo italiano\footnote{ Ângelo Mai,
filólogo italiano (1782--1854) que estudou muitos manuscritos
palimpsestos e teve sucesso em muitos casos ao fazer reaparecer o texto
primitivo.} -- aumentaram, infelizmente, em uma página a história
abominável e já muito longa da república romana. 

\section*{Capítulo XXIV}

 Depois de ter salvado a interessante Virgínia, escapo modestamente ao
seu reconhecimento; e, sempre desejoso de prestar serviço às belas,
aproveito da obscuridade de uma noite chuvosa e vou abrir furtivamente
a tumba de uma jovem vestal que o senado romano teve a barbárie de
enterrar viva, por ela ter deixado apagar-se o fogo sagrado de Vesta,
ou talvez por se ter queimado levemente nele. Caminho em silêncio pelas
ruas tortuosas de Roma com o encanto interior que precede as boas
ações, sobretudo quando elas não se dão sem perigo. Evito com cuidado o
Capitólio, por medo de despertar os gansos e, infiltrando-me através
dos guardas da porta da Colina, chego com êxito ao túmulo sem ser
percebido.

 Com o barulho que faço levantando a pedra que o cobre, a desafortunada
mostra sua cabeça despenteada pela terra úmida da tumba. Eu a vejo, ao
luar da lâmpada sepulcral, olhar a sua volta com um olhar esgazeado: em
seu delírio a infeliz vítima crê estar já nas águas do Cocito.\footnote{ Na 
mitologia grega, Cocito é o estuário dos rios do inferno.}

-- Oh Minos!, ela grita, oh juiz inexorável! Eu amava, é verdade,
sobre a terra, contra as leis severas de Vesta. Se os deuses são tão
bárbaros quanto os homens, abre, abre para mim os abismos do Tártaro!
Eu amava e amo ainda. 

-- Não, não, tu ainda não estás no reino dos mortos; vem, jovem
desafortunada, reaparece sobre a terra! Renasce à luz e ao amor! 

Enquanto isso seguro sua mão gelada pelo frio da tumba e a levanto em
meus braços, seguro-a contra meu coração e a arranco enfim desse lugar
horrível, palpitante de terror e reconhecimento. 

Evite acreditar, \textit{madame}, que qualquer interesse pessoal seja o
móvel desta boa ação. A esperança de usar em meu favor a bela ex-vestal
não entra em nada do que faço por ela porque voltaria assim ao antigo
método: posso assegurar, palavra de viajante, que, todo o tempo que
durou nosso passeio, desde a porta Colina até o lugar onde se encontra
agora o túmulo dos Scipiones, apesar da obscuridade profunda, e mesmo
nos momentos em que a fraqueza me obrigava a sustentá-la em meus
braços, não deixei de tratá-la com a consideração e o respeito devidos
a seus infortúnios, e entreguei-a escrupulosamente a seu amante que a
esperava na estrada.

\section*{Capítulo XXV}

 Uma outra vez, conduzido pela minha imaginação, encontrei-me por acaso
em meio ao rapto das Sabinas:\footnote{ Lenda que trata do início da
história de Roma: as Sabinas, mulheres de uma aldeia próxima, são
raptadas sob ordens de Rômulo para povoar a cidade. Conta-se que Rômulo
de início simplesmente as convidou para que viessem a Roma; como elas
declinaram do convite ele teria organizado uma festa para todos os
Sabinos e o rapto teria se dado nesse momento.} vi com muita surpresa
que os Sabinos tomavam a questão de uma maneira bem diferente do que a
história nos conta. Sem entender nada dessa confusão ofereci a minha
proteção a uma mulher que fugia; e não pude me impedir de rir ao
acompanhá-la quando ouvi um Sabino furioso gritar com um toque de
desespero: 

 -- Deuses imortais! Por que não levei minha mulher à festa!

\section*{Capítulo XXVI}

  Além da metade do gênero humano à qual dedico viva afeição -- eu
diria, mas me acreditariam? -- meu coração é dotado de uma tal
capacidade de ternura que todos os seres viventes e mesmo as coisas
inanimadas têm dele uma boa parte. Amo as árvores que me emprestam sua
sombra, e os pássaros que pipiam sob a folhagem, e o grito noturno da
coruja, e o barulho das torrentes: amo tudo\ldots\ amo a lua!

 Ris, \textit{mademoiselle}; é fácil achar ridículo os sentimentos que
não experimentamos; mas os corações que se parecem com o meu me
compreenderão.

 Sim, me apego com verdadeira afeição a tudo que está a minha volta. Amo
os caminhos por onde passo, a fonte de que bebo; não me separo sem
alguma pena do galho que peguei ao acaso em uma aleia: ressinto as
folhas que caem e até o zéfiro que passa. Onde está agora aquele que
agitava teus cabelos negros, Elisa, quando, sentada perto de mim nas
margens da Dora,\footnote{ Na província de Turim, parte da comuna de
Collegno.} na véspera de nossa separação, me olhavas com um silêncio
triste? Onde está teu olhar? Onde está este instante doloroso e amado?

 Oh Tempo!\ldots\ divindade terrível! não é tua foice cruel que me assusta;
só temo teus filhos horríveis, a indiferença e o esquecimento, que
fazem uma longa morte de três quartos da nossa existência. 

 Ai de mim! esse zéfiro, esse olhar, esse sorriso estão tão longe de mim
quanto as aventuras de Ariane. Só restam arrependimentos e vãs
lembranças no fundo do meu coração: triste mistura sobre a qual minha
vida ainda se sustenta, como um barco destruído pela tempestade flutua
ainda algum tempo sobre o mar agitado\ldots

\section*{Capítulo XXVII}

 Até que, pouco a pouco a água se introduzindo entre as tábuas
quebradas, o barco infeliz desaparece engolido pelo abismo; as ondas o
recobrem, a tempestade se acalma e a andorinha do mar voa baixo sobre a
superfície solitária e tranquila do oceano.

\section*{Capítulo XXVIII}

 Eu me vejo forçado a terminar aqui a explicação de meu novo método de
fazer amor, porque percebo que ele está caindo no vazio. Não será,
entretanto, fora de propósito, acrescentar ainda alguns esclarecimentos
sobre esta descoberta, que não convém geralmente a todo mundo nem a
todas as idades. Não aconselharia a ninguém colocá-lo em uso aos vinte
anos; o próprio inventor não o usava nesta altura da vida. Para tirar
dele todo o proveito possível é preciso ter experimentado todos os
sofrimentos da vida sem se ter desencorajado, e todas as alegrias sem
se ter desgostado. Questão difícil! Sobretudo útil a esta idade quando
a razão nos aconselha a renunciar aos hábitos da juventude e pode
servir de intermediária e de passagem insensível entre o prazer e a
sabedoria. Essa passagem, como observaram os moralistas, é muito
difícil. Poucos homens têm a nobre coragem de atravessá-la com
distinção; e, frequentemente, depois de ter dado o passo, entediam-se
do outro lado, e retornam sobre o fosso com os cabelos grisalhos e
grande vergonha. É isso que eles evitarão, sem sofrimento, com a minha
nova maneira de fazer amor. De fato, a maior parte de nossos prazeres
não sendo outra coisa além de um jogo de imaginação, é essencial
apresentar-lhe um pasto inocente para desviá-la dos objetos a que
devemos renunciar, mais ou menos como quando apresentamos brinquedos às
crianças enquanto lhes recusamos os doces. Desta maneira temos tempo
para nos firmarmos sobre o terreno da sabedoria sem pensar já estar lá,
e chegamos pelo caminho da loucura, o que facilitara singularmente seu
acesso a muita gente. 

 Acredito então não me ter enganado no espírito de ser útil que me fez
tomar da pena, e não posso mais que me defender do natural movimento de                      
amor-próprio que poderia legitimamente sentir ao revelar aos homens
tais verdades.

\section*{Capítulo XXIX}

 Todas essas confidências, minha querida Sophie, não te deixarão
esquecer, espero, a posição vergonhosa na qual me deixaste em minha
janela. A emoção que o aspecto do belo pé de minha vizinha me tinha
causado ainda durava e eu estava mais que nunca caído pelo perigoso
encanto da pantufa, quando um acontecimento imprevisto veio me tirar do
perigo em que estava de me precipitar do quinto andar sobre a rua. Um
morcego que girava em torno da casa e que, vendo-me imóvel há bastante
tempo, aparentemente me tomou por uma chaminé, veio de repente pousar
sobre mim e agarrar-se a minha orelha. Senti em minha bochecha o
horrível frescor de suas asas úmidas. Todos os ecos de Turim
responderam ao grito furioso que dei sem poder me conter. As sentinelas
distantes deram o ``quem vem lá'', e ouvi na rua a marcha precipitada de
uma patrulha.  

 Abandonei sem muita pena a vista do balcão, que não tinha mais nenhum
atrativo sobre mim. O frio da noite me tinha tomado. Um ligeiro tremor
me percorreu da cabeça aos pés; e, enquanto punha meu roupão para me
esquentar vi, para meu desgosto, que esta sensação de frio, junto com o
insulto do morcego, tinha bastado para mudar de novo o rumo das minhas
ideias. A pantufa mágica não teria nesse momento maior influência sobre
mim que a cabeleira de Berenice ou qualquer outra constelação. Calculei
em seguida quanto era pouco razoável passar a noite exposto às
intempéries do ar, no lugar de seguir as resoluções da natureza, que
nos ordena o sono. Minha razão, que nesse momento agia sozinha sobre
mim, me fez perceber tudo isso comprovado como uma proposição de
Euclides. Enfim fui de repente privado da imaginação e do entusiasmo e
deixado sem socorro na triste realidade. Existência deplorável! Valeria
a mesma coisa ser uma árvore seca em uma floresta ou um obelisco no
meio de uma praça.

 -- São duas máquinas estranhas, gritava então para mim mesmo, a cabeça
e o coração do homem! Levado por estes dois dirigentes de suas ações
cada vez em direções contrárias, a última que ele segue lhe parece
sempre a melhor! 

-- Oh loucura do entusiasmo e do sentimento! -- diz a fria Razão.

-- Oh fraqueza e incerteza da razão! -- diz o Sentimento. 

Quem poderá jamais, quem ousará decidir entre elas?

Pensei que seria bom tratar da questão na hora e decidir de uma vez por
todas a qual dos dois guias era conveniente confiar-se para o resto da
vida. Seguiria a partir de agora minha cabeça ou meu coração?
Examinemos.

\section*{Capítulo XXX}

 Dizendo essas palavras, percebo uma dor surda naquele pé que repousava
sobre a escada. Estava, além disso, muito cansado da posição difícil em
que tinha me mantido até então. Abaixei-me com cuidado para sentar e,
deixando as minhas pernas se pendurarem à direita e à esquerda da
janela, comecei minha viagem a cavalo. Sempre preferi, a qualquer
outra, esta maneira de viajar, e amo apaixonadamente os cavalos;
entretanto, de todos os que já vi, ou de que ouvi falar, o que mais
ardentemente desejei possuir foi o cavalo de madeira de que se fala nas
\textit{Mil e uma noites}, sobre o qual se podia viajar pelos ares e
que partia como um raio quando girávamos uma manivelinha entre suas
orelhas.

 Ora, poderíamos dizer que minha montaria se parecia muito com a das
\textit{Mil e uma noites}. Por sua posição, o viajante a cavalo sobre
sua janela se comunica por um lado com o céu, e goza do imponente
espetáculo da natureza: os meteoros e os astros estão a sua disposição;
de outro, o aspecto de seu aposento e os objetos que ele contém o
remetem à ideia de sua existência e o fazem entrar em si mesmo. Um só
movimento da cabeça substitui a manivela encantada e é suficiente para
operar uma mudança tão rápida quanto extraordinária na alma do
viajante. Ora habitando a terra, ora os céus, seu espírito e seu
coração percorrem todas as alegrias que ao homem é dado provar.

 Pressenti antecipadamente tudo o que poderia conseguir de minha
montaria. Quando me senti bem na sela e convenientemente ajeitado,
certo de não ter nada a temer de ladrões ou de passos em falso de meu
cavalo, considerei a ocasião muito favorável para me deixar levar pelo
exame do problema que deveria resolver, quanto à preeminência da razão
ou do sentimento. Mas a primeira reflexão que fiz sobre esse assunto me
fez parar, sem mais. 

-- Seria eu o responsável por me fazer juiz em tal caso? eu me dizia em
voz baixa, eu que, em minha consciência, dou de pronto ganho de causa
ao sentimento? Mas, por outro lado, se excluo as pessoas cujo coração
prevalece sobre a cabeça, quem poderia consultar? Um geômetra? Bah!
Estas pessoas estão vendidas à razão. Para decidir este ponto, seria
preciso encontrar um homem que tivesse recebido da natureza uma igual
dose de razão e sentimento e que no momento da decisão estas duas
faculdades estivessem perfeitamente em equilíbrio\ldots\ coisa impossível!
Seria mais fácil equilibrar uma república. O único juiz competente
seria então aquele que não tivesse nada em comum nem com um nem com
outra, um homem enfim sem cabeça e sem coração.

Este estranho resultado revoltou minha razão. Meu coração, por sua vez,
protestou por não ter aí nenhuma parte. Entretanto, considerava ter
raciocinado com justeza e teria, a esta ocasião, considerado da pior
maneira as minhas faculdades intelectuais se tivesse refletido que, nas
especulações da alta metafísica, como essa de que a questão trata, os
filósofos de primeira linha foram frequentemente conduzidos por
raciocínios seguidos de consequências assustadoras que influenciaram a
felicidade da sociedade humana. Eu, portanto, me consolava pensando que
o resultado de minhas especulações ao menos não faria mal a ninguém.
Deixei a questão indecidida e resolvi, para o resto de meus dias,
seguir alternadamente minha cabeça e meu coração, na medida em que um
prevalecesse sobre o outro. Acredito, de fato, que é o melhor método.
Não me fez fazer, na verdade, grande fortuna até aqui, eu me dizia. Não
importa, eu vou, descendo o caminho rápido da vida, sem medo e sem
projetos, ora rindo e ora chorando, e muitas vezes ao mesmo tempo, ou
mesmo assobiando alguma velha ária pra me distrair ao longo do caminho.
Outras vezes, colho uma margarida no canto de uma aleia, arranco as
pétalas uma a uma dizendo: 

-- Ela me ama, um pouco, muito, apaixonadamente, nada. 

A última me dá quase sempre \textit{nada}. De fato, Elisa não me ama
mais.

Enquanto me ocupo assim, a geração inteira de viventes passa: como uma
imensa onda, ela vai logo quebrar, comigo, sobre o rio da eternidade;
e, como se a tempestade da vida não fosse já suficientemente impetuosa,
como se ela nos empurrasse muito lentamente às barreiras da existência,
as nações em massa decapitam-se às pressas e precedem o termo fixado
pela natureza. Os conquistadores, levados eles próprios pelo turbilhão
rápido do tempo, divertem-se a jogar milhares de homens por terra. Ah!
\textit{Messieurs}, o que pensam? Esperem!\ldots\ estas boas pessoas iriam
morrer sua bela morte. Não veem a vaga que avança? já espuma perto do
rio\ldots\ Esperem, ainda um instante, em nome do céu; e vocês, e seus
inimigos, e eu, e as margaridas, tudo isso vai acabar! Podemos nos
espantar o suficiente por tal demência?

Vamos, é um ponto pacífico; daqui em diante eu mesmo não mais
despetalarei margaridas.

\section*{Capítulo XXXI}

 Depois de ter firmado uma regra de conduta prudente para o futuro, por
meio de uma lógica luminosa, como vimos nos capítulos precedentes,
restava-me um ponto muito importante a decidir sobre o assunto da
viagem que iria fazer. Não basta, na verdade, mover-se com um carro ou
a cavalo, é preciso ainda saber onde queremos ir. Eu estava tão cansado
das pesquisas metafísicas de que tinha acabado de me ocupar que, antes
de me decidir sobre a região do globo à qual daria preferência, quis me
repousar por algum tempo sem pensar em nada. É uma maneira de existir
que é também de minha invenção e que me foi frequentemente de muita
utilidade; mas não é dado a todo mundo saber usá-la: porquê, se é fácil
dar profundidade a suas ideias ocupando-se vigorosamente de um assunto,
não é tanto parar de repente seu pensamento como se para o balanço de
um pêndulo. Molière muito inadequadamente transformou em ridículo um
homem que se distraía dando voltas em um poço; quanto a mim, eu me
inclinaria a acreditar que este homem era um filósofo que tinha o poder
de suspender a ação de sua inteligência para repousar, operação das
mais difíceis que o espírito humano pode executar. Sei que as pessoas
que receberam esta faculdade sem a ter desejado e que geralmente não
pensam em nada me acusarão de plágio e reclamarão a prioridade da
invenção; mas o estado de imobilidade intelectual de que quero falar é
diferente daquele de que elas gozam e de que \textit{monsieur}
Necker\footnote{ Jacques Necker (1732--1804), homem de Estado
encarregado das finanças de Luís \textsc{xvi}.} fez a apologia.\footnote{
\textit{Sur le bonheur des sots}. 1782, in-12. [N. do A.]}  O meu é sempre
voluntário e só pode ser momentâneo. Para gozar dele em toda sua
plenitude, fechava os olhos apoiando-me com as duas mãos sobre a
janela, como um cavaleiro fatigado se apoia sobre a maçã da sela, e
logo a lembrança do passado, o sentimento do presente e a previsão do
porvir aniquilaram-se em minha alma.

 Como este modo de existência favorece fortemente a invasão do sono,
depois de meio minuto de gozo senti que minha cabeça tombava sobre meu
peito: abri os olhos no mesmo instante e minhas ideias retomaram seu
curso: circunstância que prova evidentemente que a espécie de letargia
voluntária de que se tratava é bem diferente do sono, já que fui
acordado pelo próprio sono. Esse incidente certamente jamais ocorreu a ninguém.

 Elevando meus olhos para o céu, percebi a Estrela Polar sobre a
cumeeira da casa, o que me pareceu de muito bom augúrio no momento em
que eu ia começar uma longa viagem. Durante o intervalo de repouso de
que acabava de gozar, minha imaginação tinha recuperado toda sua força
e meu coração estava pronto para receber as mais doces impressões, de
tal forma essa anulação passageira do pensamento pode aumentar sua
energia! O fundo de angústia que minha precária situação no mundo me
fazia surdamente experimentar foi substituído de repente por um vivo
sentimento de esperança e de coragem; senti-me capaz de afrontar a vida
e todas as chances de infortúnio ou de felicidade que ela traz consigo.

 -- Astro brilhante! -- eu gritava no êxtase delicioso que me arrebatava -- 
incompreensível produção do eterno pensamento! Tu que sozinho, imóvel
nos céus, vela desde o dia da criação sobre uma metade da terra! Tu que
diriges o navegador nos desertos do oceano e de quem um só olhar
frequentemente deu a esperança e a vida ao marinheiro atormentado pela
tempestade! Se jamais, quando uma noite serena me permitiu compreender
o céu, deixei de te procurar entre tuas companheiras, acompanha-me, luz
celeste! Ai de mim! a terra me abandona: sê hoje meu conselho e meu
guia, ensina-me qual é a região do globo onde devo fixar-me!

 Durante essa invocação, a estrela parecia brilhar mais vivamente e
alegrar-se no céu, convidando-me a me aproximar de sua influência
protetora.

 Não creio absolutamente em pressentimentos; mas creio em uma
providência divina que conduz os homens por meios desconhecidos. Cada
instante de nossa existência é uma criação nova, um ato da vontade
todo-poderosa. A ordem inconstante que produz as formas sempre novas e
os fenômenos inexplicáveis das nuvens é determinada por cada instante
até a menor parcela de água que nos compõe: os acontecimentos da nossa
vida não saberiam ter outra causa e atribuí-los ao acaso seria o cúmulo
da loucura. Posso até mesmo assegurar que algumas vezes me aconteceu
entrever os fios imperceptíveis com os quais a Providência faz os
maiores homens agirem como marionetes, enquanto eles se imaginam
conduzidos pelo mundo; um pequeno movimento de orgulho que ela lhes
sopra no coração é suficiente para fazer perecer exércitos inteiros e
para virar uma nação de pernas para o ar. 

 De qualquer modo, acreditava tão firmemente na realidade do convite que
tinha recebido da Estrela Polar que no mesmo instante me tomei de
vontade de ir para o norte; e, ainda que não tivesse nenhum ponto de
preferência nem algum destino determinado nessas regiões distantes,
quando parti de Turim no dia seguinte saí pela porta Palatina que fica
ao norte da cidade, persuadido de que a Estrela Polar não me
abandonaria. 

\section*{Capítulo XXXII}

 Estava nesse ponto de minha viagem quando fui obrigado a descer
precipitadamente do cavalo. Não teria me dado conta desta
particularidade se não devesse, conscientemente, instruir as pessoas
que gostariam de adotar esta maneira de viajar dos pequenos
inconvenientes que ela apresenta, depois de lhes ter exposto as
imensas vantagens.

 As janelas, em geral, não tendo sido originalmente inventadas para a
nova destinação que eu lhe tinha dado, foram construídas por arquitetos
que negligenciaram a possibilidade de lhes dar a forma cômoda e
arredondada de uma sela inglesa. O leitor inteligente compreenderá,
espero, sem outra explicação, a causa dolorosa que me forçou a fazer
uma parada. Desci com algum sofrimento e dei algumas voltas a pé na
extensão do meu quarto para me esticar, refletindo sobre a mistura de
penas e prazeres de que a vida é semeada, assim como sobre o tipo de
fatalidade que faz dos homens escravos das circunstâncias mais
insignificantes. Depois disso, apressei-me em remontar o cavalo, munido
de uma almofada de plumas: o que não ousaria fazer alguns dias antes,
de medo de ser vaiado pela cavalaria; mas, tendo encontrado, na
véspera, nas portas de Turim, um grupo de Cossacos, que chegava das
margens dos Palus-Meotides e do mar Cáspio, sobre almofadas parecidas,
acreditei, sem transgredir as leis da equitação, que respeito bastante,
poder adotar o mesmo uso.

 Livre da sensação desagradável que deixei adivinhar, pude me ocupar sem
inquietação do meu plano de viagem.

 Uma das dificuldades que mais me incomodavam, porque falava à minha
consciência, era saber se eu fazia bem ou mal ao abandonar minha pátria
que tinha, por sua vez, me abandonado pela metade.\footnote{ O autor
servia no Piemonte quando a Savóia, onde nasceu, foi reunida à França.
[N. do A.] O que ocorreu em 1792. [N. da T.]} Tal decisão me parecia muito
importante para que eu me decidisse levianamente. Refletindo sobre esta
palavra pátria, percebi que eu não tinha uma ideia muito clara dela. 

-- Minha pátria? Em que consiste a pátria? Seria um conjunto de casas,
de campos, de rios? Não acreditaria nisso. É talvez minha família, são
meus amigos que constituem minha pátria? Mas eles já a deixaram. Ah! aí
está, é o governo? Mas ele mudou. Bom Deus! Onde está então minha
pátria?

 Passava a mão em meu rosto num estado de inquietude inexprimível. O
amor da pátria é tão enérgico! Os arrependimentos que eu mesmo
experimentava, só de pensar em abandonar a minha, me provavam tão bem a
realidade que eu teria permanecido a cavalo toda minha vida, sem
descanso, antes de ter esgotado essa dificuldade.

 Vi logo que o amor da pátria depende de muitos elementos reunidos,
trata-se do longo hábito que o homem adquire, desde a sua infância, dos
indivíduos, do lugar e do governo. Tratava-se apenas de examinar em que
estas três bases contribuem, cada uma em sua parte, para constituir a
pátria.

 O apego a nossos compatriotas, em geral, depende do governo, e não é
outra coisa além do sentimento da força e da felicidade que nos dá em
comum; porque o verdadeiro apego se restringe à família e a um pequeno
número de indivíduos com que estamos proximamente envolvidos. Tudo que
rompe o hábito ou a facilidade de se encontrar faz os homens inimigos:
uma cadeia de montanhas forma de um lado e de outro transmontanos que
não se toleram; os habitantes da margem direita de um rio se acreditam
muito superiores aos da margem esquerda, e estes por sua vez riem de
seus vizinhos. Percebe-se esta disposição mesmo nas grandes cidades
divididas por um rio, apesar das pontes que unem suas margens. A
diferença de linguagem afasta ainda mais os homens do mesmo governo.
Enfim, mesmo a família, onde reside nossa verdadeira afeição,
frequentemente está dispersa pela pátria, muda continuamente de forma e
número; além disso, pode ser transportada. Não é, portanto, nem em
nossos compatriotas, nem na nossa família, que absolutamente reside o
nosso amor pela pátria. 

 A localidade contribui ao menos da mesma maneira para o apego que temos
em relação ao nosso país natal. Quanto a isso há uma questão muito
interessante: sempre percebemos que os montanheses são, dentre todos os
povos, os que são mais apegados a seu país, e que os povos nômades
habitam, geralmente, as grandes planícies. Qual pode ser a causa desta
diferença no apego desses povos à localidade? Se não me engano, aí
está: nas montanhas a pátria tem uma fisionomia, nas planícies não. É
uma mulher sem rosto que não saberíamos amar, apesar de todas as suas
boas qualidades. Que resta, de fato, de sua pátria local ao habitante
de uma vila em um bosque quando, depois da passagem do inimigo, a vila
é queimada e as árvores cortadas? O infeliz procura em vão, na linha
uniforme do horizonte, qualquer objeto conhecido que possa lhe fornecer
lembranças: não existe nenhum. Cada ponto do espaço lhe apresenta o
mesmo aspecto e o mesmo interesse. Este homem é nômade pelo
acontecimento, a menos que o hábito do governo o retenha; mas sua
morada será aqui ou lá, não importa; sua pátria existe em todos os
lugares em que o governo exerce sua ação: ele só terá meia-pátria.
O montanhês se apega aos objetos que tem frente aos olhos desde sua
infância e que têm formas visíveis e indestrutíveis: de todos os pontos
do vale ele vê e reconhece seu campo sobre o declive da encosta. O
barulho da torrente que borbulha entre os rochedos não é interrompido
jamais, o atalho que conduz à vila muda de direção perto de um bloco
imutável de granito. Vê em sonho o contorno das montanhas que está
pintado em seu coração como depois de ter olhado longamente os vitrais
de uma janela nós os vemos ainda ao fechar os olhos: o quadro gravado
na sua memória faz parte dele próprio e não se apaga jamais. Enfim, as
próprias lembranças se ligam à localidade, mas é preciso que ela tenha
objetos cuja origem seja ignorada e de que não possamos prever o fim.
Os antigos edifícios, as velhas pontes, tudo que carrega o caráter de
grandeza e de longa duração, substitui em parte as montanhas na afeição
das localidades; os monumentos da natureza têm, entretanto, mais força
sobre o coração. Para dar a Roma uma designação digna dela, os
orgulhosos Romanos a chamaram de \textit{a cidade das sete
}\textit{colinas}. O hábito adquirido não pode jamais ser destruído. O
montanhês, na idade madura, não se afeiçoa mais às localidades de uma
grande cidade, e o habitante das cidades não saberia se tornar um
montanhês. Daí vem talvez o fato de que um dos maiores escritores dos
nossos dias,\footnote{ Chateaubriand.} que pintou com gênio os
desertos da América, achou os Alpes mesquinhos e o Mont Blanc
consideravelmente pequeno demais.

 A parte do governo é evidente: ele é a primeira base da pátria. É ele
que produz a ligação recíproca entre os homens e que torna mais
enérgica aquela que eles dedicam naturalmente à localidade; somente
ele, pelas lembranças de felicidade ou de glória, pode ligá-los ao solo
que os viu nascer.

 O governo é bom? A pátria está com toda sua força. Torna-se vicioso? A
pátria está doente. Muda? Ela morre. É então uma nova pátria, e cada um
é responsável por adotá-la ou escolher uma outra.

 Quando, segundo Temístocles, toda a população de Atenas deixou essa
cidade, os atenienses abandonaram sua pátria ou a levaram junto com
eles nos seus navios?

 Quando Coriolano\ldots

 Bom Deus! Em que discussão me engajei! Esqueço que estou a cavalo em
minha janela.

\section*{Capítulo XXXIII}

 Tinha uma velha parenta muito espirituosa cuja conversação era das mais
interessantes; mas sua memória, inconstante e fértil alternadamente, a
fazia passar frequentemente de episódios em episódios e de digressões
em digressões, ao ponto de ser obrigada a implorar socorro aos
ouvintes:

 -- O que mesmo eu queria contar a vocês? -- dizia ela, e muitas vezes os
ouvintes tinham esquecido, o que colocava todo mundo em um embaraço
inexprimível.

 Ora, pudemos perceber que o mesmo acidente me acontece com frequência
em minhas narrações, e eu devo de fato convir em que o plano e a ordem
da minha viagem estão exatamente calcados sobre a ordem e o plano das
conversações de minha tia. Mas não peço ajuda a ninguém, porque percebi
que meu assunto vem de si mesmo e no momento em que menos espero.

\section*{Capítulo XXXIV}

 As pessoas que não aprovarem minha dissertação sobre a pátria devem
estar prevenidas de que, já há algum tempo, o sono se apoderou de mim,
apesar dos esforços que fiz para combatê-lo. Não estou, entretanto,
muito certo agora se adormeci realmente e se as coisas extraordinárias
que vou contar foram o efeito de um sonho ou de uma visão sobrenatural.

 Vi descer do céu uma nuvem brilhante que se aproximou de mim pouco a
pouco e que recobriu, como se com um véu transparente, uma jovem de
vinte e dois a vinte e três anos. Procurei em vão expressões para
descrever o sentimento que seu aspecto me fez experimentar. Sua
fisionomia, brilhando de bondade e benevolência, tinha o charme das
ilusões da juventude e era doce como os sonhos de um porvir; seu olhar,
seu sorriso tranquilo, todos os seus traços enfim, realizavam a meus olhos
o ser ideal que meu coração procurava há tanto tempo e que tinha
desistido de encontrar.

 Enquanto a contemplava em um êxtase delicioso, vi brilhar a Estrela
Polar entre os cachos de sua cabeleira negra que o vento norte
levantava e no mesmo instante palavras consoladoras se fizeram ouvir.
Que digo? Palavras! Era a expressão misteriosa do pensamento celeste
que desvelava o futuro à minha inteligência, enquanto meus sentidos
estavam acorrentados pelo sono; era uma profética comunicação do astro
favorável que eu acabava de invocar e que eu vou procurar exprimir em
uma língua humana.

 -- Tua confiança em mim não será desapontada -- dizia uma voz cujo
timbre se parecia com o som das harpas eólias.\footnote{ Instrumento de
cordas que é tocado pelo vento, sem contar com um executante.} -- Olha,
aí está o campo que te reservei; aí o bem ao qual de forma vã aspiram
os homens que pensam que a felicidade é um cálculo e que pedem à terra
o que só podem obter do céu.

 A estas palavras o meteoro reentrou na profundeza dos céus, a divindade
aérea se perdeu nas brumas do horizonte; mas, distanciando-se, jogou
sobre mim olhares que encheram meu coração de confiança e esperança.

 Logo, ardendo por segui-la, piquei com toda minha força e, como tinha
esquecido de pôr as esporas, bati o calcanhar direito contra o ângulo
de uma telha com tanta violência que a dor me acordou em um
sobressalto.

\section*{Capítulo XXXV}

 Esse acidente teve uma serventia real para a parte geológica de minha
viagem, porque me deu a ocasião de conhecer exatamente a altura de meu
quarto acima das camadas de aluvião que formam o solo sobre o qual está
construída a cidade de Turim.

 Meu coração palpitava fortemente e acabava de contar três batimentos e
meio desde o instante em que tinha picado meu cavalo, quando ouvi o
barulho de meu chinelo que tinha caído na rua, o que, cálculo feito
sobre o tempo que levam os corpos pesados em sua queda acelerada, e do
que tinha empregado as ondulações sonoras do ar para vir da rua até
minha orelha, determinou a altura de minha janela em noventa e quatro
pés, três linhas e dois décimos de linha desde o nível da pavimentação
de Turim, supondo que meu coração agitado pelo sonho batia cento e
vinte vezes por minuto, o que não pode estar muito longe da verdade.
Depois de ter falado do interessante chinelinho de minha bela vizinha
só através da relação com a ciência ousei mencionar o meu; também
previno que esse capítulo não é absolutamente feito para os sábios.

\section*{Capítulo XXXVI}

 A brilhante visão de que acabava de desfrutar me fez sentir mais
vivamente, quando acordado, todo o horror do isolamento em que me
encontrava. Passeei meu olhar em volta de mim e não vi mais que
telhados e chaminés. Ai de mim! Suspenso no quinto andar entre o céu e
a terra, envolvido por um oceano de arrependimentos, desejos e
inquietudes, só me prendia à existência por uma incerta luz de
esperança: apoio fantástico de que tinha frequentemente provado a
fragilidade! A dúvida voltou logo a meu coração mortificado pelas
decepções da vida, e acreditei firmemente que a Estrela Polar se tinha
rido de mim. Desconfiança injusta e culpável, pela qual o astro me
puniu com dez anos de espera! Oh! Se tivesse podido prever então que
todas estas promessas seriam cumpridas e que eu encontraria um dia
sobre a terra o ser adorado cuja imagem eu só tinha entrevisto no céu!
Cara Sophie, si tivesse sabido que minha felicidade excederia todas as
minhas esperanças!\ldots

 Mas não é preciso antecipar os acontecimentos: volto ao meu assunto,
não desejando inverter a ordem metódica e severa à qual me submeti na
redação de minha viagem.

\section*{Capítulo XXXVII}

 O relógio do campanário de San Fillipo soou meia-noite lentamente.
Contei cada batida, uma depois da outra, e a última me arrancou um
suspiro. ``Aí está então, digo para mim mesmo, um dia que vem
destacar-se da minha vida; e, embora as vibrações decrescentes do som
do bronze vibrassem ainda em minha orelha, a parte de minha viagem que
precedeu a meia-noite está já tão longe de mim quanto a viagem de
Ulisses ou a de Jasão. Neste abismo do passado os instantes e os
séculos têm a mesma duração. E o porvir será mais real?'' São dois nadas
entre os quais me encontro em equilíbrio como sobre o fio de uma
lâmina. Na verdade o tempo me parece algo inconcebível em si, algo que
eu estaria tentado a crer que não existe realmente e que o que nomeamos
assim não é nada além de uma punição do pensamento.  

 Alegrava-me ter encontrado essa definição do tempo, tão tenebrosa
quanto o próprio tempo, quando um outro relógio soou meia-noite, o que
me deu um sentimento desagradável. Resta-me sempre um fundo de humor
quando estou inutilmente ocupado com um problema insolúvel e achei
muito deslocada essa segunda advertência do sino a um filósofo como eu.
Mas experimentava, decididamente, um verdadeiro desapontamento alguns
segundos depois quando ouvi ao longe um terceiro sino, o do convento
dos Capuchinhos, situado do outro lado do rio Pó, soar, como por
malícia, ainda meia-noite. 

 Quando minha tia chamava a uma antiga criada de quarto, a quem ela se
afeiçoara muito embora fosse rabugenta, não se contentava, em sua
impaciência, em soar uma vez, mas puxava sem descanso o cordão da
campainha até que ela aparecesse. 

-- Venha logo, \textit{mademoiselle} Blanchet! 

E esta, descontente de se ver assim pressionada, vinha lentamente e
respondia com muito azedume antes de entrar no salão:  

-- Já vai, \textit{madame}, já vai! 

Tal foi também o sentimento que experimentei quando ouvi o sino
indiscreto dos Capuchinhos soar meia-noite pela terceira vez. 

-- Eu sei -- gritei, estendendo as mãos para o lado do relógio --, sim, eu
sei que é meia-noite: sei até demais!

É, não há dúvida, foi por um conselho insidioso do espírito
malévolo que os homens encarregaram esta hora de dividir seus dias.
Fechados em suas habitações, dormem ou se divertem, enquanto ela corta
um fio de sua existência: no dia seguinte levantam-se alegremente sem
nem cogitar que têm um dia a mais. Em vão a voz profética do bronze                           
lhes anuncia a aproximação da eternidade, em vão ela lhes repete
tristemente cada hora que acaba de escorrer; eles não ouvem nada, ou,
se ouvem, não compreendem. Oh meia-noite!\ldots\ Hora terrível! 

Não sou supersticioso, mas esta hora me inspira sempre uma espécie de
temor e tenho o pressentimento de que, se vier a morrer, será à
meia-noite. Morrerei, portanto, um dia! Como?! Morrerei? Eu que falo, eu
que me sinto e me toco, poderei morrer? Tenho alguma dificuldade em
acreditar nisso: porque enfim, nada mais natural que os outros morram,
vemos isso todos os dias. Nós os vemos passar, habituamo-nos a isso,
mas morrer eu mesmo! Morrer em pessoa! É um pouco difícil. E vocês,
senhores, que tomam essas reflexões por algaravias, aprendam que esta é
a maneira de pensar em todo o mundo, e a sua própria também. Ninguém
pensa que deve morrer. Se existisse uma raça de homens imortais a ideia
da morte os assustaria mais que a nós.

Há nisso algo que não me explico. Como pode que os homens, sem cessar
agitados pela esperança e pelas quimeras do porvir, se inquietem tão                   
pouco pelo que este porvir lhes oferece de certo e inevitável? Não
seria a própria natureza benfazeja que nos teria dado essa feliz
despreocupação para que possamos cumprir em paz nosso destino? Acredito
de fato que podemos ser um homem muito bom sem ajuntar aos males reais
da vida esta inclinação do espírito que leva às reflexões lúgubres e
sem perturbar a imaginação com negros fantasmas. Enfim, acho que é
preciso se permitir o riso, ou ao menos o sorriso, todas as vezes que a
ocasião inocente se apresenta.

Assim acaba a meditação que me tinha inspirado o relógio de San Fillipo.
Eu a teria levado mais longe se não tivesse me aparecido certo
escrúpulo sobre a severidade da moral que acabava de estabelecer. Mas,
não querendo aprofundar esta dúvida, assobiei a ária das \textit{Folies
d’Espagne},\footnote{ Música de Marin Marais, bastante
popular durante o século \textsc{xviii}.} que tem a propriedade de mudar o
curso das minhas ideias quando elas caminham mal. O efeito foi tão
imediato que terminei na mesma hora meu passeio a cavalo.

\section*{Capítulo XXXVIII}

 Antes de entrar em meu quarto dei uma olhada na cidade e nos campos
escuros de Turim, que eu ia deixar talvez para sempre, e lhes enderecei
minhas últimas despedidas. Jamais a noite me pareceu tão bela, jamais o
espetáculo que tinha sob os olhos tinha me interessado tão vivamente.
Depois de ter saudado a montanha e o templo de La Superga, descansei
dos passeios, dos campanários, de todos os objetos conhecidos de que
jamais teria imaginado sentir tanta falta, e do ar e do céu, e do rio,
cujo murmúrio surdo parecia responder às minhas despedidas. Oh! Se eu
soubesse pintar o sentimento, alternadamente terno e cruel, que enchia
o meu coração e todas as lembranças que apareciam a minha volta, da
mais bela metade da minha vida já vivida, como diabinhos, para me reter
em Turim! Mas, ai de mim! As lembranças da felicidade passada são rugas
na alma! Quando somos infelizes é preciso expulsá-las do pensamento
como a fantasmas debochados que vêm insultar nossa situação presente: é
então mil vezes melhor abandonar-se às ilusões enganadoras da esperança
e sobretudo é preciso fazer cara boa à má sorte e preservar-se de fazer
confidência de suas infelicidades a alguém. Percebi, nas viagens comuns
que fiz entre os homens que de tanto sermos infelizes acabamos nos
tornando ridículos. Nestes momentos assustadores, nada é mais
conveniente que a nova maneira de viajar de que lemos a descrição. Fiz
então uma experiência decisiva: não somente alcancei esquecer o
passado, mas ainda tomei posição sobre as penas presentes. ``O tempo as
levará -- digo a mim mesmo para me consolar --, ele leva tudo e não esquece nada
em sua passagem, e mesmo se tentamos pará-lo, mesmo se o empurramos,
com o ombro, como se diz, nossos esforços são igualmente vãos e não
mudam nada de seu curso invariável.''

 Ainda que eu geralmente me inquiete muito pouco por sua rapidez, há
tais circunstâncias, tais filiações de ideias que me fazem recordar de
uma maneira intensa. É quando os homens se calam, quando o demônio do
barulho está mudo no meio de seu templo, no meio de uma cidade
adormecida, é então que o tempo eleva sua voz e se faz ouvir em minha
alma. O silêncio e a obscuridade tornam-se seus intérpretes e me
desvelam sua marcha misteriosa, não é mais um ser de razão que só pode
se apoderar de meu pensamento, mesmo meus sentidos o percebem. Eu o
vejo no céu caçando à sua frente as estrelas do ocidente. Aí está
empurrando os rios para o mar, as névoas ao longo da colina\ldots\ Escuto:
os ventos gemem sob o esforço de suas asas rápidas; e o sino distante
treme a sua passagem terrível.

 ``Aproveitemos, aproveitemos o seu curso -- eu gritava para mim mesmo. --
Quero empregar utilmente os instantes que ele vai levar de mim.''

 Querendo tirar partido desta boa resolução, no mesmo instante em que me
inclinava para frente para me lançar corajosamente na corrida, fazendo
com a língua um certo estalo que desde sempre foi destinado a impelir
os cavalos, mas que é impossível escrever segundo as regras da
ortografia: gh!     gh!     gh!
e terminei minha excursão a cavalo com uma galopada.

\section*{Capítulo XXXIX}

 Erguia meu pé direito para descer quando senti bater fortemente no
ombro. Dizer que não fiquei assustado seria trair a verdade e esta é a
ocasião de mostrar e provar ao leitor, sem muita vaidade, o quanto
seria difícil a qualquer outro que não eu executar uma tal viagem.
Supondo ao novo viajante mil vezes mais meios e talento para a
observação do que eu jamais tenha tido, poderia ele gabar-se de encontrar
aventuras tão singulares e também tão numerosas que as que me
aconteceram no espaço de quatro horas e que estão evidentemente ligadas
ao meu destino? Se alguém duvida, que tente adivinhar quem me bateu no
ombro!

 No primeiro momento do meu susto, não refletindo sobre a situação em
que me encontrava, acreditei que meu cavalo tinha escoiceado ou que ele
me tinha arremessado contra uma árvore. Deus bem sabe quantas ideias
funestas se apresentaram para mim durante o curto espaço de tempo que
levei para virar a cabeça e olhar para o meu quarto. Vi então, como
acontece com frequência com as coisas que parecem ser as mais
extraordinárias, que a causa de minha surpresa era completamente
natural. A mesma lufada de vento que, no começo da minha viagem, tinha
aberto minha janela e fechado minha porta ao passar, e de que uma parte
tinha escorregado entre as cortinas do meu leito, voltava agora ao meu
quarto com barulho: abriu bruscamente a porta e saiu pela janela,
empurrando a vidraça contra meu ombro, o que me causou a surpresa de
que acabo de falar.

 Lembramos que foi pelo convite que esse golpe de vento tinha me trazido
que eu tinha deixado minha cama. O empurrão que eu acabava de receber
era evidentemente um convite para entrar, o qual me considerei obrigado
a aceitar.

 É belo, sem dúvida, estar assim em uma relação familiar com a noite, o
céu e os meteoros, e saber tirar partido de sua influência. Ah! as
relações que somos forçados a ter com os homens são bem mais perigosas!
Quantas vezes não tenho sido logrado em minha confiança nesses
senhores! Falava disso aqui em uma nota que suprimi -- porque ela
ficou mais longa que o texto inteiro, o que teria alterado as justas
proporções de minha viagem, cujo pequeno volume é seu maior mérito.
\ \\

\hfil\textsc{fim da expedição noturna}




