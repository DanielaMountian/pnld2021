\chapter{A morte de Ivan Ilitch}

\section{I}

No grande edifício da Justiça, no intervalo da audiência do caso dos
Melvínski, os membros do tribunal e o promotor entraram no gabinete de
Ivan Iegórovitch Chébek, e a conversa discorreu a respeito do caso
Krássov. Fiódor Vassílievitch exaltou"-se ao demonstrar que não era da
competência da corte, Ivan Iegórovitch permaneceu com sua opinião, e
Piotr Ivánovitch, que não participava da discussão desde o início, não
intervinha, folheando o recém"-recebido \emph{Boletim.}

--- Senhores! --- disse. --- Ivan Ilitch morreu.

--- É mesmo?

--- Está aqui, leia --- disse Fiódor Vassílievitch, entregando"-lhe a edição
fresca, que ainda recendia: ``É com pesar no coração que Praskóvia
Fiódorovna Goloviná comunica a parentes e conhecidos o falecimento de
seu amado esposo, o membro da Câmara de Justiça\footnote{De acordo com a
  reforma de 1864, os tribunais comuns, encarregados das causas
  convencionais (excetuando as espirituais e militares) tinham duas
  instâncias: o tribunal distrital e a câmara judicial. Ivan Ilitch era
  membro do tribunal de segunda instância. {[}\textsc{n.\,e.}{]}} Ivan Ilitch
Golovin, ocorrido em 4 de fevereiro do ano corrente de 1882. O cortejo
sai na sexta"-feira, à uma da tarde''.

Ivan Ilitch fora colega dos senhores ali reunidos, e todos gostavam
dele. Já vinha doente há algumas semanas; dizia"-se que sua doença era
incurável. Sua vaga fora deixada em aberto, mas considerava"-se que, em
caso de morte, Aleksêiev poderia ser designado para seu lugar e, para o
lugar de Aleksêiev, Vínikov ou Chtábel. De modo que, ao ouvir sobre a
morte de Ivan Ilitch, o primeiro pensamento de cada um dos senhores
reunidos no gabinete foi que significado essa morte poderia ter para
transferência ou promoção deles mesmos ou de seus conhecidos.

``Agora provavelmente vou receber a vaga de Chtábel ou Vínikov --- pensou
Fiódor Vassílievitch. = Prometeram"-me há muito tempo, e essa promoção
vai significar 800 rublos de aumento para mim, fora a chancelaria''.

``Agora terei que pedir a transferência do cunhado de Kaluga --- pensou
Piotr Ivánovitch. --- Minha mulher vai ficar muito feliz. Agora não vai
mais poder dizer que nunca fiz nada por seus parentes''.

--- Eu achava mesmo que ele nunca ia se levantar --- disse Piotr
Ivánovitch, em voz alta. --- É triste.

--- Mas, em suma, o que ele tinha?

--- Os doutores não conseguiram determinar. Ou melhor, determinaram, mas
com divergências. Quando o vi pela última vez, tive a impressão de que
ia sarar.

--- Já eu não o visitava desde os feriados. Ficava sempre me preparando
para ir.

--- Então, ele tinha patrimônio?

--- Parece que a mulher tem muito pouco. Mas é algo insignificante.

--- Sim, vamos ter que ir. Eles moravam terrivelmente longe.

--- Quer dizer, é longe do senhor. Tudo é longe do senhor.

--- Não consegue me perdoar por morar do outro lado do rio --- disse Piotr
Ivánovitch, sorrindo para Chébek. Passaram a falar de outras distâncias
da cidade, e se encaminharam à audiência.

Além das considerações sobre transferências e possíveis mudanças no
trabalho que essa morte provocava em cada um, o próprio fato da morte de
um conhecido próximo provocou, como sempre, em todos que ficaram
sabendo, uma sensação de alegria devido a ter morrido ele, e não eu.

``Aquele ali morreu; mas eu não'' --- pensou ou sentiu cada um. Os
conhecidos próximos, os assim chamados amigos de Ivan Ilitch, além disso
pensaram a contragosto que agora deveriam cumprir obrigações de decoro
bastante aborrecidas, comparecer ao funeral e apresentar os pêsames à
viúva.

Os mais próximos de todos eram Fiódor Vassílievitch e Piotr Ivánovtich.

Pitor Ivánovitch fora colega na escola de Direito, e se sentia com
obrigações para com Ivan Ilitch.

Depois de informar à mulher, durante o jantar, da morte de Ivan Ilitch,
e de considerar a hipótese do deslocamento do cunhado para seu distrito,
Piotr Ivánovitch, sem se deitar para descansar, vestiu o fraque e foi à
casa de Ivan Ilitch.

Na entrada do apartamento de Ivan Ilitch havia uma carruagem e dois
cocheiros. Embaixo, na antessala, junto ao cabide, estava encostada na
parede a tampa do caixão, coberta de brocado, com borlas e galões
polidos. Duas damas de preto tiravam as peliças. Uma, irmã de Ivan
Ilitch, era conhecida; a outra, uma dama desconhecida. Um colega de Pior
Ivánovitch, Schwartz, vinha de cima e, ao ver, do degrau superior, o
recém"-chegado, deteve"-se e piscou para ele, como que dizendo: ``Ivan
Ilítch comportou"-se de forma estúpida; conosco, a coisa é diferente''.

O rosto de Schwartz, com suíças à inglesa, e toda sua figura magra, de
fraque, possuía, como sempre, uma solenidade elegante, e tal solenidade,
sempre em contraste com seu caráter brincalhão, adquiria aqui um sabor
especial. É o que pensava Piotr Ivánovitch.

Piotr Ivánovitch cedeu passagem às damas, seguindo"-as devagar, pela
escada. Schwartz não se pôs a descer, mas ficou em cima. Piotr
Ivánovitch entendeu porque: ele obviamente queria combinar onde jogar
uíste naquela noite. As damas subiram a escada até a viúva, enquanto
Schwartz, com os lábios apertados com firmeza e seriedade, e olhar
brincalhão, apontava para Piotr Ivánovitch, com um movimento de
sobrancelha, para a direita, para a câmara motuária.

Piotr Ivánovitch entrou, como sempre acontece, confuso a respeito de
como proceder. Só sabia que fazer o sinal da cruz, nesses casos, nunca
era demais. Não estava completamente seguro de que também seria
necessário se persignar e, por isso, escolheu o caminho do meio: ao
entrar no quarto, começou a fazer o sinal da cruz, e fez menção de se
persignar. Além disso, deu uma olhada no quarto, o quanto lhe permitiram
os movimentos do braço e da cabeça. Dois jovens, sendo um colegial --- ao
que parece, os sobrinhos ---, saíam do quarto fazendo o sinal da cruz. Uma
velha estava de pé, imóvel. E uma dama, com as sobrancelhas erguidas de
forma estranha, cochichava"-lhe algo. Um sacristão de sobrecasaca,
animado, decidido, lia algo em voz alta, com expressão que excluía
qualquer objeção; o auxiliar de copeiro Guerássim, passando por Piotr
Ivánovitch com passos leves, polvilhava algo no chão. Ao ver isso, Piotr
Ivánovitch imediatamente sentiu o leve cheiro de cadáver em
decomposição. Em sua última visita a Ivan Ilitch, Piotr Ivánovitch vira
esse auxiliar no gabinete; cumpria a função de enfermeiro, e Ivan Ilitch
gostava especialmente dele. Piotr Ivánovitch fazia o sinal da cruz o
tempo todo, inclinando"-se em uma direção intermediária entre o caixão, o
sacristão e as imagens que estavam na mesa do canto. Depois, quando
achou que já tinha feito por muito tempo o movimento de se benzer com a
mão, parou e se pôs a examinar o morto.

O morto jazia, como sempre jazem os mortos, de forma especialmente
pesada, como um morto, com os membros rígidos afogados no forro do
caixão, com a cabeça curvada para sempre para o travesseiro, e exibia,
como os mortos sempre exibem, sua testa amarela de cera com entradas
acima das têmporas afundadas e o nariz saliente, como que pressionando o
lábio superior. Mudara muito e até emagrecera desde que Piotr Ivánovitch
o vira mas, como em todos os mortos, seu rosto estava mais belo e,
principalmente, mais expressivo do que em vida. No rosto havia a
expressão de que aquilo que tivera que ser feito fora feito, e bem
feito. Além disso, nessa expressão havia também uma recriminação ou
lembrança aos vivos. Piotr Ivánovitch achou essa lembrança
despropositada ou, pelo menos, sem lhe dizer respeito. Algo o fez sentir
desagrado e, por isso, Piotr Ivánovitch voltou a fazer o sinal da cruz,
apressado e, em sua opinião, apressado demais, de forma indecorosa,
virou"-se e se encaminhou para a porta. Schwartz o aguardava no quarto de
passagem, de pernas bem abertas e com as mãos para trás, brincando com
sua cartola. O mero olhar para a figura brincalhona, asseada e elegante
de Schwartz revigorou Piotr Ivánovitch. Piotr Ivánovitch compreendeu que
ele, Schwartz, estava acima daquilo e não se entregaria a impressões
desalentadoras. Seu mero aspecto falava por si: o incidente do funeral
de Ivan Ilitch não podia servir de forma alguma como pretexto suficiente
para reconhecer a ruptura da ordem da sessão, ou seja, nada poderia
impedi"-lo, naquela noite, de fazer estalar, ao tirar do embrulho, um
maço de cartas, no momento em que um lacaio separava quatro velas novas;
em suma, não havia razão para supor que aquele incidente pudesse nos
impedir de passar a noite presente de forma agradável. Ele até cochichou
isso a Piotr Ivánovitch, de passagem, convidando"-o a se juntar à partida
na casa de Fiódor Vassílievitch. Porém, pelo visto, Piotr Ivánovitch não
estava destinado a jogar uíste naquela noite. Praskóvia Fiódorovna,
mulher baixa e gorda que, apesar de todos os esforços em contrário,
dilatava"-se assim mesmo abaixo dos ombros, toda de preto, de cabeça
coberta de renda e umas sobrancelhas erguidas de jeito tão estranho
quanto as da mulher que estava de pé na frente do caixão, saiu de seus
aposentos com outras damas e, levando"-as até a porta do morto, disse:

--- O serviço fúnebre vai ser agora; entrem.

Schwartz, com uma reverência vaga, ficou parado, sem aceitar ou recusar
a proposta de forma clara. Praskóvia Fiódorovna, reconhecendo Piotr
Ivánovitch, suspirou, encaminhou"-se até ele, tomou"-o pela mão e disse:

--- Sei que o senhor era um amigo de verdade de Ivan Ilitch --- e o fitou,
esperando que ele agisse de acordo com essas palavras.

Piotr Ivánovitch sabia que, assim como antes tivera que fazer o sinal da
cruz, agora tinha que apertar a mão, suspirar e dizer: ``Pode crer!'' E
o fez. E, ao fazê"-lo, sentiu que obtivera o resultado desejado: estava
tocado, e ela também.

--- Vamos antes que comece; preciso falar com o selhor --- disse a viúva. ---
Dê"-me o braço.

Piotr Ivánovitch deu"-lhe o braço, e eles se dirigiram para um quarto
interno, passando por Schwartz, que lhe lançou uma piscada triste:
``Olha só o nosso uíste! Não leve a mal, acharemos outro parceiro.
Talvez joguemos em cinco, quando se livrar'' --- dizia seu olhar
brincalhão.

Piotr Ivánovitch soltou um suspiro ainda mais profundo e triste, e
Praskóvia Fiódorovna apertou sua mão, agradecida. Ao entrar na sala de
visitas forrada de cretone rosa, com um candeeiro mortiço, sentaram"-se à
mesa: ela no sofá, e Piotr Ivánovitch em um pufe baixinho, de molas
desarranjadas, que ficou amassado de forma irregular quando ele se
acomodou. Praskóvia Fiódorovna quisera avisá"-lo para se sentar em outra
cadeira, mas achou que esse aviso não estava de acordo com sua situação
e desistiu. Ao tomar assento naquele pufe, Piotr Ivánovitch lembrou"-se
de como Ivan Ilitch decorara a sala, pedindo seu conselho a respeito do
cretone rosa com folhas verdes. Ao tomar assento no sofá e passar em
frente à mesa (em suma, toda a sala estava cheia de coisas e móveis), a
viúva prendeu a renda preta da mantilha preta no entalhe da mesa. Piotr
Ivánovitch se levantou para desenganchar e o pufe, livre dele, começou a
se agitar e a empurrá"-lo. A viúva se pôs a desenganchar a renda, e Piotr
Ivánovitch voltou a se sentar, esmagando o pufe rebelde. Só que a viúva
não desenganchou completamente, Piotr Ivánovitch voltou a se levantar, e
o pufe voltou a se rebelar, chegando a dar um estalo. Quando tudo isso
acabou, ela tirou um lenço limpo de cambraia e se pôs a chorar. O
episódio com a renda e a luta com o pufe esfriara Piotr Ivánovitch, que
permanecia sentado, carrancudo. A situação incômoda foi desfeita por
Sokolov, o copeiro de Ivan Ilitch, ao anunciar que o lugar no cemitério
designado por Praskóvia Fiódorovna custaria duzentos rublos. Ela parou
de chorar e, com ar de sacrifício, olhou para Piotr Ivánovitch, dizendo,
em francês, que era muito duro para ela. Piotr Ivánovitch fez um gesto
silencioso para exprimir sua certeza indubitável de que não podia ser de
outra forma.

--- Fume, por favor --- ela disse, com voz ao mesmo tempo magnânima e
alquebrada, e tratou da questão do preço do lugar com Sokolov. Acendendo
o cigarro, Piotr Ivánovitch ouvia"-a fazer um detalhado interrogatório
sobre os diversos preços de terra, determinando qual deveria pegar.
Depois de resolver o lugar, decidiu também sobre os cantores. Sokolov
partiu.

--- Faço tudo eu mesma --- ela disse a Piotr Ivánovitch, afastando para o
lado os álbuns que estavam na mesa; e, ao reparar que a cinza ameaçava a
mesa, empurrou sem tardar um cinzeiro para o interlocutor, dizendo: ---
Considero hipocrisia fazer crer que o pesar me impede de me ocupar de
assuntos práticos. Pelo contrário, se algo pode, não me consolar\ldots{} mas
distrair, são os cuidados para com ele. --- Voltou a tirar o lenço, como
que preparando"-se para chorar e, de repente, como que se dominando,
animou"-se e passou a falar, calma:

--- Contudo, tenho um assunto para tratar com o senhor.

Piotr Ivánovitch inclinou"-se, sem deixar escapar as molas do pufe, que
imediatamente começaram a se mexer embaixo dele.

--- Nos últimos dias, ele sofreu de forma terrível.

--- Sofreu muito? --- perguntou Piotr Ivánovith.

--- Ah, foi terrível! Não nos últimos minutos, mas nas últimas horas não
parava de gritar. Gritou por três dias seguidos, sem poupar a voz. Foi
insuportável. Não consigo entender como suportei isso; dava para ouvir
através de três portas. Ah! O que eu suportei!

--- Mas será que estava consciente? --- perguntou Piotr Ivánovitch.

--- Sim --- ela sussurrou ---, até os últimos minutos. Despediu"-se de nós um
quarto de hora antes da morte, e ainda pediu que trouxessem
Volódia\footnote{Apelido de Vladímir. {[}\textsc{n.\,t.}{]}}.

A ideia do sofrimento do homem que conhecera tão de perto, primeiro como
um menino feliz, um escolar, depois como parceiro adulto, horrorizou de
repente Piotr Ivánovitch, apesar da consciência desagradável de seu
próprio fingimento e do daquela mulher. Voltou a avistar aquela testa, o
nariz pressionando o lábio, e temeu por si.

``Três dias de sofrimentos horríveis e a morte. Afinal, agora mesmo, a
qualquer minuto, isso também pode me acontecer'', pensou, e teve um medo
instatâneno. Porém, imediatamente, sem que ele mesmo soubesse como,
ocorreu"-lhe o pensamento banal de que aquilo sucedera a Ivan Ilitch, e
não a ele, e de que aquilo não devia e não podia lhe suceder; de que,
pensando daquela forma, estava se entregando a um estado de espírito
sombrio, o que não era adequado fazer, como estava evidente no rosto de
Schwartz. E, fazendo tal raciocínio, Piotr Ivánovitch sossegou e passou
a inquirir com interesse detalhes a respeito do fim de Ivan Ilitch, como
se a morte fosse um incidente que dizia respeito apenas a Ivan Ilitch, e
não lhe dizia respeito de forma alguma.

Depois de uma conversa sobre os vários detalhes dos sofrimentos físicos
realmente horríveis de que Ivan Ilitch padeceu (detalhes de que Piotr
Ivánovitch só ficou sabendo porque o tormento de Ivan Ilitch agira sobre
os nervos de Praskóvia Fiódorovna), a viúva, obviamente, achou
necessário passar para o assunto.

--- Ah, Piotr Ivánovitch, como é duro, como é horrivelmente duro, como é
horrivelmente duro --- e voltou a chorar.

Piotr Ivánovitch suspirou e esperou"-a assoar o nariz. Depois de ela
assoar, ele disse:

--- Pode crer\ldots{} --- e ela voltou a falar e expôs o que era, evidentemente,
o principal assunto a tratar com ele; o assunto consistia em perguntar
como obter dinheiro do erário em caso de morte do marido. Fazia cara de
que estava pedindo a Piotr Ivánovitch um conselho sobre pensão, mas ele
via que ela já saiba os menores detalhes, inclusive que ele descohecia,
a respeito do que dava para arrancar do erário em caso de morte; o que
ela desejava saber, porém, era a possibilidade de arrancar ainda mais
dinheiro, de algum jeito. Piotr Ivánovitch esforçou"-se por imaginar
algum meio mas, depois de pensar um pouco e, por bom"-tom, criticar nosso
governo por sua avareza, disse que, ao que parecia, não dava para
arrancar mais. Daí ela suspirou e, obviamente, começou a imaginar um
meio de se livrar do visitante. Ele compreendeu, apagou a
\emph{papirossa}\footnote{Cigarro com boquilha de cartão. {[}\textsc{n.\,t.}{]}},
apertou a mão e se encaminhou para a antessala.

Na sala de jantar, com o relógio que Ivan Ilitch alegrava"-se tanto por
ter comprado em um bricabraque, Piotr Ivánovitch encontrou um sacerdote
e uns conhecidos que tinham vindo para o serviço fúnebre, avistando uma
bela moça que conhecia, a filha de Ivan Ilitch. Estava toda de preto.
Sua cinura, bastante fina, parecia ainda mais fina. Tinha um ar sombrio,
decidido, quase irritado. Inclinou"-se para Piotr Ivánovitch como se ele
fosse culpado de algo. Atras da filha estava, como o mesmo ar ofendido,
um jovem rico, juiz de instrução e seu noivo, pelo que ouvira dizer.
Fez"-lhes uma reverência triste e quis ir para a câmara mortuária quando,
debaixo da escada, apareceu a figura do filho, um colegial terrivelmente
parecido com Ivan Ilitch. Era o pequeno Ivan Ilitch, como Piotr
Ivánovitch o recordava na Escola Imperial de Direito. Seus olhos eram
chorosos e impuros, como acontece com meninos de treze, catorze anos. Ao
avistar Piotr Ivánovitch, o menino começou a se encrespar, severo e
envergonhado.

Piotr Ivánovitch meneou"-lhe a cabeça e entrou na câmara mortuária. O
serviço fúnebre começara: velas, lamentos, incenso, lágrimas, soluços.
Pior Ivánovitch ficou de cenho franzido, contemplando os pés que estavam
na sua frente. Não olhou nenhuma vez para o morto, até o fim não se
entregou às influências enfraquecedoras, e foi um dos primeiros a sair.
Não havia ninguém na antessala. Guerássim, o auxiliar de copeiro, surgiu
do quarto do defunto, revolvendo com as mãos fortes todas as peliças
para encontrar a de Piotr Ivánovitch, que entregou a ele.

--- E então, irmão Guerássim? --- disse Piotr Ivánovitch, para dizer alguma
coisa. --- Triste?

--- É a vontade de Deus. Todos chegaremos lá --- disse Guerássim,
arreganhando os dentes brancos e perfeitos de mujique e, na qualidade de
homem no auge de um trabalho intenso, abriu a porta com vivacidade,
chamou o cocheiro, acomodou Piotr Ivánovitch e pulou de volta para a
entrada da casa, como que pensando no que ainda tinha que fazer.

Piotr Ivánovitch achou especialmente agradável respirar ar puro depois
do cheiro de incenso, cadáver e ácido fênico.

--- Para onde manda? --- perguntou o cocheiro.

--- Não está tarde. Ainda vou passar na casa de Fiódor Vassílievitch.

E Piotr Ivánovitch foi. E realmente surpreendeu"-os depois do fim do
primeiro \emph{rubber}\footnote{Rodada constituída de três partidas
  separadas. {[}\textsc{n.\,t.}{]}}, de modo que pôde se juntar como quinto
jogador.

\section{II}

A história pregressa da vida de Ivan Ilitch era a mais simples e
corriqueira, e a mais terrível.

Ivan Ilitch morreu aos 45 anos, membro da Câmara de Justiça. Era filho
de um funcionário público, que, em diversos ministérios e departamentos,
fizera aquela carreira que leva às pessoas à situação na qual, embora
fique claro que não estão aptas a desempenhar qualquer função
significativa, não podem ser despedidas devido ao cargo e tempo de
serviço e, por isso, recebem postos fictícios por nada fictícios
milhares de rublos, de seis a dez, com os quais vivem até a mais
avançada velhice.

Assim era o conselheiro privado Ilia Efímovitch Golovin, membro
desnecessário de diversas instituições desnecessárias.

Teve três filhos. Ivan Ilitch era o segundo. O mais velho fez a mesma
carreira do pai, só que em outro ministério, e já estava perto da idade
funcional com a qual receberia aquela inércia de vencimentos. O terceiro
filho era um fracassado. Queimara"-se em diversos empregos e agora
trabalhava na estrada de ferro: seu pai, irmãos e, especialmente, as
mulheres deles não apenas não gostavam de encontrá"-lo, como não se
lembravam de sua existência, a não ser em caso de extrema necessidade. A
irmã se casara com o barão Gref, funcionário de São Petersburgo do mesmo
naipe de seu sogro. Ivan Ilitch era \emph{le phénix de la
famille}\footnote{A fênix da família, em francẽs no original. {[}\textsc{n.\,t.}{]}},
como diziam. Não era tão frio e cuidadoso quanto o mais velho, nem tão
desesperado quanto o caçula. Era o meio termo entre eles, uma pessoa
inteligente, animada, agradável e decente. Cursou Direito junto com o
irmão menor. O caçula não terminou e foi expulso do quinto ano, enquanto
Ivan Ilitch terminou o curso bem. Na Escola de Direito, já era como
seria posteriormente, ao longo de toda a vida: uma pessoa capaz, alegre,
bonachona e sociável, porém severa cumpridora do que considerava seu
dever; e considerava seu dever tudo que era assim considerado pelas
pessoas de posição elevada. Não fora bajulador nem em criança, nem
depois, adulto, só que, desde a juventude, sentia"-se atraído pelas
pessoas de posição social elevada, como uma mosca pela luz, assimilando
seus modos, sua visão de vida, e estabelecendo com elas relações de
amizade. Todas as paixões de infância e juventude passaram por ele sem
deixar grandes traços; entregou"-se à sensualidade, à vaidade e --- no fim
do curso, nos últimos anos --- ao liberalismo, mas sempre dentro de
limites conhecidos, que o instinto lhe apontava com correção.

Na Escola de Direito, cometeu atos que, anteriormente, pareciam"-lhe
muito torpes, e lhe suscitavam aversão por si mesmo no momento em que os
cometia; porém, em seguida, ao ver que tais atos eram cometidos também
por gente de posição elevada, que não os considerava ruins, não chegou a
reconhecê"-los como bons, porém os esqueceu por inteiro, e não se
amargurava ao recordá"-los.

Saindo da Escola de Direito com a décimo classe\footnote{Criada em 1722
  por Pedro o Grande, e em vigor até a Revolução de 1917, uma tabela de
  patentes, que ia até a décima quarta classe, regulava o serviço civil
  na Rússia tsarista. A décima classe correspondia ao grau de secretário
  colegiado. {[}\textsc{n.\,t.}{]}}, e tendo recebido do pai o dinheiro para o
uniforme, Ivan Ilitch encomendou uma indumentária a Charmer, pendurou no
berloque uma medalha com a inscrição \emph{respice finem}\footnote{Considera
  o fim, em latim no original. {[}\textsc{n.\,t.}{]}}, despediu"-se do
príncipe e dos professores, jantou com os colegas no Donon e de mala,
roupa de baixo, trajes, apetrechos de barbear, itens de toalete e uma
manta, tudo novo e da moda, encomendado e adquirido nas melhores lojas,
partiu para a província com o cargo de funcionário de encargos especiais
do governador, que o pai tinha lhe arranjado.

Na província, Ivan Ilitch logo estabeleceu o mesmo tipo de situação leve
e agradável de que desfrutara na Escola de Direito. Serviu, fez carreira
e, além disso, divertiu"-se de modo agradável e decente; às vezes, ia aos
distritos por incumbência da chefia, portando"-se com dignidade com
superiores e inferiores e desempenhando, com uma fidelidade e
integridade da qual não tinha como não se orgulhar, as incumbências que
lhe eram confiadas, predominantemente relacionadas a cismas religiosos.

Nos assuntos de serviço era, apesar da juventude e inclinação à alegria
ligeira, extraordinariamente contido, formal e até severo; em sociedade,
porém, era frequentemente brincalhão, afiado e sempre bonachão, decoroso
e \emph{bom enfant}\footnote{Bom menino, em francês no original. {[}\textsc{n.\,t.}{]}}, como diziam o chefe e sua mulher, para os quais ele era
da família.

Houve na província uma ligação com uma dama que se atirara em cima do
jurista janota; houve também uma modista; houve também pileques com
\emph{Flügeladjutants}\footnote{Oficiais do séquito do tsar. {[}\textsc{n.\,t.}{]}}
de passagem, e incursões a uma rua distante após o jantar; houve também
bajulação ao chefe e até à mulher do chefe, mas tudo isso foi conduzido
em um tom tão elevado de probidade que não seria possível definir com
palavras más: tudo isso ficava sob a rubrica da máxima francesa \emph{il
faut que jeneusse se passe}\footnote{Há que se perdoar a juventude , em
  francês no original. {[}\textsc{n.\,t.}{]}}. Tudo sucedeu com mãos limpas,
camisas limpas, palavras francesas e, o principal, na mais alta
sociedade e, consequentemente, com a aprovação das pessoas em posição
elevada.

Ivan Ilitch serviu desse modo durante cinco anos, quando veio uma
mudança no trabalho. Surgiram novas instituições judiciárias; novos
homens eram necessários.

E Ivan Iitch se tornou esse novo homem.

Ofereceram a Ivan Ilitch o cargo de juiz de instrução, e Ivan Ilitch o
aceitou, apesar desse posto ser em outra província, e de ele ter que
deixar as relações estabelecidas e estabelecer novas. Os amigos de Ivan
Ilitch se reuniram, ofereceram"-lhe uma cigarreira de prata, e ele partiu
para o novo posto.

Como juiz de instrução, Ivan Ilitch foi tão \emph{comme il
faut}\footnote{Como tem que ser, em francês no original. {[}\textsc{n.\,t.}{]}} e
decente quanto fora como funcionário de encargos especiais, sabendo
separar as obrigações de serviço da vida privada, o que suscitou
admiração geral. O próprio trabalho de juiz de instrução parecia a Ivan
Ilitch muito mais interessante e atraente que o anterior. No serviço
anterior, era agradável passar livremente de uniforme de Charmer pelos
visitantes trêmulos, que aguardavam para ser recebidos, e pelos
servidores que o invejavam enquanto ia direto ao gabinete do chefe, para
se sentar com ele para um chá com \emph{papirossa;} porém, as pessoas
que dependiam diretamente de seu arbítrio eram poucas. Essas pessoas
eram apenas os \emph{isprávniks}\footnote{Comissário de polícia
  distrital da Rússia tsarista. {[}\textsc{n.\,t.}{]}} e os cismáticos religiosos,
quando era enviado em missão; e ele gostava de tratar os que dependiam
de si com cortesia e quase camaradagem, gostava de fazer sentir que ele,
que tinha o poder de esmagar, tratava"-os de forma amigável e simples.
Naquela época, essas pessoas eram poucas. Mas agora, como juiz de
instrução, Ivan Ilitch sentia que todos, todos, sem exceção, as pessoas
mais importantes e autossuficientes estavam todas em suas mãos, e que
bastava"-lhe apenas escrever palavras conhecidas no papel timbrado e
aquela pessoa importante e autossuficiente lhe seria trazida na
qualidade de acusada ou testemunha e que, se não quisesse convidá"-la a
sentar, ela ficaria de pé diante dele, respondendo às suas perguntas.
Ivan Ilitch jamais abusou desse poder, tentando, pelo contrário,
suavizar sua expressão; porém, a consciência de tal poder e da
possibilidade de suavizá"-lo constituíam a coisa mais interessante e
atraente de seu novo serviço. No trabalho em si, precisamente na
instrução, Ivan Ilitch assimilou muito rápido os meios de afastar de si
todas as circunstâncias que não se referissem ao serviço, e de enquadrar
mesmo os casos mais complicados de uma forma que eles aparecessem no
papel apenas em seu aspecto externo, o que excluía completamente sua
opinião pessoal e, principalmente, observava todas as formalidades
exigidas. Isso era uma coisa nova. E ele foi um dos primeiros a aplicar
na prática os decretos de 1864\footnote{Regulamentada por decreto de 20
  de novembro de 1864, a reforma judicial de Alexandre \textsc{ii} estabelecia a
  igualdade das partes e criava tribunais de júri, audições públicas e a
  até então inexistente atividade de advogado profissional. {[}\textsc{n.\,t.}{]}}.

Ao mudar para uma nova cidade no posto de juiz de instrução, Ivan Ilitch
estabeleceu novos conhecimentos e ligações, assumiu outra postura e
adotou um tom algo diferente. Colocava"-se com certa distância digna dos
poderes da província, mas escolhera o melhor círculo da nobreza
judiciária e endinheirada que vivia na cidade, adotando um tom de leve
descontentamento para com o governo, de um liberalismo moderado e de
civismo civilizado. Além disso, sem mudar em nada a elegância de seu
vestir, Ivan Ilitch, no novo cargo, parou de depilar o queixo e deu à
barba liberdade para crescer como quisesse.

Na nova cidade, a vida de Ivan Ilitch também se arranjou de forma muito
agradável; a sociedade que se opunha ao governador era amigável e boa;
os vencimentos eram maiores, e um grande prazer à vida foi acrescido
pelo uíste, que Ivan Ilitch começou a jogar com alegria, raciocinando de
forma rápida e fina, de modo que, em geral, estava sempre entre os
vencedores.

Depois de dois anos de serviço na nova cidade, Ivan Ilitch encontrou sua
futura esposa. Praskóvia Fiódorovna Míkhel era a moça mais atraente,
inteligente e brilhante do círculo que Ivan Ilitch frequentava. Em meio
a outros passatempos e descansos do trabalho, Ivan Ilitch estabeleceu
relações brincalhonas e leves com Praskóvia Fiódorovna.

Enquanto funcionário de encargos especiais, Ivan Ilitch dançava
bastante; como juiz de instrução, porém, dançava como exceção. Dançava
também com a ideia que, embora estivesse nas novas instituições, e
pertencesse à quinta classe, no que tangia à dança podia demonstrar a
todos que também aí era melhor do que os outros. Assim, às vezes, no fim
da noite, dançava com Praskóvia Fiódorovna, e foi basicamente durante
essas danças que a conquistou. Ela se apaixonou por ele. Ivan Ilitch não
tinha o propósito claro e determinado de se casar, porém, quando a moça
se apaixonou por ele, fez"-se a seguinte pergunta: ``Afinal, por que não
me casar?''.

A moça Praskóvia Fiódorovna era de boa família nobre, nada feia; era de
compleição pequena. Ivan Ilitch poderia almejar a um partido mais
brilhante, mas esse partido era bom. Ivan Ilitch tinha seus rendimentos,
e esperava que ela tivesse o mesmo tanto. A família era boa; ela, uma
mulher querida, boazinha e totalmente direita. Dizer que Ivan Iitch
tinha se casado por amar sua noiva e por encontrar nela compreensão por
sua visão de vida teria sido tão injusto quanto dizer que se casara
porque as pessoas de seu círculo aprovavam o par. Ivan Ilitch se casara
por ambas as considerações: tinha prazer em adquirir uma mulher daquelas
e, além disso, estava fazendo aquilo que as pessoas de posição elevada
achavam certo.

E Ivan Ilitch se casou.

O próprio processo das bodas, os primeiros tempos da vida conjugal, com
os carinhos da esposa, nova mobília, nova louça, novo enxoval, passaram
muito bem, até a gravidez da mulher, de modo que Ivan Ilitch até começou
a pensar que o matrimônio não apenas não perturbava o caráter leve,
agradável, alegre e sempre decente e aprovado pela sociedade, que ele
considerava inerente à sua vida em geral, como ainda o acentuava. Mas
daí, nos primeiros meses de gravidez, surgiu algo de novo, inesperado,
desagradável, pesado e indecente, que não dava para esperar e de que não
havia como se livrar.

Na opinião de Ivan Ilitch, sem motivo algum --- \emph{de gaité de
coeur}\footnote{Por capricho, em francês no original. {[}\textsc{n.\,t.}{]}},
como ele dizia a si mesmo ---, a mullher começou a perturbar o prazer e a
decência da vida: sem razão alguma, tinha ciúmes dele, exigia"-lhe
galanteios, chateava"-se com tudo e fazia cenas desagradáveis e rudes.

No começo, Ivan Ilitch esperava se libertar do desprazer dessa situação
com a mesma relação ligeira e decente com a vida que o ajudara antes;
tentou ignorar o estado de espírito da mulher e seguir a viver com a
leveza e o prazer de antes, convidando amigos para partidas em casa,
ensejando visitar o clube ou camaradas. Porém, certa vez, a mulher
começou a insultá"-lo com palavras tão rudes, e continuou obstinadamente
a insultá"-lo a cada vez que ele não satisfazia suas exigências,
firmemente decidida, pelo visto, a não parar até que ele se resignasse,
ou seja, ficasse em casa, tão aborrecido quanto ela, que Ivan se
horrorizou. Compreendera que a vida conjugal --- pelo menos, com sua
esposa --- não coincidia sempre com uma vida agradável e decente mas,
pelo contrário, perturbava"-a com frequência e que, por isso, era
indispensável se proteger dessa perturbação. E Ivan Ilitch passou a
buscar meios para isso. O trabalho era a única coisa que infundia
respeito em Praskóvia Fiódorovna, e Ivan Ilitch, por meio do serviço e
das obrigações dele decorrentes, começou um embate com a mulher, em
defesa de seu mundo independente.

Com o nascimento do bebê, as tentativas de alimentá"-lo e seus diversos
fracassos, com as doenças reais e imaginárias do bebê e da mãe, nas
quais se exigia a participação de Ivan Ilitch, embora não entendesse
nada do assunto, sua necessidade de encontrar um mundo fora da família
tornou"-se ainda mais imperiosa.

À medida que a mulher se tornava mais irritadiça e exigente, Ivan Ilitch
transferia cada vez mais o centro de gravidade de sua vida para o
trabalho. Passou a amar mais o serviço, e ficou mais ambicioso do que
antes.

Muito rápido, não mais que um ano depois das bodas, Ivan Ilitch
compreendeu que a vida conjugal, embora apresentasse algum conforto à
vida, era na essência uma coisa muito complicada e dura, em relação à
qual, para cumprir seu dever, ou seja, comportar"-se de maneira decente e
passível de aprovação pela sociedade, era preciso elaborar uma relação
determinada, como acontecia com o trabalho.

E Ivan Ilitch elaborou esse tipo de relação com sua vida conjugal.
Exigia da vida conjugal apenas o jantar doméstico, a dona de casa, o
leito, confortos que ela podia lhe proporcionar e, principalmente, a
decência na aparência externa, que determinava a opinião da sociedade.
De resto, procurava o prazer alegre e, se o encontrava, ficava muito
grato; porém, se encontrava oposição e rabugice, imediatamente se
retirava para seu mundo isolado do trabalho, que criara e no qual
encontrava prazer.

Ivan Ilitch era tido como bom servidor e, em três anos, tornou"-se
procurador"-assistente. As novas obrigações, sua importância, a
possibilidade de levar a julgamento e colocar na cadeia qualquer um, os
discursos públicos, o êxito que obtinha nesses casos, tudo isso o atraía
ainda mais para o trabalho.

Vieram filhos. A mulher ficou ainda mais rabugenta e brava, mas as
relações com a vida doméstica elaboradas por Ivan Ilitch fizeram"-no
quase impenetrável à sua rabugice.

Depois de sete anos de serviço na mesma cidade, Ivan Ilitch foi
transferido para o posto de procurador, em outra província. Mudaram"-se;
o dinheiro era pouco, e a mulher não gostou do lugar para onde se
mudaram. Os vencimentos talvez fossem até maiores que os anteriores, mas
a vida era mais cara; além disso, dois filhos morreram, o que fez a vida
em família se tornar ainda mais desagradável para Ivan Ilitch.

Praskóvia Fiódorovna repreendia o marido por todas as adversidades
ocorridas no novo lugar de moradia. A maioria dos temas de conversa
entre marido e mulher, principalmente a educação dos filhos, levava a
questões que lembravam discussões, e as discussões a cada instante
estavam prestes a estourar. Restavam apenas alguns raros períodos de
paixão entre os esposos, mas não duravam muito. Eram ilhas nas quais
atracavam de tempos em tempos, para depois novamente se lançarem no mar
de animosidade secreta que se exprimia no alheamento de um do outro. Tal
alheamento poderia amargurar Ivan Ilitch se achasse que as coisas não
deveriam ser assim, mas ele agora reconhecia essa situação não apenas
como normal, mas ainda como o objetivo de sua atuação familiar. Seu
objetivo consistia em se libertar cada vez mais das contrariedades,
conferindo"-lhes caráter inócuo e decente; atingia"-o ao passar cada vez
menos tempo com a família ou, quando não era possível fazer isso,
esforçava"-se para garantir sua situação com a presença de forasteiros. O
principal era que Ivan Ilitch tinha o trabalho. No mundo do trabalho se
concentrava todo o interesse de sua vida. E tal interesse o absorvia. A
consciência de seu poder, a possibilidade de destruir qualquer pessoa
que quisesse destruir, a importância, mesmo exterior, de sua entrada no
tribunal e os encontros com os subordinados, seu sucesso com superiores
e subordinados e, o principal, sua maestria na condução dos casos de que
participava, tudo isso o alegrava e, junto com as conversas com os
camaradas, os jantares e o uíste, preenchia sua vida. De modo que, em
geral, a vida de Ivan Ilitch prosseguia como ele achava que tinha que
ser: agradável e decente.

Viveu assim por mais sete anos. A filha mais velha já tinha dezesseis,
morreu mais um bebê, e sobrou o menino, colegial, objeto de discórdia.
Ivan Ilitch queria encaminhá"-lo para a Escola de Direito, mas Praskóvia
Fiódorovna, de birra, matriculou"-o no colégio. A filha estudara em casa
e crescera bem, e o menino também era bom aluno.

\section{III}

Assim transcorreu a vida de Ivan Ilitch no decurso de 17 anos depois de
seu matrimônio. Já era um procurador veterano, recusara algumas
transferências à espera de um posto mais desejável quando,
inesperadamente, aconteceu uma circunstância desagradável, que
perturbaria completamente a tranquilidade de sua vida. Ivan Ilitch
aguardava o posto de juiz presidente em uma cidade universitária, mas
Goppé ultrapassou"-o e recebeu o cargo. Ivan Ilitch se zangou, repreendeu
o colega e brigou com ele e com os chefes mais próximos, que esfriaram e
o preteriram nas nomeações seguintes.

Isso ocorreu em 1880. Esse foi o ano mais duro da vida de Ivan Ilitch.
Nesse ano ficou evidente, por um lado, que os vencimentos não lhe
bastavam; por outro, que todos o haviam esquecido, e o que lhe parecia a
maior e mais cruel das injustiças era visto pelos outros como coisa
absolutamente comum. Nem o pai considerava sua obrigação ajudá"-lo.
Sentia que todos os abandonavam, considerando sua situação, com 3500
rublos de vencimento, a mais normal, e até feliz. Só ele sabia que, com
a consciência das injustiças que lhe haviam sido feitas, as eternas
azucrinações da mulher e as dívidas que passara a fazer, vivendo acima
de seus meios, sua situação estava longe de ser normal.

No verão daquele ano, para aliviar as finanças, tirou licença e foi
passar a estação com Praskóvia Fiódorovna, na aldeia do irmão dela.

Na aldeia, sem trabalho, Ivan Ilitch pela primeira vez sentiu não apenas
tédio, mas uma angústia insuportável, decidindo que não era possível
viver daquele jeito, fazendo"-se indispensável tomar algumas medidas
drásticas.

Depois de passar uma noite inteira de insônia passeando pelo terraço,
decidiu ir a São Petersburgo fazer solicitações e, para punir os que não
souberam valorizá"-lo, transferir"-se para outro ministério.

No dia seguinte, apesar de todas os argumentos dissuasivos da mulher e
do cunhado, partiu para São Petersburgo.

Foi atrás de uma coisa: obter um posto com cinco mil de vencimentos. Não
ligava para qual seria o ministério, direção ou tipo de atividade.
Precisava apenas de um posto, um posto de cinco mil, na administração,
no banco, na estrada de ferro, nas instituições da imperatriz Maria, até
na alfândega, mas tinham que ser cinco mil, e tinha que ser fora do
ministério em que não souberam valorizá"-lo.

E eis que essa viagem de Ivan Ilitch foi coroada por um êxito
surpreendente e inesperado. Em Kursk, subiu à primeira classe F. S.
Ilin, um conhecido, que o informou de um telegrama recente, recebido
pelo governador de Kursk, noticiando que, em dias, haveria uma
reviravolta no ministério: Ivan Semiônovitch seria nomeado para o lugar
de Piotr Ivánovitch.

A reviravolta prevista, além do significado para a Rússia, tinha
significado especial para Ivan Ilitch, já que, ao impulsionar uma pessoa
nova, Piotr Ivánovitch, e obviamente também seu amigo, Zakhar
Ivánovitch, era"-lhe favorável no mais alto grau. Zakhar Ivánovitch era
camarada e amigo de Ivan Ilitch.

Em Moscou, a notícia se confirmou. E, ao chegar em São Petersburgo, Ivan
Ilitch encontrou Zakhar Ivánovitch e recebeu a promessa de um posto
seguro em seu ministério, o da Justiça.

Na semana seguinte, telegrafou à esposa:

'Zahar substituiu Miller primeiro informe recebo nomeação'.

Graças a essa mudança de pessoal, Ivan Ilitch recebeu inesperadamente,
em seu ministério, uma nomeação que o colocava dois degraus acima de
seus camaradas: cinco mil de vencimentos e três mil e quinhentos de
ajuda de custo. Todo enfado com os inimigos de até então e o ministério
foi esquecido, e Ivan Ilitch ficou completamente feliz.

Ivan Ilitch regressou à aldeia alegre, satisfeito, como há muito tempo
não estava. Praskóvia Fiódorovna também se alegrou, e houve uma trégua
entre eles. Ivan Ilitch contou como todos o homenagearam na capital,
como todos que tinham sido seus inimigos foram humilhados e agora se
rebaixavam diante dele, como o invejavam por sua posição e,
especialmente, como todos gostavam muito dele em São Petersburgo.

Praskóvia Fiódorovna ouvia tudo e fazia cara de acreditar, sem o
contradizer em nada, limitando"-se a elaborar planos para a nova
organização da vida na cidade para a qual se mudariam. E Ivan Ilitch viu
com alegria que esses planos eram os seus planos, que eles coincidiam e
que sua vida titubeante voltaria a adquirir seu caráter verdadeiro e
próprio de prazer alegre e decoro.

Ivan Ilitch partiu em pouco tempo. Tinha que assumir a função em 10 de
setembro e, além disso, precisava de tempo para se estabelecer no novo
lugar, transportar tudo da província, comprar e encomendar muita coisa;
em resumo, organizar"-se do jeito que decidira em sua alma, quase
exatamente do mesmo jeito que fora decidido também na alma de Praskóvia
Fiódorovna.

E agora, quando tudo se arranjava de forma tão exitosa, eles coincidiam
nos objetivos e, além disso, passavam pouco tempo juntos, uniram"-se com
uma harmonia que não tinham encontrado nem nos primeiros anos de vida
nupcial. Ivan Ilitch pensara em levar a família imediatamente consigo,
porém a insistência da irmã e do cunhado, que de repente se fizeram
especialmente amorosos e cordiais para com Ivan Ilitch e sua família,
tanto fez que ele partiu sozinho.

Ivan Ilitch partiu, e o estado de espírito alegre, produto do êxito e da
concórdia com a mulher, um reforçando o outro, não o abandonou o tempo
todo. Achou um apartamento encantador, do jeito que marido e esposa
tinham sonhado. Salas de recepção amplas, altas, no estilo antigo, um
gabinete cômodo e grandioso, quartos para a mulher e a filha, sala de
estudos para o filho: tudo como se tivesse sido planejado especialmente
para eles. Ivan Ilitch ocupou"-se da decoração em pessoa, escolheu o
papel de parede, comprou os móveis, especialmente de estilo antigo, que
ele achava especialmente \emph{comme il faut,} a tapeçaria, e tudo
cresceu, cresceu e alcançou o ideal que ele havia estabelecido. Quando
chegou à metade da tarefa, o resultado ultrapassou sua expectativa.
Compreendeu o caráter \emph{comme il faut,} elegante e nada vulgar que
tudo adquiriria quando estivesse pronto. Ao se deitar, imaginava como
seria o salão. Contemplando a sala de visitas ainda não concluída já via
a lareira, a tela, a estante, umas cadeirinhas espalhadas, umas
travessas e pratos nas paredes, os bronzes, quando tudo estivesse no
lugar. Alegrava"-o a ideia de como surpreenderia Pacha e
Lízanka\footnote{Diminutivos de Praskóvia e Ielizavieta. {[}\textsc{n.\,t.}{]}},
que tinham o mesmo gosto. Jamais podiam esperar por isso. Em particular,
conseguiu encontrar e comprar barato coisas velhas, que conferiam a tudo
um caráter especialmente nobre. Nas cartas, descrevia de propósito tudo
pior do que era, para surpreendê"-los. Tudo isso o ocupava tanto que
mesmo o novo serviço, por mais que apreciasse, ocupava"-o menos que o
esperava. Nas audiências, tinha minutos de distração: pensava em como
seriam as cornijas das cortinas, retas ou curvas. Isso o ocupava tanto
que era frequente fazer as coisas ele mesmo, chegando a mudar os móveis
de lugar e pendurar as cortinas. Certa vez, ao subir numa escadinha para
mostrar ao estofador desentendido como queria o drapejado, tropeçou e
caiu, porém, forte e destro, segurou"-se, apenas batendo de lado no
puxador do caixilho. A contusão causou dor, que passou logo. Durante
todo esse tempo, Ivan Ilitch sentia"-se especialmente alegre e saudável.
Escreveu: sinto ter rejuvenescido quinze anos. Pensava terminar a obra
em setembro, mas prolongou"-se até meados de outubro. Em compensação,
ficou um encanto: não era só ele quem dizia, mas todos que viam.

Na verdade, havia ali o que acontece com todas pessoas que não são ricas
de verdade, mas querem parecer ricas e, dessa forma, só se parecem umas
com as outras: damasco, ébano, flores, tapetes e bronzes, escuros e
brilhantes --- tudo que todas as pessoas de um determinado tipo fazem
para se parecer com todas as pessoas de um determinado tipo. No caso
dele, tudo era tão imitativo que nem chamava a atenção; para Ivan
Ilitch, porém, parecia muito especial. Quando foi ao encontro dos seus
na estação ferroviária, trouxe"-os para o apartamento pronto e iluminado
e um lacaio de gravata branca descerrou a porta que dava para a
antessala adornada de flores, e depois eles entraram na sala de visitas,
no gabinete e exclamaram de satisfações, ele ficou muito feliz, guiou"-os
por toda parte, inebriando"-se com seus elogios e resplandecendo de
satisfação. Nessa mesma noite, quando, ao chá, Praskóvia Fiódorovna lhe
perguntou, entre outras coisas, como havia caído, ele se riu e encenou
para as pessoas como tinha saído voando e assustado o estofador.

--- Não é à toa que faço ginástica. Outro teria se matado, mas eu só me
machuquei um pouco aqui; quando tocado, dói, mas já vai passar; é só uma
equimose.

E começaram a viver na nova moradia, na qual, como sempre, quando
estavam bem habituados, sentiram falta de só mais um aposento, e com os
novos meios, nos quais, como sempre, sentiram falta de só um pouco mais
--- uns quinhentos rublos ---, e estavam muito bem. Foram particularmente
bons os primeiros tempos, quando nem tudo tinha sido arranjado, e era
necessário providenciar coisas: ora comprar, ora encomendar, ora mudar
de lugar, ora ajustar. Embora houvesse algumas discordâncais entre
marido e mulher, ambos estavam tão satisfeitos, e havia tanto a fazer,
que tudo acabava sem grandes discussões. Quando já não havia mais o que
providenciar, instaurou"-se um certo tédio e alguma insatisfação, mas
logo fizeram conhecidos e hábitos, e a vida se preencheu.

Ivan Ilitch, depois de passar a manhã no tribunal, voltava para almoçar
e, no começo, seu estado de espírito era bom, embora sofresse um pouco
justamente com a moradia. (Qualquer mancha na toalha, no damasco, um
cordão esfarrapado da cortina o irritavam: investira tanto trabalho na
decoração que qualquer perturbação lhe doía.) Porém, no geral, a vida de
Ivan Ilitch ia do jeito que, de acordo com sua crença, a vida devia
transcorrer: leve, agradável e decente. Levantava"-se às nove, tomava
café, lia o jornal, daí envergava o uniforme e partia para o tribunal.
Lá já esatva pronta a coleira com a qual trabalharia: submetia"-se a ela
sem demora. Os peticionários, os atestados da chancelaria, a própria
chancelaria, as audiências públicas e deliberativas. Em tudo isso era
necessário excluir tudo de úmido e real, que sempre perturba a correção
do curso dos assuntos profissionais: era preciso não admitir nenhuma
relação com as pessoas que não fosse de trabalho, o motivo das relações
tinha que ser apenas de trabalho, e a própria relação apenas de
trabalho. Por exemplo, vem uma pessoa e deseja saber algo. Não sendo de
sua competência, Ivan Ilitch não pode ter nenhuma relação com essa
pessoa; porém, se o assunto da pessoa tem algo a ver com o juiz, uma
relação que possa ser expressa em papel timbrado, Ivan Ilitch faz tudo,
decididamente tudo o possível dentro dos limites de tal relação e, ao
mesmo tempo, observa uma relação similar ao que seria humano e amistoso,
ou seja, é cortês. Assim que termina a relação de trabalho, termina todo
o resto. Essa habilidade de separar o lado profissional, sem misturá"-lo
com a vida real, Ivan Ilitch dominava no mais alto grau e, com muita
prática e talento, elaborara"-a a um grau tal que até, como um virtuose,
por vezes se permitia, como que por brincadeira, misturar as relações
humanas e de trabalho. Permitia"-se isso por sentir sempre em si a força
para, quando fosse necessário, voltar a destacar o profissional e
rejeitar o humano. Para Ivan Ilitch, era uma coisa que levava não apenas
de forma leve, agradável e decente, mas até com virtuosismo. Nos
intervalos fumava, tomava chá, conversava um pouco sobre política, um
pouco sobre assuntos gerais, um pouco sobre cartas e, mais do que tudo,
sobre nomeações. E cansado, mas com a sensação do virtuose que se
desincumbira de sua parte com distinção, um dos primeiros violinos da
orquestra, voltava para casa. Em casa, a filha e a mãe tinham ido para
algum lugar, ou alguém as visitara; o filho estava no colégio, preparava
as aulas com repetidores e estudava com assiduidade aquilo que se estuda
no colégio. Tudo ia bem. Depois do jantar, desde que não houvesse
visita, Ivan Ilitch às vezes lia um livro de que muito se falava e, à
noite, ocupava"-se de negócios, ou seja, lia papéis, informava"-se sobre
as leis; comparava depoimentos e os enquadrava nas leis. Não achava isso
nem chato, nem alegre. Era chato quando poderia estar jogando uíste;
mas, quando não, era sempre melhor do que ficar sozinho, ou com a
mulher. A satisfação de Ivan Ilitch eram os pequenos jantares, para os
quais chamava damas e cavalheiros de boa situação social, e esse jeito
de passar o tempo com eles parecia"-se com o jeito habitual de passar o
tempo dessas pessoas, assim como sua sala de visitas se parecia com
todas as salas de visitas.

Certa vez, tiveram até uma noite de danças. Ivan Ilitch ficou alegre, e
tudo andou bem, só que houve uma grande briga com a mulher por causa das
tortas e bombons: Praskóvia Fiódorovna tinha seu próprio plano, mas Ivan
Iitch insistiu que tudo fosse adquirido em uma confeitaria cara,
adquiriu muitas tortas, e a briga aconteceu porque sobraram tortas, e a
conta da confeitaria ficou em quarenta e cinco rublos. A briga foi
grande e desagradável, tanto que Praskóvia Fiódorovna lhe disse:
``Imbecil, molenga''. Ele agarrou a cabeça com as mãos e, num repente de
cólera, fez menção a divórcio. Mas a noite em si foi alegre. Estava"-se
na melhor sociedade, e Ivan Ilitch dançou com a princesa Trufônova, cuja
irmã é conhecida como fundadora da sociedade ``Remova a minha
desgraça''\footnote{Paródia do nome das muitas sociedades filantrópicas
  que surgiram na Rússia na década de 1880. {[}\textsc{n.\,e.}{]}}. As alegrias
profissionais eram alegrias do amor"-próprio; as alegrias sociais eram
alegrias da vaidade; as verdadeiras alegrias de Ivan Ilitch, porém, eram
as alegrias do jogo de uíste. Confessava que, depois de tudo, depois das
mais tristes ocorrências da vida, a alegria que ardia como uma vela na
frente de todas as outras era se sentar com bons jogadores, parceiros
discretos de uíste e, impreterivelmente em quatro (em cinco era muito
penoso na hora em que um tinha que ficar de fora, por mais que fingisse
gostar muito), empreender um jogo inteligente e sério (quando as cartas
saíam), depois cear e tomar um copo de vinho. E após o uíste,
especialmente quando tinha ganhado um pouco (muito é desagradável), Ivan
Ilitch ia dormir com humor especialmente bom.

Assim viviam. O círculo social que formaram era o melhor, frequentado
por gente importante e gente jovem.

Quanto à opinião sobre o círculo de conhecidos, marido, mulher e filha
estavam completamente de acordo e, sem combinar, afastavam do mesmo modo
e se liberavam de diversos conhecidos e parentes, uns pobretões que
acorriam, cheios de ternura, à sala de visitas, com travessas japonesas
na parede. Logo esses amigos pobretões pararam de acorrer, e os Golovin
ficaram apenas com a melhor sociedade. Os jovens cortejavam Lízanka e
Petríshev, filho de Dmitri Ivánovitch Petríschev, herdeiro único e juiz
de instrução, pôs"-se a cortejar Liza de tal forma que Ivan Ilitch já
falava com Praskóvia Fiódorovna sobre organizar para eles um passeio de
troica, ou um espetáculo. Assim viviam. E tudo ia assim, sem mudanças, e
tudo estava muito bem.

\section{IV}

Todos eram saudáveis. Não dava para chamar de doença aquilo que Ivan
Ilitch mencionava às vezes, de sentir um gosto estranho na boca e um
incômodo do lado esquerdo do ventre.

Aconteceu, porém, de esse incômodo aumentar, sem se transformar ainda em
dor, mas na consciência de um peso constante do lado e em mau humor.
Esse mau humor, cada vez mais e mais forte, começou a estragar a
harmonia que se instalara na vida leve e decente da família Golovin.
Marido e mulher passaram a brigar com frequência cada vez maior, logo a
leveza e a harmonia se foram, mantendo"-se apenas a decência, e com
dificuldade. As cenas voltaram a se tornar frequentes. Voltaram a sobrar
apenas algumas ilhotas, e poucas, nas quais marido e mulher podiam se
encontrar sem explosão.

E agora Praskóvia Fiódorovna tinha motivo para dizer que o caráter do
marido era difícil. Com o hábito de exagerar que lhe era peculiar, dizia
que ele sempre tivera aquele caráter horrível, que conseguira suportar
por vinte anos devido à sua bondade. A verdade era que, agora, as brigas
começavam por causa dele. As implicâncias sempre se desencadeavam na
hora do jantar, habitualmente quando começava a tomar a sopa. Ora notava
algum estrago na louça, ora que a comida não estava boa, ora que o filho
estava com o cotovelo na mesa, ora era o penteado da filha. E culpava
Praskóvia Fiódorovna por tudo. No começo, Praskóvia Fiódorovna retrucava
e lhe dizia coisas desagradáveis, porém, por duas vezes, no começo da
refeição, sua fúria fora tamanha que ela percebeu se tratar de uma
condição doentia provocada pela alimentação e sossegou; não retrucava
mais, apenas acelerava o jantar. Praskóvia Fiódorovna atribuía"-se grande
mérito por sua resignação. Ao decidir que o marido tinha um caráter
difícil e fazia de sua vida uma desgraça, passou a ter pena de si mesma.
Quanto mais pena tinha de si, mais odiava o marido. Passou a desejar que
ele morresse, mas não podia desejar isso porque, daí, não haveria
vencimentos. Isso a deixava com ainda mais raiva. Achava"-se
terrivelmente infeliz justamente porque nem a morte dele poderia
salvá"-a; ficava com raiva, ocultava"-a, e essa ocultação fazia a raiva
ficar ainda mais forte.

Depois de uma cena na qual Ivan Ilitch fora especialmente injusto e, ao
se explicar, dissera que estava mesmo irritado devido a uma doença, ela
lhe disse que, se estava doente, tinha que se tratar, exigindo que fosse
atrás de um médico famoso.

Ele foi. Tudo aconteceu conforme o esperado; tudo foi como sempre é. A
espera, a seriedade afetada do doutor, que ele conhecia, pois era a
mesma que empregava no tribunal, as batidas, a auscultação, as perguntas
que exigiam respostas preparadas de antemão e, obviamente, supérfluas, o
ar de importância, a dizer, o que é isso, basta se sujeitar a nós e
damos um jeito em tudo, nós sabemos, sem dúvida, como dar um jeito em
tudo, sempre do mesmo jeito para qualquer pessoa. Tudo aconteceu
exatamente como no tribunal. O médico assumiu exatamente o mesmo ar que
ele assumia perante os réus.

O doutor disse: isso e isso mostra que dentro do senhor há isso e isso;
porém, caso não seja confirmado pelo exame tal e tal, então supõe"-se que
o senhor provavelmente tem aquilo e aquilo. Ao supor que seja isso,
então\ldots{} etc. Para Ivan Ilitch, só uma questão era importante: sua
situação era grave ou não? O médico, porém, ignorava essa questão
descabida. Do ponto de vista do doutor, tal questão era ociosa e indigna
de ser debatida, e havia apenas a avaliação das possibilidades: rim
solto, catarro crônico e doença no ceco. Não havia questão sobre a vida
de Ivan Ilitch, mas havia questão entre o rim solto e o ceco. E tal
discussão o médico resolveu de forma brilhante, diante dos olhos de Ivan
Ilitch, em favor do ceco, com a ressalva de que o exame de urina poderia
trazer novas provas e, daí, o caso seria reestudado. Tudo isso era,
ponto a ponto, idêntico ao que Ivan Ilitch realizara milhares de vezes
na frente dos réus, de forma iguamente brilhante. O doutor fez o seu
resumo com o mesmo brilho, e solene, até alegre, fitou o acusado por
cima dos óculos. Do resumo do médico, Ivan Ilitch concluiu que estava
mal, que isso era indiferente para o doutor e, talvez, para todo mundo,
mas estava mal. Tal conclusão afetou Ivan Ilitch de forma dolorosa,
causando uma grande pena de si mesmo e um grande ódio pelo doutor
indiferente a uma questão tão importante.

Porém, sem dizer nada, levantou"-se, colocou o dinheiro na mesa e,
suspirando, disse:

--- É provável que nós, pacientes, façamos muitas perguntas descabidas ---
disse. --- Em suma, é uma doença grave ou não?

O doutor lançou"-lhe um olhar severo através dos óculos, como que
dizendo: réu, caso não se atenha aos limites das perguntas que lhe foram
feitas, serei constrangido a tomar medidas para sua remoção da sala de
audiências.

--- Eu já lhe disse o que considero necessário e adequado --- disse o
médico. --- O restante será demonstrado pelo exame. --- E se inclinou.

Ivan Ilitch saiu devagar, sentou"-se triste no trenó e foi para casa. Por
todo o caminho não parou de rever tudo que o doutor tinha tido, tentando
transformar todo aquele palavreado confuso e científico em língua
simples, e ler a resposta à pergunta: estou mal, muito mal, ou não é
nada? E teve a impressão de que a ideia de tudo que fora dito pelo
doutor é de que estava muito mal. Ivan Ilitch achou tudo na rua triste.
Os cocheiros eram tristes, as casas eram tristes, os transeuntes, as
lojas eram tristes. Aquela dor surda, abafada, que não parava por um
segundo, parecia, devido às palavras obscuras do médico, ter ganho outro
significado, mais sério. Ivan Ilitch agora prestava atenção nela com um
sentimento novo e carregado.

Chegou em casa e começou a contar à mulher. A mulher ouvia, porém, no
meio do relato, entrou a filha, de chapéu: preparava"-se para sair com a
mãe. Com esforço, sentou"-se para ouvir aquela chatice, mas não aguentou
muito, e a mãe também não ouviu até o fim.

--- Bem, estou muito feliz --- a mulher disse ---, agora você fique de olho e
tome o remédio direito. Dê"-me a receita, vou mandar Guerássim para a
farmácia. --- E foi se vestir.

Ele não tinha tomado alento enquanto a esposa estava no quarto, e
respirou pesadamente quando ela saiu.

--- Pois bem --- disse ele. --- Talvez não seja mesmo nada\ldots{}

Começou a tomar o remédio e seguir as prescrições médicas, que haviam
sido alteradas devido ao exame de urina. Só que logo aconteceu uma certa
confusão entre esse exame e o que devia se seguir a ele. Não era
possível chegar até o doutor, e tinha a impressão de que não estava
fazendo o que lhe fora prescrito. Ou tinha esquecido, ou o médico
mentido, ou escondido alguma coisa.

Contudo, Ivan Ilitch se pôs a seguir a prescrição assim mesmo,
encontrando consolo nisso nos primeiros tempos.

A principal ocupação de Ivan Ilitch na época da consulta era cumprir
exatamente as recomedações médicas referentes à higiene, tomada de
medicamentos e observação da dor e todas as funções do organismo. Os
principais interesses de Ivan Ilitch eram as doenças e a saúde das
pessoas. Quando falavam na sua frente de doentes que tinham morrido ou
sarado, especialmente de moléstias parecidas com a sua, procurava
esconder a emoção e apurava o ouvido, interrogava e fazia associações
com o seu mal.

A dor não diminuía; Ivan Ilitch, contudo, fazia um esforço para se
obrigar a pensar que estava melhor. E conseguia se iludir quando nada o
irritava. Mas bastava ter uma contrariedadde com a mulher, um fracasso
no trabalho, cartas ruins no uíste que logo sentia a plena força de sua
doença; antes suportava esses fracassos, esperando que logo consertaria
o que estava mal, triunfaria, obteria êxito, um \emph{grand
slam}\footnote{No uíste, vencer todas as treze rodadas (\emph{tricks)}
  de uma mão do jogo. {[}\textsc{n.\,t.}{]}}. Agora, qualquer fracasso o
prostrava e o precipitava no desespero. Dizia para si mesmo: logo agora
que eu comecei a melhorar e o remédio passou a fazer efeito veio esse
maldito insucesso e contrariedade\ldots{} E ficava raivoso com o insucesso,
ou com as pessoas que lhe causavam contrariedades e o matavam, sentido
que essa raiva o matava; não conseguia, porém, se esquivar dela.
Aparentemente, deveria ver com clareza que essa exasperação com as
circunstâncias e as pessoas piorava sua moléstia e, por isso, não tinha
que prestar atenção às eventualidades dessagradáveis; contudo, fazia o
raciocínio completamente oposto: dizia precisar de tranquilidade e
ficava de olho em tudo que perturbava tal tranquilidade, irritando"-se
com a menor perturbação. Agravava sua situação o fato de que lia livros
médicos e se aconselhava com doutores. Esse agravamento era regular, de
modo que ele podia se iludir ao comparar um dia com o outro: as
diferenças eram pequenas. Porém, ao se aconselhar com médicos, tinha a
impressão de que piorava, e muito rápido. E, apesar disso,
aconselhava"-se com frequência.

Naquele mês, esteve com outra celebridade: a outra celebridade disse
quase a mesma coisa que a primeira, só que colocando as questões de
forma diferente. Aconselhar"-se com essa celebridade só fez acentuar a
dúvida e o medo de Ivan Ilitch. O amigo de um amigo --- um médico muito
bom --- diagnosticou a doença de forma totalmente diversa e, embora
prometesse a cura, suas perguntas e previsões deixaram Ivan Ilitch ainda
mais confuso, e intensificaram sua dúvida. Um homeopata formulou outro
diagnóstico diferente e prescreveu um remédio que Ivan Ilitch tomou por
uma semana, em segredo. Contudo, passada a semana, sem sentir melhora, e
tendo perdido a confiança tanto no tratamento prévio, como no atual,
abandonou"-se a um desânimo ainda maior. Certa vez, uma dama que conhecia
contou de uma cura por ícones. Ivan Ilitch surpreendeu"-se ouvindo"-a com
atenção e tentando verificar a realidade do fato. Essa ocorrência o
assustou. ``Será que fiquei tão fraco da cabeça?'' --- disse para si. ---
``Bobagem! É tudo absurdo, não devo me entregar a cismas e, uma vez que
escolhi um médico, tenho que me aferrar a seu tratamento. É o que vou
fazer. Agora acabou. Não vou pensar, e seguirei o tratamento de forma
severa, até o verão. Daí veremos! Agora é o fim das vacilações!..'' Era
fácil de falar, mas impossível de cumprir. A dor no flanco atormentava o
tempo todo, como se estivesse cada vez mais forte, tornara"-se
permanente, o gosto na boca ficara ainda mais estranho, tinha a
impressão de emitir mau hálito, e o apetite e as forças se debilitavam.
Não dava para se iludir: algo terrível, novo e muito importante, mais
importante do que qualquer coisa em sua vida estava acontecendo com Ivan
Ilitch. E só ele sabia disso, todos ao redor não entendiam, ou não
queriam entender, e pensavam que tudo continuava como antes. Isso era o
que mais torturava Ivan Ilitch. Via que as pessoas de casa ---
especialmente a mulher e a filha, que estavam no auge das saídas --- não
entendiam nada, agastavam"-se por ele estar tão macambúzio e exigente,
como se fosse culpado. Embora até tentassem ocultá"-lo, ele via como era
um estorvo, só que a mulher forjara uma certa atitude com relação à sua
doença, que mantinha independente do que ele dissesse ou fizesse. A
relação era a seguinte:

--- Vocês sabem --- dizia aos conhecidos --- que Ivan Ilitch não consegue,
como todas as pessoas de bem, seguir as prescrições do tratamento à
risca. Hoje ele toma as gotas, come o que foi recomendado e dorme na
hora; amanhã, de repente, se eu me distraio, esquece o remédio, come
esturjão (que não é recomendado) e fica jogando uíste até uma da manhã.

--- Ah, quando foi isso? --- dizia Ivan Ilitch, com enfado. --- Uma vez, na
casa de Piotr Ivánovitch.

--- E ontem, com Chébek.

--- Tudo bem, a dor não me deixava mesmo dormir.

--- Seja o que for, só que assim você nunca vai sarar e vai ficar nos
atormentando.

A relação evidente de Praskóvia Fiódorovna com a doença do marido, dita
aos outros e a ele mesmo, era de que Ivan Ilitch tinha culpa, e de que
essa doença era mais uma contrariedade que ele causava à mulher. Ivan
Ilitch sentia que isso saía dela sem querer, o que não facilitava as
coisas para ele.

No tribunal, Ivan Ilitch notou, ou achou ter notado, a mesma relação
estranha para consigo: ora tinha a impressão de que o encaravam como uma
pessoa que logo liberaria o cargo; ora, de repente, os colegas começavam
a brincar amistosamente com sua cisma, como se aquela coisa horrenda,
medonha e inaudita que se criara dentro dele, devorando"-o sem cessar e
arrastando"-o de forma irresistível sabe"-se lá para onde, fosse o mais
agradável pretexto para piada. Quem o irritava particularmente era
Schwartz, com seu jeito brincalhão, cheio de vitalidade e \emph{comme il
faut,} que remetia Ivan Ilitch a si mesmo, dez anos antes.

Os amigos vieram jogar uma partida, acomodaram"-se. Amaciaram as cartas
novas, distribuíram"-nas, separaram as de ouros, ele tinha sete. O
parceiro disse: sem trunfos, e o apoiou com dois ouros. O que mais?
Devia ficar alegre e animado: \emph{slam.} E de repente Ivan Ilitch
sentia aquela dor devoradora, aquele gosto na boca, e achava absurdo,
nessas circunstâncias, se alegrar com um \emph{slam.}

Olhou para o parceiro, Mikhail Mikháilovitch, batendo na mesa com a mão
sanguínea e, cortês e indulgente, absteve"-se de recolher a vaza,
empurrando"-a para Ivan Ilitch para lhe dar a satisfação de apanhá"-la sem
ter o trabalho de esticar muito o braço. ``O que ele está pensando, que
estou tão fraco que não consigo esticar muito o braço?'' --- pensou Ivan
Ilitch, esquecendo os trunfos, gastando um deles à toa e perdendo o
\emph{slam} por três, e o mais terrível de tudo era ver o sofrimento de
Mikhail Mikháilovitch, enquanto para ele dava na mesma. E era horrendo
pensar por que dava tudo na mesma.

Todos viram seu padecer e disseram: ``Podemos parar se estiver cansado.
Descanse''. Descansar? Não, não estava nem um pouco cansado, e
terminaram o \emph{rubber.} Todos estavam sombrios e calados. Ivan
Ilitch sentiu que fora ele quem instaurara essas sombras, e não tinha
como dissipá"-las. Eles cearam e partiram, e Ivan Ilitch ficou sozinho,
com a consciência de que sua vida estava envenenada, de que envenenava a
vida dos outros, e de que esse veneno não enfraquecia, penetrando com
cada vez mais força em todo o seu ser.

Com consciência disso e, ainda por cima, a dor física e, ainda por cima,
o horror, tinha que se deitar na cama e, frequentemente, por causa da
dor, ficava sem dormir a maior parte da noite. E, pela manhã, tinha que
se levantar de novo, vestir"-se, ir até o tribunal, falar, escrever e, se
não fosse, ficar em casa as vinte e quatro horas do dia, das quais cada
uma era uma tortura. E tinha que viver dessa forma, à beira do abismo,
sozinho, sem nenhuma pessoa que o compreendesse e tivesse compaixão.

\section{V}

Assim passou um mês, depois outro. Antes do Ano Novo, o cunhado veio à
cidade para se hospedar com eles. Ivan Ilitch estava no tribunal.
Praskóvia Fiódorovna fora às compras. Ao entrar no escritório, o dono da
casa surpreendeu"-o saudável, sanguíneo, desfazendo a própria mala. O
cunhado ergueu a cabeça ao ouvir os passos de Ivan Ilitch e fitou"-o por
um segundo, calado. Tal olhar revelou tudo a Ivan Ilitch. O cunhado
abriu a boca para exclamar, mas se conteve. Tal movimento confirmava
tudo.

--- O que foi, estou mudado?

--- Sim\ldots{} há uma mudança.

E por mais que depois induzisse o cunhado a falar de sua aparência
física, este permanecia em silêncio. Praskóvia Fiódorovna chegou, e o
cunhado ter com ela. Ivan Ilitch trancou a porta a chave e passou a se
fitar no espelho --- de frente, depois de lado. Pegou seu retrato com a
mulher e comparou com o que via no espelho. A mudança era imensa. Depois
arregaçou as mangas até o cotovelo, sentou"-se na otomana e ficou mais
sombrio do que a noite.

``Não dá, não dá'', disse para si mesmo, levantou"-se de um pulo, foi até
a mesa, abriu um caso, tentou ler, mas não conseguiu. Abriu a porta e
foi até o salão. A porta da sala de visitas estava trancada.
Aproximou"-se na ponta dos pés e começou a escutar.

--- Não, você está exagerando --- dizia Praskóvia Fiódorovna.

--- Exagerando como? Você não vê? É um homem morto, olhe"-o nos olhos. Não
há luz. Mas o que ele tem?

--- Ninguém sabe. Nikoláiev (um outro médico) disse uma coisa, mas eu não
sei. Leschetítski (um médico famoso) disse o contrário\ldots{}

Ivan Ilitch se afastou, foi até seu quarto, deitou"-se e começou a
pensar. ``O rim, o rim se mexe''. Lembrava"-se de tudo que os doutores
haviam lhe dito, como ele se soltara e como estava se mexendo. E, com um
esforço de imaginação, tentava agarrar esse rim e detê"-lo, fixá"-lo;
tinha a impressão de precisar de muito pouco. ``Não, ainda vou atrás de
Piotr Ivánovitch''. (Aquele amigo que tinha um amigo médico.) Soou a
campainha, mandou atrelar os cavalos e se preparou para sair.

--- Para onde vai, Jean? --- perguntou a mulher, com uma expressão
especialmente triste e de uma rara bondade.

A rara bondade o exasperou. Lançou"-lhe um olhar sombrio.

--- Preciso ir à casa de Piotr Ivánovitch.

Foi à casa do amigo que tinha um amigo médico. E foi com ele até o
médico. Encontrou"-o e conversou longamente.

Considerando os detalhes anatômicos e fisiológicos do que, na opinião do
doutor, acontecera com ele, Ivan Ilitch entendeu tudo.

Havia uma coisinha, uma coisinha pequenina no ceco. Tudo isso dava para
arranjar. Fortalecendo a energia de um órgão, enfraquecendo a atividade
de outro, aconteceria a absorção, e tudo estaria arranjado. Atrasou"-se
um pouco para o jantar. Jantou, falou com alegria, mas ficou um bom
tempo sem conseguir se encaminhar para suas tarefas. Por fim, foi até o
gabinete e se pôs imediatamente a trabalhar. Lia os casos, trabalhava,
porém a consciência de que adiava um importante assunto íntimo que o
ocuparia ao fim da tarefa não o abandonava. Ao encerrar seus casos,
lembrou"-se de que o assunto íntimo era a ideia do ceco. Não se entregou,
porém, a ela, indo à sala de visitas tomar chá. Havia convidados,
falava"-se, tocava"-se piano, cantava"-se; estava lá o juiz de instrução, o
desejado noivo da filha. Praskóvia Fiódorovna chegou a reparar que Ivan
Ilitch passava uma noite mais alegre do que as outras, mas ele não se
esquecia por um minuto de que adiava pensamentos importantes a respeito
do ceco. Às onze, despediu"-se e se recolheu a seu quarto. Durante a
doença, dormia sozinho, em um quartinho do lado do gabinete. Despiu"-se e
pegou um romance de Zola mas, em vez de ler, pensava. Em sua imaginação,
acontecia a desejada cura do ceco. Absorvia, expelia, reestabelecia"-se a
atividade correta. ``Sim, é tudo desse jeito --- dizia para si. --- Basta
apenas ajudar a natureza''. Lembrou"-se do remédio, levantou"-se, tomou,
deitou"-se de costas esperando por seu efeito salutar de aniquilar a dor.
``Basta tomar o remédio com regularidade e evitar influências nocivas;
agora já estou me sentindo um pouco melhor, muito melhor''. Pôs"-se a
apalpar o flanco; não doía. ``Bem, não estou mais sentindo, é verdade,
estou muito melhor''. Apagou a luz e se deitou de lado\ldots{} O ceco estava
curando, absorvia. De repente sentiu a velha conhecida, a dor surda,
abafada, obstinada, silenciosa, séria. Na boca, aquela mesma porcaria já
sabida. O coração foi sugado e a cabeça se turvou. ``Meu Deus, meu Deus!
--- proferiu. --- De novo, de novo, e não para nunca''. E, de repente, o
assunto se lhe apresentou por um ângulo completamente diferente. ``O
ceco! O rim --- disse para si mesmo. --- Não é uma questão de ceco, nem de
rim, mas de vida e\ldots{} morte. Sim, havia vida, está partindo, partindo, e
não tenho como detê"-la. Sim. Para que me enganar? Afinal, é evidente
para todos, menos para mim, que vou morrer, tratando"-se de uma questão
apenas do número de semanas ou dias --- pode até ser agora. Pois havia a
luz, e agora são trevas. Pois eu estava aqui, e agora vou para lá! Para
onde?'' Um frio o percorreu, a respiração parou. Ouvia apenas as batidas
do coração.

``Não existirei, e o que existirá? Nada existirá. E para onde vou quando
não existir mais? Será a morte? Nã, não quero''. Levantou"-se de um
salto, quis acender a vela, procurou com as mãos trêmulas, deixou
castiçal e vela caírem no chão e voltou a desabar para trás, no
travesseiro. ``Por quê? Dá na mesma --- falava para si, fitando a
escuridão com os olhos abertos. --- A morte. Sim, a morte. E ninguém sabe,
nem quer saber, nem tem pena. Estão tocando. (Escutava o estrondo de
vozes e ritornelos ao longe, atrás da porta.) Para eles dá na mesma, só
que também vão morrer. Imbecis. Eu antes, eles depois; só que eles
também. E ficam felizes. Bestas!'' O ódio o afogava. E passou a sentir
um tormento pesado e insuportável. Não era possível que todos estivessem
sempre condenados a esse medo horrendo. Levantou"-se.

``Há algo de errado; tenho que me acalmar, tenho que recapitular tudo de
novo''. E começou a recapitular. ``Sim, o começo da doença. Uma batida
de lado, e deu tudo na mesma, naquele dia e no seguinte; doeu um pouco,
depois mais, depois os doutores, depois a tristeza, a angústia, os
doutores de novo; e eu mais perto, cada vez mais perto do abismo. Menos
forças. Mais perto, mais perto. E agora definhei, não tenho luz nos
olhos. É a morte, mas eu fico pensando no ceco. Fico pensando em
consertar o ceco, só que é a morte. Será a morte?'' O terror voltou a se
apossar dele que, ofegante, curvou"-se, pôs"-se a procurar os fósforos e
bateu com o cotovelo no criado mudo. Este o atrapalhava e machucava, de
modo que ele se enfureceu, apoiou"-se com mais força e o derrubou.
Desesperado e ofegante, desabou de costas, esperando a morte imediata.

Nessa hora, as visitas estavam indo embora. Praskóvia Fiódorovna as
acompanhava. Ouviu uma queda e entrou.

--- O que você tem?

--- Nada. Deixei cair sem querer.

Ela saiu e trouxe uma vela. Ele estava deitado, com a respiração pesada
e rápida, como um homem que tinha corrido uma versta\footnote{Antiga
  medida russa equivalente a 1,067 km. {[}\textsc{n.\,t.}{]}}, fitando"-a com os
olhos parados.

--- O que você tem, Jean?

--- Na\ldots{}da. Dei\ldots{}xei ca\ldots{}ir. --- ``O que você está dizendo? Ela não vai
entender'' --- pensou.

Exatamente: não entendeu. Levantou"-se, acendeu a vela para ele e saiu
apressada: precisava acompanhar as visitas.

Quando regressou, ele estava deitado de costas, olhando para cima.

--- O que você tem, piorou?

--- Sim.

Ela abanou a cabeça e se sentou.

--- Sabe, Jean, penso se não é o caso de chamar Leschetítski aqui em casa.

Isso significava chamar um médico famoso e não poupar dinheiro. Ele deu
um sorriso venenoso e disse: ``Não''. Ela ficou sentada, achegou"-se e o
beijou na testa.

Odiava"-a com todas as forças da alma enquanto era beijado, e fez um
esforço para não afastá"-la.

--- Adeus. Se Deus quiser, você vai dormir.

--- Sim.

\section{VI}

Ivan Iitch viu que estava morrendo, e seu desespero era constante.

No fundo da alma, Ivan Ilitch sabia que estava morrendo, porém não
apenas não se acostumava a isso, como simplesmente não entendia, não
podia compreender de forma alguma.

Aquele exemplo de silogismo que aprendera na lógica de
Kiesewetter\footnote{Johann Gottfried Kiesewetter (1766--1819), filósofo
  alemão, seguidor e propagandista de Kant, autor de muitas obras,
  dentre as quais um manual de lógica traduzido em russo. Nele
  apresenta"-se como exemplo o seguinte silogismo: ``Caio é uma pessoa,
  as pessoas são mortais, logo Caio é mortal''. Caio é Caio Júlio César.
  {[}\textsc{n.\,e.}{]}} --- Caio é uma pessoa, as pessoas são mortais, logo Caio é
mortal --- por toda sua vida lhe parecera certo apenas com relação a
Caio, mas jamais a si mesmo. Pois Caio era uma pessoa, uma pessoa em
geral, e isso era totalmente justo; mas ele não era Caio, nem uma pessoa
em geral, e sempre fora em tudo, em tudo diferente de todas as outras
criaturas; fora Vânia\footnote{Diminutivo de Ivan. {[}\textsc{n.\,t.}{]}}com a
mamãe, o papai, Mítia e Volódia\footnote{Diminutivos de Dmitri e
  Vladímir. {[}\textsc{n.\,t.}{]}}, com os brinquedos, o cocheiro, a babá, depois
com Kátienka\footnote{Diminutivo de Iekatierina. {[}\textsc{n.\,t.}{]}}, com todas
as alegrias, pesares, entusiasmos da infância, juventude, mocidade. Por
acaso para Caio havia aquele cheiro da bolinha de couro listrada, de que
Vânia tanto gostava? Por acaso Caio beijava daquele jeito as mãos da
mãe, e por acaso era para Caio que as pregas de seda do vestido da mãe
farfalhavam daquele jeito? Por acaso ele protestara por causa de uns
\emph{pirojkí}\footnote{Espécie de pãezinhos recheados, assados ou
  fritos. {[}\textsc{n.\,t.}{]}} na Escola de Direito? Por acaso Caio se apaixonara
tanto? Por acaso Caio podia conduzir audiências daquela forma?

Caio é realmente mortal, e é justo que morra, mas eu, Vânia, Ivan
Ilitch, com todos meus sentimentos e ideias, sou outra coisa. Não pode
ser que me aconteça de morrer. Isso seria horrível demais.

Ele se sentia assim.

`` Se eu tivesse que morrer como Caio, então eu saberia, então minha voz
interior me diria, só que não me ocorreu nada de parecido; eu e todos
meus amigos entendemos que não temos nada a ver com Caio. Mas agora,
olha só! --- dizia a si mesmo. --- Não pode ser. Não pode ser, mas é. Como
assim? Como entender isso?''

Ele não conseguia entender e tentava expulsar esse pensamento mentiroso,
injusto, doentio, substituindo"-o por outros, justos e saudáveis. Só que
essa ideia não era apenas uma ideia, mas a realidade, que retornava e
parava na sua frente.

E convocava, um depois do outro, no lugar desses pensamentos, outros, na
esperança de neles encontrar apoio. Tentou regressar à antiga cadeia de
ideias que anteriormente protegia"-o da ideia da morte. Porém --- coisa
estranha --- tudo que antes protegia, ocultava, aniquilava a consciência
da morte agora não mais conseguia fazê"-lo. A maior parte dos últimos
tempos Ivan Ilitch passava nessas tentativas de reestabelecer a cadeia
prévia de sentimentos que protegiam da morte. Ora se dizia: ``Vou me
ocupar com o trabalho, já que vivia para ele''. E ia para o tribunal,
enxotando todas as dúvidas; participava das conversas dos colegas e,
seguindo o velho hábito, passava um olhar distraído e meditativo pela
multidão, agarrando com ambas as mãos descarnadas os braços da poltrona
de carvalho, também como de costume inclinava"-se para o colega, agitando
o processo, cochichando e depois, soerguendo os olhos de repente e se
aprumando, proferia as palavras determinadas e dava início ao caso.
Porém, de repente, no meio do processo, sem prestar nenhuma atenção ao
período de evolução do caso, a dor no flanco começava o \emph{seu}
processo de sucção. Ivan Ilitch se concentrava, expulsava"-a de sua
mente, mas \emph{ela} continuava seu curso, vinha, parava na frente dele
e o encarava; ele ficava petrificado, a luz de seus olhos se extinguia,
e voltava a se perguntar: ``Será que apenas \emph{ela} está certa?'' E
camaradas e subordinados, com assombro e desgosto, viam que aquele juiz
tão brilhante e fino se enganara, cometera um erro. Ele se sacudia,
tentava voltar a si e, de alguma forma, levava a audiência até o fim,
retornando para casa com a triste consciência de que, com os afazeres
jurídicos, não conseguia mais, como antes, esconder de si o que desejava
esconder; que, com os afazeres jurídicos, não tinha como se livrar
\emph{dela.} E o pior de tudo é que \emph{ela} o distraía não para que
ele fizesse outra coisa, mas apenas para que ele a encarasse, direto nos
olhos, encarasse"-a e, sem fazer nada, se atormentasse de forma
indescritível.

E, para se salvar dessa situação, Ivan Ilitch buscava conforto, outros
biombos, e outros biombos se formavam e pareciam salvá"-lo por pouco
tempo, mas logo voltavam não exatamente a desmoronar, mas a ficar
transparentes, como se \emph{ela} se infiltrasse através de tudo, e nada
pudesse detê"-\emph{la}.

Nos útimos tempos, acontecia de entrar na sala que decorara, naquela
sala em que caíra e --- como o ridículo de pensar naquilo o envenenava ---
por cuja construção sacrificara a vida, pois sabia que sua doença
começara com aquela contusão, e ver que na mesa de verniz havia um
vergão, um corte. Buscava a causa e a encontrava no enfeite de bronze de
um álbum deixado de lado. Pegava o álbum caro, que compusera com amor, e
se agastava com o desleixo da filha e suas amigas: aqui estava rasgado,
ali os retratos estavam virados. Colocava tudo cuidadosamente em ordem,
voltando a dobrar o enfeite.

Depois vinha"-lhe a ideia de transferir todo esse
\emph{établissement}\footnote{Estabelecimento, em francẽs no original.
  {[}\textsc{n.\,t.}{]}}com os álbuns para outro canto, perto das flores. Chamava
um lacaio: filha ou mulher acorriam em seu auxílio; não concordavam,
contradiziam, ele discutia, ficava bravo;; mas tudo estava bem, pois não
se lembrava \emph{dela,} não \emph{a} via.

Mas aí a mulher dizia, quando ele estava fazendo a mudança: ``Perdão,
deixe os outros fazerem, você vai acabar se machucando de novo'', e de
repente \emph{ela} cintilava através do biombo, ele \emph{a} avistava.
\emph{Ela} cintilava, ele ainda tinha a esperança de que \emph{ela} se
esconderia mas, sem querer, ele prestava atenção ao flanco, e lá estava,
igual, doía igual, ele já não podia esquecer, e \emph{ela} o fitava
claramente por trás das flores. Para que isso tudo?

``A verdade é que aqui, nessa cortina, eu perdi a vida, como numa
sortida. Será? Que terrível, e que estúpido! Não pode ser! Não pode ser,
mas é''.

Entrava no gabinete, deitava"-se e ficava de novo sozinho com \emph{ela.}
Olho no olho, mas não havia nada a fazer. Só olhar para \emph{ela} e
gelar.

\section{VII}

Como isso aconteceu no terceiro mês da doença de Ivan Ilitch não dava
para dizer, pois aconteceu passo a passo, mas o fato é que a mulher, a
filha, o filho, a criadagem, os médicos e, principalmente, ele mesmo,
sabiam que o único interesse dele para os outros consistia apenas em
quando finalmente desocuparia o lugar, libertaria os outros do
constrangimento causado por sua presença e a si mesmo de seus
sofrimentos.

Dormia cada vez menos; ministravam"-lhe ópio, e começaram a injetar
morfina. Só que isso não o aliviava. A angústia embotada que
experimentava semiadormecido só o aliviara no começo, como novidade,
para depois tornar"-se tão aflitiva quanto a dor em si, ou até mais.

Preparavam"-lhe uma alimentação especial, de acordo com a prescrição
médica; só que toda essa alimentação lhe parecia cada vez mais insípida
e repugnante.

Também foram feitas adaptações especiais para sua evacuação, que era um
tormento a cada vez. Um tormento devido à imundície, à indecência, ao
cheiro e à consciência de que outra pessoa tinha que participar daquilo.

Porém, foi na coisa mais desagradável que surgiu o consolo de Ivan
Ilitch. Quem sempre vinha limpar era o auxiliar de copeiro Guerássim.

Guerássim era um mujique jovem, limpo, com frescor, que engordara com a
dieta da cidade. Sempre alegre, radiante. No começo, o aspecto sempre
limpo daquele homem vestido à russa a realizar uma tarefa nojenta
deixava Ivan Ilitch embaraçado.

Certa vez, levantando"-se da comadre, sem forças para erguer as
pantalonas, desabou na poltrona macia e, horrorizado, contemplou as
coxas nuas e impotentes, com os músculos bem destacados.

De botas grossas, espalhando ao redor o aroma agradável de alcatrão e o
frescor do ar invernal, Guerássim entrou com passo ligeiro e firme,
avental limpo de cânhamo e camisa limpa de chita, mangas arregaçadas nos
braços nus, fortes e jovens e, sem olhar para Ivan Ilitch ---
contendo"-se, obviamente, para não ultrajar o doente com a alegria de
viver que irradiava em seu rosto ---, foi até a comadre.

--- Guerássim --- disse Ivan Ilitch, débil.

Guerássim se sobressaltou, obviamente com medo de ter feito alguma
besteira, e virou para o doente, com um movimento rápido, o rosto
fresco, bom, simples e jovem, no qual uma barba apenas começara a
crescer.

--- O que deseja?

--- Acho que isso deve lhe ser desagradável. Desculpe"-me. Eu não consigo.

--- Perdão, senhor. --- Os olhos de Guerássim brilharam, e ele arreganhou os
dentes brancos e jovens. --- Por que eu não me daria ao trabalho? O seu
caso é de doença.

Com mãos ágeis e fortes, cumpriu sua tarefa habitual e saiu, com passos
leves. Cinco minutos depois, com passo igualmente leve, voltou.

Ivan Ilitch continuava sentado na poltrona.

--- Guerássim --- disse, quando o outro colocou a comadre limpa e lavada ---,
por favor, ajude"-me a sair daqui. --- Guerássim foi. --- Levante"-me. É
difícil para mim sozinho, e eu mandei Dmitri embora.

Guerássim foi; com os braços fortes, e a mesma leveza de passos,
abraçou"-o, ergueu"-o de forma suave e hábil, sustentou"-o, puxou as
pantalonas com a outra mão e quis assentá"-lo. Ivan Ilitch, porém, pediu
para ser colocado no sofá. Sem esforço, e como se não apertasse,
Guerássim amparou"-o, quase carregando, e o acomodou no sofá.

--- Obrigado. Como você faz tudo\ldots{} bem, e com habilidade. Guerássim
voltou a sorrir, e fez menção de sair. Porém, Ivan Ilitch se sentia tão
bem com ele que não desejava liberá"-lo.

--- É o seguinte: aproxime, por favor, essa cadeira. Não, aquela, embaixo
dos meus pés. Fico mais aliviado com os pés para cima.

Guerássim aproximou a cadeira, ajeitou"-a sem barulho, baixando"-a no chão
de uma vez, e levantou as pernas de Ivan Ilitch; Ivan Ilitch teve a
impressão de ficar mais aliviado quando Guerássim ergueu"-lhe os pés.

--- Sinto"-me melhor com os pés para cima --- disse Ivan Ilitch. --- Ponha
aqui mais uma almofada.

Guerássim o fez. Voltou a levantar os pés e colocou a almofada. Ivan
Ilitch voltou a se sentir melhor quando Guerássim segurou"-lhe as pernas.
Quando elas foram largadas, teve a impressão de ficar pior.

--- Guerássim --- disse ---, você está ocupado agora?

--- De jeito nenhum, senhor --- disse Guerássim, que aprendera com as
pessoas da cidade a falar com os patrões.

--- O que você ainda tem que fazer?

--- O que tenho que fazer? Já fiz tudo, só falta rachar lenha para amanhã.

--- Então você poderia me levantar as pernas de novo?

--- Posso, como não. --- Guerássim levantou"-lhe as pernas, e Ivan Ilitch
teve a impressão de, naquela posição, não sentir dor alguma.

--- E a lenha, como fica?

--- Não se preocupe, por favor. Vamos dar um jeito.

Ivan Ilitch ordenou a Guerássim que se sentasse e lhe segurasse as
pernas, e passou a falar com ele. E --- coisa estranha --- teve a
impressão de se sentir melhor quando Guerássim lhe segurava as pernas.

Daí por diante, Ivan Ilitch passou a chamar Guerássim de vez em quando,
fazia"-o colocar suas pernas nos ombros e adorava conversar com ele.
Guerássim realizava"-o com leveza, de bom grado, de forma simples e com
uma benevolência que enternecia Ivan Ilitch. A saúde, força e vitalidade
de todos os outros ofendiam Ivan Ilitch; apenas a força e vitalidade de
Guerássim não o agastavam, e sim tranquilizavam.

O principal tormento de Ivan Ilitch era a mentira, aquela mentira por
alguma razão adotada por todos de que ele estava apenas doente, não
moribundo, e de que precisava apenas ficar tranquilo e se tratar, e
então tudo daria muito certo. Sabia, porém, que não importava o que
fizesse, nada sairia dali além de sofrimentos ainda mais torturantes e a
morte. E aquela mentira o atormentava por não desejarem admitir o que
todos, inclusive ele, sabiam, e que quisessem mentir a respeito da
gravidade terrível de sua situação, e que quisessem obrigá"-lo a
participar de tal mentira. Aquela mentira, uma mentira que lhe era
imposta na véspera de sua morte, uma mentira destinada a rebaixar o
terrível ato solene de sua morte ao nível de todas as visitas, cortinas,
jantares com esturjão\ldots{} era uma tortura horrível para Ivan Ilitch. E ---
estranho --- muitas vezes, quando lhe pregavam essas peças, ficava por um
fio de gritar: parem de mentir, vocês sabem e eu sei que estou morrendo,
então parem, pelo menos, de mentir. Mas jamais teve ânimo de fazê"-lo.
Via que todos que o rodeavam rebaixavam o ato medonho e horrendo de seu
falecimento ao nível de uma contrariedade casual, meio sem decoro (como
uma pessoa que, ao entrar na sala de visitas, começa a exalar mau
cheiro), aquele mesmo ``decoro'' ao qual servira por toda a vida; via
que ninguém se compadecia dele, pois ninguém queria sequer entender sua
situação. Apenas Guerássim compreendia a situação e tinha pena. Era por
isso que Ivan Ilitch estava bem com Guerássim. Sentia"-se bem quando
Guerássim segurava suas pernas, às vezes a noite inteira e, sem querer
dormir, dizia: ``Por favor, não se incomode, Ivan Ilitch, depois eu
durmo''; ou quando, subitamente mudando o tratamento para ``você'',
acrescentava: ``Se você não estivesse doente, tudo bem, mas, como está,
por que não cuidar?'' Só Guerássim não mentia, era evidente em tudo que
compreendia do que se tratava, não achava necessário esconder, e
simplesmente tinha pena do patrão fraco, a definhar. Chegou a dizer,
direto, quando Ivan o despachou:

--- Vamos todos morrer. Por que não me esforçar? --- disse, exprimindo que
não se incomodava com seu trabalho justamente porque o fazia para um
moribundo, esperando que, quando chegasse a sua hora, alguém também o
fizesse por ele.

Além da mentira, ou em consequência dela, o que mais atormentava Ivan
Ilitch era que ninguém se compadecia dele como desejaria que se
compadecesse: nos minutos subsequentes a um longo sofrimento, o que Ivan
Ilitch mais desejava, embora se envergonhasse de admitir, era que alguém
tivesse pena dele como de uma criança doente. Desejava ser acariciado,
beijado e pranteado do jeito que acariciam e tranquilizam as crianças.
Sabia que era um servidor importante, de barba grisalha e que, portanto,
isso era impossível; desejava"-o, porém, assim mesmo. Porém, nas relações
com Guerássim, havia algo perto disso e, por esse motivo, as relações
com Guerássim o consolavam. Ivan Ilitch tinha vontade de chorar, tinha
vontade de ser acariciado e pranteado, e daí vinha um colega, o servidor
Chébek e, em vez de choros e carinhos, Ivan Ilitch fazia uma cara séria,
severa, de profunda meditação e, por inércia, exprimia sua opinião a
respeito do significado de um recurso de apelação, aferrrando"-se
obstinadamente a ela. Essa mentira ao seu redor e em si mesmo foi o que
mais envenenou os últimos dias da vida de Ivan Ilitch.

\section{VII}

Era manhã. Só era manhã porque Guerássim saíra e viera o lacaio Piotr,
apagara as luzes, abrira uma cortina e começara a arrumação baixinho.
Fosse manhã, noite, sexta"-feira, domingo, tudo dava na mesma, tudo era
apenas uma coisa: a dor surda e torturante, que não sossegava por um
instante; a consciência sem esperança de uma vida sempre a fugir, mas
que ainda não se fora; a morte terrível e odiosa, sempre a avançar, que
era a única realidade, e toda aquela mentira. Para que então os dias,
semanas e horas do dia?

--- Não vai pedir chá?

``Ele precisa de ordem, de que os senhores tomem chá de manhã'' ---
pensou, e apenas disse:

--- Não.

--- Não tem vontade de passar para o sofá?

``Ele precisa colocar ordem no cômodo e eu atrapalho, sou a sujeira e a
desordem'' --- pensou, e apenas disse:

--- Não, deixe"-me.

O lacaio continuou sua movimentação. Ivan Ilitch esticou o braço. Piotr
aproximou"-se, solícito.

--- O que deseja?

--- O relógio.

Piotr pegou o relógio, que estava ao alcance da mão, e o entregou.

--- Oito e meia. Lá não se levantaram?

--- Não senhor. Vassili Ivánovitch (era o filho) foi para o colégio, e
Praskóvia Fiódorovna mandou que a acordassem, se o senhor pedisse.
Deseja?

--- Não, não precisa. --- ``Por que não provar o chá?'' --- pensou. --- Sim, o
chá\ldots{} traga.

Piotr se encaminhou para a saída. Ivan Ilitch teve medo de ficar
sozinho. ``Como retê"-lo? Sim, o remédio''. --- Piotr, dê"-me o remédio. ---
``Por que não, talvez o remédio ainda ajude''. Pegou a colher, ingeriu.
``Não, não ajuda. Tudo isso é um absurdo, um engano --- decidiu, ao
sentir o gosto conhecido, adocicado e desesperado. --- Não, não consigo
mais ter fé. Mas a dor, a dor, se pelo menos por um minuto ela
sossegasse''. E se pôs a gemer. Piotr regressou. --- Não, vá. Traga o chá.

Piotr saiu. Sozinho, Ivan Ilitch passou a gemer não tanto de dor, por
mais terrível que ela fosse, quanto de angústia. ``Tudo é igual, sempre
igual, esses dias e noites sem fim. Que seja rápida. Que quem seja
rápida? A morte, as trevas. Não, não. Tudo é melhor do que a morte!''

Quando Piotr retornou com o chá em uma bandeja, Ivan Ilitch pôs"-se a
examiná"-lo longamente, confuso, sem entender quem e o que ele era. Piotr
ficou desconcertado com esse olhar. E, quando Pitor ficou desconcertado,
Ivan Ilitch voltou a si.

--- Sim --- disse ---, o chá\ldots{} muito bem, pode servir. Só me ajude a me
lavar e botar uma camisa limpa.

E Ivan Ilitch começou a se lavar. Com pausas de descanso, lavou as mãos,
o rosto, limpou os dentes, penteou"-se e se olhou no espelho. Ficou com
medo: o mais assustador eram os cabelos achatados e apertados contra a
testa pálida.

Quando lhe trocaram a camisa, sabia que sentiria ainda mais medo se
examinasse o próprio corpo, e não se olhou. Mas tudo estava acabado.
Vestiu um avental, cobriu"-se com uma manta e se sentou à poltrona, para
o chá. Sentiu"-se aliviado por um minuto, mas bastou começar a tomar o
chá e de novo aquele mesmo gosto, aquela mesma dor. Acabou de beber com
dificuldade e se deitou, esticando as pernas. Deitou"-se e despachou
Piotr.

Tudo sempre igual. Ora cintilava uma gota de esperança, ora se
encarpelava um mar de desespero, e sempre a dor, sempre a dor, sempre a
angústia e tudo sempre único e igual. Sozinho, a angústia é terrível,
quer chamar alguém, mas sabe de antemão que, na frente dos outros, é
ainda pior. ``Talvez morfina de novo, para esquecer. Vou dizer a ele, ao
doutor, que pense em mais alguma coisa. É impossivel, impossível
assim''.

Passa assim uma hora, duas. E daí uma campainha na antessala. Pode ser o
doutor. Exatamente, é o médico, viçoso, animado, gordo, alegre com
aquela expressão de que ``alguma coisa andou assustando"-o, mas agora
vamos dar um jeito em tudo''. O doutor sabe que aqui essa expressão não
é adequada, mas já a adotou de uma vez por todas e não pode despi"-la,
como um homem que vestiu um fraque de manhã e saiu para fazer visitas.

O doutor esfrega as mãos animado e tranquilizador.

--- Estou gelado. Faz um frio de rachar. Deixem"-me esquentar --- diz, com
uma expressão de que basta deixá"-lo se esquentar um pouco que daí ele
vai consertar tudo.

--- Pois bem, e então?

Ivan Ilitch sente que o médico tem vontade de dizer ``como vão os
negócios?'', porem, ao sentir que não dá para falar assim, diz ``como
passou a noite?''

Ivan Ilitch olha para o doutor com a expressão de quem pergunta: ``será
que você nunca tem vergonha de mentir?'' O médico, porém, não deseja
entender a pergunta.

E Ivan Ilitch diz:

--- Terrível como sempre. A dor não passa, não se rende. Se houvesse algum
jeito!

--- Ah, sim, vocês, os pacientes, são sempre assim. Bem, agora, ao que
parece, eu me aqueci, e nem a asseadíssima Praskóvia Fiódorovna teria
nada a objetar contra minha temperatura. Pois bem, olá. --- E o doutor
aperta a mão.

E, deixando de lado toda a brejeirice anterior, o médico começa, com ar
sério, a examinar o doente, o pulso, a temperatura, dando início às
batidas e auscultações.

Ivan Ilitch tem a indubitável certeza de que tudo isso é um absurdo, e
um engano vazio, porém, quando o doutor, ajoelhado, se estica sobre ele,
encostando o ouvido ora em cima, ora embaixo, fazendo diversas evoluções
de ginástica em cima dele com cara de importante, Ivan Ilitch submete"-se
a isso como acontecia de se submeter aos discursos dos advogados quando
sabia que estavam sempre mentindo, e por que motivo.

De joelhos no sofá, o doutor ainda dava suas batidas quando o vestido de
seda de Praskóvia Fiódorovna farfalhou à porta, e Piotr foi repreendido
por não informá"-la da chegada do médico.

Ela entra, beija o marido e imediatamente começa a demonstrar que já
estava acordada há tempos, e só não estivera ali quando da chegada do
doutor por engano.

Ivan Ilitch examina"-a, olha"-a por inteiro e a recrimina pela brancura,
por ser roliça, pela limpaza das mãos, do pescoço, pelo brilho do cabelo
e o cintilar dos olhos cheios de vida. Odeia"-a com todas as forças do
espírito. E o toque dela o obriga a padecer de um acesso de ódio.

Sua relação com ele e a doença era sempre a mesma. Assim como o médico
elaborara uma atitude para com os pacientes que não conseguia mais
despir, ela também elaborara uma atitude para com ele --- a de que ele
não fazia o necessário, tinha culpa, e era amorosamente repreendido por
isso --- que não conseguia mais despir.

--- Mas ele não escuta! Não toma o remédio na hora. E o mais grave é que
fica deitado numa posição que provavelmente faz mal, com os pés para
cima.

E contava como ele obrigava Guerássim a lhe segurar as pernas.

O médico deu um sorriso de afável desdém, que dizia: ``Que fazer, esses
pacientes inventam às vezes cada idiotice; mas dá para perdoar''.

Quando o exame terminou, o doutor olhou para o relógio, e então
Praskóvia Fiodorovna notificou Ivan Ilitch que, quisesse ou não, ela
tinha chamado um médico famoso que, junto com Mikhail Danílovitch (esse
era o nome do médico de sempre), o examinaria e discutiria.

--- Não se oponha, por favor. Estou fazendo isso por mim mesma --- disse,
irônica, dando a entender que fazia tudo por ele e que isso bastava para
não lhe dar o direito de recusar. Ele ficou em silêncio e franziu o
cenho. Sentia que a mentira que o rodeava era tão embrulhada que ficara
difícil discernir qualquer coisa.

Tudo que fazia com ele, ela fazia apenas por si mesma; só que, ao lhe
dizer que fazia por si o que realmente fazia por si, a coisa ficava tão
improvável que ele tinha que entender o contrário.

De fato, o célebre doutor chegou às onze e meia. Voltaram a ocorrer
auscultações e conversas significativas, na frente do paciente e no
outro quarto, sobre o rim, o ceco, e as perguntas e respostas tinham um
ar tão significativo que, em vez da questão real de vida e morte, que
era a única que restava diante de Ivan Ilitch, veio a questão sobre o
rim e o ceco, que não funcionavam adequadamente, e que seriam
imediatamente atacados por Mikhail Danílovitch e pela celebridade, que
os obrigariam a se emendar.

O célebre médico despediu"-se com ar sério, porém não desesperado. E à
pergunta tímida, que Ivan Ilitch lhe endereçou com os olhos erguidos,
com um brilho de medo e esperança, se havia possibilidade de
recuperação, ele respondeu que não dava para garantir, mas que havia
possibilidade. O olhar de esperança com que Ivan Ilitch seguiu o médico
era tão penoso que, ao avistá"-lo, Praskóvia Fiódorovna chegou até a
chorar quando saiu pela porta do gabinete para pagar os honorários da
celebridade.

A elevação de espírito suscitada pela promessa de esperança do médico
durou pouco. De novo aquele mesmo quarto, de novo os quadros, cortinas,
papel de parede, frascos e o mesmo corpo enfermo e sofredor. E Ivan
Ilitch começou a gemer; aplicaram"-lhe uma injeção, e ele adormeceu.

Despertou ao cair da tarde; levaram"-lhe o jantar. Engoliu o caldo com
dificuldade; de novo a mesma coisa, de novo a chegada da noite.

Depois do jantar, às sete, Praskóvia Fiódorovna entrou no quarto, em
vestido de noite, os peitos gordos apertados e traços de pó no rosto. De
manhã ela já o tinha lembrado de que iriam ao teatro. Sarah Bernhardt
estava na cidade, e a família tinha um camarote, comprado por
insistência dele. Agora se esquecera, e o traje da mulher o ofendia.
Escondeu, porém, a ofensa, ao se lembrar de que insistira para que
adquirissem o camarote e fossem, pois seria um prazer educativo e
estético para os filhos.

Praskóvia Fiódorovna entrou satisfeita consigo mesma, porém como se
sentisse culpa. Sentou"-se e perguntou da saúde, mas ele viu que foi só
por perguntar, e não para saber, já que não havia nada para saber, e
passou a falar do que queria: que não iria de jeito nenhum, mas que o
camarote já estava comprado, que iriam Helen, a filha e Petríschev (o
juiz de instrução, noivo da filha), e que não era possível deixá"-los
sozinhos. Mas que ela gostaria mais de ficar com ele. E que ele, mesmo
sem ela, seguisse as prescrições médicas.

--- Ah, Fiódor Petróvitch (o noivo) quer entrar. Pode? Liza também.

--- Que entrem.

A filha entrou ataviada, com o jovem corpo desnudo, aquele mesmo corpo
que o obrigava a sofrer. E ela o exibia. Forte, saudável, visivelmente
apaixonada e indignada com a doença, sofrimento e morte que atrapalhavam
sua felicidade.

Entrou também Fiódor Petróvitch, de fraque, cabelo frisado \emph{à la
Capoul}\footnote{A expressão vem do tenor francês Victor Capoul
  (1839--1924), cujo penteado frisado virou moda no século \textsc{xix}. {[}\textsc{n.\,t.}{]}}, pescoço comprido e fibroso sujeito à estreiteza do
colarinho branco, peitilho branco enorme e coxas fortes apertadas em
calças preitas estreitas, luvas brancas em uma mão e claque na outra.

Atrás dele entrou, furtivo, o colegial, de uniformezinho novo,
coitadinho, de luvas e com um azul horrível em volta dos olhos, que Ivan
Ilitch sabia o que queria dizer.

O filho sempre lhe parecera lastimável. E seu olhar temeroso e condoído
era assustador. Ivan Ilitch tinha a impressão de que, além de Guerássim,
apenas Vássia\footnote{Diminutivo de Vassili. {[}\textsc{n.\,t.}{]}} o compreendia
e tinha pena.

Todos se sentaram e voltaram a perguntar da saúde. Fez"-se silêncio. Liza
perguntou à mãe do binóculo. Houve uma altercação entre mãe e filha
sobre quem o perdera. Foi desagradável.

Fiódor Petróvitch perguntou a Ivan Ilitch se ele já tinha visto Sarah
Bernhardt. Ivan Ilitch inicialmente não entendeu a pergunta, depois
disse:

--- Não; o senhor já viu?

--- Já, em \emph{Adrienne Lecouvreur}\footnote{Peça de Ernest Legouvé e
  Eugène Scribe, estreada em Paris, em 1849, e retratando a vida e a
  misteriosa morte da atriz homônima do século \textsc{xviii}. {[}\textsc{n.\,t.}{]}}.

Praskóvia Fiódorovna disse que ela estivera especialmente bem naquele
papel. A filha retrucou. Começou uma conversa sobre a elegância e
realismo de sua atuação --- o mesmo tipo de conversa que sempre acontece
nessas mesmas ocasiões.

No meio da conversa, Fiódor Petróvitch olhou para Ivan Ilitch e se
calou. Os outros olharam e se calaram. Ivan Ilitch olhava para a frente
com os olhos brilhando, visivelmente indignado com eles. Era preciso dar
um jeito naquilo, mas não havia como. Era preciso romper o silêncio de
alguma forma. Ninguém se decidia, e todos ficavam com medo de que, de
repente, a decorosa mentira desmoronasse, e todos vissem com clareza as
coisas como eram. Liza foi a primeira a se decidir. Ela rompeu o
silêncio. Queria esconder o que todos sentiram, mas acabou deixando
escapar.

--- Olha, \emph{se nós vamos,} está na hora --- disse, olhando para o
relógio, presente do pai, dando um sorríso quase imperceptível para o
jovem, com um significado que só os dois conheciam, e farfalhando o
vestido ao se levantar.

Todos se levantaram, desperiram"-se e partiram.

Quando eles saíram, Ivan Ilitch teve a impressão de se sentir mais leve:
não havia mais a mentira, que tinha saído com eles, mas a dor
permanecia. Aquela mesma dor, aquele mesmo medo, nem mais pesados, nem
mais leves. Tudo estava pior.

De novo, minuto voltou a seguir minuto, hora a seguir hora, tudo igual,
e tudo sem fim, e o mais aterrador era o fim inescapável.

--- Sim, mande vir Guerássim --- respondeu à pergunta de Piotr.

\section{IX}

A mulher voltou tarde da noite. Caminhava na ponta dos pés, mas ele a
ouvia: abriu os olhos e, apressadamente, voltou a fechá"-los. Ela queria
colocar Guerássim para dormir e substituí"-lo. Ivan Ilitch abriu os olhos
e disse;

--- Não. Vá embora.

--- Está sofrendo muito?

--- Dá na mesma.

--- Tome ópio.

Concordou e tomou. Ela se foi.

Até as três horas, permaneceu em penoso esquecimento. Teve a impressão
de estar sendo empurrado para dentro de um saco preto estreito e
profundo, que enfiavam com cada vez mais força, sem conseguir chegar até
o fim. E essa coisa terrível acontecia com sofrimento. Tinha medo e
queria se afundar, resistia e ajudava. E eis que, de repente, despencou,
caiu e acordou. Guerássim continuava sentado do mesmo jeito ao pé da
cama, cochilando tranquilo e paciente. Já ele estava deitado, com os pés
descarnados e com meias nos ombros do criado; a mesma vela no abajur, e
a mesma dor incessante.

--- Vá embora, Guerássim --- sussurrou.

--- Não é nada, senhor, fico aqui.

--- Não, vá embora.

Baixou os pés, deitou de lado sobre o braço e teve pena de si mesmo.
Esperou apenas que Guerássim fosse para o cômodo vizinho para parar de
se segurar, chorando como uma criança. Chorava devido ao desamparo, à
terrível solidão, à crueldade das pessoas, à crueldade de Deus, à
ausência de Deus.

``Por que você fez isso tudo? Por que me trouxe para cá? Para que, para
que você me submete a uma tortura tão horrível?''

Não esperava resposta, e chorava porque não havia e nem podia haver
resposta. A dor voltou a se manifestar, mas ele não se agitou nem
chamou. Dizia para si: ``De novo, bata mais! Mas por quê? O que foi que
eu lhe fiz, por quê?''

Depois sossegou, parando não apenas de chorar, mas também de respirar, e
se fez todo atenção; como se ouvisse não a voz que falava por sons, mas
a voz da alma, a torrente de ideias que se formava dentro dele.

--- Do que você precisa? --- foi a primeira coisa clara e passível de
exprimir com palavras que ouviu. --- Do que precisa? Do que precisa? ---
repetia para si. --- Do quê? --- Não sofrer. Viver --- respondeu.

E voltou a se fazer todo atenção, de forma tão tensa que nem a dor o
distraía.

--- Viver? Viver como ? --- perguntou a voz da alma.

--- Sim, viver como eu vivia antes: de um jeito bom, agradável.

--- Como você vivia antes, de um jeito bom e agradável? --- perguntou a voz.
E começou a rever na imaginação os melhores momentos de sua vida
agradável. Porém --- coisa estranha --- todos esses melhores momentos da
vida agradável agora não tinham nada da aparência que tinham antes.
Todos, menos as primeiras lembranças da infância. Lá, na infância,
houvera algo realmente agradável, que valeria a pena viver, se
regressasse. Mas a pessoa que desfrutara desse agrado não mais existia:
era como se fosse uma lembrança de outro.

Bastou começar aquilo cujo resultado era o Ivan Iitch de hoje para tudo
que então lhe parecera alegria derreter diante de seus olhos e se
converter em algo insignificante e, com frequência, abjeto.

Quanto mais distantes da infância e mais próximas do presente, mais suas
alegrias eram insignificantes e duvidosas. Começou com a Escola de
Direito. Lá ainda havia algo de verdadeiramente bom: lá havia
felicidade, lá havia amizade, lá havia esperança. Porém, nos anos mais
avançados, esses momentos bons já se faziam mais raros. Depois, no tempo
do primeiro emprego, com o governador, voltaram a surgir bons momentos:
as lembranças do amor pela mulher. Depois tudo isso se confundia, e
passava a haver menos coisas boas. Mais adiante, menos coisas boas e,
ainda mais adiante, menos ainda.

O casamento.. que acaso, que decepção, o odor da boca da mulher, a
sensualidade, o fingimento! Aquele trabalho morto, aquelas preocupações
com dinheiro, e assim um ano, dois, dez, doze, e tudo sempre igual. E
quanto mais avançava, mais morto. Era como se eu descesse uma montanha
com passos regulares, imaginando que estava a escalá"-la. Foi assim. Na
opinião geral, eu estava escalando, e a vida me escapava debaixo dos pés
na mesma medida\ldots{} E agora pronto, morra!

Mas o que é isso? Por quê? Não pode ser. Não pode ser que a vida tenha
sido tão sem sentido e abjeta. E se ela foi apenas tão abjeta e sem
sentido, então por que morrer, e morrer no sofrimento? Alguma coisa não
bate.

``Talvez eu não tenha vivido como deveria'' --- passou"-lhe de repente pela
cabeça. ``Mas como não, se eu fiz tudo como tem que ser?'' --- disse para
si, afastando de imediato, como algo completamente impossível, a única
solução para todos os enigmas da vida e da morte.

``O que você quer agora? Viver? Viver como? Viver como você vive no
tribunal, quando o oficial de justiça proclama: 'Está aberta a
sessão!..' Está aberta a sessão, a sesão está aberta --- repetia para si.
--- Começou o julgamento! Só que eu não sou culpado! --- gritava, com raiva.
--- Para quê?'' Parou de chorar e, virando o rosto para a parede, pôs"-se a
pensar em uma única e exclusiva coisa: para que por que todo aquele
horror?

Porém, por mais que pensasse, não encontrava resposta. E quando lhe
ocorria, como ocorria com frequência, a ideia de que tudo aquilo estava
acontecendo porque ele não vivera como deveria, recodava de imediato
toda a retidão de sua vida e afastava aquela ideia estranha.

\section{X}

Passaram mais duas semanas. Ivan Ilitch não se levantava mais do sofá.
Não queria ficar na cama, e se mudara para o sofá. E, deitado quase o
tempo inteiro com a cara virada para a parede, sofria sozinho os mesmos
sofrimentos insolúveis, e pensava sozinho na mesma questão insolúvel. O
que é isso? Será verdade que é a morte? E uma voz interior respondia:
sim, é verdade. Por que esses sofrimentos? E a voz respondia: é assim,
não tem porquê. E não havia nada além disso.

Desde o começo da doença, desde a primeira ida de Ivan Ilitch ao médico,
sua vida se dividia em dois estados de espírito opostos, que se
alternavam: ora o desespero e a espera por uma morte incompreensível e
horrenda, ora a esperança e o interesse em observar o funcionamento do
corpo. Ora tinha diante dos olhos apenas o rim ou o ceco desviando"-se
temporariamente do cumprimento de suas obrigações, ora tinha apenas a
morte horrenda e incompreensível, da qual não havia escapatória.

Esses dois estados de espírito se alternavam desde o começo da doença;
porém, quanto mais a moléstia avançava, mais duvidosas e fantásticas se
tornavam as considerações a respeito do rim, e mais real a consciência
da morte iminente.

Bastava recordar como era três meses atrás e como estava agora, recordar
a regularidade com que descia a montanha para que qualquer possibilidade
de esperança se esvaísse.

Nos últimos tempos, naquela solidão medonha em que se encontrava,
deitado no sofá, com a cara virada para a parede --- aquela solidão numa
cidade populosa, em meio a seus inúmeros conhecidos e familiares, uma
solidão que não podia ser mais completa nem no fundo do mar, nem na
terra ---, Ivan Ilitch vivia apenas pensando no passado. Os quadros de seu
passado se apresentavam, um atrás do outro. Começavam sempre com o tempo
mais próximo e se estendiam até o mais distante, a infância, e por lá
ficavam. Se Ivan Ilitch recordasse a ameixa passa cozida que tinham lhe
dado para comer hoje, lembrava"-se da ameixa passa francesa, crua e
enrugada, da infância, seu gosto peculiar, que deixava a boca cheia de
saliva quando se chegava ao caroço, e essa lembrança do gosto suscitava
toda uma fileira de recordações daquela época: a babá, o irmão, os
brinquedos. ``Não tenho que pensar nisso\ldots{} é doloroso demais'' --- Ivan
Ilitch dizia para si, voltando a se transportar para o presente. Um
botão no encosto do sofá e rugas no marroquim. ``O marroquim é caro,
frágil: houve discussão por causa dele. Mas houve outro marroquim e
outra discussão quando rasgamos a pasta do papai e fomos punidos, mas a
mamãe trouxe uns \emph{pirojkí}''. E voltava a se deter na infância, o
que voltava a ser doloroso, e Ivan Ilitch se empenhava em afastar aquilo
e pensar em outra coisa.

E novamente, junto com essa cadeia de recordações, havia uma outra que
lhe ia na alma, a respeito do recrudescimento e do crescimento de sua
doença. Aqui também, quanto mais recuava, mais vida havia. Havia mais
bondade na vida, e havia mais da própria vida.As duas cadeias se
fundiam. ``Assim como o tormento vai ficando cada vez pior, minha vida
inteira foi ficando cada vez pior'' --- pensava. Apenas um ponto luminoso
lá atrás, no começo da vida, e depois tudo cada vez mais negro e cada
vez mais rápido. ``Inversamente proporcional ao quadrado da distância da
morte'' --- pensou Ivan Ilitch. E essa imagem da pedra despencando com
velocidade crescente lhe calou na alma. A vida era uma sucessão de
sofrimentos crescentes, voando cada vez mais rápido para o fim, o mais
medonho dos sofrimentos. ``Estou voando\ldots{}'' Estremecia, mexia"-se,
queria resistir; mas já sabia que não havia como se opor, e novamente,
com os olhos cansados de olhar, mas impossibilitados de não olhar para o
que estava diante deles, fitava o encosto do sofá e aguardava, aguardava
aquela queda medonha, o choque e a destruição. ``Não dá para resistir''
--- dizia para si. ``Mas se ao menos entendesse para quê. Só que também
não dá. Talvez desse para explicar, se dissesse que não vivi como
deveria. Isso, porém, é impossível reconhecer'' --- dizia para si,
recordando a observância das leis, a justeza e a decência de sua vida.
``É impossível admitir isso'' --- dizia para si, com os lábios
sorridentes, como se alguém pudesse ver esse sorriso e ser ludibriado
por ele. ``Não tem explicação! Tormento, morte\ldots{} Para quê?''

\section{XI}

Assim passaram duas semanas. Nessas semanas, sucedeu o desejado por Ivan
Ilitch e sua mulher: Petríschev fez um pedido formal de casamento. Isso
aconteceu à noite. No dia seguinte, Praskóvia Fiódorovna foi até o
marido, ponderando como avisá"-lo do pedido de Fódor Petróvitch mas,
naquela mesma noite, Ivan Ilitch sofrera uma nova mudança para pior.
Praskóvia Fiódorovna encontrou"-o no mesmo sofá, mas em outra posição.
Estava deitado de costas, gemia e olhava para a frente com olhar fixo.

Ela se pôs a falar de remédios. Ele transferiu o olhar para ela. A
mulher não terminou de dizer o que começara, tamanho o ódio que aquele
olhar exprimia, justamente contra ela.

--- Por Cristo, deixe"-me morrer em paz --- disse.

Ela teve vontade de sair mas, naquela hora, a filha entrou e foi
cumprimentar. Ele olhou para a filha do mesmo jeito que para a mulher,
respondendo suas perguntas sobre a saúde de forma seca, dizendo que logo
libertaria a todos de si. Elas se calaram, se sentaram e se foram.

--- Que culpa nos temos? --- disse Liza à mãe. --- Como se fosse obra nossa!
Tenho pena do papai, mas por que ele nos atormenta?

O doutor veio na hora habitual. Ivan Ilitch respondia"-lhe ``sim, não'',
sem desviar dele o olhar exasperado e, no fim, disse:

--- Como o senhor sabe que não tem como me ajudar, deixe"-me.

--- Podemos aliviar o sofrimento --- disse o médico.

--- Não pode nem isso; deixe"-me.

O médico se encaminhou à sala de visitas e informou Praskóvia Fiódorovna
que o paciente estava muito mal, e que havia apenas um meio --- o ópio ---
de aliviar seus sofrimentos, que deviam ser horríveis.

O doutor disse que seu sofrimento físico era horrível, o que era
verdade; mas ainda mais horrível do que o sofrimento físico era o
sofrimento moral, onde residia a maior parte de seu tormento.

O sofrimento moral consistia em que, naquela noite, ao olhar para o
rosto sonolento, bonachão e de maçãs salientes de Guerássim, de súbito
lhe veio à mente: e se, de fato, toda a minha vida, a vida consciente,
tiver sido ``imprópria''?

Veio"-lhe à mente que aquilo que antes parecia uma completa
impossibilidade, que não tivesse passado a vida como devia, pudesse ser
verdade. Veio"-lhe à mente que suas intenções quase imperceptíveis de
lutar contra aquilo que as pessoas de posição elevada consideravam bom,
que ele imediatamente afastava de si, essas intenções é que podiam ser a
realidade, e tudo o resto era impróprio. Seu trabalho, a organização de
sua vida, sua família, os interesses da sociedade e do serviço, tudo
isso podia ser impróprio. Tentou defender tudo aquilo perante si mesmo.
E, de repente, sentiu toda a debilidade do que defendia. Não havia nada
a defender.

``Mas se for isso --- disse para si ---, vou sair da vida com a consciência
de que arruinei tudo que me foi dado, de que não dá para consertar, e
daí?'' Deitou"-se de costas e pôs"-se a reexaminar toda sua vida de forma
completamentee distinta. Ao ver de manhã o lacaio, depois a mulher,
depois a filha, depois o médico, cada um de seus movimentos, cada uma de
suas palavras confirmava a horrível verdade que a noite lhe revelara.
Via neles a si mesmo, tudo por que vivera, e via com clareza que tudo
aquilo era impróprio, tudo aquilo era um erro horrível e imenso, que
escondia a vida e a morte. A consciência disso intensificou, decuplicou
seu sofrimento físico. Gemia, remexia"-se e arrancava a roupa. Tinha a
impressão de que ela o sufocava e oprimia. E por isso os odiava.

Deram"-lhe uma grande dose de ópio, ele desfaleceu; ao jantar, porém,
recomeçou tudo de novo. Expulsou todo mundo e ficava se remexendo,
desvairado.

A mulher foi até ele e disse:

--- Jean, queridinho, faça isso por mim (por mim?). Isso não tem como
fazer mal, e muitas vezes ajuda. Veja, não é nada. Muitas vezes, mesmo
as pessoas saudáveis\ldots{}

Ele abriu bem os olhos.

--- O quê? A comunhão? Para quê? Não precisa! A propósito\ldots{}

Ela se pôs a chorar.

--- Sim, meu amigo? Vou chamar o nosso, ele é tão querido.

--- Maravilha, muito bem --- ele afirmou.

Quando veio o sacerdote e o confessou, ele se acalmou, sentindo algo
como um alívio por suas dúvidas e, consequentemente, do sofrimento, no
que encontrou um minuto de esperança. Voltou a pensar no ceco e na
possibilidade de consertá"-lo. Comungou com lágrimas nos olhos.

Quando o deitaram após a comunhão, sentiu leveza por uns instantes, e
voltou a surgir esperança de vida. Pôs"-se a pensar na operação que lhe
propuseram. ``Viver, quero viver'' --- dizia para si. A mulher veio
cumprimentar, disse as palavras usuais e acrescentou;

--- Não é verdade que você está melhor?

Sem olhar para ela, ele disse: sim.

Sua roupa, sua compleição, a expressão de seu rosto, o som de sua voz,
tudo isso dizia ao marido só uma coisa: ``É impróprio. Tudo por que
vivemos e você vive é uma mentira, um engano, que esconde de você a vida
e a morte''. Bastou ele pensar isso para se erguer o ódio e, junto com o
ódio, o torturante sofimento físico e, com o sofrimento, a consciência
da ruína próxima e inescapável. Aconteceu algo de novo: uns giros, umas
pontadas, um aperto na respiração.

A expressão de seu rosto, ao dizer ``sim'', era horrível. Ao dizer
aquele ``sim'', encarando"-a, virou"-se de bruços com uma rapidez rara
para a sua fraqueza e gritou:

--- Saiam, saiam, deixem"-me!

\section{XII}

Naquele momento começaram três dias de uma gritaria contínua, tão
medonha que não era possível ouvir através de duas portas sem se
horrorizar. No minuto em que respondeu à mulher, compreendeu que estava
perdido, que não tinha volta, que tinha chegado ao fim, o fim total, e a
dúvida não estava resolvida, continuava a dúvida.

--- Oh! Oh! oh! --- gritava, com diversas entonações. Começara a gritar:
``Não quero!'', e continuava a gritar na letra ``o''.

Por três dias inteiros, no decorrer dos quais o tempo não existia para
ele, debatia"-se naquele saco preto no qual uma força invisível e
invencível o enfiava. Lutava como um condenado à morte nas mãos do
carrasco, sabendo que não tinha como se salvar; e a cada minuto sentia
que, apesar de todo o esforço do combate, ficava cada vez mais próximo
do que o aterrorizava. Sentia que seu tormento consistia em ser enfiado
naquele buraco negro, e mais ainda em não poder se meter nele. O que o
impedia de se meter nele era o reconhecimento de que sua vida era boa.
Tal justificação de sua vida o prendia, não o deixava avançar e o
atormentava mais do que tudo.

De repente, uma força o golpeou no peito, no flanco, a respiração
apertou ainda mais, ele caiu no buraco e lá, no fim do buraco, algo
acendeu. Acontecia o que se passa nos vagões do trem, quando você acha
que está indo para a frente, mas vai para trás e, de repente, fica
sabendo qual é a direção certa.

--- Sim, tudo foi impróprio --- dizia para si ---, mas isso não é nada. É
possível, é possível fazer o 'próprio'. Mas o que é o 'próprio'? ---
perguntou"-se, e se calou de repente.

Isso foi no fim do terceiro dia, uma hora antes de sua morte. Nessa
mesma hora, o colegial penetrou de mansinho no quarto do pai e foi até
seu leito. O moribundo gritava o tempo todo, em desespero, e agitava as
mãos. Suas mãos caíram na cabeça do menino. O colegial tomou"-as,
levou"-as até os lábios e se pôs a chorar.

Nessa hora, Ivan Ilitch despencou, viu a luz, e revelou"-se que sua vida
não fora como devia ter sido, mas que ainda dava para consertar.
Perguntou a si mesmo: o que é 'próprio', e se calou, apurando o ouvido.
Daí sentiu que alguém lhe beijava as mãos. Abriu os olhos e viu o filho.
Teve pena dele. A mulher se aproximou. Olhou para ela. Fitava"-o com
expressão de desespero, de boca aberta, lágrimas a lhe banhar o nariz e
as faces. Teve pena dela.

``Sim, eu os estou torturando --- pensou. --- Dão pena, mas vão melhorar
quando eu morrer''. Quis dizer isso, mas não teve forças para exprimir.
``Aliás, para que falar, preciso fazer'' --- pensou. Com o olhar, indicou
o filho para a mulher e disse:

--- Leve\ldots{} dá pena\ldots{} você também\ldots{} --- Quis dizer ``desculpa'', mas disse
``licença'' e, sem forças para corrigir, fez um gesto com a mão, sabendo
que seria compreendido do jeito certo.

E de repente ficou claro para ele que o que o atormentava e não ia
embora estava indo embora de repente, de dois lados, de dez lados, de
todos os lados. Eles dão pena, é preciso fazer com que não lhes seja
doloroso. Livrá"-los e livrar"-se desse sofrimento. ``Como é bom e como é
simples --- pensou. --- E a dor? --- perguntou a si mesmo. --- Para onde ela
foi? Olha só, cadê você, dor?''

Pôs"-se a prestar atenção.

``Sim, ei"-la. Pois bem, que seja a dor''.

``E a morte? Cadê ela?''

Procurou seu antigo e habitual medo da morte e não o encontrou. Cadê
ela? Que morte? Não havia medo algum, pois não havia também a morte.

Em vez da morte, havia a luz.

--- Então é isso! --- disse de repente, em voz alta. --- Que alegria!

Para ele, tudo isso ocorreu em um instante, e o significado desse
instante não mais se alterou. Para os presentes, porém, sua agonia se
prolongou por mais duas horas. Algo fervilhava em seu peito; seu corpo
macilento estremecia. Depois o fervilhar e o estertor foram rareando.

--- Acabou! --- disse alguém acima dele.

Ele ouviu tais palavras e as repetiu na alma. ``Acabou a morte --- disse
para si. --- Ela não existe mais''.

Inalou ar, parou no meio do suspiro, aprumou"-se e morreu.
