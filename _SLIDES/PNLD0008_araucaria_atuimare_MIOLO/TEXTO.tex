\chapter{A nova ficção científica brasileira}

\section{Sobre o autor}


Não são poucos os médicos (atuantes ou não na prática da medicina) que escrevem literatura – célebres nomes como Anton Tchekhov, Guimarães Rosa, Moacyr Scliar e Arthur Conan Doyle figuram nessa lista. Fabio Atui é também um desses clínicos-literatos – médico gastroenterologista e proctologista, formou-se em 1995 pela Faculdade de Medicina da USP. É cirurgião-geral do Hospital das Clínicas de São Paulo, onde opera diversas enfermidades que assolam o trato digestivo. Também é responsável pelo ambulatório de doenças sexualmente transmissíveis e proctologia do mesmo hospital, onde atende semanalmente. Ainda no Hospital das Clínicas, Fabio é membro ativo do Navis (Núcleo de Assistência à Vítima de Violência Sexual), amparando e assistindo pessoas sexualmente agredidas.

Desde 2004, Fabio atua também como cirurgião voluntário e coordenador de cirurgia dos Expedicionários da Saúde, um grupo de médicos voluntários que levam atendimento especializado a populações indígenas em regiões isoladas. 

À toda essa vivência em contato direto com pessoas, soma-se a crença de Fabio de que a disseminação de informações de qualidade é a receita certa para que as pessoas vivam melhor. O conhecimento humano adquirido em tantos anos de dedicação à área da saúde agrega-se ao fazer artístico: Fabio assina, junto com dois outros autores, o roteiro do documentário \emph{Expedicionários}, longa-metragem narrativo do trabalho da equipe de médicos voluntários para com indígenas que habitam regiões remotas do país, onde a infraestrutura e o trabalho clínico especializado não chegam. O filme foi dirigido por Octávio Cury e lançado em 2012. 

Mas a vivência com o campo das artes acompanha Atui desde a infância: Fabio é sobrinho de Fauzi Arap, um dos principais nomes da dramaturgia brasileira, de quem toma emprestado o título \emph{Mare nostrum} e a inspiração e ideia inicial do romance.

\section{Sobre a obra}

Deslumbrado com o poder de navegar o Mare nostrum -- o oceano que interliga os nossos inconscientes --, Theo é enviado à Amazônia, onde estranhamento e familiaridade se interligam em uma jornada rumo ao próprio passado. Em contato com culturas e saberes diferentes, o aprender se impõe como urgência diante da ameaça que parece cada vez mais próxima: poderá esse mar ser cerceado e controlado? \textit{Mare nostrum: Paranã Tipi} é o segundo volume da aventura de Theo rumo ao conhecimento do Mare nostrum, que é também a busca de si mesmo no alvorecer da vida adulta.

Colocando, ao lado da discussão sobre o inconsciente e o poder do sonho, questões relativas aos povos indígenas, como seus hábitos e costumes em confronto com o pensamento de um jovem médico da cidade, \emph{Mare nostrum: Paranã Tipi} é um texto atual, num gênero de ficção que se aproxima da ficção científica, esbarrando nos subgêneros da ficção metacientífica e psicocientífica. A ficção científica é um gênero pouco explorado na escola, mas de muita identificação com o universo do adolescente, principalmente por meio de produções audiovisuais estrangeiras. Pouco se menciona de ficção com caráter científico produzida e ambientada no Brasil.

O livro parte de uma potência comum a toda a humanidade: a capacidade de sonhar. A trama ficcional não se desenvolve em torno de complexos aparatos tecnológicos de vocabulário intrincado ou especialista, facilitando a compreensão do adolescente e direcionando-o para reflexões acerca do próprio conhecimento, dos mistérios em torno da mente humana, daquilo que é pressuposto como verdade científica e da importância do acesso à informação. 

%A obra é ainda bastante interdisciplinar. Por partir de pressupostos científicos, o livro permite trabalhar em paralelo os conhecimentos do aluno sobre a Teoria da Evolução das Espécies, a teoria psicanalítica, conceitos físicos de onda, eletromagnetismo e eletrostática, além de um perene teor filosófico que permeia o enredo do início ao fim da trama.

%Em uma sociedade cada vez mais tecnicizante, cuja hipervalorização da tecnologia e da produção e consumo desenfreado alienam o jovem do instinto de perscrutação filosófica que é o princípio motor de toda a ciência, \emph{Mare nostrum} vai levantar os pilares de que conhecimento e imaginação se alimentam.

\section{Sobre o gênero}

O gênero da ficção científica é um tipo específico de ficção, que constrói o enredo a partir da ciência – real ou imaginada – e de seu impacto em determinada sociedade. Surgiu a partir do século \versal{XIX}, principalmente em razão do desenvolvimento científico e tecnológico nas áreas da Física, Química, Geologia, Astronomia e Informática, que introduziram descobertas, invenções e aparatos tecnológicos que modificaram a vivência da sociedade humana.

Muito popular em obras estrangeiras, especialmente em filmes e séries para a televisão, o gênero ainda tem poucos autores de referência no Brasil. \emph{Mare nostrum} poderia ser categorizado como ficção psicocientífica, por ficcionalizar algumas teorias psicanalíticas do inconsciente, do material onírico, daquilo que é o sonhado, ou ainda como ficção metacientífica, por orbitar a esfera da indagação da ciência acerca da própria ciência, lançando mão de pressupostos e teorias científicas para a construção do enredo.