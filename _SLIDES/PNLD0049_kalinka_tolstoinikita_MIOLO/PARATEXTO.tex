\chapter{Paratexto}\label{paratexto}

\section{Sobre o autor}\label{para1}

Aleksei Tolstói (1883--1945), da família do conde Nikolai Tolstói,
nasceu em uma cidadezinha na província de Samara e virou um conhecido e
influente escritor na União Soviética, principalmente a partir da década
de 1930. Aventurou-se por vários gêneros: foi autor de obras realistas e
históricas, como a trilogia \emph{O caminho dos tormentos} (1921--1940) e
o romance histórico inacabado \emph{Pedro \textsc{i}} (1930--1934), e de livros de
ficção científica, como \emph{Aelita} (1923) e \emph{O hiperboloide do
engenheiro Gárin} (1927). Escreveu também para jovens e crianças e
assinou uma adaptação de \emph{Pinóquio} que se tornou conhecida na
Rússia inteira: \emph{A chave de ouro ou as aventuras de Buratino}
(1936).

Foi em Paris, em 1920, que Tolstói começou a publicar \emph{A infância
de Nikita}, nos números 2, 3, 4, 5 e 6 da revista infantil \emph{Varinha
verde.} No entanto, o livro só saiu integralmente em Berlim pela editora
\emph{Guelikon} (1922) com um novo título: \emph{Uma novela sobre muitas
coisas maravilhosas (A infância de Nikita).} Na Rússia, para onde o
autor regressou em 1923, o livro recebeu novamente o título original.

Em 1918 Tolstói, acompanhado por sua terceira esposa, Natália
Krandiévskaia (1888--1963), e por seu filho Nikita, com um ano, saíra dе
Moscou para Odessa. No ano seguinte foram para Constantinopla e então
para Paris, onde Tolstói se envolvera em círculos intelectuais de
emigrados. Quando começou a publicar \emph{A infância de Nikita},
passava por problemas financeiros e estava isolado artisticamente,
vivendo de pequenas resenhas. Em 1921, partiu para Berlim com a promessa
de dirigir uma revista literária independente.

Marcada pelo tom autobiográfico, \emph{A infância de Nikita} descreve um
ano da vida de Nikita, um garoto de nove anos que morava numa
propriedade rural em Sosnovka, na província de Samara, com a mãe, o pai
e o preceptor, no fim do século \textsc{xix}. O campo, idealizado em Sosnovka, é
um depositório de experiências de iniciação, mediadas por animais, por
fenômenos da natureza e pelas estações do ano.

O desejo de retorno à Rússia --- onde Aleksei Tolstói se tornará um
escritor de renome, o ``Conde Vermelho'' --- é visível na obra, embora
seu conteúdo se afaste das tendências pedagógicas que surgiam na Rússia
de então e se aproxime da tradição russa do século \textsc{xix}.


\section{Sobre a obra}\label{para2}

\lipsum[1]

\section{Sobre o gênero}\label{para3}

\lipsum[1]

\section{Sobre os colaboradores}\label{colab}

\noindent\textbf{Moissei Mountian}, nascido na \textsc{urss}, é tradutor e parte do
conselho editorial da Kalinka. Traduziu muitos livros russos, como
\emph{O Diabo Mesquinho}, de Fiódor Sologub.

\medskip

\noindent\textbf{Irineu Franco Perpetuo} é tradutor, jornalista e crítico de
música. Entre suas muitas traduções, consta \emph{Vida e destino}, de
Vassíli Grossman.

\medskip

\noindent\textbf{Fabio Flaks} é artista plástico e arquiteto. Participou de
exposições coletivas e individuais (\textit{fabioflaks.com}).

