\chapterspecial{A vida de Harriet Jacobs}{}{Kellie Carter Jackson}

\section{Sobre a autora}

Harriet Ann Brent Jacobs nasceu em torno do outono de 1813, em Edenton,
Carolina do Norte. Como os senhores de escravos muitas vezes não
registravam as datas de nascimento das suas propriedades, a grande
maioria das pessoas escravizadas\footnote{No aparato crítico dessa edição empregou"-se a palavra ``escravizado'' no lugar de ``escravo''. Essa mudança lexical desnaturaliza o processo de escravização e a existência social do escravismo, pois a alteração do sufixo transforma o substantivo ``escravo'', que conota \emph{status} ou condição permanente, no verbo ``escravizar'', evidenciando o dinamismo da construção social da pessoa em situação de escravidão. Em inglês, o particípio e a função adjetiva do particípio se distinguem pela posição do termo em relação ao nome (``\emph{enslaved person}'', ``\emph{person enslaved}''). Em português, o valor posicional do termo não produz distinção semântica com a mesma clareza que o inglês. Nos casos em que o emprego da forma nominal ``escravizado'' gerasse ambiguidade, esta edição optou excepcionalmente pelo uso vernacular dos vocábulos. [\versal{N.~O.}]} não sabia quando havia nascido. Para
muitas, os aniversários só podiam ser identificados pela estação, como o
inverno ou o verão. Os cativos --- era o que se tentava inculcar neles --- tinham de se considerar, antes de mais nada, propriedade dos seus senhores. É interessante que,
quando jovem, Jacobs não estava ciente de que não pertencia a si mesma
ou aos seus pais. Quando eram pequenos, Jacobs e John, seu irmão mais
novo (chamado de William na narrativa), foram protegidos do sistema
complexo e violento tanto quanto seus pais conseguiam. Elijah Jacobs,
seu pai, era um carpinteiro, da propriedade do Dr.\,Andrew Knox, e
era um homem altamente inteligente e habilidoso que
tinha permissão para que ``exercesse sua profissão e administrasse
sua própria vida'', mas, apesar de poupar dinheiro, nunca conseguiu
comprar seus próprios filhos. Os pais de Jacobs pertenciam a
senhores diferentes: sua mãe Delilah e sua avó Molly pertenciam ambas a
Margaret Hornbilow. Molly, também conhecida por Tia Martha, era muito
querida na comunidade, especialmente pelas suas habilidades de
cozinheira. Ela conseguiu obter sua alforria e morava na própria casa,
ganhando a vida como padeira.

Quando tinha cerca de seis anos, Delilah, mãe de Jacobs, morreu, deixando"-a
arrasada, e igualmente incerta sobre qual seria seu destino.
Margaret, sua senhora, prometera a Delilah no seu leito de morte que
cuidaria e protegeria seus dois filhos, e, apesar de não libertá"-los
da escravidão, ela se esforçou para protegê"-los do trabalho
pesado. Margaret cumpriu sua palavra, cuidando de Jacobs e William e até
ensinando"-a a ler, escrever e costurar. Contudo, meros seis anos depois,
quando Jacobs tinha doze anos, Margaret adoeceu e morreu. Jacobs perdeu
sua senhora protetora no início da sua puberdade. Foi nesse momento que
a vida de Jacobs mudou drasticamente, e é assim que descobrimos o terror que é,
para uma menina escravizada, transformar"-se em mulher.

Em 1825, Margaret legou ``minha negra Harriet'' e ``minha escrivaninha
\& mesa de trabalho \& seu conteúdo'' para sua sobrinha de três anos,
Mary Matilda Norcom (conhecida como ``Srta. Emily Flint'' no texto). A
partir de então, Harriet e seu irmão foram mandados para a residência
dos Norcom. Como Mary era uma criança pequena, seu pai, o Dr.\,James
Norcom, foi colocado como tutor de todas
as suas propriedades. Norcom (chamado de ``Dr.\,Flint'') era um cidadão
muito respeitado na comunidade, mas a portas fechadas pretendia abusar
de Jacobs e ter relações sexuais com ela.

À medida que foi se tornando adulta, a vida de Jacobs mudou. Ela passou
a ser assolada pelas tentativas do seu senhor de explorá"-la sexualmente.
Norcom era manipulador e tirânico, e atormentava Jacobs constantemente.
A narrativa de Jacobs é a primeira a ilustrar como as mulheres
cativas eram suscetíveis à violência sexual e o quão pouco podiam
se defender. Em um dos exemplos mais descritivos do tormento sexual,
Jacobs analisa como as escravas não tinham proteções para
impedir que fossem violadas. Além disso, na fazenda, não havia
solidariedade alguma entre as mulheres brancas e as escravizadas. Jacobs
explica como a senhora tinha apenas sentimentos de ciúme equivocado e
raiva contra a vítima. Não importava se as mulheres escravizadas tinham poder para recusar os desejos dos seus senhores.

No início da história, na tentativa de rechaçar as ameaças sexuais de
Norcom, Jacobs tem um relacionamento com Samuel Tredwell Sawyer (chamado
de ``Sr.\,Sands''), um advogado branco. Jacobs tinha apenas quatorze anos
de idade quando conscientemente deu início a uma relação sexual para
obter um mínimo de proteção contra os avanços de Norcom. Não foi uma
solução perfeita, apenas uma tentativa de dissuasão, e um mal menor. Aos
vinte anos, Jacobs era a mãe de dois filhos com Sawyer, Joseph e Louisa.
Implacável, Norcom ameaçou vender os filhos de Jacobs para uma fazenda
vizinha, famosa pela sua brutalidade. Para Jacobs, Norcom sabia
expressar apenas luxúria, dominação e desdém, especialmente quando
rejeitado. A ameaça aos filhos de Jacobs fez com que ela chegasse ao seu limite.
Sua esperança era que, se ela fugisse, Sawyer, o pai dos seus filhos,
conseguiria comprar as duas crianças de Norcom. Jacobs acreditava que se
Sawyer fosse o proprietário dos seus filhos, o relacionamento entre os
dois poderia levá"-lo a conceder aos dois a sua alforria.

Em 1835, Jacobs foge da fazenda, mas em vez de se dirigir para o Norte,
esconde"-se no sótão minúsculo da avó. Por quase sete anos, Jacobs ocupou
um quartinho pequeno onde não podia se colocar de pé ou se esticar
completamente. O espaço tinha menos de três metros de comprimento, cerca
de dois metros de largura e um metro de altura, e era completamente
escuro. Ratos e camundongos corriam sobre a cama improvisada, na qual
ela só podia dormir de lado. Com cada nova estação, ela se perguntava
por quanto tempo ainda permaneceria prisioneira naquele sótão, incapaz
de sentir uma brisa refrescante ou abraçar seus filhos. Jacobs escreveu
cartas para a sua avó nas quais fingia estar no Norte, na tentativa de
convencer Norcom de que estava realmente além do seu alcance.

Em 1837, Sawyer conquistou maior proeminência e foi eleito para o
Congresso dos \versal{EUA}. De acordo com os desejos de Jacobs, ele comprou seus
filhos. Contudo, quando Sawyer se mudou para Washington D.C., ele não
alforriou Joseph ou Louisa. Foi só em 1842 que Jacobs finalmente fugiu
para o Norte. Ela conseguiu se reunir com os filhos e se estabeleceu em
Boston, um bastião abolicionista. Ela continuou a ser fugitiva, no
entanto, pois Norcom tentou recapturá"-la diversas vezes. Finalmente, em
1852, Jacobs foi comprada e libertada por Corneilia Grinnell Willis, sua
ex"-empregadora. Livre, ela finalmente pôde contar a própria história.
Após o sucesso de \emph{A Cabana do Pai Tomás}, de Harriet Beecher Stowe
(1852), Jacobs entendeu que contar sua própria história seria mais
poderoso do que qualquer obra de ficção.

Em cada capítulo, Jacobs guia seus leitores pela estrada perigosa que é
a escravidão. Boa parte da narrativa de Jacobs é dedicada a detalhar as
crueldades dessa instituição. O chicote e a violência são constantes,
infligidos por praticamente qualquer motivo, ou sem motivo nenhum, mas
era o leilão a arma mais poderosa contra as famílias escravas.
Durante o período pré"-Guerra Civil, quase um terço das famílias escravas
foi separado por vendas, seja por dívidas,
mortes, dificuldades econômicas ou despeito. Jacobs descreve uma mãe
cujos sete filhos foram vendidos e mandados para longe, todos no mesmo
dia. ``Para a mãe escrava o dia de Ano Novo chega carregado de tristezas
especiais. Ela se senta no chão frio da cabana, cuidando dos filhos que
poderão ser todos arrancados de si na manhã seguinte, e muitas vezes
anseia que ela e eles morram antes de o dia nascer'', Jacobs escreve. A
autora oferece aos seus leitores uma descrição dramática do seu
cativeiro físico e psicológico, ao mesmo tempo que aprende sobre a
cultura política e social dessa cidadezinha da Carolina do Norte e como
o mundo ao seu redor estava mudando.

Jacobs escreveu \emph{Incidentes da vida de uma escrava} entre 1853 e
1858. Foi um período turbulento na história americana, e um dos mais
violentos em termos de disputas políticas e jurídicas
em torno da escravidão. No início da década, os Estados Unidos
reformularam a infame Lei do Escravo Fugitivo, que exigia a devolução de todos os
fugitivos aos seus senhores,
independentemente de quanto tempo haviam vivido em liberdade. Ela
incentivava os caçadores de escravos, oferecendo recompensas por
capturas, e permitia que os caçadores de recompensas e os delegados
federais entrassem no Norte e até mesmo recrutassem cidadãos do norte
para recuperar ``propriedade roubada''. Desobedecer a nova lei poderia
levar a seis meses de prisão ou uma multa de 1000 dólares (cerca de
30000 dólares, corrigindo para a inflação). Em 1857, a Suprema Corte
dos \versal{EUA} decidiu o caso Dred Scott, um marco na história da escravidão,
recusando"-se a reconhecer os americanos negros como cidadãos do país.
Roger Taney, Chefe de Justiça da Suprema Corte, emitiu a declaração
infame de que os negros ``não tinham direitos que o homem branco deveria
respeitar; e que o homem negro era justo e legalmente reduzido à posição
de escravo em seu benefício''. Jacobs estava ciente de que a Lei do
Escravo Fugitivo garantiria a legalidade da sua reescravização após a
fuga. Ela também escreveu sabendo que sua liberdade recém"-adquirida não
lhe garantia seus direitos ou cidadania. Com a eleição de Abraham
Lincoln para a presidência, a Carolina do Sul declarou sua secessão da
União, com o intuito de preservar a escravidão, e até fevereiro de 1861
os estados do Mississippi, Flórida, Alabama, Geórgia, Luisiana e Texas
fizeram o mesmo. Quando \emph{Incidentes da vida de uma escrava} foi
lançado em 1861, as tensões em torno da escravidão estavam no auge; o
país estava em guerra. Apesar de livre, Jacobs publicou sob o pseudônimo
``Linda Brent'', pois não desejava incriminar seus amigos e familiares.

\section{Sobre a obra}

Dois temas se destacam durante a narrativa: meninice e maternidade. A
meninice negra é um campo emergente no qual estudiosos examinam a
representação histórica e literária das meninas negras e dos seus papéis
em suas comunidades. O próprio título do
livro de Jacobs no original, \emph{Incidents in the Life of Slave Girl}
(``incidentes na vida de uma menina escrava''), informa ao leitor que a
narrativa coloca a meninice no centro. Para as meninas escravizadas, a
inocência esmorece rápido, quando não é completamente erradicada. ``A
moça escrava é criada em uma atmosfera de medo e libidinosidade'',
escreve Jacobs. Em uma de suas passagens mais descritivas, ela discute o
que significa para uma menina ter sua inocência arrancada: ``Mesmo a
criancinha, acostumada a atender
sua senhora e os filhos, aprende antes dos doze anos de idade por que
sua senhora odeia esse escravo ou aquele''. Ela reconhece que as meninas
escravizadas entendiam até quando suas mães eram o motivo para a fúria e
o ciúme das senhoras. As meninas se tornam
``conhecedora{[}s{]} precoce{[}s{]} da maldade''. Jacobs escreveu que
para elas o som dos passos do senhor provoca
tremores. No instante em que o senhor se interessa por elas sexualmente,
entendem que não são mais crianças, e certamente
não aos olhos dele. Para as meninas escravizadas, Jacobs defende que a
beleza é a pior das maldições. Não havia momentos em que alguém como
Jacobs sequer era capaz de valorizar sua própria beleza. Apesar de
escrever suas memórias e ter uma longa vida adulta, lamenta:
``Não consigo expressar tudo o que sofri na presença desses agravos, nem
o quanto eles ainda me ferem em retrospecto''. Os conceitos de inocência
e virtude nunca eram estendidos a crianças ou mulheres cativas. É
importante observar que Norcom tinha 52 anos quando começou a perseguir
Jacobs, então com 13 anos. Ainda muito jovens, meninas negras eram
forçadas a se tornarem adultas, até mães. Jacobs afirma que seus quinze
anos foram um período muito triste na sua vida. Norcom
sussurrava obscenidades no seu ouvido. Ele era implacável; ela,
indefesa. Esse é o terror da escravidão para as mulheres.

Além da violência sexual havia a incapacidade de poder oferecer cuidado
e proteção aos próprios filhos. A maternidade foi uma das maiores
batalhas da vida de Jacobs, e seria impossível ignorar o amor pelos seus
filhos dentro da narrativa. Os filhos foram a razão para Jacobs ter
sobrevivido. As vozes dos seus filhos e até conseguir
vê"-los enquanto estava escondida eram uma fonte de vida para ela.

Diversas vezes, Jacobs escreve que desejou a morte em muitas e muitas
ocasiões até seus filhos nascerem. Quando deu à luz o seu filho,
suas motivações para viver mudaram. Pessoas diziam que seu filho era
lindo e, como qualquer mãe, ela adorava observar os filhos dormindo. ``Ele
estava plantando suas raízes no fundo da minha existência'', ela
escreve. Com o amor pelos filhos, porém, veio a dor: a escravidão era uma nuvem
negra que pairava sobre a maternidade. Legalmente, Norcom lembrava, seu
filho não pertencia a ela. Para as mulheres escravizadas, a maternidade
era uma chacota. As mães não podiam negar seu sentimento inato de amor,
mas, ao mesmo tempo, viam a morte como uma forma de salvação. Jacobs
desejava desesperadamente que seus filhos se libertassem do cativeiro e
da dor que ela sofria, mesmo que isso significasse sua morte.

Basicamente, o que Jacobs queria para sua família era um lar, um espaço
onde pudesse ser provedora, protetora e progenitora. Como a sua própria
mãe morrera jovem, a relação maternal mais forte que Jacobs tinha era
com a sua avó. Tia Martha era a única pessoa em quem podia confiar e a
única que buscou protegê"-la a qualquer custo. Sua liberdade e
independência literal e figurativamente criaram um espaço para Jacobs
escapar do cerco de Norcom. A ironia é que o único lar onde Jacobs teria
a própria proteção e poderia cuidar dos filhos (ainda que à distância)
era o sótão da avó. O lar serve ao mesmo tempo como refúgio e como
prisão.

A meninice e a maternidade são histórias que Frederick Douglass, William
Wells Brown e Solomon Northup podiam testemunhar, mas nunca viver.
\emph{Incidentes da Vida de uma Escrava} é uma obra especial e uma
perspectiva séria sobre o terror da escravidão. Douglass pode usar sua
narrativa para recontar a famosa briga com o Sr.\,Covey, na qual derrota
fisicamente o homem que tentava domá"-lo, mas Jacobs não tem essa opção.
Simplesmente rechaçar a raiva de Norcom a coloca sob um perigo terrível.
Ainda assim, a obra de Jacobs é corajosa. Durante toda a narrativa, ela
resiste ativamente ao poder de Norcom de transformar sua feminidade e
sua maternidade em arma contra ela mesma.

\section{Sobre o gênero}

A história da escravidão quase sempre é apresentada por uma ótica
masculina que enfatiza os senhores e os homens escravizados. Nos filmes
e romances, os homens aparecem fazendo o trabalho pesado na lavoura ou
sendo submetidos aos golpes mais pesados do chicote, sempre sob o jugo
de outro homem. Infelizmente, os registros históricos não são
diferentes, e algumas das mais populares narrativas de escravos foram
contadas da perspectiva masculina. A mais famosa, a \emph{Narrativa da
vida de Frederick Douglass, um escravo americano}, tornou"-se um
bestseller imediato. Publicado em 1845, o livro vendeu mais de onze mil
exemplares nos três primeiros anos após seu lançamento, sendo reimpresso
nove vezes e traduzido para o francês e o holandês para circular na
Europa. As críticas positivas da narrativa também transformaram Douglass
em uma celebridade do dia para a noite. Em 1847, foi a vez de \emph{A
narrativa de William Wells Brown, escravo fugitivo. Escrita por ele mesmo}. A narrativa de Brown também se transformou em bestseller,
perdendo apenas para Douglass em termos de vendas. Em 1853, Solomon
Northrup publicou sua autobiografia, \emph{Doze anos de escravidão},
sobre suas experiências como homem livre que foi sequestrado, vendido
como escravo e forçado a viver em cativeiro por doze anos. Juntas, essas
narrativas moldaram a forma pela qual os Estados Unidos entendiam a escravidão
durante o século \versal{XIX}.

No século \versal{XX}, as narrativas que continuaram a moldar as ideias sobre a
escravidão e a masculinidade americana foram reforçadas com o cinema e a
ficção. Em 1976, o romance \emph{Negras raízes: A saga de uma família},
de Alex Haley, teve vendagens altíssimas, e gerou uma minissérie popular para a
televisão. Até hoje, o épico de Haley sobre Kunta Kinte continua a ser
uma das minisséries de maior audiência da história da televisão,
enquanto Haley ainda é o único autor afro"-americano a ter vendido mais
de um milhão de exemplares. Na mesma veia, Hollywood apresentou diversas
histórias sobre a luta contra a escravidão de uma perspectiva masculina:
\emph{Tempo de glória}, o filme de Edward Zwick de 1989, por exemplo,
examina os feitos extraordinários de um regimento negro durante a Guerra
Civil. Outros filmes incluem \emph{Django livre}, de Quentin Tarantino,
e o oscarizado \emph{12 anos de escravidão}, do diretor Steve McQueen.
Seja nas autobiografias, na ficção ou no cinema, os homens dominaram as
experiências narrativas da escravidão.

Contudo, as mulheres foram essenciais para a sobrevivência da
escravidão, especialmente depois que o Congresso Federal encerrou o
tráfico negreiro transatlântico para os Estados Unidos em 1808. A
condição escrava dos filhos seguia a das mães. Os corpos das mulheres
foram usados para produção e reprodução. De acordo com Ned e Constance
Sublette, os corpos das mulheres escravizadas eram o motor da
escravocultura e moviam uma economia global de consumo de algodão nos
Estados Unidos. As mulheres trabalhavam em casa e no
campo, e estavam sujeitas à violência física e sexual. Nem mesmo a
gravidez protegia as mulheres do trabalho árduo, dos castigos, das
agressões ou dos leilões.

Quando a produção de algodão aumentou exponencialmente com os novos
territórios adquiridos pela Compra da Luisiana, pessoas cativas
foram forçadas a migrar dos estados superiores do Sul, como a Virgínia e
Maryland, para o Extremo Sul, ou os ``Estados Algodoeiros'', como
Geórgia, Alabama, Carolina do Sul, Mississippi, Luisiana e
Texas.\footnote{A Compra da Luisiana foi a aquisição do território
  francês da Luisiana pelos Estados Unidos em 1803. Os \versal{EUA} pagaram 50
  milhões de francos por terras que hoje incluem quinze estados
  americanos e duas províncias canadenses.} Durante esse movimento em
massa de pessoas, turmas de escravos foram agrilhoadas e
mandadas, a pé ou por vapores, para cultivar algodão. Meninas de doze a
quinze anos foram o único grupo demográfico mais comum do que os homens
negros migrando para o Sul, pois tinham a dupla capacidade de trabalhar
na lavoura e ter filhos. Em muitos sentidos, a história da escravidão
americana é a história das mulheres escravizadas.

Estima"-se que 12 milhões de africanos tenham sido levados para o Novo
Mundo, sendo que quase metade da carga humana importada para o oeste
tenha sido recebida pelo Brasil. As mulheres compunham uma parcela
significativa das pessoas roubadas para o Novo Mundo pela sua mão de
obra, e, no Brasil, mulheres escravizadas representavam
pouco menos de metade da população escrava entre 1825 e 1885.\footnote{Mary Karasch,
  ``Slave Women on the Brazilian Frontier in the Nineteenth Century'' in
  More Than Chattel: Black Women and Slavery in the Americas, ed. David
  Barry Gaspar and Darlene Clark Hine (Bloomington: University of
  Indiana, 1996), 81.} Assim, as histórias das mulheres cativas são
essenciais para entender as experiências das pessoas escravizadas.
Harriet Jacobs foi uma das primeiras autoras a ilustrar as diferenças
salientes da escravatura feminina:\footnote{Ver Mary Prince,
  \emph{The History of Mary Prince: A West Indian Slave} (London: F.
  Westley and A. H. Davis, 1831); Prince foi a primeira narrativa da
  vida de uma mulher negra a ser publicada no Reino Unido.} ``A
escravidão é terrível para os homens'', escreveu, ``mas é muito
mais terrível para as mulheres''.

As histórias das mulheres escravizadas
passaram tempo demais na periferia dos estudos acadêmicos e do
engajamento social. Reenfocar suas experiências de vida
é fundamental, pois, nas palavras da historiadora
Stephanie Camp, ``a história das mulheres não apenas agrega ao que
sabemos, ela muda o que sabemos e como o sabemos''.\footnote{Stephanie
  M. H. Camp, Closer to Freedom: Enslaved Women and Everyday Resistance
  in the Plantation South (Chapel Hill: University of North Carolina
  Press, 2004), 3.}

É minha grande honra apresentar a história de Harriet Jacobs para os
leitores brasileiros. Sua narrativa clássica, \emph{Incidentes da vida
de uma escrava}, é um testemunho da violência emocional, física e sexual
à qual as mulheres eram sujeitadas nas mãos dos seus escravizadores. A
narrativa de Jacobs revela um dos aspectos mais íntimos da vida em
cativeiro: a violência sexual e a maternidade. Suas palavras assombrosas
nos oferecem um retrato austero e chocante do que significa ser uma
mulher e uma mãe escravizada, tornando \emph{Incidentes da vida de uma escrava},
de longe, a autobiografia mais importante do gênero. Além disso, a
narrativa é total e completamente sua. Os cativos eram proibidos de
ler e escrever, mas a alfabetização de Jacobs permitiu que ela
redigisse sua própria história de forma autêntica, sem precisar fazer
concessões. Isso é importante, especialmente porque os ex"-escravizados que
permaneceram analfabetos tiveram suas histórias contadas por terceiros
ou filtradas por propagandistas abolicionistas brancos, que tinham suas
próprias motivações políticas sobre como transmitir uma história para
leitores receptivos. Os críticos passaram décadas acreditando que a
autobiografia de Jacobs era uma obra de ficção, ou, pelo menos, não suas
próprias palavras. Foi só em 1987 que a historiadora Jean Fagen Yellin
provou a autenticidade do livro. Jacobs foi a primeira escrava fugitiva
a escrever sua própria narrativa nos Estados Unidos.