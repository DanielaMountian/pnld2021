
\textbf{Pero Magalhães de Gandavo} tornou"-se um nome tão
obscuro quanto o seu livro. Nem sempre utilizou o último nome, e sabemos que
``gandavo'' é a designação dada a quem nasce em Guantes, Flandres.  Desde a
Biblioteca Lusitana, de Diogo Barbosa Machado, de meados do século
\textsc{xviii}, algumas informações somaram"-se para a invenção histórica da
vida desse nome que teve a posteridade truncada. Diz"-se que fora natural de
Braga, que o pai era flamengo, que foi moço"-de"-câmara de Dom Sebastião, que
trabalhou na Torre do Tombo como copista, que permaneceu alguns anos no Brasil
cuja história escreveu, e que, após a publicação do livro, foi nomeado provedor
da fazenda da cidade da Bahia, cargo que, diz"-se, não exerceu. Teria aberto
uma escola na região entre o Douro e o Minho, onde também casara. A maior parte
das ações e funções institucionais que se lhe atribuem constituía muito do que a
um homem de letras era digno exercer; são provavelmente verossímeis narrativos
do gênero histórico, inventados por tradições biblio"-historiográficas de
escrita de \textit{vidas} de poetas; ou são notícias derivadas desses
verossímeis em vertentes historiográficas do século \textsc{xix}. O mistério que
ronda o desaparecimento de seu livro, aplica"-se a seu estranho sobrenome, que,
sem ascendência nem descendência certas, não se sabe hoje sequer a pronúncia.
       
\textbf{História da província Santa Cruz} (1576) foi lido como ``relato de
viajante'' ou como ``nossa primeira história'', entendido como testemunho de
impressões antigas dos portugueses nas terras d'além"-mar. Contudo, esta
simples história, ou tratado descritivo, da ``costa do Brasil'' teve circulação
muito restrita à época, o que leva a crer que foi recolhida e destruída após sua
impressão, não se sabe bem por quê. Permaneceu praticamente ignorada até 1837,
quando foi reconsiderada na edição e tradução de M.~Henri Ternaux, em Paris; no
século seguinte, foi ainda vertida para o inglês por John B.~Stetson Jr. A
obscuridade do livro nos séculos seguintes à sua publicação é tanto mais
estranha se se tem em vista que, por intermédio de uma elegia e um soneto de
Camões, o livro é dedicado a um varão de armas em carreira promissora nas Índias
portuguesas, tendo sido impresso pela mesma oficina tipográfica que compôs
\textit{Os Lusíadas} (1572), apenas quatro anos mais tarde. Diferente de um
testemunho empírico, o livro é composto conforme a ideia de gênero histórico,
retoricamente regrado, em que o historiador, apoiado pelo aconselhamento ético
da Igreja Católica, tem por fim exaltar, pelo discurso, ações virtuosas de
pessoas de caráter elevado e eventos providenciais. 

\textbf{Ricardo Martins Valle} é doutor em Literatura Brasileira pela
\textsc{usp}, e professor de História Literária na Universidade Estadual do
Sudoeste da Bahia, \textsc{uesb}.

\textbf{Clara Carolina S.\,Santos} é professora, mestre em Memória e em Linguística pela Universidade Estadual do Sudoeste da Bahia, \textsc{uesb}.


