\chapter[Introdução, por Eno T. Wanke]{Introdução}

\textsc{Rodolfo Coelho Cavalcante} (1919--1986) não foi apenas mais um
cordelista nordestino. Era também um líder, um benemérito da classe
sofrida e às vezes perseguida, devido à confusão que as autoridades
faziam, no início de sua carreira, entre os poetas populares e os
camelôs. 

 Quando comecei a escrever sua biografia, \textit{Vida e luta de Rodolfo
Coelho Cavalcante}, publicada em 1983, minha grande curiosidade era
saber como uma pessoa nascida em condições tão desfavoráveis, sem
muitas oportunidades, pôde alcançar ser o que ele foi: respeitado
escritor de cordel, líder de sua classe, ouvido atentamente até em
salas universitárias e autor de tantas obras renomadas. 

 Eu tinha razão. A vida de Rodolfo foi uma luta constante contra o
ambiente hostil, contra tudo e contra todos. Jorge Amado chegou a dizer
que meu trabalho ``parecia mais um romance de aventuras do
que uma biografia baseada em fatos''. 

 Confidenciei a Rodolfo, na amizade que nos unia: 

 -- Meu caro, você poderia ter sido, não um intelectual da poesia
popular, mas um cangaceiro, um bandido! 

 Ele riu e concordou. 

\section{Infância}

 Rodolfo nasceu em Rio Largo, Alagoas, filho de humildes operários.
Aliás, naquela ocasião, a população da cidade era formada basicamente
pelos trabalhadores de duas grandes fábricas de tecido hoje
desaparecidas, a tecelagem Progresso e a Cachoeira, cada uma originando
os dois grandes bairros da cidade. 

 Seu pai, Arthur Holanda Cavalcante, era alcoólatra, o que obrigou sua
mãe, Mariazinha, a trabalhar também na fábrica Cachoeira, não obstante
já ter dois filhos pequenos para cuidar quando Rodolfo veio ao mundo. 

 Uma curiosidade: Rodolfo nasceu em 12 de março de 1919, exatamente no
momento em que a fábrica dava o ``fuéte'', ou
seja, o apito que indicava o início do trabalho. Eram quatro os apitos
matutinos destinados àquela população sem relógio: às quatro da manhã,
como despertador; às quatro e meia para dar início à faina de fazer e
tomar café; às cinco e quarenta para o pessoal tomar suas posições
junto às máquinas; e, finalmente, o chamado
``fuéte'', um apito curto indicando o início
das atividades do dia, às seis horas da manhã. 

 Bem"-humorado, Rodolfo sempre atribuiu a essa coincidência o fato de
ter vindo ao mundo para trabalhar sem descanso. E repetia, sem
disfarçar uma ponta de orgulho e uma pitada de bravata: 

 -- Olhem que vim ao mundo no momento do
``fuéte''! Realmente, teve uma vida
trabalhosa. 

 Criado desde tenra idade pelos avós maternos, Florisbela e Antônio
Coelho de Lima -- conhecidos respectivamente por ``Dona
Belinha'' e ``Velho Coelho''
-- teve junto a eles, segundo recordava, ``a mais bela
fase de sua vida''. 

 Sua avó (a quem chamava carinhosamente de ``Mãe
Belinha'') mantinha uma ``escolinha de
ABC'', onde o menino aprendeu as primeiras letras. 

 Seu avô (seu querido ``Pai Coelho'') tinha
orgulho de suas ``habilidades infantis''.
Ensinava"-lhe poemas indecentes (para os padrões da época) e fazia"-o
recitar aquelas ``barbaridades'' perante os
vizinhos e amigos, que morriam de rir ao ver o menininho
declamando"-as com desembaraço. Adorava ser o centro das atenções. 

 Quando Rodolfo contou"-me esse episódio, observei que talvez aí
residisse a origem de sua tendência ao espetáculo e à comunicação. Ele
me deu razão: 

 -- Acho que foi isso mesmo -- concordou. 

 Nesse ínterim, seus pais mudaram"-se para Maceió, onde Arthur
trabalhava numa fábrica de sabão. Aos oito anos, Rodolfo
reintegrou"-se à sua família e começou a freqüentar a escola na
capital, e a trabalhar, para garantir a comida: tinha de buscar, todos
os dias, junto com seu irmão Aristófeles, um ano mais velho, de seis a
oito latas de água para os gastos da casa. 

 Desempregado, seu pai resolveu voltar a Rio Largo, outra vez como
operário têxtil, agora da fábrica Progresso. E Rodolfo passa a ter de
trabalhar, apesar de ainda freqüentar a escola, na qual não foi além do
terceiro ano primário, máxima escolaridade que recebeu na vida. 

 Depois das aulas, vendia nas ruas frutas, tapioca feita pela mãe e
caldo de cana. Aos sábados e domingos, ia para a feira de Rio Largo
para ``pegar frete'', ou seja, levar os
cestos de compras na cabeça para os fregueses das barracas. 

 Sua mãe, muito severa, não admitia desonestidade. 

 Certa vez, ele e o irmão roubaram uma bola novinha do grupo escolar.
Foram descobertos quando, inocentemente, levaram"-na para jogar com
seus colegas na rua. A surra da mãe deixou"-lhes as nádegas esfoladas.


 Outra vez, enquanto vendia caldo de cana, acabou perdendo, num jogo de
dados, o dinheiro que ganhara. Não teve dúvidas, completou, com água, o
caldo já vendido -- o que provocou queixas dos fregueses, que
pagaram caldo aguado, junto à mãe Maria. Nova
``pisa'' monumental. 

 Sua atividade poética começou cedo, quando tomou parte nas pastoris,
cheganças e reisados de Rio Largo e de Maceió, onde ganhou fama de
``tirador de versos'' de improviso. Quando o
tenente Juarez Távora passou por Rio Largo, logo depois da vitória da
Revolução de 1930, foi designado entre os alunos da escola para
saudá"-lo em versos: 

\begin{verse}
Salve Juarez Távora, \\
O militar glorioso, \\
Estrela do nosso exército,\\ 
Que será vitorioso \\
Marchando para o porvir,\\ 
Vibrando consciencioso. 
\end{verse}

 Quando tinha 11 anos, seu pai, cada vez mais afundado na bebida, perdeu
novamente o emprego, e a família mudou"-se outra vez para Maceió.
Rodolfo e seu irmão Aristófeles passaram a sustentar a família, fazendo
biscates, vendendo bilhetes de loteria e jornais. 

 Um dia, Rodolfo conseguiu um emprego fixo, bem de acordo com sua
vocação de comunicador: seis horas por dia, na porta das Lojas
Paulistas de Maceió (que fora do Nordeste chamavam"-se Lojas
Pernambucanas), fazia propaganda dos produtos, atraindo fregueses. 

 E Rodolfo era eficiente em seu labor. Inventava versos, adaptando
letras de canções conhecidas e até compondo melodias próprias para
cantar na porta da loja. Sua mãe, graças a ele e a seu irmão (pois seu
pai bebia tudo o que ganhava com suas loterias), conseguiu economizar
para comprar uma casa própria, modesta é claro, mas que os livrava do
aluguel... 

 A casa estava quase paga, felizmente, quando a imprensa de Maceió
decidiu fazer uma campanha contra aquele absurdo de obrigarem uma
criança franzina a trabalhar na porta de uma loja... E Rodolfo perdeu o
(para ele) ótimo emprego. 

 Que fazer? Procurar nova fonte de renda. Sua mãe conseguiu colocá"-lo
como estafeta da Western Cable Telegraph Company, a companhia de
telégrafo a cabo submarino, graças a um estratagema. Simulou a idade
mínima exigida para o emprego, apesar de, na época, ter dois anos
menos. É que ele ainda não havia sido registrado em cartório. Assim, em
sua certidão de nascimento consta ter Rodolfo nascido em 1917, não em
1919, como realmente era o caso. 

 O emprego durou sete meses apenas. Até o dia em que o dono de um
armazém de açúcar prometeu"-lhe uma gratificação caso ele ficasse
depois da hora para entregar um telegrama importante esperado pela
firma. Feito isso, como o homem não se coçasse, Rodolfo reclamou o
prometido. E o gerente, assim que o menino saiu, telefonou para o chefe
dos mensageiros da Western, que imediatamente o despediu, pois a
gratificação era proibida. 

 Com medo da ``pisa'' da mãe, Rodolfo não
contou nada ao chegar em casa. No dia seguinte, vestiu o uniforme,
colocou a perneira e saiu... em direção ao Recife, a pé, seguindo a
linha férrea, e depois como passageiro clandestino, pois não tinha
dinheiro para comprar uma passagem de trem. Foi sua primeira fuga de
casa, seu grito de independência. 

\section{Tempo de andanças}


 No Recife conseguiu se arrumar fazendo biscates e propaganda nas portas
de lojas. Quando voltou para casa, três meses depois, tinha uma quantia
razoável no bolso, roupa nova e tudo o mais. Isso foi em 1932. No ano
seguinte, novo problema em casa o fez fugir de novo, desta vez com seu
irmão Aristófeles. 

 Em 1934, nova fuga aconteceu, desta vez rumo ao sul. 

 Foi uma viagem cheia de aventuras, pois era necessário ganhar o
sustento de cada dia. E haja imaginação e criatividade para isso! 

 Foram, por exemplo, ``camelôs sem
mercadoria'', vendendo pedras roliças apanhadas num
riacho, às quais atribuíram poderes mágicos de cura. Moraram numa
cadeia pública, por camaradagem de um delegado com quem fizeram
amizade. A amizade terminou, no entanto, quando eles resolveram vender
comida aos presos: pescavam piabas num rio próximo, fritavam"-nas e
serviam junto ao feijão que também cozinhavam numa fogueirinha
improvisada, tudo sem a anuência do delegado responsável. Um dia em que
o feijão estava ralo por preguiça dos meninos, os presos foram se
queixar ao delegado, que os expulsou solenemente. 

 Chegaram a cruzar com Lampião e seu bando, que os capturou, mas, ao
verificar a sua insignificância, deixou-os ir, displicente. 

 Em Sergipe, juntaram"-se a uma humilde trupe de saltimbancos. E aí
começou, aos poucos, nova fase na vida de Rodolfo, que se tornou o
palhaço Pirulito num circo maior, de propriedade de Chocolate, com quem
percorreu o Nordeste inteiro. Em Sobral, Ceará, deixou o circo para
passar ao serviço de um charlatão vendedor de remédios falsificados,
chamado Vilela. Deixou"-o em Amarração, no Piauí. Dali, os garotos
seguiram para Fortaleza, onde Rodolfo voltou à sua antiga profissão de
propagandista de lojas. 

 Depois de diversas peripécias naquela capital (inclusive uma passagem
como investigador de polícia, não obstante sua pouca idade), um golpe
de sorte: Rodolfo, agora sozinho, pois seu irmão fora tratar da vida em
outra freguesia, ganhou uma pequena fortuna (para ele) no jogo do
bicho. 

 Aplicou todo o dinheiro em roupa, rótulos de remédio, celofane e
aparelhos de mágica para atrair a freguesia e iniciou uma carreira de
charlatão ambulante (a exemplo de Vilela) e camelô pela serra de
Ubajara, na divisa entre os estados do Ceará e Piauí. 

 Vagou assim pelos caminhos do Nordeste até meados de 1937, quando
novamente se engajou num circo, como palhaço Pirulito. Mas no ano
seguinte, sabedor de que estavam precisando de um professor primário em
Porto Alegre, hoje Luzilândia, no Piauí, candidatou"-se ao cargo e,
tendo apenas o primário incompleto, passou a lecionar como professor da
Prefeitura, situação em que permaneceu até fins de 1938. 

 Em 5 de janeiro de 1939, exatamente no dia em que seu pai falecia em
Maceió, Rodolfo, embora não sabendo disso, mas cheio de saudades de
casa, iniciou a volta para sua terra. 

 Mas muita coisa ainda havia de transcorrer antes de lá chegar. Em
Parnaíba, comprou um lote de folhetos de João Martins de Athayde,
iniciando assim sua carreira de vendedor de cordel. Em Camocim, no
Ceará, foi preso enquanto mercava seus folhetos na feira. Na época, os
poetas populares eram, como já observei, perseguidos pelas autoridades.


 Declarando que era poeta por profissão, mostrou seus documentos para o
delegado. Este, fã de poesia popular, duvidou de Rodolfo e desafiou"-o
a fazer ali mesmo um acróstico com seu primeiro nome. Embora meio
atrapalhado, pegou papel e lápis, escreveu a palavra \textit{Antônio}
na vertical e, verso por verso, foi compondo a estrofe: 

\begin{verse}
A vida, aqui neste mundo \\
Nos parece fantasia. \\
Temos sempre a nosso lado \\
Os prazeres, todo dia. \\
No meio de tais prazeres, \\
Inda se encontra os sofreres, \\
Os amargos da agonia. 
\end{verse}

O delegado ficou entusiasmado e Rodolfo foi levado para a casa do juiz,
onde perdeu a feira fazendo acrósticos para todos os que lá se
encontravam naquele domingo. Ao se queixar disso, os presentes
cotizaram"-se e deram"-lhe dinheiro para compensar o prejuízo. 

 Em Fortaleza, Rodolfo escreveu o primeiro cordel de sua vida, sobre a
tragédia sucedida na praia de Iracema, na qual se afogaram um poeta
conhecido, uma meretriz e um militar que tentara salvá"-los. Foi um
sucesso, chegando a vender, em poucos dias, três mil exemplares. Isso
lhe rendeu o suficiente para voltar para casa de trem. 

 Mas ainda não foi dessa vez que chegou. Ao sair de Recife, encontrou no
trem, por acaso, o dono do circo Strigni, então muito considerado no
Nordeste, que estava sem palhaço e insistiu em contratá"-lo. Não pôde
recusar e, um dia, quando o espetáculo estava prestes a começar, soube,
por um telegrama, da morte de seu pai, ocorrida três meses antes. 

 Mesmo assim fez o espetáculo. Strigni, muito comovido, deu"-lhe mais
do que havia prometido, e ele, finalmente, voltou para casa, em abril
de 1939, após uma ausência de cinco anos. 

 Seu irmão Aristófeles, por coincidência, também regressou naqueles dias
para casa. 

 Rodolfo empregou"-se numa firma de tecidos para prover o sustento da
casa, mas ganhava pouco. Escreveu uma peça em estilo circense, Os
milagres de santa Terezinha, na qual atuariam o próprio Rodolfo,
Aristófeles e sua irmã mais nova, Nazaré. O final era de grande efeito,
pois a própria santa Terezinha entrava em cena com vestes deslumbrantes
para realizar os milagres do título. A peça foi encenada em Maceió, em
Rio Largo e em Cachoeira e rendeu bem, além de pagar o vestido da santa
e os quatro músicos que faziam o acompanhamento musical. 

 Em agosto de 1939, os dois aventureiros partiram para o mundo,
novamente para o sul, vendendo ``remédios'' e
comprando ouro quebrado, ou seja, objetos de ouro que procuravam
adquirir por onde passavam e vender a peso, em Salvador. 

 Na capital baiana, resolveram montar uma trupe de teatro de fantoches
que levaram para o interior, de cidade em cidade, em direção ao Piauí.
Além do espetáculo com bonecos, havia outras atrações como as sessões
de mágica e equilíbrio. 

 Em Conceição do Canindé, no Piauí, onde chegaram em 23 de dezembro,
Rodolfo apaixonou"-se perdidamente por uma garota chamada Hilda e...
casou"-se com ela em 31 de dezembro de 1939! Foi um namoro"-relâmpago
que mudou completamente a vida de nosso herói. A trupe se desfez, e
Rodolfo resolveu criar raízes. 

 Entretanto, ainda vagou algum tempo pela Bahia, onde nasceu seu filho
primogênito, em Salvador, logo depois falecido. Quando sua mulher
esperava o segundo filho, resolveu voltar para Conceição do Canindé,
onde nasceu sua filha Israelita. 

\section{O cordelista Rodolfo}

 Suas andanças terminaram. Em 1942 estabeleceu"-se em Teresina, capital
do Piauí, onde começou, realmente, sua carreira de cordelista. Seu
primeiro folheto -- um sucesso comercial -- chamou"-se
\textit{Os clamores dos incêndios em Teresina} e falava sobre uma série
de incêndios de barracos nos mocambos da cidade, que a população
suspeitava ter sido promovido pelo próprio governo, desejoso de acabar
com aquela zona miserável. Rodolfo morava então naquele local, e o
folheto foi seu protesto, embora ``em linguagem
elevada'', como observou o censor que o aprovou. Era o
tempo da ditadura getulista. 

 Escreveu cerca de 34 folhetos em Teresina, em cujo Mercado Público
chegou a instalar uma banca de venda de folhetos e de miudezas que,
irrequieto como era, deixou sob a responsabilidade de um parente de sua
mulher -- o qual acabou lhe ``passando a
perna''. 

 Desanimado, sem dinheiro, resolveu mudar"-se para Salvador, onde se
instalou definitivamente em agosto de 1945, desenvolvendo
extraordinariamente seu trabalho nesta capital. 

 Segundo ele, ``a praça já estava feita'',
pois os soteropolitanos eram grandes consumidores de literatura de
cordel. Havia cerca de vinte cordelistas locais em atividade, fora os
que eventualmente vinham de Sergipe, de Alagoas ou do interior para
mercar seus folhetos. 

 Rodolfo pôs logo mãos à obra. 

 Um de seus primeiros êxitos em cordel foi \textit{A volta de Getúlio},
publicado dois dias depois da queda do ditador, ocorrida em 29 de
outubro. Era profético, inclusive. Começava pelas setilhas: 

\begin{verse}
Pode o porco ser granfino, \\
Pode o pato não nadar, \\
Pode o leão ser mofino, \\
Pode o gato não miar, \\
A galinha criar dente, \\
Gente virar serpente, \\
Mas Getúlio vai voltar!

Pode um padre ser batista, \\
Protestante não cantar, \\
Católico não ir à missa, \\
Freira deixar de rezar, \\
O ateu ter salvação, \\
Cobra tocar violão, \\
Se Getúlio não voltar!
\end{verse}

 O primeiro milheiro esgotou em dois dias! E Getúlio foi um dos assuntos
favoritos de Rodolfo nos próximos anos, tanto na deposição, em 1945,
como na campanha eleitoral e reeleição à presidência da República em
1950, e no suicídio ocorrido em 24 de agosto de 1954. 

 Seus folhetos, a maioria com oito páginas mais a capa, eram impressos
em tipografias. O resto do trabalho, a composição dos versos, a
obtenção da ilustração da capa (geralmente uma xilogravura de seu amigo
Sinésio Cabral ou um clichê obtido em jornal), a dobragem, o corte da
capa, o grampeamento -- era feito em casa, ``nas horas
vagas da noite'', quer pelo próprio Rodolfo, quer
``pelos meninos'' aos quais ele dava pequena
gratificação. 

 Seu ritmo de produção era intenso: nos anos iniciais, entre quatro e
oito títulos por semana, dele ou de outros autores, comentando os mais
diversos acontecimentos. Trabalhava, geralmente, na praça Cayru, junto
ao Mercado, um dos mais movimentados pontos de Salvador. Aos poucos,
foi ampliando seus ``pontos de venda'' com os
agentes, aos quais fornecia ``em grosso'' .
Chegou a ter trinta agentes, espalhados desde o interior da Bahia até
Alagoas e Sergipe. 

 O mais famoso de seus colegas de trabalho era, sem dúvida, Cuíca de
Santo Amaro, com quem mantinha boas relações e trocava seus folhetos.
Cuíca era uma figura excêntrica, sempre de cartola, fraque, óculos
escuros, extrovertido, engraçado. O assunto de seus folhetos era
geralmente escandaloso e sensacionalista. 

 Era a antítese do próprio Rodolfo que vendia seus cordéis de terno e
gravata, apelando para o sentimentalismo, o drama, a história, a emoção
dos leitores, e contando com sua capacidade pessoal de comunicador para
vender sua mercadoria em versos. 

 Rodolfo e Cuíca tornaram"-se nomes nacionalmente conhecidos quando a
revista \textit{O cruzeiro}, publicada pelos Diários associados no Rio
de Janeiro, de enorme circulação na época, publicou, em 26 de outubro
de 1946, reportagem de oito páginas sobre os dois poetas. O artigo
``Trovadores da Bahia'' foi escrito por
Odorico Tavares e ilustrado com fotos de Pierre Verger. 

 Rodolfo também editou revistas com letras de modinhas. Prosperou tanto
que pôde construir uma casa na rua Alvarenga Peixoto, no bairro da
Liberdade, onde sua família reside até hoje. 

 Rodolfo começou a numerar seus folhetos. Quando faleceu, essa contagem
aproximava"-se do número dois mil. 

 Perguntei"-lhe qual o seu folheto de maior êxito. ``Foi
\textit{A moça que bateu na mãe e virou cachorra}, escrito no fim da
década de 1940. Já vendeu centenas de milhares de exemplares, cerca de
trinta e seis edições.'' 

\section{Rodolfo e o governador}

 Além de folheteiro, Rodolfo teve ação decisiva como defensor e líder de
sua classe, ou seja, dos compositores e mercadores de cordel. 

 Sua primeira ação nesse sentido foi quase involuntária. Ao se instalar
em Salvador, em 1945, a profissão de folheteiro era vista com
desconfiança pelas autoridades. Mas não havia diretriz. Ninguém sabia
exatamente se vender cordel era bom ou mau, se o poeta popular devia
ser equiparado ou não aos camelôs, estes sim considerados indesejáveis,
pois vendiam mercadorias sem pagar impostos, o que prejudicava os
lojistas. 

 Quando Otávio Mangabeira subiu ao governo do estado da Bahia, em 1946,
Rodolfo lançou o folheto \textit{ABC de Otávio Mangabeira}, por estar o
assunto em voga. Um dia, encostou junto ao meio"-fio do local onde
trabalhava na praça Cayru um carro de chapa branca. Desceu um oficial
de gabinete: 

 -- É você o poeta Rodolfo Cavalcante? 

 -- Sim -- respondeu ele, tremendo nas bases. 

 -- Então, venha comigo. 

 Rodolfo fechou sua mala de folhetos, entrou no carro e logo depois
estava na presença do governador. Este logo o tranqüilizou: 

 -- Gostei muito dos versos do \textit{ABC} que me dedicou. 

 Parabéns e muito obrigado pela homenagem. Aprendeu a metrificar como? 

 -- Eu escrevo por intuição, doutor. Vou escrevendo e sai tudo
metrificado... 

 Bateram um papo cordial, enquanto o governador folheava alguns dos
folhetos que Rodolfo levava. Na despedida, disse Mangabeira: 

 -- Meu caro trovador, foi um prazer conhecê"-lo. O governador da
Bahia pode fazer algo pelo amigo? Deseja alguma coisa em especial? 

 -- Sim, governador. Quero liberdade para trabalhar, como poeta do
povo! Liberdade para vender meus folhetos em praça pública sem
proibições policiais! 

 -- Que está me dizendo? Você não tem liberdade para vender seus
folhetos? 

 -- Infelizmente, não, doutor. Chega um guarda e me deixa trabalhar,
chega outro e me proíbe. Muitas vezes minha família passa fome porque
me impedem de levar o pão a meus filhos. 

 O governador indignou"-se: 

 -- Isso não pode ser! Enquanto eu for governador da Bahia, os poetas
populares terão liberdade de se expressar e de ganhar a vida em
qualquer praça desse estado! 

 E deu a Rodolfo um de seus cartões de visita, escrevendo:
``O trovador Rodolfo Cavalcante tem liberdade de trabalhar
em qualquer praça do estado da Bahia.'' 

 Rodolfo saiu do palácio contentíssimo. E, naturalmente, com uma lição
de vida importante: as autoridades não eram bichos"-papões, e podiam
ser conquistadas para a causa dos poetas populares. 

\section{O primeiro congresso de trovadores e violeiros}

 Em abril de 1950 realizou"-se em Salvador o \textsc{iii} Congresso Brasileiro
de Escritores, realizado pela Associação Brasileira de Escritores, hoje
União Brasileira de Escritores (UBE). Rodolfo, muito curioso, foi ver
os debates. Os temas e as teses impressionaram"-no. Percebeu que ali
estavam os representantes de uma classe inteira -- a dos escritores
-- para, reunidos, localizar, equacionar e resolver seus problemas. 

 Vislumbrou logo a importância do evento. Saiu dali disposto a realizar
algo semelhante pela sofrida classe dos poetas populares, e começou a
trabalhar com entusiasmo, sem deixar, é claro, de produzir e vender
seus folhetos. 

 Não tinha mãos a medir na propaganda do congresso, tanto em Salvador
como em outros lugares. Mas só em 1954 as coisas começaram a tomar
forma. Em janeiro instalava"-se a Comissão Executiva do congresso, e
Rodolfo trabalhava como nunca. Teve desde o início o apoio dos irmãos
Tavares: Odorico -- dos \textit{Diários associados}, o autor da
reportagem em \textit{O cruzeiro} -- e Cláudio, diretor da Rádio
Sociedade da Bahia. 

 Conseguiu, no \textit{Diário da Bahia}, em 1954, uma coluna diária
chamada ``Quando falam os trovadores'', que
manteve durante um ano, mesmo depois de realizado o congresso. E
começou a editar, por conta própria, o jornal \textit{A voz do
trovador}, a serviço do congresso, do qual saíram sete números, a
partir de dezembro de 1954. 

 Nessa época, Zora Seljan, esposa do escritor e jornalista Antônio
Olinto, então alto funcionário do governo federal, esteve na Bahia e
entrou em contato com Rodolfo, animando"-o a ir para o Rio de Janeiro
para incrementar a propaganda de seu Congresso. 

 Em março Rodolfo viajou para o Rio, hospedando"-se em casa de Zora,
com passagem conseguida por Orígenes Lessa, que o conhecia desde a
época da reportagem na revista O cruzeiro. Por intermédio dele,
conseguiu uma entrevista, em 4 de abril de 1955, com... o próprio
presidente da República, que era, então, Café Filho! Convidou"-o a
comparecer ao congresso, e a ser seu presidente de honra. O presidente
aceitou, mas lamentou não poder comparecer. Mandaria Orígenes Lessa
como seu representante. 

 Andou também em São Paulo, onde travou contato com a editora Prelúdio
(mais tarde Luzeiro) para resolver problemas relacionados à edição de
cordéis. 

 Finalmente, depois de cinco anos de gestação e cerca de sete meses de
labor intenso, realizou"-se o I Congresso de Trovadores e Violeiros.
Durante cinco dias, de 10 a 5 de julho de 1955, reuniram"-se em
Salvador dezenas de poetas populares (cordelistas, violeiros,
repentistas) de Salvador e de outras partes do Brasil: Rio de Janeiro,
São Paulo, Ceará, Pernambuco, Sergipe, Alagoas... Houve espetáculos,
desfiles pelas ruas da cidade e também discussões em plenário. 

 Um dos principais objetivos do conclave seria -- pensou Rodolfo
desde o início -- a fundação de uma entidade que reunisse a classe,
o que de fato sucedeu. Foram aprovados os estatutos e fundada a
Associação Nacional de Trovadores e Violeiros -- ANTY. Assinaram a ata
de fundação 87 congressistas. 

\section{O Grêmio Brasileiro de Trovadores}

 A ANTV, infelizmente, não foi muito longe. Alguns membros da diretoria
quiseram transformá"-la em instrumento do Partido Comunista, então
bastante atuante no Nordeste. Rodolfo não se conformou com a
politização da associação e acabou pedindo demissão de seu cargo de
presidente, em 25 de agosto de 1956. 

 Nunca mais se ouviu falar na ANTY. Com a saída de Rodolfo, que era sua
alma e sua vida, ela morreu definitivamente. 

 Mas nosso herói não desistiu. Em 8 de janeiro de 1958 voltou à carga
fundando nova sociedade, o Grêmio Brasileiro de Trovadores (GBT) --
que, embora mais duradouro e fecundo, foi também efêmero. Houve um erro
fundamental de tática. 

 Levados pelo equívoco de um nome comum, Rodolfo, representante dos
trovadores populares, e Luiz Otávio, líder do movimento literário
trovista que iniciava na capital do país -- o Rio de Janeiro "-
resolveram que o novo grêmio conteria as duas classes, heterogêneas
demais para poderem se misturar e viver em harmonia. 

 O movimento trovista, cujo centro era no Rio, mas cujos cultores
espalhavam"-se por todo o Brasil, era formado por intelectuais
provindos da classe média (dentistas, engenheiros, médicos, advogados)
que praticavam a \textit{trova} -- ou seja, a composição poética de
forma fixa entendida como a quadra setissílaba de sentido completo com
rimas nos esquemas ABCB, ABAB, ABBA ou AABB -- como literatura, e
denominavam"-se trovadores. Eram literatos, beletristas, gente que via
nos seus versos apenas um motivo de ordem estética e satisfação
pessoal. Eis um exemplo de trova literária, de Aparício Fernandes: 

 
\begin{verse}
Parti do Norte chorando, \\
Que coisa triste, meu Deus! \\
-- Eu vi o mar soluçando \\
E os coqueirais dando adeus! 
\end{verse}

O trovador popular, era um profissional do verso. Dependia
essencialmente do produto de seu labor versificatório para seu sustento
e de sua família. 

 Que interesses em comum poderiam ter essas duas classes, tão díspares?
Nenhum, naturalmente. 

 Diga"-se de passagem que Rodolfo, quando entrou em contato epistolar
com Luiz Otávio, passou também a praticar a trova literária, o que deve
ter contribuído para a confusão. 

 Sua maior contribuição como presidente do GBT foi, também, uma façanha:
a organização, à distância, do II Congresso de Trovadores e Violeiros,
realizado em São Paulo em setembro de 1960. Para prepará"-lo, Rodolfo
viajou em abril, novamente, para o Rio de Janeiro (onde esteve com Luiz
Otávio e sua célebre ``academia dos
trovisqueiros'') e São Paulo, onde conseguiu audiência com
o prefeito Adhemar de Barros, que cedeu o estádio do Pacaembu para as
reuniões plenárias e a hospedagem de cem trovadores populares. 

 Foi bastante pitoresco esse congresso ao qual pude comparecer, já que
morava então em Santos. Um de seus grandes resultados -- pelo menos
para os trovadores literários -- foi a eleição da chamada
``nobreza trovadoresca''. 

 Em sua passagem pelo Rio, Rodolfo havia combinado com Luiz Otávio a
eleição de Adelmar Tavares, então ainda em atividade, como Príncipe dos
Trovadores. No entanto, em plenário, houve ``uma
traição'' de um dos membros da comitiva do líder carioca
(Zálkind Piatigórski) com a apresentação da candidatura do próprio Luiz
Otávio para aquele cargo honorífico. Este aceitou, mas com a condição
de que Adelmar fosse eleito Rei da Trova. Pelo resto da vida, Luiz
Otávio ostentou e se orgulhou de seu título. 

 Além do Príncipe Luiz Otávio e do Rei Adelmar, foram ainda eleitos, na
ocasião, Lilinha Fernandes, Rainha da Trova Literária; Cego Aderaldo,
Rei dos Cantadores; José Luiz Júnior, Príncipe, e Maria das Dores,
Rainha dos Cantadores. 

 Com o passar do tempo, a disparidade entre as duas classes que formavam
o GBT ficou cada vez mais evidente e, para encurtar a história, não deu
outra: em 10 de janeiro de 1966 os trovadores literários pediram
coletivamente demissão do GBT (que já tinha seções em todo o Brasil) e
fundaram a União Brasileira de Trovadores (UBT), composta
exclusivamente de trovadores literários. (Naquela época, eu residia em
Santos, e fui o último presidente do GBT santista e o primeiro da nova
UBT local.) 

 O GBT não resistiu ao golpe e desapareceu, como a primeira agremiação
de Rodolfo. 

 Mas Rodolfo, boa alma, não guardou mágoa dos trovadores literários,
mantendo relações de amizade com todos. Eu, particularmente, que o
conhecia por correspondência desde o início do GBT, sempre o admirei e
lhe tive muito apreço -- tanto que escrevi sua biografia. 

\section{Os últimos anos: as homenagens}


 Depois do II congresso, Rodolfo continuou a luta pela vida, em meio a
diversos reveses, entre os quais sua mudança para Jequié e a morte
trágica de sua filha mais velha. Israelita suicidou"-se por causa de
um amor mal resolvido com um trovador residente na Guanabara que
prometera casamento e depois desistira sem maiores explicações. A moça
não suportou a frustração, que resultou num problema mental sério. Para
tratar de sua filha, Rodolfo transferiu"-se para Salvador, em sua
casinha da Liberdade. 

 E continuou sua luta de cordelista e líder. 

 Em novembro de 1976 reuniram"-se em Salvador trovadores e violeiros na
I Feira Regional da Literatura de Cordel, evento idealizado, como de
costume, por Rodolfo, mas agora com o apoio oficial da Divisão de
Cordel da Fundação Cultural do Estado da Bahia, liderada por Carlos
Cunha e Edilene Matos, esta última chefe da Divisão de Literatura da
entidade. Aproveitando a ocasião, Rodolfo fundou sua terceira
agremiação classista, a Ordem Brasileira de Literatura de Cordel, que
conduziu com êxito até o fim de sua vida. 

 Novo revés sofreu Rodolfo em 1984, com o arrasador incêndio do Mercado
Modelo. Seus folhetos se salvaram, mas seu ponto desapareceu, já que,
com o Mercado inativo -- embora em obras de reconstrução "- não havia
mais o movimento de sua freguesia, tanto de seus humildes leitores como
dos turistas que compravam seus folhetos por curiosidade ou para suas
coleções. 

 A Fundação Cultural veio em seu socorro, oferecendo"-lhe um emprego
que o isentou de maiores preocupações com o sustento da família: além
de Hilda, sua mulher, três filhos (Rodolfinho, Ismoca e Nilza) e mais
cinco netos, filhos de Nilza (César, Rilda, Alan, Adriano e André). 

 No fim de sua vida, recebeu diversas homenagens consagradoras. Cito
algumas: 
\begin{itemize}
\item
 em 22 de junho de 1982, a Câmara Municipal de Salvador conferiu"-lhe
o título de Cidadão, nos termos da resolução no 490/81, do qual ele
muito se orgulhava; 

\item
 em 1982, Rodolfo era assunto de tese universitária na Sorbonne,
escrita pela pesquisadora Martine Kunz e intitulada \textit{Rodolfo
Cavalcante poète populaire du NordEst Brésilien;} 

\item
 em abril de 1983 saiu o meu livro \textit{Vida e luta do trovador
Rodolfo Coelho Cavalcante} (Rio, Folha Carioca Editora), uma biografia.
Foi uma de minhas melhores realizações. Sei que lhe deu muita alegria; 

\item
 em 23 de novembro de 1984, a Academia Brasileira de Letras (ABL)
dedicou uma sessão a Rodolfo, a fim de outorgar"-lhe a Medalha Machado
de Assis, láurea máxima a que um escritor brasileiro pode almejar, um
verdadeiro Prêmio Nobel tupiniquim. Falaram Orígenes Lessa e Jorge
Amado, os patronos da premiação, além do próprio Rodolfo, com um
magnífico improviso em versos. Compareceram, além dos acadêmicos,
muitos amigos de Rodolfo. Tive, então, a honra de hospedá"-lo em minha
casa no Rio; 

\item
 na mesma ocasião, foi homenageado pela Câmara Municipal do Rio de
Janeiro; 

\item
 em 18 de março de 1986, a Fundação Cultural do Estado da Bahia
organizou uma festa, um dia inteiro de homenagens a Rodolfo. Fui à
Bahia por conta da fundação, pois no programa constava o lançamento de
minha biografia. 
end{itemize}
 A imprensa, tanto os jornais como a TV e o rádio, deu cobertura
completa do evento, o qual se iniciou às dez horas da manhã com a
inauguração, no hall de entrada da Biblioteca Pública do Estado, de uma
exposição de livros, folhetos, diplomas, medalhas e recortes sobre o
homenageado. 
\end{itemize}

 A seguir, na praça Cayru, em frente à Banca dos Trovadores, deu"-se o
lançamento de \textit{Vida e luta do trovador Rodolfo Coelho
Cavalcante,} com a presença muito honrosa de altas figuras da cultura
baiana: Jorge Amado, Zélia Gattai, Vasconcelos Maia, Wilson Lins,
Myriam Fraga, Hildegardes Vianna, Edivaldo M. Boaventura, José
Calazans, Fernando da Rocha Peres e José Coelho de Araújo, entre
outros. 

 Bule"-Bule e Zé Pedreira animaram a festa com seus improvisos. Falaram
o poeta popular Armando de Oliveira e Silva, Jorge Amado, Edivaldo
Boaventura (secretário da Cultura), \mbox{e eu}. 

 O almoço, de mais de cem talheres, no restaurante Velhos Marinheiros,
contou com a presença da fina flor da sociedade soteropolitana.
Diversas personalidades pronunciaram"-se. 

 À tarde a festa prosseguiu, na Seção de Cegos da Biblioteca Estadual,
onde foi lançada, em braile, uma antologia dos folhetos de Rodolfo,
transcritos por Jerusa Maria Ferreira de Souza, deficiente visual que
``leu'' seu discurso com a ponta dos dedos.
Foi emocionante. 

 A homenagem seguinte foi na Universidade Católica de Salvador (UcSal),
onde os alunos de Comunicação do Instituto de Letras receberam o
homenageado cantando em coro o ``Hino a Rodolfo
Cavalcante'', composto e ensaiado para a ocasião. Fui,
também, um dos oradores, e tive oportunidade de demonstrar aos
estudantes a importância de Rodolfo para o cordel. 

 A festa terminou à noite, com a projeção de dois filmes na antiga e
imponente fortaleza"-prisão de Santo Antônio, onde fica hoje o Centro
de Cultura Popular. Um filme era sobre Rodolfo e no outro ele aparecia
lendo um folheto sobre Mestre Pastinha, o criador da capoeira angola
moderna, motivo do filme. 


\section{A morte}

 Meu último contato com Rodolfo foi em Vitória, em julho de 1986,
durante o VI Seminário Nacional da Trova. Eu modificara ligeiramente a
letra de seu ``Hino dos trovadores'',
adaptando"-o para o caso dos trovadores literários. Ele aprovou a nova
versão com entusiasmo. 

 No dia 7 de outubro de 1986, às 19h 30, Rodolfo, vinha do centro, da
tipografia onde mandava fazer seus folhetos. Desembarcou de ônibus na
ladeira de São Cristóvão, a um quarteirão de sua casa, a sobraçar sua
pasta. 

 Ao atravessar a rua, um automóvel, vindo desembestado ladeira abaixo,
colheu"-o e atirou"-o longe, no asfalto. O motorista nem se importou,
e prosseguiu em sua corrida sem prestar"-lhe socorro. 

 Rodolfo foi levado para o Hospital Simões Filho e depois transferido
para o Hospital Getúlio Vargas, onde haveria condições de operá"-lo.
Mas só às 21h 30 entrou na sala de operações. 

 Além de traumatismo craniano e esmagamento do baço, quebrara ossos do
rosto, do queixo e as pernas. Três vezes seu coração parou, e na
terceira não mais se recuperou. 

 A 1h 10 do dia 8 de outubro de 1986, a notícia consternadora: Rodolfo
não resistira e morrera ainda na mesa de operação. Da hora do
atropelamento até a morte, esteve inconsciente. 

 Grande multidão acompanhou o enterro, um acontecimento negro na cidade
de Salvador. Muitos oradores ilustres se fizeram ouvir, emocionados.
Edilene Matos, por exemplo, assim se expressou: ``A
Fundação perde duplamente: o grande poeta e o exemplar funcionário.
Nada nos consola, nem mesmo a grandeza de sabê"-lo revivido para a
história da cultura de nossa terra... E aqui, neste momento, nenhuma
palavra vale para exprimir o sentimento de perda, a não ser a certeza
de que seus folhetos ficarão suspensos nos cordéis da
Eternidade''. 

 O enterro foi filmado pelo cineasta Milton Dourado Lima. Rodolfinho
colocou alguns folhetos junto ao corpo do pai. Todos cantaram o
``Hino dos trovadores''. O caixão foi
envolvido na bandeira da ordem e levado, através do cemitério do Campo
Santo, até o carneiro n° 5858 da quadra A. 

 Falaram novos oradores, cantou"-se novamente o ``Hino dos
trovadores'' e o túmulo foi fechado. 

 Assim desapareceu o poeta. Não teve chance de dizer adeus a ninguém.
Ele mesmo previu o fato na trova enviada pouco antes de sua morte ao II
Concurso de Trovas de Belém do Pará: 


\begin{verse}
Quando este mundo eu deixar, \\
A ninguém direi adeus. \\
Dos poetas quero levar \\
Suas trovas para Deus. 
\end{verse}


\section{Repercussão da morte de Rodolfo}

 A morte do poeta repercutiu nos meios literários do Brasil e até do
exterior, o que se pode verificar pelas notícias de jornal, artigos e,
especialmente, poemas lamentando o fato. Para dar apenas um exemplo,
lembro"-me de ter recebido só de Campos dos Goitacazes, estado do Rio,
cinco sonetos sobre Rodolfo feitos por membros do Instituto Campista de
Literatura: Walter Siqueira, Alves Rangel, Elcy Amorim, Constantino
Gonçalves e Paulo Roberto de Aquino Ney. E assim, muitos outros. 

 A seguir, uma relação parcial dos folhetos de cordel sobre Rodolfo,
publicados após sua morte: 

\begin{itemize}

\item
Bule"-bule (Antônio Ribeiro da Conceição). \textit{Rodolfo vive entre
nós}. Salvador, out. 1988. 

\item 
 Caboquinho (José Crispim Ramos). Morre \textit{o papa do cordel Rodolfo
Coelho Cavalcante}. Feira de Santana, Edição da Prefeitura, 1986. 
\item 
 Cavalcante, Adolfo Moreira [filho de Rodolfo]. \textit{Recordações do
meu pai Rodolfo Cavalcante.} Salvador, 1987. 
\item 
 Cavalcante Filho, Rodolfo. \textit{O adeus a Rodolfo Cavalcante}
(1917--1986). Salvador, 1986. 
\item 

Gino Frey (Higino de Freitas Júnior). \textit{Pavana em dez pés para
Rodolfo Cavalcante.} Vitória, 1986. 
\item 

Ismoca (Isaías Moreira Cavalcante) [Filho de Rodolfo]. \textit{O
encontro de Rodolfo Cavalcante com Castro Alves no céu.} Salvador, abro
1987 . 
\item 

\_\_\_\_\_\_. Saudades \textit{do grande trovador brasileiro Rodolfo
Coelho Cavalcante, o rei do cordel.} Salvador, abr. 1988. 
\item 

Lara, José. \textit{Vida, luta e morte do trovador brasileiro Rodolfo
Coelho Cavalcante.} Belo Horizonte, out. 1987. 
\item 

Macedo, Téo. O \textit{Brasil perde um herói na literatura de cordel}
s.l., s.d. 
\item 

Queiroz, Lúcia Peltier de. \textit{Homenagem a Rodolfo Coelho
Cavalcante.} s.l., s.d. 
\item 
Silva, Expedito F. \textit{O adeus de Rodolfo Coelho Cavalcante à
cultura popular brasileira.} Rio de Janeiro, Universidade Federal do
Rio de Janeiro, 1987. 
\item 

Silva, Minelvino Francisco. \textit{O encontro de dois astros luminosos,
Rodolfo e Jorge Amado.} [Exemplar consultado em casa da família de
Rodolfo.] 

\end{itemize}

 Em 1987 o norte"-americano Mark Curran publica \textit{A presença de
Rodolfo Coelho Cavalcante na moderna literatura de cordel} (Rio de
Janeiro, Nova Fronteira / Casa de Rui Barbosa). O autor, professor da
Universidade do Arizona, é especialista no cordel brasileiro. O livro
foi escrito no mesmo período em que eu escrevia a biografia de Rodolfo.
Eu me correspondia com o autor, e chegamos a ``trocar
figurinhas'' sobre Rodolfo. O livro de Curran, ao
contrário do meu, que se preocupa com o aspecto humano do personagem,
examina e analisa sua obra cordelista. 

 A lei municipal no 5064, de 8 de junho de 1987, promulgada pelo
prefeito João Gilberto Sampaio, dá o nome de Rodolfo Coelho Cavalcante
a uma das ruas da cidade paulista de Ribeirão Preto. 

\section{Conclusão}

A importância de Rodolfo Cavalcante para o movimento cordelista pode ser
comparada à de outros dois grandes nomes: Leandro Gomes de Barros
(1865--1918) -- que montou, por volta de 1906, a primeira grande
folhetaria do Recife, praticamente iniciando o gênero -- e João
Martins de Athayde (1880--1959)  -- que em 1921 adquiriu as
impressoras, a loja, os títulos dos folhetos e a rede de distribuição
da folhetaria de Leandro, conseguindo expandi"-la ainda mais, por todo
o Nordeste. 

 Rodolfo produziu muito, mas não é sua atividade pessoal como autor e
comerciante de folhetos que o torna tão importante para o movimento
cordelista. Tampouco seu trabalho na indústria do cordel, que já estava
bem firmada quando ele apareceu. Nunca, aliás, possuiu impressora
própria. Sempre mandou fazer seus folhetos. 

 Sua ação foi a favor da classe sofrida dos folheteiros, que, em grande
número, viviam -- e vivem -- em feiras, mercados, praças e locais
de peregrinação a escrever e vender seus folhetos, para ganhar a vida e
sustentar, às vezes, família numerosa. 

 Quando Rodolfo surgiu, a vida dos cordelistas não era nada fácil.
Considerados meros camelôs pelas autoridades policiais, sua presença
era geralmente indesejada. Eram escorraçados, presos e maltratados,
como vimos. 

 Publicando artigos de jornal, fazendo contatos com as autoridades
(Otávio Mangabeira em 1946, Café Filho em 1955 e Ademar de Barros em
1960, por exemplo), organizando congressos (especialmente os eventos de
1955 em Salvador e de 1960 em São Paulo), fundando associações e
agremiações de classe, Rodolfo conseguiu modificar tal situação, dando
dignidade e representatividade aos cordelistas. Não foi por acaso que a
Academia Brasileira de Literatura de Cordel do Rio de Janeiro
escolheu"-o como patrono. 

 Em março de 1990, durante o I Congresso Nacional da Trova Literária,
realizado em Salvador numa das salas da Biblioteca Estadual, apresentei
o trabalho \textit{Rodolfo, cordel e trovismo} que termina com a
proposta de uma estátua a ele dedicada. 

 Tal proposta, aprovada por unanimidade e com aplausos do plenário, eu
já apresentara em palestra pronunciada no lançamento da biografia de
Rodolfo. 

 Jorge Amado, presente naquela ocasião, apoiara minha tese, indicando
até o local onde o monumento deveria ser erigido: a praça Cayru, aos
pés do Elevador Lacerda, onde Rodolfo mercou seus folhetos durante 44
anos de sua vida. E o maior romancista brasileiro vivo justificava: 

 -- Porque Rodolfo Coelho Cavalcante é um símbolo. Erigir uma estátua
a ele não seria apenas erguer uma estátua a uma pessoa, mas, sim, a
todos os cordelistas, violeiros e trovadores que sonham seus versos e
seus ideais neste Brasil tão sofrido e tão amado. 


