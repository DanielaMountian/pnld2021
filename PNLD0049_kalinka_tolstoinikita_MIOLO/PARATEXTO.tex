\chapter{Paratexto}\label{paratexto}


\section{Sobre o autor}\label{para1}

\noindent{}Aleksei Tolstói (1883--1945) nasceu no interior da Rússia e tornou"-se um
dos mais conhecidos escritores da União Soviética (\textsc{urss}), principalmente
a partir da década de 1930.

Sua mãe, Aleksandra Bostrom (1854--1906) havia se casado em 1873 com o
conde Nikolai Tolstói (1849--1900) --- que não era o conde Lev de
\emph{Anna Kariénina} ---, mas ela se apaixonara pelo funcionário
público Aleksei Bostrom (1852--1921). Mesmo pressionada pela família, a
jovem abandonou o conde, um bom partido, e foi atrás de seu amor,
ficando isolada da sociedade depois dessa separação
escandalosa.\footnote{\textsc{tolstaia}, Elena. \emph{Chaves da felicidade.
  Aleksei Tolstói e a Petersburgo literária} (\emph{Kliútchi stchastia,
  Aleksei Tolstói i literatúrnyi Peterburg}). \emph{Моscou: Nóvoie
  Literatúrnoie Obozriénie,} 2013, p.\,15.}

O pequeno Aleksei foi alfabetizado numa escola em Samara, mas em 1892,
``ano da fome'', a família mudou"-se para uma pequena propriedade rural
em Sosnovka de seu padrasto (\textsc{tolstaia}, p.\,287). Lá o menino continuou a
ser educado em casa, no começo pela mãe, depois por preceptores. Como
escritora, Aleksandra incentivou o filho a praticar a leitura, e ele
passou a nutrir grande amor pelas duas, pela mãe e pela literatura. No
entanto, em geral, era um aluno distraído e só se interessava pela
escrita.

Em 1897, graças a esforços da mãe, ele foi matriculado numa \emph{escola
real}\footnote{\emph{Escolas reais} (\emph{realnye utchílischa}),
  existentes na Rússia até 1917, eram colégios secundários voltados
  paras as ciências naturais e exatas.} em Sýzran e, no ano seguinte,
aceito em um ginásio em Samara.

Em 1900, concluiu"-se o processo de transferência do sobrenome do pai de
sangue para Aleksei, e o menino tornou"-se mais um representante da
antiga linhagem nobre Tolstói --- há quem questione a paternidade, mas
nunca se provou nada. Com o título de conde, o jovem de 18 anos pôde se
inscrever no Instituto Técnico em São Petersburgo, para onde se mudou em
1901 (\textsc{tolstaia}, p.\,287).

Na antiga capital do império russo, Aleksei teve contato com literatos
que marcaram sua poética, como Aleksei Riémizov e Fiódor Sologub
(\textsc{tolstaia}, p.\,9), expoente do simbolismo russo, cujo ápice se deu no
primeiro decênio do séc. \textsc{xx}. Nesse sentido, além dos poemas iniciais com
traços simbolistas, pode ser destacado o gosto de Tolstói pelo
folclore e pelo conto maravilhoso (foi autor de vários).

O Instituto Técnico que cursava foi fechado por alguns meses e ele
partiu, em fevereiro de 1906, para estudar na Europa. Ali conheceu Sófia
Dýmchits (1884--1963), futura pintora de vanguarda, por meio de um colega
de estudos em Dresden, irmão dela. A beldade tinha acabado de se separar
de seu primeiro marido, Isaac Rosenfeld, um estudante de filosofia,
irmão de Bella, depois esposa de Marc Chagall. Já Tolstói estava casado
com a filha de um proprietário de terras de Samara.

Depois de regressarem à Rússia e de alguns desencontros e
desentendimentos familiares, os jovens apaixonados se casaram.
Provavelmente inspirado pela independente e talentosa Sófia, que teve
aulas com pintores como Kuzmá Petróv"-Vódkin e Leon Bakst, Tolstói tentou dedicar"-se à pintura. No entanto, desencorajado por Bakst, ele
acabou enveredando de vez para a literatura. Essa passagem pelas artes
plásticas, em todo caso, deixou marcas em suas obras, sempre imagéticas
(\textsc{tolstaia}, p.\,27).

Ao lado da esposa, embrenhou"-se nos meios artísticos e literários,
estabelecendo contatos importantes. Em uma temporada em Paris, conheceu
os poetas Nikolai Gumilióv е Maksmilian Volóchin (também pintor), dos
quais se aproximou e com quem trabalhou.

A partir de 1908, o nome de Aleksei Tolstói, que colaborava em várias
publicações e projetos, começou a circular mais e ganhar notoriedade. Em
São Petersburgo, foi um dos fundadores (\textsc{tolstaia}, p.\,141) do cabaré
\emph{Brodiátchaia sobaka} (``Cachorro vadio'', 1911). Nesse café
histórico, apresentava"-se a nata da sociedade artística russa: Anna
Akhmátova, Nadiejda Teffi, Velimir Khlébnikov, Vsiévolod Meyerhold,
Vladímir Maiakóvski, entre outros.

Em 1912, Tolstói transferiu"-se para Moscou com Sófia e passou a
frequentar novos salões de literatos. Na futura capital do país, seu
casamento entrou em crise e ele conheceu a poeta Natália Krandiévskaia
(1888--1963), que se tornará sua terceira esposa.

O processo de separação de Sófia e Aleksei foi peculiar. Como a união
não tinha sido oficializada, eles se casaram para se divorciarem em
seguida, de modo que ela ganhasse o sobrenome Tolstaia e o título de
condessa e pudesse registrar sua filha, Mariana.

Em 1915, o escritor passou a morar com Natália e, em 1917, nasceu seu
primogênito, Nikita (1917--1994) --- o escritor tivera um filho em seu
primeiro casamento, mas elе morrera alguns anos depois.

Com a Revolução de Outubro de 1917, Tolstói decidiu sair do país. Parte
de uma corrente chamada \emph{smenoviékhovstvo,} ele discordava de
concepções leninistas, mas era favorável à mudança de governo. Em 1918,
acompanhado por sua esposa Natália e por Nikita, com um ano, Aleksei
saiu dе Moscou rumo a Odessa. Depois passou a viver em Paris,
envolvendo"-se em círculos intelectuais de emigrados. Nesse momento, já
editor"-chefe do suplemento de literatura da revista \emph{A Rússia
futura} (\emph{Gradúschaia Rossia}), era visto como o escritor em
ascensão da emigração russa --- prestígio que diminuiu com a
chegada de Ivan Búnin, ganhador do Nobel em 1933.

Foi justamente em Paris, em 1920, que Aleksei Tolstói começou a publicar
\emph{A infância de Nikita} (\emph{Diétstvo Nikity}), nos números 2, 3,
4, 5 e 6 da revista infantil \emph{Varinha verde} (\emph{Zeliónaia
pálotchka}). No entanto, o livro só saiu integralmente em Berlim pela
editora \emph{Guelikon} (1922) com um novo nome: \emph{Uma novela sobre
muitas coisas maravilhosas (A infância de Nikita).} Títulos desse gênero
estavam em voga na Europa, mas na \textsc{urss} a obra voltou a ser intitulada
como originalmente era. Foi também em Berlim, por sinal, que saiu o
primeiro volume integral, \emph{As irmãs,} da trilogia \emph{O caminho
dos tormentos} (editora Moscou).

Quando começou a publicar \emph{A infância de Nikita}, o autor passava
por problemas financeiros e estava isolado política e artisticamente,
vivendo de pequenas resenhas. Em 1921, partiu para Berlim com a promessa
de dirigir uma revista literária independente.

Ao regressar para a Rússia, em 1923, ela já era parte da União
Soviética. Aleksei Tolstói estava com 40 anos e, alguns anos depois,
tornou"-se um dos escritores mais influentes do país, perdendo talvez
apenas para Maksim Górki. Tolstói foi admitido na União do Escritores
Soviéticos, nomeado membro da Academia de Ciência, condecorado com a
Ordem de Lênin, laureado três vezes com o prêmio Stálin.

Por essa razão, o percurso literário de Aleksei Tolstói ficou um tanto
encoberto pela imagem do Conde Vermelho (como foi depois apelidado), cuja
figura e obra são inevitavelmente associadas ao realismo socialista. Com
o fim da \textsc{urss}, os livros do escritor passaram a ocupar um espaço menor nas prateleiras. Os leitores contemporâneos, como observou o
filólogo Mikhail Sviérdlov,\footnote{\textsc{sviérdlov}, Mikhail. Decifrando
  Aleksei Tolstói. De \emph{Pedro \textsc{i}} a \emph{Buratino}
  (\emph{Raschifrovka: Aleksei Tolstói. Ot ``Petrá I'' do
  ``Buratino''}). Curso 55. ``Literatura Russa do Século \textsc{xx}'', temporada
  6. \emph{Arzamas.} Disponível em: \textless{}
  https://arzamas.academy/materials/1476\textgreater{}. Último acesso: 09/04/21.} estão
reaprendendo a se relacionar com o legado desse autor. Da mesma forma
que, depois de 1990, muitos nomes, censurados durante a \textsc{urss}, foram
redescobertos, outros tantos foram destronados.

Após trinta anos, já se pode passar sua obra pela triagem do tempo e ver
o que dela restou --- e uma das criações destacadas por Sviérdlov é a
``encantadora'' \emph{A infância de Nikita.} A novela, escrita no
estrangeiro, não deixa de ser uma declaração de amor à Rússia, embora
seu conteúdo se afaste, especialmente nos capítulos publicados em Paris,
das tendências pedagógicas que surgiram no governo bolchevique,
filiando"-se à tradição literária russa do século \textsc{xix}.

Elena Tolstaia (2013) não à toa despendeu esforços para reconstruir a
trajetória do escritor em período anterior à \textsc{urss}, mostrando que, embora
ele usasse, como método de representação, principalmente o realismo,
flertou com o simbolismo e recebeu com otimismo o futurismo (organizou
dois encontros com Marinetti, quando este esteve na Rússia, em 1914).
Tolstói teve contato com artistas das mais variadas tendências e
estéticas, muitos dos quais acabaram vítimas do sistema que o enalteceu,
revelando as contradições daqueles tempos turbulentos\ldots{}

Fazendo o balanço, além de avô de Tatiana Tolstaia, filha de Nikita e
uma das escritoras mais destacadas hoje na Rússia, Aleksei foi poeta,
dramaturgo, romancista, novelista, contista, tendo várias de suas obras
adaptadas para cinema e televisão; aventurou"-se por vários gêneros:
deixou obras realistas e históricas, como a já referida trilogia \emph{O
caminho dos tormentos} (1921--40), que virou em 2017 série da Netflix, e
o romance histórico inacabado \emph{Pedro I} (1930--34); livros de
ficção científica, como \emph{Aelita} (1923), também adaptado para o
público juvenil, e \emph{O hiperboloide do engenheiro Gárin} (1927).
Escreveu contos e contos maravilhosos para jovens e crianças e assinou
uma adaptação de Pinóquio que é conhecida na Rússia inteira: \emph{A
chave de ouro ou as aventuras de Buratino} (1936).


\section{Sobre a obra}\label{para2}

\noindent{}Marcada pelo tom autobiográfico, \emph{A infância de Nikita,} novela
dedicada ao filho e escrita em terceira pessoa, descreve um ano na vida
de um menino de nove anos que morava numa propriedade rural em Sosnovka,
na província de Samara, ao lado da mãe, do pai e do preceptor, no fim do
século \textsc{xix}.

A obra é constituída por 35 capítulos curtos, o que pode ser explicado
por ter sido publicada de maneira seriada. Em todo caso, como é efêmera
toda infância, são efêmeros os episódios descritos.

Além de recordações de seus tempos de menino, o escritor estabelece,
como destaca Tolstaia (2013), diálogo com a obra de sua mãe, com a de
outros autores, como Lev Tolstói е Maksim Górki, e com a sua própria.
Exemplo disso é não só o nome Vevit, que Nikita esculpe em seu trenó e
aparece em obra posterior, como também o próprio Nikita, que surge num
conto de 1921, ``Nikita Chúbin'' (depois renomeado ``As aventuras
extraordinárias de Nikita Róschin''). Aqui o menino, ``também com nove
anos e em parte um herói autobiográfico'', testemunha a Revolução de
1917 e a Guerra Civil Russa (1918--21) (\textsc{tolstaia}, p.\,271).

\asterisc

\emph{A infância de Nikita} começa com Nikita acordando em uma ``manhã
ensolarada'' de inverno e oferece o motivo condutor da narrativa, o
despertar da vida de um menino: ``Nikita deu um suspiro ao despertar e
abriu os olhos'' (p.\,9); ``Nikita acordou de felicidade. A manhã estava
clara e gelada'' (p.\,63); ``Nikita foi despertado por pardais'' (p.\,142).

Após serem apresentados Nikita, o preceptor, a mãe e o melhor amigo ao
leitor, é a vez de a menina de laço na cabeça entrar em cena. Em uma
noite tediosa de inverno, com o ``Vento'' uivando no sótão, chega Lília,
filha de Anna Apolóssovna, amiga da mãe que veio de Samara para as
festas natalinas. Sem demora, Nikita se viu apaixonado:

\begin{quote}
--- Como está vermelho --- disse Lília ---, parece uma beterraba.

E de novo se inclinou sobre a caixinha. Seu rosto tornou"-se astuto.
Nikita parecia grudado à cadeira. Não sabia o que dizer e não conseguia
sair do quarto de modo algum. A menina ria dele, mas ele não ficou
magoado nem zangado, e somente a admirava. De repente, Lília, sem
levantar os olhos, fez uma pergunta com outra voz, como se entre os dois
houvesse um segredo sobre o qual discutiam (p.\,62).
\end{quote}

A religiosidade não é explorada em particular por Tolstói, mas os
festejos tradicionais, Natal e Páscoa, exercem uma função simbólica no
enredo. Foi na noite de Natal (nascimento) que Nikita beijou pela
primeira vez Lília, a qual ``estava com um lindo vestido de sainhas de
musselina e um grande laço branco no cabelo''. Em volta da árvore
lindamente enfeitada, meninos e meninas formaram uma roda, integrando o
protagonista a um coletivo, o que se repetiu na Páscoa, quando as
crianças da aldeia se reuniram.\footnote{Na Rússia soviética, os
  feriados religiosos foram proibidos, assim como os símbolos
  relacionados com eles (desde 1929). Portanto, deixaram de ser um
  assunto em livros infantojuvenis.} Ainda do ponto de vista simbólico,
Tolstaia (p.\,295) chama atenção para as estrelas se espalhando ao longo
da novela (na árvore de Natal, na tampa da caixinha de Lília, no nome da
égua, no céu), sempre ``boas, familiares'', prenúncios de alegrias.

O amor de Nikita é descrito com notas gótico"-românticas, que são
anunciadas desde a noite nevada que deu as boas"-vindas à espalhafatosa
Anna e aos seus dois filhos. O tom romântico é acentuado quando o menino
conta o sonho que tivera com o escritório de seu estranho bisavô --- que
a tudo abandonara por um amor não correspondido ---, onde havia um
relógio de parede com um vaso em cima. Ao ouvi"-lo, Lília propõe voltar
lá, misturando sonho e realidade, para descobrirem o que havia no vaso,
e os dois encontram um anel. Tomado de paixão, Nikita dá a ela o anel e
os versinhos que escrevera num papel amassado, ``dobrado oito vezes''.
Esse episódio dá uma nuance fantástica à história, pois o autor deixa
certos fenômenos sem explicação (\textsc{tolstaia}, p.\,300). Temos ainda
elementos (algumas citações de obras de sua mãe) que conferem o toque
gótico e de mistério: o gato saindo debaixo do sofá, os quadros ganhando
vida, os reflexos da lua no assoalho, etc.

A partida de Lília, que voltou com sua mãe e seu irmão para Samara,
abalou e deprimiu Nikita. O capítulo ``A casinha sobre rodas'' talvez
seja o mais sombrio do livro, que é essencialmente alegre e estrelado.
Ao entrar no casebre:

\begin{quote}
Ele se sentia perturbado e inquieto, como se algo terrível estivesse
para acontecer, algo que não poderia ser compreendido nem perdoado. Tudo
para ele --- a terra, os animais, o gado, os pássaros --- parou de ser
compreensível e familiar e se tornou alheio, hostil, ameaçador. Algo
aconteceria, algo incompreensível e pecaminoso, e aconteceria sem falta
(p.\,108).
\end{quote}

Esse estado de ânimo, além dos sentimentos ambíguos do despertar da
sexualidade, talvez anunciasse o aparecimento cinematográfico do pai,
que quase morreu afogado ao regressar para casa no meio de uma enchente.

Nikita vivia dentro de uma família tradicional, mas não muito estável.
Seu pai tinha compulsão por gastar somas enormes em coisas inúteis, e
sua esposa, Aleksandra Leóntievna, dava o ponto de equilíbrio, contendo
os assomos do marido.\footnote{Na Rússia soviética, a família deveria
  deixar de ser o centro da vida da criança, que seria educada e
  orientada pelo Estado e por suas organizações, como a dos Pioneiros.}
De qualquer maneira, os pais eram presentes no desenvolvimento do filho,
mesmo com um preceptor em casa, Arkádi Ivánovitch, que lhe passava
ditados e lições maçantes de aritmética, dos quais Nikita se defendia
com a imaginação.

A chegada do pai espirituoso, Vassíli Nikítievitch, com as águas do
degelo, marca o início da primavera e a quaresma. Na Páscoa
(renascimento), Nikita vai sozinho assistir às matinas e fica hospedado
na casa de um amigo do pai. Lá o menino tem contato com jovens e
crianças da aldeia que se distraem com jogos antigos --- algumas dessas
brincadeiras russas do século \textsc{xix} ainda são conhecidas.

O livro traz essa dimensão lúdica e revela certas convenções da época.
Enquanto Lília brincava em casa com sua boneca, os meninos divertiam"-se
ao ar livre e faziam seus próprios brinquedos. A guerra, brincadeira
clássica de meninos, surge na obra com tudo que lhe é de direito: soldados, munição
(bolas e neve) e fortaleza.

Passada a Páscoa, em plena primavera, Nikita, com as forças renovadas,
vê o mundo encher"-se de flores e de pássaros. Tudo ao redor começa a
fervilhar em busca de novas manifestações de vida:

\begin{quote}
\textls[-30]{Todos tentavam adivinhar o próprio destino, as joaninhas, os pássaros,
as rãs, sempre surpresas com tudo, sentadas sobre as barrigas na estrada
ou nos degraus da varanda. Com o piar do cuco todo o jardim mais
alegremente ia soando e farfalhando (p.\,140).}
\end{quote}

Logo depois de ter completado 10 anos, Nikita começou a cuidar de um
estorninho, Jeltúkhin --- colocava minhocas e moscas no parapeito da
janela e fez uma casa para ele. O passarinho ficou ali até o outono,
quando partiu com as aves migratórias para a África, perda que o
protagonista não sentiu --- pelo menos isso não foi assinalado pela
narração. A partida do pássaro pode remeter à posterior partida de
Nikita para o colégio e acentua a transitoriedade da vida, que é
evidenciada o tempo todo na obra. O campo, idealizado na propriedade de Sosnovka,
é um depositório de experiências de iniciação, mediadas por animais, por
fenômenos da natureza e pelas estações do ano.

Bem definidas na Rússia, as estações, além de acompanharem os
sentimentos de Nikita, demarcam a passagem do tempo no livro. Após a
efervescência primaveril, veio o verão trazendo banhos de rio, mas
também a seca. Preocupados e desanimados, deixaram de brincar e de
sorrir em casa. Mas a sorte mudou: ao entrar voando pela janela,
Jeltúkhin, tentando chamar a atenção de Nikita, pousou no estojo do
barômetro, cuja agulha indicava ``tempestade'': ``Desabou a chuva ---
forte, abundante, torrencial''.

Enquanto as plantações se preparavam para a sega, Nikita recebeu uma
carta de sua querida Lília. De tão entusiasmado, ele saiu galopando em
seu cavalo, sentindo a vida em plenitude, sem reflexões, sem responder à
cartinha de papel rendado: Nikita é sempre descrito no processo de
viver, sem psicologismos ou lições de moral.

O verão terminava, depois da feira em Pestravka e da sega. Com o outono,
sua mãe, para evitar que o marido gastasse o dinheiro que tinha em mãos
em dois vasos chineses inúteis, resolveu partir para Samara.

Em um dia outonal, cinzento e ventoso, Nikita despediu"-se de Sosnovka
para entrar na segunda classe do ginásio. Numa versão anterior do texto,
após essa informação, o narrador acrescentava: ``Com esse acontecimento
a infância dele terminou''.

\section{Sobre o gênero}\label{para3}

\noindent{}A obra de Tolstói, mesmo tendo um narrador em terceira pessoa, pode ser
classificada como memória literária, uma vez que foi baseada em
recordações de infância do autor e este gênero confessional,
diferentemente da biografia, não estabelece com o leitor um pacto de
expressar acontecimentos verídicos. Nas memórias literárias, as
fronteiras entre realidade e ficção não são facilmente discerníveis, e o
enredo pode ser embebido de muitas fontes. Ao mesmo tempo, desconhecer
a biografia do autor não tirará o prazer da leitura desta novela sobre a
passagem da infância. Não é tarefa fácil definir o que é uma novela, mas
podemos posicioná"-la, em relação à extensão e à complexidade do enredo,
entre o conto e o romance.

\vfill

\begin{flushleft}
\textsc{daniela mountian}
\end{flushleft}


\section{Sobre os colaboradores}

\noindent\textbf{Daniela Mountian} faz pós"-doutorado,
  ``Literatura infantil russa e brasileira: uma análise comparada
  (1919--1943)'', no Departamento de Teoria Literária e Literatura
  Comparada (\textsc{usp}), com apoio da Fapesp (processo nº 2017/24139-9).

\noindent\textbf{Moissei Mountian}, nascido na \textsc{urss}, é tradutor e parte do
conselho editorial da Kalinka. Traduziu muitos livros russos, como
\emph{O Diabo Mesquinho}, de Fiódor Sologub.

\medskip

\noindent\textbf{Irineu Franco Perpetuo} é tradutor, jornalista e crítico de
música. Entre suas muitas traduções, consta \emph{Vida e destino}, de
Vassíli Grossman.

\medskip

\noindent\textbf{Fabio Flaks} é artista plástico e arquiteto. Participou de
exposições coletivas e individuais (\textit{fabioflaks.com}).

\pagebreak
\thispagestyle{empty}
\movetooddpage
\thispagestyle{empty}
\begingroup\small

\vspace*{\fill}
\begin{flushright}
A INFÂNCIA DE NIKITA\\[6pt]
Copyright © Kalinka, 2021\\[6pt]
Tradução @ Moissei Mountian\\[6pt]
Tradução © Irineu Franco Perpetuo\\[20pt]

segunda edição, 2021\\[20pt]

Esta publicação está de acordo com a reforma ortográfica.\\[6pt]

A tradução baseou-se em texto incluído em \emph{A. N. Tolstói --
ízbrannoie}. Sverdlovsk (Ekaterimburgo), \emph{Sriédnie-Urálskoie
izdátelstvo}, 1982.\\[6pt]

As notas de rodapé são da tradução com colaboração da edição.\\[6pt]
\end{flushright}
\vspace*{\fill}

\vfill


\begin{flushright}
\textbf{Editora Kalinka}\\
São Paulo/SP\\
www.kalinka.com.br
\end{flushright}
\endgroup


