\chapter{Vida e obra de Lev Tolstói}

\section{Sobre o autor}


Nascido em 9 de setembro de 1828 na propriedade Iásnaia Poliana, em Tula,
subdivisão da Federação Russa, Tolstói perdeu os pais ainda menino e foi
educado por tutores e
depois por uma tia. Ele ingressou, em 1845, na Universidade de Kazan,
mas não chegou a concluí"-la, sendo, no fim das contas, um autodidata ---
era conhecedor de muitas línguas e filosofias.

Foi durante o serviço militar, passando pelo Cáucaso e pela Crimeia, que
começou a escrever. Seu primeiro texto, \emph{Infância,} saiu em 1852 na
revista \emph{O contemporâneo.} A fase de seus longos romances, de
\emph{Guerra e paz} (1863--69) até \emph{Anna Kariénina} (1873--78),
começou após seu casamento, em 1862, com Sófia Andréievna, união que
gerou 13 filhos.

Tomado por anseios e inquietações, o escritor intercalou vida literária
e outros interesses. ``Tolstói é importante não apenas por ser o mestre
insuperado do gênero que se costumou chamar `romance psicológico do
século \textsc{xix}', mas também por seus contos breves, diários e escritos
teóricos sobre pedagogia, arte e religião'', como relata a professora Aurora Bernardini.

A década de 1880 aprofundou uma série de crises existenciais por que o
escritor havia passado e o levou a uma fase que ele próprio definiu como
sua ``redenção moral''. Já praticante do vegetarianismo, ele abriu mão
dos direitos autorais de algumas obras em prol dos camponeses e
sistematizou uma série de preceitos filosóficos e religiosos que,
reunidos, passaram a ser conhecidos como \emph{tolstoísmo}, doutrina
baseada no cristianismo, mas acrescida de outras concepções, que
repercutiu no mundo todo e fez com que Tolstói fosse excomungado da
Igreja Ortodoxa. Seu último romance foi \emph{Ressurreição} (1889).

Morreu em 1910, aos 82 anos, de pneumonia. Além da vasta obra que legou à literatura, o escritor russo também deixou uma importante contribuição à
educação e à literatura para a infância e a juventude.
Além de ter fundado, em 1859, uma escola para
camponeses na propriedade onde nascera, Iásnaia Poliana, ele criou a
\emph{Cartilha} (1871--1872), enorme manual depois desmembrado na
\emph{Nova cartilha} (1875) e em quatro \emph{Livros russos para
leitura} (1875--1885).

\section{Sobre a obra}

Publicado em 1886, o romance \textit{A morte de Ivan Ilitch} faz parte de uma série
de textos breves que um já consagrado Tolstói produzira por volta dos
seus cinquenta anos de vida. Em uma narrativa leve, mas nem por isso
menos profunda, Tolstói conta a história da vida e da morte do senhor
Ivan Ilitch Golovín, promovendo para o leitor uma reflexão acerca da
vida e da morte, enquanto narra os caminhos da vida da personagem
principal, de sua atividade junto ao judiciário, por meio da qual chega
num ponto alto de sucesso, para então adoecer e sucumbir.

A crítica sutil, salpicada com amaciada ironia, expõe as futilidades e
pequenezas de um universo de classe média na Rússia Imperial.
Curiosamente, alguns apontamentos do autor, ainda que claramente
indiquem a que se remetem, são transcendentais, ultrapassando os limites
físicos e temporais daquela sociedade. Não obstante, os lampejos cada
vez mais humanizados que se presentificam nos pensamentos da personagem
principal a aproximam de qualquer um de nós, na condição de humanos, ao
a acompanharmos pela sua áspera jornada.

Logo que o romance se inicia, com o anúncio da morte de Ivan Ilitch
Golovín, funcionário público do Judiciário imperial russo, temos já
muito clara a impressão de como fora a vida desse distinto cidadão.

A notícia saíra em um periódico local, sendo lida por seus colegas de
trabalho. De certo modo consternados, eles debatem acerca da morte do
homem, ao mesmo tempo que refletem sobre as possíveis realocações --- em
razão do óbito --- de nomes nos cargos da repartição, bem como se é de
fato necessário ter de suportar o fardo social do comparecimento ao
velório.

Um desses homens, Piotr Ivanovitch, que havia, inclusive, estado na
convivência de Ivan Ilitch desde os tempos da escola de jurisprudência,
se sente compelido a comparecer ao velório, muito em razão dos fardos
dos costumes sociais, entretanto, isso não o impediu de comentar com a
esposa sobre possíveis benesses que sua família poderia receber pela
realocação dos funcionários nos cargos do tribunal.

Deseja ser breve no velório, afinal marcara um carteado com outros
companheiros, mas, ao chegar lá, é interpelado pela viúva que o chama
para conversar, delongando assim seu passatempo lúdico. Ela deseja
indagar"-lhe sobre a possibilidade do recebimento de uma pensão mais
gorda. Ao ouvir o pedido, Piotr Ivánovitch polidamente recriminou o
Estado por sua mesquinhez, indicando não crer ser possível o recebimento
de um benefício mais robusto.

Curioso é que tal diálogo se segue às reclamações da viúva das agruras
que ela vivera nos últimos momentos do defunto, posto que, enquanto este
moribundo, gritara incessantemente por três dias. Disso, observar"-se a
falta de empatia que a agora viúva sentira nas últimas horas de
sofrimento de seu marido. Inclusive, Tolstói dá a entender que o assunto
sobre o crepúsculo da vida de Ivan Ilitch se dera unicamente para
cumprir o protocolo social.

É realizada a cerimônia, e, após uma breve conversa com um funcionário
da casa, Piotr Ivánovitch vê que ainda é tempo de participar da
jogatina, de modo que se dirige à casa do amigo onde ela ocorria.

\subsection{A trajetória de Ivan Ilitch\break crítica ao funcionalismo público}

Findado esse trecho introdutório, Tolstói passa a narrar a história de
Ivan Ilitch. Desde o início, Ivan Ilitch é pintado como um sujeitinho
mediano, nascido em uma família de posses medianas, cujo pai atuava como
funcionário público, daqueles que, na definição de Tolstói, estão lá há
muito tempo para serem mandados embora, de modo que tem"-lhes que ser
elaborado um cargo, de atribuições fictícias e de salário nada fictício,
o qual preencham.

Aliás, toda vez que o Estado se presentifica na obra, queda claro como é
um Estado, excessivo, sedentário, acomodado. Tolstói escancara a
falência da intenção de Pedro \textsc{i} com sua tabela de patentes. Observa"-se
funcionários prevaricadores, preguiçosos, corruptos. As relações se dão
por interesse e o tráfico de influência parece ser o melhor de todos os
planos de carreira.

Ora, as críticas ao funcionalismo público, às burocracias ineficazes e
aos profissionais de qualidade duvidosa são lugar comum na literatura
russa desde os tempos de Gógol, sendo a questão também amplamente
abordada em Dostoiévski. Na literatura nacional, Nelson Rodrigues
explorará também essa temática.

Com tal quadro em mente, e observando o contexto em que a personagem
principal está inserida, vê-se que esta, sem grandes brilhos, Ivan
Ilitch, cresce, estuda, e tal qual o pai ingressa no funcionalismo
público, observando rigorosamente uma cartilha social, a qual segue e
deseja seguir sem imprimir qualquer reflexão acerca desse caminho.
Correspondia ao básico exigido. Em conversas banais, cumpria o protocolo
apresentando um descontentamento leve em relação ao governo, adotando um
tom de liberalismo moderado, adequando"-se então aos modismos da época,
mas sem correr o risco de se indispor com o fornecedor de seu pão
cotidiano.

\subsection{A vida familiar e social\break de um incômodo necessário a um fardo
insuportável}

Ivan Ilitch se casou porque era isso que se esperava de um homem
classificado como decente, mas, ao longo da história, ver"-se"-á que o
matrimônio constituiu pouco mais do que uma fonte de aborrecimento.

Apesar de um início aquecido, o casamento logo esfriara, e o
relacionamento amoroso dera lugar a conflitos cotidianos por motivos
pequenos, o que fazia com que Ivan Ilitch evitasse a companhia dos
familiares, buscando sempre se refugiar no trabalho.

Nota"-se a ausência de empatia, de desejo pelo convívio familiar. Todos
ao seu redor, em verdade, parecem estranhos que, por uma determinação
social, é-lhe obrigado o convívio. Não existe envolvimento afetivo, a
família é como se fosse, perante à sociedade, o mesmo que a toga é para
o cargo de juiz, um uniforme, uma formalidade, um elemento tradicional
que confere distinção a seu possuidor.

Trata"-se do completo esvaziamento do significado das relações
familiares.

Entretanto, a despeito dessa aparente dedicação ao trabalho, seu
verdadeiro prazer, onde sentia que sobressaíam seus talentos eram nos
jogos de baralho. Mas, também a despeito de sua aparente dedicação ao
trabalho, um remanejamento das cartas no funcionalismo público em sua
província ameaçou seu conforto e possibilidade de promoção.

Frustrado, decide agir. Viaja à capital do Império, buscando realocação
e, por sorte, encontra no trem conhecidos dispostos a lhe auxiliar. Com
isso, consegue a transferência, para um cargo distante daqueles que lhe
opuseram, com um ordenado de cinco mil rublos e mais três mil e
quinhentos de abono pela mudança.

O sucesso do novo posto faz com que até parentes, há muito sem interesse
em um contato, o procurem.

\subsection{O estampido de sucesso e o ocaso da vida de Ivan Ilitch}

Satisfeito e empolgado, muda"-se adiante de sua família, com o intuito de
preparar um bom lar, que impressionasse a todos.

Contudo, nos preparativos finais, Ivan Ilitch sofre uma queda, com a
qual acaba recebendo uma pancada no flanco esquerdo. Considerando esse
acidente um infortúnio sem importância, Ivan Ilitch segue sua vida,
recebe a esposa e filhos com felicidade e até brinca, imitando sua
queda. Assume o posto, ganhando bem, um ordenado do qual faltam apenas
quinhentos rublos para uma vida satisfatória. Porém, a euforia advinda
do sucesso profissional e da mudança rapidamente findou"-se. Passou a
sentir uma dor constante, que o molestava na hora da alimentação, além
de ter a impressão de possuir um cheiro repugnante saindo de sua boca. A
harmonia conjugal também se deteriorou, retornando ao cotidiano anterior
à nova vida na capital.

O incômodo constante leva Ivan Ilitch a uma constante irritabilidade.
Ora, por que isso acontecia com ele? Por que a dor não cessava? A
perpetuação do incômodo amplificava seu mal estar físico, uma vez que
contaminava seu psicológico a todo instante.

Busca médicos, renomados, famosos, alternativos. Cada qual parece trazer
um diagnóstico, uma cura milagrosa que ocorreria em breve. Ivan Ilitch
vê nos médicos uma empáfia. A arrogância da academia, olhando com
desprezo para seus pacientes, sem o menor interesse em ouvir suas
queixas, em uma postura similar com a qual ele tivera como juiz na
condução dos procedimentos jurídicos e na oitiva dos comuns com quem
cruzara.

O desespero vai lhe consumindo a consciência. Lembra"-se dos silogismos
da época de estudo, em que aprendera que se todo homem é mortal, e Caio
é homem, logo caio é mortal. Esse pensamento o entristece. Pouco importa
que Caio é mortal, o que importa é que ele, Ivan Ilitch, também o é.

Não desejando a morte, mas tendo que suportar o incômodo incessante,
Ivan Ilitch passa a utilizar ópio para se aliviar. Enquanto padece,
reflete sobre sua história. E nos caminhos sinuosos de seu pensamento
percebe que tem poucas lembranças de verdadeira felicidade.

Irrita"-se ao ver sua família seguindo a vida normalmente, vendo que
tornou"-se um estorvo, e suspeita que talvez sempre o tenha sido. Observa
sua mulher interesseira, vê sua filha em um relacionamento com outro
carreirista e seu filho como estudante trilhando passos similares aos
seus.

Chegou a odiar a mulher, atribuindo"-lhe a razão de seus malefícios. O
que era comum, sempre atribuía a terceiros as razões de seus
infortúnios. Ela também desejara sua morte, porém se assustou com seu
desejo, afinal, a morte do marido significaria o fim do ordenado.

\subsection{A redenção em um amigo inesperado}

No final da vida, tem um pouco de alívio ao ser cuidado pelo criado
Guerássim. Um mujique, jovem, de vistosa compleição. Ao contrário dos
médicos, e todos do convívio de Ivan Ilitch, tinha ânimo e boa vontade
para auxiliar um moribundo. Levantava"-lhe as pernas, única posição em
que Ivan Ilitch sentia um pouco de conforto, quedando"-se horas assim,
para o alívio de seu patrão, sem se importar ou emitir qualquer queixa.

Tinha bom espírito para todas as suas atitudes, opondo"-se à perpétua
tonalidade \emph{blasé} vivida Ivan Ilitch em praticamente toda a sua
vida. Também fora o único que não cultivou falsas esperanças no patrão
moribundo. Sabia que ele iria morrer e lidava naturalmente com isso, não
vendo perda de tempo em seu cuidado, apesar de que a causa poderia ser
vista como perdida. Fazer bem ao próximo lhe bastava.

Nesse momento, já em seus instantes finais, Ivan Ilitch percebe o qual
insossa foi sua vida. Percebe que a razão dessa colheita insípida foi
fruto de sua própria semeadura medíocre. Lamenta o tempo perdido com
frivolidades. Percebe que toda a pompa fora desimportante, e que tudo em
sua vida se resume, com exceção de algumas recordações de sua meninice,
a uma rotina afetada.

Similarmente a Memórias Póstumas de Brás Cubas, temos em A Morte de Ivan
Ilitch uma história que começa a partir da morte da personagem principal
e, no desenrolar do livro, conhecemos com mais detalhes sua biografia. A
diferença resta no fato de aquele saber que está morto, e sua narrativa
consiste em rememorações de sua história já consciente de seu fim,
enquanto com Ivan Ilitch, embora saibamos de seu fim, percorremos sua
história pari passu com suas angústias e temores.

E é no passar dessas linhas, é-nos apresentada a genialidade de Tolstói,
que tão brandamente conduz a narrativa. No livro, saímos muito
sutilmente, do universo cotidiano, mesquinho, ganancioso e pretensioso
das repartições, para irmos, paulatinamente, quase como em adágio, ao
mais íntimo psicológico de Ivan Ilitch.

Tolstói é tão delicado no conduzir de sua história que mal percebemos
que a causa dos malefícios do protagonista foi sua queda, tal como
quando sentimos uma dor inesperada e não nos lembramos de quando nos
ferimos em um momento anterior.

O livro começa retratando a mediocridade cotidiana, porém, a narrativa
que conduz o caminho longo até a morte vai se esvaindo do relato da
pequenez, até que a personagem principal atinge um algo mais sublime.

E tão delicadamente a personagem sai da sensação de penitência por ter
gastado prodigamente a vida, para um estado mais contemplativo e
benevolente frente à sua finitude.

Já em suas últimas horas, é visitado no cômodo que passara a habitar por
seu filho. Ao ver as olheiras do filho, Ivan Ilitch se compadece do
filho. É um momento de compaixão mútua, talvez um dos primeiros momentos
de sentimentos espontâneos narrados no oceano de afetação que até então
viviam

Nutre compaixão também pela esposa, apesar de dias antes ter nutrido por
ela sentimentos tão negativos.

Também se apieda da filha, de seu noivado com outro funcionário público,
tão medíocre quanto ele fora. Talvez sentisse um incômodo por isso, uma
vez que essa questão, sempre que se apresentava, dominava seu
pensamento, tal qual uma nota aguda em contraponto à melodia.

Essa compaixão é uma espécie de amor. A percepção de que nutria um bem
querer por sua família. Um amor fundamental, remissivo, que permitiu a
paz a Ivan Ilitch, para que ele se permitisse ofuscar diante de uma
realidade maior que a dele próprio. É nessa hora ele se libertou da dor,
do terror da morte e sutilmente, após dias de tormento, fora levado em
seu último suspiro.

A plenitude atingida é o que marca seu último sopro. Ainda tentara dizer
perdão a mulher e aos filhos, porém as palavras não saíram. Mas isso
pouco importava, sua essência e consciência estavam agora harmônicas.
Não à toa que os ``amigos'' de Ivan Ilitch que compareceram ao velório
notaram que seu cadáver guardava uma expressão mais significativa do que
a que tivera em vida.

Assim se encerra o romance. Assim somos confrontados com as reflexões de
um homem moribundo; observamos a penúria de um ser em confronto consigo
mesmo, nos momentos de certeza de que não escapará do destino final e
certo.

Além disso, também somos convidados a olhar para dentro e pensarmos
sobre nós. Desse modo, Ivan Ilitch está em nós. Quando olhamos atônitos
a passagem do tempo e consternados nos debatemos a pensar se aquele
passo fora bem dado, se aquela resposta foi adequada, se há sentido na
escolha feita, se o propósito impresso foi realmente valoroso. E dessa
leitura talvez tenhamos um pequeno susto, um levantar de sobrancelhas,
um menear de cabeça e com isso compreendemos a magnitude desse romance.

\section{Sobre o gênero}

Segundo o crítico Massaud Moisés, o romance, em comparação com o conto, é essencialmente multívoca, polivalente: ``constitui"-se de uma série de unidades ou células dramáticas''. Sua primeira característica estrutural seria, portanto, a pluralidade dramática: 

\begin{quote}
Em princípio, não há limite para os núcleos
dramáticos que podem compor a ação de um romance. Ao ficcionista,
cabe selecionar os que possuem a virtualidade de se organizar harmonicamente.
E essa escolha é o grande obstáculo que se lhe 
depara, dado que infinitas
possibilidades lhe são oferecidas ao
simples golpe de vista lançada sobre os acontecimentos diários.
A imaginação, com transfundi-los e transcendê-los, faz o resto,
avultando ainda mais o número de
caminhos revelados à sua intuição.\footnote{\textsc{moisés}, Massaud. \textit{A criação literária}. São Paulo: Cultrix, 2006, p.\,172.}
\end{quote}

Como desdobramento desse núcleo dramático plural, o crítico Massaud Moisés
vê na estrutura do romance um encadeamento de conteúdos, em que ``no fim de cada episódio, procura deixar sementes de mistério ou conflito para manter aceso o interesse do leitor. É raro que esvazie o recheio dramático duma célula antes de prosseguir, pois frustraria a curiosidade do leitor''\footnote{Ibidem, p.\,114.}

Após esclarecer a ação no romance, Moisés coloca em perspectiva o tempo romanesco. Afirma que sua estrutura linear e plural lhe impõe uma limitação temporal, que faz com que esse gênero não aborde a história da personagem desde seu nascimento, mas reduz"-lhe o passado a poucas linhas, essenciais para compreender"-lhe as ações e seu modo de ser, supreendendo a personagem no momento em que está madura para agir.

\begin{quote}
O tempo do romance é o histórico, assinalado pelo relógio ou pelo calendário, ou pelas convenções sociais. O presente é categoria dominante, em que pese às referências sumárias ao pretérito. Tudo se passa como se os dias, as semanas, os meses e os anos, de efêmera importância, significassem muito. A ação desenrola"-se por inteiro no presente, aqui e agora: condensado o pretérito em breves anotações.\footnote{Ibid., p.\,115.}
\end{quote}

Por fim, ao abordar o espaço do romance, Moisés ressalta o dinamismo acelerado desse gênero literário, causado pela sucessão de episódios, que implica a ausência de uma unidade espacial. É no deslocamento físico, continua o crítico, que as personagens procuram dar cabo da angústia ou atender ao apelo da aventura, criando, mesmo em uma única cidade, uma pluralidade de espaços que distingue o romance. Segundo o crítico tal dinanismo espacial imprime ao romance uma estrutura

\begin{quote}
plástica, concreta, horizontal. Assumindo as mais das vezes a perspectiva da terceira pessoa, o autor se coloca fora dos acontecimentos, ou concede a uma personagem a direção da narrativa. A vida imaginária sobrepõe"-se à vida observada: o romancista concentra"-se em multiplicar os expedientes narrativos, formulando sucessivas células dramáticas, sem atentar para os imperativos da verossimilhança. O enredo, além de visível, não esconde nada, não dissimula profundidades dramáticas ou psicológicas: com o predomínio da ação, tudo o mais se torna menos significativo.\footnote{Ibid., p.\,117--118.}
\end{quote}

\textit{A Morte de Ivan Ilitch}, ainda que escrita no século \textsc{xix}, é um
romance atual. Tanto em sua crítica ao funcionalismo público, com suas práticas
tão distantes de um ideal cívico, quanto, e principalmente, por nos
ofertar uma reflexão acerca da vida, de sua efemeridade e fragilidade,
da certeza da morte e como com ela lidamos, dos valores empregados e das
opções feitas ao longo do viver. A obra nos faz pensar sobre nossas
ambições e vaidades, sobre aquilo que almejamos, sobre o que nos traz
bem"-estar.

Em outros termos, o romance demonstra que o homem, através dos tempos, não
é tão distinto quanto possa parecer a um primeiro olhar. Ao contrário, é
bem similar a nós. Em suas angústias mais íntimas, no imbricado de suas
relações familiares e sociais, nos anseios da vida profissional e no
temor à morte somos todos tão humanos. Assim, é possível nos vermos
refletidos na personagem, ainda que de outro tempo e outro mundo.

A grandeza da literatura russa está em ser capaz de descrever
visceralmente as situações mais cotidianas. No romance em questão, somos
colocados próximos de uma pessoa que nunca vimos, de um mundo tão
distante, em um tempo longínquo e em uma situação provavelmente distinta
da nossa quando da leitura, ainda assim, nos identificamos e nos
compadecemos, amplificando aquilo que há de mais humano em todos nós.