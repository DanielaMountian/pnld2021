
\part{Contos}

\chapter{O caso de Rute}

\hfill\emph{A Valentim Magalhães}

\bigskip

Pode abraçar sua noiva! disse com bambaleaduras na papeira flácida a
palavrosa baronesa Montenegro ao Eduardo Jordão, apontando a neta, que
se destacava na penumbra da sala como um lírio irrompido dentre os
florões da alcatifa.

Ele não se atreveu e a moça conservou"-se impassível.

--- Não se admire daquela frieza. Olhe: eu sei que Rute o ama, não porque
ela o dissesse --- esta menina é de um melindre de envergonhar a própria
sensitiva --- mas porque toda ela se altera quando ouve seu nome. Outro
dia, porque uma prima mais velha, senhora de muito respeito, ousasse pôr
em dúvida o seu bom caráter, a minha Rute fez"-se de mil cores e tais
coisas lhe disse que nem sei como a outra a aturou!

Agora, que o senhor vem pedi"-la, é que eu lhe declaro que estava morta
por que chegasse este momento. Apreciei"-o sempre como um coração e um
espírito de bom quilate.

--- Oh! minha senhora\ldots{}

--- Não lhe faço favor. Além disso, Rute está com vinte e três anos;
parece"-me ser já tempo de se casar. Há de ser uma excelente esposa: é
bondosa, regularmente instruída, nada temos poupado com a sua educação.
A mãe teve só esta filha e foi rigorosíssima na escolha das mestras e
das amigas; o padrasto tratava"-a também com muita severidade, embora
fosse carinhoso. Desde que ele morreu que nos falta alegria em casa\ldots{} A
mulher, coitada, como sabe, ficou paralítica. Foi um rude golpe\ldots{} O que
tenho a dize"-lhe, portanto, é isto: afirmo"-lhe que Rute o adora e que
não há alma mais cândida que a sua. Aí a deixo por alguns minutos; se é
o respeito por mim que lhe tolhe as palavras, concedo"-lhe plena
liberdade.

Eduardo fixou na noiva um olhar apaixonado. Na sua brancura de pétala de
camélia não tocada, Rute continuava em pé, no mesmo canto sombrio da
sala. Os seus grandes olhos negros chispavam febre e ela amarrotava com
as mãos, lentamente, em movimentos apertados, o laço branco do vestido.

A baronesa acrescentou ainda, carregando nas qualidades da neta e
fazendo ranger a cadeira de onde se erguia:

--- Rute nunca foi de lastimeiras, e, apesar de mimosa e de aparentemente
frágil, tem boa saúde. Um bom corpo ao serviço de uma excelente alma.
Dirão: ``Estas palavras ficam mal na tua boca!\ldots{}'' Pouco importa; são a
verdade. Tenho outras netas, filhas de outras filhas; tenho criado
muitas meninas, minhas e alheias, mas em nenhuma encontrei nunca tanta
altivez digna nem tanta pudicícia. Aí lha deixo; confesse"-a!

A velha saiu.

Todos os rumores da rua rolaram confusamente pela sala. A porta que se
abriu e fechou trouxe, numa raja de luz, os repiques dos sinos, o rodar
dos veículos, o sussurro abominável da cidade atarefada; mas também tudo
se extinguiu depressa. A porta fechou"-se, as janelas voltadas para o
jardim mal deixavam entrar a claridade, coada por espessas cortinas
corridas, e os noivos ficaram sós, silenciosos, contemplando"-se de face.

\asterisc

O bisavô de Rute, primeiro barão da família, fora um colecionador
afincado de móveis e de outros objetos dos tempos coloniais. Súdito de
d. João \textsc{vi}, de que a sua admirável memória acusava ainda todos os traços
já aos noventa e oito anos, era sempre o seu assunto predileto a
narração dos sucessos históricos presenciados por ele. À proporção que
se ia afastando de seus dias de moço, mais aferrado se fazia aos gostos
e às modas do seu tempo.

Só se servia em baixela assinada com os emblemas da casa bragantina e a
propósito de qualquer coisa dizia, fincando o queixo agudo entre o
indicador em curva e o polegar: --- ``Lembro"-me de uma vez em que a d.
Carlota Joaquina''\ldots{} Ou então: --- ``Em que d. João \textsc{vi}, ou d. Pedro \textsc{i}'',
etc. E em seguida lá vinha a descrição de um \emph{Te Deum}, ou de uma
procissão, a que a sua imaginação facultosa emprestava as mais
brilhantes pompas. A família tinha um sorriso condescendente para aquele
apego, já sem curiosidade, à força de ouvir repetir os mesmos fatos. Os
amigos evitavam tocar, de leve que fosse, em assuntos políticos,
receosos da longura do capítulo que o barão a propósito lhes despejasse
em cima; mas só ele, o bom, o fiel, nada percebia, e, com os olhos no
passado, toca a citar ditos e atitudes dos imperadores e a curva"-se
numa idolatria pelo espírito boníssimo da última imperatriz.

Cadeiras pesadas, de moldes coloniais, largas de assento, pregueadas no
couro lavrado de coroas e brasões fidalgos, uniam as costas às paredes,
de onde um ou outro quadro sacro pendia desguarnecido e tristonho.

Assim o quisera o pai, que até mesmo na hora suprema rejeitara um belo
crucifixo que lhe oferecia o padre, voltando os olhos suplicemente para
um outro crucifixo mais tosco, erguido sobre a cômoda, e que pertencera
a d.\,Pedro\,\textsc{i}.

Para ele, naquela cruz não estava só o Cristo; estava, de envolta com o
respeito pelos monarcas extintos, a lembrança dos seus folguedos de
moço. Talvez mesmo, num volteio súbito da memória, se lembrasse das
festas religiosas em que namorara, à sombra dos conventos, a sua
primeira mulher, e beliscara com freimas amorosas os braços gordos da
Janoca, a mulatinha mais faceira de então\ldots{} Quem sabe? talvez que na
hora da morte não se possa só a gente lembrar das coisas sérias.
Qualquer hora vivida pode ser recordada rapidamente, sem tempo de
escolha.

Como a Janoca não pertencera à história, a família ignorou"-a; e pelo ar
gélido daquela galeria de espectros palacianos não apareceu nenhum
requebro quente de mulatinha risonha que lhes desmanchasse a compostura.

Como seu pai, o segundo barão morreu quase centenário, deixando ainda
frescalhona a sua terceira mulher, que, por mau gosto ou mau conselho,
reformou o seu interior confundindo estilos, pondo no mesmo canto móveis
de gosto e utilidades opostas. A extravagância não conseguira destruir
completamente a severidade da sala.

As virgens dos quadros, de longo pescoço arqueado e rosto pequenino,
gozavam ali o doce sossego de uma meia tinta religiosa.

Mas lá dentro, os dias passavam entre o tropel da criançada, os sons do
piano e a confusão dos criados.

E era por isso que todos fugiam lá para dentro e que só Rute, nas suas
horas de inexplicável tristeza, se encerrava ali, em companhia da Madona
da Cadeira e da Virgem de S. Sixto.

Era nessa mesma sala que ela ainda estava, muda e pálida, em frente do
seu amado.

--- Rute\ldots{} balbuciou Eduardo.

Mas a moça interrompeu"-o com um gesto e disse"-lhe logo, com voz segura e
firme:

--- Minha avó mentiu"-lhe.

O noivo recuou, num movimento de surpresa; foi ela quem se aproximou
dele, com esforço arrogante e doloroso, deslumbrando"-o com o fulgor dos
seus olhos belíssimos, bafejando"-lhe as faces com seu hálito ardente.

--- Eu não sou pura! Amo"-o muito para o enganar. Eu não sou pura!

Eduardo, lívido, com latejos nas fontes e palpitações desordenadas no
coração, amparou"-se a uma antiga poltrona, e olhou espantado para a
noiva, como se olhasse para uma louca. Ela, firme na sua resolução,
muito chegada a ele, e a meia voz, para que a não ouvissem lá dentro, ia
dizendo tudo:

--- Foi há oito anos, aqui, nesta mesma sala\ldots{} Meu padrasto era um homem
bonito, forte; eu uma criança inocente\ldots{} Dominava"-me; a sua vontade era
logo a minha. Ninguém sabe! oh! não fale! não fale, pelo amor de Deus!
Escute, escute só; é segredo para toda a gente\ldots{} No fim de quatro meses
de uma vida de luxúria infernal, ele morreu, e foi ainda aqui, nesta
sala, entre as duas janelas, que eu o vi morto, estendido na
essa\footnote{Catafalco, onde se coloca um caixão ou a representação de
  um morto.}. Que libertação que foi aquela morte para a minha alma de
menina ultrajada! Ele estava no mesmo lugar em que me dera os seus
primeiros beijos\ldots{} ali! ali! oh, o danado! como lhe quero mal agora!
Não fale, Eduardo! Minha avó morreria, sofre do coração; e minha mãe
ficou paralítica com o desgosto da viuvez\ldots{} Desgosto por aquele cão! e
ela ainda me manda rezar por sua alma, a mim, que a quero no inferno! Às
vezes tenho ímpeto de lhe dizer: ``Limpa essas lágrimas; teu marido
desonrou tua filha, foi seu amante durante quatro meses\ldots{}'' Calo"-me
piedosamente; e acodem todos: que não chorei a morte daquele segundo pai
e bom amigo!

\asterisc

--- É isto a minha vida. Dou"-lhe a liberdade de restituir a sua palavra
à minha família.

Rute falara baixo, precipitando as palavras, toda curvada para Eduardo,
que lhe sentia o aroma dos cabelos e o calor da febre.

Em um último esforço, a moça fez"-lhe sinal que saísse e ele obedeceu,
curvando"-se diante dela, sem lhe tocar na mão.

\asterisc

O outro está morto há oito anos\ldots{} ninguém sabe, só ela e eu\ldots{} Está
morto, mas vejo"-o diante de mim; sinto"-o no meu peito, sobre os meus
ombros, debaixo de meus pés, nele tropeço, com ele me abraço em uma luta
que não venço nunca! Ninguém sabe\ldots{} mas por ser ignorada será menor a
culpa? Dizem todos que Rute é puríssima! Assim o creem. Deverei
contenta"-me com essa credulidade? Bastará mais tarde, para a minha
ventura, saber que toda a gente me imagina feliz? O meu amigo Daniel é
felicíssimo exatamente por ignorar o que os outros sabem. Se a mulher
dele tivesse tido a coragem de Rute, amá"-la"-ia ele da mesma maneira? Se
a minha noiva não me tivesse dito nada, não seria o morto quem se
levantasse da sepultura e me viesse relatar barbaramente as suas horas
de volúpia, que me fazem tremer de horror! E eu, ignorante, seria
venturoso, amaria a minha esposa, à sombra do maior respeito e com a
mais doce proteção\ldots{} E assim?! Poderei sempre contar o meu ciúme e não
aludir jamais ao outro?

Ele morreu há oito anos\ldots{} ela tinha só quinze\ldots{} ninguém sabe! só ela e
eu!\ldots{} e ela ama"-me, ama"-me, ama"-me! Se me não amasse e fosse em todo
caso minha noiva di"-me"-ia do mesmo modo tudo? Não\ldots{} parece"-me que
não\ldots{} não sei\ldots{} se me não amasse\ldots{} nada me diria! Daí, quem sabe?
\emph{Amo"-o muito para o enganar}\ldots{} parece"-me que lhe ouvi isto! Se eu
pudesse esquecê"-la! Não devo adorá"-la assim! É uma mulher desonrada. A
pudica açucena de envergonhar sensitivas é uma mulher desonrada\ldots{} E eu
amo"-a! Que hei de fazer agora? Abandoná"-la\ldots{} não seria digno nem
generoso\ldots{} Aquela confissão custou"-lhe uma agonia! Se ela não fosse
honesta não afrontaria assim a minha cólera, nem se confessaria àquele
que amasse só para não sentir a humilhação de o enganar. E o que é por
aí a vida conjugal senão a mentira, a mentira e, mais ainda, a mentira?

O outro está morto\ldots{} ninguém sabe, só ela e eu! Ela e eu! e que nos
importam os outros, tendo toda a mágoa em nós dois só?! Antes todos os
outros soubessem\ldots{} Não! Que será preferível ser desgraçado guardando
uma aparência digna, ou\ldots{}? Não! em certos casos há ainda alguma
felicidade em ser desgraçado\ldots{} Ela ama"-me\ldots{} eu amo"-a\ldots{} ele morreu há
oito anos\ldots{} já nem lhe falam sequer no nome\ldots{} Ninguém sabe! ninguém
sabe\ldots{} só ela e eu!

Eduardo Jordão passava agora os dias em uma agitação medonha. Atraía e
repelia a imagem de Rute, até que um dia, vencido, escreveu"-lhe
longamente, amorosamente, disfarçando, sob um manto estrelado de
palavras de amor, a irremediável amargura da sua vida. ``Que esquecesse
o passado\ldots{} ele amava"-a\ldots{} o tempo apagaria essa ideia, e eles seriam
felizes, completamente felizes.''

O casamento de Rute alvoroçava a casa. A baronesa ocupava toda a gente,
sempre abundante em palavras e detalhes. Só Rute, ainda mais arredia e
séria, se encerrava no seu quarto, sem intervir em coisa alguma.

Relia devagar a carta do noivo, em que o perdão que ela não solicitara
vinha envolvido em promessas de esquecimento. Esquecimento! como se
fosse coisa que se pudesse prometer!

A moça, de bruços na cama, com o queixo fincado nas mãos, os olhos
parados e brilhantes, bem compreendia isso.

Entraria no lar como uma ovelha batida. O perdão que o noivo lhe mandava
revoltava"-a. Pedira"-lhe ela que lhe narrasse a sua vida dele, as suas
faltas, os seus amores extintos? Não teria ele entendido a enormidade do
seu sacrifício? Seria cego? seria surdo?\ldots{} dono de um coração
impenetrável e de uma consciência muda? As suas mãos estariam só tão
afeitas a carícias que não procurassem estrangulá"-la no terrível
instante em que ela lhe dissera --- eu não sou pura? Ou então por que
não a ouvira de joelhos, compenetrado daquele amor, tão grande que assim
se desvendava todo?! Ele prometia esquecer! mas no futuro, quando se
enlaçassem, não evocariam ambos a lembrança do outro? Talvez que, então,
Eduardo a repelisse, a deixasse isolada em seu leito de núpcias, e
fugindo para a noite livre fosse chorar lá fora o sonho da sua
mocidade\ldots{}

Sim, a sua noite de núpcias seria uma noite de inferno! Se ele fosse
generoso ela adivinharia através da doçura do seu beijo os ressaibos da
lembrança do primeiro amante; e quanto maior fosse a paixão, maior seria
a raiva e o ciúme.

Esquecimento!\ldots{} sim\ldots{} talvez, lá para a velhice, quando ambos, frios e
calmos, fossem apenas amigos.

Rute pensou em mata"-se. Viver na obsessão de uma ideia humilhante era
demais para a sua altivez. Desejou então uma morte suave, que a levasse
ao túmulo com a mesma aparência de cecém cândida, de envergonhar a
própria sensitiva.

Queria um veneno que a fizesse adormecer sonhando; e quanto dera para
que nesse sonho fosse um beijo de Eduardo que lhe pousasse nos lábios!

\asterisc

De luto a casa. Ramos e coroas virginais entravam a todo o instante.
Quem saberia explicar a morte de Rute? Foram achá"-la estendida na cama,
já toda fria.

Agora estava entre as duas janelas, na grande sala sombria, espalhando
sobre o fumo da essa as suas rendas brancas e o seu fino véu de noiva.
Parecia sonhar com o desejado esposo, que ali estava a seu lado, pálido
e mudo.

Entravam já para o enterro e foi só então que uma voz disse alto, saindo
da penumbra daquela sala antiga:

--- Vai ficar com o padrasto, no mesmo jazigo\ldots{}

Eduardo fixou a porta com doloroso espanto. Estava linda! Na pele
alvíssima nem uma sombra. Os cabelos negros, mal atados na nuca,
desprendiam"-se em uma madeixa abundante, de largas ondas.

--- Que! seria ainda para o outro aquele corpo angélico, tão castamente
emoldurado nas roupas do noivado? Seria então para o outro aquela
mocidade, aquela criatura divina, que deveria ser sua?!

E a mesma voz repetiu:

--- Vai ficar com o padrasto\ldots{}

Com o padrasto, noites e dias\ldots{} fechados\ldots{} unidos\ldots{} sós! Fora para
isso que ela se matara, para ir ter com o outro! aquele outro de quem
via o esqueleto torcendo"-se na cova, de braços estendidos para a
conquista da sua amante.

Alucinado, ciumento, Eduardo arrancou então num delírio o véu e as
flores de Rute, e inclinando um tocheiro pegou fogo ao pano da essa.

E a todos que acudiram nesse instante pareceu que viam sorrir a morta em
um êxtase, como se fosse aquilo que ela desejasse\ldots{}

\chapter{A rosa branca}

\hfill{}\emph{A Magalhães de Azeredo}

\bigskip

A viúva do comandante Henriques dizia a toda a gente que, das suas duas
netinhas, dava preferência à primeira.

O verdadeiro motivo consistia em ser a neta mais velha
extraordinariamente parecida com a família Henriques, enquanto que a
mais moça pertencia toda à família do pai, um provinciano feio. Ângela,
que era a primeira, recebia continuamente presentes da avó; a outra, a
Inez, olhava com melancolia para aquelas doces manifestações de amor,
perguntando mentalmente em que desmereceria ela da ternura da mãe de sua
mãe.

Acostumaram"-se todos com aquela injustiça, menos a pobre Inezinha, que
chorava muitas vezes às ocultas.

Com o tempo veio a necessidade de Ângela entrar para um colégio. A avó
lamentou"-se, tornando"-se ainda mais indiferente para a pobre Inez e
atirando"-lhe para cima todas as culpas; era ela quem quebrava a louça
que se sumia do armário; era ela que fazia enxaquecas à mãe com a bulha
dos seus sapatos insuportáveis; era ela quem arrancava as plantas do
jardim e quem roubava os doces do guarda"-prata; era ela quem batia nos
animais, quem riscava os moveis, quem enchia de trapos e de papeis o
chão, quem impacientava as criadas e pedia dinheiro às visitas.

Ella era o demônio! e, na sua opinião, seria muito mais sensato mandá"-la
de preferência para o colégio, como pensionista, e deixar em casa a
Ângela, a quem se oferecia para pagar os mestres.

O alvitre não foi bem recebido. E Ângela teve de partir para Itu, lugar
escolhido para a sua educação.

Na véspera, à noite, recaindo a conversa sobre assuntos de
pressentimentos e de superstições, Ângela teve a fantasia de dizer à
avó:

--- Olhe, vovó, todas as manhãs há de ver no seu oratório uma rosa
branca. Será o meu pensamento que há de vir visitá"-la. No dia em que a
rosa estiver meio murcha, será um sinal de que eu estou doente; e se ela
não aparecer, será porque eu morri!

Do seu canto, Inez observou que o olhar da avó se tornara angustioso. A
pobre senhora acreditava em sonhos e em fantasmas; sabia histórias
complicadas e extravagantes; coisas extraordinárias que ela queria impor
à fé ou à incredulidade dos outros! Já agora, se a rosa branca não
surgisse todas as madrugadas aos pés da Virgem das Dores, havia de supor
que a sua Angelita tinha ido fazer companhia aos querubins!

E enquanto a sua preferida dizia descuidada e risonha: ``Eu estava
brincando\ldots{}'', a outra lia"-lhe no olhar toda a inquietação e
tristeza\ldots{}

A despedida de Ângela foi dolorosa para o coração da avó; a pobre
senhora levou o dia inteiro a chorar, encerrada no quarto, e, quando
consentiu em ir ao chá, notaram todos a extraordinária alteração da sua
fisionomia. Estava impaciente, frenética, olhando de soslaio para a
pobre Inez, com quem várias vezes ralhou sob qualquer pretexto:

--- Menina, isso são modos? Tire a mão da mesa!

E continuava depois, voltando"-se para uma visita:

--- Tanto tem a Angelita de boa, quanto esta tem de mau gênio! Pudera!
fazem"-lhe todas as vontades! Eu nunca vi!

A mãe acudiu em defesa da filha, e a questão prolongou"-se, até que a
avó, desesperada, exclamou:

--- A outra foi aos onze anos de pensionista para o colégio; pois bem,
esta tem nove, e aposto em como nem daqui a três anos irá acompanhar a
irmã!

Inez ouvia humilhada e triste aquela troca de palavras, consolando"-se
com a doçura do olhar da mãe, que caía sobre ela como uma benção.

No seu pequeno quarto, em frente à cama vazia da irmã, Inezinha
procurava em vão adormecer. Revolvia"-se entre os lençóis, olhava para o
teto, onde a luz da lamparina punha sombras, e lembrava"-se do olhar da
avó, quando a Ângela falara na rosa branca! Ah! por que lhe quereria
tanto mal a sua avó? No entanto, procurava faze"-lhe as vontades, e
tinha"-lhe até muita amizade! Realmente, a Ângela era tão boa, e tão
bonita!

Sim, achava natural que a velhinha preferisse a outra\ldots{} Mas seria
razoável que a deprimisse sempre, e assim\ldots{} diante de gente de fora?
Tentava dormir: fechava os olhos e punha"-se a rezar:

--- \emph{Ave, Maria, cheia de graça}!\ldots{} E a rosa branca? ah! se a vovó
não a encontra no oratório\ldots{} é capaz de chorar! Fazei, virgem Maria,
com que nasça uma rosa branca a vossos pés!

Se fosse eu que estivesse no colégio, a vovó estaria contente! Por que
será que não gosta de mim? É verdade que eu lhe tenho feito mal, mas sem
ser por vontade\ldots{} entornei"-lhe chá quente na mão\ldots{} quebrei o seu
espelho novo; mas o que com certeza ela não me perdoa, é eu ter batido
na Ângela! Coitadinha da Ângela! ela não se queixou\ldots{} quem teria visto?
Mas, se eu não lhe batesse, ela matava o gato da vizinha, e depois? Sim!
a vovó tem"-me raiva desde esse dia\ldots{} mas eu tenho dado tantos beijos na
Ângela! Pobre da minha irmã, que saudade ela hoje terá da sua caminha\ldots{}

Apesar dos meus beijos, a amizade da vovó não voltou. Mamãe sempre me
diz que não julgue eu isso, que a vovó adora"-me! como o saberá? Mas a
mamãe não mente; logo que diz, é porque é.

Com as mãozinhas cruzadas sobre o peito, toda envolvida na sua longa
camisa de dormir, Inez lutava com a insônia, e, para afastar os
pensamentos, recomeçava dizer: \emph{Ave, Maria, cheia de graça, o
Senhor é convosco}\ldots{}

No entanto, antevia as mãos trêmulas da avó, procurando em vão a rosa
branca entre as dobras do veludo azul do manto de Nossa Senhora. Depois
as lágrimas caindo"-lhe as duas pela face engelhada\ldots{} E tinha pena, e
tornava, cheia de fé, a suplicar:

--- Virgem Maria! fazei com que nasça uma rosa branca a vossos pés!

A luz da lamparina foi"-se tornando pálida, à proporção que os vidros da
janela se iam iluminando pela claridade exterior. Inez ergueu"-se. Nunca
tinha visto amanhecer, mas o seu fito era outro; foi cautelosamente à
janela, abriu"-a e olhou. Nuvens cor de rosa enovelavam"-se sob o céu
azul; no alto, mostrava"-se a lua, estreita como um fio de luz arqueado,
e um pouco abaixo entrebrilhava uma grande estrela esbranquiçada e fria.
Inez espreitou o oratório.

--- Nada! A lâmpada acesa, bruxuleante, difundia a sua tênue chama sobre
um ramo de flores artificiais! Voltou à janela do seu quarto, ao rés do
chão; vacilou um momento, mas, armando"-se de coragem, saltou"-a, e correu
para um recanto do jardim.

À hora do almoço, a avó apareceu risonha e tranquila, com o olhar
abrandado por uma misteriosa doçura d'alma. Passaram"-se dias, durante os
quais a pobre senhora achou sempre no seu oratório a prometida rosa
branca, que era, a seu ver, a visita do pensamento da adorada netinha!
Cada vez mais terna para a ausente, tornava"-se mais ríspida para a Inez.
A pequenita andava agora mais abatida e magra, chegando a inspirar
cuidados à família.

A história da rosa era ignorada por todos; a avó guardava o segredo da
\emph{visita} de Ângela, egoisticamente, conservando as rosas, mesmo
depois de murchas, num cofrezinho.

Um dia, estavam todos à mesa, quando o jardineiro se foi queixar de que
todas as noites ia alguém roubar uma rosa branca a uma das roseiras de
mais estimação do jardim!

Da rua não entrava ninguém; aquilo era coisa de gente da casa; pedia
providências.

Inez tornou"-se rubra; a avó estremeceu, e o dono da casa, um
colecionador fanático, prometeu um tiro a quem, sem seu consentimento,
lhe arrancasse as rosas do jardim. À noite verificou a existência de um
formoso botão. No dia seguinte o botão havia desaparecido!

Aquela persistência exasperou"-o. Começaram as indagações. A avó julgou
de seu dever intervir, contando o fato que se passava consigo, e
aconselhando paciência. Era a mão invisível de um ente sobrenatural e
piedoso, que vinha, mensageiro da sua Angelita, traze"-lhe a flor
prometida!

Essa revelação desorientou"-os. A pessoa era então, evidentemente, de
casa, e tão íntima que entrava nos quartos da família! Houve ameaças\ldots{}
Entretanto, a doce rosa branca, aquietadora dos sustos da avó, aparecia
todas as manhãs no seu oratório.

As criadas começaram a supor fantasmas, a asseverar que os viam, e de
tal forma que a própria Inez entrou de ter medo!

Uma noite deitou"-se resolvida a faltar à sua caridosa lembrança; a avó
que tivesse paciência e apreensões e lágrimas, --- ela não se arriscaria
nunca mais para poupa"-lhe esses desgostos! E ficou, como na primeira
noite, nervosa, imaginando a decepção da velha! Passou por fim
ligeiramente pelo sono; acordando, viu tamanha claridade na janela, que
supôs ser já dia. Saltou do leito, e, sem meditar, levada pelo hábito,
ainda quase a dormir, pulou para o jardim, arrastando na areia a sua
camisola branca e magoando no chão os pezinhos descalços.

A lua, em todo o esplendor, espalhava a sua luz aveludada; estava tudo
silencioso, silencioso!

Inez, no meio do caminho, ao ar fresco, compreendeu o seu engano:
levantara"-se alta noite! A bulha dos seus passos naquela solidão
horrorizou"-a. Ah! era a hora dos fantasmas, e ela não ousava olhar para
trás! caminhava sempre, com os lábios secos e os olhos muito abertos!
Foi com um movimento nervoso que arrancou da haste a triste flor
piedosa, não ousando observá"-la, porque, quando à violência do puxão a
roseira balançou os seus botões nevados, afigurou"-se"-lhe ver uma dança
macabra improvisada no ar por estranhos e pequeninos espectros! Correu
então alucinada para casa, saltou para dentro, e, sem tomar as
precauções do costume, entrou no oratório precipitadamente e atirou aos
pés da Virgem a doce rosa branca, murmurando ao mesmo tempo, com a voz
alterada pelo medo:

``\emph{Salve, Rainha\ldots{} Mãe de misericórdia\ldots{}vida e doçura\ldots{}
esperança nossa!}''\ldots{}

Não acabou. Transida de medo e de frio, cairia no chão\ldots{} se dois braços
não a amparassem meigamente.

Eram os braços da avó, que a cobria de beijos, repetindo"-lhe:

--- Como tu és boa, minha adorada Inez! como tu és boa!

\chapter{Os porcos}

\hfill{}\emph{A Artur Azevedo}

\bigskip

Quando a cabocla Umbelina apareceu grávida, o pai moeu"-a de surras,
afirmando que daria o neto aos porcos para que o comessem.

O caso não era novo, nem a espantou, e que ele havia de cumprir a
promessa, sabia"-o bem. Ela mesma, lembrava"-se, encontrara uma vez um
braço de criança entre as flores douradas do aboboral. Aquilo, com
certeza, tinha sido obra do pai.

Todo o tempo da gravidez pensou, numa obsessão crudelíssima, torturante,
naquele bracinho nu, solto, frio, resto dum banquete delicado, que a
torpe voracidade dos animais esquecera por cansaço e enfartamento.

Umbelina sentava"-se horas inteiras na soleira da porta, alisando com um
pente vermelho de celuloide o cabelo negro e corredio. Seguia assim,
preguiçosamente, com olhar agudo e vagaroso, as linhas do horizonte,
fugindo de fixar os porcos, aqueles porcos malditos, que lhe rodeavam a
casa desde manhã até a noite.

Via"-os sempre ali, arrastando no barro os corpos imundos, de pelo ralo e
banhas descaídas, com o olhar guloso luzindo sob a pálpebra mole, e o
ouvido encoberto pela orelha chata, no egoísmo brutal de concentrar em
si toda a atenção. Os leitões vinham por vezes, barulhentos e às
cambalhotas, envolverem"-se na sua saia, e ela sacudia"-os de nojo,
batendo"-lhes com os pés, dando"-lhes com força. Os porcos não a temiam,
andavam perto, fazendo desaparecer tudo diante da sofreguidão dos seus
focinhos rombudos e móveis, que iam e vinham grunhindo, babosos,
hediondos, sujos da lama em que se deleitavam, ou alourados pelo pó do
milho, que estava para ali aos montes, flavescendo ao sol.

Ah! os porcos eram um bom sumidouro para os vícios do caboclo! Umbelina
execrava"-os e ia pensando no modo de acabar com o filho d'uma maneira
menos degradante e menos cruel.

Guardar a criança\ldots{} mas como? O seu olhar interrogava em vão o
horizonte frouxelado de nuvens.

O amante, filho do patrão, tinha"-a posto de lado\ldots{} diziam até que ia
casar com outra! Entretanto achavam"-na todos bonita, no seu tipo de
índia, principalmente aos domingos, quando se enfeitava com as
maravilhas vermelhas, que lhe davam colorido à pele bronzeada e a
vestiam toda com um cheiro doce e modesto\ldots{}

Eram duas horas da madrugada, quando a Umbelina entreabriu um dia a
porta da casa paterna e se esgueirou para o terreiro.

Fazia luar; todas as coisas tinham um brilho suavíssimo. A água do
monjolo caía em gorgolões soluçados, flanqueando o rancho de sapê, e
correndo depois em fio luminoso e trêmulo pela planície fora. Flores de
gabiroba e de esponjeira brava punham lençóis de neve na extensa margem
do córrego; todas as ervas do mato cheiravam bem. Um galo cantava perto,
outro respondia mais longe, e ainda outro, e outro\ldots{} até que as vozes
dos últimos se confundiam na distância com os mais leves rumores
noturnos.

Umbelina afastou com mão febril o xale que a envolvia, e, descobrindo a
cabeça, investigou com olhar sinistro o céu profundo.

Onde se esconderia o grande Deus, divinamente misericordioso, de quem o
padre falava na missa do arraial em termos que ela não atingia, mas que
a faziam estremecer?

Ninguém pode fugir ao seu destino, diziam todos; estaria então escrito
que a sua sorte fosse essa que o pai lhe prometia --- de matar a fome
aos porcos com a carne da sua carne, o sangue do seu sangue?!

Essas coisas rolavam"-lhe pelo espírito, indeterminadas e confusas. A
raiva e o pavor do parto estrangulavam"-na. Não queria bem ao filho,
odiava nele o amor enganoso do homem que a seduzira. Matá"-lo"-ia,
esmagá"-lo"-ia mesmo, mas lançá"-lo aos porcos\ldots{} isso nunca! E voltava"-lhe
à mente, num arrepio, aquele bracinho solto, que ela tivera entre os
dedos indiferentes, na sua bestialidade de cabocla matuta.

O céu estava limpo, azul, um céu de janeiro, quente, vestido de luz, com
a sua estrela Vésper enorme e diamantina, e a lua muito grande, muito
forte, muito esplendorosa!

A cabocla espreitou com olho vivo para os lados da roça de milho, onde
ao seu ouvido agudíssimo parecera sentir uma bulha cautelosa de pés
humanos; mas não veio ninguém, e ela, abrasada, arrancou o xale dos
ombros e arrastou"-o no chão, segurando"-o com a mão, que as dores do
parto crispavam convulsivamente. O corpo mostrou"-se disforme, mal
resguardado por uma camisa de algodão e uma saia de chita. Pelos ombros
estreitos agitavam"-se as pontas do cabelo negro e luzidio; o ventre
pesado, muito descaído, dificultava"-lhe a marcha, que ela interrompia
amiúde para respirar alto, ou para agacha"-se, contorcendo"-se toda.

A sua ideia era ir ter o filho na porta do amante, matá"-lo ali, nos
degraus de pedra, que o pai havia de pisar de manhã, quando descesse
para o passeio costumado.

Uma vingança doida e cruel aquela, que se fixara havia muito no seu
coração selvagem.

A criança tremia"-lhe no ventre, como se pressentisse que entraria na
vida para entrar no túmulo, e ela apressava os passos nervosamente por
sobre as folhas da trapoeraba maninha.

Ai! iam ver agora quem era a cabocla! Desprezavam"-na? Riam"-se dela?
Deixavam"-na à toa, como um cão sem dono? Pois que esperassem! E ruminava
o seu plano, receando esquecer alguma minúcia\ldots{}

Deixaria a criança viver alguns minutos, fá"-la"-ia mesmo chorar, para que
o pai lá dentro, entre o conforto do seu colchão de paina, que ela
desfiara cuidadosamente, lhe ouvisse os vagidos débeis e os guardasse
sempre na memória, como um remorso.

Ela estava perdida. Em casa não a queriam; a mãe renegava"-a, o pai
batia"-lhe, o amante fechava"-lhe as portas\ldots{} e Umbelina praguejava alto,
ameaçando de fazer cair sobre toda a gente a cólera divina!

O luar com a sua luz brancacenta e fria iluminava a triste caminhada
daquela mulher quase nua e pesadíssima, que ia golpeada de dores e de
medo através dos campos. Umbelina ladeou a roça de milho, já seca, muito
amarelada, e que estalava ao contato do seu corpo mal firme; passou
depois o grande canavial, dum verde"-d'água que o luar enchia de doçura e
que se alastrava pelo morro abaixo, até lá perto do engenho, na
esplanada da esquerda. Por entre as canas houve um rastejar de cobras, e
ergueu"-se da outra banda, na negrura do mandiocal, um voo fofo, de ave
assustada. A cabocla benzeu"-se e cortou direito pelo terreno mole do
feijoal ainda novo, esmagando sob as solas dos pés curtos e trigueiros
as folhinhas tenras da planta ainda sem flor. Depois abriu lá em cima a
cancela, que gemeu prolongadamente nos movimentos de ida e de volta, com
que ela a impeliu para diante e para trás. Entrou no pasto da fazenda.
Uma grande mudez por todo o imenso gramado. O terreno descia numa linha
suave até o terreiro da habitação principal, que aparecia ao longe num
ponto"-branco. A cabocla abaixou"-se tolhida, suspendendo o ventre com as
mãos.

Toda a sua energia ia fugindo, espavorida com a dor física, que se
aproximava em contrações violentas. A pouco e pouco os nervos
distenderam"-se, e o quase bem"-estar da extenuação fê"-la deixa"-se ficar
ali, imóvel, com o corpo na terra e a cabeça erguida para o céu
tranquilo. Uma onda de poesia invadiu"-a toda: eram os primeiros enleios
da maternidade, a pureza inolvidável da noite, a transparência lúcida
dos astros, os sons quase imperceptíveis e misteriosos, que lhe pareciam
vir de longe, de muito alto, como um eco fugitivo da música dos anjos,
que diziam haver no céu\ldots{}

Umbelina sentia uma grande ternura toma"-lhe o coração, subi"-lhe aos
olhos. Não a sabia compreender e deixava"-se ir naquela vaga sublimemente
piedosa e triste\ldots{}

Súbito, sacudiu"-a uma dor violenta, que a tomou de assalto, obrigando"-a
a cravar as unhas no chão. Aquela brutalidade fê"-la praguejar e
ergue"-se depois raivosa e decidida. Tinha de atravessar todo o comprido
pasto, a margem do lago e a orla do pomar, antes de cair na porta do
amante.

Foi; mas as forças diminuíam e as dores repetiam"-se cada vez mais
próximas.

Lá embaixo aparecia já a chapa branca, batida do luar, das paredes da
casa.

A roceira ia com os olhos fitos nessa luz, apressando os passos
cansados. O suor caía"-lhe em bagas grossas por todo o corpo, ao tempo
que as pernas se lhe vergavam ao peso da criança.

No meio do pasto, uma figueira enorme estendia os braços sombrios, pondo
uma mancha negra em toda aquela extensão de luz. A cabocla quis
esconde"-se ali, cansada da claridade, com medo de si mesma, dos
pensamentos pecaminosos que tumultuavam no seu espírito e que a lua
santa e branca parecia penetrar e esclarecer. Ela alcançou a sombra com
passadas vacilantes; mas os pés inchados e dormentes já não sentiam o
terreno e tropeçavam nas raízes de arvores, muito estendidas e salientes
no chão. A cabocla caiu de joelhos, amparando"-se para a frente nas mãos
espalmadas. O choque foi rápido e as últimas dores do parto vieram
tolhê"-la. Quis reagir ainda e levanta"-se, mas já não pôde, e furiosa
descerrou os dentes, soltando os últimos e agudíssimos gritos da
expulsão.

Um minuto depois a criança chorava sufocadamente. A cabocla então
arrancou com os dentes o cordão da saia e, soerguendo o corpo, atou com
firmeza o umbigo do filho, e enrolou"-o no xale, sem olhar quase para
ele, com medo de o amar\ldots{}

Com medo de o amar!\ldots{} No seu coração de selvagem desabrochava
timidamente a flor da maternidade. Umbelina levantou"-se a custo com o
filho nos braços. O corpo esmagado de dores, que lhe parecia
esgarçarem"-lhe as carnes, não obedecia à sua vontade. Lá embaixo a mesma
chapa de luz alvacenta acenava"-lhe, chamando"-a para a vingança ou para o
amor. Julgava agora que, se batesse àquelas janelas e chamasse o amante,
ele viria comovido e trêmulo beijar o seu primeiro filho. Aventurou"-se
em passadas custosas a seguir o seu caminho, mas voltaram"-lhe depressa
as dores e, sentindo"-se esvair, sentou"-se na grama para descansar.
Descobriu então a meio o corpo do filho: achou"-o branco, achou"-o bonito,
e num impulso de amor beijou"-o na boca. A criança moveu logo os lábios
na sucção dos recém"-nascidos e ela deu"-lhe o peito. O pequenino puxava
inutilmente, a cabocla não tinha alento, a cabeça pendia"-lhe numa
vertigem suave, veio"-lhe depois outra dor, os braços abriram"-se"-lhe, e
ela caiu de costas.

A lua sumia"-se, e os primeiros alvores da aurora tingiam dum róseo
dourado todo o horizonte. Em cima o azul carregado da noite mudava para
um violeta transparente, esbranquiçado e diáfano. Foi no meio daquela
doce transformação da luz que Umbelina mal distinguiu um vulto negro,
que se aproximava lentamente, arrastando no chão as mamas pelancosas,
com o rabo fino, arqueado, sobre as ancas enormes, o pelo hirto,
irrompendo raro da pele escura e rugosa, e o olhar guloso, estupidamente
fixo: era uma porca.

Umbelina sentiu"-a grunhir, viu confusamente os movimentos repetidos do
seu focinho trombudo, gelatinoso, que se arregaçava, mostrando a dentuça
amarelada, forte. Um sopro frio correu por todo o corpo da cabocla, e
ela estremeceu ouvindo um gemido doloroso, dolorosíssimo, que se cravou
no seu coração aflito. Era do filho! Quis ergue"-se, apanhá"-lo nos
braços, defendê"-lo, salvá"-lo\ldots{} mas continuava a esvai"-se, os olhos mal
se abriam, os membros lassos não tinham vigor, e o espírito mesmo perdia
a noção de tudo.

Entretanto, antes de morrer, ainda viu, vaga, indistintamente, o vulto
negro e roliço da porca, que se afastava com um montão de carne
pendurado nos dentes, destacando"-se isolada e medonha naquela imensa
vastidão cor de rosa.

\chapter{A caolha}

\hfill{}\emph{A Eva Canel}

\bigskip

A caolha era uma mulher magra, alta, macilenta, peito fundo, busto
arqueado, braços compridos, delgados, largos nos cotovelos, grossos nos
pulsos; mãos grandes, ossudas, estragadas pelo reumatismo e pelo
trabalho; unhas grossas, chatas e cinzentas, cabelo crespo, de uma cor
indecisa entre o branco sujo e o loiro grisalho, desses cabelos cujo
contato parece dever ser áspero e espinhento; boca descaída, numa
expressão de desprezo, pescoço longo, engelhado, como o pescoço dos
urubus; dentes falhos e cariados.

O seu aspecto infundia terror às crianças e repulsão aos adultos; não
tanto pela sua altura e extraordinária magreza, mas porque a desgraçada
tinha um defeito horrível: haviam"-lhe extraído o olho esquerdo; a
pálpebra descera mirrada, deixando contudo, junto ao lacrimal, uma
fistula continuamente porejante.

Era essa pinta amarela sobre o fundo denegrido da olheira, era essa
destilação incessante de pus que a tornava repulsa aos olhos de toda a
gente.

Morava numa casa pequena, paga pelo filho único, operário numa oficina
de alfaiate; ela lavava roupa para os hospitais e dava conta de todo o
serviço da casa, inclusive cozinha. O filho, enquanto era pequeno, comia
os pobres jantares feitos por ela, às vezes até no mesmo prato; à
proporção que ia crescendo, ia"-se"-lhe a pouco e pouco manifestando na
fisionomia a repugnância por essa comida; até que um dia, tendo já um
ordenadozinho, declarou à mãe que, por conveniência do negócio, passava
a comer fora\ldots{}

Ela fingiu não perceber a verdade, e resignou"-se.

Daquele filho vinha"-lhe todo o bem e todo o mal.

Que lhe importava o desprezo dos outros, se o seu filho adorado lhe
apagasse com um beijo todas as amarguras da existência?

Um beijo dele era melhor que um dia de sol, era a suprema carícia para o
seu triste coração de mãe! Mas\ldots{} os beijos foram escasseando também,
com o crescimento do Antonico! Em criança ele apertava"-a nos bracinhos e
enchia"-lhe a cara de beijos; depois, passou a beijá"-la só na face
direita, aquela onde não havia vestígios de doença; agora, limitava"-se a
beija"-lhe a mão!

Ela compreendia tudo e calava"-se. O filho não sofria menos.

Quando em criança entrou para a escola pública da freguesia, começaram
logo os colegas, que o viam ir e vir com a mãe, a chamá"-lo --- o filho
da caolha.

Aquilo exasperava"-o; respondia sempre:

--- Eu tenho nome!

Os outros riam"-se e chacoteavam"-no; ele queixava"-se aos mestres, os
mestres ralhavam com os discípulos, chegavam mesmo a castigá"-los, ---
mas a alcunha pegou. Já não era só na escola que o chamavam assim.

Na rua, muitas vezes, ele ouvia de uma ou de outra janela dizerem: Olha
o filho da caolha! Lá vai o filho da caolha! Lá vem o filho da caolha!

Eram as irmãs dos colegas, meninas novas, inocentes e que, industriadas
pelos irmãos, feriam o coração do pobre Antonico cada vez que o viam
passar!

As quitandeiras, onde ia comprar as goiabas ou as bananas para o
\emph{lunch}, aprenderam depressa a denominá"-lo como os outros, e,
muitas vezes, afastando os pequenos que se aglomeravam ao redor delas,
diziam, estendendo uma mancheia de araçás, com piedade e simpatia:

--- \emph{Taí}, isso é pra \emph{o filho da caolha}!

O Antonico preferia não receber o presente a ouvi"-lo acompanhar de tais
palavras; tanto mais que os outros, com inveja, rompiam a gritar,
cantando em coro, num estribilho já combinado:

--- Filho da caolha, filho da caolha!

O Antonico pediu à mãe que o não fosse buscar à escola; e, muito
vermelho, contou"-lhe a causa: sempre que a viam aparecer à porta do
colégio os companheiros murmuravam injúrias, piscavam os olhos para o
Antonico e faziam caretas de náuseas!

A caolha suspirou e nunca mais foi buscar o filho.

Aos onze anos o Antonico pediu para sair da escola: levava a brigar com
os condiscípulos, que o intrigavam e malqueriam. Pediu para entrar para
uma oficina de marceneiro. Mas na oficina de marceneiro aprenderam
depressa a chamá"-lo --- o filho da caolha, e a humilhá"-lo, como no
colégio.

Além de tudo, o serviço era pesado e ele começou a ter vertigens e
desmaios. Arranjou então um lugar de caixeiro de venda; os seus
ex"-colegas agrupavam"-se à porta, insultando"-o, e o vendeiro achou
prudente mandar o caixeiro embora, tanto mais que a rapaziada ia"-lhe
dando cabo do feijão e do arroz expostos à porta nos sacos abertos! Era
uma contínua saraivada de cereais sobre o pobre Antonico!

Depois disso passou um tempo em casa, ocioso, magro, amarelo, deitado
pelos cantos, dormindo às moscas, sempre zangado e sempre bocejante!
Evitava sair de dia e nunca, mas nunca, acompanhava a mãe; esta
poupava"-o: tinha medo que o rapaz, num dos desmaios, lhe morresse nos
braços, e por isso nem sequer o repreendia! Aos dezesseis anos, vendo"-o
mais forte, pediu e obteve"-lhe a caolha um lugar numa oficina de
alfaiate. A infeliz mulher contou ao mestre toda a história do filho e
suplicou"-lhe que não deixasse os aprendizes humilhá"-lo; que os fizesse
terem caridade!

Antonico encontrou na oficina uma certa reserva e silêncio da parte dos
companheiros; quando o mestre dizia: sr. Antonico, ele percebia um
sorriso mal oculto nos lábios dos oficiais; mas a pouco e pouco essa
suspeita ou esse sorriso se foi desvanecendo, até que principiou a
sentir"-se bem ali.

Decorreram alguns anos e chegou a vez do Antonico se apaixonar.

Até aí, numa ou noutra pretensão de namoro que ele tivera, encontrara
sempre uma resistência que o desanimava, e que o fazia retroceder sem
grandes mágoas. Agora, porém, a coisa era diversa: ele amava! amava como
um louco a linda moreninha da esquina fronteira, uma rapariguinha
adorável, de olhos negros como veludo e boca fresca como um botão de
rosa. O Antonico voltou a ser assíduo em casa e expandia"-se mais
carinhosamente com a mãe; um dia, em que viu os olhos da morena fixarem
os seus, entrou como um louco no quarto da caolha e beijou"-a mesmo na
face esquerda, num transbordamento de esquecida ternura!

Aquele beijo foi para a infeliz uma inundação de júbilo! tornava a
encontrar o seu querido filho! pôs"-se a cantar toda a tarde, e nessa
noite, ao adormecer, dizia consigo:

--- Sou muito feliz\ldots{} o meu filho é um anjo!

Entretanto, o Antonico escrevia, num papel fino, a sua declaração de
amor à vizinha. No dia seguinte mandou"-lhe cedo a carta. A resposta
fez"-se esperar. Durante muitos dias o Antonico perdia"-se em amargas
conjeturas.

Ao princípio pensava:

``É o pudor''. Depois começou a desconfiar de outra causa; por fim
recebeu uma carta em que a bela moreninha confessava consentir em ser
sua mulher, se ele se separasse completamente da mãe! Vinham explicações
confusas, mal alinhavadas: lembrava a mudança de bairro; ele ali era
muito conhecido por \emph{filho da caolha}, e bem compreendia que ela
não se poderia sujeitar a ser alcunhada em breve de --- \emph{nora da
caolha}, ou coisa semelhante!

O Antonico chorou. Não podia crer que a sua casta e gentil moreninha
tivesse pensamentos tão práticos!

Depois o seu rancor voltou"-se para a mãe.

Ela era a causadora de toda a sua desgraça! Aquela mulher perturbara a
sua infância, quebrava"-lhe todas as carreiras, e agora o seu mais
brilhante sonho de futuro sumia"-se diante dela!

Lamentava"-se por ter nascido de mulher tão feia, e resolveu procurar
meio de separar"-se dela; considerar"-se"-ia humilhado continuando sob o
mesmo teto; havia de protegê"-la de longe, vindo de vez em quando vê"-la,
à noite, furtivamente\ldots{}

Salvava assim a responsabilidade do protetor e, ao mesmo tempo,
consagraria à sua amada a felicidade que lhe devia em troca do seu
consentimento e amor\ldots{}

Passou um dia terrível; à noite, voltando para a casa, levava o seu
projeto e a decisão de o expor à mãe.

A velha, agachada à porta do quintal, lavava umas panelas com um trapo
engordurado. O Antonico pensou: ``A dizer a verdade eu havia de sujeitar
minha mulher a viver em companhia de\ldots{} uma tal criatura?'' Estas
últimas palavras foram arrastadas pelo seu espírito com verdadeira dor.
A caolha levantou para ele o rosto, e o Antonico, vendo"-lhe o pus na
face, disse:

--- Limpe a cara, mãe\ldots{}

Ela sumiu a cabeça no avental; ele continuou:

--- Afinal, nunca me explicou bem a que é devido esse defeito!

--- Foi uma doença, respondeu sufocadamente a mãe: é melhor não lembrar
isso!

--- É sempre a sua resposta; é melhor não lembrar isso! Por quê?!

--- Porque não vale a pena; nada se remedeia\ldots{}

--- Bem! agora escute: trago"-lhe uma novidade: o patrão exige que eu vá
dormir na vizinhança da loja\ldots{} já aluguei um quarto: a senhora fica
aqui e eu virei todos os dias saber da sua saúde ou se tem necessidade
de alguma coisa\ldots{} É por força maior; não temos remédio senão
sujeitar"-nos !\ldots{}

Ele, magrinho, curvado pelo hábito de costurar sobre os joelhos, delgado
e amarelo como todos os rapazes criados à sombra das oficinas, onde o
trabalho começa cedo e o serão acaba tarde, tinha lançado naquelas
palavras toda a sua energia, e espreitava agora a mãe com olho
desconfiado e medroso.

A caolha levantou"-se, e, fixando o filho com uma expressão terrível,
respondeu com doloroso desdém:

--- Embusteiro! o que você tem é vergonha de ser meu filho! Saia! que eu
também já sinto vergonha de ser mãe de semelhante ingrato!

O rapaz saiu cabisbaixo, humilde, surpreso da atitude que assumira a
mãe, até então sempre paciente e cordata; ia com medo, maquinalmente,
obedecendo à ordem que tão feroz e imperativamente lhe dera a caolha.

Ela acompanhou"-o, fechou com estrondo a porta, e, vendo"-se só,
encostou"-se cambaleante à parede do corredor e desabafou em soluços.

O Antonico passou uma tarde e uma noite de angústia.

Na manhã seguinte o seu primeiro desejo foi voltar a casa; mas não teve
coragem: via o rosto colérico da mãe, faces contraídas, lábios
adelgaçados pelo ódio, narinas dilatadas, o olho direito saliente, a
penetrar"-lhe até o fundo do coração, o olho esquerdo arrepanhado, murcho
--- e sujo de pus; via a sua atitude altiva, o seu dedo ossudo, de
falanges salientes, apontando"-lhe com energia a porta da rua; sentia"-lhe
ainda o som cavernoso da voz, e o grande fôlego que ela tomara para
dizer as verdadeiras e amargas palavras que lhe atirara ao rosto; via
toda a cena da véspera e não se animava a arrostar com o perigo de outra
semelhante.

Providencialmente, lembrou"-se da madrinha, única amiga da caolha, mas
que, entretanto, raramente a procurava.

Foi pedir"-lhe que interviesse, e contou"-lhe sinceramente tudo que
houvera.

A madrinha escutou"-o comovida; depois disse:

--- Eu previa isso mesmo, quando aconselhava tua mãe a que te dissesse a
verdade inteira; ela não quis, aí está!

--- Que verdade, madrinha?!

--- Hei de dizer"-te perto dela; anda, vamos lá!

Encontraram a caolha a tirar umas nódoas do fraque do filho, --- queria
mandar"-lhe a roupa limpinha. A infeliz arrependera"-se das palavras que
dissera e tinha passado toda a noite à janela, esperando que o Antonico
voltasse ou passasse apenas\ldots{} via o porvir negro e vazio e já se
queixava de si! Quando a amiga e o filho entraram, ela ficou imóvel: a
surpresa e a alegria amarraram"-lhe toda a ação.

A madrinha do Antonico começou logo:

--- O teu rapaz foi suplicar"-me que te viesse pedir perdão pelo que
houve aqui ontem, e eu aproveito a ocasião para, à tua vista, contar"-lhe
o que já deverias ter"-lhe dito!

--- Cala"-te! murmurou com voz apagada a caolha.

--- Não me calo tal! Essa pieguice é que te tem prejudicado! Olha,
rapaz! quem cegou tua mãe foste tu!

O afilhado tornou"-se lívido; e ela concluiu:

--- Ah, não tiveste culpa! eras muito pequeno quando, um dia, ao almoço,
levantaste na mãozinha um garfo; ela estava distraída, e antes que eu
pudesse evitar a catástrofe, tu enterraste"-lho pelo olho esquerdo!

Ainda tenho no ouvido o grito de dor que ela deu!

O Antonico caiu pesadamente de bruços, com um desmaio; a mãe acercou"-se
rapidamente dele, murmurando trêmula:

--- Pobre filho! vês? era por isto que eu não lhe queria dizer nada!

\chapter{Incógnita}

--- O senhor conheceu"-a?

--- Talvez\ldots{}

--- Era ainda moça\ldots{}

--- Parece\ldots{}

--- Ninguém a reconheceu?

--- Ninguém\ldots{} Faz"-me o favor do seu fogo?

--- Pois não\ldots{}

Houve uma pausa; e, enquanto um dos interlocutores, o que perguntava,
examinava com interesse o interior do Necrotério, o outro ia acendendo
muito pachorrentamente o seu cigarro.

Em frente deles, sobre o mármore branco de uma das quatro mesas, estava
o cadáver de uma mulher.

Um vento salitroso espalhava a umidade do dia torvo, peneirado de
cinzas.

No seu nicho, sobre fundo azul, a Virgem da Piedade sustendo nos joelhos
o corpo inerte do Filho, parecia evocar a sua agonia como um exemplo de
dor corajosa.

O cadáver estava inchado pela absorção da água. Os cabelos curtos
empastavam"-se"-lhe no crânio sujo de areia e de partículas de algas. Os
olhos entreabertos pareciam, na sua névoa glacial, feitos da água que os
havia apagado e coagulado em dois grandes glóbulos opacos, engrandecidos
pelo pavor da onda\ldots{}

--- Como a Morte transfigura\ldots{}

O outro comentou:

--- Se estivesse como eu acostumado a isto já se não impressionaria
assim. Vá"-se embora\ldots{} está pálido, não convém abusar de uma impressão
nervosa\ldots{}

--- O senhor?

--- Sou estudante de Medicina\ldots{}

O outro, que talvez tivesse conhecido a morte\ldots{} quem lhe diria? saiu
para a rua com ar despreocupado, pensando talvez no almocinho quente, e
no vinho leve; este, ao contrário, tremia, sentia a palma das mãos
úmidas como se as tivesse passeado sobre a carne mole da defunta e um
cheiro de cadáver e de ácido fênico em tudo: na rua, no lenço, no fato,
no ar\ldots{}

Foi todo um dia de sofrimento e de obsessão doentia. Não pôde engolir
bocados que lhe não viessem náuseas, não conseguiu pousar a vista em
nenhum rosto de mulher bonita que não estremecesse pela visão da
afogada, não andava para o seu trabalho e para os seus negócios que não
lhe parecesse tropeçar no corpo entumecido da morta desconhecida. Por
que, por que, mas por quê?

Querendo reagir, procurou em vão entreter o espírito, arejá"-lo com
outras ideias. Afinal, não fora por causa dele que aquela mulher se
matara! Depois, não lia ele todas as manhãs, já sem abalo à força do
costume, tantas notícias de crimes, tão dolorosas revelações nos
jornais? Por que haveria agora este fato de o impressionar mais que
tantos outros? Então, só porque os seus olhos tinham visto aquele corpo
imundo, já a sua impassibilidade dava lugar a uma tamanha vibração
nervosa? Sem saber como, sem saber por quê, começou a reconstituir
aquela vida desconhecida: --- virgem deixara a família, a sociedade,
tudo, para fugir com o homem que a fascinara. Dias de sol, noites de
luar, meses de alucinação, e cada vez o seu amor era mais veemente e a
sua felicidade mais irradiante\ldots{} até que, pouco a pouco, principiou a
sentir arrefecerem"-se os beijos do amante a compreender as suas
ausências, a sentir a sua indiferença. A explicação viera: amava outra,
ia casar"-se com outra, estava tudo acabado.

Bastaria isso para fazer morrer uma mulher moça, com tantos dias de vida
ainda diante de si?!

Girando pelas ruas, na sua lida costumeira ele encontrou amigos que lhe
contaram casos a que sorriu por condescendência; e, andando, negou
esmolas aos pobres mais impressionadores; resvalou com indiferença o
olhar por fisionomias mortificadas; viu sem emoção passar os
estrepitosos carros da assistência pública; e com igual frieza assistiu
ao despejo de um lar onde havia berços e encontrou fechadas, por
falência, as portas de uma antiga firma da cidade. Mas ao voltar para
casa parou de novo no necrotério. Concordava que era uma estupidez, mas
não pôde resistir.

A morta já lá não estava.

--- Reconheceram"-na?

--- Não, senhor. Foi sozinha para o cemitério.

--- Antes a tivessem deixado no mar\ldots{}

--- Mais valia\ldots{}

Tinham posto o ponto final sobre uma existência humana. Quem fora? que
lhe importava\ldots{}?

Não lhe importava, mas consultou ainda as suas reminiscências:

Como o outro dissera, também ele talvez a tivesse conhecido\ldots{} Borboleta
do amor, ela teria talvez adejado numa hora fugitiva sobre a sua
existência de homem\ldots{}

Levantou os ombros, e seguiu o seu caminho sem compreender que, se
pensava tanto nela, era exatamente porque ela não representava para ele
mais do que o mistério da incógnita\ldots{}

\chapter{A morte da velha}

\hfill{}\emph{A Presciliana Duarte de Almeida}

\bigskip

Cabelos brancos, finos, em bandós; rosto redondo, amolecido, sulcado por
muitas linhas fundas; olhos azuis, caridosos e transparentes como as
pupilas das crianças; corpo pesado, grosso, baixo e curvado; pés e mãos
inchados, pernas paralíticas --- tal era a velhinha cuja vida deslizara
entre sacrifícios, que ela, na sua crença de religiosa, espera ver
transformados em flores no céu!

Muito surda, mas extraordinariamente bondosa e ativa, ela não parava de
trabalhar, na sua grande cadeira de rodas, recortando papéis para as
confeitarias. Os recursos eram minguados: o irmão, desde que se mudara
para aquele sobrado da rua do Hospício, não lhe dava vintém, e ainda se
queixava de ter de sustentar tantas bocas.

Só filhas quatro, de mais a mais doentes e pouco jeitosas. Só uma
bonita, a última, e essa era também a de melhor gênio, talvez por mais
esperançada no futuro. Mãe não tinham, e fora a velha, tia Amanda, quem
tivera com elas todo o trabalho da criação, bem como já tinha tido com o
irmão. Estava afeita. Afeita, mas cansada.

O irmão, empregado público, era viúvo, mal"-humorado e envelhecido
precocemente. A esse tinha ela criado nos braços, desde os mais tenros
meses; fora para ele uma segunda mãe. Quantas vezes contava às sobrinhas
as travessuras do seu pequeno Luciano, que aí estava agora tristonho,
achacado e impertinente!

E ela gozava relatando os episódios da meninice dele: os caprichos que
lhe satisfazia para o não ver chorar, as horas que perdia do sono para o
embalar nos braços, os sustos com as doenças e as quedas, e uma noite
que passara em claro para fazer um trajo de anjo com que Lucianinho foi
à procissão do Corpo de Deus.

--- Nesse tempo o vigário do Engenho\ldots{}

Mas as sobrinhas interrompiam"-na: queriam saber como era o vestido,
esforçando"-se por imaginar a figura do pai, agora tão enrugado e
taciturno, com seis anos apenas e vestidinho de anjo!

A velha satisfazia"-lhes a curiosidade com um sorriso de gosto: era um
vestidinho salpicado de lentejoulas e guarnecido de rendas. Nada faltara
ao irmão, --- nem a cabeleira em cachos, com o seu grande diadema cheio
de pedrarias, alto na frente, em bico; nem as asas de penas brancas,
entre as quais pusera um ramo de flores do campo, em tufos de filó; nem
as meias arrendadas, e os sapatos de cetim branco com uma roseta azul,
nem as pulseiras, o colar, o lenço guarnecido de rendas, cuja
extremidade ele oferecera graciosamente a outro anjo que ia a seu lado,
no mesmo passo. As sobrinhas ouviam"-na rindo e faziam"-na repetir certas
travessuras do pai, a que elas achavam muita graça, mas que lhes
pareciam absurdas. Custava"-lhes a crer que o pai, tão sisudo, tivesse
feito aquilo; mas a tia afirmava"-lhes tudo com segurança, mesmo diante
dele, que não protestava, e elas ficavam satisfeitas, tendo com as
antigas maldades do pai como que uma desculpa para as suas.

Entretanto, tia Amanda não parava de trabalhar; cosia as meias de toda a
gente de casa, cortava papéis de balas para uma vizinha doceira e rendas
para os pudins das confeitarias.

Ganhava pouco, e esse pouco dava"-o, tão habituada estava desde moça a
trabalhar para os outros.

A pouco e pouco a pobre velhinha foi também perdendo a memória:
confundia datas, relatava atrapalhadamente os fatos; a sua tesourinha já
se não movia com tanta delicadeza, as mãos tornaram"-se"-lhe mais pesadas,
a vista enfraqueceu; os pontos nas meias já não formavam o mesmo
xadrezinho chato e igual, e o serviço das confeitarias começou a
escassear até que lhe faltou completamente.

Nesse dia a pobrezinha chorou. O irmão não lhe dava nada\ldots{} como poderia
ela socorrer as desgraçadas que até então protegera?

No fim do mês lá foram ter com ela a viúva pobre dos sete filhos e a
comadre tísica. A velha não teve coragem de lhes contar a verdade;
corou\ldots{} e prometeu mandar"-lhes no outro dia alguma coisa. E no outro
dia mandava o que a casa de penhores lhe dera pelo seu relógio antigo, e
que ela tinha destinado para a primeira sobrinha que casasse.

Mas a história do relógio foi depressa sabida pela gente de casa.

As filhas de Luciano contaram ao pai, indignadas, que a tia o expunha ao
ridículo, mandando empenhar coisas, como se não tivesse que comer em
casa! O Luciano ouviu"-as, mordendo o bigode branco, com a indignação das
filhas a refletir"-se"-lhe nos olhos. Foi imediatamente falar à irmã.
Achou"-a cosendo na sua cadeira de rodas, os óculos caídos sobre o nariz,
a cabeça pendida.

Vendo"-o, ela sorriu"-se. Ele perguntou"-lhe num tom azedado pelo seu mau
fígado:

--- Então? é verdade que você mandou empenhar o seu relógio de ouro?

--- É, respondeu ela na sua costumada placidez.

--- Mas eu não quero isso! Hão de pensar lá fora que não lhe dou de
comer! Tome cuidado! A velha estremeceu, e nos seus olhos azuis brilhou,
fugitiva, uma expressão dolorosa.

\emph{Tome cuidado}! Quantas vezes dissera ela aquelas mesmas palavras
ao Lucianinho, nos velhos tempos! Dizia"-lhas com meiguice, alisando"-lhe
os cabelos, ou entre dois beijos: --- ``Olha meu filho, toma cuidado!
não te exponhas ao sol\ldots{} não comas frutas verdes! estuda bem as
lições\ldots{} Toma cuidado contigo, meu amor!''

E eram quase súplicas aqueles conselhos!

E aí estava agora o Luciano a dizer"-lhe colérico e brutalmente as mesmas
palavras! E ela curvava a cabeça ao irmão, e obedecia"-lhe, e temia"-o!
ela, que o criara desde pequenino, que por causa dele perdera um
casamento, que por causa dele se tinha sempre sacrificado! Era duro, mas
era assim. Há sempre mais paciência para as maldades de uma criança, do
que para as rabugices de um velho! Reconhecia isso e calava"-se.
``Luciano é doente, pensava ela, e é por isso que me trata com tão mau
humor! É doença, não é ruindade de coração\ldots{} Se ele foi sempre tão bom!
Aquilo há de passar.''

No fim do mês a questão estava esquecida, e a velha recebeu a visita da
comadre tísica e da viúva pobre. Não tinha um vintém, e resolvera dizer
isso mesmo às suas protegidas; mas exatamente nessa ocasião a tísica
mostrou"-lhe uma receita do médico, tossindo a cada palavra, com a mão
espalmada no peito; e a viúva levou"-lhe pela primeira vez o filho mais
novo, um lindo menino de olhos azuis e de cabelos loiros.

A velha enterneceu"-se e prometeu mandar no dia seguinte \emph{alguma
coisa}, tanto a uma como a outra.

Nessa mesma tarde disse ao Luciano, muito constrangida:

--- Hoje vieram cá aquelas pobres\ldots{} Coitadas! custa"-me tanto não lhes
dar esmola\ldots{} se você me pudesse emprestar\ldots{} é pouca coisa, bem vê\ldots{}

--- Acha muito o que eu ganho? não se lembra que mal me dá o ordenado
para sustentar as quatro filhas e a nós?

E como ela lhe explicasse a precária situação das duas mulheres:

--- Ora, a viúva que empregue os filhos mais velhos e ponha os outros em
asilos; e quanto à tísica\ldots{}

--- Se eu tivesse vinte anos de menos, não te pediria isto, Luciano!
Lembra"-te bem!

Mas o Luciano não se lembrou!

Ela quis referir"-se ao tempo em que o ajudava trabalhando para fora,
cuidando"-lhe dos filhos, indo muitas vezes para a cozinha, e deitando"-se
fora de horas para lhe engomar as camisas\ldots{} quis referir"-se, mas
envergonhou"-se, e disse de si para si:

--- Aquilo é doença; não é \emph{ruindade} de coração!

No entanto, o seu bom Lucianinho e as filhas comentavam entre si a
caduquice da velha. E, realmente, desde aquele dia, a paralítica decaiu
muito; incomodava toda a gente. Era preciso levá"-la ao colo para a cama,
despi"-la, vesti"-la, lavá"-la, levar lhe a comida à boca. Ela
impacientava"-se quando lhe tardavam com o almoço; gritava de lá que a
queriam matar à fome, que era melhor enterrarem"-na de uma vez. E a
criada, a quem ela dera outrora presentes, ria"-se; e as sobrinhas, que
ela tantas vezes carregara ao colo, levantavam os ombros, enfadadas.
Luciano repreendia"-as, mas ia dizendo que efetivamente a irmã era
insuportável!

Apesar de muitíssimo idosa, a pobre senhora tinha apego à vida; já muito
confusa de ideias, completamente inerte, tinha impertinências, ralhava
lá da sua cadeira de rodas com toda a gente: esta porque não lhe dava
água, aquela porque lhe apertara de propósito o cós da saia, aquela
outra porque lhe deitava veneno na comida\ldots{}

Deslizavam assim amargamente os meses quando, um dia, uma criada, muito
pálida, com os olhos esgazeados e os cabelos hirtos, entrou aos gritos
na sala de jantar, exclamando:

--- Fogo! fogo! há fogo em casa!

Levantaram"-se todos da mesa.

Por uma janela aberta entrou uma lufada de fumo; viu"-se brilhar a chama.
A porta catava tomada pelo fogo.

--- Fujam pelo telhado! gritou o Luciano.

E ouviam"-se vozes lá fora, dizendo como um eco:

--- Fujam pelo telhado!

Na sua grande cadeira de rodas, a velha presenciava aquela cena, sem se
poder mover, aterrorizada e sem voz. O irmão empurrava as filhas, atava
num guardanapo as joias tiradas à pressa de uma cômoda, punha na mão da
criada os talheres de prata, olhava para trás, para o fogo que vinha
lambendo a parede, impelido pelo vento; corria, atirava para o telhado
os móveis mais leves, pressurosamente, abria e fechava gavetas, e
saltava por fim também pela janela, para o telhado do vizinho, o único
meio de salvação que a Providência lhe oferecia !

A velha ficou só. Tentou mexer"-se, tentou gritar: debalde.

Pior que o incêndio e que o medo foi a impressão deixada pela fugida do
irmão.

O seu espírito cansado como que se esclareceu nesse momento. E dessa vez
não disse de si para si, para desculpá"-lo: ``Aquilo é doença, não é
ruindade de coração!\ldots{}''

O calor afogueava"-lhe as faces, onde há muito não subia o sangue; no
meio daquela solidão pavorosa, ouvindo o crepitar da madeira nuns
estalidos secos, a bulha surda de uma ou de outra viga que se
desmoronava, o lufe"-lufe da chama que subia, a velha sorria com ironia,
lembrando"-se da precaução do Luciano em arrecadar as coisas que ela, a
irmã abandonada, lhe ajudara a ganhar\ldots{}

E voltou de novo o olhar para a janela; então, entre o fumo já espesso,
viu desenhar"-se ali uma figura de homem.

O coração bateu"-lhe com alegria.

--- É Luciano que se lembrou de mim!\ldots{}

Era um bombeiro que lhe estendia a mão, chamando"-a. A velha fez"-lhe um
gesto, --- que se retirasse!

Nisso, um rolo de fumo negro interpôs"-se entre ambos, como um véu de
crepe. Perderam"-se de vista. O bombeiro voltou para fora, quase
asfixiado. A velha fechou os olhos e esperou a morte.

\chapter{Perfil de preta (Gilda)}

\hfill{}\emph{A Machado de Assis}

\bigskip

Suruí\footnote{Bairro do Município de Magé, no estado do Rio de Janeiro.}:
sol de rachar. Às onze horas, pela estrada quente, mal sombreada por uma
ou outra gameleira, vinha a negra Gilda da situação Fonseca, com a cesta
de taquara carregadinha de beijus, agasalhados na toalha recortada à mão
por sua senhora, d. Ricarda Maria.

A pele preta não desgosta do sol; mas era tão ardente esse de dezembro,
que a Gilda, suando em bica, meteu"-se pelo primeiro atalho para o mato
até à margem do rio. O caminho seria mais longo, paciência.

Logo que entrou na selva regalou"-se roçando as solas dos pés, queimados
pela areia da estrada descoberta, nas trapoerabas macias, onde florinhas
roxas desabrochavam à sombra de caneleiras cheirosas e de cada árvore,
que Deus nos acuda! Tinha o seu medo de andar por ali; sempre era mais
arriscado o encontro de uma cobra que pela estrada. Mas o frescor do
mato e o marulhar do rio tentavam"-na, foi andando. E tinha que andar,
porque a freguesia de S. Nicolau ainda era dali a um bom quarto de
légua, e, depois de ter oferecido os beijus em nome da ama, à sua irmã
d. Luiza, teria de voltar à situação antes do pôr do sol.

Com aquele calor\ldots{}

O cheiro agreste dos cambarás punha tontas as borboletas cor de palha.
Das altas copas dos paus"-de"-arco caía um chuvisco de ouro, em pétalas
pequeninas. Perfume e silêncio. De repente a água do rio repuxou alto;
Gilda parou; nada! Cantou um jacu, mas calou"-se logo, pressentindo
gente. A água voltara à plácida correnteza, não encontrando estorvos no
caminho.

Gilda retardava os passos, e já não deixava de sondar com o olhar
afeito, as águas moles. Súbito, numa clareira pequena, onde havia sol,
divisou junto à margem um grande peixe dorminhoco e sossegado. A pele
mosqueada do animal luzia dentro da água colorida de roxo pela copa
florida de um pé de quaresma, como uma espada enferrujada nos copos. A
água trêmula coloria"-o de lapidações de ametistas e ele dormia a sesta,
de olhos abertos, ventre roçando na areia.

Gilda pousou o balaio no chão, entalou a saia entre as pernas roliças,
e, pé ante pé, muito devagarinho, entrou no rio, agachou"-se e, zás!
Agarrou com ambas as mãos o peixe gordo, que se debateu sobressaltado,
violentamente, num reboliço gorgolhador, salpicando"-a toda. Sentindo"-o
escorregar por entre os dedos, Gilda atirou"-o para uma aberta da
clareira, sobre um pouco de mato carrasquento de roça abandonada. O
peixe arqueou"-se todo em saltos, unindo o rabo à cabeça numa ondulação
violenta, com ânsia de mergulhar de novo, no esforço de buscar a vida
que lhe roubavam. O sol secava"-lhe a pele lisa, que brilhava à luz em
reflexos de ardósia e prata ; os olhos exorbitavam"-se"-lhe, redondos como
dois globos foscos que o furor incandescia, e o corpo torcia"-se"-lhe ora
no ar, ora no chão, descrevendo curvas, num movimento incessante,
batendo na terra quente para, de um salto flexível, de acrobata doido,
atirar"-se de encontro a um tronco espinhento de paineira, sem se dar por
vencido, na ânsia de viver.

Gilda deixava"-o debater"-se, deliciada com aquela agonia longa, nervosa,
que observava com atenção alegre, no triunfo da sua força animal.

A tortura do peixe prolongava"-se; ele era valente, resistia ao ar seco,
ao sol ardente, à dureza do chão, aos embates nos espinhos que o feriam,
aos atritos dos seixos escaldantes e dos tronquinhos secos do ervaçal.
Pouco a pouco o cansaço ia"-o amolecendo, um fio de sangue escuro
corria"-lhe do ventre, um arrepio enrugava"-lhe o dorso e ficou por fim
todo estendido, batendo só com o rabo, convulsivamente, no chão áspero.
Depois nem um tremor mais; quedou"-se imóvel. Gilda cuidou"-o morto e
acocorou"-se para o ver de perto, quando, em um arranco supremo, o peixe
lhe saltou por sobre a cabeça, relanceando um fulgor de aço no ar
abafado e indo cair em um baque nas trapoerabas, quase à beira do rio.

Ouviu ele ainda o som mole das águas correndo sobre areias frias, sentiu
na pele queimada o frescor das ervinhas brandas, mais um impulso e
mergulharia na corrente salvadora\ldots{} não pôde; a carne mole não lhe
obedecia à convulsão da vontade.

Gilda cortou uma taquara, lascou"-a com força e, aproximando"-se, varou o
peixe de guelra a guelra. Ele estrebuchou languidamente, e a negra riu
empunhando o bambu, como uma lança de guerra sobre um corpo inimigo.

Foi só depois de tudo consumado que a Gilda se lembrou de que tinha de
entregar os beijus ainda quentinhos à irmã da sua senhora\ldots{} Voltou"-se;
uma mosca varejeira zumbia sabre a toalhinha branca, em lampejos de
metal azul. Um gesto da negra e ei"-la que partiu.

Deviam ser horas de se ir encaminhando para a freguesia de S. Nicolau do
Paço. Antes de prosseguir, amarrou com um cipó as taquaras em cruz,
escondeu o peixe entre folhas de inhame e depois de ter marcado o sítio
recomeçou a caminhada. Foi"-se embora, apanharia o peixe no regresso\ldots{}

Que voltas teria dado a Gilda por aqueles morros e aquelas vargens, que
só à tardinha entrou na freguesia, com a cesta de beijus, que deveria
entregar quentinhos, já muito desfalcada?

Foi talvez no mandiocal de seu Neves, quando parou ouvindo as cantigas e
vendo arrancar mandioca bonita, de lua nova\ldots{}

Não, a maior demora deveria ter sido na casa do João Romão, deitada na
esteira, no pomarzinho de tangerinas, daquelas pequeninas, que ela comia
com casca e tudo.

Nesse dia não o tinha encontrado, perdera umas duas horas a esperá"-lo,
de papo para o ar, vendo as nuvens dos mosquitos.

Por onde andaria ele?

João Romão era vadio, cantava à viola e trazia pelo beiço toda a
crioulada da redondeza. Gilda mordia"-se de ciúmes sempre que o via, lá
no engenho de d. Ricarda Maria, mais voltado para a Paula ou para a
Norberta do que para ela. Quando o censurava por isso, ele levantava os
ombros e ia dizendo que gostava de contentar toda a gente\ldots{}

Pois era sol posto quando a Gilda divisou a igreja de S. Nicolau, com o
seu mato de limoeiros perto, e as suas paredes brancas alvejando em uma
tristeza de abandono\ldots{}

Nem um badalar de sino. Voavam pombas"-rolas à procura dos ninhos e
crianças sujas cantavam em rondas na primeira rua da povoação. Gilda
apressou o passo até a uma casa velha de janelas de peitoril.

D. Luiza andava de visita a uma comadre; a preta deixou"-lhe a cesta de
beijus com a cozinheira Sophia e depois de ter engolido uma caneca de
café girou sobre os calcanhares, pensando no terror da estrada pelo
escuro. Bem faria se caminhasse sempre depressa, mas no canto da praça
viu gente ajuntada na porta da venda e foi"-se chegando curiosamente.

Falava"-se do milagre. S. Nicolau, deposto do seu trono de honra no altar
mor, fora colocado irreverentemente no chão, embaixo do coro, para que
ali lhe carminassem à vontade o rosto desbotado e lhe assinalassem os
traços já sumidos.

Deixaram"-no para ali sozinho, sem lâmpada nem vigia por toda uma feia
noite! Daí, que aconteceu? Na outra madrugada o sacristão viu com os
seus olhos carnais, que a terra havia de comer, o bom S. Nicolau do
Paço, lá no alto do seu trono condigno! Ninguém o removera; o santo
tinha subido aquela famosa altura, pelos seus próprios pés, que os não
tinha de fato, visto que a túnica de madeira, com douraduras e vernizes,
descia"-lhe até ao chão\ldots{}

Gilda estremeceu, e antes de seguir seu caminho voltou o olhar esgazeado
para o bosquezinho de limoeiros odorantes, perto da igreja.

Nossa Senhora! Arrependia"-se agora de não ter vindo direitinha dar o seu
recado logo pela manhã. Não eram as fúrias de d. Ricarda Maria, tão
impertinente, o que ela mais temia, mas as almas penadas que andassem
soltas, gemendo pelo mato. Lá a sua senhora? que se ninasse! já não
havia escravos. Agora os fantasmas, esses! S. Nicolau que a
acompanhasse.

Benzeu"-se e foi andando com o coração nas mãos, volvendo os olhos
esbugalhados para as beiras do caminho. Luzia"-lhe a esperança de pedir
pousada ao João Romão: cortaria assim a pior parte do caminho e dormiria
com ele.

Por mal dos seus pecados, a noite estava negra e um ventozinho precursor
de chuva agitava as ramagens, imitando vozes extravagantes.

Passado o negrume do mandiocal do Neves, ao dobrar mesmo a estrada, no
ângulo onde de dia tanto se enchera de araçás, Gilda estacou
boquiaberta. Através do rendilhado negro das galharias folhudas, ela viu
luzes, grandes luzes bailando vagarosamente, lá na beira do rio.

S. Nicolau me acuda! suspirou ela, com os joelhos bambos, o coração aos
pulos, estarrecida. S. Nicolau valeu"-lhe, fazendo"-a reconhecer nas luzes
archotes de bagaço de cana seca, que alumiavam o João Romão, a Norberta
e mais três parceiros, na pescaria do bagre amarelo em tocas de pedras
frias. O que enfureceu a Gilda foi ver o mulato abraçar Norberta, mesmo
ali, à vista dos outros\ldots{}

--- Que jundiá que vocês apanhem tenha veneno, diabos! rosnou ela com
desejo de irromper pelo mato e ir bater naquela gente, ruim que nem
cobra. Repeliu a ideia, estava sozinha, os outros eram muitos.

Esquecendo"-se de ir procurar o seu peixe gordo, sepultado entre folhas
de inhame junto à cruz de taquara, e que mesmo a escuridão não
permitiria encontrar, Gilda seguiu para diante, tecendo ideias de
vingança.

--- João Romão me paga, deixa está ele! Pensam que podem comungo\ldots{} não
vê!

Um uivo lamentoso atravessou a floresta e houve uma bulha de animal de
rastos. Gilda nem fez caso. A raiva tirara"-lhe o medo.

Às seis horas da manhã, d. Ricarda Maria apareceu no Engenho, e, dando
com a Gilda no trabalho, gritou"-lhe, furiosa:

--- Então, sua cachorra, é assim que você cumpre ordens?

Contra o costume a negra baixou a cabeça, humilhada e sonsa, relanceando
a vista para a Norberta que enchia um tipiti para a prensa, no meio de
uma nuvem fina de farinha que o João Romão peneirava a seu lado.
Norberta passava por ser a crioula mais bonita do Engenho. Era tafula,
vestia"-se de engomados. Pareceu à Gilda, através da névoa branca, que
ela se ria na ocasião, e teve ímpetos de lhe atirar à cara a cuia com
que levava mandioca do coxo para o forno, que a Paula remexia com a
longa pá.

Tia Teresa, a africana velha entendida em rezas e feitiços, cosia os
sacos, agachada a um canto, e, enquanto uns negros entravam com cestos
de mandioca para a raspagem, outros traziam"-na do lavador para a
cevadeira, já branquinha como ossos nus\ldots{}

D. Ricarda Maria chupou o grande buço grisalho que lhe ornava o rosto
magro e ordenou ao João Romão que deixasse a peneiragem à Rita, e fosse
ele para a máquina.

Depois voltando"-se, inquiriu:

--- O coxo está seco? Que é do Viriato?

--- Viriato tá cortando mandioca, sim senhora\ldots{} respondeu o Joaquim
velho, que entrava suando sob um fardo de aipins.

D. Ricarda Maria postou"-se ao lado da bolandeira e o mulato sentou"-se,
tanto se lhe dando fazer um serviço como o outro. A velha gritou então
que abrissem a água, e a engenhoca roncou.

--- É agora, pensou Gilda consigo, voltando"-se. Norberta olhava
embevecida para o João Romão, aproveitando a distração da patroa. O
mulato é que não podia desviar a vista do trabalho, sob pena de ficar
sem dedos ou sem braços. A máquina descrevia os seus movimentos rápidos,
impelida pela força da água, triturando, esfarelando as raízes brancas
da mandioca, num mastigar incessante.

Tia Teresa cantava num fio delgado de voz, estendendo os pés gretados
pelo chão, onde tremia uma roseta de sol caída do teto, de telha vã.

Gilda observou: estavam todos preocupados; então, avançando, disse num
berro furioso:

--- João Romão!

O mulato voltou"-se assustado e a máquina segurou"-o logo pela mão
direita, e levar"-lhe"-ia o braço se d. Ricarda Maria não o tivesse puxado
imediatamente para trás, com um movimento rápido e violento.

O sangue espadanou, houve rumor, o mulato caiu.

Gilda, vingada, num tremor de raiva e de espanto, dizia que só dera o
grito ao perceber a catástrofe. Aquela mentira sabia"-lhe tão limpa como
se fora uma verdade. Só a Norberta, fula, espumando irada, a desmentia,
xingando"-a, em avanços de animal danado:

--- Foi de propósito! prendam aquele diabo! foi de propósito! exclamava
ela debatendo"-se nas mãos das companheiras, que a continham a custo.

--- Como ele não quer mais saber dela! foi de propósito! Amaldiçoada!

Mas todas afirmavam que o caso deveria ter sido como a Gilda explicava,
por que não? Fora tudo momentâneo, e a própria d. Ricarda Maria, ali de
vigia, não se sentia habilitada nem para acusar, nem para defender\ldots{}

Eis aí por que o João Romão nunca mais seduziu as crioulas dedilhando na
viola aquelas modinhas faceiras e sentimentais.

Apesar de o ver maneta e de o saber preguiçoso, Norberta fez"-se a sua
companheira definitiva. Essa trabalha por dois, e, sempre que vê a Gilda
passar pela sua porta, cantando escarninhamente com as mãos para as
costas, ela cospe três vezes, dependura do umbral o ramo de arruda, faz
no vazio o sinal da cruz e diz de modo a fazer"-se ouvir da outra:

--- Te esconjuro, diabo!

\chapter{A nevrose da cor}

Desenrolando o papiro, um velho sacerdote sentou"-se ao lado da bela
princesa Issira e principiou a ler"-lhe uns conselhos, escritos por um
sábio antigo. Ela ouvia"-o indolente, deitada sobre as dobras moles e
fundas de um manto de púrpura; os grandes olhos negros cerrados, os
braços nus cruzados sobre a nuca, os pés trigueiros e descalços unidos à
braçadeira de ouro lavrado do leito.

Pelos vidros de cores brilhantes das janelas, entrava iriada a luz do
sol, o ardente sol do Egito, pondo reflexos fugitivos nas longas barbas
prateadas do velho e nos cabelos escuros da princesa, esparsos sobre a
sua túnica de linho fino.

O sacerdote, sentado num tamborete baixo, continuava a ler no papiro,
convictamente; entretanto a princesa, inclinando a cabeça para trás,
adormecia!

Ele lembrava"-lhe:

``A pureza na mulher é como o aroma na flor!''

``Ide confessar a vossa alma ao grande Osíris! para a terdes limpa de
toda a mácula e poderdes dizer no fim da vida: \emph{Eu não fiz derramar
lágrimas; eu não causei terror!}''

``Quanto mais elevada é a posição da mulher, maior é o seu dever de bem
se comportar.''

``Curvai"-vos perante a cólera dos deuses! lavai de lágrimas as dores
alheias, para que sejam perdoadas as vossas culpas!''

``Evitai a peste e tende horror ao sangue\ldots{}''

--- Notai bem, princesa:

\emph{E tende horror ao sangue!}

A princesa sonhava: ia navegando num lago vermelho, onde o sol estendia
móvel e quebradiça uma rede dourada. Recostava"-se num barco de coral
polido, de toldo matizado sobre varais crivados de rubis; levava os pés
mergulhados numa alcatifa de papoulas.

Quando acordou, o sacerdote, já de pé, enrolava o papiro, sorrindo com
ironia.

--- Ainda estás aí?

--- Para vos repetir: Arrependei"-vos, não abuseis da vossa posição de
noiva do senhor de todo o Egito\ldots{} lavai para sempre as vossas mãos do
sangue\ldots{}

A princesa fez um gesto de enfado, voltando para o outro lado o rosto; e
o sacerdote saiu.

Issira levantou"-se, e, arqueando o busto para trás, estendeu os braços,
num espreguiçamento voluptuoso.

Uma escrava entrou, abriu de par em par a larga janela do fundo, colocou
em frente a cadeira de espaldar de marfim com desenhos e hieróglifos na
moldura, pôs no chão a almofada para os pés, e ao lado a caçoula de onde
se evolava, enervante e entontecedor, um aroma oriental.

Issira sentou"-se, e, descansando o seu formoso rosto na mão, olhou
demoradamente para a paisagem.

O céu, azul"-escuro, não tinha nem um leve traço de nuvem. A cidade de
Tebas parecia radiante. Os vidros e os metais deitavam chispas de fogo,
como se aqui, ali e acolá, houvesse incêndio; e ao fundo, entre as
folhagens escuras das árvores ou as paredes do casario, serpeava, como
uma larga fita de aço batida de luz, o rio Nilo.

Princesa de raça, neta de um Faraó, Issira era orgulhosa; odiava todas
as castas, exceto a dos reis e a dos sacerdotes. Fora dada para esposa
ao filho de Ramazés, e, sem amá"-lo, aceitava"-o, para ser rainha.

Era formosa, indomável, mas vítima de uma doença singular: a nevrose da
cor. O vermelho fascinava"-a.

Muito antes de ser a prometida do futuro rei, chegava a cair em
convulsões ou delíquios ao ver flores de romãzeiras, que não pudesse
atingir, ou as listas encarnadas dos \emph{kalasiris}\footnote{Túnica
  egípcia.} dos homens do povo.

A medicina egípcia consultou as suas teorias, pôs em prática todos os
seus recursos, e curvou"-se vencida diante da persistência do mal.

Issira, entretanto, degolava as ovelhinhas brancas, bebia"-lhes o sangue,
e só plantava nos seus jardins papoulas rubras.

Na aldeia em que nascera e em que tinha vivido, Karnac, forrara de linho
vermelho os seus aposentos; era neles que ela bebia em taças de ouro o
precioso líquido.

Princesa e formosa, a lama levou"-lhe o nome ao herdeiro de um Ramazés; e
logo o príncipe, curioso, seguiu para essa terra.

O seu primeiro encontro foi no templo. Ele esperava"-a no centro do
enorme pátio, entre as galerias de colunas, ansiosamente. Ela vinha no
seu palanquim de seda, coberta de pérolas e de púrpura, passando
radiante entre as seiscentas esfinges que flanqueavam a rua.

Dias depois morria o pai de Issira, último descendente dos Faraós, após
a sua costumada refeição de leite e mel. O príncipe Ramazés solicitou a
mão da órfã e fê"-la transportar para o palácio real, em Tebas.

A beleza de Issira deslumbrou a corte; a sua altivez fê"-la respeitada e
temida; a paixão do príncipe rodeou"-a de prestígio e a condescendência
do rei acabou de lhe dar toda a soberania.

O seu porte majestoso, o seu olhar, ora de voltado ora de fogo, mas
sempre impenetrável e sempre dominador, impunham"-na à obediência e ao
servilismo dos que a cercavam.

Esquecera a placidez de Karnac. Lamentava só as ovelhinhas brancas que
ela imolava nos seus jardins das papoulas rubras.

A loucura do encarnado aumentou.

Os seus aposentos cobriram"-se de tapeçarias vermelhas. Eram vermelhos os
vidros das janelas; pelas colunas dos longos corredores enrolavam"-se
hastes de flores cor de sangue.

Descia às catacumbas iluminada por fogos encarnados, cortando a
grandiosa soturnidade daqueles enormes e sombrios edifícios, como uma
nuvem de fogo que ia tingindo, deslumbradora e fugidia, os sarcófagos de
pórfiro ou de granito negro.

Não lhe bastava isso; Issira queria beber e inundar"-se em sangue. Não já
o sangue das ovelhinhas mansas, brancas e submissas, que iam de olhar
sereno para o sacrifício, mas o sangue quente dos escravos revoltados,
conscientes da sua desgraça; o sangue fermentado pelo azedume do ódio,
sangue espumante e embriagador.

Um dia, depois de assistir no palácio a uma cena de pantomimas e
arlequinadas, Issira recolheu"-se doente aos seus aposentos; tinha a boca
seca, os membros crispados, os olhos muito brilhantes e o rosto
extremamente pálido.

O noivo andava por longe a visitar províncias e a caçar hienas.

Issira, estendida sobre os coxins de seda, não conseguia adormecer.
Levantava"-se, volteava no seu amplo quarto, desesperadamente, como uma
pantera ferida a lutar com a morte.

Faltava"-lhe o ar; encostou"-se a uma grande coluna, ornamentada com
inverossímeis figuras de animais entre folhas de palmeira e de lódão; e
aí, de pé, movendo os lábios secos, com os olhos cerrados e o corpo em
febre, deliberou mandar chamar um escravo.

A um canto do quarto, estendida no chão, sobre a alcatifa, dormia a
primeira serva de Issira.

A princesa despertou"-a com a ponta do pé.

Uma hora mais tarde, um escravo, obedecendo"-lhe, estendia"-lhe o braço
robusto, e ela, arregaçando"-lhe ainda mais a manga já curta do
\emph{kalasiris}, picava"-lhe a artéria, abaixava rapidamente a cabeça, e
sugava com sôfrego prazer o sangue muito rubro e quente!

O escravo passou assim da dor ao desmaio e do desmaio à morte; vendo"-o
extinto, Issira ordenou que o removessem dali, e adormeceu.

Desde então entrou a dizimar escravos, como dizimara ovelhas.

Subiam queixas ao rei; mas Ramazés, já velho, cansado e fraco, parecia
indiferente a tudo.

Ouvia com tristeza os lamentos do povo, fazendo"-lhe promessas que não
realizava nunca.

Não queria desgostar a futura rainha do Egito; temia"-a. Guardava a doce
esperança da imortalidade do seu nome. E essa imortalidade, Issira
poderia cortá"-la como a um frágil fio de cabelo. Formosa e altiva,
quando ele, Ramazés, morresse, ela, por vingança, fascinaria a tal ponto
os quarenta juízes do \emph{julgamento dos mortos}, que eles procederiam
a um inquérito fantástico dos atos do finado, apagando"-lhe o nome em
todos os monumentos, dizendo ter mal cumprido os seus deveres de rei!

Não! Ramazés não oporia a sua força à vontade da neta de um Faraó! Que a
maldita casta dos escravos desaparecesse, que todo o seu sangue fosse
sorvido com avidez pela boca rosada e fresca da princesa. Que lhe
importava, e que era isso em relação à perpetuidade do seu nome na
história?

As queixas rolavam a seus pés, como ondas marulhosas e amargas; ele
sofria"-lhes o embate, mas deixava"-as passar!

Issira, encostada à mão, olhava ainda pela janela aberta para a cidade
de Tebas, esplendidamente iluminada pelo sol, quando um sacerdote lhe
foi dizer, em nome do rei, que viera da província a triste notícia de
ter morrido o príncipe desastrosamente.

Recebeu a princesa com ânimo forte tão inesperada nova. Enrolou"-se num
grande véu e foi beijar a mão do velho Ramazés.

O rei estava só; a sua fisionomia mudara, não para a dolorosa expressão
de um pai sentido pela perda de um filho, mas para um modo de audaciosa
e inflexível autoridade. Aceitou com frieza a condolência de Issira,
aconselhando"-a a que se retirasse para os seus domínios em Karnac.

A egípcia voltou aos seus aposentos, e foi sentar"-se pensativa no dorso
de uma esfinge de granito rosado, a um canto do salão.

A tarde foi caindo lentamente; o azul do céu esmaecia; as estrelas iam a
pouco e pouco aparecendo, e o Nilo estendia"-se cristalino e pálido entre
a verdura negra da folhagem. Fez"-se noite. Imóvel no dorso da esfinge,
Issira olhava para o espaço enegrecido, com os olhos úmidos, as narinas
dilatadas, a respiração ofegante.

Pensava na volta a Karnac, no seu futuro repentinamente extinto, nesse
glorioso amanhã que se cobrira de crepes e que lhe parecia agora
interminável e vazio! Morto o noivo, nada mais tinha a fazer na corte.
Ramazés dissera"-lhe:

--- Ide para as vossas terras; deixai"-me só\ldots{}

Issira debruçou"-se da janela --- tudo negro! Sentiu rumor no quarto,
voltou"-se. Era a serva que lhe acendera a lâmpada.

Olhou fixamente para a luz; a cabeça ardia"-lhe, e procurou repousar.
Deitando"-se entre as sedas escarlates do leito, com os olhos cerrados e
as mãos pendentes, viu, em pensamento, o noivo morto, estendido no
campo, com uma ferida na fronte, de onde brotava em gotas espessas o seu
belo sangue de príncipe e de moço.

A visão foi"-se tornando cada vez mais clara, mais distinta, quase
palpável. Soerguendo"-se no leito, encostada ao cotovelo, Issira via"-o,
positivamente, a seus pés. O sangue já se não desfiava em gotas, uma a
uma, como pequenas contas de coral; caía às duas, às quatro, às seis,
avolumando"-se, até que saía em borbotões, muito vermelho e forte; Issira
sentia"-lhe o calor, aspirava"-lhe o cheiro, movia os lábios secos,
buscando"-lhe a umidade e o sabor.

A insônia foi cruel. Ao alvorecer, chamando a serva, mandou vir um
escravo.

Mas o escravo não foi. Ramazés atendia enfim ao seu povo, proibindo à
egípcia a morte dos seus súditos. Um sacerdote foi aconselhá"-la.

--- Cuidado! A justiça do Egito é severa, e vós já não sois a futura
rainha\ldots{}

Issira despediu"-o.

Perseguia"-a a imagem do noivo, coberto de sangue. A proibição do rei
revoltava"-a, acendendo"-lhe mais a febre do encarnado.

Como na véspera, o sol entrava gloriosamente pelo aposento, através dos
vidros de cor. A princesa mordia as suas cobertas de seda, torcendo"-se
sobre a púrpura do manto. De repente levantou"-se, transfigurada, e
mandou vir de fora braçadas de papoulas, que espalhou sobre o leito de
púrpura e ouro\ldots{}

Depois, sozinha, deitou"-se de bruços, estirou um braço e picou"-o bem
fundo na artéria. O sangue saltou vermelho e quente.

A princesa olhou num êxtase para aquele fio coleante que lhe escorria
pelo braço, e abaixando a cabeça uniu os lábios ao golpe.

Quando à noite a serva entrou no quarto, absteve"-se de fazer barulho,
acendeu a lâmpada de rubis, e sentou"-se na alcatifa, com os olhos
espantados para aquele sono da princesa, tão longo, tão longo\ldots{}

\chapter{As três irmãs}

\hfill{}\emph{A Zalina Rolim}

\bigskip

Havia muitos anos já que d. Teresa não via as duas irmãs. A segunda, d.
Lucinda, partira logo depois de casada, com o primeiro marido, para
Buenos Aires, e lá ficara sempre; a mais moça, d. Violeta, fora habitar
a Bahia com o seu esposo e ali estava gozando os triunfos acadêmicos dos
filhos e os respeitos delicados do seu \emph{velho}.

Mas um dia, d. Teresa, apreensiva, com medo da morte que se avizinhava,
escreveu às irmãs:

--- Que viessem ao Rio despedir"-se dela e tomar posse do que lhes
pertencia.

Interesse ou saudade\ldots{} (quem lê claro em corações tão bem ocultos?)
empurrou para as plagas natais as duas senhoras.

D. Teresa remoçou uns dias. Só ela ficara solteira e em casa dos pais,
já há tanto mortos, como um guarda fiel, depositária de todas as
relíquias da mocidade deles e delas! Assim, recomendou à governanta:

--- Olha, Emília! para a mana Lucinda arranja o quarto azul, aquele da
esquina\ldots{} era o seu quarto de solteira\ldots{} Ela gostava de canários\ldots{}
tinha sempre uma gaiola no quarto\ldots{} era isso: bota lá a gaiolinha
dourada do canário novo\ldots{} Escuta! Lava bem tudo! Ela era muito
faceira\ldots{} não te esqueças do pó de arroz, de pôr sabonete fino e
frascos de\ldots{} espera! qual era o cheiro que ela preferia?\ldots{} Ah! já sei!
jasmim! manda comprar essência da jasmins\ldots{}

--- Sim, senhora.

--- Agora, para d. Violeta prepara o quarto branco, das três janelas\ldots{}
Era o quarto dela! Vê se arranjas muitas flores\ldots{} Violeta era a nossa
jardineira!\ldots{} Olha, faze um ramo para o lavatório, outro para a cômoda.
Era assim que ela usava\ldots{} Espera! que pressa! Manda comprar essência de
violetas\ldots{} era o aroma dela!

--- Sim, senhora\ldots{}

--- Não te esqueças de nada!

--- Não, senhora\ldots{}

A governanta saiu, deixando d. Teresa aos guinchos com um ataque de
asma. Não queria morrer deixando aquela casa em mãos indiferentes. Só as
irmãs receberiam com amor aqueles trastes antigos, em que tantas vezes
rolaram juntas, onde os pães presidiam às suas travessuras de crianças e
onde, depois, os noivos as beijaram com embriaguez\ldots{} A pobre coitada
estava e desfazer"-se, sentia, a cada arranco da tosse, desmanchar"-se"-lhe
sob a pele seca e enrugada a carcaça frágil e dolorida. O seu corpo,
nunca amado, caía, como um feixe de ossos partidos, para a sepultura.
Como estariam as irmãs? A Lucinda deveria estar bem velhota! Agora a
Violeta, essa, apesar de mais moça, com tantos filhos e já tanta
netalhada, é provável que viesse trêmula e bem achacada pela velhice!
Havia já uns trinta anos que a não via\ldots{} e à outra\ldots{} uns bons
quarenta! E d. Teresa revia com saudade o rosto pálido e formoso da
esbelta Lucinda, de olhos verdes, dentes sãos, faces brancas como a
neve; e o rostinho delicado de Violeta, moreno, levemente rosado, com
uns olhos travessos e negros e uma boquinha perfumada de juventude,
muito fresca e vermelha!

E apesar de calcular"-lhes as rugas, só via diante dos olhos as figuras
louçãs e radiantes das irmãs noutros tempos\ldots{}

A mulata aprontou tudo com esmero. D. Teresa, apoiada ao seu ombro e a
uma bengala grossa, percorreu toda a casa. Ela tinha tido sempre a
singular mania de conservar as coisas nos mesmos lugares e em igual
posição. Se mandava renovar o papel de uma sala, exigia que o novo fosse
exatamente igual ao que de lá saísse; e os trastes eram polidos, os
estofos espanados com escrúpulo e as alcatifas nunca substituídas por
outras que não fossem da mesma cor e de igual desenho\ldots{} Para ela,
aquelas velharias eram preciosidades raras. Não sabia nunca, não dava
festas. Vagava no ar das suas salas um cheiro de mofo, denunciador do
triste isolamento da sua vida de solteirona, sem sobrinhos, nem
afilhados, nem ninguém!

Custava"-lhe deixar todo aquele esplendor em mãos alheias e ansiava pelas
irmãs. Por uma coincidência, chegaram no mesmo dia d. Violeta, vinda da
Bahia, e d. Lucinda, de Buenos Aires.

A manhã estava de uma beleza incomparável; o céu todo azul, atmosfera
morna, o que aprouve a d. Teresa, que pôde aliviar o peso da roupa e
cruzar sobre o vestido de seda roxo o seu belo mantelete de renda preta.
A Emília ajudou"-a naquela tarefa. Toda a roupa comparticipava daquele
cheiro de umidade. Vestido havia tanto tempo guardado, o que as rugas
fundas denunciavam, não podia cheirar a sol nem a primavera\ldots{}

No topo da escada, com a cabecinha trêmula sempre a dizer que sim, uma
das mãos apoiada à bengala, a outra sumida no braço da governanta, d.
Teresa esperava as irmãs com os olhos luminosos, molhados de lágrimas.
Elas subiam, vagarosas também, falando alto, uma com voz grave, outra em
um falsete de gaita. Haviam de ser risadinhas, lembranças da mocidade\ldots{}

D. Teresa ordenara que se abrisse o salão principal, e foram logo para
lá as três. O que ela notou, com certa alegria invejosa, foi que as
irmãs andavam mais direitas, sem necessidade de apoio. Sentaram"-se no
salão. D. Lucinda faiscava de vidrilhos, descansando a papada cor de
leite na rica seda preta da capa. Era enorme. A gordura disfarçava"-lhe
as rugas. O coquetismo da mocidade ainda mostrava os seus traços: lá
estava o cabelo pintado, caído nas fontes em duas \emph{belezas}, à moda
espanhola.

E de vez em quando saltitava um \emph{caramba}, que rebentava como uma
bomba naquela casa antiga e reservada.

D. Violeta, essa guardara alguma coisa do seu aroma de flor, para a
secura da velhice. Era pequena, muito engelhada; vinha vestida de lã
\emph{marrom}, com uma capa de pouco enfeite. O que lhe dava graça era o
cabelo muito branco e a meiguice dos seus olhos negros, habituados a
sorrir para os netos travessos.

D. Teresa era a mais acabada! Faltara"-lhe o amor, faltaram"-lhe as
sagradas agonias da maternidade, e a sua existência passiva,
concentrada, inerte, levara"-a àquele ponto, de passa seca.

As três irmãs olharam"-se com tristeza; mas o que pensaram não o
disseram. Os lábios sorriram, houve uns suspiros mal disfarçados e um
brilho de lágrimas, que pareceu molhar ao mesmo tempo os olhos de todas,
sem rolar pela face de nenhuma\ldots{} D. Lucinda rompeu o silêncio. Vinha
por pouco tempo\ldots{} o seu segundo marido, um argentino, morrera havia um
ano; tinha ainda muita coisa a liquidar\ldots{} O seu palacete não podia
ficar abandonado em mãos dos perversos enteados\ldots{} O seu palacete! Como
ela encheu a boca, descrevendo em duas palavras o luxo das suas mobílias
e da sua equipagem\ldots{}

Era conhecida e invejada na cidade toda!

D. Teresa pasmou:

--- Que! pois as suas mobílias são melhores do que\ldots{}

--- Estas?! Oh! E riu"-se com desdém. Teresa! você não imagina: isto é
horrível! \emph{Nós outras} temos coisas modernas, vindas de Paris! Meu
marido gastava todos os anos uma fortuna em quadros, em louças, em
cavalos e em roupas!

D. Teresa, pálida, com a cabecinha ainda mais trêmula, olhou para a irmã
Violeta.

--- E você?

--- Eu já não me importo com luxos\ldots{} meus netos acabam com tudo! A não
ser à missa, não vou a parte nenhuma\ldots{}

O que eu quero é ter muito espaço para as crianças e uma capela bonita.
Em minha casa celebra"-se sempre, com alguma pompa, o mês de Maria\ldots{} É o
nosso sistema.

--- Eu não conheço, modéstia à parte, casa mais completa do que a minha!
impou d. Lucinda.

--- Nem eu casa mais alegre do que a minha. Se saio, volto logo com
saudades\ldots{} murmurou d. Violeta.

D. Teresa disse, já um tanto envergonhada por tratar as irmãs por
\emph{você}, em um tom cerimonioso e encolhido:

--- Pois eu mandei pedir a\ldots{} vocês\ldots{} que viessem tomar conta das
mobílias e da casa, julgando que lhes fosse agradável\ldots{}

--- Vamos ver! interrompeu d. Lucinda, erguendo"-se com dificuldade bem
disfarçada. Emília amparou d. Teresa e seguiram todas em peregrinação.
D. Lucinda apalpava tudo e ia murmurando:

--- Esta mobília tem o estofo podre\ldots{} Olhem! e esgarçava com a unha o
damasco das poltronas.

--- Está mesmo\ldots{} afirmava d. Violeta. Assim tudo: este canapé é
medonho; eu não o quereria nem na minha cozinha! Meu Deus! esta sala de
jantar parece"-me um refeitório de convento\ldots{} E dizer que antigamente a
gente achava isto bonito\ldots{}

D. Violeta sorria; d. Teresa não chorava por vergonha, com respeito às
irmãs, que vinham mais fortes, com outros hábitos e outros gostos, cada
qual educada por um marido, com o espírito influenciado pelo espírito
deles; uma adorando o luxo, a outra a família e a igreja. Era bem certo,
o casamento e a distância roubaram"-lhe as irmãs para sempre; a Lucinda e
a Violeta de outrora estavam enterradas em algum cemitério de virgens;
aquelas duas velhas de gênios opostos\ldots{} não eram elas!

À noite, d. Teresa, opressa pela asma, não se quis recolher cedo ao seu
quarto. Emília foi dizer"-lhe com acento irônico:

--- D. Lucinda mandou tirar do quarto dela a gaiolinha. Diz que não pôde
suportar barulhos\ldots{} que o sono da manhã é o melhor!

Ao mesmo tempo aparecia d. Violeta com as flores na mão:

--- Isto não pode estar lá no quarto\ldots{} As flores devem ficar nos
jardins\ldots{} Lá em casa é o meu sistema.

\emph{Lá em casa}! pensou d. Teresa; \emph{lá em casal}! afinal cada uma
ama o que é seu, pensa no que é seu! Eu, só eu, amo esta casa, não
porque seja minha, mas porque era \emph{nossa}\ldots{} Serei melhor do que
elas? De onde me vêm esta ternura e esta saudade que elas não sentem?

D. Teresa chorou na penumbra da sala.

No dia seguinte mandou recolher ao quarto dos badulaques, no fundo do
quintal, os trastes mais antigos e de maior estimação. As irmãs zombavam
de tudo\ldots{} pois bem! deixaria escrito que se fizesse com eles uma
fogueira no dia do seu enterro. Mas não escreveu, e dois dias depois, à
hora do almoço, morreu sentada na sua cadeira de couro, com as mãos
sumidas no xale e a cabecinha pendida para o peito.

D. Violeta recolheu as imagens do oratório, como lembrança piedosa; d.
Lucinda, nada. Venderam a casa, repartiram os bens\ldots{} e foi cada uma
para o seu destino.

\chapter{O futuro presidente}

Uma\ldots{} duas\ldots{} três\ldots{} quatro\ldots{} e as horas foram soando numa lentidão
de relógio velho, até a décima pancada.

Era noite; pela janelinha aberta, do sótão, via"-se um pedaço do céu
estrelado, e nada mais.

No interior, havia um lampião de querosene sobre uma mesa de pinho; um
armário sem vidros, com cortinas de chita; cabides, máquina de costura e
uma ruma de caixas de papelão empilhadas num canto.

Junto à mesa uma mulher maltratada, magra, de olheiras fundas e dedos
calejados, curvava"-se para diante, pregando botões numa camisa para o
Arsenal.

Ao pé dela, num berço de vime, dormia regaladamente um pequerrucho,
gordo e trigueiro, com a cabeça enterrada na almofada e as mãozinhas
papudas e abertas, espalmadas sobre a colcha vermelha.

Além do tique"-taque do relógio, só se ouviam os estalidos da agulha e a
respiração regular da criança.

A mãe de vez em quando tirava da costura o seu olhar cansado e deixava"-o
cair sobre o filho. Os seus olhos verdes perdiam então pouco a pouco a
névoa de tristeza que os tornava sombrios, até irradiarem com a limpidez
das esmeraldas sem jaça.

O marido tardava; talvez passasse a noite toda fora, vigiando a linha
dos bondes, com a sua lanterna de cores, e ela aproveitaria o tempo para
coser e adiantar serviço; a vida é tão cara e eles ganhavam tão pouco\ldots{}

Pensando na dificuldade de se sustentarem, lembrava"-se do bom tempo em
que o marido era forte e ativo; agora o desgraçado tinha só uma perna e
o juízo já não era como dantes\ldots{} enfim, ajudava"-o ela; e daí a alguns
anos haveria mais alguém a auxiliá"-los.

Esse \emph{mais alguém} continuava a dormir tranquilamente, com as duas
mãozinhas abertas sobre a colcha.

Entretanto, a imaginação da mãe ia"-lhe abrindo um caminho florido e
largo através do misterioso e impenetrável futuro.

Com a costura caída nos joelhos, a cabeça voltada para o berço, ela
dizia mentalmente:

--- Ele há de ser bom, há de ser amado por toda a gente\ldots{} haverá
alegria nos olhos que o virem, e todas as mãos se estenderão para
apertar a sua mão honesta! Meu filho! Como ele dorme! Como ele é bonito!

Hei de ensiná"-lo a ser caritativo\ldots{} mas como? se nós somos tão
pobres\ldots{} Não faz mal, há de se arranjar um meio de o fazer dar esmolas!
Será abençoado assim pelos infelizes! Coitadinho! chorou tanto hoje!\ldots{}
faltou"-me o leite, talvez\ldots{} com esta vida de trabalho, não admira! E
tão manso que ele é! pobre criança!\ldots{} Pobre\ldots{} pobre! É preciso que ele
seja rico, para ter completa a felicidade! Isso é que há de ser mais
difícil\ldots{} e daí, quem sabe? talvez não\ldots{}

O pensamento ficou"-lhe suspenso nessa ideia; com um suspiro de desalento
voltou à costura, e os seus olhos foram"-se enturvando. Também o marido
tinha tido grandes esperanças de fazer fortuna; também ele arquitetara
castelos de ouro e cristal, e deitara"-se ao trabalho com amor e coragem;
também ele era probo, e digno, e leal, e aí estava quase inutilizado,
desde que a maldita máquina de um trem lhe esmigalhara uma perna,
mudando"-lhe o seu gênio desembaraçado e viril por aquela atual inércia,
doentia e triste!

A que está sujeita a gente de trabalho rude! Ela que, desde pequena, se
mostrava tímida, encolhida pelos cantos, séria e franzina, era quem mais
lidava e com maior ânimo, agora! O seu esforço seria compensado? poderia
levar ao fim a criação do filho? Chegaria a vê"-lo homem, bem"-educado,
poderoso, feliz?

Lembrava"-se de que, da última vez que tinha levado roupa ao Arsenal,
ouvira num bonde, entre dois sujeitos velhos e bem vestidos, uma
conversa que lhe causara impressão.

Falavam de pessoas de condição humilde, quase desprezíveis muitas vezes,
mas cuja inteligência, atividade e esforço conquistaram coisas
estupendas no mundo das artes, no mundo da ciência e no mundo da
política! Aludiam encomiasticamente a um rachador de lenha, que foi
chefe de estado; a um filho de um tanoeiro que chegou a marechal de
França e a príncipe; a um tecelão, nascido num subterrâneo, que foi um
grande botânico\ldots{}

Essa gente toda era apontada na História pelo seu valor extraordinário,
tendo alcançado, a par de grandes fortunas, o respeito universal!

Enquanto esperava que lhe recebessem o número, no Arsenal, ia repetindo
de si para si a conversa dos velhos, a tal ponto que a chamaram de
distraída\ldots{}

Distraída! O que ela estava era cogitando no futuro do filho!

Interrompeu de novo a costura, deu mais luz ao candeeiro, dobrou umas
camisolas já prontas, e recostou"-se um pouco, descansando as costas que
lhe doíam. O pequenito moveu"-se; ela arranjou"-lhe a coberta
delicadamente, para o não acordar, e pôs"-se a olhar para ele num êxtase.

--- Há de ser formoso, há de ser amado! virá um dia em que o solicitem
outros amores, em que a paixão de uma mulher o atraia a ponto de
esquecer"-me! O sacrifício que eu faço, as dores que sofri, as forças que
eu esgoto amamentando"-o, tendo"-o ao colo, perdendo com ele as noites,
serão coisas ignoradas, ou de que ele não faça senão uma ideia
incompleta! Meu filho! como eu já tenho ciúmes dessa outra que lhe há de
absorver toda a intensidade do seu afeto! Mas não; ela será toda
meiguice e amor, ela me ajudará a fazê"-lo feliz!\ldots{} Ele é inteligente\ldots{}
ele terá mesmo um talento notável! Será grande; será respeitado\ldots{}
chegará aos cargos mais altos\ldots{} meu filho! como ele é inocente! como
ele é puro!

Qual será o meu orgulho ouvindo chamarem"-no: ``Senhor doutor!'' e
vendo"-o deputado, a falar nas câmaras, com muita nobreza e distinção,
correto, simpático e justo! Depois\ldots{} por que não virá a ser meu filho o
presidente da República?

Neste ponto os olhos da pobre mulher lampejaram de alegria; as suas
grandes pupilas verdes tornavam"-se verdadeiramente luminosas,
atravessadas por uma alegria ofuscante, como se a sua alma fosse um
intenso foco de luz!

Presidente!\ldots{} presidente!\ldots{} Sim, ele será presidente! Quando passar
pelas ruas toda a gente o cumprimentará; e os ministros, fardados e
veneráveis, curvar"-se"-ão diante dele com o respeito devido a um
superior, nos grandes salões de um palácio onde ele habite. Terá carros
luxuosos, cavalos e criados\ldots{} À sua voz abrir"-se"-ão as prisões, os
hospitais, os asilos, todos os edifícios onde a desgraça more! Clemente,
consolará os tristes, levando"-lhes no seu conselho ou no seu perdão a
esperança e a ventura! As mães atirarão flores a seus pés; os moços
saudá"-lo"-ão alegremente e as crianças cantarão hinos agradecendo a sua
proteção, o seu amparo, a sua simpatia. A todos os recantos escuros
descerá o seu olhar luminoso! para cada chaga terá um bálsamo, para cada
mágoa um consolo, para cada vício uma reabilitação! A cadeira de veludo
que lhe destinarem em todos os lugares em que tenha de comparecer, quer
seja um lugar de festa, quer seja um lugar de dor, será sempre cercada
de flores, atiradas aí pela multidão compacta do povo, que o proclamará,
unanimemente, o melhor dos homens!

Sim! meu filho será o melhor dos homens! Triunfante, poderoso, altivo,
belo, adorado, há de levar"-me pelo braço, a mim, velha, cansada,
trêmula, e dirá à vista de toda a gente, sem se envergonhar da minha
figura nem da minha ignorância:

``Esta é minha mãe!''

A costura do Arsenal caíra no chão; a visionária mulher tinha lágrimas
nas faces, lágrimas de júbilo que aqueles pensamentos lhe davam.
Nervosa, histérica, doente, deixara"-se embalar de tal maneira pelas
douradas quimeras daquele sonho irrealizável, que o julgava já
exequível, certo.

Voava pelo azul de sua fantasia, quando ouviu na escada os passos
irregulares do marido, batendo nos degraus com a sua perna de pau.
Correu a abrir a porta.

O homem entrou carrancudo, confessando logo, à queima"-roupa, estar sem
emprego\ldots{} Implicâncias e queixas de um fiscal\ldots{} guardava as
explicações para o outro dia; estava cansado. Deitou"-se e adormeceu.

A esposa, arrefecida, gelada por semelhante notícia, voltou para a
costura; duplicaria o seu esforço, faria serão até mais tarde, talvez
toda a noite\ldots{}

No entanto o relógio cansado ia batendo, uma\ldots{} duas\ldots{} três\ldots{}
quatro\ldots{} até a décima segunda pancada da meia"-noite; e no bercinho de
vime dormia regalado o \emph{futuro presidente}, com a cabeça enterrada
na almofada e as mãozinhas papudas espalmadas sobre a colcha de cor.

(1889)

\part{Novelas}

\chapter[O laço azul]{O laço azul\subtitulo{(novela romântica)}}



\section*{\textsc{i}}

Mal o dr. Sérgio Bastos tocou a campainha de um portão à rua Haddock
Lobo, viu através das grades aparecer a chapa iluminada de um avental
branco.

--- O Sr. Isidoro Nunes\ldots{} está?

--- Sim, senhor\ldots{} A quem devo anunciar?

O dr. Sérgio tirou vagarosamente da algibeira o seu cartão, entregou"-o à
criada e caminhou atrás dela pelo jardinzinho, onde floria uma ixora
rubra.

Mal diria ele que, no fim de quarenta anos de apartamento, teria de
procurar o Isidoro para um caso tão particular e tão melindroso\ldots{}
Considerou com olhar experiente e disfarçado o exterior da casa, que não
lhe pareceu má, reconstruída há pouco, com as suas cinco janelas bem
rasgadas sobre uma varanda ladrilhada, guarnecida de trepadeiras.

A criada acomodou"-o na sala e sumiu"-se. Ele ficou a um canto do sofá,
com a cartola nova entre os dedos cuidadosos, os pés bem calçados
reluzindo sobre o tapete, o queixo enterrado entre as dobras do
colarinho. Meditava:

Que terão feito estes quarenta anos ao Isidoro?

Lembrava"-se mais do seu caráter que da sua figura. As feições
escapavam"-se"-lhe da memória, talvez as confundisse com as de outros\ldots{}
agora dos seus hábitos de docilidade e ao mesmo tempo de curiosidade
miudinha, é que não.

Se bem se lembrava, chamavam"-no no colégio de o Sabe"-tudo --- pelo
motivo de que o Isidoro estava sempre a par de todos os acontecimentos
da casa, desde os mais importantes até aos mínimos. Tal professor fora
censurado discretamente, no escritório fechado? Ele sabia"-o. O diretor
hipotecara o prédio para acudir a despesas imprevistas? Sabia"-o
igualmente; como sabia que tal criado fora substituído\ldots{} tal menino de
Campos recebera tantas latas de goiabada enviadas pela mãe, como outro
de Pernambuco tantas de caju cristalizado, mandadas pela tia. Dir"-se"-ia
que os fatos chegavam ao seu conhecimento, naturalmente, sem que ele os
tivesse procurado conhecer por mexericos e confidências. Para castigá"-lo
daquela mania de bisbilhotice e para ter comodidades à custa alheia,
quantas vezes ele, que ali estava agora confuso e impressionado naquele
canto de sofá, fizera do Isidoro, já mocinho, já buçando, gato"-sapato
nas diabruras do colégio! Pusera"-lhe então à prova a docilidade.

Era o pobre do Isidoro quem o livrava dos encargos aborrecidos de fazer
a cama, escovar a roupa, lavar os pentes, engraxar as\ldots{}

Uma porta rangeu. O Dr. Sérgio levantou"-se, dois braços magros, cobertos
de brim claro, abriram"-se diante dele.

Os quarenta anos tinham feito alguma coisa em ambos: saudades. Talvez
não fosse positivamente saudades um do outro, mas do tempo da vida em
comum no pensionato. O abraço foi longo e mudo. Depois os dois homens
sentaram"-se, observando"-se ainda com um sorriso nos olhos úmidos.

Dir"-se"-ia que cada um deles encarnava a mocidade do outro, e eram já
ambos encanecidos; dr. Sérgio, com suicinhas brancas, sem bigode,
Isidoro Nunes de bigode, branco também.

--- Que surpresa!

Dr. Sérgio perguntou então ao amigo como se pudera lembrar dele, depois
de tamanha ausência\ldots{}

--- Homem, se eu lhe fizesse essa pergunta, vá!\ldots{} mas você, que fez um
nome brilhante, que se tem posto em evidência!\ldots{} Assim, sei em que ano
se formou em S. Paulo\ldots{} em que ano casou com uma senhora da família
Bernardes, de Vassouras, fazendeira e muito formosa, de quem é viúvo\ldots{}
sei que se entregou à lavoura de corpo e alma e que tem escrito obras
importantes sobre agricultura\ldots{} O que eu não sabia, nem podia prever, é
que se lembrasse de mim e viesse um dia bater à minha porta!

Decididamente é o mesmo homem! pensou o dr. Sérgio; e logo respondeu,
com um sorrisinho fino:

--- Pois eu também não me esqueci, e a prova é que aqui estou, um pouco
pasmado de que o amigo saiba de mim mais do que eu quanto às minhas
obras, que, longe de serem importantes, são uns pequenos ensaios de
pomologia\ldots{} O mais está certo. Acrescentarei que tenho um filho de
vinte e três anos, oficial de marinha, forte, inteligente, bonitão, é
que nele concentro toda a minha felicidade.

--- Também sabia\ldots{}

--- Ah!\ldots{}

--- Sei tudo. O seu rapaz é muito considerado, muito distinto. Vou
chamar minha mulher. Ela vai ficar contente, porque já o conhece através
da minha amizade. Fui sempre muito seu apreciador, você era um rapaz
encantador, às vezes aborrecia"-me\ldots{} abusava\ldots{} mas, enfim, rapaziadas!

Riram"-se ambos.

Sim, rapaziadas! também a ele, os outros o que faziam!

--- Não: a você não! Toda a gente o respeitava, até os grandes!

Ele é que fora a vítima\ldots{} só uma manhã engraxara dez pares de botinas!
Os rapazes são perversos. Pois não guardava rancor por nenhum. Até se
ria\ldots{} e de mais de um vingara"-se depois\ldots{} sabia como? dando"-lhes
dinheiro para botas, quando os encontrava na rua com os dedos de fora.

--- Sim, alguns estão na miséria\ldots{}

--- Outros no galarim\ldots{}

--- O Barbosa é ministro.

--- E o Castro, lembra"-se, aquele a quem chamavam \emph{Zebrinha}? ---
bebe como uma esponja e pede dinheiro aos conhecidos, nas esquinas\ldots{}

--- Pois esse era rico\ldots{}

--- Talvez por isso. Homens querem"-se criados na necessidade dura. Como
eu\ldots{} como nós dois, que ambos comemos o pão que o diabo amassou, eu no
comércio, e você na sua banca de advogado, onde nos primeiros anos teve
bem grandes contrariedades.

Dr. Sérgio arregalou os olhos.

Isidoro continuou:

--- O que o salvou foi a intervenção do Roxo na questão com o Borlido e
aquele negócio do carvão. Você advogou admiravelmente bem aquela causa!
Bem vê que o não perdi de vista\ldots{}

--- Realmente!

--- A minha memória ainda regula\ldots{} Bem, vou chamar agora minha mulher.
Você talvez tivesse conhecido a família do meu sogro: Ele era filho do
barão de Filgueiras, não se lembra?\ldots{}

--- Não me recordo\ldots{}

--- O corretor Filgueiras, muito conhecido na praça . . . um ruivo,
gorducho!\ldots{}

--- Não tenho ideia\ldots{}

--- Bom homem\ldots{} Um momento. Vou chamas minha mulher\ldots{}

D. Angélica parecia ter ouvido de trás da porta o seu nome; apareceu
logo. Era uma senhora alta, gorda, já grisalha, com óculos de ouro e
vestido de ramagens sobre fundo escuro. Toda a sua doçura estava no
sorriso bom, acolhedor, e na pele muito branca, ainda fina.

O dono da casa foi abundante de adjetivos na apresentação da sua querida
companheira de trabalhos e de alegrias e também do caro amigo, de
infância, a quem reservava uma surpresa\ldots{} Dr. Sérgio estranhou lá
consigo, falou ainda um bocado de reminiscências e teve por fim de
explicar o motivo da visita. Pediu então segredo.

Isidoro levantou"-se e foi correr o reposteiro da esquerda. D. Angélica
fez o mesmo ao da direita e voltaram a sentar"-se, ardendo em
curiosidade, quando uma voz moça, muito clara, cantarolou na sala
próxima um trecho da \emph{Cavalleria Rusticana}.

--- Bonita voz! observou o advogado.

--- Há de ser a Lucila . . . afirmou, sorrindo orgulhosamente, Isidoro.

--- Talvez seja a Madalena\ldots{} retrucou d. Angélica.

Estiveram um momento à escuta. A voz calou"-se. Dr. Sérgio começou:

--- Minha senhora, como aqui o meu bom amigo está informado, eu sou
viúvo, tenho alguns bens de fortuna, boa disposição para aumentá"-los,
melhor saúde e um filho bem encaminhado, rijo e tido por um modelo entre
os rapazes. Vivo no campo e, tanto quanto eu agora me destinei à terra,
o meu Raul se dedicou ao mar. O ideal é outro, mas a pertinácia é a
mesma. Guarda"-marinha, deve partir com a sua turma, de hoje a vinte
dias, em viagem de instrução.

Estava combinado que meu filho fosse despedir"-se de mim à fazenda,
quando há dias recebi dele este telegrama\ldots{}

Dr. Sérgio desdobrou com o seu gesto pachorrento um papelinho e
ofereceu"-o a d. Angélica. Ela leu alto:

``É indispensável a sua presença aqui. Venha já. --- Raul.''

O telegrama não explicava coisa nenhuma; D. Angélica entregou"-o aberto
ao advogado, com ar meio estúpido e interrogativo.

Ele sorriu. Iam saber.

--- Quando recebi este telegrama, imaginei tudo, menos a verdade. Supus
o meu rapaz em véspera de alguma operação grave, que requisitasse a
minha presença; ou envolvido num desses dramas modernos de amores
complicados, ou arriscado a algum duelo\ldots{} Enfim, precipitei"-me para o
Rio, formulando as hipóteses mais disparatadas e mais perturbadoras. Ao
abraçá"-lo, pela primeira vez, em minha vida, depois que ele é homem, o
meu filho, chorou. Bambo, apatetado, eu nem ousava interrogá"-lo\ldots{} até
que ele se declarou. Estava apaixonado!

D. Angélica suspirou de alívio. Isidoro interrogou com interesse:

--- Paixão pura, paixão\ldots{}?

--- Perfeitamente lícita. Meu filho é o protótipo da honestidade; não
confessaria uma paixão reprovável\ldots{} Como ente humano, nunca o julguei
isento das maiores dores humanas, mas dessa quis Deus poupá"-lo, e eu
dou"-lhe graças. O amor de meu filho é por uma moça solteira, a quem ele
deseja ligar o seu destino. E embora parta de hoje a vinte dias, quer
embarcar deixando aqui não a sua noiva, mas a sua mulher. Há casos,
vê"-se, em que é melhor a certeza que a esperança! Minha senhora, quando
meu filho me disse o nome do pai dessa moça, meu coração palpitou com
mais força.

``Era o de um velho amigo! E foi por isso que ousei vir, desacompanhado,
bater à sua porta. Consente em que sua filha case com meu filho?

D. Angélica tornara"-se lívida e muda.

Isidoro torceu nervosamente a ponta do bigode interrogou com voz
engasgada:

--- Qual delas?

--- Qual delas! Chegou a vez do dr. Sérgio ficar interdito. E depois:

--- Julguei haver só uma\ldots{} Pelo menos, meu filho só conhece uma!

--- Talvez conheça as duas\ldots{}

--- Ora essa! Ele supõe até que seja filha única! lembro"-me que me
disse, e até com estes sinais: altura regular, delgada e loura\ldots{}

--- É exato\ldots{} mas isso ainda não nos esclarece suficientemente. Meu
caro amigo, há uma grave complicação na minha família; um caso
irremediável, e cujas consequências não podemos prever\ldots{}

--- Como?\ldots{} murmurou o dr. Sérgio apatetado.

--- Seu filho pensa amar uma das minhas filhas, mas talvez ame as duas.
São iguais. Tão iguais que os olhos da própria mãe às vezes as
confundem\ldots{}

--- É das boas mães confundir os filhos, observou o advogado. A minha
sempre me dava a mim o nome de meu irmão e a ele o meu!

D. Angélica voltou à vida, de que parecia suspensa, por uma sacudidela
nervosa e observou:

--- Não é só o meu amor que é igual por ambas, elas também o são entre
si de corpo e de alma\ldots{} O que uma pensa a outra pensa, o que uma
deseja, já a outra quer! Uma dessas ideias más, ou enfadonhas, que nos
indispõem muitas vezes com nós mesmos, fá"-las por vezes aborrecerem"-se e
fugir cada qual para o seu lado, como para se livrarem de si próprias.
De tal modo uma é o espelho da outra que os seus gestos são os mesmos e
o mesmo o seu modo de vestir. Há singularidades notáveis nas suas
aptidões e preferências: gostam das mesmas frutas, odeiam da mesma forma
o amarelo e o violeta, bordam com igual perfeição, a letra é
absolutamente semelhante, e o seu modo de exprimir o pensamento é
idêntico, o que faz com que uma pareça o fonógrafo da outra! Desde
pequeninas que são assim; ainda estavam no berço e eu já me via tonta
para satisfazer a ambas no mesmo instante\ldots{} Desde esse tempo que eu
pensava com susto nesta hora terrível, que havia de chegar\ldots{} e que
chegou!

D. Angélica não pôde reter as lágrimas. O marido acudiu:

--- Infelizmente a vida não se passa sempre nos berços\ldots{} Resigna"-te\ldots{}
O essencial é saber qual delas deve abandonar primeiro o ninho. O noivo
é sério, é trabalhador, de boa família, chegou a ocasião de se ir uma
embora. Que vá. Eu não as quero para freiras. É a vida. A mãe tem medo
de mata"-las, separando"-as. Meteram"-lhe isso em cabeça\ldots{} realmente são
tão unidas\ldots{} há de custar"-nos a todos\ldots{} mas enfim, ainda este
casamento proporciona a vantagem dela nos ficar em casa, depois de
casada, como se o não fora\ldots{} Durante esse tempo, um ano, talvez? . . .
ir"-se"-ão acostumando à ideia da separação. Por mim estou contente e
aprovo.

D. Angélica assoava"-se, limpava as lágrimas, desafogava"-se.

O marido continuou, voltando"-se para o amigo:

--- Deste fenômeno não deixa minha mulher de ter, talvez, certa culpa,
coisa que só a ciência poderia determinar, se soubesse responder a tudo
quanto se lhe pergunta. Imagine que a Angélica é de uma família em que
vários membros têm filhos gêmeos, e que, mal se casou, pôs"-se a desejar
que o seu primeiro parto fosse de gêmeos e que esses gêmeos fossem
meninas, e loiras e tão parecidas que se confundissem entre si! Ora,
todos os outros gêmeos da família são casais e são trigueiros. Não teria
a sua vontade tido influência neste caso fisiológico?

O dr. Sérgio abanou a cabeça, para exprimir a sua ignorância. E depois,
para fugir do assunto:

--- O que não posso compreender é como meu filho ignorasse isso, e não
tivesse nunca visto senão uma das meninas!

--- Eu explico: antigamente, quando saíamos com as duas, eram tantos os
comentários, as olhadelas, as observações, que nós, vexados, adotamos o
sistema de sair só com uma. Quando vai a Lucila, fica a Madalena; quando
vai a Madalena, fica a Lucila\ldots{} Seu filho talvez conheça as duas,
julgando conhecer uma. A minha esperança é que ele possa determinar,
pelo lugar em que a viu pela primeira vez, qual das duas terá de ser sua
mulher\ldots{} caso não haja já correspondência. Isto de moças pregam"-nos às
vezes cada peça . . . por maior que seja a vigilância\ldots{}

Sou difícil de enganar: sei tudo\ldots{} todavia não estarão as coisas mais
adiantadas do que supomos?

--- Assim estivessem; mas não estão. Meu filho, que é tão arrojado em
tudo mais, é um tímido em questões de amor. Disse"-me que não se atreveu
a declarar"-se e nem se aventuraria a pedir a menina, por quem morre de
amores, se não tivesse de partir agora. Assim, disse"-me ele, se me derem
um --- não, ninguém mais me porá aqui a vista em cima; se me derem o ---
sim, partirei feliz! Parece que houve troca de sorrisos\ldots{} olhares\ldots{}
não passou disso. Um nada para uns, a existência para outros\ldots{} Meu
filho é dos últimos. Extremamente sensível\ldots{}

D. Angélica continuava sucumbida, olhando para o vazio da sala, como se
quisesse sondar o futuro.

Dr. Sérgio mudara também de aspecto.

Cravara"-se"-lhe uma ruga funda entre as sobrancelhas. Ouvir"-se"-ia voar
uma mosca. De repente, Isidoro cortou decididamente o embaraço:

--- Traga cá amanhã à noite o seu rapaz: previna"-o primeiro como
entender, da extravagância da situação. Entretanto nada direi às minhas
filhas, para as não sobressaltar nem fazê"-las passar por um vexame, caso
seu filho desista\ldots{} Se ele não desistir, nem souber determinar
positivamente o dia em que viu a menina pela primeira vez, cada uma das
pequenas virá por sua vez à sala e a sua comoção decidirá da sua
sorte\ldots{} A surpresa que lhe anunciei era exatamente a apresentação
delas. Já agora fica tudo adiado para amanhã. Ainda mais uma explicação:
se seu filho desistir, o amigo virá sozinho e o segredo ficará entre
nós. Que tudo se decida, sem hesitações nem demoras.

D. Angélica teve um olhar súplice. Isidoro respondeu"-lhe, com expressão
penetrante:

--- Até lá\ldots{} silêncio!

\section*{\textsc{ii}}

--- Digo"-te que estás apaixonado por duas mulheres e que, na
impossibilidade de casar com ambas, é mais acertado não casar com
nenhuma\ldots{} Não me olhes assim, que não estou doido; é a verdade. Também
a mim nunca a verdade me pareceu tão absurda nem tão estorvadora! A
personalidade da tua amada desdobra"-se noutra completamente igual! Ora,
como para o amor a mulher precisa ser única --- estas não te servem, a
menos que não queiras criar embaraços futuros de muita gravidade.
Conforme me disseram os pais, e não é crível que sofram ambos desse
defeito de ótica que duplica as visões, uma é o espelho da outra, tanto
em perfeições de corpo como em defeitos ou qualidades de caráter\ldots{} Para
evitar comentários e olhadelas indiscretas, quando a família sai a
passeio uma das meninas fica em casa. Revezam"-se nos divertimentos.
Assim, conhecerás uma só?\ldots{} Conhecerás as duas?\ldots{} Não sabes?\ldots{} Nem
eu!

--- Sei! Sabe"-o ela também. O meu amor não a pode confundir! Mas que
notícia me trouxe! Como foi, como foi? Conte"-me tudo! E era para isto
que eu o esperava com tanta impaciência!

--- Devagar, descansa. Estás aceito como marido de uma delas, o que já
era previsto e não te deve surpreender; o caso agora é saber qual das
duas terás de escolher.

--- A que eu vi! A que eu amo!

--- Viste"-a uma vez só?

--- Várias vezes\ldots{}

--- Logo, ora uma, ora outra\ldots{}

--- Não. Bastará que os nossos olhos se encontrem para que tudo se
aclare, verá. As nossas almas já se compreenderam\ldots{}

--- As delas são iguais\ldots{}

--- Não creia; exageraram: mentiram. Não há duas folhas iguais numa
árvore, é absolutamente impossível haver duas pessoas assim numa
família. Desde que os pais consintam no casamento, não será por causa
disso que eu desista da minha felicidade!\ldots{}

--- Mesmo que sacrifiques a de outrem?

--- Não sacrificarei a de ninguém. Quem diz que a outra me ame? E se
assim fosse? Que me importaria; antes um sacrificado do que três! Eu amo
uma, sou amado de uma, o resto do mundo é"-me indiferente.

--- Agora.

--- Sempre. Toda a gente tem amores desencontrados, paixões a que não
corresponde. Ai de quem não é correspondido\ldots{} os sacrificados calam"-se;
os vitoriosos gozam. Eu sou vitorioso.

--- Escuta ainda a voz da prudência. Se tu pudesses levar tua noiva
contigo para bem longe da outra, ainda vá. Mas tens raízes aqui\ldots{} onde
ela tem as suas entrelaçadas às da irmã\ldots{} Evita o perigo. Para que és
homem?

--- Para tirar da vida tudo quanto eu queira que a vida me dê! Depois,
previno"-o, meu pai, de que serão baldados todos os esforços para me
separarem desta ideia. Lembre"-se de que fui sempre um obstinado e saiba
que o desejo de ser o marido daquela mulher me poria doido se o não
realizasse!

--- Quem nos diz que não desejes as duas? Vias ora uma, ora outra,
cuidando ver sempre a mesma\ldots{} Desta, guardas um sorriso que te
encantou, mas já da outra um olhar que te enterneceu\ldots{} casando com uma,
lamentarás não ter casado com a outra\ldots{}

--- Não.

--- Pensa bem; abre os ouvidos ao meu conselho. Lembra"-te de que sendo
tal a semelhança entre as duas meninas, tu acabarás depressa ou por te
enfadares da tua, ou por amares a ambas, o que ainda é mais
inconveniente\ldots{} Vale mais fugir, e já.

--- Nunca. A minha mudará. Quando eu voltar da minha viagem ela talvez
já tenha a mais o encanto da maternidade\ldots{} Mas para que perder
palavras? Conheço a que amo\ldots{} ela conhece"-me. Quero"-a. Acabou"-se\ldots{}

--- És um obstinado, como todos os namorados da tua idade, mas eu ainda
assim apelo para o teu critério de homem sensato, para pesares as
contrariedades prováveis, as situações embaraçosas que te reservará
semelhante enlace. Calada o enjoo, o tédio, que deverá causar a
convivência de uma criatura que vive ao lado de outra, em tudo sua
semelhante!\ldots{}

--- Quando elas sejam perfeitas!

--- Perfeita é a Bíblia, e que prazer proporcionaria uma biblioteca, em
que não houvesse senão exemplares dessa obra, na mesma edição!

--- O leitor afeiçoar"-se"-ia a um desses exemplares, um único, e não
faria caso dos outros\ldots{} Quando me apresentará na casa, logo?!

--- Amanhã.

--- Como tarda!

Chegou a hora da apresentação, tendo parecido as suas precedentes longas
ao filho, curtas ao pai. Pelo caminho, mais de uma vez o moço exclamou,
num desabafo feliz: --- Vou vê"-la! Ao que o velho corrigia, com um
fiozinho de ironia triste: --- Vais vê"-las\ldots{}

A sala estava iluminada, à espera.

D. Angélica e o marido, em trajes de cerimônia, e as meninas lá dentro,
com ordem de não virem à sala senão quando chamadas. O acolhimento ao
noivo foi constrangido; ele estava gelado, estúpido; foi necessário que
o pai falasse por ele, como no primeiro dia que o levou à escola.

Isidoro tinha o seu plano; fez sentar as visitas, procurou animá"-las com
meia dúzia de frases alheias ao assunto, até que achou jeito de
perguntar ao moço:

--- Quando conheceu minha filha?

--- Numa quermesse\ldots{} no Parque\ldots{} sentamo"-nos perto, ouvindo tocar os
ciganos\ldots{}

--- Foi a Madalena! exclamou Isidoro.

--- Estás enganado, foi Lucila, objetou d. Angélica.

--- Adeus, adeus!

O noivo, ardendo por decidir a dúvida, informou:

--- Ia toda de branco\ldots{}

--- Não é sinal. Andam sempre de branco, por motivo de uma promessa da
mãe.

Entreolharam"-se como a pedirem uns aos outros inspiração para resolver o
embaraço, quando o advogado interveio:

--- Parece"-me mais simples chamar a moça; não será já tempo de a
consultar?

--- É cedo\ldots{} respondeu Isidoro. Você não imagina a afinidade de
sentimentos que há naquelas criaturas\ldots{} Sejamos cautelosos e procuremos
tocar no ponto justo. Tenho o meu plano. A mãe vai lá dentro e
perguntará, como coisa sua, qual das meninas nos acompanhou ao Parque, a
ouvir os ciganos, na noite da quermesse\ldots{} Essa virá primeiramente à
sala. Assim a primeira impressão, e que foi naturalmente a mais forte
que o noivo recebeu, designará o nome da noiva.

--- Perdão, as minhas impressões eram cada vez\ldots{}

Isidoro fez um gesto, expressando, com a maior eloquência, a necessidade
do noivo interromper o desabafo; e já d. Angélica desaparecia pelo
corredor, para desempenho da sua comissão.

--- Muitas vezes o palco me tem dado a ilusão da vida; pois afirmo que é
a primeira vez que a vida me dá a impressão do palco\ldots{} comentou o dr.
Sérgio, coçando as barbinhas brancas. Só me falta ouvir o meu filho
cantar neste \emph{intermezzo} uma arieta de Planquette\ldots{} o mais está
em termos\ldots{} e olha, Raul, que estás hoje com cara de tenor!\ldots{}

Isidoro achou intempestivo o gracejo. O filho também não gostou. Suava
frio. Os minutos pareceram"-lhe longos, até que viu d. Angélica voltar e
dizer da porta com ar constrangido:

--- Foi a Lucila.

Aquele nome cortou o silêncio, como um raio de luz a escuridão.

--- Nesse caso é a Lucila a noiva. Veem, como foi simples? O senhor
viu"-a na quermesse\ldots{} está bem certo que foi lá que a viu pela primeira
vez?

--- Oh, senhor, sim!\ldots{} e depois vi"-a na\ldots{}

--- Não diga mais nada! Foi na quermesse: é a Lucila. Chama"-a, Angélica,
tem paciência.

--- Por que não hão de vir as duas? Aventurou o dr. Sérgio.

D. Angélica opôs"-se. Seria bom retardar à Madalena esse desgosto\ldots{}
Viviam tão unidas\ldots{}

--- Ora que tolice! Vocês são de uma sentimentalidade piegas. Ela há de
saber por força! Mais um minuto ou menos um minuto, que importa?
observou"-lhe o marido.

--- É mais um minuto de felicidade!

--- Pois chama lá só a Lucila\ldots{} Olhem que não conheço nada menos
prático do que o espírito das mulheres. Esta, seria capaz de passar uma
semana sem comer, só para que não faltasse nem uma migalha nas
costumadas guloseimas das filhas. Diz ela que o supérfluo, desde que dê
prazer, é necessário! E sacrificar"-se"-ia de boamente, para que não
faltassem às pequenas esses supérfluos necessários!

Raul não ouvia nada; tinha os olhos fixos na porta por onde havia de
aparecer a noiva. A seu lado os velhos conversavam! e o som das suas
vozes empastava"-se na mesma toada confusa e sem sentido. Ela tardava, e
ele impacientava"-se por vê"-la de perto, ouvir"-lhe a voz, tocar"-lhe na
mão.

De repente, um vestido branco iluminou a sala; ele levantou"-se
deslumbrado; Lucila estava na sua frente.

Dr. Sérgio, que observava a moça, notou"-lhe um estremecimento de
surpresa alegre ao deparar com o oficial de marinha. É ela e ama"-o,
pensou o velho, satisfeito.

Isidoro falou:

--- Minha filha, o sr. dr. Sérgio Bastos, meu velho amigo de infância,
vem pedir para seu filho a tua mão. Nada posso responder sem conhecer
teus sentimentos. Dirás se queres ou não ser sua esposa\ldots{}

Lucila corou, voltou"-se para a mãe, lançou"-lhe os braços ao pescoço,
sumiu o rosto nas rendas da sua blusa, e foi afogada no seio materno que
suspirou o \emph{sim}.

Estava linda. Rejubilaram"-se todos.

Só d. Angélica empalidecera. O pai puxou Lucila pela mão, aproximou"-a do
noivo, e para assegurar"-se bem:

--- Lembras"-te da primeira vez que viste este senhor?

--- Sim\ldots{} foi no Parque\ldots{} tocavam os ciganos; estava uma noite linda!

--- É isso mesmo. Concorda! E da segunda, minha filha?

--- Foi no Passeio Público\ldots{}

--- Perdão, foi nas regatas\ldots{} Lembra"-se? eu estava perto do pavilhão, e
a senhora deixou cair uma rosa\ldots{} tenho"-a aqui, como testemunho\ldots{}

--- Eu não fui às regatas; foi minha irmã\ldots{}

Isidoro achou prudente tossir, fazer barulho, desviar o curso das
reminiscências. Começou a falar do futuro. Urgia marcar o dia do
casamento. O noivo partiria dentro de poucos dias para uma viagem de
oito ou nove meses.

Uma atrapalhação! E agora era tratar do enxoval, dos papéis, e decidir
se o casamento deveria ser à capucha ou de que forma, e que fossem
chamar a Madalena para a comunicação da novidade!

Lucila precipitou"-se para o corredor; a mão da mãe reteve"-a:

--- Vai para a saleta do piano preparar as músicas. Cantarás para o teu
noivo ouvir.

Lucila obedeceu com tristeza, compreendendo a intenção da pobre senhora.
Ela não queria confrontar as duas filhas em situações diferentes.

Criadas pelo mesmo beijo, destinava"-as ao mesmo destino; no fundo da sua
alma religiosa latejara sempre a esperança de ver as duas, esposas do
mesmo esposo, na paz suave de um só convento\ldots{}

Ainda a Lucila estava na sala e já Isidoro gritava pela Madalena.

--- Madalena! Madalena!

Mal uma se sumia quando a outra apareceu: o mesmo clarão do vestido
branco, o mesmo fulgor do cabelo loiro, o mesmo sorriso escarlate, o
mesmo gesto de surpresa alegre ao deparar com o guarda"-marinha na sala.

Dr. Sérgio coçou as suíças nervosamente. Raul sentiu um baque no
coração, Isidoro mudou de aspecto e de voz:

--- Minha filha, o dr. Sérgio Bastos, meu amigo de infância, acabou de
pedir para seu filho Raul, a mão de tua irmã\ldots{}

Madalena empalideceu, os olhos encheram"-se"-lhe de água, mas ficou
imóvel.

O passo difícil fora transposto.

Isidoro, mais à vontade, tirou do escaninho de um porta"-bibelô um laço
de fita azul clara, e determinou:

--- Como é difícil ao sr. Raul, nos primeiros dias da sua convivência,
reconhecer logo à primeira vista qual das minhas duas filhas é a sua
noiva, pensei de antemão em pôr este distintivo numa delas. Têm ambas de
andar sempre de branco, seja; mas de hoje em diante Madalena adicionará
à sua toalete este laço azul\ldots{} Prega"-lho no peito, Angélica!

Quando a mãe encostou a mão ao peito da filha, sentiu"-lhe o coração
bater com força; entretanto, ela permanecia como uma estátua. Raul
fechara os olhos: também o seu coração pulsava com violência, sob a
pálida rosa murcha do dia das regatas\ldots{}

\section*{\textsc{iii}}

Uma\ldots{} duas\ldots{} três\ldots{}

Madalena ouviu aterrada as pancadas do relógio da sala de jantar. Deram
as sete; aproximava"-se a hora da visita de Raul.

A noiva já o esperava, radiante, dedilhando no piano uma valsa ligeira,
que interrompia de vez em quando, para interrogar com o ouvido o
silêncio do jardim.

Muito pálida, em pé, diante do espelho do quarto, Madalena procurava
prender no peito, bem sobre o coração, o laço azul, espalmado como uma
borboleta traspassada pelo alfinete assassino. Mas, dir"-se"-ia que os
seus dedos hábeis tinham perdido o jeito. Sentia as mãos hirtas como se
as tivesse mergulhado em neve pura, e a valsa da irmã, que esvoaçava,
pela casa como um ruflo de asas de andorinhas, penetrava"-lhe na alma
como um barulho irritante de ferros que se raspam, arrepiando"-a toda. Há
um único som grato aos ouvidos tristes: o do soluço. A dor é egoísta e
geradora da inveja.

Madalena sentia"-se só, no meio da família em festa. Todos riam. As
amigas acudiam em bando, ofereciam"-se para ajudar a compor as peças do
enxoval. A notícia voara de uma ponta da rua à outra; até lhe parecia
impossível que aquela gente toda tivesse realmente o prazer que
manifestava.

E ela não chorava; e ela assistia a todas as conversas, risadinhas,
pancadinhas no ombro, abraços, denguices, como uma estrangeira,
impassível. Oh! mas dentro dela, recalcada no coração, que revolta
contra esse Destino implacável e invencível, que a fizera em tudo igual
à irmã, para no fim atirar à outra um punhado de rosas e a ela um feixe
de espinhos! Por quê? Em que ofendera ela a Deus? Em quê?!

Com os lábios trêmulos, as faces lívidas, Madalena olhou com rancor para
a sua imagem, refletida limpidamente no cristal iluminado. Não era ela,
era a irmã, que ali estava; os seus cabelos, de ouro, rutilantes, a sua
tez de camélia, o azul ferrete dos seus olhos rasgados, o oval delicado
do seu rosto, todas as belezas de seu corpo eram uma cópia, uma
reprodução, desprezadas por inúteis, pelo único homem a quem ela amaria
na terra. Lucila tinha de vida mais uma hora do que ela, e essa hora
inconsciente dera"-lhe talvez a preferência do pai, que a impelia para a
felicidade. Afinal, se fosse Raul que escolhesse, quem sabe se não se
decidiria por ela? Sete horas: ele não tardaria, e o maldito laço azul
que a distanciava do amor não conseguia fixar"-se no seu peito
atormentado. Com um movimento de raiva Madalena atirou"-o ao chão, sem
reparar que a mãe entrava e olhava para ela, compassivamente.

--- Que é isso, Madalena!\ldots{} Tem paciência\ldots{}

A moça retraiu"-se. Depois, sacudindo"-se, numa resolução desesperada,
desabafou:

--- É que eu não posso mais, não posso mais!

--- Então, filhinha\ldots{}

--- Deviam ter"-me matado quando eu nasci\ldots{}

--- Estás louca!

--- Não viram logo que no mundo não há lugar para duas pessoas iguais?!

--- O mundo é tamanho!

--- É tamanho, que cabe todo dentro do coração de quem ama!\ldots{}

--- Palavras\ldots{} Coragem meu amor, e põe o teu laço\ldots{} Vamos\ldots{}

--- Não. Não o porei nunca mais; nunca mais! Prefiro desfear"-me;
arrancar os dentes, cortar os cabelos, queimar o rosto a vitríolo, tudo,
tudo, tudo, menos distinguir"-me de Lucila por esse pedaço de trapo . . .
Quero ser eu, estou cansada de ser --- \emph{nós}, --- quero ser eu, só
eu, embora feia, torta, aleijada, mas sozinha, eu única! Oh, que tédio,
que ódio, por que não me estrangularam, por que não perceberam que
depois de Lucila, eu era demais?!

D. Angélica suspirou, aterrada:

--- Meu amor, enlouqueceste!

--- Às vezes afigura"-se"-me que nem sou gente, sou um reflexo; como
aquilo!

E apontou para o espelho, onde o seu vulto se reproduzia.

--- Sou demais!

--- Que ideia, Madalena\ldots{} E eu?

--- A senhora é a culpada. Ter duas filhas assim!

--- Como podia eu evitar?\ldots{} bastante tenho sofrido; sempre com este
medo\ldots{} Lembra"-te que procurei inocular em' cada uma de vocês gostos
diversos\ldots{}

--- Em vão.

--- Sim, em vão. A natureza era mais forte do que eu\ldots{}

--- Maldita, mil vezes maldita, a natureza!

--- Não blasfemes!

--- Quer que eu a louve, quando ela me martiriza?!

--- Exageras o teu sofrimento, filhinha\ldots{} o tempo passará sobre esses
amores e virão outros para teu consolo! Resigna"-te.

--- Outros que, viessem seriam compartilhados por Lucila!

--- Ela será casada., terá outros cuidados\ldots{} filhos\ldots{} Serás sozinha!

--- Que importa? A alma é livre\ldots{} o coração é livre\ldots{} ela amará quem
eu amar, odiará quem eu aborrecer.

--- Tua irmã é honesta!

--- O amor não se importa com isso!

--- Madalena!

--- Não sei o que digo, perdoe"-me!

D. Angélica acariciou a filha lentamente, sentindo"-a rígida entre os
seus braços.

Indagou depois, com voz apenas sussurrada:

--- Não sabias do amor de Lucila pelo Raul, nem ela do teu?

--- Não.

--- Nem a mais pequenina confidência?

--- Nem.

--- Por quê?\ldots{}

--- Eu tinha medo de lhe despertar a curiosidade. Não queria que ela
visse Raul, para que o não amasse. Ele era meu!

--- E ela, coitadinha, já o amava!

--- E não me tinha dito nada\ldots{}

--- Teve naturalmente o mesmo receio que tu\ldots{} acomoda o teu espírito.
Promete"-me ser forte e não fazer a tolice que disseste há pouco\ldots{} Seria
um pecado, além de ser um enorme desgosto para mim\ldots{} Afinal mereço
alguma coisa\ldots{} Estás trêmula. Não convém que te vejam assim. A tua
dignidade impõe"-te o disfarce. Finge calma. Para chorar, aqui tens o meu
peito, que não sabe consolar"-te, mas sabe acolher"-te. Em todo caso, que
mais ninguém o saiba, pelo amor de Deus!

D. Angélica foi ao toucador, embeber a ponta da toalha em água,
refrescou com ela o rosto da filha, enxugou"-o de leve, aveludou"-o de pó
de arroz, e com uma voz em que Madalena sentiu um esforço poderoso, bem
subjugado, aconselhou:

--- Se eu fosse a ti, punha no peito o laço azul e iria para a sala
tocar com tua irmã\ldots{}

--- Falei a uma pedra!

--- Falaste a uma mulher que nunca deixou entrever as suas decepções\ldots{}
O sacrifício, acredita, não nasceu só para uma criatura\ldots{} Estás
desvairada, entra em ti; verás que só quero o teu bem. Ouves? teu pai lá
está gritando por mim\ldots{} não me deixa sossegada\ldots{} Coragem\ldots{} Vá!
promete"-me juízo\ldots{} Por meu gosto não saía de teu lado\ldots{} Que pretexto
darás para não usares\ldots{} o\ldots{} Já vou, homem de Deus, já vou!\ldots{} o laço
azul?

--- Madalena estendeu para a mãe a mão esguia, muito alva, num gesto de
quem pede uma esmola. D. Angélica pousou nela o laço de fita,
delicadamente, como se fora uma flor.

A voz do marido aproximava"-se, impaciente\ldots{}

--- Angélica, Angélica!

--- Cuidado, que teu pai não perceba! Sussurrou ela à filha; depois,
limpando as lágrimas que lhe corriam abundantes dos olhos, saiu,
respondendo num falsete desafinado:

--- Estou indo!

Madalena conservou"-se por um momento imóvel, como assombrada; até que
num movimento vagaroso, quase automático, prendeu ao peito o laço que
havia pouco repudiara\ldots{}

A mãe dissera: --- finge. Ela fingiria. E durante toda essa noite não
dormiu, revendo o olhar com que Raul a fixara por vezes nesse serão, em
que se não falara senão do seu noivado com a irmã. Com as mãos enlaçadas
às mãos de Lucila, ele cravara em Madalena as pupilas abrasadas, numa
interrogação incompreensível. À hora do chá os seus dedos
encontraram"-se, por acaso, puxando a mesma cadeira, e a mesma comoção no
retraírem"-se depois, aproximou as duas almas que se fugiam\ldots{} Madalena
revia todas as atitudes, todos os olhares do noivo da irmã que, na cama
fronteira, dormia o sono feliz dos que não têm medo da vida, porque
estão certos da felicidade\ldots{} Madalena fazia"-se pequena na cama, feliz
com a ideia de que Raul a amava tanto como à outra, e horrorizada ao
mesmo tempo com o pensamento do que pudesse acontecer mais tarde, quando
a irmã despertasse daquele sonho ou daquela ilusão\ldots{} Cabia"-lhe a ela
prolongar a ventura de Lucila. A mãe teria de ser desobedecida, porque
para isso só havia um meio: anular a semelhança que existia entre ambas.
Como? Matar"-se? Não. Ela queria gozar o suplício da sua abnegação,
queria dominar a natureza, dar àquele homem o sacrifício da sua beleza e
da sua mocidade. Ela falara à mãe em vitríolo, e, verdadeiramente, nem
sabia o que isso era. Lembrava"-se agora de haver no armário dos
remédios, de que o pai tinha a chave, um frasco de ácido nítrico,
comprado para limpeza de metais.

O copeiro fizera esse serviço de luvas, para que o líquido corrosivo não
lhe queimasse as mãos\ldots{}

A ideia de desfigurar"-se, derramando"-o sobre as faces, fê"-la arrepiar"-se
toda de medo, até bater os dentes, como num acesso de febre. Procurar
desfigurar"-se não seria ura crime? Teria ela coragem para levantar mão
sacrílega contra a sua beleza? Mas se essa beleza revoltava a sua alma,
e era também causa de perigo para a tranquilidade da irmã? Feia, Raul
afastaria dela a vista, num terror instintivo; linda, o seu olhar
abrasado procurá"-la"-ia como à irmã e ela talvez não tivesse forças para
resistir.

Uma cicatriz seria a salvação.

Não serviria essa intenção para redimir"-lhe a culpa perante o juízo de
Deus?

Com os olhos arregalados para a treva do quarto, Madalena passeava as
mãos piedosamente sobre o rosto acetinado, como a despedir"-se da sua
formosura. Uma grande mágoa, indefinível, torturante, entumecia"-lhe o
coração, que lhe pesava dentro do peito. Refletiu: tudo se faria como
por acidente. O vidro, explicaria ela aos outros, teria caído sobre o
seu rosto a um impulso dado por ela ao armário e o seu segredo ficaria
assim só entre ela e Raul, a quem os seus olhos diriam tudo! Antes porem
que tal horror se consumasse, só uma coisa ela pedia ao destino ingrato,
só uma ilusão que lhe deixasse da sua beleza um traço divino: um beijo,
um beijo de amor\ldots{}

Escondendo o rosto nas dobras do lençol, Madalena abafou os soluços até
a madrugada em que, exausta, adormeceu ainda soluçante. De manhã, a irmã
foi acordá"-la à cama:

--- São horas, preguiçosa!

Sim, uma grande preguiça. Até parecia doença. Mas não era. Iam ver que
passaria o dia magnificamente\ldots{}

Desde o pedido de casamento, as duas irmãs evitavam"-se mutuamente. Uma
porque estava contente; a outra porque estava triste; ambas porque se
sentiam humilhadas. Lucila por não ter forças para renunciar à
felicidade que não podia ser compartilhada pela irmã; Madalena por se
sentir ofendida no seu amor"-próprio. Nessa manhã, contudo, parecia ter
renascido entre ambas o carinho antigo, e quando entraram na sala do
almoço, d. Angélica, vendo"-as abraçadas, envolveu"-as num sorriso feliz.

O casamento estava marcado para de então a quatro dias. Isidoro virava
do avesso as algibeiras antes de sair de casa, para as despesas do
enxoval, comprado de afogadilho. Por cima do sofá, por cima das
cadeiras, espalhados por toda a parte, a trouxe"-mouxe, viam"-se jornais
de modas, cortes de vestidos, caixas de chapéus, numa desordem alegre.

Os amigos da família corriam a apresentar"-lhe as suas felicitações, e
Madalena ajudava a mãe a receber toda a gente, a determinar as coisas
naquela febre de confusão e de trabalho. De vez em quando os seus olhos
levantavam"-se para o relógio e um arrepio lhe percorria todo o corpo. O
tempo não parava, parecia até correr com maior velocidade! Por vezes
percebia os olhos de Lucila fixos nela. Então disfarçava, mudava de
posição, saía para outro compartimento, pensando:

--- De que serve fingir, se ela sabe tudo que se passa em mim?

Só na hora da visita de Raul se esqueciam as duas irmãs uma da outra,
para se absorverem nele.

\asterisc

Já Lucila esperava o noivo no salão, dedilhando uma valsa ligeira,
quando Madalena desceu disfarçadamente ao jardim, e, contornando o
canteiro que ladeava a rua da entrada, escondeu"-se no caramanchão de
madressilvas, à espera. Fora empurrada por uma tentação dolosa.
Encolhida no banco, com o olhar fixo no portão entrecerrado, esperava a
chegada de Raul, que não tardaria a vir, pensando na outra, para então
levantar"-se, fazer"-se sentir na penumbra e interrompê"-lo no caminho da
felicidade, atraindo"-o, silenciosamente, para o seu primeiro e único
beijo de amor\ldots{}

A irmã ignoraria o crime, o noivo julgaria premir, com a sua, a boca da
noiva, e a amargura da falsidade só ficaria no fundo da sua alma, como o
lodo no fundo de um lago em que se reflete o céu iluminado. Estremecia a
cada rumor de passos, latejavam"-lhe as veias, palpitava"-lhe com força o
coração.

Ele não tardaria, ele, a quem ela amava tanto, ele, que ainda guardava
no peito a rosa murcha que ela lhe atirara um dia, como penhor de toda a
sua vida. A sua vida! e ei"-la que só pedia agora um segundo de ilusão,
um beijo dado a outra, em sua boca!

Consumado o dolo, ela punir"-se"-ia. Não seria nunca mais preciso segurar
ao peito aquele maldito laço azul, que tinha arrancado agora do vestido,
amarrotando"-o entre os dedos febris. Ele não tardaria\ldots{} E se a não
percebesse? Se passasse por ela como por uma pedra? Sussurraria ela o
seu nome? fá"-lo"-ia sentir, materialmente, que ela estava ali, no
propósito de o ver antes de mais ninguém? Oh! o doce e terrível momento,
que parecia tão próximo e que tardava tanto! E se ele não viesse? Se ele
não viesse, nem por mentira ela sentiria nunca um beijo de amor! No dia
seguinte, seria impossível enganá"-lo, o seu lindo rosto já não seria
igual ao lindo rosto da irmã\ldots{}

Enfim, ele chegou. Madalena ergueu"-se trêmula, cosida à galharia negra
das trepadeiras, --- e, no mesmo instante, Lucila passou na sua frente,
quase correndo, a caminho do portão.

Tudo era para ela!

Madalena cerrou os olhos, para não ver, e quando os reabriu, os noivos
beijavam"-se, ali mesmo, a dois passos dela, crendo"-se sós, e seguiram
juntinhos até os degraus de pedra da varanda. Aí Raul voltou o rosto
pálido para o caramanchão de madressilvas, enquanto a noiva subia
adiante, ligeiramente. Madalena esgarçava com os dedos raivosos as
hastes das trepadeiras, vibrando num desespero doido. Tudo lhe fugia;
até a mentira! Mas por que voltaria Raul o rosto, para a treva onde lhe
soluçava o coração?! Que o atrairia para aquele recanto negro do jardim,
onde morria a última esperança de um gozo fictício e enganador?

Ele sentiu a minha alma\ldots{} pensava a moça, furando com a vista a ramaria
espessa. Seríamos todos desgraçados se eu não tivesse coragem\ldots{}
concluiu num suspiro; e caminhou para casa.

Onde estaria a chavinha do armário? Percebe que a mãe, desconfiada, a
pedira ao marido. Talvez a tivesse deixado no bolso do vestido com que
passara o dia\ldots{}

Meu Deus, que alegria na sala, como todos falavam e riam alto! Até d.
Angélica parecia distraída com as anedotas do dr. Sérgio\ldots{}

Encolhida no seu quarto, Madalena esperava o momento. Tudo aconteceria
como por acaso\ldots{} De repente, levantou"-se e seguiu pelo corredor, até ao
quarto dos pais. A chave deveria estar ali, dentro do bolso da saia. Mas
com que saia andara a mãe? Custou"-lhe a lembrar"-se.

O quarto recebia uma luz vaga, pelas bandeiras de vidro comunicando com
o salão. Madalena tateava os vestidos da mãe sem os reconhecer; os dedos
pareciam"-lhe insensíveis, os olhos empanados por uma nuvem\ldots{} A pouco e
pouco, afazendo"-se à penumbra, distinguiu os objetos. Revolveu toda a
roupa de um cabide de pé, inutilmente. Passou ao guarda"-vestidos, abriu
e fechou gavetas, devagarzinho como um ladrão, para voltar de novo ao
cabide, na dúvida de não ter procurado bem. E a maldita saia não
aparecia.

Subitamente acudiu"-lhe à lembrança o banheiro. A mãe tomava o seu banho
quente pouco antes de jantar. Deixara lá com certeza a saia da manhã.
Efetivamente a saia lá estava, atrás da porta, escorrida sobre os
ladrilhos brancos da parede. Madalena sentiu os cabelos eriçarem"-se e os
ombros descaíram"-lhe como sob um peso enorme. Com as costas unidas à
frialdade dos azulejos da parede fronteira, ela olhava para a saia
murcha da mãe como para um fantasma. Por fim, num assomo de energia,
amarrotou"-a com os dedos, nervosamente. O pano cedia, mole e flexível, à
pressão desses dedos, que pareciam transformados em aço. Não sentindo a
chave, Madalena procurou a algibeira, virou"-a, estava vazia. Toda a
tensão dos seus nervos se quebrou de repente. Sucedeu"-lhe o desânimo.

Incapacidade de pensar, cansaço. Foi cambaleando para o seu quarto, como
se tivesse bebido, e atirou"-se como morta na cama.

Lucila cantava na sala e d. Angélica entrou pé ante pé no quarto.

--- Filhinha?

Madalena fingiu que dormia. A mãe apalpou"-lhe a testa\ldots{} as mãos\ldots{} e
quedou"-se a escutar"-lhe a respiração\ldots{} e depois de beijá"-la na face
saiu sossegada e triste. Mas não havia tempo para entregar o corpo
àquela apatia; reagindo contra o extenuamento, Madalena levantou"-se.
Tinha a boca seca, os membros lassos; olhou para os móveis do quarto,
como a pedir uma inspiração, e viu, sobre o toucador, bem em frente,
reluzindo, à claridade frouxa do gás em lamparina, o aparelho de frisar
cabelos: ferra e fogareiro\ldots{} Aí estava um recurso! Poria o ferro em
brasa, e cortaria com ele o rosto em frente ao espelho! Somente\ldots{} isso
não pareceria obra de acaso, e ela queria dar ao acaso toda a
responsabilidade do seu crime\ldots{} Decidiu procurar ainda uma vez a
chave\ldots{} se a não encontrasse, lançaria mão do ondeador em brasa\ldots{}

Voltou ao quarto da mãe. Conversavam todos alegremente na sala. Lucila
tinha acabado de cantar. Ouviu vozes estranhas, novas visitas. Fosse
quem fosse, que lhe importava? Recomeçou a busca, revolvendo os mesmos
lugares, já sem ver, estupidamente.

Minha mãe adivinhou\ldots{} lembrou"-se do ácido nítrico!\ldots{} pensava Madalena,
sem esmorecer na busca.

Ainda não tinha examinado um lugar, a cama da mãe. Ajoelhou"-se, passeou
a mão embaixo dos travesseiros.

Encontrou a chave.

Madalena ergueu"-se, aterrada. Só lhe faltava agora um gesto, mais nada.
Um simples gesto, e tudo se consumaria! Voltou ao quarto, deu toda a
força à luz e postou"-se diante do espelho, despedindo"-se da sua beleza e
procurando na face o ponto sobre que deixaria cair o ácido corrosivo. A
ação devia ser rápida. Dominou a piedade que a sua formosura lhe
inspirava e, branca como o mármore branco, caminhou para o armário dos
remédios, na saleta contígua à copa.

\section*{\textsc{iv}}

A felicidade é uma desgraça, asseverava na sala o dr. Sérgio Bastos,
exibindo"-se, com um esboço de malícia na boca inteligente, de lábios
finos. Tanto quanto lhe permitia a excelente memória, lembrava"-se de que
os seus dias de gozo tinham originado um período de apatia e de
preguiçoso desinteresse por tudo que não fosse a própria causa desse
gozo, todo íntimo, todo pessoal. A felicidade é estéril como o deserto.

Isidoro riu"-se largamente, não tomando a sério a teoria paradoxal do
amigo; parecia"-lhe impossível que naquela idade ainda ele tivesse
semelhantes fantasias! Realmente a felicidade era considerada por cada
um a seu modo, mas nunca lhe passara pela mente tão extravagante
definição. Seria crível que o Sérgio ainda sacrificasse a sinceridade à
frase, como nos idos tempos de rapaz?

O dr. Bastos continuava: Temos da felicidade uma concepção quase sempre
em desacordo com o nosso modo de vida. Meu pai, homem de negócios
agitados, que lhe impunham uma atividade inquietadora e crescente,
afirmava que para ele a felicidade seria a sombra de uma árvore!
Amarrado à cidade pelos seus mil empreendimentos, nunca se dava ao
regalo de uns repousos campestres. Odiava a caliça, os telhados, os
muros e sonhava com o ar limpo dos grandes descampados, onde respirasse
a plenos pulmões, dentro do círculo da sombra projetada pela sua árvore
amiga. Aborrecendo a cidade, a rotina, a disciplina, construiu
quarteirões e quarteirões de casas iguais, fabricou telhas e tijolos,
abrilhantou a cidade, deu prestígio ao comércio e morreu com o ideal que
não teve tempo de realizar, nem lhe traria a consolação que esperava\ldots{}
Por que enriqueceu esse homem? porque, não se considerando feliz, tinha
pressa de chegar à felicidade. Trabalhou ferozmente, irradiou toda a sua
força, impeliu a vida para diante, na ânsia de alcançar a sombra da
árvore desejada\ldots{} Homens sem ideal são parasitas da terra. A felicidade
é uma coisa que se não alcança, atrás da qual se corre e pela qual se
morre\ldots{}

No fundo da sua consciência d. Angélica dizia consigo:

``Palavras! a felicidade é ver o marido satisfeito e as filhas
felizes\ldots{}'' E indagou levemente assustada\ldots{} --- E para o sr. Raul, a
felicidade que é? Dr. Sérgio interveio, enquanto o filho trocava com
Lucila um olhar de amor e um sorriso.

--- Para o Raul agora a felicidade única é o casamento; as fases do amor
são para todos iguais. Mais tarde, estabelecida a família, tranquilo o
coração, o gérmen das suas aspirações, que o determinaram a escolher a
carreira que escolheu, fará florir um outro ideal\ldots{}

--- Mamãe! Mamãe!

D. Angélica levantou"-se de chofre. Era a voz de Madalena irrompendo lá
de dentro, do silêncio da casa, angustiadamente; e antes que Isidoro,
estonteado, acompanhasse a mulher, que já desaparecia no corredor, Raul
atirou"-se para a porta. O pai travou"-lhe do braço com um movimento
enérgico e brusco:

--- Fica!

Lucila precipitara"-se também, acompanhando os pais. Os dois homens
viram"-se, de repente, sós e contemplavam"-se de face. Os olhos de um
buscavam o fundo da alma do outro. Nem uma palavra. Não era preciso; há
expressões que desmascaram o pensamento melhor que o mais exato dos
vocábulos. Raul estava trêmulo, numa ansiedade.

--- Que se teria passado?

--- Qualquer coisa, que te não deve importar. A tua noiva é a Lucila!

As vozes vinham agora como um murmúrio de choro abafado, mal distinto,
tal o rumor d'água dentro de um muro de aqueduto\ldots{}

--- Lembra"-te que será assim por toda a vida! suspirou Sérgio, pousando
carinhosamente a mão no ombro do filho.

--- Serei forte.

--- Tenho medo\ldots{}

--- De quê?

--- De que venhas a ser infeliz!

--- Oh\ldots{}

--- Ainda é tempo de fugir\ldots{}

--- Não\ldots{}

Dr. Sérgio calou"-se. Instantes depois Isidoro reaparecia na sala com um
sorrisinho forçado. Não fora nada; um susto apenas, um pequeno susto.
Madalena quisera por suas mãos preparar um chá, por se sentir adoentada,
a chama do fogareiro do gás propagara"-se às rendas da manga, o fogo fora
abafado, nem a mais leve queimadura\ldots{} Uma pilha de nervos, aquela
menina! A mãe lá ficara a animá"-la. Não valia a pena pensar mais
nisso\ldots{} E agora que ali estavam os três sozinhos, o que deviam era
combinar bem os últimos preparos do casamento. Já tinham carro
contratado? Por ele, tinha tudo pronto! Falara já ao pretor, também
tratara o padre\ldots{}

Lucila reapareceu por sua vez, muito pálida, com arzinho contrafeito e
tímido, de consciência que vacila. Sentou"-se a um canto, e pregou o
olhar nos florões do tapete; não quis intervir nas combinações que os
outros estavam fazendo.

Pensava:

Havia segredos naquela casa, que o noivo nunca deveria saber, que ela
seria a mais interessada em encobrir e fingir ignorar\ldots{}

Que vira um momento antes? A irmã invectivando a mãe por ter esvaziado o
vidro de um ácido corrosivo. Por quê? porque Madalena amava o mesmo
homem que ela, e tinha ciúmes e queria matar"-se. Deveria ela por isso
renunciar ao casamento? Para com a família. seria talvez esse o seu
dever, mas para com o noivo?

O destino a escolhera a ela; devia obedecer ao destino. Se ela
renunciasse a essa felicidade que a tentava, qual o proveito que de
tamanho sacrifício tiraria a irmã? Se fosse possível ceder"-lhe o noivo,
não se veria Madalena na contingência de renunciar também?

A vida é o acaso. O acaso favorecia"-a; a outra que tivesse paciência\ldots{}
Pelo desgosto da irmã ela também não podia gozar uma alegria perfeita.

A sua paixão vivia estrangulada pelo remorso. Mas o egoísmo era mais
forte\ldots{} sentia"-o, e defendê"-lo"-ia até à morte.

A alma materna previra tudo. Aberto o armário, Madalena, ao procurar
febrilmente o frasco de veneno, encontrara"-o vazio, bem lavado! Excitada
no desespero da decepção, revoltada contra a mãe que lhe penetrara os
desígnios, os gritos irromperam"-lhe do peito, raivosos, histéricos,
enquanto os dedos enclavinhados {[}parecia quererem{]} esmagar o frasco
inútil.

--- Mamãe! Mamãe!

O som da própria voz, chamou"-a á realidade. Calou"-se espavorida. Dera o
alarme. Cerrou então os dentes com força. Não responderia a nada, não
diria nada, não falaria a ninguém.

E já a mãe estava a seu lado, de braços estendidos:

--- Meu amor!

E já o pai aparecia interrogativo, aflito:

--- Que é?

Oh, se ela pudesse voar, fugir pela treva da noite fora, com o seu
sofrimento, só dela, e bem escondido! Quem não percebeu a seu lado foi a
irmã, muito pálida, com os olhos engrandecidos pelo espanto.

D. Angélica atraiu"-a ao peito, fixou"-lhe o rosto desvairado, e,
ternamente, devagarzinho, foi"-a levando para o quarto. Isidoro
acompanhou"-as, insistindo:

--- Mas afinal, que foi, que foi?

D. Angélica deitou a filha beijando"-a, como era pequenina. Madalena
então desatou a chorar em silêncio. A mãe desviou"-se para um canto do
quarto e chamando o marido, explicou em segredo:

--- Ciúmes\ldots{} também ela gosta do Raul. Era o que eu temia\ldots{} Depois te
explicarei tudo. Acho bom ires para a sala, inventa uma história que a
desculpe\ldots{}

--- Diabo, então este casamento\ldots{}

--- Está tratado e já agora há de se realizar\ldots{}

--- Tens certeza de que não te enganas?

--- Toda. Vai para a sala.

Isidoro saiu, com o olhar indeciso. Por que não lhe dera antes, Deus,
filhos homens? Estava bem certo de que teria sido mais feliz. Olhassem
para aquela cena! Se um homem seria capaz de tamanha insensatez. Ciúmes,
nervos, amor! quantas asneiras! Nem ele era de feitio a aturá"-las. Uns
feixes de nervos, as tais meninas, muito bonitinhas, muito decorativas,
mas dentro? Confusão e mais nada. Muita confusão. Nem o diabo entende as
mulheres. Quem diria que a sua Madalena, sempre tão sensata e boazinha,
gritasse assim pela mãe como uma doida, só para dizer que estava com
ciúmes da irmã!

Junto à cabeceira da filha, d. Angélica aconselhava:

--- Recolhe"-te por um mês ao colégio das Irmãs, já que não tens forças
para assistir ao casamento de Lucila. É um desgosto para mim e para ela,
que afinal não tem culpa, mas isso sempre será melhor do que praticares
qualquer violência\ldots{} Nunca te supus capaz de tais arrebatamentos. Ias
cometer um crime de que não serias só tu a vítima. Eu sofreria muito!
Não te parece que eu também mereça alguma coisa? Dize\ldots{}

--- Tudo.

--- Então\ldots{}

--- Foi um desvario\ldots{}

--- Prometes"-me ter paciência?

--- Serei irmã de caridade.

--- Não, minha filha, irás convalescer entre as tuas antigas mestras e
voltarás depois curada para a tua casa.

Deves ser forte, preciso de ti. Lucila seguirá o seu marido, é o seu
destino, tu ficarás a meu lado para me fechares os olhos na hora
derradeira; quero levar neles a tua imagem. Por enquanto, recolhe"-te ao
colégio.

Informado mais tarde dessa resolução, Isidoro não a achou acertada.
Embaraçavam"-na os comentários dos convidados das bodas. Que suporiam
eles da ausência de Madalena naquele ato solene? Se a desculpassem por
doente, não faltariam pessoas bisbilhoteiras, como a mulher do sócio,
para lhe varejarem a casa à procura da enferma.

D. Angélica observou:

--- A anormalidade da situação explica outra qualquer anormalidade.
Diremos que, como geralmente todas as gêmeas, Madalena e Lucila são
inseparáveis. Explicaremos o nervosismo da nossa filha como provocado
pela separação da irmã e a sua ausência como um ato de prudência da
nossa parte\ldots{}

--- Os outros não são tão tolos como se te afigura\ldots{}

--- Em todo caso é uma explicação.

--- É um disparate.

--- Arranja outra razão!

--- Não. Vocês as mulheres têm mais imaginação do que nós para as
mentiras\ldots{} Lavo as minhas mãos. Combina o que entenderes.

D. Angélica entendeu que a sua Madalena entraria para o colégio das
irmãs, da rua do Matoso, onde aprendera a ler e onde criara amizades
consoladoras.

Na manhã seguinte, ao entrar no quarto das filhas, encontrou"-as deitadas
na mesma cama.

Os seus lindos cabelos loiros confundiam"-se no mesmo travesseiro, os
seus braços nus entrelaçados pareciam querer unir"-se para a vida e para
a morte. Dormiam. Dir"-se"-ia que a mesma respiração levantava ao mesmo
tempo o peito de ambas. A cama de Lucila estava intacta; ela passara a
noite ao lado da irmã, enternecida e perdoada.

De pé, silenciosa, com os olhos inundados de ternura, a mãe não se
queria arredar dali, daquele quadro de felicidade que ressuscitava para
o seu coração. Suspirava porque aquele sono se prolongasse. Enquanto
durasse a inconsciência, duraria a tranquilidade. Naquelas duas
crianças, havia vinte anos que ela não via senão uma criatura única, não
alterando as suas raras contradições a harmonia que pudesse conter uma
só alma.

O casamento de Lucila encheu de povo a igreja de S. Francisco Xavier. As
amigas lamentavam a falta da Madalena.

--- Coitadinha, não tivera coragem de assistir ao casamento da irmã. Tão
unidas que eram!

Pessoas idosas, de longa prática, aconselhavam d. Angélica a ter muito
cuidado. --- As gêmeas, quando se separam, às vezes até morrem de
paixão!

Dr. Sérgio Bastos regozijava"-se de si para si da prudência da menina,
alegrando"-se com a ideia de que ela se resolvesse a professar. A sua
permanência em casa ameaçava a felicidade do filho\ldots{} Fora bem inspirada
a pequena! Afinal, ser irmã de caridade não é uma coisa do outro
mundo\ldots{} e dadas as circunstâncias que a sua perspicácia adivinhava, era
até a única solução digna. Notava que nesse dia no filho transparecia
uma preocupação qualquer, além do enleio do momento. Por que fugiriam os
olhos de Raul de encontrar os seus olhos? Por que teria voltado a cabeça
com tanta vivacidade ao ouvir comentar na sala a história do laço
azul?\ldots{}

Para ele, pai extremoso daquele filho único, a Madalena só tinha mesmo
um caminho a seguir, insistia: professar; viver para sempre longe da
família, da convivência perigosa e perturbadora do marido da irmã,
rezando, gastando a sua beleza e a sua mocidade sob a touca branca da
religiosa, enquanto a outra, amada e amante, resplandecesse em todo o
fulgor da sua graça! Para sossego do filho, pareciam"-lhe justos todos os
sacrifícios dos outros\ldots{} Que se cumprissem. Era um dever.

Quando, depois da partida do último convidado, o dr. Sérgio Bastos saiu
por sua vez da casa de Isidoro Nunes, levava a alma leve. As suas
responsabilidades estavam divididas.

O seu rapaz bem entregue, a uma família burguesa, simples, ao abrigo de
cuidados de dinheiro que são sempre apoquentadores, com boa saúde e bom
nome. Tivera tino, o Raul; podia apaixonar"-se aí por alguma mulherzinha
pobre e espevitada e acertara com uma moça dócil e simples. Quanto à
irmã. O melhor seria não pensar na irmã. Também essa parecia entregue a
um bom destino. O que o alegrava era a ideia de que haveria mais alguém,
daí em diante, a zelar pela boa saúde do Raul. Isso aliviava"-o dos
sustos e preocupações costumadas. Era adepto do casamento.

Não devera tanto ao seu? Uma boa casa e uma boa mulher tornam o homem
tranquilo e são duas excelentes condições para a prosperidade.

Para haver um \emph{mas}, havia a profissão do filho, que é a de
renúncia a todas as comodidades da terra. Marinheiro deve ter a alma
como o corpo: livre. O hábito da separação torna as criaturas
indiferentes e faz perigar a fidelidade do amor. Em todo caso, ainda
nisso o seu Raul fora feliz, casando com uma menina que ficaria sob as
vistas de uma mãe honesta e observadora. Grande alma! concluía ele,
acendendo o seu charuto, a caminho da \emph{Pensão}, onde ficaria até ao
dia da partida do filho.

Esse dia não tardou. O navio partiria às três horas. Dr. Sérgio foi
almoçar com os noivos. Eles, coitadinhos, não comeram quase nada,
misturando o sabor das lágrimas ao das azeitonas e ao do presunto, que
d. Angélica adicionara aos pratos corriqueiros, naquele almoço de
despedida. O que valeu a todos, foi Isidoro não ter descido à cidade,
porque na véspera torcera um pé. As pequenas contrariedades vêm às vezes
em socorro dos grandes embaraços. Se não fosse o pé do Isidoro, dr.
Sérgio teria ficado em face das lágrimas dos filhos, sendo para d.
Angélica, sempre cuidadosa, todo o tempo pouco para os ouvir.

Assim, disfarçaram a pena da separação, conversando voluvelmente sobre
coisas antigas do colégio; sestros de uns, glórias de outros, desgraças
de muitos\ldots{}

Isidoro não perdera de vista a maior parte dos colegas.

Sabia"-lhes a vida, comentava"-lhes a sorte. Quando a narração descambava
para a tristeza, dr. Bastos, habilidosamente, com o seu risinho fino,
puxava"-a para o caminho da graça, lembrando um caso pitoresco, citando
uma anedota\ldots{}

E nunca a nora lhe parecera tão bem como nessa manhã de choro, com o
narizinho vermelho, os olhos pisados, o corpo abandonado num vestido de
cassa branca salpicada de amoras.

\section*{\textsc{v}}

Trechos de cartas de Raul:

``Minha adorada mulherzinha. Alto mar, a bordo do \emph{Benjamin
Constant}. --- Não posso mais! A saudade adoece"-me. A tua imagem não se
esvai um momento da minha imaginação. Já não és mulher; és febre.
Latejas nos meus pulsos, palpitas no meu coração; diluis"-te na minha
voz! Estás dentro do meu cérebro e da minha alma, tão obcecado estou de
ti; e essa constante presença aumenta delirantemente a minha saudade!
Pareço estranho a todos e a tudo, porque vivo abstraído no meu amor. Só
me sinto melhor na solidão, porque a solidão está cheia de ti! A
companhia dos outros altera"-me. Sinto que me roubam à única felicidade
que me é dado fruir, quando solicitam a minha atenção, dada sempre com
má vontade, porque essa felicidade é fixar profundamente a tua
fisionomia, supor o que estarás pensando, ressentir em pensamento os
gozos da nossa curta convivência, e os que a anteciparam: os nossos
olhares confundidos, o tremor das tuas mãos fugindo às minhas, os
suspiros apenas adivinhados\ldots{} Não durmo. As insônias alteram"-me o
organismo. Tenho tido acessos. O médico de bordo insiste em dar"-me
remédios. A minha salvação está longe\ldots{} Nem tento explicar"-lhe: não me
entenderia. Nem o mais sábio dos homens compreende bem a voz de um
namorado, e eu serei de ti um namorado eterno! Escreve"-me. Escreve"-me
sempre. Dize que me amas; dize"-mo com ardor. As tuas letras serão a tua
voz!''

\dotfill{}

``Meia"-noite (a bordo). --- Deixo de falar contigo para escrever"-te.

Até a hora que assinalo aqui, desde o anoitecer, estive debruçado na
amurada, olhando para as ondas e para as estrelas, a dialogar contigo,
meu doce, meu terno bem! E nada chegará à tua alma do que te digo no
espaço infinito? Não sentes os meus beijos no teu corpo, a minha voz,
ainda que como um murmúrio, nos teus ouvidos? De tudo que se irradia da
minha alma, nada, mas nada chegará à tua? Explica"-me. Conta"-me tudo.
Como pensas em mim? Sonhas? E no sonho como te apareço? Tu a mim como
uma visão celestial, virgem candidíssima, inatingível e desesperadora! E
és minha mulher. És minha e estás tão longe dos meus braços ansiosos e
apaixonados\ldots{}

Beijo"-te, beijo"-te, beijo"-te\ldots{} e morro!''

\dotfill{}

``Lisboa. --- Só em Lisboa a primeira carta tua e tão fria, tão
incompleta! Não sei que falta à tua carta, meu amor, mas falta"-lhe
qualquer coisa. Dir"-se"-ia que a escreveste timidamente, como se eu não
fosse teu marido. Cuidaste na caligrafia. Parece uma carta copiada.
Escreve"-me como se me falasses, sem meditar, espontaneamente e não
corrijas nada. Olha, eu escrevo"-te com beijos e beijos desordenados. Não
me importa saber senão de ti, do teu afeto e do teu pensamento!''

\dotfill{}

``Hamburgo. --- Sinto na tua última carta o escrúpulo de parecer muito
íntima. Escreveste"-me pensando em outra coisa simultaneamente. Em quê?
Deixas"-me entrever uma hipótese divina, mas apenas entrever. Não és
franca. Lembra"-te de que sou teu marido e escreve"-me sinceramente, como
se falasses só para ti. Nesta carta parecias ausente de ti mesma, e eu
sofro tanto!''

\dotfill{}

Chegavam cartas de Raul por todos os correios. Lucila abria"-as
febrilmente, mas logo uma leve sombra de melancolia se lhe espalhava
pelo rosto. As cartas do marido eram sempre cartas de namorado,
idealizando bens que não pareciam ainda fruídos, como se ela continuasse
a ser a virgem candidíssima, inatingível e desesperadora, que lhe
aparecia nos sonhos. O que ela julgava perceber naquela saudade e
inconsolável sofrimento era a ideia da irmã, confundida com a dela no
cérebro do marido. Eram da irmã as mãos que fugiam às dele num tremor
mal disfarçado, eram da irmã aqueles suspiros apenas pressentidos ou
adivinhados\ldots{}

Era o mistério da irmã que alterava em Raul a saudade da esposa, a
mulher de verdade, a mulher simples e presa nos seus braços para sempre!
Que faltara na carta que escrevera ao marido e que ele recriminava?

Nem a lágrima, borrando a tinta com que no
fim' escrevera o seu nome!

O que faltava ali eram letras escritas por Madalena; tão acostumado
estava a confundi"-las, que já não compreendia uma sem a outra. E agora?!

Madalena continuaria no colégio da rua do Matoso. Ninguém a tiraria de
lá. Estava como num túmulo.

E ela? O marido não penetrara o pudor da sua confissão, ou mal se
atrevera a penetrá"-lo\ldots{} E teria assim de dizer com maior clareza que
seria mãe. Horrorizava"-se pensando que germinassem no seu seio dois
entes tais quais ela e a irmã. O alvoroto não era de alegria. Entrou
também a definhar, sem dizer a ninguém a causa desse abatimento e só à
noite, na solidão do seu quarto, as suas lágrimas ciumentas lhe
desafogavam o coração pesado e dolorido.

O palácio da ventura enganara"-a. Transposto o limiar, encontrara dentro,
como o poeta, a escuridão e o silêncio. De mais a mais a consciência
recriminava"-a pelo abandono da irmã, cuja ausência os pais não cessavam
de lamentar. A casa parecia ter sido varrida pela morte. Passavam"-se
horas sem se ouvir a voz de ninguém. Ela vagava pelos aposentos,
buscando inconscientemente a alegria que lhe faltava: a convivência da
outra\ldots{} a presença do marido; e não raras vezes percebia no gesto
disfarçado da mãe o domínio de uma dor profunda.

Corriam os meses e as cartas de Raul chegavam sempre ardendo em chamas
amorosas, até que faltaram em três correios sucessivos. Lucila alarmou a
família. Saiu da sua atonia. Rebentava em cuidados, formulava hipóteses
terríveis, telegrafou para a fazenda do sogro, chamando"-o como para a
salvar de um perigo.

Dr. Sérgio Bastos afivelou à pressa as correias da mala e atirou"-se numa
viagem expressa. A nora precipitou"-se"-lhe nos braços quando o viu.

Que diferença da suave Lucila que ele deixara, para a senhora que ele
beijava agora!

Isidoro desculpava a insensatez da filha, alarmada só com a falta de
três ou quatro cartas! Mas o dr. Bastos também vinha aflito. Ele
recebera na véspera à noite uma carta do comandante do \emph{Benjamin},
seu amigo, comunicando"-lhe que o filho tivera de ficar num hospital de
Toulon, com um tifo.

Calaram"-se todos, num grande abatimento. Dr. Sérgio pensava em partir
pelo primeiro paquete; mas não chegaria talvez a tempo nem de amortalhar
o filho. Se o comandante participava o caso depois do décimo segundo dia
da doença! Esperara antes que o mal diminuísse, mas o mal agravara"-se.
Ah! ele sabia bem o que eram essas febres da Europa. E vejam! o seu Raul
fora sempre um rapaz saudável\ldots{}

O que aterrava as senhoras era a ideia do hospital. Isidoro animava:

--- A enfermaria é melhor que a casa particular. Por esse lado, ficassem
descansadas\ldots{}

Mas Lucila não ouvia razões. Exatamente, no fim da viagem, quando não
faltavam senão dois meses! Também ela queria partir com o sogro, o seu
lugar era lá, junto à cabeceira do doente\ldots{}

Lembravam"-lhe o seu estado. Seria impossível, partir\ldots{} Insuflavam"-lhe
coragem. Dr. Bastos corria da agência do telégrafo para as agências dos
paquetes e voltava para junto da nora suado, vermelho, com os olhos
orlados de um roxo pisado, de vigília. Não havia lugar no primeiro
paquete. Tudo cheio, desesperadoramente cheio! Teimara com o agente.
Tornaria a teimar. Embarcaria de qualquer modo. Mas esse primeiro
paquete, só partiria dali a quatro dias e estaria a bordo mais de quinze
dias, antes de saber do filho amado. E a resposta do telegrama, que não
chegava! Dr. Bastos injuriava o cônsul brasileiro. Até que o pobre
cônsul respondeu:

``Raul minha casa. Livre de perigo. Escrevo''.

Novo conselho de família, em que o dr. Sérgio Bastos, já tranquilizado,
perguntava perplexo, procurando dividir com os outros a sua
responsabilidade:

--- Vou ou não vou?

Já agora, resolveram, seria mais prudente esperar a carta anunciada.

Entretanto Lucila esmorecia. Não sabia explicar o que tinha, definhava.
Chorava sem motivo conhecido, não dormia, e tinha impaciências --- até
que, numa linda madrugada, nasceu a primeira neta de Isidoro, que d.
Angélica recebeu chorando nos braços amorosos.

Lucila ia de mal a pior; ardia em febre, indiferente a tudo, amodorrada.
Era um vaivém de médicos. As conferências sucediam"-se, as enfermeiras
não descansavam; Isidoro já não saía, e dr. Sérgio instalara"-se, muito
solícito, à cabeceira da nora.

Foi assim que, numa tarde, em que todos supunham a doente acalmada e
melhor, ele a viu expirar suavemente, como quem adormece\ldots{} Pois logo no
seguinte lhe chegou às mãos uma carta do seu amigo cônsul, dizendo:

``Seu filho pensa em partir para aí no fim do mês, embora os médicos não
aprovem essa resolução, mas temem contrariá"-lo, porque ele está
excessivamente nervoso. É prudente pouparem"-lhe qualquer emoção; até que
o estado ainda melindroso em que ele está, passe de todo. Dentro de uns
trinta ou quarenta dias vocês aí o terão''.

\section*{\textsc{vi}}

Caía a tarde. A sineta do Colégio chamava as educandas à oração, na
capela. As aves procuravam o ninho nas copas ramalhudas das mangueiras
do parque.

--- Nada comove a natureza, pensava o dr. Sérgio, observando a doçura da
tarde, em que o vulto alquebrado de Isidoro Nunes, vestido de luto, se
destacava subindo a rampa do jardim.

No parlatório mal esperaram por Madalena, que apareceu logo, muito
pálida, de olhos pisados.

--- Filhinha, venho mais uma vez pedir"-te que voltes para casa\ldots{}

--- É impossível.

--- Tua mãe está muito só.

--- Pedirei a Deus que lhe dê coragem.

--- Deus quer que os filhos sirvam de amparo e de consolação a seus
pais.

--- Deus quer que eu me faça religiosa.

Isidoro baixou a cabeça desanimado, sentindo a firmeza inquebrantável da
filha. Dr. Bastos interveio, com os olhos rasos d'água, fazendo"-se muito
humilde:

--- Madalena, deixe que a nossa voz entre no seu coração. Escute: também
eu venho pedir"-lhe um sacrifício enorme, não se ofenda e pense que não é
um homem que está aqui a seu lado, mas um coração de pai. Raul deverá
chegar dentro de oito dias. Sabe\ldots{} ele esteve à morte, e os médicos
pedem"-me que lhe evite grandes comoções. É o mesmo que me ameaçarem com
a loucura do meu rapaz, caso sofra um abalo, para o que não está
prevenido de nenhum modo\ldots{} Nesta última carta então, parece de
propósito, ele só fala na alegria de abraçar a mulher e a filhinha\ldots{}

Houve uma pausa. Madalena olhava para o dr. Bastos sem pestanejar. Ele
continuou:

--- Venho suplicar"-lhe uma ação piedosa, o engano de uma hora só,
enquanto o possamos preparar para a verdade terrível\ldots{} A senhora
voltará depois para aqui, para a sua religião, mas salve meu filho, por
piedade\ldots{}

--- Não entendo\ldots{}

Isidoro auxiliou o amigo:

--- Fingirás ser Lucila, durante os primeiros instantes
da chegada do Raul\ldots{}

Madalena não respondeu, mas teve um movimento tão forte de repulsa, que
o pai suplicou, a chorar:

--- Madalena!\ldots{}

--- Não, meu pai, não\ldots{}

Dr. Sérgio deixou"-se cair numa cadeira, lívido e calado.

--- Tens o coração duro, minha filha. Uma hora na vida passa"-se
depressa. Que te importa mais um sacrifício, se com ele salvarás,
talvez, a razão de um homem?

--- Lucila não me perdoaria\ldots{}

--- Lucila morreu.

--- Agora, melhor do que nunca, ela conhece a minha alma!

--- Pois é ela, que te implora pela minha boca que protejas a razão de
Raul.

--- Ele terá de saber.

--- Mas não de repente. Sabes que ele teve uma febre cerebral, que vem
fraquíssimo, nervoso, predisposto a um mal pior que a morte\ldots{} Dr.
Sérgio imagina que a tua presença atenuará a vibração do golpe, e tua
mãe aprova essa ideia\ldots{}

Madalena cerrou as pálpebras, para reter as lágrimas: os lábios
tremiam"-lhe. Parecia de cera.

--- Tem paciência, minha filha; é mais um sacrifício!

--- Muito grande\ldots{}

--- Muito grande, mas tens a alma forte. Vem. O teu papel é doloroso.
Raul saberá pela tua boca, da morte de tua irmã\ldots{}

--- É um pecado\ldots{}

--- É uma esmola.

Para Madalena, o pecado, que ela media até ao fundo, consistia no
rompimento de um voto que fizera de fugir ao marido da irmã, até serenar
de todo o coração para a tomada do véu. A presença desse homem
alvoroçava"-lhe a alma, agora comprometida com o Senhor!

Dr. Sérgio calara"-se, desanimado, olhando para o chão muito lavado, da
sala. Isidoro fixava a filha com mágoa, mesclada de censura.

Depois de um largo silêncio, insistiu:

--- Decide"-te, Madalena\ldots{} vem!\ldots{}

--- Perdoe"-me\ldots{} não posso. Raul é um homem, não lhe devem adiar, por
meio de um embuste, o desgosto por que forçosamente há de passar!

--- Repara minha fi.

Dr. Sérgio levantou"-se com ar decidido, e, interrompendo com um gesto a
nova súplica de Isidoro, despediu"-se secamente da moça. Saíram ambos, e
já desciam a rampa, quando Madalena se atirou chorando, para a porta,
decidida a chamá"-los e a partir. Mas uma irmã de caridade que entrava no
momento interceptou"-lhe a passagem. E dois braços se abriram diante de
Madalena, que se deixou cair neles, soluçando alto.

D. Angélica acalentava a netinha, que de hora em hora achava mais
bonita, ouvindo, na sala ao lado, os desabafos do dr. Sérgio contra a
religião. Isidoro concordava; realmente, dizia ele, Deus tirava"-lhe as
filhas por todos os modos!

Ajeitando o corpinho da criança nas flanelas do cueiro, d. Angélica
decidiu ir por sua vez induzir Madalena a voltar para casa; e, para que
se não azedassem ainda mais contra a filha, não confiaria a ninguém a
sua tentativa, certa de que ela falharia também.

No dia seguinte de manhã, pôs"-se a caminho. Não estudara argumentos;
deixaria o coração falar. Madalena sobressaltou"-se ao vê"-la:

--- Como a senhora está magra\ldots{}

--- De saudades, talvez; é o meu único mal\ldots{}

Madalena cobriu"-a de beijos. D. Angélica chorou. E depois:

--- Não perguntas pela pequenina?

--- Ah\ldots{} sim.

--- É muito bonita, com um mês apenas já sorri! Começa a engordar sinto
que daqui a pouco já não poderei com ela. Fogem"-me as forças. Parece"-me
impossível que eu carregasse duas crianças ao mesmo tempo ao colo, como
carreguei.

Preciso que vás olhar por ela, já não digo por mim; mas a sorte da filha
de minha filha preocupa"-me muito. O pai\ldots{} não sei que doença teve\ldots{}
dizem que vem contra a vontade e a opinião dos médicos\ldots{} ainda muito
fraco\ldots{} depois, é homem, não pode olhar pela menina\ldots{} Uma criança dá
muito trabalho; esta vida é mesmo assim. E eu quero que a filha da minha
Lucila seja feliz\ldots{}

--- Eu\ldots{}

--- És a minha única filha, a minha esperança, a minha consolação. O teu
quarto está arrumadinho\ldots{}

Mandei lavá"-lo ontem. Fiz eu mesma hoje a tua cama\ldots{} o bercinho da
criança vai ficar no teu quarto. Tens mais saúde, tomarás conta dela
durante a noite; eu de dia, sempre é mais fácil\ldots{} É uma grande
responsabilidade, criar uma órfã de mãe! Se eu fosse mais moça\ldots{} mas,
enfim, tu olharás por ela como se foras a própria Lucila; é a minha
esperança\ldots{} é a minha vontade\ldots{} é o teu dever.

--- É o meu dever\ldots{} Mas tenho também outro dever.

--- Qual?

--- No dia, na hora mesmo em que Lucila se casava\ldots{} fiz, chorando, de
joelhos, diante do altar, a promessa de esmagar todos os sentimentos
humanos que me prendiam à terra, e ir ser religiosa.

--- Deus não aceitou a tua promessa, porque te chama para meu lado pela
boca inocente de um anjo\ldots{}

--- De mais a mais\ldots{} eu tenho medo\ldots{}

--- De quem?!

--- \emph{Dele}\ldots{}

Madalena corou até à raiz dos cabelos.

--- Tu o salvarás se quiseres, dizem\ldots{}

--- O que me propuseram é absurdo e é indigno.

--- Foi uma ideia do Sérgio, coitado; ele deseja que tu continues a
outra\ldots{} Percebo a intenção e desculpo"-a. Está acabrunhado. É muito
nosso amigo, agora não sai lá de casa\ldots{} Não imaginas o silêncio!\ldots{}
Bem; a minha Lucila deve estar com fome\ldots{} dá"-me outro beijo, mais
outro\ldots{} Nunca me dês um beijo só. Também para mim és uma continuação\ldots{}
Tive duas, não tenho nenhuma\ldots{}

Madalena olhou em silêncio para a mãe, cujo sorriso forçado mais lhe
enrugava o rosto envelhecido. Nem uma censura passara por aqueles lábios
descorados e meigos\ldots{}

--- Adeus, minha filha\ldots{} sê feliz\ldots{}

Madalena estremeceu. Uma piedade infinita encheu"-lhe os olhos d'água.

--- Feliz? longe da minha mãezinha? Não! Eu vou também, espere; seria
uma injustiça!

Minutos depois, mãe e filha desciam juntas à rua. a caminho de casa.

--- Vais ver, dizia d. Angélica; a nossa Lucila está ficando
redondinha\ldots{} Pusemo"-lhe o nome da mãe, porque assim ele continua vivo,
animando a casa\ldots{} Teu pai está um avô candongueiro. Sérgio também\ldots{}
somos três velhos piegas! Tu salvarás tua sobrinha, que sem ti ficaria
perdida de mimos\ldots{}

Um certo dia, d. Angélica fez abrir todas as janelas da casa. O berço da
pequenita foi trazido para a saleta. Toda a família se vestiu de
claro\ldots{} Era um sacrifício feito por amor da morta, para lhe salvarem o
marido, que não tardaria a chegar; já Isidoro e o Sérgio tinham ido
esperá"-lo ao cais.

D. Angélica engolia as lágrimas, revoltada no fundo contra aquela
espécie de ingratidão; mas com pena do dr. Sérgio não lhe quisera negar
essa vontade\ldots{} Madalena, toda de branco, sentara"-se ao lado do berço,
que embalava docemente. Tremia. Qual deveria ser a sua atitude? Por mais
que ensaiasse em espírito o seu papel, temia não o poder representar, e
suplicava à mãe:

--- Não me abandone nem um momento\ldots{} Não terei coragem de ficar só com
ele\ldots{}

Eram duas horas quando o automóvel parou à porta. Madalena escondeu o
rosto nas dobras do cortinado do berço. O coração batia"-lhe loucamente,
desordenadamente.

--- Coragem! disse"-lhe a mãe; e caminhou para a porta. Raul entrou de
braços abertos, chamando alto:

--- Lucila! Lucila!

Madalena colheu a criança com rapidez e apresentou"-a ao guarda"-marinha:

--- Lucila está aqui!

E com a pequenita muito unida ao peito, mal permitiu ao moço abraçá"-la.
Entretanto ele contemplava"-a, feliz:

--- Ainda me pareces mais bonita\ldots{} meu amor! Estás tão linda; mas tão
trêmula\ldots{} tão pálida\ldots{} choras?! É a comoção! Meu amor, meu amor!

Acudiram todos em auxílio de Madalena, rodeando"-a, chamando a atenção do
pai para as perfeições da filhinha. Que reparasse na formosura dos seus
olhos azuis\ldots{} E o narizinho? uma perfeição! Vira nunca boca de criança
tão bem talhada? O encanto era vê"-la nuazinha, no banho! Ele vinha
melhor do que esperavam. Ainda bem! E a viagem? Que contasse tudo!

--- Ah! a viagem reanimara"-o! Embarcara contra a vontade de todos. Mas
falaria disso depois. Agora só queria saber da sua ventura. Estava
feliz, muito feliz, ao lado da sua mulherzinha adorada.

Isidoro, dr. Sérgio, d. Angélica, prolongavam de propósito a palestra,
repetiam perguntas, diziam banalidades.

Houve um momento em que Raul exclamou:

--- Acho todos mais magros, abatidos\ldots{} esquisitos\ldots{} E Madalena?

Entreolharam"-se aflitos; mas o dr. Sérgio acudiu: que a moça ia bem.
Raul não se fartava de olhar para a mulher, confessando encontrar nela
maiores encantos, mas uma tal reserva! E lá consigo supunha: --- é a
presença dos pais que a intimida.

A situação prolongou"-se, tanto quanto a habilidade dos velhos conseguiu
prolongá"-la. Mas Madalena sufocava, não podia mais! De repente,
sacudindo a perturbação horrível que a oprimia, fez sinal aos pais que
saíssem. D. Angélica esgazeou os olhos. Estaria sonhando? pois a filha
não lhe pedira que a não abandonasse?! Percebendo o seu espanto, e para
acabar de convencê"-la. Madalena entregou"-lhe a criança, dizendo:

--- São horas de lhe dar a mamadeira, sim?

D. Angélica recebeu a criança com um gesto lento e absorto.

Tinha chegado a hora dolorosa. Dir"-se"-ia que de todos aqueles rostos iam
cair máscaras, de repente. Dr. Sérgio tremeu e pôs"-se a tatear, como à
procura, um botão de camisa que lhe não saíra do lugar.

Raul observou:

--- É extraordinário, vocês parecem"-me todos muito mudados,
misteriosos\ldots{} que houve?\ldots{} Dize\ldots{} ah, não, não digas; o que se
passou, passou, e o que me importa agora é gozar a delícia de estar a
teu lado, minha mulher!\ldots{} Mas que tens?!\ldots{} por que te afastas?
Beija"-me e deixa"-me beijar"-te! Estás linda, sabes? Linda como nunca! A
saudade espiritualizou o teu rosto, és bem a realidade do meu sonho, e a
alegria de todo o meu futuro. Agora não nos separaremos mais, nunca
mais, nunca mais! Vem, abraça"-me!\ldots{} Mas que tens tu?! por que me
foges?\ldots{} Que pudor é esse, que te alvoroça, ao ponto de fugir"-me\ldots{}
Afinal, tu és minha mu\ldots{}

Madalena recuara, e olhando fixamente para o cunhado, agora atônito,
suspenso, prendeu vagarosamente ao peito o laço azul que tinha trazido
amarfanhado nas mãos.

--- Que quer dizer isso, Luci.

Raul estacou, muito pálido.

Em pé, diante dele, cheirando a rosas, iluminada por uma paixão que já
não precisava esconder"-se e que a transfigurava, tornando"-a mais
encantadora, Madalena contemplava"-o de face.

Oh, era bem ela, essa criatura adorada cuja imagem não o abandonara um
só instante, na dolorosa ausência. Duvidando da sua razão, Raul estendeu
as mãos súplices. No vestido branco da cunhada o laço azul tomava a seus
olhos proporções enormes. Desorientado, inquiriu, passados instantes,
com os lábios brancos a tremerem:

--- Lucila, a minha Lucila?\ldots{}

Madalena apontou para o céu.

--- És então Mada\ldots{}

Por única resposta, a moça abriu"-lhe os braços. Ele precipitou"-se,
apertando"-a ao peito e repetindo como um doido:

--- É mentira, é mentira, dize"-me que me mentiste, e que me amas,
Lucila!

E Madalena murmurou quase desmaiada:

--- Adoro"-te\ldots{}

\asterisc

No fim de alguns meses de luto, dr. Sérgio Bastos pediu ao seu velho
amigo Isidoro Nunes a mão de sua filha para o Raul.

Nessa mesma tarde d. Angélica saiu sozinha e foi enfeitar de flores a
sepultura de Lucila. Era o seu culto. Era a sua devoção. No meio das
lágrimas, a pobre senhora pensava, rezando, que não há amor que triunfe
sem sacrificar outro amor\ldots{}

\chapter{O dedo do velho}

\begin{flushright}
\small\textit{Whether this be\\
Or be not, I'll not swear}\\\emph{Tempest}, \textsc{shakespeare}
\end{flushright}

\section*{\textsc{i}}

--- Quem está aí?

--- Quem está aí?

Eram quase duas horas. Claudino Senra estaria acordado havia uns dois
minutos quando sentiu dar volta à chave da porta do terraço para a
biblioteca, contígua ao seu quarto de dormir.

Receando estar ainda iludido por um resto de sonho, repetiu a pergunta:

--- Quem está aí?

Nenhuma resposta, como das outras vezes. Soerguido na cama, com um
cotovelo fincado no colchão, ele arregalou os olhos para a porta e
apurou o ouvido. Alguém deslizava agora pela sala, com pés de lã e abria
a grande estante de jacarandá envidraçada,
cuja porta rangeu num gemido, que foi logo abafado. Não, agora não podia
haver dúvida,
aquilo não era uma ilusão; ele tateou a cabeceira à procura do cordão
elétrico e acendeu a lâmpada, ao mesmo tempo que pensava, com um
calafrio:

--- Tenho ladrões em casa. O estúpido do Antão esqueceu"-se de fechar bem
as portas\ldots{} E agora? O melhor seria talvez ficar quieto na cama, e
fingir que durmo; porque, afinal, não sei se terei de lutar com um homem
só, ou se com mais\ldots{} Seria mais prudente talvez deixar"-me roubar e não
arriscar a minha vida; mas agora é tarde, devem ter percebido que acendi
a lâmpada.

Estas reflexões passaram como um relâmpago pelo cérebro de Claudino, ao
mesmo tempo que ele saltava da cama e tirava da gaveta do criado"-mudo o
seu velho revólver, de cujas boas funções já infelizmente duvidava um
pouco. Vacilou ainda: seria
melhor correr à biblioteca ou esperar ali a pé firme e de arma em punho,
que aparecesse alguém? Esperou, até que viu abrir"-se de par em par a
porta do seu quarto, sem o mais leve ruído. Quis fazer fogo; o gatilho
do revólver estalou em seco. Mas fazer fogo contra quem? A porta parecia
ter sido aberta por um sopro que arrefecia toda a temperatura do
aposento.

Com os cabelos em pé, e a coragem dos momentos decisivos, Claudino
caminhou até à biblioteca, num arremesso, para estacar entre os batentes
procurando alguém com a vista. Não viu ninguém; mas reparou que a sala
estava iluminada pela lâmpada móvel da sua secretária, que ele apagara
ao ir deitar"-se. Quem a reacendera? Quem tinha aberto a porta do seu
quarto? Alguém, que se escondia, naturalmente agora por detrás do
reposteiro da porta, ou que fugira ao senti"-lo aproximar"-se. Havia de
ser isso.

A essa ideia, Claudino caminhou para a porta do terraço, apontando o
revólver para diante, com braço firme; temendo ao mesmo tempo que atrás
do reposteiro estivessem escondidos malfeitores que lhe saltassem em
cima. Com assombro verificou que por trás dos reposteiros não havia
ninguém e que a porta para o exterior fora, além de fechada à chave,
trancada por uma larga barra de ferro, pelo seu criado Antão. Na
biblioteca não havia outras portas além dessa e a do seu dormitório. As
janelas estavam fechadas. Claudino investigou então por baixo dos
móveis, atrás do biombo e atrás de uma colcha da índia pendurada na
parede, à guisa de ornamento.

Não vendo ninguém, voltou ao seu quarto de cama, esquadrinhou todos os
cantos, abriu o guarda"-casacas, arredou o divã do canto, sacudiu os
almofadões, como à espera de ver cair deles os ladrões em forma de
farelo e, não vendo nada, nada, refletiu que tinha sido vítima de uma
ilusão. Fora certamente ele quem se esquecera de apagar a lâmpada,
embora tivesse a lembrança nítida de o haver feito. Preferia duvidar de
si a acreditar que uma pessoa qualquer tivesse penetrado em sua casa
pelo buraco da fechadura\ldots{} O que tinha agora a fazer, era ir apagar a
luz e vir dormir sossegadamente o seu resto de noite. Para isso voltou
quase tranquilo à biblioteca, mas logo ao entrar observou que a lâmpada
mudara de posição, inclinando"-se agora para a estante de jacarandá
completamente escancarada.

Claudino estremeceu violentamente. Quem teria vindo inclinar a haste da
sua lâmpada para o lado exatamente oposto àquele em que a tinha visto,
ainda há poucos minutos?! No assoalho não havia alçapões; nas paredes
não existiam saídas falsas; para o teto não havia escadas nem
aberturas\ldots{} que seria aquilo? Lembrou"-se então de ter ouvido ainda da
cama o ranger da velha estante manuelina de vidraçaria lavrada, onde o
pai em vida acumulara as grandes obras clássicas das mais famosas
literaturas, e que ele jamais folheara, por falta de tempo e de
curiosidade.

A luz derramada pela lâmpada incidia agora francamente para as lombadas
bolorentas dos livros velhos, alinhados no armário. Teriam tido os
clássicos seculares a fantasia de virem do outro mundo rever os seus
pensamentos escritos?

O silêncio era absoluto. Quantas horas faltariam ainda para que a luz do
sol viesse espancar aquela agonia extravagante? Claudino quis saber. O
homem enlaça todos os acontecimentos extraordinários da sua vida à ideia
exata do tempo. Nas circunstâncias as mais desorientadoras e
imprevistas, não se esquece de consultar o relógio, como a mais
expressiva das testemunhas a invocar, ou a citar depois. Nos assaltos,
nos assassínios, nos naufrágios, nos incêndios, em todos os lances de
delírio, de susto, ou de clamor, há sempre alguém que se não esquece de
fixar no relógio um olhar de interrogação, e que guarde na lembrança
agitada a sua resposta impassível.

Procurando reagir contra as próprias impressões e tentar com toda a
clareza de consciência uma prova decisiva, Claudino apagou de novo a
lâmpada da sala e voltou para o quarto, ainda iluminado, a ver as horas
no seu cronômetro. Eram duas.

Curvava"-se ele ainda sobre a mesinha de cabeceira, quando um leve
estalido o fez voltar depressa a cabeça para a porta e perceber que a
lâmpada que ele tinha apagado, acabara de ser reacendida. Um frio de
doença percorreu"-lhe o corpo. Estaria ele louco? Apalpou"-se todo para
certificar"-se se seria ele mesmo em carne e osso quem ali estava de pé,
hirto de susto, no seu próprio quarto, junto à cama desfeita.
Ocorreu"-lhe então chamar o criado. Era um idiota, mas em tais
conjunturas até na companhia de um gato encontraria alívio. O diabo era
que, para acordar o bruto do Antão, teria de percorrer um corredor
comprido e de passar por várias portas\ldots{} Teve medo. Um medo de mulher,
um medo de criança. E pôs"-se a pensar, com um esforço horrível:

Estarei eu bêbedo?\ldots{} Com quem estive antes de me deitar?\ldots{} mas eu
nunca me embebedei!\ldots{} quem sabe\ldots{} mas por onde andei eu ontem?\ldots{} a
que horas me recolhi a casa?\ldots{}

O seu espírito estava incapaz de conjugar duas ideias. Varrera"-se"-lhe
tudo da lembrança. Moveu repetidamente a língua e os lábios, a
experimentar se sentia no paladar o sabor do vinho terrível que o
tivesse embriagado, mas a boca não lhe sabia a nada; e nunca o nada lhe
pareceu tão complicado. Querendo atribuir os fenômenos que presenciava,
a qualquer alteração do seu organismo físico, acabou por se querer
convencer de que fumara na véspera em demasia.

Sim, não podia ser outra coisa.

O que lhe competia agora fazer era voltar serenamente à sala, abrir a
janela à frescura do ar livre e esperar que tudo passasse. Assim fez,
mas ao transpor a soleira da porta, estacou boquiaberto.

Em cima da mesa, bem exposto à claridade da lâmpada elétrica, estava
aberto um dos grandes livros da estante e, como se mão invisível o
manuseasse, as suas folhas viravam"-se lentamente, sossegadamente\ldots{}

Senhor! continuaria o delírio?! Com os olhos esbugalhados, a garganta
áspera e seca como se tivesse engolido alfinetes, Claudino aproximou"-se
da mesa, apoiou"-se nela com as duas mãos e olhou. Viraram"-se ainda
algumas folhas, depois o livro permaneceu
por algum tempo imóvel, até que sobre o papel amarelado apareceu um dedo
de homem, um dedo velho que, desusando sob várias linhas, fixou"-se por
fim sob uma só frase.

Era um dedo indicador, pálido, magro, nodoso, de pele engelhada, e unha
curta, da forma e da cor de uma escama de peixe. Claudino compreendeu a
insistência: ele deveria ler aquelas palavras e compreender"-lhes o
sentido. Mas como? Na sua confusão nem sabia em que língua elas estavam
escritas! Olhava para os termos impressos no claro português de
Francisco Rodrigues Lobo, como se olhasse para hieróglifos enigmáticos.

Entretanto, o dedo insistia, insistia com uma tal firmeza, com tal
obstinação, que Claudino acabou por compreender:

``Socorre Lourenço e põe os olhos no seu exemplo''.

Não bastava compreender: seria preciso ler em voz alta, porque o dedo
continuava da primeira à última palavra, apontando sempre aquela mesma
linha em toda a página de prosa cerrada.

Claudino conseguiu por fim articular sílaba por sílaba, numa voz que não
lhe parecia a sua, a esquisita frase apontada pelo dedo do velho --- que
então se desvaneceu, deixando aberto sobre a mesa o livro oferecido pelo
autor a sua majestade o senhor d. João V.

E nada mais de anormal se passou na biblioteca de Claudino, nessa noite
de assombro. Tonto, e duvidando sempre da sua imaginação, correu a
escancarar a janela e a debruçar"-se para a ma. Queria companhia; a
solidão apavorava"-o. Embaixo, os varredores, como sombras de bruxaria,
moviam as compridas vassouras espalmadas, de raízes secas. Dois deles
desciam a encosta com a posição da vassoura invertida, esgalhando para o
ar, como dedos descarnados de mãos enormes, a sua rama nua. Um nevoeiro,
muito delgado, envolvia"-lhes as figuras negras, desmaterializando"-as; em
todo caso aquele \emph{rus}, \emph{russ}, \emph{russs}, das vassouras
municipais e aqueles vultos de gente viva, asseguraram uma relativa
tranquilidade aos nervos do Claudino. O ar fresco dissipou"-lhe
também um pouco o terror das impressões e ele conseguiu refletir um
momento:

Quem seria esse Lourenço, a quem deveria levar auxílio, e qual a espécie
de auxílio? Pôs"-se então a ver se descobria no vasto círculo dos seus
conhecimentos, alguém com um tal nome. Mas, nem parente, nem amigo, nem
conhecido! O acertado seria voltar para dentro e procurar uma
elucidação. Mas hesitou. Havia alguma coisa que o prendia à janela\ldots{} Um
dos varredores ousou cantarolar a meia voz, soturnamente, para não
acordar os que dormiam, e ele teve ímpetos de lhe gritar que cantasse
mais alto, que despertasse toda a gente daquele sono, que parecia o da
morte\ldots{} Medo? estaria ele com medo aos trinta e três anos, não tendo
nunca antes em sua vida experimentado semelhante impressão?

Aos sete, lembrava"-se bem, ia a qualquer quarto escuro sem temor e
àquela mesma biblioteca quantas vezes subira sozinho à noite, por ordem
do pai, a buscar tal ou tal livro, sem que lhe tremesse nas mãos o
castiçal com a vela? Se ao menos pudesse coordenar agora duas ideias que
o orientassem! Mas, ao contrário disso, os seus pensamentos, desligados,
voltavam"-se para ele na forma ansiosa da interrogação.

A voz soturna do varredor, cantarolando na sua melopeia monótona, de
baixo profundo, dilatava ainda mais a sua impressão de mistério e de
sofrimento. Olhou para as costas e para as palmas das mãos, fixou as
unhas, bateu de novo rijamente no peito, como a verificar se era
realmente ele quem estava ali, dentro daquele enredo. Estaria doido? por
que razão? a sua saúde era excelente\ldots{} os seus negócios corriam bem\ldots{}
os seus amores não o atormentavam\ldots{} Não; ele não estava doido: vira,
ouvira e lera aquelas coisas singulares em perfeito estado mental.
Entretanto, se no dia seguinte fosse contar a algum médico o que se
acabava de passar, que diria o tal médico? --- Nervos! e indagaria sem
tardança se teria havido alguns casos de loucura na sua família\ldots{} E
assim como ao médico, se contasse essa história aos seus amigos, todos
se ririam dele e sempre que o cumprimentassem não se esqueceriam de lhe
perguntar pela sua alma do outro mundo\ldots{} Seria ele sonâmbulo? Não.
Tinha bem nítida a consciência de se ter levantado da cama perfeitamente
desperto, de ter ouvido dar volta à chave do terraço e ouvido passos
pela biblioteca. E a lâmpada? e aquele dedo curto, engelhado, de unha
cortada rente e redondinha como a escama de um peixe?

Que extravagante visão\ldots{}

O homem calou"-se. Uma carroça rodou na calçada, e as sombras dos
varredores sumiram"-se no fundo da rua enevoada.

Claudino atirou"-se na cama como um fardo.

\asterisc

Aquela casa da rua do Costa Bastos era herança de família, que toda
morara ali desde os tempos do velho desembargador Aleixo, duas vezes
ministro de sua majestade o Imperador e pai do juiz dr. Cláudio, morto
de uma congestão ao formular uma sentença famosa. Não havia mulheres em
casa; Claudino sublocara o primeiro andar a um alfaiate
italiano, reservando para si e seu criado Antão os compartimentos
superiores. Pouco lhe importava, de resto, que a casa fosse alegre ou
triste: não a ocupava senão para mudar de roupa ou para dormir.
Associado numa casa comissária alemã, saía para a rua todas as manhãs às
oito horas, sendo sacudido do seu sono às sete pelo seu velho criado.
Era rijo de carne, de ar desanuviado, não acreditava em Deus nem no
diabo; mastigava bons bifes com os seus trinta e dois dentes naturais,
tinha sempre mulheres apaixonadas por si e começava exatamente agora a
sentir"-se enleado por uma intriga de amor, deliciosa.

Quando nessa manhã o Antão foi chamá"-lo, ele deu um berro, que o
deixasse dormir. O criado não se convenceu. Tinha ordem para, em casos
de relutância ir até ao beliscão. Até ao murro. Insistiu, até ver o
patrão arregalar os olhos.

O senhor hoje custou a acordar\ldots{} pois olhe, ontem até se deitou mais
cedo do que o costume\ldots{} Eu também sou assim: quanto mais durmo mais
vontade tenho de dormir. Seu café está frio\ldots{}

--- Tive um pesadelo horrível, Antão.

--- Alguma coisa que o senhor comeu lá fora.

--- Não comi nada.

--- Então foi fraqueza\ldots{}

--- Qual fraqueza!

--- O senhor devia estar mesmo incomodado, para se esquecer do lampião
aceso, e da janela aberta!

--- Hein?!

--- A lâmpada da secretária\ldots{}

--- Cala"-te! Vai aquecer o café\ldots{}

Então não tinha sido um pesadelo!\ldots{} A lâmpada\ldots{} Toda a cena noturna se
reproduziu no seu espírito e ele saltou para o escritório, a verificar
se lá estaria também o livro em cima da mesa\ldots{} Sim, tanto o livro em
cima da mesa como a estante escancarada de par em par! Claudino esfregou
os olhos, curvou"-se para a página do velho volume de Rodrigues Lobo e
logo a sua vista caiu em cima desta frase:

``Socorre Lourenço e põe os olhos no seu exemplo''.

Quando o Antão voltou com o café, encontrou o patrão com cara de idiota.

--- Dize"-me cá, Antão. Lembras"-te de algum amigo meu com o nome de
Lourenço?

Não, senhor. Só se for seu Gil.

--- Ora, que ideia! Gil é Gil, Lourenço é Lourenço!

--- Então, seu Braga.

--- Esse é Ernesto\ldots{} Outra coisa: não ouviste nenhum rumor esta noite,
na vizinhança do teu quarto?

--- Homem, parece que o alfaiate esta noite cozeu na máquina até que
horas! O senhor também ouviu?

--- Não. Eu não ouvi nada. Hoje não vou ao trabalho. Prepara almoço para
mim, depois hás de ir levar uma carta lá ao negócio.

Antão espantou"-se, calado. Era a primeira vez que tal acontecia em dia
de semana. Claudino tornou a ler a frase, examinou em silêncio todos os
cantos da casa, e, ao mesmo tempo que fazia a sua toalete, indagava da
memória ingrata onde teria ele jamais conhecido um Lourenço. Com o
espírito mais repousado, sabia agora perfeitamente ter estado, na
véspera à noite, em casa de uma família burguesa e pacata, onde bebera,
por satisfazer certas instâncias, uma xícara de leite, mugido, com
certeza, de uma vaquinha também pacata e mansa\ldots{}

Em todo caso tomou na sua carteira a data do dia e anotou a
singularidade da aventura noturna em que se sentira envolvido. Deliberou
depois ir consultar um médico amigo, embora tivesse receio do
ridículo\ldots{}

Foi só ao finalizar o almoço, ao descascar a sua deliciosa seleta, que
ele se lembrou de repente de um rapaz alto e pálido que vira por duas ou
três vezes em casa de Cora e que, se bem se recordava, parecia ter esse
nome, embora o denominassem quase sempre pelo apelido.

Era um sujeito meditativo, dado a leituras filosóficas e pianista
distinto. Mas que motivo poderia haver para que ele se interessasse por
uma criatura com quem mal tinha trocado meia dúzia de frases e cuja
existência lhe era indiferente? Não, não podia ser esse. Se algum
Lourenço havia a quem devesse socorrer, teria de ser um outro!

Chamou o criado e recomendou"-lhe que lhe remexesse a carteira, os
bolsos, as gavetas, a casa, a ver se encontrava algum cartão de visita
ou carta assinada cora esse nome.

E, enquanto o Antão mergulhava os dedos nodosos nas algibeiras do seu
\emph{smoking} e da sua casaca, ele pensava de si para si, se não
estaria prolongando estupidamente uma aventura a que a luz do dia
deveria ter posto o ponto final.

``Fui vítima de uma alucinação. Quem tirou o livro da estante e acendeu
a lâmpada fui eu\ldots{} sonhando, talvez, e tornei a deitar"-me\ldots{} acordei
sob a impressão de um sonho\ldots{} e tudo mais decorreu disso mesmo. Não foi
outra cousa\ldots{} Em todo caso, já agora, por curiosidade, se encontrar o
endereço do tal Lourenço, irei procurá"-lo. Não sei como lhe hei de
explicar a visita, mas há de ocorrer"-me um pretexto, se, com a presteza
com que lhe levar o socorro, ainda o encontrar vivo!\ldots{}''

O endereço apareceu, depois de muito trabalho.

Escovando"-se para sair, Claudino ia dizendo:

--- Tu és um grande descobridor, Antão. Se eu estivesse de veia,
levava"-te ao bispo, para crismar"-te com o nome de Cristóvão Colombo\ldots{}
Dá"-me o meu chapéu\ldots{}

--- O senhor leva a chave da porta?

--- Não. Espera hoje por mim; faze a tua cama na biblioteca, quero que
observes o meu sono. Se me vires levantar de noite, sacode"-me com força;
entendeste?

--- Entendi\ldots{}

--- Janto fora. Com que, então, este senhor Lourenço R. Trigoso, mora no
Engenho de Dentro?\ldots{} É longe como diabo\ldots{} Pois vou lá!

Claudino desceu a escada, mas, antes de pôr o pé na rua, chamou o Antão
para uma outra ordem:

--- Olha: deixa também na biblioteca um regador cheio d'água fria. Se eu
não acordar aos teus safanões, molha"-me sem piedade!

\section*{\textsc{ii}}

O automóvel ia a toda velocidade. Sentindo o ar bater"-lhe no rosto,
Claudino desanuviava"-se de pensamentos enervantes. Teria tempo de
cogitar nas suas fantasmagorias quando chegasse ao seu destino; preferia
entreter agora a imaginação com os seus negócios, que iam bem, e os seus
amores, que também não iam mal.

Tinha exatamente no bolso, havia já dois dias, uma carta de Cora,
marcando"-lhe uma entrevista para a noite imediata, dando aquele passo,
dizia ela, animada pela ausência do marido, que partia para o Sul a
escolher gado para uma das suas propriedades rurais. Abençoada ideia de
homem! que todos os touros, carneiros e cavalos que ele adquirisse se
multiplicassem assombrosamente, em seus pastos, nos mais numerosos e
belos exemplares das raças respectivas. Para ele bastava"-lhe a glória e
a delícia de beijar"-lhe a mulher; e ora isso não era fortuna de que
outro qualquer se pudesse gabar. Cora, com toda aquela estonteadora
beleza de loira, que Deus lhe dera e o diabo lhe acrescentara com as
suas artes, era de uma honestidade proverbial na própria roda dos
maledicentes. Ele mesmo não se sabia explicar a facilidade daquela
conquista. Havia só dois meses que frequentava a casa dela e já há
muitos dias as suas mãos se encontravam em movimentos furtivos e os seus
pés se procuravam embaixo da mesma mesa em que o marido, cabeludo c
trigueiro como um turco, fazia e desfazia paciências, muito calado,
ruminando cifras.

Coubera"-lhe a ele, Claudino, a dita de quebrar o gelo daquela mulher
virtuosa e de lhe dar frêmitos de desejo ao corpo esguio e branco como
os das virgens das iluminuras. Raciocinando um pouco, isso afinal não o
devia admirar, visto que a sua modéstia não ia até à tolice de
desconhecer o valor do seu espírito e o garbo das suas exterioridades.
Seria mesmo um imbecil se se não considerasse um bonito rapaz e ainda
com a circunstância favorabilíssima de se vestir sempre no melhor
alfaiate do Rio de Janeiro. A sua perspicácia era também bastante aguda
para perceber que todas as mulheres, inclusive a doce Cora, olhavam para
as suas casimiras com ar de absoluta aprovação. Apesar, porém, de todas
essas vantagens, poderosíssimas, ele não ousara esperar que o beijo
definitivo dessa senhora honesta viesse tão depressa\ldots{} Cora mostrava
saber aproveitar com afã as ocasiões, ciência que muita gente ignora; e
nisso revelava ainda ser uma mulher superior. Um triunfo em toda a
linha, aquele, mas que viria depois? Como acabaria? Por mais que o
seduzissem certas intrigas, tinha"-lhes sempre medo do fim. De mais a
mais o marido de Cora era um homem de fisionomia medieval, trescalante
a ciúme\ldots{} Enfim, fosse Deus servido da melhor fôrma, o que lhe
importava agora era reformar, para as suas entrevistas de amor, aquela
casa da rua Costa Bastos, forrá"-la de novo, pintá"-la de fresco. Seria
isso até um magnífico pretexto para se desfazer de alguns móveis velhos
que por preguiça conservava em casa, como aquela almanjarra da
biblioteca, carregada de livros e de traças, e como o oratório, em que a
irreverência do Antão metia as caixas dos seus charutos e as latinhas de
graxa para os seus sapatos! Mas esse plano desfez"-se à lembrança de que
a linda Cora não saía sozinha, vendo"-se por isso na contingência, pobre
mártir, de marcar as suas entrevistas de amor para a própria casa.
Contanto que o diabo do marido não aparecesse\ldots{} O que seria preciso era
influi"-lo a ir escolher animais todos os meses às feiras do Sul, e que
ficasse por lá, pelo menos quinze ou vinte dias de cada vez\ldots{} porque
isso de se sujeitar ao papel ridículo do esconde"-esconde em casa alheia,
não estava de acordo com os seus hábitos. Gostava do fruto proibido,
sim, mas saboreado com sossego, delicadamente, demoradamente.

Refastelando"-se ainda mais no banco, Claudino lembrou"-se que no dia
seguinte teria de ir mais cedo para o escritório, obrigado pela urgência
do correio do Norte, aquela estopada de tantas cartas parecidas. Na
verdade, a vida só era boa como ele a levava naquele instante: correndo
aventuras na asa do vento. Seria entretanto prudente ir imaginando o que
houvesse de dizer a esse tal Lourenço, se o encontrasse em casa. O que
lhe não confessaria, nem a pau, era a razão extravagante, mesmo absurda,
da sua visita. Mas que motivo arranjaria ele, homem de comércio, para se
apresentar pela primeira vez numa casa estranha, em dia útil e a horas
de trabalho?

Ainda era tempo de voltar para trás: e estava quase resolvido a fazê"-lo,
quando o automóvel parou em frente a um portão de ferro entreaberto. Era
ali.

--- Entro? não entro? vacilava Claudino, embaraçado: --- Afinal, é
ridículo isto que eu estou fazendo\ldots{} mas já agora, que diabo, o melhor
é ir até ao fim.

A sua convicção e a sua esperança era saber logo ao primeiro toque de
campainha que o sr. Lourenço não estava em casa. Mas enganou"-se. O sr.
Lourenço estava.

Um jardineiro, engelhado e baixinho, correu à entrada, e, mesmo sem
indagar de quem se tratava, afirmou que sua senhoria podia entrar ---
como se já estivesse prevenido da sua vinda!

A casa, alta, cor de oca, de dois andares, ficava dentro, ao fundo de
uma rua de bambus, cujas pontas se arqueavam em cima, formando túnel.
Era uma sombra deliciosa por toda a extensão de uns cinquenta metros.
Caminhando por ela, Claudino pensava a cada passo:

--- Mas que irei eu dizer ao homem\ldots{} Ora, já se viu uma cousa assim?!
Mau! de mais a mais há cães em casa!

Efetivamente, ao desembocar da rua dos bambus, viu no patamar dos quatro
degraus de pedra da entrada, um cão ruivo, que, sentado sobre as patas
traseiras, fixava nele um par de olhos redondos, mais semelhantes aos
olhos das focas que aos dos cães.

Era um animal esquisito, de pelo eriçado e duro como os das escovas de
arame. Claudino hesitou um momento em passar por ele; mas, vendo a sua
impassibilidade, subiu a escada, pensando:

--- É extraordinário, nem mesmo o cão se alvoroça com a chegada de um
estranho\ldots{}

E, aproximando"-se da porta, ia empurrar o botão da campainha, quando viu
aparecer, através dos vidros, uma senhora alta, de bandos brancos, que
veio depressa ao seu encontro e disse, em ar de censura, antes mesmo de
lhe ter ouvido a mais leve pergunta:

--- Como o senhor tardou!

--- Minha senhora\ldots{} eu\ldots{}

--- Agora, não perca tempo. Faça o favor de entrar.

--- Haverá talvez alguma confusão. Tomam"-me, naturalmente, por uma outra
pessoa\ldots{}

Ela devia ser surda, porque respondeu:

--- O quarto dele é no segundo andar. Venha comigo. Contanto que o
senhor ainda chegue a tempo!

--- Julgará V. Ex.\textsuperscript{ª} que sou médico?!

Ela não respondeu; tomou"-lhe o chapéu e a bengala das mãos, suspirando:

--- Foi uma noite horrível!

--- Talvez que V. Ex.\textsuperscript{ª} pense que sou tabelião?\ldots{}

--- Não percamos tempo, observou ela, sem o atender, e guiando"-o através
de um corredor largo até à escada do pavimento superior. Aí, fazendo"-lhe
sinal para que subisse na frente, aconselhou:

--- Tenha cuidado com os degraus, que são um pouco traiçoeiros, sr.
Claudino.

Sr. Claudino\ldots{} Sr. Claudino! Como poderia aquela senhora, que ele
jamais vira em dias de sua vida, saber assim o seu nome?! Continuaria a
ilusão? Estaria ele doido varrido, ou dentro ainda de um sonho? Para ter
uma sensação da realidade, beliscou disfarçadamente uma das coxas.
Doeu"-lhe. Não há dúvida, sou eu, pensou Claudino, estacando.

--- Já agora não o faça esperar; tenha paciência, sim? disse a seu lado
a senhora dos bandos. E a sua voz estava impregnada de uma tal tristeza
e doçura que o moço subiu a escada toda em um só fôlego. Fosse o que
fosse, estava morto por saber a verdade daquela história, ainda inédita
nos anais da sua vida.

A escada desembocava em cima numa saleta circundada de portas e de
armários. Uma mulatinha magra revolvia roupas brancas num deles, sem nem
ao menos ter a curiosidade de voltar a cabeça, para ver quem passava.
Desse modo, Claudino varejava pela primeira vez todo o interior daquela
casa alheia, sem despertar a atenção dos seus moradores, como se lhes
fosse familiar! Penetrava agora num outro corredor, ao fundo do qual se
abria uma janela para a galharia em flor de um grande pé de hibiscos. De
cada lado desse corredor havia duas portas; foi na segunda da direita
que a senhora dos bandos parou, fazendo"-lhe o gesto de esperar. Ele esperou.

Começava a sentir"-se nervoso, impaciente. Ardia por acabar com aquilo e
voltar para a sua vida interrompida. Não era homem para mistérios e
enredos sobrenaturais; toda a sua ufania era de ser um bom animal, de
carnes rijas e espírito lavado de preconceitos.

No mesmo minuto em que a senhora reapareceu, abrindo a porta do quarto
para que ele entrasse, assaltou"-o de chofre a ideia de estar sendo
vítima de um embuste qualquer. Que o esperaria dentro desse quarto?
Teriam tido aquele trabalho todo para lhe roubarem o relógio e uma
carteira insignificante? Nesse relance passou"-lhe pelo espírito a
probabilidade de um assalto, de um assassínio, ou imposição da sua
assinatura para fins odiosos; mas, a lembrança do dedo do velho, furando
a luva de neblina que envolvia toda a mão para apontar imperativamente
uma frase, ordenando que se socorresse um homem, afastou"-lhe tais
pensamentos da cabeça. E agora já não havia tempo para suposições. Era
ver e ouvir.

--- Faça o obséquio de entrar\ldots{}

O quarto tresandava a iodofórmio, e ainda por cima estava com as
venezianas fechadas. Claudino tropeçou num tapete, depois num banquinho
de pés. Trazia os olhos cheios da luz do sol, não via nada na meia
escuridade. Alguém lhe segurou no braço, pelo cotovelo e o levou até a
uma poltrona antiga, onde o fez sentar"-se. Era ainda a senhora dos
cabelos brancos que o dirigia.

Foi preciso que ele tivesse descansado alguns minutos em silêncio, para
que pouco a pouco pudesse distinguir os objetos que o cercavam. Estava
em frente a uma cama de ferro dourado onde um moço repousava, recostado
em almofadões. Claudino reconheceu
nesse moço o pianista da casa de Cora e já com isso não sentiu surpresa,
de tal jeito os sucessos se iam desenrolando.

No fim, tudo se há de explicar, pensava, voltando a cabeça para um dos
ângulos do quarto, onde, a um cochichar de vozes femininas, se juntava o
rumor d'água despejada pelo gargalo de uma moringa. Notou então com
estranheza, que nesse canto do quarto, como nos outros três, havia
grandes blocos de gelo em bacias de ágata. As vozes eram da senhora de
cabelos brancos e de uma moça pálida, de trança solta e avental azul mal
atado sobre o vestido claro. Eram ambas da mesma estatura e ocorreu a
Claudino que fossem mãe e filha. A presença daquelas duas mulheres
tranquilizava"-o. A mais nova, tendo enchido um copo d'água, veio
oferecê"-lo ao doente, com todo o jeito e carinho. Depois de ter bebido a
água, Lourenço ordenou, com um gesto de cabeça, às duas senhoras que
saíssem, e as duas caminharam para a porta; mas a moça voltou de lá
ainda a alisar os lençóis da cama, a olhar de perto, longa,
amorosamente, para o rosto de Lourenço; depois, fixou Claudino como a
pedir"-lhe piedade e saiu fechando a porta sobre si.

``Cheguei ao fundo do poço; é agora'', pensou Claudino consigo. E disse:

--- Às suas ordens, sr. Lourenço!

--- Obrigado. Espero não perder palavras, mas primeiro jure que não
repetirá a ninguém o que me vai ouvir.

Jurar! Até a fórmula empregada pelo doente tinha um sabor novo para
Claudino. Repugnando"-lhe fazer um juramento, ele calou"-se, embaraçado. A
voz do outro, embora fraca, era tão clara, tão limpa, tão bem
articuladas lhe saíam as palavras dentre os beiços desmaiados, que seria
impossível perder"-lhe uma única sílaba. Procurando evitar a solenidade
de um compromisso, cujo valor não podia ainda avaliar, Claudino
perguntou:

--- Não é para furtar"-me a servi"-lo, mas desejaria saber por que e para
que vim eu aqui. Estou agindo sob o poder de um mistério que não
compreendo!

--- Ninguém pode compreender mistérios\ldots{} Mas vá primeiro dar volta à
chave daquela porta, não quero que ninguém nos ouça. Ninguém! A menina
que acaba de sair deste quarto é minha prima e minha noiva\ldots{} \emph{Era
minha noiva}, eu já devo falar da vida como de um assunto passado\ldots{}
Faça favor: examine bem tudo; devemos estar sós. Minha tia é surda, mas
a curiosidade é capaz de fazer o prodígio que a medicina e o tempo não
conseguiram fazer. E se ela me ouvisse, não teria depois razão para
chorar por mim\ldots{} Ninguém?

--- Ninguém.

--- Aproxime"-se e esforce"-se por entender"-me, mas, já que não quer
jurar, dê"-me a sua palavra de honra de que não referirá a ninguém o que
me vai ouvir.

--- Dou"-lhe a minha palavra de honra.

--- Escute: vimo"-nos duas vezes: a primeira, cruzando"-nos numa sala onde
eu entrava, quando o senhor saía; a segunda, na mesma sala em que o
senhor conversava alegremente, enquanto eu tocava músicas após músicas,
a pedido de uma mulher\ldots{}

--- Não me esqueci\ldots{} O senhor nesse dia tocou uma sonata, cujas
harmonias me ficaram na memória\ldots{}

--- Era a \emph{Sonata à minha noiva}, inspirada por esta pobre
criança\ldots{} Foi a última vez que a toquei\ldots{}

--- Há de tocá"-la outra vez no dia do seu casamento, se não for ainda
mais cedo\ldots{}

--- Nunca mais! Olhe:

E Lourenço, afastando com a mão esquerda a colcha de sobre o busto,
mostrou o braço direito ligado no pulso. Tinham"-lhe amputado a mão.
Claudino não pôde reprimir um movimento de horror.

--- Escute. Não podemos perder tempo. A minha história é esta:
Solicitado por uma mulher casada, fui ontem à meia"-noite a uma
entrevista de amor em sua casa, e justamente no instante em que,
pousando a mão no parapeito do seu terraço, eu ia saltar para dentro, vi
reluzir na treva o raio branco de uma lâmina e senti que alguém de
dentro me decepava mão.

--- O marido!\ldots{}

--- Fosse quem fosse, pouco importa. Eu não posso condenar ninguém. Caí
na rua e arrastei"-me de bruços até aos trilhos dos \emph{bonds}, com a
intenção ou de morrer de uma vez esmagado pelo primeiro carro que
passasse, ou de, sendo encontrado ali, ver atribuído a um acidente
vulgar o meu desastre.

--- Salvava assim a reputação de uma mulher

--- Salvava a minha. Foi na minha que pensei. A figura da minha noiva
apresentou"-se"-me diante dos olhos como um remorso. Nunca a amei tanto.
Felizmente, ela acreditou\ldots{} Dê"-me água.

Claudino tremia ao deitar água no copo.

--- Obrigado. Sentindo esvair"-me em sangue, bem vizinho da morte, sabe
no que pensei? Que na mão que me tinham cortado, ficava um anel com o
nome da minha noiva e a data do nosso primeiro e único beijo. Então, uma
indignação sacudiu todo o meu corpo e o meu pensamento voou para o
senhor.

--- Para mim?! por que para mim? Mas isto tudo é um pesadelo!

--- Há muita coisa na vida, que parece sonho; se é que tudo não é sempre
irrealidade! Pensei no senhor como a única pessoa capaz de ir buscar
esse anel e trazê"-lo à minha noiva, ou a mim, se eu ainda estiver vivo.
Ficando em poder do meu assassino, esse anel se me afiguraria como uma
ameaça ao sossego da minha memória. Ele seria em todo o tempo testemunha
de uma verdade, que precisa ser enterrada comigo. Compreende?

--- Compreendo.

--- Lembro"-me confusamente do resto: um automóvel da Assistência,
médicos discutindo ao redor de mim, e a minha chegada aqui à casa de
minha tia\ldots{} Conscientemente às vezes, e inconscientemente outras vezes,
eu chamava pelo seu nome, como pelo da única pessoa capaz de salvar"-me.
Agora já sabe tudo.

--- Não sei nada. Falou"-me em ir buscar um anel. Onde? Como?

--- Em casa dela.

--- Mas onde é essa casa e quem é ela?

--- Cora. Pois não tinha adivinhado?

--- Cora?!

--- Não a condene e acredite que ela é uma mártir\ldots{} Só por muito amor
consentiria em ser minha\ldots{} Só por muito amor! Leve"-lhe a minha
carteira, onde há uma carta sua, que não deve ser lida pela minha noiva.
Percebe agora por que pensei no senhor? Cora dizia sempre que o estimava
como a um irmão. Vá vê"-la, console"-a, que ela também deve ter sofrido
muito\ldots{} entregue"-lhe a carta de que lhe falei e peça"-lhe o meu anel, em
que está gravado o nome de Beatriz, vítima inocente de toda esta
tragédia\ldots{} Mais água!\ldots{} O senhor está tiritando\ldots{} é também por causa
do gelo com que encheram o quarto\ldots{} Vá, não perca tempo. Eu já não
tenho forças\ldots{} se pudesse trazer também a minha mão!\ldots{} Que saudades eu
teria do piano se continuasse a viver!\ldots{} A minha carteira está embaixo
do meu travesseiro\ldots{} leve"-a\ldots{} É essa\ldots{} adeus\ldots{}

Lourenço desmaiava, lívido, rolando a cabeça no almofadão. Claudino
correu a abrir a porta e logo deparou com a noiva do pianista, em pé, de
encontro à parede fronteira, à espera de poder entrar. Depois de
assegurar"-se de que o doente voltava a si, Claudino desandou todo o
caminho por que tinha andado, até aquele segundo andar. Embaixo, a
senhora dos bandos brancos, que ele sabia agora ser a dona da casa,
correu ao seu encontro, resumindo numa palavra inexpressiva toda a sua
curiosidade ansiosa:

--- Então?!

Claudino gritou:

--- O mal não me parece de morte. Há muita gente por aí que vive gorda e
satisfeita depois de ter passado por desgraças idênticas. V.
Ex.\textsuperscript{ª} deve ter encontrado muitas vezes, por essas ruas,
pessoas a quem falte uma perna ou um braço e que, entretanto, não
parecem mais tristes do que qualquer de nós\ldots{}

--- Tem razão; mas serão pessoas de outra natureza. Meu sobrinho é
excessivamente impressionável. O senhor voltará a vê"-lo?

--- Sim\ldots{} logo à noite, ou amanhã de manhã; entretanto desejaria que me
informasse de uma cousa:

--- Pois não\ldots{}

--- Quem foi que me mandou chamar?

--- Quem foi que o mandou chamar?!

--- Sim, porque evidentemente V. Ex.\textsuperscript{ª}, como todos da
sua casa, esperava por mim!

--- Certamente.

--- Ora, como eu hoje não fui à minha casa comercial e ignoram aqui o
meu domicílio particular, calculei que alguém da sua família tivesse
mandado para o meu escritório um recado, que eu não recebi\ldots{}

--- Não. Nós todos supúnhamos em casa que Lourenço o tivesse avisado por
meio de qualquer médico ou empregado da Assistência, visto que desde que
chegou participou"-nos a sua visita, continuando a falar no seu nome com
ansiedade crescente. Mas se o senhor não recebeu nenhum chamado, como
veio até cá?

--- Vim por acaso.

--- Hein?!

Claudino retraiu"-se. Talvez não fosse conveniente dizer tudo. Inventou
então a combinação de uma entrevista marcada por Lourenço havia dias e
saiu apressado e nervoso.

No terraço, o mesmo cão ruivo, eriçado, de olhos de foca, deixou"-o
passar sem nem sequer fazer ouvir uma rosnadela. Os bambus moviam"-se,
agitando sombras na areia do jardim, e por uma esquisitice da imaginação
torturada, Claudino, dando ao \emph{chauffeur}
o endereço de Cora, pensava na figura negra do varredor das ruas, alta
noite, arranhando nas pedras a sua espalmada vassoura de raízes e
cantarolando soturnamente aquela toada monótona de que não entendera as
palavras, mas que lhe ficara no ouvido.

\section*{\textsc{iii}}

Tudo agora, na rua, parecia esquisito e artificial aos olhos de
Claudino. Os tipos mais vulgares, as cenas mais comuns, acordavam"-lhe
impressões curiosas.

Estaria ele de fato no Rio de Janeiro, no meio de sua gente?

Sim, não podia haver dúvidas. Ainda poucos minutos antes tinha passado
entre aqueles dois cartazes da Caxambu e da Brahma, e vira naquela
padaria da esquina o mesmo homem barrigudo, de nariz erguido, cheirando
o ar da sua porta. Para comprovar a identidade do lugar, bastava"-lhe, de
resto, olhar para aqueles soberbos renques de palmeiras imperiais,
alinhadas paralelamente por detrás do muro de uma chácara sua conhecida,
e ver, ao fundo da planura do bairro, os recortes da serra da Tijuca.

A verdade é que ele vinha do fundo de uma aventura absurda, como se
tivesse vindo do porto de um outro mundo. A realidade afigurava"-se"-lhe
mais irreal do que a irrealidade lhe parecera extravagante. Os homens
por que passava tinham todos a expressão inconsciente de autômatos,
agindo ao influxo de um só sopro: o egoísmo, em direção do mesmo ideal:
o dinheiro. E esse ideal comum uniformizava todas as fisionomias. Não
havia por ali caras nem raças diferentes: havia almas iguais, almas
feitas às dúzias, como os casacos, que o seu inquilino alfaiate da rua
Costa Bastos cortava, de uma assentada, em camadas sobrepostas de pano,
para o seu comércio de fancaria\ldots{}

O sol das cinco horas coloria tudo de amarelo. Havia fachadas de casas,
que lembravam caras de cadáveres; árvores de calçada carregadas de
moedas de ouro, representadas por discos de luz movediça. Moleques e
garotos, jogando o botão, acocorados nas pedras, guinchando como
macacos, discutindo e esbordoando"-se mutuamente, reproduziam a mesma
ideia dominadora e essencial dos adultos, menos demonstrativos, mas
igualmente possessos. O mesmo veneno, que circulava nas veias dos
indivíduos, como que se transmitia às coisas inertes: nas tabuletas das
casas comerciais as letras dos dizeres ressumavam, de inchadas, o suor
da ganância, pela qual a humanidade gasta as suas melhores forças,
combatendo até ao último suspiro. Por quê? para quê? Para servir a
matéria. Para regalar o corpo\ldots{} A algumas janelas assomavam cabeças de
meninas, enfeitadas de fitas. Era ainda a tabuleta insincera e ridícula.
Dentro daquelas cabeças só deveria existir uma ideia: atrair o marido,
como nas tabuletas só havia a de atrair o freguês. Uma mulher carregando
o filho ao colo e ainda um fardo à cabeça, deu"-lhe a impressão de uma
força muscular admirável, ao serviço do mesmo ideal abjeto. No mundo
vasto havia lugar para outras aspirações: por que não ia aquela mulher
para o campo, roer ervas como os cabritos e dormir à sombra de árvores
cheirosas? Assim não morreria de fome, evitaria sacrifícios deprimentes
e a sua alma teria espaço para agitar"-se e crescer até às estrelas, que
palpitam no céu aveludado e assombroso. Pela primeira vez comparou a
agremiação das cidades aos pátios dos hospícios, onde todos os loucos
fossem atacados da mesma mania sob formas diferentes. De fato, desde a
criança ainda inconsciente, que estende a mão tentando segurar qualquer
objeto que brilhe diante de seus olhos, até ao velho moribundo, que no
último momento de vida procura ainda apanhar nas bainhas dos lençóis ou
no espaço coisas que só ele vê, a mais enraivecida e poderosa vontade
dos homens é --- possuir. Essa ideia escraviza"-os, torna"-os em máquinas,
verdadeiros moinhos de carne e osso, em que as ideias se trituram para o
pão da sua fome fenomenal.

Pobres fantasmas esses, que ele via, caminhando pelas calçadas daquele
bairro. Nunca a população da rua se mostrara aos seus olhos tão
mesquinha nos seus trajes, nem tão cansada nos seus passos. Sentia, por
vezes, a impressão de que se descesse do automóvel e fosse tocar, embora
só com as pontas dos dedos, nessa gente, ela se desfaria em pó, como
certas árvores mortas, corroídas pelos vermes\ldots{}

Quem lhe asseguraria que na vida não fosse tudo ilusão? E sendo tudo
ilusão, por que haveria de existir o absurdo?

O \emph{chauffeur} voltou"-se:

--- O senhor disse rua do Uruguai?

--- Sim; rua do Uruguai.

Estavam perto da casa de Cora; daí a uns três ou quatro minutos ele
bateria à sua porta. E se ela não estivesse só? Se o hirsuto marido se
intrometesse no caso, cioso de saber qual o assunto daquela sua
conversa, tão esquisita e tão inesperada? Oh! a terrível complicação dos
maridos! E que fogueira seria essa, que tão minaz e devoradoramente
consumia aquele delicado e pálido corpo de mulher? Seria crível que ela
amasse dois homens ao mesmo tempo e a ambos fizesse simultaneamente a
mesma promessa?

Espicaçou"-o o desejo de comparar a carta escrita por ela a Lourenço,
àquela que ela lhe dirigira a ele próprio. Nada mais fácil. Trazia"-as
ambas no bolso. Esboçou o gesto de tirar a carteira de Lourenço da
algibeira em que a enterrara, mas não levou a ação a efeito. Para quê?
Iguais ou diferentes, as duas cartas voltariam juntas para a mão que as
escrevera. Gêmeas ou não, eram filhas ambas do mesmo sentimento
enganador. De resto, Cora era"-lhe agora insuportável. A sua
espiritualidade, que o seduzira, transformava"-se subitamente numa
materialidade suína, que lhe dava náuseas. Não era uma mulher
apaixonada, era uma mulher doente. Não devia ser submetida ao regímen
dos beijos de amor, mas ao das duchas geladas. Não lhe daria tempo para
defender"-se. Tratá"-la"-ia com aspereza, usaria para com ela frases curtas
e sumárias. Previa cenas terríveis, mas de nenhum modo se esqueceria de
que ia como delegado de Lourenço, e nada mais. Nada mais.

Contudo, ao entrar em casa dela teve um estremecimento. Lembrou"-lhe a
última visita, em que ouvira ao Lourenço aquela sua famosa composição à
noiva, ao mesmo tempo que bebia nas grandes pupilas de Cora, as mais
lindas e ardentes promessas de amor\ldots{}

O criado, antes de lhe abrir a porta da sala, fê"-lo esperar um momento
no gabinete do patrão, que tinha saído havia pouco, mas não deveria
tardar\ldots{}

Claudino respirou.

O compartimento não era grande e a dona da casa deixara nele, como
vestígio da sua passagem, o seu chapelão branco de jardinagem, que se
balançava de uma maçaneta da porta, suspenso pelos atilhos de fitas
cor"-de"-rosa.

Aquele chapéu evocou"-lhe uma recordação suave:

Fora numa tarde assim que ele tinha um dia passeado ao lado de Cora,
pela chácara, vendo"-a colher flores, e sentindo aquelas rendas brancas
roçarem"-lhe o queixo, de vez em quando, numa leve carícia
inconsciente\ldots{} Mas, logo um pensamento o mortificou: --- Teria ela tido
hoje a coragem de passear?

Voltando, atormentado, o rosto para o outro lado, viu, com sobressalto,
a lâmina curva de um alfanje, sobre o pano escuro da mesa.

``Foi com aquele alfanje que ele decepou a mão de Lourenço''.

E sem tirar os olhos do aço polido da arma, repetiu, de si para si, num
calafrio: --- ``foi com aquele alfanje que ele decepou a mão de
Lourenço\ldots{}''

O criado tardava em abrir a sala. Claudino, muito impaciente e nervoso
para esperar sentado, pôs"-se a passear de um lado para o outro. De
repente, parou em frente a uma grande lousa escolar, em que leu estas
palavras, escritas a giz:

\begin{quote}
\textsc{fábula persa}

``Há três coisas que nunca se obtêm por meio de outras três: riquezas
por desejos, mocidade por arrebiques, saúde por medicamentos; assim como
há três coisas, que três circunstâncias tornam valiosas: socorrer os
famintos quando têm fome, falar verdade quando se está irado, perdoar
tendo"-se o poder de castigar''.
\end{quote}

A mesma letra acrescentava, noutra linha:

--- ``O poder e o direito''.

A que mais abaixo respondia uma letra miúda, de mulher:

--- ``Poder sim. Direito não''.

Continuarei ainda por muito tempo dentro do enigma? perguntava a si
próprio Claudino, quando o criado voltou e o conduziu à sala de visitas.

O piano estava aberto e o ambiente saturado pelo aroma de um grande ramo
de murta. Espalhadas pelo tapete e acumuladas no sofá, sobrepunham"-se,
umas às outras, mais de vinte almofadas de seda. Os \emph{stores},
descidos a meio, deixavam entrar no aposento uma claridade leitosa, que
adoçava uma grande tela escura, emoldurada a prata, onde o rosto e as
mãos pálidas de Cora emergiam da lã, cor de chumbo, de um vestido de
inverno. Claudino deu as costas ao retrato e voltou"-se para a porta, à
espera.

Apenas uns segundos e a dona da casa apareceu, envolvida num roupão
marfim"-velho, que lhe descia em pregas, desde os ombros delicados até ao
chão. Não sorria, como nas outras vezes. Vinha séria, olhos orlados de
violeta. Os cabelos louros, presos na nuca por dois grampos de ouro,
iluminavam"-na com uma claridade de auréola. A poucos passos de Claudino,
talvez por lhe estranhar o aspecto rude e altivo, estacou, com ar
interrogativo.

Ele disse:

--- Compreende que só uma razão muito poderosa me obrigaria a vir à sua
casa a uma hora tão inconveniente e de um modo tão inesperado. Não
proteste; é isto mesmo. Preciso falar"-lhe e o que tenho a dizer"-lhe
di"-lo"-ei sem rodeios, bruscamente, porque assim é preciso. Sei do que se
passou esta noite no seu terraço. Venho, mandado por Lourenço,
trazer"-lhe uma carta e pedir"-lhe um anel\ldots{}

Não pôde continuar: Cora desfalecia. Para a não ver rolar no assoalho,
ele estendeu depressa os braços e amparou"-a. Pesava. Através das pregas
moles do vestido, sentiu"-lhe a carne do corpo sem colete, abandonando"-se
todo. Na esperança de que a vertigem fosse momentânea, Claudino não quis
gritar pelo auxílio de ninguém. Entretanto, Cora arrefecia, e uma
palidez marmórea se estendia por toda ela. Pareceria morta se dos seus
olhos semicerrados não descessem lágrimas grossas, que lhe escorriam
em fio pelas faces.

Claudino fê"-la sentar"-se, reclinando"-a nas sedas moles dos almofadões.
Uma grande, uma enorme tristeza se evolava daquela prostração. Que lhe
revelaria ela ao despertar? E se se fechasse obstinadamente no seu
silêncio e no seu segredo, como ousaria ele interrogá"-la?

Começava a ter pena daquele sofrimento, mas precisava saber de tudo, ir
até à impiedade.

Supusera tratar"-se de uma vertigem passageira, mas Cora ainda mal
respirava. E se, entretanto, o marido chegasse? Não lhe tinha dito o
criado que ele não deveria tardar? Talvez fosse prudente abrir uma
janela, fazer com que o ar renovasse aquela atmosfera saturada do cheiro
forte da murta e que entretinha, naturalmente, prolongando"-a, a síncope
de Cora. Ela respirava apenas, cada vez mais pálida e mais fria.

Claudino começava a impacientar"-se, a ter medo. Foi abrir a janela,
voltou a sentar"-se perto do divã. E se ela morresse? Via"-lhe os dentes
cerrados, --- as olheiras cada vez mais escuras\ldots{} Tornou a erguer"-se e
ia chamar alguém, quando ela voltou a si, desafogando"-se num suspiro.

--- Passou? perguntou"-lhe Claudino, quase solícito.

--- Passou\ldots{} respondeu ela num débil fio de voz.

--- Se eu pudesse adiar esta entrevista, creia que o faria sem a
relutância de um segundo; mas as circunstâncias obrigaram"-me à maior
urgência. Lourenço não pôde esperar\ldots{}

--- Lourenço?!

--- Sim\ldots{}

--- Mas\ldots{} como soube?

--- Isso pouco nos importa agora. Lembre"-se de que seu marido pode
chegar de um momento para o outro! O anel estará com ele?!

--- Não\ldots{}

--- Nesse caso dê"-mo depressa!

O tom de Claudino ia"-se tornando áspero. Cora fixou nele os olhos
inundados e, tirando de vagar um anel de ouro do dedo, ofereceu"-lho, sem
dizer uma única palavra.

--- É este? perguntou ele.

Ela acenou que sim, com a cabeça. Não podia haver dúvida; lá estava
gravado o nome da noiva de Lourenço, no interior do aro. Claudino não
tinha o gesto impaciente. Depois de ter examinado a joia friamente,
guardou"-a, com um movimento, vagaroso, na mesma carteira de onde tirou
duas cartas que entregou a Cora, observando a impressão que isso causava
nela, e dando"-lhe tempo a que se explicasse. Ela, porém, parecia não
entendê"-lo e recebeu as cartas maquinalmente, sem desviar os olhos dos
de Claudino, que se conservava calado, sufocando o desejo de interrogar,
de saber\ldots{} Por fim, vendo que ela não falava, ergueu"-se:

--- Agora, minha senhora, que está tudo acabado entre nós, permita que
lhe agradeça a sua atenção e me despeça\ldots{}

Uma nuvem passou pela fronte de Cora; depois, ela segurou nervosamente,
com ambas as mãos as duas mãos de Claudino e, levantando"-se rente a ele,
disse"-lhe perto da boca, precipitando as palavras:

--- Escuta; não me julgues mal; escuta! A carta era para ti\ldots{} Lourenço
veio em teu lugar\ldots{} era a ti que eu chamava, era a ti\ldots{}

--- Não!

--- Sim! olha para os meus olhos: verás que não minto. Eu nunca te
menti; é só a ti que eu amo, quero que saibas, quero que tenhas a
certeza. Escuta! Eu tinha acabado de escrever a Lourenço, pedindo"-lhe a
\emph{Sonata à minha noiva} e estava a escrever"-te, avisando"-te de que
poderias antecipar a tua visita, vir uma noite antes, quando ouvi os
passos de meu marido. Percebes a minha confusão? A portadora esperava as
cartas ao pé de mim. Assustada, troquei precipitadamente os sobrescritos
de ambas e entreguei"-as\ldots{} Oh! tu não acreditas e é a verdade, é a
verdade!

--- Mas nesse caso eu deveria ter recebido a carta de Lourenço\ldots{} em que
lhe pedia a sonata.

--- Sim!

--- Oh! mas isso é um subterfúgio e peço"-lhe que repare que eu não a
acuso!

--- Um subterfúgio? Não! Não te vás embora; escuta, espera, não me
condenes sem me ouvir. Foi horrível, quero que saibas tudo, mas preciso
falar depressa, depressa, porque \emph{ele} pode chegar e eu tenho medo.
Escuta: meu marido tinha resolvido partir ontem, à tarde, para o Sul, e
quando eu já o supunha embarcado, vi"-o entrar de novo pela casa.
Achou"-me pálida, notou certa perturbação em mim\ldots{} expliquei"-lhe que
era pela alegria de o tornar a ver\ldots{} Não sei se acreditou\ldots{} Disse"-me
então que vinha passar apenas algumas horas comigo, visto que o paquete
só levantaria ferro às sete horas da tarde. Estás compreendendo? Não me
olhes com esse olhar de dúvida! Escuta ainda: no primeiro momento de
liberdade corri a dizer à minha velha Antônia que podia ir levar ao seu
destino as cartas que eu lhe entregara e ela saiu. Eu não tinha nada a
recear. Meu marido deveria partir às sete horas da noite! À proporção
que a tarde avançava, eu ia"-me tornando mais nervosa, mais impaciente,
porque ele se acomodava como quem tencionasse ficar! Às seis horas,
alguém o avisou pelo telefone que o vapor adiava ainda a partida para
hoje de manhã! Então comecei a girar pela casa, como louca. Antônia não
tinha voltado da rua, nem voltaria, por ter de ir fazer quarto a um
doente, não sei em que casa nem em que lugar\ldots{} O meu desespero era tão
visível, que meu marido já não disfarçava as suas suspeitas\ldots{} Oh! não
me fujas, é a verdade, a verdade! Escuta, escuta, meu amor. Eu tinha
medo de que viesses, que ele te pressentisse e te matasse! Todo o meu
corpo tremia numa irritação nervosa que eu não podia dominar. Desde o
escurecer até à meia"-noite, o mais leve ruído, folha de árvore caindo na
areia, sopro de aragem levando um papel, qualquer som, por mais
indistinto, fazia"-me vibrar em grandes sobressaltos, que lhe causavam
uma estranheza silenciosa, mas sensível. Às dez horas obrigou"-me a
deitar"-me e a fechar a casa\ldots{} Seria preciso dormir cedo, para se fazer
madrugada\ldots{} Dormir! Mas as minhas pupilas queimavam"-me as pálpebras de
cada vez que eu tentava fechar os olhos, e o coração arrebentava"-me o
peito\ldots{} À meia"-noite alguém bateu devagarinho na veneziana\ldots{} Era o
sinal. Quis levantar"-me sem ser pressentida, julgando"-o a dormir; uns
pulsos de ferro obrigaram"-me à inatividade\ldots{} Ele escutava os sons de
passos tímidos embaixo da janela\ldots{} e, pensando que esses passos fossem
teus, eu desmaiara, de terror! Oh! é a verdade! escuta: meu marido
adivinhou o motivo da minha aflição, amarrou"-me ao leito com o lençol,
e, despendurando o seu velho alfanje do alto da cabeceira, deslizou de
mansinho para o terraço\ldots{} Não tenho palavras para contar"-te o meu
terror. Tentei gritar, para avisar"-te que fugisses, mas a voz morrera"-me
na garganta\ldots{} Que tempo estive assim? Um minuto? uma hora? não sei\ldots{}
De repente uma voz cortou o ar da noite\ldots{} um grito agudo e a voz de meu
marido rouquejou num rugido de raiva, abrasado e vingativo. Ah! mil anos
que eu viva, hei de sempre, sempre, sempre, ouvir aquela voz! Voltando
para dentro, ele me disse que, embaixo da minha janela, Lourenço estava
estendido na calçada, esvaindo"-se em sangue\ldots{} e, para aterrar"-me mais,
mostrou"-me a mão, que lhe decepara, passando"-a, numa carícia repetida,
pelo meu rosto e pelo meu colo\ldots{} Depois, rindo com escárnio do meu
desespero, enfiou"-me no dedo o anel de Lourenço, como aliança do crime e
da morte\ldots{} Que noite! que noite!\ldots{} Oh! mas tu não vieste, é a ti que
eu amo, o meu engano salvou"-te a vida. É a ti que eu amo, leva"-me daqui,
quero ser tua\ldots{} só tua\ldots{} eternamente tua!

Claudino sentia o corpo de Cora debatendo"-se de encontro ao seu coração.
Ela exprimira"-se vertiginosamente, quase sem tomar fôlego. Arquejava
agora de cansaço. A verdade surgia entre ambos, vestida de lágrimas, mas
lágrimas radiantes como estrelas. Ela amava"-o, não lhe mentira\ldots{} e era
linda; para que saber mais? Para que desejar mais?

--- Ele não tarda\ldots{} vai"-te embora\ldots{} Mas, antes, dize"-me alguma coisa:
tu não me disseste nada!\ldots{} Dize que me amas\ldots{} que crês em mim\ldots{} e
dize"-me que me perdoas\ldots{} mas que perdoas o quê?! Eu não tive culpa!\ldots{}
é só a ti que eu quero, meu amor\ldots{} meu grande amor, meu único amor!

Houve ura silêncio, em que, olhos nos olhos, eles se contemplaram,
longamente, profundamente.

A dor espiritualizara o rosto de Cora, que todo se erguia para Claudino,
numa súplica. Enleado e vencido, ele abaixou a cabeça e os lábios
trêmulos de ambos iam encontrar"-se num beijo, quando um encadeamento de
acordes em surdina vibrou ali mesmo, dentro da sala. Claudino ergueu o
busto, de súbito. Era a \emph{Sonata à minha noiva}, que ele estava
ouvindo, mas tocada por quem?

Cora gemeu:

--- Não crês em mim, não crês em mim!

--- Escuta.

--- Por que me repeles? É só a ti que eu amo\ldots{}

--- Escuta!

--- Escuta, escuta! o quê?!

A harmonia abria"-se na frase larga e comovida, principal motivo da
Sonata, em que Lourenço sonhara a noiva, errando num vasto campo de
açucenas em flor\ldots{}

--- Dá"-me a tua boca\ldots{} suspirou Cora, ainda num gemido.

Claudino não a ouviu e, desembaraçando"-se dela com esforço, atirou"-a
sobre os almofadões do divã e saiu precipitadamente.

Começava a escurecer e o ar mudara. Ao fundo da rua já se via, sobre o
borrão negro de um morro, uma grande talhada de lua, cor de laranja.
Claudino fez sinal para que o automóvel se aproximasse e quando o
\emph{chauffeur} lhe perguntou para onde o deveria conduzir, teve
vontade de gritar: --- Para o Hospício! mas conteve"-se e deu de novo o
endereço de Lourenço. Puxou do relógio. Eram sete e meia. Sentia"-se
nervoso, febricitante, com a imagem de Cora diante dos olhos e os
compassos da \emph{Sonata à minha noiva} a se reproduzirem
constantemente na sua memória. Era um caos; era um tormento. Mas, Cora
amava"-o, amava"-o, e em tudo isso salvava"-se ao menos o seu orgulho de
homem\ldots{}

Fechou os olhos\ldots{} sorriu.

\section*{\textsc{iv}}

O portão da chácara estava escancarado; e pelo extenso chão da alameda,
a luz do luar, coada pela ramaria dos bambus, lembrava uma quantidade de
ratos brancos, que se movessem, perplexos, um pouco para a direita, um
pouco para a esquerda, sem saber qual a direção, que deveriam tomar
definitivamente. Ao fundo, na clareira circular em que a casa assentava,
toda iluminada como para uma noite de festa, o cão ruivo de olhos de
foca uivava espaçadamente, de focinho erguido para a lua, já toda branca
e em pleno azul\ldots{}

Claudino percorreu a alameda, cismando no modo por que deveria responder
à curiosidade ansiosa de Lourenço. Dever"-lhe"-ia conservar a ilusão do
amor de Cora, ou repetir"-lhe as suas palavras?

Deveria conservar"-lhe a ilusão. Mas se ele ficasse bom, não resultaria
depois da sua mentira piedosa uma complicação desastrada? O melhor seria
entregar"-lhe o anel, adiando para mais tarde outras explicações. Em todo
caso, ele, de nenhum modo, poderia dizer ao outro:

--- ``Olha que sou eu a quem Cora ama! Tu, pobre de ti, foste vítima de
uma zombaria trágica do acaso. Os deuses protegem"-me; leva para a cova a
tua humilhação, se morreres, ou não me guardes rancor se continuares
neste mundo; porque, afinal, eu não tenho culpa''.

Subindo os poucos degraus do patamar de pedra, não teve nem sequer o
trabalho de bater à porta, porque ela, como o portão, também estava
aberta de par em par. Na sala da entrada não estava vivalma. Claudino
não ousou bater palmas; esperou um momento, a ver se via alguém.

O silêncio da casa, só cortado pelos uivos do cão lá fora e pelo rumor
de passos apressados no segundo andar, dar"-lhe"-ia a entender qualquer
coisa de terrível significação, se as sucessivas comoções por que tinha
passado não lhe tivessem perturbado o raciocínio. De mais a mais estava
com fome. Naquela ocasião tal contingência envergonhava"-o; um herói de
aventuras não deve pensar em bifes e ele ansiava por se atirar a uma
ceia no \emph{Brahma} ou no \emph{Paris}.

Infelizmente, não poderia sair dali sem ter cumprido a sua missão.
Sentou"-se numa cadeira de vime e tossiu alto, para dar sinal da sua
presença.

Tinha ideia de ter ouvido a um médico qualquer que o fenômeno da fome,
como o do sono, é frequente em certos indivíduos, por motivo de
excitações nervosas ou comoções intelectuais. Certas atrizes dramáticas,
ao cabo de três ou quatro atos de luta, em que vibram de paixão e fazem
tremer a plateia de entusiasmo, restauram ordinariamente as suas forças
combalidas, com um bom naco de vitela e um escorregadio copo de
Colares\ldots{} Outras pessoas, à iminência de um grande perigo, o melhor por
que se decidem é por se deitarem a dormir! O pai fora um desses. Na hora
em que ele, Claudino, entrava no mundo, pondo em grande risco a preciosa
vida da mãe, o pai, nervosíssimo, aterradíssimo, roncava alto como um
bem"-aventurado!

Aquilo, portanto, era da natureza; era humano, e ele não tinha razões
para se vexar consigo mesmo de sentir uma necessidade material numa hora
toda de espiritualidades\ldots{} Ao contrário, tal urgência consolava"-o,
porque o fazia pensar:

``Começo a sentir"-me outra vez gente!\ldots{}'' E ser alguém ainda é a coisa
melhor da vida. Impaciente por acabar com tudo e ir"-se embora,
decidia"-se a bater palmas, estrondosamente, quando viu um criado
atravessar a sala e dirigir"-se para o corredor com uma bandeja de
canequinhas de café.

Pigarreou. O criado voltou"-se:

--- O senhor é servido?

Claudino apressou"-se era dizer que sim.

O café quente e forte restaurou"-lhe as forças abatidas.

O criado esperava, em pé e calado, em frente dele.

--- Poderei falar à dona da casa?

--- A senhora está lá em cima\ldots{}

--- Sei.

--- Ah, o senhor sabe?

--- Presumo, pela bulha dos passos. Diga"-me, poderá entregar"-lhe este
cartão, imediatamente?

--- Vou mandá"-lo pela criada. A senhora está acompanhando a filha, que
tem estado muito aflita\ldots{}

--- Compreendo. Faço"-lhe notar que tenho pressa.

O criado murmurou ainda, quase imperceptivelmente:

--- Foi uma desgraça! e voltou as costas a Claudino, que tirava da
algibeira a sua cigarreira. Teria ao menos tempo para tragar uma fumaça
antes que a dona da casa aparecesse. O sabor do café impôs"-lhe a
necessidade do cigarro. Recostando"-se mais na
cadeira, começou a fumar, cerrando as pálpebras.

Cora, ao debater"-se de encontro ao seu corpo, deixara"-lhe um pouco do
seu perfume. Ele como que a trazia ainda consigo. Que teria ela pensado
do seu brusco abandono? Seria crível que não tivesse ouvido, tão bem
como ele, o encadeamento de acordes, da \emph{Sonata}? Recompondo bem a
cena, verificava que não. Cora não ouvira nada. E como poderia ele
explicar"-se aquela sensação do som, tão nítida, tão perfeita,
experimentada por ele numa hora tão alheia a todas as coisas
exteriores?! Que onda de ar lhe teria levado aquelas notas evocadoras e
dolorosas, ao ouvido deleitado por outra música e justamente no instante
divino da reconciliação?

Sabia haver certas doenças em que o paciente é perseguido por vozes que
não existem senão no maquinismo dos seus próprios ouvidos, vozes
extravagantes, alucinadoras, como rangidos de portas ferrugentas,
guinchos, latidos ou sopros. Nos sonhos, também, toda a gente que sonha,
se lembra de se ter deliciado uma ou outra vez, com melodias em surdina
ou estremecido à bulha de fanfarras turbulentas, e não há cirurgião a
quem certos doentes cloroformizados, ao passarem da vigília para o sono,
não descrevam a audição de certos sons: melodias suaves de violino,
vibrantes repiques ou badaladas de sinos. Todos esses casos, porém,
estão explicados pela ciência. Mas o seu?

Ele não se podia convencer de ter estado a dormir e gabava"-se de ser um
homem robusto, um homem são em todos os sentidos, limpo de pele e de
razão. Entretanto, era forçoso confessar ter sido nessa tarde vítima de
um fenômeno nervoso, e dos mais singulares e despropositados. Começava a
afligir"-se com o que pudesse Cora pensar a seu respeito. Que espécie de
sentimento seria agora o seu para com ele?

Pensando nela, revia"-lhe as atitudes ondeantes, o fulgor dos olhos
inundados de lágrimas, a palidez da sua pele quente e cheirosa, a
flexibilidade do seu corpo moço e gentil. Sentia ainda na palma das mãos
o contato das suas mãos pequeninas e nos braços o enlaçamento frenético
dos seus braços, delicados, mas fortes como a hera. Realmente, a vida
tem os seus mistérios e cada acontecimento a sua filosofia enigmática.
Bastava ver, como até mesmo os mais frívolos desses acontecimentos eram
interpretados diferentemente por um comentador!

Sem entrar na intimidade de um facto, a sua origem, o seu
desenvolvimento e os fatores, muitas vezes inconscientes da sua ação,
ninguém o pôde criticar com justiça\ldots{}

Para prova, ali estava ele, todo mergulhado numa intriga, debatendo"-se
nas suas águas de cores mais variadas que as das diferentes fases do
Nilo e de sabor mais amargo que as do Atlântico; e que variedade de
ideias lhe tinha ela infiltrado no cérebro? Tantas, tantíssimas, que já
escapavam à análise da sua memória atordoada.

E onde estava a justiça das suas apreciações? Onde?

Vistas pelo lado de fora, todas as casas bonitas parecem abrigar a
felicidade. Mas ide lá dentro, e vereis muitas vezes o sofrimento que ha
no interior. E' preciso não julgar todas as cousas pela sua
exterioridade, visto que os próprios sentimentos materiais têm paredes
como os prédios, e não podem ser julgados com verdade senão quando eles
próprios nos franqueiam as portas da sua intimidade. As aparências
tinham feito de Cora um monstro, mas penetrado o seu segredo, ela
brilhava agora aos seus olhos com o resplendor da inocência\ldots{}

Um rumor de saias que se aproximavam obrigou Claudino a atirar o seu
cigarro pela janela.

Lá fora o cão de olhos de foca uivava mais baixo, mais espaçadamente,
cansado.

--- Senhor Claudino\ldots{}

--- Minha senhora!

--- Desculpe"-me se o fiz esperar. Mas minha filha precisa muito de mim!

--- Será apenas por um curto momento, minha senhora. Queria ainda
dever"-lhe a fineza de me aproximar de seu sobrinho\ldots{}

--- Lourenço?!

--- Sim. Preciso falar"-lhe.

--- Pois não lhe disseram?

--- Nada.

--- Lourenço\ldots{}

E ela fez um gesto, exprimindo que ele tinha morrido. A língua negara"-se
a articular a palavra terrível e grossas lágrimas saltaram"-lhe dos
olhos.

--- Quando?!

--- Inda agora, às sete e meia\ldots{} Ele tinha"-o incumbido de uma missão
qualquer, muito delicada, não é verdade?

--- Por que me pergunta isso, minha senhora?

--- Porque ele não morreu pensando em nós\ldots{} o seu olhar voltava"-se para
a porta numa ansiedade, como se esperasse alguém. E esse alguém creio
que deveria ser o senhor\ldots{}

--- Talvez\ldots{}

A própria noiva, ajoelhada à sua cabeceira, não recebeu o seu último
suspiro, que se diria dado a outro amor que estivesse bem longe\ldots{} Houve
mesmo um instante em que me pareceu distinguir um nome de mulher no
movimento dos seus lábios quase frios\ldots{}

--- E esse nome, minha senhora?

--- Era, se bem pude entender, o nome de --- Cora\ldots{}

Claudino não pôde reter um estremecimento, que procurou logo disfarçar:

--- Deveria ter sido ilusão de V. Ex.\textsuperscript{ª}: seu sobrinho
adorava a noiva com toda a alma\ldots{}

--- Se ele não mereceu completamente o amor de minha filha, muito me
pesará que ela sofra demasiadamente por ele\ldots{}

--- Sua filha tem razão de o chorar.

--- O senhor afirma"-o?

--- Sim\ldots{}

--- A morte de meu sobrinho está, entretanto, envolvida em qualquer
mistério que o senhor conhece.

--- Eu? não.

--- O senhor, sim. Mas, descanse, que não procurarei violar à força o
seu segredo. Adorei Lourenço e fiz por ele o que teria feito por um
filho; isso bastaria para justificar toda a curiosidade que eu tivesse
pela sua vida, que defendi desde o dia em que o recebi, ainda pequenino,
nos braços.

--- Mas o que a faz acreditar num mistério, que naturalmente não
existiu?\ldots{}

--- Tudo. O próprio médico que o tratou e que, antes dos outros, lhe
ligou o pulso, afirmou que não poderia confundir o esmagamento da carne
com o talho de um instrumento cortante. Lourenço teve a mão decepada
pela lâmina de qualquer arma branca. Não é da minha opinião?

--- Não sei\ldots{} porque não vi.

--- Não quer dizer e eu já lhe confessei: não tenho intenções de
vingá"-lo. Se ele próprio respeitou o seu ofensor, não serei eu quem o
acuse\ldots{} Nem mesmo para estancar as lágrimas de minha filha eu lançarei
mão de um escândalo. Somente, não acrescentarei à fogueira da sua
saudade a lenha da minha\ldots{}

--- O dever das boas mães é exatamente o de procurarem dissipar os
desgostos das filhas\ldots{}

--- Quando esses desgostos não são comuns; porque, acredite na minha
experiência: há sempre uma certa consolação em não se sofrer sozinha\ldots{}
Mas já agora, antes de subir para ver Lourenço, deixe"-me perguntar"-lhe:
por que me queria falar?

--- Exatamente para receber de V. Ex.\textsuperscript{ª} a permissão de
vê"-lo\ldots{}

--- Ah! sim\ldots{} o senhor já me tinha dito. Perdoe a minha cabeça\ldots{}

Como a tia de Lourenço fizesse menção de retirar"-se, Claudino
atalhou"-lhe o movimento com um gesto:

--- Perdão. Preciso falar"-lhe, sim. Se, entretanto, V.
Ex.\textsuperscript{ª} tem muito cuidado em sua filha, eu esperarei aqui
alguns minutos.

--- Seria abusar da sua bondade. Percebo a sua fadiga.

--- Diga antes a minha comoção.

--- Era há muito tempo amigo de Lourenço?

--- Conhecia"-o apenas\ldots{}

--- É extraordinário!

--- Sim, é extraordinário.

--- Já que me permite, irei ver minha filha.

Prefere esperar"-me aqui ou velar uma hora lá em cima?

--- Cumprirei o dever de velar até à meia"-noite\ldots{}

--- Obrigada. Subamos então.

--- Se, entretanto, a filha de V. Ex.\textsuperscript{ª} precisar que eu
vá chamar um médico ou\ldots{}

--- Não. Ela está bem rodeada. Para estas coisas não há médicos.

--- Há: o tempo\ldots{}

--- Esse, cura matando\ldots{} Tenha a bondade de seguir"-me.

Como da primeira vez, ela recomendou cuidado com a escada,
maquinalmente, limpando os olhos do choro. Em cima, na sala dos
armários, havia roupas em desordem, revolvidas à pressa; e ao fundo do
corredor a janela aberta para o luar desenhava um quadro argênteo
cortado em diagonal por um galho escuro do hibisco. O quarto de Lourenço
estava aberto. Claudino entrou e a dona da casa acompanhou"-o até à beira
do leito.

--- Veja como ele parece dormir. Toda a agitação da fisionomia se
transformou, logo após o passamento, na mais pura serenidade. Isso
serve"-me ao menos de consolação\ldots{} O senhor disse que o conhecia apenas.
Se tivesse sido seu amigo íntimo deveria adorá"-lo. Era um crédulo, ura
sensível, um idealista\ldots{} Nunca imaginei que ele me pudesse morrer
dentro de um segredo\ldots{} porque para ele a mentira não existia. Toda a
gente era leal. Corria como uma criança para as ilusões, grato ao menor
afago\ldots{} Quem sabe se não foi por uma ilusão que ele morreu?!

Claudino sentiu que as pupilas da tia de Lourenço se fixavam1 nele,
interrogativamente.

Ele disfarçou, com uma frase banal:

--- Está descansado.

--- Sim. Está descansado\ldots{} Vou ver minha filha. Daqui a pouco estarei à
sua disposição. Permita que me retire um momento\ldots{}

--- Oh! minha senhora!

Ela saiu. Claudino ficou pensando:

``Realmente, é espantosa a perspicácia das mulheres! Esta senhora
descreveu"-me a alma do sobrinho como se a tivesse visto ao espelho. E
não podia deixar de ser assim.

Ele era ura idealista, um sensível, um crédulo; verdadeira alma de
artista, ainda não arranhada pelas asperezas da luta pela vida\ldots{} Assim,
bem se compreende que a beleza de Cora o tivesse fascinado com toda a
sua graça mórbida e sedutora e que ele corresse, sem refletir, ao seu
primeiro aceno, como as crianças correm para a ilusão\ldots{} Morrera ao
menos sem ter conhecido o desengano, murmurando o seu nome, na suposição
de ser amado\ldots{}''

E nascia"-lhe agora no coração uma piedade imensa por aquele moço,
sacrificado pelo acaso, em seu lugar. Parecia"-lhe uma hipocrisia estar
ali a lastimá"-lo, quando àquele sucesso devia o ar que estava respirando
e o gosto inigualável de poder dizer: eu vivo, eu penso, eu amo.

Claudino suspendeu a corrente dos seus pensamentos e ficou um momento
parado, como para descortinar qualquer dúvida que surgisse no fundo da
sua imaginação, ainda vaga e indistinta\ldots{} Oh! ele não tinha aquela fé
ilimitada que tivera o outro. Conhecia os embustes do mundo, era
desconfiado\ldots{} Em frente àquele corpo inerte e àquela desgraça
irremediável, assaltava"-o a ideia de ser impossível, de ser uma
ingenuidade quase fantástica, o ter Lourenço corrido para uma entrevista
sem estar previamente, e, por várias circunstâncias, prevenido para ela.
Nenhum homem, mesmo o mais idealista, acredita que uma senhora o chame
em condições tão excepcionais, sem que antes se tenham encontrado os
seus olhos no mesmo desejo amoroso e veemente.

Agora, que já não sentia nos braços o calor do corpo de Cora, nem lhe
via a comoção dos olhos inundados, podia analisar mais livremente a
situação. Onde estará a verdade? Estivesse onde estivesse, seria justo
que, depois de ter enterrado aquele pobre iludido, ele continuasse a
gozar o amor da mulher que o matara, embora inconscientemente? E teria
essa mulher sido sincera, absolutamente sincera para com ele? Só uma
cousa lho poderia provar; a carta de que ela lhe falara, a carta escrita
a Lourenço e que ele deveria ter recebido, como o desgraçado Lourenço receberá a
sua\ldots{} Entretanto, onde estava essa carta?

Claudino começou a passear, pensativo, pelo quarto. Recompunha os
factos.

Tinha saído nesse dia muito mais tarde do que o costume, e estava por
isso habilitado para assegurar que tal carta não lhe entrara em casa,
nem levada por mão própria, como lhe dissera Cora, nem pelo correio, nem
pelo diabo! Ocorreu"-lhe então que
a mensageira, essa velha tia Antônia, que ele vira apenas uma vez,
furtivamente, em lugar de se dar ao trabalho de vir à sua residência na
rua Costa Bastos, tivesse ido de preferência ao seu escritório
comercial, por ser mais perto. Seria isso? Deveria ser isso.

Tal suspeita serenou"-o um pouco.

Verificaria o caso de manhã, o mais cedo que lhe fosse possível. Agora
impacientava"-se pela verdade. Mas onde a encontraria? Toda a gente pensa
que traz a verdade dentro de si e ninguém a conhece. Há um minuto ele
tinha uma convicção. Agora já tinha outra. Que o esperaria de ali a
alguns instantes? Se por um milagre aquele morto ressuscitasse e lhe
fizesse do seu amor uma confidência completa, então poderia julgar.
Acreditava na voz dos homens mais que na das mulheres. Mas o morto não
acordaria jamais daquele sono, nem ele acreditava em milagres, que
reputava frutos do charlatanismo, da escamoteação, ou de bem combinadas
experiências de física\ldots{}

Na sua educação não tinham colaborado mulheres.

A mãe deixara"-o apenas recém"-nascido, o pai e o avô tinham"-no criado sem
crendices nem preconceitos. De resto, ele não tinha curiosidades
intelectuais. Queria viver no presente e pelo modo mais confortável que
lhe fosse possível. Lembrava"-se que uma vez, em pequeno, contando"-lhe
alguém que um certo Simão, contemporâneo de S. Pedro, fazia falar o seu
burro, o seu cão, o seu gato e a sua cabra, tudo o que lhe aprouvesse e
em qualquer língua, e que esse mesmo taumaturgo transformava a sua vara
inerte de nogueira (cortada como todas as varas mágicas, à meia"-noite do
terceiro dia da lua nova, com uma faca nova também), em uma serpente
viva e temerosa e que, com o simples fato de a espetar na terra, a fazia
num relance enramar, florir e frutificar como a mais fecunda árvore dos
nossos pomares, o avô, vendo"-o impressionado, interrompeu a narração
para levá"-lo a um circo da Cidade Nova, onde um ventríloquo fazia dizer
a um porco preto e anafado deliciosos versos de Musset\ldots{}

Voltando para a casa, o avô explicava"-lhe:

--- Era assim que Simão fazia falar o seu burro, o seu cão, o seu gato e
a sua cabra, bichos aliás mais espertos do que o suíno, que nos
apresentou este palhaço. Não acredites nunca nas coisas que não
entenderes, porque assim darás fraca ideia da tua razão.

E quanto às árvores brotadas e secularizadas em um minuto, não são
milagres da antiguidade, mas artes que ainda hoje exercem certos homens
da índia, denominados faquires\ldots{}

O avô gostava de varrer"-lhe da imaginação a poeira dourada das crenças e
das ilusões.

Por tudo isso, mais extraordinária lhe parecia agora a sua situação. Não
acreditando em fantasmas nem em casos sobrenaturais, sem curiosidade
pelos segredos mais ocultos da natureza, como explicar a si próprio a
visão daquele dedo de velho, alta noite, na sua sala deserta, e o
seguimento da história iniciada por ele? Como compreender a audição da
\emph{Sonata à minha noiva}, na hora exata da morte do seu autor e,
justamente, no instante em que a sua confiança ia ser traída?
Afigurava"-se"-lhe que se o avô vivesse ainda ele correria a abrir"-lhe o
coração atônito, embora com a certeza de não ser esclarecido.

Como a ignorância é suave e deleitosa! Percebia agora que o tempo
passado no cumprimento de deveres materiais, conquanto às vezes
cansativos, era o melhor da vida. Os dizeres simples das suas cartas
comerciais, os seus livros de ``Deve e Haver'', as suas cifras
positivas, sem embustes nem mistérios, como tudo isso era bom e honesto!
E haver quem se cansasse correndo atrás de ideias e de fantasmagorias,
sem visar outra recompensa senão a glória! Pobres doidos! De que servira
a esse Lourenço consumir o seu cérebro e a sua alma em estudos profundos
e demasiados e chegar a considerar"-se por isso um ente superior, se
tinha de morrer como poderia morrer qualquer bruto, por uma ofensa
física? Logo a mão! Realmente, o Destino ou a sua comadre Fatalidade
preparam tramas bem singulares! Se em vez de terem a Lourenço decepado a
mão, lhe tivessem cortado um pé, ou vibrado uma punhalada no peito ou
nas costas, ele não se teria deixado morrer, se o golpe não fosse
mortal. A ideia de nunca mais poder percorrer o teclado do seu piano com
os seus dez dedos fascinadores, tirara"-lhe forças para a reação. Antes
ser estúpido; a estupidez defende melhor os indivíduos de qualquer
acidente, mesmo o mais natural, do que o talento e a imaginação. Depois,
os asnos têm o atilamento natural da espécie, um certo instinto que os
faz fugir do perigo, enquanto que os homens intelectuais se deixam
enganar facilmente. Lourenço, quando abandonava a música, refugiava"-se
no livro. A filosofia e a harmonia dos mestres punham"-no fora do
ambiente comum às outras
pessoas. E o resultado fora aquela ingenuidade, aquela credulidade cega,
que o levara à morte, sem reflexões, logo ao primeiro aceno de uma
mulher\ldots{} E mal pudera ele pensar que esse aceno era feito a outro, a
outro\ldots{}

Claudino estremeceu, como se lhe competisse ter remorsos daquela obra do
acaso; e por fugir de olhar para Lourenço, chegou"-se à janela. O céu
estava recamado de estrelas. Também eles, os longínquos astros, têm sido
desde remotos séculos interrogados ansiosamente pela humanidade,
desejosa de conhecer o seu futuro e o seu segredo. Não bastara à
Superstição a Terra para rastejar, e ela deu ao seu corpo de réptil asas
de águia, com que se remontasse às regiões siderais. Dos simples
pescadores da Caldéia até os tempos modernos, que infinidade de almas
têm invocado os astros, nos temores da sua covardia! Lá estava radiando
a linda Vésper, amante de Marte e filha de Saturno, ao mesmo tempo que a
lua enchia o céu e a terra com a sua luz clara e untuosa. Claudino
deixou"-se ficar encostado ao umbral da janela, olhando para a noite, a
rever Cora no seu roupão lasso, cor de marfim, desmaiada entre os
almofadões de seda, ou a debater"-se depois nos seus braços, com um
desespero de mulher amorosa. Fora a primeira vez que a vira chorar, e
aquelas lágrimas cristalinas e grossas não lhe saíam do sentido.
Renascia"-lhe, com a memória do seu abraço e do contato quente do seu
corpo, o desejo de a tornar a ver, de a interrogar repetidamente, sem
descanso, fazendo"-a dizer tudo, tudo. Para maior desespero, o beijo,
sustado pela \emph{Sonata}, ficara como a queimar"-lhe os beiços\ldots{}
Alguém entrava no quarto e chamava"-o. Era a dona da casa.

--- Não aceito o seu sacrifício por mais tempo. Acabam de chegar alguns
amigos que velarão o corpo. Queira acompanhar"-me até à sala. É
meia"-noite.

Claudino inclinou"-se. Antes de sair, olhou para a cama. A viração
brincava com os cabelos do morto e ele pensou: ``Amanhã a estas horas
nem essa carícia ele terá, coitado''.

Percorreram de novo o corredor, a saleta dos armários, e escada, e
encontraram"-se no mesmo ponto da sala onde tinham conversado, horas
antes.

--- O senhor desejava falar"-me? Estou pronta a ouvi"-lo.

--- Supus ter muita coisa a dizer"-lhe, mas em verdade só tenho a dar a
V. Ex.\textsuperscript{ª} este anel, que, por sua vez, V.
Ex.\textsuperscript{ª} entregará à senhora sua filha\ldots{}

--- O anel de Lourenço!

--- Sim, minha senhora.

--- E foi o senhor quem o foi buscar?!

--- Sim\ldots{} minha senhora.

--- Aonde?

Claudino não respondeu. Olharam"-se fixamente, demoradamente.

--- Ele proibiu"-lhe que nos dissesse a verdade?

--- Minha senhora\ldots{}

--- A sua resposta não o comprometeria.

--- Mas eu nada sei!

--- Compreendo. Minha pobre filha!

--- Acredite que Lourenço a adorava!

--- Oh, o senhor não sabe nada, nada, deve portanto ignorar também isso!

--- Isso eu não o ignoro, porque ele mo disse.

--- Quando?

--- Poucas horas antes da sua morte.

--- Mas se ele amava minha filha, por que pensou em outra mulher?

--- Não sei se ele pensou em outra mulher.

--- O senhor é inabalável. Seja. Além da saudade que esta morte me
deixa, fica"-me também o espinho de uma suspeita dolorosa. Contanto que
ele fique só comigo\ldots{}

--- V. Ex.\textsuperscript{ª} saberá dissipar qualquer apreensão da
senhora sua filha\ldots{}

--- Não sou forte em dissimulações.

--- O coração das mães ensina todos os prodígios.

--- O meu coração está cansado. Acredite: tudo eu poderia esperar de
Lourenço, menos que ele, tão cândido e tão bom, se envolvesse um dia em
aventuras desleais, porque era, com certeza, casada essa mulher

--- Eu já lhe disse, minha senhora, que nada sei.

--- Tenho eu a certeza de que sabe tudo.

--- Pois bem, e se assim fosse? Se o próprio ofendido não tivesse
querido que a sua história transparecesse, o meu dever, embora a
conhecesse em todos os seus detalhes, não seria o de me calar?

--- Sim, todos nós devemos respeitar a vontade dos mortos. Ele
entregou"-lhe o seu segredo, o senhor defende"-o nobremente. Não o acuso;
admiro"-o; e se desejo saber toda a verdade, que adivinho por instinto,
não é por espírito de curiosidade, mas só pelo da vingança. Não lhe peço
nem lhe perguntarei mais nada\ldots{} Torna"-nos"-emos a ver?

Ele fez um gesto vago, de ignorância. Olharam"-se mais uma vez em
silêncio e despediram"-se com um demorado e mudo aperto de mão.

\section*{\textsc{v}}

Já o sol ia alto quando Claudino saltou da cama, atroando a casa com o
grito:

--- Antão!

E como o Antão não surgisse logo ali, como os demônios das mágicas, ele
sapateou com fúria no assoalho. Ora que estúpido, deixa"-lo dormir
daquele feitio! E o escritório, comi a sua correspondência para o Norte?
E o enterro de Lourenço, que àquela hora já deveria ir a caminho do
Caju? Apostava quanto quisessem em como o maluco do criado havia de
estar lá em baixo, com o outro maluco do alfaiate, dissertando sobre os
destinos da Itália, unida ao do Brasil republicano!

Ia dar outro berro cora toda a força dos seus pulmões valentes, quando o
Antão apareceu entre portas, com ar submisso, pronto para a arremetida
do amo terrível.

--- O senhor chamou?

--- O senhor chamou! E ainda você me pergunta isso a mim, que estou a
esgoela"-me ha mais de uma hora aqui, pelo seu nome! Sabe que horas são?
Nove!

--- Faltam dez.

--- Nove!

--- Como o senhor ontem se queixou de cansado e se deitou muito tarde,
tive hoje pena de o acordar. Assim mesmo, chamei"-o duas vezes, uma às
sete, como de costume, e outra às oito horas.

--- Qual chamou, qual nada! E se me chamou, como o fez você que eu não o
ouvi?

--- Da primeira vez disse só: --- São sete horas, patrão! e abri a
veneziana.

--- E da segunda?

--- Da segunda, com sua licença, sacudi"-o.

--- Com muita piedade, hein?!

--- Sim\ldots{} com algum respeito.

--- Algum! algum! Não quero que me respeite, já disse! Em semelhantes
circunstâncias tem licença de ir até ao safanão. E agora? Tinha nada
menos que o correio e um defunto à minha espera! Deveria te"-me posto em
pé à força, com murros ou com água fria. Os regadores não ficaram na
biblioteca?

--- Não, senhor.

--- Pois eu não lhe tinha dito?\ldots{}

--- Pensei que fosse brincadeira!

--- Está doido! Pois eu ia entrete"-me a brincar com você?\ldots{}

--- Pareceu"-me tão esquisito\ldots{}

--- É preciso cumprir as minhas ordens sem as comentar; prepare"-me o
café e o banho e corra ao escritório; diga ao meu sócio que não poderei
ir antes do meio"-dia e que me mande as cartas que lá houver para mim.
Avie"-se!

Antão franziu as sobrancelhas. Não estava acostumado àqueles modos
ríspidos. Que teria sucedido ao patrão, para que ele, geralmente amável,
se tornasse assim tão áspero? E o mau humor aumentava com o correr dos
minutos, tanto, que ainda ele na copa aquecia o café e já Claudino
gritava do quarto, estrondosamente:

--- Então, esse café vem ou não vem?

O criado perdeu o tino, serviu mal e, à hora de sair para o recado ao
armazém, ainda correu a lustrar os borzeguins do patrão, que se regalava
embaixo do chuveiro.

A água desfez"-lhe a impaciência. Claudino começou a fazer a sua toalete
sozinho e com toda a calma, o sono livrara"-o de envergar o terno preto
naquela manhã de sol.

Afinal, a sua companhia não fizera falta ao Lourenço no seu último
passeio pelas ruas da cidade; prestara"-lhe serviços mais importantes; e
quanto a isso não o acusava a consciência, visto que não fora por culpa
sua que a linda Cora tinha trocado as cartas na sua precipitação de
adúltera medrosa. No silêncio da casa deserta e depois de algumas horas
de sono, tinha mais lucidez de espírito para julgar e apreciar os fatos.
Pôs"-se a meditar: sob que força tinha ele agido naquela pavorosa
intriga? Sob a força de um poder oculto, em que até então não tinha
acreditado e que negaria ainda para o futuro, a todo o transe, por temor
do ridículo, não querendo compromete"-se com ideias religiosas ou
sobrenaturais. De mais a mais, se o seu caso transparecesse, não
faltaria quem viesse meter o nariz na sua vida e raspa"-lhe, com unhas
irreverentes, a crosta da inteligência e da reputação. Era egoísta.
Queria passar despercebido pelo meio da turba, para andar à vontade;
tanto mais que esse segredo bem guardado, da"-lhe"-ia liberdade para amar
a sua Cora e deixa"-se amar por ela, com todos os frenesis da sua paixão
voluptuosa. Porque, embora ele a amasse, percebia ser amado por ela com
redobrada intensidade.

Não vira na véspera como, ainda agitada, combalida, desorientada pelas
cenas cruéis em que fora obrigada a tomar parte, ela lhe caíra nos
braços, com tanta sinceridade e tanta comoção? Sabendo do que o marido
era capaz, tendo"-o presente, na imaginação, tinto de sangue ainda fresco
de uma vítima apenas suspeitada, ela não o enlaçara nos braços, a ele,
Claudino, sem temer de ser surpreendida e morta a seu lado por esse
mesmo marido sanguinário e feroz?

Que maior prova de amor poderia ele, em toda a sua vida, obter de uma
mulher? Era verdade que, se continuasse a amá"-la, correria o risco de
ter a mesma sorte de Lourenço; entretanto, --- oh! inconsequências da
paixão! --- aquela história de sabor medieval exacerbava"-lhe o desejo de
prosseguir no caminho perigoso

``Senhor, este Rio de Janeiro é a terra das complicações, --- refletia
ele, correndo sobre o colarinho o nó da sua gravata. Ainda anteontem eu
era um homem tranquilo, bem orientado, e aqui estou hoje indeciso, se
serei mesmo um homem ou apenas um ponto de interrogação!''

Completada a toalete chegou à janela, a ver se lobrigava o criado. Ardia
por ler a carta de Cora, única prova que ela apresentava em sua defesa.
A bem dizer, nem devia ser uma carta. Apenas um bilhete:

«Mande"-me a sua \emph{Sonata à minha noiva}, etc.''

Fora com certeza isso que ela escrevera a Lourenço e pusera, com a
precipitação do susto, dentro do envelope com o seu nome. Como certo
gênero de cartas amorosas não têm geralmente outra rubrica senão --- Meu
querido, ou --- Meu adorado, ou coisa nenhuma, --- o que é mais comum,
Lourenço não poderia, com efeito, ter percebido, nem sequer desconfiado,
ser a carta que recebera de Cora dirigida a outro homem! Enfim, não
valia agora a pena pensar nas lucubrações que por ventura pudesse ter
tido um homem já enterrado a essas horas, no quente chão do Caju.

O diabo era que o patife do Antão não aparecia; havia já tempo de sobra
para estar de volta com a desejada carta. Quando, por fim, o criado
chegou, vermelhaço e ofegante, antes mesmo de abrir a boca, ouviu ainda
da escada uma saraivada de impropérios. O desgraçado, chegando acima,
fixou no amo impaciente um olhar de espanto e murmurou:

--- O senhor desculpe, mas\ldots{}

--- Qual mas! Não admito \emph{mas}! Dê"-me essa papelada.

--- Seis cartas e um jornal\ldots{}

Enquanto Claudino rasgava o primeiro sobrescrito, o Antão perguntou,
limpando o suor que lhe escorria em bagas do rosto aflito:

--- O senhor quer que lhe arrume a mala?

--- A mala! para quê?

--- Para a sua viagem.

--- Que viagem?!

--- O senhor parte hoje para Buenos Aires.

--- Está doido!

--- Não estou doido, não senhor.

A primeira carta não era a de Cora, mas de um freguês de Pernambuco, que
se dirigia pessoalmente a ele, pedindo"-lhe amostras de couros e oleados.
Claudino atirou"-a para o chão, com raiva\ldots{} O criado recomeçou:

--- A sua passagem já está comprada. O senhor parte num paquete da Mala
Real Inglesa\ldots{}

--- Você está pra aí a dizer coisas que eu não entendo. Com o dia de
ontem acabaram"-se as mistificações. Deixe"-me ler\ldots{}

A segunda carta era de uma viúva pedindo"-lhe para arranjar um lugar para
o filho --- \emph{rapazinho muito trabalhador} --- no seu armazém.

Claudino varejou essa carta, por debaixo da mesa, com uma praga e,
enquanto tateava outra, Antão apressou"-se:

--- Fui eu mesmo à agência com uma carta do seu sócio, o sr. Jorge.
Cabine de primeira classe, tudo do melhor.

--- Irra! Que inferno! Cala"-se você ou não?

--- É que\ldots{} o sr. Jorge\ldots{}

A terceira carta era um convite para um sarau musical na residência de
um médico dos subúrbios.

--- O sr. Jorge --- continuou Antão --- ordenou que lhe dissesse isto
logo ao chegar, para seu governo. É negócio urgente.

--- Você põe"-me tonto. Espere. Deixe"-me ver tudo isto e depois falará.

A quarta, a quinta e a sexta carta eram contas de camisaria, prospectos
de uma nova companhia de cerâmica e um pedido de resgate para umas
cautelas de penhor de joias. Nada mais. Claudino sacudiu ainda as folhas
de uma revista, a ver se cairia de dentro o famoso bilhete de Cora, e
não achando coisa nenhuma, fixou interrogativamente a cara ainda
transtornada do Antão.

--- Você não teria perdido nada pelo caminho?

--- Nada, não senhor. Pus todos os papéis dentro desta algibeira.

--- É impossível!

--- O senhor se quiser pode perguntar lá na loja, porque o seu ajudante
até contou as cartas quando mas entregou. Eram seis.

--- Tem certeza disso?

--- Como de estar aqui.

Claudino começou a passear nervosamente pela sala.

Com que então a senhora dona Cora mentira"-lhe! Para livra"-se de uma
recriminação direta ou de uma queixa importuna, resolvera inventar
aquele velho \emph{truc} de comédia, em que ele caíra de quatro, como um
asno que era! Excelente atriz, a tal senhora! Depois da vinda de certas
celebridades femininas teatrais ao Rio, era isso que se via: toda a
mulher mais ou menos elegante julga"-se com direito a inventar e a
representar o seu papel de tragédia ou de farsa, na sociedade!

Execrável animal, a mulher formosa! Tem magnetismos de jacaré e
tentáculos de polvo. Fuja quem puder, antes de entrar no raio da sua
ação. Com aqueles abraços de fogo, aqueles soluços de súplica, aquele
franzir de lábios a pedir beijos, na hora da angústia em que deveria
estar de joelhos rezando pelo outro, que morrera de amor, e por amor
dela, Cora tivera imaginação para inventar alvitres que a salvassem
momentaneamente da vergonha e do crime\ldots{} E ele que se deixara convencer
da sua inocência e que se arrependera de a ter repudiado no doce
instante de se beijarem!\ldots{} Benéfico espírito, o que lhe cantara ao
ouvido esses compassos da música salvadora, e que o obrigaram a fugir!
Agora tudo se iluminava diante dos seus olhos. Lourenço não morrera
iludido, mas convencido. Por mais inexperiente e idealista que seja um
homem, ele não caminha para um abismo ao primeiro aceno de qualquer
tentação. Há sempre um momento de dúvida e de indecisão. Lourenço
correra para a morte, na certeza de correr para o amor: sem vacilar.
Logo, sabia que era esperado, que era querido por Cora. Sabia"-o,
positivamente. O ludibriado tinha sido ele, Claudino, e mais ninguém!

Antão, antes de se retirar para mudar de casaco, aventurou ainda, com
voz tímida:

--- Nesse caso posso fazer outro qualquer serviço?

--- Hein?! É verdade. Que me dizia você há pouco?

--- Perguntava se o senhor queria que lhe arranjasse a mala, visto que
não há tempo a perder.

--- Mas que trapalhada é essa?!

--- A casa precisa dos seus serviços em Buenos Aires. Foi o que me disse
o sr. Jorge. Ele pede para o senhor ir conversar com ele às três horas;
se não puder irá ele ao cais às cinco e meia, para lhe dar as suas
instruções. Lamenta que o senhor não tenha aparecido, porque tem estado
muito atrapalhado. Pensava que estivesse doente.

--- Bem. Arruma a mala.

--- Como de costume?

--- Como de costume.

--- Sim, senhor.

--- E dizer que eu poderia comprometer os meus negócios por causa de uma
aventura sem pés nem cabeça! pensou Claudino, recomeçando agitadamente
no seu passeio, até ir, em uma das voltas, parar em frente à grande
estante de vidraçaria lavrada da biblioteca. Abriu"-a. Sentiu cheiro a
mofo e um rastejar suspeito de baratas. Havia papéis em desordem, e
livros deitados, por detrás de outros livros. Quantos anos de trabalho,
de paciência, representavam essas obras de estudo e de literatura
clássica, na consulta das quais o pai consumira também tantos anos e
tanta paciência!

Claudino teve um triste sorriso de piedade pela loucura daquela
consumição. Valeria a pena queimarem o cérebro na fabricação de frases,
que se esquecem, ou que outros reproduzem depois com outra forma,
igualmente mortal? Não seria ele quem martirizasse os seus pobres olhos
naquelas cinzas de um passado que o não interessava absolutamente. Com
muito trabalho, tocando nos livros com as pontas dos dedos, como se
fossem animais putrefatos, Claudino conseguiu achar o volume de
Francisco Rodrigues Lobo, que o Antão sepultara por detrás de outras
obras mais ou menos suas contemporâneas. Procurando a luz da janela,
folheou então o livro até achar uma frase sublinhada à mão.

Leu: ``Socorre Lourenço e põe os olhos no seu exemplo''.

Estava na terminação desta frase toda a razão da intervenção paterna. Os
olhos encheram"-se"-lhe de água à lembrança do dedo do pai, apontando"-lhe,
tremulamente, mas pertinazmente, aquelas palavras que resumiam então uma
ordem incompreensível. Claudino fechou devagarinho o livro e levou"-o aos
lábios, devotamente. Depois sentou"-se à secretária, e escreveu:

``Minha Senhora.

Sumo"-me da sua vida com a certeza de lhe não deixar saudades. Por mim,
procurarei esquecê"-la. Peço"-lhe, entretanto, que reze por ele, e que não
zombe de mim.

Claudino''.

Mandada a carta, sentiu que se tinha partido para sempre o fio daquela
aventura extraordinária, sem que por isso deixasse de pensar, por uma
obsessão dolorosa, nessa mulher perturbadora, toda feita de luz e de
perfume. Deveria odiá"-la e amava"-a ainda, mais do que nunca, como um
doido! Se ela não lhe tivesse mentido, se de fato o bilhete a Lourenço
lhe tivesse chegado, ele seria nesse instante o mais feliz dos homens.
Um riso nervoso sacudiu"-lhe o corpo à ideia de ter sido ludibriado por
Cora. Fechou os olhos, esmagou as pálpebras com as pontas dos dedos
gelados e quedou"-se assim, revendo"-a, adorando"-a, maldizendo"-a\ldots{}

``Meu amor, meu grande amor, meu único amor!''

Ela dissera"-lhe estas palavras iluminada pelo fulgor da paixão, unindo
ao seu o corpo fremente, o seu corpo divino, toda, toda dele\ldots{}

Claudino sentia agora que a sua vida passada e futura convergia
inteiramente para esse minuto criado pela aliança trágica da mentira com
o amor. Seria então certo que o homem tanto mais ama quanto menos
confia? Aí tinha a prova de tal argumento: Cora, que fora até então para
ele como que uma promessa, mais lisonjeira à sua vaidade do que ao seu
coração, transformou"-se de repente no único motivo da sua existência,
numa obsessão dolorosa, só por ter deixado de ser a mulher certa, pronta
a servi"-lo através de todos os sacrifícios, entre sorrisos, como a um
deus.

Agora estava tudo acabado, e se ela fora forte em enganá"-lo, ele
sê"-lo"-ia ainda mais em fugi"-lhe, em repudiá"-la, num gesto de desprezo.
Entretanto, os olhos de Cora, os seus braços amorosos, o resplendor dos
seus cabelos, a palidez da sua fronte alta e o som da sua voz musical e
intensa: ``meu amor, meu grande amor, meu único amor!'' imploravam"-lhe
que voltasse, que voltasse, que voltasse para o beijo interrompido, para
um beijo eterno!

Às duas horas, o Antão julgou dever intervir.

--- A mala está pronta.

--- A mala?!\ldots{} ah! sim\ldots{}

--- Agora vou chamar um carregador.

--- Bem\ldots{} vai\ldots{}

--- Não será bom chamar também um táxi, para o senhor ir ao armazém
combinar os seus negócios?

--- Talvez seja melhor\ldots{}

--- É melhor, porque já é tarde.

--- Tenho tempo. Escuta: não quero voltar para esta casa. Amanhã mesmo
combina com o sr. Jorge sobre os meios de se fazer leilão de todos os
trastes e de se reformar isto para alugar.

--- E\ldots{} e depois?

--- Acabaremos, como todos os solteiros, numa pensão ou num hotel\ldots{}
Vida execrável, Antão!

O criado baixou a cabeça, entristecido. Claudino limpou os olhos.

--- Vende"-se também a cama que pertenceu ao sr. desembargador Aleixo?

--- Também.

--- E os livros?

--- Vende"-se tudo. De mais a mais, para que nos servem os livros?

--- Pra nada, lá isso é verdade\ldots{}

--- Estes ao menos podem gaba"-se de terem engordado muitas gerações de
traças\ldots{} o que prova que não há nada inútil neste mundo. Hoje fui
injusto para com você. Tome lá e não me queira mal.

Claudino pôs dinheiro nas mãos do criado e fez"-lhe sinal que saísse.

\section*{\textsc{vi}}

Eram oito horas da noite quando o Antão voltou do cais, de ver o amo
partir para Buenos Aires. Vinha cansado, morto por fumar o seu cigarro,
de pernas estendidas no sofá, mas ao entrar na escada o alfaiate do
andar térreo, gritou por ele:

--- \emph{Seu} Antão!

--- Que é?

--- Faça favor.

--- Há alguma novidade?

--- Eu lhe digo: anteontem veio aqui uma mulher com uma carta para o seu
patrão\ldots{}

--- E daí?

--- Como nem o senhor nem ele estivessem em casa, ela pediu"-nos que a
guardássemos para entregar depois.

--- E o senhor esqueceu"-se\ldots{}

--- Não me esqueci. Saí para umas provas e supus que a minha pequena lha
tivesse dado; tenho andado tão atarefado de trabalho, que nem tempo
tenho para comer, acredite. Só agora, fazendo a limpeza de sábado na
oficina, foi que topei com a peste da carta no meio dos retalhos do
gavetão.

--- Paciência\ldots{}

--- Mas inda o pior não é isso. Parece que o diabo do aprendiz, achando
a carta cheirosa, quis verificar se ela trazia flores dentro e
abriu"-a\ldots{}

--- Oh! diabo!

--- Eu podia não lhe dizer nada e pôr o papel ao cisco; mas pode
trata"-se de negócio de importância e não quero. Tome"-a lá.

--- Obrigado. Boa noite.

--- Boa noite.

Antão subiu. Em cima, depois de ter acendido a lâmpada e de se ter
refestelado na cadeira de balanço, julgou prudente ler aquela carta, a
ver se valeria a pena transmiti"-la ao patrão. A curiosidade obrigava"-o a
esse bom serviço\ldots{} Assim, leu"-a, tornou a lê"-la, comparou o papel lilás
das suas folhas ao papel igualmente lilás do sobrescrito, e pasmou. Não
entendia nada.

O escrito dizia assim:

``Sr. Lourenço.

Peço"-lhe que me mande a sua \emph{Sonata à minha noiva}, que desejo
estudar. Sua amiga muito grata,

Cora''.

Antão ficou um momento perplexo, revirando a carta entre os dedos;
depois amarrotou"-a e atirou"-a com um gesto decidido para a cesta dos
papéis inúteis.

\bigskip

\begin{center}
\textsc{fim}
\end{center}
