\chapterspecial{Vida e obra}{}{Rodrigo Jorge Ribeiro Neves}
\hedramarkboth{vida e obra}{}

\section{Sobre a autora}

Em um casarão na rua do Lavradio, no Centro do Rio de Janeiro, nasceu
Júlia Lopes de Almeida em 24 de setembro de 1862. Com uma produção
literária expressiva em gêneros diversos, Júlia foi romancista,
contista, cronista e dramaturga. Em 1881, aos dezenove anos, publicou
seus primeiros textos em \emph{A Gazeta de Campinas}, jornal da cidade
para onde se mudara com a família ainda na infância. Aos 22 anos, em
1884, começou a escrever para um dos principais periódicos brasileiros,
\emph{O País}, colaboração que se estendeu por mais de trinta anos. A
atividade literária e jornalística em importantes veículos da imprensa
da época exerceu influência decisiva na sua atuação intelectual e
artística.

Filha de um casal de portugueses, Valentim José da Silveira Lopes e
Adelina Pereira Lopes, Júlia mudou"-se, em 1886, para Lisboa, onde deu
início a sua carreira de escritora. No ano seguinte, com a irmã Adelina
Lopes Vieira, publicou \emph{Contos infantis}. Em 1887, casou"-se com o
também escritor Filinto de Almeida, então diretor do periódico carioca
\emph{A Semana Illustrada}, que contou com a frequente colaboração de
Júlia Lopes. Por meio de folhetins em \emph{O País}, lançou, em 1888,
seu primeiro romance, \emph{Memórias de Marta}, quando retornou ao
Brasil. Desde então, foi uma escritora prolífica e engajada, abordando
temas como a República, a escravidão e o papel da mulher nas esferas
pública e privada da sociedade, com o Rio de Janeiro como um de seus
principais cenários. Dentre seus livros, destaca"-se o romance \emph{A
falência}, de 1901, retrato contundente de um país que mudava de regime
e se modernizava, mas permanecia preso a estruturas arcaicas de
exploração e desigualdades.

Júlia Lopes de Almeida foi uma das escritoras mais importantes da virada
do século \textsc{xix} para o \textsc{xx}, sendo um dos principais nomes da \emph{Belle Époque} carioca. Esteve entre os idealizadores da Academia Brasileira de
Letras, mas foi preterida a assumir uma das cadeiras entre os fundadores
por ser mulher, já que a maioria dos membros decidiu acompanhar a
tradição da Academia Francesa de Letras, modelo seguido pela agremiação
no Brasil, que contava apenas com homens no quadro. Seu marido, Filinto
de Almeida, ao contrário, ocupou a cadeira de número 3, embora
reconhecesse, em entrevista a João do Rio, que quem deveria estar na
Academia era Júlia, e não ele.

A respeito do não ingresso de Lopes de Almeida na \textsc{abl}, a pesquisadora Michele Fanini, autora da tese de doutorado ``Fardos e fardões: Mulheres na Academia Brasileira de Letras (1987--2003)'', comenta: 

\begin{quote}
Júlia Lopes de Almeida participou, juntamente com seu cônjuge, Filinto de Almeida, de muitas das reuniões que culminariam na criação da \textsc{abl}. Lúcio de Mendonça, um dos idealizadores da agremiação, chegou a elaborar uma lista extraoficial com os nomes daqueles que, segundo ele, deveriam figurar como seus membros fundadores. Publicada no \textit{Estado de São Paulo}, em 1896, a lista trazia o nome de uma única escritora: o de Júlia Lopes de Almeida. Até onde nos é dado saber, a tímida ressonância da indicação entre os demais postulantes (à exceção de Filinto de Almeida, Lúcio de Mendonça, José Veríssimo e Valentim Magalhães), amparada na alegação pretensamente impessoal de que a agremiação estaria sendo concebida à imagem e semelhança de sua congênere francesa, a \textit{Académie Française de Lettres}, em cujo Regimento Interno a expressão \textit{homme de lettres} adquiria sentido literal, culminou em um desfecho sugestivo, que viria a assumir os contornos de uma gentileza compensatória: o ingresso de Filinto de Almeida, que passou a ser considerado por alguns como o ``acadêmico consorte''. Filinto de Almeida chegou a fazer a seguinte afirmação em entrevista concedida ao dândi João do Rio: ``Não era eu quem devia estar lá [na \textsc{abl}], era ela''. Júlia Lopes de Almeida protagonizou o primeiro e mais emblemático vazio institucional da \textsc{abl} produzido pela barreira do gênero. Gostaria, no entanto, de mencionar o papel fundamental desempenhado por Claudio Lopes de Almeida, neto da escritora, que foi quem cuidou de seu arquivo pessoal até fins de 2010, quando então passou a ser custodiado pela \textsc{abl}. Como Júlia Lopes de Almeida fez coincidir sua trajetória literária e social com a produção de seu arquivo, a preservação de sua memória muito se deve à cuidadosa atuação do neto e, atualmente, ao Arquivo da \textsc{abl}.\footnote{Disponível em: \emph{https://bit.ly/3qudvYV}.}
\end{quote}

A escritora chegou a morar novamente em Portugal, onde publicou suas
primeiras peças teatrais, e depois na França, onde sua obra foi
traduzida e divulgada. Participou ativamente de diversas associações
femininas e discutiu temas relacionados ao Brasil e à mulher em
conferências no país e no exterior, bem como em alguns de seus livros.
Faleceu no Rio de Janeiro em 30 de maio de 1934, por complicações renais e linfáticas decorrentes da febre amarela.

Mesmo sendo uma das autoras mais importantes de seu tempo e admirada
pelos seus pares, o nome de Júlia Lopes de Almeida não resistiu aos
mecanismos de apagamento do cânone. No entanto, sua obra vem sendo
resgatada nos últimos anos por estudiosos de diversas áreas das
humanidades, com reedições de seus principais livros. Além disso, a
atualidade das questões discutidas em sua obra e a moderna sofisticação
de sua escrita são também fatores determinantes para que sua leitura
seja cada vez mais necessária.

\section{Sobre a obra}

Esta coletânea reúne algumas narrativas curtas de Júlia Lopes de
Almeida, dividida em duas seções: ``Contos'' e ``Novelas''. Os livros
dos quais foram extraídos os textos são, respectivamente, \emph{Ânsia
eterna} (1903) e \emph{A isca} (1922). Embora não sejam os únicos
volumes de narrativas curtas da escritora, foram selecionados por
apresentarem algumas das características da narrativa de Júlia Lopes e
dos temas que permeiam sua obra. Por isso, este livro não se propõe a
ser uma síntese ou um panorama da multifacetada e expressiva produção
literária da autora, mas um convite à discussão sobre questões presentes
em suas temáticas, bem como um estímulo a conhecer suas demais obras.

\emph{Ânsia eterna} foi publicado pela primeira vez, no Rio de Janeiro,
pela H. Garnier. Em 1938, foi lançada uma reedição póstuma pela editora
A Noite, com correções feitas pela autora. Uma das principais
influências desse livro, e de outros que marcam o estilo de Júlia Lopes
de Almeida, são os contos do escritor francês Guy de Maupassant
(1850--1893). Embora os textos de \emph{Ânsia eterna} fujam um pouco do
universo da obra de Júlia Lopes, ao abordar o insólito e o fantástico, a
começar pelo título do volume, eles não deixam de discutir as questões
caras à escritora, como o papel da mulher e o retrato da sociedade
escravocrata. Para esta coletânea, foram selecionados dez contos: ``O
caso de Rute'', ``A rosa branca'', ``Os porcos'', ``A caolha'',
``Incógnita'', ``A morte da velha'', ``Perfil de preta (Gilda)'', ``A
nevrose da cor'', ``As três irmãs'' e ``O futuro presidente''. Muitos
deles são dedicados a escritores e intelectuais de sua geração, como
Arthur Azevedo e Machado de Assis.

Já a edição de \emph{A isca} foi um trabalho da Livraria Leite Ribeiro,
também no Rio de Janeiro. O livro é constituído de quatro novelas, das
quais selecionamos duas para esta coletânea, ``O laço azul'' e ``O dedo
do velho''. Com o subtítulo ``novela romântica'', a primeira traz à tona
o lugar da mulher na constituição familiar, sua posição em tempos de
guerra e as dinâmicas das relações entre seus membros. E isto através da
questão do duplo, representada por duas irmãs gêmeas, um dos temas
recorrentes da prosa de ficção moderna. A segunda novela foi publicada
pela primeira vez em \emph{A Illustração Brazileira}, em 1909, com o
subtítulo ``romance''. Assim como em alguns contos de \emph{Ânsia
eterna}, ``O dedo do velho'' também se reveste do insólito no
desenvolvimento de sua história, além de apresentar alguns índices da
modernidade, nas referências ao automóvel e à urbanização.

Para esta edição, foi atualizada a grafia segundo o Novo Acordo
Ortográfico da Língua Portuguesa. Palavras como ``oiro'', ``coiro'',
``doiradas'', ``loiça'', ``óptica'' e ``cousa'', embora contempladas no
Vocabulário Ortográfico da Língua Portuguesa, foram substituídas pelas
suas formas contemporâneas do Português Brasileiro, como ``ouro'',
``couro'', ``douradas'', ``louça'', ``ótica'' e ``coisa''. A pontuação
da autora também foi conservada, salvo em casos que podem levar a
ambiguidades ou estejam em desacordo com regras sintáticas, como a
exclusão de vírgulas separando sujeito e predicado. Decidimos manter
ainda colocações pronominais, como próclises, mesóclises e ênclises,
empregadas pela autora. Expressões em língua estrangeira foram grifadas
em itálico.

Além das obras acima citadas, Júlia Lopes de Almeida deixou os romances \textit{A família Medeiros}, \textit{A viúva Simões}, \textit{Cruel amor}, \textit{A intrusa},
\textit{A Silveirinha}, \textit{A casa verde} (com o marido Filinto de Almeida), \textit{Pássaro tonto} e \textit{O funil do diabo}; além dos livros de contos \textit{Traços e iluminuras}, \textit{Era uma vez\ldots} e \textit{A caolha} e as peças teatrais \textit{A herança}, \textit{O caminho do céu}, \textit{A última entrevista}, \textit{A senhora marquesa}, \textit{O dinheiro dos outros}, \textit{Vai raiar o sol} e \textit{Laura}.

\section{Sobre o gênero}

Marcada pelos contos de Guy de Maupassant (1850-1893), assim como pelos
romances de Émile Zola (1840-1902), Júlia Lopes imprime em suas obras
uma forte influência do naturalismo e do realismo francês. Algumas das
características presentes na sua produção literária são a objetividade,
em contraposição ao sentimentalismo, o antropocentrismo, as duras
críticas à sociedade brasileira e o cientificismo na análise de seus
personagens, influenciados pelo meio, raça e contexto histórico, de
acordo com o determinismo, e cujo comportamento é associado a causas
biológicas, segundo o biologismo. A zoomorfização também é um elemento
recorrente nas obras de Júlia Lopes, atribuindo características animais
a seres humanos. No entanto, a escritora não deixou de escrever aquilo
considerado mais adequado para uma mulher da época, como \emph{O Livro
das Noivas e Maternidade}.

É importante ressaltar o contexto histórico no qual Júlia Lopes de
Almeida está inserida, a começar pelo ano de seu nascimento. Em 1862 o
Brasil rompe relações com o Reino Unido na Questão Christie, como
consequência de tensões entre as coroas, principalmente por conta da
persistência da~escravidão~no Brasil. Após a Proclamação da República em
1889, importantes transformações políticas, econômicas, sociais e
culturais marcaram o país na virada do século. A Primeira República,
também conhecida como República Velha, vivenciou graves~crises devido às
disputas geradas pelas forças políticas ainda fragmentadas e à
desvalorização da moeda acompanhada do súbito crescimento da inflação.
Júlia Lopes viveu um período de consolidação das instituições
republicanas, de uma~economia agroexportadora, de revoltas populares,
civis e militares, contra o sistema político e social, e de entrada no
século \textsc{xx}, a chamada Era dos Extremos.

Em termos de estrutura, o conto é um gênero literário de extensão curta, por excelência moderno. Já a novela está entre o conto e o
romance em termos de extensão. Enquanto o primeiro se caracteriza pela
presença de poucos personagens, uma única ação, um só conflito e um
drama, além de uma limitação do tempo e do espaço, a novela é mais ampla
e plural quanto a esses elementos.

Segundo o crítico Massaud Moisés, a novela, em comparação com o conto, é essencialmente multívoca, polivalente: ``constitui"-se de uma série de unidades ou células dramáticas. De onde se segue que a primeira característica estrutural da novela é sua pluralidade dramática: ao invés do conto, que gira em torno de um conflito, a novela focaliza vários. E cada um deles apresenta começo, meio e fim''.\footnote{\textsc{moisés}, Massaud. \textit{A criação literária}. São Paulo: Cultrix, 2006, p.\,113.}

Ainda nas palavras do crítico literário:

\begin{quote}
O novelista não esgota por completo o conteúdo de uma unidade para depois efetuar o mesmo com as seguintes: no fim de cada episódio, procura deixar sementes de mistério ou conflito para manter aceso o interesse do leitor. É raro que esvazie o recheio dramático duma célula antes de prosseguir, pois frustraria a curiosidade do leitor.

{[}\ldots{]}

Em suma multiplicidade dramática, numa corrente horizontal. Por isso, o número de páginas pode crescer à vontade: a pluralidade pressupõe uma estrutura aberta, de modo que novos episódios possam adicionar"-se numa cadeia sucessiva, assim como o fim provisório da narrativa implica a multivocidade.\footnote{Ibid., p.\,114.}
\end{quote}

Após esclarecer a ação na novela, Moisés coloca em perspectiva o tempo novelesco. Afirma que a estrutura linear e plural da novela lhe impõe uma limitação temporal, que faz com que esse gênero não aborde a história da personagem desde seu nascimento, mas reduz"-lhe o passado a poucas linhas, essenciais para compreender"-lhe as ações e seu modo de ser, supreendendo a personagem no momento em que está madura par agir.

\begin{quote}
O tempo da novela é o histórico, assinalado pelo relógio ou pelo calendário, ou pelas convenções sociais. O presente é categoria dominante, em que pese às referências sumárias ao pretérito. Tudo se passa como se os dias, as semanas, os meses e os anos, de efêmera importância, significassem muito. A ação desenrola"-se por inteiro no presente, aqui e agora: condensado o pretérito em breves anotações.\footnote{Ibid., p.\,115.}
\end{quote}

Por fim, ao abordar o espaço da novela, Moisés ressalta o dinamismo acelerado desse gênero literário, causado pela sucessão de episódios, que implica a ausência de uma unidade espacial. É no deslocamento físico, continua o crítico, que as personagens procuram dar cabo da angústia ou atender ao apelo da aventura, criando, mesmo em uma única cidade, uma pluralidade de espaços que distingue a novela. Segundo o crítico tal dinanismo espacial aproxima a novela do conto:

\begin{quote}
À semelhança do conto, a estrutura da novela caracteriza"-se por ser plástica, concreta, horizontal. Assumindo as mais das vezes a perspectiva da terceira pessoa, o autor se coloca fora dos acontecimentos, ou concede a uma personagem a direção da narrativa. A vida imaginária sobrepõe"-se à vida observada: o novelista concentra"-se em multiplicar os expedientes narrativos, formulando sucessivas células dramáticas, sem atentar para os imperativos da verossimilhança. O enredo, além de visível, não esconde nada, não dissimula profundidades dramáticas ou psicológicas: com o predomínio da ação, tudo o mais se torna menos significativo.\footnote{Ibid., p.\,117--118.}
\end{quote}

\section{Sobre nossa equipe}

Rodrigo Ribeiro Neves é crítico literário e pesquisador, com doutorado em Estudos de Literatura e mestrado em Letras pela Universidade Federal Fluminense (\textsc{uff}). Foi pesquisador visitante na Princeton University, nos \textsc{eua}, e bolsista da Fundação Casa de Rui Barbosa. Atuou como docente de literatura brasileira na Universidade Federal Fluminense (\textsc{uff}) e na Universidade Federal do Rio de Janeiro (\textsc{ufrj}). Desenvolveu pesquisa de pós"-doutorado no Instituto de Estudos Brasileiros da Universidade de São Paulo (\textsc{ieb"-usp}) e na Universidad de Alcalá, na Espanha.
