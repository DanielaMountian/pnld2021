\chapter{O amor invisível numa terra de parasitas}

\begin{flushright}
\textsc{ieda lebensztayn}
\end{flushright}\medskip

\section{Sobre o autor}

\noindent{}Fazendeiro, escritor para crianças e adultos, editor, empresário,
defensor do petróleo nacional: a intensidade com que Monteiro Lobato
experienciou as várias faces de sua vida transparece na vitalidade de
seus contos, frutos de sua sensibilidade, observação crítica,
conhecimentos literários e trabalho intelectual e artístico.

José Bento Renato Monteiro Lobato nasceu em Taubaté, São Paulo, a 18 de
abril de 1882, que ficou consagrado como Dia Nacional do Livro Infantil,
e faleceu em São Paulo, a 4 de julho de 1948.

Escritor de literatura infantojuvenil, contista, jornalista, editor,
tradutor, pintor e fotógrafo. Aos onze anos, mudou seu nome para José
Bento, por causa das iniciais gravadas no castão da bengala do pai,
J.B.M.L. Apesar de sua inclinação para as artes plásticas, cursou a
Faculdade de Direito do Largo São Francisco, em São Paulo, por imposição
do avô, o Visconde de Tremembé. Formado em 1904, voltou a Taubaté, onde
foi nomeado promotor público interino, transferido, em 1907, para
Areias, São Paulo. Enviou artigos para \emph{A Tribuna}, de Santos,
traduções para o jornal \emph{O Estado de S. Paulo} e caricaturas para a
revista \emph{Fon"-Fon!}, do Rio de Janeiro. Em 1911 herdou, com as duas
irmãs, a fazenda do avô. Publicou, em 1914, os artigos ``Velha praga'' e
``Urupês'' em \emph{O Estado de S. Paulo}, criando o personagem Jeca
Tatu. Em 1917, vendeu a fazenda e se mudou para São Paulo.

Escreveu em \emph{O Estado de S. Paulo} o artigo ``A propósito da
Exposição de Malfatti'' (``Paranoia ou mistificação?''), de crítica
contra as vanguardas, abrindo polêmica com os modernistas. Em 1918,
estreou com o livro de contos \emph{Urupês}, que esgotou 30 mil
exemplares entre 1918 e 1925, e comprou a \emph{Revista do Brasil},
lançando as bases da indústria editorial no país. \emph{Cidades mortas},
originalmente publicado em 1919, numa edição da \emph{Revista do
Brasil}, reúne os primeiros escritos de Lobato, ainda estudante em
Taubaté, e contos que escreveu antes de viajar a Nova York para ocupar
um posto no Consulado brasileiro.

Criando uma rede de distribuição, com vendedores autônomos e
consignatários, revolucionou o mercado livreiro. Em 1920, fundou a
editora Monteiro Lobato \& Cia, que publicou obras de Lima Barreto, Léo
Vaz, Oswald de Andrade, Ribeiro Couto, Menotti del Picchia, Guilherme de
Almeida, Oliveira Viana e Amadeu Amaral, entre muitos outros. No mesmo
ano, lançou \emph{A menina do Narizinho Arrebitado}, primeira da série
de histórias com que Lobato criou a literatura brasileira dedicada às
crianças, formando gerações de leitores. Em 1924, com capital ampliado e
nova denominação, Companhia Gráfico"-Editora Monteiro Lobato, sua editora
monta o maior parque gráfico da América Latina. Porém, no ano seguinte,
dificuldades financeiras o levam a vender a \emph{Revista do Brasil} e
liquidar a editora. Mudou"-se para o Rio de Janeiro e fundou a Companhia
Editora Nacional.

Adido comercial em Nova York de 1927 até 1930, voltou ao Brasil com
ideias para a exploração de ferro e petróleo. Fundou empresas de
prospecção, mas, contrariando interesses multinacionais e fazendo
oposição, em artigos e entrevistas, ao governo Vargas, foi preso por
seis meses em 1941. Recebeu indulto depois de cumprir metade da pena,
mas o governo mandou apreender e queimar seus livros infantis.

Em 1944, Lobato recusou indicação para a Academia Brasileira de Letras.
Em 1946, tornou"-se sócio da editora Brasiliense. Embarcou para a
Argentina e fundou em Buenos Aires a Editorial Acteon, retornando no ano
seguinte a São Paulo.

Principais publicações:

\begin{enumerate}
\item Livros para crianças: \emph{O Saci} (1921); \emph{Fábulas} (1922);
\emph{Reinações de Narizinho} (1931); \emph{Viagem ao céu} (1932);
\emph{Caçadas de Pedrinho} (1933); \emph{História do Mundo para as}
\emph{Crianças} (1933); \emph{Emília no País da Gramática} (1934);
\emph{Aritmética da Emília} (1935); \emph{Memórias da Emília} (1936);
\emph{O Poço do Visconde} (1937); \emph{O Picapau Amarelo} (1939);
\emph{A Reforma da Natureza} (1941); \emph{A Chave do Tamanho} (1942);
\emph{Os doze trabalhos de Hércules}, dois volumes (1944);

\item Livros para adultos: \emph{Urupês} (1918); \emph{Cidades}
\emph{mortas} (1919); \emph{Ideias de Jeca Tatu} (1919); \emph{Negrinha}
(1920); \emph{O macaco que se fez homem} (1923); \emph{Mundo da lua}
(1923); \emph{O presidente negro/O choque das raças} (1926);
\emph{Ferro} (1931); \emph{América} (1932); \emph{O escândalo do
petróleo} (1936); \emph{A barca de Gleyre: quarenta anos de
correspondência literária entre Monteiro Lobato e Godofredo Rangel}
(1944).
\end{enumerate}

Lobato traduziu e adaptou diversas obras, entre as quais: \emph{Da
história da filosofia}, de Will Durand; \emph{Memórias}, de André
Maurois; \emph{Por quem os sinos dobram}, de Ernest Hemingway;
\emph{Crepúsculo dos ídolos e Anticristo}, de Friedrich Nietzsche;
\emph{Robinson Crusoé}, de Daniel Defoe; \emph{Mogli, o menino lobo}, de
Rudyard Kipling; \emph{Aventuras de Tom Sawyer}, de Mark Twain;
\emph{Pollyana}, de Eleanor H. Porter; \emph{Moby Dick}, de Herman
Melville; \emph{Tarzan}, de Edgar Rice Burroughs.

\section{Sobre a obra}

Certamente os leitores, se ainda não se debruçaram sobre as páginas de
Monteiro Lobato, já ouviram falar das personagens do \emph{Sítio do
Pica"-Pau Amarelo}, que ganharam vida nos livros e tempos depois nas
séries de \textsc{tv}. Mas sempre é tempo de conhecer, por meio do estilo da
escrita de Lobato, as aventuras e diálogos vividos pela boneca"-gente
Emília, pelo sábio espiga de milho Visconde de Sabugosa, pelas crianças
Pedrinho e Narizinho, pelas boas senhoras Tia Nastácia e Dona Benta,
pelo leitão Marquês de Rabicó, o Burro Falante, o Quindim. Então,
lembrando aqui a riqueza da obra infantojuvenil de Lobato, fica
reforçado o convite para agora conhecer uma seleção da obra voltada para
o público adulto, a que não faltam o espírito crítico, como o da Emília,
o humor, o suspense, reviravoltas e a sabedoria do desejo de compreender
os conflitos humanos.

Monteiro Lobato publicou os volumes de contos \emph{Urupês} (1918),
\emph{Cidades mortas} (1919), \emph{Negrinha} (1920) e \emph{O macaco
que se fez homem} (1923), tendo estampado alguns deles anteriormente na
imprensa, em especial na \emph{Revista do Brasil}, que ele adquiriu e
passou a editar em 1918. À edição original de \emph{Negrinha}, composta
do conto homônimo, de ``As fitas da vida'', ``O drama da geada'',
``Bugio moqueado'', ``O jardineiro Timóteo'' e ``O colocador de
pronomes'', Lobato depois acrescentou textos. Em 1923, ele preparou
\emph{Contos escolhidos}, edição voltada para o público de colégios,
cuja folha de rosto indicava: ``Adotado no Colégio Mackenzie e em outros
estabelecimentos de ensino, para leitura secundária''. E editou duas
antologias de seus contos, reagrupando"-os significativamente em:
\emph{Contos leves: Cidades mortas e outros} (1935) e \emph{Contos
pesados (Urupês, Negrinha e O macaco que se fez homem)} (1940).

Com os contos de Monteiro Lobato, conhecemos melhor a realidade social
brasileira e experienciamos os dramas e momentos poéticos de personagens
como o comprador de fazendas, o Jeca Tatu, o estafeta, Negrinha, o
jardineiro Timóteo, o galo Peva, o menino Pedrinho. Impressiona a
atualidade da matéria e dos conflitos configurados pela arte de Lobato,
que nos possibilita entender melhor as iniquidades da sociedade
brasileira de origem colonial e escravocrata, rir e ou chorar das
histórias narradas, algumas das quais foram adaptadas para o cinema.

Tendo em vista a riqueza dos contos lobatianos, mostrou"-se necessária a
organização desta edição segundo um critério temático e formal. Além das
organizadas por ele, de difícil acesso hoje, as antologias de Lobato
conhecidas se ocupam da seleção de textos e os ordenam em seções
conforme apenas a pertença aos volumes originais; ou seja, alguns de
\emph{Urupês}, outros de \emph{Cidades mortas}, de \emph{Negrinha} e de
\emph{O macaco que se fez homem}. Já o reagrupamento temático aqui
apresentado visa a chamar a atenção do leitor para a amplitude de
assuntos que motivaram a pena de Lobato e para a força do seu estilo,
capaz de dar forma a aspectos e tons diversos de questões semelhantes.

Apresentam"-se aqui, pois, quatro seções: ``Os protagonistas da
história'' traz narrativas que nos levam a compreender a história de
povos originários brasileiros, de africanos e seus descendentes, bem
como de senhores e fazendeiros decadentes, num país de origem
escravocrata, cuja marca continua sendo a violência; ``Amor invisível''
concentra a sensibilidade de Monteiro Lobato para com os seres
desamparados, especialmente crianças e animais, em uma sociedade que
privilegia o dinheiro e o consumismo; ``Os parasitas donos e os
olhodarruáveis'' tece a representação de personagens como o caboclo Jeca
Tatu e o estafeta Biriba, bem diferentes, mas todos explorados pelos
poderosos. ``Terra para rir, ou chorar'' reúne contos que revelam
insuficiências da realidade, em especial brasileira, por meio de um
efeito cômico, provocando a consciência crítica e a sensibilidade
estética do leitor. Em cada seção, os escritos de Lobato se sucedem em
ordem cronológica.

Destaca"-se o conto ``Negrinha'', publicado em 1920 no volume homônimo.
Desperta comoção e consciência histórica dos leitores: tece a
representação crítica da naturalização da violência escravocrata,
causada por uma senhora falsamente caridosa, contra uma menina órfã
descendente de escravos; e, a um tempo, expressa com poesia trágica o
sofrimento da criança, que, habituada brutalmente à privação de tudo, só
sorria, e para dentro, diante do relógio cuco e morreu de tristeza
depois de haver conhecido a maravilha que é brincar de boneca e saber
que tem uma alma.

Considerando, com Schopenhauer (\emph{Sobre o fundamento da moral}), ser
a compaixão o fundamento da moral, sobressai a lição ética do conto, em
que o narrador flagra o momento único em que a senhora cruel foi gente e
se apiedou da criança, permitindo"-lhe brincar com as outras.
Tragicamente, no conto, o momento raro de plenitude da órfã redundará na
percepção do vazio de sua vida e na sua morte.

Mas tal lição ética --- o avesso da violência, do racismo, de
preconceitos, da insensibilidade para com o outro --- fica para o leitor.
E assume dimensão política em contos como o atualíssimo ``Um suplício
moderno'', de 1916, em que, depois de conhecermos o tipo social do
estafeta, do encarregado de fazer entregas para os outros, acompanhamos
a história pessoal do estafeta Biriba. É uma história cômica e trágica:
segundo o neologismo criado por Lobato, ele não passa de um
``olhodarruável'', vergado aos donos do poder, apesar de interesseiro,
desejoso de ser parasita como os políticos que apoia.

``Quero ajudar o Brasil'' demanda atenção. Deixa ver como a linguagem,
inclusive a utilizada por Lobato, pode ter um teor racista. Porém, o
sentido do texto é antirracista: ressalta o amor do homem
afrodescendente pelo país, capaz do ato impensável de sacrificar seu
dinheiro. Ele tem a ``sublime serenidade'', pois é capaz de exercer um
``dever de consciência''.

Interessa perceber a forma como o narrador expressa a comoção que a
atitude do homem negro lhe causa. Combinam"-se o gesto de esconder as
lágrimas e uma formulação eufemística para o choro: ``Em certas ocasiões
só mesmo derrubando uma caneta e custando a achá"-la, porque há umas tais
glândulas que nos turvam os olhos com umas aguinhas
impertinentes\ldots{}''.

A força daquela atitude inacreditável, de sacrificar"-se pelo Brasil,
leva a refletir"-se sobre o limite ético que deve frear os interesses
comerciais e as propagandas: embora o propósito do narrador fosse
defender a campanha pelo petróleo, não era digno prejudicar aquele homem
singular que, mesmo pobre, priorizava o bem comum.

Sobressai a atualidade de certas frases desse texto de Lobato, que
desmascaram a estrutura iníqua da sociedade brasileira, desde a origem
colonial e escravocrata: ``A calúnia é a rainha da técnica''; ``Os
salários no Brasil são a miséria que sabemos''; ``mas no Brasil não há
negros ricos''; ``Essa coisa chamada Brasil, que é de vender, que até os
ministros vendem, ele queria ajudar\ldots{}''.

``Tragédia dum capão de pintos'' é o conto de Lobato escolhido por
Graciliano Ramos para integrar a antologia que ele organizou nos anos
1940, de contos das diversas regiões do país. O narrador configura a
singela história do amor do galo Peva pelos três filhotes órfãos, que
termina em revolta e tragédia, dada a indiferença dos homens e da
própria natureza por aquele afeto puro. De forma criativa, Lobato traduz
o olhar dos animais em relação aos homens, que inclui desconfiança,
incompreensão, submissão e medo. E deparamos com a falta de atenção dos
homens, cegos às manifestações de afeto dos animais e à dor que lhes
causam.

O melhor então é mergulhar nos contos de Lobato.

\section{Sobre o gênero}

\begin{quote}
O conto é, do ângulo dramático, unívoco, univalente. [\ldots]
Etimologicamente preso à linguagem teatral,
``drama'' significava ``ação''. E com o tempo passou a designar
toda peça destinada à representação. Na época romântica, dado o
princípio da fusão de gêneros, entendia-se por drama o misto de
tragédia e comédia. Transferido para a prosa de ficção, o termo
``drama'' entrou a significar ``conflito'', ``atrito''. Nesse caso,
``ação'' ``conflito'' se tonaram equivalentes, uma vez que toda
ação pressupõe conflito, e este, promove a ação, ou por meio dela
se manifesta; em suma, ambos se implicam mutuamente.

O conto é, pois, uma narrativa unívoca, univalente: constitui
uma \textit{unidade dramática}, uma \textit{célula dramática}, visto gravitar ao
redor de um só conflito, um só drama, uma só ação. Caracteriza-se,
assim, por conter \textit{unidade de ação}, tomada esta como a sequência de atos praticados pelos protagonistas, ou de acontecimentos de
que participam. A ação pode ser externa, quando as personagens se
deslocam no espaço e no tempo, e interna, quando o conflito se
localiza em sua mente.\footnote{\textsc{moisés}, Massaud. \textit{A criação literária}. São Paulo: Cultrix, 2006, p.\,40.}
\end{quote}

Partindo da definição de Massaud Moisés sobre o conto, evidencia"-se a principal característica desse gênero literário: a unidade de conflito, condensada em ações que se completam em um único enredo. Ao conto, ainda seguindo Moisés, aborrecem as divagações e os excessos, pois há uma concentração de efeitos e pormenores essenciais, em sua brevidade, para o bom funcionamento do conto.
Cada construção, cada palavra nesse gênero tem sua razão de existir, pois integra a economia global da narrativa.

Apesar da brevidade de sua forma, o conto desdobra"-se em muitas direções e implicações, e o faz a partir de elementos restritos: a unidade dramática, como já mencionada, assim como a presença de poucas personagens e a limitação espacial e temporal. Um ótimo exemplo é o conto ``Missa do galo'', de Machado de Assis, em que o narrador, Nogueira, conta a sua experiência de uma única noite na companhia de sua hospedeira, D.\,Conceição. Apesar de unidade temporal (a noite que antecede a Missa do galo), espacial (uma sala na casa de D.\,Conceição) e da redução dramática, basicamente, à interação entre duas personagens, Conceição e Nogueira, esse conto desdobra"-se em muitas direções. A companhia de Conceição desperta a sexualidade de Nogueira, e seu impacto é tão profundo que o narrador relembra aos leitores esse acontecimento de sua juventude. As intenções da anfitriã, narradas e, logo, distorcidas pela memória de Nogueira, também são ambíguas, levantando as mais diversas questões e interpretações.

Como reflete o escritor argentino Julio Cortázar, o conto consegue, de forma muito concisa, despertar ``uma realidade infinitamente mais vasta que a do seu mero argumento'', influindo ``em nós com uma força que nos faria suspeitar da modéstia do seu conteúdo aparente, da brevidade do seu texto''.\footnote{\textsc{CORTÁZAR}, Julio. \textit{Valise de cronópio}. São Paulo: Editora Perspectiva, 2008, p.\,155.}

Apesar da aparente banalidade do argumento, o conto abre essa possibilidade de desenvolver o tema em profundidade, em contraposição à aparente concisão narrativa. Realiza plenamente, assim, o que Cortázar define como o gênero do conto:

\begin{quote}
Um escritor argentino, muito amigo do boxe, dizia"-me que nesse combate que se trava entre um texto apaixonante e o leitor, o romance ganha sempre por pontos, enquanto que o conto deve ganhar por \textit{knock"-out}. É verdade, na medida em que o romance acumula progressivamente seus efeitos no leitor, enquanto que um bom conto é incisivo, mordente, sem trégua desde as primeiras frases. Não se entenda isto demasiado literalmente, porque o bom contista é um boxeador muito astuto, e muitos dos seus golpes iniciais podem parecer pouco eficazes quando, na realidade, estão minando já as resistências mais sólidas do adversário.
Tomem os senhores qualquer grande conto que seja de sua preferência, e analisem a primeira página. Surpreender"-me"-ia se encontrassem elementos gratuitos, meramente decorativos. O contista sabe que não pode proceder acumulativamente, que não tem o tempo por aliado; seu único recurso é trabalhar em profundidade, verticalmente, seja para cima ou para baixo do espaço literário.\footnote{Ibid., p.\,152.}
\end{quote}

\subsection{FONTES/EDIÇÕES DE CONTOS DE LOBATO}

\emph{Urupês}. São Paulo: Edição da Revista do Brasil, 1918; 1919; 1920;
São Paulo: Monteiro Lobato \& Cia. Editores, 1921; 1923; São Paulo:
Companhia Editora Nacional, 1937, Coleção Os Grandes Livros Brasileiros,
vol. 10; São Paulo: Livraria Martins Editora, 1944, Biblioteca de
Literatura Brasileira, vol. \textsc{viii}; \emph{Obras completas}, vol. 1, São
Paulo: Brasiliense, 1946\ldots{} 1994; São Paulo: Globo, 2007; 2 ed.
2009.

\emph{Cidades mortas (contos e impressões)}. São Paulo: Revista do
Brasil, 1919; 1920; São Paulo: Monteiro Lobato \& Cia., 1921; 1923;
\emph{Obras completas}, vol. 2. São Paulo: Brasiliense, 1946\ldots{};
1996; São Paulo: Globo, 2007; 2 ed., 2009.

\emph{Negrinha}: contos. São Paulo: Revista do Brasil; Monteiro Lobato
\& Cia., 1920; 1922; 1923; \emph{Obras completas}, vol. 3. São Paulo:
Brasiliense, 1946\ldots{} 1994; São Paulo: Globo, 2007; 2 ed., 2009.

\emph{O macaco que se fez homem}. São Paulo: Monteiro Lobato \& Cia.,
1923; 2 ed. São Paulo: Globo, 2010.

\emph{Contos escolhidos}. São Paulo: Cia Gráfico"-Editora Monteiro Lobato, 1923.

\emph{Contos leves: Cidades mortas e outros}. São Paulo: Companhia
Editora Nacional, 1935. Coleção Os Grandes Livros Brasileiros, n. 5.

\emph{Urupês, outros contos e coisas}. São Paulo: Companhia Editora
Nacional, 1943.

\emph{Contos pesados} (\emph{Urupês}, \emph{Negrinha} e \emph{O macaco
que se fez homem}). São Paulo, Rio de Janeiro: Companhia Editora Nacional, 1940.
Coleção Os Grandes Livros Brasileiros, vol. 2.

\emph{Contos completos}. São Paulo: Biblioteca Azul, 2014.

%\section{REFERÊNCIAS BIBLIOGRÁFICAS SOBRE MONTEIRO LOBATO}

\begin{bibliohedra}
\tit{AZEVEDO}, Carmen Lucia de; \textsc{camargos}, Marcia Mascarenhas de Rezende \&
\textsc{sacchetta}, Vladimir. \emph{Monteiro Lobato: furacão na Botocúndia}. 3
ed. São Paulo: Ed. Senac, 2001.

\tit{BEDÊ}, Ana Luiza Reis. \emph{Monteiro Lobato e a presença francesa em} A
barca de Gleyre. São Paulo: Annablume, 2007.

\tit{CAMARGOS}, Marcia. \emph{Juca e Joyce: memórias da neta de Monteiro
Lobato}. São Paulo: Moderna, 2007.

\tit{CAMPOS}, André Luiz Vieira de. \emph{A República do Picapau Amarelo: uma
leitura de Monteiro Lobato}. São Paulo: Martins Fontes, 1986.

\tit{CAVALHEIRO}, Edgard. \emph{A correspondência entre Monteiro Lobato e Lima
Barreto}. Rio de Janeiro: \textsc{mec}/Serviço de Documentação, 1955. Caderno de
Cultura, 76.

\titidem. \emph{Monteiro Lobato, vida e obra}. 2 tomos. São
Paulo: Companhia Editora Nacional, 1955.

\tit{CHIARELLI}, Tadeu. \emph{Um Jeca nos vernissages}. São Paulo: Edusp,
1995.

\tit{DANTAS}, Paulo (org.). \emph{Vozes do tempo de Lobato}. São Paulo: Traço
Editora, 1982.

\tit{DUARTE}, Lia Cupertino. \emph{Lobato humorista: a construção do humor nas
obras infantis de Monteiro Lobato}. São Paulo: Editora Unesp, 2006.

\tit{\emph{JECA TATU}}, Roteiro baseado no personagem Jeca Tatu, de Monteiro
Lobato. Argumento: Amácio Mazzaropi; diretor de produção: Felix Aidar;
direção: Milton Amaral. \textsc{pam} Filmes, Produções Amácio Mazzaropi, 1959.

\tit{KOSHIYAMA}, Alice Mitika. \emph{Monteiro Lobato: intelectual, empresário,
editor}. São Paulo: Edusp, 2006. Coleção Memória Editorial, 4.

\tit{LAJOLO}, Marisa. \emph{Monteiro Lobato, livro a livro: obra adulta}. São
Paulo: Editora Unesp, 2014.

\titidem. \emph{Monteiro Lobato: um brasileiro sob medida}. São
Paulo: Moderna, 2000.

\titidem \& \textsc{ceccantini}, João Luís (orgs.). \emph{Monteiro Lobato,
livro a livro: obra infantil}. São Paulo: Editora Unesp; Imprensa
Oficial do Estado de São Paulo, 2008.

\tit{LANDERS}, Vasda Bonafini. \emph{De Jeca a Macunaíma: Monteiro Lobato e o
modernismo}. Rio de Janeiro: Civilização Brasileira, 1988.

\tit{LUCA}, Tania Regina de. \emph{Revista do Brasil: um diagnóstico para a
(n)ação}. São Paulo: Editora da Unesp, 1999.

\tit{MARTINS}, Milena Ribeiro. \emph{Lobato edita Lobato: história das edições
dos contos lobatianos}. 429 p. Tese (doutorado) -- Universidade Estadual
de Campinas, Instituto de Estudos da Linguagem, Campinas, \textsc{sp}, 2003.
Disponível em: \emph{http://www.repositorio.unicamp.br/handle/REPOSIP/270369}.

\tit{MONTEIRO LOBATO}, \emph{site} oficial:
\emph{http://www.monteirolobato.com/}.

\tit{NUNES}, Cassiano. \emph{Monteiro Lobato: o editor do Brasil}. Rio de
Janeiro: Contraponto/Petrobras, 2000.

\titidem. \emph{Novos estudos sobre Monteiro Lobato}. Brasília:
Editora UnB, 1998.

\titidem. \emph{O patriotismo difícil: a correspondência entre
Monteiro Lobato e Artur Neiva}. São Paulo: Copidart, 1981.

\tit{\emph{O COMPRADOR DE FAZENDAS}}, Filme adaptado do conto de Monteiro
Lobato, do volume \emph{Urupês} (1918). Direção de Alberto Pieralisi.
São Paulo, Companhia Cinematográfica Maristela, 1950. Comédia, P\&B:
\emph{https://www.youtube.com/watch?v\=LcdfdfD9\_Bs}.

\tit{\emph{O COMPRADOR DE FAZENDAS}}, Filme adaptado do conto de Monteiro
Lobato, do volume \emph{Urupês} (1918). Direção de Alberto Pieralisi.
Rio de Janeiro, Embrafilme -- Empresa Brasileira de Filmes S.A., 1972:
\emph{https://www.youtube.com/watch?v\=C9OrDOQWm5o}.

\tit{PASSIANI}, Enio. \emph{Na trilha do Jeca: Monteiro Lobato e a formação do
campo literário no Brasil}. Bauru (\textsc{sp}): Edusc, Editora da Universidade
do Sagrado Coração/Associação Nacional de Pós"-Graduação em Ciências
Sociais, 2003.

\tit{ZILBERMAN}, Regina (org.). \emph{Atualidade de Monteiro Lobato: uma
revisão crítica}. Porto Alegre: Mercado Aberto, 1983.
\end{bibliohedra}