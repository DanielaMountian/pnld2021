\chapter{Apresentação}

Esta antologia é um convite para conhecer a obra de Monteiro Lobato
voltada para o público adulto, a que não faltam o espírito crítico, o
humor, o suspense, reviravoltas e a sabedoria do desejo de compreender
os conflitos humanos.

Monteiro Lobato publicou os volumes de contos \emph{Urupês} (1918),
\emph{Cidades mortas} (1919), \emph{Negrinha} (1920) e \emph{O macaco
que se fez homem} (1923), tendo estampado alguns deles anteriormente na
imprensa, em especial na \emph{Revista do Brasil}, que ele adquiriu e
passou a editar em 1918. À edição original de \emph{Negrinha}, composta
do conto homônimo, de ``As fitas da vida'', ``O drama da geada'',
``Bugio moqueado'', ``O jardineiro Timóteo'' e ``O colocador de
pronomes'', Lobato depois acrescentou textos. Em 1923, ele preparou
\emph{Contos escolhidos}, edição voltada para o público de colégios,
cuja folha de rosto indicava: ``Adotado no Colégio Mackenzie e em outros
estabelecimentos de ensino, para leitura secundária''. E editou duas
antologias de seus contos, reagrupando"-os significativamente em:
\emph{Contos leves: Cidades mortas e outros} (1935) e \emph{Contos
pesados (Urupês, Negrinha e O macaco que se fez homem)} (1940).

Com os contos de Monteiro Lobato, conhecemos melhor a realidade social
brasileira e experienciamos os dramas e momentos poéticos de personagens
como o comprador de fazendas, o Jeca Tatu, o estafeta, Negrinha, o
jardineiro Timóteo, o galo Peva, o menino Pedrinho. Impressiona a
atualidade da matéria e dos conflitos configurados pela arte de Lobato,
que nos possibilita entender melhor as iniquidades da sociedade
brasileira de origem colonial e escravocrata, rir e ou chorar das
histórias narradas, algumas das quais foram adaptadas para o cinema.

Tendo em vista a riqueza dos contos lobatianos, mostrou-se necessária a
organização desta edição segundo um critério temático e formal. Além das
organizadas por ele, de difícil acesso hoje, as antologias de Lobato
conhecidas se ocupam da seleção de textos e os ordenam em seções
conforme a pertença aos volumes originais; ou seja, alguns de
\emph{Urupês}, outros de \emph{Cidades mortas}, de \emph{Negrinha} e de
\emph{O macaco que se fez homem}. Já o reagrupamento temático aqui
apresentado visa a chamar a atenção do leitor para a amplitude de
assuntos que motivaram a pena de Lobato e para a força do seu estilo,
capaz de dar forma a aspectos e tons diversos de questões semelhantes.

Apresentam-se aqui, pois, quatro seções: ``Os protagonistas da
história'' traz narrativas que nos levam a compreender a história de
povos originários brasileiros, de africanos e seus descendentes, bem
como de senhores e fazendeiros decadentes, num país de origem
escravocrata, cuja marca continua sendo a violência; ``Amor invisível''
concentra a sensibilidade de Monteiro Lobato para com os seres
desamparados, especialmente crianças e animais, em uma sociedade que
privilegia o dinheiro e o consumismo; ``Os parasitas donos e os
olhodarruáveis'' tece a representação de personagens como o caboclo Jeca
Tatu e o estafeta Biriba, bem diferentes, mas todos explorados pelos
poderosos. ``Terra para rir, ou chorar'' reúne contos que revelam
insuficiências da realidade, em especial brasileira, por meio de um
efeito cômico, provocando a consciência crítica e a sensibilidade
estética do leitor. Em cada seção, os escritos de Lobato se sucedem em
ordem cronológica.

No conjunto desta antologia, o leitor encontrará, portanto, um panorama
dos contos de Monteiro Lobato, no qual poderá divisar o alcance da obra
desse autor, marcada pelo retrato sensível e fiel da realidade
brasileira do seu tempo, que não perdeu, em boa medida, a atualidade.


\part{\textsc{os protagonistas da história}}

\chapter{Negrinha\footnote[*]{A primeira edição
  de \emph{Negrinha} (1920) era composta dos seguintes contos:
  ``Negrinha'', ``As fitas da vida'', ``O drama da geada'', ``Bugio
  moqueado'', ``O jardineiro Timóteo'' e ``O colocador de pronomes''.
  Nota da edição de 1955.}}

Negrinha era uma pobre órfã de sete anos. Preta? Não; fusca, mulatinha
escura, de cabelos ruços e olhos assustados.

Nascera na senzala, de mãe escrava, e seus primeiros anos vivera"-os
pelos cantos escuros da cozinha, sobre velha esteira e trapos imundos.
Sempre escondida, que a patroa não gostava de crianças.

Excelente senhora, a patroa. Gorda, rica, dona do mundo, amimada dos
padres, com lugar certo na igreja e camarote de luxo reservado no céu.
Entaladas as banhas no trono (uma cadeira de balanço na sala de jantar),
ali bordava, recebia as amigas e o vigário, dando audiências, discutindo
o tempo. Uma virtuosa senhora, em suma --- ``dama de grandes virtudes
apostólicas, esteio da religião e da moral'', dizia o reverendo.

Ótima, a dona Inácia.

Mas não admitia choro de criança. Ai! Punha"-lhe os nervos em carne viva.
Viúva sem filhos, não a calejara o choro da carne de sua carne, e por
isso não suportava o choro da carne alheia. Assim, mal vagia, longe, na
cozinha, a triste criança, gritava logo nervosa:

--- Quem é a peste que está chorando aí?

Quem havia de ser? A pia de lavar pratos? O pilão? O forno? A mãe da
criminosa abafava a boquinha da filha e afastava"-se com ela para os
fundos do quintal, torcendo"-lhe em caminho beliscões de desespero.

--- Cale a boca, diabo!

No entanto, aquele choro nunca vinha sem razão. Fome quase sempre, ou
frio, desses que entanguem pés e mãos e fazem"-nos doer\ldots{}

Assim cresceu Negrinha --- magra, atrofiada, com os olhos eternamente
assustados. Órfã aos quatro anos, por ali ficou feito gato sem dono,
levada a pontapés. Não compreendia a ideia dos grandes. Batiam"-lhe
sempre, por ação ou omissão. A mesma coisa, o mesmo ato, a mesma palavra
provocava ora risadas, ora castigos. Aprendeu a andar, mas quase não
andava. Com pretexto de que às soltas reinaria no quintal, estragando as
plantas, a boa senhora punha"-a na sala, ao pé de si, num desvão da
porta.

--- Sentadinha aí, e bico, hein?

Negrinha imobilizava"-se no canto, horas e horas.

--- Braços cruzados, já, diabo!

Cruzava os bracinhos a tremer, sempre com o susto nos olhos. E o tempo
corria. E o relógio batia uma, duas, três, quatro, cinco horas --- um
cuco tão engraçadinho! Era seu divertimento vê"-lo abrir a janela e
cantar as horas com a bocarra vermelha, arrufando as asas. Sorria"-se
então por dentro, feliz um instante.

Puseram"-na depois a fazer crochê, e as horas se lhe iam a espichar
trancinhas sem fim.

Que ideia faria de si essa criança que nunca ouvira uma palavra de
carinho? Pestinha, diabo, coruja, barata descascada, bruxa, pata"-choca,
pinto gorado, mosca"-morta, sujeira, bisca, trapo, cachorrinha,
coisa"-ruim, lixo --- não tinha conta o número de apelidos com que a
mimoseavam. Tempo houve em que foi bubônica. A epidemia andava na berra,
como a grande novidade, e Negrinha viu"-se logo apelidada assim --- por
sinal que achou linda a palavra. Perceberam"-no e suprimiram"-na da lista.
Estava escrito que não teria um gostinho só na vida --- nem esse de
personalizar a peste\ldots{}

O corpo de Negrinha era tatuado de sinais, cicatrizes, vergões. Batiam
nele os da casa todos os dias, houvesse ou não houvesse motivo. Sua
pobre carne exercia para os cascudos, cocres e beliscões a mesma atração
que o ímã exerce para o aço. Mão em cujos nós de dedos comichasse um
cocre, era mão que se descarregaria dos fluidos em sua cabeça. De
passagem. Coisa de rir e ver a careta\ldots{}

A excelente dona Inácia era mestra na arte de judiar de crianças. Vinha
da escravidão, fora senhora de escravos --- e daquelas ferozes, amigas
de ouvir cantar o bolo e estalar o ``bacalhau''. Nunca se afizera ao
regime novo --- essa indecência de negro igual a branco e qualquer
coisinha: a polícia! ``Qualquer coisinha'': uma mucama assada ao forno
porque se engraçou dela o senhor; uma novena de relho\footnote{Surra de
  chicote durante nove dias. Nota da edição de 1946.} porque disse:
``Como é ruim a sinhá!''\ldots{}

O Treze de Maio tirou"-lhe das mãos o azorrague, mas não lhe tirou da
alma a gana. Conservava Negrinha em casa como remédio para os frenesis.
Inocente derivativo.

--- Ai! Como alivia a gente uma boa roda de cocres bem fincados!\ldots{}

Tinha de contentar"-se com isso, judiaria miúda, os níqueis da crueldade.
Cocres: mão fechada com raiva e nós de dedos que cantam no coco do
paciente. Puxões de orelha: o torcido, de despegar a concha (bom! bom!
bom! gostoso de dar!) e o a duas mãos, o sacudido. A gama inteira dos
beliscões: do miudinho, com a ponta da unha, à torcida do umbigo,
equivalente ao puxão de orelha. A esfregadela: roda de tapas, cascudos,
pontapés e safanões à uma --- divertidíssimo! A vara de marmelo,
flexível, cortante: para ``doer fino'' nada melhor!

Era pouco, mas antes isso do que nada. Lá de quando em quando vinha um
castigo maior para desobstruir o fígado e matar as saudades do bom
tempo. Foi assim com aquela história do ovo quente.

Não sabem? Ora! Uma criada nova furtara do prato de Negrinha --- coisa
de rir --- um pedacinho de carne que ela vinha guardando para o fim. A
criança não sofreou a revolta --- atirou"-lhe um dos nomes com que a
mimoseavam todos os dias.

--- ``Peste?'' Espere aí! Você vai ver quem é peste --- e foi contar o
caso à patroa.

Dona Inácia estava azeda, necessitadíssima de derivativos. Sua cara
iluminou"-se.

--- Eu curo ela! --- disse. E desentalando do trono as banhas foi para a
cozinha, qual perua choca, a rufar as saias.

--- Traga um ovo.

Veio o ovo. Dona Inácia mesma pô"-lo na água a ferver; e de mãos à cinta
gozando"-se na prelibação da tortura, ficou de pé uns minutos, à espera.
Seus olhos contentes envolviam a mísera criança que, encolhidinha a um
canto, aguardava trêmula alguma coisa de nunca visto. Quando o ovo
chegou a ponto, a boa senhora chamou:

--- Venha cá!

Negrinha aproximou"-se.

--- Abra a boca!

Negrinha abriu a boca, como o cuco, e fechou os olhos. A patroa, então,
com uma colher, tirou da água ``pulando'' o ovo e zás! na boca da
pequena. E antes que o urro de dor saísse, suas mãos amordaçaram"-na até
que o ovo arrefecesse. Negrinha urrou surdamente, pelo nariz. Esperneou.
Mas só. Nem os vizinhos chegaram a perceber aquilo. Depois:

--- Diga nomes feios aos mais velhos outra vez, ouviu, peste?

E a virtuosa dama voltou contente da vida para o trono, a fim de receber
o vigário que chegava.

--- Ah, monsenhor! Não se pode ser boa nesta vida\ldots{} Estou criando
aquela pobre órfã, filha da Cesária --- mas que trabalheira me dá!

--- A caridade é a mais bela das virtudes cristãs, minha senhora ---
murmurou o padre.

--- Sim, mas cansa\ldots{}

--- Quem dá aos pobres empresta a Deus.

A boa senhora suspirou resignadamente.

--- Inda é o que vale\ldots{}

Certo dezembro vieram passar as férias com Santa Inácia duas sobrinhas
suas, pequenotas, lindas meninas louras, ricas, nascidas e criadas em
ninho de plumas.

Do seu canto na sala do trono Negrinha viu"-as irromperem pela casa como
dois anjos do céu --- alegres, pulando e rindo com a vivacidade de
cachorrinhos novos. Negrinha olhou imediatamente para a senhora, certa
de vê"-la armada para desferir contra os anjos invasores o raio dum
castigo tremendo.

Mas abriu a boca: a sinhá ria"-se também\ldots{} Quê? Pois não era crime
brincar? Estaria tudo mudado --- e findo o seu inferno --- e aberto o
céu? No enlevo da doce ilusão, Negrinha levantou"-se e veio para a festa
infantil, fascinada pela alegria dos anjos.

Mas a dura lição da desigualdade humana lhe chicoteou a alma. Beliscão
no umbigo, e nos ouvidos o som cruel de todos os dias: ``Já para o seu
lugar, pestinha! Não se enxerga?''.

Com lágrimas dolorosas, menos de dor física que de angústia moral ---
sofrimento novo que se vinha acrescer aos já conhecidos ---, a triste
criança encorujou"-se no cantinho de sempre.

--- Quem é, titia? --- perguntou uma das meninas, curiosa.

--- Quem há de ser? --- disse a tia num suspiro de vítima. --- Uma
caridade minha. Não me corrijo, vivo criando essas pobres de Deus\ldots{} Uma
órfã. Mas brinquem, filhinhas, a casa é grande, brinquem por aí afora.

```Brinquem!' Brincar! Como seria bom brincar!'', refletiu com suas
lágrimas, no canto, a dolorosa martirzinha, que até ali só brincara em
imaginação com o cuco.

Chegaram as malas e logo:

--- Meus brinquedos! --- reclamaram as duas meninas.

Uma criada abriu"-as e tirou os brinquedos.

Que maravilha! Um cavalo de pau!\ldots{} Negrinha arregalava os olhos. Nunca
imaginara coisa assim tão galante. Um cavalinho! E mais\ldots{} Que é aquilo?
Uma criancinha de cabelos amarelos\ldots{} que falava ``mamã''\ldots{} que
dormia\ldots{}

Era de êxtase o olhar de Negrinha. Nunca vira uma boneca e nem sequer
sabia o nome desse brinquedo. Mas compreendeu que era uma criança
artificial.

--- É feita?\ldots{} --- perguntou extasiada.

E, dominada pelo enlevo, num momento em que a senhora saiu da sala a
providenciar sobre a arrumação das meninas, Negrinha esqueceu o
beliscão, o ovo quente, tudo, e aproximou"-se da criaturinha de louça.
Olhou"-a com assombrado encanto, sem jeito, sem ânimo de pegá"-la.

As meninas admiraram"-se daquilo.

--- Nunca viu boneca?

--- Boneca? --- repetiu Negrinha. --- Chama"-se Boneca?

Riram"-se as fidalgas de tanta ingenuidade.

--- Como é boba! --- disseram. --- E você, como se chama?

--- Negrinha.

As meninas novamente torceram"-se de riso; mas, vendo que o êxtase da
bobinha perdurava, disseram, apresentando"-lhe a boneca:

--- Pegue!

Negrinha olhou para os lados, ressabiada, com o coração aos pinotes. Que
aventura, santo Deus! Seria possível? Depois, pegou a boneca. E, muito
sem jeito, como quem pega o Senhor Menino, sorria para ela e para as
meninas, com assustados relanços de olhos para a porta. Fora de si,
literalmente\ldots{} Era como se penetrara no céu e os anjos a rodeassem, e
um filhinho de anjo lhe tivesse vindo adormecer ao colo. Tamanho foi o
seu enlevo que não viu chegar a patroa, já de volta. Dona Inácia
entreparou, feroz, e esteve uns instantes assim, presenciando a cena.

Mas era tal a alegria das hóspedas ante a surpresa estática de Negrinha,
e tão grande a força irradiante da felicidade desta, que o seu duro
coração afinal bambeou. E pela primeira vez na vida foi mulher.
Apiedou"-se.

Ao percebê"-la na sala Negrinha havia tremido, passando"-lhe num relance
pela cabeça a imagem do ovo quente e hipóteses de castigos ainda piores.
E incoercíveis lágrimas de pavor assomaram"-lhe aos olhos.

Falhou tudo isso, porém. O que sobreveio foi a coisa mais inesperada do
mundo --- estas palavras, as primeiras que ela ouviu, doces, na vida:

--- Vão todas brincar no jardim, e vá você também, mas veja lá, hein?

Negrinha ergueu os olhos para a patroa, olhos ainda de susto e terror.
Mas não viu mais a fera antiga. Compreendeu vagamente e sorriu.

Se alguma vez a gratidão sorriu na vida, foi naquela surrada carinha\ldots{}

Varia a pele, a condição, mas a alma da criança é a mesma --- na
princesinha e na mendiga. E para ambas é a boneca o supremo enlevo. Dá a
natureza dois momentos divinos à vida da mulher: o momento da boneca ---
preparatório, e o momento dos filhos --- definitivo. Depois disso, está
extinta a mulher.

Negrinha, coisa humana, percebeu nesse dia da boneca que tinha uma alma.
Divina eclosão! Surpresa maravilhosa do mundo que trazia em si e que
desabrochava, afinal, como fulgurante flor de luz. Sentiu"-se elevada à
altura de ente humano. Cessara de ser coisa --- e doravante ser"-lhe"-ia
impossível viver a vida de coisa. Se não era coisa! Se sentia! Se
vibrava!

Assim foi --- e essa consciência a matou.

Terminadas as férias, partiram as meninas levando consigo a boneca, e a
casa voltou ao ramerrão habitual. Só não voltou a si Negrinha. Sentia"-se
outra, inteiramente transformada.

Dona Inácia, pensativa, já a não atenazava tanto, e na cozinha uma
criada nova, boa de coração, amenizava"-lhe a vida.

Negrinha, não obstante, caíra numa tristeza infinita. Mal comia e
perdera a expressão de susto que tinha nos olhos. Trazia"-os agora
nostálgicos, cismarentos.

Aquele dezembro de férias, luminosa rajada de céu trevas adentro do seu
doloroso inferno, envenenara"-a.

Brincara ao sol, no jardim. Brincara!\ldots{} Acalentara, dias seguidos, a
linda boneca loura, tão boa, tão quieta, a dizer ``mamã'', a cerrar os
olhos para dormir. Vivera realizando sonhos da imaginação.
Desabrochara"-se de alma.

Morreu na esteirinha rota, abandonada de todos, como um gato sem dono.
Jamais, entretanto, ninguém morreu com maior beleza. O delírio rodeou"-a
de bonecas, todas louras, de olhos azuis. E de anjos\ldots{} E bonecas e
anjos remoinhavam"-lhe em torno, numa farândola do céu. Sentia"-se
agarrada por aquelas mãozinhas de louça --- abraçada, rodopiada.

Veio a tontura; uma névoa envolveu tudo. E tudo regirou em seguida,
confusamente, num disco. Ressoaram vozes apagadas, longe, e pela última
vez o cuco lhe apareceu de boca aberta.

Mas, imóvel, sem rufar as asas.

Foi"-se apagando. O vermelho da goela desmaiou\ldots{}

E tudo se esvaiu em trevas.

Depois, vala comum. A terra papou com indiferença aquela carnezinha de
terceira --- uma miséria, trinta quilos mal pesados\ldots{}

E de Negrinha ficaram no mundo apenas duas impressões. Uma cômica, na
memória das meninas ricas.

--- Lembras"-te daquela bobinha da titia, que nunca vira boneca?

Outra de saudade, no nó dos dedos de dona Inácia.

--- Como era boa para um cocre!\ldots{}

\chapter{O jardineiro Timóteo\footnote[*]{Texto de 1920, publicado no livro \emph{Negrinha}.}}


O casarão da fazenda era ao jeito das velhas moradias coloniais: frente
com varanda, uma ala e pátio interno. Neste ficava o jardim, também à
moda antiga, cheio de plantas antigas, cujas flores punham no ar um
saudoso perfume de antanho. Quarenta anos havia que lhe zelava dos
canteiros o bom Timóteo, um preto branco por dentro. Timóteo o plantou
quando a fazenda se abria e a casa inda cheirava a reboco fresco e
tintas de óleo recentes, e desde aí --- lá se iam quarenta anos ---
ninguém mais teve licença de pôr a mão em ``seu jardim''.

Verdadeiro poeta, o bom Timóteo.

Não desses que fazem versos, mas dos que sentem a poesia sutil das
coisas. Compusera, sem o saber, um maravilhoso poema onde cada plantinha
era um verso que só ele conhecia, verso vivo, risonho ao reflorir anual
da primavera, desmedrado e sofredor quando junho sibilava no ar os
látegos do frio. O jardim tornara"-se a memória viva da casa. Tudo nele
correspondia a uma significação familiar de suave encanto, e assim foi
desde o começo, ao riscarem"-se os canteiros na terra virgem ainda
recendente à escavação. O canteiro principal consagrara"-o Timóteo ao
``Sinhô"-Velho'', tronco da estirpe e generoso amigo que lhe dera carta
de alforria muito antes da Lei Áurea. Nasceu faceiro e bonito, cercado
de tijolos novos vindos do forno para ali inda quentes, e embutidos no
chão como rude cíngulo de coral; hoje, semidesfeitos pela usura do tempo
e tão tenros que a unha os penetra, esses tijolos esverdecem nos musgos
da velhice.

``Veludo de muro velho'' é como chama Timóteo a essa muscínea invasora,
filha da sombra e da umidade. E é bem isso, porque o musgo foge sempre
aos muros secos, vidrentos, esfogueados de sol, para estender
devagarinho o seu veludo prenunciador de tapera sobre os muros
alquebrados, de emboço já carcomido e todo aberto em fendas.

Bem no centro erguia"-se um nodoso pé de jasmim"-do"-cabo, de galhos negros
e copa dominante, ao qual o zeloso guardião nunca permitiu que outra
planta sobre"-excedesse em altura. Simbolizava o homem que o havia
comprado por dois contos de réis, dum importador de escravos de Angola.

--- Tenha paciência, minha negra! --- conversava ele com as roseiras de
setembro, teimosas em espichar para o céu brotos audazes. --- Tenha
paciência, que aqui ninguém olha de cima para o Sinhô"-Velho.

E sua tesoura afiada punha abaixo, sem dó, todos os rebentos temerários.

Cercando o jasmineiro havia uma coroa de periquitos, e outra menor de
cravinas. Mais nada.

--- Ele era homem simples, pouco amigo de complicações. Que fique ali
sozinho com o periquito e as irmãzinhas do cravo.

Dos outros canteiros, dois eram em forma de coração.

--- Este é o de Sinhazinha; e como ela um dia há de casar, fica a par
dele o canteiro do Sinhô"-Moço.

O canteiro de Sinhazinha era de todos o mais alegre, dando bem a imagem
de um coração de mulher rico de todas as flores do sentimento. Sempre
risonho, tinha a propriedade de prender os olhos de quantos penetravam
no jardim. Tal qual a moça, que desde menina se habituara a monopolizar
os carinhos da família e a dedicação dos escravos, chegando esta a ponto
de ao sobrevir a Lei Áurea nenhum ter ânimo de afastar"-se da fazenda.
Emancipação? Loucura! Quem, uma vez cativo de Sinhazinha, podia jamais
romper as algemas da doce escravidão?

Assim ela na família, assim o seu canteiro entre os demais. Livro
aberto, símbolo vivo, crônica vegetal, dizia pela boca das flores toda a
sua vidinha de moça. O pé de flor"-de"-noiva, primeira ``planta séria''
ali brotada, marcou o dia em que foi pedida em casamento. Até então só
vicejavam nele flores alegres de criança --- esporinhas, bocas"-de"-leão,
``borboletas'', ou flores amáveis da adolescência --- amores"-perfeitos,
damas"-entre"-verdes, beijos"-de"-frade, escovinhas, miosótis.

Quando lhe nasceu, entre dores, o primeiro filho, plantou Timóteo os
primeiros tufos de violeta.

--- Começa a sofrer\ldots{}

E no dia em que lhe morreu esse malogrado botãozinho de carne rósea, o
jardineiro, em lágrimas, fincou na terra os primeiros goivos e as
primeiras saudades. E fez ainda outras substituições: as alegres
damas"-entre"-verdes cederam o lugar aos suspiros"-roxos, e a sempre"-viva
foi para o canto onde viçavam as ridentes bocas"-de"-leão.

Já o canteiro de Sinhô"-Moço revelava intenções simbólicas de energia.

Cravos vermelhos em quantidade, roseiras fortes, ouriçadas de espinhos;
palmas"-de"-santa"-rita, de folhas laminadas; junquilhos nervosos.

E tudo mais assim.

Timóteo compunha os anais vivos da família, anotando nos canteiros, um
por um, todos os fatos de alguma significação. Depois, exagerando, fez
do jardim um canhenho de notas, o verdadeiro diário da fazenda.
Registrava tudo. Incidentes corriqueiros, pequenas rusgas de cozinha, um
lembrete azedo dos patrões, um namoro de mucama, um hóspede, uma geada
mais forte, um cavalo de estimação que morria --- tudo memorava ele, com
hieróglifos vegetais, em seu jardim maravilhoso.

A hospedagem de certa família do Rio --- pai, mãe e três sapequíssimas
filhas --- lá ficou assinalada por cinco pés de ora"-pro"-nóbis. E a venda
do pampa calçudo, o melhor cavalo das redondezas, teve a mudança de dono
marcada pela poda dum galho do jasmineiro.

Além desta comemoração anedótica, o jardim consagrava uma planta a cada
subalterno ou animal doméstico. Havia a roseira"-chá da mucama de
Sinhazinha; o sangue"-de"-adão do Tibúrcio cocheiro; a rosa"-maxixe da
mulatinha Cesária, sirigaita enredeira, de cara fuxicada como essa flor.
O Vinagre, o Meteoro, a Manjerona, a Teteia, todos os cães que na
fazenda nasceram e morreram, ali estavam lembrados pelo seu pezinho de
flor, um resedá, um tufo de violetas, uma touça de perpétuas. O cão mais
inteligente da casa, Otelo, morto hidrófobo, teve as honras duma
sempre"-viva rajada.

--- Quem há de esquecer um bicho daqueles, que até parecia gente?

Também os gatos tinham memória. Lá estava a cinerária da gata branca
morta nos dentes do Vinagre, e o pé de alecrim relembrativo do velho
gato Romão.

Ninguém, a não ser Timóteo, colhia flores naquele jardim. Sinhazinha o
tolerava desde o dia em que ele explicou:

--- Não \emph{sabem}, Sinhazinha! Vão lá e atrapalham tudo. Ninguém
\emph{sabe} apanhar flor\ldots{}

Era verdade. Só Timóteo sabia escolhê"-las com intenção e sempre de
acordo com o destino. Se as queriam para florir a mesa em dia de anos da
moça, Timóteo combinava os buquês como estrofes vivas. Colhia"-as
resmungando:

--- Perpétua? Não. Você não vai pra mesa hoje. É festa alegre. Nem você,
dona violetinha!\ldots{} Rosa"-maxixe? Ah! Ah! Tinha graça a Cesária em festa
de branco!\ldots{}

E sua tesoura ia cortando os caules com ciência de mestre. Às vezes
parava, a filosofar:

--- Ninguém se lembra hoje do anjinho\ldots{} Pra que, então, goivo nos
vasos? Quieto fique aqui o senhor goivo, que não é flor de vida, é flor
de cemitério\ldots{}

E sua linguagem de flores? Suas ironias, nunca percebidas de ninguém?
Seus louvores, de ninguém suspeitados? Quantas vezes não depôs na mesa,
sobre um prato, um aviso a um hóspede, um lembrete à patroa, uma censura
ao senhor, composto sob forma dum ramalhete? Ignorantes da língua do
jardim, riam"-se eles da maluquice do Timóteo, incapazes de lhe alcançar
o fino das intenções.

Timóteo era feliz. Raras criaturas realizam na vida mais formoso delírio
de poeta. Sem família, criara uma família de flores; pobre, vivia ao pé
de um tesouro.

Era feliz, sim. Trabalhava por amor, conversando com a terra e as
plantas --- embora a copa e a cozinha implicassem com aquilo.

--- Que tanto resmunga o Timóteo! Fica ali mamparreando horas, a
cochichar, a rir, como se estivesse no meio duma criançada\ldots{}

É que na sua imaginação as flores se transfiguravam em seres vivos.
Tinham cara, olhos, ouvidos\ldots{} O jasmim"-do"-cabo, pois não é que lhe dava
a bênção todas as manhãs? Mal Timóteo aparecia, murmurando ``A bênção,
Sinhô'', e já o velho, encarnado na planta, respondia com voz alegre:
``Deus te abençoe, Timóteo''.

Contar isso aos outros? Nunca! ``Está louco'', haviam de dizer. Mas bem
que as plantinhas falavam\ldots{}

--- E como não hão de falar, se tudo é criatura de Deus, homessa!\ldots{}

Também dialogava com elas.

--- Contentinha, hein? Boa chuva a de ontem, não?

--- \ldots{}

--- Sim, lá isso é verdade. As chuvas miúdas são mais criadeiras, mas
você bem sabe que não é tempo. E o grilo? Voltou? Voltou, sim, o
ladrão\ldots{} E aqui roeu mais esta folhinha\ldots{} Mas deixe estar que eu curo
ele!

E punha"-se a procurar o grilo. Achava"-o.

--- Seu malfeitor!\ldots{} Quero ver se continua agora a judiar das minhas
flores.

Matava"-o, enterrava"-o.

--- Vira esterco, diabinho!

Pelo tempo da seca era um regalo ver Timóteo a chuviscar amorosamente
sobre as flores com o seu velho regador.

--- O sol seca a terra? Bobice!\ldots{} Como se o Timóteo não estivesse aqui
de chovedor na mão.

``Chega também, ué! Então quer sozinho um regador inteiro? Boa moda! Não
vê que as esporinhas estão com a língua de fora?

``E esta boca"-de"-leão, ah! ah!, está mesmo com uma boca de cachorro que
correu veado! Tome lá, beba, beba!

``E você também, seu resedá, tome lá seu banho, pra depois namorar
aquela dona hortênsia, moça bonita do `zoio' azul\ldots{}''

E lá ia\ldots{}

Plantas novas que abrolhavam o primeiro botão punham alvoroço de noivo
no peito do poeta, que falava do acontecimento na copa, provocando as
risadinhas impertinentes da Cesária.

--- Diabo do negro velho, cada vez caducando mais! Conversa com flor
como se flor fosse gente.

Só a moça, com o seu fino instinto de mulher, lhe compreendia as
delicadezas do coração.

--- Está aqui, Sinhá, a primeira rainha"-margarida deste ano!

Ela fingia"-se extasiada e punha a flor no corpete.

--- Que beleza!

E Timóteo ria"-se, feliz, feliz\ldots{}

Certa vez falou"-se na reforma do jardim.

--- Precisamos mudar isto --- lembrou o moço, de volta dum passeio a São
Paulo. --- Há tantas flores modernas, lindas, enormes, e nós toda a vida
com estas cinerárias, estas esporinhas, estas flores caipiras\ldots{} Vi lá
crisandálias magníficas, crisântemos deste tamanho e uma rosa nova,
branca, tão grande que até parece flor artificial.

Quando soube da conversa, Timóteo sentiu gelo no coração. Foi agarrar"-se
com a moça. Ele também conhecia essas flores de fora, vira crisântemos
em casa do coronel Barroso, e as tais dálias mestiças no peito duma
faceira, no leilão do Espírito Santo.

--- Mas aquilo nem é flor, Sinhá! Coisas da estranja que o Canhoto
inventa para perder as criaturas de Deus. Eles lá que plantem. Nós aqui
devemos zelar das plantas de família. Aquela dália rajada, está vendo? É
singela, não tem o crespo das dobradas; mas quem troca uma menina de
sainha de chita cor"-de"-rosa por uma semostradeira da cidade, de muita
seda no corpo mas sem fé no coração? De manhã ``fica assim'' de abelhas
e cuitelos em roda dela!\ldots{} E eles sabem, eles não ignoram quem merece.
Se as das cidades fossem de mais estimação, por que é que esses
bichinhos de Deus ficam aqui e não vão pra lá? Não, Sinhá! É preciso
tirar essa ideia da cabeça de Sinhô"-Moço. Ele é criança ainda, não sabe
a vida. É preciso respeitar as coisas de dantes\ldots{}

E o jardim ficou.

Mas um dia\ldots{} Ah! Bem sentira"-se Timóteo tomado de aversão pela família
dos ora"-pro"-nóbis! Pressentimento puro\ldots{} O ora"-pro"-nóbis pai voltou e
esteve ali uma semana em conciliábulo com o moço. Ao fim desse tempo,
explodiu como bomba a grande notícia: estava negociada a fazenda,
devendo a escritura passar"-se dentro de poucos dias.

Timóteo recebeu a nova como quem recebe uma sentença de morte. Na sua
idade, tal mudança lhe equivalia a um fim de tudo. Correu a agarrar"-se à
moça, mas desta vez nada puderam contra as armas do dinheiro os seus
pobres argumentos de poeta.

Vendeu"-se a fazenda. E certa manhã viu Timóteo arrumarem"-se no trole os
antigos patrões, as mucamas, tudo o que constituía a alma do velho
patrimônio.

--- Adeus, Timóteo! --- disseram alegremente os senhores"-moços,
acomodando"-se no veículo.

--- Adeus! Adeus!\ldots{}

E lá partiu o trole, a galope\ldots{} Dobrou a curva da estrada\ldots{} Sumiu"-se
para sempre\ldots{}

Pela primeira vez na vida Timóteo esqueceu de regar o jardim. Quedou"-se
plantado a um canto, a esmoer o dia inteiro o mesmo pensamento doloroso:

--- Branco não tem coração\ldots{}

Os novos proprietários eram gente da moda, amigos do luxo e das
novidades. Entraram na casa com franzimentos de nariz para tudo.

--- Velharias, velharias\ldots{}

E tudo reformaram. Em vez da austera mobília de cabiúna, adotaram móveis
pechisbeques, com veludinhos e frisos. Determinaram o empapelamento das
salas, a abertura de um \emph{hall}, mil coisas esquisitas\ldots{} Diante do
jardim, abriram"-se em gargalhadas.

--- É incrível! Um jardim destes, cheirando a Tomé de Sousa, em pleno
século das crisandálias!

E correram"-no todo, a rir, como perfeitos malucos.

--- Olha, Yvette, esporinhas! É inconcebível que inda haja esporinhas no
mundo!

--- E periquito, Odete! Pe"-ri"-qui"-to!\ldots{} --- disse uma das moças,
torcendo"-se em gargalhadas.

Timóteo ouvia aquilo com mil mortes na alma. Não restava dúvida, era o
fim de tudo, como pressentira: aqueles bugres da cidade arrasariam a
casa, o jardim e o mais que lembrasse o tempo antigo. Queriam só o
moderno.

E o jardim foi condenado. Mandariam vir o Ambrogi para traçar um plano
novo de acordo com a arte moderníssima dos jardins ingleses. Reformariam
as flores todas, plantando as últimas criações da floricultura alemã.
Ficou decidido assim.

--- E para não perder tempo, enquanto o Ambrogi não chega ponho aquele
macaco a me arrasar isto --- disse o homem apontando para Timóteo.

--- Ó tição, vem cá!

Timóteo aproximou"-se, com ar apatetado.

--- Olha, ficas encarregado de limpar este mato e deixar a terra
nuazinha. Quero fazer aqui um lindo jardim. Arrasa"-me isto bem
arrasadinho, entendes?

Timóteo, trêmulo, mal pôde engrolar uma palavra:

--- Eu?\ldots{}

--- Sim, tu! Por que não?

O velho jardineiro, atarantado e fora de si, repetiu a pergunta:

--- Eu? Eu, arrasar o jardim?

O fazendeiro encarou"-o, espantado da sua audácia, sem nada compreender
daquela resistência.

--- Eu? Pois me acha com cara de criminoso?

E não podendo mais conter"-se explodiu num assomo estupendo de cólera ---
o primeiro e o único de sua vida.

--- Eu vou mas é embora daqui, morrer lá na porteira como um cachorro
fiel. Mas olhe, moço, que hei de rogar tanta praga que isto há de virar
uma tapera de lacraias! A geada há de torrar o café. A peste há de levar
até as vacas de leite!

Não há de ficar aqui nem uma galinha, nem um pé de vassoura! E a família
amaldiçoada, coberta de lepra, há de comer na gamela com os cachorros
lazarentos!\ldots{} Deixa estar, gente amaldiçoada! Não se assassina assim
uma coisa que dinheiro nenhum paga. Não se mata assim um pobre negro
velho que tem dentro do peito uma coisa que lá na cidade ninguém sabe o
que é. Deixa estar, branco de má casta! Deixa estar, caninana! Deixa
estar!\ldots{}

E fazendo com a mão espalmada o gesto fatídico, saiu às arrecuas,
repetindo cem vezes a mesma ameaça:

--- Deixa estar! Deixa estar!\ldots{}

E longe, na porteira, ainda espalmava a mão para a fazenda, num gesto
mudo:

--- Deixa estar!\ldots{}

Anoitecia. Os curiangos andavam a espacejar silenciosos voos de sombra
pelas estradas desertas. O céu era todo um recamo fulgurante de
estrelas. Os sapos coaxavam nos brejos e vaga"-lumes silenciosos piscavam
piques de luz no sombrio das capoeiras.

Tudo adormecera na terra, em breve pausa de vida para o ressurgir do dia
seguinte.

Só não ressurgirá Timóteo. Lá agoniza ao pé da porteira. Lá morre. E lá
o encontrará a manhã enrijecido pelo relento, de borco na grama
orvalhada, com a mão estendida para a fazenda num derradeiro gesto de
ameaça:

--- Deixa estar!\ldots{}

\chapter{Os negros\footnote[*]{Texto de 1922, publicado no livro \emph{Negrinha}.}}

\section*{I}

\noindent{}Viajávamos certa vez pelas regiões estéreis por onde há um século,
puxado pelo Negro, o carro triunfal de Sua Majestade o Café passou,
quando grossas nuvens reunidas no céu entraram a desmanchar"-se.

Sinal certo de chuva.

Para confirmá"-lo, um vento brusco, raspante, veio quebrar o mormaço,
vascolejando a terra como a preveni"-la do iminente banho meteórico.
Remoinhos de poeira sorviam folhas secas e gravetos, que lá
torvelinhavam em espirais pelas alturas.

Sofreando o animal, parei, a examinar o céu.

--- Não há dúvida --- disse ao meu companheiro ---, têmo"-la e boa! O
remédio é acoutar"-nos quanto antes nalgum socavão, que água vem aí de
rachar.

Circunvaguei o olhar em torno. Morraria áspera a perder"-se de vista, sem
uma casota de palha a acenar"-nos com um ``Vem cá''.

--- E agora? --- exclamou desnorteado o Jonas, marinheiro de primeira
viagem que tudo fiava da minha experiência.

--- Agora é galopar. Atrás deste espigão fica uma fazenda em ruínas, de
má nota, mas único oásis possível nesta emergência. Casa do Inferno,
chama"-lhe o povo.

--- Pois toca para o inferno, já que o céu nos ameaça --- retorquiu
Jonas, dando de esporas e seguindo"-me por um atalho.

--- Tens coragem? --- gritei"-lhe. --- Olha que é casa mal"-assombrada!\ldots{}

--- Bem"-vinda seja. Anos há que procuro uma, sem topar coisa que preste.
Correntes que se arrastam pela calada da noite?

--- Dum preto velho que foi escravo do defunto capitão Aleixo, fundador
da fazenda, ouvi coisas de arrepiar\ldots{}

Jonas, a criatura mais gabola deste mundo, não perdeu vasa duma
pacholice:

--- De arrepiar a ti, que a mim, bem sabes, só me arrepiam correntes de
ar\ldots{}

--- Acredito, mas toca, que o dilúvio não tarda.

O céu enegrecera por igual. Um relâmpago fulgurou, seguido de formidável
ribombo, que lá se foi às cabeçadas pelos morros até perder"-se distante.
E os primeiros pingos vieram, escoteiros, pipocar no chão ressecado.

--- Espora, espora!

Em minutos vingávamos o espigão, de cujo topo vimos a casaria maldita,
tragada a meio pelo mataréu invasor. Os pingões mais e mais se
amiudavam, e já eram água de molhar quando a ferradura das bestas
estrepitou, com faíscas, no velho terreiro de pedra. Sururucados por ele
adentro rumo a um telheiro em aberto, lá apeamos afinal, esbaforidos,
mas a salvo da molhadela.

E as bategas vieram, furiosas, em cordas d'água a prumo, como devia ser
no chuveiro bíblico do dilúvio universal.

Examinei o couto. Telheiro de carros e tropa, derruído em parte. Os
esteios, da cabiúna eterna, tinham os nabos\footnote{Parte mais grossa
  dum esteio, que fica enterrada. Nota da edição de 1946.} à mostra ---
tantos enxurros correram por ali erodindo o solo. Por eles marinhava a
caetaninha,\footnote{Melão"-de"-são"-caetano. Nota da edição de 1946.} essa
mimosa alcatifa dos tapumes, toda rosetada de flores amarelas e
pingentada de melõezinhos de bico, cor de canário.

Também aboboreiras viçavam na tapera, galgando vitoriosas pelos espeques
para enfolharem no alto, entremeio das ripas e caibros a nu. Suas flores
grandalhudas, tão caras às mamangavas, manchavam de amarelo"-pálido o tom
cru da folhagem verde"-negra.

Fora, a pouca distância do telheiro, a ``casa"-grande'' se erguia,
vislumbrada apenas através da cortina d'água.

E a água a cair.

E a trovoada a escalejar seus ecos pela morraria intérmina.

E o meu amigo, tão calmo sempre e alegre, a exasperar"-se:

--- Raio de peste de tempo desgraçado! Já não posso almoçar em Vassouras
amanhã, como pretendia.

--- Chuva de corda não dura hora --- consolei"-o.

--- Sim, mas será possível alcançar o tal pouso do Alonso ainda hoje?

Consultei o pulso.

--- Cinco e meia. É tarde. Em vez de Alonso, temos que gramar o Aleixo.
E dormir com as bruxas, mais a alma do capitão infernal.

--- Inda é o que nos vale --- filosofou o impenitente Jonas. --- Que
assim, ao menos, haverá o que contar amanhã.

\section*{II}

O temporal durou meia hora e ao cabo amainou, com os relâmpagos
espacejados e os trovões a roncarem muito longe dali. Apesar de próxima
a noite, inda tínhamos uma hora de luz para sondar o terreiro.

--- Há de morar aqui por perto algum urumbeva --- disse eu. --- Não
existe tapera sem lacraia. Vamos à cata desse abençoado urupê.

Encavalgamos de novo e saímos a rodear a fazenda.

--- Acertaste, amigo! --- exclamou de repente Jonas, ao divisar uma
casinhola erguida entre moitas, a duzentos passos de distância. ---
Bico"-de"-papagaio, pé de mamão, terreiro limpo; é o urumbeva sonhado!

Para lá nos dirigimos e já do terreiro gritamos o ``Ó de casa!''. Uma
porta abriu"-se, enquadrando o vulto dum negro velho, de cabelos ruços.
Com que alegria o saudei\ldots{}

--- Pai Adão, viva!

--- Vassuncristo! --- respondeu o preto.

Era dos legítimos\ldots{}

--- Pra sempre! --- gritei eu. --- Estamos aqui trancados pela chuva e
impedidos de prosseguir viagem. Tio Adão há de\ldots{}

--- Tio Bento, pra servir os brancos.

--- Tio Bento há de arranjar"-nos pouso por esta noite.

--- E boia --- acrescentou Jonas ---, visto que temos a caixa das
empadas a tinir.

O excelente negro sorriu"-se, com a gengiva inteira à mostra, e disse:

--- Pois é apeá. Casa de pobre, mas de bom coração. Quanto a ``de
comer'', comidinha de negro velho, já sabe\ldots{}

Apeamos, alegremente.

--- Angu? --- chasqueou o Jonas.

O negro riu"-se:

--- Já se foi o tempo do angu com ``bacalhau''\ldots{}\footnote{Chicote de
  vários rabos com que se chibatavam os negros. Nota da edição de 1946.}

--- E não deixou saudades, hein, tio Bento?

--- Saudades não deixou, não, eh! eh!\ldots{}

--- Para vocês, pretos; porque entre os brancos muitos há que choram
aquele tempo de vacas gordas. Não fosse o Treze de Maio e não estava
agora eu aqui a arrebentar as unhas neste raio de látego, que encruou
com a chuva e não desata. Era servicinho do pajem\ldots{}

Desarreamos as bestas e depois de soltá"-las penetramos na casinha,
sobraçando os arreios. Vimos, então, que era pequena demais para nos
abrigar aos três.

--- Amigo Bento, olha, não cabemos tanta gente aqui. O melhor é
acomodar"-nos na casa"-grande, que isto cá não é casa de bicho"-homem, é
ninho de cuitelo\ldots{}

--- Os brancos querem dormir na casa mal"-assombrada? --- exclamou
admirado o preto. --- Não aconselho, não. Alguém já fez isso mas se
arrependeu depois.---

Arrepender"-nos"-emos também depois, amanhã, mas já com a dormida no papo
--- disse Jonas.

E como o preto abrisse a boca:

--- Você não sabe o que é coragem, tio Bento. Escoramos sete. E almas do
outro mundo, então, uma dúzia! Vamos lá. Está aberta a casa?

--- A porta do meio emperrou, mas à força de ombros deve abrir.

--- Abandonada há muito tempo?

--- ``Quizano!'' Desde que morreu o último filho do capitão Aleixo ficou
assim, ninho de morcego e suindara.

--- E por que a abandonaram?

--- ``Descabeçada'' do moço. Pra mim, castigo de Deus. Os filhos pagam a
ruindade dos pais, e o capitão Aleixo, Deus que me perdoe, foi mau, mau,
mau inteirado. Tinha fama! Aqui em dez léguas de roda, quem queria
ameaçar um negro reinador era só dizer: ``Espera, diabo, que te vendo
pro capitão Aleixo''. O negro ficava que nem uma seda!\ldots{} Mas o que ele
fez, os filhos pagaram. Eram quatro: Sinhozinho, o mais velho, que
morreu ``masgaiado'' num trem; Nhá Zabelinha\ldots{}

\section*{III}

Enquanto o preto falava, insensivelmente fomos caminhando para a casa
maldita.

Era o casarão clássico das antigas fazendas negreiras. Assobradado,
erguido em alicerces e muramento de pedra até meia altura e daí por
diante de pau a pique. Esteios de cabreúva, entremostrando"-se picados a
enxó nos trechos donde se esboroara o reboco. Janelas e portas em arco,
de bandeiras em pandarecos. Pelos interstícios da pedra amoitavam"-se as
samambaias; e nas faces de sombra, avenquinhas raquíticas. Num cunhal
crescia anosa figueira, enlaçando as pedras na terrível cordoalha
tentacular. À porta de entrada ia ter uma escadaria dupla, com alpendre
em cima e parapeito esborcinado.

Pus"-me a olhar para aquilo, invadido da saudade que sempre me causam
ruínas, e parece que em Jonas a sensação era a mesma, pois que o vi
muito sério, de olhar pregado na casa, como quem recorda. Perdera o bom
humor, o espírito brincalhão de inda há pouco. Emudecera.

--- Está visto --- murmurei depois dalguns minutos. --- Vamos agora à
boia, que não é sem tempo.

Voltamos.

O negro, que não parara de falar, dizia agora de sua vida ali.

--- Morreu tudo, meu branco, e fiquei eu só. Tenho umas plantas na beira
do rio, palmito no mato e uma paquinha lá de vez em quando na ponta do
chuço. Como sou só\ldots{}

--- Só, só, só?

--- ``Suzinho, suzinho!'' A Merência morreu, faz três anos. Os filhos,
não sei deles. Criança é como ave: cria pena, avoa. O mundo é grande ---
andam pelo mundo avoando\ldots{}

--- Pois, amigo Bento, saiba que você é um herói e um grande filósofo
por cima, digno de ser memorado em prosa ou verso pelos homens que
escrevem nos jornais. Mas filósofo de pior espécie está me parecendo
aquele sujeito\ldots{} --- concluí referindo"-me ao Jonas, que se atrasara e
parara de novo em contemplação da casa.

Gritei"-lhe:

--- Mexe"-te, ó poeta que ladras às lagartixas! Olha que saco vazio não
se põe de pé, e temos dez léguas a engolir amanhã.

Respondeu"-me com um gesto vago e ficou"-se no lugar, imóvel.

Larguei mão do cismabundo e entrei na casinhola do preto, que, acendendo
luz --- um candeeiro de azeite --- foi ao borralho buscar raízes de
mandioca assada. Pô"-las sobre um mocho, quentinhas, dizendo:

--- É o que há. Isto e um restico de paca moqueada.

--- E achas pouco, Bento? --- disse eu, metendo os dentes na raiz
deliciosa. --- Não sabes que se não fosse tua providencial presença
teríamos de manducar viradinho de brisas com torresmos de zéfiros até
alcançarmos a venda do Alonso amanhã? Deus que te abençoe e te dê no céu
um mandiocal imenso, plantado pelos anjos.

\section*{IV}

Caíra de todo a noite. Que céu! Alternavam estrelas vivíssimas com
rebojos negros de nuvens acasteladas. Na terra, escuridão de breu,
rasgada de piques de luz pelas estrelinhas avoantes. Uma coruja berrava
longe, num esgalho morto de perobeira.

Que solidão, que espessura de trevas é a de uma noite assim, no deserto!
Nesses momentos é que um homem bem compreende a origem tenebrosa do
Medo\ldots{}

\section*{V}

Acabada a magra refeição, observei ao preto:

--- Agora, amigo, é agarrarmos estas mantas e pelegos, mais a luz, e
irmo"-nos à casa"-grande. Dormes lá conosco, à guisa de para"-raios de
almas. Topas?

Contente de ser"-nos útil, tio Bento sobraçou a quitanda e deu"-me a levar
o candeeiro. E lá fomos pelo escuro da noite, a chapinhar nas poças e na
grama empapada.

Encontrei Jonas no mesmo lugar, absorto em frente à casa.

--- Estás louco, rapaz? Não comeres, tu que estalavas de fome, e ficares
aí como perereca diante da cascavel?

Jonas olhou"-me dum modo estranho e como única resposta esganiçou um
``deixa"-me''. Fiquei a encará"-lo por uns instantes, deveras desnorteado
por tão inexplicável atitude. E foi assim, de rugas na testa, que
galguei a escadaria musgosa do casarão.

Estava perra de fato a porta, como dissera o negro, mas com valentes
ombradas abria"-a no preciso para dar passagem a um homem. Mal entramos,
morcegos às dezenas, assustados com a luz, debandaram às tontas, em
voejos surdos.

--- Macacos me lambam se isto aqui não é o quartel"-general de todos os
ratos de asas deste e dos mundos vizinhos!

--- E das suindaras, patrãozinho. Mora aqui um bandão delas que até dá
medo --- acrescentou o preto, ao ouvir"-lhes os pios no forro.

A sala de espera toava com o restante da fazenda. Paredes lagarteadas de
rachas, escorridas de goteiras, com vagos vestígios do papel. Móveis
desaparelhados, duas cadeiras Luís \textsc{xv}, de palhinha rota, e mesa de
centro do mesmo estilo, com o mármore sujo pelo guano dos morcegos. No
teto, tábuas despregadas, entremostrando rombos escuros.

Lúgubre\ldots{}

--- Tio Bento --- disse eu, procurando iludir com palavras a tristeza do
coração, --- isto aqui cheira"-me à sala nobre do sabá das bruxas. Que
não venham hoje atropelar"-nos, nem apareça a alma do capitão"-mor a nos
infernizar o sono. Não é verdade que a alma do capitão"-mor vagueia por
aqui a desoras?

--- Dizem --- respondeu o preto. --- Dizem que aparece ali na casa do
tronco, não às dez, mas à meia"-noite, e que sangra as unhas a arranhar
as paredes\ldots{}

--- E depois vem cá arrastar correntes pelos corredores, hein? Como é
pobre a imaginativa popular! Sempre e em toda parte a mesma ária das
correntes arrastadas! Mas vamos ao que serve. Não haverá um quarto
melhor do que isto, nesta hospedaria de mestre tinhoso?

--- Haver, há --- trocadilhou sem querer o preto ---, mas é o quarto do
capitão"-mor. Tem coragem?

--- Ainda não estás convencido, Bento, de que sou um poço de coragem?

--- Poço tem fundo --- retrucou ele, sorrindo filosoficamente.

--- O quarto é aqui à direita.

Dirigi"-me para lá. Entrei. Quarto amplo e em melhor estado que a sala de
espera. Guarneciam"-no duas velhas marquesas de palhinha bolorenta, além
de várias cadeiras rotas. Na parede, um retrato na moldura clássica da
época, dourada, de cantos redondos, com florões. Limpei com o lenço a
poeira acumulada no vidro e vi que era um daguerreótipo esmaiado,
representando imagem de mulher.

Bento percebeu a minha curiosidade e explicou:

--- É o retrato da filha mais velha do capitão Aleixo, Nhá Zabé, uma
moça tão desgraçada\ldots{}

Contemplei longamente aquela antigualha venerável, vestida à moda da
época.

--- Tempo das anquinhas, hein, Bento? Lembras"-te das anquinhas?

--- Se me lembro! A sinhá velha, quando vinha da cidade, era assim que
elaandava, que nem uma perua choca\ldots{}

Recoloquei na parede o daguerreótipo e pus"-me a arranjar as marquesas,
arrumando numa e noutra pelegos, à guisa de travesseiros. Em seguida fui
ao alpendre, de luz na mão, a ver se amadrinhava o meu relapso
companheiro. Era demais aquela maluquice! Não jantar e agora ficar"-se
ali ao relento\ldots{}

\section*{VI}

Perdi meu requebrado. Chamei"-o, mas nem com o ``deixa"-me'' respondeu
desta vez. Tal atitude pôs"-me seriamente apreensivo.

--- Se lhe desarranja a cabeça, aqui nestas alturas\ldots{}

Torturado por esta ideia, não pude sossegar. Confabulei com o Bento e
resolvemos sair em procura do transviado.

Fomos felizes. Encontramo"-lo no terreiro, em face da antiga casa do
tronco. Estava imóvel e mudo.

Ergui"-lhe a luz à altura do rosto. Que estranha expressão a sua! Não
parecia o mesmo --- não era o mesmo. Deu"-me a impressão de retesado no
último arranco duma luta suprema, com todas as energias crispadas numa
resistência feroz. Sacudi"-o com violência.

--- Jonas! Jonas!

Inútil. Era um corpo largado da alma. Era um homem ``vazio de si
próprio''. Assombrado com o fenômeno, concentrei todas as minhas forças
e, ajudado pelo Bento, trouxe"-o para casa.

Ao penetrar na sala de espera, Jonas estremeceu; parou, arregalou os
olhos para a porta do quarto. Seus lábios tremiam. Percebi que
articulavam palavras incompreensíveis. Precipitou"-se, depois, para o
quarto e, dando com o daguerreótipo de Izabel, agarrou"-o com frenesi,
beijou"-o, rompido em choro convulsivo. Em seguida, como exausto duma
grande luta, caiu sobre a marquesa, prostrado, sem articular nenhum som.

Inutilmente interpelei"-o, procurando a chave do enigma. Jonas permanecia
vazio\ldots{} Tomei"-lhe o pulso: normal. A temperatura: boa. Mas largado,
como um corpo morto.

Fiquei ao pé dele uma hora, com mil ideias a me azoinarem a cabeça. Por
fim, vendo"-o calmo, fui ter com o preto.

--- Conta"-me o que sabes desta fazenda --- pedi"-lhe. --- Talvez que\ldots{}

Meu pensamento era deduzir das palavras do negro algo explicativo da
misteriosa crise.

\section*{VII}

Nesse entremeio zangara de novo o tempo. As nuvens recobriam
inteiramente o céu, transformado num saco de carvão. Os relâmpagos
voltaram a fulgurar, longínquos, acompanhados de reboos surdos. E para
que ao horror do quadro nenhum tom faltasse, a ventania cresceu, uivando
lamentosa nas casuarinas.

Fechei a janela.

Mesmo assim, pelas frinchas o assobio lúgubre entrava a me ferir os
ouvidos\ldots{}

Bento falou em voz baixa, receoso de despertar o doente. Contou como
viera ali, comprado pelo próprio capitão Aleixo, na feira de escravos do
Valongo, molecote ainda. Disse da formação da fazenda e do caráter cruel
do senhor.

--- Era mau, meu branco, como deve ser mau o Canhoto. Judiava da gente à
toa, pelo gosto de judiar. No começo não era assim, mas foi piorando com
o tempo.

``No caso da Liduína\ldots{} A Liduína era uma bonita crioula aqui da
fazenda. Muito viva, desde bem criança passou da senzala pra
casa"-grande, como mucama de Sinhazinha Zabé\ldots{}

``Isso foi\ldots{} deve fazer sessenta anos, antes da guerra do Paraguai. Eu
era molecote novo e trabalhava aqui dentro, no terreiro. Via tudo. A
mucama, uma vez que Sinhazinha Zabé veio da Corte passar as férias na
roça, protegeu o namoro dela com um portuguesinho, e foi então\ldots{}''

Na marquesa, onde dormia, Jonas estremeceu. Olhei. Estava sentado e em
convulsões. Os olhos exorbitados fixavam"-se nalguma coisa invisível para
mim. Suas mãos crispadas mordiam a palhinha rota.

Agarrei"-o, sacudi"-o.

--- Jonas, Jonas, que é isso?

Olhou"-me sem ver, com a retina morta, num ar de desvario.

--- Jonas, fala!

Tentou murmurar uma palavra. Seu lábios tremeram na tentativa de
articular um nome. Por fim enunciou"-o, arquejante:

--- Izabel\ldots{}

Mas aquela voz já não era a voz de Jonas. Era uma voz desconhecida. Tive
a sensação plena de que um ``eu'' alheio lhe tomara de assalto o corpo
vazio. E falava por sua boca, e pensava com seu cérebro. Não era Jonas,
positivamente, quem estava ali. Era ``outro''!\ldots{}

Tio Bento, ao pé de mim, olhava assombrado para aquilo, sem compreender
coisa nenhuma; e eu, num horroroso estado de superexcitação, sentia"-me à
beira do medo pânico. Não fossem os trovões ecoantes e o ululo da
ventania nas casuarinas denunciarem"-me lá fora um horror talvez maior, e
é possível que não resistisse ao lance e fugisse da casa maldita como um
criminoso. Mas ali ao menos havia luz, aquele humilde candeeiro de
azeite, no momento mais precioso do que todos os bens da terra.

Estava escrito, entretanto, que ao horror dessa noite de trovoada e
mistério não faltaria uma nota sequer. Assim foi que, altas horas, a luz
principiou a esmorecer. Estremeci, e fiquei de cabelos eriçados quando a
voz do negro murmurou a única frase que eu não queria ouvir:

--- O azeite está no fim\ldots{}

--- E há mais lá em tua casa?

--- Era o restinho\ldots{}

Estarreci\ldots{}

Os trovões ecoavam longe, e o uivar do vento nas casuarinas era o mesmo
de sempre. Parecia empenhada a natureza em pôr em prova a resistência
dos meus nervos. Súbito, um estalido no candeeiro. A luz bruxuleou um
clarão final e extinguiu"-se.

Trevas. Trevas absolutas\ldots{}

Corri à janela. Abria"-a.

As mesmas trevas lá fora\ldots{}

Senti"-me sem olhos.

Procurei a cama às apalpadelas e caí de bruços na palhinha bolorenta.

\section*{VIII}

Pela madrugada começou Jonas a falar sozinho, como quem se recorda. Mas
não era o meu Jonas quem falava --- era o ``outro''.

Que cena!\ldots{}

Tenho até agora gravadas a buril no cérebro todas as palavras dessa
misteriosa confidência, proferida pelo íncubo no silêncio das trevas
profundas. Mil anos que viva e nunca se me apagará da memória o ressoar
daquela voz de mistério. Não reproduzo suas palavras da maneira como as
enunciou. Seria impossível, sobre nocivo à compreensão de quem lê. O
``outro'' falava ao jeito de quem pensa em voz alta, como a recordar.
Linguagem taquigráfica, ponho"-a aqui traduzida em língua corrente.

\section*{IX}

``Meu nome era Fernão. Filho de pais incógnitos quando me conheci por
gente já rolava no mar da vida como rolha sobre a onda. Ao léu, solto
nos vaivéns da miséria, sem carinhos de família, sem amigos, sem ponto
de apoio no mundo.

Era no Reino, na Póvoa do Varzim; e do Brasil, a boa colônia preluzida
em todas as imaginações como o Eldorado, eu ouvia os marinheiros de
torna"-viagem contarem maravilhas.

Fascinado, deliberei emigrar.

Parti um dia para Lisboa, a pé, como vagabundinho de estrada. Caminhada
inesquecível, faminta, mas rica dos melhores sonhos da minha existência.
Via"-me na terra nova feito mascate de bugigangas. Depois, vendeiro;
depois, comerciante com casa"-forte no Rio. Depois, já casado com linda
cachopa, via"-me de novo na Póvoa, rico, morando em quinta, senhor de
vinhedo e terras de semeadura.

Assim embalado em sonhos áureos, alcancei o porto de Lisboa, onde passei
o primeiro dia no cais, namorando os navios surtos no Tejo. Um havia em
aprestos para largar de rumo à colônia, a caravela \emph{Santa Tereza}.
Acamaradando"-me com velhos marujos de gandaia por ali, consegui nela,
por intermédio deles, o engajamento necessário.

--- Lá, foges --- aconselhou"-me um --- e afundas para o sertão. E
mercadejas, e enriqueces, e voltas cá excelentíssimo. É o que faria eu
se tivesse os verdes anos que tens.

Assim fiz e, grumete do \emph{Santa Tereza}, boiei no oceano, rumo às
terras de ultramar.

Aportamos em África para recolher pretos de Angola, metidos nos porões
como fardos de couro suado com carne viva por dentro. Pobres pretos!
Desembarcado no Rio, tive ainda ocasião de vê"-los no Valongo, seminus,
expostos à venda como reses. Os pretendentes chegavam, examinavam"-nos,
fechavam negócio.

Foi assim, nessa tarefa, que conheci o capitão Aleixo. Era um homem
alentado, de feições duras, olhar de gelo. Trazia botas, chapéu largo e
rebenque na mão. Atrás dele, como sombra, um capataz mal"-encarado.

O capitão notou o meu tipo, fez perguntas e ao cabo propôs"-me serviço em
sua fazenda. Aceitei e fiz a pé, em companhia do lote de negros
adquiridos, essa viagem pelo interior de um país onde tudo me era
novidade.

Chegamos.

Sua fazenda, formada de pouco tempo, ia então no apogeu, riquíssima de
canaviais, gado e café em inícios. Deram"-me servicinhos leves,
compatíveis com a idade e a minha nenhuma experiência da terra. E,
sempre subindo de posto, ali continuei até ver"-me homem.

A família do capitão morava na Corte. Os filhos vinham todos os anos
passar temporadas na roça, enchendo a fazenda de travessuras loucas. Já
as meninas, então no colégio, lá se deixavam ficar mesmo nas férias. Só
vieram uma vez, com a mãe, dona Teodora --- e foi isso a minha
desgraça\ldots{}

Eram duas, Inês, a caçula, e Izabel, a mais velha, lindas meninas de
luxo, radiosas de mocidade. Eu as via de longe, como nobres figuras de
romance, inacessíveis, e lembro"-me do efeito que naquele sertão bruto,
asselvajado pela escravaria retinta, fazia a presença das meninas ricas,
sempre vestidas à moda da Corte. Eram princesinhas de conto de fadas que
só provocam uma atitude: adoração.

Um dia\ldots{}

Aquela cachoeira --- lá lhe ouço o remoto rumorejo --- era a piscina da
fazenda. Escondida numa grota, como joia de cristal vivo a defluir com
permanente escachoo num engaste rústico de taquaris, caetês e
ingazeiros, formava um recesso grato ao pudor dos banhistas.

Um dia\ldots{}

Lembro"-me bem --- era domingo e eu, de vadiagem, saíra cedo a
passarinhar. Seguia pela margem do ribeirão tocaiando os pássaros
ribeirinhos.

Um pica"-pau"-de"-cabeça"-vermelha zombou de mim. Errei a bodocada e, metido
em brios, afreimei"-me em persegui"-lo. E, salta daqui, salta dali, quando
dei acordo estava embrenhado na grota da cachoeira, onde, num galho de
ingá, pude visar melhor a minha presa e espeloteá"-la.

Caiu a avezinha longe do meu alcance; barafustei pela trama dos taquaris
para colhê"-la. Nisto, por uma aberta na verdura, avistei embaixo a bacia
de pedra onde a água chofrava. Mas estarreci. Duas ninfas nuas brincavam
na espuma. Reconheci"-as. Eram Izabel e sua mucama dileta, da mesma
idade, a Liduína.

O improviso da visão ofuscou"-me os olhos. Quem há insensível à beleza da
mulher em flor e, a mais, vista assim em nudez num quadro agreste
daqueles?

Izabel deslumbrou"-me.

Corpo escultural, nesse período entontecedor em que florescem todas as
promessas da puberdade, diante dele senti a explosão subitânea dos
instintos. Ferveu"-me nas veias o sangue. Fiz"-me cachoeira de apetites.
Vinte anos! O momento das erupções incoercíveis\ldots{}

Imóvel como estátua, ali me quedei em êxtase o tempo que durou o banho.
E estou ainda com o quadro na imaginação. A graça com que ela, de cabeça
erguida, boca entreaberta, apresentava os pequeninos seios ao jato das
águas\ldots{} Os sustos e gritinhos nervosos quando gravetos derivantes lhe
esfrolavam a epiderme\ldots{} Os mergulhos de sereia na bacia de pedra e o
emergir do corpo aljofrado de espuma\ldots{}

Durou minutos o banho fatal. Depois vestiram"-se numa laje a seco e lá se
foram, contentes como borboletinhas ao sol.

Fiquei"-me por ali, extático, rememorando a cena mais linda que meus
olhos viram.

Impressão de sonho\ldots{}

Águas de cristal rumorejantes; frondes orvalhadas pendidas para a linfa
como a lhe escutar o murmúrio; um raio de sol matutino, coado pelas
franças, a pintalgar de ouro tremeluzente a nudez menineira das náiades.

Quem poderá esquecer um quadro assim?''

\section*{X}

``Essa impressão matou"-me. Matou"-nos.''

\section*{XI}

``Saí dali transformado.

Não era mais o humilde serviçal da fazenda, contente de sua sorte. Era
um homem branco e livre que desejava uma mulher formosa.

Daquele momento em diante minha vida iria girar em torno dessa
aspiração. Nascera em mim o amor, vigoroso e forte como as ervas loucas
da tiguera. Dia e noite só um pensamento ocuparia meu cérebro: Izabel.
Um só desejo: vê"-la. Um só objetivo à minha frente: possuí"-la.

Todavia, apesar de branco e livre, que abismo me separava da filha do
fazendeiro! Eu era pobre. Era um subalterno. Era nada.

Mas o coração não raciocina, nem o amor olha para conveniências sociais.
E assim, desprezando obstáculos, cresceu o amor no meu peito como
crescem rios em tempo de cheia.

Aproximei"-me da mucama e, depois de lhe cair em graça e lhe conquistar a
confiança, contei"-lhe um dia a minha tortura.

--- Liduína, tenho um segredo na alma que me mata, mas tu poderás
salvar"-me. Só tu. Preciso do teu socorro\ldots{} Juras auxiliar"-me?

Ela espantou"-se da confidência, mas, insistida, rogada, implorada,
prometeu tudo quanto pedi.

Pobre criatura! Tinha alma irmã da minha e foi ao compreender sua alma
que pela primeira vez alcancei todo o horror da escravidão\ldots{}

Abri"-lhe o meu peito e revelei"-lhe em frases candentes a paixão que me
consumia.

Liduína a princípio assustou"-se. Era grave o caso. Mas quem resiste à
dialética dos apaixonados? E Liduína, vencida afinal, prometeu
auxiliar"-me.''

\section*{XII}

``A mucama agiu por partes, fazendo desabrochar o amor no coração da
senhora sem que esta o percebesse. A princípio, uma vaga e discreta
referência à minha pessoa.

--- Sinhazinha conhece o Fernão?

--- Fernão?!\ldots{} Quem é?

--- Um moço que veio do reino e toma conta do engenho\ldots{}

--- Se já o vi, não me lembro.

--- Pois repare nele. Tem uns olhos\ldots{}

--- É teu namorado?

--- Quem me dera!\ldots{}

Foi essa a abertura do jogo. E assim, aos poucos, em dosagem hábil, hoje
uma palavra, amanhã outra, no espírito de Izabel nasceu a curiosidade
--- passo número um do amor.

Certo dia Izabel quis ver"-me.

--- Falas tanto nesse Fernão, nos olhos desse Fernão, que estou curiosa
de vê"-lo.

E viu"-me.

Eu estava no engenho, dirigindo a moagem da cana, quando as duas
apareceram de copo na mão. Vinham com o pretexto da garapa.

Liduína achegou"-se a mim e:

--- Seu Fernão, uma garapinha de espuma para Sinhá Izabel.

A menina olhou"-me de frente, mas não lhe pude sustentar o olhar. Baixei
os meus olhos, conturbado. Eu tremia, balbuciava apenas, nessa ebriez do
primeiro encontro.

Dei ordens aos pretos e logo jorrou da bica um jato fofo de garapa
espumejante. Tomei o copo da mão da mucama, enchi"-o e ofereci"-o à
náiade. Ela o recebeu com simpatia, bebeu aos golinhos e pagou"-me o
serviço com um gentil `obrigada', olhando"-me de novo nos olhos.

Pela segunda vez baixei os meus.

Saíram.

Mais tarde Liduína contou"-me o resto --- um pequenino diálogo.

--- Tinhas razão --- dissera"-lhe Izabel ---, é um bonito rapaz. Mas não
lhe vi bem os olhos. Que acanhamento! Parece que tem medo de mim\ldots{} Duas
vezes que o olhei de frente, duas vezes que os baixou.

--- Vergonha --- disse Liduína. --- Vergonha ou\ldots{}

--- \ldots{} ou quê?

--- Não digo\ldots{}

A mucama, com o seu fino instinto de mulher, compreendeu que não era
ainda tempo de pronunciar a palavra amor. Pronunciou"-a dias mais tarde,
quando percebeu a menina suficientemente madura para ouvi"-la sem
escândalo.

Passeavam pelo pomar da fazenda, então no auge da florescência.

O ar embriagava, tanto era o perfume nele solto.

Abelhas aos milhares, e colibris, zumbiam e esfuziavam num delírio
orgíaco.

Era a festa anual do mel.

Percebendo em Izabel o trabalho dos amavios ambientes, Liduína
aproveitou o ensejo para um passo a mais.

--- Quando eu vinha vindo vi Seu Fernão sentado na pedra do muro. Uma
tristeza\ldots{}

--- Que será que ele tem? Saudades da terra?

--- Quem sabe?! Saudades ou\ldots{}

--- \ldots{} ou quê?

--- \ldots{} ou amor.

--- Amor! Amor! --- disse Izabel sorvendo com volúpia o ar embalsamado.
--- Que linda palavra, Liduína! Eu, quando vejo um laranjal assim
florido, a palavra que me vem à ideia é essa: amor! Mas amará ele a
alguém?

--- Pois de certo. Quem não ama neste mundo? Os passarinhos, as
borboletas, as vespas\ldots{}

--- Mas a quem amará ele? A alguma preta do eito, com certeza\ldots{} --- e
Izabel riu"-se desabaladamente.

--- Aquele? --- fez Liduína num muxoxo. --- Não é desses, não,
Sinhazinha. Moço pobre, mas de condição. Para mim, até penso que ele é
filho dalgum fidalgo do reino. Anda por aqui escondido\ldots{}

Izabel quedou"-se pensativa.

--- Mas a quem amará, então, aqui, neste deserto de brancas?

--- Pois as brancas\ldots{}

--- Que brancas?

--- Dona Inezinha\ldots{} Dona Izabelinha\ldots{}

A mulher desapareceu por um momento para ceder o lugar à filha do
fazendeiro.

--- Eu? Engraçadinha! Era só o que faltava\ldots{}

Liduína calou"-se. Deixou que a semente lançada corresse o prazo da
germinação. E, vendo um casal de borboletas a perseguirem"-se com
estalidos de asas, mudou o rumo à conversa.

--- Sinhazinha já reparou nestas borboletas de perto? Têm dois números
debaixo das asas --- oito, oito. Quer ver?

Correu atrás delas.

--- Não pegas! --- gritou Izabel, divertida.

--- Mas pego esta aqui --- retrucou Liduína apanhando outra, lerdota, e
trazendo"-a a espernejar entre os dedos.

--- É ver uma casca de árvore com musgo. Espertalhona! Assim se
disfarça, que ninguém a percebe quando está sentadinha. É como o
periquito, que está gritando numa árvore, em cima da cabeça da gente, e
a gente nada vê. Por falar em periquito, por que Sinhazinha não arranja
um casal?

Izabel tinha o pensamento longe dali. A mucama bem o sentia, mas muito
de indústria continuava na tagarelice.

--- Dizem que se querem tanto, os periquitos, que quando um morre o
companheiro se mata. Tio Adão teve um assim, que se afogou numa pocinha
d'água no dia em que a periquita morreu. Só entre os pássaros há coisas
dessas\ldots{}

Izabel continuava absorta. Mas em dado momento quebrou o mutismo.

--- Por que te lembraste de mim nesse negócio do Fernão?

--- Por quê? --- repetiu Liduína cavorteiramente. --- Porque é tão
natural isso\ldots{}

--- Alguém te disse alguma coisa?

--- Ninguém. Mas se ele ama de amor, aqui neste sertão, e ficou assim
agora, depois que Sinhazinha chegou, a quem há de amar?\ldots{} Ponha o caso
em si. Se Sinhazinha fosse ele, e ele fosse Sinhazinha\ldots{}

Calaram"-se ambas e o passeio terminou no silêncio de quem dialoga
consigo mesmo.''

\section*{XIII}

``Izabel dormiu tarde essa noite. A ideia de que sua imagem enchia o
coração de um homem esvoaçava"-lhe na imaginação como as abelhas no
laranjal.

--- Mas é um subalterno! --- alegava o Orgulho.

--- Que importa, se é um moço rico de bons sentimentos? --- retorquia a
Natureza.

--- E bem pode ser que fidalgo!\ldots{} --- acrescentava, insinuante, a
Fantasia.

A Imaginação também veio à tribuna.

--- E pode vir a ser um poderoso fazendeiro. Quem era o capitão Aleixo
na idade dele? Um simples arreador\ldots{}

Já era o Amor quem assoprava tais argumentos.

Izabel ergueu"-se da cama e foi à janela. A lua em minguante quebrava de
tons cinéreos o escuro da noite. Os sapos no brejal coaxavam
melancólicos. Vaga"-lumes tontos riscavam fósforos no ar.

Era aqui\ldots{} Era aqui neste quarto, era aqui nesta janela!\ldots{}

Eu a espiava de longe, nesse estado de êxtase que o amor provoca na
presença do objeto amado. Longo tempo a vi assim, imersa em cisma.
Depois fechou"-se a persiana, e o mundo para mim se encheu de trevas.''

\section*{XIV}

``No outro dia, antes que Liduína abordasse o tema dileto, disse"-lhe
Izabel:

--- Mas, Liduína, que é amor?

--- Amor? --- respondeu a arguta mucama em quem o instinto substituía a
cultura. --- Amor é uma coisa\ldots{}

--- \ldots{} que\ldots{}

--- \ldots{} que vem vindo, vem vindo\ldots{}

--- \ldots{} e chega!

--- \ldots{} e chega e toma conta da gente. Tio Adão diz que o amor é doença.
Que a gente tem sarampo, catapora, tosse comprida, caxumba e amor ---
cada doença no seu tempo.

--- Pois eu tive tudo isso --- replicou Izabel --- e não tive amor\ldots{}

--- Sossegue que não escapa. Teve as piores e não há de ter a melhor?
Espere que um dia ele vem\ldots{}

Silenciaram.

Súbito, agarrando o braço da mucama, Izabel encarou"-a a fito nos olhos.

--- És minha amiga do coração, Liduína?

--- Um raio me parta neste momento se\ldots{}

--- És capaz dum segredo, mas dum segredo eterno, eterno, eterno?

--- Um raio me parta se\ldots{}

--- Cala a boca.

Izabel vacilava.

Depois, nessa ânsia de confidência que nasce ao primeiro luar do amor,
disse, corando:

--- Liduína, parece"-me que estou ficando doente\ldots{} da doença que
faltava.

--- Pois é tempo --- exclamou a finória arregalando os olhos. ---
Dezessete anos\ldots{}

--- Dezesseis.

E Liduína, cavilosa:

--- Algum fidalguinho da Corte?

Izabel vacilou de novo; por fim disse:

--- Eu tenho um namorado no Rio --- mas é namoro só. Amor, amor, desse
que bole cá dentro com o coração, desse que vem vindo, vem vindo e
chega, não! Não, lá\ldots{}

E em cochicho ao ouvido da mucama, corando:

--- Aqui!\ldots{}

--- Quem? --- perguntou Liduína, simulando espanto.

Izabel não respondeu com palavras. Ergueu"-se e:

--- Mas é um comecinho só. Vem vindo\ldots{}''

\section*{XV}

``O amor veio vindo e chegou. Chegou e destruiu todas as barreiras.
Destruiu nossas vidas e acabou destruindo a fazenda. Estas ruínas, estas
corujas, este morcegal, tudo não passa da florescência de um grande
amor\ldots{}

Por que há de ser a vida assim? Por que hão de os homens, à força de
orgulho, impedir que o botão da maravilhosa planta passe a flor? E por
que hão de transformar o que é céu em inferno, o que é perfume em dor, o
que é luz em negrume, o que é beleza em caveira?

Izabel, mimo de fragilidade feminil avivada de graça brasília, tinha o
quê perturbador das orquídeas. Sua beleza não era ao molde da beleza
rechonchuda e corada, forte e sadia, das cachopas da minha terra. Por
isso mesmo mais fortemente me seduzia a pálida princesinha tropical.

Ao inverso, o que em mim a seduzia era a força varonil e transbordante,
e a nobre rudeza dos meus instintos, que iam até a audácia de pôr os
olhos na altura em que ela pairava.''

\section*{XVI}

``O primeiro encontro foi\ldots{} casual. Meu acaso chamava"-se Liduína. Seu
gênio instintivo fê"-la a boa fada de nossos amores.

Foi assim.

Estavam as duas no pomar diante duma pitangueira enrubescida de frutos.

--- Lindas pitangas! --- disse Izabel. --- Sobe, Liduína, e apanha um
punhado.

Aproximou"-se Liduína da pitangueira e fez vãs tentativas para trepar.

--- Impossível, Sinhazinha, só chamando alguém. Quer?

--- Pois vai chamar alguém.

Liduína partiu correndo e Izabel teve a previsão nítida de quem viria.
De fato, momentos depois apareci eu.

--- Senhor Fernão, desculpe"-me --- disse a moça. --- Pedi àquela maluca
que chamasse algum preto para colher pitangas --- e foi ela incomodá"-lo.

Perturbado pela sua presença e com o coração aos pulos, gaguejei, para
dizer algo:

--- São pitangas que quer?

--- Sim. Mas falta uma cestinha que Liduína foi buscar.

Pausa.

Izabel, tão senhora de si, percebi"-a nesse momento embaraçada como eu.
Não tinha o que dizer. Silenciava. Por fim:

--- Moem cana hoje? --- perguntou"-me.

Gaguejei que sim e novo silêncio se fez. Para quebrá"-lo, Izabel gritou
em direção da casa:

--- Anda depressa, rapariga! Que lesmice\ldots{}

E depois, para mim:

--- Não tem saudades de sua terra?

Despregou"-se"-me a língua. Perdi o embaraço. Respondi que tive, mas não
as tinha mais.

--- Os primeiros anos passei"-os a suspirar à noite, saudoso de tudo de
lá. Só quem emigrou sabe a dor do fruto arrancado à árvore.
Conformei"-me, afinal. E hoje\ldots{} o mundo inteiro para mim está aqui
nestas montanhas.

Izabel compreendeu"-me a intenção e quis perguntar"-me por quê. Mas não
teve ânimo. Saltou para outro assunto.

--- Por que motivo só as pitangas desta árvore prestam? As outras são
tão azedas\ldots{}

--- Vai ver --- disse eu --- que esta árvore é feliz e as outras não. O
que azeda os homens e as coisas é a desgraça. Fui doce como a lima, logo
que vim para cá. Hoje sou amargo\ldots{}

--- Julga"-se infeliz?

--- Mais do que nunca.

Izabel arriscou"-se:

--- Por quê?

Respondi intrepidamente:

--- Dona Izabel, que é menina rica, não imagina a posição desgraçada de
quem é pobre. O pobre forma neste mundo uma casta maldita, sem direito a
coisa nenhuma. O pobre não pode nada\ldots{}

--- Pode, sim. Pode uma coisa\ldots{}

--- ?

--- Deixar de ser pobre.

--- Não falo da riqueza do dinheiro. Essa é fácil de alcançar, depende
apenas de esforço e habilidade. Falo de coisas mais preciosas que o
ouro. Um pobre, tenha o coração que tiver, seja a mais nobre das almas,
não tem o direito de erguer os olhos para certas \emph{alturas}\ldots{}

--- Mas se a \emph{altura} quiser descer até ele? --- retrucou audaciosa
e vivamente a menina.

--- Esse caso acontece às vezes nos romances. Na vida, nunca\ldots{}

Calamo"-nos de novo. Neste entremeio Liduína reapareceu, esbaforida, com
a cestinha na mão.

--- Custou"-me a achar --- disse a velhaca, justificando a demora. ---
Estava caída atrás do toucador.

O olhar que lhe lançou Izabel dizia: `mentirosa!'.

Tomei a cesta e preparei"-me para trepar à árvore.

Izabel, porém, interveio:

--- Não! Não quero mais pitangas. Vão tirar"-me o apetite para a garapa
do meio"-dia. Ficam para outra vez.

E para mim, amável:

--- Queira desculpar"-me\ldots{}

Saudei"-a, ébrio de felicidade, e lá me fui de aleluias na alma, com o
mundo a dançar em torno de mim.

Izabel seguiu"-me com o olhar, pensativamente.

--- Tinha razão, Liduína, é um rapagão que vale todos os pelintras da
Corte. Mas, coitado!\ldots{} Queixa"-se tanto do seu destino\ldots{}

--- Bobagens --- muxoxou a mucama, trepando à pitangueira com agilidade
de macaco.

Vendo aquilo, Izabel sorriu e murmurou, entre repreensiva e maliciosa:

--- Você, Liduína\ldots{}

A rapariga, que tinha entre os dentes alvíssimos o vermelho duma
pitanga, esganiçou uma risada velhaca.

--- Pois Sinhazinha não sabe que sou mais sua amiga do que sua
escrava?''

\section*{XVII}

``O amor é o mesmo em toda parte e em todos os tempos. Aquele enleio do
primeiro encontro é o eterno enleio dos primeiros encontros. Aquele
diálogo à sombra da pitangueira é o eterno diálogo da abertura. Assim,
nosso amor, tão novo para nós, reproduzia um jogo velho qual o mundo.

Nascera em Izabel e em mim um sexto sentido maravilhoso.
Compreendíamo"-nos, adivinhávamo"-nos e descobríamos meios de inventar os
mais imprevistos encontros --- encontros deliciosos, em que um olhar
bastava para a permuta de mundos de confidências\ldots{}

Izabel amou"-me.

Que período de vida, esse!

Eu sentia"-me alto como as montanhas, forte como o oceano e todo a
coruscar de estrelas por dentro.

Era rei.

A terra, a natureza, os céus, a lua, a luz, a cor, tudo existia para
ambiente do meu amor. Não era mais vida aquele meu viver, sim um êxtase
contínuo.

Alheado de tudo, uma só coisa eu via, duma só coisa me alimentava.

Riquezas, poderio, honras --- que vale tudo isso ante a sensação divina
de amar e ser amado?

Nessa ebriedade vivi --- quanto tempo não sei. O tempo não contava para
o meu amor. Vivia --- tinha a impressão de que só nessa época entrara a
viver. Antes, a vida não me fora mais que simples agitação animalesca.

Poetas! Como vos compreendi a voz interior ressoada em rimas, como me
irmanei convosco no esvoaçar pelos intermúndios do sonho!\ldots{}

Liduína comportava"-se como a fada boa dos nossos destinos. Sempre
vigilante, a ela devíamos inteirinho o mar de felicidade em que
boiávamos. Lépida, mimosa, travessa, a gentil crioula enfeixava em si
toda a artimanha da raça perseguida --- e todo o gênio do sexo
escravizado à prepotência do homem.

Entretanto, o bem que nos fizeste como se avinagrou para ti, Liduína!\ldots{}
Em que fel horroroso se transfez para ti, afinal\ldots{}

Eu sabia que o mundo é governado pelo monstro Estupidez. E que Sua
Majestade não perdoa o crime de Amor. Mas nunca supus que esse monstro
fosse a fera delirante que é --- tão sanguissedenta, tão requintada em
ferócia. Nem que houvesse monstro mais bem servido que esse.

Que comitiva numerosa traz!

Que servos diligentes possui!

A sociedade, as leis, os governos, as religiões, os juízes, as morais,
tudo que é força social organizada presta mão forte à Estupidez
Onipotente.

E assanha"-se em punir, em torturar o ingênuo que, conduzido pela
natureza, arrosta com os mandamentos da megera.

Ai dele se comete um crime de lesa"-Estupidez! Mãos de ferro
constringem"-lhe a garganta. Seu corpo rola por terra, espezinhado; seu
nome perpetua"-se com pechas infames.

Nosso crime --- que lindo crime: amar! --- foi descoberto. E a
monstruosa engrenagem de aço triturou"-nos, ossos e alma, aos três\ldots{}''

\section*{XVIII}

``Uma noite\ldots{}

A lua, bem no alto, empalidecia as estrelas e eu, triste, velava,
rememorando o último encontro com Izabel. Fora à tardinha, numa volta do
ribeirão, à sombra dum tufo de marianeiras cacheadas de frutos. Mãos
unidas, cabeça contra cabeça, num enlevo de comunhão de alma,
assistíamos ao alvoroto da peixaria assanhada na disputa das frutinhas
amarelas que a espaços pipocavam na água remansosa do rio. Izabel,
absorta, mirava aquelas ariscas linguinhas de prata, apinhadas em torno
das iscas.

--- Sinto"-me triste, Fernão. Tenho medo da nossa felicidade. Qualquer
coisa me diz que isso vai ter fim --- e fim trágico\ldots{}

Minha resposta foi aconchegá"-la inda mais ao meu peito.

Um bando de saíras e sanhaços, de pouso nas marianeiras, entraram a
debicar energicamente os cachos da frutinha silvestre. E o espelho das
águas piriricou ao chuveiro das migalhas caídas. Coalhou"-se o rio de
lambaris famintos, engalfinhados num delírio de rega"-bofe, com saltos de
prata faiscantes no ar.

Izabel, sempre absorta, dizia:

--- Como são felizes!\ldots{} E são felizes porque são livres. Nós --- pobres
de nós!\ldots{} Nós somos inda mais escravos do que os escravos do eito\ldots{}

Duas viuvinhas pousaram numa haste de peri emersa da margem fronteira. A
vara vergou"-se"-lhes ao peso, oscilou uns instantes e estabilizou"-se de
novo. E o lindo casal permaneceu imóvel, juntinho, comentando talvez,
como nós, a festa glutona dos peixes.

Izabel murmurou num sorriso de infinita melancolia:

--- Que cabecinha sossegada eles têm\ldots{}

Eu rememorava frase por frase esse último encontro com a minha amada,
quando, dentro da noite, ouvi bulha à porta.

Alguém corria o ferrolho e entrava.

Sentei"-me na cama, de sobressalto.

Era Liduína. Tinha os olhos esgazeados de pavor e foi em voz arquejante
que atropelou as derradeiras palavras que lhe ouvi na vida.

--- Fuja! O capitão Aleixo sabe tudo. Fuja, que estamos perdidos\ldots{}

Disse, e esgueirou"-se para o terreiro como sombra.''

\section*{XIX}

``O choque foi tamanho que me senti vazio de cérebro. Parei de pensar\ldots{}

O capitão Aleixo\ldots{}

Lembro"-me bem dele. Era o plenipotenciário de Sua Majestade a Estupidez
nestas paragens. Frio e duro, não reconhecia sensibilidade em carne
alheia. Recomendava sempre aos feitores a sua receita de bem conduzir os
escravos: `angu por dentro e relho por fora, sem economia e sem dó'.

Consoante tal programa, a vida na fazenda escoava"-se entre trabalhos de
eito, comezaina farta e ``bacalhau''.

Com o tempo desenvolveu"-se nele a crueldade inútil. Não se limitava a
impor castigos: ia presenciá"-los. Gozava de ver a carne humana
avergoar"-se aos golpes do couro cru.

Ninguém, entretanto, estranhava aquilo. Os pretos sofriam como
predestinados à dor. E os brancos tinham como dogma que de outra maneira
não se levavam pretos.

O sentimento de revolta não latejava em ninguém, salvo em Izabel, que se
fechava no quarto, de dedos fincados nos ouvidos, sempre que na casa do
tronco o ``bacalhau'' arrancava urros a um pobre infeliz.

A mim, em começo, também me era indiferente a dor alheia. Ao depois ---
depois que o amor me floriu a alma de todas as flores do sentimento ---,
aquelas barbaridades diárias punham"-me fremente de cólera.

Uma vez tive ímpetos de estrangular o déspota. Foi o caso dum vizinho
que lhe trouxera um cão de fila para vender.''

\section*{XX}

``--- É bom? Bem bravo? --- perguntou o fazendeiro examinando o animal.

--- Uma fera! Para apanhar negro fugido, nada melhor.

--- Não compro nabos em sacos --- disse o capitão. --- Experimentemo"-lo.

Ergueu os olhos para o terreiro que fulgurava ao sol. Deserto. A
escravaria inteira na roça. Mas naquele momento o portão se abriu e um
preto velho entrou, cambaio, de jacá ao ombro, rumo ao chiqueiro dos
porcos. Era um estropiado do eito que pagava o que comia tratando da
criação.

O fazendeiro teve uma ideia. Tirou o cão da corrente e atiçou"-o contra o
preto.

--- Pega, Vinagre!

O mastim partiu como bala e instante depois ferrava o pobre velho, dando
com ele em terra. Estraçalhou"-o\ldots{}

O fazendeiro sorria"-se com entusiasmo.

--- É de primeira --- disse ao sujeito. --- Dou"-lhe cem mil"-réis pelo
Vinagre.

E como o sujeito, assombrado daqueles processos, lamentasse a desgraça
do estraçalhado, o capitão fez cara de espanto.

--- Ora bolas! Um caco de vida\ldots{}''

\section*{XXI}

``Pois foi esse homem que vi subitamente penetrar no meu quarto, essa
noite, logo depois que se sumiu Liduína. Acompanhavam"-no dois feitores,
como sombras. Entrou e fechou a porta sobre si. Parou a alguma
distância. Olhou"-me e sorriu.

--- Vou dar"-te uma bela noivinha --- disse ele. E num gesto ordenou aos
carrascos que me amarrassem.

Despertei da vacuidade. O instinto de conservação retesou"-me todas as
energias e, mal os capangas vieram a mim, atirei"-me a eles com furor de
onça fêmea a quem roubam os cachorrinhos.

Não sei quanto tempo durou a luta horrorosa; sei apenas que a tantas
perdi os sentidos em virtude das violentas pancadas que me racharam a
cabeça.

Quando despertei pela madrugada vi"-me por terra, com os pés doridos
entalados no tronco. Levei a mão aos olhos sujos de pó e sangue e
entrevi à minha esquerda, no extremo do madeiro hediondo, um corpo
desmaiado de mulher.

Liduína\ldots{}

Percebi ainda que havia mais gente ali.

Olhei.

Dois homens de picaretas abriam um largo rombo no espesso muro de taipa.

Outro, um pedreiro, misturava cal e areia no chão, rente a uma pilha de
tijolos.

O fazendeiro também ali estava, de braços cruzados, dirigindo o serviço.
Vendo"-me desperto, aproximou"-se do meu ouvido e murmurou com gélido
sarcasmo as últimas palavras que ouvi sobre a terra:

--- Olhe! A tua noivinha é aquela parede\ldots{}

Compreendi tudo: iam emparedar"-me vivo\ldots{}''

\section*{XXII}

Aqui se interrompeu a história do ``outro'', como a ouvi naquela
horrorosa noite. Repito que não a ouvi assim, nessa ordem literária, mas
murmurada em solilóquio, aos arrancos, às vezes entre soluços, outras
num cicio imperceptível. Tão estranha era essa forma de narrar que o
velho tio Bento não apanhou coisa nenhuma.

E foi com ela a me doer no cérebro que vi chegar a manhã.

--- Bendita sejas, luz!

Ergui"-me, alvoroçado.

Abri a janela, todo a renascer"-me dos horrores noturnos.

O sol lá estava espiando"-me dentre a copa do arvoredo. Seus raios de
ouro invadiram"-me a alma. Varreram dela os frocos de trevas que a
entenebreciam qual cabelugem de pesadelo.

O ar lavado e alerta encheu"-me os pulmões da delirante vida matutina.
Respirei"-o alegremente, em haustos largos.

E Jonas? Dormia ainda, repousado de feições.

Era ``ele'' outra vez. O ``outro'' fugira com as trevas da noite.

--- Tio Bento --- exclamei ---, conte"-me o resto da história. Que fim
teve Liduína?

O velho preto recomeçou a contá"-la a partir do ponto em que a
interrompera na véspera.

--- Não! --- gritei eu ---, dispenso isso tudo. Só quero saber que fim
teve Liduína depois que o capitão deu sumiço ao moço.

Tio Bento abriu cara de espanto.

--- Como o meu branco sabe disso?

--- Sonhei, tio Bento.

Ele permaneceu ainda uns instantes admirado, custando a crer. Depois
narrou:

--- Liduína morreu no chicote, a coitadinha --- tão na flor, dezenove
anos\ldots{} O Gabriel e o Estevão, os carrascos, retalharam o seu corpinho
de criança com os rabos do ``bacalhau''\ldots{} A mãe dela, que só na hora do
castigo soube do acontecido na véspera, correu feito louca para a casa
do tronco. No momento em que empurrou a porta e olhou, uma chicotada
cortava o seio esquerdo da filha. Antonia deu um grito e caiu para trás
como morta.

Apesar do radioso da manhã meus nervos fremiram às palavras do preto.

--- Basta, basta\ldots{} De Liduína basta. Só quero agora saber o que sucedeu
a Izabel.

--- Nhá Zabé ninguém mais viu ela na fazenda. Foi levada para a Corte e
acabou mais tarde no hospício, é o que dizem.

--- E Fernão?

--- Esse sumiu. Ninguém nunca soube dele --- nunca, nunca\ldots{}

Jonas acabava de despertar. E ao ver luz no quarto sorriu. Queixava"-se
de peso na cabeça.

Interpelei"-o sobre o eclipse noturno de sua alma, mas Jonas mostrou"-se
alheio a tudo. Enrugou a testa, recordando"-se.

--- Lembro"-me que uma coisa me invadiu, que fui empolgado, que lutei com
desespero\ldots{}

--- E depois?

--- Depois?\ldots{} Depois um vácuo\ldots{}

Saímos para fora.

A casa maldita, mergulhada na onda de luz matutina, perdera o aspecto
trágico.

Disse"-lhe adeus --- para sempre\ldots{}

--- \emph{Vade retro}!\ldots{}

E fomo"-nos à casinhola do preto engolir o café e arrear os animais.

De caminho espiei pelas grades da casa do tronco: na taipa grossa da
parede havia um trecho murado a tijolo\ldots{}

Afastei"-me horripilado.

E guardei comigo o segredo da tragédia de Fernão. Só eu no mundo a
conhecia, contada por ele mesmo, oitenta anos após a catástrofe.

Só eu!

Mas como não sei guardar segredo, revelei"-o em caminho ao Jonas.

Jonas riu"-se à larga e disse, estendendo"-me o dedo minguinho:

--- Morde aqui!\ldots{}

\chapter{Marabá\footnote[*]{Texto de 1923, publicado no livro \emph{O macaco que se fez homem}.}}

Bom tempo houve em que o romance era coisa de aviar com receitas à
vista, qual faz o honesto boticário com os seus xaropes.

Quer trabuco histórico? Tome tanto de Herculano, tanto de Walter Scott,
um pajem, um escudeiro e o que baste de Briolanjas, Urracas e Guterres.

Quer indianismo? Ponha duas arrobas de Alencar, uns laivos de Fenimore,
pitadas de Chateaubriand, graúnas \emph{quantum satis}, misture e mande.

Receitas para tudo. Para começo (fórmula Herculano): ``Era por uma
dessas tardes de verão em que o astro"-rei etc., etc.''.

E para fim (fórmula Alencar): ``E a palmeira desapareceu no
horizonte\ldots{}''.

Arrumado o cenário da natureza, surgia, lá em Portugal, um lidador com o
seu espadagão, todo carapaçado de ferro e ereto no lombo de árdego
morzelo; ou, aqui no Brasil, um cacique de feroz catadura, todo arco,
flechas e inúbias.

E vinha, ou uma castelã de olhos com cercadura de violetas, ou uma
morena virgem nua, de pulseira na canela e mel nos lábios.

E não tardava um donzel trovadoresco que ``cantava'' a castelã, ou um
guerreiro branco que fugia com a Iracema à garupa.

Depois, a escada de corda, o luar, os beijos --- multiplicação da
espécie à moda medieval; ou um sussurro na moita --- multiplicação da
espécie à moda natural.

A tantas o pai feroz descobria tudo e, à frente dos seus peões, voava à
caça do sedutor em desabalada corrida, rebentando dúzias de corcéis; ou
o cacique de rabos de arara na cabeça erguia as mãos para o céu de Tupã
implorando vingança.

E dom Bermudo, apanhando o trovador pirata, o objurgava em estilo de
catedral, com a toledana erguida sobre sua cabeça:

--- Mentes pela gorja, perro infame!

Ou o cacique, filando o guerreiro branco, o trazia para a taba ao som da
inúbia e lá o assava em fogueira de pau"-brasil; vingança tremenda, porém
não maior que a de dom Bermudo a fender o crânio do pajem e a
arrancar"-lhe o coração fumegante para depô"-lo no regaço da castelã
manchada.

E a moça desmaiava, e o leitor chorava e a obra recebia etiqueta de
histórica, se passada unicamente entre Dons e Donas, ou de indianista,
se na manipulação entravam ingredientes do empório Gonçalves Dias,
Alencar \& Cia.

Veio depois Zola com o seu naturalismo, e veio a psicologia e a
preocupação da verdade, tudo por contágio da ciência que Darwin, Spencer
e outros demônios derramaram no espírito humano.

Verdade, Verdade!\ldots{} Que musa tirânica! Como fez mal aos romancistas ---
e como os \emph{força} a ter talento!

Foram"-se as receitas, os figurinos. Cada qual faça como entender,
contanto que não discrepe do \emph{veritas super omnia}, latim que em
arte significa mentir com verossimilhança.

--- Tudo isso para quê? --- perguntará o leitor atônito.

É que trago nos miolos uma novela tão ao sabor antigo, tão fora da moda,
que não me animo a impingi"-la sem preâmbulo. E não é feia, não. Vem de
Alencar, esse filho de alguma Sherazade aimoré, que a todos nós, na
juventude, nos povoou a imaginação de lindas coisas inesquecíveis. E
compõe"-se de um guerreiro branco, duas virgens das selvas, caciques,
danças guerreiras, fuga heroica etc.

Chama"-se ``Marabá'' e principia assim:

Era por uma dessas noites enluaradas de verão, em que a natureza parece
chovida de cinzas brancas.

Dorme a taba, e dorme a floresta circundante, sem sussurros de brisas,
nem regorjeio de aves.

Só o urutau pia longe, e uma ou outra suindara perpassa, descrevendo
voos de veludo ao som dum \emph{clu}, \emph{clu}, \emph{clu}\ldots{} que ora
se aproxima, ora se perde distante.

No centro do terreiro, atado a um poste da canjerana rija, o prisioneiro
branco vela. Foi vencido em combate cruento, teve todos os seus homens
trucidados e vai agora pagar com a vida o louco ousio de pisar terra
aimoré. Será sacrificado pela manhã ao romper do sol, cabendo ao potente
Anhembira, cacique invicto, a honra de fender"-lhe o crânio com a
ivirapema de pau"-ferro.

Seu corpo será destroçado pelas horrendas megeras da tribo, sua carne
devorada pelos ferozes canibais.

O guerreiro branco rememora com melancolia o viver tão breve --- sua
meninice de ontem, o engajamento numa nau, a viagem por mar, as
aventuras nas terras novas de Santa Cruz, norteadas pela desmedida
ambição do ouro.

É louro e tem olhos azuis. Em suas veias corre o melhor sangue do reino.

Seu avô caiu nas Índias, varado duma zagaia cingalesa; seu pai, nos
sertões inóspitos dos Brasis, acabou na paralisia do curare que seta
fatal lhe inoculou.

Chegara a vez do mal"-aventurado rebento último dessa estirpe de
heróis\ldots{}

Em redor, guerreiros cor de bronze, exaustos da dança e bêbados de
cauim, jazem estirados, as mãos soltas dos tacapes terríveis. Também
dormita o velho pajé, de cócoras rente à ocara, com o maracá em silêncio
ao lado.

Que mais? Sim, a lua\ldots{} A lua que no alto passeia o seu crescente.

Súbito, um vulto se destaca de moita vizinha e aproxima"-se cauteloso,
com pés sutis de corça arisca.

É Iná, a mais formosa virgem das selvas, oriunda do sangue cacical de
Anhembira, o Morde"-corações.

A virgem caminha em direção ao prisioneiro. Para"-lhe defronte e por
instantes o contempla, como presa de indecisas ideias.

Por fim decide"-se e, ligeira como a irara, desfaz os nós da muçurana
fatal e dá de beber ao guerreiro branco o trago de cauim desentorpecedor
dos músculos adormentados. Em seguida mira"-o a furto nos olhos,
perturbada, e num gesto indica"-lhe a mata, sussurrando em língua da
terra:

--- Foge!

O guerreiro branco vacila. Não conhece a mata, que é imensa, e teme
encontrar em seu seio morte mais cruel que a pelo tacape de Anhembira.

Iná compreende o seu enleio e, tomando"-lhe a mão, leva"-o consigo;
conhece a mata a palmo e sabe o caminho de pô"-lo a seguro em sítio até
onde não ousa alongar"-se a gente aimoré.

A noite inteira caminham, e só quando um grande rio de águas negras lhes
tranca o passo é que a virgem morena se detém. Aponta o rio ao moço
guerreiro e nesse gesto diz que está finda a sua missão, pois que o rio
leva ao mar e o mar é o caminho dos guerreiros brancos.

O moço tem o peito a estourar de gratidão e amor, e como não pode
significá"-los com palavras lusas, recorre ao esperanto da natureza:
abraça a virgem morena, beija"-a e, a céu aberto, ao som múrmuro das
águas eternas, louco de paixão, a possui.

Reticências.

Ao romper da madrugada:

--- É a cotovia que canta!\ldots{} --- diz ela.

--- Não; é o rouxinol --- retruca Romeu.

--- É a cotovia\ldots{}

--- É o rouxinol\ldots{}

Vence a cotovia. O moço beija"-a pela última vez e parte. Não esquece,
porém, de enfiar no dedo de Julieta um anel --- joia indispensável ao
desfecho da nossa tragédia.

\section*{PRIMEIRO ATO}

A tribo está apreensiva. As velhas murmuram e o pajé inquieta"-se.

--- Marabá! --- sussurram todos.

Castigo de Tupã? Sinal do céu que marca o termo da glória de Anhembira,
o chefe da tribo?

Uma criança nascera ali, de olhos azuis e loura, evidentemente marabá. E
nascera de Iná, a virgem bronzeada em cujas veias corre o sangue do
grande morubixaba.

Traição!

A mãe mentira à raça, e do contato com o estrangeiro invasor, cruel
inimigo que do seio do mar surgiu para desgraça do povo americano, teve
aquela filha. O louro dos cabelos, o azul dos olhos, a alvura da pele
denunciavam claramente o imperdoável crime.

--- Marabá! --- sussurram todos.

E um vago terror espalha"-se pela tribo.

O pajé reúne em concílio os velhos para decidirem sobre o gravíssimo
caso.

E após longas ponderações a assembleia resolve o sacrifício da pequena
marabá, em holocausto aos manes irritados da tribo.

Levam a sentença ao cacique, que é pai, mas que antes de pai é o chefe,
o inexorável guardião da Lei velha como o tempo.

Anhembira cerra o sobrecenho, baixa a cabeça e queda"-se imóvel como a
própria estátua da dor.

\emph{Entre parêntesis.}

Uma coisa me espanta: que haja inda hoje, nestes nossos atropelados dias
modernos, quem \emph{escreva} romances! E quem os \emph{leia}!\ldots{}

Conduzir por trezentas páginas a fio um enredo, que estafa!

Nada disso. Sejamos da época. A época é apressada, automobilística,
aviatória, cinematográfica, e esta minha ``Marabá'', no andamento em que
começou, não chegaria nunca ao epílogo.

Abreviemo"-la, pois, transformando"-a em entrecho de filme. Vantagem
tríplice: não maçará o pobre do leitor, não comerá o escasso tempo do
autor e ainda pode ser que acabe filmada, quando tivermos por cá miolo e
ânimo para concorrer com a Fox ou a Paramount.

Vá daqui para diante a cem quilômetros por hora, dividida em
\emph{quadros} e \emph{letreiros}.

\section*{QUADRO}

Enquanto Anhembira, de cabeça derrubada sobre o peito, medita sobre a
sentença que condenou a criança loura, uma índia velha corre a avisar
Iná.

Iná é mãe e as mães não vacilam. Toma a filhinha nos braços e foge para
as selvas\ldots{}

\section*{QUADRO}

Lindo cenário. Trecho de mata virgem trancado de cipoeira, trançado de
taquaruçus. Vê"-se à direita um velho tronco de enorme jequitibá ocado. É
nesse oco que mora a menina loura de olhos azuis. A mãe ajeitou"-o para
esconderijo seguro; tapetou"-o de musgos macios; fez dele um ninho de
meter inveja às aves.

Ali dorme o lindo anjo, filho do amor a céu aberto. Ali recebe a mãe
inquieta, que de fuga lhe traz o seio nutriz. De fuga, pois a tribo
ignora o estratagema e está certa de que a filha de Anhembira arrojou ao
abismo das águas o fruto maldito do seu ventre.

\section*{LETREIRO}

\emph{Marabá cresceu no sombrio da mata, como a ninfa mimosa do ermo.
Iná ensinou"-lhe a vida e deu"-lhe armas com que abatesse as aves que piam
no subosque, e a caça ligeira que entoca, e os peixes faiscantes que se
alapam nas pedras.}

\section*{QUADRO}

Marabá despede"-se de sua mãe.

Já pode viver por si e quer seguir para ermos distantes onde não chegue
o som das inúbias de Anhembira --- lá onde o rio é como um deus
irrequieto que ora escabuja nas fragas, ora brinca com as pétalas mortas
remoinhantes em seus remansos.

Iná despede"-se da filha e, repetindo o gesto do guerreiro branco,
põe"-lhe no dedo o anel de núpcias.

\section*{QUADRO}

A vida solitária de Marabá. Seu namoro com o rio. Nele banha"-se e
mergulha e nada, com a linda coma loura flutuante, e nele mira seus
olhos feitos de pedaços do céu.

É seu amante, é seu deus o rio eterno. É o ser vivo em cuja companhia
refoge à depressão do ermo absoluto.

\section*{LETREIRO}

\emph{Em Marabá confluem duas psíquicas --- a da terra, herdada de sua
mãe, e a do moço louro vindo de além"-mar, duma plaga distante que em
sonhos indecisos sua alma em botão adivinha.}

\section*{QUADRO}

Mas pouco cisma, a linda Marabá. O tempo lhe é escasso para a delirante
vida de ninfa que é o seu viver ali.

Ora perde a manhã inteira na perseguição do gamo que veio beber ao rio;
ora galga a pedranceira em prodígios de arrojo para colher uma flor que
se abriu no mais alto da penha.

Persegue borboletas --- e que quadro é vê"-la no campo, veloz como a
gazela, a loura cabeleira solta ao vento!

Sua nudez de virgem esplende em fulgor de escultura divina. Deus a
esculpiu --- e escultor nenhum jamais concebeu corpo assim, de linhas
mais puras, seios mais firmes, ancas mais esgalgas, braços de torneio
mais fino.

Tem a nudez divina, Marabá --- porque existe a nudez humana: das
criaturas que convivem entre humanos e sofrem todos os vincos da
humanidade.

Marabá não viciou sua nudez no contato humano; é nua como é nu o lírio
--- sem saber que o é.

Mas é mulher. Adivinha de instinto que as flores fê"-las Deus para a
mulher, e colhe"-as, e tece"-as em guirlandas, e com elas enfeita os
cabelos e o colo e a cintura. E assim, toda flores, mira"-se no espelho
das águas e sorri. E porque sorri, logo salta, alegre, e dança. E porque
dança, anima as selvas da luz maravilhosa que os helenos ensinaram ao
mundo.

Súbito, um rumor fá"-la estacar. A filha de Dionísio se apaga e surge
Diana. Ei"-la de arco em punho, em louca desabalada, na pista do cervo
incauto que lhe interrompeu a bela improvisação coreográfica.

Quem lhe ensinou a dançar?

Tudo. O sangue estuante em suas veias, o vento que agita a fronde das
jiçaras, o remoinho das águas, as aves. Viu dançarem os tangarás, um
dia, e desde esse momento sua vida é uma contínua e maravilhosa criação
em que a alma da terra americana se exsolve em movimentos rítmicos.

Sempre mulher, Marabá amansou uma veadinha de leite e tem"-na consigo
como inseparável companheira, dócil às suas expansões de carinho. Com a
pequena corça brinca horas a fio, e abraça"-a, e beija"-a no mimoso
focinho róseo. Que festa a vida de Marabá!

Ninguém a vence em riquezas. Ouro, dá"-lhe o sol às catadupas, e todo só
para ela. Perfume, não em frascos microscópicos o tem, mas ambiente,
perenal; as flores só exalam para ela, e todas as brisas se ocupam em
trazê"-lo de longe, tomado da corola das orquídeas mais raras.

E as abelhas ofertam"-lhe o mel puríssimo; e os ingazeiros de beira"-rio
dão"-lhe a nívea polpa dos seus frutos invaginados; e cem árvores da
floresta parecem precipitar a maturescência de suas bagas rubras, roxas,
verdoengas, para que mais cedo os alvos dentes da ninfa as mordam com
delícia.

E os dias de Marabá são assim um delírio de luz, de perfumes, de
movimentos sadios e livres, capaz de enlouquecer a imaginação dos pobres
seres chamados homens, que vivem em prisões chamadas cidades, dentro de
gaiolas chamadas casas, com poeira para os pulmões em vez de ar, catinga
de gasolina em vez de vida\ldots{}

\section*{NOTA A MR. CECIL B. DE MILLE}

Este papel de Marabá tem que ser feito por Annette Kellermann. Como,
porém, Annette já está madura e Marabá é o que existe de mais botão,
torna"-se preciso inventar um processo que rejuvenesça de trinta anos a
intérprete.

\section*{QUADRO}

Um dia, um caçador tresmalhado surpreende a ninfa no banho. É Ipojuca, o
filho dileto de Anhembira e seu sucessor no cacicado. Três dias e três
noites correu ele em perseguição de um jaguar; mas no momento em que
dobrava o arco para desferir a flecha certeira, descaiu"-lhe das mãos a
arma e seus olhos se dilataram de assombro.

O corpo nu da virgem loura emergira das águas à sua frente.

--- Iara?

No primeiro momento o medo sobressaltou"-o --- mas o sangue de Anhembira
reagiu em suas veias, e não seria o filho do guerreiro que jamais
conheceu o medo quem tremesse diante de mulher, Iara que fosse.

E Ipojuca imobilizou"-se à margem do rio, em muda contemplação, até que a
ninfa, percebendo"-o, fugisse para o lado oposto, mais arisca do que a
tabarana.

Ipojuca atravessou o rio e logo mergulhou na floresta, em sua
perseguição. Jamais as ninfas venceram a faunos na corrida. Foi assim na
Grécia; seria assim sob o céu de Colombo. O filho do cacique alcançou"-a.
Seu braço de ferro enlaçou"-a; suas mãos potentes quebraram"-lhe a
resistência e dobraram"-lhe a cabeça loura para o beijo de núpcias.

Mas a virgem vencida abriu para o macho vitorioso os grandes olhos azuis
e, encarando"-o a fito, murmurou a tremenda palavra que afasta:

--- Sou Marabá!

Ipojuca estarrece, como fulminado pelo raio, e deixa que a presa loura
fuja para o recesso das selvas.

\section*{QUADRO}

Ipojuca, o vencedor vencido, caminha de cabeça baixa, absorto em sonhos.
Vai de regresso à taba. O jaguar que tinha perseguido cruza"-se"-lhe à
frente. Ipojuca não o vê. A seta que lhe destinara cravou"-lha Eros no
coração.

\section*{QUADRO}

Na taba, Ipojuca, desde que regressou, vive arredio. Pensa. A cabeça lhe
estala. Travam"-se de razões seu cérebro e seu coração --- o dever de
solidariedade para com a tribo e o amor. Um impõe"-lhe o desprezo da
criatura maldita; outro pede"-a para o beijo.

\section*{LETREIRO}

\emph{Vence Amor --- o eterno vencedor, e Ipojuca volta ao ermo em
procura de Marabá.}

\section*{QUADRO}

A virgem loura, desde o encontro fatal, perdida tem a sua serenidade de
lírio. Cisma. Horas e horas passa imóvel, com o olhar absorto. Sua
veadinha ao lado inutilmente espera as carícias de sempre. Marabá não a
vê. Marabá esqueceu"-a.

Como esqueceu as borboletas amarelas que douram o úmido em redor da laje
onde jaz reclinada. Como não vê o casal de martins"-pescadores que a três
passos a espiam curiosos.

Marabá só vê o guerreiro de pele bronzeada que a subjugou com o braço
potente, que lhe premiu com violência a carne virgem, que lhe derramou
na alma um veneno mortal.

Marabá só vê o seu guerreiro. Vê"-lhe o vulto ereto, firme e forte como
os penedos. Vê"-lhe a musculatura mais rija que o tronco da peroba. Vê o
fogo que seus olhos chispam.

E com tamanha nitidez o vê que para ele estende os braços, amorosamente.
E Ipojuca, pois era Ipojuca em pessoa e não sua sombra o que ela via,
cai"-lhe nos braços e esmaga"-lhe nos lábios o primeiro beijo.

\section*{QUADRO}

Idílio. Marabá espera o seu guerreiro no alto de uma canjerana. Ipojuca
chega, procura"-a, chama"-a, aflito.

A resposta é um punhado de bagas rubras que a virgem lhe lança da
fronde. Ágil como o gorila, Ipojuca abarca o tronco da canjerana e
marinha galhos acima.

Ao ser alcançada, Marabá despenha"-se no rio e mergulha.

Susto do índio, logo seguido de alegria ao vê"-la emergir além. Lança"-se
à água, persegue"-a --- e são dois peixes de pasmosa agilidade que
brincam. Agarra"-a --- e a luta finda"-se na doce quebreira dos beijos.

\section*{QUADRO}

Moema, a formosa virgem por Anhembira destinada para esposa de Ipojuca,
desconfia dos modos de seu noivo. Aquelas contínuas ausências, aquele
incessante cismar, seu alheamento a tudo, dizem"-lhe com clareza que uma
rival se interpõe entre ambos.

E, como desconfia, segue"-o cautelosa. E tudo descobre, pois alcança o
rio onde, o coração varado de crudelíssima flecha, assiste, oculta em
propícia moita, às expansões amorosas dos ternos amantes. Adivinha quem
é a rival, pois que inda tem vivo na memória o caso da marabazinha
misteriosamente desaparecida.

\section*{QUADRO}

Moema regressa à tribo e, sequiosa de vingança, denuncia ao pajé o
esconderijo da virgem maldita.

O velho reúne os guerreiros, arenga"-os, incita"-os à vingança antes que
volte Anhembira, alongado numa expedição de vindita contra os brancos
invasores. Receia que o cacique perdoe à neta, movido pelas lágrimas da
velha Iná.

\section*{QUADRO}

Os guerreiros em marcha para a vingança.

\section*{QUADRO}

\textls[-20]{Surpreendidos pelos índios, os amantes fogem rio abaixo numa piroga. (É
difícil explicar o aparecimento desta providencial piroga, mas não
impossível. Derivou rio abaixo, por exemplo, e ali ficou enredada numa
tranqueira. Não esquecer de introduzir num dos quadros anteriores um
\emph{close"-up} da piroga.)}

Os índios metem"-se em outras pirogas. (Mais pirogas! É que não derivou
uma só, sim várias\ldots{}) E remam com fúria na esteira dos fugitivos.

\section*{QUADRO}

Continua a perseguição. Não há flechaços, para evitar"-se o perigo de
ferir"-se Ipojuca. Perseguição silenciosa, à força de remos que estalam.

\section*{QUADRO}

A noite vem e a regata continua ao luar.

\section*{QUADRO}

E descem os fugitivos até que, de súbito, dão de cara com um fortim
português.

\section*{LETREIRO}

\emph{Entre dois fogos!}

\section*{QUADRO}

Os remos caem das mãos de Ipojuca. Marabá aninha"-se"-lhe ao peito rijo,
indiferente à morte --- que nada há mais suave do que acabar assim, a
dois, em pleno apogeu do delírio do amor.

\section*{QUADRO}

Os índios perseguidores ganham terreno. São avistados pelos portugueses,
que logo acodem com os seus trabucos de boca de sino e abrem fuzilaria.

\section*{QUADRO}

Os perseguidores fogem desordenadamente. Ipojuca, ferido no peito, é
aprisionado juntamente com Marabá.

\section*{QUADRO}

Na praia, ao lado do seu arco, Ipojuca estorce"-se nas dores da agonia,
enquanto Marabá é levada à presença do capitão do forte, que demora um
minuto para apresentar"-se.

\section*{QUADRO}

Rodeiam"-na os lusos e admiram"-lhe a beleza do tipo europeu.

Nisto o capitão do fortim aparece. Interroga"-a; examina"-a cheio de
pasmo, como que tomado de vagos pressentimentos.

Marabá tem o anel que Iná lhe deu.

O capitão examina"-o e, assombrado, o reconhece.

--- Minha filha! --- exclama.

E numa delirante explosão de amor paterno abraça"-a e beija"-a com
frenesi.

\section*{QUADRO}

Ipojuca, a distância, estorce"-se na agonia. Vê a cena e, sem compreender
o que se passa, julga que o capitão, como um sátiro, rouba"-lhe a amante
querida.

Reúne as últimas forças, toma do arco, ajusta uma flecha e despede"-a
contra Marabá.

\section*{QUADRO}

A flecha crava"-se no peito da virgem loura, que desfalece e morre nos
braços do pai atônito, enquanto na praia o heroico Ipojuca exala o
derradeiro suspiro, murmurando:

\section*{LETREIRO}

\emph{--- Minha ou de ninguém!}

(Acendem"-se as luzes e enxugam"-se as lágrimas.)

\part{\textsc{o amor invisível}}

\chapterspecial{O fisco}{Conto de Natal\,\footnote[*]{Texto de 1918, publicado no livro \emph{Negrinha}.}}{}

\section*{PRÓLOGO}

\noindent{}No princípio era o pântano, com valas de agrião e rãs coaxantes. Hoje é
o parque do Anhangabaú, todo ele relvado, com ruas de asfalto, pérgola
grata a namoriscos noturnos, a Eva de Brecheret, a estátua dum
adolescente nu que corre --- e mais coisas. Autos voam pela via central,
e cruzam"-se pedestres em todas as direções. Lindo parque,
civilizadíssimo.

Atravessando"-o certa tarde, vi formar"-se ali um bolo de gente, rumo ao
qual vinha vindo um polícia apressado.

``\emph{Fagocitose}'', pensei. A rua é a artéria; os passantes, o
sangue. O desordeiro, o bêbado, o gatuno são os micróbios maléficos,
perturbadores do ritmo circulatório. O soldado de polícia é o glóbulo
branco --- o \emph{fagócito} de Metchnikoff.

Está de ordinário parado no seu posto, circunvagando olhares atentos.
Mal se congestiona o tráfego pela ação antissocial do desordeiro, o
fagócito move"-se, caminha, corre, cai a fundo sobre o mau elemento e
arrasta"-o para o xadrez.

Foi assim naquele dia.

Dia sujo, azedo. Céu dúbio, de decalcomania vista pelo avesso. Ar
arrepiado.

Alguém perturbara a paz do jardim, e em redor desse rebelde logo se
juntou um grupo de glóbulos vermelhos, vulgo passantes. E lá vinha agora
o fagócito fardado restabelecer a harmonia universal.

O caso girava em torno de uma criança maltrapilha, que tinha a tiracolo
uma caixa tosca de engraxate, visivelmente feita pelas suas próprias
mãos. Muito sarapantado, com lágrimas a brilharem nos olhos cheios de
pavor, o pequeno murmurava coisas de ninguém atendidas. Sustinha"-o pela
gola um fiscal da Câmara.

--- Então, seu cachorrinho, sem licença, hein? --- exclamava entre
colérico e vitorioso o mastim municipal, focinho muito nosso conhecido.
É um que não é um mas sim legião, e sabe ser tigre ou cordeiro conforme
o naipe do contraventor.

A miserável criança evidentemente não entendia, não sabia que coisa era
aquela de licença, tão importante, reclamada assim a empuxões brutais.
Foi quando entrou em cena o polícia.

Este glóbulo branco era preto. Tinha beiço de sobejar e nariz invasor de
meia cara, aberto em duas ventas acesas, relembrativas das cavernas de
Trofônio. Aproximou"-se e rompeu o magote com um napoleônico
``Espalha!''.

Humildes alas se abriram àquele Sésamo, e a Autoridade, avançando,
interpelou o Fisco:

--- Que encrenca é esta, chefe?

--- Pois este cachorrinho não é que está exercendo ilegalmente a
profissão de engraxate? Encontrei"-o banzando por aqui com estes troços,
a fisgar com os olhos os pés dos transeuntes e a dizer ``Engraxa,
freguês''. Eu vi a coisa de longe.

Vim pé ante pé, disfarçando e, de repente, \emph{nhoc}! ``Mostre a
licença'', gritei. ``Que licença?'', perguntou ele com arzinho de
inocência. ``Ah, você diz que licença, cachorro? Está me debochando,
ladrão? Espera que te ensino o que é licença, trapo!'' E agarrei"-o. Não
quer pagar a multa. Vou levá"-lo ao depósito, autuar a infração para
proceder de acordo com as posturas --- concluiu com soberbo entono o
cariado canino da Maxila Fiscal.

O solene Mata"-Piolho da Manopla Policial concordou.

--- É isso mesmo. Casca"-lhes!

E chiando por entre os dentes uma cusparada de esguicho, deu a sua
sacudidela suplementar no menino. Depois voltou"-se para os basbaques e
ordenou com império de soba africano:

--- Circula, paisanada! É ``purivido'' ajuntamentos de mais de um.

Os glóbulos vermelhos dispersaram"-se em silêncio. O buldogue lá seguiu
com o pequeno nas unhas. E o Pau de Fumo, em atitude de Bonaparte em
face das pirâmides, ficou, de dedo no nariz e boca entreaberta, a gozar
a prontidão com que, num ápice, sua energia resolvera o tumor maligno
formado na artéria sob a sua fiscalização.

\section*{O BRÁS}

Também lá, no princípio, era o charco --- terra negra, fofa, turfa
tressuante, sem outra vegetação além dessas plantinhas miseráveis que
sugam o lodo como minhocas. Aquém da várzea, na terra firme e alta, São
Paulo crescia. Erguiam"-se casas nos cabeços, e esgueiravam"-se ladeiras
encostas abaixo: a Boa Morte, o Carmo, o Piques; e ruas, Imperador,
Direita, São Bento. Poetas cantavam"-lhe as graças nascentes:

\emph{Ó Liberdade, ó Ponte Grande, ó Glória\ldots{}}

Deram"-lhe um dia o Viaduto do Chá, esse arrojo\ldots{} Os paulistanos pagavam
sessenta réis para, ao atravessá"-lo, conhecerem a vertigem dos abismos.
E em casa narravam a aventura às esposas e mães, pálidas de espanto. Que
arrojo de homem, o Jules Martin, que construíra aquilo!

Enquanto São Paulo crescia o Brás coaxava. Enluravam"-se naquele brejal
legiões de sapos e rãs. À noite, do escuro da terra um coral subia de
coaxos, \emph{panpans} de ferreiro, latidos de mimbuias, \emph{glu"-glus}
de untanhas; e por cima, no escuro do ar, vaga"-lumes ziguezagueantes
riscavam fósforos às tontas.

E assim foi até o dia da avalanche italiana.

Quando lá no Oeste a terra roxa se revelou mina de ouro das que pagam
duzentos por um, a Itália vazou para cá a espuma da sua transbordante
taça de vida. E São Paulo, não bastando ao abrigo da nova gente,
assistiu, atônito, ao surto do Brás.

Drenos sangraram em todos os rumos o brejal turfoso; a água escorreu; os
espavoridos sapos sumiram"-se aos pulos para as baixadas do Tietê; rã
comestível não ficou uma para memória da raça; e, breve, em substituição
aos guembês, ressurtiu a cogumelagem de centenas e centenas de casinhas
típicas --- porta, duas janelas e platibanda.

Numerosas ruas, alinhadas na terra cor de ardósia que já o sol
ressequira e o vento erguia em nuvens de pó negro, margearam"-se com
febril rapidez desses prediozinhos térreos, iguais uns aos outros, como
saídos do mesmo molde, pífios, mas únicos possíveis então. Casotas
provisórias, desbravadoras da lama e vencedoras do pó à força de preço
módico.

E o Brás cresceu, espraiou"-se de todos os lados, comeu todo o barro
preto da Mooca, bateu estacas no Marco da Meia Légua, lançou"-se rumo à
Penha, pôs de pé igrejas, macadamizou ruas, inçou"-se de fábricas, viu
surgirem avenidas e vida própria, e cinemas, e o Colombo, e o namoro, e
o corso pelo Carnaval. E lá está hoje enorme, feito a cidade do Brás,
separado de São Paulo pelo faixão vermelho da Várzea aterrada --- Pest
da Buda à beira do Tamanduateí plantada.

São duas cidades vizinhas, distintas de costumes e de almas já bem
diversas.

Ir ao Brás é uma viagem. O Brás não é ali, como o Ipiranga; é lá do
outro lado, embora mais perto que o Ipiranga. Diz"-se vou ao Brás como
quem diz vou à Itália. Uma Itália agregada como um bócio recente e
autônomo a uma \emph{urbs} antiga, filha do país; uma Itália função da
terra negra, italiana por sete décimos e \emph{algo nuevo} pelos
restantes.

O Brás trabalha de dia e à noite gesta. Aos domingos fandanga ao som do
bandolim. Nos dias de festa nacional (destes tem predileção pelo 21 de
Abril: vagamente o Brás desconfia que o barbeiro da Inconfidência,
porque barbeiro, havia de ser um patrício), nos dias feriados o Brás vem
a São Paulo. Entope os bondes no travessio da Várzea e cá ensardinha"-se
nos autos: o pai, a mãe, a sogra, o genro e a filha casada no banco de
trás; o tio, a cunhada, o sobrinho e o Pepino escoteiro no da frente;
filhos miúdos por entremeio; filhos mais taludos ao lado do motorista;
filhos engatinhantes debaixo dos bancos; filhos em estado fetal no
ventre bojudo das matronas. Vergado de molas, o carro geme sob a carga e
arrasta"-se a meia velocidade, exibindo a Pauliceia aos olhos arregalados
daquele exuberante cacho humano.

Finda a corrida, o auto debulha"-se do enxame no Triângulo e o bando toma
de assalto as confeitarias para um regabofe de \emph{spumones}, gasosas,
croquetes. E tão a sério toma a tarefa que ali pelas nove horas não
restam iscas de empada nos armários térmicos, nem vestígios de sorvete
no fundo das geladeiras. O Brás devora tudo, ruidosa, alegremente, e com
massagens ajeitadoras do abdômen sai impando bem"-aventurança estomacal.
Caroços de azeitonas, palitos de camarões, guardanapos de papel, pratos
de papelão seguem nas munhecas da petizada como lembrança da festa e
consolo ao bersalherzinho que lá ficou de castigo em casa, berrando com
goela de Caruso.

Em seguida, toca para o cinema! O Brás abarrota os de sessão corrida. O
Brás chora nos lances lacrimogêneos da Bertini e ri nas comédias a gás
hilariante da L"-Ko mais do que autorizam os mil e cem da entrada. E
repete a sessão, piscando o olho: é o jeito de dobrar a festa em
extensão e obtê"-la a meio preço --- quinhentos e cinquenta réis, uma
pechincha.

As mulheres do Brás, ricas de ovário, são vigorosíssimas de útero.
Desovam quase filho e meio por ano, sem interrupção, até que se acabe a
corda ou rebente alguma peça essencial da gestatória.

É de vê"-las na rua. Bojudas de seis meses, trazem um Pepininho à mão e
um choramingas à mama. À tarde o Brás inteiro chia de criançalha a
chutar bolas de pano, a jogar pião, ou a piorra, ou o tento de telha, ou
o tabefe, com palavreados mistos de português e dialetos de Itália.
Mulheres escarranchadas às portas, com as mãos ocupadas em manobras de
agulha de osso, espigaitam para os maridos os sucessos do dia, que eles
ouvem filosoficamente, cachimbando calados ou cofiando a bigodeira à
Humberto Primo.

De manhã esfervilha o Brás de gente estremunhada a caminho das fábricas.

A mesma gente reflui à tarde aos magotes --- homens e mulheres de cesta
no braço, ou garrafas de café vazias penduradas do dedo; meninas,
rapazes, raparigotas de pouco seio, galantes, tagarelas, com o namoro
rente.

Desce a noite, e nos desvãos de rua, nos becos, nas sombras, o amor
lateja. Ciciam vozes cautelosas das janelas para os passeios; pares em
conversa disfarçada nos portões emudecem quando passa alguém ou tosse lá
dentro o pai.

Durante o escuro das fitas, nos cinemas, há contatos, longos,
febricitantes; e quando nos intervalos irrompe a luz, não sabem os
namorados o que se passou na tela --- mas estão de olhos langues, em
quebreira de amor.

É o latejar da messe futura. Todo aquele eretismo por música, com cicios
de pensamentos de cartão"-postal, estará morto no ano seguinte ---
legalizado pela Igreja e pelo juiz, transfeita a sua poesia em choro de
criança e nas trabalheiras sem"-fim da casa humilde.

Tal menina rosada, leve de andar, toda requebros e dengues, que passa na
rua vestida com graça e atrai os olhares gulosos dos homens, não a
reconhecereis dois anos depois na lambona filhenta que deblatera com o
verdureiro a propósito do feixe de cenouras em que há uma menor que as
outras.

Filho da lama negra, o Brás é como ela um sedimento de aluvião. É São
Paulo, mas não é a Pauliceia. Ligados pela expansão urbana, separa"-os
uma barreira. O velho caso do fidalgo e do peão enriquecido.

\section*{PEDRINHO, SEM SER CONSULTADO, NASCE}

Viram"-se, ele e ela. Namoraram"-se. Casaram.

Casados, proliferaram. Eram dois. O amor transformou"-os em três. Depois
em quatro, em cinco, em seis\ldots{}

Chamava"-se Pedrinho o filho mais velho.

\section*{A VIDA}

De pé na porta a mãe espera o menino que foi à padaria. Entra o pequeno
com as mãos abanando.

--- Diz que subiu; custa agora oitocentos.

A mulher, com uma criança ao peito, franze a testa desconsolada.

--- Meu Deus! Onde iremos parar? Ontem era a lenha; hoje é o pão\ldots{} Tudo
sobe. Roupa, pela hora da morte. José ganhando sempre a mesma coisa. Que
será de nós, Deus do céu!

E voltando"-se para o filho:

--- Vá a outra padaria, quem sabe se lá\ldots{} Se for a mesma coisa, traga
só um pedaço.

Pedrinho sai. Nove anos. Franzino, doentio, sempre mal alimentado e
vestido com os restos das roupas do pai.

Trabalha este num moinho de trigo, ganhando jornal insuficiente para a
manutenção da família. Se não fosse a bravura da mulher, que lavava para
fora, não se sabe como poderiam subsistir. Todas as tentativas feitas
com o intuito de melhorarem a vida com indústrias caseiras esbarraram no
óbice tremendo do Fisco. A fera condenava"-os à fome. Assim escravizados,
José perdeu aos poucos a coragem, o gosto de viver, a alegria. Vegetava,
recorrendo ao álcool para alívio de uma situação sem remédio.

Bendito sejas, amável veneno, refúgio derradeiro do miserável, gole
inebriante de morte que faz esquecer a vida e lhe resume o curso!
Bendito sejas!

Apesar de moça, vinte e sete apenas, Mariana aparentava o dobro. A
labuta permanente, os partos sucessivos, a chiadeira da filharada, a
canseira sem"-fim, o serviço emendado com o serviço, sem folga outra além
da que o sono força, fizeram da bonita moça que fora a escanzelada besta
de carga que era.

Seus dez anos de casada\ldots{} Que eternidade de canseiras!\ldots{}

Rumor à porta. Entra o marido. A mulher, ninando a pequena de peito,
recebe"-o com a má nova.

--- O pão subiu, sabe?

Sem murmurar palavra o homem senta"-se, apoiando nas mãos a cabeça.

Está cansado.

A mulher prossegue:

--- Oitocentos réis o quilo agora. Ontem foi a lenha; hoje é o pão\ldots{} E
lá?

Sempre aumentaram o jornal?

O marido esboçou um gesto de desalento e permaneceu mudo, com o olhar
vago. A vida era um jogo de engrenagens de aço entre cujos dentes se
sentia esmagar. Inútil resistir. Destino, sorte.

Na cama, à noite, confabulavam. A mesma conversa de sempre. José acabava
grunhindo rugidos surdos de revolta. Falava em revolução, saque. A
esposa consolava"-o, de esperança posta nos filhos.

--- Pedrinho tem nove anos. Logo estará em ponto de ajudar"-nos. Um pouco
mais de paciência e a vida melhora.

Aconteceu que nessa noite Pedrinho ouviu a conversa e a referência à sua
futura ação. Entrou a sonhar. Que fariam dele? Na fábrica, como o pai?
Se lhe dessem a escolher, iria a engraxador. Tinha um tio no ofício, e
em casa do tio era menor a miséria. Pingavam níqueis.

Sonho vai, sonho vem, brota na cabeça do menino uma ideia, que cresceu,
tomou vulto extraordinário e fê"-lo perder o sono. Começar já, amanhã,
por que não? Faria ele mesmo a caixa; escovas e graxa, com o tio
arranjaria. Tudo às ocultas, para surpresa dos pais! Iria postar"-se num
ponto por onde passasse muita gente. Diria como os outros: ``Engraxa,
freguês!'', e níqueis haviam de juntar"-se no seu bolso. Voltaria para
casa recheado, bem tarde, com ar de quem as fez\ldots{} E mal a mãe começasse
a ralhar, ele lhe taparia a boca despejando na mesa o monte de dinheiro.
O espanto dela, a cara admirada do pai, o regalo da criançada com a
perspectiva da ração em dobro! E a mãe a apontá"-lo aos vizinhos: ``Estão
vendo que coisa? Ganhou, só ontem, primeiro dia, dois mil"-réis!''. E a
notícia a correr\ldots{} E murmúrios na rua quando o vissem passar: ``É
aquele!''.

Pedrinho não dormiu essa noite. De manhãzinha já estava a dispor a
madeira dum caixote velho sob forma de caixa de engraxate ao molde
clássico. Lá a fez.

Os pregos, bateu com o salto de uma velha botina. As tábuas, serrou
pacientemente com um facão dentado. Saiu coisa tosca e mal"-ajambrada, de
fazer rir a qualquer carapina e pequena demais --- sobre ela só caberia
um pé de criança igual ao seu. Mas Pedrinho não notou nada disso, e
nunca trabalho nenhum de carpintaria lhe pareceu mais perfeito.

Conclusa a caixa, pô"-la a tiracolo e esgueirou"-se para a rua, às
escondidas. Foi à casa do tio e lá obteve duas velhas escovas fora de
uso, já sem pelos, mas que à sua exaltada imaginação se afiguraram
ótimas. Graxa, conseguiu alguma raspando o fundo de quanta lata velha
encontrou no quintal.

Aquele momento marcou em sua vida um apogeu de felicidade vitoriosa. Era
como um sonho --- e sonhando saiu para a rua. Em caminho viu o dinheiro
crescer"-lhe nas mãos, aos montes. Dava à família parte, e o resto
encafuava.

\textls[-5]{Quando enchesse o canto da arca onde tinha suas roupas, montaria um
``corredor'', pondo a jornal outros colegas. Aumentaria as rendas!
Enriqueceria! Compraria bicicletas, automóvel, doces todas as tardes na
confeitaria, livros de figura, uma casa, um palácio, outro palácio para
os pais. Depois\ldots{}}

Chegou ao parque. Tão bonito aquilo --- a relva tão verde, tosadinha\ldots{}
Havia de ser bom o ponto. Parou perto de um banco de pedra e, sempre
sonhando as futuras grandezas, pôs"-se a murmurar para cada passante,
fisgando"-lhe os pés:

--- Engraxa, freguês!

Os fregueses passavam sem lhe dar atenção. ``É assim mesmo'', refletia
consigo o menino; ``no começo custa. Depois se afreguesam.''

Súbito, viu um homem de boné caminhando para o seu lado. Olhou"-lhe para
as botinas. Sujas. Viria engraxar, com certeza --- e o coração bateu"-lhe
apressado, no tumulto delicioso da estreia. Encarou o homem já a cinco
passos e sorriu com infinita ternura nos olhos, num agradecimento
antecipado em que havia tesouros de gratidão.

Mas em vez de lhe espichar o pé, o homem rosnou aquela terrível
interpelação inicial:

--- Então, cachorrinho, que é da licença?

\section*{EPÍLOGO? NÃO! PRIMEIRO ATO\ldots{}}

Horas depois o fiscal aparecia em casa de Pedrinho com o pequeno pelo
braço.

Bateu. O pai estava, mas quem abriu foi a mãe. O homem nesses momentos
não aparecia, para evitar explosões. Ficou a ouvir do quarto o
bate"-boca.

O fiscal exigia o pagamento da multa. A mulher debateu"-se, arrepelou"-se.
Por fim, rompeu em choro.

--- Não venha com lamúrias --- rosnou o buldogue ---; conheço o truque
dessa aguinha nos olhos. Não me embaça, não. Ou bate aqui os vinte
mil"-réis, ou penhoro toda esta cacaria. Exercer ilegalmente a profissão!
Ora dá"-se! E olhe cá, madama, considere"-se feliz de serem só vinte. Eu é
de dó de vocês, uns miseráveis; senão, aplicava o máximo. Mas se resiste
dobro a dose!

A mulher limpou as lágrimas. Seus olhos endureceram, com uma chispa má
de ódio represado a faiscar. O Fisco, percebendo"-o, motejou:

--- Isso. É assim que as quero --- tesinhas, ah, ah.

Mariana nada mais disse. Foi à arca, reuniu o dinheiro existente ---
dezoito mil"-réis ratinhados havia meses, aos vinténs, para o caso
dalguma doença, e entregou"-os ao Fisco.

--- É o que há --- murmurou com tremura na voz.

O homem pegou o dinheiro e gostosamente o afundou no bolso, dizendo:

--- Sou generoso, perdoo o resto. Adeuzinho, amor!

E foi à venda próxima beber dezoito mil"-réis de cerveja.

Enquanto isso, no fundo do quintal, o pai batia furiosamente no menino.

\chapter{As fitas da vida\footnote[*]{Texto de 1920, publicado no livro \emph{Negrinha}.}\,\footnote[**]{Na primeira   edição o título deste conto era simplesmente ``Fitas da vida''. Nota da edição de 1955.}}

Perambulávamos ao sabor da fantasia, noite adentro, pelas ruas feias do
Brás, quando nos empolgou a silhueta escura duma pesada mole tijolácea,
com aparência de usina vazia de maquinismos.

--- Hospedaria dos Imigrantes --- informa o meu amigo.

--- É aqui, então\ldots{}

Paramos a contemplá"-la. Era ali a porta do Oeste Paulista, essa Canaã em
que o ouro espirra do solo; era ali a antessala da Terra Roxa --- essa
Califórnia do rubídio, oásis cor de sangue coalhado onde cresce a árvore
do Brasil de amanhã, uma coisa um pouco diferente do Brasil de ontem,
luso e perro; era ali o ninho da nova raça, liga, amálgama, justaposição
de elementos étnicos que temperam o neobandeirante industrial, antijeca,
antimodorra, vencedor da vida à moda americana.

Onde pairam os nossos Walt Whitmans, que não veem estes aspectos do país
e os não põem em cantos? Que crônica, que poema não daria aquela casa da
Esperança e do Sonho! Por ela passaram milhares de criaturas humanas, de
todos os países e de todas as raças, miseráveis, sujas, com o estigma
das privações impresso nas faces --- mas refloridas de esperança ao
calor do grande sonho da América. No fundo, heróis, porque só os heróis
esperam e sonham.

Emigrar: não pode existir fortaleza maior. Só os fortes atrevem"-se a
tanto. A miséria do torrão natal cansa"-os e eles se atiram à aventura do
desconhecido, fiando na paciência dos músculos a vitória da vida. E
vencem.

Ninguém, ao vê"-los na Hospedaria, promíscuos, humildes, quase muçulmanos
na surpresa da terra estranha, imagina o potencial de força neles
acumulado, à espera de ambiente propício para explosões magníficas.

Cérebro e braço do progresso americano, gritam o Sésamo às nossas
riquezas adormidas. Estados Unidos, Argentina, São Paulo devem dois
terços do que são a essa varredura humana, trazida a granel para aterrar
os vazios demográficos das regiões novas. Mal cai no solo novo,
transforma"-se, floresce, dá de si a apojadura farta com que se aleita a
Civilização.

Aquela Hospedaria\ldots{} Casa do Amanhã, corredor do futuro\ldots{}

Por ali desfilam, inconscientes, os formadores duma raça nova.

--- Dei"-me com um antigo diretor desta almanjarra --- disse o meu
companheiro ---, ao qual ouvi muita coisa interessante acontecida cá
dentro. Sempre que passo por esta rua, avivam"-se"-me na memória vários
episódios sugestivos, e entre eles um, romântico, patético, que até
parece arranjo para terceiro ato de dramalhão lacrimogêneo. O
romantismo, meu caro, existe na natureza, não é invenção dos Hugos; e
agora que se fez cinema, posso assegurar"-te que muitas vezes a vida
plagia o cinema escandalosamente.

``Foi em 1906, mais ou menos. Chegara do Ceará, então flagelado pela
seca, uma leva de retirantes com destino à lavoura de café, na qual
havia um cego, velho de mais de sessenta anos. Na sua categoria dolorosa
de indesejável, por que cargas"-d'água dera com os costados aqui? Erro de
expedição, evidentemente. Retirantes que emigram não merecem grande
cuidado dos prepostos ao serviço. Vêm a granel, como carga incômoda que
entope o navio e cheira mal. Não são passageiros, mas fardos de couro
vivo com carne magra por dentro, a triste carne de trabalho, irmã da
carne de canhão.

``Interpelado o cego por um funcionário da Hospedaria, explicou sua
presença por engano de despacho. Destinavam"-no ao Asilo dos Inválidos da
Pátria, no Rio, mas pregaram"-lhe às costas a papeleta do `Para o eito' e
lá veio. Não tinha olhos para guiar"-se, nem teve olhos alheios que o
guiassem. Triste destino o dos cacos de gente\ldots{}

``--- Por que para o Asilo dos Inválidos? --- perguntou o funcionário.
--- É voluntário da Pátria?

``--- Sim --- respondeu o cego ---, fiz cinco anos de guerra no Paraguai
e lá apanhei a doença que me pôs a noite nos olhos. Depois que ceguei
caí no desamparo. Para que presta um cego? Um gato sarnento vale mais.

``Pausou uns instantes, revirando nas órbitas os olhos esbranquiçados.
Depois:

``--- Só havia no mundo um homem capaz de me socorrer: o meu capitão.
Mas, esse, perdi"-o de vista. Se o encontrasse --- tenho a certeza! ---,
até os olhos me era ele capaz de reviver. Que homem! Minhas desgraças
todas vêm de eu ter perdido meu capitão\ldots{}

``--- Não tem família?

``--- Tenho uma menina --- que não conheço. Quando veio ao mundo, já
meus olhos eram trevas.

``Baixou a cabeça branca, como tomado de súbita amargura.

``--- Daria o que me resta de vida para vê"-la um instantinho só. Se o
meu capitão\ldots{}

``Não concluiu. Percebera que o interlocutor já estava longe, atendendo
ao serviço, e ali ficou, imerso na tristeza infinita da sua noite sem
estrelas.

``O incidente, entretanto, impressionara o funcionário, que o levou ao
conhecimento do diretor. O diretor da Imigração era nesse tempo o major
Carlos, nobre figura de paulista dos bons tempos, providência humanizada
daquele departamento. Ao saber que o cego fora um soldado de 70,
interessou"-se e foi procurá"-lo. Encontrou"-o imóvel, imerso no seu eterno
cismar.

``--- Então, meu velho, é verdade que fez a campanha do Paraguai?

``O cego ergueu a cabeça, tocado pela voz amiga.

``--- Verdade, sim, meu patrão. Fui soldado do 33.

``--- O 33 de São Paulo? Como isso, se você é do Norte? --- objetou o
major.

``--- Verdade, sim, meu patrão. Vim no 13, e logo depois de chegar ao
império do Lopes entrei em fogo. Tivemos má sorte. Na batalha de Tuiuti
nosso batalhão foi dizimado como milharal em tempo de chuva de pedra.
Salvamo"-nos eu e mais um punhado de camaradas. Fomos incorporados ao 33
paulista para preenchimento dos claros, e nele fiz o resto da campanha.

``O major Carlos também era veterano do Paraguai, e por coincidência
servira no 33. Interessou"-se, pois, vivamente pela história do cego,
pondo"-se a interrogá"-lo a fundo.

``--- Quem era o seu capitão?

``O cego suspirou.

``--- Meu capitão era um homem que se eu o encontrasse de novo até a
vista me era capaz de dar! Mas não sei dele, perdi"-o --- para mal meu\ldots{}

``--- Como se chamava?

``--- Capitão Boucault.

``Ao ouvir esse nome o major sentiu eletrizarem"-se"-lhe as carnes num
arrepio intenso; dominou"-se, porém, e prosseguiu:

``--- Conheci esse capitão. Foi meu companheiro de regimento. Mau homem,
por sinal, duro para com os soldados, grosseiro\ldots{}

``O cego, até ali vergado na atitude humilde do mendigo, ergueu
altivamente o busto e, com indignação a fremir na voz, disse com
firmeza:

``--- Pare aí! Não blasfeme! O capitão Boucault era o mais leal dos
homens, amigo, pai do soldado. Perto de mim ninguém o insulta. Conheci"-o
em todos os momentos, acompanhei"-o durante anos como sua ordenança e
nunca o vi praticar o menor ato de vileza.

``O tom firme do cego comoveu estranhamente o major. A miséria não
conseguira romper no velho soldado as fibras da lealdade, e não há
espetáculo mais arrebatador do que o de uma lealdade assim vivedoira até
aos limites extremos da desgraça. O major, quase rendido, sobresteve"-se
por um instante. Depois, friamente, prosseguiu na experiência.

``--- Engana"-se, meu caro. O capitão Boucault era um covarde\ldots{}

Um assomo de cólera transformou as feições do cego. Seus olhos anuviados
pela catarata revolveram"-se nas órbitas, num horrível esforço para ver a
cara do infame detrator. Seus dedos crisparam"-se; todo ele se retesou,
como fera prestes a desferir o bote. Depois, sentindo pela primeira vez
em toda a plenitude a infinita fragilidade dos cegos, recaiu em si,
esmagado. A cólera transfez"-se"-lhe em dor, e a dor assomou"-lhe aos olhos
sob forma de lágrimas. E foi lacrimejando que murmurou em voz apagada:

``--- Não se insulta assim um cego\ldots{}

``Mal pronunciara estas palavras, sentiu"-se apertado nos braços do
major, também em lágrimas, que dizia:

``--- Abrace, amigo, abrace o seu velho capitão! Sou eu o antigo capitão
Boucault\ldots{}

``Na incerteza, aparvalhado ante o imprevisto desenlace e como receoso
de insídia, o cego vacilava.

``--- Duvida? --- exclamou o major. --- Duvida de quem o salvou a nado
na passagem do Tebiquari?

``Àquelas palavras mágicas a identificação se fez e, esvanecido de
dúvidas, chorando como criança, o cego abraçou"-se com os joelhos do
major Carlos Boucault, a exclamar num desvario:

``--- Achei meu capitão! Achei meu pai! Minhas desgraças se
acabaram!\ldots{}''

``E acabaram"-se de fato.

``Metido num hospital sob os auspícios do major, lá sofreu a operação da
catarata e readquiriu a vista.

``Que impressão a sua quando lhe tiraram a venda dos olhos! Não se
cansava de `ver', de matar as saudades da retina. Foi à janela e sorriu
para a luz que inundava a natureza. Sorriu para as árvores, para o céu,
para as flores do jardim. Ressurreição!\ldots{}

``--- Eu bem dizia! --- exclamava a cada passo. --- Eu bem dizia que se
encontrasse o meu capitão estava findo o meu martírio. Posso agora ver
minha filha! Que felicidade, meu Deus!\ldots{}

``E lá voltou para a terra dos verdes mares bravios onde canta a
jandaia. Voltou a nado --- nadando em felicidade. A filha, a filha!\ldots{}

``--- Eu não dizia? Eu não dizia que se encontrasse o meu capitão até a
luz dos olhos me havia de voltar?''

\chapter{Era no Paraíso\ldots{}\footnote[*]{Texto de 1923, publicado no livro \emph{O macaco que se fez homem}.}}

Era no paraíso e deus estava contente. Tinha criado a luz, as estrelas,
o ar, a água e por fim criou a Vida, semeando"-a sob milhares de formas
por cima da terra fresquinha e nua. E esfervilhou de viventes o orbe,
aqui bactéria e mastodonte, ali musgo e baobá, além craca e baleia --- a
suma variedade de aspectos dentro da perfeita unidade de plano.

E Deus, que achara aquilo bom, deliberou consolidar sua obra de vida
\emph{per secula seculorum} com o invento da Fome e do Amor, dois
apetites tremendos engastados no âmago das criaturas à guisa de
moto"-contínuo da Perpetuação. E cofiando a imensa barba branca, velha
como o Tempo, lançou a palavra mágica que tudo move e tudo explica:

--- Comei"-vos uns aos outros e nos intervalos amai!

Em seguida elaborou para regência da animalidade o Código da Sabedoria
Ingênita.

Não deu esse nome ao Código, visto como, no começo, não existindo homem,
não existiam nomes.

--- Não existindo homens?\ldots{}

Sim, o homem não estava nos planos do Criador. Esta revelação mirífica,
que ainda há de roer pelos alicerces as caducas verdades oficiais (e
talvez me conquiste o prêmio Nobel), está ansiosinha por me fugir da
pena. Que fuja, que se espoje no espírito do leitor. Adeus, filha!\ldots{}

Não era escrito esse Código. Lei escrita vale por pura invenção humana
--- donde a rapidez com que envelhecem os códigos humanos e as humanas
leis. Escrever é fixar e fixar é matar. Perpétuo movimento, a vida é
infixa. Entretanto, se o não escreveu, foi além Jeová: impregnou com ele
cada uma das criaturas recém"-formadas, de modo que ao nascer já viessem
ricas da sabedoria infusa e agissem automaticamente de acordo com os
imutáveis preceitos da lei natural.

Este saber sem aprender receberia do homem o nome de Intuição, assim
como o Código Ingênito receberia o nome de Instinto. Os futuros homens
se caracterizariam pelo vezo de dar nome às coisas, gozando"-se da fama
de sábios os que com maior entono e mais pomposamente as nomeassem.
Grande doutor, o que tomasse o pulso a um doente, lhe espiasse a língua
e gravibundo dissesse, tirando do nariz os óculos de ouro:
\emph{polinevrite metabólica}; e, grande mestre, o que apontasse o dedo
para um grupo de estrelas e declarasse com voz firme: \emph{constelação
do Centauro}. Doença e estrelas, com ou sem nome, seguiriam o seu curso
prefixo --- mas nada de louvores ao médico que apenas dissesse:
\emph{doença}, ou ao mestre que humilde murmurasse: \emph{astros}. Paga
ou louvor não os teria o ignorante, isto é, o homem que não sabe nomes.
Viva o nome!

Assim, inoculou Deus em todos os seres a sabedoria da vida e pô"-los no
orbe como notas cromáticas do \emph{pot"-pourri} sinfônico de cuja
audição integral somente os seus ouvidos gozariam o privilégio.

E Deus achou que estava ótimo.

Grandes coisas tinha feito. A gravitação dos mundos era jogo de
movimentos que mais tarde derrubaria o queixo a Newton --- mas não
passava de mecânica pura.

A concepção do éter, da luz, do calor, assombrosas invenções eram ---
mas mecânica fria.

O bonito fora a criação da Vida, porque, obra de arte das mais
autênticas, só ela dava medida completa dos imensos recursos do alto
engenho de Deus.

Quanta afinação no tumulto aparente! A bactéria às voltas com o
mastodonte, o musgo em simbiose com o baobá, a craca aparasitada à
baleia\ldots{} Vida em vida, vida devorando vida, vida sobrepondo"-se à vida,
vida criando vida\ldots{} O perpétuo ressoar dos uivos de cólera, berros de
dor, guinchos de alegria, gemidos de gozo sonorizando o perpétuo
agitar"-se das formas --- voo de ave, arranque de tigre, coleio de serpe,
rabanar de peixe, tocaiar de sáurio\ldots{}

Tão pitoresca saiu a ópera \textsc{vida} que o Sumo Esteta a elegeu para recreio
de sua Eterna Displicência. E, debruçado na amplidão, as longas barbas
dispersas ao vento, o contemplativo Jeová antecipou a figura do sábio
que no fundo dos laboratórios cisma sobre o microscópio.

Ora, pois, certo dia de estuporante mormaço, um casal de chimpanzés
dormitava beatificamente no esgalho de enorme embaúba. Digeriam as
bananas comidas e prelibavam, risonhos, as bananas da manhã seguinte.

Eram chimpanzés como os demais, sábios de sabedoria inculcada pelo
Eterno, e bem"-comportadinhas notas da ópera paradisíaca.

Mas Éolo suspirou no seu antro e um forte pé de vento deu, que
vascolejou com frenesi a árvore e fez o chimpanzé macho, perdido o
equilíbrio, precipitar"-se de ponta"-cabeça ao chão.

Seria aquilo um tombo como qualquer outro, sem consequências funestas,
se a malícia da serpente não houvesse colocado ao pé da embaúba uma
grande laje, na qual se chocou o crânio do infeliz desarvorado.

Perdeu os sentidos o macaco; e a macaca, presa de grande aflição, pulou
incontinênti a socorrê"-lo. Rondou"-lhe em torno aos guinchos, soprou"-lhe
nos olhos, amimou"-o, beliscou"-lhe as carnes insensíveis e, por fim,
convencida de que estava bem morto, deu de ombros, já com a ideia na
escolha de quem lhe consolasse a viuvez.

Mas não morrera o raio do chimpanzé. Minutos depois entreabria os olhos,
piscava sete vezes e levava as mãos à fronte, significando que lhe doía.

Neste comenos funga no juncal próximo um tigre. Desde o Paraíso que os
tigres ``adoram'' os macacos, como desde o Paraíso que os macacos
arrenegam dos tigres. Em virtude de tal divergência, a fungadela felina
valeu por frasco de amoníaco nas ventas do contuso. Pôs"-se de pé, inda
tonto e, ajudado da companheira, marinhou embaúba acima, rumo ao galho
de pouso, onde, a bom recato, pudesse distrair a dor de cabeça com a
linda cena que é um tigre faminto à caça de bicho que não seja
chimpanzé.

Desde essa desastrada queda nunca mais funcionou normalmente o cérebro
do pobre macaco. Doíam"-lhe os miolos, e ele queixava"-se de vágados e de
estranho mal"-estar.

É que sofrera seriíssima lesão.

Digo isto porque sou homem e sei dar nomes aos bois; homem ignorante,
porém, não vou mais longe, nem ponho nome grego à lesão. Afirmo apenas
que era lesão, certo de que me entendem os meus incontáveis colegas em
ignorância nomenclativa.

Lesão grave, gravíssima, e de resultados imprevisíveis à própria
presciência de Jeová.

A Bíblia já tratou do assunto; de modo simbólico, entretanto, fugindo de
tomar a Queda ao pé da letra. Moisés, redator do Gênesis, tinha
veleidades poéticas --- mas não previra Darwin, nem a força do prêmio
Nobel como áureo pai de grandes descobertas. Moisés poetizou\ldots{} Fez um
Adão, uma Eva, uma serpente e um pomo, que certos exegetas declaram ser
a maçã, e outros, a banana. Compôs assim uma peça com a mestria
consciente de Edgar Poe ao carpinteirar \emph{O corvo}, mas sem deixar,
como Poe, um estudo da psicologia da composição, onde demonstrasse que
fez aquilo por a + b e com bem estudada pontaria. E foi pena! Quanto
papel, tinta e sangue tal esclarecimento não pouparia à humanidade,
sempre rixenta na interpretação dos textos bíblicos!

Vem daí que é o Gênesis uma peça de fina psicologia, e por igual
penetrante nas cabeças duras e nas dos Pascais, permeabilíssimas; o que
escasseia ao Gênesis é acordo com a verdade dos fatos. Essa verdade,
mais preciosa que o diamante Cullinan, eu a achei sob o montão de
cascalho das hipóteses e sem nenhum alarde aqui a estampo de graça. Já é
ser generoso! Tenho nas unhas a verdade das verdades e não requeiro do
Congresso um prêmio de cinquenta contos! Contento"-me com um apenas\ldots{}

A partir da Queda, o nosso macaco entrou a mudar de gênio. Sua cabeça
perdeu o frescor da antiga despreocupação e deu de elaborar uns
mostrengozinhos, informes, aos quais, com alguma licença, caberia o nome
de ideias.

Vacilava, ele que nunca vacilara e sempre agira com os soberbos impulsos
do automatismo. Entre duas bananas pateteava na escolha tomado de
incompreensíveis indecisões --- e por vezes perdeu ambas, iludido por
monos de bote pronto que não vacilavam nem escolhiam.

Para galgar de um ramo a outro calculava agora não só a distância como a
força do salto --- e errava, ele que antes da lesão nunca errara pulo.

Até em suas relações sentimentais com a velha companheira o chimpanzé
variou. Ganho de malsãs curiosidades, examinava as outras macacas do
bando, comparava"-as à sua e cometia o pecado de desejar a macaca do
próximo.

Como também claudicasse na escolha das frutas, comeu diversas impróprias
à alimentação símia, daí provindo as primeiras perturbações
gastrointestinais observadas na higidez do Paraíso --- enterites,
colites, disenteria ou o que seja.

Quando iam águias pelo céu, punha"-se a contemplar os seus harmoniosos
voos, com vagos anseios nas tripas e muito desejo na alma de ser águia.
Era a inveja a nascer, má cuscuta que vicejaria luxuriantemente na
execrável descendência desse mono. Invejou as aves que dormiam em ninho
fofo e os animais que moravam em boas tocas de pedra. Abandonou o viver
em árvore, prescrito para os da sua laia pelo Código Ingênito, e deu de
andar sobre a terra de pé sobre as patas traseiras, com as dianteiras
--- futuras mãos --- ocupadas em construir ninho, como os via fazer às
perdizes, ou toca, como as tem o tatu.

E sempre nervoso e inquieto, e descontente com a ordem das coisas
estabelecida no Éden, imaginava mudanças e ``melhoramentos''. E variava
e tresvariava, e malucava, arrastando consigo a pobre companheira que,
sem nada compreender de tudo aquilo, em tudo o imitava passivamente,
dócil e meiga.

Aconteceu o que tinha de acontecer. A admirável disciplina reinante no
Éden viu"-se logo perturbada pelo estranho proceder do macaco, advindo
daí murmurações e por fim queixas a Jeová. E tais e tantas foram as
queixas, que o Sumo, zangado com a nota desafinadora da sua música
divina, ordenou ao anjo Gabriel que pusesse no olho da rua o sustenido
anárquico.

Até esse ponto vai certo Moisés. Onde começa a fazer poesia é daí por
diante. De fato, Jeová ordenou a expulsão do rebelde e são Gabriel deu
para executá"-la os primeiros passos. A curiosidade, porém, que dizem
feminina mas aqui se vê que é divina, fez o Criador reconsiderar.

--- Suspende, Gabriel! Estou curioso de ver até que extremos irá o
desarranjo mental do meu macaco.

Era Gabriel o \emph{Sarrazani} daquele jardim zoológico e, graças ao
convívio com o Eterno, adquirira alguma coisa da divina presciência.
Assim foi que objetou:

--- Vossa Eternidade me perdoe, mas se lá deixamos o trapalhão aquilo
vira em ``humanidade''\ldots{}

--- Sei disso --- retorquiu o Soberano Senhor de todas as coisas. --- A
lesão do cérebro do meu macaco põe"-no à margem da minha Lei Natural e
fa"-lo"-á discrepar da harmonia estabelecida. Nascerá nele uma
\emph{doença}, que seus descendentes, cheios de orgulho, chamarão
inteligência --- e que, ai deles!, lhes será funestíssima. Esse mal,
oriundo da Queda, transmitir"-se"-á de pais a filhos --- e crescerá
sempre, e terrivelmente influirá sobre a terra, modificando"-lhe a
superfície de maneira muito curiosa. E, deslumbrados por ela, os homens
ter"-se"-ão na conta de criaturas privilegiadas, entes à parte no
universo, e olharão com desprezo para o restante da animalidade. E será
assim até que um senhor Darwin surja e prove a verdadeira origem do
\emph{Homo sapiens}\ldots{}

--- ?!

--- Sim. Eles nomear"-se"-ão \emph{Homo sapiens} apesar do teu sorriso,
Gabriel, e ter"-se"-ão como feitos por mim de um barro especial e à minha
imagem e semelhança.

--- ?!!

\textls[-10]{--- Os demais chimpanzés permanecerão como eu os criei; só o ramo agora
a iniciar"-se com a prole do lesado é que se destina a sofrer a
diferenciação mórbida, cuja resultante será cair o governo da terra nas
unhas de um bicho que não previ.}

--- ?!!!

--- Essa inteligência se caracterizará pela ânsia de ver"-me através das
coisas, e para que bem a compreendas, Gabriel, te direi que será como
asas sem ave, luz sem sol, dedos sem pés\ldots{}

Gabriel não compreendeu coisa nenhuma da longa definição de Jeová --- e
como sucederia o mesmo com os meus leitores, interrompo"-a nos dedos sem
pés. Até aí ainda a percepção é possível; mas no ponto em que Jeová lhe
assinalou a essência última, nem Einstein pescaria um \emph{x}\ldots{}

Vendo o ar aparvalhado de Gabriel, o Criador pulou da metagênese abaixo
e falou fisicamente.

--- Essa inteligência apurará aos extremos a crueldade, a astúcia e a
estupidez. Por meio da astúcia se farão eles engenhosos, porque o
engenho não passa da astúcia aplicada à mecânica. E à força de engenho
submeterão todos os outros animais, e edificarão cidades, e esfuracarão
montanhas, e rasgarão istmos, destruirão florestas, captarão fluidos
ambientes, domesticarão as ondas hertzianas, descobrirão os raios
cósmicos, devassarão o fundo dos mares, roerão as entranhas da terra\ldots{}

Gabriel estremeceu. Apavorou"-o a força futura da inteligência nascente;
mas Jeová sorriu, e quando Jeová sorria Gabriel serenava.

--- Nada receies. Essa inteligência terá alguns atributos da minha, como
o carvão os tem do diamante, mas estará para a minha como o carvão está
para o diamante. A fraqueza dela provirá da sua jaça de origem.
Inteligência sem memória, inteligência de chimpanzé, o homem
\emph{esquecerá} sempre. Esquecerá o que ensinei aos seus precursores
peludos e esquecerá de colher a boa lição da experiência nova.

``Seu engenho criará engenhosíssimas armas de alto poder destrutivo ---
e empolgados pelo ódio se estraçalharão uns aos outros em nome de
pátrias, por meio de lutas tremendas a que chamarão guerras, vestidos
macacalmente, ao som de músicas, tambores e cornetas --- esquecidos de
que não criei nem ódio, nem corneta, nem pátria.

``E transporão mares, e perfurarão montes, e voarão pelo espaço, e
rodarão sobre trilhos na vertigem louca de vencer as distâncias e chegar
depressa --- esquecidos de que eu não criei a pressa nem o trilho.

``E viverão em guerra aberta com os animais, escravizando"-os e
matando"-os pelo puro prazer de matar --- esquecidos de que eu não criei
o prazer de matar por matar.

``E inventarão alfabetos e línguas numerosas, e disputarão sem tréguas
sobre gramática, e quanto mais gramáticas possuírem menos se entenderão.
E se entenderão de tal modo imperfeito que aclamarão o messias do
entendimento geral um doutor Zamenhoff\ldots{}''

--- Já sei! Um que proporá a supressão das línguas.

Jeová sorriu.

--- Não! Apenas o criador de mais uma. E eles elaborarão ciências e
excogitarão toda a mecânica das coisas, adivinhando o átomo e o planeta
invisível, e saberão tudo --- menos o segredo da vida.

``E um Pascal, muito cotado entre eles, dará murros na cabeça, na
tortura de compreender os \emph{xx} supremos --- e os homens admirarão
grandemente esses murros.

``E criarão artes numerosas, e terão sumos artistas e jamais alcançarão
a única arte que implantei no Éden --- a arte de ser biologicamente
feliz.

``E organizarão o parasitismo na própria espécie, e enfeitar"-se"-ão de
vícios e virtudes igualmente antinaturais. E inventarão o Orgulho, a
Avareza, a Má"-Fé, a Hipocrisia, a Gula, a Luxúria, o Patriotismo, o
Sentimentalismo, o Filantropismo, a Colocação dos Pronomes ---
esquecidos de que eu não criei nada disso e só o que eu criei é.

``E em virtude de tais e tais macacalidades, a inteligência do homem não
conseguirá nunca resolver nenhum dos problemas elementares da vida, em
contraste com os outros seres, que os terão a todos solvidos de maneira
felicíssima.

``Não saberá comer; e ao lado das minhas abelhas, de tão sábio regime
alimentar --- sábio porque por mim prescrito ---, o homem morrerá de
fome ou indigestão, ou definhará achacoso em consequência de erros ou
vícios dietéticos.

``Não saberá morar --- e ao lado das minhas aranhas, tão felizes na casa
que lhes ensinei, habitarão ascorosas espeluncas sem luz, ou palácios.

``Não resolverá o problema da vida em sociedade, e experimentará mil
soluções, errando em todas. E revoluções tremendas agitarão de espaço em
espaço os homens no desespero de destruir o parasitismo criado pela
inteligência --- e as novas formas de equilíbrio surgidas afirmar"-se"-ão
com os mesmos vícios das velhas formas destruídas. E o homem olhará com
inveja para os meus animaizinhos gregários, que são felizes porque
seguem a minha lei sapientíssima.

``E não solverá o problema do governo; e mais formas de governo invente,
mais sofrerá sob elas --- \emph{esquecido de que não criei governo}. E
criará o Estado, monstro de maxilas leoninas, por meio do qual a minoria
astuta parasitará cruelmente a maioria estúpida. E a fim de manter nédio
e forte esse monstro, os sábios escreverão livros, os matemáticos
organizarão estatísticas, os generais armarão exércitos, os juízes
erguerão cadafalsos, os estadistas estabelecerão fronteiras, os
pedagogos atiçarão patriotismos, os reis deflagrarão guerras tremendas e
os poetas cantarão os heróis da chacina --- para que jamais a guerra
cesse de ser uma permanente.

``Queres ver ao vivo, Gabriel, o que vai ser a chimpanzeização do mundo?
Corre essa cortina do futuro e espia por um momento a humanidade.''

Gabriel correu a cortina do futuro e espiou. E viu sobre a crosta da
terra uma certa poeira movediça. Mas, ansioso de detalhes, Gabriel
microscopou e distinguiu uma dolorosa caravana de chimpanzés pelados, em
atropelada marcha para o desconhecido.

Miserável rebanho! Uns grandes, outros pequenos; estes louros, aqueles
negríssimos --- nada que recordasse a perfeição somática dos outros
viventes, tão iguaizinhos dentro do tipo de cada espécie. Que feia
variedade! Ao lado do Apolo, o torto, o capenga, o cambaio, o corcovado,
o corcunda, o raquítico, o trôpego, o careteante, o zanaga, o zarolho, o
careca, o manco, o cego, o tonto, o surdo, o espingolado, o nanico\ldots{}
Caricaturas móveis, com os mais grotescos disparates nas feições, era
impossível apanhar"-lhes de pronto o tipo"-padrão. E Gabriel evocou
mentalmente a linda coisa que é um desfile de abelhas ou pinguins, no
qual não há um só indivíduo que destoe do padrão comum.

Da manada humana subia um rumor confuso. Gabriel desencerrou os ouvidos
e pôde distinguir sons para ele inéditos: tosse, espirros, escarradelas,
fungos, borborigmos, ronqueira asmática, gemidos nevrálgicos, ralhos,
palavrões de insulto, blasfêmias, gargalhadas, guinchos de inveja,
rilhar de dentes, bufos de cólera, gritos histéricos\ldots{}

Depois observou que à frente das multidões caminhavam seres de escol,
semideuses lantejoulantes, vestidos fantasiosamente, pingentados de
cristaizinhos embutidos em engastes metálicos, com penas de aves na
cabeça, cordões e fitas, crachás e miçangas\ldots{}

--- Quem são?

--- Os chefes, os magnatas, os reis: os condutores de povos.
Conduzem"-nos\ldots{} não sabem para onde.

E viu, entremeio à multidão, homens armados, tangendo o triste rebanho a
golpes de espada ou vergalho. E viu uns homens de toga negra que liam
papéis edavam sentenças, fazendo pendurar de forcas miseráveis
criaturas, e a outras cortar a cabeça, e a outras lançar em ergástulos
para o apodrecimento em vida. E viu homens a cavalo, carnavalescamente
vestidos, empenachados de plumas, que arregimentavam as massas,
armavam"-nas e atiravam"-nas umas contra as outras. E viu que depois de
tremenda carnificina um grupo abandonava o campo em desordem, e outro,
atolado em sangue e em carne gemebunda, cantava o triunfo num delírio
orgíaco, ao som de músicas marciais. E viu que os homens de penacho
organizadores das chacinas eram tidos em elevadíssima conta. Todos os
aplaudiam, delirantes, e os carregavam em charolas de apoteose. E viu
que a multidão caminhava sempre inquieta e em guarda, porque o irmão
roubava o irmão, e o filho matava o pai, e o amigo enganava o amigo, e
todos se maldiziam e se caluniavam, e se detestavam e jamais se
compreendiam\ldots{}

Horrorizado, Gabriel cerrou a cortina do futuro e disse ao Criador:

--- Se vai ser assim, cortemos pela raiz tanto mal vindouro. Um
chimpanzé a menos no Paraíso e estará evitado o desastre.

--- Não! --- respondeu o Criador. --- Tenho um rival: o Acaso. Ele criou
o homem, provocando a lesão desse macaco, e quero agora ver até a que
extremos se desenvolverá essa criatura aberrante e alheia aos meus
planos.

Gabriel piscou por uns momentos (catorze vezes ao certo), desnorteado
pela expressão ``quero ver'' jamais caída dos lábios do Senhor. Haveria
porventura algo fechado, ou obscuro, à presciência divina?

E Gabriel ousou interpelar Jeová.

--- Não sois, então, Senhor, a Presciência Absoluta?

Jeová franziu os sobrolhos terríveis e murmurou apenas:

--- Eu Sou, e se Sou, Sou também O que se não interpela.

Gabriel encolheu"-se como fulminado pelo raio e sumiu"-se da presença do
Eterno com pretexto de uma vista de olhos pelo Éden.

Linda tarde! O sol moribundo chapeava debruns de cobre nos gigantescos
samambaiuçus, a cuja sombra dormitavam megatérios de focinhos metidos
entre as patorras.

As \emph{arqueopterix} desajeitadonas chocavam na areia seus grandes
ovos. Um urso das cavernas catava as pulgas da companheira com a
minuciosa atenção dum entomologista apaixonado, e de longe vinham urros
de estegossauros perseguidos por mutucões venenosos.

Ao fundo dum vale de avencas viçosas como bambus, dois labirintodontes
amavam"-se em silencioso e pacato idílio, não longe de um leão fulvo que
comia a carne fumegante da gazela caçada.

Aves gorjeavam amores nos ramos; serpes monstruosas magnetizavam
monstruosas rãs; flores carnívoras abriam a goela das corolas para a
apanha de animaizinhos incautos.

Paz. Paz absoluta. Felicidade absoluta. A Vida comia a Vida e a Vida
amava para que não se extinguisse a Vida --- tudo rigorosamente de
acordo com a senha divina.

Só Adão, o macaco lesado, discrepava, piscando os olhinhos vivos, como a
ruminar certa ideia.

Gabriel parou perto dele e deixou"-se ficar a observá"-lo. Viu que Adão,
de olhos ferrados numa toca de onça, \emph{raciocinava}: ``Ela sai e eu
entro, e fecho a porta com uma pedra, e a casa fica sendo minha\ldots{}''.

Eva, a macaca ilesa, permanecia muda ao lado, embevecida no macho
pensante. Não o compreendia --- não o compreenderia nunca! ---, mas
admirava"-o, \emph{imitava"-o} e obedecia"-lhe passivamente.

Nisto, a onça deixou o antro e foi tocaiar uma veadinha.

--- Acompanhe"-me! --- disse Adão à companheira, e ambos precipitaram"-se
para a toca da onça, cuja entrada fecharam por dentro com uma grande
pedra roliça. E ficaram \emph{donos}.

Gabriel, que acompanhara toda aquela maromba, acendeu um cigarro de
papiro, baforou para o céu três fumaças e murmurou:

--- Ele já é inteligência. Ela não passa de imitação. É lógico: só ele
foi lesado no cérebro; mas vão ver que Eva, a instintiva, ainda acabará
fingindo"-se lesada\ldots{}

E o primeiro difamador da mulher foi jogar sua partida de gamão com o
Todo"-Poderoso.

\chapter{Tragédia dum capão de pintos\footnote[*]{Texto de 1923, publicado no livro \emph{O macaco que se fez homem}.}}

Nasceram na mesma semana um pinto, um peruzinho e um marreco. Até aqui,
nada. Todos os dias vêm ao mundo marrecos, perus e pintos sem que isso
ponha comichões na pena dos novelistas. O estranho do caso foi que
nasceram irmãos, contra todos os preceitos biológicos.

--- ??

Explica"-se. Tio Pio, preto cambaio que tomava conta do terreiro, tivera
a ideia de reunir sob certa galinha, em choco sobre apenas cinco ovos,
mais três de perua e dois de marreca salvos de ninhadas infelizes,
conseguindo assim dar vida àquela estranha irmandade de nova espécie.

Dos nove ovos só vingaram três, e lá estavam os produtos já crescidotes
sob a guarda solícita do Peva"-de"-raça, capão de pintos posto a pajeá"-los
para que dona galinha não perdesse tempo com tão pífia ninhada.

Triste sorte na fazenda a dos galos cotós de pernas! Tio Pio os punha de
parte para capões de pintos, transformando os belicosos ``clarins da
aurora'' em tristes eunucos, bichos metade galo, metade galinha,
senhores de crista, espora e cauda flamante não mais destinadas a
seduzir frangas, senão a divertir pintinhos.

Peva"-de"-raça tinha este nome pelas razões que o nome indica. Mas vá
lição para os leitores da cidade, gente que de galos e galinhas só
conhece os da torre das igrejas e as que aparecem ao jantar em molho
pardo. \emph{Peva}: perna curta; \emph{de raça}: raça estrangeira.

--- A mó que Plimu --- explicava Pio aos interpelantes.

Excelente sujeito o Peva! Tomara os órfãos no primeiro dia sem nenhuma
relutância e dera com eles criados à custa de infinitos de pachorra.

Muitos dissabores sofreu. O marrequinho, sobretudo, causou"-lhe sérios
aborrecimentos.

Havia na fazenda um tanque bordado de taboas esbeltas, rico de traíras e
sapinhos de cauda. Esse tanque era a mania do lindo pompom de arminho
amarelo. Quantas vezes não ficou o Peva à beira d'água seguindo de olhos
aflitos as evoluções do mimoso palmípede, que nela penetrava e nadava, e
mergulhava com louca afoiteza, inconcebível para o velho capão!

\textls[-10]{Já os outros não o afligiam tanto. Divertiam"-no até. O capão gostava de
ver o peruzinho em caça às moscas. Magricela e tonto, como sabia marcar
a presa, achegar"-se com extrema lentidão e de repente --- \emph{zás!}
--- uma bicada certeira!}

O pinto, esse era mestre em travessuras. Subia"-lhe às costas,
tenteando"-se nas asinhas, e trepava"-lhe pelo pescoço até alcançar a
crista, cujas carúnculas bicava.

Era muito cauteloso, o Peva. Se vinha chuva, punha"-se logo de agacho
para abrigo dos guris --- de dois apenas, que o terceiro, o marreco,
nenhum caso d'água fazia, antes pelava"-se por chuva, só recolhendo ao
sentir"-se entanguido.

E era muito metódico, o Peva. Mal a tarde fechava a carranca anunciativa
da noite, lá ia ele de rumo ao terreiro aninhar"-se rente ao muro, sempre
no mesmo lugar. Escarrapachava"-se ali ao jeito das galinhas e esperava
que os órfãos, depois dumas derradeiras voltas por perto, viessem
chegando e se metessem dentro da plumosa casa viva.

Entrava primeiro o peru, um friorento de marca; depois o pinto; o
marreco por último.

E o Peva cochilava, transfeito em esquisito animal de quatro cabeças: a
sua, grande, cristuda, e mais três cabecinhas curiosas, que abriam
seteiras na plumagem e espiavam o mistério do mundo a envolver"-se nas
sombras da noite.

Aquela singularidade deu nome e renome aos três bichinhos. Quantos
pintos, perus e marrecos houvesse na fazenda eram todos conhecidos por
pinto, peru e marreco, genericamente. Só eles se personalizavam. Eram o
Pinto Sura, o Peruzinho do Capão e o Reco"-Reco. Seres privilegiados,
libertos da disciplina comum do galinheiro, tornaram"-se logo as
criaturinhas mais populares daquele pequeno mundo. Viviam soltos sem lei
nem grei, como boêmios errantes encontradiços por toda parte --- nos
chiqueiros, nos pastos, ao pé das tulhas, à porta das cozinhas, onde
quer que aparecesse fartura de milho, siriris e quireras.

Havia na fazenda outros animais populares. Havia a Ruça, mulinha de
carroça, bastante velha e próxima da aposentadoria. Só trabalhava em
serviços leves de terreiro, puxando a ``carrocinha de dentro''.
Pertencera à tropa, transportara muito café para a cidade, sempre com
carga de oito arrobas, façanha de que, com saudades, se recordava agora.

Entre as vacas era a Princesa a mais popular. Vaca de estimação.
Enriquecera a fazenda de numerosos filhos, entre os quais o possante
Beethoven, agora pastor do rebanho. Dera ainda a Rosita, leiteira de
truz fiel à estirpe e certa nas doze garrafas diárias. E quantas outras
crias que já andavam por sua vez de bezerrinho novo, ou na canga, a
puxar carros! Vivia às soltas, livre de cercas, sempre no pasto dos
porcos, ocupando o tempo em mascar babosamente boas palhas de milho.

Quem mais? Sim, o Vinagre --- fiel guardião da ``casa"-grande'', veadeiro
de fama outrora, hoje um dorminhoco que o que fazia era cochilar ao sol,
de focinho entre as patas e olhos lacrimejantes. Todo ele era passado.
Durante as sonecas vinham agitá"-lo pesadelos, nos quais reviviam as
cenas violentas das caçadas de antanho. E o glorioso veterano acuava a
dormir.

\textls[-15]{Os homens nunca prestam grande atenção aos animais que os rodeiam.
Brutinhos, dizem, e desprezam"-nos. Mas a verdade é que a esses nossos
manos o que os inferioriza é não gozarem o dom da fala, pelo menos de
fala inteligível para nós, visto como pensam e superiormente raciocinam,
possuindo sobre os homens e as coisas ideias terrivelmente lógicas.}

\textls[-15]{Ali na fazenda eram todos concordes num ponto: a supremacia de Tio Pio
sobre os demais seres humanos. Era Tio Pio a atenção que nada esquece, a
justiça que dá e pune, o amor que compreende, o deus que cura, a ordem
que tudo simplifica.}

Para o trio do Peva era Tio Pio o Recolhe"-ovos, o Deita"-ninhadas, o
Matapiolho, o Varre"-galinheiro, o Pega"-frango, o Arruma"-ninho, o
Traz"-quirera, o Rebenta"-cupim, o Espanta"-cachorro --- modalidades várias
dum alto espírito de providência.

Para a Princesa era o Traz"-milho, o Tira"-leite, o Prende"-bezerro, o
Esvurma"-berne, o Fecha"-porteira, o Bota"-no"-pasto.

Para a mulinha era o Põe"-carroça, o Arruma"-arreios, o Escova"-pelo, o
Daração.

Para o Vinagre era o Lava"-cachorro, o Traz"-angu, o Atiça"-atiça, o
Pregapontapés.

Só ele, entre tantos homens da fazenda, revelava"-se, apesar de preto,
claro de intenções e compreensível; só ele não podia desaparecer sem
grave dano geral. Lembravam"-se de como todos padeceram certa ocasião em
que Tio Pio caiu de cama. Houve desordem grossa. Pintos morreram de
fome; Vinagre emagreceu; a Princesa viu"-se privada de palha; o Peva
dormiu fora do terreiro pela primeira vez. Ao cabo de dez dias, quando o
preto ressurgiu, recém"-sarado, foi como se repontasse o sol em seguida a
longo tempo de chuvas. Que alegria!

As demais criaturas humanas afiguravam"-se"-lhes misteriosas e sobretudo
ilógicas. Impossível ao Vinagre entender o patrão. Já de cara alegre, já
de cara amarrada, recebia"-o alternativamente com carinho ou pontapés. E
o velho cachorro filosofava: como é que um mesmo ato meu, sempre gesto
de afago e submissão, ora recebe prêmio, ora castigo? Não entendia\ldots{}

E muito menos o entendiam o Peva, a Princesa e a Ruça. Sua presença no
curral ou no pasto era signo certo de calamidade --- morte, prisão,
tortura. ``Mate aquele boi'', ``Pegue aquele frango'', ``Arreie aquele
cavalo'', ``Cape aquele porco''. Mate, pegue, arreie, cape, venda,
esfole --- não se lhe ouviam outras palavras. E toda gente corria
pressurosa a executar"-lhe as ordens, por mais tirânicas que fossem.

Igualmente incompreensíveis eram os filhotes do homem. Que criaturinhas
variáveis, irrequietas, cruéis! Sempre de vara na mão, perseguiam
abelhas e borboletas, esmagavam os sapos, atropelavam as galinhas. Ao
vê"-las, Vinagre disfarçadamente saía para longe e o Peva bandeava"-se com
seus órfãos para o outro lado dalgum vedo. Só a Princesa nenhum caso
deles fazia, certa do terror que lhes inspiravam os seus longos chifres.

Já a Dona, mulher do Senhor, não infundia medo senão às aves. Terrível
inimiga do galinheiro! Depredava os ovos e condenava à morte justamente
os mais belos frangos e as mais respeitáveis matronas de pena ---
``galinhas velhas'', como dizia a ingrata.

Para os outros animais a Dona significava apenas ignorância. Era a
``Perguntativa'' e a ``Muda"-cor''. Hoje de cor"-de"-rosa, amanhã de azul,
não usava cor fixa. E vivia interrogando:

--- Pio, que burro é esse?

--- Não é burro, sinhá, é a mulinha Ruça.

Perguntava sempre. Que \emph{caroços} eram aqueles na vaca? Que boi
estava \emph{rinchando} no pasto? Que \emph{trepadeira} andavam a tirar
das árvores?

Viera duma cidade grande, havia pouco tempo, cheia de gritinhos e medo
aos bichos. Ignorava tudo, fora pilhar ninhos. \emph{Papa"-ovo},
apelidou"-a o Peva, como já havia apelidado Tio Pio de \emph{É"-hora} e
aos demais camaradas da fazenda de \emph{Sim"-Senhores}, porque \emph{Sim
Senhor} era o estribilho com que habitualmente retrucavam a todas as
ordens do Dono.

Por uma tarde igual às outras, recolhia"-se Peva ao pouso do costume
seguido dos três órfãos já marmanjões. No céu, a caraça vermelha do sol
escondia"-se detrás do morro, e na terra os primeiros grilos ensaiavam as
asas cricrilantes. Rente à porteira a mulinha, solta no pasto minutos
antes, espojava"-se regalada.

--- Boa tarde! --- saudou"-a o Peva. --- Cansadinha, hein?

A mula interrompeu a cabriola e abanou as orelhas como quem diz: ``É
verdade''. Depois falou:

--- Acho prudente que tome cuidado com seus filhos. A Perguntativa anda
interessada por eles --- e isso é mau sinal. Vi"-a em conversa com É"-hora
e pilhei este pedacinho: ``O marreco do capão está no ponto''. Não sei o
que quer dizer, mas boa coisa não será.

O Peva enrugou a testa, apreensivo. Jamais a Perguntativa se referia a
alguma ave sem que sobreviesse desgraça. ``Está no ponto'' --- que
quereria dizer aquilo?

A mulinha ignorava"-o. Sabia de algumas palavras triviais, conhecia o
\emph{pegue}, o \emph{prenda}, o \emph{mate} --- mas o \emph{está no
ponto} era"-lhe coisa nova.

--- Quem há de saber disto é o Vinagre. Mora na casa"-grande e entende a
língua dos homens melhor do que nenhum de nós. Consulte"-o, e não deixe
também de consultar a Princesa, cuja experiência da vida é grande.

Peva se foi à Princesa, que encontrou mascando as palhas do costume.

--- \emph{Está no ponto} --- poderá dizer"-me, senhora Princesa, que
coisa significa na língua dos homens?

A vaca interrompeu a mascação e disse:

--- Já ouvi essa palavra aplicada ao meu filho segundo, o Barroso. Tinha
ele dois anos e meio. O Dono passava em companhia de um Sim"-Senhor.
Avistou de longe o meu Barroso no pasto e ordenou: ``Aquele boizinho
está no ponto. Carro com ele!''. No dia seguinte laçaram"-no, meteram"-no
na canga e o pobre do meu garrote muito que padeceu a puxar um carro
pesadíssimo. Deste incidente concluo que \emph{estar no ponto} quer
dizer \emph{carro}.

Peva, um tanto curto de ideias, tremeu ante aquela revelação. Horror,
meterem no carro ao seu querido marrequinho! Em seguida duvidou. Andar
no carro era coisa que só vira fazer aos bois. Não podia ser. A vaca
errara evidentemente.

``Resta"-me consultar o Vinagre'', refletiu, e todo pepé, com ruguinhas
de apreensão na crista, foi ter com o velho cachorro.

Vinagre não resolveu o enigma, embora respondesse como o mais sábio dos
oráculos.

--- Pode ser muita coisa. A linguagem dos homens varia, ora quer dizer
isto, ora aquilo. Mas que não é coisa boa, isso eu asseguro.

Nesse dia o capão, seguido dos órfãos, recolheu"-se ao pouso habitual sem
a despreocupação de outrora. Custou"-lhe conciliar o sono. Não lhe saíam
da cabeça as palavras misteriosas e de sentido inapreensível. Por fim
dormiu e sonhou. Sonhou que ao lado do Barroso jungiam ao carro o pobre
marrequinho.

O sonho virou pesadelo e Peva sofreu horrores ante o quadro do filho
adotivo a debater"-se sob a monstruosa canga\ldots{}

No dia seguinte, no momento da ração de milho, Tio Pio inesperadamente
agarrou o marrequinho pelas pernas e lá se foi com ele para a Cozinha.

Aflitíssimo, tomado de imenso desespero, Peva inda alimentou esperanças
de vê"-lo. Mas a noite chegou e com ela a primeira desilusão de sua vida.
Nada do marreco. Pela manhã, nada. Meio"-dia, nada.

À hora do jantar encontrou Vinagre roendo uns ossos no terreiro.

--- Que é isso, amigo?

--- Ossos de marreco.

--- De marreco! --- exclamou Peva, surpreso.

--- Sim. Que admiras? Que os marrecos tenham ossos? Têm"-nos, e
excelentes\ldots{}

Peva estarreceu. Compreendia afinal o tremendo sentido das palavras
misteriosas. \emph{Está no ponto} significava condenação à morte.
Horror!\ldots{}

Guardou consigo, entretanto, aquela mágoa. Nada disse ao peruzinho nem
ao frango, prevendo para os dois sorte idêntica.

--- Bem triste a vida sob o domínio cruel do homem! Nada de bom vem
deles\ldots{} --- filosofou.

Nessa mesma tarde Peva cruzou"-se com a Princesa e disse"-lhe:

--- Erraste, Princesa. \emph{Está no ponto} quer dizer \emph{morte}.

A vaca parou a mastigação da palha e sorriu da ingenuidade do Peva. Ela
tinha tanta certeza de que queria dizer \emph{carro}\ldots{}

A vida na fazenda rolava na mesmice de sempre. Tudo continuava. A Ruça,
a puxar a carrocinha; a Princesa, a mascar palhas; o Vinagre, a acuar em
sonhos. Só na tribo do Peva a alegria não era a mesma. Saudades do
marreco. Várias vezes o frango indagou do destino de Reco"-Reco, forçando
o capão a mentir. ``Anda de viagem, uma longa viagem\ldots{} Um dia volta.''

Mas com que tristeza punha os olhos no tanque ou nas poças de enxurro
que se formavam em dias de aguaceiro, pensando lá consigo: ``Nunca
mais!''\ldots{}

O tempo corre, as estações se sucedem. A primavera anunciou"-se nos mil
botões que se arredondavam nas laranjeiras. Os órfãos do capão já eram
mais companheiros de ciscagem do que filhotes pipilantes. Já dispensavam
a sua solícita assistência. O peruzinho, grandalhudo e bem empenado,
fez"-se independente. O frango punha crista, com as esporas abotoadinhas.
Mudara de gênio, e se via alguma franga ia arrastar"-lhe a asa até que
algum galo de verdade o escorraçasse.

Certa manhã a Perguntativa veio assistir à amilhagem das aves. Fez
várias perguntas e deu várias ordens ao Pio, concluindo, de dedo
apontado para o frango:

--- Está pedindo panela, aquele!

--- Qual, Sinhá? O Sura?

--- Sura quer dizer sem rabo? É. É ele mesmo.

Peva, que tudo ouvira, engasgou"-se com o grão de milho que tinha no
bico, perdeu a fome e incontinenti saiu do bando. Embora não
compreendesse o sentido daquelas palavras, previu que ``boa coisa não
seria'', como filosofava o Vinagre.

E acertou. O frango, no dia imediato, desapareceu misteriosamente. Peva
procurou"-o por todos os cantos e, desconfiado, foi rondar os fundos da
Cozinha na esperança de ouvi"-lo piar lá dentro. Não ouviu pio nenhum ---
mas encontrou penas suspeitas no monte de lixo\ldots{}

Adquirida a certeza do novo desastre, fez"-se ainda mais tristonha a vida
do pobre capão. A Cozinha! Era nas goelas daquele horrendo Moloch que
sucessivamente iam desaparecendo os seus queridos órfãos. Engolira o
marreco, engolira o frango\ldots{} Engoliria também o peruzinho, por que não?

Velho e desalentado, com o coração sempre saudoso dos travessos
garotinhos que criara, tornou"-se macambúzio. Inda passeava com o peru,
apesar da cada vez maior independência deste. Chegou a notar que era
ele, Peva, quem o acompanhava agora. Notou"-o, mas procurou iludir"-se e
simulava amadrinhá"-lo, como outrora\ldots{}

Pela força do hábito inda dormiam juntos, no antigo pouso ao pé do muro.
Mas logo o peru, que é amigo de poleiro, elegeu um, cômodo, em certa
escada velha, e o capão teve de acompanhá"-lo na mudança. E ali passaram
a dormir juntinhos e encorujados no mesmo degrau.

Assim viveram até a chegada do Ano"-Bom.

Na véspera a Perguntativa apareceu no momento do dar milho e disse ao
Pio:

--- Olhe, amanhã temos o peru. Não esqueça de comprar pinga.

Desta feita Peva não vacilou quanto ao sentido da expressão. \emph{Está
no ponto} --- \emph{panela} --- \emph{temos o peru} --- deviam ser
frases equivalentes. Estava pois condenado a entrar para a Cozinha o seu
derradeiro filho\ldots{}

Cheio de resignação e com a alma em transes, Peva passou o dia num
canto, jururu, remoendo as doces recordações de outrora. Ao cair da
noite recolheu"-se. Empoleirou"-se na velha escada e achou muito natural
que o peru não comparecesse.

Dormiu tarde e teve o sono agitado de contínuas estremeções de angústia.

No dia seguinte notou movimento fora do comum na casa"-grande. Vinha
gente de longe, mulheres de trole, homens a cavalo. Vinagre, esquecido
da soneca do costume, entrava e saía, abanando a cauda com vivacidade de
cachorro novo.

Num destes vaivéns Peva o deteve.

--- Que há na casa"-grande? Tanta gente\ldots{}

--- Há peru --- respondeu o cão. --- Quando há peru, os homens se
assanham, vestem roupas novas, brincam e dançam. Tenho notado que a
presença do peru à mesa provoca nos homens uma espécie de delírio, como
entre as galinhas a queda de içás.

Esta observação do cachorro, embora muito lisonjeira para a raça dos
perus, não consolou nada ao nosso Peva, que se sentia ganho menos de
tristeza que de funda indiferença pela vida. O sucessivo sacrifício dos
filhotes calejara"-lhe por partes o coração. No dia do marreco a dor que
sentiu foi verdadeira dor de pai; em seguida, pela morte do frango, a
sua dor foi dor de pai adotivo; agora, ao perder o peru, a dor era calma
e resignada. Dor de filósofo. Compreendia, afinal, que a vida foi e é
assim, e não melhora\ldots{}

Os capões inspiram desprezo aos galos e talvez piedade irônica às
senhoras galinhas. Por isso Peva, em sua triste solidão, deambulava pelo
terreiro como criatura sem lugar na vida. As lindas frangas, as viçosas
poedeiras, e até as velhas galinhas aposentadas, tinham pela sua honesta
companhia um profundo desdém. E como nem os frangotes o procuravam, o
isolamento do triste eunuco era completo.

Esse errar à toa fê"-lo notado de Tio Pio, que se lembrou de pô"-lo a
criar nova ninhada.

--- Anda vadiando aqui, este diabo\ldots{} Espera que te arrumo.

Agarrou"-o, levou"-o ao galinheiro, esfregou"-lhe urtiga no abdome e
deitou"-o sobre uma ninhada de dez pintos nascidos na véspera.

Não ofereceu Peva a menor resistência. Deixou fazer. Agachou"-se como
dantes e cobriu lindamente os gentis recém"-nascidos.

Altas horas, porém, ergueu"-se e tomou rumo do poleiro, abandonando aos
frios da noite a roda de vidinhas pipilantes. Não mais queria exercer a
profissão de mãe. Para quê?

--- Se têm de morrer na Cozinha, morram agora enquanto ainda não lhes
tenho amor.

Os pintos amanheceram mortos, entanguidos de frio.

Quando Tio Pio tomou conhecimento do desastre, ficou furioso.

--- Cachorro! Você fez mas paga!

Houve um corre"-corre. A galinhada assustadiça debandou; os marrecos
meteram"-se no tanque.

Cotó de pernas, frouxo de asas, Peva pouco resistiu à perseguição do
negro.

Rendeu"-se e, seguro pelas patas, de cabeça para baixo: com as ideias
perturbadas pela congestão do cérebro, por sua vez transpôs a soleira da
Cozinha, insaciável sorvedouro de vidas, odioso túmulo de Reco"-Reco, do
Sura, do Peru e agora do venerável tutor da estranha irmandade\ldots{}

Quem na manhã do dia seguinte passasse pelo fundo da horta veria no
monte de lixo um punhado de penas escaldadas, escorridas, sem cor, sujas
de cinza. E veria duas pernas rugosas de longas esporas recurvas. E
veria ainda uma dolorosa cabeça de crista violácea, com os olhos
semiabertos, em cujas pupilas de vidro várias formiguinhas se miravam.

Horríveis, aqueles despojos?

Um urubu pousado ali perto não pensava assim\ldots{}

\chapter{Duas cavalgaduras\footnote[*]{Texto de 1923, publicado no livro \emph{O macaco que se fez homem}.}}

Um grande amigo dos livros, o estudante Batista de Ribeiro
Couto.\footnote{\emph{O crime do estudante Batista} (1922), livro de
  contos de Ribeiro Couto. Nota da edição de 1946.}

Na sua dolorosa miséria de rapaz pobre, solto sem padrinhos na voragem
carioca, desses bons amigos se socorria para desafogo da alma crestada
ao vento das decepções. Falhava"-lhe o sonhado emprego? Abria \emph{Dom
Casmurro} e logo a malícia de Capitu empolgava, levando"-o para casos bem
distantes do seu dorido caso pessoal. Traía"-o algum amigo? O moço
embarcava para Florença no \emph{Lys Rouge}, hospedava"-se com Miss Bell
e, de visita às igrejas com Dechartre, ei"-lo embriagado no ardente amor
da condessa.

O estômago, porém, é Sancho. Não digere contemplações. Exige pão. E a
fome, um dia, apresentou ao estudante o seu inexorável \emph{ultimatum}:
Mata"-me ou mato"-te.

Um só recurso lhe restava: reduzir a pão duro os seus amados livros.

Fê"-lo, mas com que mágoa! Como vacilou na escolha da primeira vítima! E
como lhe doeu o sórdido negocismo do belchior, miserável depreciador da
``mercadoria'' com o fito de obtê"-la pelo mínimo!

Era este belchior certo judeu mulato com um sebo à rua do Catete. Mulato
de barbicha irônica, própria para coçadelas nos momentos de engatilhar o
preço.

Tinha um jeito irritante de tomar os livros e ler o título por baixo dos
óculos, como se os cheirasse. Tipo desagradável de múmia ressurreta, em
perfeita harmonia com a sordidez da casa.

Que vitrina! Já ali se lhe anunciava a alma. Livros encardidos,
brochuras de cantos surrados, canetas de vintém, lápis ``quebra a
ponta'', tinteiros de refugo --- tudo desbotado pelo sol e tamisado pela
horrível poeira negra da rua. Dentro, um cheiro de velhice, misto de
mofo e ranço --- bafo proveniente metade da múmia, metade das estantes
prenhes de brochuras infectas.

Pois foi nas garras de tal aranha barbada que o pobre contemplativo
caiu, e um a um lhe sorvia ela todos os volumes da amada biblioteca,
sempre a ratinhar, a rosnar, a espichar níqueis para o que valia notas.

Uma vez recebeu o moço más notícias de casa e instante pedido de uma
linda irmãzinha que deixara em Catalão. Era forçoso servi"-la, inda que
houvesse de vender a alma ao diabo.

O jeito era um só: negociar em bloco os livros restantes. Que vá, que
vá! Uma grande dor única é de preferir"-se a mil dorezinhas parceladas.
Que vá tudo!

Contou"-os. Trezentos. Pelo preço médio que o judeu lhe pagava por
unidade, obteria com aquele sacrifício os duzentos mil"-réis necessários
e mais uns bicos. Que vá.

Batista retesou"-se de alma, amordaçou o coração, meteu na carroça os
velhos amigos e, como vai para a guilhotina o condenado, foi com eles
para a rua do Catete.

O judeu examinou os volumes um por um, cheirou"-os, sopesou"-os e depois
de longas manobras, engasgos, meias palavras e coçadelas da barbicha,
abriu a oferta.

--- Dou"-lhe quarenta mil"-réis, moço, por ser para o senhor. E lamba as
unhas, hein?

Tomado de súbita onda de cólera homicida, o estudante não lambeu as
unhas: lambeu"-lhe a vida. Estrangulou"-o\ldots{}

Havia eu lido esse formoso conto e ficara com os tipos gravados em
relevo na memória, tanta nitidez dera à pintura o autor. O judeu mulato,
sobretudo, passara a viver dentro de mim em lugar de honra na ``sala de
Harpagão''.

Somos todos nós uns museus de tipos apanhados na rua ou colhidos na
literatura. Museus classificados, com salas disto e daquilo. A minha
sala dos usurários encerrava bom número de Shylockzinhos modernos,
fisgados à porta de cartórios ou diretamente nos antros onde costumam
empoleirar"-se como harpias pacientes à espera dos náufragos da vida.
Ombro a ombro conviviam eles com os patriarcas do clã --- mestre
Harpagão, tio Grandet e o João Antunes, de Camilo Castelo Branco.

Lida a novela de Couto, entrou para a sala mais um --- o judeu mulato do
Catete, tipo de tal vida que uma suspeita breve me tomou: ``Este diabo
existe. Não pode ser ficção. Há nele traços que se não inventam. E se
existe, hei de vê"-lo''.

E pus"-me a procurá"-lo em certo dia de folga.

Fui feliz. Logo adiante do palácio do Catete certa vitrina atraiu"-me a
atenção. Acerquei"-me dela com cara de Colombo. Aqueles livros
desbotados, aquelas canetas\ldots{} Tudo exato!

Mas\ldots{} e aquele coelhinho?\ldots{}

Sim, havia a mais, na sórdida vitrina, um coelhinho de lã do tamanho de
um punho fechado. Encardido, os olhos de louça já bambos, as longas
orelhas roídas, visivelmente brinquedo já muito brincado.

Aquele coelhinho!

Uma criança existe de quem o usurário comprou o coelhinho\ldots{}

Meu Deus! Poderá haver em corpo humano almas assim?

Shakespeare, Balzac: que fraca imaginação a vossa! Criastes Shylock,
Grandet, mas a potência do vosso gênio não previu este caso extremo. O
judeu mulato reabilita os vossos heróis e atinge a suprema expressão do
sórdido.

Furtou o coelhinho à criança\ldots{}

Furtou"-o com a gazua dum níquel\ldots{}

Privou a pobrezinha do seu único brinquedo, do seu único amigo,
talvez\ldots{}

\emph{Abra"-se um parêntesis.}

Aqui intervém a imaginação.

Bastou que meus olhos vissem na sórdida vitrina o coelhinho de lã para
que a irrequieta rainha Mab me viesse cabriolar na cachola.

E todo um drama infantil se me antolhou, nitidamente.

Era um menino de poucos anos, filho de pais miseráveis.

O homem bebia e a mãe definhava nas unhas ``da pertinaz moléstia''.
Minto: da tísica. ``Pertinaz moléstia'' é a tísica dos ricos\ldots{}

O clássico operário bêbado, em suma, e a clássica mãe tuberculosa. É
sempre assim nos romances e é sempre assim na vida, essa impiedosa
plagiária dos romances.

Reina a miséria na cafua úmida em que vivem, ele a delirar o seu eterno
delírio alcoólico, ela a tossir os pulmões cavernosos --- e a triste
criança, sempre de olhos assustados, a criar"-se um mundinho de sonhos
para refúgio da almazinha que teima em ser alma.

Só tem um amigo essa criança: o coelhinho de lã que a mãe lhe deu em
certo dia de doença grave.

Excelente quinino! A febre cedeu incontinenti e dois dias depois o
enfermo se punha de pé.

Desde aí ficou sendo o coelhinho o amigo único da criança triste, seu
confidente de todas as horas, seu irmãozinho mais novo.

Conversavam o dia inteiro, brincavam, contavam"-se mutuamente lindas
histórias; e à noite, muito abraçados, dormiam o sono dos anjos e dos
coelhos.

Aquele coelhinho de lã\ldots{}

É preciso ser Dickens para compreender o papel dos brinquedos únicos na
vida das crianças miseráveis.

O comum dos homens não vê nisso coisa nenhuma.

Triste coisa, o comum dos homens\ldots{}

Um dia, o pai desapareceu.

Inutilmente a tísica o esperou até altas horas, e o esperou no dia
seguinte, e o esperou a semana inteira.

Desapareceu, e está dito tudo.

Na vida os miseráveis desaparecem, tal qual nos romances.

Vida, romance; romance, vida: será tudo um?

A tísica piorou, e certa manhã não pôde erguer"-se da cama.

E a fome veio.

E foi mister vender, hoje isto, amanhã aquilo, todos os trapos e cacos
da mansarda em crise.

A \emph{mansarda}! Que lindo efeito faz em romance esta palavra lúgubre!
A \emph{mansarda}!\ldots{}

Vendeu"-se tudo.

Luizinho era o leva e traz.

Levava o trapo, o caco, e trazia os níqueis do pão. E assim até que as
reservas se esgotaram e a mansarda ficou nua como Jó.

--- E agora?

A tísica lançou os olhos cansados pelas paredes nuas, pelos cantos nus.

Nada. Só viu o coelhinho. Mas era um crime sacrificar o coelhinho de
lã\ldots{}

Resistiu ainda algum tempo.

Por fim, disse:

--- Vai, meu filho, vai vender o coelhinho de lã\ldots{}

A criança relutou, mas cedeu ao cabo de muitas lágrimas. A fome
impunha"-lhe aquele sacrifício: trocar o seu tesouro por um pão.

O que chorou nessa manhã!

Como apertava contra o peito o amiguinho, sem ânimo de notificá"-lo da
tragédia iminente!

Resolveu mentir.

--- Sabe? --- disse ao coelho. --- Vou pôr você numa casa que tem
vitrina para a rua. Fica lá sentadinho a ver quem passa, os bondes, os
automóveis tão bonitos! E eu vou todos os dias espiar você através do
vidro. Quer?

O coelhinho não compreendeu aquilo e desconfiou.

--- Mas por quê? Estou tão bem aqui\ldots{}

Não era fácil iludi"-lo; a fome, porém, é capciosa e Luizinho continuou a
mentir:

--- É cá uma coisa que sei. Uma pândega! Por enquanto é segredo. Fica
você lá quietinho uns tempos, depois volta para cá de novo e eu conto a
história.

O coelhinho de lã piscou para o menino, cavorteiramente. Gostava desses
mistérios\ldots{}

Luizinho levou"-o ao belchior. Mostrou"-o ao judeu; ofereceu"-lho. O aranho
tomou o coelhinho entre os dedos rapinantes, examinou"-o, apalpou"-o,
cheirou"-o e abrindo a gaveta suja tirou de dentro o menor níquel.

--- Toma!

Luizinho ressentiu"-se. Já conhecia o valor do dinheiro; achou aquilo
``pouco demais''. Vendo, porém, pela cara do judeu que era inútil
insistir, pegou do níquel, beijou o coelhinho e disparou a correr.

No dia seguinte reapareceu no Catete. Parou diante da vitrina e longo
tempo esteve a namorar o amigo, trocando com ele sinais de inteligência.
O coelhinho piscava"-lhe com uma vontade doida de rir e ele piscava para
o coelhinho com uma vontade doida de chorar. E assim todos os dias, a
semana inteira.

--- A semana inteira, senhor novelista? Não estou compreendendo nada.
Vosmecê disse que o último recurso dos famintos fora o coelhinho de lã,
que trocaram por um pão. Ora, comido o pão, e nada mais havendo para
vender, manda a lógica que mãe e filho tenham morrido de fome.

--- Obrigado, senhor lógico! Vejo que leu Stuart Mill e Bain, mas que
nunca leu Dickens, nem Escrich, nem Montepin. Devia ser como dizes, se a
vida fosse feita pelos lógicos. Mas Deus não era lógico, era apenas
romancista. Não morreram, não, nem mãe nem filho. E não morreram porque
justamente naquele dia o pai bêbado reapareceu\ldots{}

--- Oh!\ldots{}

--- Sim, meu Bain, reapareceu. E sabe que mais? Reapareceu regenerado\ldots{}

--- Oh! Oh!\ldots{}

--- \ldots{} e com dinheiro no bolso. Quer mais? E rico! Quer mais? E
milionário, com a sorte grande da Espanha no papo. Quer mais? Quer mais?
Nos romances á o epílogo e não sabe que o epílogo é o esparadrapo que
une os bordos da ferida?, o dedo de Deus que recompensa?, o suspiro de
consolo que nos reconcilia com a vida?

--- Mas isto, afinal de contas, é vida ou romance?

--- Grande tolo\ldots{} É a vida com a lição da arte. A arte corrige a vida,
dizendo"-lhe: se não és assim, megera, devias sê"-lo; se não procedeste
assim, harpia, devias ter procedido; se não fizeste o bêbado reaparecer
no momento oportuno, carcaça, devias tê"-lo feito. A arte ensina à vida o
seu dever.

Imagina tu, amigo lógico, que quando Deus criou o mundo\ldots{}''

\emph{Feche"-se o parêntesis.}

Mas acordei. A rainha Mab fugiu"-me do cérebro a galope em sua
carruagenzinha \emph{made by the joiner squirrel}, e entrei no belchior.

Lá estava no balcão o judeu mulato com sua barbicha de bode, os óculos
de latão, o gorro sebento.

Não morrera, o aranho; apesar de estrangulado na novela de Ribeiro
Couto, passava muito bem de saúde, o infame.

Era ele mesmo!

Naquele momento cheirava o lombo de um livro que um novo estudante
Batista lhe oferecera.

Enquanto negociavam, pus"-me à espreita disfarçadamente.

Exatinho! Couto fotografara"-o com objetiva Zeiss. Até a voz\ldots{}

--- \emph{Hum! Hum!} --- fungou ele depois de lido o título. --- Oscar
Wilde\ldots{} Isto não se vende, já passou da moda. Tenho carradas de
\emph{Dorian Gray}\ldots{} A pior coisa que ele escreveu\ldots{}

--- Mas quanto oferece? --- indagou o estudante, aborrecido de tantas
micagens.

--- Por ser freguês, pago sete tostões. E lamba as unhas, que hoje me
pegou de veia!

O meu estudante Batista não fez como o de Ribeiro Couto. Não lhe lambeu
a vida. Lambeu"-lhe os sete níqueis oferecidos e saiu a pegar o bonde,
displicentemente.

--- E o senhor, que deseja? --- disse"-me então o pirata, depois de
encafuar o livro na estante.

Eu não desejava coisa nenhuma, além de vê"-lo, apalpá"-lo, cheirá"-lo,
talvez estrangulá"-lo de verdade. Não obstante, fiz"-me de tolo.

--- Ando à procura de um livro. Um livro de Wilde. Tem aí qualquer coisa
deste escritor?

A fisionomia do estrangulado iluminou"-se.

--- Tenho a melhor coisa que Wilde escreveu, \emph{O retrato de Dorian
Gray}, conhece? --- disse, puxando fora da estante o volume adquirido
momentos antes.

--- Coisa papa"-fina!

Tomei o livro, folheei"-o. Edição francesa vulgar. Valeria, novo, quatro
mil"-réis.

--- Quanto pede?

--- Seis mil"-réis, por ser para o amiguinho.

Sorri"-me por dentro e por fora. Larguei o volume e acendi o cigarro.

--- Não me interessa. É caro.

--- Caro? Um livro destes, nesta encadernação, deste editor, deste
autor? Nem me diga isso! E o senhor deve saber que \emph{Dorian Gray} é
a obra"-prima de Oscar Wilde.

Meus dedos se crisparam. Que prazer estrangular aquela harpia!
Contive"-me, porém.

--- E aquele coelhinho? --- perguntei"-lhe. --- Quanto?

--- Que coelhinho? --- exclamou o aranho, mudando de cara.

--- Um que está na vitrina.

--- Ah, sim\ldots{} Aquele coelhinho não vendo.

--- Por que o expõe, então?

--- Expu"-lo ao sol. Mora aqui na minha mesa, mas como a casa é úmida
ponho"-o às vezes lá para evitar o bolor.

Diabo! O homem principiava a desnortear"-me. Tinha em casa um objeto que
não vendia. Era lá possível que um judeu daqueles não vendesse até a
alma?

Insisti:

--- Dou"-lhe cinco mil"-réis pelo coelhinho.

--- Já lhe disse que não é de venda. Cinco mil"-réis! Nem cinco contos,
sabe?

Revoltei"-me. Veio"-me à imaginação toda a tragédia do Luizinho e tive
ímpetos de insultá"-lo.

Contive"-me e disse apenas:

--- No entanto, furtou"-o a uma pobre criança miserável\ldots{}

O meu Shylock abriu a mais expressiva cara de espanto que já topei na
vida. Depois encarou"-me a fito e seus olhos lacrimejaram. Sentou"-se,
como aniquilado de súbita dor e explicou"-me, em voz entrecortada:

--- Não sou casado, não tenho filhos, não tenho ninguém no mundo. Mas
tive uma criança. Enjeitaram"-na aqui à minha porta e recolhi"-a. Criei"-a.
Durante sete anos constituiu a minha única alegria. O Antoninho\ldots{} Um
dia veio a gripe e levou"-o para o céu. Seu último brinquedo foi esse
coelhinho de lã. Conservo"-o aqui na minha mesa como joia preciosa, pois
me fala do Antoninho melhor que um livro aberto. Como quer que o venda?
Não há no mundo o que para mim valha esse coelhinho\ldots{}

Foi à vitrina e recolheu o brinquedo. Pô"-lo sobre a mesa ao lado do
tinteiro.

E depois de uma pausa exclamou, olhando"-o com um sorriso que me pareceu
divino:

--- Tinha um nome. O Antoninho só dizia ``o Labi''\ldots{}

--- ?

--- Sim, Rabi\ldots{} Quer dizer rabicó, sem cauda. O Antoninho trocava o
\emph{r} pelo \emph{l}.

Saí da casa do judeu completamente desorientado. Fui ao telégrafo e
expedi ao autor de \emph{O crime do estudante Batista} o seguinte
despacho: ``Couto, somos duas cavalgaduras!''.

\chapter{Fatia de vida\footnote[*]{Texto de 1923, publicado no livro \emph{O macaco que se fez homem}.}}


Não era homem querido, o doutor Bonifácio Torres. Não era querido pela
ponderosa razão de pensar com sua própria cabeça. Para ser querido é
força pensar como toda gente.

``Toda gente!''

Moloch social cujos mandamentos havemos de seguir de cabecinha baixa,
sob pena dos mais engenhosos castigos. Um deles: incidir na pecha de
esquisitice.

``É um esquisitão.''

Inútil dizer mais. O homem marcado vê"-se logo posto de través e à
margem, como o leproso. Torna"-se um indesejável. É um suspeito. Haja
meio e eliminam"-no do grêmio como a um corpo estranho, de malsão
convívio.

Assombramo"-nos ao recordar os crimes de grupo que enchem a história ---
Santo Ofício, guerras, matanças religiosas. Transportados à época vemos
que o progredir humano não passa da consolidação das vitórias do
``esquisitão'' sobre ``Toda gente''.

``Toda gente'' não tolerava dúvidas sobre a fixidez da Terra. Vem um
esquisitão e diz: A Terra move"-se em redor do Sol. ``Toda gente'', por
intermédio de seus representantes legais, agarra o velho pelo gasnete e
força"-o a retratar"-se.

--- Renega a heresia, infame, ou asso"-te já na fogueira!

Galileu baixou a cabeça encanecida e abjurou. E a Terra, que começara a
girar em torno ao Sol, teve que mudar de política e imobilizar"-se por
muito tempo ainda. Hoje roda livremente. O monstro deu"-lhe essa
liberdade\ldots{}

Como se vê, apesar da guerra que ``Toda gente'' move aos esquisitões as
ideias destes influenciam e aos poucos transformam a mentalidade do
Moloch. No começo o monstro encarcera, esquarteja, empala, sufoca.
Depois volta atrás, medita e murmura: ``Ele tinha razão!'', e adere com
a maior inocência.

``Toda gente'' tem hoje a caridade como dogma infalível, e por esse
motivo encarou com assombro o doutor Bonifácio quando o esquisitão
sorriu a uma frase nédia e lisa do cônego Eusébio. O cônego Eusébio,
conspícuo representante legal do Moloch, dissera no tom solene dos que
monopolizam a verdade sobre o orbe:

--- Não há virtude mais sublime. Só ela tem forças para resolver a
questão social. Aquele movimento belíssimo durante a epidemia da gripe
em São Paulo --- que réplica de escachar o espírito que nega! Todos à
urna, governos, matronas, meninas, associações, todos empenhados em
lenir o sofrimento dos pobres, como que a derramar Deus nos corações!\ldots{}

O doutor Bonifácio sorrira e o padre olhara"-o de revés, com saudades,
quem sabe, do bem"-aventurado tempo em que sorrisos assim recebiam a
réplica do fogo pio.

--- Sorri"-se o herege? --- interpelou o padre. --- Nega até a caridade?

--- Não nego --- respondeu mansamente o filósofo ---, porque não nego
nem afirmo coisa nenhuma. Negam e afirmam os atores, os que se agitam no
palco da vida. Eu tenho meu lugar na plateia e, como não represento,
observo. E como observo, sorrio --- sorrio para não chorar\ldots{}

--- Seja mais claro.

--- Serei. Quando o reverendo se abriu em louvores à caridade, não
desfiz nessa cristianíssima virtude. Apenas me lembrei de certo drama a
que assisti --- e, repito, sorri para não chorar\ldots{}

Depois de breve pausa de interrogativa expectação o doutor Bonifácio
principiou.

--- Isaura, a minha lavadeira\ldots{}

As anedotas têm força de ímã. Vários curiosos aproximaram"-se e ficaram a
ouvir.

--- Minha lavadeira, como todas as lavadeiras, era uma pobre mulher de
incomparável heroísmo, desse que os épicos não cantam, o Estado não
recompensa e ninguém sequer observa. Para mim, entretanto, é a forma
nobre por excelência do heroísmo --- a luta silenciosa contra a miséria.

--- Que esquisitice!

--- Porque é heroísmo ininterrupto, sem tréguas --- continuou o doutor
Bonifácio ---, sem momento de repouso e, além disso, sem nenhuma
esperança de qualquer espécie de paga.

--- Vamos ao caso\ldots{}

--- Viúva com quatro filhos, a heroica Isaura matava"-se no trabalho
incessante. Aquelas mãos vermelhas e curtidas\ldots{} Aqueles braços
requeimados\ldots{} Que máquinas! Era do movimento deles que vinha o sustento
da casa. Parassem, repousassem --- e a Fome, esquálida megera que ronda
os bairros pobres, meter"-se"-ia portas adentro\ldots{}

--- Romantismo\ldots{} ``Esquálida megera''\ldots{}

--- No primeiro sábado da Grande Gripe, Isaura, minha pontualíssima
lavadeira, não me apareceu como de costume com a sua bandeja de roupa
lavada. Em lugar dela veio uma vizinha.

``--- A Isaura? --- perguntei"-lhe.

``--- Anda às voltas com os filhos. Deu lá a `espanhola' e a pobre está
que está numa roda"-viva.

``--- Hei de ir vê"-la, coitada\ldots{}

``--- É caridade, senhor. A pobre é bem capaz de endoidecer\ldots{}

``Não fui. Impediu"-mo a própria gripe, cujos primeiros sintomas nesse
mesmo dia comecei a sentir. Passei de molho três semanas e quando me
levantei, e me preparava para ir ver Isaura, eis que ela me reaparece em
pessoa.

``Em que estado, porém! Envelhecera vinte anos, tinha os cabelos
brancos, os olhos no fundo, o ar de uma coisa vencida pelo destino. E
tossia.

``--- Sente"-se e conte"-me tudo.''

Sentou"-se e, sem derramar uma só lágrima, pois já as chorara todas,
narrou"-me a sua tragédia.

``Tinha em casa uma filha de dezoito anos, que trabalhava na costura;
outra de dezesseis, que a ajudava na lavagem; um filho de quinze,
entregador de roupa, e mais uma netinha de seis anos, órfã.

``A gripe apanhou"-os a todos e a ela também. Mas a pobre criatura não
soube disso, não o notou. Como perceber que estava doente se suas
faculdades eram poucas para atentar nos filhos? E lá sarou de pé, sem um
remédio. E como ela também sarariam os filhos todos se\ldots{}''

O doutor Bonifácio voltou"-se para o cônego.

--- \ldots{} se a caridade não interviesse\ldots{}

--- Já sei onde quer bater --- exclamou o cônego. --- Mas cumpre notar
que quando falo de caridade não me refiro à assistência pública, nem
sequer à filantropia. Falo da caridade sentimento, da caridade virtude
cristã --- concluiu baforando o cigarro, alegre, com ar de quem cortou
vazas.

O doutor Bonifácio prosseguiu:

--- \ldots{} se a caridade sentimento não sobreviesse por intermédio do
coração bondoso de uma vizinha. Esta vizinha, compadecida daquele
angustioso transe, telefonou a um posto médico narrando o caso e pedindo
assistência. A ambulância veio justamente durante a ausência da Isaura,
que saíra a compras, e levou"-lhe todos os filhos para o Hospital da
Imigração.

``Corriam boatos apavorantes a respeito deste hospital improvisado, onde
--- murmuravam --- só se recebiam os pobres bem pobres e o tratamento
era o que devia ser, porque pobre bem pobre não é bem gente. De modo que
nada apavorava tanto o povinho miúdo como ir para a Imigração.

``Assim, ao voltar da rua e saber do acontecido Isaura estarreceu. Foi
como se o próprio inferno houvesse aberto as goelas e engolido os
adorados doentes. Quem zelaria por eles? Sozinhas no meio de
desconhecidos, de enfermeiros mercenários, que seria das pobres
crianças?

``Correu para aqueles lados, inquirindo às tontas: `A Imigração? Onde
fica a Imigração?'. `É por aqui.' `Dobre à direita.' `É lá naquela casa
grande', informavam"-na pelo caminho.

``Chegou. Bateu. Esperou à porta um tempo enorme. Entravam e saíam
pessoas apressadas, médicos, ajudantes, homens de avental. `Não é
comigo', diziam. `Espere.' `Bata outra vez.'

``Afinal, uma alma caridosa\ldots{}''

--- Ca"-ri"-do"-sa --- repetiu o cônego, sorrindo.

--- \ldots{} uma alma caridosa apareceu e deu"-lhe a informação pedida. Os
filhos estavam lá, mais a netinha. A de dezesseis anos, porém, atacada
de tifo.

``--- Tifo?! --- exclamou, alanceada, a pobre mãe.

``A alma caridosa enterrou mais fundo o punhal:

``--- Sim, tifo, e do bravo.

``A mulher já não ouvia. De olhos esbugalhados, como fora de si, repetia
a esmo a palavra tremenda --- `Tifo!' Conhecia"-o muito bem. Fora a
doença malvada que lhe arrebatara o marido.

``--- Quero vê"-la, quero ver minha filha!\ldots{}

``--- Impossível!

``Isaura lutou, insistiu.

``Inútil.

``A porta fechou"-se com chave e a pobre mulher se viu despejada na rua.

``Andou muito tempo à toa, como ébria, sem destino. `Olha a louca!',
gritavam os moleques. E parecia mesmo, se não louca, pelo menos aluada.

``Súbito Isaura resolveu"-se. Havia de ver os filhos. Era mãe. `São meus,
o mundo nada tem com eles. Eu os tive, eu os criei, só eu os quero no
mundo. São tudo para mim. Como gentes estranhas me roubam assim os
filhos, me impedem que eu, mãe, os veja? Nem ver, apenas ver? Oh, isso é
demais.''

``Havia de vê"-los.

``Galvanizada pela resolução, Isaura correu a implorar socorro de um
homem influente cuja roupa lavava.

``O influente deu"-lhe uma carta. `Vá com isto que as portas se abrem.'

``Nova corrida ao hospital. Nova espera angustiosa. Por fim a mesma alma
caridosa\ldots{}''

O doutor Bonifácio entreparou, olhando para o sacerdote. E, como desta
vez ele silenciasse, prosseguiu:

--- Por fim a alma caridosa reapareceu e disse à desolada mãe:

``--- Posso ir lá dentro saber de seus filhos, mas deixá"-la entrar, não!

``--- E a carta?

``--- Inútil. É expressamente proibido.

``--- Pois dê"-me notícias de meus filhos, então.

``A alma caridosa foi saber dos doentinhos e a triste mãe, embrulhada em
seu xale humilde, ficou a um canto, esperando. Minutos depois reaparecia
a alma caridosa.

``--- Olhe, sua filha morreu.

``--- Morr\ldots{}

``E os olhos da miseranda mãe exorbitaram, seus dedos se crisparam\ldots{}

``--- Morreu!\ldots{} Mas qual delas?

``--- Uma delas.

``--- Mas qual? Qual?\ldots{}

``Já eram gritos lancinantes que lhe saíam da boca. A alma caridosa
fechou a porta e sumiu"-se\ldots{}

``O infinito desespero de Isaura nessa noite em casa, a revolver"-se na
cama, a remorder o travesseiro\ldots{} `Qual? Qual das minhas filhas
morreu?\ldots{}' A dor requintava"-se ante a incerteza. `Seria a Inesinha?
Seria a Marietinha?' E o cérebro lhe estalava na ânsia de adivinhar.
`Qual delas, meu Deus?'

``São dores que a palavra não diz. Imagina"-as a imaginação de cada um.
Adiante.

``No outro dia a mulher correu de novo ao hospital. Repete"-se a mesma
cena --- a ansiosa espera de sempre, os pedidos com lágrimas a saltarem
dos olhos. O ambiente é o mesmo --- de indiferença geral. Só não há
indiferença na alma caridosa, que reaparece e pergunta:

``--- Que quer de novo, santinha?

``--- Meus filhos\ldots{} saber\ldots{}

``--- Seus filhos? Não estão mais aqui. Foram removidos para o hospital
do

Isolamento, os dois.

``--- Os dois?!\ldots{}

``--- Os dois, sim, porque a mais pequena também morreu.

``--- A minha netinha morreu?!\ldots{}

``--- Coragem, minha velha, a vida é isto mesmo.

``E a porta fechou"-se pela última vez.''

As três ou quatro pessoas reunidas em torno do doutor Bonifácio ansiavam
pelo final da história. ``E depois?'', era a sugestão de todos os olhos.

O doutor Bonifácio prosseguiu:

--- Depois? Depois a gripe declinou, a normalidade foi se restabelecendo
e os dois filhos restantes voltaram à casa materna. Em que estado! O
menino, semimorto, cadavérico, e a Inês (só ao vê"-la chegar soube Isaura
qual das duas morrera) e a Inês com uma tosse de tuberculosa. E ali
ficaram, destroços de horrível naufrágio, aqueles três miseráveis
molambos de vida, sob a assistência da negra enfermeira --- a Fome.
Continuaram a viver, sem saber como, por instinto --- num desvario, numa
alucinação\ldots{}

``Da última vez que vi a pobre Isaura, disse"-me ela, entre dois acessos
de tosse:

``--- Tudo porque me levaram de casa os filhos. Se ficassem nada lhes
teria acontecido. A nossa vizinha, tão boa, coitada, quis fazer o bem e
fez a nossa desgraça. É um perigo ser muito bom\ldots{}''

O doutor Bonifácio calou"-se. O cônego não achou que fosse caso de
comentar. A roda dissolveu"-se em silêncio.

\chapter{«Quero ajudar o Brasil\ldots{}»\footnote[*]{Texto de 1938, publicado no livro \emph{Negrinha}.}}

Já contei este caso. Vou contá"-lo de novo. Hei de contá"-lo toda a vida,
porque é um grande conforto de alma. É a coisa mais bonita que ainda vi.

Foi no começo de nossa tremenda campanha pró"-petróleo. Havíamos com
Oliveira Filho e Pereira de Queiroz lançado a Companhia Petróleos do
Brasil --- em que ambiente, santo Deus! Tudo contra. Todos contra. O
Governo contra. Os homens de dinheiro contra. Os bancos contra. A
``sensatez'' contra.

Ceticismo absoluto em todas as camadas. Uma guerra surda por baixo,
subterrânea, que naquele tempo não sabíamos donde emanava. Guerra de
difamação ao ouvido --- a pior de todas. As coisas ditas em voz alta não
causam efeito; ao ouvido, sim.

--- Fulano é um escroque.

Enunciadas assim ao natural não impressionam a ninguém, tanto andamos
afeitos a ouvir acusações dessas. Mas a mesma frase dita muito em
reserva, ao ouvido, com a mão em tapa"-som, ``para que ninguém mais
ouça'', cala fundo, faz"-se imediatamente crida --- e quem a recebe corre
a propagá"-la como dogma.

A guerra contra os promotores da nova companhia era assim: de ouvido em
ouvido, as mãos sempre em tapa"-som --- \emph{para que ninguém mais
ouvisse o que era preciso que todos soubessem}. A calúnia é a rainha da
técnica.

Nos seus manifestos os incorporadores haviam sido em extremo leais.
Admitiam a possibilidade de fracasso, com perda total do capital
empatado. Pela primeira vez na vida comercial deste país se propunha ao
público um negócio com admissão das duas faces: vitória esplêndida, em
caso de encontro do petróleo, ou perda total dos dinheiros invertidos,
no caso reverso. Esta franqueza impressionou. Inúmeros subscritores
vieram arrastados por ela.

--- Vou tomar tantas ações só por terem os senhores mencionado a
hipótese da perda total dos dinheiros. Isso me convenceu de que se trata
de negócio sério. Os negócios não sérios só acenam com lucros, jamais
com possibilidades de perda.

A lealdade dos incorporadores foi vencendo o público miúdo. Só aparecia
no escritório gente simples, tentada pelas vantagens tremendas do
negócio em caso de sucesso. O raciocínio de todos era o mesmo de na
compra dum bilhete das grandes loterias do Natal.

Os incorporadores levaram o escrúpulo a ponto de lembrar a cada novo
subscritor a hipótese da perda total do dinheiro.

--- Sabe que corre o risco de perder o seu cobre? Sabe que se não
tocarmos em petróleo o fracasso da empresa será completo?

--- Sei. Li o manifesto.

--- Mesmo assim subscreve?

--- Mesmo assim.

--- Então assine.

E desse modo iam sendo as ações absorvidas pelo público.

Certo dia entrou"-nos pela sala um preto modestamente vestido, de ar
humilde. Recado de alguém, certamente.

--- Que deseja?

--- Quero tomar umas ações.

--- Para quem?

--- Para mim mesmo.

Oh! O fato surpreendeu"-nos. Aquele homem tão humilde a querer comprar
ações. E logo no plural. Quereria duas, com certeza, uma para si, outra
para a mulher. Isso importaria em duzentos mil"-réis, quantia que já pesa
num orçamento de pobre. Quantos sacrifícios não teria de fazer o casal
para pôr de lado duzentos mil"-réis ratinhados ao salário miserável? Para
um ricaço tal quantia corresponde a um níquel; para um operário é uma
fortuna, é um capital. Os salários no Brasil são a miséria que sabemos.

Repetimos ao extraordinário preto a cantiga de sempre.

--- Sabe que há mil dificuldades neste negócio e que corremos o risco de
perder a partida, com destruição de todo o capital empatado?

--- Sei.

--- E mesmo assim quer tomar ações?

--- Quero.

--- Está bem. Mas se houver fracasso não se queixe de nós. Estamos a
avisá"-lo com toda a lealdade. Quantas ações quer? Duas?

--- Quero trinta.

Arregalamos os olhos e, duvidando dos nossos ouvidos, repetimos a
pergunta.

--- Trinta, sim --- confirmou o preto.

Entreolhamo"-nos. O homem devia estar louco. Tomar trinta ações, empatar
três contos de réis num negócio em que a gente mais endinheirada não se
atrevia a ir além de algumas centenas de mil"-réis, era evidentemente
loucura. Só se aquele homem de pele preta estava escondendo o leite ---
se era rico, muito rico.

Na América existem negros riquíssimos, até milionários; mas no Brasil
não há negros ricos. Teria aquele, por acaso, ganho algum pacote na
loteria?

--- Você é rico, homem?

--- Não. Tudo quanto tenho são estes três contos que juntei na Caixa
Econômica. Sou empregado na Sorocabana há muitos anos. Fui juntando de
pouquinho em pouquinho. Hoje tenho três contos.

--- E quer pôr tudo num negócio que pode falhar?

--- Quero.

Entreolhamo"-nos de novo, incomodados. Aquele raio de negro nos
atrapalhava seriamente. Forçava"-nos a uma inversão de papéis. Em vez de
acentuarmos as probabilidades felizes do negócio, passamos a acentuar as
infelizes. Enfileiramos todos os contras. Quem nos ouvisse, jamais
suporia estar diante de incorporadores duma empresa que pede dinheiro ao
público --- mas de difamadores dessa empresa. Chegamos a afirmar que
pessoalmente não tínhamos muitas esperanças de vitória.

--- Não faz mal --- respondeu o preto na sua voz inalteravelmente
serena.

--- Faz, sim! --- insistimos. --- Jamais nos perdoaríamos se fôssemos os
causadores da perda total das reservas duma vida inteira. Se quer mesmo
arriscar, tome duas ações só. Ou, três. Trinta é demais. Não é negócio.
Ninguém põe tudo quanto possui num cesto só, e muito menos num cesto
incertíssimo como este. Tome três.

--- Não. Quero trinta.

--- Mas por quê, homem de Deus? --- indagamos, ansiosos por descobrir o
segredo daquela decisão inabalável.

Seria a cobiça? Crença de que com trinta ações ficaria milionário em
caso de jorrar o petróleo?

--- Venha cá. Abra o seu coração. Diga tudo. Qual o verdadeiro motivo de
você, um homem humilde, que só tem três contos de réis, insistir desta
maneira em jogar tudo neste negócio? Ambição? Pensa que pode ficar um
Matarazzo?

--- Não. Não sou ambicioso --- respondeu ele serenamente. --- Nunca
sonhei em ficar rico.

--- Então por que é, homem de Deus?

--- É que eu quero ajudar o Brasil\ldots{}

Derrubei a caneta debaixo da mesa e levei uma porção de tempo a
procurá"-la. Maneco Lopes fez o mesmo, e foi embaixo da mesa que nos
entreolhamos, com caras que diziam: ``Que caso, hein?''. Em certas
ocasiões só mesmo derrubando uma caneta e custando a achá"-la, porque há
umas tais glândulas que nos turvam os olhos com umas aguinhas
impertinentes\ldots{}

Nada mais tínhamos a dizer. O humilde negro subscreveu as trinta ações,
pagou"-as e lá se foi, na sublime serenidade de quem cumpriu um dever de
consciência.

Ficamos a olhar uns para os outros, sem palavras. Que palavras
comentariam aquilo? Essa coisa chamada Brasil, que é de vender, que até
os ministros vendem, ele queria ajudar\ldots{} De que brancura deslumbrante
nos saíra aquele negro! E como são negros certos ministros brancos!

O incidente calou fundo em nossas almas. Cada um de nós jurou lá por
dentro levar avante a campanha do petróleo custasse o que custasse,
sofrêssemos o que sofrêssemos, houvesse o que houvesse. Tínhamos de nos
manter na altura daquele negro.

A campanha do petróleo tem sofrido variados desenvolvimentos. Guerra
grande. Luta peito a peito. E se o desânimo não nos vem nunca, é que as
palavras do negro ultrabranco não nos saem dos ouvidos. Nos momentos
trágicos das derrotas parciais (e têm sido muitas), nos momentos em que
os lidadores no chão ouvem o juiz contar o tempo do nocaute, aquelas
palavras sublimes fazem que todos se ergam antes do \textsc{dez} fatal.

``--- É preciso ajudar o Brasil\ldots{}''

Hoje sabemos de tudo. Sabemos das forças invisíveis, externas e
internas, que puxam para trás. Sabemos os nomes dos homens. Sabemos da
sabotagem sistemática, dos móveis da difamação ao ouvido, do perpétuo
dar para trás da administração. Isso, entretanto, deixa de ser obstáculo
porque é menor que a força haurida nas palavras do negro.

Abençoado negro! Um dia teu nome será revelado. O primeiro poço de
petróleo em São Paulo não terá o nome de nenhum ministro nem presidente.
Terá o teu. Porque talvez tenham sido tuas palavras a secreta razão da
vitória. Os teus três contos foram mágicos. Amarraram"-nos para sempre.
Trancaram com pregos a porta da deserção\ldots{}

\part{\textsc{terra para rir ou chorar}}

\chapter{Gens ennuyeux\footnote[*]{Texto de 1901, publicado no livro \emph{Cidades mortas}.}\,\footnote[**]{Com este trabalho Monteiro
  Lobato, então acadêmico de Direito, concorreu ao concurso de contos
  instituído pelo~jornal \emph{O Onze de Agosto}. Obteve o primeiro
  lugar, tendo o júri sido composto por Amadeu Amaral, Garcia Redondo e
  Silvio de Almeida. ``Gens ennuyeux'' foi publicado no \emph{O Onze}
  \emph{de Agosto}, de 12 de outubro de 1904. Nota da edição de 1955.}}

--- Queres ir? --- indagou Lino, espichando"-me um convite. Li: \emph{A
Sociedade Científica,} ahn, ahn\ldots{} \emph{convida}, ahn\ldots{} \emph{a
conferência versará sobre a História da Terra}.

--- É; a tese é catita; vais?

--- Está"-me apetecendo conhecê"-los aos nossos sábios.

--- Sábios --- rosnei ---, \emph{gens ennuyeux}\ldots{}

--- Nem sempre --- contraveio Lino. --- O assunto é magnífico --- e
depois, que diabo!, uma penitenciazinha de vez em quando, por amor à
ciência\ldots{}

--- Pois vamos --- resolvi com intrepidez.

--- Às oito, rua tal.

--- Lá estarei sem falta.

\asterisc

Ao assomarmos à porta já as cadeiras do grande salão se pintalgavam de
graves sobrecasacas científicas, encimadas por carecas luzidias, em cujo
espelho punha gangrenas de luz (perdão, Apolo!) a luz violácea do arco
voltaico.

Entramos com religiosa compostura, pisando com passos humílimos o
augusto piso do Pagode da Ciência.

No rosto do meu amigo vi uma leve expressão de terror sagrado. Os
quíchuas, quando davam de chofre com o Eldorado, haviam de ficar
assim\ldots{}

Lino comovia"-se deveras e foi balbuciante que cochichou:

--- Sábios, hein?

Sentamo"-nos devagarinho e pusemo"-nos a olhar. Novas sobrecasacas
chegavam, aos magotes de três e quatro, compenetradas, pensabundas. Eram
novos sábios de variegado estilo. Havia o estilo"-fiambre: gente
vermelha, com sangue à flor da pele em permanente congestão. O
estilo"-melado: gênero de importação alemã. O estilo"-\emph{ball}: queijos
de Palmira com o vermelho substituído por um palor circular de
cabelugens ralas. O estilo"-clorose: rapazelhos de peito cavo e barba a
espontar ingenuamente, macilentos de tez, olhos de bezerro disentérico,
em cujas meninas --- meninas dos olhos --- pareciam boiar hipotenusas de
braços dados a binômios de Newton.

À nossa destra suava uma rubra apoplexia alemã, enchouriçada em
sobrecasaca de debrum contemporânea do iguanodonte, cujas costuras
cediam à pressão das enxúndias comprimidas; sua mão gordita, recoberta
de dourados pelinhos, alisava a grelha cor de fogo como quem alisa um
gato de luxo.

Mais adiante, um amplo burguês, barbaçudo, verrugoso, bexiguento,
fungava a suar.

À sua frente, sorrindo com bondade em meio dum grupinho amigo, uma
espécie de criatura do sexo neutro, acondicionada em alpaca, sem um só
enfeite e cujos cabelos grisalhos se erguiam em ríspido pericote sob a
copa acartolada dum chapéu masculino. Discutia Cuvier.

--- É a doutora Mariote\ldots{} --- sussurrou"-me o Lino. --- Uma sábia
sapientíssima!\ldots{}

Mais além, um oculista de nomeada; depois, um pomólogo; em seguida um
filósofo, uma parteira, um charlata, um lente de geometria, um
fisiopsicopatologista.

Nós, miserandos intrusos, vexados da nossa espessa ignorância a dois,
comentávamos baixinho, com respeitosa deferência, as efígies hirsutas
daqueles paredros que davam de tu a Minerva. Lino nem falava: ciciava
tatibitate. Aquela face da sociedade nos era de todo desconhecida. Tudo
ali cheirava a novidade. O próprio ar nada tinha do ar comum das ruas:
pairava nele um cheirinho sutil a raízes cúbicas.

À frente do salão havia uma comprida mesa em cujo centro o presidente da
Sociedade --- um rolete de homem cor de salame --- cofiava os bigodinhos
ruivos, bamboleando no ar pés que não alcançavam o chão. Ladeavam"-no
dois bonitos secretários a remexerem atas. Sobre a mesa, enfileirada,
uma récua de bichos pré"-históricos em miniatura --- estegossauros,
plesiossauros, iguanodontes e um mamutezinho que escancarava a goela
vermelha num urro mudo.

\emph{--- Dlin}, \emph{dlin}, \emph{dlin!\ldots{}} Está aberta a sessão ---
rosnou o presidencial salame.

O secretário mascou a ata --- \emph{tá}, \emph{tá}, \emph{tá}\ldots{}

--- Tem a palavra o conferencista.

Corre pela sala o bisbilho da curiosidade. Galga a tribuna um homem.
Roliço e pipote, tem a calva resplandecente, traz casaca, óculos e
convicção profunda. Prepara os papéis, tosse.

Novo \emph{psst!} desliza pelo salão. Cai nele o silêncio curioso da
expectativa.

--- Minhas senhoras e meus senhores! Me parece que a outro e não a mim,
que sou o mais modesto membro da Sociedade\ldots{}

Entreolhamo"-nos àquele \emph{me} com piscadelas gramaticais, e
entregamos nossos quatro ouvidos às palavras do Sábio. Após o exórdio da
praxe, o orador veste o escafandro da observação, apoia"-se no pau
ferrado da crítica, encavalga na penca os nasóculos da análise e, sem
tir"-te, cai no mergulho do fundo sombrio das idades. Vai aos períodos
\emph{eos} examinar \emph{gneiss} e \emph{micaxistos}; mostra exemplares
ao auditório, descreve"-os com minúcia. Narra como vieram os primeiros
vegetais --- samambaiuçus enormes e molengos --- e como à sombra deles
foram surgindo bichinhos tontos, sem experiência da vida, admiradíssimos
de verem casa tão grande posta a seres tão pequenos. Fala com a
segurança de um feto arborescente, testemunha ocular daquilo, transfeito
em sábio moderno. Diz e rediz. Vai e volta --- porque o \emph{gneiss}
pra aqui, porque o \emph{gneiss} pra lá, porque o \emph{gneiss}, o
\emph{gneiss}, o \emph{gneiss}\ldots{}

Depois agarra os \emph{trilobitas}, os \emph{amonitas} e mói, remói,
tremói, pulveriza os pobres bichinhos, digressiona, gesticula, sua: o
\emph{amonita}\ldots{} \emph{porque o trilobita}\ldots{} não obstante o
\emph{amonita}\ldots{} \emph{bita}\ldots{} \emph{nita}\ldots{} e \emph{nita} e
\emph{bita}, lá borbota ele ciência pura, híspida, hirsuta, inexorável,
num fluxo que berra por tampões de percloreto de ferro.

O tempo corre, e da torneira aberta deflui caudaloso o jorro
hermafrodita do palavreado greco"-latino. O espelho da sua careca
tremeluz de inspiração. Seu dedo pontifical coleia riscos explicatórios.
E a linfa científica a jorrar, a jorrar durante quinze, trinta minutos,
uma hora, hora e meia\ldots{}

O esgoelado urro do mamutezinho já não é mais urro, sim bocejo
formidoloso. E não o único. Pela sala outros se escancaram,
incoercíveis. A doutora reprime os seus com caretas. Algumas
sobrecasacas cochilam. O burguês das verrugas resfolga com maior
estrépito e mais bagas de suor na testa.

E na tribuna a ciência a correr\ldots{} a farragem fóssil a desfilar
inesgotável numa sarabanda sem fim: porque o \emph{gneiss}, o
\emph{micaxisto}\ldots{} não obstante o \emph{bita}, o \emph{nita}\ldots{} os
conglomerados da Westfália, as superposições devonianas, a sedimentação
terciária, \emph{tá}, \emph{tá}, \emph{tá, tá}\ldots{}

Nesse ponto penetrou na sala um delicioso casal, pisando de leve os
passinhos de lã preventivos dos \emph{pssts}. Ele, alto e elegante; ela,
mimosa e feminina, tom exótico de teteia cara. Sentam"-se. Ele abre os
ouvidos. Ela espevita o \emph{lorgnon} e corre os olhos vivos de malícia
irônica pela assembleia inteira: pousa"-os por fim na figura salpiconesca
do orador.

Lino segue"-os.

--- Que graciosos! --- diz, furando"-me as costelas a cotoveladas ---;
repara na ironia daqueles dois diamantes negros. Pousam na careca do
homem\ldots{} alisam"-na com bonomia malandra\ldots{} agora descem, examinam o
nariz\ldots{} Riem"-se os marotos --- e da verruga talvez\ldots{} Tentam
arrancá"-la\ldots{} irritam"-se\ldots{} fogem da penca\ldots{} examinam o feitio da
sobrecasaca. Bom, deixaram em paz o homem\ldots{} passeiam pela sala\ldots{} dão
com o chapéu da doutora Mariote\ldots{} Como se riem perdidamente os
moleques!

Enquanto os olhos do meu amigo estudam os maliciosos olhos da linda
criatura, barafustam"-se os meus pela goela do mamutezinho que o dedo do
sábio apontava naquele momento.

--- \ldots{} e apareceu então --- dizia ele --- um animal de pelos duros e
pretos, de presas recurvadas, cujo esqueleto foi encontrado na
embocadura do Iena e se chamou mamute\ldots{}

Lino arrancou"-me de golpe às goelas do monstro e ao caçanje do sábio.

--- Vê como ela boceja com graça.

De fato, a petulante boquinha da moça escondia no leque um bocejo
saciado; saciado e contagioso, porque logo em seguida o sociólogo
escancarou o seu, o pomólogo lá no fundo abriu outro, e o alemão da
nossa direita reprimiu um que prometia levar as lampas ao do mamute.

--- Dez horas já! --- espantou"-se Lino, consultando o relógio. --- Há
esperanças de fim?

--- Qual! --- gemi. --- Ele ainda está no megatério.

--- E é comprido o megatério?

--- Enorme. E tem vasta parentela. Só depois de descritos os
gliptodontes, os megáceros, os rinoceros e as hienas é que há esperanças
de entrarmos na terra do nosso avô pitecantropo. Coragem!

Às dez e mais inda o corrimento paleontológico continuava copioso, sem
sintomas de exaustão. Sistemas sobre sistemas amontoavam"-se, induções
sobre induções, num mascar monótono de realejo elétrico. Nossas nádegas
protestavam. Novos bocejos insolentes amiudavam exigências: queriam sair
já e já, queriam passagem franca, bocas bem escancaradas --- e nós
lutávamos por conter"-lhes a má"-criação.

E o chafariz científico a despejar.

--- Há esperanças --- sussurrei para o Lino. --- Já estamos no
\emph{Homo sapiens}.

--- Bendito sejas, ó rei da criação!

Era verdade. O sábio penetrara no homem. Mais cinquenta minutos de seca
e pingou o ponto, convidando a assistência a examinar de perto os
fósseis amontoados sobre a mesa.

Estrepitaram palmas, e após o \emph{uf!} de ressurreição encheu o
recinto o sussurro do ``à vontade'', das cadeiras recuadas, do frufrutar
surdo dos capotes enfiados, dos espreguiçamentos risonhos.

--- Que gostosura, um fim de seca!

A assistência aflui aos magotes para junto à mesa a fim de examinar os
bichos. Fomos na onda. Todos comentavam, queriam pegar, apalpar os
fósseis, cheirá"-los, prová"-los.

Com um estegossauro de palmo e meio seguro pelo cangote, o sociólogo
explicava ao pomólogo ``de como pela restauração de Cuvier se tinha ali
um elo da vasta cadeia da evolução que Darwin descobrira''.

Ao centro da mesa o conferencista desfazia"-se em amabilidades de
caixeiro, fragmentando sua ciência e distribuindo"-a em pílulas.

--- Olhe, doutor --- dizia o filólogo ---, olhe a \emph{baculite} de
transição de que falei.

E para outro sujeito:

--- Já viu, doutor, o magnífico exemplar de \emph{hipurite} que nos veio
de Berlim?

Nisto ouvi ao meu lado um resfôlego adiposo; voltei"-me: era o burguês
das verrugas, com a toucinhenta consorte pelo braço, a examinar uma
lasca de pedra azulega que de mão em mão viera ter às suas. O bicharoco
olhava a pedra como quem olha talismã. Não resisti, atirei"-lhe a esmo:

--- É o \emph{gneiss}.

O burguês encarou"-me com o respeito devido a Quem Sabe e, virando"-se
para a mulher, repetiu gravemente:

--- Este é o \emph{gneiss}, Maricota.

Dona Maricota tomou"-o nos dedos, examinou"-o sob todas as faces e em
seguida passou"-o a uma sua amiga, gaguejando de geológica emoção:

--- O \emph{gneiss}, Nhanhã!

Na rua esfumada pela garoa, um friozinho de tiritar. De golas erguidas
estugamos o passo, enquanto íamos extraindo a moralidade da festa.

Ciência e Arte nasceram para viver juntas, porque Arte é harmonia e
Ciência é verdade. Quando se divorciam, a verdade fica desarmônica e a
harmonia falsa. Se este senhor sábio trouxesse pela mão direita a
Ciência e pela esquerda a Arte, para fundi"-las no momento de falar, que
coisa esplêndida não faria de um tal tema! Trouxe uma só e por isso
maçou"-nos, empanturrou"-nos a alma de coisas duras, indigeríveis,
misturadas com mil pronomes fora dos mancais. Além disso\ldots{}

Foi"-nos impossível prosseguir na filosofia. Um carro passava estalando
rumorosamente as pedras da rua. Dentro vinha a nossa diva.

--- Ela\ldots{}

--- A Verdade e a Harmonia\ldots{}

Nossas bocas emudeceram, porque a imaginação, tomando as rédeas nos
dentes, nos levava a galope no encalço da teteia de olhos negros.

\chapter{Cabelos compridos\footnote[*]{Texto de 1904, publicado no livro \emph{Cidades mortas}.}}

--- Coitada da Das Dores, tão boazinha\ldots{}

Das Dores é isso, só isso --- boazinha. Não possui outra qualidade. É
feia, é desengraçada, é inelegante, é magérrima, não tem seios, nem
cadeiras, nem nenhuma rotundidade posterior; é pobre de bens e de
espírito; e é filha daquele Joaquim da Venda, ilhéu de burrice ebúrnea
--- isto é, dura como o marfim. Moça que não tem por onde se lhe pegue
fica sendo apenas isso --- boazinha.

--- Coitada da Das Dores, tão boazinha\ldots{}

Só tem uma coisa a mais que as outras --- cabelo. A fita da sua trança
toca"-lhe a barra da saia. Em compensação, suas ideias medem"-se por
frações de milímetro, tão curtinhas são. Cabelos compridos, ideias
curtas --- já o dizia Schopenhauer.

\textls[-10]{A natureza pôs"-lhe na cabeça um tabloide homeopático de inteligência, um
grânulo de memória, uma pitada de raciocínio --- e plantou a cabeleira
por cima. Essa mesquinhez por dentro. Por fora ornou"-lhe a asa do nariz
com um grão de ervilha, que ela modestamente denomina verruga,
arrebitou"-lhe as ventas, rasgou"-lhe a boca de dimensões comprometedoras
e deu"-lhe uns pés\ldots{} Nossa Senhora, que pés! E tantas outras pirraças
lhe fez que ao vê"-la todos dizem comiserados:}

--- Coitada da Das Dores, tão boazinha\ldots{}

Das Dores só faz o que as outras fazem e porque as outras o fazem. Vai à
igreja aos domingos de livrinho na mão, ouve a missa, ouve a prédica,
reza. Nunca falhou um dia. Se lhe perguntarem o porquê daqueles atos,
responderá, muito admirada da pergunta:

--- Mas se todas vão!

O grande argumento de Das Dores é esse: as outras. Ouve o sermão do
padre e chora nos lances trágicos, não porque compreenda algo daquela
retórica, nem porque sinta vontade de chorar --- mas porque as outras
choram.

Toma tudo quanto ouve ao pé da letra, incapaz que é de galgar do
concreto ao abstrato. Se ouve falar em ``fazer pé de alferes'', fica a
pensar em pés e mãos de alferes e tenentes.

--- Tão boazinha a Das Dores\ldots{}

Uma vez foi à prédica de um padre em missão pela zona, orador famoso
pelas muitas almas que desatolara do chafurdeiro de Satanás. Ouviu"-lhe
muita coisa que não entendeu, mas entendeu um pedacinho que terminava
assim:

``Meditai, meus irmãos, refleti em cada uma das palavras das vossas
orações cotidianas, pois do contrário não terão elas nenhum valor''.

Das Dores saiu da igreja impressionada com o estranho conselho e se foi
de consulta à tia Vicência, velha sabidíssima em mezinhas e teologias.

--- Tia Vicência viu o que o seu cônego disse? Pra gente pensar em cada
palavra senão a reza não vale?\ldots{}

A tia mastigou um ``pois é'' que dava toda a razão ao padre.

--- Que coisa, não? --- foi o comentário final de Das Dores, que
continuava a achar esquisitíssima aquela ideia.

\textls[-20]{À noite era seu costume rezar umas tantas orações preventivas dos mil
males possíveis no dia seguinte. Mas até ali as rezara qual um
fonógrafo, \emph{psi}, \emph{psi}, \emph{psi}, amém. Tinha agora que
pensar nas palavras. Diabo! Havia de ficar engraçada a reza\ldots{}}

Caiu a noite. Das Dores meteu"-se na cama, cobriu a cabeça com o lençol e
deu início à novidade. Abriu com o Padre"-Nosso.

--- \emph{Padre"-Nosso que estais no céu}; padre, padre; os padres, padre
Pereira, padre vigário\ldots{} Padre Luís\ldots{} Coitado, já morreu e que morte
feia --- estuporado!\ldots{} Padre\ldots{} Que ideia do seu cônego mandar a gente
pensar nas palavras! Nem se pode rezar direito\ldots{}

``\ldots{} \emph{nosso}; nosso é o que é da gente; nossa casa; nossa vida;
nosso pai\ldots{} Pra quem seria que foi o Nosso"-Pai ontem? Para a nhá Veva
não é, que ela já melhorou. Seria para o major Lesbão? Coitado! Quem
sabe se a estas horas já não está no outro mundo? Bom homem, aquele\ldots{}
Tão caridoso\ldots{} Ó diabo! Estou me distraindo! `Nosso', `nosso'\ldots{} Em
certas palavras não se tem jeito de pensar\ldots{}

``\ldots{} \emph{que estais no céu}: estar no céu, que lindeza não será! Os
anjos voando, as estrelinhas, Nossa Senhora tão bonita com o Menino no
braço, os santos passeando de lá para cá\ldots{} O céu; céu; céu da boca; céu
azul. Por que será que se diz céu da boca?

``\ldots{} \emph{santificado}, san"-ti"-fi"-ca"-do; que é santo; dia santificado,
dia santo\ldots{}

``\ldots{} \emph{seja vosso nome}; nome; nome bonito\ldots{} Nome feio! Quantos
tapas levei na boca por dizer nomes feios! Quem me ensinava era aquela
bruxa da Cesária. Peste de negrinha! Onde andará ela? `Nome de gente';
`nome de cachorro'. Gustavo, bonito nome. Está ali um que se quisesse\ldots{}
Mas nem me enxerga, o mauzinho; é só a Loló praqui, a Loló prali, aquela
caraça de broa\ldots{} Gustavo é o nome de homem mais bonito para mim. De
mulher é\ldots{} Rosinha? Não. Merência? Não\ldots{} `Home', a falar verdade
nenhum. Gustavo. Gustavinho\ldots{} Ahn, que sono!

``\emph{O pão nosso}; pão; pão\ldots{} Por que será que quando a gente repete
muitas vezes uma palavra ela perde o jeito e fica assim esquisita? Pão;
pão; pã"-o\ldots{} Por falar em pão, como anda minguando o pão do Zé Padeiro!
E que pão ruim! Azedo\ldots{} Pão sovado; pão de cará; pão de Petrópolis\ldots{}

``\ldots{} \emph{de cada dia}; dia; dia; marido da noite; dia de sol; dia de
chuva; dia das almas; dia de anos; dia bonito\ldots{} E que dia bonito fez
ontem! Vão ver que domingo chove. É sempre assim. Havendo uma festinha,
chove mesmo. Amanhã, se fizer bom dia, vou à casa da Iná. Coitada da
Iná! Acontece cada coisa nesta vida\ldots{}

``\ldots{} \emph{dai"-nos hoje}; hoje, hoje\ldots{} Que é que eu fiz hoje? Ahn! Que
soneira!

``\ldots{} \emph{e livrai"-nos Senhor}; senhor; ilustríssimo senhor Gustavo de
Silva. Bonito nome! Senhor amado; Senhor morto; senhor; se"-nhor, nhor,
nhor"-se\ldots{}

``\ldots{} \emph{de todo o mal}; mal; mal\ldots{} mal\ldots{} al\ldots{}''

Os olhos de Das Dores fecharam"-se, o corpo moleou e seu sono foi um só
até romper o dia. Ao despertar lembrou"-se logo do caso da véspera.
Sorriu. Achou que a ideia do cônego --- um padre de tanta fama! --- não
passava de grossa asneira. E pela primeira vez na vida duvidou.

--- Ora, titia --- foi ela dizer à tia Vicência ---, aquilo é asneira.
Se a gente for pensar em cada palavra, não pode rezar direito. O cônego
que me perdoe, mas ele disse uma grande bobagem\ldots{}

Não se sabe se a tia lhe deu razão ou não; mas o fato é que Das Dores
continuou a rezar pelo sistema antigo, mais rápido, mais correntio e com
certeza mais agradável a Deus. Quem se saiu mal do incidente foi o pobre
missionário. Cada vez que se referiam a ele perto de Das Dores, ela
floria a cara de uma risadinha irônica.

--- Está aí um que pode estar dizendo as coisas que eu\ldots{}

E concluía a frase com o mais convencido muxoxo de pouco"-caso.

\chapter{O fígado indiscreto\footnote[*]{Texto de 1904, publicado no livro \emph{Cidades mortas}.}\,\footnote[**]{Na primeira publicação
  deste trabalho --- \emph{Revista do Brasil,} n. 39, março de 1919 ---
  vinha o seguinte subtítulo: ``Ou o rapaz que saía fora de si''. Nota
  da edição de 1955.}}

Que há um deus para o namoro e outro para os bêbados está provado ---
\emph{a contrario sensu}. Sem eles, como explicar tanto passo falso sem
tombo, tanto tombo sem nariz partido, tanta beijoca lambiscada a medo
sem maiores consequências afora uns sobressaltos desagradáveis, quando
passos inoportunos põem termo a duos de sofá em sala momentaneamente
deserta?

Acontece, todavia, que esses deuses, ao jeito dos de Homero, também
cochilam: e o borracho parte o nariz de encontro ao lampião, ou a futura
sogra lá apanha Romeu e Julieta em flagrante contato de mucosas
petrificando"-os com o clássico: ``Que pouca"-vergonha!\ldots{}''.

Outras vezes acontece aos protegidos decaírem da graça divina.

Foi o que sucedeu a Inácio, o calouro, e isso lhe estragou o casamento
com a Sinharinha Lemos, boa menina a quem cinquenta contos de dote
faziam ótima.

Inácio era o rei dos acanhados. Pelas coisas mínimas avermelhava, saía
fora de si e permanecia largo tempo idiotizado.

O progresso do seu namoro foi, como era natural, menos obra sua que da
menina, e da família de ambos, tacitamente concertadas numa conspiração
contra o celibato do futuro bacharel. Uma das manobras constou do
convite que ele recebeu para jantar nos Lemos, em certo dia de
aniversário familiar comemorado a peru.

Inácio barbeou"-se, laçou a mais formosa gravata, floriu de orquídeas a
botoeira, friccionou os cabelos com loção de violetas e lá foi, de roupa
nova, lindo como se saíra da fôrma naquela hora. Levou consigo,
entretanto, para mal seu, o acanhamento --- e daí proveio a
catástrofe\ldots{}

Havia mais moças na sala, afora a eleita, e caras estranhas, vagamente
suas conhecidas, que o olhavam com a benévola curiosidade a que faz jus
um possível futuro parente.

Inácio, de natural mal firme nas estribeiras, sentiu"-se já de começo um
tanto desmontado com o papel de galã à força que lhe atribuíam. Uma das
moças, criaturinha de requintada malícia, muito ``saída'' e
``semostradeira'', interpelou"-o sobre coisas do coração, ideias
relativas ao casamento e também sobre a ``noivinha'' --- tudo com meias
palavras intencionais, sublinhadas de piscadelas para a direita e a
esquerda.

Inácio avermelhou e tartamudeou palavras desconchavadas, enquanto o
diabrete maliciosamente insistia: ``Quando os doces, seu Inácio?''.

Respostas mascadas, gaguejadas, ineptas, foram o que saiu de dentro do
moço, incapaz de réplicas jeitosas sempre que ouvia risos femininos em
redor de si. Salvou"-o a ida para a mesa.

Lá, enquanto engoliam a sopa, teve tempo de voltar a si e arrefecer as
orelhas. Mas não demorou muito no equilíbrio. Por dá cá aquela palha o
pobre rapaz mudava"-se de si para fora, sofrendo todos os horrores
consequentes. A culpada aqui foi a dona da casa. Serviu"-lhe dona Luísa
um bife de fígado sem consulta prévia.

Esquisitice dos Lemos: comiam"-se fígados naquela casa até nos dias mais
solenes.

Esquisitice do Inácio: nascera com a estranha idiossincrasia de não
poder sequer ouvir falar de fígado. Seu estômago, seu esôfago e talvez o
seu próprio fígado tinham pela víscera biliar uma figadal aversão. E não
insistisse ele em contrariá"-los: amotinavam"-se, repelindo
indecorosamente o pedaço ingerido.

Nesse dia, mal dona Luísa o serviu, Inácio avermelhou de novo, e
novamente saiu de si. Viu"-se só, desamparado e inerme ante um problema
de inadiável solução. Sentiu lá dentro o motim das vísceras; sentiu o
estômago, encrespado de cólera, exigir, com império, respeito às suas
antipatias. Inácio parlamentou com o órgão digestivo, mostrou"-lhe que
mau momento era aquele para uma guerra intestina. Tentou acalmá"-lo a
goles de clarete, jurando eterna abstenção para o futuro. Pobre Inácio!
A porejar suor nas asas do nariz, chamou a postos o heroísmo, evocou
todos os martírios sofridos pelos cristãos na era romana e os padecidos
na era cristã pelos heréticos; contou um, dois, três e \emph{glug!},
engoliu meio fígado sem mastigar. Um gole precipitado de vinho rebateu o
empache. E Inácio ficou a esperar, de olhos arregalados, imóvel, a
revolução intestina.

Em redor a alegria reinava. Riam"-se, palestravam ruidosamente, longe de
suspeitarem o suplício daquele mártir posto a tormentos de uma nova
espécie.

--- Você já reparou, Miloca, na ``ganja'' da Sinharinha? --- disse a
sirigaita de ``beleza'' na testa. --- Está como quem viu o passarinho
verde\ldots{} --- e olhou de soslaio para Inácio.

O calouro, entretanto, não deu fé da tagarelice; surdo às vozes do
mundo, todo se concentrava na auscultação das vozes viscerais. Além
disso, a tortura não estava concluída: tinha ainda diante de si a
segunda parte do fígado engulhento.

Era mister atacá"-la e concluir de vez a ingestão penosa. Inácio
engatilhou"-se de novo e --- um, dois, três: \emph{glug!} --- lá rodou,
esôfago abaixo, o resto da miserável glândula.

Maravilha! Por inexplicável milagre de polidez, o estômago não reagiu.
Estava salvo Inácio. E como estava salvo, voltou lentamente a si, muito
pálido, com o ar lorpa dos ressuscitados. Chegou a rir"-se. Riu"-se
alvarmente, de gozo, como riria Hércules após o mais duro dos seus
trabalhos. Seus ouvidos ouviam de novo os rumores do mundo, seu cérebro
voltava a funcionar normalmente e seus olhos volveram outra vez às
visões habituais.

Estava nessa doce beatitude, quando:

--- Não sabia que o senhor gostava tanto de fígado --- disse dona Luísa,
vendo"-lhe o prato vazio. --- Repita a dose.

O instinto de conservação de Inácio pulou em guarda. E fora de si outra
vez o pobre moço exclamou, tomado de pânico:

--- Não! Não! Muito obrigado!\ldots{}

--- Ora, deixe"-se de luxo! Tamanho homem com cerimônias em casa de
amigos. Coma, coma, que não é vergonha gostar de fígado. Aqui está o
Lemos, que se pela por uma isca.

--- Iscas são comigo --- confirmou o velho. --- Lá isso não nego. Com
elas ou sem elas, nunca as enjeitei.\footnote{\emph{``}Iscas com elas ou
  sem elas'' é como os restaurantes portugueses anunciam fígado com ou
  sem batatas. Nota da edição de 1946.} Tens bom gosto, rapaz.
Serve"-lhe, serve"-lhe mais, Luísa.

E não houve salvação. Veio para o prato de Inácio um novo naco --- este
formidável, dose dupla.

Não se descreve o drama criado no seu organismo. Nem um Shakespeare, nem
Conrad --- ninguém dirá nunca os lances trágicos daquela estomacal
tragédia sem palavras. Nem eu, portanto. Direi somente que à memória de
Inácio acudiu o caso de Nora de Ibsen na \emph{Casa de bonecas}, e
disfarçadamente ele aguardou o milagre.

E o milagre veio! Um criado estouvadão, que entrava com o peru, tropeçou
no tapete e soltou a ave no colo de uma dama. Gritos, rebuliço, tumulto.
Num lampejo de gênio, Inácio aproveitou"-se do incidente para agarrar o
fígado e metê"-lo no bolso.

Salvo! Nem dona Luísa nem os vizinhos perceberam o truque --- e o jantar
chegou à sobremesa sem maior novidade.

Antes da dançata lembrou alguém recitativos e a espevitadíssima Miloca
veio ter com Inácio.

--- A festa é sua, doutor. Nós queremos ouvi"-lo. Dizem que recita
admiravelmente. Vamos, um sonetinho de Bilac. Não sabe? Olhe o luxinho!
Vamos, vamos! Repare quem está no piano. \emph{Ela}\ldots{} Nem assim?
Mauzinho!\ldots{} Quer decerto que a Sinharinha insista?\ldots{} Ora, até que
enfim! A \emph{Douda de Albano}? Conheço, sim, é linda, embora um pouco
fora da moda. Toque a \emph{Dalila}, Sinharinha, bem \emph{piano}\ldots{}
assim\ldots{}

Inácio, vexadíssimo, vermelhíssimo, já em suores, foi para o pé do piano
onde a futura consorte preludiava a \emph{Dalila} em surdina. E declamou
a \emph{Douda de Albano}. Pelo meio dessa hecatombe em verso, ali pela
quarta ou quinta desgraça, uma baga de suor escorrida da testa parou"-lhe
na sobrancelha, comichando qual importuna mosca. Inácio lembrou"-se do
lenço e saca"-o fora. Mas com o lenço vem o fígado, que faz \emph{plaf!}
no chão. Uma tossida forte e um pé plantado sobre a infame víscera,
manobras de instinto, salvam o lance.

Mas desde esse momento a sala começou a observar um extraordinário
fenômeno. Inácio, que tanto se fizera rogar, não queria agora sair do
piano. E mal terminava um recitativo, logo iniciava outro, sem que
ninguém lho pedisse. É que o acorrentava àquele posto, novo Prometeu, o
implacável fígado\ldots{}

Inácio recitava. Recitou, sem música, o \emph{Navio negreiro}, \emph{As
duas ilhas}, \emph{Vozes da África}, \emph{O Tejo era sereno}.

Sinharinha, desconfiada, abandonou o piano. Inácio, firme. Recitou
\emph{O corvo} de Edgar Poe, traduzido pelo senhor João Kopke; recitou
\emph{Quisera amar"-te}, o \emph{Acorda donzela}; borbotou poemetos,
modinhas e quadras.

Num canto da sala Sinharinha estava chora"-não"-chora. Todos se
entreolhavam. Teria enlouquecido o moço?

Inácio, firme. Completamente fora de si (era a quarta vez que isso lhe
acontecia naquela festa) e, falto já de recitativos de salão, recorreu
aos \emph{Lusíadas}. E declamou \emph{As armas e os barões},
\emph{Estavas linda Inês}, \emph{Do reino a rédea leve}, o
\emph{Adamastor} --- tudo!\ldots{}

E esgotado Camões ia"-lhe saindo um ``ponto'' de Filosofia do Direito ---
\emph{A escola de Bentham} ---, a coisa última que lhe restava de cor na
memória, quando perdeu o equilíbrio, escorregou e caiu, patenteando aos
olhos arregalados da sala a infamérrima víscera de má morte\ldots{}

O resto não vale a pena contar. Basta que saibam que o amor de
Sinharinha morreu nesse dia; que a conspiração matrimonial falhou; e que
Inácio teve de mudar de terra. Mudou de terra porque o desalmado major
Lemos deu de espalhar pela cidade inteira que Inácio era, sem dúvida, um
bom rapaz, mas com um grave defeito: quando gostava de um prato não se
contentava de comer e repetir --- ainda levava escondido no bolso o que
podia\ldots{}

\chapter{O pito do reverendo\footnote[*]{Texto de 1906, publicado no livro \emph{Cidades mortas}.}\,\footnote[**]{Este conto foi publicado
  na \emph{Revista do Brasil}, n. 42, de junho de 1919, com o título
  ``Gramática viva'' e o subtítulo ``De como se formam locuções
  familiares''. Nota da edição de 1955.}}

Itaoca é uma grande família com presunção de cidade, espremida entre
montanhas, lá nos confins do Judas, precisamente no ponto onde o demo
perdeu as botas. Tão isolada vive do resto do mundo que escapam à
compreensão dos forasteiros muitas palavras e locuções de uso local,
puros itaoquismos. Entre eles este, que seriamente impressionou um
gramático em trânsito por ali: \emph{Maria, dá cá o pito!}

Usado em sentido pejorativo para expressar decepção ou pouco"-caso, e
aplicado ao próprio gramático, mal descobriram que ele era apenas isso e
não ``influência política'', como o supunham, descreve"-se aqui o fato
que lhe deu origem. E pede"-se perdão aos gramaticões de má morte pelo
crime de introduzir a anedota na tão sisuda quão circunspecta ciência de
torturar crianças e ensandecer adultos.

O reverendo tomou do estojo os velhos óculos de ouro, encavalgou"-os no
batatão nasal e leu pausadamente a carta do compadre, que dava notícias,
pedia"-as, e comunicava a próxima ida para ali do doutor Emerêncio do
Val, ``nosso ministro em Viena d'Áustria, homem de muito saber e
distinção de maneiras, um desses diplomatas à antiga, como já os não há
nesta república que etc. etc.'', em viagem de recreio pelo interior, a
matar saudades do país.

O reverendo coçou o toitiço com dedos sornas e releu a carta demorando o
pensamento nos trechos que pintavam o alto figurão itinerante, em via de
honrar"-lhe a casa com a sua nobilíssima presença.

Verdade é que dispensava tal honraria, boa seca à pacatez do seu viver
abacial, repartido entre missinhas de cinco mil"-réis (mais um frango),
cachimbadas de muito bom fumo de corda e os pitéus (senão ainda a
ternura, como propalavam as más"-línguas) da ótima caseira e afilhada, a
Maria Prequeté. Culpa toda sua, aliás. Quem lhe mandara a ele possuir a
melhor casa de Itaoca e ser, modéstia à parte, um homem de luzes
notórias, autor de vários acrósticos em latim?

Já doutra feita hospedara um eloquente inspetor agrícola e, logo depois,
o tal sábio que colecionava pedrinhas --- grande falta de serviço! Um
diplomata agora\ldots{} Ahn! A coisa variava\ldots{}

Que viesse, respondeu ao compadre, mas não esperasse encontrar na roça
desses ``confortos e excelências de vida que é de hábito nas grandes
terras''.

Escrita a resposta, foi o reverendo à cozinha conferenciar com a caseira
sobre a hospedagem e longamente confabularam sobre o pato a
sacrificar"-se (se o patão de peito branco ou aquele mais novo com que a
viúva do João das Bichas lhe pagara a missa, a gatuna); sobre a toalha
de mesa e a roupa de cama; sobre o tratamento a dispensar --- Vossa
Excelência, Vossa Senhoria ou Vossa Diplomacia.

Após longo bate"-boca, salpicado de injúrias em calão e algum latim,
assentaram no pato da missa, na toalha de renda e no Vossa Excelência.

Combinadas essas minúcias, uma nuvem de nostalgia ensombrou a nédia cara
do reverendo. Os olhos penduraram"-se"-lhe no vago, saudosos, e de lá só
desciam para envolver, com ternura viciosa, o velho pito de barro que
lhe fedia na mão.

Notou a Prequeté aquelas sombras e:

--- Acorda, boi sonso! Amode que está ervado?\ldots{}

O reverendo abriu"-se. Era o pito. Eram já saudades do velho pito\ldots{} Pois
não ia privar"-se desse amigo de tantos anos durante a estada do
``empata''? Tinha educação. Não desejava impressionar mal a um homem de
raro primor de maneiras. E o pito, se é bom, é também plebeu e, mais que
plebeu, chulo.

Reconhecia"-o, reconhecia"-o\ldots{}

Entretanto, três, quatro dias --- sabia lá a quantos iria a seca? --- de
abstenção forçada, sem que a boca sentisse o bendito contato do saboroso
canudo amarelo de sarro?\ldots{} Doloroso\ldots{}

E o reverendo sorveu com delícia uma baforada maciça. Tragou"-a. Depois,
recostada a cabeça ao espaldar, semicerrados os olhos, semiaberta a
boca, deixou"-se fumegar gostosamente, como piúca de queimada. Coisas
boas da vida!\ldots{}

Mas que remédio? O homem fora diplomata e em Viena d'Áustria!
Confabulara com arquiduques e cardeais. Homem de requintes, portanto.
Era forçoso transigir com o pito, o rico pito, o amor do pito. Sim,
porque a dignidade do clero antes de tudo! Lá isso\ldots{}

Uma semana depois nova carta anunciava que ``o tal das Europas'' em tal
data repontaria por ali.

Grande alvoroço de saia e batina. A Prequeté arregaçou as mangas ---
braços a Machado de Assis tinha a morena! --- e pôs de pernas para o ar
a casa. Varreu, esfregou, escovou tudo, demoliu teias de aranha, limpou
o vidro do lampião, matou o pato e desfez com decoada os muitos pingos
de gema de ovo que constelavam a batina do padrinho.

--- Arre, que até parece uma gemada! --- reguingou ela, entre
repreensiva e caçoísta. Depois, relanceando"-lhe o olhar pelo alto da
cabeça:

--- Chi!\ldots{} A coroa está que é uma tapera! --- exclamou.

E, expedita, \emph{zás! zás!} deu nela uma alimpa de tesoura.

--- E o breviário? --- inquiriu de súbito o padre.

Andava de muito tempo sumido, o raio do livro; procura que procura,
descobrem"-no afinal no quarto dos badulaques, feito calço duma cômoda
capenga. A Prequeté --- maravilhosa caseira! --- com uma dedada de banha
pô"-lo escorreito e envernizado, a fingir com tanta perfeição uso diário
que nem Deus desconfiaria da marosca.

--- Que mais? --- disse ela depois, plantando"-se a distância para uma
vista de conjunto no seu restaurado padrinho. E como de alto a baixo
tudo estivesse a contento: ``Está mesmo \emph{pshut!}'', concluiu,
brejeira, borrifando"-lhe por cima um chuvilho de Água Florida, para
disfarçar o ranço.

Ficou o padre um amor de reverendo, liso e bem amanhado como cônego de
oleografia. Ele próprio o reconheceu ao espelho e, nadando nas delícias
daquele carinho sem par --- e muito agradável a Deus, pois não! ---,
sorriu"-se babosamente, acariciando"-a no queixo:

--- Esta marota!

Conclusa a arrumação, da coroa do padre à cozinha, postou"-se a Prequeté
de vigia à janela, indagando os extremos da rua, enquanto o reverendo,
lindo como no dia da sua primeira missa, passeava pela saleta a chupar
as derradeiras cachimbadas.

Súbito:

--- ``Evem'' vindo o \emph{reis}! --- exclamou a atalaia.

O reverendo meteu o pito na gaveta, passou a mão no breviário e
assumindo cara de circunstância rumou para a porta da rua. Instantes
depois defrontava"-o um cavaleiro. O padre correu a segurar"-lhe a rédea e
o estribo.

--- Queira apear"-se vossa excelência, que esta choupana é de vossa
excelência. Sou o padre vigário de Itaoca, humilde servo de vossa
excelência.

O diplomata, como que ressabiado com tão respeitosa acolhida, deixou"-se
descavalgar. Mas sem garbo, esquerdão e reles, como aí um pulha
qualquer.

Entrou.

Trocaram"-se rapapés, palacianos da parte do reverendo, mal achavascados
(quem o diria?) da parte do cortesão que conversara arquiduques e
cardeais. Houve etiquetas revividas, sempre claudicantes do lado
diplomático. Houve cerimônia.

Mas o doutor não era positivamente o que se esperava. Já no físico
desiludia. Em vez duma fina figura de mundano, saíra"-lhes um magrela de
barba recrescida, roupa surrada, chambão e alvar. Enfim, pensou lá
consigo o reverendo, o hábito não faz o monge. Quem sabe, sob aquelas
aparências vulgares e talvez rebuscadas, não luzia o espírito de um
Talleyrand ou as manhas dum Metternich?

Foram para a mesa e no decurso do jantar acentuou"-se a desilusão. O
homem comia com a faca, baforava no copo, chupava os dentes. Um puro pai
da vida.

Observando"-o por cima dos óculos, o reverendo piscava para a caseira,
que, da cozinha, pela fresta da porta, torcia o nariz à pífia excelência
excursionista. Ao trincar o pato, desastre. O doutor deixou cair no chão
um osso, que logo apanhou, muito encalistrado. Depois, às voltas com a
asa do palmípede, falseou"-se"-lhe a faca, resultando espirrar"-lhe à cara
um chuvisco de arroz. A Prequeté por sua vez espirrou lá dentro uma
risadinha de mofa, acompanhada dum mortificante \emph{ché!}\ldots{}

O reverendo entrou"-se de dúvidas. Era lá possível que o doutor Emerêncio
do Val fosse um estupor daqueles?

À sobremesa caiu a conversa sobre a política, e o doutor desmanchou"-se
em bobagens graúdas. Enquanto asneava, o padre ia matutando lá consigo:

``E eu com cerimônias, e eu com bobices, e eu querendo até privar"-me do
pito por amor a um cretino destes! Fumo"-lhe nas ventas e já!''

Nisto veio o café. Enquanto o ingeriam, o doutor entrou a falar de
remédios, farmácias e projetos de estabelecimento.

O reverendo, decifrando o mistério, deteve a xícara no ar.

--- Mas\ldots{} mas então o senhor\ldots{}

--- Sou farmacêutico, e vim estudar a localidade a ver se é possível
montar aqui uma botica. Portei em sua casa porque\ldots{}

O padre mudou de cara.

--- Então não é o doutor Emerêncio, o diplomata?

--- Não tenho diploma, não senhor, sou farmacêutico prático\ldots{}

O padre sorveu dum trago o café e refloriu a cara de todos os sorrisos
da beatitude; desabotoou a batina, atirou com os pés para cima da mesa,
expeliu um suculento arroto de bem"-aventurança e berrou para a cozinha:

--- Maria, dá cá o pito!

\chapter[De como quebrei a cabeça à mulher do Melo]{De como quebrei a cabeça\\ à mulher do Melo\footnote[*]{Texto de 1906, publicado no livro \emph{Cidades mortas}.}}
\hedramarkboth{De como quebrei a cabeça\ldots}{}

--- Olha, esperam"-te hoje em casa para o jantar.

--- Impossível. Não janto fora.

--- Abre uma exceção e vai.

--- Impossível, já disse. Não insistas.

--- Põe de lado a esquisitice e vai.

--- Não é esquisitice, meu caro, é sibaritismo e prudência. Tenho para
mim que comer é uma das boas coisas da vida. Mas comer o que se quer,
como se quer, quando se quer. Gosto, por exemplo, de lombo de porco, mas
a meu modo, assado cá dum jeito que sei. Se o como fora de casa, nunca o
tenho ao sabor do meu paladar. Gosto ainda de comer quando tenho fome.
Detesto o horário forçado, almoço às onze, jantar às seis, haja ou não
apetite. Ora, a não ser em minha casa, onde não tenho horário, raramente
o apetite coincidirá com o momento do bródio. Esta circunstância, aliada
ao fato de ser induzido a comer o que está na mesa e não o que me pede a
veneta, leva"-me a recusar
sistematicamente convites para jantar.

--- Mas, homem de Deus, para tudo há remédio. Farás tu mesmo o cardápio,
darás as receitas e só se porá a mesa à voz do teu apetite.

--- Não. Em tua casa são todos de tal modo amáveis que receio não chegar
à sobremesa sem cometer um homicídio.

--- !!!

--- Nunca te contei o meu rompimento com a família Melo? Éramos
amicíssimos de longos anos e sê"-lo"-íamos até hoje se não fosse a minha
imprudência aceitando um convite para lá jantar em dia de anos da dona
Vidoca.

Havia à mesa umas dez pessoas, todas íntimas, e as filhas, os genros ---
um povaréu. Dona Vidoca, como sabes, é uma criatura excessivamente
amável e nesse dia excedeu"-se. Serviu"-me sopa, ela própria, mas
carregando a mão como se eu fora um frade. Arrepiou"-me aquele
pantagruelismo brutal, mas calei a exasperação e ingeri com paciência
toda a maranha de fios amarelos, boiantes num caldo untuoso. Mal
absorvera a última colherada, a boa senhora, sem consulta prévia, atocha
feijão num prato e passa"-mo.

``--- Não, minha senhora, muito obrigado!

``--- Ora, coma! Deixe"-se de história. Coma feijão que dá sustância.

``Não houve escapatória possível; tive que aceitar o truculento prato de
caroços pretos, coisa que detesto. Olhei para a rodela escura, cor de
chocolate, que se me esparramava pelo prato inteiro sem deixar
transparecer uma nesga sequer da louça branca, enchi"-me de resignação e
empreendi o trabalho de Hércules que era trasladar tudo aquilo para o
estômago. Mas meu sangue começou a esquentar e senti o nó das cóleras
surdas a subir"-me à garganta.

Estava eu em meio da empreitada, quando vi a excelente senhora dirigir
para o meu prato um enorme naco de carne fisgado no garfo.

``--- Doutor, um \emph{pedacinho} de carne assada?

``Gaguejei, mal firme nas estribeiras:

``--- Mas, minha senhora, eu\ldots{}

``--- Sempre com cerimônias! Olhe que aqui não se usa disso! Coma lá!

``E soltou"-me no prato o boi\ldots{}

``Senti bagas de suor frio borbulharem"-me na testa. O nó da garganta
engrossou. Baixei a cabeça, resignado, e encetei silenciosamente a
mastigação, matutando sobre o modo de dar cabo daquilo. Comer tudo era
impossível; deixar no prato, impolidez\ldots{}

``--- Agora um pouco de arroz!

``Lancei um olhar facinoroso à santa criatura, que o interpretou de
maneira errônea, como de assentimento.

``--- Eu bem vi que estava querendo arroz.

``--- Impossível, dona Vidoca! Peço"-lhe perdão, mas estou satisfeito.
Como pouco e o que tenho no prato janta"-me por três dias.

``--- Luxento! Coma lá!

``E \emph{zás!}, uma, duas, três colheradas, das grandes.

``Uma onda de sangue escureceu"-me a vista. Tive ímpetos de saltar pela
janela. Contive"-me, porém, e com a resignação dos verdadeiros mártires
recomecei a mastigar.

``--- Um pastelzinho agora?

``Era demais! A virtuosa criatura abusava da minha situação. Recusei
desabridamente, áspero.

``--- Já sei por que não quer\ldots{} É que foram feitos por mim\ldots{} Mas deixe
estar\ldots{}

``--- Dona Vidoca! Pelo amor de Deus! --- gaguejei.

``--- Unzinho só! Para me dar opinião sobre o tempero da massa, sim?

Apare lá estezinho tostadinho, sim?

``Conheces o meu gênio, sabes com que facilidade saio fora de mim e
cometo as maiores loucuras. Esse estado de superexcitação nervosa
preludia por um tremor da voz e excessiva quentura nas faces. Naquele
momento, sentindo os pródromos da erupção, entreguei"-me a esforços
sobre"-humanos para conter a fera que mora em mim. E contive"-a. Curvei de
novo a cabeça e levei à boca mais umas garfadas.

``Aqui Melo principia a trinchar o leitão.

``Refleti: se mo oferecem, estouro. E fiquei de sobreaviso, engatilhado
para o revide.

``Não tardou muito que dona Vidoca espetasse no garfo uma alentadíssima
costela de leitão e fizesse pontaria para o meu lado.

``Ah! Perdi a tramontana! Agarrei na garrafa que estava na minha frente
e abri a cabeça da santa criatura com uma pancada horrível!

``De nada mais me lembro. Ouvi um berro, um clamor. Senti o pânico em
redor de mim e corri para a rua como um ébrio. Foi quando\ldots{}''

Não concluiu. O amigo havia abalado.

\section*{Nota da primeira edição (1946)}

Esta história deu origem a curioso incidente. Publicada em julho de
1906, sob o pseudônimo de Antão de Magalhães, no \emph{Minarete}, que
circulava não só em Pinda como nas cidades vizinhas, caiu sob os olhos
de um hoteleiro da cidade de São Bento, de nome Melo e por coincidência
esposo de uma senhora de apelido Vidoca. O excelente homem viu no artigo
alusões pessoais e ofensivas a ele e sua família --- e apresentou
queixa"-crime. Aqui vai a petição, transcrita do \emph{Minarete}:

\emph{Ilmo. e Exmo. Sr. Dr. Juiz de Direito desta Comarca.}

\emph{Diz F. F. Melo, por seu procurador, que, sentindo"-se ofendido, com
sua família, pelo injurioso artigo do} Minarete\emph{, periódico de
imprensa desta cidade, ora junto, distribuído por mais de quinze
pessoas, intitulado ``De como quebrei a cabeça à mulher do Melo'' de 19
de julho de 1906, assinado por Antão de Magalhães, edição nº 159, e
querendo a bem de seus direitos promover a responsabilidade criminal do
autor, que não é pessoa conhecida, pelas injúrias que afetam ao
suplicante e sua família, vem requerer a V. Exa. que se digne mandar
intimar ao editor ou gerente da tipografia do dito periódico, senhor
José Monteiro Salgado, que é quem assumiu a responsabilidade da
publicação do} Minarete \emph{perante a Câmara, preliminarmente, para
exibir em juízo o}

\emph{respectivo autógrafo, em dia, lugar e hora previamente designados,
requerendo também o suplicante a V. Exa. para isso uma audiência
extraordinária, visto ser urgente a diligência etc. etc. Nestes termos,
o suplicante requer que D. e A. esta, com os documentos inclusos, se
proceda na forma da lei, a fim de que, terminadas as diligências, a
exibição do referido autógrafo e pagas as custas do processo, sejam os
autos originais entregues ao procurador do suplicante independente de
traslado, para deles fazer o uso que convier ao suplicante.}

\emph{P. deferimento E. R. M. Pinda, 26 de julho de 1906. Com a proc.
inclusa --- o advogado J. M. F. J.}

O processo não foi por diante, irrisório que era. Apesar disso, a
brincadeira custou ao escamado hoteleiro perto de um conto de réis\ldots{}

\chapter{A vida em Oblivion\footnote[*]{Texto de 1908, publicado no livro \emph{Cidades mortas}.}\,\footnote[**]{Na primeira edição o
  título deste capítulo era ``Coisas do meu diário'', com o subtítulo
  ``Oblivion''. Nota da edição de 1955.}}

\section*{Os três livros}

\noindent{}A cidadezinha onde moro lembra soldado que fraqueasse na marcha e, não
podendo acompanhar o batalhão, à beira do caminho se deixasse ficar,
exausto e só, com os olhos saudosos pousados na nuvem de poeira erguida
além.

Desviou"-se dela a civilização. O telégrafo não a põe à fala com o resto
do mundo, nem as estradas de ferro se lembram de uni"-la à rede por
intermédio de humilde ramalzinho.

O mundo esqueceu Oblivion, que já foi rica e lépida, como os homens
esquecem a atriz famosa logo que se lhe desbota a mocidade. E sua vida
de vovó entrevada, sem netos, sem esperança, é humilde e quieta como a
do urupê escondido no sombrio dos grotões.

Trazem"-lhe os jornais o rumor do mundo, e Oblivion comenta"-o com
discreto parecer. Mas como os jornais vêm apenas para meia dúzia de
pessoas, formam estas a aristocracia mental da cidade. São ``Os Que
Sabem''. Lembra o primado dos Dez de Veneza, esta sabedoria dos Seis de
Oblivion.

Atraídos pelas terras novas, de feracidade sedutora, abandonaram"-na seus
filhos; só permaneceram os de vontade anemiada, débeis, faquirianos.
``Mesmeiros'', que todos os dias fazem as mesmas coisas, dormem o mesmo
sono, sonham os mesmos sonhos, comem as mesmas comidas, comentam os
mesmos assuntos, esperam o mesmo correio, gabam a passada prosperidade,
lamuriam do presente e pitam --- pitam longos cigarrões de palha,
matadores do tempo.

Entre as originalidades de Oblivion uma pede narrativa: o como da sua
educação literária. Promovem"-se três livros venerandos, encardidos pelo
uso, com as capas sujas, consteladas de pingos de vela --- lidos e
relidos que foram em longos serões familiares por sucessivas gerações.
São eles: \emph{La Mare d'Auteuil}, de Paul de Kock, para o uso dos
conhecedores do francês; uns volumes truncados do \emph{Rocambole}, para
enlevo das imaginações femininas; e \emph{Ilha maldita}, de Bernardo
Guimarães, para deleite dos paladares nacionalistas.

O dono primitivo seria talvez algum padre morto sem herdeiros. Depois, à
força de girarem de déu em déu, esses livros forraram"-se à propriedade
individual. Quem, por exemplo, deseja ler o \emph{Rocambole} diz na
rodinha da farmácia:

--- Onde andará o \emph{Rocambole}?

Informam"-no logo, e o candidato toma"-o das mãos do detentor último,
ficando desde esse momento como o seu novo depositário. Processo
sumaríssimo e inteligente.

Quando se esgotou a minha provisão de livros e, ignorante ainda da
riqueza literária da terra, deliberei decorrer ao estoque local,
dirigi"-me a um dos Seis. O homem enfunou"-se de legítimo orgulho ao
dar"-me os informes pedidos.

--- Temos obras de fôlego, poucas mas boas, e para todos os paladares.

Gênero pândego, para divertir, temos, ``por exemplo'', \emph{La Mare
d'Auteuil}, de Paul de Kock. Impagável!

--- Obrigado. De Kock, nem a tuberculina.

--- Temos o célebre \emph{Rocambole}, ``gênero imaginoso''; infelizmente
está incompleto; faltam uns dezessete volumes.

--- Não me serve o resto.

--- E temos uma obra"-prima nacional, a \emph{Ilha maldita}, do ``nosso''
Bernardo Guimarães.

Parando aí o catálogo, era forçoso escolher.

No concerto dos nossos romancistas, onde Alencar é o piano querido das
moças e Macedo a sensaboria relambória dum flautim piegas, Bernardo é a
sanfona. Lê"-lo é ir para o mato, para a roça --- mas uma roça adjetivada
por menina de Sion, onde os prados são \emph{amenos}, os vergéis
\emph{floridos}, os rios \emph{caudalosos}, as matas \emph{viridentes},
os píncaros \emph{altíssimos}, os sabiás \emph{sonorosos}, as rolinhas
\emph{meigas}. Bernardo descreve a natureza como um cego que ouvisse
contar e reproduzisse as paisagens com os qualificativos surrados do mau
contador. Não existe nele o vinco enérgico da impressão pessoal. Vinte
vergéis que descreva são vinte perfeitas e invariáveis amenidades.
Nossas desajeitadíssimas caipiras são sempre lindas morenas cor de
jambo.

Bernardo falsifica o nosso mato. Onde toda a gente vê carrapatos,
pernilongos, espinhos, Bernardo aponta doçuras, insetos maviosos, flores
olentes.

Bernardo mente. Mas como mente menos que o Paul de Kock ou o truculento
Ponson, pai do \emph{Rocambole}, escolhi"-o.

Veio o livro. Volume velho como um monumento egípcio e como ele
revestido de inscrições. Cada leitor que passava ia deixando o rastro
gravado a lápis.

``Li e gostei'', dizia um, ``Li e apreciei'', afirmava certa senhorita.
Inscrição quase em cuneiforme rezava ``Fulano leu e apreciou o talento
do grande escritor brasileiro''. Outro versificava: ``Já foi lido ---
Pelo Walfrido''. Tal moça notara parcimoniosamente: ``Li'' e assinou. Um
amigo da ordem inversa pôs: ``Li e muito gostei''.

Houve quem discordasse. ``Li e não gostei'', declarou um fulano.

O patriotismo literário dum anônimo saiu a campo em prol do autor: ``Os
porcos preferem milho a pérolas'', escreveu ele embaixo.

Monograma complicadíssimo subscrevia isto: ``O \emph{Rocambole} diverte
mais''.

E assim, por quanto espaço em branco tinha o livro, margens ou fins de
capítulo, as apreciações se alastravam com levíssimas variantes ao
sóbrio ``Li e gostei'' inicial. Havia nomes bem antigos, de pessoas
falecidas, e nomes das meninas casadeiras da época.

Os intelectuais de Oblivion bebiam à farta naquela veneranda fonte. Em
Bernardo abeberavam"-se de ``estilo e boa linguagem'', conforme afirmou
um; no \emph{Rocambole} truncado exercitavam os músculos da imaginativa;
e no Paul de Kock, os eleitos, os Sumos (os que sabiam francês!)
fartavam"-se da \emph{grivoiserie} permitida a espíritos superiores.

Essa trindade impressa bastava à educação literária da cidade. Feliz
cidade! Se é de temer o homem que só conhece um livro, a cidade que só
conhece três é de venerar. Veneração, entretanto, que não virá, porque o
mundo desconhece totalmente a pobrezinha da Oblivion\ldots{}

\chapter{Vidinha ociosa\footnote[*]{Texto de 1908, publicado no livro \emph{Cidades mortas}.}}

\section*{APÓLOGO}

\noindent{}O velho Torquato dá relevo ao que conta à força de imagens engraçadas ou
apólogos. Ontem explicava o mal da nossa raça: \emph{preguiça de
pensar}. E restringindo o asserto à classe agrícola:

--- Se o Governo agarrasse um cento de fazendeiros dos mais ilustres e
os trancasse nesta sala, com cem machados naquele canto e uma floresta
virgem ali adiante; e se naquele quarto pusesse uma mesa com papel, pena
e tinta, e lhes dissesse: ``Ou vocês \emph{pensam} meia hora naquele
papel ou botam abaixo aquela mata'', daí a cinco minutos \emph{cento e
um} machados pipocavam nas perobas!\ldots{}

\section*{A MESMICE}

Um coronel inglês suicidou"-se ``\emph{tired of buttoning and
unbuttoning}'' --- cansado de abotoar e desabotoar a farda.

A vida em Oblivion é um perpétuo ``\emph{buttoning and unbuttoning}''
que não desfecha no suicídio.

Salvam"-na a botica e o jogo. A botica, porque nela há uma sessão
permanente de mexerico, e o mexerico é a ambrosia dos lugarejos pobres.
E o jogo, porque quem perdeu não pode suicidar"-se antes da desforra, e
quem ganhou vai alegre, a cantarolar que afinal de contas a vida é boa.
Dessa forma escapam todos ao cansaço da mesmice.

\section*{A FOLHINHA}

A folhinha inventou"-a algum boticário do interior para uso de sua
cidade"-aldeia, onde correm os dias tão iguais e parecidos que só por
meio dela podemos distinguir uma segunda duma terça ou quarta"-feira.

Um só dia tem feição própria: o domingo. Assinala"-o a roupa limpa, a
roupa nova, a roupa preta que surge pelas ruas a tomar sol no corpo de
toda gente.

Redobram de movimento as praças. Caras novas de gente extramuros dão
ares de sua graça. Há mercado cedo, missas até as onze; depois, pelo
resto da tarde, continuam a assinalar o Dia do Senhor caboclos e negros
encachaçados, aglomerados pelas vendas. Vendem elas mais pinga nesse dia
do que durante a semana inteira. Todos voltam para casa mais ou menos
chumbeados. Os ``de cair'' dormem na cidade. Os de pinga exaltada, no
xadrez. E assim transcorre o belo domingo sem necessidade de irmos à
folhinha para sabermos que dia é.

\section*{TOURADAS}

Transformaram o antigo velódromo em circo de touros; metade das
arquibancadas virou \emph{Sombra}, a mil"-réis; e a outra metade,
\emph{Sol}, a quinhentos. Num camarote enfeitado de cetim amarelo e
verde está um \emph{inteligente} pegado a laço e imensamente bronco. Ao
seu lado, um \emph{clarim} tuberculoso; cada vez que sopra na corneta
falta"-lhe fôlego para um som completo --- e o povo ri"-se.

Toureiro de verdade há um, o Antônio Corajoso, empresário, bilheteiro e
assessor do \emph{inteligente}. Mais dois açougueiros vestidos de
\emph{toreros}, com o competente rabicho, completam a \emph{cuadrilla}.

A cada passinho Corajoso berra para o \emph{inteligente}: ``Dê ordem de
recolhida, faça isto, faça aquilo''. E o pobre"-diabo se vê tonto para
conciliar uma burrice inata com os deveres do cargo.

O povo vaia ou aplaude num tom amolecado que é toda a graça da festa.
Reles, mas divertido. ``Feche a boca, negro! Está com fome?'' (isto para
um toureiro mulato). ``Recolham esse canivete aleijado!'' (para um
zebuzinho preto muito magro). ``Hu! hu! Tira leite dessa vaca, ó canudo
de pito!''

Uma farpa fere um boi na veia; o sangue começa a correr. Enternecimento
geral. Para"-se a tourada para remendar"-se o boi. Laçam"-no, cosem"-lhe a
ferida --- operação demorada que consome vinte minutos. Tomado de
piedade, o povo não consente que farpeiem os restantes.

Há palhaço --- um palhaço que faz jus ao cinturão de ouro do
Desenxabimento e da Moleza. Tem preguiça até de andar, preferindo
apanhar marradas a correr. Lá quando a banda de música ataca a valsa
\emph{Amoureuse}, o ladrão atravessa a arena dançando. Mas dança com
tamanha preguiça que o povo rompe num berreiro: ``Lincha o cínico!
Mata!''. E chovem"-lhe em cima toda sorte de desaforos --- e cascas de
pinhão\ldots{}

Remata a festa a ``pantomina'', como diz o programa. Consiste no
\emph{Pançudo}, figura de um cômico prodigioso. Tem tanto de largo como
de alto. Perfeita esfera encimada por uma cabeça e ``embaixada'' por
dois pés. É um homem acolchoado. Mal aparece, em passinhos miúdos e
lentos, uma voz o denuncia: ``É o Zé de Mamã! Aí, negro safado!''. E
toda a gente morre de rir, adivinhando o pobre preto, muito sério, a
suar em bicas dentro da couraça de colchões. O boi investe, marra"-o,
arremessa"-o longe. Os toureiros reerguem"-no. Nova investida, novo
rebolar. E assim até que o touro, desconfiado, se recuse à pagodeira.
Soa por fim o toque de recolher e, todo esburacado, com a palhaça à
mostra, lá vai para os bastidores o pobre Zé de Mamã, rolado qual uma
pipa.

\section*{A ENXADA E O PARAFUSO}

Cada terra com seu uso. O nosso teatrinho sempre usou campainha para as
chamadas. Campainha é eufemismo. Havia lá dentro uma enxada velha,
pendurada de um arame, com um parafuso de cama, cabeçudo, ao lado. Os
sinais eram batidos ali.

Veio um mambembe pernóstico e calou a enxada, substituindo os seus
sonidos por três pancadas no assoalho.

No primeiro dia o povo da plateia entreolhou"-se ao ouvir aquilo, e lá
pelo poleiro houve risadas e assobios. O delegado resolveu intervir.

--- Este mambembe parece que está mangando conosco!

Explicações. O empresário provou que aquele sistema era a última moda de
Paris. Os espectadores remexeram"-se, desconfiados. Estavam nessa
indecisão, quando o major dirimiu a pendenga com o peso de sua
autoridade:

--- Mas isto aqui não é Paris!\ldots{}

--- Bravos! Bravos!

E a velha enxada sonorosa voltou a ser tangida com o parafuso de cabeça.

\section*{RABULICES}

Nos dias de júri reúnem"-se os advogados e rábulas na antessala do
tribunal, os primeiros a virem, os últimos a saírem, como gente que
procura gozar, bem gozado, um ambiente poucas vezes fornecido pelas
circunstâncias. E, como peixes n'água, à vontade, dão trela à comichão
mexeriqueira da rabulice, esquecendo"-se em interminável prosa sobre
processos, atos judiciários, movimento forense, nomeações, negócios
profissionais, pilhérias jurídicas. As cabeças estão abarrotadas de
leis, regulamentos, decretos e fatos jurídicos, de modo a só tomarem
conhecimento das relações entre o fato e a lei escrita, e nunca entre o
fato e a lei natural --- o que é próprio do filósofo. Na natureza só
veem coisas fungíveis, infungíveis, móveis, imóveis, semoventes, bens,
\emph{res}

\emph{nullius}, artigos de \emph{enfiteuse} --- a carne e o osso, enfim,
da propriedade. Essa janelinha que o artista e o filósofo trazem aberta
para a natureza bruta, ou para a humanidade, vistas, uma como turbilhão
de forças em perene esfervilhar, outra como oceano de paixões onde se
debate o \emph{Homo} --- animal filho da natureza, todo ele vegetação
viçosa de instintos irredutíveis ---, o homem de leis abre"-a para a rede
de fios que a Lei trama e destrama, fios que atam os homens entre si ou
à Natureza convertida em \emph{propriedade}.

E toda a maranha velhaca que isso é engloba"-se dentro da mais bela
concepção do idealismo --- a Justiça.

\section*{PÉ NO CHÃO}

Fica no extremo da rua o Grupo Escolar, de modo que a meninada passa e
repassa à frente da minha janela. Notei que muitas crianças sofriam dos
pés, pois traziam um no chão e outro calçado. Perguntei a uma delas:

--- Que doença de pés é essa? Bicho arruinado?

O pequeno baixou a cabeça com acanhamento; depois confessou:

--- É ``inconomia''.

Compreendi. Como nos Grupos não se admitem crianças de pé no chão,
inventaram as mães pobres aquela pia fraude. Um pé vai calçado; o outro,
doente de imaginário mal crônico, vai descalço. Um par de botinas dura
assim por dois. Quando o pé de botina em uso fica estragado,
transfere"-se a doença de um pé para outro, e o pé de botina de reserva
entra em funções. Destarte, guardadas as conveniências, fica o dispêndio
cortado pelo meio. Acata"-se a lei e guarda"-se o cobre.

Benditas sejam as mães engenhosas!

\section*{BARQUINHA DE PAPEL}

Quando chove, logo que passa o aguaceiro e o enxurro transforma a rua
num sistema de rios e riachos lamacentos, começam a derivar barquinhas
de papel. A casa do Joaquim, o moleque"-chefe da rua, vira estaleiro.
Saem de lá as grandes, com bandeirolas. A mocinha de frente também
deita, a medo, a sua; e quem seguir esta barquinha verá o rapaz moreno,
que mora na outra esquina e está à janela, correr à sarjeta, apanhá"-la e
ler risonho a mensagem a lápis da sua namorada\ldots{}

\section*{O HEREGE}

Os filhos do capitão Zarico brincam todos os dias debaixo da minha
janela. É a ciranda, é o pegador, é a senhora pastora. A preta Esméria
fica o tempo todo com o caçula ao colo, vigiando"-os. Ainda hoje estava
lá, às voltas com o pequerrucho.

--- Quem tirou o toucinho daqui?

--- Foi o gato.

--- Que é do gato?

--- Está no mato.

--- Que é do mato?

--- O fogo queimou.

--- Que é do fogo?

--- A água apagou.

--- Que é da água?

--- O boi bebeu.

--- Que é do boi?

--- Está dizendo missa\ldots{}

--- Credo! --- resmungou a preta. --- Tão pequenino e já herege como o
pai\ldots{}

\section*{JUQUITA}

É Juquita o terror da bicharia miúda. Cães e gatos conhecem"-no de longe.
Esta manhã encontrei"-o a brincar com um sanhaço semimorto que, de
repente, não se sabe como, sumiu. O menino procurava"-o quando passei.

--- Não viu o meu sanhaço? --- perguntou"-me.

--- Com certeza algum gato o pegou --- sugeri.

--- Gato! --- e Juquita riu"-se com a maior comiseração da minha
ingenuidade. --- Não há gato que tenha coragem de chegar perto de mim.

\section*{O JESUÍNO}

Quando os juízes de fato se fecham (ou são fechados) na sala secreta,
ficam à porta de guarda os dois oficiais de justiça. O único
interessante é o Jesuíno, mulato velhusco, grandalhão, lento no falar
como um carro de boi ladeira acima.

Desfila o seu rosário de aventuras, onde ele sempre trunfa às avessas.
Tem absorvido muita pancada, e até cargas de chumbo. Como é homem da
lei, não reage senão por meio da lei. É comezinho ir citar um caboclo na
roça, ser hospedado a guatambu e vir dar conta ao juiz da façanha com
vergões pelo corpo, galos na testa, e às vezes descadeirado. Considera a
pancada um osso do ofício. Conta de um soco tão violento que o derribou
a duas braças de distância. Como os valentões exageram as proezas,
Jesuíno exagera os martírios que padeceu a bem da lei.

Isso no fundo é ganância de gorjetas. A parte por amor da qual levou
pancada paga"-lhe os galos.

Mas nesse caso do soco há um apêndice --- para os colegas, onde não há
de vir gorjeta. Conta que mal se ergueu, meio tonto, e se aprumou, o
escachameirinho veio"-lhe para cima de porrete e o desancou sem dó. Mas
ele afinal atracou"-se ao bicho e conseguiu ferrar"-lhe as munhecas no
gasnete. Deitou o ``sojeito'' no chão, socou um joelho na boca do
estômago, e leu"-lhe na cara o mandado. Só não disse com que mão tirou do
bolso o papel (pois as duas estavam ferradas no pescoço do intimado).
Mas é pormenor sem importância esse. Depois fugiu a cavalo. Diz que a
arma do oficial de justiça é a pena. O ``sojeito'' puxa pela garrucha; o
oficial puxa da pena, tira o papel do bolso, e --- Espere aí! Vá
berrando e pregando tiros enquanto eu escrevo; vamos a ver quem pode
mais!

Carlyle esqueceu de incluir no seu livro famoso esta categoria do herói
obscuro da intimação judicial.

Para realce da sua grandeza de alma, contraposta à ferócia do
``sojeito'', Jesuíno conta como este lhe apareceu no dia seguinte ao
pega. Jesuíno disse consigo: ``Vou mostrar como se recebe um inimigo com
civilização''. Fê"-lo entrar, mandou vir café e não tocou na sova. A
folhas tantas o homem quis explicar a sua loucura da véspera. Jesuíno
interrompeu: ``Eu nada tenho contra o senhor, porque o senhor agravou e
esbofeteou mas foi o doutor Juiz e é com ele que tem de avir"-se''.

Com esta sutileza vai traspassando ao meritíssimo a bordoeira velha ---
porque afinal, como ``homem'', nunca levou pancada. ``Queria só ver esse
peitudo que erguesse a mão para mim! Ia parar no inferno!''

\chapter{O colocador de pronomes\footnote[*]{Texto de 1920, publicado no livro \emph{Negrinha}.}\,\footnote[**]{Na primeira edição, este conto era encerrado
  com a seguinte nota: ``Do espólio de Aldrovando Cantagalo faziam parte
  numerosos originais de obras inéditas, entre os quais citaremos:
  \emph{O acento circunflexo} --- 3 volumes; \emph{A vírgula no
  hebraico} --- 5 volumes; \emph{A crase} --- 10 volumes. Pesaram todos,
  por junto, 4 arrobas que renderam, vendidos a 3 tostões o quilo, 18
  mil réis''. Nota da edição de 1946.}}

Aldrovando Cantagalo veio ao mundo em virtude dum erro de gramática.

Durante sessenta anos de vida terrena pererecou como um peru em cima da
gramática.

E morreu, afinal, vítima dum novo erro de gramática.

Mártir da gramática, fique este documento da sua vida como pedra angular
para uma futura e bem merecida canonização.

Havia em Itaoca um pobre moço que definhava de tédio no fundo de um
cartório. Escrevente. Vinte e três anos. Magro. Ar um tanto palerma.
Ledor de versos lacrimogêneos e pai duns acrósticos dados à luz no
\emph{Itaoquense}, com bastante sucesso.

Vivia em paz com as suas certidões quando o frechou venenosa seta de
Cupido. Objeto amado: a filha mais moça do coronel Triburtino, o qual
tinha duas, essa Laurinha, do escrevente, então nos dezessete, e a Do
Carmo, encalhe da família, vesga, madurota, histérica, manca da perna
esquerda e um tanto aluada.

Triburtino não era homem de brincadeiras. Esgoelara um vereador
oposicionista em plena sessão da Câmara e desde aí se transformou no
tutu da terra. Toda gente lhe tinha um vago medo; mas o amor, que é mais
forte que a morte, não receia sobrecenhos enfarruscados nem tufos de
cabelos no nariz.

Ousou o escrevente namorar"-lhe a filha, apesar da distância hierárquica
que os separava. Namoro à moda velha, já se vê, pois que nesse tempo não
existia a gostosura dos cinemas. Encontros na igreja, à missa, troca de
olhares, diálogos de flores --- o que havia de inocente e puro. Depois,
roupa nova, ponta de lenço de seda a entremostrar"-se no bolsinho de cima
e medição de passos na rua dela, nos dias de folga. Depois, a serenata
fatal à esquina, com o
\emph{Acorda, donzela\ldots{}}
sapecado a medo num velho pinho de empréstimo. Depois, bilhetinho
perfumado.

Aqui se estrepou\ldots{}

Escrevera nesse bihetinho, entretanto, apenas quatro palavras, afora
pontos exclamativos e reticências:

\emph{Anjo adorado!}

\emph{Amo"-lhe!}

\begin{quote}
\emph{P\ldots{}}
\end{quote}

Para abrir o jogo bastava esse movimento de peão.

Ora, aconteceu que o pai do anjo apanhou o bilhetinho celestial e,
depois de três dias de sobrecenho carregado, mandou chamá"-lo à sua
presença, com disfarce de pretexto --- para umas certidõezinhas,
explicou.

Apesar disso o moço veio um tanto ressabiado, com a pulga atrás da
orelha.

Não lhe erravam os pressentimentos. Mal o pilhou portas aquém, o coronel
trancou o escritório, fechou a carranca e disse:

--- A família Triburtino de Mendonça é a mais honrada desta terra, e eu,
seu chefe natural, não permitirei nunca --- nunca, ouviu? --- que contra
ela se cometa o menor deslize.

Parou. Abriu uma gaveta. Tirou de dentro o bilhetinho cor"-de"-rosa,
desdobrou"-o.

--- É sua esta peça de flagrante delito?

O escrevente, a tremer, balbuciou medrosa confirmação.

--- Muito bem! --- continuou o coronel em tom mais sereno. --- Ama,
então, minha filha e tem a audácia de o declarar\ldots{} Pois agora\ldots{}

O escrevente, por instinto, ergueu o braço para defender a cabeça e
relanceou os olhos para a rua, sondando uma retirada estratégica.

--- \ldots{} é casar! --- concluiu de improviso o vingativo pai.

O escrevente ressuscitou. Abriu os olhos e a boca, num pasmo. Depois,
tornando a si, comoveu"-se e com lágrimas nos olhos disse, gaguejante:

--- Beijo"-lhe as mãos, coronel! Nunca imaginei tanta generosidade em
peito humano! Agora vejo com que injustiça o julgam aí fora!\ldots{}

Velhacamente o velho cortou"-lhe o fio das expansões.

--- Nada de frases, moço, vamos ao que serve: declaro"-o solenemente
noivo de minha filha!

E, voltando"-se para dentro, gritou:

--- Do Carmo! Venha abraçar o teu noivo!

O escrevente piscou seis vezes e, enchendo"-se de coragem, corrigiu o
erro.

--- Laurinha, quer o coronel dizer\ldots{}

O velho fechou de novo a carranca.

--- Sei onde trago o nariz, moço. Vassuncê mandou este bilhete à
Laurinha dizendo que ama"-``lhe''. Se amasse a ela deveria dizer
amo"-``te''. Dizendo ``amo"-lhe'' declara que ama a uma terceira pessoa, a
qual não pode ser senão a Maria do Carmo. Salvo se declara amor à minha
mulher\ldots{}

--- Oh, coronel\ldots{}

--- \ldots{} ou à preta Luzia, cozinheira. Escolha!

O escrevente, vencido, derrubou a cabeça, com uma lágrima a escorrer
rumo à asa do nariz. Silenciaram ambos, em pausa de tragédia. Por fim o
coronel, batendo"-lhe no ombro paternalmente, repetiu a boa lição da sua
gramática matrimonial.

--- Os pronomes, como sabe, são três: da primeira pessoa --- quem fala,
e neste caso vassuncê; da segunda pessoa --- a quem se fala, e neste
caso Laurinha; da terceira pessoa --- de quem se fala, e neste caso Do
Carmo, minha mulher ou a preta. Escolha!

Não havia fuga possível.

O escrevente ergueu os olhos e viu Do Carmo que entrava, muito lampeira
da vida, torcendo acanhada a ponta do avental. Viu também sobre a
secretária uma garrucha com espoleta nova ao alcance do maquiavélico
pai. Submeteu"-se e abraçou a urucaca, enquanto o velho, estendendo as
mãos, dizia teatralmente:

--- Deus vos abençoe, meus filhos!

No mês seguinte, solenemente, o moço casava"-se com o encalhe, e onze
meses depois vagia nas mãos da parteira o futuro professor Aldrovando, o
conspícuo sabedor da língua que durante cinquenta anos a fio coçaria na
gramática a sua incurável sarna filológica.

Até aos dez anos não revelou Aldrovando pinta nenhuma. Menino vulgar,
tossiu a coqueluche em tempo próprio, teve o sarampo da praxe, mais a
caxumba e a catapora. Mais tarde, no colégio, enquanto os outros enchiam
as horas de estudo com invenções de matar o tempo --- empalamento de
moscas e moidelas das respectivas cabecinhas entre duas folhas de papel,
coisa de ver o desenho que sai ---, Aldrovando apalpava com erótica
emoção a gramática de Augusto Freire da Silva. Era o latejar do
furúnculo filológico que o determinaria na vida, para matá"-lo, afinal\ldots{}

Deixemo"-lo, porém, evoluir e tomemo"-lo quando nos serve, aos quarenta
anos, já a descer o morro, arcado ao peso da ciência e combalido de
rins. Lá está ele em seu gabinete de trabalho, fossando à luz dum
lampião os pronomes de Filinto Elísio. Corcovado, magro, seco, óculos de
latão no nariz, careca, celibatário impenitente, dez horas de aulas por
dia, duzentos mil"-réis por mês e o rim volta e meia a fazer"-se lembrado.

Já leu tudo. Sua vida foi sempre o mesmo poento idílio com as veneráveis
costaneiras onde cabeceiam os clássicos lusitanos. Versou"-os um por um
com mão diurna e noturna. Sabe"-os de cor, conhece"-os pela morrinha,
distingue pelo faro uma seca de Lucena duma esfalfa de Rodrigues Lobo.
Digeriu todas as patranhas de Fernão Mendes Pinto. Obstruiu"-se da broa
encruada de frei Pantaleão do Aveiro. Na idade em que os rapazes correm
atrás das raparigas, Aldrovando escabichava belchiores na pista dos mais
esquecidos mestres da boa arte de maçar. Nunca dormiu entre braços de
mulher. A mulher e o amor --- mundo, diabo e carne eram para ele os
alfarrábios freiráticos do quinhentismo, em cuja soporosa verborreia
espapaçava os instintos lerdos, como porco em lameiro.

Em certa época viveu três anos acampado em Vieira. Depois vagamundeou,
como um Robinson, pelas florestas de Bernardes.

Aldrovando nada sabia do mundo atual. Desprezava a natureza, negava o
presente. Passarinho, conhecia um só: o rouxinol de Bernardim Ribeiro. E
se acaso o sabiá de Gonçalves Dias vinha bicar ``pomos de Hespérides''
na laranjeira do seu quintal, Aldrovando esfogueteava"-o com apóstrofes:

--- Salta fora, regionalismo de má sonância!

A língua lusa era"-lhe um tabu sagrado que atingira a perfeição com frei
Luís de Sousa, e daí para cá, salvo lucilações esporádicas, vinha
chafurdando no ingranzéu barbaresco.

--- A ingresia de hoje --- declamava ele --- está para a Língua como o
cadáver em putrefação está para o corpo vivo.

E suspirava, condoído dos nossos destinos:

--- Povo sem língua!\ldots{} Não me sorri o futuro de Vera Cruz\ldots{}

E não lhe objetassem que a língua é organismo vivo e que a temos a
evoluir na boca do povo.

--- Língua? Chama você língua à garabulha bordalenga que estampa
periódicos? Cá está um desses galicígrafos. Deletreemo"-lo ao acaso.

E, baixando as cangalhas, lia:

--- \emph{Teve lugar ontem}\ldots{} É língua esta espurcícia negral? Ó meu
seráfico frei Luís, como te conspurcam o divino idioma estes sarrafaçais
da moxinifada!

--- \ldots{} \emph{no Trianon}\ldots{} Por quê, Trianon? Por que este perene
barbarizar com alienígenos arrevesos? Tão bem ficava --- a
\emph{Benfíca}, ou, se querem neologismo de bom cunho --- o
\emph{Logratório}\ldots{} Tarelos é que são, tarelos!

E suspirava deveras compungido.

--- Inútil prosseguir. A folha inteira cacografa"-se por este teor. Ai!
Onde param as boas letras de antanho? Fez"-se peru o níveo cisne. Ninguém
atende à lei suma --- Horácio! Impera o desprimor, e o mau gosto vige
como suprema regra. A gálica intrujice é maré sem vazante. Quando
penetro num livreiro o coração se me confrange ante o pélago de óperas
barbarescas que nos vertem cá mercadores de má morte. E é de notar,
outrossim, que a elas se vão as preferências do vulgacho. Muito não faz
que vi com estes olhos um gentil mancebo preferir uma sordícia de Oitavo
Mirbelo, \emph{Canhenho duma dama de servir},\footnote{Octave Mirbeau,
  \emph{Journal d'une Femme de Chambre}. Nota da edição de 1946.} creio,
à\ldots{} adivinhe ao quê, amigo? À \emph{Carta de Guia} do meu divino
Francisco Manuel!\ldots{}

--- Mas a evolução\ldots{}

--- Basta. Conheço às sobejas a escolástica da época, a ``evolução''
darwínica, os vocábulos macacos --- pitecofonemas que ``evolveram'',
perderam o pelo e se vestem hoje à moda de França, com vidro no olho.
Por amor a frei Luís, que ali daquela costaneira escandalizado nos ouve,
não remanche o amigo na esquipática sesquipedalice.

Um biógrafo ao molde clássico separaria a vida de Aldrovando em duas
fases distintas: a estática, em que apenas acumulou ciência, e a
dinâmica, em que, transfeito em apóstolo, veio a campo com todas as
armas para contrabater o monstro da corrupção.

Abriu campanha com memorável ofício ao Congresso, pedindo leis
repressivas contra os ácaros do idioma.

--- Leis, senhores, leis de Dracão, que diques sejam, e fossados, e
alcaçares de granito prepostos à defensão do idioma. Mister sendo, a
forca se restaure, que mais o baraço merece quem conspurca o sacro
patrimônio da sã vernaculidade, que quem o semelhante a vida tira. Vêde,
senhores, os pronomes, em que lazeira jazem\ldots{}

Os pronomes, ai!, eram a tortura permanente do professor Aldrovando.
Doía"-lhe como punhalada vê"-los por aí pre ou pospostos contra regras
elementares do dizer castiço. E sua representação alargou"-se nesse
pormenor, flagelante, concitando os pais da pátria à criação dum Santo
Ofício gramatical.

Os ignaros congressistas, porém, riram"-se da memória e grandemente
piaram sobre Aldrovando as mais cruéis chalaças.

--- Quer que instituamos patíbulo para os maus colocadores de pronomes!
Isto seria autocondenar"-nos à morte! Tinha graça!

Também lhe foi à pele a imprensa, com pilhérias soezes. E depois, o
público. Ninguém alcançara a nobreza do seu gesto, e Aldrovando, com a
mortificação na alma, teve que mudar de rumo. Planeou recorrer ao
púlpito dos jornais. Para isso mister foi, antes de nada, vencer o seu
velho engulho pelos ``galicígrafos de papel e graxa''. Transigiu e,
breve, desses ``pulmões da pública opinião'' apostrofou o país com o
verbo tonante de Ezequiel. Encheu colunas e colunas de objurgatórias
ultraviolentas, escritas no mais estreme vernáculo.

Mas não foi entendido. Raro leitor metia os dentes naqueles
intermináveis períodos engrenados à moda de Lucena; e ao cabo da
aspérrima campanha viu que pregara em pleno deserto. Leram"-no apenas a
meia dúzia de Aldrovandos que vegetam sempre em toda parte, como notas
rezinguentas da sinfonia universal.

A massa dos leitores, entretanto, essa permaneceu alheia aos flamívomos
pelouros da sua colubrina sem raia. E por fim os ``periódicos''
fecharam"-lhe a porta no nariz, alegando falta de espaço e coisas.

--- Espaço não há para as sãs ideias --- objurgou o enxotado ---, mas
sobeja, e pressuroso, para quanto recende à podriqueira!\ldots{} Gomorra!
Sodoma! Fogos do céu virão um dia alimpar"-vos a gafa!\ldots{} --- exclamou,
profético, sacudindo à soleira da redação o pó das cambaias botinas de
elástico.

Tentou em seguida ação mais direta, abrindo consultório gramatical.

--- Têm"-nos os físicos (queria dizer médicos), os doutores em leis, os
charlatas de toda espécie. Abra"-se um para a medicação da grande
enferma, a língua. Gratuito, já se vê, que me não move amor de bens
terrenos.

Falhou a nova tentativa. Apenas moscas vagabundas vinham esvoejar na
salinha modesta do apóstolo. Criatura humana nem uma só lá apareceu a
fim de emendar"-se filologicamente.

Ele, todavia, não esmoreceu.

--- Experimentemos processo outro, mais suasório.

E anunciou a montagem da ``Agência de Colocação de Pronomes e Reparos
Estilísticos''.

Quem tivesse um autógrafo a rever, um memorial a expungir de cincas, um
calhamaço a compor"-se com os ``afeites'' do lídimo vernáculo, fosse lá
que, sem remuneração nenhuma, nele se faria obra limpa e escorreita.

Era boa a ideia, e logo vieram os primeiros originais necessitados de
ortopedia, sonetos a consertar pés de versos, ofícios ao Governo pedindo
concessões, cartas de amor.

Tais, porém, eram as reformas que nos doentes operava Aldrovando, que os
autores não mais reconheciam suas próprias obras. Um dos clientes chegou
a reclamar.

--- Professor, vossa senhoria enganou"-se. Pedi limpa de enxada nos
pronomes, mas não que me traduzisse a memória em latim\ldots{}

Aldrovando ergueu os óculos para a testa:

--- E traduzi em latim o tal ingranzéu?

--- Em latim ou grego, pois que o não consigo entender\ldots{}

Aldrovando empertigou"-se.

--- Pois, amigo, errou de porta. Seu caso é ali com o alveitar da
esquina.

Pouco durou a Agência, morta à míngua de clientes. Teimava o povo em
permanecer empapado no chafurdeiro da corrupção\ldots{}

O rosário de insucessos, entretanto, em vez de desalentar exasperava o
apóstolo.

--- Hei de influir na minha época. Aos tarelos hei de vencer. Fogem"-me à
férula os maraus de pau e corda? Ir"-lhes"-ei empós, filá"-los"-ei pela
gorja!\ldots{} Salta rumor!

E foi"-lhes ``empós''. Andou pelas ruas examinando dísticos e tabuletas
com vícios de língua. Descoberta e ``asnidade'', ia ter com o
proprietário, contra ele desfechando os melhores argumentos catequistas.

Foi assim com o ferreiro da esquina, em cujo portão de tenda uma
tabuleta --- ``Ferra"-se cavalos'' --- escoicinhava a santa gramática.

--- Amigo --- disse"-lhe pachorrentamente Aldrovando ---, natural a mim
me parece que erres, alarve que és. Se erram paredros, nesta época de
ouro da corrupção\ldots{}

O ferreiro pôs de lado o malho e entreabriu a boca.

--- Mas da boa sombra do teu focinho espero --- continuou o apóstolo ---
que ouvidos me darás. Naquela tábua um dislate existe que seriamente à
língua lusa ofende. Venho pedir"-te, em nome do asseio gramatical, que o
expunjas.

--- ???

--- Que reformes a tabuleta, digo.

--- Reformar a tabuleta? Uma tabuleta nova, com a licença paga? Estará
acaso rachada?

--- Fisicamente, não. A racha é na sintaxe. Fogem ali os dizeres à sã
gramaticalidade.

O honesto ferreiro não entendia nada de nada.

--- Macacos me lambam se estou entendendo o que vossa senhoria diz\ldots{}

--- Digo que está a forma verbal com eiva grave. O ``ferra"-se'' tem que
cair no plural, pois que a forma é passiva e o sujeito é ``cavalos''.

O ferreiro abriu o resto da boca.

--- O sujeito sendo ``cavalos'' --- continuou o mestre ---, a forma
verbal é ``ferram"-se'' --- ``ferram"-se cavalos!''

--- Ahn! --- respondeu o ferreiro ---, começo agora a compreender. Diz
vossa senhoria que\ldots{}

--- \ldots{} que ``ferra"-se cavalos'' é um solecismo horrendo e o certo é
``ferram"-se cavalos''.

--- Vossa senhoria me perdoe, mas o sujeito que ferra os cavalos sou eu,
e eu não sou plural. Aquele ``se'' da tabuleta refere"-se cá a este seu
criado. É como quem diz: Serafim ferra cavalos --- Ferra Serafim
cavalos. Para economizar tinta e tábua abreviaram o meu nome, e ficou
como está: Ferra Se (rafim) cavalos. --- Isto me explicou o pintor, e
entendi"-o muito bem.

Aldrovando ergueu os olhos para o céu e suspirou.

--- Ferras cavalos e bem merecias que te fizessem eles o mesmo!\ldots{} Mas
não discutamos. Ofereço"-te dez mil"-réis pela admissão dum ``m'' ali\ldots{}

--- Se vossa senhoria paga\ldots{}

Bem empregado dinheiro! A tabuleta surgiu no dia seguinte
dessolecismada, perfeitamente de acordo com as boas regras da gramática.
Era a primeira vitória obtida e todas as tardes Aldrovando passava por
lá para gozar"-se dela.

Por mal seu, porém, não durou muito o regalo. Coincidindo a entronização
do ``m'' com maus negócios na oficina, o supersticioso ferreiro atribuiu
a macaca alteração dos dizeres e lá raspou o ``m'' do professor.

A cara que Aldrovando fez quando no passeio desse dia deu com a vitória
borrada! Entrou furioso pela oficina adentro, e mascava uma apóstrofe de
fulminar quando o ferreiro, às brutas, lhe barrou o passo.

--- Chega de caraminholas, ó barata tonta! Quem manda aqui, no serviço e
na língua, sou eu. E é ir andando, antes que eu o ferre com um bom par
de ferros ingleses!

O mártir da língua meteu a gramática entre as pernas e moscou"-se.

``\emph{Sancta simplicitas}!'', ouviram"-no murmurar na rua, de rumo à
casa, em busca das consolações seráficas de frei Heitor Pinto. Chegado
que foi ao gabinete de trabalho, caiu de borco sobre as costaneiras
venerandas e não mais conteve as lágrimas, chorou\ldots{}

O mundo estava perdido e os homens, sobremaus, eram impenitentes. Não
havia desviá"-los do ruim caminho, e ele, já velho, com o rim a rezingar,
não se sentia com forças para a continuação da guerra.

--- Não hei de acabar, porém, antes de dar a prelo um grande livro, onde
compendie a muita ciência que hei acumulado.

E Aldrovando empreendeu a realização de um vastíssimo programa de
estudos filológicos. Encabeçaria a série um tratado sobre a colocação
dos pronomes, ponto onde mais claudicava a gente de Gomorra.

Fê"-lo, e foi feliz nesse período de vida em que, alheio ao mundo, todo
se entregou, dia e noite, à obra magnífica. Saiu trabuco volumoso, que
daria três tomos de quinhentas páginas cada um, corpo miúdo. Que
proventos não adviriam dali para a lusitanidade! Todos os casos
resolvidos para sempre, todos os homens de boa vontade salvos da
gafaria! O ponto fraco do brasileiro falar resolvido de vez! Maravilhosa
coisa\ldots{}

Pronto o primeiro tomo --- \emph{Do pronome Se} ---, anunciou a obra
pelos jornais, ficando à espera da chusma de editores que viriam
disputá"-la à sua porta. E por uns dias o apóstolo sonhou as delícias da
estrondosa vitória literária, acrescida de gordos proventos pecuniários.

Calculava em oitenta contos o valor dos direitos autorais, que, generoso
que era, cederia por cinquenta. E cinquenta contos para um velho
celibatário como ele, sem família nem vícios, tinha a significação duma
grande fortuna. Empatados em empréstimos hipotecários, sempre eram seus
quinhentos mil"-réis por mês de renda a pingarem pelo resto da vida na
gavetinha onde, até então, nunca entrara pelega maior de duzentos.
Servia, servia!\ldots{} E Aldrovando, contente, esfregava as mãos, de ouvido
alerta, preparando frases para receber o editor que vinha vindo\ldots{}

Que vinha vindo mas não veio, ai!\ldots{} As semanas se passaram sem que
nenhum representante dessa miserável fauna de judeus surgisse a chatinar
o maravilhoso livro.

--- Não me vêm a mim? Salta rumor! Pois me vou a eles!

E saiu em via"-sacra, a correr todos os editores da cidade.

Má gente! Nenhum lhe quis o livro sob condições nenhumas. Torciam o
nariz, dizendo: ``Não é vendável!'' ou: ``Por que não faz antes uma
cartilha infantil aprovada pelo Governo?''.

Aldrovando, com a morte na alma e o rim dia a dia mais derrancado,
retesou"-se nas últimas resistências.

--- Fá"-la"-ei imprimir à minha custa! Ah!, amigos! Aceito o cartel. Sei
pelejar com todas as armas e irei até ao fim. Boa fé!\ldots{}

Para lutar era mister dinheiro e bem pouco do vilíssimo metal possuía na
arca o alquebrado Aldrovando. Não importa! Faria dinheiro, venderia
móveis, imitaria Bernardo de Pallissy, não morreria sem ter o gosto de
acaçapar Gomorra sob o peso da sua ciência impressa. Editaria ele mesmo
um por um todos os volumes da obra salvadora.

Disse e fez.

Passou esse período de vida alternando revisão de provas com
padecimentos renais. Venceu. O livro compôs"-se, magnificamente revisto,
primoroso na linguagem como não existia igual.

Dedicou"-o a frei Luís de Sousa:

\begin{quote}
À memoria daquele que me sabe as dores,

\textsc{o autor.}
\end{quote}

Mas não quis o destino que o já trêmulo Aldrovando colhesse os frutos de
sua obra. Filho dum pronome impróprio, a má colocação doutro pronome lhe
cortaria o fio da vida.

Muito corretamente havia ele escrito na dedicatória: \ldots{} \emph{daquele
que me sabe}\ldots{} e nem poderia escrever doutro modo um tão conspícuo
colocador de pronomes. Maus fados intervieram, porém --- até os fados
conspiram contra a língua! --- e por artimanha do diabo que os rege
empastelou"-se na oficina esta frase. Vai o tipógrafo e recompõe"-na a seu
modo\ldots{} \emph{daquele que sabe"-me as dores}\ldots{} E assim saiu nos
milheiros de cópias da avultada edição.

Mas não antecipemos.

Pronta a obra e paga, ia Aldrovando recebê"-la, enfim. Que glória!
Construíra, finalmente, o pedestal da sua própria imortalidade, ao lado
direito dos sumos cultores da língua.

A grande ideia do livro, exposta no capítulo \textsc{vi} --- ``Do método
automático de bem colocar os pronomes'' ---, engenhosa aplicação duma
regra mirífica por meio da qual até os burros de carroça poderiam zurrar
com gramática, operaria como o ``914'' da sintaxe, limpando"-a da
avariose produzida pelo espiroqueta da pronominúria.

A excelência dessa regra estava em possuir equivalentes químicos de uso
na farmacopeia alopata, de modo que a um bom laboratório fácil lhe seria
reduzi"-la a ampolas para injeções hipodérmicas, ou a pílulas, pós ou
poções para uso interno.

E quem se injetasse ou engolisse uma pílula do futuro \textsc{pronominol
cantagalo} curar"-se"-ia para sempre do vício, colocando os pronomes
instintivamente bem, tanto no falar como no escrever. Para algum caso de
pronomorreia aguda, evidentemente incurável, haveria o recurso do
\textsc{pronominol nº 2}, onde entrava a estricnina em dose suficiente para
libertar o mundo do infame sujeito.

Que glória! Aldrovando prelibava essas delícias todas quando lhe entrou
casa adentro a primeira carroçada de livros. Dois brutamontes de mangas
arregaçadas empilharam"-nos pelos cantos, em rumas que lá se iam; e
concluso o serviço um deles pediu:

--- Me dá um mata"-bicho, patrão!\ldots{}

Aldrovando severizou o semblante ao ouvir aquele ``Me'' tão fora dos
mancais, e tomando um exemplar da obra ofertou"-a ao ``doente''.

--- Toma lá. O mau bicho que tens no sangue morrerá asinha às mãos deste
vermífugo. Recomendo"-te a leitura do capítulo sexto.

O carroceiro não se fez rogar; saiu com o livro, dizendo ao companheiro:

--- Isto no sebo sempre renderá cinco tostões. Já serve!\ldots{}

Mal se sumiram, Aldrovando abancou"-se à velha mesinha de trabalho e deu
começo à tarefa de lançar dedicatórias num certo número de exemplares
destinados à crítica. Abriu o primeiro, e estava já a escrever o nome de
Rui Barbosa quando seus olhos deram com a horrenda cinca:

``daquele \textsc{que sabe"-me} as dores.''

--- Deus do céu! Será possível?

Era possível. Era fato. Naquele, como em todos os exemplares da edição,
lá estava, no hediondo relevo da dedicatória a frei Luís de Sousa, o
horripilantíssimo --- ``\emph{que sabe"-me}\ldots{}''

Aldrovando não murmurou palavra. De olhos muito abertos, no rosto uma
estranha marca de dor --- dor gramatical inda não descrita nos livros de
patologia ---, permaneceu imóvel uns momentos.

Depois empalideceu. Levou as mãos ao abdômen e estorceu"-se nas garras de
repentina e violentíssima ânsia.

Ergueu os olhos para Frei Luís de Souza e murmurou:

--- Luís! Luís! Lamma Sabachtani!

E morreu.

De que, não sabemos --- nem importa ao caso. O que importa é
proclamarmos aos quatro ventos que com Aldrovando morreu o primeiro
santo da gramática, o mártir número 1 da Colocação dos Pronomes.

Paz à sua alma.

\chapter{O rapto\footnote[*]{Texto de 1923, publicado no livro \emph{O macaco que se fez homem}.}}

Sou oculista.

Dentre tantas especialidades abertas ao anel de pedra verde, barafustei
pela oftalmologia, movido de nobres razões sentimentais. Lutar contra a
noite, arrebatar presas à treva: poderá existir profissão mais
abençoada?

Assim pensei, e jamais me arrependi de o ter pensado. Minha melhor paga
nunca foi o dinheiro ganho em troca dos milagres da faca de De
Graefe,\footnote{Instrumento cirúrgico usado nas operações de catarata.
  Nota da edição de 1946.} senão o êxtase da triste criatura imersa na
escuridão ao ver"-se de súbito restituída à luz.

O oculista, fora dos grandes centros, é um animal andejo. Não pode
estacionar permanentemente no mesmo ponto, a exemplo dos colegas que
curam todas as moléstias conhecidas e \emph{quibusdam aliis}. Encontra
em cada zona um reduzido grupo de clientes, curados os quais, ou
desenganados, força é que se abale de freguesia.

Fiz"-me andejo. Andei de déu em déu, por ceca e meca, desfazendo
cataratas, recompondo nervos ópticos; e se não enriqueci, vale um
tesouro o livro da minha carreira clínica, tão cheio o tenho de
impressões suculentas de psicologia ou pitoresco.

Estampo cá uma delas: o caso do cego de Rio Manso. Não é caso cômico e
não será trágico; duvido, porém, que me apresentem outro mais humano ---
e de tão grande rigor de lógica.

Rio Manso é viloca que os fados plantaram seis léguas além de Itaguaçu,
cidadezinha onde permaneci três meses de consultório aberto. Parti para
Rio Manso --- lembro"-me tão bem! --- bifurcado em aspérrimo sendeiro de
aluguel, avatar evidente do Rocinante, salvo o trote, que o tinha capaz
de desfazer em pandarecos a nobre vestimenta de lata do herói manchego.

Meu Sancho era o Geremário, excelente cabrocha a quem extirpei uma
catarata e que virou desde aí o meu fidelíssimo escudeiro.

Nem eu nem ele conhecíamos o caminho. Não obstante, funcionou Geremário
como perfeita bússola, agudíssimo que é o senso de orientação adquirido
pela gente da roça no traquejo da vida ao ar livre.

A terra é para eles um mapa vivo; e o chão das estradas, um roteiro
luminoso. Conhecem a primor a linguagem dos sinais impressos no solo
vermelho --- sulcos de carros, pegadas de animais, galhos partidos,
restos de fogueirinhas --- e os leem como nós lemos a letra de forma.
Foi assim que o arguto Geremário em certo ponto da viagem murmurou
convictamente, com os olhos postos no caminho:

--- Estamos chegando!

Olhei em redor e nada vi senão a mesma morraria desnuda, as mesmas
samambaias. Nada denunciativo de povoado próximo.

--- Como sabe, se nunca viajou destas bandas?

O meu cabrocha sorriu com malícia e explicou:

--- A estrada está piorando. Estrada ruim, Câmara Municipal perto\ldots{}

De fato, o caminho, bom até ali, principiava a esburacar"-se. Pus"-me a
observar a mudança, rápida transição para pior, até que, dobrada uma
curva, de chofre avistamos as primeiras casas da vila.

--- Não disse? --- exclamou jubiloso o pajem. --- Câmara Municipal é
marca que não nega\ldots{}

Ri"-me por fora, e por dentro admirei a suave ironia daquela agudeza de
altos quilates.

Todos os nossos povoados possuem o mesmo aspecto suburbano --- a mesma
somática, como diria o meu velho professor de patologia, no seu
preciosismo de acadêmico.

A estrada principia de repente a margear"-se de humildes casebres de sapé
e barro, com cercas de bambu atrepadas do melão"-de"-são"-caetano, ou
cercas vivas de pinhão"-do"-paraguai, cactos e outras plantas da zona. Aos
poucos os casebres melhoram. Começam a surgir casas de telha, já
rebocadas, já caiadas; e vendinhas; e tendas de ferradores; e assim vai
em gradação insensível até virar rua, com passeios e espaçados lampiões
de querosene.

Também a categoria social dos moradores acompanha tal ascensão. De
mendigos, de velhos negros capengas, de sórdidas pretas que se espiolham
ao sol --- perfeita varredura humana de entristecedor aspecto ---, a
população passa a jornaleiros, a gente pobre mas arranjadinha, até
chegar à ``gente limpa''. E como a rua, no crescendo em que vai,
desfecha em praça --- o largo da matriz, com gramados, coreto de música
e casas de comércio ---, assim também as ``almas'' sobem do mendigo roto
ao senhor doutor delegado e ao excelentíssimo senhor coronel N. N.,
chefe da política local, semideus, dono e tutu"-marambaia da terra.

Ao entrar em Rio Manso, vencidos os primeiros casebres, chamou"-me a
atenção um berreiro. Em certa casinhola fechada ia rolo velho, surra ou
luta, a avaliar pelos gritos que de lá vinham. Não posso ver dessas
coisas sem intervir.

Parei à porta e com rompante de autoridade dei com a argola do relho.

--- Que é lá isso aí?

O rumor interno cessou, mas ninguém me respondeu. Nisto aproximaram"-se
alguns vizinhos, de mãos no bolso e ar velhaco.

--- Que terra é esta? --- gritei. --- Mata"-se gente dentro das casas e
ninguém se move?\ldots{}

Retrucou"-me um deles:

--- Se a gente fosse se incomodar cada vez que o Bento Cego desce o
guatambu nos filhos\ldots{}

Bento Cego\ldots{} O caso interessava"-me. Pedi informações.

--- É um cego que mora aqui, o Bento. Ele gosta da sua pinguinha. Bebe
às vezes demais, vira valente e mete a lenha nos filhos. Tranca a porta
e é, como diz o outro, pancada de cego!

Fiquei na mesma e, vendo que o sujeito não me adiantava o expediente,
bati de novo na porta com o cabo do relho. Abriu"-ma dessa feita um
rapazinho aí dos seus catorze anos. Interpelei"-o. O menino, a coçar"-se,
olhou para a gente reunida atrás de mim e riu"-se.

--- Bem se vê que o senhor não é daqui. Papai é assim mesmo. Bebe seus
martelinhos e quando esquenta a cabeça o gosto dele é bater. \emph{Nós
deixa}, e até \emph{se diverte} com isso\ldots{}

Assombrei"-me. Um pai cujo gosto é bater na prole e filhos que se
divertem com a surra! Mas como cada roca tem seu fuso e eu não conhecia
o uso daquela terra, não pedi mais --- toquei para o hotel, vivamente
interessado pelo estranho costume daquela família.

Armei tenda em Rio Manso e pus"-me a consertar olhos. Entrementes,
enfronhei"-me na história do Bento Cego. Nascera arranjado, filho dum
fiscal da Câmara, e quando casou morava em casa própria, legada pelo pai
e sita em rua de procissão. Maus negócios fizeram"-no perdê"-la e passar à
rua mais modesta.

Vieram filhos, vieram doenças, macacoas de toda espécie, urucas, e
Bento, a decair mais e mais, foi rolando para pior até acabar cego, à
beira da cidade, na zona da mendicância.

Como e por quê?

Era Bento um triste incapaz. Não prestava para coisa nenhuma. Começasse
por onde começasse, seu destino seria sempre aquele, acabar na rua
chorando esmolas. Bobo em negócios, tinha, entretanto, fumos de esperto.
Piscava o olho a cada transação que fazia, e quando os arregalava via"-se
logrado, tungado, embrulhado, furtado pelos ``passadores de perna''.

Fez"-se barganhista, e jamais a barganha\footnote{Existe pelo interior a
  ``arte da barganha'', em que na troca de um objeto por outro o mais
  esperto ganha uma ``volta'' em dinheiro. Nota da edição de 1946.} lhe
deu o menor lucro. Começou pela casa. Barganhou"-a por outra, muito
inferior, tentado pela ``volta''. Em três meses comeu a ``volta'' e
ficou a nenhum em matéria monetária.

Mas a tentação da ``volta'' não o abandonou mais. Iria barganhando e
comendo as ``voltas'': solução mirífica, pensou ele piscando o olho.

E assim fez.

Casão por casa, casa por casinha, casinha por dois carros e quatro
juntas de bois, os carros por dois cavalos, os dois cavalos por uma
besta de fama que fazia e acontecia e não sei quem dava por ela
oitocentos ``bagos'' --- um negocião, sempre um negocião!

A ciganagem espigatória\footnote{Os que na barganha sabem lograr o
  parceiro ingênuo. Nota da edição de 1946.} viu nele uma perfeita mina
incapaz de resistir ao sésamo ``volta''!

E tantas voltas deram no pisca"-olho, que Bento se viu por fim com toda a
herança paterna reduzida à mula, que não valia nem metade do preço. O
freguês dos oitocentos bagos era fantástico e por muito feliz se deu ele
de passá"-la adiante por duzentos e sessenta mil"-réis, mais uma garrucha
velha de lambuja.

Os filhos, já taludos por esse tempo, saíram ao pai. Nunca frequentaram
escolas, nem queriam saber de trabalho. Não se ``sojeitavam''. Pelas
vendas, à toa pelas ruas, viraram os piores moleques da terra e
transformaram num inferno a casa do Bento. Exigências, brigas diárias,
palavrões imundos e uma lambança das mais sórdidas. E como o pai,
frouxíssimo de caráter, nunca tivesse ânimo de lhes torcer o pepino,
eles acabaram torcendo o pepino ao pai.

Tratavam"-no como alguém trata cachorro, aos pontapés, e por fim, quando
a miséria chegou e faltou um dia feijão à panela, foram às últimas ---
espancaram"-no.

Bento não reagiu.

Reagir como, se eram três e ele não chegava a um? Resignou"-se.
Estimulados por tamanha covardia, entraram os filhos a repetir as doses,
a amiudarem"-nas, até o meterem para ali, num canto, bode expiatório e
armazém de pancadas.

Bento deixou de ser homem. Passou a coisa humana, triste molambo de
carne pensante, tímida, apavorada; desprezado de todos, seu consolo
único era o álcool, em cujo sopor vivia agora imerso.

Tal situação durou até a venda da besta. Aí explodiu. Quando entraram em
casa os duzentos e sessenta mil"-réis, mais a garrucha, Bento anunciou
que ia aplicá"-los num excelente negócio. Fartos de excelentes negócios,
os filhos opuseram"-se. Ele havia que repartir o cobre.

Bento resistiu, retesando as vagas fibras da energia ainda restante em
sua alma. Os filhos quebraram"-lhe a cara com o cabo da garrucha e
fugiram com o dinheiro.

Datou daí a cegueira do homem; do espancamento resultou traumatismo do
nervo óptico e consequente catarata.

Bento passou a mendigo.

Viúvo que era, sem cão em casa, arranjou um cão, um porrete, um negrinho
sarambé para guia e iniciou vida nova.

Como em Rio Manso não existissem cegos, todos se apiedaram dele.
Davam"-lhe roupas velhas, chapéus, mantimentos, dinheiro --- afora
consolações verbais.

Resultou disso que uma relativa abundância veio substituir"-se à miséria
de até então. Chapéus, possuía"-os às dúzias, e de todos os formatos,
inclusive cartola! Calças, paletós e coletes, às pilhas. Até fraques e
uma formosa sobrecasaca de debrum vieram enriquecer"-lhe o guarda"-roupa.

Bento dizia:

--- Deus dá nozes a quem não tem dentes. Agora que é um corpo só na
casa, tanta roupa, até fraque\ldots{}

Mas os filhos marotos cheiraram de longe a reviravolta da fortuna e
bateu"-lhes a pacuera do arrependimento. Hoje um, amanhã outro, vieram os
três, cabisbaixos e humílimos, implorar perdão ao velho.

Que não perdoará um cego, inda mais pai? Bento perdoou"-os e readmitiu"-os
em casa. A esmola sempre farta havia de dar para todos.

E deu. Nunca daí por diante faltou feijão à panela, nem roupa ao corpo,
nem dinheirinho para o resto, inclusive cachaça e fumo.

Milagre! Aquele homem que de olhos perfeitos jamais conseguira coisa
alguma na vida além do desprezo público e da pancada dos filhos recebia
agora provas de carinho, gozava certa consideração, fazia"-se chefe da
casa, respeitado, ouvido --- e até temido!

Acostumou"-se a mandar e a ser obedecido. E não o fizessem! E não o
fizessem depressa! Sua mão, outrora tão frouxa, esmagava agora todas as
resistências. Sua vontade encorpou, enrijou, deitou os galhos da veneta.

Até da viuvez se remendou o Bento. Surgiu logo uma parenta pobre que lhe
escreveu propondo"-se a morar com ele e cuidar da casa. Veio a mulher,
arrumou"-se, deu boa aparência de limpeza e ordem ao tugúrio da lambança
e do desmazelo, fazendo coisa fina, que a toda gente causava pasmo.

Bento chegou a pensar na aquisição da casinha, e para isso foi apartando
cobres.

Mais tarde, novo parente em petição de miséria veio achegar"-se à sua
sombra --- um misantropo que lhe contava lorotas e lia capítulos do
Bertoldo e da história de \emph{Carlos Magno e os doze pares de França.}

Bento era fanático de Roldão e nunca admitiu que fosse lida a segunda
parte do livro, em que Bernardo Del Carpio vence os doze pares.

--- Mentira! Não venceu nada --- dizia ele. --- Veja se um Bernardo,
seja donde diabo for, é lá capaz de aguentar uma só lambada da durindana
de Roldão! Venceu coisa nenhuma\ldots{}

Uma nuvem apenas toldava a paz da família restaurada. Bento bebia, e se
errava a dose, sorvendo a mais um martelo que fosse, esquentava a
cabeça. Aspectos da vida antiga vinham"-lhe então à memória: o caso da
besta, a cena da pancadaria, e Bento, com grande furor, apostrofava os
filhos criminosos. Em seguida castigava"-os. Corria os ferrolhos das
portas e, chispando maldições tremendas, deslombava"-os à cega.

Os filhos suportavam o tratamento sem a mínima reação. Mereciam"-no e,
além disso, era tão gostosa aquela vidinha esmolenga\ldots{}

Foi por essas alturas que cheguei a Rio Manso, e o caso do Bento, que
desde o primeiro dia me interessara à curiosidade, interessou"-me depois
à piedade.

Resolvi curá"-lo.

Examinei"-o e vi que cegara em virtude de catarata de origem traumática,
sob forma de fácil remoção. A faca de De Graefe punha"-o bom em três
tempos.

Propus"-lhe o tratamento.

--- Deus que o abençoe! Que vontade tenho de ver de novo o sol! O sol,
as cores, as gentes\ldots{} Só quem perdeu a vista sabe o que valem os olhos.
Esta noite sem fim\ldots{}

--- Terá fim a tua, meu velho. O caso é simples e tenho a certeza de
pôr"-te sãozinho como dantes. Apronto"-te um quarto em minha casa, donde
só sairás curado.

--- Deus o ouça! Sempre pensei em procurar curar"-me. Mas não havia
médico por aqui, era preciso ir longe, viagem cara\ldots{} Se os ``videntes''
soubessem o que é a cegueira\ldots{}

``Videntes''! Ele chamava videntes aos que enxergam\ldots{}

--- Pois está combinado. Amanhã cedo vais ao meu consultório e amanhã
mesmo te opero. E verás de novo o sol, as flores, o céu\ldots{}

A fisionomia do cego irradiava.

--- Sabe o que mais desejo ver? --- disse revirando nas órbitas os olhos
branquicentos. --- A cara dos meus filhos. Eram tão maus e são hoje tão
bonzinhos\ldots{}

No dia seguinte, cedo, preparada a ferramenta, fiquei à espera do meu
homem.

Oito, nove horas, dez, onze e nada. Bento não aparecia.

--- Geremário, já aprontou o quarto do cego?

--- Não, senhor.

--- Por quê? Não ordenei isso ontem?

Geremário sorriu maliciosamente.

--- O homem não vem, seu doutor. Vai ver que não vem. Pois se a sorte
dele é ser cego\ldots{}

Revoltou"-me aquele cinismo de opinião e ordenei"-lhe com rispidez que
cumprisse minhas ordens sem mais filosofias. E inda de vincos na testa
saí de rumo à casa do Bento.

Encontrei"-a fechada. Bati e ninguém me respondeu. Insistia nisso quando
à janela do casebre fronteiro assomou a trunfa duma bodarrona em camisa.

--- Pode dizer"-me que fim levou a gente desta casa? --- perguntei"-lhe.

--- Seu Bento? Seu Bento foi"-se embora. Ali pelas dez da noite os filhos
``vinheram'' com um carro de boi e um recado seu.

--- Meu?\ldots{}

--- Seu sim! Que o doutor mandou dizer que fosse já, já, por causa da
operação --- uma história comprida. Seu Bento trepou no carro, com
aquela coruja que mora com ele, mais o leitor de livros, e as roupas, e
o cachorro, e o negrinho, e a cacaria inteira. Até uma cartola desta
altura levaram! Depois o carro seguiu por esse mundo afora. Os filhos
consumiram com ele\ldots{}

Fiquei parvo, inteiramente desnorteado de ideias.

A boda prosseguiu:

--- Mas se ele só presta porque é cego\ldots{} Se sarasse, toda a família
afundava na miséria outra vez\ldots{}

No meu primeiro ímpeto de dar queixa à polícia disparei para a casa do
delegado. A meio caminho, porém, estava arrefecida essa inspiração e, ao
chegar à delegacia, gelada de todo. Parei à porta. Vacilei.

Em seguida dei de ombros, convencido de que o Geremário tinha razão e
tinha razão a boda, e os filhos do cego tinham razão, e todo mundo tem
razão.

Polícia! A polícia viria romper ineptamente esse maravilhoso equilíbrio
das coisas de que resulta a harmonia universal.

Rodei para casa.

Logo ao entrar apareceu"-me o Geremário com ar de quem adivinhou tudo.

--- Ponha o almoço --- ordenei"-lhe secamente.

--- Sim, senhor. E\ldots{} posso desarrumar o quarto do cego?

Olhei bem para ele, ainda irritado. Mas a irritação caiu logo. Que culpa
tinha o Geremário de conhecer a vida melhor do que eu?

Humilhei"-me e respondi apenas:

--- Desarrume\ldots{}

\chapter{Sorte grande\footnote[*]{Texto de 1939, publicado no livro \emph{Negrinha}.}}

Foi numa quieta cidadezinha entrevada, dessas que se alheiam do mundo
com a discrição humilde dos musgos. Havia lá a gente do Moura, o
arrecadador de taxas municipais no mercado. A morte arrecadou o Moura
muito fora de tempo e propósito. Consequência: viúva e sete filhos na
``dependura''.

Dona Teodora, quarentona que nunca soubera a significação da palavra
descanso, viu"-se de trabalhos dobrados. Encher sete estômagos, vestir
sete nudezas, educar outras tantas individualidades\ldots{} Se houvesse
justiça no mundo, quantas estátuas a certos tipos de mães!

A vida em tais lugarejos lembra a dos liquens na pedra. Tudo se encolhe
no ``limite'' --- no mínimo que a civilização comporta. Não há
``oportunidades''. Os meninos mal empenam emigram. As meninas, como não
podem emigrar, viram moças; as moças passam a ``tias''; e as tias
evoluem para velhinhas enrugadas como o maracujá murcho --- sem que
nunca venha ensejo para a realização dos dois grandes sonhos: casamento
ou ocupação decentemente remunerada.

Os empreguinhos públicos, de paga microscópica, são tremendamente
disputados. Quem se aferra a um, dali só é arrancado pela morte --- e
passa a vida invejado. Uma só saída para as mulheres, afora o casamento:
a meia dúzia de cadeiras das escolinhas locais.

O mulherio de Santa Rita lembra os rizomas de gladíolos de certas casas
de ``cera e sementes'' pouco frequentadas. O dono do negócio os expõe
numa cesta à porta, à espera do freguês eventual. Não aparece freguês
nenhum --- e o homem os vai retirando da cesta à proporção que murcham.
Mas o estoque não diminui porque entram sempre rizomas novos. O dono da
casa de ``cera e sementes'' de Santa Rita é a Morte.

A boa mãe revoltava"-se. Tinha culpa de terem vindo ao mundo as cinco
meninas e os dois meninos, e de nenhum modo admitia que elas virassem
maracujás secos e eles se estiolassem na lambança viciosa dos
zés"-ninguém.

O problema não era totalmente insolúvel com os meninos, porque podia
mandá"-los para fora no momento oportuno --- mas as meninas? Como
arranjar a vida de cinco moças numa terra em que havia seis para cada
homem casadouro --- e só cinco cadeirinhas?

A mais velha, Maricota, herdara o temperamento, a valentia materna.
Estudou o que pôde e como pôde. Fez"-se professora --- mas já estava nos
vinte e quatro e nem sombra de colocação. As vagas iam sempre para as de
maior peso político, ainda que analfabetas. Maricota, um peso"-pluma, que
poderia esperar?

Mesmo assim dona Teodora não desanimava.

--- Estudem. Preparem"-se. De repente qualquer coisa acontece e vocês se
arrumam.

Os anos, entretanto, passavam sem que a esperadíssima ``qualquer coisa''
viesse --- e os apertos recresciam. Por muito que trabalhassem em
cocadas, bordados de enxoval e costurinhas, a renda não se distanciava
do zero.

Dizem que as desgraças gostam de vir juntas. Quando a situação dos
Mouras atingiu o ponto perigoso da ``dependura'', nova calamidade
sobreveio. Maricota recebeu do céu um estranho castigo: a singularíssima
doença que lhe atacou o nariz.

No começo não deram importância ao caso; só no começo, porque a doença
entrou a progredir, com desorientação de todos os entendidos em medicina
das redondezas. Nunca, verdadeiramente nunca, ninguém soubera por lá de
coisa assim.

O nariz da moça crescia, engordava, engrouvinhava, lembrando o de certos
bêbados incorrigíveis. A deformação nessa parte do rosto é sempre
desastrosa. Dá à fisionomia um ar cômico. Todos se apiedavam da Maricota
--- mas riam"-se sem querer.

A maldade dos lugarejos tem a insistência de certas moscas. Aquele nariz
foi virando o prato predileto do Comentário. Nos momentos de escassez de
assunto era infalível porem"-no à mesa.

--- Se aquilo pega, ninguém mais planta rabanetes em Santa Rita. É só
levar a mão ao rosto e colher um\ldots{}

--- E dizem que está crescendo\ldots{}

--- Se está! A moça já não põe o pé na rua --- nem para a missa. Aquela
negrinha, cria de dona Teodora, me disse que já não é nariz --- é
beterraba\ldots{}

--- Sério?

--- Cresce tanto que se a coisa continua vamos ter um nariz com uma moça
atrás e não uma moça com um nariz na frente. O maior, o principal,
ficará sendo o rabanete\ldots{}

Nos galinheiros também é assim. Quando aparece uma ave doente, ou
ferida, as sãs correm"-na a bicadas --- e bicam"-na até destruí"-la. Em
matéria de maldade o homem é galináceo. A tal ponto chegou a de Santa
Rita que quando aparecia alguém de fora não vacilavam em enfileirar
entre as curiosidades locais a doença da moça.

--- Temos várias coisas dignas de ver"-se. Há a igreja, cujo sino tem um
som sem igual no mundo. Bronze do céu. Há o pé de cacto da casa do major
Lima, com quatro metros de roda na altura do peito. E há o rabanete da
Maricota\ldots{}

O visitante espantava"-se, está claro.

--- Rabanete?

O informante desfiava a crônica do famoso nariz com invençõezinhas
cômicas de sua lavra. ``Não poderei ver isso?'' ``Creio que não, porque
ela já não tem ânimo de pôr o pé na rua --- nem para a missa.''

Chegou o momento de recorrer aos médicos especialistas. Como por lá não
houvesse nenhum, dona Teodora lembrou"-se de um doutor Clarimundo,
especialista de todas as especialidades na cidade próxima. Tinha de
mandar"-lhe a filha. O nariz de Maricota estava ficando clamoroso demais.
Mas\ldots{} mandar como? A distância era grande. Viagem por água --- pelo rio
São Francisco, em cuja margem direita se assentava Santa Rita. O
percurso custaria dinheiro; e custariam dinheiro a consulta, o
tratamento, a estada lá --- e onde o dinheiro? Como reunir os duzentos
mil"-réis necessários?

Não há barreiras para o heroísmo das mães. Teodora redobrou de faina,
operou milagres de gênio e por fim reuniu o dinheiro da salvação.

Chegou o dia. Muito vexada de mostrar"-se em público depois de tantos
meses de segregação, Maricota embarcou para a viagem de dois dias.
Embarcou num gaiola --- o \emph{Comandante Exupério} --- e logo que se
viu a bordo tratou de descobrir um cantinho em que ficasse a salvo da
curiosidade dos passageiros. Inutilmente. Deu logo nos olhos de vários,
sobretudo nos dum moço de bom aspecto, que entrou a mirá"-la com singular
insistência. Maricota esgueirou"-se de sua presença e, de bruços na
amurada, fingiu"-se absorta na contemplação da paisagem. Fraude pura,
coitadinha. A única paisagem que via era a sua --- a nasal. O
passageiro, entretanto, não a largava.

--- Quem é essa moça? --- quis saber, e um de boca perdigotante, também
embarcado em Santa Rita, regalou"-se em contar pormenorizadamente tudo
quanto sabia a respeito.

O moço refranziu a testa. Reconcentrou"-se a meditar. Por fim seus olhos
brilharam.

--- Será possível? --- murmurou em solilóquio, e resolutamente
encaminhou"-se na direção da triste criatura absorvida na contemplação da
paisagem.

--- Perdão, minha senhora, eu sou médico e\ldots{}

Maricota voltou para ele os olhos, muito vexada, sem saber o que dizer.
Como um eco, repetiu:

--- Médico?\ldots{}

--- Sim, médico, e o seu caso está me interessando profundamente. Se é o
que suponho, talvez que\ldots{} Mas, venha cá, conte"-me tudo, conte"-me como
isso começou. Não se vexe. Sou médico --- e para os médicos não há
segredos. Vamos\ldots{}

Maricota, depois de alguma resistência, contou tudo, e à medida que
falava o interesse do moço recrescia.

--- Com licença --- disse ele, e pôs"-se a examinar"-lhe o nariz, sempre
com perguntas cujo alcance a moça não percebia.

--- Como é seu nome? --- atreveu"-se a indagar Maricota.

--- Doutor Cadaval.

A expressão do médico lembrava a do garimpeiro que encontra um diamante
de valor fabuloso --- um Cullinan! Nervosamente ele insistia:

--- Conte, conte\ldots{}

Queria saber tudo; como aquilo começara, como se desenvolvera, que
perturbação ela sentira e outras coisinhas técnicas. E as respostas da
moça tinham o condão de aumentar"-lhe o entusiasmo. Por fim:

--- Maravilhoso! --- exclamou. --- Um caso único de boa sorte\ldots{}

Tais exclamações desnortearam a doente. ``Maravilhoso?'' Que
maravilhamento poderia causar a sua desgraça? Chegou a ressentir"-se. O
médico tentou sossegá"-la.

--- Perdoe"-me, dona Maricota, mas o seu caso é positivamente
extraordinário. De momento não posso firmar parecer --- estou sem
livros; mas macacos me lambam se o que a senhora tem não é um rinofima
--- um \textsc{rinofima}, imagine!

Rinofima! Aquela palavra estranha, dita naquele tom de entusiasmo, em
coisa nenhuma melhorou a situação de atrapalhamento de Maricota. O fato
de sabermos o nome de uma doença não nos consola nem cura.

--- E que tem isso? --- perguntou ela.

--- Tem, minha senhora, que é uma doença raríssima. Pelo que sei a
respeito, não se conhece ainda um só caso em toda América do Sul\ldots{}
Compreende agora o meu entusiasmo de profissional? Médico que descobre
casos únicos é médico de nome feito\ldots{}

Maricota começava a compreender.

Longamente Cadaval debateu a situação, informando"-se de tudo --- da
família, do objeto da viagem. Ao saber de sua ida à cidade próxima em
busca do doutor Clarimundo, revoltou"-se.

--- Qual Clarimundo, minha senhora! Esses médicos da roça não passam de
perfeitas cavalgaduras. Formam"-se e afundam nos lugarejos, nunca leem
nada! Atrasadíssimos. Se a senhora vai consultá"-lo, perderá o seu tempo
e o seu dinheiro. Ora, o Clarimundo!

--- Conhece"-o?

--- Claro que não, mas adivinho. Conheço a classe. O seu caso, minha
senhora, é a maravilha das maravilhas, desses que só podem ser tratados
pelos grandes médicos dos grandes centros --- e estudado pelas
academias. A senhora vai mas é para o Rio de Janeiro. Tive a sorte de
encontrá"-la e não a largo mais. Ora estar! Um rinofima destes nas mãos
do Clarimundo! Tinha graça\ldots{}

A moça alegou que a sua pobreza não lhe permitia tratar"-se na capital.
Eram paupérrimos.

--- Sossegue. Eu farei todas as despesas. Um caso como o seu vale ouro.
Rinofima! O primeiro observado na América do Sul! Isso é ouro em barra,
minha senhora\ldots{}

E tanto falou, e tanto gabou a beleza do rinofima, que Maricota deu de
sentir uns começos de orgulho. Depois de duas horas de debates e
combinações, já estava outra --- sem vexame nenhum dos passageiros ---,
a exibir pelo tombadilho o seu rabanete como quem exibe algo fascinante.

O doutor Cadaval era um moço extremamente expansivo, dos que não param
de falar. O empolgamento em que ficou fê"-lo debater o assunto com todos
de bordo.

--- Comandante --- disse ao capitão horas depois ---, aquilo é uma
preciosidade sem par. Único na América do Sul, imagine! O sucesso que
vou fazer no Rio --- na Europa! É dessas coisas que arrumam a carreira
de um médico. Um rinofima! Um ri"-no"-fi"-ma, capitão!\ldots{}

Não houve passageiro que se não inteirasse da história do rinofima da
moça --- e o sentimento de inveja tornou"-se geral. Evidentemente
Maricota fora marcada pelo Destino. Possuía algo único, uma coisa de
fazer a carreira de um médico e de figurar em todos os tratados de
medicina. Muitos houve que instintivamente correram os dedos pelo nariz
na esperança de apalpar um comecinho da maravilha\ldots{}

Maricota, ao recolher"-se à cabina, escreveu à mãe:

``Tudo está mudando da maneira mais esquisita, mamãe! Encontrei a bordo
um médico distintíssimo, que ao dar com o meu nariz abriu a boca no
maior entusiasmo. Eu só queria que a senhora visse. Acha que é uma
grande --- uma grandíssima coisa, a coisa mais rara do mundo, única na
América do Sul, imagine! Disse que vale um tesouro, que para ele foi o
mesmo que ter encontrado um tal diamante Cullinan. Quer que eu vá para o
Rio de Janeiro. Paga tudo. Como aleguei que somos muito pobres, prometeu
que depois da operação me arranja um lugar de professora. Imagine, eu
professora no Rio de Janeiro! Que ponta, hein? Estou que não caibo em
mim. Professora no Rio!\ldots{} Até a vergonha lá se foi. Passeio com o nariz
bem à mostra, alto. E, coisa incrível, mamãe, todos me olham com inveja!
Inveja, sim --- eu leio nos olhos de todos. Decore esta palavra:
\textsc{rinofima}. É o nome da doença. Ah, eu só queria ver a cara desses bobos
de Santa Rita que tanto caçoavam de mim --- quando souberem\ldots{}''

Maricota mal conseguiu dormir essa noite. Grande mudança de ideias se
operava em sua cabeça. Qualquer coisa a advertia de que era chegado o
momento de uma grande tacada. Tinha de tirar vantagens da situação --- e
como ainda não dera resposta definitiva ao doutor Cadaval, deliberou
executar um plano.

No dia seguinte o médico abordou"-a de novo.

--- Então, dona Maricota, está resolvida, afinal?

A moça estava resolvidíssima; mas, boa mulher que era, fingiu.

--- Não sei ainda. Escrevi à mamãe\ldots{} Há a minha situação pessoal e a da
minha gente. Para que eu vá ao Rio preciso ficar sossegada quanto a
estes dois pontos. Tenho dois irmãos e quatro irmãs --- e como é? Ficar
lá no Rio sem eles, impossível. E como deixá"-los sozinhos em Santa Rita,
se sou o esteio da casa?

O doutor Cadaval refletiu uns momentos. Depois disse:

--- Os rapazes eu posso colocar facilmente. Já suas irmãs, não sei. Que
idade têm elas?

--- Alzira, a logo abaixo de mim, está com vinte e cinco anos. Muito boa
criatura. Borda que é um primor. Bonitinha.

--- Se tem essas prendas, poderemos colocá"-la numa boa casa de modas. E
as outras?

--- Há a Anita, com vinte e dois, mas essa só sabe ler e escrever
versos. Sempre teve um jeito extraordinário para a poesia.

O doutor Cadaval coçou a cabeça. Colocar uma poetisa não é nada fácil
--- mas veria. Há os empregos do Governo, nos quais cabem até os poetas.

--- Há a Olga, com vinte anos, que só pensa em casar. Essa não quer
outro emprego. Nasceu para o casamento --- e lá em Santa Rita está
secando porque não há homens --- todos emigram.

--- Arranjaremos um bom casamento para a Olga --- prometeu o médico.

--- E há a Odete, com dezenove anos, que ainda não revelou disposição
para coisa nenhuma. Boa criatura, mas muito criançola, bobinha.

--- Vai ser outro casamento --- sugeriu o médico. --- Arranja"-se.
Arranjaremos a vida de todos.

O doutor Cadaval ia prometendo com aquela facilidade porque no íntimo
não tinha intenção de colocar tanta gente. Poderia, sim, arrumar a vida
de Maricota --- depois de operá"-la. Mas o resto da família que se
fomentasse.

Assim não sucedeu, entretanto. As aperturas da vida tinham dado a
Maricota um senso das realidades verdadeiramente totalitário. Percebendo
que aquela oportunidade era a maior da sua vida, resolveu não deixá"-la
escapar. De modo que ao chegar ao Rio, antes de entregar"-se ao
tratamento e exibir na Academia de Medicina o seu caso único, impôs
condições. Alegou que sem a irmã Alzira não tinha jeito de ficar sozinha
na capital --- e o remédio foi a vinda de Alzira. Mal pilhou lá a irmã,
insistiu em colocá"-la --- porque não tinha o menor propósito ficarem as
duas nas costas do médico. ``Assim, a Alzira acanha"-se e volta.''

Ansioso por dar início à exploração do rinofima, o médico pulou para
arranjar a colocação da Alzira. E depois disso deu novos pulos para
mandar vir e colocar a Anita. E depois da Anita chegou a vez da Olga. E
depois da Olga chegou a vez da Odete. E depois da Odete chegou a vez de
dona Teodora e dos dois rapazes.

O caso da Olga foi difícil. Casamento! Mas Cadaval teve uma ideia filha
do desespero: intimou um seu ajudante no consultório, português
quarentão de nome Nicéforo, a casar"-se com a menina. \emph{Ultimatum} da
Moral.

--- Ou casa"-se ou vai para o olho da rua. Não quero mais saber de
auxiliares solteirões.

Nicéforo, tipo bastante pai da vida, coçou a cabeça mas casou"-se --- e
foi o mais feliz dos Nicéforos.

A família já estava toda arrumada, quando Maricota se lembrou de dois
primos. O médico, porém, resistiu.

--- Não. Isso também é demais. Se continua assim, a senhora acaba
forçando"-me a arranjar um bispado para o padre de Santa Rita. Não e não.

A vitória do doutor Cadaval foi verdadeiramente estrondosa. Encheram"-se
as revistas médicas e os jornais com a notícia da solene apresentação à
Academia de Medicina do belíssimo caso --- único na América do Sul ---
dum maravilhoso rinofima, o mais belo dos rinofimas. As publicações
estrangeiras acompanharam as nacionais. O mundo científico de todos os
continentes ficou sabendo de Maricota, do seu ``rabanete'' e do eminente
doutor Cadaval Lopeira --- luminar da ciência médica sul"-americana.

Dona Teodora, felicíssima, não cessava de comentar o estranho curso dos
acontecimentos.

--- Bem se diz que Deus escreve direito por linhas tortas. Quando havia
eu de imaginar, ao nos surgir aquela horrível coisa no nariz de minha
filha, que era para o bem geral de todos!

Restava a parte última --- a operação. Maricota, entretanto, ainda nas
vésperas do dia marcado vacilava.

--- Que acha, mamãe? Deixo ou não deixo que o doutor me opere?

Dona Teodora abriu a boca.

--- Que ideia, menina! Claro que deixa. Pois há de ficar toda vida assim
com esse escândalo na cara?

Maricota não se decidia.

--- Podemos demorar um pouco mais, mamãe. Tudo quanto nos veio de bom
saiu do rinofima. Quem sabe se nos rende mais alguma coisa? Há ainda o
Zezinho a colocar --- e o pobre do Quindó, que nunca achou emprego\ldots{}

Mas dona Teodora, arquifarta do rabanete, ameaçou de levá"-la de volta
para Santa Rita, se ela teimasse na asneira de retardar, por um só dia,
a operação. E Maricota foi operada. Perdeu o rinofima, ficando com um
nariz igual ao de todas as outras, apenas levemente enrugadinho em
consequência dos enxertos de epiderme.

Quem positivamente desapontou foi a gente maldosa do lugarejo. O
maravilhoso romance de Maricota era comentado em todas as rodinhas com
grandes exageros --- até com o exagero de que ela estava noiva do doutor
Cadaval.

--- Como a gente se engana neste mundo! --- filosofou o farmacêutico.
--- Todos pensamos que aquilo fosse doença --- mas o verdadeiro nome de
tais rabanetes, sabem qual é?

--- ?

--- Sorte grande, minha gente! Sorte grande da Espanha\ldots{}

\chapter{Dona Expedita\footnote[*]{Texto de 1939, publicado no livro \emph{Negrinha}.}}

--- \ldots{}

--- Minha idade? Trinta e seis\ldots{}

--- Então, venha.

Sempre que dona Expedita se anunciava no jornal, dando um número de
telefone, aquele diálogo se repetia. Seduzidas pelos termos do anúncio,
as donas de casa telefonavam"-lhe para ``tratar'' --- e vinha
inevitavelmente a pergunta sobre a idade, com a também inevitável
resposta dos trinta e seis anos. Isso desde antes da Grande Guerra. Veio
o 1914 --- ela continuou nos trinta e seis. Veio a batalha do Marne;
veio o armistício --- ela firme nos trinta e seis. Tratado de Versalhes
--- trinta e seis. Começos de Hitler e Mussolini --- trinta e seis.
Convenção de Munique --- trinta e seis\ldots{}

A futura guerra a reencontrará nos trinta e seis. O mais teimoso dos
empaques! Dona Expedita já está ``pendurada'', escorada de todos os
lados, mas não tem ânimo de abandonar a casa dos trinta e seis anos ---
tão simpática!

E, como só tem trinta e seis anos, veste"-se à moda dessa idade, um pouco
mais vistosamente do que a justa medida aconselha. Erro grande! Se à
força de cores claras, ruges e batons, não mantivesse aos olhos do mundo
os seus famosos trinta e seis, era provável que desse a ideia duma bem
aceitável matrona de sessenta\ldots{}

Dona Expedita é ``tia''. Amor só teve um lá pela juventude, do qual às
vezes, nos ``momentos de primavera'', ainda fala. Ah, que lindo moço! Um
príncipe. Passou um dia a cavalo pela sua janela. Passou na tarde
seguinte e ousou um cumprimento. Passou e repassou durante duas semanas
--- e foram duas semanas de cumprimentos e olhares de fogo. E só. Não
passou mais --- desapareceu da cidade para sempre.

O coração da gentil Expedita pulsou intensamente naqueles maravilhosos
quinze dias --- e nunca mais. Nunca mais namorou ou amou ninguém --- por
causa da casmurrice do pai.

Seu pai era um caturra de barbas à Von Tirpitz, português irredutível,
desses que fogem de certos romances de Camilo e reentram na vida. Feroz
contra o sentimentalismo. Não admitia namoros em casa, e nem que se
pronunciasse a palavra casamento. Como vivesse setenta anos, forçou as
duas únicas filhas a se estiolarem ao pé da sua catarreira crônica.
``Filhas são para cuidar da casa e da gente.''

Morreu, afinal, e arruinado. As duas ``tias'' venderam a casa para
pagamento das contas e tiveram de empregar"-se. Sem educação técnica, os
únicos empregos antolhados foram os de criada"-grave, dama de companhia
ou ``tomadeira de conta'' --- graus levemente superiores à crua
profissão normal de criada comum. O fato de serem de ``boa família''
autorizava"-as ao estacionamento nesse degrau um pouco acima do último.

Um dia a mais velha morreu. Dona Expedita ficou só no mundo. Que fazer,
senão viver? Foi vivendo e especializando"-se em lidar com patroas. Por
fim distraía"-se com isso. Mudar de emprego era mudar de ambiente --- ver
caras novas, coisas novas, tipos novos. Um cinema --- o seu cinema! O
ordenado, sempre mesquinho. O maior de que se lembrava fora de cento e
cinquenta mil réis. Caiu depois para cento e vinte; depois para cem;
depois oitenta. Inexplicavelmente as patroas iam"-lhe diminuindo a paga a
despeito da sua permanência na linda idade dos trinta e seis anos\ldots{}

Dona Expedita colecionava patroas. Teve"-as de todos os tipos e naipes
--- das que obrigam as criadas a comprar o açúcar com que adoçam o café
às que voltam para casa de manhã e nunca lançam os olhos sobre o caderno
de compras. Se fosse escritora teria deixado o mais pitoresco dos
livros. Bastava que fixasse metade do que viu e ``padeceu''. O capítulo
das pequeninas decepções seria dos melhores --- como aquele caso dos
quatrocentos mil"-réis\ldots{}

Foi certa vez em que, saída de um emprego, andava em procura de outro.
Nessas ocasiões costumava encostar"-se à casa de uma família que se dera
com a sua, e lá ficava um mês ou dois até conseguir nova colocação.
Pagava a hospedagem fazendo doces, no que era perita, sobretudo num
certo bolo inglês que mudou de nome, passando a chamar"-se o ``bolo de
dona Expedita''. Nesses interregnos comprava todos os dias um jornal
especializado em anúncios domésticos, no qual lia atentamente a seção do
``procura"-se''. Com a velha experiência adquirida, adivinhava pela
redação as condições reais do emprego.

--- Porque ``elas'' publicam aqui uma coisa e querem outra --- comentava
filosoficamente, batendo no jornal. --- Para esconder o leite, não há
como as patroas!

E ia lendo, de óculos na ponta do nariz: ``Precisa"-se duma senhora de
meia idade para servicinhos leves''.

--- Hum! Quem lê isto pensa que é assim mesmo --- mas não é. O tal
servicinho leve não passa de isca --- é a minhoca do anzol. A mim é que
não me enganam, as biscas\ldots{}

Lia todos os ``procura"-se'', com um comentário para cada um, até que se
detinha no que lhe cheirava melhor. ``Precisa"-se duma senhora de
meia"-idade para serviços leves em casa de fino tratamento.''

--- Este, quem sabe? Se é casa de fino tratamento, pelo menos fartura há
de haver. Vou telefonar.

E vinha a telefonada do costume com a eterna declaração dos trinta e
seis anos.

O hábito de lidar com patroas manhosas levou"-a a lançar mão de vários
recursos estratégicos; um deles: só ``tratar'' pelo telefone e não
dar"-se como ela mesma. ``Estou falando em nome duma amiga que procura
emprego.'' Desse modo tinha mais liberdade e jeito de sondar a
``bisca''.

--- ``Essa amiga é uma excelente criatura'' --- e vinham bem dosados
elogios. --- ``Só que não gosta de serviços pesados.''

--- ``Que idade?

--- ``Trinta e seis anos. Senhora de muito boa família --- mas por menos
de cento e cinquenta mil"-réis nunca se empregou.

--- ``É muito. Aqui o mais que pagamos é cento e dez --- sendo boa.

--- ``Não sei se ela aceitará. Hei de ver. Mas qual é o serviço?

--- ``Leve. Cuidar da casa, fiscalizar a cozinha, espanar --- arrumar\ldots{}

--- ``Arrumar? Então é arrumadeira que a senhora quer?''

E dona Expedita pendurava o fone, arrufada, murmurando: ``Outro
ofício!''.

O caso dos quatrocentos mil"-réis foi o seguinte. Ela andava sem emprego
e a procurá"-lo na seção do ``procura"-se''. Súbito, esbarrou com esta
maravilha: ``Precisa"-se duma senhora de meia"-idade para fazer companhia
a uma enferma; ordenado, quatrocentos mil"-réis''.

Dona Expedita esfregou os olhos. Leu outra vez. Não acreditou. Foi em
busca duns óculos novos adquiridos na véspera. Sim. Lá estava escrito
quatrocentos mil"-réis!\ldots{}

A possibilidade de apanhar um emprego único no mundo fê"-la pular. Correu
a vestir"-se, a pôr o chapeuzinho, a avivar as cores do rosto e voou
pelas ruas afora. Foi dar com os costados numa rua humilde; nem rua era
--- numa ``avenida''. Defronte à casa indicada --- casinha de porta e
duas janelas --- havia uma dúzia de pretendentes.

--- Será possível? O jornal saiu agorinha e já tanta gente por aqui?

Notou que entre as postulantes predominavam senhoras bem"-vestidas, com o
aspecto de ``damas envergonhadas''. Natural que assim fosse porque um
emprego de quatrocentos mil"-réis era positivamente um fenômeno. Nos
seus\ldots{} trinta e seis anos de vida terrena jamais tivera notícia de
nenhum. Quatrocentos por mês! Que mina! Mas como um emprego assim em
casa tão modesta? ``Já sei. O emprego não é aqui. Aqui é onde se trata
--- casa do jardineiro, com certeza\ldots{}''

Dona Expedita observou que as postulantes entravam de cara risonha e
saíam de cabeça baixa. Evidentemente a decepção da recusa. E o seu
coração batia de gosto ao ver que todas iam sendo recusadas. Quem sabe?
Quem sabe se o destino marcara justamente a ela como a eleita?

Chegou por fim a sua vez. Entrou. Foi recebida por uma velha na cama.
Dona Expedita nem precisou falar. A velha foi logo dizendo:

--- Houve erro no jornal. Mandei por quarenta mil"-réis e puseram
quatrocentos\ldots{} Tinha graça eu pagar quatrocentos a uma criada, eu que
vivo à custa do meu filho, sargento da polícia, que nem isso ganha por
mês\ldots{}

Dona Expedita retirou"-se com cara exatamente igual à das outras.

O pior da luta entre criados e patroas é que estas são compelidas a
exigir o máximo, e as criadas, por natural defesa, querem o mínimo.
Nunca jamais haverá acordo, porque é choque de totalitarismo com
democracia.

Um dia, entretanto, dona Expedita teve a maior das surpresas: encontrou
uma patroa absolutamente identificada com suas ideias quanto ao ``mínimo
ideal'' --- e, mais que isso, entusiasmada com esse minimalismo --- a
ajudá"-la a minimizar o minimalismo!

Foi assim. Dona Expedita estava pela vigésima vez na tal família amiga,
à espera de nova colocação. Lembrou"-se de recorrer a uma agência, para a
qual telefonou. ``Quero uma colocação assim, assim, de duzentos
mil"-réis, em casa de gente arranjada, fina e, se for possível, em
fazenda. Serviços leves, bom quarto, banho. Aparecendo qualquer coisa
deste gênero, peço que me telefonem'' --- e deu o número do aparelho e
da casa.

Horas depois retinia a campainha do portão.

--- É aqui que mora madame Expedita? --- perguntou em língua atrapalhada
uma senhora alemã, cheia de corpo, de bom aspecto.

A criadinha que atendeu disse que sim, fê"-la entrar para o \emph{hall}
de espera e foi correndo avisar dona Expedita. ``Uma estrangeira gorda,
querendo falar com madame!''

--- Que pressa, meu Deus! --- murmurou a solicitada, correndo ao espelho
para os retoques. --- Nem três horas faz que telefonei. Agência boa,
sim\ldots{}

Dona Expedita apareceu no \emph{hall} com um excessozinho de ruge nos
beiços de múmia. Apareceu e conversou --- e maravilhou"-se, porque pela
primeira vez na vida encontrava a patroa ideal. A mais \emph{sui
generis} das patroas, de tão integrada no ponto de vista das ``senhoras
de meia"-idade que procuram serviços leves''.

O diálogo travou"-se num crescendo de animação.

--- Muito boa tarde! --- disse a alemã com a maior cortesia. --- Então
foi madame quem telefonou para a agência?

O ``madame'' causou espécie a dona Expedita.

--- É verdade. Telefonei e dei as condições. A senhora gostou?

--- Muito, mas muito mesmo! Era exatamente o que eu queria. Perfeito.
Mas vim ver pessoalmente, porque o costume é anunciarem uma coisa e a
realidade ser outra.

A observação encantou dona Expedita, cujos olhos brilharam.

--- A senhora parece que está pensando com a minha cabeça. É justamente
isso o que se dá, vivo eu dizendo. As patroas escondem o leite. Anunciam
uma coisa e querem outra. Anunciam serviços leves e botam em cima das
pobres criadas a maior trabalheira que podem. Eu falei, eu insisti com a
agência: servicinhos leves\ldots{}

--- Isso mesmo! --- concordou a alemã, cada vez mais encantada. ---
Serviços leves, bem leves, porque afinal de contas uma criada é gente
--- não é burro de carroça.

--- Claro! Mulheres de certa idade não podem fazer serviços de mocinhas,
como arrumar, lavar, cozinhar quando a cozinheira não vem. Ótimo! Quanto
à acomodação, falei à agência em ``bom quarto''\ldots{}

--- Exatamente! --- concordou a alemã. --- Bom quarto --- com janelas.
Nunca pude conformar"-me com isso de as patroas meterem as criadas em
desvãos escuros, sem ar, como se fossem malas. E sem banheiro em que
tomem banho.

Dona Expedita era toda risos e sorrisos. A coisa lhe estava saindo
maravilhosa.

--- E banho quente! --- acrescentou com entusiasmo.

--- Quentíssimo! --- berrou a alemã batendo palmas. --- Isso para mim é
ponto capital. Como pode haver asseio numa casa onde nem banheiro há
para as criadas?

--- Ah, minha senhora, se todas as patroas pensassem assim! --- exclamou
dona Expedita erguendo os olhos para o céu. --- Que felicidade não seria
o mundo! Mas no geral as patroas são más --- e iludem as pobres criadas,
para agarrá"-las e explorá"-las.

--- Isso mesmo! --- apoiou a alemã. --- A senhora está falando como um
livro de sabedoria. Para cada cem patroas haverá cinco ou seis que
tenham coração --- que compreendam as coisas\ldots{}

--- Se houver! --- duvidou dona Expedita.

O entendimento das duas era perfeito: uma parecia o ``dublê'' da outra.
Debateram o ponto dos ``serviços leves'' com tal mútua compreensão que
os serviços ficaram levíssimos, quase nulos --- e dona Expedita viu
erguer"-se diante de si o grande sonho de sua vida: um emprego em que não
fizesse nada, absolutamente nada\ldots{}

--- Quanto ao ordenado --- disse ela (que sempre pedia duzentos para
deixar por oitenta) ---, fixei"-o em duzentos\ldots{}

Avançou isso medrosamente e ficou à espera da inevitável repulsa. Mas a
repulsa do costume pela primeira vez não veio. Bem ao contrário disso, a
alemã concordou com entusiasmo.

--- Perfeitamente! Duzentos por mês --- e pagos no último dia de cada
mês.

--- Isso! --- berrou dona Expedita levantando"-se da cadeira. --- Ou no
comecinho. Essa história de pagamento em dia incerto nunca foi comigo.
Dinheiro de ordenado é sagrado.

--- Sacratíssimo! --- urrou a alemã levantando"-se também.

--- Ótimo --- exclamou dona Expedita. --- Está tudo como eu queria.

--- Sim, ótimo --- repetiu a alemã. --- Mas a senhora também falou em
fazenda\ldots{}

--- Ah, sim, fazenda. Uma fazenda boa, com bastante frutas, bastante
leite, bastante ovos porque há fazendas muito feias.

O quadro da fazenda bonita, toda frutas, leite e ovos, extasiou a alemã.
Que maravilha\ldots{}

Dona Expedita continuou:

--- Gosto muito de lidar com pintinhos.

--- Pintos? Ah, é o maior dos encantos! Adoro os pintos --- as
ninhadas\ldots{} O nosso entendimento vai ser absoluto, madame\ldots{}

O êxtase de ambas sobre a vida de fazenda foi subindo numa vertigem.
Tudo quanto havia de sonhos incubados naquelas almas refloriu viçoso.
Infelizmente a alemã teve a ideia de perguntar:

--- E onde fica a \emph{sua} fazenda, madame?

--- A \emph{minha} fazenda --- repetiu dona Expedita refranzindo a
testa.

--- Sim, a sua fazenda --- a fazenda para onde madame quer que eu vá\ldots{}

--- Fazenda pra onde eu quero que a senhora vá? --- tornou a repetir
dona Expedita, sem entender coisa nenhuma. --- Fazenda, eu? Pois se eu
tivesse fazenda lá andava a procurar emprego?

Foi a vez de a alemã arregalar os olhos, atrapalhadíssima. Também não
estava entendendo coisa nenhuma. Ficou uns instantes no ar. Por fim:

--- Pois madame não telefonou para a agência dizendo que tinha um
emprego assim, assim, na sua fazenda?

--- \emph{Minha} fazenda uma ova! Nunca tive fazenda. Telefonei
procurando emprego, se possível numa fazenda, isso sim\ldots{}

--- Então, então, então\ldots{} --- e a alemã enrubesceu como uma papoula.

--- Pois é --- disse dona Expedita percebendo afinal o quiproquó. ---
Estamos aqui feito duas idiotas, cada qual querendo emprego e pensando
que a outra é a patroa\ldots{}

O cômico da situação fê"-las rirem"-se --- e gostosamente, já retomadas à
posição de ``senhoras de meia"-idade que procuram serviços leves''.

--- Esta foi muito boa! --- murmurou a alemã levantando"-se para sair.
--- Nunca me aconteceu coisa assim. Que agência, hein?

Dona Expedita filosofou.

--- Eu bem que estava desconfiada. A esmola era demais. A senhora ia
concordando com tudo que eu dizia --- até com os banhos quentes! Ora,
isso nunca foi linguagem de patroa --- dessas biscas. A agência errou,
talvez por causa do telefone, que estava danado hoje --- além do que sou
meio dura dos ouvidos\ldots{}

Nada mais havia a dizer. Despediram"-se. Depois que a alemã bateu o
portão, dona Expedita fechou a porta, com um suspiro arrancado do fundo
das tripas.

--- Que pena, meu Deus! Que pena não existirem no mundo patroas que
pensem como as criadas\ldots{}

\chapter{Herdeiro de si mesmo\footnote[*]{Texto de 1939, publicado no livro \emph{Negrinha}.}}

O povo de Dois Rios não cessava de comentar a inconcebível ``sorte'' do
coronel Lupércio Moura, o grande milionário local. Um homem que saíra do
nada. Que começara modesto menino de escritório dos que mal ganham para
os sapatos, mas cuja vida, dura até aos trinta e seis anos, fora daí por
diante a mais espantosa subida pela escada do Dinheiro, a ponto de aos
sessenta ver"-se montado numa hipopotâmica fortuna de sessenta mil contos
de réis.

Não houve o que Lupércio não conseguisse da Sorte --- até o posto de
coronel, apesar de já extinta a pitoresca instituição dos coronéis. A
nossa velha Guarda Nacional era uma milícia meramente decorativa, com os
galões de capitão, major e coronel reservados para coroamento das vidas
felizes em negócios. Em todas as cidades havia sempre um coronel: o
homem de mais posses. Quando Lupércio chegou aos vinte mil contos, a
gente de Dois Rios sentiu"-se acanhada de tratá"-lo apenas de ``senhor
Lupércio''. Era pouquíssimo. Era absurdo que um detentor de tanto
dinheiro ainda se conservasse ``soldado raso'' --- e por consenso
unânime promoveram"-no, com muita justiça, a coronel, o posto mais alto
da extinta milícia.

Criaturas há que nascem com misteriosa aptidão para monopolizar
dinheiro. Lembram ímãs humanos. Atraem a moeda com a mesma inexplicável
força com que o ímã atrai a limalha. Lupércio tomara"-se ímã. O dinheiro
procurava"-o de todos os lados, e uma vez aderido não o largava mais.
Toda gente faz negócios em que ora ganha, ora perde. Ficam ricos os que
ganham mais do que perdem e empobrecem os que perdem mais do que ganham.

Mas caso de homens de mil negócios sem uma só falha, existia no mundo
apenas um --- o do coronel Lupércio. Até aos trinta e seis anos ganhou
dinheiro de modo normal, e conservou"-o à força da mais acirrada
economia. Juntou um pecúlio de quarenta e cinco contos e quinhentos
mil"-réis como o juntam todos os forretas. Foi por essas alturas que sua
vida mudou. A Sorte ``encostou"-se'' nele, dizia o povo. Houve aquela
tacada inicial de Santos e a partir daí todos os seus negócios foram
tacadas prodigiosas. Evidentemente, uma Força Misteriosa passara a
protegê"-lo.

Que tacada inicial fora essa? Vale a pena recordá"-la.

Certo dia, inopinadamente, Lupércio apareceu com a ideia, absurda para o
seu caráter, de uma estação de veraneio em Santos. Todo mundo se
espantou.

Pensar em veraneio, em flanar, botar dinheiro fora, aquela criatura que
nem sequer fumava para economia dos níqueis que custam os maços de
cigarros? E quando o interpelaram, deu uma resposta esquisita:

--- Não sei. Uma coisa me empurra para lá\ldots{}

Lupércio foi para Santos. Arrastado, sim, mas foi. E lá se hospedou no
hotelzinho mais barato, sempre atento a uma só coisa: o saldo que lhe
ficaria dos quinhentos mil"-réis que destinara à ``maluquice''. Nem
banhos de mar tomou, apesar da grande vontade, para economia dos vinte
mil"-réis da roupa de banho. Contentava"-se com ver o mar.

Que enlevo de alma lhe vinha da imensidão líquida, eternamente a aflar
em ondas e a refletir os tons do céu! Lupércio extasiava"-se diante de
tamanha beleza.

``Quanto sal! Quantos milhões de milhões de toneladas de sal!'', dizia
lá consigo --- e seus olhos em êxtase ficavam a ver pilhas imensas de
sacas de sal amontoadas por toda a extensão das praias.

Também gostava de assistir à puxada das redes dos pescadores,
enlevando"-se no cálculo do valor da massa de peixes recolhida. Seu
cérebro era a mais perfeita máquina de calcular que o mundo ainda
produzira.

Num desses passeios afastou"-se mais que de costume e foi ter à Praia
Grande. Um enorme trambolho ferrugento semienterrado na areia chamou"-lhe
a atenção.

--- Que é aquilo? --- indagou dum passante.

Soube tratar"-se dum cargueiro inglês que vinte anos antes dera à costa
naquele ponto. Uma tempestade arremessara"-o à praia onde encalhara e
ficara a afundar"-se lentissimamente. No começo o grande casco aparecia
quase todo de fora --- ``mas ainda acaba engolido pela areia'', concluiu
o informante.

Certas criaturas nunca sabem o que fazem nem o que são, nem o que as
leva a isto e não aquilo. Lupércio era assim. Ou andava assim agora,
depois do ``encostamento'' da Força. Essa Força o puxava às vezes como o
cabreiro puxa para a feira um cabrito --- arrastando"-o. Lupércio veio
para Santos arrastado. Chegara até aquele casco arrastado --- e era a
contragosto que permanecia diante dele, porque o sol estava terrível e
Lupércio detestava o calor. Travava"-se dentro dele uma luta. A Força
obrigava"-o a atentar no casco, a calcular o volume daquela massa de
ferro, o número de quilos, o valor do metal, o custo do desmantelamento
--- mas Lupércio resistia. Queria sombra, queria escapar ao calor
terrível. Por fim venceu. Não calculou coisa nenhuma --- e fez"-se de
volta para o hotelzinho com cara de quem brigou com a namorada ---
evidentemente amuado.

Nessa noite todos os seus sonhos giraram em torno do casco velho. A
Força insistia para que ele calculasse a ferralha, mas mesmo em sonhos
Lupércio resistia, alegava o calor reinante --- e os pernilongos. Oh,
como havia pernilongos em Santos! Como calcular qualquer coisa com o
termômetro perto de quarenta graus e aquela infernal música anofélica?
Lupércio amanheceu de mau humor, amuado. Amuado com a Força.

Foi quando ocorreu o caso mais inexplicável de sua vida: o casual
encontro de um corretor de negócios que o seduziu de maneira estranha.
Começaram a conversar bobagens e gostaram"-se. Almoçaram juntos.
Encontraram"-se de novo à tarde para o jantar. Jantaram juntos e
depois\ldots{} a farrinha!

A princípio a ideia de farra tinha assustado Lupércio. Significava
desperdício de dinheiro --- um absurdo. Mas como o homem lhe pagara o
almoço e o jantar, era bem possível que também custeasse a farrinha.
Essa hipótese fez que Lupércio não repelisse de pronto o convite, e o
corretor, como se lhe adivinhasse o pensamento, acudiu logo:

--- Não pense em despesas. Estou cheio de ``massa''. Com o negocião que
fiz ontem, posso torrar um conto sem que meu bolso dê por isso.

A farra acabou diante de uma garrafa de uísque, bebida cara que só
naquele momento Lupércio veio a conhecer. Uma, duas, três doses.
Qualquer coisa levitante começou a desabrochar dentro dele. Riu"-se à
larga. Contou casos cômicos. Referiu cem fatos de sua vida e depois, oh,
oh, oh, falou em dinheiro e confessou quantos contos possuía no banco!

--- Pois é! Quarenta e cinco contos --- ali na batata!

O corretor passou o lenço pela testa suada. \emph{Uf}! Até que enfim
descobrira o peso metálico daquele homem. A confissão dos quarenta e
cinco contos era algo absolutamente aberrante na psicologia de Lupércio.
Artes do uísque, porque em estado ``normal'' ninguém nunca lhe
arrancaria semelhante confissão. Um dos seus princípios instintivos era
não deixar que ninguém lhe conhecesse ``ao certo'' o valor monetário.
Habilmente despistava os curiosos, dando a uns a impressão de possuir
mais, e a outros a de possuir menos, ao que realmente possuía. Mas
``\emph{in whiskey veritas}'', diz o latim --- e ele estava com quatro
boas doses no sangue.

O que se passou dali até a madrugada Lupércio nunca o soube com clareza.
Vagamente se lembrava de um estranhíssimo negócio em que entravam o
velho
casco do cargueiro inglês e uma companhia de seguros marítimos.

Ao despertar no dia seguinte, ao meio"-dia, numa ressaca horrorosa,
tentou reconstruir o embrulho da véspera. A princípio, nada; tudo
confusão. De repente, empalideceu. Sua memória começava a abrir"-se.

--- Será possível?

Fora possível, sim. O corretor havia ``roubado'' os seus quarenta e
cinco contos! Como? Vendendo"-lhe o ferro"-velho. Esse corretor era agente
da companhia que pagara o seguro do cargueiro naufragado e ficara dona
do casco. Havia muitos anos que recebera a incumbência de apurar
qualquer coisa daquilo --- mas nunca obtivera nada, nem cinco, nem três,
nem dois contos --- e agora o vendera àquele imbecil por quarenta e
cinco!

A entrada triunfal do corretor no escritório da companhia, vibrando no
ar o cheque! Os abraços, os parabéns dos companheiros tomados de
inveja\ldots{}

O diretor da sucursal fê"-lo vir ao escritório.

--- Quero que receba o meu abraço --- disse"-lhe. --- A sua façanha vem
pô"-lo no primeiro lugar entre os nossos agentes. O senhor acaba de
tornar"-se a grande estrela da Companhia.

Enquanto isso, lá no hotelzinho, Lupércio amarfanhava o travesseiro
desesperadamente. Pensou na polícia. Pensou em contratar o melhor
advogado de Santos. Pensou em dar tiro --- um tiro na barriga do infame
ladrão; na barriga, sim, por causa da peritonite. Mas nada pôde fazer. A
Força lá dentro o inibia. Impedia"-o de agir neste ou naquele sentido.
Forçava"-o a esperar.

--- Mas esperar que coisa?

Ele não sabia, não compreendia, mas sentia aquela impulsação tremenda
que o forçava a esperar. Por fim, exausto da luta, ficou de corpo
largado --- vencido. Sim, esperaria. Não faria nada --- nem polícia, nem
advogado, nem peritonite, apesar de ser um caso de escroqueria pura,
desses que a lei pune.

E como não tivesse ânimo de regressar a Dois Rios, deixou"-se ficar em
Santos num empreguinho dos mais modestos --- esperando, esperando\ldots{} não
sabia o quê.

Não esperou muito. Dois meses depois rebentava a Grande Guerra, e a
tremenda alta dos metais não demorou a sobrevir. No ano seguinte
Lupércio revendeu o casco do \emph{Sparrow} por trezentos e vinte contos
de réis. A notícia encheu Santos --- e o corretor estrela foi tocado da
companhia de seguros quase a pontapés. O mesmo diretor que o promovera
ao ``estrelato'' despediu"-o com palavras ferozes:

--- Imbecil! Esteve anos e anos com o \emph{Sparrow} e vai vendê"-lo por
uma ninharia justamente nas vésperas da valorização. Rua! Faça"-me o
favor de nunca mais me pôr os pés aqui, seu coisa!

Lupércio voltou para Dois Rios com os trezentos e vinte contos no bolso
e perfeitamente reconciliado com a Força. Daí por diante nunca mais
houve amuos, nem hiatos na sua ascensão ao milionarismo. Lupércio dava
ideia do demônio. Enxergava no mais escuro de todos os negócios.
Adivinhava. Recusava muitos que todos consideravam da China, para
realizar outros que todos refugavam --- e o que inevitavelmente sucedia
era o fracasso desses negócios da China e a vitória dos de todos
refugados.

No jogo dos marcos alemães o mundo inteiro perdeu --- menos Lupércio. Um
belo dia deliberou ``embarcar nos marcos'', contra o conselho de todos
os prudentes locais. A moeda alemã estava a cinquenta réis. Lupércio
comprou milhões e mais milhões, empatou nela todas as suas
disponibilidades. E com espanto geral o marco principiou a subir. Foi a
sessenta, a setenta, a cem réis. O entusiasmo pelo negócio tornou"-se
imenso. Iria a duzentos, a trezentos réis, diziam todos --- e não houve
quem não se atirasse à compra daquilo.

Quando a cotação chegou a cento e dez réis, Lupércio foi à capital
consultar um banqueiro das suas relações, verdadeiro oráculo em finanças
internacionais --- o ``infalível'', como diziam nas rodas bancárias.

--- Não venda --- foi o conselho do homem. --- A moeda alemã está
firmíssima, vai a duzentos, pode chegar mesmo a oitocentos --- e só
então será o momento de vender.

As razões que o banqueiro deu para demonstrar matematicamente o asserto
eram de perfeita solidez; eram a própria evidência materializada em
raciocínio.

Lupércio ficou absolutamente convencido daquela matemática --- mas
arrastado pela Força encaminhou"-se para o banco onde tinha os seus
marcos --- arrastado como o cabritinho que o cabreiro conduz à feira ---
e lá, em voz sumida, submisso, envergonhado, deu ordens para a venda
imediata dos seus milhões.

--- Mas, coronel --- objetou o empregado a quem se dirigiu ---, não acha
que é erro vender agora que a alta está numa vertigem? Todos os
prognósticos são unânimes em garantir que teremos o marco a duzentos, a
trezentos, e isso antes de um mês\ldots{}

--- Acho, sim, que é isso mesmo --- respondeu Lupércio, como que
agarrado pela garganta. --- Mas quero, sou ``forçado'' a vender. Venda
já, já, hoje mesmo.

--- Olhe, olhe\ldots{} --- disse ainda o empregado. --- Não se precipite.
Deixe essa resolução para amanhã. Durma sobre o caso.

A Força quase estrangulou Lupércio, que com os últimos restos de voz
apenas pôde dizer:

--- É verdade, tem razão --- mas venda, e hoje mesmo\ldots{}

No dia seguinte começou a degringolada final dos marcos alemães, na
descida vertiginosa que os levou ao zero absoluto.

Lupércio, comprador a cinquenta réis, vendera"-os pelo máximo da cotação
alcançada --- e justamente na véspera de debacle! O seu lucro foi de
milhares de contos.

Os contos de Lupércio foram vindo aos milhares, mas também lhe vieram
vindo os anos, até que um dia se convenceu de estar velho e
inevitavelmente próximo do fim. Dores aqui e ali --- doencinhas
insistentes, crônicas. Seu organismo evidentemente decaía à proporção
que a fortuna aumentava. Ao completar os sessenta anos Lupércio tomou"-se
de uma sensação nova, de pavor --- o pavor de ter de largar a
maravilhosa fortuna reunida. Tão integrado estava no dinheiro, que a
ideia de separar"-se dos milhões lhe parecia uma aberração da natureza.
Morrer! Teria então de morrer, ele que era diferente dos outros homens?
Ele que viera ao mundo com a missão de chamar a si quanto dinheiro
houvesse? Ele que era o ímã atrator da limalha?

O que foi a sua luta com a ideia da inevitabilidade da morte não cabe em
descrição nenhuma. Exigiria volumes. Sua vida ensombreceu. Os dias iam
se passando e o problema se tornava cada vez mais angustioso. A morte é
um fato universal. Até aquela data não lhe constava que ninguém houvesse
deixado de morrer. Ele, portanto, morreria também --- era o inevitável.
O mais que poderia fazer era prolongar a vida até os setenta, até
oitenta. Poderia mesmo chegar a quase cem, como o Rockefeller --- mas ao
cabo teria de ir"-se, e então? Quem ficaria com os duzentos ou trezentos
mil contos que deveria ter por essa época?

Aquela história de herdeiros era o absurdo dos absurdos para um
celibatário de sua marca. Se a fortuna era dele, só dele, como deixá"-la
a quem quer que fosse? Não. Tinha de descobrir um jeito de não morrer,
ou\ldots{}

Lupércio interrompeu"-se no meio do raciocínio, tomado de súbita ideia.
Uma ideia tremenda, que por minutos o deixou de cérebro paralisado.
Depois sorriu.

--- Sim, sim\ldots{} Quem sabe? --- e seu rosto iluminou"-se de uma luz nova.
As grandes ideias emitem luz\ldots{}

Desde esse momento Lupércio revelou"-se outro, com preocupações que nunca
tivera antes. Não houve em Dois Rios quem o não notasse.

--- O homem mudou completamente --- diziam. --- Está se
espiritualizando. Compreendeu que a morte vem mesmo e começa a
arrepender"-se da sua feroz materialidade.

Lupércio fez"-se espiritualista. Comprou livros, leu"-os, meditou"-os.
Passou a frequentar o centro espírita local e a ouvir com a maior
atenção as vozes do Além, transmitidas pelo Chico Vira, o famoso médium
da zona.

--- Quem havia de dizer! --- era o comentário geral. --- Esse usurário,
que passou a vida inteira só pensando em dinheiro e nunca foi capaz de
dar um tostão de esmola, está virando santo. E vão ver que faz como o
Rockefeller: deixa toda a fortuna para o Asilo de Mendigos\ldots{}

Lupércio, que nunca lera coisa nenhuma, estava agora se tornando um
sábio, a avaliar pelo número de livros que adquiria. Entrou a estudar a
fundo. Sua casa fez"-se centro de reuniões de quanto médium aparecia por
lá --- e muitos de fora vieram a Dois Rios a convite seu. Generosamente
hospedava"-os, pagava"-lhes a conta do hotel --- coisa inteiramente
aberrante dos seus princípios financeiros. O assombro da população não
tinha limites.

Mas o doutor Dunga, diretor do Centro Espírita, começou a estranhar uma
coisa: o interesse do coronel Lupércio pela metapsíquica centrava"-se num
só ponto --- a reencarnação. Só isso o preocupava realmente. Pelo resto
passava como gato por brasas.

--- Escute, irmão --- disse ele um dia ao doutor Dunga. --- Há na teoria
da reencarnação um ponto para mim obscuro e que no entanto me apaixona.
Por mais autores que eu leia, não consigo firmar as ideias.

--- Que ponto é esse? --- indagou o doutor Dunga.

--- Vou dizer. Já não tenho dúvidas sobre a reencarnação. Estou
plenamente convencido de que a alma, depois da morte do corpo, volta ---
reencarna"-se em outro ser. Mas em quem?

--- Como em quem?

--- Em quem, sim. Meu ponto é saber se a alma do desencarnado pode
escolher o corpo em que vai novamente encarnar"-se.

--- Está claro que escolhe.

--- Até aí vou eu. Sei que escolhe. Mas ``quando'' escolhe?

O doutor Dunga não percebia o alcance da pergunta.

--- Escolhe quando chega o momento de escolher --- respondeu.

A resposta não contentou o coronel. O momento de escolher! Bolas! Mas
que momento é esse?

--- Meu ponto é o seguinte: saber se a alma de um vivo pode
antecipadamente escolher a criatura em que vai futuramente encarnar"-se.

O doutor Dunga estava tonto. Fez cara de não entender nada.

--- Sim --- continuou Lupércio. --- Quero saber, por exemplo, se a alma
de um vivo pode antes de morrer marcar a mulher que vai ter um filho em
quem essa alma se encarne.

A perplexidade do doutor Dunga recrescia.

--- Meu caro --- disse por fim Lupércio ---, estou disposto a pagar até
cem contos por uma informação segura --- seguríssima. Quero saber se a
alma de um vivo pode antes de desencarnar"-se escolher o corpo da sua
futura reencarnação.

--- Antes de morrer?

--- Sim\ldots{}

--- Em vida ainda?

--- Está claro\ldots{}

O doutor Dunga quedou"-se pensativo. Estava ali uma hipótese em que
jamais refletira e sobre que nada lera.

--- Não sei, coronel. Só vendo, só consultando os autores --- e as
autoridades. Nós aqui somos bem pouco neste assunto, mas há mestres na
Europa e nos Estados Unidos. Podemos consultá"-los.

--- Pois faça"-me o favor. Não olhe as despesas. Darei cem contos, e até
mais, em troca de uma informação segura.

--- Sei. Quer saber se ainda em vida do corpo podemos escolher a
criatura em que vamos reencarnar"-nos\ldots{}

--- Exatamente.

--- E por que isso?

--- Maluquices de velho. Como ando a estudar as teorias da reencarnação,
lógico que me interesso pelos pontos obscuros. Os pontos claros esses já
os conheço. Não acha natural a minha atitude?

O doutor Dunga teve de achar naturalíssima aquela atitude.

Enquanto as cartas de consulta cruzavam o oceano, endereçadas às mais
famosas sociedades psíquicas do mundo, o estado de saúde do coronel
Lupércio agravou"-se --- e concomitantemente se agravou a sua pressa pela
solução do problema. Chegou a autorizar pedido de resposta pelo
telégrafo --- custasse o que custasse.

Certo dia o doutor Dunga, tomado de vaga desconfiança, foi procurá"-lo em
casa. Encontrou"-o mal, respirando com esforço.

--- Nada ainda, coronel. Mas a minha visita tem outro fim. Quero que o
amigo fale claro, abra esse coração. Quero que me explique a verdadeira
causa do seu interesse pela consulta. Francamente, não acho natural
isso. Sinto, percebo, que o coronel tem uma ideia secreta na cabeça\ldots{}

Lupércio olhou"-o de revés, desconfiado. Mas resistiu. Alegou que era
apenas curiosidade. Como nos seus estudos sobre a reencarnação nada vira
sobre aquele ponto, viera"-lhe a lembrança de esclarecê"-lo. Só isso\ldots{}

O doutor Dunga não se satisfez. Insistiu:

--- Não, coronel, não é isso, não. Eu sinto, eu vejo, que o senhor tem
uma ideia oculta na cabeça. Seja franco. Bem sabe que sou seu amigo.

Lupércio resistiu ainda por algum tempo. Por fim confessou, com
relutância.

--- É que estou no fim, meu caro --- e tenho de fazer o testamento\ldots{}

Não disse mais, nem foi preciso. Um clarão iluminou o espírito do doutor
Dunga. O coronel Lupércio, a mais pura encarnação humana do dinheiro,
não admitia a ideia de morrer e deixar a fortuna aos parentes. Não se
conformando com a hipótese de separar"-se dos sessenta mil contos,
pensava em fazer"-se o herdeiro de si mesmo em outra reencarnação\ldots{}
Seria isso?

Dunga olhou"-o firmemente, sem dizer palavra. Lupércio leu"-lhe o
pensamento nos olhos inquisidores. Corou --- pela primeira vez na vida.
E, baixando a cabeça, abriu o coração.

--- Sim, Dunga, é isso. Quero que vocês me descubram a mulher em que vou
nascer de novo --- para fazê"-la em meu testamento a depositária da minha
fortuna\ldots{}

\chapter{A policitemia de Dona Lindoca\footnote[*]{Texto sem data, publicado no livro \emph{Negrinha}.}}

Dona Lindoca não era feliz. Quarentona bem puxada apesar dos trinta e
sete anos em que fizera finca"-pé, via pouco a pouco chegar a velhice com
seu empaste de feições, rugas e macacoas.

Não era feliz, porque nascera com o gênio da ordem e do asseio
meticuloso --- e gente assim passa a vida a amofinar"-se com criados e
coisinhas. E como também nascera casta e amorosa, não ia com o desamor e
desrespeito do mundo. O marido jamais lhe retribuíra o amor com os mimos
entressonhados em noiva. Não tinha ``caídos'', nem usava para a sua
sensibilidade, sempre menineira, desses pequeninos nadas cariciosos que
para certas criaturas constituem a suprema felicidade na terra.

Isso, porém, não traria a dona Lindoca mal de monta, excedente a
suspiros e queixas às amigas, se a certeza da infidelidade do Fernando
não viesse um dia estragar tudo. Estava a boa senhora a escovar"-lhe o
paletó quando sentiu vago aroma suspeito. Foi logo aos bolsos --- e
apanhou o corpo do delito num lencinho perfumado.

--- Fernando, você deu agora para usar perfume? --- indagou a santa
esposa aspirando o lenço comprometedor. --- E ``\emph{Coeur de
Jeannette}'', inda mais\ldots{}

O marido, pegado de surpresa, armou a cara mais alvar de toda a sua
coleção de ``caras circunstanciais'' e murmurou o primeiro rebate
sugerido pelo instinto de defesa:

--- Você está sonhando, mulher\ldots{}

Mas teve que render"-se à evidência, logo que a esposa lhe chegou ao
nariz o crime.

Há coisas inexplicáveis, por mais lépida que seja a presença de espírito
de um homem traquejado. Lenço cheiroso em bolso de marido que jamais
usou perfume, eis uma. Põe em ti o caso, leitor, e vai estudando desde
já uma saída honrosa para a hipótese de te suceder o mesmo.

--- Pilhéria de mau gosto do Lopes\ldots{}

O melhor que lhe acudiu foi lançar à conta do espírito brincalhão do seu
velho amigo Lopes mais aquela. Dona Lindoca, está claro, não engoliu a
grosseira pílula --- e desde aquele dia entrou a suspirar suspiros de um
novo gênero, com muita queixa às amigas sobre a corrupção dos homens.

Mas a realidade era diferente de tudo aquilo. Dona Lindoca não era
infeliz; seu marido não era um mau marido; seus filhos não eram maus
filhos. Gente toda ela muito normal, vivendo a vida que todas as
criaturas normais vivem.

Dava"-se apenas o que se dá sempre na existência da generalidade dos
casais pacíficos. A peça matrimonial ``Multiplicai"-vos'' tem um segundo
ato em excesso trabalhoso na procriação e criação dos rebentos. É uma
dobadoura de anos, na qual os atores principais mal têm tempo de cuidar
de si, tanto lhes monopolizam as energias os cuidados absorventes da
prole. Nesse período longo e rotineiro, quanto perfume vago não trouxe
da rua o doutor Fernando! Mas o olfato da esposa, sempre saturado com o
cheirinho das crianças, jamais deu tento de nada.

Um dia, porém, começou a dispersão. Casaram"-se as filhas e os filhos
foram deixando o borralho um por um, como passarinhos que já sabem fazer
uso das asas. E como o esvaziamento do lar ocorreu no período muito
curto de dois anos, o vácuo trouxe a dona Lindoca uma penosa sensação de
infelicidade.

O marido não mudara em coisa nenhuma, mas como só agora dona Lindoca
tinha tempo de dar"-lhe atenção, parecia"-lhe mudado. E queixava"-se dos
seus eternos negócios fora de casa, da sua indiferença, do seu
``desamor''. Certa vez, perguntou"-lhe ao jantar:

--- Fernando, que dia é hoje?

--- Treze, filha.

--- Treze só?

--- Está claro que 13 só. Impossível que fosse 13 e mais alguma coisa. É
da aritmética.

Dona Lindoca arrancou um suspiro dos mais sucados.

--- Essa aritmética antigamente era bem mais amável. Pela aritmética
antiga, hoje não seria 13 só e sim 13 de julho\ldots{}

O doutor Fernando bateu na testa.

--- É verdade, filha! Não sei como me escapou que é hoje dia dos teus
anos. Esta cabeça\ldots{}

--- Essa cabeça não falha quando as coisas a interessam. É que para você
eu já passei\ldots{} Mas console"-se, meu caro. Não me ando sentindo bem
e breve deixarei você livre no mundo. Poderá então, sem remorso,
regalar"-se com as Jeannettes\ldots{}

Como as recriminações alusivas ao caso do lenço perfumado fossem uma
``\emph{scie}'', o marido adotara a boa política de ``passar'', como no
pôquer. ``Passava'' todas as alusões da esposa, meio eficaz de torcer em
germe o pepino de um debate tão inútil quão indigesto. Fernando
``passou'' a Jeannette e aceitou a doença.

--- Sério? Sente qualquer coisa, Lindoca?

--- Uma ansiedade, uma canseira, isto desde que vim de Teresópolis.

--- Calor. Estes verões cariocas derrancam até aos mais pintados.

--- Sei quando é calor. O mal"-estar que sinto deve ter outra causa.

--- Nervoso, então. Por que não vai ao médico?

--- Já pensei nisso. Mas a qual médico?

--- Ao Lanson, filha. Que ideia! Pois não é o médico da casa?

--- Deus me livre. Depois que matou a mulher do Esteves? Isso quer
você\ldots{}

--- Não matou tal, Lindoca. É maldade inventada por aquela caninana da
Marocas. Ela é que diz isso.

--- Ela e todos. Voz corrente. Além do mais, depois daquele caso da
corista do Trianon\ldots{}

O doutor Fernando espirrou uma gargalhada.

--- Não diga mais nada! --- exclamou. --- Adivinho tudo. A eterna mania.

Sim, era a mania. Dona Lindoca não perdoava infidelidade de marido, nem
no seu nem no das outras. Em matéria de moralidade sexual não cedia
milímetro. Como fosse de natural casta, exigia castidade de todo mundo.
Daí o desmerecerem ante seus olhos todos os maridos que na voz das
comadres andavam de amores fora do ninho conjugal. Aquele doutor Lanson
perdera"-se no conceito de dona Lindoca não porque houvesse ``matado'' a
mulher do Esteves --- pobre tuberculosa que mesmo sem médico tinha de
morrer ---, mas porque andara às voltas com uma corista.

A gargalhada do marido enfureceu"-a.

--- Cínicos! São todos os mesmos\ldots{} Pois não vou ao Lanson. É um sujo.
Vou ao doutor Lorena, que é homem limpo, decente, um puro.

--- Vai, filha. Vai ao Lorena. A pureza desse médico, que eu cá chamo
hipocrisia requintada, com certeza lhe há de ajudar muito a terapêutica.

--- Vou, sim, e nunca mais me há de entrar aqui outro médico. De
Lovelaces ando eu farta --- concluiu dona Lindoca sublinhando a
indireta.

O marido olhou"-a de soslaio, sorriu filosoficamente e, ``passando'' o
``Lovelaces'', pôs"-se a ler os jornais.

No dia seguinte, dona Lindoca foi ao consultório do médico puritano e
voltou radiante.

--- Tenho uma policitemia --- foi logo dizendo. --- Garante ele que não
é grave, embora requeira tratamento sério e longo.

--- Policitemia? --- repetiu o marido com vincos na testa, sinal de que
entendia suas pitadas de medicina.

--- Que espanto é esse? Policitemia, sim, a doença da rainha Margarida e
da grã"-duquesa Estefânia, disse"-me o doutor. Mas cura"-me, assegurou ---
e ele sabe o que diz. Como é fino o doutor Lorena! Como sabe falar!\ldots{}

--- Sobretudo falar\ldots{}

--- Já vem você. Já começa a implicar com o homem só porque é um puro\ldots{}
Pois, quanto a mim, só sinto tê"-lo conhecido agora. É um médico decente,
sabe? Fino, amável, muito religioso. Religioso, sim! Não perde a missa
das onze na Candelária. Diz as coisas de um modo que até lisonjeia a
gente. Não é um sujo como o tal Lanson, que anda metido com atrizes, que
vê humores em tudo e põe as clientes nuas para examiná"-las.

--- E o teu Lorena, como as examina? Vestidas?

--- Vestidas, sim, está claro. Não é nenhum libertino. E se o caso exige
que a cliente se dispa em parte, ele aplica os ouvidos mas fecha os
olhos. É decente, ora aí está! Não faz do consultório casa de encontros.

--- Venha cá, minha filha. Noto que você fala com leviandade de sua
doença. Tenho minhas noções de medicina e parece"-me que essa tal
policitemia\ldots{}

--- Parece nada. O doutor Lorena afirmou"-me que não é coisa de matar,
embora de cura lenta. Doença até distinta, de fidalgos.

--- De rainhas, grãs"-duquesas, sei\ldots{}

--- Só que exige muito tratamento --- sossego, regime alimentar, coisas
impossíveis nesta casa.

--- Por quê?

--- Ora essa. Quer você que uma dona de casa possa cuidar de si tendo
tanta coisa em que olhar? Vá a pobre de mim deixar de matar"-se na
trabalheira, para ver como isto vira de pernas para o ar. Tratamento na
regra, só para essas que tomam o marido das outras. A vida é para
elas\ldots{}

--- Deixemos isso, Lindoca, até cansa.

--- Mas vocês não se cansam delas.

--- Elas, elas! Que elas, mulher? --- exclamou já exasperado o marido.

--- As perfumadas.

--- Bolas.

--- Não briguemos. Basta. O doutor\ldots{} ia"-me esquecendo. O doutor Lorena
quer que você apareça por lá, no consultório.

--- Para quê?

--- Ele dirá. Das duas às cinco.

--- Muita gente a essa hora?

--- Como não? Um médico daqueles\ldots{} Mas a você não fará esperar. É
negócio à parte da clínica. Vai?

O doutor Fernando foi. O médico desejava adverti"-lo de que a doença da
dona Lindoca era grave, havendo perigo sério caso o tratamento que
prescrevera não fosse seguido à risca.

--- Muito sossego, nada de contrariedades, mimos. Principalmente mimos.
Indo tudo a contento, num ano poderá estar boa. Do contrário, teremos
mais um viúvo em pouco tempo.

A possibilidade da morte da esposa, quando assim se antolha pela
primeira vez a um marido de coração sensível, abala profundamente. O
doutor Fernando deixou o consultório e rodando para casa ia a recordar o
tempo róseo do namoro, o noivado, o casamento, o enlevo dos primeiros
filhos. Não era mau marido.

Poderia até figurar entre os ótimos, no juízo dos homens que se perdoam
uns aos outros os pequenos arranhões no pacto conjugal, filhos da
curiosidade adâmica.

Já as mulheres não compreendem assim, e dão demasiado vulto a
borboleteios que muitas vezes só servem para valorizar as esposas aos
olhos dos maridos.

Assim é que a notícia da gravidade da moléstia de dona Lindoca despertou
em Fernando um certo remorso, e o desejo de redimir com carinhos de
noivo os anos de indiferença conjugal.

--- Pobre Lindoca. Tão boa de coração\ldots{} Se azedou um bocado, a culpa
foi só minha. O tal perfume\ldots{} Se ela pudesse compreender a absoluta
insignificância do frasco donde emanou aquele perfume\ldots{}

Ao entrar em casa indagou logo da esposa.

--- Está em cima --- respondeu a criada.

Subiu. Encontrou"-a no quarto, numa preguiçosa.

--- Viva a minha doentezinha! --- e abraçou"-a e beijou"-a na testa.

Dona Lindoca espantou"-se.

--- Ué! Que amores esses agora? Até beijos, coisas que me dizias fora da
moda\ldots{}

--- Vim do médico. Confirmou"-me o diagnóstico. Não há gravidade nenhuma,
mas exige tratamento de rigor. Muito sossego, nada de amofinações, nada
que abale o moral. Vou ser o enfermeiro da minha Lindoca e hei de pô"-la
sãzinha.

Dona Lindoca arregalou os olhos. Não reconhecia no indiferente Fernando
de tanto tempo aquele marido amável, tão perto do padrão com que sempre
sonhara. Até diminutivos\ldots{}

--- Sim --- disse ela ---, tudo isso é facil de dizer, mas sossego de
fato, repouso absoluto, como, nesta casa?

--- Por que não?

--- Ora, você será o primeiro a dar"-me aborrecimentos.

--- Perdoe"-me, Lindoca. Compreenda a situação. Confesso que não fui
contigo o esposo entressonhado. Mas tudo mudará. Você está doente e isso
vai fazer que tudo renasça --- até o velho amor dos vinte anos, que não
morreu nunca, apenas encasulou"-se. Não imagina como me sinto cheio de
ternura para com a minha mulherzinha. Estou todo lua de mel por dentro.

--- Os anjos digam amém. Só receio que com tanto tempo o mel já esteja
azedo\ldots{}

Apesar de mostrar"-se assim tão incrédula, a boa senhora irradiava. O seu
amor pelo marido era o mesmo dos primeiros tempos, de modo que aquela
ternura a fez logo reflorir, à imitação das árvores desfolhadas pelo
inverno a um chuvisco de primavera.

E a vida de dona Lindoca de fato mudou. Os filhos passaram a vir vê"-la
com frequência --- logo que o pai os advertiu da vida periclitante da
boa mãe. E mostravam"-se muito carinhosos e solícitos. Os parentes mais
chegados, também por influxo do marido, amiudaram as visitas, de tal
jeito que dona Lindoca, sempre queixosa outrora de isolamento, se fosse
queixar"-se agora seria de solicitude excessiva.

Veio uma tia pobre do interior tomar conta da casa, chamando a si todas
as preocupações amofinantes.

Dona Lindoca sentia um certo orgulho da sua doença, cujo nome lhe soava
bem aos ouvidos e fazia abrir a boca aos visitantes ---
\emph{policitemia}\ldots{} E como o marido e os demais lhe lisonjeassem a
vaidade enaltecendo o chique das policitemias, acabou por considerar"-se
uma privilegiada.

Falavam muito na rainha Margarida e na grã"-duquesa Estefânia como se
fossem pessoas da casa, havendo um dos filhos conseguido e posto na
parede o retrato de ambas. E certa vez em que os jornais deram um
telegrama de Londres noticiando achar"-se enferma a princesa Mary, dona
Lindoca sugeriu logo, convencidamente:

--- Vai ver que uma policitemia\ldots{}

A prima Elvira trouxe de Petrópolis uma novidade de sensação.

--- Viajei com um doutor Maciel na barca. Contou"-me que a baronesa de
Pilão Arcado também está com policitemia. E também aquela grandalhona
loura, mulher do ministro francês --- a Grouvion.

--- Sério?

--- Sério, sim. É doença de gente graúda, Lindoca. Este mundo!\ldots{} Até em
questão de doenças as bonitas vão para os ricos e as feias vão para os
pobres! Você, a Pilão Arcado e a Grouvion com policitemia --- e lá a
minha costureirinha do Catete, que morre dia e noite em cima da máquina
de costura, sabe o que lhe deu? Tísica mesentérica\ldots{}

Dona Lindoca fez cara de nojo.

--- Eu nem sei onde ``essa gente'' apanha tais coisas\ldots{}

Outra ocasião, ao saber que uma sua ex"-criada de Teresópolis fora ao
médico e viera com diagnóstico de policitemia, exclamou, incrédula, a
sorrir com superioridade:

--- Duvido! A Liduína com policitemia? Duvido!\ldots{} Vai ver que quem disse
tal bobagem foi o Lanson, aquela toupeira.

A casa virou perfeita maravilha de ordem. As coisas surgiam à hora e no
ponto, como se anões invisíveis estivessem a prover tudo. A cozinheira,
ótima, fazia pitéus de arregalar o olho. A arrumadeira alemã dava ideia
de uma abelha em forma de gente. A tia Gertrudes era uma governante de
casa como jamais existiu outra.

E nenhum barulho, todos na ponta dos pés, com \emph{pssius} aos
estouvados. E presentinhos. Os filhos e noras jamais esqueciam a boa
mamãe, ora com flores, ora com os doces de que ela mais gostava. O
marido fizera"-se caseiro. Deu jeito aos negócios e pouco saía, e à noite
nunca, passando a ler para a esposa os crimes dos jornais nas raras
vezes em que não tinham visitas.

Dona Lindoca começou a viver vida de céu aberto.

--- Como me sinto feliz agora! --- dizia. --- Mas para que nada haja
perfeito, tenho a policitemia. Verdade é que esta doença não me incomoda
em nada. Não a sinto absolutamente --- além de que é doença fina\ldots{}

O médico vinha vê"-la amiúde, mostrando boa cara à doente e má ao marido.

--- Demora ainda, meu caro. Não nos iludamos com aparências. As
policitemias são insidiosas.

O curioso era que dona Lindoca realmente não sentia coisa nenhuma. O
mal"-estar, a ansiedade do começo que a levara a consultar o médico, de
muito que havia passado. Mas quem sabia da sua doença não era ela e sim
o médico.

De modo que enquanto ele não lhe desse alta teria de continuar nas
delícias daquele tratamento.

Certa vez chegou a dizer ao doutor Lorena:

--- Sinto"-me boa, doutor, completamente boa.

--- Parece"-lhe, minha senhora. O característico das policitemias é
iludir assim os doentes, e pô"-los derreados, ou liquidados, à menor
imprudência. Deixe"-me cá levar o barco a meu modo, que para outra coisa
não queimei as pestanas na escola. A grã"-duquesa Estefânia também se
julgou boa, certa vez, e contra o parecer do médico assistente deu"-se
alta a si própria\ldots{}

--- E morreu?

--- Quase. Recaiu e foi um custo pô"-la de novo no ponto em que estava. O
abuso, minha senhora, a falta de confiança no médico, tem levado muita
gente para o outro mundo\ldots{}

E repetiu ao marido aquele parecer, com grande encanto de dona Lindoca,
que não cessava de abrir"-se em elogios ao grande clínico.

--- Que homem! Não é à toa que ninguém diz ``isto'' dele, neste Rio de
Janeiro das más"-línguas. ``Amantes, minha senhora'', declarou ele outro
dia à prima Elvira, ``ninguém me apontará jamais nenhuma.''

O doutor Fernando ia se saindo com uma ironia à moda antiga, mas
recolheu"-se a tempo, por amor ao sossego da esposa, com a qual jamais
esgrimira depois da doença. E resignou"-se a ouvir o estribilho de
sempre: ``É um homem puro e muito religioso. Fossem todos assim e o
mundo seria um paraíso''.

Durou seis meses o tratamento de dona Lindoca e duraria doze, se um belo
dia não rebentasse um grande escândalo --- a fuga do doutor Lorena para
Buenos Aires com uma cliente, moça da alta sociedade.

Ao receber a notícia dona Lindoca recusou"-se a dar crédito.

--- Impossível! Há de ser calúnia. Vai ver como ele logo aparece por
aqui e tudo se desmente.

O doutor Lorena jamais apareceu; o fato confirmou"-se, fazendo dona
Lindoca passar pela maior desilusão de sua vida.

--- Que mundo, meu Deus! --- murmurava. --- Em que mais acreditar, se
até o doutor Lorena faz dessas?

O marido rejubilou"-se por dentro. Sempre vivera engasgado com a pureza
do charlatão, comentada todos os dias em sua presença sem que ele
pudesse explodir o grito da alma que lhe punha um nó na garganta: ``Puro
nada! É um pirata igual aos outros''.

O abalo moral não fez dona Lindoca recair enferma, como era de supor.
Sinal de que estava perfeitamente curada. Para melhor certificar"-se
disso o marido lembrou"-se de consultar outro médico.

--- Pensei no Lemos de Souza --- sugeriu ele. --- Está com muito nome.

--- Deus me livre! --- acudiu logo a doente. --- Dizem que é amante da
mulher do Bastos.

--- Mas trata"-se de um grande clínico, Lindoca. Que importa o que lá do
seu namoro dizem as más"-línguas? Neste Rio ninguém escapa.

--- A mim importa muito. Não quero. Veja outro. Escolha um decente.
Sujeiras não admito aqui.

Depois de comprido debate acordaram em chamar o Manuel Brandão,
professor da Escola e já em adiantado grau de senilidade. Não constava
que fosse amante de ninguém.

Veio o novo doutor. Examinou cuidadosamente a doente e ao cabo concluiu
com absoluta segurança.

--- Vossa Excelência não tem nada --- disse ele. --- Absolutamente nada.

Dona Lindoca pulou, muito lépida, da sua preguiçosa.

--- Então sarei de uma vez, doutor?

--- Sarou\ldots{} se é que esteve doente. Não consigo ver sinal nenhum em seu
organismo de doença presente ou passada. Quem foi o médico?

--- O doutor Lorena\ldots{}

O velho clínico sorriu e, voltando"-se para o marido:

--- É o quarto caso de doença imaginária que o meu colega Lorena (aqui
entre nós, um refinadíssimo patife) leva a explorar durante meses.
Felizmente raspou"-se para Buenos Aires, ou ``desinfetou'' o Rio, como
dizem os capadócios.

Foi um assombro. O doutor Fernando abriu a boca.

--- Mas então\ldots{}

--- É o que lhe digo --- reafirmou o médico. --- A sua senhora teve
qualquer crise nervosa que passou com o repouso. Mas, policitemia,
nunca! Policitemia!\ldots{} Até me espanta que tão grosseiramente pudesse o
tal Lorena iludir a todos com essa pilhéria\ldots{}

A tia Gertrudes voltou para sua casa no interior. Os filhos foram se
tornando mais parcos nas visitas e os demais parentes idem. O doutor
Fernando retomou a vida de negócios e nunca mais teve tempo de ler
crimes para a desconsolada esposa, sobre cujos ombros recaiu a velha
trabalheira de zelar pela casa.

Em suma, a infelicidade de dona Lindoca voltou com armas e bagagens,
fazendo"-a suspirar suspiros ainda mais profundos que os de outrora.
Suspiros de saudade. Saudade da policitemia\ldots{}

\part[os parasitas donos, os decadentes e os olhodarruáveis]{\textsc{os parasitas donos, os decadentes\\ e os olhodarruáveis}}

\chapter{Café! Café!\footnote[*]{Texto de 1900, publicado no livro \emph{Cidades mortas}.}}

E o velho major recaiu em cisma profunda. A colheita não prometia pouco:
florada magnífica, tempo ajuizado, sem ventanias nem geadas. Mas os
preços, os preços! Uma infâmia! Café a seis mil"-réis, onde se viu isso?
E ele que anos atrás vendera"-o a trinta! E este Governo, santo Deus, que
não protege a lavoura, que não cria bancos regionais, que não obriga o
estrangeiro a pagar o precioso grão a peso de ouro!

E depois não queriam que ele fosse monarquista\ldots{} Havia de ser, havia de
detestar a República porque era ela a causa de tamanha calamidade, ela
com seus Campos Sales de bobagem.

Que tempos! Pois até o Chiquinho Alves, um menino que ele vira em
fraldas de camisa brincando na rua, não estava agora na chapa oficial
para deputado? Que tempos!

E com as magras mãos de velho engorovinhado o major torcia com frenesi
os bigodes amarelos de sarro.

Todo ele recendia a passado e rotina. Na cabeça já branca habitavam
ideias de pedra. Como essas famílias de caboclos que vegetam ao pé dos
morros numa choça de palha, cercada de taquara, com um terreirinho,
moenda e o chiqueiro e toda a imensidade azul e verde das serras e dos
céus a insulá"-las da civilização, assim a cabeça do major. As primeiras
ideias que ali abicaram, e isso já de sessenta anos, nas remotas eras do
bê"-á"-bá na escola do Ganimedes, meteram a foice na capoeira, fincaram os
paus da cerca, aprumaram os esteios da morada, cobriram"-na de sapé; e
lentamente, à medida que vinham entrando, compelidas pela vara de
marmelo e a rija palmatória do feroz pedagogo, foram erigindo a casa
mental do nosso herói. Depois, no começo da vida prática, como
administrador da fazenda paterna, novas ideias e novos conhecimentos,
filhos da experiência, tiveram guarida na choça daquele cérebro,
acrescendo"-o de mais
uns puxados ou telheirinhos. Juízos sobre o Governo, apreciações sobre
Suas Majestades, conceitos transmitidos por pais de família e coronéis
da Guarda Nacional, ideias religiosas embutidas pelo roliço padre
Pimenta, oráculo da família, receitas para quebrantos, a trenzama toda
moral e intelectual da sua psíquica de matuto ricaço, por lá se arrumou
com o tempo, apesar do acanhamento da choça e das dependências. Para o
chiqueirinho foram as anedotas frescas e as chalaças pesadas aprendidas
na botica do Zeca Pirula. E ficou nisso o meu major; se uma ideiazita
nova voava para ele, batia de peito em seus ouvidos moucos, como
rolinhas em paredes caiadas, caindo morta no chão; ou como borboleta em
casa aberta, entrava por uma orelha e saía por outra. Ficou naquilo o
major Mimbuia, uma pedra, um verdadeiro monólito que só cuidava de
colher café, de secar café, de beber café, de adorar o café. Se algum
atrevido ousava insinuar"-lhe a necessidadezinha de plantar outras
coisinhas, um mantimentozinho humilde que fosse, Mimbuia fulminava"-o com
apóstrofes.

--- O café dá para tudo. Isso de plantar mantimento é estupidez. Café.
Só café.

--- Mas, com seu perdão, major, se algum dia, que Deus nos livre, o café
baixar e\ldots{}

--- O café não baixa e se baixar sobe de novo. Vocês não entendem dessa
história --- e depois, olhe, eu não admito ideias revolucionárias em
minha casa, já ouviu?

E estava acabado, o pedreiro"-livre murchava as orelhas e abalava de rabo
encolhido.

Veio, porém, a baixa; as excessivas colheitas foram abarrotando os
mercados, dia a dia os estoques do Havre e de Nova York aumentavam. Os
preços baixavam sempre, cada vez mais; chegaram a dez mil"-réis, a nove,
a oito, a seis. O major ria"-se e limpando as unhas profetizava:

--- Em janeiro o café está a trinta e cinco mil"-réis.

Chegou janeiro; o café desceu a cinco mil e quinhentos. ``Em fevereiro
eu aposto que vai a quarenta!'' Foi a cinco.

O major emagrecia. ``Em março eu juro pela alma de meu pai, que Deus
haja, como o café há de subir a quarenta e cinco mil"-réis!'' O café em
março desceu a quatro.

O major enlouquecia. Estava à míngua de recursos, endividado, a fazenda
penhorada, os camaradas desandando, os credores batendo à porta. Já ia
para três anos que o produto das safras não bastava para cobrir o
custeio. Três déficits sucessivos devoraram"-lhe as economias e
estancaram as fontes. Mas o velho não desanimava. O cafezal estava um
brinco, sem um pezinho de capim. As casas desmoronavam, o mato viçava
nos terreiros, invadindo as tulhas, inundando tudo de clara verdura
vitoriosa, o caruru já estava cansado de nascer nos lugares proibidos
onde outrora, nem bem repontava medroso, já vinha um negro cambaio a
arrancá"-lo sem dó. O major passava a mandioca assada e canjica: nem
pitava mais daqueles longos cigarros de palha, por economia. Todo
dinheirinho que entrava das vendas do gado, de pedaços de terra, de
empréstimos, de velhas dívidas pagas, tudo ia para o Moloch insaciável
do cafezal. Chegado o tempo da colheita, colhia muito, as safras eram
ótimas, porém o produto das vendas nenhum alívio trazia à situação,
antes agravava"-a com um novo déficit. E como não, se o café estava
beirando os três mil a arroba e lhe saía a seis a produção de cada uma?

Aconselharam"-lhe o plantio de cereais; o feijão andava caro, o milho
dava bom lucro. Nada! O homem encolerizava"-se e rugia:

--- Não! Só café! Só café! Há de subir, há de subir muito. Sempre foi
assim. Só café. Só café!

E ninguém o tirava dali. A fazenda era uma desolação; a penúria,
extrema; os agregados andavam esfomeados, as roupas em trapo, imundos,
mas a trabalhar ainda, a limpar café, a colher café, a socar café. Os
salários, caídos no mínimo, uma ninharia, o quanto bastasse para matar a
fome. O velho roía as unhas rancorosamente, vomitando injúrias contra os
tempos modernos, contra a estrangeirada, o Governo, os comissários, numa
cólera perene, e trabalhava no eito com os camaradas a limpar café, a
colher café.

--- Sobe, há de subir, há de chegar a trinta mil"-réis.

Para sustentar a luta vendeu uma nesga da fazenda --- um pedaço da sua
própria carne.

Depois vendeu outra, mais outra e outra. O Moloch insaciável, porém,
engoliu tudo e pediu mais. Ele vendeu mais: vendeu os pastos, vendeu por
fim a casa de morada com todas as benfeitorias e foi residir num
ranchinho no cafezal.

A situação piorava, os preços continuavam a cair, o velho já estava sem
unhas para roer e sem mandioca para se alimentar. Só possuía o cafezal,
sempre limpo, sempre sem um matinho. Um dia desertou uma leva de
camaradas: outros seguiram aqueles e em breve Mimbuia viu"-se
completamente só no seu ranchinho do cafezal. Levantava"-se antes de
clarear o dia e saía de enxada em punho, numa raiva surda, a capinar, a
capinar o dia inteiro como um possesso.

Depois, como o cafezal fosse grande e ele um só, o mato brotou
luxuriante, numa alegria verde"-clara de vitória. O velho, possesso,
dentes cerrados, surdo ao sol e à chuva, seminu, esfarrapado e
macilento, baba a escorrer dos cantos da boca, torrado pela soalheira,
sujo de terra, já não podendo vencer o mato exuberante, andava a
arrancar as ervas mais atrevidas ou graúdas, catando uma aqui, outra
ali.

A luta era gigantesca, de vida ou de morte. Pelo cafezal todo as ruas
outrora vermelhas e varridas eram extensas faixas do verde vitorioso. A
beldroega alastrava"-se, o caruru já florescia, o picão derrubava as
sementes novas para nova seara mais farta e pujante.

Pintassilgos inúmeros trilavam pelo chão banqueteando"-se à farta nas
sementes dos capins. As rolinhas rebolavam, arrulhando, roliças, de
papinho duro. Os tico"-ticos, como legiões de bárbaros, tagarelavam
fabricando ninhos, pondo ovos, chocando"-os, tirando ninhadas famintas. O
sol rompia todas as madrugadas, fecundo, forte, vencedor, criando seiva
intensa, acariciando as ervas transbordantes. Chuvas contínuas davam à
terra magnífica um fofo de alfobre. O velho Mimbuia estava um espectro,
já nu de todo, os olhos esbugalhados a se revirarem nas órbitas com
desvario. Um espectro sem carnes, só pele calcinada e ossos pontiagudos.
Mas quando a boca se abria naquela barba hirsuta, o que vinha era uma
coisa só:

--- Há de subir, há de subir, há de chegar a sessenta mil"-réis em julho.
Café, café, só café!\ldots{}

\chapter{O plágio\footnote[*]{Texto de 1905, publicado no livro \emph{Cidades mortas}.}}

--- Você sai, Nenesto, com um tempo destes?

--- Não há outro.

--- Dia de São Bartolomeu, inda mais?\ldots{}

--- Importa"-me lá o santo.

--- Está bem. Depois não se arrependa\ldots{}

Isto dizia dona Eucaris ao ``queixo"-duro'' do seu marido Ernesto
d'Olivais, ao vê"-lo tomar o chapéu do cabide para sair.

Fora, remoinhava o vento, anunciando tempestade próxima. Por castigo,
nem bem caminhara o teimoso duzentos passos e desaba o aguaceiro. Tão
repentino que mal teve tempo de barafustar por um sebo adentro, no
instante preciso em que o belchior cerrava a última folha da porta.
Mesmo assim resfriou"-se e foi com três espirros que retribuiu à saudação
do homem.

\emph{--- Atchim!\ldots{}}

--- Viva!

--- \emph{Atchim!\ldots{}}

--- Viva!

--- \emph{Atchim}\ldots{} Brr! Pra burro! Espirro pra burro. \emph{C'est
le diable}.

(Século trinta! Se por acaso um exemplar deste livro chegar ao
conhecimento dos teus fariscadores de antigualhas, não se assombrem eles
com a expressão curralina do meu Ernesto. Nem quebrem a cabeça a
interpretá"-la com ajuda da filologia comparada, da veterinária e mais
ciências conexas. Cá fica a chave do enigma. A expressão ``pra burro''
viveu correntia pelas imediações da Grande Guerra, com significação de
abundante, excessivo ou estupendo. Nascida nalguma cocheira, alargou"-se
às ruas e passou destas aos salões. Penetrou até na retórica amorosa.
Romeus houve que, pintando a formosura das respectivas Julietas,
substituíram o arcaico \emph{linda como os amores} por este soberbo jato
de impressionismo cavalar: \emph{É linda pra burro!} Não obstante, as
Julietas casavam com eles e eram felizes. Lá se entendiam.)

O belchior era francês, e Ernesto taramelava na língua adotiva do senhor
Jacques d'Avray o necessário para embrulhar língua com um belchior
francês. Sabia diferençar \emph{femme sage} de \emph{sage femme},
distinguia \emph{chair} de \emph{viande} e alambicava a primor os
\emph{uu} gauleses. Além disso tinha ciência de vários idiotismos,
usando amiúde o \emph{qu'est"-ce que c'est que ça?}; sabia de cor a
história do \emph{Didon dit"-on}, além de uma dúzia de prosopopeias de
alto calibre, forrageadas nos \emph{Miseráveis} de Victor Hugo --- o que
já é bagagem glóssica de peso para um carrapato orçamentário com seis
anos de sucção.

Tais conhecimentos, mensalmente postos em jogo, bastavam para espezinhar
a paciência do livreiro, a quem Ernesto, em todo dia 2 de cada mês,
tomava alugado um bacamarte de Escrich, matador das horas vazias da
repartição.

Naquela tarde, porém, Ernesto não queria livros, sim um teto, razão pela
qual falhou o usual encetamento da seca. (Esse ritual começava assim:
\emph{Qu'est"-ce que vous avez de nouveau, monsieur?})

Fora, em rogougos sibilantes, o vento pulverizava a chuva.

Tinha de esperar.

Ernesto esperou. Esperou a remexer as estantes, a folhear revistas, a
ler a meia"-voz os títulos dourados. De longe em longe tomava dum volume
e perguntava ao francês acurvado na escrituração de um livro de capa
preta:

--- \emph{Combien, monsieur?\ldots{}}

E a resposta do homem repicava invariavelmente:

\emph{--- C'est très salé, c'est très salé, c'est très salé} ---
estribilho trauteado em surdina até que novo livro lhe empolgasse a
atenção.

Empolgou"-lha, logo depois, uma brochura esborcinada: \emph{A maravilha},
de Ernesto Souza.

--- Olé! Um xará! \emph{Combien, monsieur?}

O livreiro, sem maior atenção, rosnou qualquer coisa, enquanto Ernesto,
absorto no manuseio do livro, ia murmurando maquinalmente o \emph{très
salé}\ldots{} Leu"-lhe o período inicial e o final, vezo antigo adquirido no
colégio, onde colecionava num caderninho a primeira e a última frase de
quanto livro lhe transitava pela carteira.

\emph{A maravilha} era um desses romances esquecidos, que trazem o nome
do autor à frente duma comitiva de identificações, à laia de passaporte
à posteridade, muito em moda no tempo do onça:

\textsc{alfredo maria jacuacanga}

\emph{(Natural do Recife)}

\emph{3º anista da Escola de Medicina da Bahia}

\emph{ou}

\textsc{doutor cornélio rodrigues fontoura}

\emph{Ex"-lente disto, ex"-diretor daquilo, ex"-membro do Pedagogium,
ex"-deputado provincial, ex"-cavaleiro da Cruz Preta etc. etc.}

Romances descabelados, onde há lágrimas grandes como punhos, punhais
vingativos e virtudes premiadíssimas de par com vícios arquicastigados
pela intervenção final e apoteótica do Dedo de Deus --- livros que a
traça rendilhou nos poucos exemplares escapos à função, sobre todas
bendita, de capear bombas de foguetes.

O período final rezava assim: ``E um rubro fio de sangue correu do níveo
seio da donzela apunhalada como uma víbora de coral num mármore pagão''.

Ernesto, \emph{né} de Oliveira mas d'Olivais por contingências
estéticas, enrubesceu de apolíneo prazer. E assoou"-se, demonstração
muito sua de entusiasmo chegado a ponto de arrepio.

--- Sim, senhor! Está aqui uma frase soberba! ``Como víbora de
coral\ldots{}''.

Magnífico! E este ``mármore pagão''\ldots{}

Foi ter com o \emph{monsieur} e leu"-lha ``com alma''; mas o tipo,
absorvido numa edição, miou apenas o \emph{oui}, \emph{oui}, sem sequer
erguer a cabeça.

Ernesto não comprou o livro (não era 2 do mês), mas escondeu"-o num
desvão para que até o dia aquisitivo ninguém lhe pusesse a vista em
cima.

Entrementes a chuva amainara. Ernesto entreabriu a porta para a rua
murmurejante e resolveu abalar.

--- \emph{Monsieur, au revoir!}

--- \emph{Oui, oui} --- miou pela última vez o belchior.

Na rua endireitou para casa, ruminado que, sim, senhor, era ter fogo
sagrado! Uma frase daquelas fazia um nome. O xará tinha talento. Bem
dizia Victor Hugo nos \emph{Miseráveis} que o gênio\ldots{} é o gênio.

E foi pelo caminho a redizê"-la com cariciosa unção, a remirá"-la de todos
os lados, sob todas as luzes. Degustou"-a em surdina inúmeras vezes; pela
forma, revendo o jeito com que a fixaram no papel os caracteres
tipográficos; pelas correções associadas, evocando vagos helenismos
clássicos que o padre mestre Jordão lhe embutira no cérebro a
palmatoadas --- Frineia, o cão de Alcebíades, as Termópilas, o barril de
Diógenes.

Por fim, à noite, já a preciosa frase se lhe incrustara nos miolos, no
lugar onde costumavam encruar as ideias fixas. Chegou a repeti"-la à dona
Eucaris.

Mas dona Eucaris, uma criatura sovada, toda virtudes conjugais e
preocupações caseiras, interrompeu"-o prosaicamente:

--- E você trouxe, Ernesto, o pavio de lampião que encomendei?

Ernesto d'Olivais arrepanhou a cara num assomo de dó ante a chinfrinice
mental da companheira. Dó, despeito e meia cólera, coisa rara em seu imo
de amanuense gomoso e manso.

--- Que pavio? Que me importa o pavio? Quem fala aqui de pavio? Ora, não
me aborreça com histórias de pavio!

E voltando"-se para o canto (que a cena se passava na cama) embezerrou. O
sono dessa noite não foi bom conselheiro, e no dia seguinte Ernesto
andou pela repartição mais meditativo que do costume, com olhos parados
--- olhos de cobra morta que olham sem ver.

É que uma ideia\ldots{}

Não era bem uma ideia ainda, mas células vagas, destroços vogantes de
ideias mortas, lampejos de ideias futuras, coisas tão afins que ao cabo
de três dias se englobavam numa ideia"-mãe de imperiosa vitalidade.

--- Escrever um conto, uma simples ``variedade'', em linguagem bem
caprichada, com floreados bem bonitos, arabescos de alto estilo\ldots{} Duas
ou três personagens --- não gostava de muita gente. --- Um conde, uma
condessa pálida, a cidade de Três Estrelinhas, o ano de 18\ldots{} Como
enredo, uma paixão violenta da condessa de \textsc{x} pelo pintor Gontran ---,
gostava muito deste nome. A cena, já se sabe, passava"-se na França, que
nunca achara jeito em personagens nacionais, vivendo em nosso meio, ao
nosso lado. Perdiam o encanto. A narrativa vinha crescendo até
engastar"-se naquele final\ldots{} oh, sim!\ldots{} naquele final, porque, em suma,
o conto só viveria para justificar a exibição daquela joia de ``celinio
lavor''. E logo abaixo o seu nome por extenso: Ernesto da Cunha Olivais.

Esse remate furtado ao xará d'\emph{A maravilha} insinuou"-se aos poucos
na consciência de Ernesto como coisa muito sua, propriedade artística
indiscutível.

\emph{A maravilha}, ora! Um miserável caco de livro cuja existência
ninguém conhecia\ldots{}

Plágio? Como plágio? Por que plágio? É tão comum duas criaturas terem a
mesma ideia\ldots{} Coincidência apenas\ldots{} E, além disso, quem daria pela
coisa?

Ernesto era literato.

``Fazer literatura'' é a forma natural da calaçaria indígena. Em outros
países o desocupado caça, pesca, joga o murro. Aqui beletra. Rima
sonetos, escorcha contos ou tece desses artiguetes inda não
classificados nos manuais de literatura, onde se adjetiva sonoramente
uma aparência de ideia, sempre feminina, sem pés e raramente sem cabeça,
que goza a propriedade, aliás preciosa, de deixar o leitor na mesma. A
gramática sofre umas tantas marradas, os tipógrafos lá ganham sua vida,
as beldades se saboreiam na cândi"-adjetivação e o sujeito autor lucra
duas coisas: mata o tempo, que entre nós em vez de dinheiro é uma
simples maçada, e faz jus a qualquer academia de letras, existente ou
por existir, de Sapopemba a Icó.

Ernesto não fugira à regra. Em moço, enquanto vivia às sopas do pai à
espera de que lhe caísse do céu amanuensado, fundara \emph{A Violeta},
órgão literário e recreativo, com charadas, sonetos, variedades e mais
mimos de Apolo e Minerva. Redigiu depois certa folha ``crítica,
científica e literária'' com dois \emph{tt}, \emph{O Combatente}, que
morreu aos sete meses, combatendo a gramática até no derradeiro transe.
Compôs nesse intervalo, e publicou, um livro de sonetos, cuja impressão
deu com o pai na miséria.

Incompreendido pelo público, que não percebia o advento de um novo
gênio, Ernesto amargou como peroba da miúda, deixou crescer grenha e
barba, esgrouviou"-se, virou"-se e disse cobras cascavéis do país, do
público, da crítica, de José Veríssimo e da ``cambada'' da Academia de
Letras. Citava amiúde Schopenhauer e Kropotkin, mostrando tendências
para saltar dum pessimismo inofensivo ao perigoso niilismo russo. Foi
quando o pai, farto das atitudes teatrais do filho, meteu"-o numa roda de
guatambu e pô"-lo fora de casa com um valente pontapé:

--- Vá ganhar a vida, seu anarquista de borra!

Ernesto, jururu, achegou"-se a um tio influente na política e afinal
cavou o empreguinho. No empreguinho amou, casou e tomou a seu cargo a
seção ``Conselhos Úteis'' d'\emph{O Batalhador}. Estava nisso quando
ventou, choveu, entrou no sebo, pilhou \emph{A maravilha} e patinhou
como Hamlet no pego da indecisão, até que\ldots{}

Ernesto, em tiras de papel de Governo, lançou em belo cursivo um lindo
começo bem arredondado:

``Era por uma dessas noites de abril, em que o céu recamado de estrelas
lembra um manto negro com mil buraquinhos\ldots{}''

Na roda de orçamentívoros que domingueiramente bebericavam o chá com
torradas de dona Eucaris, todos afinados pela cravelha do Ernesto ---
vítimas imbeles da incompreensão ---, o conto estampado n'\emph{O Lírio}
causou agradável surpresa. O João Damasceno foi o primeiro a dar"-lhe um
abraço num vai e vem de café.

--- Olha, li o teu ``Never more'' n'\emph{O Lírio}. Esplêndido! O final,
então, divino! Tens miolo, meu caro! Pagas o chope?

Nesse dia Ernesto contou à esposa toda a vida do João, terminando
cismático:

--- É um caráter, Eucaris, um nobilíssimo caráter\ldots{}

O capitão Prelidiano, chefe da sua seção, foi comedido e pausado como
convinha à eminência do seu tamanco:

--- Li o seu trabalho, senhor Ernesto, e gostei; termina com
brilhantismo; continue, continue\ldots{}

E o Claro Vieira? Fora brutal, esse.

--- Que ótimo fecho arranjaste para o teu conto! O resto está pulha, mas
o final é \emph{un morceau de roi!}

O que nessa noite dona Eucaris ouviu relativo ao caráter baixo, infame e
vil do Claro\ldots{}

Ernesto entrou"-se de receios. Pareceu"-lhe que o Claro estava no segredo
do ``encontro de ideias''. Como medida de precaução deu busca aos sebos
em cata de quanto exemplar d'\emph{A maravilha} empoava por lá.
Encontrou meia dúzia, adquiriu"-os e queimou"-os, com grande assombro de
dona Eucaris, que duvidou da integridade dos miolos maritais ao vê"-lo
transfeito em Torquemada de inocentes brochuras carunchosas.

Mas nem assim sossegou.

--- Quem me assegura não existirem outras, espalhadas aí pelas
bibliotecas públicas? Se ao menos houvesse eu variado a forma,
conservado apenas a ideia\ldots{}

Fora audacioso, não havia dúvida. Fora tolo, pois não.

--- Sou uma besta, bem mo dizia o pai\ldots{}

Ernesto arrependeu"-se do plagiato --- sim, porque, afinal de contas,
vamos e venhamos, era um plágio aquilo! Sua consciência proclamava"-o de
cabeça erguida, reagindo contra as chicanas peitadas em provar o
contrário. E Ernesto arrependia"-se, sobretudo por causa do ``Dizem\ldots{}''
d'\emph{O Cromo}. Constava ser Claro o enredeiro daquelas maldades --- e
Claro era impiedoso na mofina. Sabia revestir as palavras dum jossá
urente de urtiga.

Fizera mal, sim, porque, afinal de contas, um plágio\ldots{} é sempre um
plágio.

Quando no domingo seguinte recebeu \emph{O Cromo}, tremeu ao correr os
olhos pelo ``Dizem\ldots{}''. Mas não vinha nada e respirou. No ``Recebemos e
Agradecemos'' havia boa referência ao conto, muito elogiosa para o
remate.

Também \emph{A Dalila} desse dia trouxe algo: ``O conto do senhor F. é
um desses etc. etc. O final é uma dessas frases que chispam beleza
helênica etc.''.

--- O final, sempre o final! Estão todos apostados em fazerem"-me perder
a paciência. Ora pistolas!

Ernesto deblaterou contra os jornalistas, contra os amigos, contra os
dez exemplares d'\emph{O Lírio} em seu poder --- dez arautos do seu
crime. E queimou"-os.

Na repartição, a um novo elogio do Damasceno Ernesto rompeu
desabridamente.

--- Ora vá ser besta na casa da sogra!

Damasceno abriu a boca.

Nas palavras mais inocentes o pobre autor via alusões irônicas, diretas,
claras, brutais. Num simples ``bom dia'' enxergava risinhos de mofa. O
próprio capitão Prelidiano, honestíssima cavalgadura incapaz de ironias,
afigurava"-se"-lhe o chefe da tropa.

Conspiravam contra ele, não havia dúvida.

Ernesto pôs"-se em guarda. Fugiu dos amigos. Deu cabo do mate
domingueiro. Não podia sequer ouvir falar em literatura, o assunto
dileto de tantos anos. Emagreceu.

Dona Eucaris, pensabunda, matutava:

--- Serão lombrigas?

E deu"-lhe quenopódio às ocultas.

--- Afinal\ldots{}

--- Afinal? É o diabo ser a vida tão pouco romântica como é! Os casos
mais interessantes descambam a meio para o mais reles prosaísmo. Este do
Ernesto d'Olivais, por exemplo. Merecia fim trágico, duelo ou
quebramento de cara. Quando nada, uma remoçãozinha a pedido.

Mas seria mentir. Nem toda gente encontra, como Ernesto, remates de
estrondo à mão.

É o caso deste caso.

Ernesto adoeceu, mas sarou. O quenopódio revelou"-se um porrete para o
seu mal. Depois, com o decorrer do tempo, esqueceu o plágio. Os amigos
esqueceram o ``Never more''. \emph{O Lírio} morreu como morrem
\emph{Lírios}, \emph{Dalilas} e \emph{Cromos}: calote na tipografia.
Ernesto engordou. Já é major. Tem seis filhos. Continua a fazer
literatura --- clandestinamente, embora. E, se encontrar a talho de
foice um novo final de estrondo, plagiará de novo.

Moralidade há nas fábulas. Na vida, muito pouca --- ou nenhuma\ldots{}

\chapter{Cidades mortas\footnote[*]{Texto de 1906, publicado no livro \emph{Cidades mortas}.}}

A quem em nossa terra percorre tais e tais zonas, vivas outrora, hoje
mortas, ou em via disso, tolhidas de insanável caquexia, uma verdade,
que é um desconsolo, ressurte de tantas ruínas: nosso progresso é nômade
e sujeito a paralisias súbitas. Radica"-se mal. Conjugado a um grupo de
fatores sempre os mesmos, reflui com eles duma região para outra. Não
emite peão. Progresso de cigano, vive acampado. Emigra, deixando atrás
de si um rastilho de taperas.

A uberdade nativa do solo é o fator que o condiciona. Mal a uberdade se
esvai, pela reiterada sucção de uma seiva não recomposta, como no velho
mundo, pelo adubo, o desenvolvimento da zona esmorece, foge dela o
capital --- e com ele os homens fortes, aptos para o trabalho. E
lentamente cai a tapera nas almas e nas coisas.

Em São Paulo temos perfeito exemplo disso na depressão profunda que
entorpece boa parte do chamado Norte.

Ali tudo foi, nada é. Não se conjugam verbos no presente. Tudo é
pretérito.

Umas tantas cidades moribundas arrastam um viver decrépito, gasto em
chorar na mesquinhez de hoje as saudosas grandezas de dantes.

\textls[-10]{Pelas ruas ermas, onde o transeunte é raro, não matracoleja sequer uma
carroça; de há muito, em matéria de rodas, se voltou aos rodízios desse
rechinante símbolo do viver colonial --- o carro de boi. Erguem"-se por
ali soberbos casarões apalaçados, de dois e três andares, sólidos como
fortalezas, tudo pedra, cal e cabiúna; casarões que lembram ossaturas de
megatérios donde as carnes, o sangue, a vida para sempre refugiram.}

Vivem dentro, mesquinhamente, vergônteas mortiças de famílias fidalgas,
de boa prosápia entroncada na nobiliarquia lusitana. Pelos salões
vazios, cujos frisos dourados se recobrem da pátina dos anos e cujo
estuque, lagarteado de fendas, esboroa à força de goteiras, paira o
bafio da morte. Há nas paredes quadros antigos, \emph{crayons},
figurando efígies de capitães"-mores de barba em colar. Há sobre os
aparadores Luís \textsc{xv} brônzeos candelabros de dezoito velas, esverdecidos
de azinhavre. Mas nem se acendem as velas, nem se guardam os nomes dos
enquadrados --- e por tudo se agruma o bolor râncido da velhice.

São os palácios mortos da cidade morta.

\textls[-10]{Avultam em número, nas ruas centrais, casas sem janelas, só portas, três
e quatro: antigos armazéns hoje fechados, porque o comércio desertou
também. Em certa praça vazia, vestígios vagos de ``monumento'' de vulto:
o antigo teatro --- um teatro onde já ressoou a voz da Rosina Stoltz, da
Candiani\ldots{}}

Não há na cidade exangue nem pedreiros, nem carapinas; fizeram"-se estes
remendões; aqueles, meros demolidores --- tanto vai da última
construção. A tarefa se lhes resume em especar muros que deitam ventres,
escorar paredes rachadas e remendá"-las mal e mal. Um dia metem abaixo as
telhas: sempre vale trinta mil"-réis o milheiro --- e fica à inclemência
do tempo o encargo de aluir o resto.

\textls[-10]{Os ricos são dois ou três forretas, coronéis da Briosa, com cem apólices
a render no Rio; e os sinecuristas acarrapatados ao orçamento: juiz,
coletor, delegado. O resto é a \emph{mob}: velhos mestiços de miserável
descendência, roídos de opilação e álcool; famílias decaídas, a viverem
misteriosamente umas, outras à custa do parco auxílio enviado de fora
por um filho mais audacioso que emigrou. ``Boa gente'', que vive de
aparas.}

\textls[-30]{Da geração nova, os rapazes debandam cedo, quase meninos ainda; só ficam
as moças --- sempre fincadas de cotovelos à janela, negaceando um marido
que é um mito em terra assim, donde os casadouros fogem. Pescam, às
vezes, as mais jeitosas, o seu promotorzinho, o seu delegadozinho de
carreira --- e o caso vira prodigioso acontecimento histórico, criador
de lendas.}

Toda a ligação com o mundo se resume no cordão umbilical do correio ---
magro estafeta bifurcado em pontiagudas éguas pisadas, em eterno ir e
vir com duas malas postais à garupa, murchas como figos secos.

Até o ar é próprio; não vibram nele \emph{fonfons} de auto, nem cornetas
de bicicletas, nem campainhas de carroça, nem pregões de italianos, nem
\emph{ten"-tens} de sorveteiros, nem \emph{plás"-plás} de mascates sírios.
Só os velhos sons coloniais --- o sino, o chilreio das andorinhas na
torre da igreja, o rechino dos carros de boi, o cincerro de tropas
raras, o taralhar das baitacas que em bando rumoroso cruzam e recruzam o
céu.

Isso, nas cidades. No campo não é menor a desolação. Léguas a fio se
sucedem de morraria áspera, onde reinam soberanos a saúva e seus
aliados, o sapé e a samambaia. Por ela passou o Café, como um Átila.
Toda a seiva foi bebida e, sob forma de grão, ensacada e mandada para
fora. Mas do ouro que veio em troca nem uma onça permaneceu ali,
empregada em restaurar o torrão. Transfiltrou"-se para o Oeste, na avidez
de novos assaltos à virgindade da terra nova; ou se transfez nos
palacetes em ruína; ou reentrou na circulação europeia por mão de
herdeiros dissipados.

À mãe fecunda que o produziu nada coube; por isso, ressentida, vinga"-se
agora, enclausurando"-se numa esterilidade feroz. E o deserto lentamente
retoma as posições perdidas.

Raro é o casebre de palha que fumega e entremostra em redor o
quartelzinho de cana, a rocinha de mandioca. Na mor parte os
escassíssimos existentes, descolmados pelas ventanias, esburaquentos,
afestoam"-se do melão"-de"-são"-caetano --- a hera rústica das nossas
ruínas.

\textls[-5]{As fazendas são escoriais de soberbo aspecto vistas de longe,
entristecedoras quando se lhes chega ao pé. Ladeando a casa"-grande,
senzalas vazias e terreiros de pedra com viçosas guanxumas nos
interstícios. O dono está ausente. Mora no Rio, em São Paulo, na Europa.
Cafezais extintos. Agregados dispersos. Subsistem unicamente, como
lagartixas na pedra, um pugilo de caboclos opilados, de esclerótica
biliosa, inermes, incapazes de fecundar a terra, incapazes de abandonar
a querência, verdadeiros vegetais de carne que não florescem nem
frutificam --- a fauna cadavérica de última fase a roer os derradeiros
capões de café escondidos nos grotões.}

--- Aqui foi o Breves. Colhia oitenta mil arrobas!\ldots{}

A gente olha assombrada na direção que o dedo cicerone aponta. Nada
mais!\ldots{} A mesma morraria nua, a mesma saúva, o mesmo sapé de sempre. De
banda a banda, o deserto --- o tremendo deserto que o Átila Café criou.

Outras vezes o viajante lobriga ao longe, rente ao caminho, uma ave
branca pousada no topo dum espeque. Aproxima"-se devagar ao chouto
rítmico do cavalo; a ave esquisita não dá sinais de vida; permanece
imóvel. Chega"-se inda mais, franze a testa, apura a vista. Não é ave, é
um objeto de louça\ldots{} O progresso cigano, quando um dia levantou
acampamento dali, rumo a Oeste, esqueceu de levar consigo aquele
isolador de fios telegráficos\ldots{} E lá ficará ele, atestando mudamente
uma grandeza morta, até que decorram os muitos decênios necessários para
que a ruína consuma o rijo poste de ``candeia'' ao qual o amarraram um
dia --- no tempo feliz em que Ribeirão Preto era ali\ldots{}

\chapter{O luzeiro agrícola\footnote[*]{Texto de 1910, publicado no livro \emph{Cidades mortas}.}}

\section*{I}

\noindent{}Sizenando Capistrano é o inspetor agrícola do vigésimo distrito.
Incumbe"-lhe fomentar a pecuária, elaborar relatórios, ensinar o uso de
máquinas agrícolas, preconizar a policultura, combater a rotina e ao fim
de cada mês perceber na coletoria a realidade de setecentos mil"-réis.

Antes de inspetor Capistrano fora poeta. Cultivara as musas. Não sabia
que coisa era pé de café, mas entendia de pés métricos, pés"-quebrados e
fazia pé de alferes a todas as divas do Parnaso. Tal cultura,
entretanto, emagrecia"-o. A sua produção de endecassílabos, alexandrinos,
quadras, odes, sonetos, poemas, vilancetes, églogas, sátiras, anagramas,
logogrifos, charadas elétricas e enigmas pitorescos, conquanto copiosa,
não lhe dava pão para a boca, nem cigarro para o vício. A palidez de
Capistrano, sua cabeleira à Alcides Maia, sua magreza à Fagundes Varela,
seu \emph{spleen} à Lord Byron e suas atitudes fatais, ao invés de lhe
aureolarem a face dos nimbos da poesia, comiseravam o burguês, que, ao
vê"-lo deslizar como alma penada pela cidade, horas mortas, de mãos no
bolso e olho nostalgicamente ferrado na lua, murmurava condoído:

--- Não é poesia, não, coitado, é fome\ldots{}

O editor artilhava a cara de carrancas más quando Capistrano lhe surgia
escritório adentro com a maçaroca de versos candidatos à edição.

--- São versos puros, senhor, versos sentidos, cheios de alma. Virão
enriquecer o patrimônio lírico da humanidade.

--- E arruinar o meu patrimônio econômico --- retorquia a fera. --- Do
lirismo bastam"-me aquelas prateleiras que editei no tempo em que era
tolo e não se vendem nem a peso.

--- Ó vil metal! --- murmurava o poeta, franzindo os lábios num repuxo
de supremo enojo. --- Ó mundo vil! Ó torpe humanidade! Em que te
distingues, Homem, rei grotesco da criação, do suíno toucinhento que
espapaça nos lameiros? Manes de Juvenal! Eumênides! Musas de Cólera!
Inspirai"-me versos candentes com que cauterize até aos penetrais da alma
este verme orgulhoso e mesquinho! Baudelaire, dá"-me os teus venenos\ldots{}

--- Rapazes --- berrava o livreiro à caixeirada ---, ponham"-me este vate
no olho da rua!

Ante o \emph{manu militari} irretorquível, o poeta apanhava a papelada
lírica e moscava"-se para a zona neutra do passeio, onde, readquirida
altivez ossiânica, objurgava para dentro da loja hostil:

--- A Posteridade me vingará, javardos!

E sacudia à porta do editor o pó das suas sandálias, que no caso eram
surradas e já risonhas botinas de bezerro. Em seguida, remessando para
trás a cabeleira, num repelão, ia fincar"-se sinistramente à esquina
próxima, em torva atitude, à espera dum conhecido esfaqueável, a quem,
com gestos soberbos de Bergerac, extorquisse um níquel.

Cansado, entretanto, de ouvir estrelas em jejum, de amar a lua no céu
sem possuir um queijo na terra, acatou a voz do estômago e quebrou a
lira --- para viver. Meteu a tesoura nas melenas, deu brilho aos
sapatos, desfatalizou o semblante, substituiu o ar absorto do aedo pelo
ar avacalhado do pretendente, e à força de pistolões guindou"-se às
cumeadas do Morro da Graça.\footnote{Residência do general Pinheiro
  Machado, o mandão da política na época. Nota da edição de 1946.} Todo
mundo o recomendou ao Gaúcho Onipotente, porque todos andavam fartos
daquela perpétua fome lírica a deambular pelas ruas, caçando rimas e
filando cigarros. Que fosse acarrapatar"-se ao Estado. O Estado é um boi
gordo, semelhante àquela estátua equestre de Hindenburg, feita de
madeira, em que os alemães pregavam pregos de ouro. A diferença está em
que no Estado, em vez de tachas de ouro, pregam"-se Capistranos vivos.

Foi apresentado ao Pinheiro.

--- Então, menino, que quer?

--- Um empreguinho qualquer que Vossa Onipotência haja por bem
conceder"-me.

--- E para que presta você, menino?

--- Eu? Eu\ldots{} fui poeta. Cantei o amor, a Mulher, a Beleza, as manhãs
cor"-de"-rosa, as auroras boreais, a natureza, enfim. Romântico,
embriaguei"-me na Taverna de Hugo. Clássico, bebi o mel do Himeto pela
taça de Anacreonte. Evoluído para o parnasianismo, burilei mármores de
Paros com os cinzéis de Herédia. Quando quebrei a lira, estava
ascendendo ao cubismo transcendental. Sim, general, sou um gênio
incompreendido, novo Asverus a percorrer todas as regiões do ideal em
busca da Forma Perfeita. Qual Prometeu, vivi atado ao potro do
\emph{Inania Verba}, onde me roeu o Abutre da Perfeição Suprema. Fui um
Torturado da Forma\ldots{}

O general, que era amigo das belas imagens, iluminou o rosto de um
sorriso promissor.

--- Poeta --- disse ele ---, eu também sou poeta. Rimo homens. Componho
poemas herói"-cômicos. Conheces a \emph{Hermeida}? É obra minha. Amo as
belas imagens e tenho lançado algumas imortais. ``A mulher de César!''
``Os levitas do Alcorão!'' Hein? Tu me caíste em graça e, pois,
acolho"-te sob o meu pálio. Que queres ser?

--- Inspetor.

--- De quarteirão?

--- Isso não.

--- Agrícola?

--- Ou avícola\ldots{}

--- De que região?

--- Não faço questão.

--- Sê"-lo"-ás do vigésimo distrito. Conheces as culturas rurais?

--- Já cultivei batatas gramaticais.

--- E de pecuária, entendes? Distingues um Zebu dum galo Brama? Um pampa
dum murzelo?

--- Já cavalguei Pégaso em pelo.

--- Conheces a suinocultura? Sabes como se cria o canastrão?

--- Sei trincá"-lo com tutu de feijão.

--- És um gênio, não há que ver. Talvez faça de ti, um dia, presidente
da República. Teu nome?

--- Sizenando. Capistrano é sobrenome.

--- Cá me fica. Vai, que estás aí, estás fomentando a agricultura como
inspetor do vigésimo distrito, com setecentos bagos por mês. Os poetas
dão ótimos inspetores agrícolas e tu tens dedo para a coisa. Vai, levita
do Ideal\ldots{}

\section*{II}

Sizenando Capistrano, mal se pilhou transformado de famélico ouvidor de
estrelas em peça mestra do Ministério da Agricultura\ldots{} casou,
luademelou três meses e por fim compareceu perante o ministro para saber
em que rumos nortear a sua atividade.

O ministro franziu a testa: é tão difícil dar ocupação aos fósforos
ministeriais\ldots{} Pensou um bocado e:

--- Escreva um relatório --- sugeriu.

--- Sobre que, Excia.?

--- Sobre qualquer coisa. Relate, vá relatando. A função capital do
nosso ministério é produzir relatórios de arromba sobre o que há e o que
não há. Relate.

--- Mas, Excia., eu desejava ao menos uma sugestãozinha emanada do alto
critério de V. Excia., sobre o tema do relatório que a bem da lavoura V.
Excia., com tanto descortino, me incumbe de escrever\ldots{}

--- Já disse: sobre qualquer coisa que lhe dê na veneta. Relate, vá
relatando e depois apareça.

Sizenando saiu tonto com os processos expeditos do doutor
Grifado\footnote{Um ministro da Agricultura da época que não era doutor,
  mas não protestava contra o tratamento. Nota da edição de 1946.} com
assento na pasta, e passou três meses de papo ao ar, procurando uma tese
conveniente.

Como por essa época a lua de mel entrasse em plena minguante, houve
certo dia rusga brava ao jantar, e a consorte, mulherzinha de pelo
crespo no nariz, pespegou"-lhe pela cara com um prato de salada de
beldroega. Tal o célebre estalo que abriu a inteligência do padre
Antônio Vieira em menino, aquele obus culinário teve a estranha ação de
iluminar os refolhos cerebrais do inspetor.

--- Eureca! --- berrou ele radiante. E com um grande riso de gozo na
cara emplastada de verdura, ergueu"-se da mesa precipitadamente e correu
ao escritório. A mulherzinha, entre colérica e pasmada, perguntou de si
para si:

--- Estará louco?

Sizenando deitou mãos à tarefa e levou a cabo um estudo
botânico"-industrial da beldroega, com afã tal que, transcorridos dez
meses, dava a prelo o \emph{Relatório sobre o} Papalvum
brasiliensis\emph{, vulgo beldroega, e sua aplicação na culinária}.

O ano seguinte gastou"-o em rever as provas do calhamaço, a modo de
escoimá"-lo dos mínimos vícios de linguagem. O antigo torturado da Forma
ressurtia ali\ldots{} Saiu obra papa"-fina, em ótimo papel e com muitas
gravuras elucidativas. Entre estas, em belo destaque, os retratos do
ministro e do diretor da Agricultura, do Marechal Hermes, do tenente
Pulquério, do Frontim, do Pinheiro e mais protuberantes beldroegas do
momento. Pronta a edição, embaraçou"-se Sizenando quanto ao destino a
dar"-lhe. Que fazer de tanta beldroega?

Foi ao ministro.

--- Excelência! De acordo com as sábias ordens de V. Excia., venho
comunicar a V. Excia. que se acha pronta a edição do relatório sobre o
\emph{Papalvum}.

--- Que papalvo? Que relatório? --- inquiriu o ministro, deslembrado.

--- O que V. Excia. me incumbiu de escrever.

--- Quando?

--- Haverá dois anos.

--- Não me recordo, mas é o mesmo. Mande a papelada para o forno de
incineração da Casa da Moeda.

Sizenando abriu a maior boca deste mundo. Compreendendo aquela
estuporação, o ministro sorriu.

--- Então? Que queria que eu fizesse de cinco mil exemplares de um
relatório sobre a beldroega? Que o pusesse à venda? Ninguém o compraria.
Que o distribuísse grátis? Ninguém o aceitaria. Se é assim, se sempre
foi assim, se sempre será assim com todas as publicações deste
ministério, o mais prático é passar a edição diretamente da tipografia
ao forno. Isso evitará a maçada de nos preocuparmos com ela e de a
termos por aí a atravancar os arquivos. Não acha vossa senhoria que é o
mais razoável? Retire o que quiser e forno com o resto.

--- E depois, que devo fazer? --- indagou Sizenando, ainda tonto com o
expeditismo ministerial.

--- Escrever outro relatório --- respondeu sem vacilar o ministro.

--- Para ser queimado novamente? --- atreveu"-se a murmurar o poeta
inspetor.

--- Está claro, homem! Para que diabo despendeu o Governo tanto dinheiro
na montagem do forno? Está claro que para incinerar as notas velhas e os
relatórios novos. Deste modo se conservam em perpétua atividade o
pessoal da Imprensa, o do Forno e o dos Ministérios. Veja como é sábia a
nossa organização administrativa! A montagem do forno foi a melhor ideia
do Governo passado. Antes dele a Imprensa Nacional vivia entulhada de
impressos; a produção de relatórios, função capital deste ministério,
periclitava; e era tudo uma desordem, um desequilíbrio capaz de induzir
o Governo à supressão da Imprensa e do meu ministério. O forno sanou a
situação. O \emph{fervet opus} é magnífico e a espada de Dâmocles está
para sempre arredada de nossas cabeças. Hein? Vá. Escreva outro
relatório, sobre\ldots{} sobre\ldots{} o caruru, por exemplo.

Sizenando deixou o gabinete do ministro profundamente meditativo. S.
Excia. derrancara"-o!

Viu com dor de alma as chamas do Forno lerem aquele relatório tão
bemacabadinho, tão de encher o olho\ldots{} E sacou seis meses de licença com
vencimentos para descansar.

Esgotada a licença, ia Sizenando começar a pensar em preparar"-se para
escolher o papel e a tinta com que relatasse o caruru quando a política
apeou da administrança o doutor Grifado. Sizenando deixou que
transcorressem mais seis meses, ao termo dos quais se apresentou ao novo
ministro para lhe sondar a orientação.

O novo ministro era bacharel em ciências jurídicas e sociais, ex"-chefe
de polícia e tão entendido em agricultura como em arqueologia inca. Mas
lera uns números de \emph{Chácaras e Quintais} e ali se abeberara de
umas tantas noções sobre avicultura, policultura, criação de canários
etc. Fez dessas uras o seu programa. No discurso de apresentação, ao
empossar"-se no cargo, emitiu os seguintes conceitos, louvadíssimos pelos
circunstantes, empregados no ministério quase todos e verdadeiros
hortaliças em matéria agrícola.

--- A monocultura, senhores, é o grande mal; a policultura é o grande
bem; no dia em que produzirmos cebola, alho, batata, repolho, coentro,
alpiste, cerefólio, grão"-de"-bico, tremoço, quiabo, espargo, espinafre,
alcachofra\ldots{}

(Um arrepio de entusiasmo percorreu a espinha dos assistentes, que se
entreolharam gozosos, como quem diz: Temos homem pela proa!)

--- \ldots{} cebolinha, couve"-flor, sorgo, soja amarela, centeio, aveia,
figos da Trácia, uvas de Corinto, violetas de Parma\ldots{}

--- Bravíssimo!

--- \ldots{} violetas de Parma\ldots{} e outros cereais europeus (vermelhidão no
rosto), a prosperidade nacional se assentará num soco basáltico, do qual
não a arrancarão as mais rijas lufadas dos vendavais econômicos.
Conduzir a pátria a essa Canaã da policultura: eis a mira permanente dos
meus esforços, eis o meu programa, eis o fim supremo colimado pela minha
atividade. Espero, pois, que etc. etc.

Palmas, bravos, guinchos, silvos e outros sons denunciadores de
entusiasmo em grau de ebulição estrugiram pela sala. O ministro foi
abraçado e beijado --- nas mãos. Aquele salvaria a pátria, não havia a
menor dúvida!

O novo ministro da Agricultura era positivamente uma águia --- igual às
anteriores. Tinha programa. Visava confundir a rotina monocultora com
demonstrações práticas das magnificências da policultura mecânica.

Sizenando recebeu ordem de ir desatolar a vigésima região do atascal da
rotina. Aquela gente ainda vivia em pleno período da pedra lascada do
café; era mister tangê"-la à estação áurea da policultura, da avicultura,
da sericultura, da criação de canários hamburgueses etc., preluzida no
discurso do ministro.

Chegando à sede do distrito, com séquito numeroso e abundante farragem
mecânica, Sizenando distribuiu convites para a inauguração dum curso
prático. Escolheu para campo de demonstração um ``rapador''\footnote{Pasto
  de aluguel muito sovado; rapado. Nota da edição de 1946.} a um
quilômetro da cidade, e lá, no dia emprazado, reuniu os convivas. Veio o
prefeito municipal, o porteiro da Câmara, o coletor federal, o promotor
público, três jornalistas, quatro professores, o diretor do grupo
escolar com a meninada, o vigário da paróquia, o fiscal da iluminação
pública, o zelador do cemitério, o carcereiro, dois guarda"-chaves da
Central, cinco inspetores de quarteirão, o delegado, o cabo do
destacamento --- e \emph{um} fazendeiro recém"-despojado da sua
propriedade por dívidas. A turma docente e os bois do arado formavam
grupo à parte.

Sizenando trepou a um cupim e pronunciou breve alocução alusiva à
personalidade sobre"-excelente do ministro, e ao papel dos novos métodos
racionais na agricultura moderna.

--- O novo método, meus senhores, é baseado na ciência pura. Vem dos
laboratórios de braços dados à química. Começarei pela demonstração do
arado, ou charrua, a pedra angular de todo o progresso agrícola. Senhor
Primeiro Arador, arado para a frente!

Despegou"-se da turma um capataz, que empurrou para perto do cupim
tribunício um belo arado de disco. Rodearam"-no os circunstantes, como a
um animal raro.

--- Eis, meus senhores, um arado de disco. Esta parte se chama cabo;
esta é a roda, serve para rodar; estas rodelas são os discos, servem
para sulcar a terra; este ferrinho é a manivela graduadora; este
pauzinho é o balancim. Aqui se atrelam os bois e cá toma assento o
condutor.

A assistência abria a boca.

--- Vejamo"-lo agora em ação. Senhor Primeiro Condutor de Primeira
Classe, atrelar!

Adiantou"-se da turma um carreiro e tangeu os bois para a máquina,
jungindo"-os à canga. Os assistentes riram"-se. Acharam imensa graça no
Tomé Pichorra, que nunca fora senão o Tomé Pichorra, carreiro,
transformado em Primeiro Condutor de Primeira Classe! Era de
primeiríssima.

--- Senhor Primeiro Arador, arar!

O Primeiro Arador saltou à boleia e empunhou as manivelas. O Primeiro
Condutor aguilhoou a junta de bois.

--- `amo, Bordado! Puxa, Malhado!

Os dois caracus moveram"-se pesadamente. A terra, sulcada pelo ferro,
abriu"-se em leivas. Sizenando exultou.

--- Vejam, senhores, que maravilha! Faz o trabalho de vinte homens, além
de que deixa a terra desatada, com grande receptividade para a
meteorização atmosférica --- o que equivale a um adubamento copioso.

Este pedacinho encantou sobremodo ao zelador do cemitério, o qual não
conteve um sincero ``Muito bem!''.

Sizenando agradeceu com um gesto de cabeça. O arado deu umas tantas
voltas e emperrou. A banda de música, para disfarçar a entaladela,
rompeu o ``Vem cá, mulata''. E assim terminou a primeira parte da bela
demonstração agrícola.

A segunda constituiu no destorroamento e no gradeamento da terra, feitos
com o mesmo luxuoso aparato. Havia Primeiro e Segundo Destorroador,
Primeiro e Segundo Gradeador. Um mimo de hierarquia!

Ao terminar o serviço, a banda zabumbou um tanguinho.

A terceira parte foi absorvida pelo plantio de cebolas, batatas, alho,
alfafa e outras salvações nacionais.

--- Os senhores verão --- concluiu Sizenando --- que maravilhosa messe
vai brotar, farta, deste torrão sáfaro e ingrato só porque aplicamos
sumariamente os processos modernos da cultura racional, os quais
centuplicam a produção e diminuem o trabalho. A máquina agrícola é a
verdadeira alavanca do progresso!

--- Protesto! A alavanca do progresso sempre foi a imprensa ---
contraveio um jornalista, cioso da velha prerrogativa.

--- Será --- retrucou Sizenando ---; mas se uma, a imprensa, alçaprema o
progresso moral, a outra, a máquina agrícola, alçaprema o progresso
econômico!

--- Bravíssimo! --- rugiu o zelador do cemitério, inimigo pessoal do Zé
Tesoura. --- Isto é que é!

--- Sim, senhor, muito bem! --- grunhiram outros.

Rubro de gozo pelo sucesso da tirada, Capistrano espichou o dedo para a
filarmônica, a pedir o hino nacional.

Desbarretaram"-se todos. Ereto sobre o pedestal de cupim, Capistrano
imobilizou"-se em atitude de religiosa unção, de olhos fixos no futuro da
pátria. E à derradeira nota pôs fim à festa com um escarlate viva à
República com três ``erres''.

Acompanharam"-no, como um eco, o coletor, o zelador do cemitério, o
agente do correio e os funcionários federais demissíveis, além dos bois,
que mugiram.

Meses mais tarde procedeu"-se à colheita. As cebolas haviam apodrecido na
terra, devido às chuvas; os alhos vieram sem dentes, devido ao sol; as
batatas não foram por diante, devido às vaquinhas; as outras
``policulturas'' negaram fogo devido à saúva, à quenquém, à geada, a
isto e mais aquilo.

Não obstante, seguiu para o Rio um soporoso relatório de trezentas
páginas onde Capistrano, entre outras maravilhas, notava: ``Os
resultados práticos do nosso método demonstrativo \emph{in loco} têm
sido verdadeiramente assombrosos! Os lavradores acodem em massa às
lições, aplaudem"-nos com delírio e, de volta às suas terras, lançam"-se
com furor à cultura poli, em tão boa hora lembrada pelo claro espírito
de V. Excia. O Senhor Ministro pode felicitar"-se de ter aberto de par em
par as portas da idade de ouro da agricultura nacional''.

Os jornais transcreveram com gabos estes e outros pedacinhos de ouro. E
o conde de Afonso Celso se encheu de mais um bocado de ufania por este
nosso maravilhoso país.

\chapter{Velha praga\footnote[*]{Texto de 1914, publicado no livro \emph{Urupês}.}}

Andam todos em nossa terra por tal forma estonteados com as proezas
infernais dos belacíssimos ``vons'' alemães, que não sobram olhos para
enxergar males caseiros.

Venha, pois, uma voz do sertão dizer às gentes da cidade que se lá fora
o fogo da guerra lavra implacável, fogo não menos destruidor devasta
nossas matas, com furor não menos germânico.

Em agosto, por força do excessivo prolongamento do inverno, ``von Fogo''
lambeu montes e vales, sem um momento de tréguas, durante o mês inteiro.

Vieram em começos de setembro chuvinhas de apagar poeira e, breve, novo
``verão de sol'' se estirou por outubro adentro, dando azo a que se
torrasse tudo quanto escapara à sanha de agosto.

A serra da Mantiqueira ardeu como ardem aldeias na Europa, e é hoje um
cinzeiro imenso, entremeado aqui e acolá de manchas de verdura --- as
restingas úmidas, as grotas frias, as nesgas salvas a tempo pela cautela
dos aceiros. Tudo mais é crepe negro.

À hora em que escrevemos, fins de outubro, chove. Mas que chuva cainha!
Que miséria d'água! Enquanto caem do céu pingos homeopáticos, medidos a
conta"-gotas, o fogo, amortecido mas não dominado, amoita"-se insidioso
nas piúcas,\footnote{Tocos semicarbonizados. Nota da edição de 1946.} a
fumegar imperceptivelmente, pronto para rebentar em chamas mal se limpe
o céu e o sol lhe dê a mão.

Preocupa à nossa gente civilizada o conhecer em quanto fica na Europa
por dia, em francos e cêntimos, um soldado em guerra; mas ninguém cuida
de calcular os prejuízos de toda sorte advindos de uma assombrosa queima
destas. As velhas camadas de húmus destruídas; os sais preciosos que,
breve, as enxurradas deitarão fora, rio abaixo, via oceano; o
rejuvenescimento florestal do solo paralisado e retrogradado; a
destruição das aves silvestres e o possível advento de pragas
insetiformes; a alteração para pior do clima com a agravação crescente
das secas; os vedos e aramados perdidos; o gado morto ou depreciado pela
falta de pastos; as cento e um particularidades que dizem respeito a
esta ou aquela zona e, dentro delas, a esta ou aquela ``situação''
agrícola.

Isto, bem somado, daria algarismos de apavorar; infelizmente no Brasil
subtrai"-se; somar ninguém soma\ldots{}

É peculiar de agosto, e típica, esta desastrosa queima de matas; nunca,
porém, assumiu tamanha violência, nem alcançou tal extensão, como neste
tortíssimo 1914 que, benza"-o Deus, parece aparentado de perto com o
célebre ano 1000 de macabra memória. Tudo nele culmina, vai logo às do
cabo, sem conta nem medida. As queimas não fugiram à regra.

Razão sobeja para, desta feita, encararmos a sério o problema. Do
contrário a Mantiqueira será em pouco tempo toda um sapezeiro sem fim,
erisipelado de samambaias --- esses dois términos à uberdade das terras
montanhosas.

Qual a causa da renitente calamidade?

É mister um rodeio para chegar lá.

A nossa montanha é vítima de um parasita, um piolho da terra, peculiar
ao solo brasileiro como o \emph{Argas} o é aos galinheiros ou o
\emph{Sarcoptes mutans} à perna das aves domésticas. Poderíamos,
analogicamente, classificá"-lo entre as variedades do \emph{Porrigo
decalvans}, o parasita do couro cabeludo produtor da ``pelada'', pois
que onde ele assiste se vai despojando a terra de sua coma vegetal até
cair em morna decrepitude, nua e descalvada. Em quatro anos a mais
ubertosa região se despe dos jequitibás magníficos e das perobeiras
milenárias --- seu orgulho e grandeza, para, em achincalhe crescente,
cair em capoeira, passar desta à humildade da vassourinha e, descendo
sempre, encruar definitivamente na desdita do sapezeiro --- sua tortura
e vergonha.

Este funesto parasita da terra é o \textsc{caboclo}, espécie de homem baldio,
seminômade, inadaptável à civilização, mas que vive à beira dela na
penumbra das zonas fronteiriças. À medida que o progresso vem chegando
com a via férrea, o italiano, o arado, a valorização da propriedade, vai
ele refugindo em silêncio, com o seu cachorro, o seu pilão, a
pica"-pau\footnote{Espingarda de carregar pela boca. Nota da edição de
  1946.} e o isqueiro, de modo a sempre conservar"-se fronteiriço, mudo e
sorna. Encoscorado numa rotina de pedra, recua para não adaptar"-se.

É de vê"-lo surgir a um sítio novo para nele armar a sua arapuca de
``agregado''; nômade por força de vagos atavismos, não se liga à terra,
como o campônio europeu: ``agrega"-se'', tal qual o \emph{Sarcoptes},
pelo tempo necessário à completa sucção da seiva convizinha; feito o
que, salta para diante com a mesma bagagem com que ali chegou.

Vem de um sapezeiro para criar outro. Coexistem em íntima simbiose: sapé
e caboclo são vidas associadas. Este inventou aquele e lhe dilata os
domínios; em troca o sapé lhe cobre a choça e lhe fornece fachos para
queimar a colmeia das pobres abelhas.

Chegam silenciosamente, ele e a ``sarcopta'' fêmea, esta com um filhote
no útero, outro ao peito, outro de sete anos à ourela da saia --- este
já de pitinho na boca e faca à cinta. Completam o rancho um cachorro
sarnento --- Brinquinho, a foice, a enxada, a pica"-pau, o pilãozinho de
sal, a panela de barro, um santo encardido, três galinhas pevas e um
galo índio. Com estes simples ingredientes, o fazedor de sapezeiros
perpetua a espécie e a obra de esterilização iniciada com os
remotíssimos avós.

Acampam.

Em três dias uma choça, que por eufemismo chamam casa, brota da terra
como um urupê. Tiram tudo do lugar, os esteios, os caibros, as ripas, os
barrotes, o cipó que os liga, o barro das paredes e a palha do teto. Tão
íntima é a comunhão dessas palhoças com a terra local, que dariam ideia
de coisa nascida do chão por obra espontânea da natureza --- se a
natureza fosse capaz de criar coisas tão feias.

Barreada a casa, pendurado o santo, está lavrada a sentença de morte
daquela paragem.

Começam as requisições. Com a pica"-pau o caboclo limpa a floresta das
aves incautas. Pólvora e chumbo adquire"-os vendendo palmitos no povoado
vizinho. É este um traço curioso da vida do caboclo e explica o seu
largo dispêndio de pólvora; quando o palmito escasseia, rareiam os
tiros, só a caça grande merecendo sua carga de chumbo; se o palmital se
extingue, exultam as pacas: está encerrada a estação venatória.

Depois ataca a floresta. Roça e derruba, não perdoando ao mais belo pau.
Árvores diante de cuja majestosa beleza Ruskin choraria de comoção, ele
as derriba, impassível, para extrair um mel"-de"-pau escondido num oco.

Pronto o roçado, e chegado o tempo da queima, entra em funções o
isqueiro. Mas aqui o \emph{Sarcoptes} se faz raposa. Como não ignora que
a lei impõe aos roçados um aceiro de dimensões suficientes à
circunscrição do fogo, urde traças para iludir a lei, cocando destarte a
insigne preguiça e a velha malignidade.

Foi neste momento que o viu o poeta:

\emph{Cisma o caboclo à porta da cabana.}\footnote{Verso de Ricardo
  Gonçalves. Nota da edição de 1946.}

Cisma, de fato, não devaneios líricos, mas jeitos de transgredir as
posturas com a responsabilidade a salvo. E consegue"-o. Arranja sempre um
álibi demonstrativo de que não esteve lá no dia do fogo.

Onze horas.

O sol quase a pino queima como chama. Um \emph{Sarcoptes} anda por ali,
ressabiado. Minutos após crepita a labareda inicial, medrosa, numa touça
mais seca; oscila incerta; ondeia ao vento; mas logo encorpa, cresce,
avulta, tumultua infrene e, senhora do campo, estruge fragorosa com
infernal violência, devorando as tranqueiras, estorricando as mais altas
frondes, despejando para o céu golfões de fumo estrelejado de faíscas.

É o fogo de mato!

E como não o detém nenhum aceiro, esse fogo invade a floresta e caminha
por ela adentro, ora frouxo, nas capetingas\footnote{Capins de mato
  dentro, sempre ralos, magrelas. Nota da edição de 1946.} ralas, ora
maciço, aos estouros, nas moitas de taquaruçu; caminha sem tréguas,
moroso e tíbio quando a noite fecha, insolente se o sol o ajuda.

E vai galgando montes em arrancadas furiosas, ou descendo encostas a
passo lento e traiçoeiro até que o detenha a barragem natural dum rio,
estrada ou grota noruega.\footnote{Grota fria onde não bate o sol. Nota
  da edição de 1946.}

Barrado, inflete para os flancos, ladeia o obstáculo, deixa"-o para trás,
esgueira"-se para os lados --- e lá continua o abrasamento implacável.

Amordaçado por uma chuva repentina, alapa"-se nas piúcas, quieto e
invisível, para no dia seguinte, ao esquentar do sol, prosseguir na
faina carbonizante.

Quem foi o incendiário? Donde partiu o fogo? Indaga"-se, descobre"-se o
Nero: é um urumbeva qualquer, de barba rala, amoitado num
litro\footnote{A terra se mede pela quantidade de milho que nela pode
  ser plantada; daí, um alqueire, uma quarta, um litro de terra. Nota da
  edição de 1946.} de terra litigiosa.

E agora? Que fazer? Processá"-lo?

Não há recurso legal contra ele. A única pena possível, barata, fácil e
já estabelecida como praxe, é ``tocá"-lo''.

Curioso este preceito: ``Ao caboclo, toca"-se''. Toca"-se, como se toca um
cachorro importuno, ou uma galinha que vareja pela sala. E tão afeito
anda ele a isso, que é comum ouvi"-lo dizer: ``Se eu fizer tal coisa o
senhor não me toca?''.

Justiça sumária --- que não pune, entretanto, dado o nomadismo do
paciente.

Enquanto a mata arde, o caboclo regala"-se.

--- Eta fogo bonito!

No vazio de sua vida semisselvagem, em que os incidentes são um jacu
abatido, uma paca fisgada na água ou o filho novimensal, a queimada é o
grande espetáculo do ano, supremo regalo dos olhos e dos ouvidos.

Entrado setembro, começo das ``águas'', o caboclo planta na terra em
cinzas um bocado de milho, feijão e arroz; mas o valor da sua produção é
nenhum diante dos males que para preparar uma quarta de chão ele semeou.

O caboclo é uma quantidade negativa. Tala cinquenta alqueires de terra
para extrair deles o com que passar fome e frio durante o ano. Calcula
as sementeiras pelo máximo da sua resistência às privações. Nem mais,
nem menos. ``Dando para passar fome'', sem virem a morrer disso, ele, a
mulher e o cachorro --- está tudo muito bem; assim fez o pai, o avô;
assim fará a prole empanzinada que naquele momento brinca nua no
terreiro.

Quando se exaure a terra, o agregado muda de sítio. No lugar fica a
tapera e o sapezeiro. Um ano que passe e só este atestará a sua estada
ali; o mais se apaga como por encanto. A terra reabsorve os frágeis
materiais da choça e, como nem sequer uma laranjeira ele plantou, nada
mais lembra a passagem por ali de Manoel Peroba, de Chico Marimbondo, de
Jeca Tatu ou outros sons ignaros, de dolorosa memória para a natureza
circunvizinha.

\chapter{Urupês\footnote[*]{Texto de 1914, publicado no livro (De \emph{Urupês}.}}

Esboroou"-se o balsâmico indianismo de Alencar ao advento dos Rondons
que, ao invés de imaginarem índios num gabinete, com reminiscências de
Chateaubriand na cabeça e \emph{Iracema} aberta sobre os joelhos,
metem"-se a palmilhar sertões de Winchester em punho.

Morreu Peri, incomparável idealização dum homem natural como o sonhava
Rousseau, protótipo de tantas perfeições humanas que no romance, ombro a
ombro com altos tipos civilizados, a todos sobreleva em beleza de alma e
corpo.

Contrapôs"-lhe a cruel etnologia dos sertanistas modernos um selvagem
real, feio e brutesco, anguloso e desinteressante, tão incapaz,
muscularmente, de arrancar uma palmeira, como incapaz, moralmente, de
amar Ceci.

Por felicidade nossa --- e de dom Antônio de Mariz ---, não os viu
Alencar; sonhou"-os qual Rousseau. Do contrário lá teríamos o filho de
Araré a moquear a linda menina num bom braseiro de pau"-brasil, em vez de
acompanhá"-la em adoração pelas selvas, como Ariel benfazejo do
Paquequer.

A sedução do imaginoso romancista criou forte corrente. Todo o clã
plumitivo deu de forjar seu indiozinho refegado de Peri e Atala. Em
sonetos, contos e novelas, hoje esquecidos, consumiram"-se tabas inteiras
de aimorés sanhudos, com virtudes romanas por dentro e penas de tucano
por fora.

Vindo o público a bocejar de farto, já cético ante o crescente
desmantelo do ideal, cessou no mercado literário a procura de bugres
homéricos, inúbias, tacapes, borés, piagas e virgens bronzeadas. Armas e
heróis desandaram cabisbaixos, rumo ao porão onde se guardam os móveis
fora de uso, saudoso museu de extintas pilhas elétricas que a seu tempo
galvanizaram nervos. E lá acamam poeira cochichando reminiscências com a
barba de dom João de Castro, com os franquisques de Herculano, com os
frades de Garrett e que tais\ldots{}

Não morreu, todavia.

Evoluiu.

O indianismo está de novo a deitar copa, de nome mudado. Crismou"-se de
``caboclismo''. O cocar de penas de arara passou a chapéu de palha
rebatido à testa; a ocara virou rancho de sapé; o tacape afilou, criou
gatilho, deitou ouvido e é hoje espingarda trouxada; o boré descaiu
lamentavelmente para pio de inambu; a tanga ascendeu a camisa aberta ao
peito.

Mas o substrato psíquico não mudou: orgulho indomável, independência,
fidalguia, coragem, virilidade heroica, todo o recheio em suma, sem
faltar uma azeitona, dos Peris e Ubirajaras.

Este setembrino rebrotar duma arte morta inda se não desbagoou de todos
os frutos. Terá o seu ``I"-Juca"-Pirama'', o seu ``Canto do Piaga'' e
talvez dê ópera lírica.

Mas, completado o ciclo, em flor da ilusão indianista virão destroçar o
inverno os prosaicos de ídolos --- gente má e sem poesia. Irão os
malvados esgaravatar o ícone com as curetas da ciência. E que feias se
hão de entrever as caipirinhas cor de jambo de Fagundes Varela! E que
chambões e sornas os Peris de calça, camisa e faca à cinta!

Isso, para o futuro. Hoje ainda há perigo em bulir no vespeiro: o
caboclo é o ``Ai Jesus!'' nacional.

É de ver o orgulhoso entono com que respeitáveis figurões batem no peito
exclamando com altivez:

--- Sou raça de caboclo!

Anos atrás o orgulho estava numa ascendência de tanga, inçada de penas
de tucano, com dramas íntimos e flechaços de curare.

Dia virá em que os veremos, murchos de prosápia, confessar o verdadeiro
avô:

--- Um dos quatrocentos de Gedeão trazidos por Tomé de Sousa\footnote{Tomé
  de Sousa veio ao Brasil com um carregamento de quatrocentos degredados
  e uns tantos jesuítas. Nota da edição de 1946.} num barco daqueles
tempos, nosso mui nobre e fecundo \emph{Mayflower}. Porque a verdade nua
manda dizer que entre as raças de variado matiz, formadoras da
nacionalidade e metidas entre o estrangeiro recente e o aborígine de
tabuinha no beiço, uma existe a vegetar de cócoras, incapaz de evolução,
impenetrável ao progresso. Feia e sorna, nada a põe de pé.

Quando Pedro I lança aos ecos o seu grito histórico e o país desperta
estrouvinhado à crise duma mudança de dono, o caboclo ergue"-se, espia e
acocora"-se de novo.

Pelo 13 de Maio, mal esvoaça o florido decreto da Princesa e o negro
exausto larga num \emph{uf!} o cabo da enxada, o caboclo olha, coça a
cabeça, imagina e deixa que do velho mundo venha quem nele pegue de
novo.

Em 15 de Novembro troca"-se um trono vitalício pela cadeira quadrienal. O
país bestifica"-se ante o inopinado da mudança.\footnote{Aristides Lobo:
  ``O país assistiu bestificado à proclamação da República''. Nota da
  edição de 1946.} O caboclo não dá pela coisa.

Vem Floriano; estouram as granadas de Custódio; Gumercindo bate às
portas de Roma; Incitatus derranca o país.\footnote{O presidente Hermes
  da Fonseca! Nota da edição de 1946.} O caboclo continua de cócoras, a
modorrar\ldots{}

Nada o esperta. Nenhuma ferrotoada o põe de pé. Social, como
individualmente, em todos os atos da vida, Jeca, antes de agir,
acocora"-se.

Jeca Tatu é um piraquara do Paraíba, maravilhoso epítome de carne onde
se resumem todas as características da espécie.

Ei"-lo que vem falar ao patrão. Entrou, saudou. Seu primeiro movimento,
após prender entre os lábios a palha de milho, sacar o rolete de fumo e
disparar a cusparada de esguicho, é sentar"-se jeitosamente sobre os
calcanhares. Só então destrava a língua e a inteligência.

--- Não vê que\ldots{}

De pé ou sentado as ideias se lhe entramam, a língua emperra e não há de
dizer coisa com coisa. De noite, na choça de palha, acocora"-se em frente
ao fogo para ``aquentálo'', imitado da mulher e da prole.

Para comer, negociar uma barganha, ingerir um café, tostar um cabo de
foice, fazê"-lo noutra posição será desastre infalível. Há de ser de
cócoras.

Nos mercados, para onde leva a quitanda domingueira, é de cócoras, como
um faquir do Bramaputra, que vigia os cachinhos de brejaúva ou o feixe
de três palmitos.

Pobre Jeca Tatu! Como és bonito no romance e feio na realidade!

Jeca mercador, Jeca lavrador, Jeca filósofo\ldots{}

Quando comparece às feiras, todo mundo logo adivinha o que ele traz:
sempre coisas que a natureza derrama pelo mato e ao homem só custa o
gesto de espichar a mão e colher --- cocos de tucum ou jiçara,
guabirobas, bacuparis, maracujás, jataís, pinhões, orquídeas; ou
artefatos de taquara"-poca --- peneiras, cestinhas, samburás, tipitis,
pios de caçador; ou utensílios de madeira mole --- gamelas, pilõezinhos,
colheres de pau.

Nada mais.

Seu grande cuidado é espremer todas as consequências da lei do menor
esforço --- e nisto vai longe.

Começa na morada. Sua casa de sapé e lama faz sorrir aos bichos que
moram em toca e gargalhar ao joão"-de"-barro. Pura biboca de bosquímano.
Mobília, nenhuma. A cama é uma espipada esteira de peri posta sobre o
chão batido.

Às vezes se dá ao luxo de um banquinho de três pernas --- para os
hóspedes. Três pernas permitem equilíbrio; inútil, portanto, meter a
quarta, o que ainda o obrigaria a nivelar o chão. Para que assentos, se
a natureza os dotou de sólidos, rachados calcanhares sobre os quais se
sentam?

Nenhum talher. Não é a munheca um talher completo --- colher, garfo e
faca a um tempo?

No mais, umas cuias, gamelinhas, um pote esbeiçado, a pichorra e a
panela de feijão.

Nada de armários ou baús. A roupa, guarda"-a no corpo. Só tem dois
parelhos; um que traz no uso e outro na lavagem.

Os mantimentos apaiola nos cantos da casa.

Inventou um cipó preso à cumeeira, de gancho na ponta e um disco de lata
no alto: ali pendura o toucinho, a salvo dos gatos e ratos.

Da parede pende a espingarda pica"-pau, o polvarinho de chifre, o são
Benedito defumado, o rabo de tatu e as palmas bentas de queimar durante
as fortes trovoadas. Servem de gaveta os buracos da parede.

Seus remotos avós não gozaram maiores comodidades. Seus netos não
meterão quarta perna ao banco. Para quê? Vive"-se bem sem isso.

Se pelotas de barro caem, abrindo seteiras na parede, Jeca não se move a
repô"-las. Ficam pelo resto da vida os buracos abertos, a entremostrarem
nesgas de céu.

Quando a palha do teto, apodrecida, greta em fendas por onde pinga a
chuva, Jeca, em vez de remendar a tortura, limita"-se, cada vez que
chove, a aparar numa gamelinha a água gotejante\ldots{}

Remendo\ldots{} Para quê?, se uma casa dura dez anos e faltam ``apenas'' nove
para ele abandonar aquela? Esta filosofia economiza reparos.

Na mansão de Jeca a parede dos fundos bojou para fora um ventre
empanzinado, ameaçando ruir; os barrotes, cortados pela umidade, oscilam
na podriqueira do baldrame. A fim de neutralizar o desaprumo e prevenir
suas consequências, ele grudou na parede uma Nossa Senhora enquadrada em
moldurinha amarela --- santo de mascate.

--- Por que não remenda essa parede, homem de Deus?

--- Ela não tem coragem de cair. Não vê a escora?

Não obstante, ``por via das dúvidas'', quando ronca a trovoada Jeca
abandona a toca e vai agachar"-se no oco dum velho embiruçu do quintal
--- para se saborear de longe com a eficácia da escora santa.

Um pedaço de pau dispensaria o milagre; mas entre pendurar o santo e
tomar da foice, subir ao morro, cortar a madeira, atorá"-la, baldeá"-la e
especar a parede, o sacerdote da Grande Lei do Menor Esforço não vacila.
É coerente.

Um terreirinho descalvado rodeia a casa. O mato o beira. Nem árvores
frutíferas, nem horta, nem flores --- nada revelador de permanência.

Há mil razões para isso; porque não é sua a terra; porque se o
``tocarem'' não ficará nada que a outrem aproveite; porque para frutas
há o mato; porque a ``criação'' come; porque\ldots{}

--- Mas, criatura, com um vedozinho por ali\ldots{} A madeira está à mão, o
cipó é tanto\ldots{}

Jeca, interpelado, olha para o morro coberto de moirões, olha para o
terreiro nu, coça a cabeça e cuspilha.

--- Não paga a pena.

Todo o inconsciente filosofar do caboclo grulha nessa palavra
atravessada de fatalismo e modorra. Nada paga a pena. Nem culturas, nem
comodidades. De qualquer jeito se vive.

Da terra só quer a mandioca, o milho e a cana. A primeira, por ser um
pão já amassado pela natureza. Basta arrancar uma raiz e deitá"-la nas
brasas. Não impõe colheita, nem exige celeiro. O plantio se faz com um
palmo de rama fincada em qualquer chão. Não pede cuidados. Não a ataca a
formiga. A mandioca é sem"-vergonha.

Bem ponderado, a causa principal da lombeira do caboclo reside nas
benemerências sem conta da mandioca. Talvez que sem ela se pusesse de pé
e andasse. Mas enquanto dispuser de um pão cujo preparo se resume no
plantar, colher e lançar sobre brasas, Jeca não mudará de vida. O vigor
das raças humanas está na razão direta da hostilidade ambiente. Se a
poder de estacas e diques o holandês extraiu de um brejo salgado a
Holanda, essa joia do esforço, é que ali nada o favorecia. Se a
Inglaterra brotou das ilhas nevoentas da Caledônia, é que lá não medrava
a mandioca. Medrasse, e talvez os víssemos hoje, os ingleses, tolhiços,
de pé no chão, amarelentos, mariscando de peneira no Tâmisa. Há bens que
vêm para males. A mandioca ilustra este avesso de provérbio.

Outro precioso auxiliar da calaçaria é a cana. Dá rapadura, e para Jeca,
simplificador da vida, dá garapa. Como não possui moenda, torce a pulso
sobre a cuia de café um rolete, depois de bem macetados os nós; açucara
assim a beberagem, fugindo aos trâmites condutores do caldo de cana à
rapadura.

Todavia, \emph{est modus in rebus}. E assim como ao lado do restolho
cresce o bom pé de milho, contrasta com a cristianíssima simplicidade do
Jeca a opulência de um seu vizinho e compadre que ``está muito bem''. A
terra onde mora é sua. Possui ainda uma égua, monjolo e espingarda de
dois canos. Pesa nos destinos políticos do país com o seu voto e nos
econômicos com o polvilho azedo de que é fabricante, tendo amealhado com
ambos, voto e polvilho, para mais de quinhentos mil"-réis no fundo da
arca.

Vive num corrupio de barganhas nas quais exercita uma astúcia nativa
muito irmã da de Bertoldo. A esperteza última foi a barganha de um
cavalo cego por uma égua de passo picado. Verdade é que a égua mancava
das mãos, mas inda assim valia dez mil"-réis mais do que o rocinante
zanaga.

Esta e outras celebrizaram"-lhe os engrimanços potreiros num raio de mil
braças, granjeando"-lhe a incondicional e babosa admiração de Jeca, para
quem, fino como o compadre, ``home''\ldots{} nem mesmo o vigário de Itaoca!

Aos domingos vai à vila bifurcado na magreza ventruda da Serena; leva
apenso à garupa um filho e atrás o potrinho no trote, mais a mulher, com
a criança nova enrolada no xale. Fecha o cortejo o indefectível
Brinquinho, a resfolgar com um palmo de língua de fora.

O fato mais importante de sua vida é sem dúvida votar no Governo. Tira
nesse dia da arca a roupa preta do casamento, sarjão furadinho de traça
e todo vincado de dobras; entala os pés num alentado sapatão de bezerro;
ata ao pescoço um colarinho de bico e, sem gravata, ringindo e mancando,
vai pegar o diploma de eleitor às mãos do chefe Coisada, que lho retém
para maior garantia da fidelidade partidária.

Vota. Não sabe em quem, mas vota. Esfrega a pena no livro eleitoral,
arabescando o aranhol de gatafunhos e que chama ``sua graça''.

Se há tumulto, chuchurreia de pé firme, com heroísmo, as porretadas
oposicionistas, e ao cabo segue para a casa do chefe, de galo cívico na
testa e colarinho sungado para trás, a fim de novamente lhe depor nas
mãos o ``dipeloma''.

Grato e sorridente, o morubixaba galardoa"-lhe o heroísmo, flagrantemente
documentado pelo latejar do couro cabeludo, com um aperto de munheca e a
promessa, para logo, duma inspetoria de quarteirão.

Representa este freguês o tipo clássico do sitiante já com um pé fora da
classe. Exceção, díscolo que é, não vem ao caso. Aqui tratamos da regra
e a regra é Jeca Tatu.

O mobiliário cerebral de Jeca, à parte o suculento recheio de
superstições, vale o do casebre. O banquinho de três pés, as cuias, o
gancho de toucinho, as gamelas, tudo se reedita dentro de seus miolos
sob a forma de ideias: são as noções práticas da vida, que recebeu do
pai e sem mudança transmitirá aos filhos.

O sentimento de pátria lhe é desconhecido. Não tem sequer a noção do
país em que vive. Sabe que o mundo é grande, que há sempre terras para
diante, que muito longe está a Corte com os graúdos e mais distante
ainda a Bahia, donde vêm baianos pernósticos e cocos.

Perguntem ao Jeca quem é o presidente da República.

--- O homem que manda em nós tudo?

--- Sim.

--- Pois de certo que há de ser o imperador.

Em matéria de civismo não sobe de ponto.

--- Guerra? Te esconjuro! Meu pai viveu afundado no mato pra mais de
cinco anos por causa da guerra grande.\footnote{Guerra do Paraguai. Nota
  da edição de 1946.} Eu, para escapar do ``reculutamento'', sou inté
capaz de cortar um dedo, como o meu tio Lourenço\ldots{}

Guerra, defesa nacional, ação administrativa, tudo quanto cheira a
governo resume"-se para o caboclo numa palavra apavorante ---
``reculutamento''.

Quando em princípios da presidência Hermes andou na balha um
recenseamento esquecido a Offenbach, o caboclo tremeu e entrou a casar
em massa. Aquilo ``haverá de ser reculutamento'', e os casados, na voz
corrente, escapavam à redada.

A sua medicina corre parelhas com o civismo e a mobília --- em
qualidade. Quantitativamente, assombra. Da noite cerebral
pirilampejam"-lhe apózemas, cerotos, arrobes e eletuários escapos à
sagacidade cômica de Mark Twain. Compendia"-os um Chernoviz não escrito,
monumento de galhofa onde não há rir, lúgubre como é o epílogo. A rede
na qual dois homens levam à cova as vítimas de semelhante farmacopeia é
o espetáculo mais triste da roça.

Quem aplica as mezinhas é o ``curador'', um Eusébio Macário de pé no
chão e cérebro trancado como moita de taquaruçu. O veículo usual das
drogas é sempre a pinga --- meio honesto de render homenagem à deusa
Cachaça, divindade que entre eles ainda não encontrou heréticos.

Doenças hajam que remédios não faltam.

Para bronquite, é um porrete cuspir o doente na boca de um peixe vivo e
soltá"-lo: o mal se vai com o peixe água abaixo\ldots{}

Para ``quebranto de ossos'', já não é tão simples a medicação. Tomam"-se
três contas de rosário, três galhos de alecrim, três limas de bico, três
iscas de palma benta, três raminhos de arruda, três ovos de pata preta
(com casca; sem casca desanda) e um saquinho de picumã; mete"-se tudo
numa gamela d'água e banha"-se naquilo o doente, fazendo"-o tragar três
goles da zurrapa. É infalível!

O específico da brotoeja consiste em cozimento de beiço de pote para
lavagens. Ainda há aqui um pormenor de monta: é preciso que antes do
banho a mãe do doente molhe na água a ponta de sua trança. As brotoejas
saram como por encanto.

Para dor de peito que ``responde na cacunda'', cataplasma de
``jasmim"-decachorro'' é um porrete.

Além desta alopatia, para a qual contribui tudo quanto de mais
repugnante e inócuo que existe na natureza, há a medicação simpática,
baseada na influição misteriosa de objetos, palavras e atos sobre o
corpo humano.

O ritual bizantino dentro de cujas maranhas os filhos de Jeca vêm ao
mundo, e do qual não há fugir sob pena de gravíssimas consequências
futuras, daria um in"-fólio de alto fôlego ao Sílvio Romero bastante
operoso que se propusesse a compendiá"-lo.

Num parto difícil, nada tão eficaz como engolir três caroços de feijão
mouro, de passo que a parturiente veste pelo avesso a camisa do marido e
põe na cabeça, também pelo avesso, o seu chapéu. Falhando esta simpatia,
há um derradeiro recurso: colar no ventre encruado a imagem de são
Benedito.

Nesses momentos angustiosos outra mulher não penetre no recinto sem
primeiro defumar"-se ao fogo, nem traga na mão caça ou peixe: a criança
morreria pagã. A omissão de qualquer destes preceitos fará chover mil
desgraças na cabeça do chorincas recém"-nascido.

A posse de certos objetos confere dotes sobrenaturais. A
invulnerabilidade às facadas ou cargas de chumbo é obtida graças à flor
da samambaia.

Esta planta, conta Jeca, só floresce uma vez por ano, e só produz em
cada samambaial uma flor. Isto à meia"-noite, no dia de são Bartolomeu. É
preciso ser muito esperto para colhê"-la, porque também o diabo anda à
cata. Quem consegue pegar uma, ouve logo um estouro e tonteia ao cheiro
de enxofre --- mas livra"-se de faca e chumbo pelo resto da vida.

Todos os volumes do Larousse não bastariam para catalogar"-lhe as
crendices, e como não há linhas divisórias entre estas e a religião,
confundem"-se ambas em maranhada teia, não havendo distinguir onde para
uma e começa outra.

A ideia de Deus e dos santos torna"-se jecocêntrica. São os santos os
graúdos lá de cima, os coronéis celestes, debruçados no azul para
espreitar"-lhes a vidinha e intervir nela ajudando"-os ou castigando"-os,
como os metediços deuses de Homero. Uma torcedura de pé, um estrepe, o
feijão entornado, o pote que rachou, o bicho que arruinou --- tudo
diabruras da corte celeste, para castigo de más intenções ou atos.

Daí o fatalismo. Se tudo movem cordéis lá de cima, para que lutar,
reagir? Deus quis. A maior catástrofe é recebida com esta exclamação,
muito parenta do ``Allah Kébir'' do beduíno.

E na arte?

Nada.

A arte rústica do campônio europeu é opulenta a ponto de constituir
preciosa fonte de sugestões para os artistas de escol. Em nenhum país o
povo vive sem a ela recorrer para um ingênuo embelezamento da vida. Já
não se fala no camponês italiano ou teutônico, filho de alfobres
mimosos, propícios a todas as florações estéticas. Mas o russo, o
hirsuto mujique a meio atolado em barbárie crassa. Os vestuários
nacionais da Ucrânia nos quais a cor viva e o sarapantado da
ornamentação indicam a ingenuidade do primitivo; os isbas da Lituânia,
sua cerâmica, os bordados, os móveis, os utensílios de cozinha, tudo
revela no mais rude dos campônios o sentimento da arte.

No samoiedo, no pele"-vermelha, no abexim, no papua, um arabesco ingênuo
costuma ornar"-lhes as armas --- como lhes ornam a vida canções
repassadas de ritmos sugestivos.

Que nada é isso, sabido como já o homem pré"-histórico, companheiro do
urso das cavernas, entalhava perfis de mamutes em chifres de rena.

Egresso à regra, não denuncia o nosso caboclo o mais remoto traço de um
sentimento nascido com o troglodita.

Esmerilhemos o seu casebre: que é que ali denota a existência do mais
vago senso estético? Uma chumbada no cabo do relho e uns zigue"-zagues a
canivete ou fogo pelo roliço do porretinho de guatambu. É tudo.

Às vezes surge numa família um gênio musical cuja fama esvoaça pelas
redondezas. Ei"-lo na viola: concentra"-se, tosse, cuspilha o pigarro,
fere as cordas e ``tempera''. E fica nisso, no tempero.

Dirão: e a modinha?

A modinha, como as demais manifestações de arte popular existentes no
país, é obra do mulato, em cujas veias o sangue recente do europeu, rico
de atavismos estéticos, borbulha de envolta com o sangue selvagem,
alegre e são do negro.

O caboclo é soturno.

Não canta senão rezas lúgubres.

Não dança senão o cateretê aladainhado.

Não esculpe o cabo da faca, como o cabila.

Não compõe sua canção, como o felá do Egito.

No meio da natureza brasílica, tão rica de formas e cores, onde os ipês
floridos derramam feitiços no ambiente e a infolhescência dos cedros, às
primeiras chuvas de setembro, abre a dança dos tangarás; onde há abelhas
de sol, esmeraldas vivas, cigarras, sabiás, luz, cor, perfume, vida
dionisíaca em escacho permanente, o caboclo é o sombrio urupê de pau
podre, a modorrar silencioso no recesso das grotas.

Só ele não fala, não canta, não ri, não ama.

Só ele, no meio de tanta vida, não vive\ldots{}

\chapter{A nuvem de gafanhotos\footnote[*]{Texto de 1914, publicado no livro \emph{O macaco que se fez homem}.}}

Ser empregado público de inferior categoria e por mal de pecados
demissível: será isso programa que seduza alguém?

--- É.

E para Pedro Venâncio mais que seduzia --- sorria. Foi, pois, com enlevo
de alma que recebeu a notícia de sua nomeação para fiscal da Câmara
Municipalzinha de Itaoca.

--- Vou sossegar --- disse consigo, esfregando as mãos de contentamento.

--- Cavei o meu osso e agora é roê"-lo pela vida afora na santa paz do
Senhor.

E ferrou o dente no ossinho.

Mas acontece que há osso e osso. Osso de bom tutano e osso pedra"-pomes.
No andar dos tempos verificou Venâncio que o tal ossinho era desses que
embotam os dentes sem dar o mínimo de suco.

Gastar a vida inteira naquilo? É ser tolo, cochichou"-lhe a humana
ambição de melhoria, engenhosa fada a quem se devem todos os progressos
do mundo. Assim espicaçado, entrou Venâncio a fariscar tutanos. Recorreu
antes de mais nada à loteria, pois que é a Sorte Grande o supremo engodo
dos pés"-rapados. Venham gasparinhos! Todas as semanas adquiria um --- e
sonhava. O mesmo vendeiro que lhe fornecia aos sábados a semanal quarta
de feijão, os semanais oito litros de arroz e o semanal cento de
cigarros, juntava na conta mil"-réis de sonhos. E Venâncio, comido o
feijão, fumado o cigarro, sonhava. Sonhava o doce beijo da Fortuna, boa
deusa que o despegaria do atoleiro com um simples toque de sua asa
potente.

Em matéria de cultura não era Venâncio de todo cru. Lia suas coisas e
tinha lá suas ideias. Revelara desde cedo grande embocadura para a
lavoura e documentava o pendor assinando quanta publicação oficial
existe. Publicações gratuitas\ldots{}

Assim, nas palestras da farmácia ninguém piava sobre lavoura sem que ele
pulasse no meio com a sua colher torta. E era de ver o calor da sua
argumentação e a riqueza das suas citações estatísticas.

Fazendeiro que nesses momentos passasse havia que parar e abrir bem
aberta a boca. Venâncio possuía planos grandiosos para salvar o café e
pô"-lo aí a quarenta mil"-réis a arroba\ldots{}

--- Quarenta mil"-réis, Venâncio? Não acha meio muito?

Venâncio incendiava"-se.

--- Por que muito? Não somos os maiores produtores? Não temos o quase
privilégio dessa cultura? Se é assim, o lógico é que imponhamos o preço.
Eu disse quarenta, não foi? Pois digo agora quarenta e cinco! Digo
cinquenta!

--- !!!

--- Não se espantem. Eu provo que pode ser assim e que os americanos têm
que gemer ali no dolarzinho, queiram ou não queiram!

--- !!!

--- Quei"-ram ou não quei"-ram! --- reafirmava o salvador, escandindo as
palavras.

E provava.

Também extinguia em menos de um ano a lagarta"-rosada, mais o curuquerê;
e triplicava a corrente imigratória; e extraía o azoto do ar, pondo o
adubo ao alcance de todos, a cem réis o quilo, talvez mesmo a setenta.

--- Porque, como os senhores sabem, a química agrícola demonstra que\ldots{}

E demonstrava.

Num desses rompantes demonstrativos o coronel da terra, de passagem pela
rua, deteve"-se a ouvi"-lo e, finda a tirada, disse"-lhe à queima"-roupa:

--- Que excelente ministro da Agricultura não daria você! Duvido que os
Calmons e os Bezerras entendam mais de lavoura\ldots{}

--- Está caçoando, coronel! --- murmurou Venâncio com modéstia, embora
no íntimo convencido da justiça da apreciação.

--- Falo sério. Bem sabe que não brinco.

Os circunstantes sorriram discretamente, enquanto o massa de ministro se
lambia todo, como boi feliz.

Em casa repetiu à esposa a opinião do chefe político.

--- Brincadeira dele, Pedro! --- objetou a sensatíssima consorte. ---
Não está vendo?

--- Brincadeira nada! O coronel é homem que não brinca, você bem sabe\ldots{}

Desde esse dia, imaginariamente, Venâncio transformou"-se num maravilhoso
ministro da Agricultura. Plantou"-se de armas e bagagens no casarão da
Praia Vermelha e com raro tino administrativo salvou o país. Que
eficácia de medidas! Que sábias leis protetoras! Que maravilhosos
resultados! Lagarta nos algodoais? Nem umazinha para remédio! Curuquerê?
Nem sombra! O café trepou à casa dos quarenta\ldots{}

--- Por arroba?

--- Por dez quilos, homem!

E, firmíssimo, revelava tendências para alta ainda maior. Os mais
pessimistas já concediam que não era de admirar fosse a cinquenta.

A borracha do Norte arrancou"-se ao marasmo em que emperrava e voltou a
ser um Pactolo de esterlinas.

Azoto andava por aí aos pontapés, como um trambolho.

E na cabeça de Venâncio os sonhos lotéricos desapareceram trocados pelos
sonhos administrativos, muito mais amplos e de muito maior alcance
patriótico.

A consequência foi que Venâncio se eternizou no Ministério. Vários
presidentes se sucederam sem que nenhum ousasse tocar em sua pasta. Era
sagrado aquele ministro de gênio, que salvara o país, enriquecera a
lavoura, desafogara o comércio, consolidara a indústria e que, adorado
pela nação, teria estátua em vida.

Que teria? Que teve! Por mais que em sua infinita modéstia o grande
ministro recusasse tal homenagem, a gratidão nacional teimou em
glorificá"-lo no bronze.

Inesquecível a manhã em que Venâncio, de lágrimas nos olhos, viu
rasgarem"-se os véus do seu monumento.

\textsc{ao salvador da pátria,}

\textsc{o povo agradecido.}

Agradecido ou enriquecido? A turvação dos olhos não lhe permitiu
distinguir a expressão exata --- e por longo tempo semelhante dúvida o
torturou.

Mas a grande recompensa teve"-a ele em casa, ouvindo à esposa estas
deliciosas palavras:

--- Agora, sim, Venâncio, acredito que você é mesmo o que dizia. Até
estátua!\ldots{}

A boa senhora só se convencia com provas de bronze\ldots{}

O doloroso, porém, era o contraste das duas vidas --- ministro por
dentro e fiscal da Câmara por fora, obrigado a interromper a matutação
de um projeto salvador da pátria para ir, de bonezinho na cabeça, cercar
na rua carros de boi não aferidos\ldots{}

Um ano se passou assim, no qual os gasparinhos falharam lamentavelmente.
O mesmo dinheiro; zero, zero, zero; o mesmo dinheiro; zero, zero. Os
seus rapapés à Sorte Grande recebiam da grande cortesã apenas esta magra
resposta. Tábuas sobre tábuas; carranca amarrada sempre e jamais o
sorrisozinho de uma ``aproximação'' para consolo.

Mas um dia\ldots{}

Nesse dia Venâncio disputava com a esposa, que pedia dinheiro para umas
compras.

--- Estamos com a louça reduzida a cacos. Xícara de chá, duas e
desbeiçadas. De café, três e sem asas. Ontem, quando aquele cacetão do
Freitas esteve aqui, fui obrigada a pedir emprestada uma xícara da
vizinha. Veja que vergonha\ldots{}

Venâncio relutou.

--- Mas por que é que quebram a louça? O ano passado, lembro"-me, eu
mesmo comprei meia dúzia de cada.

Dona Fortunata pôs as mãos na cintura.

--- Por que quebram? A pergunta é bem idiotazinha\ldots{} A louça quebra"-se
porque é quebrável. Se fosse inquebrável não se quebraria. Parece
incrível que um homem já indicado para ministro\ldots{}

--- Não admito ironias! Quer louça? Compre com o dote que trouxe\ldots{}

--- Já esperava por essa resposta. Está mesmo uma resposta de
ministro\ldots{} do coronel --- concluiu dona Fortunata venenosamente.

Venâncio, engasgado de cólera, ia replicar, quando a porta da sala se
abriu e o vendeiro irrompeu como um pé de vento:

--- Deixe ver o seu bilhete! Se é o 3.743, deu a tacada!

O improviso do lance transformou em estupor a cólera de Venâncio, que
entrou a piscar, numa tonteira, como quem leva porretada no crânio.

--- Quê? Que há? --- tartamudeava ele.

O vendeiro bateu o pé, impaciente.

--- O bilhete, homem! Deixe ver o seu bilhete, homem de Deus! Parece
estuporado\ldots{}

Custou a Venâncio encontrar na papelada agrícola que lhe enchia os
bolsos o raio do bilhete. Suas mãos tremiam e o cérebro andava"-lhe à
roda.

Por fim achou"-o.

Era o 3.743.

Pegara os vinte contos.

Estas revoluções operadas pela sorte em cérebros venancinos não há aí
quem as conte. É banho de ópio, é fumarada de haxixe, é gole de cocaína,
é bebedeira que rompe toda a velha cristalização dos miolos. A ebriez do
ouro vale pela soma da essência última de todas as mais ebriedades. Só
ela abre a gaiola a ``todos'' os sonhos e põe o homem leve, com
pequeninas asas em cada célula do corpo.

No caso do Venâncio, porém, não houve muita vacilação. Sua diretriz
estava traçada pelo insopitável pendor agrícola.

Uma fazenda, uma grande fazenda, a melhor fazenda do município --- a
fazenda"-modelo da zona. Da zona? Do país, por que não? E depois --- quem
sabe? --- o ministério, desta vez de verdade. O mundo dá tantas
voltas\ldots{}

E faria isto mais aquilo, e mais isto e mais aquilo. Meu Deus! Como a
fazenda se foi aperfeiçoando, e a que requintes de primor atingiu!
Legiões de curiosos vinham de longe visitá"-la, e pasmavam. A fama
corria, os jornais estudavam"-na em artigos longos. Por fim o Governo,
impressionado com a voz pública, mandava examiná"-la e propunha"-lhe
compra. Era forçoso que pertencesse ao patrimônio da nação uma coisa
daquelas para que todos pudessem aprender na maravilhosa escola as
palavras últimas do aperfeiçoamento agrícola.

Mas vendê"-la? A um particular, nunca! À nação, sim, coagido pelo
patriotismo. Isso mesmo, porém sob uma condição! Oh, sim, uma condição
\emph{sine qua non}: darem"-lhe a pasta da Agricultura\ldots{}

--- Porque eu, senhores, farei do Brasil inteiro o mimo que fiz da minha
fazenda. Um vergel florido! A nova Califórnia! O paraíso terreal!\ldots{}

O Governo chorava de emoção e dava"-lhe a pasta, sob as aclamações do
povo agradecido\ldots{}

Infelizmente, os vinte contos não eram elásticos e Venâncio teve que
arrepiar da vertigem megalomaníaca e adquirir um pequeno sítio aí de
trinta contos de réis. Deu quinze à vista e ficou a dever quinze sob
hipoteca.

Sítio velho, de terras cansadas; mas isso mesmo queria ele, para
estrondosa demonstração do axioma tantas vezes berrado na botica:

--- Não há terras más, há más cabeças. Com a química agrícola na mão
esquerda e o arado na direita, eu faço o Saara produzir milho de pipoca!

--- Mas Venâncio\ldots{}

--- Não há ``mas'', há ``más''; más cabeças, já disse. De pipoca!

Tinha agora de provar o asserto.

Começou mudando o nome antigo --- Sítio do Embirussu --- por este muito
mais adiantado --- Granja"-Modelo de Pomona.

Apesar do lindo nome, o sítio permaneceu a pinoia que sempre fora.
Barba"-de"-bode, guanxuma, saúva, cupins, joveva, geadas --- todos os
mimos da brasileiríssima deusa Praga.

Em compensação, no tocante ao pitoresco poucos haveria mais
bem"-arranjados. Tudo velho e musgoso e carcomido, como o quer a
estética. Vate de cabeleira que ali caísse desentranhava"-se logo em
sonetos do mais repassado bucolismo; e o pintor de paisagens encontrava
quadrinhos já feitos, encantadores, que era um gosto trasladar para a
tela.

As paineiras laterais à casa faziam em setembro o enlevo dos colibris e
das abelhas --- mas a paina produzida mal dava para encher um
travesseiro.

O pomar, velhíssimo, lembrava um ninho de faunos tocadores de avena;
laranjeiras de cinquenta anos, pitangueiras altíssimas, ameixeiras
musgosas, jabuticabeiras, romeiras --- o que há de virgiliano e
romântico e sombrio e parasitado. Renda, porém, zero.

Tudo mais pelo mesmo teor.

Venâncio mediu com os olhos penetrantes a grandeza da sua tarefa e
sorriu. Tinha tanta convicção de transmutar aquele bucolismo em fonte de
lucros\ldots{}

Começou pelas aves. Em vez daquele sórdido restolho de galinhame da
terra, sem sangue de \emph{pedigree}, venham Leghorns para ovos e
Orpingtons para carne. Imbecil o fazendeiro que não adota as belas raças
americanas!

A mesma coisa com os porcos. Nada de canastrões ou tatuzinhos, tardios
ou degenerados. Venham o Yorkshire, o Duroc"-Jersey !

E venham mudas de boas árvores frutíferas, caquis, ameixas"-do"-japão,
damascos, maçãs, peras, tudo isto com explicações ao eterno nariz
torcido da esposa:

--- Porque você vê, Fortunata, dá o mesmo trabalho e vale cinco vezes
mais.

Um ovo de Orpington, por exemplo: quanto vale no Rio? Dois mil"-réis;
mais que uma dúzia de ovos crioulos!

E venham sementes de capim"-de"-rodes para as pastagens.

E venha um aradinho de disco, e agora uma semeadeira, e uma carpideira,
e uma grade\ldots{}

E venha isto e mais aquilo --- e as novidades vinham vindo e os cinco
contos iam indo muito mais depressa do que ele o imaginou.

Tudo isso não seria nada se não viesse também uma coisa bem fora dos
cálculos de Venâncio: visitas.

Um belo dia o correio trouxe uma carta do Rio: ``\ldots{} e soubemos que V.
está de maré, empacotado pela sorte grande (200 ou 500?) e já montado em
linda fazenda. E como andamos todos aqui muito amarelos, e a Bibi
necessitada, a conselho médico, de ares de campo, lembramo"-nos de passar
uns dias aí, se o caro parente não levar isso a mal\ldots{}''.

--- ``Caro parente''?!\ldots{}

Venâncio releu a missiva.

--- Quem será este novo parente, Ladislau Teixeira?

Consultou a mulher. Dona Fortunata refranziu a testa.

--- Vai ver que é aquele filho da Carola\ldots{}

--- ??

--- \ldots{} que casou por lá com uma tipa de beiço rachado\ldots{}

--- Ahn!\ldots{}

--- \ldots{} e esteve uma vez em Itaoca um ano atrás.

--- Em casa do Estevinho, sei\ldots{}

--- Isso. Um tal Lalau.

--- Sei, sei\ldots{} Mas que diabo de parentesco tem ele comigo? Só se por
parte de Adão e Eva\ldots{}

--- Você já reparou, Venâncio, quantos parentes estão aparecendo agora?

--- É verdade. Com este, cinco. E amigos, então? Nunca imaginei que os
possuísse tantos\ldots{}

Venâncio respondeu que a casa, casa de pobres, estava às ordens; que
viessem.

Vieram. Quinze dias depois um trole despejava no terreiro um senhor de
meia"-idade, sua esposa Filoca, três filhas empalamadas, Bibi, Babá,
Bubu, e mais uma preta mucama. Venâncio reconheceu"-os vagamente, mas por
delicadeza fingiu intimidade.

--- Bem"-vindos sejam à casa do parente pobre!

Lalau abraçou"-o carinhosamente.

--- Não diga isso! Você é hoje a glória da família. Recebeu a recompensa
que merecia. Quantas vezes eu não disse à Filoca: aquele nosso parente
vai longe, porque quem planta colhe. Não é verdade, Filoca?

Dona Filoca sibilou através do beiço rachado uma confirmação plena:

--- É sim! Nós nunca duvidamos do futuro do ``primo'' Venâncio.

Venâncio ficou sabendo que eram primos\ldots{}

Nisto um novo trole assomou à porteira. Lalau explicou:

--- Ia"-me esquecendo\ldots{} Vieram conosco umas vizinhas, moças muito
boazinhas, as Seixas. Não te avisei na carta porque foi coisa de última
hora. Devem ser parentas de dona Fortunata, ao que me disseram\ldots{}

Venâncio interrogou furtivamente a esposa com o olhar e esta
respondeu"-lhe com um imperceptível movimento de beiço.

Apearam do segundo trole três moças e uma negrinha. Lalau apresentou"-as.

--- Dona Fafá, dona Fifi, dona Fufu.

As moças abraçaram os fazendeiros com grande cordialidade e abriram"-se
em louvores às belezas bucólicas.

--- Veja, Fifi, que coisa estupenda esta paineira!

--- Nem diga! E aquele maravilhoso beija"-flor? Que belezinha! Como
ficaria bem no meu chapéu azul\ldots{}

E Babá para Venâncio:

--- Que ar, primo! Que pureza de ar! A vida aqui deve ser um encanto. E
que apetite dá! Eu, que não como nada, seria capaz de devorar um leitão
inteiro hoje!

A Bibi conversava com a ``prima'' Fortunata:

--- Leite há muito, já sei. Fazenda quer dizer fartura. Lá na capital o
leite é água de polvilho, e caríssimo! É como os ovos: pela hora da
morte e metade chocos. Sua galinhada, quantas dúzias põe por dia?

E a Fifi para a Bubu:

--- Pesei"-me antes de vir: 49 quilos, veja que miséria! Mas daqui não
saio sem alcançar 58! Ah, não saio! O meu peso normal deve ser este, diz
o médico.

Dona Fortunata atendia a todos, sorrindo amavelmente, enquanto Lalau, já
no pomar, investia contra as laranjas com fúria de ``retirante''.

--- A minha conta, quando me pilho num pomar, são três dúzias. Pelo"-me
por laranjas!

Venâncio, armando cara alegre, dizia"-lhe que era chupar, chupar\ldots{}

Mas lá consigo pensava que naquela toada não venderia aquele ano uma
dúzia sequer. Só o Lalau daria cabo da safra inteira em quinze dias\ldots{}

À decima quinta laranja Lalau parou, entupido.

--- Estou por aqui! --- grugulejou, riscando no pescoço o nível do
caldo.

E, confidencial, ao ouvido do primo:

--- Agora, que ninguém nos ouve, diga lá a verdade: duzentos ou
quinhentos contos?

Venâncio não teve ânimo de pronunciar a palavra vinte. Também não quis
mentir, e marombou:

--- Não chega lá. Tirei apenas uns cobrinhos\ldots{}

O primo cutucou"-lhe a barriga:

--- Está escondendo o leite? Faz muito bem, que isso de arrotar grandeza
é transformar"-se em ``fruteira'': todo mundo pega a aproveitar"-se.

E dando"-lhe o braço:

--- Conselho de velho: defenda os arames, enforque a cobreira! Do
contrário, começam a aparecer amigos e parentes que não acabam mais.

Venâncio entreparou pasmado.

--- É o que lhe digo --- prosseguiu Lalau. --- Enquanto não possuímos
nada, ninguém se importa com a gente. Mas logo que a maré chega, brotam
da terra aproveitadores --- como cogumelos!

Venâncio pasmou dois pontos mais, e Lalau, lendo a seu modo aquele
pasmo, insistiu:

--- É o que lhe digo! Como cogumelos! Você é inexperiente ainda, não tem
os anos que tenho, e deve, portanto, ouvir"-me. Como parente próximo,
zelo pela família e faço grande empenho em abrir os seus olhos contra a
caterva de parasitas que vai por este mundo de Cristo. Quer saber de uma
coisa? Foi por esse motivo que eu vim. Motivo real! O resto foi
pretexto, você compreende. Eu disse à Filoca: é preciso abrir os olhos
ao primo; dinheiro escorrega das mãos como peixe e se lhe não acudo com
os meus conselhos, adeus sorte grande! Vê? Foi por este motivo que vim.

Inda atônito, Venâncio balbuciou umas palavras de agradecimento pela
generosa intenção, e Lalau, colhendo nova laranja, continuou:

--- Porque, cá comigo, é assim: para salvar um parente não poupo
sacrifícios! Ah, não poupo! Vou longe atrás dele, gasto dinheiro, mas
aviso"-o. Pensa que não foi um sacrifício esta minha viagem? Só de trem,
duzentos mil réis! Mas, como já disse, não olho a despesas. É parente? É
amigo? Não olho a despesas. Ah, não olho! Não acha que devo ser assim?

--- Está claro --- sussurrou Venâncio.

--- Parece claro, mas poucos pensam deste modo e, em vez de sacrificarem
um bocado das suas comodidades e virem abrir os olhos ao parente em
perigo, sabe o que fazem?

--- ?

--- Vêm explorá"-lo. Vêm ex"-plo"-rá"-lo, primo! Admira"-se? Pois saiba que o
mundo está cheio de gente assim. Olhe, eu conheço um caso que\ldots{}

Nessa noite o casal de fazendeiros passou a dormir na cozinha. Tiveram
que ceder seu quarto ao Lalau e à esposa. As B\ldots{} acomodaram"-se na sala
de espera. As F\ldots{}, numa alcova. As duas criadas, na despensa. Ficou a
casa repleta, tendo a cozinheira de dormir fora, no paiol.

Venâncio perdeu o sono. Altas horas inda matutava:

--- Não sei como está para ser! De um momento para outro, onze bocas a
mais\ldots{}

--- E que bocas! --- observou dona Fortunata. --- Como comem! A tal
Fifi, que é um bilro e parece viver de brisas, bebeu um litro de leite
para ``rebater'' meia dúzia de ovos. E sabe o que disse, toda
espevitada? ``Isto é para começar\ldots{} O médico mandou"-me ir aumentando as
doses aox poucox\ldots{}'' Veja você!

--- Parece que chegaram da seca do Ceará! Lalau chupou duma assentada
quinze laranjas, e das de umbigo\ldots{}

--- Esse não me admiro, que é homem e grandalhão. Mas aquele figo seco
da tal prima Filoca? Com partes de enfastiada, foi à cozinha e chamou
para o bucho todos os torresmos que eu tinha guardado para você. Dizem
que é o ar\ldots{}

--- Ar! Ar! Eu respiro o mesmo ar e nunca tenho apetite. Esfaimados por
natureza é o que eles são.

--- E depois isto de comer à custa alheia deve ser um regalo! ---
concluiu dona Fortunata, valente criatura que jamais provara um quitute
que não fosse preparado por suas próprias mãos.

O sono custou a vir, mas veio, e com ele um sonho. Sonhou Venâncio que
uma nuvem de gafanhotos vinda do Sul se abatera no sítio, deixando"-o nu
em pelo, sem folha nas árvores, nem soca de capim nos pastos.

Despertou sobressaltado. A manhã ia alta, com réstias de sol a coarem"-se
pelos vidros. Saltou da cama e foi à janela. Um vulto caminhava rumo ao
pomar, de pijama, faca de mesa na mão, assobiando despreocupadamente o
\emph{pé de anjo}.

--- Lá vai ele --- murmurou Venâncio. --- Lá vai às laranjas"-baianas\ldots{}

--- Quem? --- indagou a esposa, interrompendo o amarrar da saia.

--- Ora quem! O gafanhoto"-mor.

E como a esposa fizesse cara de interrogação, Venâncio contou"-lhe o
sonho da nuvem.

Dona Fortunata concluiu o nó da saia apreensivamente:

--- Queira Deus não dê certo!

Deu certo. Nunca um sonho profético antepintou o futuro com maior
precisão. Os hóspedes devoraram o sítio do Venâncio em poucas semanas.
Foram"-se todos os porcos, transfeitos em torresmos, lombo assado e
linguiça. Os lindos leitõezinhos que brincavam no terreiro acabaram no
espeto, um por um. O mesmo destino tiveram as aves, com exceção do casal
de Orpingtons, amarelas, que muito tentou a gula dos hóspedes, mas que
Venâncio, por precaução, mandou esconder em casa de um vizinho. Os ovos,
porém, se perderam.

--- Sabe --- disse dona Fortunata ao marido uma noite (era sempre à
noite, na cama, que murmuravam contra a praga dos gafanhotos) ---, sabe
que a ninhada de ovos de raça já se foi?

--- Não me diga! --- exclamou Venâncio.

--- Pois escondi"-os num canto, no quarto dos badulaques, mas aquele pau
de virar tripa da Bubu meteu o nariz lá e descobriu"-os e veio berrando
muito lampeira: ``Prima, suas galinhas estão botando no quarto dos
cacaréus. Olhe que lindos ovos encontrei lá! Duas dúzias: a continha
certa para hoje''. Expliquei"-lhe o caso, contei que eram ovos de raça,
caros, que você reservava para chocar. Sabe o que a bisca respondeu?
``Ora, não seja somítica. Nós vamos embora logo e suas galinhas ficam
por aqui botando ovos pelo resto da vida.''

Venâncio suspirou.

Um mês. Dois meses. Três meses.

No dia em que os hóspedes se foram, Venâncio mais a esposa deram uma
volta pelo sítio, em desconsoladora inspeção. Tudo deserto. Nem um
frango no galinheiro, nem uma goiaba no pomar, nem um porquinho na ceva.

--- Comeram até o cachaço! --- murmurou Venâncio, sacudindo a cabeça.

Na horta, as leiras de couve só apresentavam talos esguios --- folhas
nenhuma. Os pés de abóbora davam dó: nem uma aboborinha, nem um broto\ldots{}

--- Como eles gostavam de cambuquira! --- recordou dona Fortunata.

Finda a inspeção, um olhou para o outro, com desanimadíssimos focinhos.

--- E agora? --- indagou a mulher.

--- Agora? --- repetiu Venâncio. --- Agora é fazer a trouxa e tocar para
Itaoca antes que morramos de fome.

--- E volta você para o empreguinho?

--- Que remédio? Os ``primos'' devoraram a carne; tenho que roer o osso.

E foi graças ao apetite daqueles bem"-aventurados primos que Itaoca viu
reintegrar"-se em seu seio um precioso elemento social. As palestras da
botica andavam mortas, e sempre que se ventilava um ponto agrícola todos
lamentavam a ausência do argumentador seguro, que sempre detivera com
tanto brilho a palma da vitória.

Mas a volta de Venâncio foi uma decepção. O antigo entusiasmo
murchara"-lhe e nunca mais em sua vida piou sobre o tema favorito. E se
acaso falavam perto dele em pragas da lavoura, geada, ferrugem,
curuquerê ou o que seja, sorria melancolicamente, murmurando de si para
si:

--- Conheço uma muito pior\ldots{}

E conhecia.

\chapter{Um suplício moderno\footnote[*]{Texto de 1916, publicado no livro \emph{Urupês}.}}

Todas as crueldades de que foi useira a Inquisição para reduzir
heréticos, as torturas requintadas da ``questão'' medieval, o
empalamento otomano, o suplício chinês dos mil pedaços, o chumbo em
fusão metido a funil gorgomilos adentro --- toda a velha ciência de
martirizar subsiste ainda hoje encapotada sob hábeis disfarces. A
humanidade é sempre a mesma cruel chacinadora de si própria, numerem"-se
os séculos anterior ou posteriormente a Cristo. Mudam de forma as
coisas; a essência nunca muda. Como prova denuncia"-se aqui um avatar
moderno das antigas torturas: o estafetamento.

Este suplício vale o torniquete, a fogueira, o garrote, a polé, o touro
de bronze, a empalação, o bacalhau, o tronco, a roda hidráulica de
surrar. A diferença é que estas engenharias matavam com certa rapidez,
ao passo que o estafetamento prolonga por anos a agonia do paciente.

Estafeta"-se um homem da seguinte maneira: o Governo, por malévola
indicação dum chefe político, hodierno sucedâneo do ``familiar'' do
Santo Ofício, nomeia um cidadão estafeta do correio entre duas cidades
convizinhas não ligadas por via férrea.

O ingênuo vê no caso honraria e negócio. É honra penetrar na falange
gorda dos carrapatos orçamentívoros que pacientemente devoram o país; é
negócio lambiscar ao termo de cada mês um ordenado fixo, tendo
arrumadinha, no futuro, a cama fofa da aposentadoria.

Note"-se aqui a diferença entre os ominosos tempos medievos e os
sobreexcelentes da democracia de hoje. O absolutismo agarrava às brutas
a vítima e, sem tir"-te nem \emph{habeas corpus}, trucidava"-a; a
democracia opera com manhas de Tartufo, arma arapucas, mete dentro
rodelas de laranja e espera aleivosamente que, \emph{sponte sua}, caia
no laço o passarinho. Quer vítimas ao acaso, não escolhe. Chama"-se a
isto --- arte pela arte\ldots{}

Nomeado que é o homem, não percebe a princípio a sua desgraça. Só ao
cabo de um mês ou dois é que entra a desconfiar; desconfiança que por
graus se vai fazendo certeza, certeza horrível de que o empalaram no
lombilho duro do pior matungo das redondezas, com, pela frente, cinco,
seis, sete léguas de tortura a engolir por dia, de mala postal à garupa.

Eis as puas do aparelho de tormento, as tais léguas! Para o comum dos
mortais, uma légua é uma légua; é a medida duma distância que principia
aqui e acaba lá. Quem viaja, feito o percurso, chega e é feliz.

As léguas do estafeta, porém, mal acabam voltam ``da capo'', como nas
músicas. Vencidas as seis (suponhamos um caso em que sejam só seis)
renascem na sua frente de volta. É fazê"-las e desfazê"-las. Teia de
Penélope, rochedo de Sísifo, há de permeio entre o ir e o vir a má
digestão do jantar requentado e a noite maldormida; e assim um mês, um
ano, dois, três, cinco, enquanto lhes restarem, a ele nádegas e ao
sendeiro lombo.

Quando cruza um viandante a jornadear, morde"-o a inveja: aquele breve
``chegará'', ao passo que para o estafeta tal verbo é uma irrisão. Mal
apeia, derreado, com o coranchim em fogo, ao termo dos trinta e seis mil
metros da caminheira, come lá o mau feijão, dorme lá a má soneca e a
aurora do dia seguinte estira"-lhe à frente, à guisa de ``Bom dia!'', os
mesmos trinta e seis mil metros da véspera, agora espichados ao
contrário\ldots{}

Breve o animal, pisado, dá de si, fraqueia. Já os topes o cavaleiro
galga a pé. Não possui meios de adquirir outra montada. O ordenado
vai"-se"-lhe em milho e ``rapador'' para a alimária, água de sal para os
semicúpios e mais remédios às pisaduras de ambos, cavalgante e
cavalgado. Não sobeja sequer para roupa.

Dá"-lhe o Estado --- o mesmo que custeia enxundiosas taturanas
burocráticas a contos por mês, e baitacas parlamentares a duzentos
mil"-réis por dia ---, dá"-lhe o generoso Estado\ldots{} cem mil"-réis mensais.
Quer dizer, ``um real'' por nove braças de tormento. Com um vintém
paga"-lhe trezentos e trinta metros de suplício. Vem a sair a sessenta
réis o quilômetro de martírio. Dor mais barata é impossível.

O estafeta entra a definhar de canseira e fome. Vão"-se"-lhe as carnes, as
bochechas encovam, as pernas viram parênteses dentro dos quais mora a
barriga do desventurado rocim.

Além das calamidades fisiológicas, econômicas e sociais, chovem"-lhe em
cima as meteorológicas. O tempo inclemente não lhe poupa judiarias. No
verão não se dói o sol de assá"-lo como se assam pinhões nas cinzas. Se
chove, de nenhuma gota se livra. Pelos fins de maio, à entrada do frio,
é entanguido como um súdito de Nicolau exilado nas Sibérias que devora
as léguas infernais. No dia de são Bartolomeu, agarrado de unhas à crina
da escanzelada égua, é por milagre que não os despeja a ambos,
perambeiras abaixo, o endemoninhado vento.

O patrão"-Governo pressupõe que ele é de ferro e suas nádegas são de aço;
que o tempo é um permanente céu com ``brisas fagueiras'' ocupadas em
soprar sobre os caminhantes os olores da ``balsamina em flor''.

Pressupõe ainda que os cem mil"-réis do salário são uma paga real de
lamber as unhas. E, nestas angelicais pressuposições, quando há crises
financeiras e lhe lembram economias, corta seus cinco, seus dez mil"-réis
no pingue ordenado, para que haja sobras permitidoras de ir à Europa um
genro em
comissão de estudos sobre ``a influência zigomática do periélio solar no
regime zaratústrico das democracias latinas''.

E assim o exército dos estafetas, dia a dia mais encanifrado,
encalacrado de dívidas, enchagado de pisaduras, ao sol de dezembro ou à
garoa entanguente de junho, trota, trota sem cessar, morro acima, morro
abaixo, por atoleiros e areões, caldeirões e escorregadouros, sacudido
pela miseranda cavalgadura que de tanto padecer, coitada, já nem jeito
de cavalo tem.

O lombo delas é todo uma chaga viva; as costelas, um ripado. Caricaturas
contristadoras do nobre \emph{Equus,} um dia rebentam de fome, exaustas,
a meio de viagem.

O estafeta toma às costas os arreios, a mala, e conclui a caminheira a
pé. Nesse dia chega fora de horas, e o agente do correio oficia ao
centro sobre a ``irregularidade''.

O centro move"-se; faz correr um papelório através de várias salas onde,
comodamente espapaçada em poltronas caras, a burocracia gorda palestra
sobre espiões alemães. Depois de demorada viagem o papelório chega a um
gabinete onde impa em secretária de imbuia, fumegando o seu charuto, um
sujeito de boas carnes e ótimas cores. Este vence dois contos de réis
por mês; é filho de algo; é cunhado, sogro ou genro de algo; entra às
onze e sai às três, com folga de permeio para uma ``batida'' no frege da
esquina.

O canastrão corre os olhos mortiços de lombeira por sobre o papel e
grunhe:

--- Estes estafetas, que malandros!

E assina a demissão daquele a bem do serviço público.

(E se isso não acontece, acontece pior. Certa vez o agente do correio
duma cidadezinha paulista oficiou ao centro queixando"-se do estafeta. O
centro respondeu autorizando"-o a ``punir com severidade o faltoso''. O
agente medita a sério sobre o caso; depois, mostrando o ofício ao
estafeta, e com muita dor de coração, ferra"-lhe em nome do Governo a
maior sova de chicote de que há memória no lugar. Em seguida oficia ao
centro dando conta do desempenho da missão e declarando que o serviço
ficaria interrompido por uma quinzena, visto o paciente estar de cama, a
curar"-se com salmoura\ldots{})

O supliciado, posto no olho da rua, sem saúde, sem cavalo, sem nádegas,
coberto de dívidas, com o fígado e mais vísceras fora do lugar em
virtude do muito que ``chacoalharam'', vê"-se logo rodeado pela chusma de
credores, ávidos como urubus de charqueada. Como está nu, mais nu que
Jó, não pode pagar a nenhum --- e ganha fama de caloteiro.

--- Parecia um homem sério, e no entanto roubou"-me cinco alqueires de
milho --- diz o da venda, calabrês gordo, enricado no passamento de
notas falsas.

--- Tomou"-me emprestados 100 mil"-réis para a compra de um cavalo, a
jurinho de amigo (cinco por cento ao mês), já lá vão cinco anos, e por
muito favor pagou"-me o premiozinho e deu os arreios por conta. Que
ladrão! --- diz o onzeneiro, sócio do outro na nota falsa.

A loja de fazenda chora umas calças de algodão mineiro que lhe fiou em
tempo. A farmácia, um quilo de sal"-amargo falsificado. Abeberado de
insultos, o mártir só vê pela frente uma saída: fincar o pé na estrada e
fugir\ldots{} fugir para uma terra qualquer onde o desconheçam e o deixem
morrer em paz.

Destarte, o moderno suplício do estafetamento, além de charquear as
carnes duma criatura humana limpa de crimes, dá"-lhe ainda de lambuja uma
bela mortezinha moral. Tudo isto a fim de que não falte aos soletradores
de tais e tais bibocas do sertão o pábulo diário da graxa preta em fundo
branco, por meio do qual se estampam em língua bunda as facadas que Pé
Espalhado deu em Camisa Preta, o queijo que furtou Baianinho ao Manoel
da Venda, o romance traduzido de Jorge Ohnet, o salvamento da pátria
pela alta volataria nacional, o palavreado gordo das ligas disto e
daquilo, a descoberta de espiões onde nada há que espiar, a policultura,
o zebu, o analfabetismo, o aliadismo, o germanismo, as potocas da Havas
e quanta papalvice grela por massapês e terras roxas deste país das
arábias.

A política do coronel Evandro em Itaoca deu com o rabo na cerca desde
que em tal pleito o competidor Fidêncio, também coronel, guindou a
cotação dos votos de gravata a quinhentos mil"-réis, e a dos votos de pé
no chão a dois parelhos de roupa, mais um chapéu.

O primeiro ato do vencedor foi correr a vassoura do Olho da Rua em tudo
quanto era \emph{olhodarruável} em matéria de funcionalismo público.
Entre os varridos estava a gente do correio, inclusive o estafeta, para
cuja substituição inculcou"-se ao Governo o Izé Biriba.

Era este Biriba um caranguejo humano, lerdo de maneiras e atolambado de
ideias, com dois percalços tremendos na vida --- a política e o topete.

O topete consistia num palmo de grenha teimosa em lhe cair sobre a
testa, e tão insistente nisto que gastava ele metade do dia erguendo a
mão esquerda à altura da fronte para, num movimento maquinal, botar pra
arriba a crina rebelde. A política escusa dizer o que é.

Coligados ambos, topete e política comiam"-lhe o tempo inteiro, de jeito
a não lhe deixar folga nenhuma para o amanho do sítio, que, afinal,
roído pelo cupim da hipoteca, lá foi parar nas unhas dum onzeneiro
ladrão.

Montou em seguida botequim mas faliu. Enquanto Biriba arrumava o topete
os fregueses surrupiavam"-lhe os mata"-bichos; e nas cavaqueiras políticas
os correligionários, de passo que expeliam diatribes contra o governo,
sorviam capilés refrescantes e mascavam bolinhos de peixe por conta da
vitória futura.

Além do topete tinha Biriba o sestro do ``sim senhor'' alçado às funções
de vírgula, ponto e vírgula, dois"-pontos e ponto final de todas as
parvoiçadas emitidas pelo parceiro; e às vezes, pelo hábito, quando o
freguês parando de falar entrava a comer, continuava ele escandindo a
``sim senhores'' a mastigação do bolinho filado.

Ao tempo da queda do outro e subida de sua gente, andava Biriba reduzido
à conspícua posição de ``fósforo'' eleitoral. No pleito trabalhara como
nenhum. Deram"-lhe as piores missões --- acuar eleitores tabaréus
embibocados nos socavões das serras, negociar"-lhes a consciência,
debater preço de votos, barganhá"-los com éguas lazarentas e provar aos
desconfiados, com argumentos de cochicho ao ouvido, que o Governo estava
com eles.

Após a vitória sentiu pela primeira vez um gozo integral de coração,
cabeça e estômago.

Vencer! Oh, néctar! Oh, ambrosia incomparável!

O nosso homem regalou as vísceras com o petisco dos deuses. Até que
enfim os negrores da vida de misérias lhe alvorejavam em aurora. Comer à
farta, serrar de cima\ldots{} Delícias do triunfo!

Que lhe daria o chefe?

No antegozo da pepineira iminente, viveu a rebolar"-se em cama de rosas
até que rebentou sua nomeação para o cargo de estafeta.

Sem queda para aquilo, quis relutar, pedir mais; na conferência que teve
com o chefe, entretanto, as objeções que lhe vinham à boca
transmutavam"-se no habitual ``sim senhor'', de modo a convencer o
coronel de que era aquilo o seu ideal.

--- Veja, Biriba, quanto vale a felicidade! Pilha um empregão! Vai
Regino para agente e você para estafeta.

O mais que ele pôde alegar foi que não tinha cavalgadura.

--- Arranja"-se --- resolveu de pronto o coronel. --- Tenho lá uma égua
moura legítima, de passo picado, que vale duzentos mil"-réis. Por ser
para você, dou"-a por metade. O dinheiro? É o de menos. Você toma"-o de
empréstimo a Leandrinho. Arranja"-se tudo, homem.

O arranjo foi adquirir Biriba uma égua trotona pelo dobro do valor, com
dinheiro tomado a três por cento ao tal Leandro, que outra coisa não era
senão o testa de ferro do próprio Fidêncio. Destarte, carambolando, o
matreiro chefe punha a juros o pior sendeiro da fazenda, além de
conservar pelo cabresto da gratidão ao idiota estafetado.

Iniciou Biriba o serviço: seis léguas diárias a fazer hoje e a desfazer
amanhã, sem outra folga além do último dia dos meses ímpares.

Inda bem se fora devorar as léguas na só companhia da chupada mala
postal. Mas não lhe saiu serena assim a empresa. Como Itaoca não
passasse de mesquinho lugarejo empoleirado no espinhaço da serra e
desprovido de tudo, não transcorria vez sem que os amigos políticos não
viessem com encomendas a aviar na cidade. À hora de partir surgiam
aproveitadores com listinhas de miudezas, ou negras com recados.

--- Sinhá disse assim pra suncê comprar três carretéis de linha
cinquenta, um papel de agulhas, uma peça de cadarço branco, cinco maços
de grampo miúdo e, se sobejar um tostão, pra trazer uma bala de apito
pro seu Juquinha.

Todos aqueles artigos existiam em Itaoca, um tantinho mais caros, porém;
o encomendá"-los fora visava apenas à economia do tostão da bala de
apito.

--- Sim senhor, sim senhor!\ldots{}

Não lhe escapava da boca outro som, embora o exasperasse a contínua
repetição do abuso.

Além das pequenas encomendas, pouco trabalhosas, surgiam outras de
vulto, como levar um cavalo arreado ao senhor Fulano que vinha em tal
dia, acompanhar a mulher de Etcetrano, e que tais. Tibúrcia, cozinheira
preta do coletor, cada vez que ia de férias descansar à cidade, era
Biriba o indicado para conduzi"-la.

Foi como o conheci, guardando cesta às amazonas. De viagem para Itaoca,
a meio caminho topo num homem encavalgado na mais avariada égua que
jamais meus olhos viram. À garupa iam malas do correio e vários picuás;
no santo"-antônio, mais picuás além duma vassoura nova enganchada nos
arreios com a palha para cima. Estava parado, em atitude idiotizada,
segurando pelo cabresto um cavalinho de silhão. Abordei"-o, pedindo fogo.
Aceso o cigarro, indaguei de quem montava a cavalgadura vazia.

--- ``Não vê'' que estou acompanhando a dona Engrácia, que é parteira em
Itaoca. Ela apeou um bocadinho e\ldots{}

Ouvi rumor atrás: saía do mato uma mulheraça rúbida, de saias tufadas de
goma, tendo na cabeça um toucadinho coevo de S. M. Fidelíssima\ldots{} Para
não vexá"-la pus"-me a caminho, não sem, voltando a cara de soslaio,
regular"-me com os apuros do estafeta para entalar nas andilhas as cinco
arrobas da parteira aliviada.

E descomposturas\ldots{}

--- Seu Biriba, não foi linha quarenta que eu encomendei. O senhor
parece bobo!

Quando a fazenda era má:

--- Não viu que a chita desbotava? Que moda!

Doía"-lhe, sobretudo, carretear para a execrável gente da oposição. O
coronel contrário não se pejava de por intromissão de terceiro, neutro
ou oposicionista encapotado, abusar da boa"-fé do mártir. Lembrava"-se
Biriba, com dor de alma, de um bode de raça que lhe dera grandes
trabalhos pelo caminho --- e várias marradas de lambuja; afinal,
chegando, verificou que vinha para o inimigo.

Toda gente gozou do caso, entre espirros de riso e galhofa.

--- É um pax"-vóbis Biriba! Trazer o bode da oposição! Quiá! quiá! quiá!

Estas e outras foram"-lhe azedando os fígados e as vísceras
circunvizinhas. Biriba emagreceu. Biriba amarelou.

A égua, coitada, perdeu a feição cavalar. Seu lombo selara em meia"-lua,
de modo que por um nadinha não raspavam o chão os pés do cavaleiro.
Montado, Biriba afundava. Sua cabeça caía quase ao nível duma linha
tirada da anca às orelhas da égua. Horrendamente pisada, trazia a bicha
nos olhos permanentes lágrimas de dor; mas em vez de tanta mazela mover
ao dó o coração dos itaoquenses, regalava"-os, e eram chufas sem fim e
piadas idiotas acerca do ``Estafeta da Triste Figura mais a sua
Bucéfala'', como os batizou um engraçado local.

Lazarento como eles, só o Cunegundes, cão sem dono, coberto de sarna,
que perambulava a esmo pela cidade, fugindo a moscas e pontapés. Pois
não lhe mudaram o nome para Biribinha? Cachorrada!

Não tardou muito viesse o Governo dar sua volta ao torniquete, cortando
dez mil"-réis no ordenado dos estafetas --- para salvar"-se em certa
ocasião de apuros financeiros. E salvou"-se, esta é que é!\ldots{}

A roupa no fio. À entrada das chuvas uma alma caridosa deu"-lhe uma velha
capa de borracha; mas no primeiro aguaceiro verificou Biriba que tal
capote vazava como peneira, de modo a piorar"-lhe a situação com a
sobrecarga dum panejamento absorvedor de litros de água.

Biriba, perdida a paciência, murmurou.

Ai! Soube"-o logo o chefe e fê"-lo vir a contas.

--- É certo que o senhor me anda arrenegando do emprego que lhe demos?
Queria, acaso, ser eleito senador ou vice"-presidente? Um pedaço de
porcalhão que andava aí lambendo embira, morre não morre de fome, passa,
por generosidade nossa, a ocupar um cargo federal com ordenado
relativamente bom (aqui Biriba tossiu um\ldots{} ``Sim senhor''), encontra
todas as facilidades, recebe um bom animal e ainda se queixa? Que quer
então Vossa Excelência?

Biriba entumeceu"-se de coragem e declarou querer uma coisa só: a
demissão. Estava doente, surradíssimo, ameaçado de perder de um momento
para outro a égua e as nádegas. Queria mudar de vida.

--- Muda"-se, então, de vida assim do pé pra mão? Quer abandonar os
amigos? E a disciplina partidária onde fica, meu caro palerma?

Não convinha a ninguém a saída do Biriba. Quem mais serviçal?
Lembravam"-se dos estafetas anteriores, malcriados, inimigos de trazer um
papel de agulha fosse para quem fosse. Não sairia. Itaoca impunha"-lhe o
sacrifício de ficar.

Mas a tortura do diário chocalhar por sete léguas das vísceras de Biriba
acabou por desconjuntar nele o cimento da lealdade partidária. O mártir
abriu os olhos. Lembrou"-se com saudades dos ominosos tempos do coronel
Evandro, das delícias do botequim e até do calamitoso período da
degradação ``fosfórica''. Piorara após o triunfo, não havia dúvida.

Este livre exame de consciência --- crede"-me --- foi o início da queda
do coronel Fidêncio em Itaoca. Biriba, o firme esteio, apodrecia pelo
nabo; viria abaixo, e com ele a cumeeira do pardieiro político. A víbora
da traição armara ninho em sua alma.

Como o novo pleito se aproximasse, nova vitória lhe seria novo termo de
martírio. Biriba ponderou de si para sua égua que a salvação de ambos
estava na derrota. Demitiam"-no, e ele, veterano e mártir do fidencismo,
continuaria com jus ao apoio do partido, sem padecer por via coccigiana
o contato odioso das sete horas diárias de socado.

Deliberou trair.

Na véspera da eleição incumbiu"-o Fidêncio de trazer da cidade um papel
importantíssimo para o tribofe das urnas. Sei lá o que era! Um
``papel''. A palavra ``papel'' dita assim em tom de mistério traz no
bojo ``coisas''\ldots{}

Fidêncio frisou a gravidade da incumbência --- a maior prova de
confiança jamais dada por ele a um cabo eleitoral.

--- Veja lá! A nossa sorte está nas suas mãos. Isto é que é confiança,
hein?

Partiu Biriba. Recebeu na cidade o ``papel'' e rodou para trás. A meio
caminho, porém, tomou por uma errada, foi ter à biboca dum negro velho,
soltou a égua, pegou de prosa com o gorila. Caiu a noite: Biriba
deixou"-se ficar. Alvoreceu o dia seguinte: Biriba quieto. Dez dias se
passaram assim. Ao cabo, arreou a égua, montou e botou"-se para Itaoca
como se nada houvera acontecido.

Foi um assombro a sua aparição. Baldadas as tentativas para apanhá"-lo no
dia do pleito e nos posteriores, deram"-no como papado pelas onças, ele,
égua, mala postal e ``papel''. Vê"-lo agora surgir sãozinho da silva foi
um abrir de boca e um pasmar à vila inteira. Que houve? Que não houve?

A todas as perguntas Biriba armava na cara a suprema expressão da
idiotia. Nada explicava. Não sabia de nada. Sono cataléptico? Feitiço?
Não compreendia o sucedido. Afigurava"-se"-lhe ter partido na véspera e
estar de volta no dia certo.

Ficaram todos maravilhados, com asníssimas caras.

Fidêncio delirava na cama, com febre cerebral. Perdera a eleição
redondamente.

--- Derrota fedida --- arrotavam os vencedores, atochando foguetes de
assobio.

Em consequência do inexplicável eclipse do estafeta senhoreou"-se do
rebenque o ex"-ominoso Evandro. Começou a derrubada. O olho da rua
recebeu em seu seio tudo quanto cheirava a fidencismo. A vassoura da
demissão, porém, poupou a\ldots{} Biriba.

O novo cacique aproximou"-se dele e disse:

--- Demiti toda a canalha, Biriba, menos a você. Você é a única coisa
que se salva da quadrilha de Fidêncio. Fique sossegado, que do seu
lugarzinho ninguém o arranca, nem que o céu chova torqueses.

Pela derradeira vez em Itaoca Biriba balbuciou o ``Sim senhor''. À noite
deu um beijo no focinho da égua e saiu de casa pé ante pé. Ganhou a
estrada e sumiu.

E nunca mais ninguém lhe pôs a vista em cima\ldots{}

\chapter{«Pollice verso»\footnote[*]{Texto de 1916, publicado no livro \emph{Urupês}.}}

Dos dezesseis filhos do coronel Inácio da Gama cedo revelou o caçula
singulares aptidões para médico. Pelo menos assim julgara o pai, como
quer que o encontrasse na horta interessadíssimo em destripar um
passarinho agonizante.

--- Descobri a vocação de Nico --- disse o arguto sujeito à mulher. ---
Dá um ótimo esculápio. Inda agorinha o vi lá fora dissecando um sanhaço
vivo.

Hão de duvidar os naturalistas estremes que o homem dissesse
``dissecando''. Um coronel indígena falar assim com este rigor de
glótica é coisa inadmissível aos que avaliam o gênero inteiro pela meia
dúzia de pafúncios agaloados do seu conhecimento. Pois disse. Este
coronel Gama abria exceção à regra; tinha suas luzes, lia seu jornal,
devorara em moço o \emph{Rocambole}, as \emph{Memórias de um médico} e
acompanhava debates da Câmara com grande admiração por Rui Barbosa,
Barbosa Lima, Nilo e outros. Vinha"-lhe daí um certo apuro na linguagem,
destoante do achavascado ambiente glóssico da fazenda, onde morava.

Quem nada percebeu foi dona Joaquininha, a avaliar pelo ar emparvecido
que deu à cara.

--- Dissecando --- explicou superiormente o marido --- quer dizer
destripando.

--- E deixou você que ele cometesse semelhante malvadeza? --- exclamou a
excelente senhora, compadecida.

--- Lá vens com a pieguice!\ldots{} Deixa"-lo brincar, que é da idade, eu em
pequeno fazia piores e nem por isso virei nenhum ogre.

(Outra vez! ``Ogre''! O homem nascera precioso. Este ogre devia ser
reminiscência do Ogre da Córsega, Napoleão chamado. Perdoem"-lho à guisa
de compensação à parcimônia da esposa, cujo vocabulário era dos mais
restritos.)

Dona Joaquina fechou a cara, e quando o pequeno facínora entrou do
quintal pediu"-lhe contas da perversidade, asperamente. O coronel, que
nesse momento lia na rede as folhas recém"-chegadas, houve por bem
interromper a ingestão de um flamante discurso sobre a questão do Amapá
para acudir em apoio ao fedelho.

--- Uma vez que será médico, não vejo mal em ir"-se familiarizando com a
anatomia\ldots{}

--- A anatomia está ali! --- rematou a encolerizada senhora apontando a
vara de marmelo oculta atrás da porta. --- Eu que saiba que o senhor me
anda com judiarias aos pobres animaizinhos, que te disseco o lombo com
aquela anatomia, ouviu, seu carniceiro?

O menino raspou"-se; o coronel retomou resignado o fio do discurso; e o
caso do sanhaço ficou por ali.

Mas não ficou por ali a malvadez de Nico. Acautelava"-se agora. Era às
escondidas que ``depenava'' moscas, brinquedo muito curioso, consistente
em arrancar"-lhes todas as pernas e asas para gozar o sofrimento dos
corpinhos inertes. Aos grilos cortava as saltadeiras, e ria"-se de ver os
mutilados caminharem como qualquer bichinho de somenos.

Gatos e cães farejavam"-no de longe, aterrorizados. Fora ele quem cortara
o rabo ao mísero Joli da agregada Emiliana, e era quem descadeirava
todos os gatos da fazenda. Isso, longe. Em casa, um anjinho. E assim,
anjo internamente e demônio extramuros, cresceu até à mudança de voz.
Entrou nesse período para um colégio, e deste pulou para o Rio,
matriculado em medicina.

O emprego que lá deu aos seis anos do curso soube"-o ele, os amigos e as
amigas. Os pais sempre viveram empulhados, crentes de que o filho era
uma águia a plumar"-se, futuro Torres Homem de Itaoca, onde, vendida a
fazenda, então moravam. Nesta cidade tinham em mente encarreirar o
menino, para desbanque dos quatro esculápios locais, uns onagros, dizia
o coronel, cuja veterinária rebaixava os itaoquenses à categoria de
cavalos.

Pelas férias o doutorando aparecia por lá, cada vez ``mais outro'',
desempenado, com tiques de carioca, ``ss'' sibilantes, roupas caras e
uns palavreados técnicos de embasbacar.

Quando se formou e veio de vez, estava já definitivo, nos 24 anos. Não
se lhe descreve aqui a cara, porque retratos por meio de palavras têm a
propriedade de fazer imaginar feições às vezes opostas às descritas.
Dir"-se"-á unicamente que era um rapaz espigado, entre louro e castanho,
bonito mas antipático --- com o olhar de Stuart Holmes, diziam as
meninas doutoras em cinemas. No queixo trazia barba de médico francês,
coisa que muito avulta a ciência do proprietário. Doentes há que entre
um doutor barbudo e um glabro, ambos desconhecidos, pegam sem tir"-te no
peludo, convictos de que pegam no melhor.

O doutor Inacinho, entretanto, aborrecia aquele meio acanhado ``onde não
havia campo''.

``Isto aqui'' --- contava em carta aos colegas do Rio --- ``é um puro
degredo. Clínica escassa e mal pagante, sem margem para grandes lances,
e inda assim repartida por quatro curandeiros que se dizem médicos,
perfeitas vacas de Hipócrates, estragadores de pepineira com suas
consultinhas de cinco mil"-réis. O cirurgião da terra é um Doyen de
sessenta anos, emérito extrator de bichos"-de"-pé e cortador de verrugas
com fio de linha. Dá iodureto a todo mundo e tem a imbecilidade de
arrotar ceticismo, dizendo que o que cura é a natureza. Estes rábulas é
que estragam o negócio'' --- etc.

Negócio, pepineira, grandes lances --- está aqui a psicologia do novo
médico. Queria pano verde para as boladas gordas.

``Além disso'' --- continuava ---, ``é"-me insuportável a ausência da
Yvonne e de vocês. Não há cá mulheres, nem gente com quem uma pessoa
palestre. Uma pocilga! As boas pândegas do nosso tempo, hein?''

Ora aqui está: Yvonne, os amigos, as pândegas foram o melhor do curso.
Com mão diurna e noturna manuseou"-os a estes tratadistas de anatomia, da
fisiologia, da calaçaria, e agora torturavam"-no saudades.

Yvonne volta à pátria, deixando cá a meia dúzia de amantes que depenara
a morrerem de saudades dos seus encantos. Antes de ir"-se deu a cada
parvo uma estrelinha do céu, para que, a tantas, se encontrassem nela os
amorosos olhares. Os seis idiotas todas as noites ferravam os olhos, um
no ``Taureau'' (ela distribuíra as constelações em francês), outro na
``Écrevisse'', outro na ``Chevelure de Bérenice'', o quarto, no
``Bélier'', o quinto em ``Antarés'', e o derradeiro na ``Épi de la
Vièrge''.

A garota morria de rir no colo dum apache montmartre, contando"-lhe a
história cômica dos seis parvos brasileiros e das seis constelações
respectivas. Liam juntos as seis cartas recebidas a cada vapor, nas
quais os protestos amorosos em temperatura de ebulição faziam perdoar a
ingramaticalidade do francês antártico. E respondiam de colaboração, em
carta circular, onde só variava o nome da estrela e o endereço.

Esta circular era o que havia de terno. Queixava"-se a rapariga de
saudades, ``essa palavra tão poética que fora aprender no Brasil, o belo
país das palmeiras, do céu azul, e dos michês''. Acoimava"-os de
ingratos, já em novos amores, ao passo que a pobrezinha, solitária e
triste ``\emph{comme la} juriti'', consagrava os dias a rememorar o doce
passado.

Eis explicada a razão pela qual, nas noites límpidas, ficava Inacinho à
janela, pensativo, de olhos postos na ``Chevelure de Bérenice''.

O sonho do moço era enriquecer às rápidas para reatar a gostosura do
idílio interrompido.

--- Paris!\ldots{} --- balbuciava à meia"-voz nos momentos de devaneio,
semicerrando os olhos no antegozo do paraíso. Sonhava"-se lá, riquinho,
com Yvonne pelo braço, flanando no ``Bois'', tal qual nos romances; e a
realização deste sonho era o alvo de todos os seus anelos. Jurara à
amiga ir ter com ela logo que a prosperidade lhe abastasse meios. O
tempo, entretanto, corria sem que nenhuma piabanha de vulto lhe caísse
na rede. Tardava a bolada\ldots{}

Entre os médicos antigos de Itaoca o doutor Inacinho gozava péssimo
renome --- se renome péssimo pode ser coisa de gozo.

--- Uma bestinha! --- dizia um. --- Eu fico pasmado mas é de saírem da
Faculdade cavalgaduras daquele porte! É médico no diploma, na barbicha e
no anel do dedo. Fora daí, que cavalo!

--- E que topete! --- acrescentava outro. --- Presumido e pomadista como
não há segundo. Não diz ``humores'' ou ``sífilis''; é \emph{mal
luético.} Eu o que queria era pilhá"-lo numa conferência, para
escachar\ldots{}

O pai, já viúvo então, esse babava"-se de orgulho. Filho médico, e ainda
por cima destabocado e bem falante como aquele\ldots{} Era de moer de inveja
aos mais. Enlevava"-o, sobretudo, aquele modo alcandorado de exprimir"-se.
Revia"-se no filho, o coronel\ldots{}

--- A terminologia inteira da ciência alopata, coisas em grego e latim,
circunvolve naquela cabecinha --- disse ele uma vez ao vigário, que o
olhou de revés, por cima dos óculos, ao som daquele mirífico
\emph{circunvolve}.

E assim corria o tempo, entre as diatribes das duas ciências, a moça e a
velha, com entremeio dos belos vocábulos que o coronel nunca perdia de
meter na falação.

Entrementes adoeceu o major Mendanha, capitalista aposentado com
trezentas apólices federais, o Rockefeller de Itaoca. Deu"-lhe uma súbita
aflição, uma canseira, e a mulher alvoroçou"-se.

--- Não é nada, isto passa --- acalmou ele.

--- Passará ou não!\ldots{} O melhor é chamar um médico.

--- Qual, médico! Isto é nada.

Não era tão nada assim, como pretendia. À noite agravou"-se"-lhe o
malestar, e o velho, apreensivo, cedeu às instâncias da esposa. Chamar a
qual deles, porém?

--- Pois o Moura --- disse a mulher, para quem o da sua confiança era
este Moura.

--- Deus me livre! --- retrucou o doente. --- Aquilo é homem mal
azarado. Pois não foi quem tratou Zeca, Peixoto, Jerônimo? E não
esticaram a canela todos três?

--- O doutor Fortunato, então\ldots{}

--- Fortunato! Já esqueceu você do que me ele fez por ocasião do júri, o
tranca? Cobrar cinquenta mil"-réis por um atestado falso? Não me pilha
mais um vintém, o pirata\ldots{}

No doutor Elesbão não se falou: era adversário político.

--- Chama"-se Galeno\ldots{}

--- É tão mosca"-morta Galeno\ldots{} --- gemeu o doente com cara de
desconsolo. --- Andou anos a tratar Faria do Hotel como diabético, e já
o dava por morto quando um curandeiro da roça o pôs saníssimo com um
coco"-da"-baía comido em jejum. Eram solitárias os diabetes do homem\ldots{} Só
se vier o filho de Inácio?!

Aqui foi a mulher quem protestou.

--- Eu, a falar a verdade, prefiro a ruindade de Galeno, a má sorte de
Moura, e até Elesbão\ldots{}

--- Esse, nunca!\ldots{} --- interrompeu o velho, num assomo de rancor
político.

--- \ldots{} do que a antipatia do tal doutorzinho. Os outros ao menos têm a
experiência da vida, ao passo que este\ldots{}

--- Este, quê?

--- Este, Mendanha, é moço bonito, que o que quer é dinheiro e pândega,
você não vê?

--- Qual!\ldots{} --- emberrinchou o teimoso. --- Sempre há de saber um pouco
mais que os velhos; aprendeu coisas novas. No caso de Nhazinha Leandro,
não a pôs boa num ápice?

--- Também que doença! Prisão de ventre\ldots{}

--- Seja prisão ou soltura, o caso foi que a curou. Mande chamar o
menino.

--- Olhe, olhe! Depois não se arrependa!\ldots{}

--- Mande, mande chamá"-lo e já, que não me estou sentindo bem.

Inacinho veio. Interrogou detidamente o major, tomou"-lhe o pulso,
auscultou"-o com o semblante carregado e disse, depois de longa pausa:

--- Não diagnostico por enquanto, porque não sou leviano como ``certos''
por aí. Sem auscultação estetoscópica nada posso dizer. Voltarei mais
tarde.

--- Vê? --- disse Mendanha à esposa logo que o moço partiu. --- Fosse
Moura, ou qualquer dos tais, e já dali da porta vinha berrando que era
isto mais aquilo. Este é consciencioso. Quer fazer uma auscultação, quê?

--- Estereoscópica, parece.

--- Seja o que for. Quer fazer a coisa pelo direito, é o que é.

Voltou o moço logo depois e com grande cerimonial aplicou o instrumento
no peito magro do doente. Vincou de novo a fisionomia das rugas da
concentração e concluiu com imponente solenidade:

--- É uma pericardite aguda agravada por uma flegmasia hepático"-renal.

O doente arregalou o olho. Nunca imaginara que dentro de si morassem
doenças tão bonitas, embora incompreensíveis.

--- E é grave, doutor? --- perguntou a mulher, assustada.

--- É e não é! --- respondeu o sacerdote. --- Seria grave se, modéstia
de lado, em vez de me chamarem a mim chamassem a um desses mata"-sanos
que por aí rabulejam. Comigo é diferente. Tive no Rio, na clínica
hospitalar, numerosos casos mais graves e a nenhum perdi. Fique
descansada que porei o seu marido completamente são dentro de um mês.

--- Deus o ouça! --- rematou a mulher acompanhando"-o até a porta e já
meio reconciliada com a ``antipatia''.

--- Então? --- perguntou"-lhe o doente. --- Fiz ou não fiz bem em chamar
este moço?

--- Parece\ldots{} Deus queira tenhamos acertado, porque isto de médicos é
sorte.

--- Não é tanto assim --- reguingou o velho. --- Os que sabem,
conhecem"-se por meia dúzia de palavras, e este moço ou muito me engano
ou sabe o que diz. Fosse Fortunato\ldots{}

E riu"-se lá consigo ao imaginar as doencinhas caseiras que Fortunato
descobriria nele\ldots{}

A doença do major Mendanha ninguém soube qual fosse. O lindo diagnóstico
de Inacinho não passava de mera sonoridade pelintra. Bacorejara ao moço
que o velho tinha o coração fraco e qualquer maromba no fígado. Isto
porque lhe doía, a ele, aqui no ``vazio''; aquilo por ser natural.
Confessá"-lo com esta sem"-cerimônia, porém, seria fazer clínica à moda de
Fortunato, e desmoralizar"-se. Além do mais, quem sabe lá se não estaria
ali o sonhado lance? Prolongar a doença\ldots{} Engordar a maquia\ldots{}

Inácio não enxergava em Mendanha o doente, mas uma bolada maior ou
menor, conforme a habilidade do seu jogo. A saúde do velho importava"-lhe
tanto como as estrelas do céu --- exceção feita à ``Cabeleira de
Berenice''. Como desadorasse a medicina, não vendo nela mais que um meio
rápido de enriquecer, nem sequer lhe interessava o ``caso clínico'' em
si, como a muitos. Queria dinheiro, porque o dinheiro lhe daria Paris,
com Yvonne de lambuja. Ora, o major tinha trezentas apólices\ldots{} Dependia
pois da sua artimanha malabarizar aquele fígado, aquele coração, aquelas
palavras gregas e, num prestidigitar manhoso, reduzir tudo a uns tantos
contos de réis bem sonantes.

Mandou carta à francesinha: ``Os negócios melhoraram. Estou metido em
uma empresa que se me afigura rendosa. Saindo tudo a contento, tenho
esperanças de inda este ano beijar"-te sob a luz da terna confluente dos
nossos olhares\ldots{}''.

O velho piorou com a medicação. Injeções hipodérmicas, cápsulas,
pílulas, poções, não houve terapêutica que se não experimentasse
desastrosamente.

--- É mais grave o caso do que eu supunha --- disse o doutor à mulher
--- e os escrúpulos do meu sacerdócio aconselham"-me a pedir conferência
médica. Os colegas da terra são os que a senhora sabe; entretanto,
submeto"-me a ouvi"-los.

--- Não, doutor! Mendanha não quer ouvir falar nos seus colegas; só tem
confiança no doutor Inácio Gama.

--- Nesse caso\ldots{}

Inacinho voltou para casa esfregando as mãos. Estava só em campo, com
todos os ventos favoráveis. Paris corria"-lhe ao encontro\ldots{}

Malgrado seu, na semana seguinte, inesperadamente, o raio do major
apresentou melhoras. Sarava, o patife! E a Inácio palpitou que com mais
uma quinzena daquela arribação o homem se punha de pé.

Fez os cálculos: trinta visitas, trinta injeções e tal e tal: três
contos. Uma miséria! Se morresse, já o caso mudava de figura, poderia
exigir vinte ou trinta.

Era costume dos tempos fazerem"-se os médicos herdeiros dos clientes.
Serviços pagos em caso de cura aí com centenas de mil"-réis, em caso de
morte reputavam"-se em contos. Se os interessados relutavam no pagamento,
a questão subia aos tribunais, com base no arbitramento. Os árbitros,
mestres do mesmo ofício, sustentavam o pedido por coleguismo, dizendo em
latim: ``\emph{Hodie mihi, cras tibi''}, cuja tradução médica é:
``Prepare"-se você para me fazer o mesmo, que também pretendo dar a minha
cartada''.

Inácio ponderou tudo isto. Mediu prós e contras. Consultou acórdãos. E
tão absorvido no problema andou que à noite se deixava ficar à janela
até tarde, mergulhado em cismas, sem erguer os olhos para a Berenice
estelar.

O que a sua cabeça pensou ninguém o saberá jamais. Têm as ideias para
escondê"-las a caixa craniana, o couro cabeludo, a grenha; isso por cima;
pela frente têm a mentira do olhar e a hipocrisia da boca. Assim
entrincheiradas, elas, já de si imateriais, ficam inexpugnáveis à
argúcia alheia. E vai nisso a pouca de felicidade existente neste mundo
sublunar. Fosse possível ler nos cérebros claro como se lê no papel e a
humanidade crispar"-se"-ia de horror ante si própria\ldots{}

Positivo como era Inacinho, supomos que meteu em equação o problema das
duas vidas.

Primeira hipótese:

Cura do major = 3 contos.

Três contos = Itaoca, pasmaceira etc.

Segunda hipótese:

Morte do major = 30 contos.

Trinta contos = Paris, Yvonne, ``Bois''\ldots{}

Depois desta sólida matemática, esta anavalhante filosofia: ``A morte é
um preconceito. Não há morte. Tudo é vida. Morrer é transitar de um
estado para outro. Quem morre, transforma"-se. Continua a viver
inorganicamente, transmutado em gases e sais, ou organicamente, feito
lucílias, necróforas e uma centena de outras vidinhas esvoaçantes. Que
importa para a universal harmonia das coisas esta ou aquela forma? Tudo
é vida. A vida nasce da morte. Eu preciso, eu `quero' viver a minha
vida. Há óbices no caminho? Afasto"-os\ldots{}''.

Fiquemos por aqui. Não há tempo para filosofias, porque o major Mendanha
piorou subitamente e lá agoniza. Morreu.

O atestado de óbito deu como causa mortis flegmatite complicada com
necrose elipsodal. Podia batizá"-la de embolia estourada, nó cego na
tripa, tuberculose mesentérica, estupor granuloso peristáltico, ou
qualquer outro dos cem mil modos de morrer à grega.

Morreu, e está dito tudo. Morreu, e o doutor Inacinho apresentou no
inventário uma conta de chegar: 35 contos de réis.

Os herdeiros impugnaram o pagamento. Move"-se a traquitana da Justiça.
Mói"-se o palavreado tabelionesco. Saem das estantes carunchosos trabucos
romanos. Procede"-se ao arbitramento.

Os árbitros são Fortunato e Moura, os quais disseram entre si:

--- Que grande velhaco! Mata o homem e ainda por cima quer ficar"-se
herdeiro! O tratamento, alto e malo, não vale cem mil"-réis. Que valha
duzentos. Que valha um conto ou três. Mas trinta e cinco? É ser
ladrão!\ldots{}

No laudo, entretanto, acharam relativamente módico o pedido --- sem
dizer relativo ao quê.

A Justiça engoliu aquele papel, gestou"-o com outros ingredientes da
praxe e, a cabo de prazos, partejou um monstrozinho chamado sentença, o
qual obrigava o espólio a aliviar"-se de trinta e cinco contos de réis em
proveito do médico, mais as custas da esvurmadela forense. Inacinho,
radiante, embolsou os cobres e reconciliou"-se com os dois colegas que,
afinal de contas, não eram os cretinos que supusera.

--- Colegas, o passado, passado; agora, para a vida e para a morte!

--- Pois está visto! --- disse Fortunato. --- Tolo andou você em abrir
luta com os que ajudam o negócio. O coleguismo: eis a nossa grande
força!\ldots{}

--- Tem razão, tem razão. Criançada minha, ilusões, farofas que a idade
cura\ldots{}

Que mais? Que voou a Paris? É claro. Voou e lá está sob o pálio da
grenha astral, a passear com a Yvonne no ``Bois''.

Ao pai escreveu:

``Isto é que é vida! Que cidade! Que povo! Que civilização! Vou
diariamente à Sorbonne ouvir as lições do grande Doyen e opero em três
hospitais. Voltarei não sei quando. Fico por cá durante os 35 contos, ou
mais, se o pai entender de auxiliar"-me neste aperfeiçoamento de
estudos.''

A Sorbonne é o apartamento em Montmartre onde compartilha com o apache
de Yvonne o dia da rapariga. Os três hospitais são os três cabarés mais
à mão.

Não obstante, o pai cismou naquilo cheio de orgulho, embora pesaroso:
não estar viva Joaquininha para ver em que alturas pairava Nico --- Nico
do sanhaço estripado\ldots{} Em Paris! Na Sorbonne!\ldots{} Discípulo querido do
Doyen, o grande, o imenso Doyen!\ldots{}

Mostrou a carta aos médicos reconciliados.

--- Isso de hospitais --- gemeu o invejoso Fortunato --- é uma mina. Dá
nome. Para botar nos anúncios é de primeiríssima.

--- E o Doyen? --- murmurou, baboso, o embevecido pai. --- Não há como a
gente apropinquar"-se das celebridades\ldots{}

--- É isso mesmo --- concluiu Moura, relanceando um olhar a Fortunato
num comentário mudo àquele mirífico apropinquamento. E os dois
enxugaram, a uma, os copos da cerveja comemorativa mandada abrir pelo
bem"-aventurado coronel.

\chapter{O engraçado arrependido\footnote[*]{Texto de 1917, publicado no livro \emph{Urupês}.}\,\footnote[**]{O conto ``O
  engraçado arrependido'' foi publicado na \emph{Revista do Brasil}, n.
  16, de abril de 1917, com o título de ``A gargalhada do coletor''.
  Nota da edição de 1955.}}

Francisco Teixeira de Souza Pontes, galho bastardo duns Souza Pontes de
trinta mil arrobas afazendados no Barreiro, só aos trinta e dois anos de
idade entrou a pensar seriamente na vida.

Como fosse de natural engraçado, vivera até ali à custa da veia cômica,
e com ela amanhara casa, mesa, vestuário e o mais. Sua moeda corrente
eram micagens, pilhérias, anedotas de inglês e tudo quanto bole com os
músculos faciais do animal que ri, vulgo homem, repuxando risos ou
matracolejando gargalhadas.

Sabia de cor a \emph{Enciclopédia do riso e da galhofa} de Fuão
Pechincha, o autor mais dessaborido que Deus botou no mundo; mas era tal
a arte do Pontes, que as sensaborias mais relambórias ganhavam em sua
boca um chiste raro, de fazer os ouvintes babarem de puro gozo.

Para arremedar gente ou bicho, era um gênio. A gama inteira das vozes do
cachorro, da acuação aos caititus ao uivo à lua, e o mais, rosnado ou
latido, assumia em sua boca perfetibilidade capaz de iludir aos próprios
cães --- e à lua.

Também grunhia de porco, cacarejava de galinha, coaxava de
untanha,ralhava de mulher velha, choramingava de fedelho, silenciava de
deputado governista ou perorava de patriota em sacada. Que vozeio de
bípede ou quadrúpede não copiava ele às maravilhas, quando tinha pela
frente um auditório predisposto?

Descia outras vezes à pré"-história. Como fosse de algumas luzes, quando
os ouvintes não eram pecos ele reconstituía os vozeirões paleontológicos
dos bichos extintos --- roncos de mastodontes ou berros de mamutes ao
avistarem"-se com peludos \emph{Homos} repimpados em fetos arbóreos ---
coisa muito de rir e divulgar a ciência do senhor Barros Barreto.

Na rua, se pilhava um magote de amigos parados à esquina, aproximava"-se
de mansinho e --- \emph{nhoc!} --- arremessava um bote de munheca à
barriga da perna mais a jeito. Era de ver o pinote assustado e o
``Passa!'' nervoso do incauto, e logo em seguida as risadas sem fim dos
outros, e a do Pontes, o qual gargalhava dum modo todo seu, estrepitoso
e musical --- música de Offenbach.

Pontes ria parodiando o riso normal e espontâneo da criatura humana,
única que ri além da raposa bêbeda; e estacava de golpe, sem transição,
caindo num sério de irresistível cômico.

Em todos os gestos e modos, como no andar, no ler, no comer, nas ações
mais triviais da vida, o raio do homem diferençava"-se dos demais no
sentido de amolecá"-los prodigiosamente. E chegou a ponto de que escusava
abrir a boca ou esboçar um gesto para que se não torcesse em risos a
humanidade. Bastava sua presença.

Mal o avistavam, já as caras refloriam; se fazia um gesto, espirravam
risos; se abria a boca, espigaitavam"-se uns, outros afrouxavam os coses,
terceiros desabotoavam os coletes. E se entreabria o bico, Nossa
Senhora!, eram cascalhadas, eram rinchavelhos, eram guinchos, engasgos,
fungações e asfixias tremendas.

--- É da pele, este Pontes!

--- Basta, homem, você me afoga!

E se o pândego se inocentava, com cara palerma:

--- Mas que estou fazendo? Se nem abri a boca\ldots{}

--- Quá, quá, quá --- a companhia inteira, desmandibulada, chorava no
espasmo supremo dos risos incoercíveis.

Com o correr do tempo não foi preciso mais que seu nome para deflagrar a
hilaridade. Pronunciando alguém a palavra ``Pontes'', acendia"-se logo o
estopim das fungadelas pelas quais o homem se alteia acima da
animalidade que não ri.

Assim viveu Pontes até a idade de Cristo, numa parábola risonha, a rir e
fazer rir, sem pensar em nada sério --- vida de filante que dá momos em
troca de jantares e paga continhas miúdas com pilhérias de truz.

Um negociante caloteado disse"-lhe um dia entre frouxos de riso babado:

--- Você ao menos diverte, não é como o major Carapuça que caloteia de
carranca.

Aquele recibo sem selo mortificou seu tanto ao nosso pândego; mas a
conta subia a quinze mil"-réis --- valia bem a pelotada. Entretanto, lá
ficou a lembrança dela espetada como alfinete na almofadinha do
amor"-próprio. Depois vieram outros e outros, estes fincados de leve,
aqueles até a cabeça.

Tudo cansa. Farto de tal vida, entrou o hilarião a sonhar as delícias de
ser tomado a sério, falar e ser ouvido sem repuxo de músculos faciais,
gesticular sem promover a quebra da compostura humana, atravessar uma
rua sem pressentir na peugada um coro de ``Lá vem o Pontes!'' em tom de
quem se espreme na contenção do riso ou se ajeita para uma barriga das
boas.

Reagindo, tentou Pontes a seriedade.

Desastre.

Pontes sério mudava de tecla, caía no humorismo inglês. Se antes
divertia como o Clown, passava agora a divertir como o Tony.

O estrondoso êxito do que a toda a gente se afigurou uma faceta nova da
sua veia cômica verteu mais sombras na alma do engraçado arrependido.
Era certo que não poderia traçar outro caminho na vida além daquele, ora
odioso? Palhaço, então, eternamente palhaço à força?

Mas a vida de um homem feito tem exigências sisudas, impõe gravidade e
até casmurrice dispensáveis nos anos verdes. O cargo mais modesto da
administração, uma simples vereança, requer na cara a imobilidade da
idiotia que não ri. Não se concebe vereador risonho. Falta ao dito de
Rabelais uma exclusão: o riso é próprio à espécie humana, fora o
vereador.

Com o dobar dos anos a reflexão amadureceu, o brio cristalizou"-se, e os
jantares cavados deram a saber"-lhe a azedo. A moeda pilhéria
tornou"-se"-lhe dura ao cunho; já a não fundia com a frescura antiga; já
usava dela como expediente de vida, não por folgança despreocupada, como
outrora. Comparava"-se mentalmente a um palhaço de circo, velho e
achacoso, a quem a miséria obriga a transformar reumatismo em caretas
hílares como as quer o público pagante.

Entrou a fugir dos homens e despendeu bons meses no estudo da transição
necessária ao conseguimento de um emprego honesto. Pensou no balcão, na
indústria, na feitoria duma fazenda, na montagem dum botequim --- que
tudo era preferível à paspalhice cômica de até ali.

Um dia, bem maturados os planos, resolveu mudar de vida. Foi a um
negociante amigo e sinceramente lhe expôs os propósitos regeneradores,
pedindo por fim um lugar na casa, de varredor que fosse. Mal acabou a
exposição, o galego e os que espiavam de longe à espera do desfecho
torceram"-se em estrondoso gargalhar, como sob cócegas.

--- Esta é boa! É de primeiríssima! Quá! quá! quá! Com que então\ldots{} Quá!
quá! quá! Você me arruína os fígados, homem! Se é pela continha dos
cigarros, vá embora que me dou por bem pago! Este Pontes tem cada uma\ldots{}

E a caixeirada, os fregueses, os sapos de balcão e até passantes que
pararam na calçada para ``aproveitar o espírito'' desbocaram"-se em
``quás'' de matraca até lhe doerem os diafragmas.

Atarantado e seriíssimo, Pontes tentou desfazer o engano.

--- Falo sério, e o senhor não tem o direito de rir"-se. Pelo amor de
Deus não zombe de um pobre homem que pede trabalho e não gargalhadas.

O negociante desabotoou o cós da calça.

--- Fala sério, pff! Quá! quá! quá! Olha, Pontes, você\ldots{}

Pontes largou"-o em meio da frase e se foi com a alma atenazada entre o
desespero e a cólera. Era demais. A sociedade o repelia, então?
Impunha"-lhe uma comicidade eterna?

Correu outros balcões, explicou"-se como melhor pôde, implorou. Mas por
voz unânime o caso foi julgado como uma das melhores pilhérias do
``incorrigível'' --- e muita gente o comentou com a observação do
costume:

--- Não se emenda o raio do rapaz! E olhem que já não é criança\ldots{}

Barrado no comércio, voltou"-se para a lavoura. Procurou um velho
fazendeiro que despedira o feitor e expôs"-lhe o seu caso.

Depois de ouvir"-lhe atentamente as alegações, conclusas com o pedido do
lugar de capataz, o coronel explodiu num ataque de hilaridade.

--- O Pontes capataz! Ih! Ih! Ih!

--- Mas\ldots{}

--- Deixe"-me rir, homem, que cá na roça isto é raro. Ih! Ih! Ih! É muito
boa! Eu sempre digo: graça como o Pontes, ninguém!

E berrando para dentro:

--- Maricota, venha ouvir esta do Pontes. Ih! Ih! Ih!

Nesse dia o infeliz engraçado chorou. Compreendeu que não se desfaz do
pé para mão o que levou anos a cristalizar"-se. A sua reputação de
pândego, de impagável, de monumental, de homem do chifre furado ou da
pele, estava construída com muito boa cal e rijo cimento para que assim
esboroasse de chofre.

Urgia, entretanto, mudar de tecla, e Pontes volveu as vistas para o
Estado, patrão cômodo e único possível nas circunstâncias, porque
abstrato, porque não sabe rir nem conhece de perto as células que o
compõem. Esse patrão, só ele, o tomaria a sério --- o caminho da
salvação, pois, embicava por ali.

Estudou a possibilidade da agência do correio, dos tabelionatos, das
coletorias e do resto. Bem ponderados os prós e contras, os trunfos e
naipes, fixou a escolha na coletoria federal, cujo ocupante, major
Bentes, por avelhantado e cardíaco, era de crer não durasse muito. Seu
aneurisma andava na berra pública, com rebentamento esperado para
qualquer hora.

O ás de Pontes era um parente do Rio, sujeito de posses, em via de
influenciar a política no caso da realização de certa reviravolta no
Governo. Lá correu atrás dele e tantas fez para movê"-lo à sua pretensão
que o parente o despediu com promessa formal.

--- Vai sossegado que, em a coisa arrebentando por cá e o teu coletor
rebentando por lá, ninguém mais há de rir"-se de ti. Vai, e avisa"-me da
morte do homem sem esperar que esfrie o corpo.

Pontes voltou radioso de esperança e pacientemente aguardou a sucessão
dos fatos, com um olho na política e outro no aneurisma salvador. A
crise afinal veio; caíram ministros, subiram outros e entre estes um
politicão negocista, sócio do tal parente. Meio caminho já era andado.
Restava apenas a segunda parte.

Infelizmente, a saúde do major encruara, sem sinais patentes de declínio
rápido. Seu aneurisma, na opinião dos médicos que matavam pela alopatia,
era coisa grave, de estourar ao menor esforço; mas o precavido velho não
tinha pressa de ir"-se para melhor, deixando uma vida onde os fados lhe
conchegavam tão fofo ninho, e lá engambelava a doença com um regime
ultrametódico. Se o mataria um esforço violento, sossegassem, ele não
faria tal esforço.

Ora, Pontes, mentalmente dono daquela sinecura, impacientava"-se com o
equilíbrio desequilibrador dos seus cálculos. Como desembaraçar o
caminho daquela travanca? Leu no \emph{Chernoviz} o capítulo dos
aneurismas, decorou"-o; andou em indagações de tudo quanto se dizia ou se
escreveu a respeito; chegou a entender da matéria mais que o doutor
Iodureto, médico da terra, o qual, seja dito aqui à puridade, não
entendia de coisa nenhuma desta vida.

O pomo da ciência, assim comido, induziu"-o à tentação de matar o homem,
forçando"-o a estourar. Um esforço o mataria? Pois bem, Souza Pontes o
levaria a esse esforço! ``A gargalhada é um esforço'', filosofava
satanicamente de si para si. ``A gargalhada, portanto, mata. Ora, eu sei
fazer rir\ldots{}''

Longos dias passou Pontes alheio ao mundo, em diálogo mental com a
serpente.

--- Crime? Não! Em que código fazer rir é crime? Se disso morresse o
homem, culpa era da sua má aorta.

A cabeça do maroto virou picadeiro de luta onde o ``plano'' se batia em
duelo contra todas as objeções mandadas ao encontro pela consciência.
Servia de juiz à sua ambição amarga, e Deus sabe quantas vezes tal juiz
prevaricou, levado de escandalosa parcialidade por um dos contendores.

Como era de prever, a serpente venceu, e Pontes ressurgiu para o mundo
um tanto mais magro, de olheiras cavadas, porém com um estranho brilho
de resolução vitoriosa nos olhos. Também notaria nele o nervoso dos
modos quem o observasse com argúcia --- mas a argúcia não era virtude
sobeja entre os seus conterrâneos, além de que estados da alma do Pontes
eram coisa de somenos\emph{,} porque Pontes\ldots{}

--- Ora, Pontes\ldots{}

O futuro funcionário forjicou, então, meticulosos planos de campanha. Em
primeiro era mister aproximar"-se do major, homem recolhido consigo e
pouco amigo de lérias; insinuar"-se"-lhe na intimidade; estudar suas
venetas e cachacinhas até descobrir em que zona do corpo tinha ele o
calcanhar de Aquiles.

Começou frequentando com assiduidade a coletoria, sob pretextos vários,
ora para selos, ora para informações sobre impostos, que tudo era ensejo
de um parolar manhoso, habilíssimo, calculado para combalir a rispidez
do velho.

Também ia a negócios alheios, pagar cisas, extrair guias, coisinhas;
fizera"-se muito serviçal para os amigos que traziam negócios com a
fazenda.

O major estranhou tanta assiduidade e disse"-lho, mas Pontes
escamoteou"-se à interpelação montado numa pilhéria de truz, e perseverou
num bem calculado dar tempo ao tempo que fosse desbastando as arestas
agressivas do cardíaco.

Dentro de dois meses já se habituara Bentes àquele serelepe, como lhe
chamava, o qual, em fim de contas, lhe parecia um bom moço, sincero,
amigo de servir e sobretudo inofensivo\ldots{} Daí a lá em dia de acúmulo de
serviço pedir"-lhe um obséquio, e depois outro, e terceiro, e tê"-lo
afinal como espécie de adido à repartição, foi um passo. Para certas
comissões não havia outro. Que diligência! Que finura! Que tato!
Advertindo certa vez o escrevente, o major puxou aquela diplomacia como
lembrete.

--- Grande pasmado! Aprenda com o Pontes, que tem jeito para tudo e inda
por cima tem graça.

Nesse dia convidou"-o para jantar. Grande exultação na alma de Pontes! A
fortaleza abria"-lhe as portas.

Aquele jantar foi o início duma série em que o serelepe, agora factótum
indispensável, teve campo de primeira ordem para evoluções táticas.

O major Bentes, entretanto, possuía uma invulnerabilidade: não ria,
limitava suas expansões hílares a sorrisos irônicos. Pilhéria que levava
outros comensais a erguerem"-se da mesa atabafando a boca nos guardanapos
encrespava apenas os seus lábios. E se a graça não era de superfina
agudeza, ele desmontava sem piedade o contador.

--- Isso é velho, Pontes, já num almanaque \emph{Laemmert} de 1850 me
lembra de o ter lido.

Pontes sorria com ar vencido; mas lá por dentro consolava"-se, dizendo,
dos fígados para o rim, que se não pegara daquela, doutra pegaria.

Toda a sua sagacidade enfocava no fito de descobrir o fraco do major.
Cada homem tem predileção por um certo gênero de humorismo ou chalaça.
Este morre por pilhérias fesceninas de frades bojudos. Aquele pela"-se
pelo chiste bonacheirão da chacota germânica. Aquele outro dá a vida
pela pimenta gaulesa. O brasileiro adora a chalaça onde se põe a nu a
burrice tamancuda de galegos e ilhéus.

Mas o major? Por que não ria à inglesa, nem à alemã, nem à francesa, nem
à brasileira? Qual o seu gênero?

Um trabalho sistemático de observação, com a metódica exclusão dos
gêneros já provados ineficientes, levou Pontes a descobrir a fraqueza do
rijo adversário: o major lambia as unhas por casos de ingleses e frades.
Era preciso, porém, que viessem juntos. Separados, negavam fogo.
Esquisitices do velho. Em surgindo bifes vermelhos, de capacete de
cortiça, roupa enxadrezada, sapatões formidolosos e cachimbo, juntamente
com frades redondos, namorados da pipa e da polpa feminina, lá abria o
major a boca e interrompia o serviço da mastigação, como criança a quem
acenam com cocada. E quando o lance cômico chegava, ele ria com gosto,
abertamente, embora sem exagero capaz de lhe destruir o equilíbrio
sanguíneo.

Com infinita paciência Pontes bancou nesse gênero e não mais saiu dali.
Aumentou o repertório, a gradação do sal, a dose de malícia, e
sistematicamente bombardeou a aorta do major com os produtos dessa hábil
manipulação.

Quando o caso era longo, porque o narrador o floria no intento de
esconder o desfecho e realçar o efeito, o velho interessava"-se
vivamente, e nas pausas manhosas pedia esclarecimento ou continuação.

--- E o raio do bife? E daí? Mister John apitou?

Embora tardasse a gargalhada fatal, o futuro coletor não desesperava,
confiando no apólogo da bilha que de tanto ir à fonte lá ficou. Não era
mau o cálculo. Tinha a psicologia por si --- e teve também por si a
quaresma.

Certa vez, findo o Carnaval, reuniu o major os amigos em torno a uma
enorme piabanha recheada, presente dum colega. O entrudo desmazorrara a
alma dos comensais e a do anfitrião, que estava naquele dia contente de
si e do mundo, como se houvera enxergado o passarinho verde. O cheiro
vindo da cozinha, valendo por todos os aperitivos de garrafaria, punha
nas caras um enternecimento estomacal.

Quando o peixe entrou, cintilaram os olhos do major. Pescado fino era
com ele, inda mais cozido por Gertrudes. E naquele bródio primara
Gertrudes num tempero que excedia às raias da culinária e se guindava ao
mais puro lirismo.

--- Que peixe! Vatel o assinaria com a pena da impotência molhada na
tinta da inveja --- disse o escrevente, sujeito lido em Brillat"-Savarin
e outros praxistas do paladar.

Entre goles de rica vinhaça ia a piabanha sendo introduzida nos
estômagos com religiosa unção. Ninguém se atrevia a quebrar o silêncio
da bromatológica beatitude.

Pontes pressentiu oportuno o momento do golpe. Trazia engatilhado o caso
dum inglês, sua mulher e dois frades barbadinhos, anedota que elaborara
à custa da melhor matéria cinzenta de seu cérebro, aperfeiçoando"-a em
longas noites de insônia. Já de dias a tinha de tocaia, só aguardando o
momento em que tudo concorresse para levá"-la a produzir o efeito máximo.

Era a derradeira esperança do facínora, seu último cartucho. Negasse
fogo e, estava resolvido, metia duas balas nos miolos. Reconhecia
impossível manipular"-se torpedo mais engenhoso. Se o aneurisma lhe
resiste ao embate, então é que o aneurisma era uma potoca, a aorta uma
ficção, o \emph{Chernoviz} um palavrório, a medicina uma miséria, o
doutor Iodureto uma cavalgadura e ele, Pontes, o mais chapado sensaborão
ainda aquecido pelo sol --- indigno, portanto, de viver.

Matutava assim Pontes, negaceando com os olhos da psicologia a pobre
vítima, quando o major veio ao seu encontro: piscou o olho esquerdo ---
sinal de predisposição para ouvir.

--- É agora! --- pensou o bandido. E com infinita naturalidade, pegando
como por acaso uma garrafinha de molho, pôs"-se a ler o rótulo.

\emph{--- Perrins}; \emph{Lea and Perrins}. Será parente daquele
\emph{lord} Perrins que bigodeou os dois frades barbadinhos?

Inebriado pelos amavios do peixe, o major alumiou um olho concupiscente,
guloso de chulice.

--- Dois barbadinhos e um \emph{lord}! A patifaria deve ser marca
\textsc{x.p.t.o.} Conta lá, serelepe.

E, mastigando maquinalmente, absorveu"-se no caso fatal.

A anedota correu capciosa pelos fios naturais até as proximidades do,
narrada com arte de mestre, segura e firme, num andamento estratégico em
que havia gênio. Do meio para o fim a maranha empolgou de tal forma o
pobre velho que o pôs suspenso, de boca entreaberta, uma azeitona no
garfo detida a meio caminho. Um ar de riso --- riso parado, riso
estopim, que não era senão o armar bote da gargalhada --- iluminou"-lhe o
rosto.

Pontes vacilou. Pressentiu o estouro da artéria. Por uns instantes a
consciência brecou"-lhe a língua, mas Pontes deu"-lhe um pontapé e com voz
firme puxou o gatilho.

O major Antônio Pereira da Silva Bentes desferiu a primeira gargalhada
da sua vida, franca, estrondosa, de ouvir"-se no fim da rua, gargalhada
igual à de Teufalsdröckh diante de Jean Paul Richter. Primeira e última,
entretanto, porque no meio dela os convivas, atônitos, viram"-no cair de
bordo sobre o prato, ao tempo que uma onda de sangue avermelhava a
toalha.

O assassino ergueu"-se alucinado; aproveitando a confusão, esgueirou"-se
para a rua, qual outro Caim. Escondeu"-se em casa, trancou"-se no quarto,
bateu dentes a noite inteira, suou gelado. Os menores rumores
retransiam"-no de pavor. Polícia?

Semanas depois é que entrou a declinar aquele transtorno da alma que
toda gente levara à conta de mágoa pela morte do amigo. Não obstante,
trazia sempre nos olhos a mesma visão: o coletor de bruços no prato,
golfando sangue, enquanto no ar vibravam os ecos da sua derradeira
gargalhada.

E foi nesse deplorável estado que recebeu a carta do parente do Rio.
Entre outras coisas dizia o ás: ``Como não me avisaste a tempo, conforme
o combinado, só pelas folhas vim a saber da morte de Bentes. Fui ao
ministro mas era tarde, já estava lavrada a nomeação do sucessor. A tua
leviandade fez"-te perder a melhor ocasião da vida. Guarda para teu
governo este latim: \emph{tarde venientibus ossa}, quem chega tarde só
encontra os ossos --- e sê mais esperto para o futuro''.

Um mês depois descobriram"-no pendente duma trave, com a língua de fora,
rígido. Enforcara"-se numa perna de ceroula.

Quando a notícia deu volta pela cidade, toda gente achou graça no caso.
O galego do armazém comentou para os caixeiros:

--- Vejam que criatura! Até morrendo fez chalaça. Enforcar"-se na
ceroula! Esta só mesmo de Pontes\ldots{}

E reeditaram em coro meia dúzia de ``quás'' --- único epitáfio que lhe
deu a sociedade.

\chapter{O comprador de fazendas\footnote[*]{Texto de 1917, publicado no livro \emph{Urupês}.}}

Pior fazenda que a do Espigão, nenhuma. Já arruinara três donos, o que
fazia dizer aos praguentos: ``Espiga é o que aquilo é!''.

O detentor último, um Davi Moreira de Sousa, arrematara"-a em praça,
convicto de negócio da China; mas já lá andava, também ele, escalavrado
de dívidas, coçando a cabeça, num desânimo\ldots{}

Os cafezais em vara, ano sim ano não batidos de pedra ou esturrados de
geada, nunca deram de si colheita de entupir tulha. Os pastos
ensapezados, enguaxumados, ensamambaiados nos topes, eram acampamentos
de cupins com entremeios de macegas mortiças, formigantes de carrapatos.
Boi entrado ali punha"-se logo de costelas à mostra, encaroçado de
bernes, triste e dolorido de meter dó.

As capoeiras substitutas das matas nativas revelavam pela indiscrição
das tabocas a mais safada das terras secas. Em tal solo a mandioca
bracejava a medo varetinhas nodosas; a cana"-caiana assumia aspecto de
caninha, e esta virava um taquariço magrela dos que passam incólumes
entre os cilindros moedores.

Piolhavam os cavalos. Os porcos escapos à peste encruavam na magrém
faraônica das vacas egípcias.

Por todos os cantos imperava o ferrão das saúvas, dia e noite entregues
à tosa dos capins para que em outubro se toldasse o céu de nuvens de
içás, em saracoteios amorosos com enamorados savitus.

Caminhos por fazer, cercas no chão, casas de agregados engoteiradas,
combalidas de cumeeira, prenunciando feias taperas. Até na moradia
senhorial insinuava"-se a breca, aluindo panos de reboco, carcomendo
assoalhos. Vidraças sem vidro, mobília capengante, paredes
lagarteadas\ldots{} intacto que é que havia lá?

Dentro dessa esborcinada moldura, o fazendeiro, avelhuscado por força
das sucessivas decepções e, a mais, roído pelo cancro feroz dos juros,
sem esperança e sem conserto, coçava cem vezes ao dia a coroa da cabeça
grisalha.

Sua mulher, a pobre dona Isaura, perdido o viço do outono, agrumava no
rosto quanta sarda e pé de galinha inventam os anos de mãos dadas à
trabalhosa vida.

Zico, o filho mais velho, saíra"-lhes um pulha, amigo de erguer"-se às
dez, ensebar a pastinha até as onze e consumir o resto do dia em
namoricos mal azarados.

Afora este malandro tinham Zilda, então nos dezessete, menina galante,
porém sentimental mais do que manda a razão e pede o sossego da casa.
Era um ler Escrich, a rapariga, e um cismar amores de Espanha.

Em tal situação só havia uma aberta: vender a fazenda maldita para
respirar a salvo de credores. Coisa difícil, entretanto, em quadra de
café a cinco mil"-réis, botar unhas num tolo das dimensões requeridas.
Iludidos por anúncios manhosos alguns pretendentes já haviam abicado ao
Espigão; mas franziam o nariz, indo"-se a arrenegar da pernada sem abrir
oferta.

--- De graça é caro! --- cochichavam de si para consigo.

O redemoinho capilar de Moreira, a cabo de coçadelas, sugeriu"-lhe um
engenhoso plano mistificatório: entreverar de caetés, cambarás,
unhas"-de"-vaca e outros padrões de terra boa, transplantados das
vizinhanças, a fímbria das capoeiras e uma ou outra entrada acessível
aos visitantes. Fê"-lo, o maluco, e mais: meteu em certa grota um
pau"-d'alho trazido da terra roxa, e adubou os cafeeiros margeantes ao
caminho no suficiente para encobrir a mazela do resto.

Onde um raio de sol denunciava com mais viveza um vício da terra, ali o
alucinado velho botava a peneirinha\ldots{}

Um dia recebeu carta de um agente de negócios anunciando novo
pretendente. ``Você tempere o homem'', aconselhava o pirata, ``e saiba
manobrar os padrões que este cai. Chama"-se Pedro Trancoso, é muito rico,
muito moço, muito prosa, e quer fazenda de recreio. Depende tudo de você
espigá"-lo com arte de barganhista ladino.''

Preparou"-se Moreira para a empresa. Advertiu primeiro aos agregados para
que estivessem a postos, afiadíssimos de língua. Industriados pelo
patrão, estes homens respondiam com manha consumada às perguntas dos
visitantes, de jeito a transmutar em maravilhas as ruindades locais.

Como lhes é suspeita a informação dos proprietários, costumam os
pretendentes interrogar à socapa os encontradiços. Ali, se isso
acontecia --- e acontecia sempre, porque era Moreira em pessoa o
maquinista do acaso ---, havia diálogos desta ordem:

--- Gea por aqui?

--- Coisinha, e isso mesmo só em ano brabo.

--- O feijão dá bem?

--- Nossa Senhora! Inda este ano plantei cinco quartas e malhei
cinquenta alqueires. E que feijão!

--- Berneia o gado?

--- Qual o quê! Lá um ou outro carocinho de vez em quando. Para criar,
não existe terra melhor. Nem erva nem feijão"-bravo.\footnote{Plantas
  venenosas para o gado. Nota da edição de 1946.} O patrão é porque não
tem força. Tivesse ele os meios e isto virava um fazendão.

Avisados os espoletas, debateram"-se à noite os preparativos da
hospedagem, alegres todos com o reviçar das esperanças emurchecidas.

--- Estou com palpite que desta feita a ``coisa'' vai! --- disse o filho
maroto. E declarou necessitar, à sua parte, de três contos de réis para
estabelecer"-se.

--- Estabelecer"-se com quê? --- perguntou admirado o pai.

--- Com armazém de secos e molhados em Volta Redonda\ldots{}

--- Já me estava espantando uma ideia boa nessa cabeça de vento. Para
vender fiado à gente da Tudinha, não é?

O rapaz, se não corou, calou"-se; tinha razões para isso.

Já a mulher queria casa na cidade. De há muito trazia de olho uma de
porta e janela, em certa rua humilde, casa baratinha, de arranjados.

Zilda, um piano --- e caixões e mais caixões de romances.

Dormiram felizes essa noite e no dia seguinte mandaram cedo à vila em
busca de gulodices de hospedagem --- manteiga, um queijo, biscoitos.

Na manteiga houve debate.

--- Não vale a pena! --- reguingou a mulher. --- Sempre são seis
mil"-réis.

Antes se comprasse com esse dinheiro a peça de algodãozinho que tanta
falta me faz.

--- É preciso, filha! Às vezes uma coisa de nada engambela um homem e
facilita um negócio. Manteiga é graxa --- e a graxa engraxa!

Venceu a manteiga.

Enquanto não vinham os ingredientes, meteu dona Isaura unhas à casa,
varrendo, espanando e arrumando o quarto dos hóspedes; matou o menos
magro dos frangos e uma leitoa manquitola; temperou a massa do pastel de
palmito, e estava a folheá"-la quando:

--- Ei, vem ele! --- gritou Moreira da janela, onde se postara desde
cedo, muito nervoso, a devassar a estrada por um velho binóculo; e sem
deixar o posto de observação foi transmitindo à ocupadíssima esposa os
pormenores divisados.

--- É moço\ldots{} Bem trajado\ldots{} Chapéu"-panamá\ldots{} Parece Chico Canhambora\ldots{}

Chegou, afinal, o homem. Apeou"-se. Deu cartão: Pedro Trancoso de
Carvalhais Fagundes. Bem"-apessoado. Ares de muito dinheiro. Mocetão e
bem falante mais que quantos até ali aparecidos.

Contou logo mil coisas com o desembaraço de quem no mundo está de pijama
em sua casa --- a viagem, os acidentes, um mico que vira pendurado num
galho de imbaúba.

Entrados que foram para a saleta de espera, Zico, incontinênti,
grudou"-se de ouvido ao buraco da fechadura, a cochichar para as mulheres
ocupadas na arrumação da mesa o que ia pilhando à conversa.

Súbito, esganiçou para a irmã, numa careta sugestiva:

--- É solteiro, Zilda!

A menina largou disfarçadamente os talheres e sumiu"-se.

Meia hora depois voltava trazendo o melhor vestido e no rosto duas
redondinhas rosas de carmim.

Quem a essa hora penetrasse no oratório da fazenda notaria nas vermelhas
rosas de papel de seda que enfeitavam o santo Antônio a ausência de
várias pétalas, e aos pés da imagem uma velinha acesa. Na roça, o ruge e
o casamento saem do mesmo oratório.

Trancoso dissertava sobre variados temas agrícolas.

--- O canastrão? Pff! Raça tardia, meu caro senhor, muito agreste. Eu
sou pelo Poland China. Também não é mau, não, o Large Black. Mas o
Poland! Que precocidade! Que raça!

Moreira, chucro na matéria, só conhecedor das pelhancas famintas, sem
nome nem raça, que lhe grunhiam nos pastos, abria insensivelmente a
boca.

--- Como em matéria de pecuária bovina --- continuou Trancoso --- tenho
para mim que, de Barreto a Prado, andam todos erradíssimos. Pois não!
Er"-radís"-si"-mos! Nem seleção, nem cruzamento. Quero a adoção
i"-me"-di"-a"-ta das mais finas raças inglesas, o Polled Angus, o Red
Lincoln. Não temos pastos? Façamo"-los. Plantemos alfafa. Fenemos.
Ensilemos. Assis\footnote{Assis Brasil. Nota da edição de 1946.}
confessou"-me uma vez\ldots{}

Assis! Aquele homem confessava os mais altos paredros da agricultura!
Era íntimo de todos eles --- Prado,\footnote{Antônio Prado. Nota da
  edição de 1946.} Barreto,\footnote{Luís Pereira Barreto. Nota da
  edição de 1946.} Cotrim\ldots{}\footnote{Eduardo Cotrim, homem de
  muita autoridade em assuntos de pecuária, na época. Nota da edição de
  1946.} E de ministros!

--- Eu já aleguei isso ao Bezerra\ldots{}\footnote{José Bezerra,
  ministro da Agricultura. Nota da edição de 1946.}

Nunca se honrara a fazenda com a presença de cavalheiro mais distinto,
assim bem relacionado e tão viajado. Falava da Argentina e de Chicago
como quem veio ontem de lá. Maravilhoso!

A boca de Moreira abria, abria, e acusava o grau máximo de abertura
permitida a ângulos maxilares, quando uma voz feminina anunciou o
almoço.

Apresentações.

Mereceu Zilda louvores nunca sonhados, que a puseram de coração aos
pinotes. Também os teve a galinha ensopada, o tutu com torresmos, o
pastel e até a água do pote.

--- Na cidade, senhor Moreira, uma água assim, pura, cristalina,
absolutamente potável, vale o melhor dos vinhos. Felizes os que podem
bebê"-la!

A família entreolhou"-se; nunca imaginaram possuir em casa semelhante
preciosidade, e cada um insensivelmente sorveu o seu golezinho, como se
naquele instante travassem conhecimento com o precioso néctar. Zico
chegou a estalar a língua\ldots{}

Quem não cabia em si de gozo era dona Isaura. Os elogios à sua culinária
puseram"-na rendida; por metade daquilo já se daria por bem paga da
trabalheira.

--- Aprenda, Zico --- cochichava ela ao filho ---, o que é educação
fina.

Após o café, brindado com um ``delicioso!'', convidou Moreira o hóspede
para um giro a cavalo.

--- Impossível, meu caro, não monto em seguida às refeições; dá"-me
cefalalgia.

Zilda corou. Zilda corava sempre que não entendia uma palavra.

--- À tarde sairemos, não tenho pressa. Prefiro agora um passeiozinho
pedestre pelo pomar, a bem do quilo.

Enquanto os dois homens em pausados passos para lá se dirigiam, Zilda e
Zico correram ao dicionário.

--- Não é com S --- disse o rapaz.

--- Veja com C --- alvitrou a menina.

Com algum trabalho encontraram a palavra cefalalgia.

--- ``Dor de cabeça!'' Ora! Uma coisa tão simples\ldots{}

À tarde, no giro a cavalo, Trancoso admirou e louvou tudo quanto ia
vendo, com grande espanto do fazendeiro que, pela primeira vez, ouvia
gabos às coisas suas. Os pretendentes em geral malsinam de tudo, com
olhos abertos só para defeitos; diante de uma barroca, abrem"-se em
exclamações quanto ao perigo das terras frouxas; acham más e poucas as
águas; se enxergam um boi, não despregam a vista dos bernes.

Trancoso, não. Gabava! E quando Moreira, nos trechos mistificados, com
dedo trêmulo assinalou os padrões, o moço abriu a boca.

--- Caquera? Mas isto é fantástico!\ldots{}

Em face do pau"-d'alho culminou"-lhe o assombro.

--- É maravilhoso o que vejo! Nunca supus encontrar nesta zona vestígios
de semelhante árvore! --- disse, metendo na carteira uma folha como
lembrança.

Em casa abriu"-se com a velha.

--- Pois, minha senhora, a qualidade destas terras excedeu de muito à
minha expectativa. Até pau"-d'alho! Isto é positivamente famoso!\ldots{}

Dona Isaura baixou os olhos. A cena passava"-se na varanda. Era noite.
Noite trilada de grilos, coaxada de sapos, com muitas estrelas no céu e
muita paz na terra. Refestelado numa cadeira preguiçosa, o hóspede
transfez o sopor da digestão em quebreira poética.

--- Este cri"-cri de grilos, como é encantador! Eu adoro as noites
estreladas, o bucólico viver campesino, tão sadio e feliz\ldots{}

--- Mas é muito triste!\ldots{} --- aventurou Zilda.

--- Acha? Gosta mais do canto estridente da cigarra, modulando cavatinas
em plena luz? --- disse ele, amelaçando a voz. --- É que no seu
coraçãozinho há qualquer nuvem a sombreá"-lo\ldots{}

Vendo Moreira assim atiçado o sentimentalismo, e dessa feita passível de
consequências matrimoniais, houve por bem dar uma pancada na testa e
berrar:

--- Oh, diabo! Não é que ia me esquecendo do\ldots{}

Não disse do que, nem era preciso. Saiu precipitadamente, deixando"-os
sós.

Prosseguiu o diálogo, mais mel e rosas.

--- O senhor é um poeta! --- exclamou Zilda a um regorjeio dos mais
sucados.

--- Quem o não é debaixo das estrelas do céu, ao lado duma estrela da
terra?

--- Pobre de mim! --- suspirou a menina, palpitante.

Também do peito de Trancoso subiu um suspiro.

Seus olhos alçaram"-se a uma nuvem que fazia no céu as vezes da Via
Láctea, e sua boca murmurou em solilóquio um rabo de arraia desses que
derrubam meninas.

--- O amor!\ldots{} A Via Láctea da vida!\ldots{} O aroma das rosas, a gaze da
aurora! Amar, ouvir estrelas\ldots{} Amai, pois só quem ama entende o que
elas dizem.

Era zurrapa de contrabando; não obstante, ao paladar inexperto da menina
soube a fino moscatel. Zilda sentiu subir à cabeça um vapor. Quis
retribuir. Deu busca aos ramilhetes retóricos da memória em procura da
flor mais bela. Só achou um bogari humílimo:

--- Lindo pensamento para um cartão"-postal!

Ficaram no bogari; o café com bolinhos de frigideira veio interromper o
idílio nascente.

Que noite aquela! Dir"-se"-ia que o anjo da bonança distendera suas asas
de ouro por sobre a casa triste. Via Zilda realizar"-se todo o Escrich
deglutido. Dona Isaura gozava"-se da possibilidade de casá"-la rica.
Moreira sonhava quitações de dívidas, com sobras fartas a tilintar"-lhe
no bolso. E imaginariamente transfeito em comerciante, Zico fiou, a
noite inteira, em sonhos, à gente da Tudinha, que, cativa de tanta
gentileza, lhe concedia afinal a ambicionada mão da pequena.

Só Trancoso dormiu o sono das pedras, sem sonhos nem pesadelos. Que bom
é ser rico!

No dia imediato visitou o resto da fazenda, cafezais e pastos, examinou
criação e benfeitorias; e como o gentil mancebo continuasse no enlevo,
Moreira, deliberado na véspera a pedir quarenta contos pela Espiga,
julgou de bom aviso elevar o preço. Após a cena do pau"-d'alho,
suspendeu"-o mentalmente para quarenta e cinco; findo o exame do gado, já
estava em sessenta. E quando foi abordada a magna questão, o velho
declarou corajosamente, na voz firme de um \emph{alea jacta}:

--- Sessenta e cinco! --- e esperou de pé atrás a ventania.

Trancoso, porém, achou razoável o preço.

--- Pois não é caro --- disse ---, está um preço bem mais razoável do
que imaginei.

O velho mordeu os lábios e tentou emendar a mão.

--- Sessenta e cinco, sim, mas\ldots{} o gado fora!\ldots{}

--- É justo --- respondeu Trancoso.

--- \ldots{} e fora também os porcos!\ldots{}

--- Perfeitamente.

--- \ldots{} e a mobília!

--- É natural.

O fazendeiro engasgou; não tinha mais o que excluir e confessou de si
para consigo que era uma cavalgadura. Por que não pedira logo oitenta?

Informada do caso, a mulher chamou"-lhe \emph{pax"-vobis}.

--- Mas, criatura, por quarenta já era um negocião! --- justificou"-se o
velho.

--- Por oitenta seria o dobro melhor. Não se defenda. Eu nunca vi
Moreira que não fosse palerma e sarambé. É do sangue. Você não tem
culpa.

Amuaram um bocado; mas a ânsia de arquitetar castelos com a imprevista
dinheirama varreu para longe a nuvem. Zico aproveitou a aura para
insistir nos três contos do estabelecimento --- e obteve"-os. Dona Isaura
desistiu da tal casinha. Lembrava"-se agora de outra maior, em rua de
procissão --- a casa de Eusébio Leite.

--- Mas essa é de doze contos --- advertiu o marido.

--- Mas é outra coisa que não aquele casebre! Muito mais bem repartida.
Só não gosto da alcova pegada à copa; escura\ldots{}

--- Abre"-se uma claraboia.

--- Também o quintal precisa de reforma; em vez do cercado das
galinhas\ldots{}

Até noite alta, enquanto não vinha o sono, foram remendando a casa,
pintando"-a, transformando"-a na mais deliciosa vivenda da cidade. Estava
o casal nos últimos retoques, dorme não dorme, quando Zico bateu à
porta.

--- Três contos não bastam, papai; são precisos cinco. Há a armação, de
que não me lembrei, e os direitos, e o aluguel da casa, e mais
coisinhas\ldots{}

Entre dois bocejos o pai concedeu"-lhe generosamente seis.

E Zilda? Essa vogava em alto"-mar dum romance de fadas. Deixemo"-la vogar.

Chegou enfim o momento da partida. Trancoso despediu"-se. Sentia muito
não poder prolongar a deliciosa visita, mas interesses de monta o
chamavam. A vida do capitalista não é livre como parece\ldots{} Quanto ao
negócio, considerava"-o quase feito; daria a palavra definitiva dentro de
semana.

Partiu Trancoso, levando um pacote de ovos --- gostara muito da raça de
galinhas criada ali; e um saquito de carás --- petisco de que era mui
guloso. Levou ainda uma bonita lembrança, o rosilho de Moreira, o melhor
cavalo da fazenda. Tanto gabara o animal durante os passeios, que o
fazendeiro se viu na obrigação de recusar uma barganha proposta e
dar"-lho de presente.

--- Vejam vocês! --- disse Moreira, resumindo a opinião geral. --- Moço,
riquíssimo, direitão, instruído como um doutor e no entanto amável,
gentil, incapaz de torcer o focinho como os pulhas que cá têm vindo. O
que é ser gente!

À velha agradara sobretudo a sem"-cerimônia do jovem capitalista. Levar
ovos e carás! Que mimo!

Todos concordaram, louvando"-o cada um a seu modo. E assim, mesmo
ausente, o gentil ricaço encheu a casa durante a semana inteira.

Mas a semana transcorreu sem que viesse a ambicionada resposta. E mais
outra. E outra ainda.

Escreveu"-lhe Moreira, já apreensivo e nada. Lembrou"-se dum parente
morador na mesma cidade e endereçou"-lhe carta pedindo que obtivesse do
capitalista a solução definitiva. Quanto ao preço, abatia alguma coisa.
Dava a fazenda por cinquenta e cinco, por cinquenta e até por quarenta,
com criação e mobília.

O amigo respondeu sem demora. Ao rasgar do envelope, os quatro corações
da Espiga pulsaram violentamente: aquele papel encerrava o destino de
todos quatro.

\textls[-10]{Dizia a carta: ``Moreira. Ou muito me engano ou estás iludido. Não há
por aqui nenhum Trancoso Carvalhais capitalista. Há o Trancosinho, filho
de nhá Veva, vulgo Sacatrapo. É um espertalhão que vive de barganhas e
sabe iludir aos que o não conhecem. Ultimamente tem corrido o estado de
Minas, de fazenda em fazenda, sob vários pretextos. Finge"-se às vezes
comprador, passa uma semana em casa do fazendeiro, a caceteá"-lo com
passeios pelas roças e exames de divisas; come e bebe do bom, namora as
criadas, ou a filha, ou o que encontra --- é um vassoura de marca! ---,
e no melhor da festa some"-se. Tem feito isto um cento de vezes, mudando
sempre de zona. Gosta de variar de tempero, o patife. Como aqui Trancoso
só há este, deixo de apresentar ao pulha a tua proposta. Ora Sacatrapo a
comprar fazenda! Tinha graça\ldots{}''.}

O velho caiu numa cadeira, aparvalhado, com a missiva sobre os joelhos.
Depois o sangue lhe avermelhou as faces e seus olhos chisparam.

--- Cachorro!

As quatro esperanças da casa ruíram com fragor, entre lágrimas da
menina, raiva da velha e cólera dos homens.

Zico propôs"-se a partir incontinênti na peugada do biltre, a fim de
quebrar"-lhe a cara.

--- Deixe, menino! O mundo dá voltas. Um dia cruzo"-me com o ladrão e
justo contas.

Pobres castelos! Nada há mais triste que estes repentinos
desmoronamentos de ilusões. Os formosos palácios da Espanha, erigidos
durante um mês à custa da mirífica dinheirama, fizeram"-se taperas
sombrias. Dona Isaura chorou até os bolinhos, a manteiga e os frangos.

Quanto a Zilda, o desastre operou como pé de vento através de paineira
florida. Caiu de cama, febricitante. Encovaram"-se"-lhe as faces. Todas as
passagens trágicas dos romances lidos desfilaram"-lhe na memória;
reviu"-se na vítima de todos eles. E dias a fio pensou no suicídio.

\textls[-10]{Por fim habituou"-se a essa ideia e continuou a viver. Teve azo de
verificar que isso de morrer de amores, só em Escrich.}

Acaba"-se aqui a história --- para a plateia; para as torrinhas segue
ainda por meio palmo. As plateias costumam impar umas tantas finuras de
bom gosto e tom muito de rir; entram no teatro depois de começada a peça
e saem mal as ameaça o epílogo.

Já as galerias querem a coisa pelo comprido, a jeito de aproveitar o
rico dinheirinho até ao derradeiro vintém. Nos romances e contos pedem
esmiuçamento completo do enredo; e se o autor, levado por fórmulas de
escola, lhes arruma para cima, no melhor da festa, com a caudinha
reticenciada a que chama ``nota impressionista'', franzem o nariz.
Querem saber --- e fazem muito bem --- se Fulano morreu, se a menina
casou e foi feliz, se o homem afinal vendeu a fazenda, a quem e por
quanto.

Sã, humana e respeitabilíssima curiosidade!

--- Vendeu a fazenda o pobre Moreira?

Pesa"-me confessá"-lo: não! E não a vendeu por artes do mais inconcebível
quiproquó de quantos tem armado neste mundo o diabo --- sim, porque
afora o diabo, quem é capaz de intrincar os fios da meada com laços e
nós cegos, justamente quando vai a feliz remate o crochê?

O acaso deu a Trancoso uma sorte de cinquenta contos na loteria. Não se
riam. Por que motivo não havia Trancoso de ser o escolhido, se a sorte é
cega e ele tinha no bolso um bilhete? Ganhou os cinquenta contos,
dinheiro que para um pé"-atrás daquela marca era significativo de grande
riqueza.

De posse do bolo, após semanas de tonteira deliberou afazendar"-se.
Queria tapar a boca ao mundo realizando uma coisa jamais passada pela
sua cabeça: comprar fazenda. Correu em revista quantas visitara durante
os anos de malandragem, propendendo, afinal, para a Espiga. Ia nisso,
sobretudo, a lembrança da menina, dos bolinhos da velha e a ideia de
meter na administração ao sogro, de jeito a folgar"-se uma vida vadia de
regalos, embalada pelo amor de Zilda e os requintes culinários da sogra.
Escreveu, pois, a Moreira anunciando"-lhe a volta, a fim de fechar"-se o
negócio.

Ai, ai, ai! Quando tal carta penetrou na Espiga houve rugidos de cólera,
entremeio a bufos de vingança.

--- É agora! --- berrou o velho. --- O ladrão gostou da pândega e quer
repetir a dose. Mas desta feita curo"-lhe a balda, ora se curo! ---
concluiu, esfregando as mãos no antegozo da vingança.

No murcho coração da pálida Zilda, entretanto, bateu um raio de
esperança. A noite de sua alma alvorejou ao luar de um ``Quem sabe?''.
Não se atreveu, todavia, a arrostar a cólera do pai e do irmão,
concertados ambos num tremendo ajuste de contas. Confiou no milagre.
Acendeu outra velinha a santo Antônio\ldots{}

O grande dia chegou. Trancoso rompeu à tarde pela fazenda, caracolando o
rosilho.

Desceu Moreira a esperá"-lo embaixo da escada, de mãos às costas.

Antes de sofrear as rédeas, já o amável pretendente abria"-se em
exclamações.

--- Ora viva, caro Moreira! Chegou enfim o grande dia. Desta vez
compro"-lhe a fazenda.

Moreira tremia. Esperou que o biltre apeasse e mal Trancoso, lançando as
rédeas, dirigiu"-se"-lhe de braços abertos, todo risos, o velho saca de
sob o paletó um rabo de tatu e rompe"-lhe para cima com ímpeto de
queixada.

--- Queres fazenda, grandessíssimo tranca? Toma, toma fazenda, ladrão!
--- e \emph{lepte}, \emph{lepte}, finca"-lhe rijas rabadas coléricas.

O pobre rapaz, tonteando pelo imprevisto da agressão, corre ao cavalo e
monta às cegas, de passo que Zico lhe sacode no lombo nova série de
lambadas de agravadíssimo ex"-quase cunhado.

Dona Isaura atiça"-lhe os cães:

--- Pega, Brinquinho! Ferra, Joli!

O mal azarado comprador de fazendas, acuado como raposa em terreiro, dá
de esporas e foge a toda, sob uma chuva de insultos e pedras. Ao cruzar
a porteira inda teve ouvidos para distinguir na grita os desaforos
esganiçados da velha:

--- Comedor de bolinhos! Papa"-manteiga! Toma! Em outra não hás de cair,
ladrão de ovo e cará!\ldots{}

E Zilda?

\textls[-5]{Atrás da vidraça, com os olhos pisados do muito chorar, a triste menina
viu desaparecer para sempre, envolto em uma nuvem de pó, o cavaleiro
gentil dos seus dourados sonhos.}

Moreira, o caipora, perdia assim naquele dia o único negócio bom que
durante a vida inteira lhe deparara a Fortuna: o duplo descarte --- da
filha e da Espiga\ldots{}

\chapter{O drama da geada\footnote[*]{Texto de 1919, publicado no livro \emph{Negrinha}.}}

Junho. Manhã de neblina. Vegetação entanguida de frio. Em todas as
folhas o recamo de diamantes com que as adereça o orvalho.

Passam colonos para a roça, retransidos, deitando fumaça pela boca.

Frio. Frio de geada, desses que matam passarinhos e nos põem sorvete
dentro dos ossos.

Saímos cedo a ver cafezais, e ali paramos, no viso do espigão, ponto
mais alto da fazenda. Dobrando o joelho sobre a cabeça do socado, o
major voltou o corpo para o mar de café aberto ante nossos olhos e disse
num gesto amplo:

--- Tudo obra minha, veja!

Vi. Vi e compreendi"-lhe o orgulho, sentindo"-me orgulhoso também de tal
patrício. Aquele desbravador de sertões era uma força criadora, dessas
que enobrecem a raça humana.

--- Quando adquiri esta gleba --- disse ele ---, tudo era mata virgem,
de ponta a ponta. Rocei, derrubei, queimei, abri caminhos, rasguei
valos, estiquei arame, construí pontes, ergui casas, arrumei pastos,
plantei café --- fiz tudo. Trabalhei como negro cativo durante quatro
anos. Mas venci. A fazenda está formada, veja.

Vi. Vi o mar de café ondulando pelos seios da terra, disciplinado em
fileiras de absoluta regularidade. Nem uma falha! Era um exército em pé
de guerra. Mas bisonho ainda. Só no ano vindouro entraria em campanha.
Até ali, os primeiros frutos não passavam de escaramuças de colheita. E
o major, chefe supremo do verde exército por ele criado, disciplinado,
preparado para a batalha decisiva da primeira safra grande, a que
liberta o fazendeiro dos ônus da formação, tinha o olhar orgulhoso dum
pai diante de filhos que não mentem à estirpe.

O fazendeiro paulista é alguma coisa séria no mundo. Cada fazenda é uma
vitória sobre a fereza retrátil dos elementos brutos, coligados na
defesa da virgindade agredida. Seu esforço de gigante paciente nunca foi
cantado pelos poetas, mas muita epopeia há por aí que não vale a destes
heróis do trabalho silencioso. Tirar uma fazenda do nada é façanha
formidável. Alterar a ordem da natureza, vencê"-la, impor"-lhe uma
vontade, canalizar"-lhe as forças de acordo com um plano preestabelecido,
dominar a réplica eterna do mato daninho, disciplinar os homens da lida,
quebrar a força das pragas\ldots{} --- batalha sem tréguas, sem fim, sem
momento de repouso e, o que é pior, sem certeza plena da vitória.
Colhe"-a muitas vezes o credor, um onzeneiro que adiantou um capital
caríssimo e ficou a seu salvo na cidade, de cócoras num título de
hipoteca, espiando o momento oportuno para cair sobre a presa como um
gavião.

--- Realmente, major, isto é de enfunar o peito! É diante de espetáculos
destes que vejo a mesquinharia dos que lá fora, comodamente, parasitam o
trabalho do agricultor.

--- Diz bem. Fiz tudo, mas o lucro maior não é meu. Tenho um sócio voraz
que me lambe, ele só, um quarto da produção: o Governo. Sangram"-na
depois as estradas de ferro --- mas destas não me queixo porque dão
muita coisa em troca. Já não digo o mesmo dos tubarões do comércio, esse
cardume de intermediários que começa ali em Santos, no zangão, e vai
numa cadeia até o torrador americano. Mas não importa! O café dá para
todos, até para a besta do produtor\ldots{} --- concluiu, pilheriando.

Tocamos os animais a passo, com os olhos sempre presos ao cafezal
intérmino. Sem um defeito de formação, as paralelas de verdura ondeavam,
acompanhando o relevo do solo, até se confundirem ao longe em massa
uniforme. Verdadeira obra de arte em que, sobrepondo"-se à natureza, o
homem lhe impunha o ritmo da simetria.

--- No entanto --- continuou o major ---, a batalha ainda não está
ganha. Contraí dívidas; a fazenda está hipotecada a judeu"-franceses. Não
venham colheitas fartas e serei mais um vencido pela fatalidade das
coisas. A natureza depois de subjugada é mãe; mas o credor é sempre
carrasco\ldots{}

A espaços, perdidas na onda verde, perobeiras sobreviventes erguiam
fustes contorcidos, como galvanizadas pelo fogo numa convulsão de dor.
Pobres árvores! Que destino triste, verem"-se um dia arrancadas à vida em
comum e insuladas na verdura rastejante do café, como rainhas
prisioneiras à cola de um carro de triunfo! Órfãs da mata nativa, como
não hão de chorar o conchego de outrora? Vede"-as. Não têm o desgarre, o
frondoso de copa das que nascem em campo aberto. Seu engalhamento, feito
para a vida apertada da floresta, parece agora grotesco; sua altura
desmesurada, em desproporção com a fronde, provoca o riso. São mulheres
despidas em público, hirtas de vergonha, não sabendo que parte do corpo
esconder. O excesso de ar as atordoa, o excesso de luz as martiriza ---
afeitas que estavam ao espaço confinado e à penumbra sonolenta do
\emph{habitat}.

Fazendeiros desalmados --- não deixeis nunca árvores pelo cafezal\ldots{}
Cortai"-as todas, que nada mais pungente do que forçar uma árvore a ser
grotesca.

--- Aquela perobeira ali --- disse o major --- ficou para assinalar o
ponto de partida deste talhão. Chama"-se a peroba do Ludgero, um baiano
valente que morreu ao pé dela estrepado numa jiçara\ldots{}

Tive a visão do livro aberto que seriam para o fazendeiro aquelas
paragens.

--- Como tudo aqui lhe há de falar à memória, major!

--- É isso mesmo. Tudo me fala à recordação. Cada toco de pau, cada
pedreira, cada volta de caminho tem uma história que sei, trágica às
vezes, como essa da peroba, às vezes cômica --- pitoresca sempre.
Ali\ldots{} --- está vendo aquele toco de jerivá? Foi por uma tempestade
de fevereiro. Eu abrigara"-me num rancho coberto de sapé, e lá em
silêncio esperávamos, eu e a turma, o fim do dilúvio, quando estalou um
raio quase em cima das nossas cabeças.

```Fim do mundo, patrão!' --- lembro"-me que disse, numa careta de pavor,
o defunto Zé Coivara\ldots{} E parecia!\ldots{} Mas foi apenas o fim de um velho
coqueiro, do qual resta hoje --- \emph{sic transit}\ldots{} --- esse pobre
toco\ldots{} Cessada a chuva, encontramo"-lo desfeito em ripas.''

Mais adiante abria"-se a terra em boçoroca vermelha, esbarrondada em
coleios até morrer no córrego. O major apontou"-a, dizendo:

--- Cenário do primeiro crime cometido na fazenda. Rabo de saia, já se
sabe. Nas cidades e na roça, pinga e saia são o móvel de todos os
crimes. Esfaquearam"-se aqui dois cearenses. Um acabou no lugar; outro
cumpre pena na correição. E a saia, muito contente da vida, mora com o
\emph{tertius}. A história de sempre.

E assim, de evocação em evocação, às sugestões que pelo caminho iam
surgindo, chegamos à casa de moradia, onde nos esperava o almoço.
Almoçamos, e não sei se por boa disposição criada pelo passeio matutino
ou por mérito excepcional da cozinheira, o almoço desse dia ficou"-me na
memória gravado para sempre. Não sou poeta, mas se Apolo algum dia me
der na cabeça o estalo do padre Vieira, juro que antes de cantar Lauras
e Natércias hei de fazer uma beleza de ode à linguiça com angu de fubá
vermelho desse almoço sem par, única saudade gustativa com que descerei
ao túmulo\ldots{}

Em seguida, enquanto o major atendia à correspondência, saí a espairecer
pelo terreiro, onde me pus de conversa com o administrador. Soube por
ele da hipoteca que pesava sobre a fazenda e da possibilidade de outro,
não o major, vir a colher o fruto do penoso trabalho.

--- Mas isso --- esclareceu o homem --- só no caso de muito azar ---
chuva de pedra ou geada, daquelas que não vêm mais.

--- Que não vêm mais, por quê?

--- Porque a última geada grande foi em 1895. Daí para cá as coisas
endireitaram. O mundo, com a idade, muda, como a gente. As geadas, por
exemplo, vão"-se acabando. Antigamente ninguém plantava café onde o
plantamos hoje. Era só de meio morro acima. Agora não. Viu aquele
cafezal do meio? Terra bem baixa; no entanto, se bate geada ali é sempre
coisinha --- um tostado leve. De modo que o patrão, com uma ou duas
colheitas, paga a dívida e fica o fazendeiro mais ``prepotente'' do
município.

--- Assim seja, que grandemente o merece --- rematei.

Deixei"-o. Dei umas voltas, fui ao pomar, estive no chiqueiro vendo
brincar os leitõezinhos e depois subi. Estava um preto dando nas
venezianas da casa a última demão de tinta. Por que será que as pintam
sempre de verde? Incapaz por mim de solver o problema, interpelei o
preto, que não se embaraçou e respondeu sorrindo:

--- Pois veneziana é verde como o céu é azul. É da natureza dela\ldots{}

Aceitei a teoria e entrei.

À mesa a conversa girou em torno da geada.

--- É o mês perigoso este --- disse o major. --- O mês da aflição. Por
maior firmeza que tenha um homem, treme nesta época. A geada é um eterno
pesadelo. Felizmente a geada não é mais o que era dantes. Já nos permite
aproveitar muita terra baixa em que os antigos nem por sombras plantavam
um só pé de café. Mas, apesar disso, um que facilitou, como eu, está
sempre com a pulga atrás da orelha. Virá? Não virá? Deus sabe!\ldots{}

Seu olhar mergulhou pela janela, numa sondagem profunda ao céu límpido.

--- Hoje, por exemplo, está com jeito. Este frio fino, este ar parado\ldots{}

Ficou a cismar uns momentos. Depois, espantando a nuvem, murmurou:

--- Não vale a pena pensar nisto. O que tem de ser lá está gravado no
livro do destino.

--- Livra"-te dos ares!\ldots{} --- objetei.

--- Cristo não entendia de lavoura --- replicou o fazendeiro sorrindo.

E a geada veio! Não geadinha mansa de todas os anos, mas calamitosa,
geada cíclica, trazida em ondas do Sul.

O sol da tarde, mortiço, dera uma luz sem luminosidade e raios sem calor
nenhum. Sol boreal, tiritante. E a noite caíra sem preâmbulos.

Deitei"-me cedo, batendo o queixo, e na cama, apesar de enleado em dois
cobertores, permaneci entanguido uma boa hora antes que ferrasse no
sono. Acordou"-me o sino da fazenda, pela madrugada. Sentindo"-me
enregelado, com os pés a doerem, ergui"-me para um exercício violento.
Fui para o terreiro.

O relento estava de cortar as carnes --- mas que maravilhoso espetáculo!
Brancuras por toda parte. Chão, árvores, gramados e pastos eram, de
ponta a ponta, um só atoalhado branco. As árvores imóveis, inteiriçadas
de frio, pareciam emersas dum banho de cal. Rebrilhos de gelo pelo chão.
Águas envidradas. As roupas dos varais, tesas, como endurecidas em goma
forte. As palhas do terreiro, os sabugos de ao pé do cocho, a telha dos
muros, o topo dos moirões, a vara das cercas, o rebordo das tábuas ---
tudo polvilhado de brancuras, lactescente, como chovido por um saco de
farinha. Maravilhoso quadro! Invariável que é a nossa paisagem, sempre
nos mansos tons do ano inteiro, encantava sobremodo vê"-la de súbito
mudar, vestir"-se dum esplendoroso véu de noiva --- noiva da morte,
ai!\ldots{}

Por algum tempo caminhei a esmo, arrastado pelo esplendor da cena. O
maravilhoso quadro de sonho breve morreria, apagado pela esponja de ouro
do sol. Já pelos topes e faces de batedeira andavam"-lhe os raios na
faina de restaurar a verdura. Abriam manchas no branco da geada,
dilatavam"-nas, entremostrando nesgas do verde submerso.

Só nas baixadas, encostas noruegas ou sítios sombreados pelas árvores, é
que a brancura persistia ainda, contrastando sua nítida frialdade com os
tons quentes ressurretos. Vencera a vida, guiada pelo sol. Mas a
intervenção do fogoso Febo, apressada demais, transformara em desastre
horroroso a nevada daquele ano --- a maior de quantas deixaram marca nas
embaubeiras de São Paulo.

A ressurreição do verde fora aparente. Estava morta a vegetação. Dias
depois, por toda parte, a vestimenta do solo seria um burel imenso, com
a sépia a mostrar a gama inteira dos seus tons ressecos. Pontilhá"-lo"-ia
apenas, cá e lá, o verde"-negro das laranjeiras e o esmeraldino
sem"-vergonha da vassourinha.

Quando regressei, sol já alto, estava a casa retransida no pavor das
grandes catástrofes. Só então me acudiu que o belo espetáculo que eu até
ali só encarara pelo prisma estético tinha um reverso trágico: a ruína
do heroico fazendeiro. E procurei"-o ansioso.

Tinha sumido. Passara a noite em claro, disse"-me a mulher; de manhã, mal
clareara, fora para a janela e lá permanecera imóvel, observando o céu
através dos vidros. Depois saíra sem ao menos pedir o café, como de
costume. Andava a examinar a lavoura, provavelmente.

Devia ser isso. Mas como tardasse a voltar --- onze horas e nada ---, a
família entrou"-se de apreensões.

Meio"-dia. Uma hora, duas, três e nada.

O administrador, que a mandado da mulher saíra a procurá"-lo, voltou à
tarde sem notícias.

--- Bati tudo e nem rasto. Estou com medo de alguma coisa\ldots{} Vou
espalhar gente por aí, à cata.

Dona Ana, inquieta, de mãos enclavinhadas, só dizia uma coisa:

--- Que será de nós, santo Deus! Quincas é capaz duma loucura\ldots{}

Pus"-me em campo também, em companhia do capataz. Corremos todos os
caminhos, varejamos grotas em todas as direções --- inutilmente.

Caiu a tarde. Caiu a noite --- a noite mais lúgubre de minha vida ---
noite de desgraça e aflição.

Não dormi. Impossível conciliar o sono naquele ambiente de dor, sacudido
de choro e soluços.

Certa hora os cães latiram no terreiro, mas silenciaram logo.

Rompeu a manhã, glacial como a da véspera. Tudo apareceu geado
novamente.

Veio o sol. Repetiu"-se a mutação da cena. Esvaiu"-se a alvura, e o verde
morto da vegetação envolveu a paisagem num sudário de desalento.

Em casa repetiu"-se o corre"-corre do dia anterior, o mesmo vaivém, o
mesmo ``quem sabe?'', as mesmas pesquisas inúteis.

À tarde, porém --- três horas ---, um camarada apareceu esbaforido,
gritando de longe, no terreiro:

--- Encontrei! Está perto da boçoroca!\ldots{}

--- Vivo? --- perguntou o capataz.

--- Vivo, sim, mas\ldots{}

Dona Ana surgira à porta e, ao ouvir a boa"-nova, exclamou, chorando e
sorrindo:

--- Bendito sejas, meu Deus!\ldots{}

Minutos depois partimos todos de rumo à boçoroca e a cem passos dela
avistamos um vulto às voltas com os cafeeiros requeimados.
Aproximamo"-nos. Era o major. Mas em que estado! Roupa em tiras, cabelos
sujos de terra, olhos vítreos e desvairados. Tinha nas mãos uma lata de
tinta e uma brocha --- brocha do pintor que andava a olear as
venezianas. Compreendi o latido dos cães à noite\ldots{}

O major não se deu conta da nossa chegada. Não interrompeu o serviço:
\emph{continuou a pintar, uma a uma, do risonho verde esmeraldino das
venezianas, as folhas requeimadas do cafezal morto\ldots{}}

Dona Ana, estarrecida, entreparou atônita. Depois, compreendendo a
tragédia, rompeu em choro convulso.

\chapter{Toque outra\footnote[*]{Texto sem data, publicado no livro \emph{Cidades mortas}.}}

--- Ora toque, Sinhazinha, toque!

--- Mas eu não sei\ldots{}

--- Não faz mal, toque assim mesmo, não se faça de rogada. Aquela
valsinha\ldots{}

A pálida menina geme novos luxinhos faceiros, torce os pingentes da
almofada e por fim levanta"-se, toda dengues, a desculpar"-se.

--- Vou errar tudo, não tenho estudado há muitos dias, estou
esquecida\ldots{}

--- Não faz mal, toque!\ldots{}

Sinhazinha senta"-se ao piano, folheia a maçaroca de músicas e
preguiçosamente abre diante de si uma valsa de Aurélio Cavalcanti.

E toca: \emph{blem}, \emph{blem}, \emph{belelém}\ldots{}

A sala então, que só por aquilo esperava, afunda na conversa. O barulho
do piano, abafando o tom geral da palestra, dá azo à delícia dos duos,
em que cada um pega de cochicho com quem mais o atende. As matronas,
donas de casa, caem no assunto dileto --- os criados!

--- Ai, os criados! Que gente, prima! Que pestes! Não fazem ``isto'' sem
uma pessoa estar em cima; se vão a compras, roubam no troco\ldots{} E não se
lhes diga uma palavrinha! Pedem a conta e dizem desaforos, os
demônios\ldots{}

As meninas rodeiam o moço, que impa como um galo e desdobra o farnel da
banalidade tão cara às mulheres; todas ouvem"-no atentas, bebem"-lhe os
ditos, riem das suas pilhérias, acham"-no ``levado''.

Titinha diz, sorvendo"-o com os olhos:

--- Este seu Raul é mesmo da pele!

Num desvão da janela cochicha"-se um namoro; a das Dores conta à do Carmo
que não gosta mais do Luisinho por umas certas coisas que viu no último
baile. Do Carmo comenta, sentenciosa:

--- Os homens! Os homens!\ldots{}

Duas em outro canto riem perdidamente, em casquinadas argentinas.

Nisto Sinhazinha acaba a valsa. A sala dá pela coisa, interrompe a
tagarelice e pede mais:

--- Muito bem, Sinhazinha, muito bem! Toque outra!\ldots{}

Sinhazinha ataca uma \emph{schottisch}.

A sala retoma os temas interrompidos.

--- Mas\ldots{} como eu ia contando\ldots{}

Impossível negar as vantagens sociais da música.

\chapter{Os pequeninos\footnote[*]{Texto de 1939, publicado no livro \emph{Negrinha}.}}

Ouvi certa vez uma conversa inesquecível. A esponja de doze anos não a
esmaeceu em coisa nenhuma. Por que motivo certas impressões se gravam de
tal maneira e outras se apagam tão profundamente?

Eu estava no cais, à espera do \emph{Arlanza}, que me ia devolver de
Londres um velho amigo já de longa ausência. O nevoeiro atrasara o
navio.

--- Só vai atracar às dez horas --- informou"-me um sabe"-tudo de boné.

Bem. Tinha eu de matar uma hora de espera dentro dum nevoeiro
absolutamente fora do comum, dos que negam aos olhos o consolo da
paisagem distante. A visão morria a dez passos; para além, todas as
formas desapareciam no algodoamento da névoa. Pensei nos \emph{fogs}
londrinos que o meu amigo devia trazer na alma e comecei a andar por ali
à toa, entregue a esse trabalho, tão frequente na vida, de ``matar o
tempo''. Minha técnica em tais circunstâncias se resume em recordar
passagens da vida. Recordar é reviver. Reviver os bons momentos tem as
delícias do sonho.

Mas o movimento do cais interrompia amiúde o meu sonho, forçando"-me a
cortar e a reatar de novo o fio das recordações. Tão cheio de nós foi
ele ficando que o abandonei. Uma das interrupções me pareceu mais
interessante que a evocação do passado, porque a vida exterior é mais
viva que a interior --- e a conversa dos três carregadores era
inegavelmente ``água"-forte''.

Três portugueses bem típicos, já maduros; um deles de rosto
singularmente amarrotado pelos anos. Um incidente qualquer ali do cais
dera origem à conversa.

--- Pois esse caso, meu velho --- dizia um deles ---, me lembra a
história da ema que tive num cercado. Também ela foi vítima dum
animalzinho muitíssimo menor, e que seria esmagado, como esmagamos
moscas, se lhe ficasse ao alcance do bico --- mas não ficava\ldots{}

Esse começo assanhou a curiosidade dos companheiros.

--- Como foi? --- perguntaram.

--- Eu nesse tempo estava de cima, dono de terras, com casa minha, meus
animais de cocheira, família. Foi um ano antes daquela rodada que me
levou tudo\ldots{} Peste de mundo! Tão bem que eu ia indo e afundei, perdi
tudo, tive de rolar morro abaixo até bater com o lombo neste cais,
entregue ao mais baixo dos serviços, que é o de carregador\ldots{}

--- Mas como foi o caso da ema?

Os ouvintes não queriam filosofias; ansiavam por pitoresco --- e o homem
por fim contou, depois de sacar o cachimbo, enchê"-lo, acendê"-lo. Devia
ser história das que exigem pontuação a baforadas.

--- Eu morava em minhas terras, lá onde vocês sabem --- na Vacaria, zona
de campos e mais campos, aquela planura sem"-fim. E há lá muita ema.
Conhecem? É a avestruz do Brasil, menor que a avestruz africana, mas
mesmo assim um avejão dos mais alentados. Que força tem! Domar uma ema
corresponde a domar um potro. Exige o mesmo muque. Mas são aves de boa
índole. Domesticam"-se facilmente e eu andava querendo ter uma em meus
cercados.

--- São de utilidade? --- perguntou o utilitário da roda.

--- De nenhuma; apenas enfeitam a casa. Aparece um visitante. ``Viu
minha ema?'' --- e lá o levamos a examiná"-la de perto, a assombrar"-se do
tamanhão, a abrir a boca diante dos ovos. São assim como uma
laranja"-baiana das graúdas.

--- E o gosto?

--- Nunca provei. Ovos para mim só os de galinha. Mas, como ia dizendo,
fiquei com ideia de apanhar uma ema nova para domesticá"-la --- e um belo
dia eu mesmo o consegui graças a ajuda dum quiriquiri.

A história começava a interessar. Os companheiros do narrador ouviam"-no
suspensos.

--- Como foi? Ande logo.

--- Foi num dia em que saí a cavalo para uma chegada à fazendinha do
João Coruja, que morava a uns seis quilômetros do meu rancho. Montei no
meu pampa e fui varando a macega. Aquilo lá não há caminho, só trilhas
de vai"-um pelo capim rasteiro. Os olhos alcançam longe naquele mar de
verde sujo que some na distância. Fui andando. De repente vi a uns
trezentos metros longe qualquer coisa que se movia na macega. Parei.
Firmei a vista. Era uma ema a dar voltas num círculo estreito. ``Que
diabo disto será aquilo?'', perguntei comigo mesmo. Emas eu vira muitas,
mas sempre a pastarem sossegadas ou a fugirem no galope, nadando com as
asas curtas. Assim a dar voltas era novidade. Fiquei de rugas na testa.
Que será? A gente da roça conhece muito bem a natureza de tudo; se vê
qualquer coisa na ``forma da lei'', não se espanta porque é o natural;
mas se vê qualquer coisa fora da lei, fica logo de orelha em pé ---
porque não é o natural. Que tinha aquela ema para dar tantas voltas em
torno do mesmo ponto? Não era da lei. A curiosidade me fez esquecer o
negócio do João Coruja. Torci a rédea ao pampa e lá me fui para a ema.

--- E ela fugiu no galope\ldots{}

--- O natural seria isso, mas não fugiu. Ora, não há ema que não fuja do
homem --- nem ema, nem animal nenhum. Nós somos o terror da bicharia
toda. Parei o pampa a cinco passos dela e nada, nada da ema fugir. Nem
me viu; continuou nas suas voltas, com ar aflito. Pus"-me a observá"-la,
intrigado. Seria seu ninho ali? Não era. Não havia sinal de ninho. A
pobre ave girava e regirava, fazendo movimentos de pescoço sempre na
mesma direção, para a esquerda, como se quisesse alcançar qualquer coisa
com o bico. A roda que fazia era de raio curto, aí duns três metros, e
pelo amassamento do capim calculei que já havia dado umas cem voltas.

--- Interessante! --- murmurou um dos companheiros.

--- Foi o que pensei comigo mesmo. Mais que interessante:
esquisitíssimo. Primeiro, não fugir de mim; segundo, continuar nas
voltas aflitas, sempre com aqueles movimentos de pescoço para a
esquerda. Que seria? Apeei e fui chegando. Olhei"-a de bem perto. ``A
coisa é embaixo da asa'', vi logo. A pobre criatura tinha qualquer coisa
sob a asa, e aquelas voltas e aquele movimento de pescoço eram para
alcançar o sovaco. Aproximei"-me mais. Segurei"-a. A ema, arquejante, não
fez a menor resistência. Deixou"-se agarrar. Ergui"-lhe a asa e vi\ldots{}

Os ouvintes suspenderam o fôlego.

--- \ldots{} e vi uma coisa vermelha atracada ali, uma coisa que se assustou
e voou, e foi pousar num galho seco a vinte passos de distância. Sabem o
que era? Um quiriquiri\ldots{}

--- Que é isso?

--- Um gaviãozinho dos menores que existem, assim do tamanho dum sanhaço
--- um gaviãozinho"-carijó.

--- Mas não disse que era vermelho?

--- Estava vermelho do sangue da ema. Agarrara"-se"-lhe ao sovaco, que é
um ponto despido de penas, e aferrara"-se à carne com as unhas, enquanto
com o bico ia arrancando nacos de carne viva e devorando"-os. Aquele
ponto do sovaco é o único sem defesa num corpo de ema, porque ela não o
alcança com o bico. É como esse ponto que temos nas costas e não podemos
coçar com as unhas. O quiriquiri conseguira localizar"-se ali e estava a
seguro de bicadas.

``Examinei a ferida. Pobre ema! Uma ferida enorme, assim dum palmo de
diâmetro e onde o bico do quiriquiri fizera menos mal que suas garras,
pois, como tinha de manter"-se aferrado, ia mudando as garras à proporção
que a carne dilacerada cedia. Nunca vi ferida mais arrepiante.''

--- Coitada!

\textls[-5]{--- As emas são duma estupidez famosa, mas o sofrimento abriu a
inteligência daquela. Fê"-la compreender que eu era o seu salvador --- e
a mim entregou"-se como quem se entrega a um deus. O alívio que minha
chegada lhe produziu, fazendo que o quiriquiri a largasse, iluminou"-lhe
os miolos.}

--- E o gaviãozinho?

--- Ah, o patife, muito vermelho do sangue da ema, lá ficou no galho
seco à espera de que eu me afastasse. Pretendia retomar ao banquete!
``Eu já te curo, malvado!'', exclamei, sacando o revólver. Um tiro.
Errei. O quiriquiri voou para longe.

--- E a ema?

--- Levei"-a para casa, curei"-a e tive"-a lá por uns meses num cercado.
Por fim soltei"-a. Não vai comigo isso de escravizar os pobres
animaizinhos que Deus fez para vida solta. Se no cercado estava livre
dos quiriquiris, era em compensação uma escrava saudosa das correrias
pelo campo. Se fosse consultada, certamente que preferiria os riscos da
liberdade à segurança da escravidão. Soltei"-a. ``Vai, minha filha, segue
o teu destino. Se outro quiriquiri te apanhar, arruma"-te lá com ele.''

--- Mas então é assim?

--- Um velho caboclo da zona informou"-me que aquilo é frequente. Esses
minúsculos gaviõezinhos procuram as emas. Ficam traiçoeiramente a
rondá"-las, à espera de que se descuidem e levantem a asa. Eles, então,
rápidos como setas, lançam"-se; e se conseguem alcançar"-lhes o sovaco,
ali enterram as garras e ficam como carrapatos. E as emas, apesar de
imensas comparadas com eles, acabam vencidas. Caem exaustas; morrem; e
os malvadinhos repastam"-se no carname durante dias.

--- Mas como eles sabem? É o que mais admiro\ldots{}

--- Ah, meu caro, a natureza está inçada de coisas assim, que para nós
são mistérios. Com certeza houve um quiriquiri que por acaso fez isso
uma primeira vez, e como deu certo ensinou a lição aos outros. Estou
convencido de que os animais ensinam uns aos outros o que vão
aprendendo. Oh, vocês, criaturas da cidade, não imaginam que coisas
interessantes há na natureza da roça\ldots{}

\textls[-10]{O caso da ema foi comentado sob todos os ângulos --- e deu um broto. Fez
sair da memória do carregador de cara amarrotada uma história vagamente
similar, em que bichinhos muito pequenos destruíram a vida moral dum
homem.}

--- Sim, destruíram a vida dum bicho imensamente maior, como sou eu em
comparação com as formigas. Fiquem vocês sabendo que a mim aconteceu
coisa ainda pior que o acontecido à ema. Fui vítima dum formigueiro\ldots{}

Todos arregalaram os olhos.

--- Só se já foste hortelão e as formigas te comeram a fazenda ---
sugeriu um.

--- Nada disso. Comeram"-me mais que a fazenda, comeram"-me a alma.
Destruíram"-me moralmente --- mas foi sem querer. Pobrezinhas! Não as
culpo de nada.

--- Conta lá isso depressa!, Manuel. O \emph{Arlanza} não tarda.

E o velho contou.

--- Eu era o fiel da firma Toledo \& Cia., com obrigação de tomar conta
daquele grande armazém da rua Tal. Vocês sabem que tomar conta dum
depósito de mercadorias é coisa séria, porque o homem se torna o único
responsável por tudo quanto entra e sai. Ora, eu, português dos antigos,
desses de antes quebrar que torcer, fui escolhido para ``fiel'' porque
era fiel --- era e sou. Não valho nada, sou um pobre homem ao léu, mas
honradez está aqui. Meu orgulho sempre foi esse. Criei reputação desde
menino. ``O Manuel é dos bons; quebra mas não torce.'' Pois não é que as
formigas me quebraram?

--- Conta lá isso depressa\ldots{}

--- A coisa foi assim. Na qualidade de fiel do armazém, nada entrava nem
saía sem ser por minhas mãos. Eu fiscalizava tudo e com tal severidade
que Toledo \& Cia. juravam sobre mim como sobre a Bíblia. Certa vez
entrou lá uma partida de trinta e dois sacos de arroz, que contei,
conferi e fiz empilhar a um canto, junto a uma pilha de velhos caixões
que lá estavam encostados de muito tempo. Trinta e dois. Contei"-os e
recontei"-os e escrevi no livro de entradas trinta e dois, nem mais um,
nem menos um. E no dia seguinte, conforme velho hábito meu, ainda me fui
à pilha e recontei os sacos. Trinta e dois.

``--- Pois muito que bem. O tempo se passa. O arroz lá fica meses à
espera de negócio, até que um dia recebo do escritório ordem para
entregá"-lo ao portador. Vou dirigir a entrega. Fico na porta do armazém
conferindo os sacos que por ali passavam à costas de dois carregadores
---um, dois, vinte, trinta e um\ldots{} Faltava o último.

\textls[-28]{``---Anda com isso! --- berrei ao carregador que fora buscá"-lo, mas o
bruto aparece"-me lá dos fundos com as mãos vazias:}

``---Não há mais nada.

``---Como não há mais nada? --- exclamei. --- São trinta e dois. Falta
um. Vá buscá"-lo, vá ver.

``Ele foi e voltou na mesma:

``---Não há mais nada.

``---Impossível!

``E fui eu mesmo fazer a verificação e nada achei. Misteriosamente
desaparecera um saco de arroz da pilha\ldots{}

``--- Aquilo pôs"-me tonto de cabeça. Esfreguei os olhos. Cocei"-me.
Voltei ao livro de entradas; reli o assento; claro como o dia: trinta e
dois. Além disso eu lembrava"-me muito bem daquela partida por causa dum
incidente agradável. Logo que terminei a contagem eu havia dito `trinta
e dois, última dezena do camelo!', e aproveitei o palpite na venda da
esquina. Mil réis na dezena trinta e dois: de tarde apareceu"-me o
empregadinho com oitenta mil réis. Dera o camelo com trinta e dois.

``--- Vocês bem sabem que essas coisas a gente não esquece. Eram pois
trinta e duas sacas ---e como então só estavam lá trinta e uma? Pus"-me a
parafusar. Furtar ninguém furtara, porque eu era o mais fiel dos fiéis,
não arredava pé da porta e dormia lá dentro. Janelas gradeadas de ferro.
Porta, uma só. Que ninguém furtara o saco de arroz era coisa que eu
juraria perante todos os tribunais do mundo, como o jurava para a minha
consciência. Mas a saca de arroz desaparecera\ldots{} e como era?

``--- Tive de comunicar ao escritório o desaparecimento --- e foi o
maior vexame da minha vida. Porque nós, operários, temos a nossa honra,
e a minha honra era aquela ---era ser o único responsável por tudo
quanto entrasse e saísse daquele depósito.\looseness=-1

``Chamaram"-me ao escritório.

``---Como explica a diferença, Manuel?

``Cocei a cabeça.

``---Meu senhor ---respondi ao patrão --- bem quisera eu explicá"-la, mas
por mais que torça os miolos não consigo. Recebi os trinta e dois sacos
de arroz; contei"-os e recontei"-os, e tanto eram trinta e dois que nesse
dia deu essa dezena e `mamei' do vendeiro da esquina oitenta `paus'. O
arroz demorou lá meses. Agora recebo ordem para entregá"-lo ao caminhão.
Vou presidir à retirada e só encontro trinta e um. Furtá"-lo, ninguém o
furtou; isso juro, porque a entrada do armazém é uma só e eu sempre fui
cão de fila ---mas o fato é que o saco de arroz desapareceu. Não sei
explicar o mistério.

``As casas comerciais têm que seguir certas normas, e se eu fosse o
patrão faria o que ele fez. Já que era o Manuel o responsável único, se
não havia explicação para o mistério, pior para o Manuel.

``--- Manuel --- disse o patrão --- a nossa confiança em você sempre foi
completa, como você muito bem sabe, confiança de doze anos; mas o arroz
não podia ter"-se evaporado como água ao fogo. E como desapareceu um saco
podem desaparecer mil. Quero que você mesmo nos diga o que devemos
fazer.

``Respondi como devia.

``---O que há a fazer, meu senhor, é despedir o Manuel. Ninguém furtou a
saca de arroz, mas a saca de arroz confiada à guarda do Manuel
desapareceu. O que o patrão tem a fazer é fazer o que o Manuel faria se
estivesse em seu lugar: despedi"-lo e contratar outro.

``O patrão disse:

``--- Muito lamento ter de agir assim, Manuel, mas tenho sócios que me
fiscalizam os atos, e serei criticado se não fizer como você mesmo me
aconselha.''

O velho carregador parou para avivar o cachimbo.

--- E foi assim, meus caros, que depois de doze anos de serviço no
armazém de Toledo \& Cia. fui para o olho da rua, suspeitado de ladrão
por todos os meus colegas. Se ninguém podia furtar aquele arroz e o
arroz desaparecera, qual o culpado? O Manuel, evidentemente.

``Fui para a rua, meus caros, já velhusco e sem carta de recomendação,
porque recusei a que a firma me quis dar por esmola. Em boa consciência,
que carta poderiam dar"-me os senhores Toledo \& Cia.?

``Ah, o que sofri! Saber"-me inocente e sentir"-me suspeitado --- e sem
meios de defesa. Roubar é roubar, seja mil rés, sejam contos. Cesteiro
que faz um cesto faz um cento. E eu, que era um homem feliz porque
compensava a minha pobreza com a fama de honestidade sem par, rolei para
a classe dos duvidosos. E o pior era o rato que me roía os miolos. Os
outros podiam satisfazer"-se atribuindo a mim o furto, mas eu, que sabia
da minha inocência, não arrancava aquele rato da cabeça. Quem tiraria de
lá o saco de arroz? Esse pensamento ficou"-me lá dentro como um berne dos
cabeludos.

``Dois anos se passaram, em que envelheci dez. Um dia recebo recado da
firma, `que aparecesse no escritório'. Fui.

``---Manuel ---disse"-me o mesmo chefe que me despedira --- o misterioso
desaparecimento do saco de arroz está decifrado e você reabilitado da
maneira mais completa. Ladrões tiraram de lá o arroz sem que você
visse\ldots{}

``---Não pode ser, meu senhor! Tenho orgulho do meu trabalho de guarda.
Sei que ninguém entrou lá durante aqueles meses. Sei.

``O chefe sorriu.

``---Pois saiba que inúmeros ladrõezinhos entraram e saíram com o arroz.

``Fiquei tonto. Abri a boca.

``---Sim, as formigas\ldots{}

``---As formigas? Não estou entendendo nada, patrão\ldots{}

``Ele contou então tudo. A partida dos trinta e dois sacos fora
arrumada, como já disse, junto a uma pilha de velhos caixões vazios. E o
último saco ficava pouco acima do nível do último caixão --- disso eu me
lembrava perfeitamente. Fora esse o saco desaparecido. Pois bem. Um belo
dia o escritório dá ordem ao novo fiel para remover de lá os caixões. O
fiel executa"-a --- mas ao fazê"-lo nota uma coisa: grãos de arroz
derramados no chão, em redor dum olheiro de formigas saúvas. Foram as
saúvas as roubadoras da saca de arroz número trinta e dois!''

--- Como?

--- Subiram pelos interstícios da caixotaria e furaram o saco último, o
qual ficava um pouco acima do nível do último caixão. E foram retirando
os grãos um a um. Com o progressivo esvaziar"-se, o saco perdeu o
equilíbrio e escorregou da pilha para cima do último caixão --- e nessa
posição as formigas completaram o esvaziamento\ldots{}

--- E\ldots{}

--- Os senhores Toledo \& Cia. pediram"-me desculpas e ofereceram"-me de
novo o lugar, com paga melhorada a título de indenização. Sabem o que
respondi?

\textls[-28]{``Meus senhores, é tarde. Já não me sinto o mesmo. O desastre matou"-me
por dentro. Um rato roubou"-me todo o arroz que havia dentro de mim.
Deixou"-me o que sou: carregador do porto, saco vazio. Já não tenho
interesse em nada. Continuarei portanto carregador. É serviço de menos
responsabilidade --- além de que este mundo é uma pinoia. Pois um mundo
onde uns bichinhos inocentes dão cabo da alma dum homem, então isso é lá
mundo? Obrigado, meus senhores!'', e saí.}

Nesse momento o \emph{Arlanza} apitou. O grupo dissolveu"-se e também eu
fui colocar"-me a postos. O amigo de Londres causou"-me má impressão.
Magro, corcovado.

--- Que te aconteceu, Marinho?

--- Estou com os pulmões afetados.

Hum!, sempre a mesma coisa --- o pequenininho a derrear o grande.
Quiriquiri, saúva, bacilo de Koch\ldots{}


