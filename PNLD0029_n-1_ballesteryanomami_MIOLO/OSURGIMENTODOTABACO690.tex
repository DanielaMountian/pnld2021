\chapterspecial{O {surgimento} {do} {tabaco}}{}{}
 

 

\letra{E}{sta é a história de Hãxoriwë}, o dono do tabaco. Antes ninguém usava o
tabaco, porque ninguém conhecia suas sementes, nem as soprava para
semear. 
%NOTA: Os Yanomami consomem tabaco em rolos feitos com as folhas umedecidas e misturadas com cinzas. Esses rolos são colocados entre o lábio inferior e a gengiva, e são chamados em português de brejeiras. Os Yanomami apreciam muito o tabaco e o consomem constantemente. 

 ``É desse jeito que se coloca o tabaco no lábio!'' Ninguém pensava
assim. Eles não conheciam o tabaco; por isso, ninguém andava com
brejeira no lábio, ninguém o usava, pois o desconheciam. 

Nessa época, Hãxoriwë morava sozinho, não tinha esposa nem
filho. Quando Horonamɨ por acaso o encontrou, ele fez
perguntas a Hãxoriwë. Horonamɨ o encontrou pois era pajé e se deslocava
facilmente. Quando Horonamɨ o encontrou, ele o viu comendo a
fruta \emph{pahi}, um tipo de ingá. Hãxoriwë estava comendo, mas não
usava tabaco. Ele tinha vontade de usar tabaco, por isso chorava.
Hãxoriwë chorava. Estava sofrendo por causa do tabaco, e assim nos
ensinou a ter vontade de usar o tabaco --- por isso choramos quando não
tem tabaco. 

Horonamɨ apareceu naquele momento; Hãxoriwë estava comendo. Ele comia
frutas \emph{pahi} sem parar. Os galhos estavam cheios de frutas
agrupadas, que estavam penduradas nos galhos carregados. Horonamɨ o viu
comer. Horonamɨ estava vindo sem nada, não tinha brejeira, mas fez
aparecer no seu lábio um tabaco sem cor. Ele fez aparecer o
tabaco \emph{taratara}.\footnote{  Trata"-se de uma variedade forte de tabaco, muito apreciada.}  Enquanto Horonamɨ ainda estava
de pé, ele perguntou a Hãxoriwë: 

--- Quem é você? Você aí, quem é? 

--- Não pergunte quem sou! Sou Hãxoriwë! ---disse ele. --- Meu
filho,\footnote{  Modo carinhoso usado por parentes mais velhos ao se dirigerem a
parentes mais novos, mais especificamente entre pais e filhos ou avós e
netos.}  é você? 

--- Sim.

--- Você, quem é você?

--- Sou Horonamɨ, sou Horonamɨ --- disse. --- O que você está comendo? 

--- Não pergunte o que é! --- retrucou. --- Eu como fruta. Eu como
fruta. É a fruta \emph{pahi}! --- disse Hãxoriwë. 

Quando ele disse isso, Horonamɨ olhou. Ele queria fazer aparecer o
tabaco. Ele não fez aparecer o tabaco da forma que o conhecemos, pois
ninguém, sequer ele mesmo, sabia preparar o tabaco depois de soprar as
sementes e de misturar as folhas com cinzas. Como Horonamɨ era pajé, ele
fez sair o tabaco de dentro de Hãxoriwë. Depois de fazer sair o tabaco
sem cor, ele o usou. Hãxoriwë olhou e quando viu o tabaco: 

--- Hɨ̃ɨɨ! --- chorou logo. 

Era um ardil para que Horonamɨ lhe desse o tabaco: 

--- Brejeira! Meu filho! Brejeira! --- chorou Hãxoriwë. 

--- Hɨ̃ɨɨ! Meu sogro! Você está sofrendo tanto assim?! 

--- Sim! Estou querendo, meu filho! Divida o que você tem no
lábio! --- chorou ele.

--- Meu sogro está sofrendo muito, mesmo! Me dê algumas das frutas que
você está comendo e eu lhe darei tabaco para você provar! --- disse
Horonamɨ. 

Com essa conversa, Hãxoriwë jogou uma ou duas frutas. Ele estava
sovinando as frutas, guardando"-as só para si. Horonamɨ experimentou as
frutas. 

Depois de chupar as frutas, os caroços caíam por si sós, de tão maduras:

``Hɨ̃ɨɨ! Prohu! Prohu!'' elas faziam ao cair. 

--- Sogro! As sementes estão moles. Tem muitas frutas ali grudadas, tire
para mim! 

--- Não, primeiro me passe a brejeira! 

Hãxoriwë nos ensinou essa palavra: brejeira. Assim, quando Horonamɨ a
guardou no lábio, ele disse: 

--- Minha brejeira! 

Não apareceu logo esse nome, tabaco.\footnote{  Nesta narrativa os dois termos são tratados como sinônimos.}  Ele só apareceu
quando Hãxoriwë pronunciou essa palavra, até então desconhecida.
Horonamɨ lhe deu a brejeira. Horonamɨ aproveitou a situação e pediu
outras frutas. Assim, Hãxoriwë lhe deu mais uma, mais uma e mais uma.
Essas frutas penduradas, depois de colhidas, pareciam cachos de banana. 

--- Vamos, meu sogro! Experimente! --- disse Horonamɨ. --- Prova!

``Tëɨ!'', Hãxoriwë caiu. 

--- Dê aqui! Traga aqui! --- choramingou. 

Como Hãxoriwë estava chorando, Horonamɨ lhe deu o tabaco e ele
logo o colocou no lábio. Quando o colocou na boca, ele já ficou tonto, e
tremia de tontura. Ele chorava, embriagado. A força do tabaco o pegou
imediatamente. Ainda com o tabaco na boca ele cuspiu, e a espuma caiu no
chão. Onde a espuma caiu, surgiu um broto de tabaco, que logo cresceu e
se espalhou de uma vez. As folhas de tabaco logo ficaram grandes, como as folhas da jurubeba. 

Horonamɨ fez aparecer o tabaco através de Hãxoriwë. O conhecimento das sementes foi transmitido, por isso nossos antepassados as pegaram e hoje nós usamos o tabaco, apesar de ele se originar do cuspe de Hãxoriwë. 

--- Meu sogro, depois de melhorar, você dirá: é só tabaco! --- disse
Horonamɨ. 

Enquanto Hãxoriwë estava pendurado e inebriado, uma espuma grande saiu
da sua boca, por causa da força do tabaco. Ele se engasgou e cuspiu, e
foi dessa espuma que surgiu o tabaco, do cuspe de Hãxoriwë, que se tornou
tabaco. 

E um dia, quando os antepassados foram de \emph{wayumɨ}, como de
costume, um deles encontrou o tabaco. Assim, fizeram se multiplicar as
sementes e ficaram conhecendo o tabaco. 

Quem fez aparecer o tabaco? Nós já sabemos, não foi outro que o fez
aparecer. Não foi um Yanomami comum. 

Havia nessa época os Yanomami do xapono\emph{ }Warahiko, e foram eles
que encontraram o tabaco, foi um deles. Quando viram o tabaco,
disseram: 

--- Õooãa! Uau! Uma plantação de tabaco! 

Foram eles que pronunciaram o nome do tabaco. Em uma região
ali perto, moravam dois Wãimaãtori, de outro xapono. Quando os do
xapono\emph{ }Warahiko encontraram um deles, lhe contaram a respeito do
tabaco. 

--- Meu filho! Qual é o nome disso? 
--- Ah, é tabaco! --- assim retrucaram os dois Wãimaãtori. 

Foi assim que aconteceu: Hãxoriwë, os Warahikoteri e os dois Wãimaãtori
descobriram o tabaco primeiro. Foi assim que o uso do tabaco se
desenvolveu. Os \emph{napë} não fizeram surgir o tabaco depois de soprar
as sementes. Foi a partir do lugar onde surgiu o tabaco que ele se
espalhou por todo canto. Assim foi. 

Como surgiu o tabaco? Já sabemos: Hãxoriwë iniciou o processo quando
Horonamɨ fez aparecer o tabaco, enquanto Hãxoriwë estava olhando. É obra
de Horonamɨ, foi ele quem o fez surgir. Ele é um grande pajé, por isso,
o maior. 

Depois de o tabaco se espalhar, quando os Warahikoteri eram Yanomami,
eles até desmaiaram com a força do tabaco \emph{taratara}. Sofreram de
tontura. Os dois Wãimaatori que moravam mais além, apesar de serem
resistentes ao tabaco, também desmaiaram e ficaram duros por causa da
força do tabaco \emph{taratara}. Mas depois eles melhoraram. Foi assim
que, em seguida, pegaram as sementes de tabaco e as espalharam,
fazendo"-as se multiplicarem aqui. Assim foi.

Hãxoriwë morava aqui. Depois da história do sofrimento de Hãxoriwë,
surge a história do encontro de Horonamɨ com o Tatu.

