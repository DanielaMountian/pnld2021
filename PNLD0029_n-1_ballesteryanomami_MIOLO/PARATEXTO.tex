\chapterspecial{Como foi feito este livro}{}{Anne Ballester Soares}

\section{Sobre o autor}

Os Yanomami habitam uma grande extensão da floresta amazônica, que cobre
parte dos estados de Roraima e do Amazonas, e também uma parte da
Venezuela. Sua população está estimada em 35 mil pessoas, que falam
quatro línguas diferentes, todas pertencentes a um pequeno tronco
linguístico isolado. Essas línguas são chamadas yanomae, ninam, sanuma e
xamatari.

As comunidades de onde veio este livro são falantes da língua xamatari
ocidental, e ficam no município de Barcelos, no estado do Amazonas, na
região conhecida como Médio Rio Negro, em torno do rio Demini. 
 
Em 2008, as comunidades Ajuricaba, do rio Demini, Komixipɨwei, do rio
Jutaí, e Cachoeira Aracá, do rio Aracá --- todas situadas no município
de Barcelos, estado do Amazonas --- decidiram gravar e transcrever todas
as histórias contadas por seus pajés. Elas conseguiram fazer essas
gravações e transcrições com o apoio do Prêmio Culturas Indígenas de
2008, promovido pelo Ministério da Cultura e pela Associação Guarani
Tenonde Porã.

No mês de junho de 2009, o pajé Moraes, da comunidade de Komixipɨwei,
contou todas as histórias, auxiliado pelos pajés Mauricio, Romário e
Lauro. Os professores yanomami Tancredo e Maciel, da comunidade de
Ajuricaba, ajudaram nas viagens entre Ajuricaba e Barcelos durante a
realização do projeto. Depois, no mês de julho, Tancredo e outro
professor, Simão, me ajudaram a fazer a transcrição das gravações, e
Tancredo e Carlos, professores respectivamente de Ajuricaba e
Komixipɨwei, me ajudaram a fazer uma primeira tradução para a língua
portuguesa.  

Fomos melhorando essa tradução com a ajuda de muita gente: Otávio
Ironasiteri, que é professor yanomami na comunidade Bicho"-Açu, no rio
Marauiá, o linguista Henri Ramirez, e minha amiga Ieda Akselrude de
Seixas. Esse trabalho deu origem ao livro \emph{Nohi patama Parahiteri
pë rë kuonowei të ã} --- \emph{História mitológica do grupo Parahiteri},
editado em 2010 para circulação nas aldeias yanomami do Amazonas onde se
fala o xamatari, especialmente os rios Demini, Padauiri e Marauiá. 
 
Em 2013, a editora Hedra propôs a essas mesmas comunidades e a mim que
fizéssemos uma reedição dos textos, retraduzindo, anotando e ordenando as
narrativas para apresentar essas histórias para adultos e para crianças de todo
o Brasil. Assim, o livro original deu origem a diversos livros com as muitas
histórias contadas pelos pajés yanomami.  

\section{Sobre a obra}

Este livro reúne histórias contadas por pajés yanomami do rio Demini, sobre os tempos antigos, quando seres que hoje são animais e espíritos eram gente como os Yanomami de hoje. Estas histórias contam como o mundo veio a ser como ele é agora. 

Trata"-se de um saber sobre a origem do mundo e dos conhecimentos dos Yanomami que as pessoas aprendem e amadurecem ao longo da vida, por isto este é um livro para adultos. As crianças yanomami também conhecem estas histórias, mas sugerimos que os pais das crianças de outros lugares as leiam antes de compartilhá-las com seus filhos.
%E com a ajuda do \versal{PROAC},
%programa de apoio da Secult-SP e da antropóloga Luísa Valentini, que organiza a
%série Mundo Indígena, publicamos agora uma versão bilíngue das principais
%narrativas coletadas, com o digno propósito de fazer circular um livro que
%seja, ao mesmo tempo, de uso dos yanomami e dos \textit{napë} – como eles nos chamam. 

\subsection{Para ler as palavras yanomami}


Foi adotada neste livro a ortografia elaborada pelo lingüista Henri Ramirez, que é a mais utilizada no Brasil e, em particular, nos programas de alfabetização de comunidades yanomami. 

\begin{itemize}
\item[/ɨ/] vogal alta, emitida do céu da boca, e que soa próximo a I e U
\item[/ë/] vogal entre o E e o O do português
\item[/w/] U curto, como em “língua”
\item[/y/] I curto, como em “Mário”
\item[/e/] vogal E, como em português
\item[/o/] O, como em português
\item[/u/] U, como em português
\item[/i/] U, como em português
\item[/a/] A, como em português
\item[/p/] como P ou B em português
\item[/t/] como T ou D em português
\item[/k/] como C de “casa”
\item[/h/] como o RR em “carro”, aspirado e suave
\item[/x/] como X em “xaxim”
\item[/s/] como S em “sapo”
\item[/m/] como M em “mamãe”
\item[/n/] como N em “nada”
\item[/r/] como R em “puro”
\end{itemize}

\section{Sobre o gênero}



 

 

 

 

 

 
