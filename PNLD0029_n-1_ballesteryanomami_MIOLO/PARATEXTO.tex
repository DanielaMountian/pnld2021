\chapter{Como foi feito este livro}

\begin{flushright}
\textsc{anne ballester soares}
\end{flushright}
\bigskip

\section{Sobre o autor}

\noindent{}Os Yanomami habitam uma grande extensão da floresta amazônica, que cobre
parte dos estados de Roraima e do Amazonas, e também uma parte da
Venezuela. Sua população está estimada em 35 mil pessoas, que falam
quatro línguas diferentes, todas pertencentes a um pequeno tronco
linguístico isolado. Essas línguas são chamadas yanomae, ninam, sanuma e
xamatari.

As comunidades de onde veio este livro são falantes da língua xamatari
ocidental, e ficam no município de Barcelos, no estado do Amazonas, na
região conhecida como Médio Rio Negro, em torno do rio Demini. 
 
Além dessas quatro línguas, no Brasil há um
total de 154 línguas indígenas faladas por todo o território brasileiro,
mas antes do Brasil colônia, tínhamos entre 600 e 1000 línguas. O
trabalho de registro feito pelos pesquisadores em conjunto com as
comunidades indígenas é de extrema importância para conseguir conservar
a história desses idiomas. Não somente para consultas posteriores, mas
para a organização da própria comunidade.

Em 2008, as comunidades Ajuricaba, do rio Demini, Komixipɨwei, do rio
Jutaí, e Cachoeira Aracá, do rio Aracá --- todas situadas no município
de Barcelos, estado do Amazonas --- decidiram gravar e transcrever todas
as histórias contadas por seus pajés. Elas conseguiram fazer essas
gravações e transcrições com o apoio do Prêmio Culturas Indígenas de
2008, promovido pelo Ministério da Cultura e pela Associação Guarani
Tenonde Porã.

No mês de junho de 2009, o pajé Moraes, da comunidade de Komixipɨwei,
contou todas as histórias, auxiliado pelos pajés Mauricio, Romário e
Lauro. Os professores yanomami Tancredo e Maciel, da comunidade de
Ajuricaba, ajudaram nas viagens entre Ajuricaba e Barcelos durante a
realização do projeto. Depois, no mês de julho, Tancredo e outro
professor, Simão, me ajudaram a fazer a transcrição das gravações, e
Tancredo e Carlos, professores respectivamente de Ajuricaba e
Komixipɨwei, me ajudaram a fazer uma primeira tradução para a língua
portuguesa.  

Fomos melhorando essa tradução com a ajuda de muita gente: Otávio
Ironasiteri, que é professor yanomami na comunidade Bicho"-Açu, no rio
Marauiá, o linguista Henri Ramirez, e minha amiga Ieda Akselrude de
Seixas. Esse trabalho deu origem ao livro \emph{Nohi patama Parahiteri
pë rë kuonowei të ã} --- \emph{História mitológica do grupo Parahiteri},
editado em 2010 para circulação nas aldeias yanomami do Amazonas onde se
fala o xamatari, especialmente os rios Demini, Padauiri e Marauiá. 
 
Em 2013, a editora Hedra propôs a essas mesmas comunidades e a mim que
fizéssemos uma reedição dos textos, retraduzindo, anotando e ordenando as
narrativas para apresentar essas histórias para adultos e para crianças de todo
o Brasil. Assim, o livro original deu origem a diversos livros com as muitas
histórias contadas pelos pajés yanomami.  

\section{Sobre a obra}

Este livro reúne histórias contadas por pajés yanomami do rio Demini, sobre os tempos antigos, quando seres que hoje são animais e espíritos eram gente como os Yanomami de hoje. Estas histórias contam como o mundo veio a ser como ele é agora. 

Trata"-se de um saber sobre a origem do mundo e dos conhecimentos dos Yanomami que as pessoas aprendem e amadurecem ao longo da vida, por isto este é um livro para adultos. As crianças yanomami também conhecem estas histórias, mas sugerimos que os pais das crianças de outros lugares as leiam antes de compartilhá-las com seus filhos.

%E com a ajuda do \versal{PROAC},
%programa de apoio da Secult-SP e da antropóloga Luísa Valentini, que organiza a
%série Mundo Indígena, publicamos agora uma versão bilíngue das principais
%narrativas coletadas, com o digno propósito de fazer circular um livro que
%seja, ao mesmo tempo, de uso dos yanomami e dos \textit{napë} – como eles nos chamam. 

Ao apresentar narrativas sobre os mitos de criação sob a perspectiva
Yanomami, \emph{O Surgimento da Noite} desvela outros modos de existência
e crenças que transbordam da lógica ocidental e eurocêntrica.
Apesar do título do livro fazer referência ao surgimento da noite, as
narrativas encontradas ao decorrer da leitura abordam o surgimento de
outros elementos também. Encontraremos o surgimento do tabaco, do cipó e
da banana através das aventuras do personagem \emph{Horonamɨ.}

\emph{Horonamɨ} é um grande pajé que surgiu dele mesmo, assim como é
relatado na narrativa. Surgiu junto com as florestas e ensinou aos
Yanomamis como morar nelas. Além de compartilhar os conhecimentos com os
Yanomamis, ele também compartilhou suas histórias com os estrangeiros.



\subsection{a lógica cristã e o mito de criação indígena}

Assim que se iniciou a colonização do Brasil, os portugueses acreditavam
que tudo aquilo que não se pareciam com eles poderia ser denominado de
selvagem. A Coroa Portuguesa sempre foi muito conhecida por ser
extremamente católica, logo a sua história é repleta de perseguições e
intolerância em relação as novas religiões.

As missões jesuíticas são as provas de como a intolerância cultural era
altamente disseminada no século \textsc{xvi} e \textsc{xvii}. A metrópole envia padres que
tinham como missão catequizar a comunidade indígena e os fazerem
acreditar na existência de um Deus, em conceitos maniqueístas e a
supremacia da cultura europeia.

Sob a perspectiva cristão, o mundo foi criado em sete dias por um Deus
único e onipresente. Após os sete dias, a humanidade foi criada a partir
de Adão e Eva. Tudo o que existe no mundo foi criado exclusivamente por
um único ser. Quando acompanhamos as narrativas dos Yanomamis, não temos
a figura de um Grande Criador e único, mas temos a criação dos elementos
e o surgimento dos fenômenos físicos, como a noite ou o dia, a partir de
aventuras vividas em conjunto pela comunidade.

A ideia de coletivo e a concepção do surgimento de elementos a partir de
ações ou reações é a linha de raciocínio das comunidades. Não existe uma
figura única que deva ser respeitada, mas existe um conjunto de ações da
natureza e todos os seus elementos e quando uma comunidade trabalha em
conjunto, a natureza poderá fornecer novos conceitos e outros novos
elementos.

De leitura rápida, \emph{O Surgimento da Noite} é recheado de
histórias fantásticas. A valorização da natureza e de seus elementos é a
peça fundamental para entender a cultura Yanomami. O modo de narrar
apresentado é muito semelhante as aventuras de \textit{Macunaíma}, obra de Mario
de Andrade. O autor modernista mergulhou nas histórias brasileiras para
construir seu herói e suas aventuras.

É importante conhecermos e valorizarmos os relatos e as culturas que já
existiam no Brasil antes da colonização. Os registros fonéticos e
fonológicos realizados pelos pesquisadores são como um tesouro que devem
ser compartilhados e bem cuidados. Ler as narrativas Yanomamis é poder
se conectar com o Brasil anterior aquele que aprendemos nas aulas de
história.

\section{Sobre o gênero}

Essas narrativas podem ser classificas como mitologias indígenas pois relacionam"-se ao contexto específico de produção de mitologias e crenças dos povos originários.
Como esses povos primeiros eram ágrafos, isto é não tinham um sistema de escrita, estas narrativas eram tradicionalmente transmitidas pela oralidade.
A narrativa oral é caracterizada por alguns elementos específicos: normalmente é entoada, como uma canção, o que ajuda na memorização da história; revela mitos e histórias relacionados à cosmovisão de um povo; envole a criação de um espaço diferente, uma demarcação de sacralidade em contraposição ao espaço profano do cotidiano, se formos pensar com o mitólogo romeno Mircea Eliade.

Entre o povo Guarani, por exemplo, são comuns as rodas para fumar petygua, cachimbo de tabaco, para a contação de histórias. Já os Hupd’ah, povo do Alto Rio Negro, narram suas histórias em rodas que fumam tabaco e mascam coca.
Entre os Yanomami, a prática é a ingestão de \textit{yãkoana}, pó feito com cascas de árvores secas e pulverizadas, que iniciam o indígena no conhecimento xamânico de seu povo.

Como relata o líder indígena Davi Kopenawa em \textit{A queda do céu}, o consumo
de \textit{yãkoana} está intimamente relacionado à transmissão das narrativas tradicionais yanomami, pois permite ao xamã ouvir as palavras de \textit{Omama}, demiurgo da cosmogonia yanomami, possibilitando"-lhe a transmissão de histórias e narrativas que atravessam gerações. É através da \textit{yãkoana}, igualmente, que o xamã pode visualizar os \textit{xapiri}, os espíritos da floresta, presentes em cada elemento natural, seja um animal, árvore ou mesmo a terra e a água.
Percebe"-se, assim, como a narrativa tradicional yanomami está intrinsecamente ligada à construção de espaços e rituais específicos.

Registradas em livro, essas narrativas aproximariam"-se do gênero conto,
que ``remonta aos primórdios da própria arte literária''.
De linguagem concisa, por vezes poética, o conto é unívoco e univalente, sob a perspectiva do ângulo dramático:

\begin{quote}
Etimologicamente preso à linguagem teatral,
``drama'' significava ``ação''. E com o tempo passou a designar
toda peça destinada à representação. Na época romântica, dado o
princípio da fusão de gêneros, entendia-se por drama o misto de
tragédia e comédia. Transferido para a prosa de ficção, o termo
``drama'' entrou a significar ``conflito'', ``atrito''. Nesse caso,
``ação'' ``conflito'' se tonaram equivalentes, uma vez que toda
ação pressupõe conflito, e este, promove a ação, ou por meio dela
se manifesta; em suma, ambos se implicam mutuamente.

O conto é, pois, uma narrativa unívoca, univalente: constitui
uma \textit{unidade dramática}, uma \textit{célula dramática}, visto gravitar ao
redor de um só conflito, um só drama, uma só ação. Caracteriza-se,
assim, por conter \textit{unidade de ação}, tomada esta como a sequência de atos praticados pelos protagonistas, ou de acontecimentos de
que participam. A ação pode ser externa, quando as personagens se
deslocam no espaço e no tempo, e interna, quando o conflito se
localiza em sua mente.\footnote{\textsc{moisés}, Massaud. \textit{A criação literária}. São Paulo: Cultrix, 2006, p.\,40.}
\end{quote}

Partindo da definição de Massaud Moisés sobre o conto, evidencia"-se a principal característica desse gênero literário: a unidade de conflito, condensada em ações que se completam em um único enredo. Ao conto, ainda seguindo Moisés, aborrecem as divagações e os excessos, pois há uma concentração de efeitos e pormenores essenciais, em sua brevidade, para o bom funcionamento do conto.
Cada construção, cada palavra nesse gênero tem sua razão de existir, pois integra a economia global da narrativa.

São contos, no entanto, com uma nítida dimensão mítica, pois têm
seus componentes essenciais na esfera do sagrado, buscando equacionar
grandes questões espirituais e materiais dessa sociedade.
Nota"-se que, tradicionalmente, não existia uma diferença entre a narrativa histórica e a mítica, pois através do próprio mito se explicava a criação do mundo, dos seres viventes e da sociedade tal qual encontra"-se.
Como é o caso, visto acima, dos mitos de criação cristãos.

Vale ressaltar, por fim, que, apesar de o português ter se sobressaído como a língua oficial do Brasil, por se tratar do idioma do colonizador, muitas das palavras e dos mitos indígenas foram assimilados pelos falantes do português.
Abaixo segue uma tabela para entender a pronúncia dessas palavras, possibilitando um mergulho ainda maior no universo das narrativas yanomami.

\subsection{Para ler as palavras yanomami}


Foi adotada neste livro a ortografia elaborada pelo linguista Henri Ramirez, que é a mais utilizada no Brasil e, em particular, nos programas de alfabetização de comunidades yanomami. 

\begin{itemize}
\item[/ɨ/] vogal alta, emitida do céu da boca, e que soa próximo a I e U
\item[/ë/] vogal entre o E e o O do português
\item[/w/] U curto, como em “língua”
\item[/y/] I curto, como em “Mário”
\item[/e/] vogal E, como em português
\item[/o/] O, como em português
\item[/u/] U, como em português
\item[/i/] U, como em português
\item[/a/] A, como em português
\item[/p/] como P ou B em português
\item[/t/] como T ou D em português
\item[/k/] como C de “casa”
\item[/h/] como o RR em “carro”, aspirado e suave
\item[/x/] como X em “xaxim”
\item[/s/] como S em “sapo”
\item[/m/] como M em “mamãe”
\item[/n/] como N em “nada”
\item[/r/] como R em “puro”
\end{itemize}








 

 

 

 

 

 
