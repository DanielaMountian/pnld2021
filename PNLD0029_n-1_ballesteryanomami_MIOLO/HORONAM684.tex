\chapterspecial{{Horonam}ɨ}{}{}
 
%Esta deve ser a primeira história do livro. Importante, por causa do contexto e das notas.

\section*{Quem nos fez?}

Esta é a verdadeira história de nosso surgimento: quando a floresta era
virgem, apareceu Horonamɨ, personagem principal de nossa história, por
causa de seus ensinamentos. O grande pajé\footnote{  Ser pajé, nestas histórias, quer dizer que o personagem em questão é ou
tem a capacidade de se transformar em espírito e, com isso, fazer coisas
extraordinárias.}  yanomami
Horonamɨ surgiu dele mesmo; surgiu ao mesmo tempo que esta floresta e
foi quem ensinou os Yanomami a morar nela. Assim foi o início. 

Não existia Yanomami como os de hoje, nem outro ser humano. 

Ele propagou sua sabedoria para que nossa história fosse sempre lembrada
e discutida, como fazemos agora. Aconteceu bem antes de os tuxauas
yanomami passarem a existir como existem
hoje.\footnote{  No Amazonas, onde vivem as comunidades de Ajuricaba e Komixipɨwei,
usa"-se ``tuxaua'' ou ``liderança'' para designar a pessoa de referência
de uma comunidade indígena, por essa razão optou"-se por esses termos na
tradução.}  Horonamɨ foi o primeiro habitante da floresta e
nos ensinou a morar nela, assim como ensinou também aos estrangeiros,
os \emph{napë}.\footnote{   O termo \emph{napë} designa os estrangeiros, em geral os brancos, ou quem adotou seus costumes.}  Ele não tinha pai, mas mesmo assim
ele surgiu. Ele surgiu em uma floresta maravilhosa. 

Quem morava com Horonamɨ? Horonamɨ morava com seu cunhado, Wɨyanawë,
que, apesar de não ter desposado sua irmã, era seu verdadeiro
cunhado.\footnote{  Os Yanomami, tradicionalmente, não podem chamar uns aos outros por seus nomes próprios, por isso usam termos de parentesco. Quando não há
consanguinidade, são usados termos de afinidade, como cunhado ou sogro.
Cunhado é também um termo positivo, na medida em que indica alguém em
quem se pode confiar.} Horonamɨ sempre o levava consigo nos
períodos que passavam dentro da mata, chamados \emph{wayumɨ}, e ensinou
os descendentes como ir de \emph{wayumɨ}.\footnote{   Longas estadias coletivas na floresta. Em geral são motivadas pela falta de comida no xapono. A comunidade pode se dividir em vários grupos quando se trata de um xapono populoso, e se desloca num vasto círculo, fazendo acampamentos sucessivos.}

Apesar de sua mãe não ter parido Horonamɨ, pois ele surgiu de repente, o
nome de sua mãe era Yotoama. O pajé Horonamɨ foi quem procurou e
descobriu nossa comida, nosso conhecimento da floresta e o habitat dos
animais, para que, quando os Yanomami ocupassem a floresta, eles fossem
capazes de aplacar sua fome de carne. 

Ele descobriu o nome dos animais quando eles viviam como nós. Apesar de
serem animais, antes eles viviam do mesmo modo que os Yanomami. 

Como ele fez aparecer a água para acalmar a sede dos Yanomami? Ele abriu
várias veredas na floresta. Abriu veredas em todas as direções, de forma
que elas nunca sumam e que sempre bebamos água. 

Horonamɨ tinha seu próprio xapono\footnote{  Os xaponos são as casas coletivas circulares onde moram os Yanomami. Cada
casa corresponde a uma comunidade; em geral não se fazem duas casas numa
mesma localidade.}, onde moravam
também seus aliados, que se tornaram muito importantes. 

Como se chamava o xapono pertencente a Horonamɨ? Esse xapono chamava"-se
Horona. 

O xapono vizinho, que ficava do outro lado do rio, se chamava
Menawakoari. Os primeiros habitantes desse xapono\emph{ }também se
chamavam Menawakoari. Penewakoari era o tuxaua e morava com o grupo dos
Kapurawëteri. O tuxaua dos que moravam com Horonamɨ se chamava
Penewakoari. Kapurawë era o nome do xapono\emph{ }e da região dos
Kapurawëteri.\footnote{  ``Habitantes'': Em alguns casos o xapono tem o nome de seu tuxaua.}

Penewakoari morava com eles e estava destinado a se transformar num
monstro. Penewakoari depois se transformou no monstro Xõewëhena, faminto
de carne e comedor de crianças. Mas, quando ainda era Yanomami,
Penewakoari morava no xapono Kapurawëteri, vizinho ao
xapono\emph{ }Horona.

Nesses xaponos moravam poucas pessoas. Com o tempo, nos xaponos vizinhos
foram aparecendo mais tuxauas. Os primeiros tuxauas que viviam nos
xaponos\emph{ }vizinhos, os xaponos dos aliados, não eram nossos
antepassados, eram outros. Sobre eles se contaram estas histórias.