%!TEX root=LIVRO.tex
\chapterspecial{Sob o pôr do sol}{}{}
 

Longe, muito longe, há um belo País que nenhum olho humano jamais viu
em vigília. Ele fica Sob o Pôr do Sol, onde o horizonte distante desenha
os limites do dia, e onde as nuvens, esplêndidas em luz e cor, prometem
a glória e a beleza que o cerca.

Algumas vezes, podemos vê"-lo em sonhos.

De vez em quando se achegam Anjos, ternamente, abanando com suas grandes
asas brancas os cenhos franzidos, e repousam as mãos de bálsamo sobre os
olhos dormentes. Então, o espírito do adormecido levanta voo. Ele se alça do
ofuscamento e das trevas da temporada noturna. Veleja para longe
através das nuvens púrpuras. Apressa"-se pela vasta amplidão de
luz e ar. Voa pelo azul intenso da abóbada celeste e, estendendo"-se pelo
longínquo horizonte, repousa no belo Reino Sob o Pôr do Sol.

Esse País é como o nosso em vários sentidos. Tem homens e mulheres,
reis e rainhas, ricos e pobres; tem casas, e árvores, e campos, e
pássaros, e flores. Há ali dia e também noite, e calor e frio, e doença
e saúde. Os corações dos homens e das mulheres, e de garotos e de garotas,
batem como os daqui. Há as mesmas tristezas e as mesmas alegrias, e as
mesmas esperanças e os mesmos medos.

Se uma criança daquele País estivesse ao lado de uma criança daqui, você
não poderia apontar a diferença entre elas, exceto que somente as roupas
são diferentes. Elas falam a mesma língua que nós. Não sabem que
são diferentes de nós, e não sabemos que somos diferentes delas. Quando
vêm até nós em seus sonhos, não sabemos que são estranhas; e quando
vamos ao País delas em nossos sonhos, parecemos estar em casa. Talvez
isso ocorra porque os lares das pessoas boas estão em seus corações; e,
em qualquer lugar em que possam estar, terão paz.

O País Sob o Pôr do Sol foi por longas eras um Reino fantástico e
agradável. Nada havia que não fosse belo e doce e agradável. Foi somente
quando chegou o pecado que as coisas começaram a perder sua perfeita
beleza. Mas até mesmo agora é uma terra fantástica e agradável.

Porque lá o sol é forte, às margens de todas as estradas estão plantadas
grandes árvores que espalham seus galhos grossos. Assim, os viajantes
encontram abrigo quando passam. Os marcos à margem são fontes de água fresca e
agradável, tão clara e cristalina que, quando o viajante chega a uma
delas, senta no banco de pedra talhada a seu lado e dá um suspiro de
alívio, pois sabe que haverá descanso.

Quando é pôr do sol aqui, lá é o meio do dia. As nuvens se ajuntam e com
suas sombras livram o Reino do calorão. Daí, por um curto tempo, tudo
adormece.

Essa hora agradável e pacífica é chamada de Hora do Descanso.

Quando ela chega, os pássaros param seu canto, e repousam sob as amplas
calhas das casas ou nos galhos das árvores, onde eles se juntam aos
troncos. Os peixes param de agitar"-se e descansam sob as pedras, com suas
barbatanas e caudas tão imóveis como se estivessem mortos. As ovelhas e
o gado descansam sob as árvores. Os homens e as mulheres deitam"-se em
redes estendidas entre as árvores ou nas varandas de suas casas.
Então, quando o sol para de resplandecer intensamente e as nuvens se
dissipam, todas as coisas vivas acordam.

As únicas coisas vivas que não dormem na Hora do Descanso são os
cachorros. Eles ficam deitados, muito quietos, somente meio dormindo,
com um olho aberto e uma orelha levantada, mantendo vigilância o tempo
todo. Assim, se algum estranho chega durante o momento de Descanso, os
cães se levantam e o observam calmamente, sem latir, para não
perturbar ninguém. Eles sabem se o recém"-chegado é inofensivo; e, quando
é assim, deitam"-se novamente, e o estranho também se deita, até que
termine a Hora do Descanso.

Mas se os cães pensam que o estranho veio para causar malefícios, eles
latem alto e rosnam. As vacas começam a mugir e as ovelhas a balir, e os
pássaros a gorjear e a cantar suas notas mais altas, mas sem música; e
até mesmo os peixes começam a se agitar de lá para cá e a espirrar água.
Os homens acordam e saltam de suas redes, e agarram suas armas. Então, o
intruso passa por maus momentos. Imediatamente é levado à Corte e
julgado, e, se a sentença o considerar culpado, é encarcerado ou banido.

Depois os homens voltam para suas redes, e todas as coisas vivas
novamente se retiram até que a Hora do Descanso termine.

À noite acontece o mesmo que na Hora do Descanso, caso um intruso venha
para causar malefícios. À~noite, somente os cães e os doentes e suas
enfermeiras estão acordados.

Qualquer um só pode deixar o País Sob o Pôr do Sol seguindo numa única
direção. Aqueles que vão para lá em sonhos, ou aqueles que vêm em sonhos para
nosso mundo, vêm e vão sem saber como. Mas, se um habitante tentar
deixá"-lo, só conseguirá de uma única maneira. Se tentar de outras maneiras, vagueará
infinitamente, dando voltas sem perceber, até chegar ao único lugar por
onde pode partir.

Esse lugar é chamado de Portal, e ali os Anjos mantêm guarda.

O palácio do Rei fica exatamente no meio do País, e as estradas
estendem"-se a partir dele para todos os lados. Quando o Rei se posta no
topo da torre, que se ergue a uma grande altura no meio de seu palácio,
consegue estender seu olhar pelas estradas, que são todas bem retas.

Elas parecem se tornar mais e mais estreitas à medida que seguem
adiante, até que por fim se perdem totalmente na ampla distância.

Em volta do palácio do Rei estão reunidas casas de grandes nobres, cada
uma de um tamanho proporcional ao posto de seu dono. Ao lado delas, vêm
as casas dos menos nobres; e depois aquelas de todas as outras pessoas,
tornando"-se cada vez menores à medida que se vai mais adiante.

Toda casa, grande e pequena, ergue"-se no meio de um jardim que tem uma
fonte e um curso d'água, e grandes árvores, e canteiros de belas flores.

Mais ao longe, em direção ao Portal, o país torna"-se cada vez mais
selvagem. Para além dele, há densas florestas e grandes montanhas
repletas de cavernas profundas, tão escuras quanto a noite. Ali,
animais selvagens e todas as coisas cruéis têm sua morada.

Então surgem pântanos e brejos e lamaçais profundos e instáveis, e densas
selvas. Depois tudo se torna tão selvagem que a estrada some
completamente.

Nenhum homem sabe o que há nos lugares selvagens mais além. Alguns
dizem que os Gigantes que ainda existem vivem ali, e que todas as
plantas venenosas crescem lá. Dizem que há um vento iníquo que
carrega as sementes de todas as coisas más e as espalha sobre a terra.
Há alguns que dizem que esse mesmo vento iníquo também espalha as Doenças e
as Pragas que existem ali. Outros dizem que a Fome vive lá nos pântanos,
e que se aproxima silente se os homens são maus -- tão
maus que os Espíritos que guardam essa terra choram muito
amargamente quando não a veem passar.

Murmuram que a Morte tem seu reino nos Ermos além dos pântanos, e que vive
num castelo tão terrível de se olhar que ninguém jamais o viu e viveu
para contar como ela é. Também dizem que todas as coisas más que vivem nos
pântanos são os desobedientes Filhos da Morte, que deixaram seus lares e
não conseguem mais encontrar o caminho de volta.

Mas nenhum homem sabe onde fica o Castelo do Rei Morte. Todos os homens
e mulheres, garotos e garotas, e mesmo as crianças pequenas devem viver
de tal modo que, quando tiverem de entrar no Castelo e ver o Rei
macabro, não tenham medo de contemplar seu rosto.

Por muito tempo, a Morte e seus Filhos permaneceram fora do Portal e
tudo dentro dele era alegria.

Mas eis que veio um tempo em que tudo
mudou. Os corações se esfriaram e endureceram de orgulho à medida que a
prosperidade aumentava e os homens não prestaram mais atenção às lições que
lhes tinham sido ensinadas. Então, quando lá dentro houve frieza e
indiferença e desdém, os Anjos em guarda perceberam nos terrores lá de fora
os meios de punição e a lição que poderia fazer bem.

As boas lições vieram -- como muito frequentemente vêm as coisas boas --
depois de dor e de provação, e elas ensinaram muito. A~história da sua
vinda guarda uma lição para o bom entendedor.

No Portal, dois Anjos vigiavam e mantinham guarda constantemente. Esses
anjos eram tão majestosos e tão vigilantes, e sempre tão firmes em sua
guarda, que havia somente um nome para ambos. Cada um deles ou o par,
quando interpelado, seria chamado pelo nome inteiro. Um deles conhecia
tanto quanto o outro sobre qualquer coisa que pudesse ser conhecida.
Isso não era tão estranho, pois ambos conheciam tudo. O~nome deles era
Fid"-Def.

Fid"-Def estavam de guarda no Portal. Ao lado deles havia uma
Criança"-Anjo, mais bela do que a luz do sol. A~silhueta de sua bela
forma era tão suave que sempre parecia estar desvanecendo no ar; parecia
uma luz viva e sagrada.

Ela não ficava em pé como os outros Anjos, mas flutuava para cima e para
baixo e por todo lado. Algumas vezes era somente uma pequena mancha, e,
então, de repente, sem parecer passar por qualquer mudança,
tornava"-se maior do que os grandes Espíritos Guardiões, os
mesmos desde sempre.

Fid"-Def amavam a Criança"-Anjo, e às vezes, quando porventura ela se
levantava, eles abriam suas grandes asas brancas, sobre as quais ela
subia. E, com suas próprias asas, belas e delicadas, arejava os
rostos deles suavemente quando se viravam para falar.

Mas a Criança"-Anjo nunca cruzava o limiar. Ela olhava para o ermo ao
longe, mas nunca colocava nem mesmo a ponta de sua asa para além do
Portal.

Ela sempre fazia perguntas para Fid"-Def, e parecia querer saber o que
havia lá fora, e como tudo lá diferia de tudo daqui.

As perguntas e as respostas dos Anjos não eram como as nossas perguntas e
respostas, pois não havia necessidade de fala. No momento em que vinha o
pensamento de querer saber alguma coisa, a pergunta era feita e a
resposta era dada. Mas, mesmo assim, a pergunta não deixava de ser feita
pela Criança"-Anjo e respondida por Fid"-Def; e, se conhecêssemos a
não"-língua que os Anjos estavam não"-falando, teríamos ouvido Fid"-Def
falando com Fid"-Def da seguinte maneira:

``Chiaro não é belo?''

``É muito belo. Ele será um novo poder no Reino.''

Aqui, Chiaro, que estava apoiado com um pé na pluma da asa de Fid"-Def,
disse:

``Digam, Fid"-Def, o que são aqueles Seres além do Portal, de aparência
horrível?''

Fid"-Def responderam:

``São os Filhos do Rei Morte. O~mais horrível de todos, envolto em
trevas, é Skooro, um Espírito Mau.''

``Como eles parecem horríveis!''

``Muito horríveis, caro Chiaro. E~esses Filhos da Morte querem cruzar o
Portal e entrar no Reino.''

Chiaro, diante da terrível notícia, ergueu"-se para o alto, e ficou tão
grande que todo o País Sob o Pôr do Sol passou a brilhar. Logo depois,
entretanto, foi diminuindo, diminuindo, até que virou somente uma
mancha, como o facho colorido visto num quarto escuro quando o sol entra
por uma fresta. Ele perguntou aos Anjos do Portal:

``Digam"-me, Fid"-Def, por que os Filhos da Morte querem entrar?''

``Porque, querida Criança, eles são malvados, e querem corromper os
corações dos moradores do Reino.''

``Mas me digam, Fid"-Def, eles conseguem entrar? Tenho certeza que, se o
Pai"-Supremo diz `Não!', eles têm de ficar para sempre fora do Reino.''

Depois de uma pausa veio a resposta dos Anjos do Portal:

``O Pai"-Supremo é mais sábio do que até mesmo os Anjos podem conceber.
Ele expulsou os malvados com seus próprios truques, e fez o caçador cair
em sua própria armadilha. Os Filhos da Morte, quando entram -- como
estão prestes a fazer -- fazem muitas coisas boas no Reino ao qual
querem fazer mal. Pois veja!, os corações das pessoas estão corrompidos.
Eles esqueceram as lições que lhes foram ensinadas. Não sabem o
quanto deveriam ser gratos por sua sorte, pois não conhecem a tristeza.
Deve haver alguma dor ou pesar ou tristeza para que possam ver
o erro de seus caminhos.''

Enquanto falavam, os Anjos choraram de dor pelos pecados do povo e pelo
sofrimento que ele teria de suportar.

A Criança"-Anjo respondeu assombrada:

``Então esse, que é o Ser mais horrível, também está para entrar no
Reino\ldots{} Ai! Ai!''

``Querida Criança'', disseram os Espíritos Guardiães enquanto a
Criança"-Anjo deslizou para seus peitos, ``a você está incumbido um
grande dever. Os Filhos da Morte estão prestes a entrar. A~você foi
confiada a vigilância sobre esse Ser horrível, Skooro. Onde quer que ele vá,
lá você deverá estar também; assim, nada de mal poderá acontecer --, exceto
somente aquilo que é pretendido ou permitido.''

A Criança"-Anjo, maravilhada pela grandeza da confiança, decidiu que sua
tarefa deveria ser bem"-feita. Fid"-Def continuaram:

``Você deve saber, querida Criança, que sem a escuridão não há medo
algum do invisível; e nem mesmo a escuridão da noite consegue assustar caso
haja luz dentro da alma. Ao bom e ao puro não há medo, seja das coisas más
da terra, seja dos Poderes invisíveis. A~você é confiada a guarda do
puro e do verdadeiro. Skooro lançará sua sombra sobre eles; mas
sua missão é penetrar em seus corações e por sua própria luz gloriosa tornar a
sombra dos Filhos da Morte invisível e desconhecida.''

``Mas você se manterá afastado de quem faz o mal, isto é, dos perversos, e dos mal"-agradecidos, e
dos impiedosos, e dos impuros,~e dos falsos você se manterá afastado; e
assim, quando o procurarem para que lhes dê conforto -- como sempre
haverão de fazer --, eles não o verão. Eles irão ver somente a sombra que a sua
luz distante fará parecer ainda mais escura, pois a sombra estará em
suas próprias almas.''

``Mas, oh!, Criança, nosso Pai é bom para além do acreditável. Ele ordena
que, caso uma pessoa seja má e se arrependa, você voará imediatamente até
ela e a confortará, e a ajudará, e a animará, e compelirá a sombra para
longe. Caso a pessoa apenas finja se arrepender, intencionando ser novamente
má quando o perigo passar, ou caso ela aja somente por medo,
então você esconderá sua claridade para que a sombra cresça sobre ela e
escureça ainda mais. Agora, querido Chiaro, torne"-se invisível.
Aproxima"-se a hora em que será permitido ao Filho da Morte entrar no
Reino. Ele tentará entrar sorrateiramente, e nós vamos deixar, pois
devemos trabalhar invisíveis e incógnitos para desempenharmos nossa
função.''

Então a Criança"-Anjo desvaneceu"-se lentamente, a fim de que nenhum olho
-- nem mesmo os de Fid"-Def -- pudesse vê"-lo; e os Espíritos Guardiães se
postaram como sempre ao lado do Portal.

A Hora do Descanso chegou, e tudo estava quieto no Reino.

Quando os Filhos da Morte, bem ao longe nos pântanos, viram que nada
estava em movimento, exceto os Anjos que estavam como
sempre de guarda, resolveram tentar mais uma vez entrar no
Reino.

Assim, dividiram"-se em muitas partes. Cada parte assumiu uma forma
diferente, mas todas juntas se moveram em direção ao Portal. Dessa maneira, os
Filhos da Morte chegaram bem perto do limiar do Reino.

Eles vieram sobre a asa de um pássaro que passava; numa nuvem que
deslizava lentamente pelo céu; nas cobras que rastejavam sobre a terra;
nos vermes, e ratos, e toupeiras que se arrastavam sob ela; nos peixes
que nadavam e nos insetos que voavam. Vieram por terra e água e ar.

Então, sem obstáculos ou impedimentos, e de muitas formas, os Filhos da
Morte entraram no país Sob o Pôr do Sol; e a partir daquela hora tudo
naquele belo Reino mudou.

Os Filhos da Morte não se mostraram imediatamente. Um a um, os
espíritos mais arrojados entre eles, espreitando o Reino com passadas cruéis, preenchiam todos os corações com terror à medida que
avançaram.

Entretanto, cada um e todos eles deixaram uma lição para o bem nos
corações dos moradores do Reino.