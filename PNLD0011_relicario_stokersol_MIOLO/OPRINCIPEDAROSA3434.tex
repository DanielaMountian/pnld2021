%!TEX root=LIVRO.tex
\chapterspecial{O Príncipe da Rosa}{}{}
 

Há muito, muito tempo -- há tanto tempo que, se alguém tenta pensar tão
longe no passado, é ainda mais longe \mbox{--,} o Rei Mago reinava no País Sob
o Pôr do Sol. Era um rei velho, sua barba branca crescera tanto que
quase tocava o chão. E~todo seu reinado transcorrera no esforço de fazer
seu povo feliz.

Ele tinha um único filho, de quem gostava muito. Esse filho, o Príncipe
Zaphir, era bastante merecedor do afeto de seu pai, pois era tão bom
quanto se pode ser. Ainda era apenas um garoto, e nunca tinha visto
o belo e doce semblante de sua mãe, que morrera quando era apenas um
bebê. Amiúde ficava muito triste por não ter mãe quando pensava que os
outros garotos tinham mães carinhosas em cujos joelhos aprendiam a
rezar, que vinham lhes dar beijos à noite em suas camas. Ele achava
estranho que muitas das pessoas pobres nos domínios de seu pai tivessem
mães, enquanto ele, o príncipe, não tinha. Quando pensava assim,
tornava"-se muito humilde; pois sabia que nenhum poder, ou riqueza, ou
juventude, ou beleza salvaria qualquer pessoa do destino de todos os
mortais, e que a única coisa bela no mundo, cuja beleza dura para
sempre, é uma alma justa e pura. Ele sempre lembrava, entretanto, que,
embora não tivesse mãe, tinha um pai que o amava muito e, assim,
consolava"-se e ficava contente.


\imagemmedia{}{./img/01.png}

Costumava devanear muito sobre diversas coisas; e frequentemente,
até mesmo durante a luminosa Hora do Descanso, quando todas as pessoas
dormiam, ia para o~bosque perto do palácio e pensava e pensava sobre
tudo o que era belo e verdadeiro, enquanto seu fiel cão Gomus se
agachava a seus pés e às vezes balançava a cauda, como que para dizer:

``Aqui estou eu, príncipe. Também não estou dormindo.''

O Príncipe Zaphir era tão bom e amável que nunca machucou qualquer coisa
viva. Se visse à sua frente um verme rastejando na estrada, passava
cuidadosamente acima dele a fim de não o machucar. Se visse uma mosca
caída na água, ele a levantava com cuidado e a soltava, com as asas
livres, no ar claro e glorioso: tão bom era que todos os animais que
uma vez o tinham visto o reconheciam, e, quando se sentava em seu lugar
favorito no bosque, ali surgia um zunzum alegre vindo de todos os
seres vivos. Aqueles insetos brilhantes, cujas cores mudam de hora em
hora, mostravam suas cores mais vivas e se expunham ao cintilar da luz
do sol que penetrava oblíqua entre os galhos das árvores. Os insetos
ruidosos se cobriam com seus abafadores para não perturbá"-lo; e os
passarinhos descansando nas árvores abriam seus olhos redondos e
brilhantes, e saíam, e piscavam os olhos à luz, e assobiavam canções
jubilosas de boas"-vindas com todas as suas notas mais doces.

Acontece assim sempre com pessoas afáveis e amorosas. Os seres vivos que
têm vozes tão doces quanto as de um homem ou de uma mulher e que falam
idiomas próprios, apesar de não os podermos entender, falam a tais
pessoas com notas alegres e lhes desejam boas"-vindas a seu modo
peculiar e belo.

O Rei Mago tinha orgulho de seu garoto corajoso, bom e belo, e gostava
de vê"-lo muito bem vestido. E~todas as pessoas adoravam observar seu
rosto límpido e sua vestimenta vistosa. O~Rei mandou os grandes
mercadores procurarem perto e longe até que conseguissem a maior e mais
fina pena que já fora vista. Ele colocou essa pena na frente de um belo
chapéu, da cor de um rubi, e a fixou com um broche feito de um grande
diamante. Deu esse chapéu a Zaphir em seu aniversário.

Quando o Príncipe Zaphir andava pelas ruas, as pessoas viam de longe
a grande pluma branca acenando. Todos ficavam alegres ao vê"-la
e corriam às janelas e às portas, inclinando"-se em reverência, sorrindo
e agitando as mãos enquanto seu belo príncipe passava. Zaphir sempre se
inclinava e sorria de volta; amava seu povo e se rejubilava com o
amor que tinham por ele.

Na Corte do Rei Mago, Zaphir tinha uma companheira, a qual amava
muito. Era a Princesa Bluebell, filha de outro rei que fora injustamente
privado de seu domínio por um inimigo cruel e traidor, e que havia
procurado o Rei Mago pedindo ajuda, e que veio a morrer em sua Corte depois
de viver ali por muitos, muitos anos. Mas o Rei Mago acolheu sua
filhinha órfã e criou"-a como sua própria filha.

Uma grande vingança havia recaído sobre o usurpador maléfico. Os
Gigantes haviam atacado seus domínios e o assassinaram e toda a sua
família, e mataram todas as pessoas do reino, e destruíram até mesmo
todos os animais, exceto os selvagens, que eram como os próprios
Gigantes. Então as casas começaram a vir abaixo devido à velhice e à
deterioração, e os belos jardins tornaram"-se selvagens e abandonados. E~assim, quando depois de muitos longos anos os Gigantes se cansaram e
voltaram para sua remota terra natal, o país que a Princesa Bluebell
possuía era de uma desolação tão vasta que ninguém que lá entrasse adivinharia
que ali já havia morado gente.

A Princesa Bluebell era muito jovem e muito, muito bela. Ela, como o
Príncipe Zaphir, nunca conhecera o amor materno, pois, quando nova, sua
mãe também morrera. Ela amava muito o Rei Mago, mas amava o Príncipe
Zaphir mais do que todo o resto do mundo. Eles sempre haviam sido
companheiros, e não havia um pensamento sequer no coração dele que ela
não conhecesse quase antes de ele o pensar. O~Príncipe Zaphir amava"-a
também, mais ternamente do que podem dizer as palavras, e por ela
faria qualquer coisa, sem importar o tamanho do perigo. Tinha
esperança de que, quando se tornassem homem e mulher, ela se casaria com
ele, e ambos ajudariam o Rei Mago a reinar em seu domínio justa e
sabiamente, e não haveria dor ou pobreza por todo o país caso
pudessem evitar.

O Rei Mago havia mandado fazer dois pequenos tronos; e quando se
sentava cerimoniosamente em seu grande trono, as duas crianças se
sentavam uma de cada lado e aprendiam como ser Rei e Rainha.

A Princesa Bluebell usava um robe de arminho como o de uma Rainha, e um
pequeno cetro e uma pequena coroa, e o Príncipe Zaphir tinha uma espada,
tão brilhante quanto um raio de luz, pendurada numa bainha dourada.

Os cortesãos costumavam se reunir atrás do trono do Rei. E~muitos deles
eram notáveis e bons, e outros eram somente vaidade e egoísmo.

Havia Phlosbos, o Primeiro Ministro, um homem muito, muito velho com uma
barba longa parecida com seda branca que carregava um bastão branco com
um anel de ouro incrustado nele.

Havia Janisar, o Capitão da Guarda, com bigodes selvagens e uma armadura
pesada como vestimenta.

E depois havia Tufto, um cortesão antigo, um velho tolo que não fazia
nada senão vaguear em torno dos grandes nobres e prestar"-lhes
deferência; e todos, de alto a baixo, desprezavam"-no muito. Era
gordo e não tinha cabelos ou pelos no rosto, nem mesmo as sobrancelhas;
era -- oh!, muito engraçado com sua grande cabeça calva, bem branca e
macia.

Havia Sartorius, um cortesão jovem e tonto, que pensava que a roupa era a
coisa mais importante do mundo, e que por isso se vestia com as melhores
roupas que conseguia comprar. Mas as pessoas apenas riam e até mesmo
gargalhavam dele, pois não há honra que venha de roupas bonitas, mas somente do
que está no próprio homem que as veste. Por toda parte, Sartorius sempre
tentava forçar passagem para estar na frente dos outros e exibir suas
belas roupas, e pensava que, como não tentavam repeli"-lo,
os outros cortesãos reconheciam seu direito de ser o primeiro.
Não era bem assim, no entanto; eles apenas o desprezavam e não fariam o que
ele fazia.

Havia também Skarkrou, que era exatamente o oposto de Sartorius, e que
pensava -- ou fingia pensar -- que a falta de asseio era uma coisa boa.
E~era tão ou mais orgulhoso de seus trapos do que Sartorious o era
de suas belas roupas. Também era desprezado, pois era~vaidoso, e sua
vaidade tornava"-o ridículo.

Então havia Gabbleander, que nada mais fazia além de falar desde a manhã
até a noite, e que falaria da noite até a manhã se conseguisse alguém
para ouvi"-lo. Também riam dele, pois as pessoas não conseguem falar
sempre com juízo se falam demais. As coisas tolas são lembradas, mas as
sábias são esquecidas. E~assim, os tagarelas vêm a ser considerados
tolos.

Mas ninguém deve pensar que toda a Corte do bom Rei Mago era como essas
pessoas. Não! Havia muitas, muitas pessoas boas, e grandiosas, e nobres,
e corajosas. Mas a vida é assim em qualquer país, até mesmo no País Sob
o Pôr do Sol: há tolos tanto quanto sábios, covardes tanto quanto
corajosos e homens maus tanto quanto homens bons.

Crianças que desejam se tornar homens importantes e bons ou mulheres
boas e nobres devem tentar conhecer bem todas as pessoas que encontram.
Assim, perceberão que não há ninguém que não tenha uma parcela de
bondade; e quando virem grandes tolices, ou um pouco de malvadeza, ou um
pouco de covardia, ou algum erro ou fraqueza em outra pessoa, devem
examinar a si mesmas cuidadosamente. Então verão que, talvez, elas
próprias também tenham alguns defeitos -- ainda que eles não se revelem da
mesma forma -- e que devem tentar vencê"-los. Assim, elas se
tornarão melhores à medida que crescem; e os outros as examinarão, e ao
descobrirem que elas não têm defeitos, irão amá"-las e honrá"-las.

Bem, um dia o Rei Mago estava sentado em seu trono com seu manto e sua
coroa, segurando seu cetro na mão.

À sua direita estava sentada a Princesa Bluebell com seu manto, sua coroa e seu
cetro, tendo ao lado seu cãozinho Smg.

Esse cão era de longe o favorito. Antes, fora chamado de Sumog
porque o cão de Zaphir se chamava Gomus, e este era seu nome escrito ao
contrário. Mas então foi nomeado Smg porque esse era um nome que não~se
poderia gritar, mas somente sussurrar. Bluebell não tinha necessidade de
mais do que isso, pois Smg nunca estava longe, e ficava sempre perto de
sua dona e a protegia.

À esquerda do Rei estava sentado o Príncipe Zaphir em seu pequeno trono
com sua espada brilhante e sua imponente pena.

Mago estava fazendo leis para o bem de seu povo. Em volta dele estavam
reunidos todos os cortesãos e muitas pessoas estavam no salão, e outras
muitas lá fora na rua.

De repente, ouviu"-se um ruído alto -- estalos de chicote e sopros de
trombeta --; e o ruído foi se aproximando cada vez mais, e as pessoas na
rua começaram a murmurar. Surgiram gritos altos, o Rei parou para ouvir,
e as pessoas viraram a cabeça para ver quem vinha. A~multidão se abriu,
e um mensageiro de botas e esporas, e coberto de poeira, correu para o
salão e ajoelhou"-se ante ao Rei sobre um dos joelhos, estendendo um
papel que o Rei Mago pegou e leu avidamente. O~povo esperou em silêncio
para ouvir as notícias.

O Rei ficou profundamente tocado, mas como sabia que seu povo estava
ansioso, ficando de pé, disse a todos:

``Meu povo, um grave perigo assalta nosso Reino. Soubemos, por este
despacho da província de Sub"-Tegmine, que um terrível Gigante surgiu dos
pântanos para lá da Terra"-de"-Ninguém e está devastando o país. Mas não
tema, meu povo, pois hoje muitos soldados se apresentarão com suas
armas, e ao pôr do sol de amanhã o Gigante terá sucumbido,
acreditamos.''

As pessoas curvaram suas cabeças com murmúrios de agradecimento, e foram
todos quietos para suas casas.

Naquela noite, um corpo de soldados selecionados saiu com corações
valentes para lutar contra o Gigante, e as pessoas os saudaram pelo
caminho.

\smallskip

Por todo o dia seguinte e a noite seguinte, as pessoas, bem como o Rei,
ficaram muito ansiosas; e na segunda manhã esperaram por notícias que
dissessem que o Gigante havia sido derrotado.

Mas nenhuma notícia chegou até o anoitecer; e então um homem cansado,
coberto com poeira e sangue, e ferido de morte, entrou na cidade
arrastando"-se.

As pessoas abriram caminho, e ele foi para diante do trono, curvou"-se e
disse:

``Ai! Rei, tenho de lhe dizer que seus soldados foram mortos -- todos
exceto eu. O~Gigante triunfa e avança em direção a cidade.''

Tendo dito isso, a dor de seus ferimentos aumentou tanto que gritou
diversas vezes e caiu; e quando o ergueram, estava morto.

Diante da triste notícia que trouxe, surgiu um lamento baixo vindo
do povo. As viúvas dos soldados mortos soltaram um grito alto e breve, e
se dirigiram ao trono do Rei, prostrando"-se diante dele, levantando as
mãos para cima e dizendo:

``Oh, Rei! Oh, Rei!'', e não puderam dizer mais nada por causa do choro.

Então o coração do Rei ficou muito, muito magoado, e ele tentou
consolá"-las, mas seu melhor consolo estava em suas lágrimas -- pois as
lágrimas de amigos ajudam a aliviar os problemas. Falou ao povo,
dizendo:

``Ai! Eram muito poucos os nossos soldados. Hoje à noite enviaremos um
exército, e talvez o Gigante sucumba.''

Naquela noite, um exército aguerrido, com muitas máquinas de guerra, com
bandeiras tremulando e bandas tocando, partiu contra o Gigante.

No comando do exército, Janisar, o capitão, cavalgava ao brilho do pôr do sol com sua armadura de aço incrustada de ouro reluzente. Os adornos
escarlates e alvos de seu grande cavalo de guerra negro mostravam"-se
esplêndidos. Com ele, a alguma distância pelo caminho, cavalgava o
Príncipe Zaphir em seu palafrém branco.

O povo todo se reuniu para desejar ao exército sucesso em sua partida; e
muitas pessoas tolas que acreditavam na sorte atiraram sapatos velhos
depois da passagem deles. Um desses sapatos acertou Sartorius, que
estava como de costume forçando a dianteira para se~exibir,~e deixou seu
olho roxo, e a sujeira preta do sapato sujou sua roupa nova,
estragando"-a. Outro sapato -- pesado, com saltos de ferro -- acertou
Tufto enquanto ele conversava com Janisar bem no topo de sua cabeça
calva, e cortou"-a, e então todos riram.

Imagine como é desprezado um homem de quem as pessoas riem quando se
machuca. O~velho Tufto caiu e ficou com muita raiva, e então as pessoas
riram ainda mais; pois nada é mais engraçado do que uma pessoa que está
com tanta raiva que perde todo o autocontrole.

Todas as pessoas aclamavam à medida que o exército passava. Mesmo as
pobres viúvas dos soldados mortos aclamavam. E~os homens que
partiam olhavam"-nas e decidiam que venceriam ou morreriam como
bravos soldados em serviço.

A Princesa Bluebell subiu com o Rei Mago para o topo da torre do palácio,
e juntos assistiram aos soldados partindo em marcha. O~rei logo se
retirou, mas Bluebell continuou lá, observando os capacetes cintilando e
reluzindo ao poente, até que o sol mergulhou no horizonte.

Bem naquele momento, o Príncipe Zaphir, que havia retornado, juntou"-se a
ela. Então, ao crepúsculo, no topo da torre, com muitos milhares de
corações batendo ávidos e ansiosos na cidade abaixo deles, e com o belo
céu acima, as duas crianças se ajoelharam e rezaram pelo sucesso do
exército na manhã seguinte.

\smallskip
Naquela noite, ninguém dormiu na cidade.

\smallskip
No dia seguinte, as pessoas estavam cheias de ansiedade. E à medida que
o dia gradualmente avançava e as notícias não vinham, ficaram ainda
mais ansiosas.

Ao anoitecer, ouviram ao longe o som de um grande tumulto. Sabiam que a
batalha continuava; e assim esperaram e esperaram por notícias.

Absolutamente ninguém foi se deitar naquela noite; mas, por toda a
cidade, fogueiras de vigília foram acesas e todos ficaram acordados
esperando notícias.

Mas nenhuma notícia veio.

Então o medo se tornou tão grande que os rostos dos homens e das
mulheres ficaram tão brancos e seus corações tão frios quanto a neve.
Por um tempo muito longo ficaram em silêncio, pois pessoa alguma ousava
falar.

Finalmente, uma das viúvas dos soldados mortos levantou"-se e disse:

``Vou me levantar e ir ao campo de batalha para ver o que está
acontecendo lá, e trarei notícias para aquietar vossos pobres e aflitos
corações.''

Então muitos homens ergueram"-se e disseram:

``Não! Não deve ser assim. Nós iremos. Seria uma vergonha para nossa
Cidade se uma mulher fosse a um lugar onde homens não conseguiriam ir.
Nós iremos.''

Mas ela lhes respondeu com um sorriso tristonho:

``Ah! Não tenho medo da morte, já que meu corajoso marido foi morto. Não
desejo mais viver. Vocês devem defender a cidade, eu irei.''

Imediatamente, ela saiu da cidade na manhã cinzenta e fria em direção ao
campo de batalha. À~medida que se afastava e desaparecia na distância,
dava ao povo ansioso a impressão de um fantasma da Esperança evanescendo
diante deles.

O sol nasceu e brilhou nos céus até que a hora do descanso chegou; mas
os homens não se importaram com a hora, sempre vigiando e esperando.

Nesse instante, viram de longe a silhueta de uma mulher correndo.
Dirigiram"-se até ela e descobriram que era a viúva. Ela colocou"-se no
meio deles e gritou:

``Ai! Ai! Ah! O nosso exército está desbaratado. Nossos mais fortes
estão sob o domínio do orgulho de sua força. O~Gigante triunfa e temo
que tudo esteja perdido.''

Do povo se ergueu um grande lamento, e um silêncio recaiu sobre todos, tão
grande era o medo.

Então o Rei reuniu sua Corte inteira e seu povo, e aconselhou"-se quanto
ao melhor a se fazer. Muitos pareciam pensar que um novo exército
deveria partir, um exército formado por todos aqueles que estavam dispostos a
morrer, se necessário, pelo bem do País. Mas havia muita perplexidade.

Enquanto discutiam, o Príncipe Zaphir permaneceu quieto, sentado em seu
trono. E~seus olhos mais de uma vez se encheram de lágrimas diante do
pensamento do sofrimento de seu povo amado. Então, levantou"-se e se pôs
diante do trono.

Houve silêncio até que ele começasse a falar.

Quando o Príncipe se pôs, de chapéu nas mãos, diante do Rei, havia no
rosto dele um olhar de tamanha determinação que aqueles que o
perceberam não puderam deixar de renovar a esperança. O~Príncipe
falou:

``Oh, Rei, Pai, antes que decida algo, escute"-me. É~certo que, se há
perigo no Reino, o primeiro que deve enfrentá"-lo é o Príncipe, em quem o
povo confia. Se há dor a ser sentida, quem deve senti"-la antes do povo? Se
a morte vem a qualquer um, certamente deveria atingir primeiro o seu
cadáver. Rei, Pai, espere somente um dia. Deixe"-me partir amanhã para
enfrentar o Gigante. Esta viúva lhe contou que ele está dormindo agora,
após o combate. Amanhã eu o encontrarei em batalha. Se eu vier a sucumbir,
então será hora de arriscar a vida de seu povo; mas se quem
sucumbir for o Gigante, então tudo estará bem.''

O Rei Mago sabia que o Príncipe havia falado bem, e apesar de afligi"-lo
ver seu amado filho indo ao encontro de tal perigo, não tentou
impedi"-lo, e disse:

\imagemmedia{}{./img/02.png}


``Oh, filho, filho digno de ser rei, falaste bem! Que seja como
queres.''

Então o povo deixou o Salão, e o Rei Mago e a Princesa Bluebell beijaram
Zaphir. Bluebell lhe disse:

``Zaphir, você fez o certo'', e olhou para ele, orgulhosa.

Imediatamente, o príncipe foi deitar"-se, para que pudesse dormir e assim
estar forte para o dia seguinte.

Por toda aquela noite, os ferreiros e os armeiros e os ourives
trabalharam duro e rápido. Até o raiar do dia, as fornalhas brilharam e
as bigornas soaram; e todas as mãos habilidosas nessas artes trabalharam
com esforço.

Pela manhã, levaram ao Salão, e colocaram diante do trono como um
presente ao Príncipe Zaphir uma armadura como antes nunca se vira.

Era trabalhada em aço e ouro, e feita toda com lamelas. Cada lamela era
como uma folha diferente, e era toda polida e brilhante como o sol.
Entre as folhas havia joias, e muitas outras mais estavam presas nelas
como gotas de orvalho. Assim, a armadura cintilava à luz até ofuscar os
olhos de quem a olhasse -- pois os habilidosos armeiros acreditavam que,
quando o Príncipe lutasse, seu inimigo poderia ser parcialmente cegado
com o brilho e, desse modo, errar seus golpes.

O capacete era como uma flor; a insígnia do Príncipe fora fundida
em sua superfície, e a pena e o grande diamante de seu chapéu foram fixados na
frente.

Depois de se paramentar todo, o príncipe parecia tão nobre e corajoso
que o povo aclamou aos gritos que ele venceria; e renovou grandes
esperanças.

Então seu pai, o Rei, abençoou"-o, e a Princesa Bluebell beijou"-o, verteu
algumas lágrimas e deu"-lhe uma graciosa rosa, que ele fixou em seu
capacete.

Entre brados do povo, o Príncipe Zaphir partiu para lutar contra o
Gigante.

Seu cão, Gomus, queria ir com ele, mas não podia ser levado. Então Gomus
se aquietou e uivou, pois sabia que seu querido amo estava em perigo e
desejou estar com ele.

Depois que o Príncipe partiu, a Princesa Bluebell subiu ao topo da torre
e observou"-o até que ele estivesse tão longe a ponto de não mais
poder ver o cintilar de sua bela armadura à luz do sol. Inicialmente,
enquanto se despedia de Zaphir -- e sabia que poderia ser
uma despedida para sempre --, não derramou nenhuma lágrima para não causar
dor a seu amado Príncipe, pois sabia que ele estava rumando para a
batalha e precisaria de toda sua coragem e de toda sua firmeza. Assim, a
última imagem que Zaphir viu no rosto de sua Bluebell foi um sorriso
amável, esperançoso e confiante. Por isso, partiu para a batalha
fortalecido pelo pensamento de que o coração dela o acompanhava, e de
que, embora o corpo dela estivesse longe, seu espírito estava ao seu lado.

Depois de ele já ter se afastado realmente para bem longe da vista,
tendo ela ficado sozinha no topo da torre, Bluebell derramou muitas lágrimas.
E~o grande medo em seu coração de que Zaphir pudesse ser morto deixou"-a
fatalmente triste. Pensou que poderia acontecer de ele ser morto pelo
maléfico Gigante, que já havia destruído dois exércitos, e que, então,
nunca mais iria vê"-lo -- nunca mais veria o amor em seus olhos
queridos e verdadeiros --, nunca mais ouviria os tons de sua voz tenra e
doce --, nunca mais sentiria o bater de seu coração grandioso, generoso.

E então chorou, oh!, muito amargamente. Mas, enquanto chorava,
ocorreu"-lhe o pensamento de que a vida não está em poder dos
homens, ou mesmo dos gigantes; e, assim, enxugou suas lágrimas,
ajoelhou"-se e rezou com o coração humilde, consolando"-se ao se
levantar, assim como as pessoas sempre ficam quando rezam com
sinceridade.

Depois desceu ao grande salão, mas o Rei Mago não estava lá. Ela o
procurava para consolá"-lo, pois sabia que o coração dele devia estar
sofrendo por seu filho em perigo.

Encontrou"-o em seus aposentos, e ele, também, estava rezando.
Ajoelhou"-se ao seu lado, e eles -- o velho Rei e a criança órfã --
colocaram os braços em torno um do outro e rezaram juntos. E~assim ambos
se consolaram.

\imagemmedia{}{./img/03.png}

Juntos, esperaram, e esperaram pacientemente, pelo retorno de seu amado.
Toda a cidade esperou também; e nem de dia, nem de noite houve sono no
País Sob o Pôr do Sol, pois todos estavam aguardando o retorno do
Príncipe.

Quando Zaphir deixou a cidade, rumou sempre adiante em direção ao
Gigante, até o sol brilhar alto nos céus, tão brilhante que sua armadura
dourada reluzia como fogo. E~então andou sob a proteção das árvores, e
não parou nem mesmo na Hora do Descanso, mas continuou sempre em frente.

Ao anoitecer, ouviu e viu coisas estranhas.

Ao longe, o chão parecia tremer, e ecoava um estrondo surdo de rochas
sendo destruídas e de florestas sendo derrubadas. Esses eram os sons dos
passos do Gigante vindo em direção à cidade. Mas o Príncipe Zaphir,
apesar de os sons serem terríveis, não teve medo e avançou bravamente.
Então, começou a encontrar muitas coisas vivas, que passavam por ele a
toda velocidade -- pois eram as mais rápidas de suas espécies e, por
isso, haviam fugido do Gigante antes dos demais.

Elas vinham, em centenas e milhares, sua quantidade aumentando mais e
mais à medida que o tempo passava, e à medida que o Príncipe e o Gigante
se aproximavam.

Lá estavam todos os animais do campo, e todas as aves do ar, e todos os
insetos que voam e rastejam. Leões e tigres, e cavalos e ovelhas, e
ratos e gatos e camundongos, e galos e galinhas, e raposas e gansos e
perus, todos estavam misturados, grandes e pequenos, e todos estavam tão
atemorizados pelo Gigante que se esqueceram de ter medo uns dos outros.
Assim, fugiam juntos, gatos e ratos, lobos e carneiros, raposas e
gansos; os fracos não tinham medo, nem os mais fortes queriam fazer mal
algum.

Entretanto, à medida que vinham, todas as coisas vivas pareciam saber
que o Príncipe Zaphir era mais corajoso do que elas, e abriam caminho
para ela passar. As mais fracas, e as mais atemorizadas, não
continuavam em fuga, e tentavam chegar o mais perto possível do
Príncipe; e muitas preferiam segui"-lo, retornando em direção ao Gigante,
a não ficarem perto dele.

Mais adiante, depois de um tempo, encontrou todos os animais velhos
que não conseguiam fugir tão rápido quanto os demais, e todos os pobres seres
vivos feridos, e todos aqueles que eram lentos. Esses, também, não
tentaram ir mais longe, pois sabiam que estariam mais seguros perto de
um homem corajoso do que em fuga desamparada.

Então o Príncipe Zaphir viu algo, ainda muito longe, que parecia uma
portentosa montanha.

Estava se movendo em sua direção, e seu coração bateu alto, parte com o
pensamento na batalha vindoura, parte com esperança.

O Gigante aproximava"-se cada vez mais. Seus passos esmagavam as rochas,
e com sua poderosa clava varria as florestas de seu caminho.

As criaturas vivas atrás do Príncipe Zaphir tremeram de medo e
esconderam suas caras na poeira. Alguns animais, como algumas pessoas
tolas, pensam que se não veem algo que não desejam ver, aquilo deixa então
de existir.

Muito tolo da parte deles.

Então, com o Gigante já próximo, o Príncipe Zaphir sentiu que era
chegada a hora da batalha.


\imagemmedia{}{./img/04.png}

Quando ficou cara a cara com o inimigo mais poderoso do que qualquer
coisa que já tinha visto, Zaphir sentiu"-se como nunca se sentira antes. Não é que
estivesse com medo do Gigante, pois se sentia com tanta coragem que,
pelo bem de seu povo, poderia vir a morrer alegremente da forma mais
dolorosa. É~que havia se dado conta de que coisa pequenina ele era
naquele mundo tão vasto.

Viu, mais claramente do que jamais havia visto, que era apenas um ponto
-- um mero átomo -- em meio ao enorme mundo vivo; e, num instante, percebeu que,
se a vitória lhe coubesse, não seria porque seu braço era forte ou seu
coração valente, mas porque foi desejada por Aquele que governa o
universo.

Então, humildemente, o Príncipe Zaphir rezou pedindo forças.
Despiu sua esplêndida armadura, que brilhava como um sol na terra,
tirou o esplêndido capacete e deitou ao lado a rutilante espada; e tudo
ficou a seu lado como um amontoado de coisas sem vida.

Era uma bela visão a daquele jovem garoto ajoelhado ao lado da armadura
descartada. O~amontoado brilhante era belíssimo, cintilando no claro pôr
do sol com milhões de lampejos coloridos, chegando até mesmo a parecer
uma coisa viva. No entanto, parecia triste, miserável e desprezível ao lado
do rapaz, que ali se ajoelhou e rezou humildemente, com seus olhos
profundamente sérios, acesos pela verdade e pela confiança que jaziam em
seu coração limpo e em sua alma pura.

A armadura reluzente parecia o trabalho das mãos do homem -- como o era de fato,
o trabalho das mãos de homens bons e verdadeiros. Mas o belo garoto,
ajoelhado em confiança e com fé, era o trabalho das mãos de Deus.

Enquanto rezava, o Príncipe Zaphir reviu toda sua vida passada, desde o
primeiro dia de que conseguia se lembrar até aquele exato momento, face
a face com o Gigante. Não houve nenhum pensamento indigno que tivesse
tido, e nenhuma palavra rude que tivesse dito, e nenhum olhar colérico
que tivesse provocado dor em outra pessoa que não tenha voltado à sua
mente. Muito o afligiu haver tantos, pois se amontoavam tão rápida e
abundantemente que ficou espantado justamente com a quantidade.

É sempre assim, as coisas erradas que fazemos -- ainda que possam
parecer pequenas no momento, e ainda que as ignoremos por causa da
dureza de nossos corações -- retornam a nós com amargura quando o
perigo nos leva a pensar no pouco que fizemos para merecer ajuda e no
muito que fizemos para merecer punição.

O coração do Príncipe Zaphir foi purificado pelo arrependimento de todas
as coisas erradas feitas no passado, e pela sublime resolução de ser bom
no futuro. E~quando sua humilde reza terminou, ele levantou"-se e sentiu
em seus braços uma força que não conhecia. Sabia que não era a sua
própria força, mas que ele era o humilde instrumento da salvação de seu
amado povo. E,~em seu coração, ficou agradecido.

O Gigante logo viu o brilho da áurea armadura, e percebeu que mais um
inimigo se aproximava.

Deu um rugido estrondoso de raiva e fúria, que soou como o eco de um
trovão. Pelas colinas distantes o som ecoou, ribombou através dos vales
ao longe e dissipou"-se em resmungos e rosnados baixos, como os de
animais selvagens em cavernas e rochedos montanhosos.

Era sempre com esse estrondo que o Gigante começava suas lutas, a fim de
atemorizar seus inimigos. Mas o coração valente do Príncipe não tremeu
de medo. Ele ficou mais valente do que nunca quando ouviu o barulho;
pois sabia que era preciso ter ainda mais coragem para que seu povo, e
até mesmo o Rei, seu pai, e Bluebell, não caíssem sob o poder do Gigante.

Enquanto entre as pedras e as florestas as passadas do Gigante se embatiam,
e enquanto subia em volta de seus pés o pó da desolação que ela causavam, o
Príncipe Zaphir juntou do riacho alguns seixos arredondados.

Ele encaixou um deles na funda que carregava.

Assim que levantou seu braço para rodopiar a funda em volta de sua
cabeça, o Gigante o viu, riu e apontou desdenhoso em sua direção com
suas grandes mãos, que eram mais brutas do que as garras de tigres. A~risada que~o Gigante trovejou era tão terrível -- tão rude e raivosa e
medonha que as coisas vivas que haviam levantado os tímidos olhos para
observar a luta enterraram novamente as cabeças na terra, e tremeram de
medo.

Mas justo quando o Gigante riu para escarnecer de seu inimigo, sua
perdição foi proferida.

A funda girou em volta da cabeça do Príncipe Zaphir, e o seixo sibilante
voou. Acertou bem na têmpora do Gigante, e foi com a risada de escárnio
em seus lábios, e com sua mão estendida apontando com menosprezo, que
ele caiu de bruços.

Enquanto caía, emitiu um único grito, mas um grito tão alto que
percorreu as colinas e os vales como o estrondo de um trovão. Em meio ao
som, as coisas vivas novamente se acovardaram e fraquejaram de medo.

\smallskip
Ao longe, as pessoas da cidade ouviram o poderoso som, mas não
sabiam o que significava.

Quando o grande corpo do Gigante caiu de bruços, a terra tremeu por
muitas milhas ao seu redor devido ao choque. E~quando sua grande clava caiu
de sua mão, derrubou muitas árvores altas da floresta.

Então, o Príncipe Zaphir caiu de joelhos e rezou com gratidão fervorosa
por sua vitória.

Rapidamente se levantou e, porque sabia da amarga ansiedade do Rei e do
povo, nem parou para recolher sua armadura, mas dirigiu"-se rápido para a
cidade a fim de levar as boas"-novas.

A noite havia caído agora e o caminho estava escuro; mas o Príncipe
Zaphir tinha confiança e seguiu adiante pela escuridão, com coração
valente e esperançoso.

Logo, as coisas vivas que eram nobres circundaram"-no com gratidão, e
todos que puderam o seguiram de perto. Havia muitos animais nobres --
leões e tigres e ursos, bem como animais domésticos. E~seus grandes
olhos fogosos pareciam lampiões e ajudaram"-no em seu caminho.

Entretanto, à medida que se aproximaram da Cidade, os animais selvagens
começaram a se retrair, pois, apesar de confiarem em Zaphir, temiam os
outros homens. Rosnaram um pequeno rosnado de pesar e pararam, e o
Príncipe Zaphir continuou sozinho.

Por toda a noite a cidade permanecera acordada. Na corte, o Rei Mago e a
Princesa Bluebell esperavam e vigiavam juntos, as mãos dadas. O~povo nas
ruas se sentou em volta de suas fogueiras de vigília, e as pessoas só
ousavam falar em sussurros.

Assim, a longa noite passou.

Por fim, o céu do oriente começou a clarear; e então uma risca de fogo
rubro disparou pelo horizonte e o sol nasceu em sua glória. E~assim
fez"-se dia. O~povo, quando viu a luz e ouviu o cantar revigorado dos
pássaros, teve esperança. E~aguardou ansiosamente pela vinda do
Príncipe.

Nem o Rei Mago, nem a Princesa Bluebell ousaram subir para o alto da
torre; e esperaram pacientemente no salão. Seus rostos estavam pálidos
como a morte.

As sentinelas da cidade e aqueles que se juntaram a elas observavam a
longa estrada, esperando ver em algum momento a armadura áurea do
Príncipe Zaphir reluzindo à luz esfuziante da manhã e sua grande pluma
branca, que conheciam muito bem, acenando à brisa. Sabiam que poderiam
vê"-la de longe e, por isso, de vez em quando davam só uma olhada para o
horizonte.

De repente, houve brados de todas as pessoas -- e depois uma quietude
repentina.

Todos se levantaram, e esperaram por notícias.

Pois, oh!, que alegria!, lá, entre eles -- despojado de sua armadura
brilhante e de sua pluma que acenava, mas cheio de viço -- estava seu
amado Príncipe.

Havia vitória em seu olhar.

Ele sorriu para o povo, levantando as mãos como que abençoando, e apontou
para o palácio do Rei, como para dizer:

``Nosso Rei! Ele tem o direito de ouvir as mais novas notícias.''

Passou e foi entrando no salão, todas as pessoas a segui"-lo.

\smallskip
Quando o Rei Mago e a Princesa Bluebell ouviram o brado e sentiram a
quietude que se seguiu, seus corações começaram a bater forte e
aguardaram muito apreensivos.

A Princesa Bluebell sentiu um calafrio e chorou um pouco, aproximou"-se
do Rei e apoiou seu rosto em seu peito.

Enquanto apoiava e escondia seu rosto junto ao Rei, sentiu"-o
sobressaltar. Ela rapidamente levantou os olhos, e ali -- oh!, alegria
das alegrias! -- estava seu amado Zaphir entrando no saguão, com todo o
povo a segui"-lo. O~Rei desceu de seu trono e tomou"-o nos braços,
beijando"-o; Bluebell também colocou seus braços em torno dele e o beijou
na boca.

O Príncipe Zaphir pôs"-se a falar e disse:

``Oh!, Rei, meu Pai, e oh!, Povo! -- Deus foi bom para conosco e Seu
braço deu"-nos a vitória. Veja! O Gigante sucumbiu no orgulho de sua
força!''

Então, do meio do povo surgiu um tamanho brado que o teto tremeu
e o barulho percorreu toda a Cidade nas asas do vento. A~multidão contente bradou mais e mais, até que o som transbordou em ondas
por todo o Domínio, e em Sob o Pôr do Sol, naquela hora, nada houve senão
alegria. O~Rei chamou Zaphir de seu Filho Valente, e a Princesa Bluebell
beijou"-o novamente, chamando"-o de seu Herói.

Naquele mesmo momento, lá longe na floresta, o Gigante jazia sucumbido
pelo orgulho de sua força -- a coisa mais vil de todo o mundo --, e
sobre seu cadáver corriam raposas e arminhos. As cobras rastejavam em
torno de seu corpo; e ali, também, arrastavam"-se todos os piores seres
vivos que haviam fugido dele quando ele estava vivo.

Vindos de longe, os abutres se reuniram ao redor de sua presa.

Perto do Gigante abatido, brilhando na luz, jazia a armadura áurea. A~grande pluma branca erguia"-se do capacete e continuava a acenar na
brisa.

Quando o povo saiu para ver o Gigante morto, descobriu que ervas
daninhas já haviam crescido onde seu sangue tinha escorrido, mas também
que em volta da armadura que o Príncipe despiu havia crescido um anel de
graciosas flores. A~mais bela de todas era uma roseira em
flor, pois a rosa que a Princesa Bluebell tinha lhe dado havia criado
raízes e florescido novamente, formando uma coroa de rosas vivas em
volta do capacete e inclinando"-se para a haste da pluma.

Então o povo levou de volta, respeitosamente, a armadura dourada; o
Príncipe Zaphir, porém, disse que não fora tal armadura, mas sim um
coração verdadeiro, a melhor proteção, e que não ousaria vesti"-la
novamente.

Então a penduraram na Catedral entre as grandes bandeiras antigas e os
capacetes dos cavaleiros de outrora, como um memorial da vitória sobre o
Gigante.

O Príncipe Zaphir tirou do capacete a pena que o Rei, seu pai, havia
antes lhe dado e usou"-a novamente em seu chapéu. A~rosa que florescera
foi plantada no centro do jardim do palácio, e cresceu tanto que muitas
pessoas podiam se sentar debaixo dela, abrigando"-se do sol devido à abundância de
suas flores.


\imagemmedia{}{./img/05.png}
Quando o aniversário do Príncipe Zaphir chegou, o povo já havia feito,
em segredo, grandes preparos.

Quando se levantou de manhã para ir à Catedral, todo o povo havia se
reunido e formado uma fila de cada lado do caminho. Toda pessoa, velha e
nova, segurava~uma rosa. Aqueles que tinham muitas rosas trouxeram uma
para quem não tinha; e cada pessoa tinha somente uma rosa para que todos
pudessem ser iguais aos olhos do Príncipe que amavam. Haviam
retirado todos os espinhos dos caules para que os pés do Príncipe não
fossem machucados. À~medida que ele passava, o povo jogava suas rosas no
caminho, até que toda a longa rua tornou"-se um tapete de flores.

Depois que o Príncipe passava, as pessoas se inclinavam e recolhiam as rosas
que seus pés haviam tocado, tendo"-as em grande estima.

Durante toda a vida, a cada aniversário do Príncipe, o povo repetia a cerimônia. Quando Zaphir e Bluebell se casaram, cobriram o caminho deles com
rosas da mesma forma, pois o povo os amava muito.

Por muito tempo e felizes viveram O Príncipe da Rosa -- pois assim o
chamaram -- e sua bela esposa, a Princesa Bluebell.

Quando, chegando ao termo de seus dias, o Rei Mago faleceu -- pois todos
os homens falecem --, eles reinaram como Rei e Rainha. Reinaram com
justiça e altruísmo, sempre renunciando aos próprios interesses e
lutando para deixar as pessoas boas e felizes.

Eles foram abençoados pela paz.
