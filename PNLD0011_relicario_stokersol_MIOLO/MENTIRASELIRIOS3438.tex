%!TEX root=LIVRO.tex
\chapterspecial{Mentiras e lírios}{}{}
 

\letra{C}{laribel} vivia em paz e feliz com seu pai e sua mãe desde o tempo em que
era um bebezinho até quando, aos dez anos, foi para a escola.

Seus pais eram pessoas boas e adoráveis, que amavam a verdade e tentavam
sempre andar no caminho dos justos. Ensinaram a Claribel todas as
coisas boas, e sua mãe, Fridolina, costumava levá"-la todo dia quando ia
visitar e consolar os doentes.

Quando Claribel foi para a escola, ficou ainda mais feliz, pois não
somente tinha sua casa como sempre a tivera, mas também tinha muitos
amigos novos que eram da sua idade e os quais viria a conhecer e a amar.
A~professora era muito boa e muito gentil e muito velha, com um belo
cabelo branco e um rosto doce e gentil que nunca parecia duro ou sério,
exceto quando alguém contava uma mentira. Então, o sorriso desaparecia
de seu rosto; e era como a mudança no céu quando o sol se punha, e então
ficava séria e chorava silenciosamente. Se a criança que tinha sido
malvada confessasse o erro e prometesse nunca mais contar uma mentira, o
sorriso retornaria como a luz do sol. Mas se a criança insistisse na
mentira, seu rosto se tornava sério, e depois esse olhar sério ficava na
memória do mentiroso, mesmo quando ela não estivesse presente.

Todo dia ela contava a todas as crianças sobre a beleza da Verdade e como
uma mentira era uma coisa muito obscura e terrível. Também lhes
contava histórias do Belo Livro; uma que ela amava, e que eles amavam
também, era sobre a Bela Cidade onde as pessoas boas vão viver depois
daqui.

As crianças nunca se cansavam de ouvir sobre aquela Cidade, límpida como
cristal de jaspe, com seus doze portões com nomes escritos neles; e
faziam perguntas à Professora sobre o Anjo que mediu a Cidade com um
junco dourado. Sempre perto do fim da história, a voz da Professora se
tornava muito séria, e um silêncio se entranhava nas crianças, e elas
ficavam mais próximas umas das outras, espantadas, quando ela lhes contava
que, fora daquela bela cidade,
``todo aquele que adorava contar uma mentira'' era condenado a ficar de pé para sempre.

Então, a boa Professora lhes contava que coisa terrível seria ficar
ali fora, e perder toda a beleza e a glória eterna que havia lá dentro.
E~tudo por um erro que nenhum ser humano precisava cometer -- contar uma
mentira. As pessoas não ficam muito bravas, mesmo se um erro for
cometido, se a verdade for contada de uma vez; mas se um erro fosse
piorado por uma mentira, então todo mundo ficava bravo com razão. Se
homens e mulheres, até mesmo pais e mães que amam seus filhinhos com
muito carinho, ficam bravos, o quanto mais bravo vai ficar Deus contra
quem se comete o pecado da mentira?

Claribel amava essa história e muitas vezes chorava quando pensava nas
pobres pessoas que tinham de ficar fora da Bela Cidade para sempre, mas
nunca pensou que ela mesma iria contar uma mentira. Na verdade,
nunca pensou, até que veio a tentação. Quando as pessoas pensam muito
bem de si mesmas, perigam cometer um pecado, pois, se não ficarmos
sempre atentos para o mal, certamente faremos algo errado; e porque
Claribel não temia mal algum, era facilmente levada ao pecado.

As crianças estavam todas envolvidas com seus problemas de matemática. Algumas
delas sabiam a aritmética, conseguiam suas respostas e provavam"-nas; mas
algumas não conseguiam a resposta certa, e outras empacavam e não
conseguiam resposta alguma. Algumas crianças levadas nem mesmo tentavam
chegar nas respostas, mas faziam desenhos em suas lousas e escreviam
seus nomes. Claribel tentou resolver seus problemas, mas não
conseguia lembrar 9 vezes 7, e ao invés de começar em ``duas vezes um
são dois'' e ir aumentando, ficou sem vontade e com preguiça e
desistiu do problema, e começou a fazer desenhos e desistiu deles também.
Olhou para a janela pensando em algo para desenhar e viu nos vidros
de baixo flores coloridas pintadas para impedir que as crianças olhassem
para as pessoas lá fora durante as lições. Claribel olhou fixo para uma
dessas flores, um lírio, e começou a desenhá"-lo.



Skooro a viu olhando e começou seu trabalho maléfico. Para ajudá"-la a
fazer o que não devia, ele tomou a forma de um lírio e se colocou com
formas muito apagadas na lousa, de modo que ela tinha somente de desenhar
em volta de seus contornos, e então desenharia um lírio. Bem, não
é errado desenhar um lírio, e se Claribel o tivesse desenhado na hora
certa, teria sido elogiada; mas uma coisa boa pode se tornar uma
coisa má se for feita de modo errado -- e assim era com o lírio de
Claribel.


Depois de um instante, a Professora pediu as lousas. Quando Claribel
trouxe a dela, sabia que tinha feito algo errado e estava arrependida; mas só
estava arrependida porque estava com medo de ser punida. Quando a
Professora pediu as respostas, ela baixou a cabeça e disse que não
tinha conseguido.

\imagemmedia{}{./img/14.png}

``Você tentou?'', perguntou a Professora.

``Sim'', ela respondeu, sentindo que tinha tentado por um tempo.

``Ficou com preguiça?'', perguntaram"-lhe. ``Você fez alguma coisa além de
seus problemas?'' Então ela percebeu que, se contasse, teria problemas por
ter ficado com preguiça; e, então, esquecendo tudo sobre a Cidade de Jaspe
e aqueles que estão condenados a ficar fora de seus belos portões,
respondeu que não tinha feito mais nada a não ser os problemas. A~professora acreditou em sua palavra -- pois ela sempre fora sincera
-- e disse:

``Você ficou confusa, suponho, minha querida. Deixe"-me ajudá"-la'', e
gentilmente lhe mostrou como resolver o problema.

Quando estava voltando para sua carteira, Claribel abaixou sua cabeça,
pois sabia que havia contado uma mentira, e, apesar de nunca
precisar ser descoberta, ficou triste e se sentiu como se estivesse do
lado de fora da Cidade cintilante. Mesmo naquele momento, se ela tivesse
corrido para a professora e tivesse dito: ``Eu errei; mas serei de novo
uma criança melhor'', tudo ficaria bem; mas ela não o fez, e a todo
minuto que passava isso se tornou mais difícil de fazer.

Logo depois a aula terminou, e Claribel foi triste para casa. Ela não se
interessou em brincar, pois havia contado uma mentira, e seu coração
estava pesaroso.

Quando chegou a hora de dormir, ela deitou"-se cansada, mas não conseguiu
dormir; e chorou amargamente, pois não conseguiu rezar. Estava
arrependida de ter contado uma mentira, e achou bem difícil o fato de que
sua aflição não era suficiente para deixá"-la novamente feliz. Mas sua
consciência disse: ``vai confessar amanhã?'', mas pensou que não
seria necessário, pois o pecado havia chegado ao fim e ela não havia
feito mal a ninguém. Mas todo o tempo ela sabia que havia feito algo errado.
Tivesse a professora falado sobre isso, teria dito:``É sempre assim,
querida. Um pecado não pode ser expiado até que a vergonha tenha vindo
primeiro; pois, sem a vergonha e o reconhecimento da culpa, o coração não
pode ficar limpo de pecados''.

Finalmente, Claribel chorou até dormir.

Então, quando dormiu, a Criança Anjo entrou furtivamente no quarto e
passou acima de suas pálpebras, de modo que até mesmo em seu sono ela
vira a bela luz, e pensou sobre a Cidade como uma pedra jaspe, límpida
como cristal, com seus doze portões com nomes escritos nele. Sonhou que
vira o Anjo com o junco dourado medindo a cidade, e Claribel ficou tão
feliz que se esqueceu totalmente de seu pecado. A~Criança Anjo conhecia
todos os pensamentos dela, e ficou menor e menor até que toda a sua luz
se extinguiu. E~para Claribel, em seu sonho, tudo pareceu
escurecer, e percebeu que estava de pé do lado de fora do portão da
Bela Cidade. O~Anjo, que segurava o junco dourado de medir, estava nas
ameias da cidade, e, com uma voz terrível, disse:

``Claribel, ficai no lado de fora; vós adorais contar uma mentira.''

``Oh, não'', disse Claribel, ``Não a adoro.''

``Então por que não confessais vosso erro?''

Claribel calou"-se. Mas não iria confessar seu pecado, pois seu
coração estava firme; o Anjo levantou seu junco dourado e, veja!, brotou
um belo lírio. Então, o Anjo disse:

``Os lírios crescem somente para os puros, que vivem dentro da cidade;
vós deveis ficar aqui fora entre os mentirosos.''

Claribel viu as paredes jaspe diante de si se elevando cada vez mais
alto, e soube que eram uma barreira eterna e que deveria
ficar do lado de fora da Bela Cidade para sempre. E, em angústia e horror,
sentiu o quão profundo fora seu pecado, e desejou confessá"-lo.

Skooro viu que ela estava se arrependendo, pois ele, também, podia ver
seus pensamentos, e com a escuridão de sua presença tentou apagar todo o
sonho da Bela Cidade.

Mas a Criança Anjo infiltrou"-se em seu coração e deixou"-o leve; a
semente da penitência cresceu e floresceu.

Claribel acordou cedo, levantou"-se, e contou à sua professora seu
pecado, e ficou feliz mais uma vez.

Por toda sua vida ela amou os lírios, pois refletira sobre sua mentira e
a penitência que fizera por causa dela, e que os lírios crescem dentro da Cidade
Jaspe, que é somente para os puros.


