\chapterspecial{O Gigante Invisível}{}{}
 

O tempo segue em frente no País Sob o Pôr do Sol tanto quanto aqui.

Muitos anos se passaram: e eles acarretaram muitas mudanças. E~agora,
encontramo"-nos em uma época em que as pessoas que viveram no tempo do
bom Rei Mago dificilmente teriam reconhecido seu próprio Reino se o
tivessem visto novamente.

De fato, ele tristemente mudou. Não havia mais o mesmo amor ou a mesma
reverência em relação ao rei -- não havia mais a paz perfeita. As
pessoas haviam se tornado mais egoístas e mais gananciosas, e tentavam
tomar tudo o que podiam para si mesmas. Havia poucos muito ricos e havia
muitos pobres. A~maioria dos belos jardins havia sido devastada. Casas
haviam sido erguidas próximas ao entorno do palácio, e em algumas delas
viviam muitas pessoas que somente podiam pagar por parte de uma casa.

Todo o belo País havia tristemente mudado, e mudada foi a vida dos
moradores dele. O~povo tinha quase se esquecido do Príncipe Zaphir, que
morrera há muitos, muitos anos; e rosas não foram mais espalhadas pelos
caminhos. Aqueles que viviam agora no País Sob o Pôr do Sol riam da
ideia de outros Gigantes, e eles não os temiam porque não os tinham
visto. Alguns deles diziam:

``Ora! O que há para temer? Mesmo se tivessem existido gigantes, eles já
não existem mais''.

E assim as pessoas cantavam e dançavam e banqueteavam como antes, e
somente pensavam em si mesmas. Os Espíritos que vigiavam o Reino estavam
muito, muito tristes. Suas grandes asas quiméricas definhavam quando
estavam em seus postos nos Portais do Reino. Eles escondiam seus rostos,
e seus olhos estavam turvos pelo choro constante, de modo que eles não
cuidavam se qualquer coisa má passasse por eles. Tentaram fazer as
pessoas pensar em seus atos maléficos, mas não podiam deixar seus
postos, e as pessoas ouviam seus gemidos de dor na estação da noite e
diziam:

``Ouça o suspirar da brisa; que doce!''

Assim é também sempre conosco: quando ouvimos o vento suspirando e
gemendo e choramingando em volta de nossas casas em noites solitárias
não pensamos que nossos Anjos possam estar se lamentando por nossas
maldades, mas somente pensamos que uma tempestade está vindo. Os Anjos
choravam sempre, e sentiam a tristeza da mudez -- pois, apesar de
poderem falar, aqueles a quem falavam não queriam ouvir.

Enquanto o povo ria diante da ideia de Gigantes, havia um velho que
balançava a cabeça e replicava"-lhes, quando os ouvia, dizendo:

``A Morte tem muitos filhos, e ainda há Gigantes nos pântanos. Vocês
podem não os ver, talvez --, mas eles estão lá, e o único reduto de
segurança está em uma terra de corações pacientes e leais''.

O nome desse bom velho era Knoal, e ele vivia em uma casa construída com
grandes blocos de pedra no meio de um local selvagem longe da cidade.

Na cidade havia muitas casas velhas e grandes, com vários andares; e
nessas casas viviam muitas pessoas pobres. Quanto mais alto você subia
as grandes escadas íngremes, mais pobres eram as pessoas que ali viviam,
de forma que nos sótãos havia pessoas tão pobres que, quando a manhã
vinha, eles não sabiam se teriam algo para comer durante o dia todo.
Isso era muito, muito triste, e crianças boas teriam chorado se tivessem
visto a dor deles.

Em um desses sótãos vivia solitária uma mocinha chamada Zaya. Ela era
uma órfã, pois seu pai havia falecido há muitos anos e sua pobre mãe,
que havia trabalhado exaustivamente por muito tempo para sua querida
filhinha -- sua única filha --, havia morrido não fazia muito.

A pobrezinha Zaya chorou muito amargamente quando viu sua querida mãe
morta, e ela tinha ficado tão triste e desconsolada por tanto tempo que
se esquecera de não ter meios para viver. Entretanto, as pessoas pobres
que viviam na casa haviam lhe dado parte de sua própria comida para que
ela não passasse fome.

Então, depois de um tempo, ela tentara trabalhar por si mesma e ganhar o
próprio sustento. Sua mãe havia lhe ensinado a fazer flores de papel;
então ela fez um monte de flores e, quando juntou uma cesta inteira
delas, saiu para as ruas e as vendeu. Ela fazia flores de vários tipos,
rosas e lírios, violetas, fura"-neves, prímulas, resedas e muitas outras
flores belas que somente florescem no País Sob o Pôr do Sol. Algumas
delas ela podia fazer sem qualquer modelo, mas outras ela não conseguia;
assim, quando ela queria um modelo, tomava seu maço de folhas, tesouras,
cola, pincéis e todas as coisas que usava e ia para o jardim, do qual
uma mulher era a dona, onde cresciam muitas flores belas. Ali ela se
sentava e trabalhava, observando as flores que queria.

Algumas vezes ela ficava muito triste, e suas lágrimas caíam espessas e
rápidas quando pensava em sua pobre mãe falecida. Ela parecia amiúde
sentir que sua mãe a estava observando a fim de olhar seu sorriso tenro
à luz do sol refletido na água; então seu coração se alegrava e ela
cantava tão docemente que os pássaros a rodeavam e paravam seus próprios
cantos para escutá"-la.

Ela e os pássaros se tornaram grandes amigos, e algumas vezes, quando
ela cantava uma música, eles todos entoavam juntos, enquanto se sentavam
em volta dela em círculo, algumas notas que pareciam claramente dizer:

``Cante para nós de novo. Cante para nós de novo''.

Então ela cantava de novo. Depois, pedia que eles cantassem, e eles
cantavam até que houvesse um concerto. Depois de um tempo, os pássaros
conheciam"-na tão bem que iam a seu quarto, e faziam ali até mesmo seus
ninhos e seguiam"-na para onde ela fosse.

As pessoas costumavam dizer:

``Olhe a menina com os pássaros. Ela mesma deve ser meio pássaro, pois
olhe como os pássaros a amam e a conhecem''.De tantas pessoas virem e
dizerem coisas como essa, alguns indivíduos tolos realmente acreditavam
que ela era meio pássaro e balançavam a cabeça quando pessoas
inteligentes riam delas, dizendo:

``De fato, ela deve ser. Ouçam ela cantando: sua voz é mais doce do que
até mesmo os pássaros''.

Então lhe foi dado um apelido; e garotos levados chamavam"-na assim na
rua. E~o apelido era ``Passarona''. Mas Zaya não se importava com a
alcunha; e, apesar de frequentemente os garotos levados assim a
chamarem, pretendendo lhe causar sofrimento, ela não desgostava; ao
contrário, pois ela se glorificava tanto no amor e na confiança de seus
amigos de voz doce que ela gostaria que a comparassem a eles.

De fato, seria bom para os garotinhos e as garotinhas levados se fossem
tão bons e inofensivos quanto os passarinhos que trabalham o dia todo
para seus filhotinhos indefesos, construindo ninhos e trazendo comida, e
sentando"-se pacientemente a chocar seus pequenos ovos manchados.

Em uma noite, Zaya estava sentada sozinha em seu sótão muito triste e
desolada. Era uma noite muito agradável de verão, e ela estava sentada
na janela, observando a cidade. Ela podia ver muitas ruas que iam em
direção à grande catedral, cuja espira se erguia ao céu muito mais alta
que a grande torre do palácio do rei. Quase não havia sopro de vento, e
a fumaça subia diretamente das chaminés, tornando"-se cada vez mais
indistinta até que desaparecia completamente.

Zaya estava muito triste. Pela primeira vez em muitos dias, seus
pássaros estavam todos longe dela, e ela não sabia aonde eles tinham
ido. Parecia"-lhe como se eles a houvessem abandonado; e ela estava tão
sozinha, pobrezinha, que derramou lágrimas amargas. Estava pensando na
história que há muito tempo sua falecida mãe havia lhe contado de como o
Príncipe Zaphir havia matado o Gigante, e ela imaginou como era o
príncipe, e pensou o quanto alegres as pessoas devem ter sido quando
Zaphir e Bluebell eram rei e rainha. Então imaginou se havia crianças
famintas naqueles tempos bons, e se, de fato, como as pessoas diziam, não
mais havia Gigantes. Então ela pensou e pensou enquanto continuava seu
trabalho em frente à janela aberta.

De repente, ela desviou o olhar de seu trabalho e fitou o outro lado da
cidade. Lá ela viu uma coisa terrível -- algo tão terrível que ela
emitiu um gritinho de medo e de espanto, e debruçou"-se na janela,
fazendo sombra aos olhos com sua mão para ver mais claramente.

No céu para além da cidade ela viu uma vasta Forma de sombra com seus
braços erguidos. Ela estava enrolada em um grande robe de névoas que a
cobria, desvanecendo"-se no ar de modo que a menina podia apenas ver o
rosto e as mãos espectrais e macabras.

A Forma era tão portentosa que a cidade abaixo dela parecia um brinquedo
de criança. Estava ainda longe da cidade.

O coração da garotinha pareceu ficar paralisado de medo quando pensou:
``Os Gigantes, então, não estão mortos. Esse é outro deles''.

Ela desceu correndo rapidamente as escadas altas e saiu para a rua. Ali
ela viu algumas pessoas e gritou a elas:

``Olhem! Olhem! O Gigante, o Gigante!'', e apontou em direção à Forma
que ela ainda via se movendo lentamente na direção da cidade.

As pessoas olharam para cima, mas não podiam ver coisa alguma; eles
então riram e disseram:

``Essa criança está louca''.

Então a pobrezinha Zaya ficou assustada mais do que nunca, e correu pela
rua ainda gritando:

``Olhem! Olhem! O Gigante, o Gigante!'' Mas ninguém lhe prestou atenção
e todos disseram: ``Essa criança está louca'', e continuavam em seus
caminhos.

Então, os garotos levados se aproximaram dela e bramaram:

``A Passarona perdeu seus colegas. Ela vê um pássaro maior no céu e ela
o quer''. E~faziam rimas sobre ela, cantando"-as enquanto dançavam em
círculo.

Zaya fugiu deles; correu apressada pelo meio da cidade, indo para o
campo fora dela, pois ainda via a grande Forma no ar diante de si.

À medida que avançava, e aproximava"-se mais e mais do Gigante, ele se
tornava um pouco mais escuro. Ela podia apenas enxergar as nuvens, mas a
forma de um Gigante pendendo vagamente no ar ainda era visível.

Uma névoa fria rodeou"-a quando o Gigante pareceu vir em sua direção.
Então, pensou em todas as pessoas pobres na cidade, e teve esperança de
que o Gigante as poupasse; ajoelhou"-se diante dele, ergueu suas mãos em
súplica e gritou:

``Oh, grande Gigante! Poupe"-as, poupe"-as!''

Mas o Gigante seguia em frente como se não a tivesse escutado. Ela
gritou ainda mais alto:

``Oh, grande Gigante! Poupe"-as, poupe"-as!'' E curvou sua cabeça e
chorou; o Gigante, apesar de bem lentamente, ainda continuava a avançar
para a cidade.

Não longe, havia um velho parado em pé na porta de uma pequena casa
construída com grandes pedras, mas a menina não o viu. Seu rosto
mostrava um olhar de medo e de espanto, e quando ele viu a criança se
ajoelhar e erguer as mãos, ele se aproximou e escutou sua voz. Quando a
ouviu dizer ``Oh, grande Gigante'', ele murmurou para si:

``Então é mesmo como eu temia. Há mais Gigantes, e realmente esse é
outro deles''. Ele olhou para cima, mas nada viu, e murmurou novamente:

``Eu não vejo, porém essa criança consegue ver; e no entanto eu temia,
pois algo me disse que havia perigo. Realmente, o conhecimento é mais
cego do que a inocência''.

A menina, ainda não sabendo que havia algum ser humano perto dela,
gritou novamente, um grande grito de aflição:

``Oh!, não, não, grande Gigante, não cause mal às pessoas. Se alguém
deve sofrer, que seja eu. Leve"-me. Estou disposta a morrer, mas
poupe"-as. Poupe"-as, grande Gigante; e faça comigo o que bem entender''.
Mas o gigante não prestou atenção.

E Knoal -- pois era ele o velho -- sentiu seus olhos se encherem de
lágrimas, e disse a si mesmo:

``Oh!, nobre criança, quão corajosa ela é!, ela se sacrificaria!''E,
aproximando"-se dela, colocou a mão na cabeça da menina.

Zaya, que estava novamente arqueando a cabeça, assustou"-se e olhou em
torno quando sentiu o toque. Entretanto, quando ela viu que era Knoal,
consolou"-se pois sabia o quão sábio e bom ele era, e sentiu que, se
alguma pessoa poderia ajudá"-la, seria ele. Ficou perto dele e escondeu o
rosto em seu peito; ele fez carinho em seu cabelo e a consolou. Mas,
ainda, ele não conseguia ver nada.

A névoa fria passou, e quando Zaya olhou para cima, viu que o Gigante
havia passado e que estava se movendo em direção a cidade.

``Venha comigo, minha filha'', disse o velho. E~os dois se levantaram e
entraram na casa construída com grandes pedras.

Quando Zaya entrou, ela se espantou, pois, pasmem!, dentro era como uma
tumba. O~velho sentiu"-a ter calafrios, pois ainda a mantinha perto dele,
e disse:

``Não chore, pequenina, e não tema. Este lugar me lembra, e a todos que
nele entram, que à tumba todos retornaremos ao fim. Não tema, pois este
se tornou um lar alegre para mim''.

Então a menina ficou aliviada, e começou a examinar mais atentamente seu
entorno. Ela viu todo tipo de instrumentos curiosos, e muitas ervas
estranhas e comuns, e plantas medicinais penduradas para secar em cachos
nas paredes. O~velho observou"-a em silêncio e depois disse:

``Minha filha, você viu a aparência do Gigante quando ele passou?''

Ela respondeu: ``Sim''.

``Pode descrever o rosto dele para mim?'', ele perguntou novamente.

D'onde então ela começou a lhe contar tudo o que havia visto. De como o
Gigante era tão grande que todo o céu parecia preenchido. De como os
grandes braços estavam abertos, velados em seu robe, até que muito longe
a manta se perdia no ar. De como o rosto era o de um homem forte,
impiedoso, porém sem malícia; e de como os olhos eram cegos.

O velho arrepiou"-se enquanto ouvia, pois sabia que o Gigante era muito
terrível; e seu coração chorou pela cidade amaldiçoada onde tantos
pereceriam em meio a seus pecados.

Eles decidiram partir e alertar novamente o povo amaldiçoado; sem
atraso, o velho e a menina correram para a cidade.

Quando deixaram a casinha, Zaya viu o Gigante atrás deles, movendo"-se
ainda em direção a cidade. Eles se apressaram; e quando passaram através
da névoa fria, Zaya olhou para trás e viu o Gigante atrás deles.

Rapidamente, chegaram a cidade.

Era uma visão estranha o velho e aquela menina correndo para contar às
pessoas sobre a terrível Praga que estava recaindo sobre elas. A~longa
barba branca do velho e os cachos dourados da criança eram arrastados
pelo vento de tão rápido que vinham. Os rostos de ambos estavam brancos
como a morte. Atrás deles, visto apenas pelos olhos da mocinha de
coração puro quando ela olhava para trás, o espectral Gigante vinha
sempre lento, pendendo uma sombra escura no ar do fim da tarde.

Mas as pessoas na cidade nunca viam o Gigante. E~quando o velho e a
menina as alertavam, elas ainda assim não prestavam atenção, mas
zombavam deles e escarneciam deles, dizendo:

``Ora! Não há mais Gigantes agora'', e continuavam em seus afazeres,
rindo e zombando.

Então, o velho se colocou em um lugar elevado entre eles, no degrau mais
baixo da grande fonte, com a menina ao seu lado, e falou assim:

``Oh!, povo, moradores deste Reino, sejam alertados a tempo. Esta
criança, de coração puro, em torno de cuja inocência até mesmo os
passarinhos, que temem os homens, e as mulheres reúnem"-se em paz, viu
esta noite no céu a forma de um Gigante que avança continuamente,
ameaçador, em direção à nossa cidade. Acreditem, oh!, acreditem; e
fiquem alertas enquanto podem. A~mim mesmo, como a vocês, o céu está
limpo; e, no entanto, vejam que eu acredito. Pois, escutem"-me: ignorando
completamente que outro Gigante invadiu nossa terra, sentei pensativo em
minha moradia. E, sem causa ou motivo, veio ao meu coração um medo
repentino pela segurança de nossa cidade. Eu me levantei, olhei ao norte
e ao sul e ao leste e ao oeste, e para o alto e para baixo, mas nunca
pude enxergar um sinal de perigo. Então eu disse a mim mesmo: `Meus
olhos estão turvos devido a uma centena de anos observando e esperando,
e assim não consigo enxergar.' E, no entanto, oh!, povo, moradores deste
reino, apesar de o século ter esmorecido meus olhos externos, ele aguçou
meus olhos internos -- os olhos de minha alma. Novamente eu saí e,
veja!, esta menina se ajoelhara e implorara ao Gigante, invisível para
mim, para poupar a cidade; mas ele não a escutou, ou, se escutou, não
respondeu e ela caiu de bruços no chão. Então viemos para cá para
alertá"-los. Dali, diz a menina, ele avança para a cidade. Oh, sejam
alertados! Alertados a tempo''.

Ainda assim, as pessoas não prestaram atenção, mas zombaram e riram
ainda mais, dizendo:

``Ora!, a menina e o velho estão loucos''; e foram para suas casas --
para dançar e festejar como antes.

Então os garotos levados vieram e zombaram deles, e disseram que Zaya
perdera seus pássaros e ficara louca; e eles fizeram músicas e
cantaram"-nas enquanto dançavam em círculo.

Zaya estava tão dolorosamente angustiada pelo pobre povo que não prestou
atenção aos garotos cruéis. Vendo que ela não prestava atenção a eles,
alguns se tornaram ainda mais rudes e malvados. Eles se afastaram um
pouco e arremessaram coisas neles, zombando ainda mais.

Então, triste no coração, o velho se levantou, e tomou a menina pela
mão, e a levou para bem dentro da floresta, abrigando"-a consigo na casa
feita de grandes pedras. Naquela noite Zaya dormiu com o cheiro
adocicado de ervas secas em torno dela, e o velho segurou sua mão para
que não tivesse medo.

De manhã, Zaya se levantou cedo e acordou o velho, que havia dormido em
sua cadeira.

Ela foi até a porta e olhou para fora; então uma vibração de alegria
recaiu sobre seu coração. Pois, do lado de fora, como se esperando para
vê"-la, estavam todos os seus passarinhos e muitos, muitos mais. Quando
os pássaros viram a menina, entoaram alguns sons alegres e altos, e
voaram loucamente pelo entorno com muita alegria -- alguns deles
sacudindo as asas e tão engraçados que ela não pôde evitar sorrir um
pouco.

Quando Knoal e Zaya tomaram seu café da manhã simples e repartiram"-no
com seus amiguinhos de penas, partiram com corações pesarosos para
visitar a cidade, e tentar mais uma vez alertar o povo. Os pássaros
voavam em torno deles à medida que avançavam, e para incentivá"-los
cantavam o mais alegre que podiam, apesar de seus coraçõezinhos estarem
abatidos.

Enquanto andavam, viram diante si o grande Gigante brumoso. E~ele agora
havia avançado até as fronteiras da cidade.

Mais uma vez eles alertaram as pessoas, e grandes grupos cercaram"-nos,
mas somente zombaram deles mais do que nunca. E~garotos levados jogaram
gravetos e pedras nos passarinhos, matando alguns deles. A~pobre Zaya
chorou amargamente e o coração de Knoal ficou muito triste. Depois de um
tempo, quando eles já tinham chegado à fonte, Zaya olhou para cima e
ressaltou"-se, com alegre surpresa, pois não via mais o Gigante. Ela
bramou alegre e as pessoas riram, dizendo:

``Criança esperta! Ela vê que não vamos acreditar nela e finge que o
Gigante foi embora''.

Eles cercaram"-na, zombeteiros, e alguns deles disseram:

``Vamos colocá"-la na fonte e afundá"-la, como uma lição a mentirosos que
nos assustam''. Eles então se aproximaram dela com ameaças. Ela se
agarroua Knoal, que ficou terrivelmente sério quando ela disse que não
via mais o Gigante, e que estava agora como se em um sonho, pensativo.
Mas, ao seu toque, ele pareceu acordar; e falou severamente ao povo,
censurando"-o. Mas eles também gritaram com ele e disseram que, como ele
havia ajudado Zaya em sua mentira, ele também seria afundado; e eles se
aproximaram para pôr as mãos em ambos.

A mão de um deles, que era um chefe do grupo, já estava estendida,
quando ele emitiu um grito baixo e pressionou sua mão a seu lado; e,
enquanto os outros se viraram para olhar a ele, assustados, ele gritou
com grande dor, berrando horrivelmente. Mesmo enquanto as pessoas
olhavam, seu rosto enegreceu muito, e ele caiu na frente deles,
retorcendo"-se um pouco em terror, e correu, gritando:

``O Gigante! O Gigante! Ele está mesmo entre nós!''

Eles temeram ainda mais por não poderem vê"-lo.

Mas antes de conseguirem sair da praça do mercado, no centro da qual
estava a fonte, muitos caíram mortos, e os cadáveres ficaram no chão.

Ali no centro o velho e menina ajoelharam, rezando;e os pássaros
pousaram em volta da fonte, mudos e parados, e nenhum som se escutava,
exceto os gritos das pessoas ao longe. Então, suas lamentações soaram
cada vez mais altas, pois o Gigante -- a Praga -- estava entre e em
volta deles, e não havia escapatória, pois era tarde demais para fugir.

Ah! No País Sob o Pôr do Sol houve muito choro naquele dia. E~quando a
noite chegou, pouco se dormiu, pois havia medo em alguns corações e dor
em outros. Nada estava parado, exceto os mortos, que jaziam rígidos pela
cidade, tão parados e sem vida que até mesmo a fria luz da lua e as
sombras das nuvens passando sobre eles não podiam lhes fazer parecer que
haviam vivido.

E por muitos dias houve dor e pesar e morte no País Sob o Pôr do Sol.

Knoal e Zaya fizeram tudo o que puderam para ajudar o pobre povo, mas
era realmente difícil ajudá"-los, pois o Gigante invisível estava entre
eles, errando para lá e para cá pela cidade, de modo que ninguém podia
dizer onde ele colocaria sua mão gélida a seguir.

Algumas pessoas fugiram da cidade; mas não adiantava, pois, fossem do
jeito que fossem e nunca fugindo rápido o bastante, eles ainda ficavam
dentro do alcance do Gigante invisível. De vez em quando, ele
transformava seus corações mornos em gelo com seu sopro e seu toque, e
eles caíam mortos.

Alguns, como aqueles dentro da cidade, foram poupados e alguns deles
morreram de fome, e o resto rastejou tristemente de volta para a cidade,
e viveram ou morreram entre seus amigos. E~tudo isso era, oh!, tão
triste, pois nada havia senão pesar e medo e choro de manhã à noite.

Agora, veja como os passarinhos amigos de Zaya ajudaram"-na em sua
necessidade.

Eles pareceram ver a vinda do Gigante quando ninguém -- nem mesmo a
menina -- podia ver qualquer coisa, e conseguiram contá"-la quando havia
perigo, como se conseguissem mesmo falar.

No começo, Knoal e ela iam para casa feita de grandes pedras todas as
noites para dormir e voltavam a cidade de manhã, ficando com o pobre
povo doente, consolando"-o e alimentando"-o, dando"-lhe os remédios que
Knoal, com sua grande sabedoria, sabia que lhes fariam bem. Assim, eles
salvaram muitas preciosas vidas humanas, e aqueles que foram resgatados
ficaram muito agradecidos e, desde então e para sempre, viveram da
maneira mais sagrada e altruísta.

Depois de alguns dias, entretanto, eles descobriram que o pobre povo
doente precisava de ajuda mais de noite do que de dia, e então os dois
ficavam o tempo todo na cidade, ajudando o povo abatido dia e noite.

De manhã bem cedo, Zaya saía para respirar o ar da manhã; e ali,
recém"-acordada do sono, estariam seus amigos emplumados esperando por
ela. Eles entoavam canções de alegria, vinham e empoleiravam"-se em seus
ombros e em sua cabeça, beijando"-a. Então, se ela fosse em direção a
qualquer lugar onde, durante a noite, a Praga houvesse posto sua mão
mortífera, eles se agitariam em frente a ela, tentariam impedi"-la e
gritariam, em sua própria língua:

``Volte! Volte!''

Eles ciscavam de seu pão e bebiam de sua xícara antes que ela os
tocasse; e quando havia perigo -- pois a mão fria do Gigante estava por
toda parte --, eles bradariam:

``Não, não!'', e ela não tocaria a comida, ou não deixaria qualquer um
tocá"-la. Frequentemente ocorria que, mesmo enquanto ciscando o pão ou
bebendo da xícara, um pobre passarinho caía, sacudia suas asas e morria.
Mas todos aqueles que morriam, morriam com um trilo de alegria, olhando
para sua pequena mestra, por quem eles haviam alegremente falecido.
Sempre que passarinhos achavam que o pão e a xícara estavam puros e
livres do perigo, eles olhavam para Zaya vivamente, e batiam suas asas e
tentavam piar, parecendo tão travessos que a pobre e triste menininha
apenas sorria.

Havia um pássaro velho que sempre se demorava mais, e frequentemente
dava muito mais ciscadas no pão quando este era bom para que conseguisse
uma refeição substanciosa. Algumas vezes ele continuava a se alimentar
até que Zaya balançasse o dedo a ele e dissesse:

``Guloso!'', e ele saltava para longe como se não tivesse feito nada.

Havia outro passarinho querido -- um tordo, com peito tão vermelho
quanto o pôr do sol -- que amava Zaya mais do que se pode imaginar.
Quando ele experimentava a comida e descobria que era seguro comê"-la,
ele tirava pequenos pedacinhos em seu bico, voava e colocava na boca
dela.

Cada passarinho que bebia da xícara de Zaya e gostava levantava sua
cabeça para fazer as graças; e desde então os passarinhos fazem a mesma
coisa, e nunca se esquecem de fazer as graças -- como algumas crianças
mal"-agradecidas o fazem.

Assim, Knoal e Zaya viveram, apesar de muitos à sua volta terem morrido,
e o Gigante ainda permaneceu na cidade. Tantas pessoas morreram que
começou a se imaginar se muitos sobraram, pois somente quando a cidade
começou a ficar rarefeita que o povo pensou no vasto número de pessoas
que haviam vivido nela.

A pobre e pequenina Zaya havia ficado tão pálida e magra que ela parecia
uma sombra, e a forma de Knoal estava mais curva com os sofrimentos de
algumas poucas semanas do que estivera com seu centenário todo. Mas,
apesar de os dois estarem fatigados e desgastados, eles ainda mantinham
seu bom trabalho de ajudar os doentes.

Muitos dos passarinhos estavam mortos.

Uma manhã, o velho ficou muito fraco -- tão fraco que mal podia ficar em
pé. Zaya temeu por ele e disse:

``Você está doente, pai?'', pois ela sempre o chamava de pai agora.

Ele lhe respondeu com uma voz, ah!, rouca e baixa, mas muito, muito
carinhosa:

``Minha criança, temo que o fim se aproxima: leve"-me para casa para que
eu possa morrer lá''.

Às suas palavras, Zaya emitiu um grito baixo e caiu de joelhos ao lado
dele, enterrando sua cabeça em seu peito, e chorou amargamente enquanto
o abraçava forte. Mas ela tinha pouco tempo para chorar, pois o velho
lutava para ficar em pé e, vendo que ele desejava ajuda, limpou as
lágrimas e o ajudou.

O velho tomou seu cajado e, com Zaya ajudando"-o a lhe dar apoio, chegou
até a fonte no meio da praça do mercado. E~ali, no degrau mais baixo,
ele sucumbiu, como se exausto. Zaya sentiu"-o ficar frio como o gelo, e
soube que a mão fria do Gigante havia sido posta em cima dele.

Então, sem saber o motivo, olhou para onde ela havia visto pela última
vez o Gigante, quando Knoal e ela haviam ficado ao lado da fonte. E,
veja!, quando ela olhou, segurando a mão de Knoal, viu a forma sombrosa
do terrível Gigante, que havia estado por muito tempo invisível, ficando
mais e mais nítida entre as nuvens.

Seu rosto estava sério como sempre, e seus olhos ainda estavam cegos.

Zaya gritou ao Gigante, ainda segurando Knoal pelas mãos bem forte:

``Ele não, ele não! Ó, poderoso Gigante! Ele não! Ele não!'', e arqueou
a cabeça para baixo e chorou.

Havia uma tal angústia em seu coração que aos olhos cegos do Gigante
sombroso vieram lágrimas que caíram como orvalho na testa do velho.
Knoal falou a Zaya:

``Não se aflija, minha menina. Estou contente por você ver novamente o
Gigante, pois eu tenho esperança de que ele deixe nossa cidade livre de
infelicidades. Sou a última vítima, e morro contente''.

Então Zaya ajoelhou"-se ao Gigante, e disse:

``Poupe"-o! Oh! Poupe"-o e me leve! Mas o poupe! Poupe"-o!''

O velho levantou"-se com os cotovelos, ainda deitado, e falou a ela:

``Não se aflija, pequenina, e não lamente. De verdade, sei que você
alegremente daria sua vida pela minha. Mas nós devemos dar pelo bem dos
outros aquilo que para nós é mais caro do que nossas vidas. A~bênção,
minha pequenina, e seja boa. Adeus! Adeus!''

Quando ele falou a última palavra, ficou frio como a morte, e seu
espírito partiu.

Zaya ajoelhou"-se e rezou; e quando ela olhou para cima, viu o Gigante
sombroso se afastando.

O Gigante virou"-se quando passou, e Zaya viu que seus olhos cegos estavam
apontados em sua direção como se ele tentasse enxergar. Ele levantou os
grandes braços umbrosos, drapejou, extático, em sua mortalha de névoa,
como se a abençoando; e ela pensou que o vento que passou por ela
sussurrando levava o eco das palavras:

``Inocência e devoção salvam o reino''.

Imediatamente ela viu ao longe o grande Gigante Praga se afastando para
as fronteiras do Reino, passando entre os Espíritos Guardiães e pelo
Portal em direção aos desertos para o além… para sempre.
