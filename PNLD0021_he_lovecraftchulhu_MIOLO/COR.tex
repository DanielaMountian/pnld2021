\chapter*{}

A oeste de Arkham as colinas se erguem selvagens, e há vales com bosques
profundos que machado algum tocou. Há grotas estreitas e escuras onde as
árvores se inclinam fantasticamente, e onde regatos esguios escoam sem
jamais ter recebido lampejo de sol. Nos declives mais suaves há fazendas
antigas e de pedra, com cabanas rústicas e cobertas de musgo, ruminando
eternamente sobre os velhos segredos da Nova Inglaterra, ao abrigo dos
grandes desfiladeiros; mas agora todas se acham vazias, as largas
chaminés desmoronando e os lados de telha que incham perigosamente sob
os telhados baixos, de mansarda.

Toda a gente antiga se foi, e estrangeiros não gostam de morar lá.
Franco"-canadenses tentaram, italianos tentaram, e os poloneses chegaram
e partiram. Não é por algo que se possa ver, ouvir ou com que se possa
lidar, mas por algo que se imagina. O lugar não faz bem para a
imaginação, nem traz sonhos repousantes à noite. Tem de ser isso o que
mantém os estrangeiros longe, pois o velho Ammi Pierce jamais lhes
contou o que recorda dos dias estranhos. Ammi, cuja cabeça anda um pouco
esquisita faz anos, é o único que ainda resta, ou que ainda fala dos
dias estranhos; e se atreve a fazê"-lo porque sua casa é muito próxima
dos campos abertos e das estradas utilizáveis na região de Arkham.

Houve certa vez uma estrada, por sobre as colinas e através dos vales,
que corria exatamente onde hoje está a brenha dos infernos;\footnote{\emph{Blasted
  heath} é uma expressão shakespeareana, e foi utilizada no
  \emph{Macbeth} (Ato \textsc{i}, cena 3, v.78), quando Macbeth pergunta às três
  bruxas o motivo de terem chamado a atenção dele e de Banquo naquele
  lugar inóspito.} mas as pessoas deixaram de usá"-la e uma nova estrada
foi aberta, curvando"-se longe em direção ao sul. Traços da antiga ainda
podem ser encontrados entre as ervas daninhas de um ermo que retorna, e
alguns persistirão mesmo quando metade das depressões for inundada para
o novo reservatório. Daí os bosques escuros serão derrubados e a brenha
dos infernos adormecerá fundo sob as águas azuis cuja superfície será um
espelho ao céu, com estrias ao sol. E os segredos dos dias estranhos vão
se juntar aos segredos das profundezas; vão se juntar às lendas ocultas
do velho oceano e a todos os mistérios da terra primeva.

Quando fui às colinas e aos prados, inspecionar para o novo
reservatório, disseram"-me que o lugar era maligno. Eles o disseram em
Arkham, e por ser uma cidade muito velha, repleta de histórias de
bruxas, pensei que esse mal fosse algo que as avós vinham sussurrando às
crianças por séculos. O nome ``brenha dos infernos'' me parecia mui
singular e teatral, e me perguntava como aquilo entrara para o folclore
de uma gente puritana. Então vi por mim mesmo aquele escuro emaranhado
de vales e declives a oeste e deixei de especular sobre o que quer que
não fosse seu próprio mistério ancestral. Era manhã quando o vi, mas a
sombra sempre espreita por lá. As árvores cresciam espessas demais, e
seus troncos eram grandes demais para qualquer bosque saudável da Nova
Inglaterra. Havia silêncio demasiado nas turvas aleias que medeiam o
lugar, e o chão era macio demais no atapetado de musgo úmido com
infinitos anos de decomposição.

Nos espaços abertos, sobretudo junto à linha da antiga estrada, havia
pequenas fazendas de encosta; por vezes, com todas as construções ainda
de pé, por vezes com apenas uma ou duas, e por vezes com uma solitária
chaminé ou porões diminutos. Ervas daninhas e espinhos reinavam, e uma
furtiva vida selvagem farfalhava sob a folhagem rasteira. Havia em tudo
uma névoa de inquietude e opressão; um toque do irreal e do grotesco,
como se algum elemento vital de perspectiva ou do \emph{chiaroscuro}
viesse deturpado. Não me surpreendi que os estrangeiros não
permanecessem, essa não era uma região em que dormir. Era demasiado como
uma paisagem de Salvator Rosa;\footnote{Salvator Rosa (1615--1673), pintor
  italiano nascido em Arenella, famoso, entre outras coisas, por suas
  paisagens escuras, desoladas, tempestuosas e íngremes.} era demasiado
como uma xilogravura proibida de conto de terror.

Mas nem mesmo isso era pior do que a brenha dos infernos. Eu o soube no
momento em que cheguei lá, no fundo de um amplo vale; pois nenhum outro
nome caberia a tal coisa, nem outra coisa caberia em tal nome. Era como
se o poeta tivesse cunhado o termo após ter visto esta região
específica. Teria de ser --- pensava a olhar para aquilo --- o resultado
de um incêndio; mas por que nada mais brotava de novo naqueles cinco
acres de desolação cinzenta, alastrando"-se abertos ao céu como uma
imensa mancha de corrosão ácida entre os bosques e campos? Situava"-se em
grande parte ao norte da antiga linha da estrada, mas engolia um pouco
do outro lado. Sentia curiosa relutância em me aproximar, e enfim o fiz
apenas porque o trabalho me levava a passar por ali e além. Não havia
vegetação de tipo algum naquela vasta área, somente uma fina poeira, ou
borralho cinzento, sobre a qual vento algum parecia jamais ter soprado.
As árvores nas imediações eram enfermiças e atrofiadas, e muitos troncos
mortos estavam em pé ou jaziam apodrecendo nas beiras. Enquanto
caminhava apressado, vi os tijolos e pedras caídos de uma chaminé e uma
adega antigas à minha direita, e a bocejante bocarra negra de um poço
esquecido, cujos vapores estagnados produziam estranhos efeitos com os
matizes de luz solar. Até a longa e escura subida da floresta parecia
convidativa em comparação, e não me espantei mais com os sussurros
assustados da gente de Arkham. Não havia casa ou ruína por perto; mesmo
nos velhos tempos o lugar terá sido solitário e remoto. E no crepúsculo,
receando retornar por aquele lugar agourento, tomei o caminho tortuoso
para a cidade pela longa estrada ao sul. Desejei vagamente que algumas
nuvens se juntassem, pois um receio anormal a respeito dos profundos
vazios celestes se infiltrara em minha alma.

~À noite perguntei aos idosos em Arkham sobre a brenha dos infernos, e
sobre o que queria dizer a expressão ``dias estranhos'', que tantos
murmuravam evasivamente. Não pude, contudo, obter boas respostas, exceto
a de que todo o mistério era muito mais recente do que sonhara. Não se
tratava de uma questão de lendas antigas, mas do que acontecera durante
a vida dos que falavam. Havia ocorrido nos anos oitenta, e uma família
desaparecera ou fora morta. Meus interlocutores não eram exatos; e como
todos me disseram para não prestar atenção aos loucos contos do velho
Ammi Pierce, procurei"-o na manhã seguinte, tendo ouvido que morava
sozinho no chalé antigo e caduco, lá onde as árvores começaram a ficar
muito espessas. Era lugar pavorosamente arcaico, e começara a exsudar o
leve odor miasmático que se agarra às casas que existem há muito.
Somente batendo com persistência consegui despertar o homem idoso, e
quando timidamente arrastou os pés à porta pude notar que não estava
nada feliz em me ver. Não era frágil como o imaginara; mas seus olhos
descaíam de maneira curiosa, e as roupas descuidadas, mais a barba
branca, o faziam parecer muito gasto e lúgubre. Sem saber como lançá"-lo
às suas histórias, fingi questão de negócios: disse a ele sobre minha
inspeção e fiz perguntas vagas sobre o distrito. Era muito mais
inteligente e educado do que eu fora levado a pensar, e antes que
percebesse já havia compreendido tanto do assunto quanto qualquer pessoa
com quem conversara em Arkham. Não era como outros rústicos que eu
conhecia das regiões onde se construíam reservatórios. Da sua parte, não
protestava quanto às milhas de bosque velho e terra arável a serem
submersas, ou talvez protestasse se seu lar não estivesse fora dos
limites do futuro lago. Alívio era tudo o que demonstrava; alívio com
relação ao fim daqueles antigos vales escuros, pelos quais perambulara
toda sua vida. Estariam melhor sob a água, agora --- melhor sob a água,
desde os dias estranhos. E, com essa abertura, sua voz roufenha
recolheu"-se grave, enquanto seu corpo se inclinou para a frente e com o
indicador direito começou a apontar, trêmulo e impressionante.

Foi então que ouvi a história, e, ao seguir a voz errante em seus
arranhados e sussurros, sentia calafrios apesar do dia de verão. Tive de
interromper várias vezes o meu interlocutor em suas divagações, recompor
pontos científicos que ele sabia somente por uma gasta memória de
papagaio de professores, ou preencher as lacunas onde seu senso de
lógica e continuidade se desintegrava. Quando terminou, não me
surpreendi que sua mente tenha sofrido, ou que a gente de Arkham não
falasse muito da brenha dos infernos. Apressei"-me em voltar ao meu hotel
antes do pôr"-do"-sol, por não desejar que as estrelas me encontrassem
ainda ao ar livre; e no dia seguinte retornei a Boston para desistir da
minha posição. Não podia entrar de novo naquele turvo caos de velha
floresta e encosta, ou enfrentar outra vez aquela brenha miserável, onde
o poço negro bocejava profundamente, ladeado de pedras e tijolos caídos.
O reservatório será construído em breve, e todos aqueles segredos
antigos estarão a salvo para sempre sob as funduras aquáticas. Mas,
mesmo assim, não acredito que gostaria de visitar aquele lugar à noite
--- ao menos não quando as estrelas sinistras aparecem; e nem que me
pagassem beberia a nova água da cidade de Arkham.

Tudo começou, disse o velho Ammi, com o meteorito. Antes disso havia
apenas as lendas aberrantes da época dos processos por bruxaria, e mesmo
então esses bosques do oeste não eram tão temidos quanto a pequena ilha
em Miskatonic, onde o diabo mantinha sua corte junto de um curioso altar
de pedra mais antigo do que os índios. Não eram bosques assombrados, e o
crepúsculo fantástico nunca fora terrível antes dos dias estranhos.
Então houve aquela branca nuvem ao pino do meio"-dia, aquela série de
explosões no ar e o pilar de fumaça no vale distante dentro do bosque.
Já à noite todos tinham ouvido falar da enorme rocha que caíra do céu e
se alojara no chão ao lado do poço de Nahum Gardner. Era essa a casa que
ficava no lugar onde a brenha dos infernos surgiria --- a graça de casa
branca de Nahum Gardner, em meio a seus férteis jardins e pomares.

Nahum fora à cidade contar a todos sobre a pedra, e dera uma passada no
Ammi Pierce no caminho. Ammi tinha quarenta anos, e todas as coisas
esquisitas ficaram marcadas de modo indelével em sua mente. Ele e sua
esposa foram com três professores da Universidade Miskatonic, que vieram
às pressas na manhã seguinte, para ver o bizarro visitante do
desconhecido espaço sideral, e se perguntaram por que Nahum havia dito
que era tão grande no dia anterior. Encolheu, Nahum disse apontando o
enorme monte de terra sobre o chão rasgado e a grama carbonizada perto
da arcaica picota de poço em seu jardim frontal; mas os sábios
responderam que pedras não encolhem. Seu calor persistia
insistentemente, e Nahum afirmou que brilhara ligeiramente à noite. Os
professores experimentaram"-na com um martelo petrográfico e perceberam
maciez incomum. Era, de fato, tão macia como se fosse quase de plástico;
e assim, ao invés de lascar, arrancaram uma amostra para os testes na
faculdade. Levaram"-na em um velho balde emprestado da cozinha de Nahum,
pois até mesmo aquela fração recusava"-se a se resfriar. No caminho de
volta pararam na casa de Ammi para descansar, e pareciam pensativos
quando a Sra. Pierce observou que o fragmento estava encolhendo e
queimando o fundo do balde. É verdade, não se tratava de muito, mas
talvez tivessem retirado menos do que supunham.

No dia que se seguiu --- foi tudo em junho de 1882 ---, os professores
marcharam ao local novamente e com grande entusiasmo. Ao passar pela
casa de Ammi, contaram"-lhe as coisas esquisitas que a amostra fizera, e
como desaparecera por completo quando a puseram num béquer. O béquer se
fora também, e os sábios discutiram a afinidade da estranha pedra com o
silício. Agira de modo inacreditável naquele laboratório tão organizado;
não reagira a nada, sequer havia liberado gases obstruídos quando
aquecida em carvão, inteiramente negativa ao teste da pérola de
bórax,\footnote{O \emph{teste}, ou \emph{ensaio da pérola de bórax} é uma
  análise química para identificação de metal: quando aquecida junto com
  a amostra, a pérola de bórax recebe coloração indicativa do metal
  presente.} provando"-se não"-volátil em absoluto a qualquer
temperatura, incluindo a do maçarico de oxi"-hidrogênio. Mostrou"-se
altamente maleável numa bigorna, e no escuro sua luminosidade era muito
acentuada. Recusando obstinadamente a se resfriar, logo pusera a
faculdade num estado de verdadeira agitação; e quando foi aquecida
diante do espectroscópio, exibindo faixas brilhantes e diferentes de
quaisquer cores conhecidas no espectro comum, houve muita conversa
ofegante sobre novos elementos, bizarras propriedades ópticas, e outras
coisas que homens de ciência, perplexos, costumam dizer quando
confrontados com o desconhecido.

Quente como estava, testaram"-na em um cadinho com todos os reagentes
adequados. Água não resultou em nada. Ácido hidroclorídrico, o mesmo.
Ácido nítrico e mesmo água"-régia meramente silvaram e espirraram contra
a tórrida invulnerabilidade. Ammi tinha dificuldade de recordar todas
essas coisas, mas reconheceu alguns solventes quando os mencionei na
ordem habitual de uso. Havia amoníaco e soda cáustica, álcool e éter, o
nauseante dissulfeto de carbono e uma dúzia de outros; mas ainda que o
peso se reduzisse progressivamente com o passar do tempo, e o fragmento
parecesse também ir progressivamente se resfriando, não havia mudança
nos solventes que apontasse seu ataque à substância. Não obstante era,
sem dúvida alguma, um metal. Para começar, era magnético; e, depois da
imersão nos solventes ácidos, parecia haver traços leves dos padrões de
Widmannstätten,\footnote{Os padrões, ou figuras de Widmanstätten
  (Lovecraft grafa com dois ``n'', como acima) são ranhuras específicas
  encontradas nos meteoritos de ferro. O nome vem do conde Alois von
  Beckh Widmanstätten, diretor da coleção imperial de louça de Viena,
  que foi quem primeiro observou aqueles padrões metalográficos.}
encontrados no ferro de meteoritos. Quando o resfriamento chegou a um
ponto considerável, o teste continuou em vidro; e foi num béquer de
vidro que puseram todas as lascas tiradas ao fragmento original durante
os procedimentos. Na manhã seguinte tanto as lascas quanto o béquer
haviam desaparecido sem deixar traço, e apenas uma mancha de queimado
marcava o lugar na estante de madeira onde haviam estado.

Isso tudo os professores contaram a Ammi ao passar por sua casa, e ele
uma vez mais os acompanhou para ver o mensageiro rochoso das estrelas,
embora desta vez sua esposa não tenha se juntado ao grupo. Havia então
diminuído, com toda a certeza, e sequer os sóbrios professores podiam
duvidar da verdade do que viram. A terra afundara em toda a volta
daquela minguante massa marrom, perto do poço; se antes media uns bons
sete pés, agora mal seriam cinco. Estava ainda quente, e os sábios
estudaram com curiosidade sua superfície, removendo ainda uma outra
parte, maior, com o martelo e o cinzel. Escavaram mais fundo desta vez
e, ao retirar aquela porção, notaram que o núcleo da coisa não era
exatamente homogêneo.

Haviam descoberto o que parecia ser o lado de um grande glóbulo
colorido, incrustado na substância. A cor, que lembrava algumas das
faixas no estranho espectro do meteoro, era quase impossível de
descrever; e apenas por analogia foi que, de todo modo, a chamaram
\emph{cor}. Sua textura era polida, e sob leves batidas parecia prometer
tanto fragilidade quanto vazio. Um dos professores aplicou"-lhe uma leve
pancada com um martelo, e daí abriu"-se com um pequeno estalo nervoso.
Não houve nenhuma emissão, e todos os traços da coisa desapareceram com
o perfurar. Deixou apenas um espaço esférico vazio de por volta de três
polegadas de diâmetro, e todos pensaram que seria provável encontrar
outros assim que a substância de revestimento se consumisse.

Conjectura seria algo vão; logo, depois de uma tentativa fútil de achar
novos glóbulos fazendo perfurações, os perquiridores novamente se foram
com seu novo espécime --- que se provou, no entanto, tão exasperante no
laboratório quanto o fora seu predecessor. A despeito de ser quase de
plástico, ter calor, magnetismo e leve luminosidade, de resfriar"-se
ligeiramente em ácidos poderosos, de possuir um espectro desconhecido,
de se consumir no ar e de atacar compostos de silício com destruição
mútua como resultado, não apresentava quaisquer características
identificáveis; e ao fim dos testes os cientistas da faculdade foram
forçados a reconhecer que não sabiam como classificá"-lo. Não era algo
desta Terra, mas um pedaço do grande exterior; e, como tal, dotado de
propriedades daquele exterior e obediente a leis daquele exterior.

Naquela noite houve uma tempestade, e quando os professores foram ao
Nahum no dia seguinte se viram diante de uma amarga decepção. A rocha,
magnética como fora, deve ter tido alguma propriedade elétrica peculiar;
pois havia ``atraído o raio'', como disse Nahum, com singular
persistência. Por seis vezes em uma hora o fazendeiro viu o raio atingir
o buraco do jardim dianteiro, e, quando a tempestade passou, nada
restava senão um fosso irregular junto à antiga picota, em parte
obstruída por causa da terra que afundara. Cavar era infrutífero, e os
cientistas verificaram o desaparecimento completo. O fracasso fora
total; de modo que nada restava fazer senão voltar ao laboratório e
testar de novo o fragmento evanescente que com cuidado se deixara
envolto em chumbo. O fragmento durou uma semana, ao fim da qual nada de
útil haviam descoberto sobre aquilo. Quando se acabou, resíduo algum
ficou para trás, e com o tempo os professores mal podiam ter certeza de
que haviam visto com olhos despertos aquele críptico vestígio dos
insofismáveis golfos exteriores; aquela solitária mensagem bizarra de
outros universos e outros reinos de matéria, força e entidade.

Como era natural, os jornais de Arkham fizeram muito barulho sobre o
incidente e seu patrocínio acadêmico, e enviaram repórteres para
conversar com Nahum Gardner e sua família. Ao menos um diário de Boston
mandou um escriba, e Nahum logo se tornou um tipo de celebridade local.
Ele era um tipo esguio, cordato, de mais ou menos cinquenta anos, que
vivia com a esposa e três filhos numa herdade simpática do vale. Ele e
Ammi se visitavam com frequência, assim como suas esposas; e Ammi
tinha"-lhe apenas elogios depois de todos aqueles anos. Parecia levemente
orgulhoso da atenção que seu sítio atraiu, e falou com frequência do
meteorito nas semanas seguintes. Aqueles meses de julho e agosto foram
quentes, e Nahum trabalhou duro cortando feno no pasto de dez acres que
atravessava o Ribeirão de Chapman, sua caleça ruidosa abrindo sulcos
fundos nas aleias umbrosas do caminho. O labor cansava"-o mais do que em
outros anos, e sentiu que a idade começava a se tornar aparente.

Então veio o tempo da colheita. As peras e maçãs aos poucos
amadureceram, e Nahum jurava que suas hortas prosperavam como nunca
antes. As frutas cresciam com tamanhos fenomenais e com lustro insólito,
e em tal abundância que barris extra foram encomendados para acolher a
futura safra. Mas com o amadurecimento veio a decepção cruel: pois a
despeito daquela gama de luxuriantes ilusões, sequer uma prestava"-se ao
consumo. No sabor salutar das peras e maçãs insinuara"-se sorrateiro um
amargor adoentado, de modo que mesmo a menor das mordidas induzia a um
desgosto duradouro. O mesmo com melões e tomates, e Nahum notou triste
que a colheita toda estava perdida. Veloz para conectar eventos,
declarou que o meteorito envenenara o solo, e agradeceu aos céus que a
maior parte das outras semeaduras se encontravam nas terras mais altas
ao longo da estrada.

O inverno chegou cedo, e foi rigoroso. Ammi viu Nahum menos do que
costumava, e observou que começava a parecer preocupado. O resto de sua
família também parecia se tornar taciturno; e já não eram muito
frequentes no ir à igreja ou nas ocasiões dos vários eventos sociais da
vida no campo. Causa alguma pôde se encontrar para essa reserva, ou
melancolia, ainda que os membros da família confessassem aqui e ali que
a saúde não lhes ia bem, além de um vago sentimento de inquietação. O
próprio Nahum ofereceu a melhor definição quando disse que lhe
perturbavam certas pegadas na neve. Eram as marcas habituais de inverno
dos esquilos vermelhos, coelhos brancos e raposas, mas o fazendeiro
incomodado afirmava notar algo não muito certo em sua natureza e
arranjo. Não era específico, mas parecia pensar que não fossem tão
características da anatomia e dos hábitos de esquilos e coelhos e
raposas como deveriam ser. Ammi ouvia essa conversa sem interesse até
uma noite quando passava em seu trenó diante da casa de Nahum, ao voltar
de Clark's Corners. Havia lua, e um coelho correra atravessando a
estrada, e os saltos do coelho eram longos demais para o gosto de Ammi e
de seu cavalo. Este último, de fato, quase fugira antes de ser retido a
rédea firme. A partir de então Ammi considerou com mais respeito, e se
perguntou por que os cães de Gardner pareciam tão curvados, tremendo
todas as manhãs. Notava"-se, haviam perdido todo impulso de latir.

Em fevereiro, os garotos McGregor, de Meadow Hill, estavam no campo
atirando em marmotas, e não muito longe da casa de Gardner acertaram um
espécime muito peculiar. As proporções de seu corpo pareciam
ligeiramente alteradas, de modo tão esquisito que seria impossível
descrever, e sua face ganhara uma expressão que ninguém jamais vira numa
marmota. Os garotos ficaram genuinamente apavorados, e jogaram a coisa
fora de imediato, de modo que somente sua história grotesca chegara às
pessoas da região. Mas o susto dos cavalos perto da casa de Nahum
tornara"-se então algo compreendido, e toda a base para um ciclo de
lendas sussurradas começava a tornar forma.

Havia gente que jurava que a neve derretia mais rápido em volta da casa
de Nahum do que em qualquer outro lugar, e no começo de março houve uma
espantada discussão no armazém do Potter em Clark's Corners. Stephen
Rice passara diante da casa de Gardner pela manhã e notara o repolho
gambá\footnote{\emph{Symplocarpus foetidus}, ou, em inglês,
  \emph{skunk"-cabbage}, planta que cresce por volta de 50 cm do chão e
  emite odor fétido, nascendo sobretudo em pântanos; planta termogênica,
  da família das aráceas, pode crescer também em solo gelado.} surgindo
da lama em meio ao bosque, além da estrada. Nunca se viram coisas de
tamanho semelhante, e elas tinham estranhas cores que não se pode pôr em
palavras. Suas formas eram monstruosas, e o cavalo resfolegara diante de
um odor que surpreendera Stephen por não ter precedentes. Naquela tarde
muitas pessoas passaram por lá para ver o florescimento anormal, e todos
concordaram que plantas desse tipo jamais deveriam brotar em um mundo
saudável. As más frutas do outono anterior foram mencionadas o tempo
todo, e o boca"-a"-boca era o de que o solo de Nahum estava envenenado.
Claro que era o meteorito; e, lembrando como os homens da faculdade
acharam aquela rocha estranha, muitos fazendeiros falaram"-lhes do
assunto.

Certo dia fizeram uma visita a Nahum; mas, não tendo apreço algum por
histórias extravagantes e folclore, foram bastante conservadores no que
inferiram. As plantas eram realmente ímpares, mas todos os repolhos
gambá são mais ou menos ímpares em forma e odor e matiz. Talvez algum
elemento mineral da rocha penetrara o solo, mas logo deveria passar.
Quanto às pegadas e os cavalos assustados --- é claro que se tratava de
meros causos interioranos que um fenômeno tal como como o aerólito sem
dúvida iniciara. Não havia realmente nada que homens sérios pudessem
fazer em casos de tanta conversa fiada, uma vez que rústicos
supersticiosos dizem e acreditam em qualquer coisa. E assim, durante os
dias estranhos, os professores se mantiveram afastados, por desdém.
Apenas um deles, tendo recebido um frasco de poeira para análise em um
trabalho para a polícia, pouco mais de um ano depois, lembrou que a cor
esquisita daquele repolho gambá fora muito próxima à de uma das faixas
de luz anômalas apresentadas pelo fragmento de meteoro ao
espectroscópio, e pelo glóbulo quebradiço incrustado na rocha abissal.
As amostras nesse caso de análise forneceram as mesmas faixas ímpares
inicialmente, mas depois perdeu"-se essa propriedade.

As árvores floresceram prematuras em torno da casa de Nahum, e à noite
oscilavam sinistras ao vento. O segundo filho de Nahum, Thaddeus, rapaz
de quinze anos, jurava que elas oscilavam mesmo quando não havia vento;
mas sequer os boatos dariam crédito a isso. No entanto, é certo que
havia desassossego no ar. Toda a família Gardner desenvolveu o hábito de
ouvir em silêncio atento, mas não qualquer som que pudessem
conscientemente identificar. Esse ouvir era, de fato, mais um produto de
momentos quando a consciência parecia um pouco fugir. Infelizmente, tais
momentos cresceram semana a semana, até que se tornou comum dizer que
``tinha algo de errado com essa gente do Nahum''. Quando a primeira
saxífraga surgiu, tinha outra cor estranha; não bem como aquela do
repolho gambá, mas claramente aparentada, e igualmente desconhecida para
quem quer que a tenha visto. Nahum levou algumas flores para Arkham e
mostrou"-as ao editor da \emph{Gazeta}, mas o dignitário não fez mais do
que escrever um artigo humorístico sobre elas, no qual os medos escuros
dos rústicos eram atacados por mui polida ridicularização. Fora um erro
de Nahum contar a um estólido homem da cidade sobre o modo como as
grandes, desproporcionais borboletas manto"-de"-luto se comportavam em
relação àquelas saxífragas.

Abril trouxe consigo um tipo de loucura às pessoas do campo, e se
iniciou o desuso da estrada que passava perto da casa de Nahum, levando
a seu completo abandono. Foi a vegetação. Todas as árvores do pomar
floriram com cores estranhas, e do solo rochoso do jardim e dos pastos
adjacentes brotaram plantas que somente um botânico poderia ligar com a
flora apropriada da região. Não se achavam cores saudáveis em parte
alguma, exceto na grama e nas folhas verdes; mas, por todo lado, aquelas
variantes febris e prismáticas de um adoentado tom primário de base, sem
lugar entre as tonalidades conhecidas. As \emph{calças de
holandês}\footnote{\emph{Dicentra cucullaria}, flor de folhas brancas e
  base amarela, chamada \emph{Dutchman breeches} em inglês.} se tornaram
um tipo de sinistra ameaça, e as sanguinárias cresciam insolentes em sua
perversão cromática. Ammi e os Gardner achavam que a maioria das cores
tinha um tipo de familiaridade assombrada, e decidiram que lembravam
aquele glóbulo quebradiço no meteoro. Nahum arou e plantou o pasto de
dez acres e o lote nas terras altas, mas não tocou a terra em volta da
casa. Sabia que seria inútil, e esperava que as estranhas plantas do
verão sugassem todo o veneno do solo. Estava agora preparado para quase
tudo, e se acostumara à sensação, junto de si, de algo aguardando para
se fazer ouvir. O fato de que os vizinhos afastavam"-se de sua casa teve
efeito nele, claro; mas foi pior em sua esposa. Os garotos lidavam
melhor, estando na escola o dia todo; mas não podiam evitar o terror
provocado pelos boatos. Thaddeus, jovem particularmente sensível, sofreu
mais.

Em maio vieram os insetos, e a casa de Nahum se tornou um pesadelo de
zumbidos e rastejar. A maior parte das criaturas não parecia muito comum
quanto a aspecto e movimento, e seus hábitos noturnos contrariavam toda
a experiência prévia. Os Gardner puseram"-se a observar à noite ---
observar aleatoriamente em todas as direções em busca de algo\ldots{} que
não saberiam definir. Foi quando se aperceberam que Thaddeus estivera
certo sobre as árvores. A Sra. Gardner foi a segunda a presenciar pela
janela, observando os galhos inchados do bordo contra o céu aceso de
lua. Os ramos certamente se moveram, e não havia vento. Tem de ser a
seiva. Estranheza permeava tudo o que crescia, agora. Não foi,
entretanto, ninguém da família de Nahum quem fez a próxima descoberta. O
costume os havia dessensibilizado, e o que não podiam mais ver foi visto
de relance por um tímido vendedor de moinhos de Bolton, que passou por
lá certa noite sem conhecer as histórias locais. O que ele disse em
Arkham recebeu um parágrafo curto na \emph{Gazeta}; e lá foi que todos
os fazendeiros, incluindo Nahum, viram aquilo primeiro. A noite fora
escura e os lampiões da carruagem falharam, mas perto de uma fazenda no
vale, que todos sabiam, pela descrição, ser a de Nahum, a escuridão era
menos espessa. Uma luminosidade tênue mas nítida parecia ser inerente a
toda a vegetação, relva, folhas e flores igualmente, e em um momento um
pedaço de fosforescência, destacando"-se, pareceu se agitar furtiva no
jardim próximo ao celeiro.

A relva até então parecia intocada, e as vacas pastavam livremente no
terreno junto à casa, mas lá pelo fim de maio o leite começou a ficar
ruim. Nahum então levou as vacas para as terras altas, e o problema
acabou. Não muito depois, a mudança na relva e nas folhas fez"-se
evidente. Todo o verdor tornara"-se cinzento, e desenvolvia uma
singularíssima qualidade quebradiça. Ammi agora era a única pessoa a
visitar o lugar, e suas visitas tornavam"-se cada vez mais escassas.
Quando a escola fechou, os Gardner ficaram virtualmente sem contato
algum com o mundo, e por vezes deixavam Ammi resolver seus afazeres na
cidade. Decaíam, de modo curioso, tanto física quanto mentalmente, e
ninguém se surpreendeu quando a notícia da loucura da Sra. Gardner se
espalhou.

Aconteceu em junho, perto do aniversário da queda do meteoro, e a pobre
mulher berrava a respeito de coisas no ar que ela não podia descrever.
Em seu desvario não havia um substantivo específico, mas só verbos e
pronomes. Coisas se moviam e mudavam e flutuavam, e orelhas se eriçavam
a impulsos que não eram inteiramente sonoros. Algo havia sido levado ---
ela estava sendo drenada de algo --- algo que não deveria agarrava"-se a
ela --- alguém tinha de afastar aquilo --- nada jamais se aquietava à
noite --- os muros e janelas se alteravam. Nahum não a enviou ao asilo
do condado, mas deixou que perambulasse pela casa enquanto fosse
inofensiva para si e para os outros. Mesmo quando sua expressão mudou,
ele nada fez. Mas quando os garotos passaram a temê"-la, e Thaddeus quase
desmaiou pelo modo como o rosto dela se transfigurava à sua frente, ele
decidiu mantê"-la trancada no sótão. Em julho ela deixara de falar e
rastejava de quatro, e antes do fim do mês Nahum entretinha a louca
noção de que ela emitia uma luz sutil no escuro, como agora claramente
percebia ser o caso com a vegetação circunstante.

Foi pouco antes disso que os cavalos debandaram. Algo os despertara à
noite, e seus relinchos e coices nas baias haviam sido terríveis.
Parecia não haver o que os acalmasse, e quando Nahum abriu a porta do
estábulo todos voaram para fora como gamos apavorados. Levou uma semana
para rastrear os quatro, e quando foram encontrados viu"-se que estavam
já inúteis e indomáveis. Algo estalara em seus cérebros, e tiveram de
ser sacrificados para seu próprio bem. Nahum tomara um cavalo emprestado
de Ammi para cortar feno, mas viu que ele não se aproximava do celeiro.
Evitava, empacava, rinchava, e por fim não se pôde fazer nada senão
conduzi"-lo pelo jardim enquanto os homens usavam sua própria força para
arrastar a carroça pesada perto o bastante do palheiro para um
carregamento apropriado. E o tempo todo a vegetação ia se acinzentando e
esfarelando. Mesmo as flores cujo matiz era tão estranho agora se
acinzentavam, e o fruto surgia cinzento, reduzido e sem sabor. Ásters e
varas"-de"-ouro floriam cinzentas e distorcidas, e as rosas e as zínias e
as malvas no jardim dianteiro eram coisas de aparência tão blasfema que
o garoto mais velho de Nahum, Zenas, as cortou todas. Os insetos
estranhamente inchados morreram mais ou menos naquele período, e também
as abelhas que haviam deixado as colmeias para ir ao bosque.

Em setembro toda a vegetação rapidamente esboroava em um tipo de pó
cinzento, e Nahum temeu que as árvores morressem antes que o veneno
deixasse o solo. Sua esposa agora tinha períodos de berros horrorosos, e
ele e os garotos viviam em estado de permanente tensão nervosa. Evitavam
as pessoas, por sua vez, e quando a escola voltou os garotos não foram.
Mas foi Ammi, em uma de suas raras visitas, quem primeiro notou que a
água do poço já não era boa. Tinha um sabor ruim que não era exatamente
fétido nem exatamente salgado, e Ammi advertiu seu amigo para que
cavasse outro poço em terreno mais alto e usasse até que o solo
estivesse bem de novo. Nahum, no entanto, ignorou o aviso porque àquela
altura se calejara contra coisas estranhas e desagradáveis. Ele e os
garotos continuaram a usar o suprimento maculado de modo tão indiferente
e mecânico quanto comiam suas refeições mirradas e malcozidas, e
executavam suas tarefas monótonas e ingratas pelos dias a esmo. Havia
algo de uma estólida resignação neles todos, como se caminhassem
parcialmente em outro mundo, entre as linhas de guardas sem nome, em
direção a um destino familiar e fatal.

Thaddeus enlouqueceu em setembro, após uma visita ao poço. Fora lá com
um balde e voltara de mãos vazias, emitindo guinchos agudos e girando os
braços, e às vezes tinha um ataque de riso nervoso ou murmúrios sobre
``as cores que se movem lá embaixo''. Dois em uma só família já era
muito mal, mas Nahum demonstrou bravura. Permitiu que o garoto corresse
por aí durante uma semana, até que começasse a tropeçar e se ferir, e
daí trancou"-o em um espaço no sótão, no lado oposto ao de sua mãe. O
modo como gritavam um ao outro detrás das portas trancadas era
horripilante, especialmente para o pequeno Merwin, que fantasiava os
dois falando em alguma língua horrenda que não pertencia a esta Terra.
Merwin tornava"-se espantosamente imaginativo, e sua inquietude piorou ao
trancarem o irmão, que era seu melhor companheiro de brincadeiras.

Quase ao mesmo tempo, a mortandade entre os animais começou. As aves
domésticas se acinzentaram e morreram muito rápido, com sua carne seca e
nociva, como se via ao corte. Porcos tornaram"-se descomedidamente
gordos, e logo passaram a sofrer mudanças repulsivas que ninguém
conseguia explicar. Sua carne era evidentemente inútil, e Nahum chegava
ao fim de suas forças. Nenhum veterinário rural queria se aproximar de
sua casa, e o veterinário vindo de Arkham mostrava"-se abertamente
perplexo. Os suínos começavam a se tornar cinzentos e quebradiços, e
logo caíam aos pedaços antes de morrer, seus olhos e focinhos sofrendo
alterações singulares. Era inexplicável, porque nunca se alimentaram com
os vegetais maculados. Então, algo atacou as vacas. Certas áreas ou, por
vezes, o corpo todo se tornava insolitamente enrugado ou comprimido, e
colapsos atrozes ou mesmo desintegrações eram comuns. Nos últimos
estágios --- e morte era sempre o
resultado --- acontecia de acinzentarem, tornando"-se farelentas como o
que acometia os porcos. Não havia questão sobre envenenamento, uma vez
que todos os casos ocorreram em um celeiro fechado e tranquilo.
Mordedura alguma de coisas que, rondando, trouxessem o vírus, pois qual
criatura viva neste mundo poderia atravessar objetos sólidos? Deve ser
apenas uma enfermidade natural --- e, ainda assim, qual enfermidade
poderia infligir tais efeitos estava além da capacidade de qualquer um
imaginar. Quando chegou a hora da colheita não havia um animal sequer
vivendo no lugar, pois o gado e os galináceos estavam mortos e os cães
haviam fugido. Esses cães, três em número, haviam desaparecido certa
noite e nunca se soube mais deles. Os cinco gatos fugiram algum tempo
antes, mas sua partida mal se sentira, uma vez que agora já não parecia
haver mais ratos, e apenas a Sra. Gardner fizera dos graciosos felinos
seus animais de estimação.

No dia dezenove de outubro Nahum cambaleou para dentro da casa de Ammi
com notícias horrorosas. A morte chegara para o pobre Thaddeus em seu
espaço no sótão, e acontecera de um modo que não se podia contar. Nahum
cavou uma sepultura no terreno familiar com gradil, atrás da fazenda, e
pôs lá o que encontrou. Não poderia ter sido nada vindo de fora, porque
a pequena janela com barras e a porta trancada estavam intactas; mas
fora tal como havia acontecido no celeiro. Ammi e sua esposa consolaram
o homem afligido o melhor que puderam, mas estremeceram ao fazê"-lo.
Terror absoluto parecia se materializar em torno dos Gardner e de tudo o
que tocavam, e em particular a presença de um deles em casa era como um
sopro das regiões sem nome, e inomináveis. Ammi acompanhou Nahum de
volta a casa com a maior relutância, e fez o que pôde para acalmar os
soluços histéricos do pequeno Merwin. Zenas não precisava se acalmar. Há
algum tempo já não fazia nada a não ser olhar fixamente para o vazio e
obedecer o que seu pai lhe dizia; e Ammi pensou que esse era um destino
misericordioso. Uma vez ou outra os gritos de Merwin eram respondidos
fracamente de dentro do sótão, e, respondendo a um olhar inquiridor,
Nahum disse que sua esposa se tornava muito frágil. Ao cair da noite,
Ammi arranjou um modo de ir embora; pois nem mesmo a amizade poderia
fazê"-lo ficar naquele lugar quando começava o discreto brilho na
vegetação e as árvores não se sabia se se moviam ou não sem o vento.
Ammi tinha a sorte de não ser mais imaginativo. Mesmo com as coisas do
jeito que estavam sua mente se alterou muito pouco; mas se tivesse sido
capaz de conectar as coisas e refletir sobre todos os portentos à sua
volta, inevitavelmente teria se tornado um arrematado maníaco.
Apressou"-se de volta a casa ao pôr"-do"-sol, com os berros da mulher
insana e os da criança em colapso ressoando horrivelmente em seus
ouvidos.

Três dias depois, Nahum arrastou"-se para dentro da cozinha de Ammi no
começo da manhã e, na ausência de seu anfitrião, gaguejou ainda outra
vez um conto desesperado enquanto a Sra. Pierce o ouvia em contrito
pânico. Dessa vez era sobre o pequeno Merwin. Ele se fora, e se fora na
alta noite com um lampião e um balde para água, e não voltara mais.
Vinha se desfazendo fazia dias, e tinha dificuldade de se situar. Urrava
contra tudo. Ouvira"-se um guinchado maníaco vindo do jardim, mas antes
que seu pai pudesse chegar à porta, o garoto se fora. Não havia a
luminosidade do lampião que levara e, do menino, nem sinal. Na hora,
Nahum pensou que o lampião e o balde tinham ido também; mas quando veio
a aurora e o homem rastejou de volta de sua busca noturna pelos bosques
e campos, encontrou algumas coisas muito curiosas próximas do poço.
Havia uma massa de ferro esmagada e, aparentemente, como que derretida,
que certamente fora antes o lampião; enquanto ao lado um fecho curvo e
argolas torcidas, em parte fundidos, pareciam sugerir os restos do
balde. Era tudo. Nahum estava além da imaginação, a sra. Pierce, pálida,
e Ammi, quando chegou em casa e ouviu a história, não conseguia opinar.
Merwin se fora, e era inútil falar com as pessoas das redondezas, que
agora evitavam os Gardner. Inútil, também, falar com as pessoas da
cidade, em Arkham, que riam de tudo. Thad se fora, e agora Merwin. Algo
se acercava, cada vez mais insidioso, e aguardava por ser visto e
sentido e ouvido. Nahum seria o próximo, e queria que Ammi cuidasse de
sua esposa e de Zenas, se sobrevivessem. Tudo aquilo tinha de ser algum
tipo de julgamento, ainda que não soubesse por que, pois, ao que sabia,
sempre andara com retidão nos caminhos do Senhor.

Por mais de duas semanas, Ammi não teve notícias de Nahum; daí,
preocupado com o que pudesse ter acontecido, superou seus receios e foi
à casa dos Gardner fazer uma visita. Não vinha fumaça da grande chaminé,
e por um momento o visitante receou o pior. O aspecto da fazenda toda
era chocante --- grama murcha e cinzenta e folhas pelo chão, vinhas se
desfazendo numa ruína de restos em muros arcaicos e arestas, e grandes
árvores nuas arranhando o céu cinzento de novembro com tão estudada
malevolência que Ammi não conseguia afastar a sensação de que isso se
devesse a certa mudança sutil no torneado dos galhos. Mas Nahum estava
vivo, apesar de tudo. Frágil, estendido num sofá na cozinha de teto
baixo, mas inteiramente consciente e capaz de dar ordens simples a
Zenas. O aposento era frio como a morte; e como Ammi tremia, o anfitrião
gritava roucamente a Zenas por mais lenha. Lenha, de fato, era de
primeira necessidade, pois a lareira cavernosa se achava apagada e
vazia, com uma nuvem de fuligem soprando no vento frio que descia pela
chaminé. Logo Nahum perguntou"-lhe se com a lenha extra ficava mais
confortável, e então Ammi viu o que acontecera: a fibra mais forte se
partira enfim, e a mente do fazendeiro desafortunado fechara"-se a novas
aflições.

Avançando com tato, Ammi não conseguiu informações muito claras sobre o
desaparecimento de Zenas. ``No poço --- ele vive no poço ---'' foi tudo
o que o nebuloso pai pôde dizer. Lampejou então na mente do visitante
uma lembrança fortuita da esposa insana, e daí mudou sua linha de
questionamento. ``Nabby? Por quê? aqui está ela!'', foi a resposta
surpreendente do pobre Nahum, e Ammi logo notou que teria de buscar por
si. Deixando o balbuciante inofensivo no sofá, tomou as chaves
penduradas no prego junto à porta e subiu as escadas rangentes para o
sótão. Era muito apertado e infecto lá em cima, e som algum se ouvia em
qualquer direção; das quatro portas à vista, apenas uma se encontrava
trancada, e nessa testou várias das chaves do molho que pegara. A
terceira mostrou"-se a correta e, após algum esforço, Ammi escancarou
aberta a portinhola branca.

Estava muito escuro lá dentro, pois a janela era pequena e em parte
obscurecida por barras rústicas de madeira; e Ammi não conseguia ver
nada no piso de longas tábuas. O fedor estava além do suportável, e
antes de prosseguir teve de recuar a outro aposento para só ressurgir
com seus pulmões cheios de ar respirável. Ao entrar viu algo escuro no
canto, e ao notá"-lo com mais clareza gritou imediatamente. Gritando,
pensou que uma nuvem fortuita eclipsara a janela, e um segundo depois
sentiu"-se roçado como se por uma odiosa corrente de vapor. Cores
estranhas dançavam diante de seus olhos; e se o presente horror não o
tivesse entorpecido, teria pensado naquele glóbulo do meteoro que o
martelo do geólogo despedaçara, e na vegetação mórbida que brotara na
primavera. Naquela situação, pensou apenas na monstruosidade blasfema
que o confrontava, e que bem claramente partilhara a sina inominável do
jovem Thaddeus e dos animais. Mas a coisa terrível a respeito desse
horror era que aquilo, de modo lento e perceptível, se movia enquanto
continuava a se esfacelar.

Ammi não me forneceu mais detalhes sobre essa cena, mas a forma naquele
canto não retornou em sua história como objeto em movimento. Há coisas
que não podem ser mencionadas, e o que se faz por gesto humano é às
vezes cruelmente julgado pela lei. Inferi que nada que se movesse fora
deixado naquele espaço do sótão, e que deixar uma coisa se movendo lá
teria sido um ato monstruoso a ponto de condenar qualquer ser capaz de
consciência a um tormento eterno. Ninguém senão um estólido fazendeiro
teria desmaiado ou enlouquecido, mas Ammi saiu lúcido pela portinhola e
trancou o segredo maldito atrás de si. Haveria que lidar com Nahum,
agora; precisa ser alimentado, tratado e removido para algum lugar onde
possam cuidar dele.

Iniciando a descida das escadas escuras, Ammi ouviu um baque lá embaixo.
Pensou mesmo que um grito se engasgara, e rememorou nervosamente o vapor
pegajoso que o havia roçado naquele apavorante cômodo de cima. Que
presença despertara com sua entrada e seu grito? Paralisado por um medo
indefinido, ouviu ainda outros sons vindos de baixo. Sem dúvida, havia
algo como um arrastar pesado, e um som detestável, viscoso, como se de
um tipo demoníaco e imundo de sucção; com um sentido associativo levado
a alturas febris, pensou inexplicavelmente no que vira lá em cima. Deus
meu! Que mundo insólito de sonho era esse em que se perdera? Não ousava
mover"-se para frente ou para trás, mas ficava ali tremendo na volta
escura da escada embutida. Cada mínima parte da cena se imprimiu a ferro
quente em seu cérebro. Os sons, a sensação da expectativa horripilante,
as trevas, a inclinação acentuada dos degraus estreitos
--- e, piedoso céu!\ldots{} a
luminosidade débil e inconfundível de todo o madeiramento à vista:
degraus, corrimãos, ripas expostas e as traves também!

Foi quando um relincho delirante irrompeu do cavalo de Ammi, do lado de
fora, seguido de imediato por um tropel que indicava fuga desvairada. Em
outro momento, cavalo e carruagem desapareceram de distância audível,
deixando um homem aterrorizado nas escadas escuras a adivinhar qual
teria sido a causa. Mas não era tudo. Ouvira"-se ainda outro som lá fora.
Um espirrar de líquido --- água
--- deve ter sido o poço. Ele
deixara Herói solto ali perto, e uma das rodas da carruagem terá raspado
no cimo, derrubando uma pedra. E a pálida fosforescência reluzia naquele
detestável madeiramento antigo. Deus! como era antiga, a casa! A maior
parte construída antes de 1670, e o telhado de mansarda datava no máximo
de 1730.

Um sutil arranhar no assoalho ao fim da escada agora soava
distintamente, e Ammi apertou mais forte o bastão pesado que pegara no
sótão com algum propósito. Lutando com seus nervos, terminou a descida e
se moveu com ousadia em direção à cozinha. Mas não completou o caminho,
porque o que buscava não se encontrava mais lá. Vinha a seu encontro, e
vivia, de certa forma. Se aquilo rastejara ou se fora arrastado por
força externa, Ammi não saberia dizer, mas a morte estivera lá. Tudo
havia acontecido na última meia"-hora, mas o colapso, o acinzentar e a
desintegração estavam já muito adiantados. Vinha horrivelmente
desmanchado, e fragmentos ressequidos descamavam. Ammi não conseguiria
tocá"-lo, e olhava horrorizado para aquela distorcida paródia do que fora
um rosto. ``O que houve, Nahum --- o que houve?'' sussurrou, e os lábios
rachados, intumescidos, puderam apenas crepitar uma última resposta.

``Nada\ldots{} nada\ldots{} a cor\ldots{} queima\ldots{} fria e úmida\ldots{} mas
queima\ldots{} vivia no poço\ldots{} eu vi\ldots{} que nem um fumo\ldots{} que
nem as frô na primavera passada\ldots{} o poço brilhava na noite\ldots{}
Thad e Mêrnie e Zenas\ldots{} tudo vivo\ldots{} sugano a vida de tudo\ldots{}
naquela rocha\ldots{} tem que tê vino naquela rocha\ldots{} invenenô o lugar
todo\ldots{} num sei o que quer\ldots{} a coisa redonda que os ôme da
facudade arrancaro da rocha\ldots{} rebentaro\ldots{} era da mema cor\ldots{}
tudo igual, que nem as frô e as pranta\ldots{} tinha que tê mais deles\ldots{} semente\ldots{} semente\ldots{} elas crescia\ldots{} vi pela primera veiz
essa semana\ldots{} pegô forte no Zenas\ldots{} meninão graúdo, cheio de
vida\ldots{} derruba suas ideia e daí pega ocê\ldots{} torra ocê todo\ldots{}
na água do poço\ldots{} cê tava certo nisso\ldots{} água má\ldots{} Zenas
nunca voltô do poço\ldots{} num sai\ldots{} agarra ocê\ldots{} cê vê que tem
um troço vino, mas nem adianta\ldots{} vi muitas veiz, dêis que pegô o
Zenas\ldots{} cadê a Nabby, Ammi?\ldots{} minha cabeça num tá bem, não\ldots{}
num sei quanto tempo faiz que dei comida pra ela\ldots{} vai catá ela se
nóis num cuidá\ldots{} só uma cor\ldots{} a cara dela tá ficano daquela cor
virada da noite\ldots{} e queima e suga\ldots{} vem de um lugar onde as coisa
num é que nem aqui\ldots{} um dos professô foi que disse\ldots{} tava certo\ldots{} prestenção, Ammi, vai fazê mais coisa\ldots{} suga a vida de\ldots{}''

E isso foi tudo. Aquilo que falara não podia mais falar, porque se
desmontara por completo. Ammi estendeu uma toalha de mesa, xadrez
vermelha, sobre o que restara, e cambaleou pela porta dos fundos em
direção aos campos. Subiu o declive dos pastos de dez acres de extensão
e arrastou"-se para casa, pela estrada ao norte e pelos bosques. Não
podia passar por aquele poço de que seu cavalo fugira. Havia olhado pela
janela, e notara que pedra alguma faltava ao beiral. Enfim, a carruagem
afobada não havia deslocado coisa alguma --- o jorro d'água fora de
alguma outra coisa --- algo que
entrou no poço após fazer o que fizera ao pobre Nahum\ldots{}

Quando Ammi chegou em casa o cavalo e a carruagem já estavam lá, o que
pusera sua esposa em ataques de ansiedade. Tranquilizando"-a sem maiores
explicações, foi de uma vez a Arkham e notificou as autoridades de que a
família Gardner já não existia. Não entrou em detalhes, mas relatou as
mortes de Nahum e Nabby --- a de Thaddeus já era sabida
---, e mencionou que a causa
parecia ser a mesma estranha afecção que vitimara os animais. Ele também
declarou que Merwin e Zenas haviam desaparecido. Houve um interrogatório
considerável no posto policial, e ao fim instaram Ammi a levar três
guardas à fazenda dos Gardner, juntamente com o magistrado, o legista e
o veterinário que tratara dos animais afetados. Ele foi, muito contra
sua vontade, pois a tarde já avançava e temia o cair da noite naquele
lugar maldito, mas era de algum conforto ter tantas pessoas consigo.

Os seis homens se foram numa caleche, seguindo a carruagem de Ammi, e
chegaram na fazenda infestada por volta das quatro horas. Ainda que
acostumados a experiências abomináveis, como eram os guardas, nenhum
permaneceu impassível ao que se encontrou no sótão e sob a toalha xadrez
vermelha, lá embaixo. O aspecto geral da fazenda com sua desolação
cinzenta era suficientemente terrível, mas aqueles dois objetos se
desintegrando iam além de todos os limites. Ninguém era capaz de
olhá"-los por muito tempo, e mesmo o legista admitiu que não havia lá
muito o que examinar. Amostras podiam ser analisadas, claro, e então se
ocupou de obtê"-las --- e segue"-se
que houve certo desenvolvimento desconcertante no laboratório da
universidade, para onde os dois frascos de pó foram levados. Sob o
espectroscópio, ambas as amostras emitiam um espectro desconhecido, em
que muitas das faixas aberrantes eram precisamente como as que o
estranho meteoro gerara no ano anterior. A propriedade de emissão desse
espectro desapareceu em um mês, o pó consistindo então apenas de
fosfatos alcalinos e carbonatos.

Ammi não teria contado aos homens sobre o poço se achasse que
pretenderiam fazer algo ali, e naquele momento. Aproximava"-se o
pôr"-do"-sol, e ele estava ansioso por ir embora. Mas não conseguia evitar
olhar nervoso e de soslaio para a borda de pedra junto da enorme picota,
e quando um detetive o questionou ele admitiu que Nahum temia algo lá
embaixo --- e tanto que jamais
havia considerado procurar Merwin e Zenas lá dentro. Depois disso nada
mais importava senão explorar imediatamente o poço, e assim Ammi teve de
aguardar, tremendo enquanto balde após balde de água fétida era içado e
entornado no chão que se encharcava. Os homens inspiravam aquele fluido
com desgosto, e perto do fim já tapavam os narizes contra a fedentina
que iam desvendando. Não foi um trabalho tão longo quanto temiam, pois a
água estava fenomenalmente baixa. Não há necessidade de falar com muita
exatidão do que encontraram. Merwin e Zenas estavam ambos lá, em parte,
embora os vestígios fossem sobretudo esqueléticos. Também havia um
pequeno cervo e um cachorro grande, mais ou menos nas mesmas condições,
e uns tantos ossos de animais menores. O lodo e o limo ao fundo pareciam
inexplicavelmente porosos e borbulhantes, e um homem que desceu
apoiando"-se nas mãos e com um longo bastão notou que poderia fincar a
haste de madeira em qualquer profundidade na lama daquele chão sem achar
qualquer obstrução sólida.

Baixara o crepúsculo, e assim trouxeram lampiões de dentro de casa.
Então, quando perceberam que mais nada ganhariam com o poço, todos
entraram e conferenciaram na velha sala"-de"-estar enquanto a luz
intermitente de uma meia"-lua espectral brincava pálida com a desolação
cinzenta de fora. Os homens estavam francamente perplexos com o caso
todo, e não encontravam elemento comum que fosse convincente para ligar
as estranhas condições dos vegetais, a doença desconhecida de animais e
humanos e as mortes inexplicáveis de Merwin e Zenas no poço contaminado.
Haviam ouvido as histórias habituais da região, é verdade; mas não
podiam acreditar que qualquer coisa contrária às leis da natureza
tivesse ocorrido. Sem dúvida o meteoro envenenara o solo, mas a moléstia
de pessoas e animais que não comeram coisa alguma nascida daquele solo
era ainda outra questão. Seria a água do poço? Bastante provável. Talvez
seja uma boa ideia analisá"-la. Mas que peculiar loucura fizera ambos os
garotos pular no poço? Suas ações haviam sido tão similares
--- e os fragmentos mostraram que
ambos sofreram aquela morte cinzenta e quebradiça. Por que tudo se
tornava tão cinzento e quebrável?

Foi o magistrado, sentado perto de uma janela que dava para o jardim,
quem primeiro notou um cintilar sobre o poço. Era já noite fechada, e os
campos aberrantes pareciam ligeiramente luminosos com mais do que a
vacilante luz do luar; mas essa nova fulgurância era algo definido e
distinto, e se assemelhava a um suave raio de holofote disparado de
dentro do buraco escuro, gerando reflexos opacos nas poças exíguas do
chão onde a água fora despejada. Possuía uma cor muito esquisita, e, com
todos os homens apinhados junto à janela, Ammi foi acometido de
sobressalto violento, pois aquele estranho raio de um miasma hediondo
era"-lhe de matiz familiar. Havia visto a cor antes, e receava o
significado daquilo. Ele a havia visto no glóbulo quebradiço e asqueroso
do aerólito de dois verões passados, a havia visto na vegetação absurda
da primavera, e pensava tê"-la visto por um átimo, naquela mesma manhã,
contra a pequena janela com barras naquele canto do sótão terrível, onde
coisas inomináveis aconteceram. Relanceara por um segundo, e uma
corrente pegadiça e odiosa de vapor roçara por ele
--- e o pobre Nahum fora tomado
por algo daquela cor. Ele o dissera ao fim --- dissera que haviam sido o
glóbulo e as plantas. Depois daquilo, a fuga e o jato d'água no poço
--- e agora aquele mesmo poço
vomitava sobre a noite um raio pálido e insidioso da mesma tintura
demoníaca.

Louvável a presença de espírito de Ammi, que, mesmo naquele momento
tenso, se intrigou com um ponto essencialmente científico. Não pôde
deixar de especular sobre ter captado a impressão de um vapor num
vislumbre diurno, contra uma janela aberta ao céu matutino, e numa
exalação noturna vista como uma névoa fosforescente contra a paisagem
escura e esfacelada. Não estava certo --- era contra a Natureza --- e
pensou naquelas temíveis palavras finais de seu desolado amigo, ``Vem de
um lugar onde as coisa num é que nem aqui\ldots{} um dos professô foi que
disse\ldots{}''

Todos os três cavalos lá fora, amarrados a um par de tocos secos na
estrada, agora relinchavam e pateavam freneticamente. O cocheiro correu
à porta para fazer algo, mas Ammi pousou a mão trêmula sobre seu ombro.
``Num vai lá'', sussurrou. ``Tem uns trem que a gente num sabe aí. Nahum
falô que a coisa vive no poço e suga a vida fora d'ocê. Falô que tem que
tê nascido de uma bola redonda que nem a que todo mundo viu na rocha do
meteoro que caiu faz um ano em junho. Suga e queima, ele falô, e é só
uma nuvem de cor, que nem aquela ali fora, qu'ocê nem vê direito e num
sabe dizê o que é. Nahum achô que se alimenta de tudo vivo e fica cada
veiz mais forte. Falô que viu ela semana passada. Deve de sê um trem de
bem longe no céu, que nem os ôme da facudade dissero que a rocha do
meteoro era. O jeito que é feito, e o jeito que anda, num é desse mundo
de Deus. É um trem aí de bem longe''.

E assim os homens hesitaram enquanto a luz vinda do poço ficava mais
forte e os cavalos escoiceavam e relinchavam num frenesi crescente. Foi
um momento de fato temível, com o terror naquela casa antiga e
amaldiçoada, os quatro monstruosos punhados de fragmentos --- dois da
casa e dois do poço --- no barraco dos fundos, e aquele feixe de
iridescência desconhecida e diabólica das profundezas limosas na frente.
Ammi refreara o cocheiro por impulso, esquecendo quão ileso ele mesmo
saíra do roçar viscoso daquele vapor colorido no espaço do sótão, mas
talvez tenha sido bom ter agido como agiu. Ninguém jamais saberá o que
estava lá fora naquela noite; e apesar de a blasfêmia do espaço até
então não ter ferido humanos de mente sã, não era possível saber o que
poderia ter feito naquele momento final, e com sua força aparentemente
ampliada e os sinais particulares de intenção que logo mostraria sob
aquele céu nublado e à luz do luar.

Naquele preciso momento um dos detetives junto da janela engoliu seco.
Os outros voltaram os olhos para ele e rapidamente seguiram seu olhar
para o alto, para o ponto onde seu movimento casual havia parado. Não
havia necessidade de palavras. O que se discutia da fofoca interiorana
já não se discutiria, e é por causa da coisa que cada homem daquele
grupo concordou em apenas sussurrar sobre no futuro que nunca mais se
falou em Arkham a respeito dos dias estranhos. É preciso iniciar com a
premissa de que não havia vento àquela hora da noite. Surgiu um, não
muito depois, mas naquele momento não havia nada. Mesmo as pontas secas
da erva"-rinchão, cinzentas e infestadas, e a franja da capota da caleche
estavam imóveis. E mesmo assim, naquela tensa, nefanda calmaria, os
galhos altos e nus de todas as árvores no jardim se moviam. Eles se
contraíam num mórbido espasmo, arreganhando"-se numa insanidade
convulsiva, epilética, contra as nuvens sob a luz do luar; arranhavam
impotentes o ar nocivo como se torcidos por alguma linha alienígena e
sem corpo ligada a horrores subterrâneos que se reviravam e lutavam sob
as raízes escuras.

Nem um homem sequer respirou por uns tantos segundos. Daí uma nuvem de
escuridão mais profunda estendeu"-se sobre a lua e a silhueta de galhos
preênseis se dissipou momentaneamente. Diante disso soou um grito
uníssono, abafado de espanto, mas rouco e quase idêntico em todas as
gargantas. Pois o terror não se dissipara com a silhueta, e num temível
instante de maior escuridão os circunstantes viram, serpenteando no topo
das árvores, mil mínimos pontos de ímpia irradiância, entornando de cada
ramo como fogo de santelmo ou as flamas que desciam às cabeças dos
apóstolos no Pentecostes. Monstruosa constelação de luz antinatural,
como um enxame de vagalumes, saciados em cadáveres, dançando sarabandas
infernais sobre um pântano repulsivo; e sua cor era daquela mesma
inominável intrusão que Ammi passara a discernir e temer. O feixe de
fosforescência vindo do poço tornava"-se mais e mais brilhante com o
tempo, trazendo às mentes dos homens encolhidos uma sensação de agouro e
anormalidade que superava em muito qualquer outra imagem que suas mentes
conscientes pudessem conjurar. Já não \emph{brilhava}, mas
\emph{transbordava}; e, ao deixar o poço, aquele fluxo sem forma, na cor
indefinível, parecia fluir diretamente para o céu.

O veterinário estremecia, e caminhou até a porta da frente para descer a
pesada barra extra, trancando"-a. Ammi não tremia menos, e teve de
cutucar e apontar, por falta de uma voz estável, quando quis chamar
atenção para a crescente luminosidade das árvores. O relinchar e
escoicear dos cavalos se tornara totalmente pavoroso, mas sequer uma
alma entre eles, naquela velha casa, teria se arriscado a sair, nem por
todo o dinheiro do mundo. O brilho das árvores aumentava com o tempo,
enquanto seus galhos incansáveis pareciam se esticar mais e mais para o
alto. A madeira da picota de poço agora brilhava, e logo um policial
apontou mudamente para alguns galpões e colmeias perto do muro de pedra,
a oeste. Começavam a brilhar, também, embora os veículos dos visitantes
parecessem, até então, ilesos. Houve um tumulto selvagem e o som de
galope na estrada, e ao reabastecer o lampião para ver melhor, Ammi
percebeu que a parelha de tordilhos desenfreados havia rompido as
amarras e arrastara consigo a caleche.

O choque serviu a soltar várias línguas, e trocaram sussurros
constrangidos: ``Se espalha para tudo o que é orgânico nas redondezas'',
murmurou o legista. Ninguém respondeu, mas o homem que estivera no poço
cogitou que seu longo bastão talvez tenha agitado algo intangível. ``Foi
horrível'', acrescentou. ``Não tinha fundo, só uma gosma, bolhas e a
sensação, lá, de algo à espreita.'' O cavalo de Ammi ainda escoiceava e
urrava ensurdecedor na estrada lá fora, e quase abafando o sutil tremor
de seu dono, que murmurava suas reflexões sem forma. ``Veio daquela
rocha\ldots{} foi cresceno na fundura\ldots{} pegô tudo vivo\ldots{} se
alimentô deles, corpo e arma\ldots{} Thad e Mêrnie, Zenas e Nabby\ldots{}
Nahum foi o útimo\ldots{} todos bebêro da água\ldots{} ficô forte neles\ldots{} veio de longe, onde as coisa num é que nem aqui\ldots{} e agora tá
vôtano pra casa\ldots{}''

Nesse ponto, quando de repente a coluna de cor desconhecida refulgiu
mais forte e começou a se trançar em fantásticas sugestões de forma que
cada espectador mais tarde descreveu de modo diverso, ouviu"-se do pobre
Herói, amarrado, um som que ninguém antes, ou desde então, ouviu de um
cavalo. Todos naquela sala do andar de baixo taparam os ouvidos, e Ammi
afastou o rosto da janela, com horror e náusea. Palavras jamais poderiam
expressar --- quando Ammi olhou novamente, a criatura infeliz jazia,
pilha de carne inerte sobre o chão enluarado, entre as lascas farpadas
da carruagem. Foi o que se soube de Herói, até que o enterraram no dia
seguinte. Mas não era hora para lamentos, pois quase no mesmo instante
um detetive silenciosamente chamou a atenção para algo terrível na
própria sala onde estavam. Na ausência da luz de lampião era claro que
uma tênue fosforescência começara a permear o aposento todo. Fulgurava
no piso de tábuas amplas e no fragmento de tapete de pano, cintilando
sobre os caixilhos das pequenas janelas. Subia e descia nos cantos,
resplandecia sobre a estante e a lareira, e infectava mesmo as portas e
a mobília. Tornava"-se mais forte a cada minuto, e por fim ficou óbvio
que tudo o que era vivo e saudável deveria deixar a casa.

Ammi levou"-os à porta dos fundos e ao caminho do alto, pelos campos do
pasto de dez acres. Caminharam tropeçando como se num sonho, e não
ousaram olhar para trás até que estivessem bem distantes e em terreno
mais elevado. Ficaram felizes com aquele caminho, pois jamais poderiam
ter ido pela frente, pelo poço. Era suficientemente ruim ter de passar
pelo celeiro e pelos barracos fulgurantes, e aquelas árvores brilhando
no pomar com seus contornos retorcidos, demoníacos; mas, graças aos
céus, os galhos faziam suas piores torções no topo. A lua afundou sob
algumas nuvens muito escuras quando cruzaram a ponte rústica sobre o
Ribeirão de Chapman, e dali até campo aberto foi um cego tatear.

Quando olharam para trás, na direção do vale e da distante casa dos
Gardner, lá no fundo, tiveram uma visão temível. Toda a fazenda brilhava
com aquela horrenda mescla de cor; árvores, construções e mesmo a grama
e as ervas que ainda não haviam adquirido aquele tom cinza, farelento e
letal. Os ramos todos se franziam na direção do céu, com línguas da
flama funesta no topo, e as reluzentes fagulhas do mesmo fogo monstruoso
crepitavam sobre as vigas da casa, o celeiro e os barracos. A cena era
como uma das visões de Fuseli,\footnote{Johann Heinrich Füssli
  (1741--1825), ou Henry Fuseli, foi um pintor suíço, bastante incomum,
  cujo tema favorito e notório eram perturbadoras imagens sobrenaturais.
  Influenciou William Blake (1757--1827).} e sobre todo o resto reinava
aquele tumulto de luminosidade amorfa, aquele arco"-íris alienígena e
unidimensional de críptico veneno vindo do poço --- inquieto, sondando,
saltando, buscando, cintilando, premindo e malignamente borbulhando em
seu cromatismo cósmico e incompreensível.

E então, de surpresa, a coisa horrenda disparou verticalmente em direção
ao céu, como um foguete ou meteoro, sem deixar atrás de si rastro algum,
desaparecendo no meio de um buraco em forma de um círculo curiosamente
regular em meio às nuvens, antes que alguém tivesse a chance de engolir
ou gritar. Nenhum dos que observavam pôde esquecer aquela visão, e Ammi
olhou pasmo para as estrelas de Cisne,\footnote{A constelação do Cisne,
  que tem 50 estrelas visíveis, das quais Deneb é uma. É também
  constelação considerada desde a antiguidade como a origem da vida, à
  qual se orientam construções na Terra, e que ativou a imaginação de
  muitos que especulam sobre vida fora dela.} Deneb brilhando mais do
que as outras, onde a cor desconhecida se dissolveu na Via Láctea. Mas
seu olhar logo foi chamado de volta à Terra por um estrondo no vale. Foi
apenas isso. Somente um estrondo e um rebentar de madeira, e não uma
explosão, como outros do grupo juraram. No entanto o resultado foi o
mesmo, pois, em um instante febril e caleidoscópico, irrompeu daquela
fazenda amaldiçoada, perdida, um cataclisma eruptivo e faiscante de
centelhas e substância não"-naturais, borrando a vista dos poucos que o
presenciaram, e enviando ao zênite um tal dilúvio de fragmentos
fantásticos e coloridos que nosso universo precisa repudiar. Por meio de
rápidos vapores que fechavam"-se atrás de si, os fragmentos seguiram a
grande morbidez que desaparecera, e no segundo seguinte haviam
desaparecido também. Atrás e embaixo havia só uma escuridão à qual os
homens não ousavam voltar, e em toda parte se levantava um vento
crescente que parecia varrer tudo em negras e álgidas rajadas de espaço
interestelar. Guinchava e uivava, e açoitava os campos e o bosques
distorcidos num louco frenesi cósmico, e em pouco tempo o grupo temeroso
percebeu que de nada adiantaria aguardar que a lua mostrasse o que
restara da fazenda de Nahum.

Pasmos demais até para aventar teorias, os sete homens sofridos
marcharam de volta a Arkham pela estrada do norte. Ammi estava pior do
que seus colegas, e implorou que o acompanhassem até sua cozinha ao
invés de seguirem direto para a cidade. Não desejava cruzar sozinho os
bosques notívagos, varados de vento, da estrada principal no caminho de
casa, pois tivera um outro choque de que os demais haviam sido poupados.
Quando os outros do grupo, naquele monte tempestuoso, puseram os olhos
firmemente na estrada, Ammi por um instante lançou o olhar para trás, na
direção daquele vale sombrio de desolação, que abrigara até há pouco seu
amigo de má"-estrela. E daquele lugar distante, afligido, ele vira algo
se erguendo frágil, apenas para afundar de novo naquele lugar de onde o
grande horror sem forma disparara ao céu. Era só uma cor --- mas cor
alguma desta Terra ou sequer dos céus. Mas pelo fato de que Ammi
reconhecera aquela cor, e sabia que aquele tênue remanescente ainda
devia espreitar de dentro do poço, nunca mais pôde estar realmente bem.

Ammi jamais se aproximaria daquele lugar. Faz mais de meio século desde
que aquele horror aconteceu, mas nunca estivera mais lá, e ficará feliz
quando o novo reservatório apagar aquilo. Ficarei feliz, também, porque
não me agradou o modo como a luz do sol mudou de cor em torno da
abertura daquele poço abandonado, quando passei. Espero que a água seja
sempre muito profunda --- mesmo assim, eu jamais a beberia. Não creio
que visitarei a região de Arkham outra vez. Três dos homens que
estiveram com Ammi retornaram na manhã seguinte para ver as ruínas à luz
do dia, mas não havia propriamente ruínas, apenas os tijolos da chaminé,
as pedras do porão, alguns detritos minerais e metálicos aqui e ali, e o
beiral daquele poço nefando. Com a exceção do cavalo morto de Ammi, que
rebocaram e enterraram, e a carruagem que lhe levaram de volta, tudo o
que vivera tinha acabado. Restavam cinco acres insólitos de cinzenta
poeira desértica, e nada jamais voltou a nascer ali. Até hoje se
alastram a céu aberto como um enorme espaço devorado por ácido nos
bosques e nos campos, e os poucos que ousaram pousar os olhos nele a
despeito dos contos rurais o chamaram ``brenha dos infernos''.

Os contos rurais são esquisitos. Poderiam ser ainda mais esquisitos se
as pessoas da cidade e os químicos da universidade se interessassem o
bastante para analisar a água daquele poço em desuso, ou a poeira
cinzenta que vento algum parece dispersar. Botânicos também deveriam
estudar a flora atrofiada nos limites daquele lugar, pois talvez
lançassem luz à opinião interiorana de que o flagelo está se
espalhando
--- pouco a pouco, talvez uma
polegada por ano. As pessoas dizem que a cor da relva nas proximidades
não está lá muito certa na primavera, e que coisas selvagens deixam
pegadas esquisitas na neve suave do inverno. A neve nunca parece ser tão
pesada na brenha dos infernos quanto em outras partes. Cavalos --- os
poucos que restaram nesta era dos motores --- tornam"-se inquietos no
vale silencioso; e caçadores não podem depender de seus cães muito perto
daquela nódoa de poeira acinzentada.

Dizem que as influências mentais são muito más, também. Muitos se
tornaram esquisitos, anos após o desaparecimento de Nahum, e sempre lhes
faltou força para fugir. E então aqueles de mente forte deixaram, todos,
a região, e só os forasteiros tentaram viver nas propriedades velhas e
decadentes. Não podiam permanecer, no entanto; por vezes é mesmo de se
especular que impressões, além das nossas, seus estoques furiosos e
fantásticos de sussurrada magia lhes terão dado. Seus sonhos noturnos,
afirmam, são bastante horríveis naquele lugar grotesco; e, sem dúvida
alguma, já a aparência daquele domínio escuro é o bastante para atiçar
uma imaginação mórbida. Nenhum viajante jamais deixou de sentir uma
sensação de estranheza naquelas ravinas profundas, e artistas se
arrepiam ao pintar bosques espessos cujo mistério é tanto do espírito
quanto do olho. Eu mesmo estou curioso sobre a sensação que me veio da
minha longa caminhada antes de Ammi me contar seu conto. Quando chegou o
pôr"-do"-sol, me veio um desejo difuso de que algumas nuvens se juntassem,
pois uma peculiar hesitação a respeito dos profundos vazios do céu acima
rastejara para dentro da minha alma.

Não me perguntem a minha opinião. Não sei --- eis tudo. Não havia
ninguém a quem questionar, fora Ammi; pois o grupo de Arkham não falava
sobre os dias estranhos, e todos os três professores que viram o
aerólito e seu glóbulo colorido morreram. Havia outros glóbulos --- e
tudo depende disso. Um deve ter se alimentado e escapado, e é provável
que houvesse outro, que se atrasou. Sem dúvida, ainda está no fundo do
poço --- sabia que havia algo
errado com a luz do sol que vi sobre aquele beiral miasmático. Os
rústicos dizem que a praga se arrasta uma polegada por ao, então talvez
haja um tipo de crescimento, ou nutrimento, ainda agora. Seja lá qual
for o filhote de demônio gestando ali, deve estar retido por alguma
coisa, ou teria rapidamente se espalhado. Estaria enlaçado às raízes
daquelas árvores que cavoucam em busca de ar? Um dos causos atuais em
Arkham é sobre carvalhos gordos que brilham e se movem como não
deveriam, à noite.

O que isso é, só Deus sabe. Em termos materiais, suponho que a coisa que
Ammi descreveu seria um gás, mas esse gás obedecia a leis que não são do
nosso cosmo. Não era fruto de mundos e sóis tais como os que brilham nos
telescópios e nas pranchas fotográficas dos nossos observatórios. Aquilo
não era um sopro dos céus, cujos movimentos e dimensões nossos
astrônomos medem ou julgam vastos demais para medir. Era apenas uma cor
que caiu do espaço --- um
mensageiro apavorante de domínios sem forma do infinito, para além de
toda a Natureza como a conhecemos; de domínios cuja mera existência
atordoa o cérebro e nos entorpece com golfos negros extracósmicos que
escancara diante de nossos olhos alucinados.

Duvido muito que Ammi tenha conscientemente mentido para mim, e não acho
que seu conto fosse todo uma aberração de loucura, como o pessoal da
cidade havia advertido. Algo terrível veio às colinas e vales naquele
meteoro, e algo terrível --- embora eu não saiba em que proporção ---
permanece lá ainda. Ficarei feliz de ver a água chegar. Enquanto isso,
espero que nada aconteça a Ammi. Ele viu tanto da
coisa --- e sua influência fora
tão insidiosa. Por que nunca foi capaz de se mudar de lá? E como
lembrava daquelas últimas palavras de Nahum --- ``num sai\ldots{} agarra
ocê\ldots{} cê vê que tem um troço vino, mas nem adianta\ldots{}'' Ammi é um
senhor muito bom --- quando a gangue do reservatório chegar para
trabalhar devo escrever ao engenheiro"-chefe para que fique de olho
atento nele. Odiaria pensar em Ammi como naquela monstruosidade
cinzenta, retorcida e quebradiça que insiste mais e mais em perturbar o
meu sono.
