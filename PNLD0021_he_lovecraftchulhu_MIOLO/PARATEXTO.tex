\chapter{Lovecraft e o horror}

\begin{flushright}
\textsc{dirceu villa}
\end{flushright}
\bigskip

\section{Sobre o autor}

\noindent{}Howard Phillips Lovecraft (1890--1937) já foi largamente traduzido e lido
no Brasil, e os fatos de sua vida são também bastante notórios: nascido
em Providence, Rhode Island, no topo dos \versal{EUA}, lugar quase enfiado nas
águas do Atlântico, Lovecraft veria seu pai morrer ainda cedo de
complicações da sífilis, que o enlouqueceram (dizia coisas bizarras ao
filho), e sua mãe, em consequência, teria passado o resto da vida em
escuríssima amargura (paparicando Lovecraft e punindo"-o psicologicamente
na mesma medida), o que fez com que o garoto ficasse muito próximo do
avô, Whipple Van Buren Phillips (1833--1904), rico empreendedor. Teve uma
infância com o mesmo aspecto de seus contos, obrigado a percorrer
quartos escuros, imaginando diabretes de Doré perseguindo"-o com
tridentes pela casa, descobrindo o sexo com desgosto em livros de
anatomia e decidindo aos cinco anos, nietzscheanamente, que Deus não
passava de um mito, após ouvir dizer o mesmo sobre o Papai
Noel.\footnote{O que se pode ler no importante e interessantíssimo
  documento autobiográfico, o ensaio ``A Confession of Unfaith'' (1922),
  no qual Lovecraft afirma que pouco antes já se dizia um devoto
  muçulmano e se dava o nome de Abdul Alhazred (por seu fascínio pelas
  \emph{Mil e uma noites}) mais tarde imortalizado como o \emph{poeta
  árabe insano} de sua ficção.}

A infância --- que ao menos tivera o conforto da riqueza do avô ---
acabou também cedo, com a desgraça empresarial seguida da morte do velho
homem por infarto: Lovecraft, a mãe e uma tia solteira foram então
lançados à pobreza e a um tanto de desespero. Obviamente, sua vida
escolar sofreu com as agruras familiares, e Lovecraft percorreu a escola
de modo errático até que o trajeto fosse interrompido em definitivo por
um colapso nervoso no segundo grau; o mais próximo que chega de
prosseguir com alguma educação formal é um curso de química que faz por
correspondência. Essa história frustrada lhe deixou marcas profundas de
um desejo de se provar intelectualmente diferenciado, especial, o que se
nota mesmo em sua escrita de ficção, muitas vezes na forma de avaliação
da inteligência (ou falta de) de seus personagens.

Interessado em ciências e de imaginação vivíssima, também alimentada
pela biblioteca do avô (onde desde muito cedo leu livros robustos como
as \emph{Metamorfoses} de Ovídio e \emph{A rima do velho marinheiro} de
Coleridge, entre outros) e por uma rotina de recitação de Shakespeare
com a mãe, Lovecraft era por assim dizer um \emph{nerd avant la lettre},
uma fina sensibilidade cultivada --- e estraçalhada --- no alto da
sociedade industrial e mecânica, entre horrores psicológicos, ciência,
timidez erótica, desajuste social e medo do mundo, suscetível a colapsos
nervosos, e insone.

Lovecraft não parece ter sido um tipo agradável, aspecto biográfico que
partilha com aquele de quem --- não sem bons motivos --- se diz que
descende: Edgar Allan Poe (1809--1849). Mas a passagem de Poe para
Lovecraft explica"-nos igualmente um pouco da história dos \versal{EUA}, o país de
ambos, e onde ambos viveram quase anônimos: se Poe era um alcoólatra
neurótico, Lovecraft foi um ultraconservador paranoico, repleto de
preconceitos enraizados e violentos. Penso que a \emph{doença} --- se
pudermos utilizar a palavra com alguma licença poética --- de Poe como a
de Lovecraft é a \emph{doença da percepção}. Os dois notaram um complexo
de horrores futuros, ainda sem forma, mas que perturbavam suas finas
percepções. Se Poe herdou as visões perturbadoras do alemão E.\,T.\,A.
Hoffmann (1776--1822),\footnote{Contos como ``Die Automate'' (Os autômatos)
  e ``Der Sandmann'' (O homem da areia) são duas das mais importantes
  peças desse gênero de literatura, cujos desdobramentos reais Sigmund
  Freud teria a felicidade de nomear ``Das Unheimliche'' (O inquietante,
  1919), no valioso ensaio que define um tipo recente de horror, o do
  familiar"-estranho, que veremos adiante.} Lovecraft herdaria, por sua
vez, as de Poe.\footnote{E há quem diga que Stephen King herdou de
  Lovecraft. O que não parece muito exato, porque o maciço e contínuo
  sucesso multimilionário torna King menos perceptivo. Como King mesmo
  disse, sua literatura é o equivalente de um Big Mac. Nada contra o Big
  Mac, nem contra King, mas isso explica por que o autor detestou a
  adaptação cinematográfica do seu \emph{The Shining} (O iluminado,
  1977) por Stanley Kubrick naquela obra"-prima de 1980: sua percepção,
  como artista, está limitada aos \emph{efeitos}, não vê
  \emph{estrutura}.}

Haveria apenas um outro ponto fundamental para entender a estrutura
mental do horror lovecraftiano: Mary Shelley (1797--1851) com seu
\emph{Frankenstein} (1818). Lá se encontra pela primeira vez o tipo de
horror científico que se entrevira nos autômatos de Hoffmann, ele mesmo
um passo adiante das narrativas que o grupo de Byron leu naquela famosa
estadia na Suíça, o \emph{Gespensterbuch} (O livro de fantasmas, 1811--1815)
de Johann August Apel e Friedrich Laun, em que seus organizadores reúnem
e reescrevem antigas narrativas folclóricas de horror germânico. Mary
Shelley leva essa narrativa a um ponto que não se poderia imaginar
antes, trazendo o foco a uma absoluta \emph{hybris} da razão.\footnote{Para
  notar como Mary Shelley antecipou questões profundas da humanidade
  basta lembrar da frase de Robert Oppenheimer (1904--1967), um dos
  grandes físicos a desenvolver o Projeto Manhattan, o da bomba de
  hidrogênio, após a explosão das bombas de Hiroshima e Nagasaki: ``os
  cientistas conheceram o pecado''. Mary Shelley o soubera muito antes,
  como a arte em geral o sabe.}

O subtítulo a \emph{Frankenstein}, ``Prometeu moderno'', é precisamente
o ponto: a luz da ciência, como sabemos, projeta sombras largas, e
Shelley o nota, porque afirma que escreveria dos ``medos misteriosos da
nossa natureza'', mas também, e sobretudo, do que surgiu nas conversas
dos escritores reunidos sobre ``a natureza do princípio da vida'', em
particular de um experimento do Dr.\,Erasmus Darwin (1731--1802), que lhe
deixou a hipótese\footnote{``Eles falaram'' {[}\emph{eles} são Byron,
  Shelley, Polidori{]} ``dos experimentos do Dr.\,Darwin (não falo do que
  o doutor realmente fez ou disse que fez, mas, mais ajustado ao meu
  objetivo, o que então teria sido feito por ele), que preservou um
  pedaço de verme num vidro até que de alguma maneira extraordinária
  começou a se mexer com movimentos voluntários. Não é assim, afinal,
  que se daria vida. Talvez o cadáver tenha sido reanimado; galvanismo
  já dera exemplo de tais coisas: talvez as partes que compunham a
  criatura pudessem ser manufaturadas, recompostas, e dotadas de calor
  vital''. ``Author's introduction'', \emph{in}: \textsc{shelley}, Mary.
  \emph{Frankenstein or, The Modern Prometheus}. New York: The Modern
  Library, 1993.} que o próprio Lovecraft depois revisitaria em
``Herbert West -- Reanimator'' (1921--1922).

De resto, como sabemos ao menos desde a frase atribuída a Joseph Heller,
autor de \emph{Catch"-22} (Ardil"-22, 1961), ``o fato de ser
paranoico não quer dizer que não estejam atrás de você''. O século \versal{XX}
geraria uma quantidade realmente espantosa de indivíduos visionários e
adoecidos, desconfiados da máquina gigantesca gerada por um Estado
crescentemente policial, guerras de dimensão nunca antes vista e a ação
viciante da propaganda midiática narcótica para as massas. Este século
\versal{XXI} segue e aprofunda o costume, quando as teorias da conspiração (um
bom número delas já nem mais \emph{teorias}, mas \emph{fatos de
conspiração}) são a mais popular vertente dos horrores escondidos sob a
aparência cotidiana de normalidade. Diria que Lovecraft desempenha um
papel estrutural nisso, e eis porque, como veremos, ele é onipresente
hoje.

\section{Sobre a obra}

Aqui estão reunidos dois contos dos mais representativos da produção ficcional de Lovecraft:
\textit{O chamado de Cthulhu} (1928) e \textit{A cor que caiu do espaço} (1927).
No primeiro somos apresentados à figura mais popular de Lovecraft, centro da série sobre os Grandes Antigos, as gigantescas e incompreensíveis criaturas anteriores a esta Terra. É a cristalização, numa imagem, de um tipo específico de terror chamado ``cósmico'': mas um cósmico íntimo e literário. Em seu Cthulhu, um monstro que dorme no fundo do mar --- verde, sombrio, doentio, descomunal e de dimensões inqualificáveis ---, o autor procedeu a uma metamorfose do próprio Kraken, monstro marinho e cefalópode da mitologia escandinava, para encontrar um código de seus próprios horrores: mas que funcionou bem, porque o verdadeiro mergulho no medo de um é o mergulho no medo de todos. Um dos grandes clássicos de horror do século \textsc{xx}, \textit{O chamado de Cthulhu} permite um desconcertante passeio pelo universo macabro de um dos grandes mestres do horror.


Já no segundo conto, temos o momento decisivo de Lovecraft em direção
ao horror cósmico inspirado pela ficção
científica. Na história, um vilarejo a oeste de Arkham vê-se ameaçado quando um
meteoro cai na propriedade de um fazendeiro local e traz consigo uma estranha
aberração cromática que afeta a flora e a fauna da região --- e cria o cinzento e
estéril “descampado maldito” onde nada cresce.

Nesse conto, Lovecraft cria um dos conceitos mais complexos que a
literatura fantástica pôde desenvolver: o de que se o mundo sofresse a
interferência de algo externo à experiência, essa intervenção, mesmo que
imotivada, produziria um efeito estrutural em quem a presenciasse, e de
modo irreconciliável com os demais, desencadeando atritos que cresceriam
até o conflito aberto. Fisicamente, a \emph{razão isotópica} de um
meteorito, uma pedra que vem do espaço, é diferente da de uma pedra
achada na Terra, protegida dos \emph{raios cósmicos}, ou da
\emph{radiação cósmica}, e por isso há entre elas uma diferença
fundamental, ainda que os elementos de constituição se achem na tabela
periódica: Lovecraft expande o dado científico em uma invenção
hipotética dos limites da civilidade, dos limites da percepção. Pode"-se
até mesmo especular que a famosíssima série iniciada na \versal{TV} por Rod
Serling (1924--1975) nos anos 1950, \emph{Twilight Zone} --- que ficou
conhecida como \emph{Além da imaginação} no Brasil ---, terá partido de
ficções anteriores como essa de Lovecraft em específico.

É um experimento intelectual, e por isso o termo \emph{ficção
especulativa} é tão apropriado para o melhor do que se escreve em
fantasia: está uns tantos passos adiante do que se chamou, com Émile
Zola no século \versal{XIX}, \emph{romance de tese}. O mesmo se dá em ``O chamado
de Cthulhu'', se lemos um dos trechos com especial atenção: o trecho de
quando os marinheiros estão na estranha ilha ciclópica e presenciam as
impossibilidades da geometria ``não euclidiana'' das construções.
Lovecraft descreve uma notável invenção ficcional do que depois, com a
física, poderíamos chamar \emph{anomalia de escala}, proposta dentro do
conto como as regras clássicas do conceito de realidade em nível humano
sendo distorcidas quando se encontram coisas de fora desse espectro,
formas nas quais um ângulo supostamente reto pode se ``comportar'' como
obtuso.

A fantasia, como gênero de arte e literatura, se apropria de aspectos
dessa lenta sedimentação para amplificar os efeitos de uma visão
particular. A fantasia tem, em sua origem, o mesmo ponto de
partida das antigas alegorias, porque ambas estabelecem um domínio
paralelo ao que está sendo dito, no sentido de que o que se diz é apenas
a instância inicial do que se projeta \emph{a partir} do texto. Mas a
despeito de suas enormes ambições, os subgêneros fantásticos já eram
rejeitados \emph{per se}, como vimos, estabelecendo"-se antes em
publicações comerciais modestíssimas e esnobadas tanto pela crítica como
pelo ``bom gosto'' do respeitável público.\footnote{A efemeridade da
  coisa colaborava para o conjunto de preconceitos: não eram feitas para
  durar, impressas em papel vagabundo e para o consumo incauto das
  massas. Ao menos desde Charles Baudelaire (1821--1867), o maligno poeta
  das \emph{Fleurs du Mal}, o jornalismo, publicado em folhas volantes,
  era visto como algo passageiro, sem a importância de uma arte da
  duração, e espelho dos novos tempos afobados e rudes. Conquanto isso
  em parte seja verdade, deve"-se lembrar que muito daquilo que foi e é
  feito para durar não passa de má arte e má literatura: em todos os
  veículos a maioria nunca presta, mas em todos os veículos, não importa
  como nem por quê, haverá sempre um núcleo de verdadeira qualidade
  diferencial.}

 

\section{Sobre o gênero}

\begin{quote}
O conto é, do ângulo dramático, unívoco, univalente. [\ldots]
Etimologicamente preso à linguagem teatral,
``drama'' significava ``ação''. E com o tempo passou a designar
toda peça destinada à representação. Na época romântica, dado o
princípio da fusão de gêneros, entendia-se por drama o misto de
tragédia e comédia. Transferido para a prosa de ficção, o termo
``drama'' entrou a significar ``conflito'', ``atrito''. Nesse caso,
``ação'' ``conflito'' se tonaram equivalentes, uma vez que toda
ação pressupõe conflito, e este, promove a ação, ou por meio dela
se manifesta; em suma, ambos se implicam mutuamente.

O conto é, pois, uma narrativa unívoca, univalente: constitui
uma \textit{unidade dramática}, uma \textit{célula dramática}, visto gravitar ao
redor de um só conflito, um só drama, uma só ação. Caracteriza-se,
assim, por conter \textit{unidade de ação}, tomada esta como a sequência de atos praticados pelos protagonistas, ou de acontecimentos de
que participam. A ação pode ser externa, quando as personagens se
deslocam no espaço e no tempo, e interna, quando o conflito se
localiza em sua mente.\footnote{\textsc{moisés}, Massaud. \textit{A criação literária}. São Paulo: Cultrix, 2006, p.\,40.}
\end{quote}

Partindo da definição de Massaud Moisés sobre o conto, evidencia"-se a principal característica desse gênero literário: a unidade de conflito, condensada em ações que se completam em um único enredo. Ao conto, ainda seguindo Moisés, aborrecem as divagações e os excessos, pois há uma concentração de efeitos e pormenores essenciais, em sua brevidade, para o bom funcionamento do conto.
Cada construção, cada palavra nesse gênero tem sua razão de existir, pois integra a economia global da narrativa.

Apesar da brevidade de sua forma, o conto desdobra"-se em muitas direções e implicações, e o faz a partir de elementos restritos: a unidade dramática, como já mencionada, assim como a presença de poucas personagens e a limitação espacial e temporal. Um ótimo exemplo é o conto ``Missa do galo'', de Machado de Assis, em que o narrador, Nogueira, conta a sua experiência de uma única noite na companhia de sua hospedeira, D.\,Conceição. Apesar de unidade temporal (a noite que antecede a Missa do galo), espacial (uma sala na casa de D.\,Conceição) e da redução dramática, basicamente, à interação entre duas personagens, Conceição e Nogueira, esse conto desdobra"-se em muitas direções. A companhia de Conceição desperta a sexualidade de Nogueira, e seu impacto é tão profundo que o narrador relembra aos leitores esse acontecimento de sua juventude. As intenções da anfitriã, narradas e, logo, distorcidas pela memória de Nogueira, também são ambíguas, levantando as mais diversas questões e interpretações.

Como reflete o escritor argentino Julio Cortázar, o conto consegue, de forma muito concisa, despertar ``uma realidade infinitamente mais vasta que a do seu mero argumento'', influindo ``em nós com uma força que nos faria suspeitar da modéstia do seu conteúdo aparente, da brevidade do seu texto''.\footnote{\textsc{CORTÁZAR}, Julio. \textit{Valise de cronópio}. São Paulo: Editora Perspectiva, 2008, p.\,155.}

Apesar da aparente banalidade do argumento, o conto abre essa possibilidade de desenvolver o tema em profundidade, em contraposição à aparente concisão narrativa. Realiza plenamente, assim, o que Cortázar define como o gênero do conto:

\begin{quote}
Um escritor argentino, muito amigo do boxe, dizia"-me que nesse combate que se trava entre um texto apaixonante e o leitor, o romance ganha sempre por pontos, enquanto que o conto deve ganhar por \textit{knock"-out}. É verdade, na medida em que o romance acumula progressivamente seus efeitos no leitor, enquanto que um bom conto é incisivo, mordente, sem trégua desde as primeiras frases. Não se entenda isto demasiado literalmente, porque o bom contista é um boxeador muito astuto, e muitos dos seus golpes iniciais podem parecer pouco eficazes quando, na realidade, estão minando já as resistências mais sólidas do adversário.
Tomem os senhores qualquer grande conto que seja de sua preferência, e analisem a primeira página. Surpreender"-me"-ia se encontrassem elementos gratuitos, meramente decorativos. O contista sabe que não pode proceder acumulativamente, que não tem o tempo por aliado; seu único recurso é trabalhar em profundidade, verticalmente, seja para cima ou para baixo do espaço literário.\footnote{Ibid., p.\,152.}
\end{quote}

\subsection{Pulp Fiction: Weird Tales}\label{pulp}

Os contos de Lovecraft inscrevem"-se em uma tradição específica, característica da ficção cientifica moderna: \emph{pulp fiction}, nome que foi alçado ao patamar das expressões
cultuadas como se traduzisse um tipo de atmosfera, ou um tipo de
atitude, como também seus antecedentes britânicos dos \emph{penny
dreadfuls}.\footnote{Não por acaso, \emph{Penny Dreadful} recentemente
  foi o nome de uma série baseada naquele tipo de narrativa seriada em
  papel barato do fim do século \versal{XIX} na Inglaterra, e reunia personagens
  originalmente desconexos, como os de \emph{Dracula}, o dr.
  Frankenstein e sua criatura, Dr.\,Jekyll, Dorian Gray, um lobisomem,
  etc. Obviamente, o seriado veio depois das séries de \versal{HQ}s escritas por
  Alan Moore, com arte de Kevin O'Neill, \emph{The League of
  Extraordinary Gentlemen} (A Liga Extraordinária, iniciada em 1999) e
  que se utilizava precisamente da curiosa reunião de personagens
  famosos da literatura fantástica da Era Vitoriana na Inglaterra, já
  dentro do subgênero \emph{steampunk}.} O \emph{pulp} vinha igualmente
do esforço de reduzir o custo de produção das edições, empregando a
polpa da árvore no fazer dela papel barato para as revistas de consumo
popular\emph{.} Essas publicações, nos \versal{EUA}, começam no fim do século \versal{XIX}
e atingem seu pico de popularidade entre os anos 1920 e 1930, com
títulos como \emph{Argosy}, \emph{Spicy Mystery}, \emph{Marvel Tales},
\emph{Blue Book}, \emph{Horror Stories} e outros. Essas publicações,
homenageadas e levadas ao centro das atenções ao menos desde \emph{Pulp
Fiction} (1992), o filme de Quentin Tarantino, eram essas edições
ordinárias em brochura, com capas chamativas ou escandalosas, vendidas
por centavos em banca de rua, com textos que contavam histórias
misteriosas, eróticas ou horripilantes: às vezes, as três coisas juntas.

Parte da nascente indústria de entretenimento estadunidense,
condicionava o estilo dos autores a buscar de qualquer forma a atenção
pública (em geral de modo aberrante ou escandaloso para os padrões da
época), porque o sistema dependia necessariamente da popularidade dos
títulos, e então a crítica da época ignorava essas brochuras,
desprezadas como má literatura apelativa; parte dessa crítica não se
dedicava sequer a tentar entender o fenômeno, mas buscava impor uma
visão artística imóvel em seus cânones, moralista e carola sobre os
escândalos escritos --- como, mais tarde, aconteceria com as próprias
\versal{HQ}s sob a censura do Macarthismo e do \emph{Comics Code} (de meados da
década de 1940 e 1950), que higienizaria as publicações de suas
inventadas imoralidades.

\emph{Weird Tales} (1922--1954, com algumas tentativas de retorno aqui e
ali) foi a publicação \emph{pulp} na qual Lovecraft publicou parte
considerável de seus contos, e onde começou sua notória série de
Cthulhu, com a história que leremos nesta edição; lá também foi onde
Robert\,E.\,Howard, seu amigo, começou a publicar \emph{Conan, o bárbaro},
as aventuras do bruto cimeriano hoje famosíssimo pelas \versal{HQ}s da Marvel
Comics e pelos (quem diria?) já velhos filmes com o então monossilábico
Arnold Schwarzenegger no papel principal. A revista, por seu acidentado
histórico de publicação, não obstante pôde contar com aqueles e outros
nomes notáveis, já na época entre seus pares, e universalmente em
retrospecto. Eram escritores que herdavam uma tradição antiga, como
vimos, mas também recente de uma nova cristalização de subgêneros, e que
ainda não tinham lugar de visibilidade, em parte porque fruto da
indústria de entretenimento, em parte porque exploravam, pioneiros,
novos aspectos da arte da escrita dentro daqueles âmbitos.

Lovecraft era um dos poucos que, ainda que condicionasse em parte seu
estilo àquelas necessidades do tipo de publicação, sequer precisava
tentar se acomodar ao público: sua imaginação era tão obsessiva e tão
múltipla que suas histórias se sobressaíam mesmo entre as mais
estranhas; sua escrita, orgulhosa e sentenciosa, repleta de
subentendidos e sombrias maquinações inumanas, de estilo hiperbólico,
repetitivo e grandíloquo, iria se tornar não apenas imediatamente
atraente como depois seria o padrão para muitos que o seguiram em
explorar aberrações perceptivas, o estranho"-familiar ou criaturas
multidimensionais perturbando a existência aparente deste mundo.

