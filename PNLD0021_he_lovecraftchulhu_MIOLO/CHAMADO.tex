
\part[o chamado de cthulhu]{\textsc{o chamado de cthulhu}\break\normalsize\textsc{(achado entre os papéis do falecido francis wayland thurston, de boston)}}

\chapter*{}
\thispagestyle{empty}

\epigraph{De tais grandes poderes ou seres talvez seja possível supor uma
sobrevivência\ldots{} uma sobrevivência de um período imensamente remoto
quando\ldots{} a consciência se manifestava, talvez, em formas e
configurações desde há muito recolhidas, antes da maré da humanidade
avançar\ldots{} formas das quais só a poesia e as lendas agarraram uma
memória fugidia e as chamaram deuses, monstros, criaturas míticas de
todos os tipos e espécies\ldots{}}{\textsc{algernon blackwood}\footnotemark
}

\footnotetext{A citação de abertura vem do romance \emph{The Centaur} (O Centauro, 1911), de Algernon Blackwood
  (1869--1951). Lovecraft terá se baseado no estilo de Blackwood em criar
  expectativa e encerrar a narrativa no ápice de uma revelação, como
  acontece em \emph{O Centauro}.}


\chapter{O horror em argila}


\noindent{}A coisa mais misericordiosa do mundo é, penso, a inabilidade da mente
humana em correlacionar todas as suas partes. Vivemos numa plácida ilha
de ignorância em meio aos mares negros do infinito, e não nos foi dado
viajar longe. As ciências, cada uma se distendendo em um sentido
específico, até agora nos prejudicaram pouco; mas algum dia a
recomposição desse conhecimento dissociado abrirá visões tão aterradoras
da realidade, e da nossa temível posição nela, que ficaremos loucos com
a revelação ou fugiremos da luz mortal para a paz e a segurança de uma
nova idade das trevas.

Teósofos teceram hipóteses sobre a espantosa grandeza do ciclo cósmico,
em que o nosso mundo e a raça humana não formam senão incidentes
passageiros. Aludiram a estranhas permanências em termos que gelariam o
sangue se não se mascarassem de um insípido otimismo. Mas não veio deles
o vislumbre singular de eras proibidas que me arrepiam quando me vêm ao
pensamento, e que me apavoram quando me vêm em sonho. Esse vislumbre,
como todos os terríveis vislumbres da verdade, faiscou-me numa
recomposição acidental de coisas distantes --- nesse caso, um velho
fragmento de jornal e as notas de um professor já falecido. Espero que
ninguém mais chegue a essa recomposição; por certo, se eu viver, jamais
oferecerei conscientemente um elo de tão odiosa cadeia. Penso que também
o professor decidira manter silêncio sobre a parte que conhecia, e que
teria destruído suas notas caso a morte súbita não o tivesse
surpreendido.

Meu conhecimento do assunto se iniciou no inverno de 1926--27, com a
morte de meu tio-avô George Gammell Angell, professor emérito de línguas
semíticas na Universidade Brown, em Providence, Rhode Island. O
professor Angell era amplamente considerado uma autoridade em inscrições
antigas, e os diretores dos mais prestigiosos museus recorriam a ele com
frequência; de modo que seu falecimento à idade de noventa e dois anos
deve ser lembrado por muitos. No meio local, o interesse foi
intensificado pela obscuridade da causa da morte. O professor fora
acometido de algo quando retornava do barco em Newport; tombou
subitamente, segundo testemunhas, depois de levar um encontrão de um
negro com aparência de marujo, que viera de um dos estranhos becos
escuros na encosta íngreme, formando um atalho da beira-mar para a casa
do falecido, em Williams Street. Os médicos foram incapazes de encontrar
qualquer enfermidade discernível, mas concluíram, após debate
tumultuado, que alguma obscura lesão cardíaca --- induzida pela vigorosa
subida de uma colina tão abrupta, por um homem já tão idoso --- fora
responsável por seu fim. No momento não vi razão para discordar daquela
sentença, mas recentemente me sinto inclinado a desconfiar --- e a mais 
do que desconfiar. 


Sendo o herdeiro e o executor testamentário de meu tio-avô, pois morrera
viúvo e sem filhos, esperava-se que eu examinasse seus papéis com alguma
minúcia; e com esse objetivo trouxera todo o conjunto de seus arquivos e
caixas para os meus alojamentos em Boston. Muito do material que
relacionei será publicado mais tarde pela Sociedade Arqueológica
Americana, mas uma caixa me parecia sobremaneira intrigante, e decidi
não mostrá-la a outros olhos. Havia sido lacrada, e não encontrei a
chave até que me ocorresse examinar o chaveiro que o professor carregava
sempre em seu bolso. Pude então abri-la, mas ao fazê-lo me vi
confrontado por uma pior e ainda mais sólida barreira: pois qual poderia
ser o significado do peculiar baixo-relevo em argila, das digressões,
incisões e dos rabiscos desconexos que encontrei? Teria o meu tio se
tornado, em seus últimos anos, crédulo das mais superficiais imposturas?
Resolvi procurar o escultor excêntrico, responsável por esse aparente
distúrbio na paz de espírito de um homem idoso.

O baixo-relevo era um retângulo grosseiro com espessura de menos de uma
polegada, e com por volta de seis polegadas de área; obviamente de
origem moderna. Suas incisões, entretanto, eram distantes do moderno em
atmosfera e sugestão; pois apesar de os caprichos do cubismo e do
futurismo serem muitos, e brutais, não reproduzem a regularidade
críptica que se acha na escrita pré-histórica. E algum tipo de escrita o
grosso daquelas incisões parecia de fato ser, embora minha memória, a
despeito da muita familiaridade com os papéis e coleções do meu tio, não
fosse de modo algum capaz de identificar que espécie em particular, ou
sequer supor suas mais remotas ligações.

Logo acima do que parecia linguagem hieroglífica estava uma figura de
caráter evidentemente ilustrativo, ainda que a execução subjetiva
impedisse de se fazer uma ideia muito clara de sua natureza. Parecia ser
um tipo de monstro, ou símbolo representando monstro, de forma que só
uma fantasia mórbida poderia conceber. Se disser que minha imaginação
algo extravagante extraía de lá as imagens simultâneas de um polvo, um
dragão e uma caricatura humana, não serei infiel ao espírito da coisa.
Uma cabeça carnuda, tentaculada, arrematava um corpo grotesco e escamoso
de asas rudimentares; mas era o \emph{esquema geral} do todo que o fazia
escandalosamente assustador. Por trás da figura estava a sugestão vaga
de um fundo arquitetônico ciclópico.\footnote{Tipo de construção feito de
  rochas apoiadas umas nas outras, sem material de juntura.}

A escrita acompanhando essa excentricidade era, com exceção de uma pilha
de recortes de jornal, recente e da própria mão do professor Angell, sem
a pretensão de estilo literário. O que parecia ser o documento principal
se intitulava ``\versal{CULTO DE CTHULHU}'' em caracteres cuidadosamente
impressos para evitar a leitura errônea de uma palavra tão
insólita.\footnote{Efetivamente insólita. Sobre ela, Lovecraft escreveu
  em carta de 1934 a Duane Rimel, também autor de fantasia e horror, que
  ``deveria representar uma tentativa humana desajeitada de fixar a
  fonética de uma palavra \emph{absolutamente não-humana}. O nome da
  entidade infernal foi inventada por seres cujos órgãos vocais não eram
  como o do homem, daí não haver relação com o equipamento humano para
  fala (\ldots{}) O som efetivo --- tanto quanto os órgãos humanos poderiam
  imitá-lo ou as letras humanas poderiam registrá-lo --- devem ser algo
  como \emph{Khlûl'hloo}, com a primeira sílaba pronunciada guturalmente
  e espessa. O \emph{u} é como aquele em \emph{full}; e a primeira
  sílaba não é sonoramente diferente de \emph{klul}, uma vez que
  \emph{h} representa a aspereza gutural. A segunda sílaba não é muito
  pronunciada --- o som \emph{l}, mudo''. ``Notes'', \emph{in}:
  Lovecraft, H.\,P. \emph{Tales} (Peter Straub, editor), New York, The
  Library of America, 2005, p. 830.} O manuscrito se dividia em duas
seções, a primeira das quais se intitulava ``1925 --- Sonho e Trabalho
de Sonho de H.\,A. Wilcox, Thomas St., 7, Providence, \versal{R.I.}'', e a
segunda, ``Narrativa do Inspetor John R.\,Legrasse, Bienville St., 121,
New Orleans, La., em 1908 A.\,A.\,S.\,Hiptc. --- Notas do Mesmo, \& Rel. do
Prof. Webb''. Os outros papéis do manuscrito eram todos notas breves,
algumas delas relatos de sonhos incomuns de diversas pessoas, outras,
ainda, citações de revistas e livros teosóficos (em particular
\emph{Atlântida} e a \emph{Lemúria Perdida}, de W.\,Scott-Elliot),\footnote{\emph{The
  Story of Atlantis} (1896) e \emph{The Lost Lemuria} (1904), de William
  Scott-Elliot (1849--1919), teósofo inglês.} e o resto, comentários
sobre as sociedades secretas e as seitas ocultas de longa história, com
citações de passagens em obras de referência mitológica e antropológica
como \emph{O Ramo Dourado,} de Frazer e \emph{O Culto das Bruxas na
Europa Ocidental}, da Srta.\,Murray.\footnote{\emph{The Golden Bough} (os
  doze volumes foram publicados entre 1906--15), obra monumental de James
  George Frazer (1854--1941), antropólogo e folclorista escocês, e
  \emph{The Witch-Cult in Western Europe} (1921), de Margaret Alice
  Murray (1863--1963), egiptóloga, antropóloga, folclorista, arqueóloga e
  historiadora anglo-indiana.} Os recortes em sua maioria aludiam a
doenças mentais \emph{outré} e a surtos de insanidade coletiva na
primavera de 1925.

A primeira metade do manuscrito principal contava uma história muito
peculiar. Ao que parece, em primeiro de março de 1925 um jovem magro e
lúgubre, de aspecto neurótico e exaltado, procurara o professor Angell
trazendo o singularíssimo baixo-relevo de barro, que então se encontrava
sobremaneira úmido e fresco. Em seu cartão de visita vinha o nome de
Henry Anthony Wilcox, e meu tio o identificara como o filho mais novo de
uma ótima família que conhecia superficialmente, e que há pouco estivera
estudando escultura na Escola de Design de Rhode Island e morava só no
edifício Fleur-de-Lys, próximo à instituição. Wilcox era um jovem
precoce de conhecido gênio, mas de grande excentricidade, e havia desde
a infância chamado a atenção por histórias estranhas e sonhos ímpares
que tinha o hábito de contar. Chamava a si mesmo ``psiquicamente
hipersensível'', mas a gente sisuda da antiga cidade comercial o
descartava apenas como um ``esquisito''. Jamais se integrando aos de seu
tipo, perdera gradualmente a visibilidade social e era agora conhecido
só de um pequeno grupo de estetas de outras cidades. Até o Clube de Arte
de Providence, ansioso por preservar seu conservadorismo, o considerava
realmente um caso perdido.

Na ocasião da visita, dizia o manuscrito do professor, o escultor
abruptamente pediu que o anfitrião o beneficiasse com seu conhecimento
arqueológico para identificar os hieróglifos no baixo-relevo. Falava
numa maneira de sonho, afetada, o que sugeria altivez e simpatia
distante; e meu tio mostrou-se algo incisivo na resposta, porque o
notável frescor do tablete implicava afinidade com tudo, menos
arqueologia. A réplica do jovem Wilcox, impressionando meu tio o
suficiente para recordá-la e registrá-la \emph{verbatim}, era de um
matiz tão fantasticamente poético que coloria todo o seu modo de falar,
e que desde então julguei muito característico dele. Disse, ``É novo, de
fato, pois o fiz ontem à noite em um sonho de estranhas cidades; e
sonhos são mais antigos do que a inquietante Tiro, ou a contemplativa
Esfinge, ou a Babilônia cingida de jardins''.

Foi quando começou aquele conto divagante que de súbito fisgou uma
memória adormecida e ganhou o febril interesse do meu tio. Houvera um
leve tremor de terra na outra noite, o mais significativo já sentido na
Nova Inglaterra em alguns anos; e a imaginação de Wilcox fora afetada
agudamente. Cidades ciclópicas de blocos titânicos e monolitos
projetados aos céus, gotejando todas um lodo verde e sinistras de horror
latente. Hieróglifos haviam coberto as muralhas e os pilares, e de um
ponto indeterminado no fundo vinha uma voz que não era uma voz: uma
sensação caótica que só a fantasia transmutaria em som, mas que tentava
representar pela quase impronunciável mixórdia de letras ``\emph{Cthulhu
fhtagn}''.

Essa mixórdia verbal era a chave para a recordação que excitara e
perturbara o professor Angell. Questionou o escultor com minúcia
científica; e estudou com intensidade quase frenética o baixo-relevo no
qual o jovem havia se achado trabalhando arrepiado e trajando apenas
seus pijamas, quando o despertar se insinuou confusamente sobre ele. Meu
tio culpara a idade avançada, disse Wilcox depois, pela lentidão em
reconhecer tanto os hieróglifos quanto o plano pictórico. Muitas das
suas questões pareceram enormemente fora de propósito ao visitante, em
especial aquelas que tentavam conectá-lo a estranhos cultos e
sociedades; e Wilcox não compreendia as repetidas promessas de silêncio
que lhe eram oferecidas em troca da admissão de filiação a algum amplo
grupo místico, ou religioso pagão. Quando o professor Angell se
convenceu de que o escultor de fato ignorava qualquer culto ou sistema
de tradição críptica, ele assediou seu visitante com pedidos
sobre
relatos futuros de sonhos. Isso deu frutos regulares, pois, passada a
primeira entrevista, o manuscrito registra contatos diários do jovem,
durante os quais relatava fragmentos alarmantes de imagens noturnas,
cujo fardo era sempre alguma terrível paisagem ciclópica de rocha escura
e gotejante, e uma voz ou inteligência subterrânea berrando
monotonamente em enigmáticos impactos sensoriais, irregistráveis senão
como mera algaravia. Os dois sons mais frequentemente repetidos eram os
representados pelas palavras ``\emph{Cthulhu}'' e ``\emph{R'lyeh}''.

Em 23 de março, o manuscrito continuava, Wilcox não apareceu; e buscas a
seus aposentos revelaram que, acometido de um tipo obscuro de febre, o
haviam levado para a casa de sua família em Waterman Street. Gritara no
meio da noite, acordando vários outros artistas no prédio, e daí
manifestara apenas alternâncias de inconsciência e delírio. Meu tio
imediatamente telefonou para a família, e daquele momento em diante
manteve vigília estrita sobre o caso; chamava com frequência o
consultório do Dr.\,Tobey em Thayer Street, que lhe disseram ser o
responsável. A mente febril do jovem, aparentemente, vivia em coisas
estranhas, e o doutor estremecia, vez ou outra, quando ele falava delas.
As coisas incluíam não apenas uma repetição do que havia sonhado antes,
mas chegavam furiosamente a algo gigantesco ``com milhas de altura'', e
que caminhava, ou arrastava-se em algum lugar. Em nenhum momento
descreveu em detalhe esse ponto, mas palavras frenéticas e ocasionais,
como as reproduzidas pelo Dr.\,Tobey, convenceram o professor de que
aquilo deveria ser idêntico à monstruosidade sem nome que buscara
representar em sua escultura de sonho. Qualquer referência a esse ponto,
acrescentava o doutor, era invariavelmente um prelúdio para o retorno do
jovem à letargia. Sua temperatura, curiosamente, não ia muito além do
comum, mas sua condição geral, por outro lado, sugeria antes febre
verdadeira do que uma condição mental.

Em 2 de abril, por volta das 3 da tarde, todos os traços da afecção de
Wilcox cessaram subitamente. Sentou-se ereto na cama, surpreso de
encontrar-se em casa, e completamente ignorante do que houvera, em sonho
ou realidade, desde a noite de 22 de março. Recebendo alta de seu
médico, retornou a seus aposentos em três dias; mas para o professor
Angell ele já não servia de assistência: todos os indícios de sonhos
estranhos haviam desaparecido com a sua recuperação, e meu tio não
manteve registro dos pensamentos noturnos após uma semana de relatos,
inúteis e irrelevantes, de visões inteiramente comuns.

Aqui se encerrava a primeira parte do manuscrito, mas referências a
certas notas esparsas deram-me muito mais material para pensar ---
tanto, na verdade, que apenas o arraigado ceticismo então formador da
minha filosofia pode explicar minha contínua desconfiança com respeito
ao artista. As notas em questão eram as que descreviam os sonhos de
várias pessoas, cobrindo o mesmo período no qual o jovem Wilcox recebera
suas estranhas visitações. Meu tio, ao que parece, instituíra rápido um
conjunto prodigiosamente vasto de questões entre quase todos os amigos
que podia questionar sem impertinência, pedindo-lhes relatórios noturnos
de sonhos e as datas de quaisquer visões dignas de nota no passado
recente. A recepção a esse pedido parece ter sido variada; mas, no
mínimo, terá recebido mais respostas do que qualquer pessoa comum
poderia lidar sem uma secretária. Essa correspondência original não foi
preservada, mas suas notas formaram um farto e significativo compêndio.
Gente mediana na sociedade e nos negócios --- o tradicional ``sal da
terra'' da Nova Inglaterra --- forneceu um resultado quase inteiramente
negativo, embora alguns casos esparsos de inquietas e disformes
impressões noturnas apareçam aqui e ali, sempre entre 23 de março e 2 de
abril --- o período de delírio do jovem Wilcox. Homens da ciência foram
afetados ligeiramente mais, ainda que quatro casos de descrição vaga
sugiram relances fugidios de estranhas paisagens, e em um caso
mencione-se o pavor de algo anormal.

Foi dos artistas e dos poetas que vieram as respostas pertinentes, e
estou certo de que o pânico teria se espalhado se tivessem podido
comparar notas. Assim como estava, com a falta das cartas originais,
tive em parte a suspeita de que o compilador fizera perguntas
tendenciosas, ou de que teria editado a correspondência para
corroboração do que, de modo latente, já estivesse decidido a ver. Por
isso continuava a sentir que Wilcox, conhecendo de alguma forma os
antigos dados em posse de meu tio, aplicava um truque no cientista
veterano. Aquelas respostas de estetas contavam uma história
perturbadora: de 28 de fevereiro a 2 de abril, uma larga proporção deles
havia sonhado coisas muito bizarras, e os sonhos imensuravelmente mais
fortes em intensidade aconteciam durante o período do delírio do
escultor. Mais de um quarto dos que fizeram relatos descreveram cenas, e
como que sons, não diferentes dos que Wilcox descrevera; e alguns dos
sonhadores confessaram um medo agudo da gigantesca coisa sem nome,
visível quase ao fim. Um caso, que a nota descreve com ênfase, era muito
triste. O indivíduo, arquiteto bem conhecido, com inclinação para a
teosofia e o ocultismo, tornou-se violentamente insano na data da crise
do jovem Wilcox e faleceu muitos meses depois, após brados incessantes
implorando para ser salvo de um fugitivo habitante do inferno. Se meu
tio houvesse se referido aos casos por nome ao invés de apenas por
número, eu teria tentado corroborá-los com alguma investigação pessoal;
mas assim como estava, tive sucesso em rastrear apenas uns poucos. Todos
esses, no entanto, tinham notas completas. Imaginava com frequência se
os demais objetos dos questionários do professor se sentiam tão
perplexos quanto essa porção deles. É bom que nenhuma explicação jamais
lhes alcançará.

Os recortes de jornais, como disse, abordavam casos de pânico, mania e
excentricidade durante o período considerado. O professor Angell deve
ter mantido todo um escritório para os recortes, porque o número de
extratos era descomunal e as fontes se espalhavam pelo globo. Aqui se
achava um suicida noturno em Londres, onde um homem dormindo solitário
pulara de uma janela após um grito chocante. Ali, do mesmo modo, uma
carta incoerente para o editor de um jornal na América do Sul, onde um
fanático deduz um futuro lúgubre a partir das visões que tivera. Um
despacho vindo da Califórnia descreve uma colônia teosófica onde todos
vestiam robes brancos para um ``evento glorioso'' que jamais chega,
enquanto itens vindos da Índia falam cautelosamente de uma séria
agitação entre os nativos, perto do fim de março. Orgias vodu se
multiplicam no Haiti, e postos africanos reportam murmurações
agourentas. Oficiais americanos nas Filipinas acham certas tribos
incômodas por volta do mesmo momento, e policiais em Nova York são
atacados por levantinos histéricos na noite de 22 para 23 de março. No
oeste da Irlanda também correm rumores aberrantes e lendas, e um pintor
de temas fantásticos, chamado Ardois-Bonnot, expõe uma blasfema
``Paisagem Onírica'' no salão de primavera em Paris, 1926. Tão numerosos
são os problemas registrados nos asilos para insanos que só um milagre
pode ter evitado que a comunidade médica notasse estranhos paralelismos
e traçasse conclusões mistificantes. Um punhado bizarro de recortes, que
seja dito; e hoje eu mal posso imaginar o nível de racionalidade rude
com o qual o pus de lado. Mas então eu estava convencido de que o jovem
Wilcox sabia das questões antigas mencionadas pelo professor.


\chapter{A narrativa do inspetor Legrasse}

\noindent{}As questões antigas que fizeram o sonho e o baixo-relevo do escultor tão
significativos para o meu tio compunham o assunto da segunda metade de
seu longo manuscrito. Já uma vez antes, parece, o professor Angell
notara os contornos infernais da monstruosidade inominável, intrigara-se
com os hieróglifos desconhecidos e ouvira as sílabas nefastas que apenas
poderiam ser transcritas como ``\emph{Cthulhu}''; e tudo isso numa
conexão tão horrível e tantalizante que não surpreende o ter procurado o
jovem Wilcox com questões e pedidos por informação.

A mais antiga experiência ocorrera em 1908, dezessete anos antes, quando
a Sociedade Arqueológica Americana teve seu encontro anual em St.\,Louis.
O professor Angell, como cabia a alguém de sua autoridade e de seus
êxitos, tivera um papel destacado em todas as deliberações; e foi um dos
primeiros a serem abordados pelos muitos leigos que aproveitavam a
oportunidade de fazer perguntas para conseguir respostas exatas, e de
apresentar problemas para a solução de especialistas.

O principal desses leigos, e logo o foco de interesse para o encontro
todo, era um homem de meia-idade e aparência convencional que viajara
desde New Orleans por certa informação especial que não poderia ser
obtida de nenhuma fonte local. Seu nome era John Raymond Legrasse, e
era, de profissão, inspetor da polícia. Trouxera consigo o assunto de
sua visita, uma estatueta de pedra, grotesca e repulsiva, aparentemente
muito antiga, cuja origem não tinha noção de como determinar. Não se
deve supor que o inspetor Legrasse tivesse ainda que um vago interesse
em arqueologia. Ao contrário, seu desejo por esclarecimento era atiçado
por considerações puramente profissionais. A estatueta, ídolo, fetiche
ou o que quer que fosse, havia sido apreendida alguns meses antes nos
pântanos arborizados ao sul de New Orleans durante uma batida a um
suposto encontro vodu; e tão singulares e horrendos eram os ritos
ligados a ela que a polícia não teve como não perceber que esbarrara em
um culto sombrio totalmente desconhecido, e infinitamente mais diabólico
do que mesmo o mais sombrio de todos os círculos de vodu africano. Sobre
sua origem, a despeito das histórias erráticas e inacreditáveis
extraídas de membros capturados, absolutamente nada se pôde descobrir;
daí a ansiedade da polícia por qualquer repertório antiquário que
pudesse ajudá-los a estabelecer o símbolo assustador, e a partir disso
rastrear o culto até a sua fonte.

O inspetor Legrasse estava bem pouco preparado para a sensação que sua
oferta causou. Um vislumbre da coisa fora o suficiente para lançar as
pessoas reunidas lá em um estado de tenso excitamento, e não perderam
tempo em se ajuntar a seu lado para espiar a diminuta figura, cuja
extrema estranheza e ainda o ar de genuína e abismal antiguidade
sugeriam enfaticamente hipóteses arcaicas e não exploradas. Nenhuma
escola de escultura reconhecível havia animado esse terrível objeto, e,
no entanto, séculos ou mesmo milhares de anos pareciam gravados na
superfície baça e esverdeada da pedra irreconhecível.

A figura, que finalmente passava de mão em mão para estudo próximo e
cuidadoso, tinha entre sete e oito polegadas de altura, e era de
artesanato artístico requintado. Representava um monstro de contornos
vagamente antropoides, mas com uma cabeça como que de polvo, cuja face
era uma massa de tentáculos, o corpo, escamoso de aparência elástica,
garras prodigiosas nos pés da frente e de trás, e longas asas estreitas
nas costas. Essa coisa, que parecia infusa de malignidade abominável e
antinatural, tinha corpulência algo inchada, e se acocorava com
perversidade num bloco retangular, ou um pedestal, coberto de caracteres
indecifráveis. As pontas das asas tocavam a extremidade da parte de trás
do bloco, o assento ocupava o meio, enquanto as garras longas e curvas
das pernas traseiras, dobradas e agachadas, agarravam a borda dianteira
e tomavam um quarto da distância até a base do pedestal. A cabeça do
cefalópode se inclinava para a frente, de modo que as pontas dos
tentáculos faciais resvalavam na parte de cima das enormes patas
dianteiras fincadas nos joelhos elevados da criatura. O aspecto completo
sugeria, de modo anormal, algo vivo e sutilmente mais temível por sua
origem de todo desconhecida. Sua vasta, pavorosa e incalculável idade,
algo inequívoco; e ainda assim não apontava sequer uma ligação com
quaisquer dos tipos conhecidos de arte pertencente à infância da
civilização --- ou, a bem da verdade, com qualquer outro período.
Totalmente único e à parte, mesmo o seu material era um mistério; pois a
pedra luzidia, verde-escura, mosqueada e estriada de dourado ou
iridescente não se parecia com nada em geologia ou mineralogia. Os
caracteres na base eram igualmente exasperantes; e nenhum dos membros
presentes, a despeito de representarem metade do conhecimento
especializado nesse campo em todo o mundo, conseguia formar ainda que
uma noção de sua mais remota família linguística. Aquilo, como o tema e
o material, pertencia a algo horrivelmente remoto e diverso da
humanidade como a conhecemos; algo assustadoramente sugestivo de ciclos
de vida antigos e profanos, nos quais o nosso mundo e os nossos
conceitos não têm parte.

E, enquanto os membros todos balançavam as cabeças e confessavam a
derrota diante do dilema do inspetor, houve ainda assim um homem naquela
conferência que suspeitou de uma bizarra familiaridade na forma
monstruosa e na escrita, e que de imediato contou, com certa reserva,
uma peculiar trivialidade de seu conhecimento. Essa pessoa foi o finado
William Channing Webb, professor de antropologia da Universidade de
Princeton, e explorador de não pouca monta. O professor Webb havia
tomado parte, quarenta e oito anos antes, em uma expedição à Groenlândia
e à Islândia, buscando inscrições rúnicas que antes falhara em
encontrar; e, lá no alto, na costa da Groenlândia Ocidental, encontrara
uma tribo singular, ou culto de esquimós degenerados, cuja religião, uma
forma curiosa de adoração do demônio, lhe deu calafrios por suas
deliberadas sede de sangue e repugnância. Era uma fé que outros esquimós
conheciam pouco, e que mencionavam sempre estremecidos, dizendo que
aquilo vinha de eras horrivelmente antigas, de antes mesmo que o mundo
se fizesse. Além de rituais inomináveis e sacrifícios humanos, havia
certos rituais hereditários exóticos, dedicados a um demônio supremo
mais antigo, ou \emph{tornasuk};\footnote{``Os groenlandeses não fazem
  preces nem sacrifícios, e não praticam rito algum; eles creem, não
  obstante, na existência de certos seres sobrenaturais. O principal e
  mais poderoso desses seres é o \emph{Torngarsuk}, invocado sobretudo
  pelos pescadores, e que eles por vezes representam sob a forma de um
  urso, por vezes sob a de um homem com um só braço, por vezes, enfim,
  sob a forma de uma grandíssima criatura humana como um dos dedos da
  mão'', \emph{in}: Plancy, J.\,Collin de. \emph{Dictionnaire Infernale}
  (sixième édition, augmentée de 800 articles nouveaux), Paris, Henri
  Plon, Imprimeur-Éditeur, 1863, p.661.} e desse, o professor Webb
tomara o cuidadoso registro fonético de um \emph{angekok}, ou
mago-sacerdote, expressando os sons em alfabeto romano tão bem quanto
pudesse. Mas, no momento, era de importância capital o fetiche que
aquele culto prezara, e em torno do qual dançavam quando a aurora saltou
sobre os montes gelados. Era, declarou o professor, um baixo-relevo bem
rústico, de pedra, abrangendo uma imagem horrenda e uns escritos
crípticos. E, tanto quanto sabia, um paralelo tosco, em todas as
características essenciais, daquela coisa bestial agora diante da
conferência.

Esses dados, recebidos com suspense e espanto pelos membros reunidos,
provou ser duplamente estimulante para o inspetor Legrasse: começou de
imediato a cobrir de perguntas o seu informante. Tendo anotado e copiado
um rito oral entre os membros do culto que seus homens haviam prendido
no pântano, ele rogou ao professor que lembrasse, o melhor que pudesse,
as sílabas registradas entre os esquimós diabólicos. Daí se seguiu uma
exaustiva comparação de detalhes, e um momento de silêncio reverente
quando, detetive e cientista, concordaram sobre a identidade possível de
uma frase comum aos dois ritos infernais, com mundos de distância entre
si. O que, em substância, tanto os feiticeiros esquimós e os sacerdotes
do pântano da Louisiana haviam cantado a seus ídolos aparentados era
algo como isto --- as divisões entre palavras conjecturaram-se das
pausas tradicionais na frase, tal como cantada em voz alta:

``\emph{Ph'nglui mglw'nafh Cthulhu R'lyeh wgah'nagl fhtagn.}''

Legrasse estava um ponto adiante do professor Webb, pois vários dentre
seus prisioneiros mestiços haviam repetido a ele o que os mais antigos
celebrantes lhes tinham dito significar as palavras. Esse texto, tal
como estava, queria dizer mais ou menos:

``\emph{Em sua morada em R'lyeh o morto Cthulhu aguarda sonhando.}''

E logo, em resposta a um pedido geral e urgente, o inspetor Legrasse
contou, do modo mais completo possível, sua experiência com os
adoradores no pântano; história à qual posso ver que meu tio atribuía
profunda importância. Sabia aos sonhos mais selvagens dos criadores de
mitos e dos teósofos, e revelava um grau de imaginação cósmica tão
espantoso entre aqueles pardos e párias quanto se podia esperar.

Em 1\textsuperscript{o} de novembro de 1907 chegaram à polícia de
New Orleans chamados frenéticos da região do pântano e da lagoa, ao sul.
Os ocupantes clandestinos lá, sobretudo uns descendentes primitivos, mas
de boa índole, dos Lafitte,\footnote{Jean e Pierre Lafitte, piratas
  franceses que, no começo do século \versal{XIX}, se estabeleceram na Louisiana.}
estavam transidos de horror por causa de algo desconhecido que os havia
abordado no meio da noite. Era aparentemente vodu, mas vodu de um tipo
bem mais terrível do que conheciam; e algumas de suas mulheres e
crianças haviam desaparecido desde que o malévolo tambor começara seu
incessante batuque das profundezas da assombrada escuridão do bosque,
onde nenhum dos moradores ousaria pisar. Havia gritos insanos e berros
aflitos, cantos de gelar o sangue e flamas que dançavam demoníacas; e,
acrescentava o apavorado mensageiro, as pessoas já não aguentavam mais
aquilo.

Então um destacamento de vinte policiais, enchendo duas carruagens e um
automóvel, saiu naquela tarde com o temeroso ocupante como guia. Ao fim
da via transitável eles desceram, e chapinharam em silêncio por milhas
no meio do bosque de ciprestes, onde o dia jamais chegou. Feias raízes e
entrenós malignos que pendiam da barba-de-velho os assolavam, e de vez
em quando uma pilha de pedras úmidas ou o fragmento de um muro pútrido
intensificava, pelo indício de mórbida habitação, uma depressão que cada
árvore malformada e toda ilhota de fungos combinava-se para criar.
Adiante, o assentamento dos ocupantes: um amontoado miserável de
barracas pairava à vista; e moradores histéricos corriam para fora a se
aglomerar junto das lanternas que balançavam. O batucar abafado dos
tambores agora se ouvia fraco e distante, muito distante; e um aulido
azedo vinha a intervalos irregulares, com a mudança do vento. Um brilho
avermelhado também se parecia filtrar através da pálida vegetação
rasteira, para além das infinitas aleias da floresta noturna. Relutantes
até mesmo em ficar a sós de novo, todos os acuados ocupantes se
recusavam definitivamente a avançar um centímetro que fosse para de onde
vinha aquela adoração profana, e então o inspetor Legrasse e seus
dezenove colegas meteram-se sem guia nas negras arcadas do horror que
lhes eram desconhecidas.

A região que a polícia agora adentrava era tradicionalmente de má
reputação, sobretudo incógnita e impenetrada por homens brancos. Havia
lendas de um lago escondido, jamais vislumbrado por olhos mortais, onde
vivia uma coisa pólipa, enorme, sem forma e branca, com olhos luminosos;
e alguns dos ocupantes murmuravam que demônios com asas de morcego
voavam das cavernas para a floresta, em adoração àquilo, à meia-noite.
Diziam que a coisa estivera lá desde antes de D'Iberville, antes de La
Salle,\footnote{Pierre Le Moyne d'Iberville (1661--1706), soldado,
  explorador e administrador colonial, o francês d'Iberville foi um dos
  fundadores de La Luoisiane, agora a Louisiana; René-Robert Cavalier,
  Sieur de La Salle (1643--1687), explorador e comerciante de peles que
  expedicionou pelo rio Mississippi.} antes dos índios, e mesmo antes
dos animais e pássaros bons do bosque. Um vivo pesadelo, e presenciá-lo
seria a morte. Mas fazia os homens sonhar, e assim sabiam o suficiente
para manter distância. Aquela orgia vodu estava, de fato, no exato
limite externo da área repulsiva, mas já a sua localização era ruim o
bastante; daí talvez o porquê de o próprio lugar de adoração ter
horrorizado os ocupantes mais do que os sons e incidentes ofensivos.

Só poesia, ou loucura, poderia fazer justiça aos ruídos ouvidos pelos
homens de Legrasse ao embrenharem-se mais e mais no negro atoleiro em
direção à luminosidade vermelha e aos tambores abafados. Há qualidades
vocais específicas dos humanos, e qualidades vocais específicas dos
animais: e é terrível ouvir uma quando a fonte deveria fazer soar a
outra. Fúria animal e licenciosidade orgíaca aqui se misturavam, em
alturas demoníacas, com êxtases de uivos e guinchados que irrompiam e
reverberavam pelos bosques noturnos como tempestades pestilentes, vindas
dos abismos do inferno. De vez em quando o ululado menos rítmico cessava
e, do que parecia um coro de vozes roucas, bem treinado, erguia-se num
cantochão aquela frase horrenda, ou ritual:

``\emph{Ph'nglui mglw'nafh Cthulhu R'lyeh wgah'nagl fhtagn.}''

E então, chegando a um ponto onde as árvores eram mais esguias, os
homens subitamente se viram diante daquele espetáculo. Quatro deles
cambalearam, um desmaiou e dois foram impelidos a uma gritaria
frenética, que, afortunadamente, a cacofonia insana da orgia não
permitia ouvir. Legrasse lançou a água do pântano no rosto do homem que
desmaiou, e todos ficaram ali tremendo e quase hipnotizados de horror.
Em uma clareira natural do pântano ficava uma ilha gramada, com por
volta de um acre de extensão, sem árvores e toleravelmente seca. Nela
agora saltava e se contorcia uma horda de anormalidade humana que só
poderia ser descrita como algo que um Sime ou um Angarola\footnote{Sidney
  Herbert Sime (1865--1941), artista inglês, conhecido por suas
  ilustrações das obras fantásticas de Lord Dunsany (1878--1957); Anthony
  Angarola (1893--1929), artista estadunidense que ilustrou, por exemplo,
  \emph{A Kingdom of Evil, a Continuation of the Journal of Fantazius
  Mallare} (1924), de Ben Hecht (1893--1964).} pintaria. Desprovidas de
roupa, essas crias híbridas estavam zurrando, mugindo e se retorcendo em
volta de uma fogueira monstruosa em forma de anel, no centro da qual,
revelado por fendas ocasionais na cortina de fogo, estava um grande
monolito granítico de uns dois metros e meio de altura, no topo do qual,
incongruente em seu tamanho diminuto, ficava entalhada a nociva
estatueta. Pendurados em dez andaimes dispostos num amplo círculo de
intervalos regulares, tendo o monolito cercado de fogo como centro,
estavam os corpos arruinados dos ocupantes indefesos que haviam
desaparecido. Era dentro do círculo que a roda de adoradores pulava e
rugia, e a direção geral em que a massa se movia era da esquerda para a
direita, num bacanal infinito entre a roda de corpos e a roda de fogo.

Talvez tenha sido só a imaginação, e talvez tenham sido só ecos o que
induziu um dos homens, um espanhol suscetível, a supor que ouvia
respostas antifonais ao ritual a partir de um ponto distante e sem luz,
ainda mais fundo naquele bosque de antiga lenda e horror. Esse homem,
Joseph D. Galvez, eu depois encontrei e questionei, e ele comprovou ser
intrigantemente imaginativo. Chegou ao ponto de sugerir um tênue bater
de asas amplas, um relance de olhos brilhantes e um montanhoso maciço
branco além das árvores mais remotas --- mas suponho que andasse ouvindo
muita superstição nativa.

Na verdade, a pausa horrorizada dos homens foi de duração
comparativamente breve. O dever veio antes; e a despeito do fato de que
talvez houvesse uma centena de celebrantes pardos na turba, a polícia
confiava em suas armas de fogo e mergulhou com determinação na nauseante
balbúrdia. Por cinco minutos o alvoroço e o caos resultantes foram
indescritíveis. Atracaram-se selvagemente, tiros foram disparados e
fugas se fizeram; porém, ao fim, Legrasse foi capaz de contar uns
quarenta e sete prisioneiros mal-encarados, a quem obrigou que se
vestissem depressa e formassem uma fila em meio a duas colunas de
policiais. Cinco dos adoradores estavam mortos, e os dois gravemente
feridos foram levados embora em macas improvisadas por seus colegas
prisioneiros. A imagem no monolito, é claro, foi removida cuidadosamente
e apreendida por Legrasse.

Examinados na delegacia após um percurso de muita tensão e cansaço,
constatou-se que os prisioneiros eram uns tipos de extração muito baixa,
mestiços e mentalmente aberrantes. A maioria vivia do mar, e um punhado
de negros e mulatos, sobretudo das Índias Ocidentais, ou portugueses de
Brava, das ilhas de Cabo Verde, dava um colorido de voduísmo ao culto
heterogêneo. Mas antes que se fizesse o interrogatório tornou-se
manifesto que algo muito mais profundo e antigo do que fetichismo negro
estava envolvido. Degradadas e ignorantes como fossem, as criaturas
mantinham consistência surpreendente na ideia central de sua fé
asquerosa.

Adoravam, assim o diziam, os Grandes Antigos que viveram muitas eras
antes de que houvesse o homem, e vieram do céu ao mundo ainda jovem.
Esses Antigos já se foram, para dentro da terra ou ao fundo do mar; mas
seus corpos mortos haviam contado os segredos em sonhos aos primeiros
homens, que assim formaram um culto jamais extinto. Esse era o seu
culto, e os prisioneiros disseram que sempre existira, e sempre
existiria, escondido nos ermos distantes e nos lugares escuros de todo o
mundo, até que o grande sacerdote Cthulhu se erguesse de sua escura
morada na imponente cidade de R'lyeh sob as águas, e submetesse a terra
novamente a seu jugo. Um dia faria o chamado, quando as estrelas
estivessem prontas, e o culto secreto estaria sempre aguardando para
liberá-lo.

No ínterim, nada mais havia a ser dito. Havia um segredo que nem mesmo a
tortura poderia extrair. A humanidade não se via absolutamente só entre
as coisas conscientes da Terra, pois as formas se arrancam do escuro
para visitar seus poucos fiéis. Mas essas não eram os Grandes Antigos.
Homem algum vira os Grandes Antigos. O ídolo entalhado era o grande
Cthulhu, mas ninguém saberia dizer se os outros eram precisamente como
ele. Ninguém agora era capaz de ler a antiga escrita, mas as coisas eram
passadas boca-a-boca. O ritual cantado não era o segredo --- esse jamais
fora dito em voz alta, mas apenas em sussurros. O canto significava
apenas isto: ``Em sua morada em R'lyeh o morto Cthulhu aguarda
sonhando''.

Apenas dois dos prisioneiros foram considerados sãos o suficiente para
ser enforcados, e o resto foi internado em várias instituições. Todos
negaram participar dos assassinatos rituais, e declararam que a matança
fora feita pelos Asas Negras, que lhes vieram de seu imemorial ponto de
encontro no bosque assombrado. Porém, a respeito daqueles aliados
misteriosos, nenhum relato coerente se pôde extrair. O que a polícia de
fato extraiu veio principalmente de um mestiço tremendamente idoso
chamado Castro, que alegava ter navegado a portos estranhos e falado com
líderes imortais do culto nas montanhas da China.

O velho Castro lembrava partes de uma lenda horrorosa que humilhava as
especulações dos teósofos, e fazia o homem e o mundo parecerem de fato
recentes e transitórios. Houve eras em que as outras Coisas dominavam a
Terra, e Elas haviam feito grandes cidades. Resquícios d'Elas, lhe
haviam dito os chineses imortais, podiam ser encontrados como rochas
ciclópicas nas ilhas do Pacífico. Todas tinham morrido vastas épocas no
tempo antes da chegada do homem, mas certas artes poderiam revivê-Las
quando as estrelas retornassem às posições corretas no ciclo da
eternidade. Elas vinham, de fato, das estrelas, e traziam Suas imagens
Consigo.

Esses Grandes Antigos, Castro prosseguiu, não eram compostos
inteiramente de carne e osso. Tinham forma --- pois não o provava aquela
imagem estelar? ---, mas a forma não era feita de matéria. Quando as
estrelas estivessem certas, Eles mergulhariam de mundo em mundo através
do céu; mas quando estivessem erradas, não poderiam viver. Mas ainda que
não mais vivessem, Eles nunca realmente morriam. Repousam todos em
moradas de pedra em Sua grande cidade de R'lyeh, preservados pelos
feitiços do poderoso Cthulhu para um ressurreição gloriosa quando as
estrelas e a Terra outra vez estivessem prontas para Eles. Naquele
momento, então, alguma força de fora deveria servir para liberar Seus
corpos. Os feitiços que Os haviam preservado intactos da mesma maneira
impediam-Nos de fazer um movimento inicial, e Eles podiam apenas
repousar despertos no escuro e pensar, enquanto incontáveis milhões de
anos se passavam. Sabiam tudo o que estava ocorrendo no universo, mas
Seu modo de fala era o pensamento transmitido. Mesmo agora, Eles falavam
em Suas tumbas. Quando, após infinidades de caos, os primeiros homens
surgiram, os Grandes Antigos falaram aos sensitivos moldando seus
sonhos; pois apenas assim Sua linguagem alcançava a mente carnal dos
mamíferos.

Então, sussurrou Castro, aqueles primeiros homens formaram o culto em
torno de pequenos ídolos que os Grandes Antigos lhes mostraram; ídolos
trazidos de estrelas escuras em eras turvas. Aquele culto jamais
morreria antes que as estrelas se ajustassem de novo, e os sacerdotes
secretos trariam o grande Cthulhu de Sua tumba para reviver Seus súditos
e retomar Seu domínio da Terra. O momento seria fácil perceber, pois a
humanidade se tornaria como os Grandes Antigos: livre e selvagem e além
do bem e do mal, lançando as leis e a moral de lado, com todos os homens
gritando e matando e se regozijando de prazer. Então os Grandes Antigos,
liberados, ensinariam a eles novos modos de gritar e matar e
regozijar-se de prazer, e toda a Terra se inflamaria em um holocausto de
êxtase e liberdade. Enquanto isso, o culto, pelos ritos apropriados,
deveria manter viva a memória daqueles costumes antigos e adumbrar a
profecia de seu retorno.

Em tempos distantes, homens escolhidos falaram por sonho com os Antigos
em suas tumbas, mas algo acontecera. A grande cidade pétrea de R'lyeh,
com seus monolitos e sepulcros, afundara sob as ondas; e as águas
profundas, repletas de um mistério primevo através do qual sequer o
pensamento pode passar, cortaram o intercâmbio espectral. Mas a memória
não morre jamais, e os altos sacerdotes disseram que a cidade se
ergueria novamente quando as estrelas estivessem certas. Daí saíram da
terra os espíritos negros da terra, de bolor e sombra, e cheios de
turvos rumores entreouvidos em cavernas sob o fundo esquecido do oceano.
Deles, no entanto, o velho Castro não ousava falar muito. Interrompeu-se
de repente e persuasão ou sutileza alguma pôde obter mais naquele
sentido. O \emph{tamanho} dos Antigos, também, ele curiosamente
declinara mencionar. Sobre o culto, disse pensar que o centro ficava em
meio aos desertos inviáveis da Arábia, onde Irem, a Cidade dos
Pilares,\footnote{Cidade perdida mencionada no Corão.} sonha escondida e
intocada. Não estava ligado ao culto das bruxas da Europa, e era
virtualmente desconhecido para além de seus membros. Livro algum jamais
o mencionou, embora os chineses imortais tenham dito que havia
duplos-sentidos no \emph{Necronomicon}\footnote{\emph{Necronomicon} é o
  título de um grimório fictício, daquele autor árabe fictício, ambos
  inventados por Lovecraft. O nome original em árabe seria \emph{Al
  Azif}, que propõe o ruído dos insetos à noite como sendo o ruído das
  vozes dos demônios. O livro foi famosamente usado na trilogia de
  filmes de horror sobrenatural de Sam Raimi, \emph{Evil Dead}
  (1981--1992), mas é referido lá como antigo livro sumério. Raimi, além
  disso, usou na abertura do filme a sonoplastia de insetos para efeito
  perturbador, assim como William Friedkin o fez em \emph{The Exorcist}
  (1973).} do árabe insano Abdul Alhazred, que os iniciados podiam ler
como quisessem, em especial o dístico controverso:

\begin{quote}
\forceindent{}Morto não está se eterno pode adormecer,

E em estranhas eras mesmo a morte irá morrer.
\end{quote}

Legrasse, muito impressionado e não pouco perplexo, perguntara em vão a
respeito das conexões históricas do culto. Castro aparentemente havia
dito a verdade quando alegou que era de todo secreto. As autoridades na
Universidade de Tulane não conseguiam lançar alguma luz, seja sobre o
culto, seja sobre a imagem, e agora o detetive chegara às maiores
autoridades no país e não pôde encontrar mais do que a história do
professor Webb na Groenlândia.

O interesse febril despertado na conferência por conta da narrativa de
Legrasse, corroborada como fora pela estatueta, é ecoada na
correspondência subsequente daqueles que compareceram ao evento, embora
se encontrem notas escassas nas publicações formais da sociedade.
Cautela é o primeiro cuidado dos que estão acostumados a se confrontar
com ocasionais charlatanice e impostura. Legrasse emprestou por algum
tempo a imagem ao professor Webb, mas à morte deste ela lhe fora
devolvida e permanece em sua possessão, onde a vi não faz muito. É de
fato uma coisa terrível, e de afinidade indubitável com a escultura de
sonho do jovem Wilcox.

Não me surpreendeu que meu tio estivesse empolgado pela história do
escultor, pois quais pensamentos não surgiriam após saber-se o que
Legrasse descobrira do culto, e ouvindo de um jovem que \emph{sonhara}
não apenas a figura e os exatos hieróglifos da imagem achada no pântano
e no diabólico tablete da Groenlândia, mas que tivera \emph{em seus
sonhos} ao menos três das palavras exatas da fórmula pronunciada tal e
qual pelos esquimós diabolistas e aqueles vira-latas da Louisiana? O
início imediato de uma investigação muito meticulosa pelo professor
Angell era apenas natural, apesar de que eu suspeitava, reservadamente,
que o jovem Wilcox ouvira sobre o culto de algum modo indireto, e que
havia inventado uma série de sonhos para ampliar e continuar o mistério
às custas do meu tio. As narrativas de sonho e os recortes compilados
pelo professor eram, é claro, uma corroboração forte; mas a
racionalidade da minha mente e a extravagância do assunto como um todo
me levaram a adotar as conclusões mais sensatas. Portanto, após
novamente estudar em detalhe o manuscrito, e correlacionar as notas
teosóficas e antropológicas com a narrativa de Legrasse sobre o culto,
viajei para Providence para ver o escultor e lhe dar a reprimenda que
julguei apropriada por seu abuso descarado de um homem culto e idoso.

Wilcox ainda morava sozinho no edíficio Fleur-de-Lys, em Thomas Street,
uma imitação vitoriana medonha da arquitetura bretã do século \versal{XVII},
ostentando sua fachada de estuque em meio às casas coloniais charmosas
na velha colina; e exatamente sob a sombra do melhor campanário
georgiano da América eu o encontrei trabalhando em seus aposentos; e
devo conceder que, pelos exemplos espalhados lá por toda parte, seu
gênio é de fato profundo e autêntico. Em algum tempo se ouvirá falar
dele, acredito, como um dos grandes decadentes; pois cristalizou em
argila e um dia espelhará em mármore aqueles pesadelos e fantasias que
Arthur Machen evoca em prosa, e Clark Ashton Smith\footnote{Arthur
  Machen (1863--1947) escritor e místico galês, cuja fama literária no
  meio fantástico se dá especialmente pela novela \emph{The Great God
  Pan} (1894), admirada por Lovecraft e também por Stephen King, que a
  considera ``talvez a melhor história de horror da língua inglesa'';
  Clark Ashton Smith (1893--1961), escritor e ilustrador estadunidense:
  seu universo visual tem muita semelhança com o tipo de imagens
  construídas por Lovecraft, mas também (e talvez sobretudo) seu estilo
  de escrita. Seus poemas em verso e prosa trazem títulos como
  ``Eidolon'', ``The Dream-God's Realm'', ``The Abyss Triumphant'' e
  ``The Abomination of Desolation''. Este último, por exemplo, começa da
  seguinte maneira: ``Do deserto de Soom se diz que jaz no extremo
  inexplorável do mundo, entre as terras pouco conhecidas e as que
  sequer se conjecturou. É temida pelos viajantes, por suas areias nuas
  e moventes, sem oásis, e rumores dizem que um estranho horror reside
  ali''.} torna visíveis em verso e pintura.

Sombrio, frágil e algo descuidado com seu aspecto, voltou-se
languidamente quando bati à sua porta e perguntou, sem se levantar, o
que queria ali. Eu lhe disse quem era, e ele demonstrou algum interesse,
pois meu tio havia despertado sua curiosidade ao sondar-lhe os estranhos
sonhos, mas sem nunca explicar a razão para o estudo. Não ampliei seu
conhecimento sobre o assunto, mas busquei pegá-lo com algum ardil. Em
pouco tempo eu estava convencido de sua absoluta sinceridade, pois
falava dos sonhos de modo a não deixar dúvidas. Os sonhos e seu resíduo
subconsciente haviam influenciado sua arte profundamente, e me mostrou
uma estátua mórbida cujos contornos quase me fizeram estremecer com a
potência de sua negra sugestão. Ele não se recordava de ter visto o
original dessa coisa, com a exceção de seu próprio baixo-relevo de
sonho, mas seu traçado se formara insensivelmente sob suas mãos. Era,
sem dúvida, a figura gigante a respeito da qual delirara. Deixou logo
claro que nada sabia do culto obscuro, excetuando o que o catecismo
incansável do meu tio lhe havia deixado; e novamente lutei para
encontrar algum modo pelo qual tivesse recebido as bizarras impressões.

Falava de seus sonhos de uma maneira estranhamente poética, fazendo-me
ver com terrível vividez a úmida cidade ciclópica de rochas verdes e
viscosas --- cuja
\emph{geometria}, ele notava estranhamente, estava \emph{toda errada}
--- e ouvir com apavorada expectativa o incessante chamado subterrâneo,
em parte mental: ``\emph{Cthulhu fhtagn}'', ``\emph{Cthulhu fhtagn}''.
Essas palavras faziam parte daquele temido ritual que dizia da vigília
de sonho do morto Cthulhu em sua cripta de pedra em R'lyeh, e me senti
abalado, a despeito das minhas crenças racionais. Wilcox, agora tinha
certeza, ouvira sobre o culto de maneira casual, e logo o esquecera em
meio à massa de suas leituras e imaginações igualmente bizarras. Depois,
em virtude daquela força impressionante, encontrara expressão
subconsciente nos sonhos, no baixo-relevo e na terrível estátua que eu
agora contemplava; de modo que sua impostura com meu tio fora na verdade
muito inocente. O jovem era um tipo em parte um pouco afetado, em parte
mal-educado, de que nunca pude gostar; mas estava agora suficientemente
disposto a admitir tanto seu gênio quanto sua honestidade. Despedi-me
amigável, e desejei-lhe todo o sucesso que seu talento promete.

A questão do culto ainda continuou a me fascinar, e por vezes eu tinha
visões de fama pessoal advinda de pesquisas sobre sua origem e conexões.
Visitei New Orleans, falei com Legrasse e outros daquele grupo antigo de
busca no pântano, vi a imagem apavorante e até mesmo alguns daqueles
prisioneiros pardos que haviam sobrevivido. O velho Castro,
infelizmente, morrera há alguns anos. O que ouvi então, de modo tão
gráfico e em primeira-mão, embora não fosse mais do que uma confirmação
detalhada do que meu tio havia escrito, instigou-me de novo; pois tinha
certeza de estar no encalço de uma religião muito real, muito secreta e
muito antiga, cuja descoberta faria de mim um antropólogo de destaque.
Minha atitude ainda era de absoluto materialismo, \emph{como gostaria
que ainda fosse}, e descartava com uma perversidade quase inexplicável a
coincidência das anotações de sonho e os recortes peculiares coletados
pelo professor Angell.

Uma coisa de que comecei a suspeitar, e que temo agora saber: que a
morte de meu tio tenha sido tudo menos natural. Ele tombou em uma rua
estreita na colina, que subia de uma antiga zona portuária infestada de
uns pardos estrangeiros, e depois de um empurrão descuidado de um
marinheiro negro. Não esqueci do sangue mestiço e das buscas marinhas
dos membros do culto na Louisiana, e não me surpreenderia se desvendasse
métodos secretos e agulhas envenenadas tão implacáveis e de ciência tão
antiga quanto as crenças e os ritos crípticos. Legrasse e seus homens, é
verdade, foram deixados em paz; mas na Noruega certo marujo que viu
coisas está morto. Não teriam as perquirições de meu tio, após encontrar
os dados do escultor, chegado a ouvidos sinistros? Creio que o professor
Angell morreu porque sabia demais, ou porque estava prestes a sabê-lo.
Se o mesmo acontecerá comigo ainda está por se ver, porque agora também
soube muito.


\chapter[A loucura vinda do mar \bigskip]{A loucura vinda do mar}

\noindent{}Se o céu um dia quiser conceder-me uma dádiva, será o apagamento total
dos resultados de um mero acaso que fixou meu olho em certo jornal velho
de forrar estante. Não era algo em que eu normalmente esbarraria durante
um dia ordinário, pois se tratava de antigo exemplar de jornal
australiano, o \emph{Sydney Bulletin,} de 18 de abril de 1925. Escapara
até mesmo ao escritório de recortes que, no momento daquela tiragem,
andava coletando material avidamente para a pesquisa do meu tio.

Eu havia em grande parte desistido das minhas investigações sobre aquilo
que o professor Angell chamava o ``Culto de Cthulhu'', e estava
visitando um amigo muito culto, curador de um museu local e
mineralogista destacado em Paterson, New Jersey. Um dia, ao examinar os
espécimes da reserva dispostos preliminarmente nas estantes de estoque
na sala dos fundos do museu, meu olho captou uma curiosa figura em um
dos jornais velhos estendidos sob as pedras. Era o \emph{Sydney
Bulletin} que mencionei, porque meu amigo tem contatos em todos os
pontos imagináveis no exterior; e a reprodução era um recorte de
meio-tom de uma horrenda imagem de pedra, quase idêntica à que Legrasse
encontrara no pântano.

Ansiosamente desobstruindo a folha de seus preciosos itens, analisei-a
em detalhe e fiquei desapontado em constatar que se estendia pouco. O
que sugeria, entretanto, era de significado portentoso para minha
empreitada pendente; e cuidadosamente rasguei o pedaço para ação
imediata. Lia-se o seguinte:

\begin{quote}
\versal{MISTERIOSO NAVIO À DERIVA É ENCONTRADO NO MAR}

Vigilant Chega Rebocando Barco Neozelandês Armado e Deserto.

Um Sobrevivente e um Morto Encontrados a Bordo.

Relato de Batalha Desesperada e de Mortes no Mar.

Marujo Resgatado Recusa

Detalhar Estranha Experiência.

Ídolo Incomum Encontrado em Sua Posse.

Será Instaurado Inquérito.
\end{quote}

\emph{Vigilant}, o cargueiro da Morrison Co., com destino a Valparaíso,
aportou nesta manhã no cais de Darling Harbour, rebocando o barco a
vapor, desabilitado, mas que viu batalha e continha muitas armas, de
nome \emph{Alert}, de Dunedin, \versal{N.Z.}, avistado em 12 de abril na latitude
sul 34º~21', longitude oeste 152º 17', com um homem vivo e um morto a bordo.

O \emph{Vigilant} deixou Valparaíso em 25 de março, e em 2 de abril foi
impulsionado muito ao sul da rota designada por tempestades
excepcionalmente duras e ondas monstruosas. Em 12 de abril o navio
perdido foi avistado e, embora aparentemente deserto, notou-se,
abordado, que continha um sobrevivente em estado de semi-delírio e um
outro evidentemente morto há mais de uma semana. Aquele que estava vivo
agarrava-se a um horrível ídolo de pedra, de origem desconhecida, com
uns trinta centímetros de altura, sobre cuja natureza todas as
autoridades da Universidade de Sydney, da Royal Society e do Museu em
College Street professam completa perplexidade, e que o sobrevivente diz
ter encontrado na cabine do barco, num pequenino altar entalhado de
padrão comum.

Esse homem, após recobrar os sentidos, contou uma história extremamente
estranha de pirataria e carnificina. Chama-se Gustaf Johansen, norueguês
de alguma inteligência, e era o segundo imediato da escuna de dois
mastros \emph{Emma}, de Auckland, que navegou para Callao em 20 de
fevereiro com uma equipagem de onze homens. O \emph{Emma}, ele diz,
sofreu atraso e foi lançado bem ao sul de sua rota graças à grande
tempestade de 1º de março, e a 22 de março, na
latitude sul 49º~51', longitude
oeste 128º~34′, encontrou o \emph{Alert} operado por uma tripulação
esquisita e mal-encarada de canacas e de miscigenados. Recebendo ordem
direta de voltar, o capitão Collins recusou, e assim a estranha
tripulação começou a disparar selvagemente e sem aviso na direção da
escuna, com uma bateria de canhões de bronze especialmente pesada, que
fazia parte do equipamento do vapor. Os homens do \emph{Emma} deram
combate, diz o sobrevivente, e, apesar de a escuna começar a afundar com
os disparos abaixo da linha de flutuação, eles conseguiram arremessá-la
ao lado de seu inimigo e abordá-lo, atracando-se com a tripulação
selvagem no convés do vapor, sendo forçados a matá-los todos, seu número
sendo ligeiramente superior, por causa do modo aberrante e desesperado,
ainda que confuso, com que lutavam.

Três dos homens do \emph{Emma}, incluindo o capitão Collins e o primeiro
imediato Green, foram mortos; e os oito remanescentes, sob as ordens do
segundo imediato Johansen, procederam a navegação do barco capturado,
prosseguindo na direção original para ver se existia algum motivo para a
ordem de voltar. No outro dia, ao que parece, levantaram e desembarcaram
em uma ilha pequena, embora não se soubesse existir ilha alguma naquela
parte do oceano; e seis dos homens morreram de alguma forma em terra
firme, ainda que Johansen seja estranhamente reticente sobre essa parte
de sua história, e se concentre apenas na queda deles em um precipício
rochoso. Mais tarde, aparentemente, ele e um companheiro embarcaram no
vapor e tentaram manejá-lo, mas foram fustigados pela tempestade de 2 de
abril. Daí até o momento do resgate no dia 12 o homem lembra pouco, e
sequer rememora quando William Briden, seu companheiro, morreu. A morte
de Briden não tem causa aparente, e teria ocorrido por estresse ou
exposição aos elementos. Informações por telegrama, vindas de Dunedin,
registram que o \emph{Alert} era bem conhecido lá como comerciante
ilhéu, e tinha má reputação na zona portuária. Era de propriedade de um
grupo peculiar de miscigenados, cujos encontros frequentes e excursões
noturnas aos bosques atraíram não pouca curiosidade; e havia zarpado com
muita afobação logo depois da tempestade e dos tremores de terra de
1º de março. Nosso correspondente em Auckland
atesta a excelente reputação do \emph{Emma} e de sua tripulação, e
Johansen é descrito como homem sóbrio e de valor. O almirantado vai
instaurar um inquérito com relação a todo o ocorrido, a começar de
amanhã, e no qual fará todo o esforço de induzir Johansen a falar com
mais liberdade do que o fez até o momento.

Isso foi tudo, além da figura com a imagem infernal; mas que sucessão de
ideias precipitou em minha mente! Ali havia novos e inteiros compêndios
de dados sobre o Culto de Cthulhu, e evidência de que possuía estranhos
interesses tanto no mar como na terra. Qual motivo teria levado a
tripulação híbrida a ordenar a volta do \emph{Emma} quando navegavam por
aí com seu ídolo horrendo? Qual seria a ilha desconhecida em que seis
dos tripulantes do \emph{Emma} morreram, e sobre a qual Johansen era tão
sigiloso? O que a investigação do vice-almirantado desvendou, e o que se
sabia do nocivo culto em Dunedin? E o mais maravilhoso de tudo, que
ligação profunda e além do natural era essa entre as datas, que conferia
significância maligna e agora inegável ao variado desenrolar de
acontecimentos, tão cuidadosamente anotado por meu tio?

Em 1º de março --- para nós 28 de fevereiro, de
acordo com a Linha Internacional de Datas --- o terremoto e a tempestade
vieram. Vindo de Dunedin, o \emph{Alert} e sua perniciosa equipagem
corriam ansiosamente, como se convocados de maneira irresistível, e do
outro lado da Terra poetas e artistas haviam começado a sonhar com uma
estranha, úmida cidade ciclópica, enquanto um jovem escultor moldava
durante o sono a forma do temido Cthulhu. Em 23 de março a tripulação do
\emph{Emma} desembarcou numa ilha desconhecida e a deixou com seis
homens mortos; naquela data os sonhos de homens sensitivos ganharam
amplificada vividez e se turvaram com o terror da busca maligna de um
monstro gigante, enquanto um arquiteto enlouquecia e um escultor
repentinamente mergulhava em delírio! E quanto a essa tempestade do dia
2 de abril --- data na qual todos os sonhos da cidade úmida cessaram, e
Wilcox emergiu ileso do cativeiro de estranha febre? E quanto a tudo
isso --- e àquelas pistas que deixou o velho Castro sobre os Grandes
Antigos, nascidos das estrelas e submersos, e o seu reino porvir, seu
culto fiel e \emph{seu domínio sobre os sonhos}? Estaria eu cambaleando
à beira de horrores cósmicos além da capacidade humana de suportar? Se
sim, devem ser horrores apenas da mente, pois de algum modo o dois de
abril pusera um fim a quaisquer ameaças monstruosas que houvessem
iniciado seu cerco à alma humana.

Naquela noite, após um dia de arranjos e telegramas apressados,
despedi-me de meu anfitrião e tomei um trem para São Francisco. Em menos
de um mês estava em Dunedin, onde, entretanto, descobri que pouco se
sabia dos estranhos membros de culto que haviam frequentado as velhas
tavernas da costa. A escória da região dos portos era ordinária demais
para merecer menção especial, embora houvesse uma vaga conversa sobre
certa incursão em terra que um daqueles mestiços fizera, na qual uma
tênue batucada e uma flama vermelha tinham sido notadas em colinas
distantes. Em Auckland soube que Johansen retornara \emph{com o cabelo
loiro todo embranquecido} após um perfunctório e inconclusivo
interrogatório em Sydney, e havia em seguida vendido seu chalé em West
Street e viajado com a esposa para seu antigo lar em Oslo. De sua
alarmante experiência ele não diria a seus amigos mais do que contou aos
oficiais do almirantado, e tudo o que puderam fazer foi me passar seu
endereço em Oslo.

Depois disso fui a Sydney e falei inutilmente com marujos e membros da
corte do vice-almirantado. Vi o \emph{Alert}, agora vendido e posto a
uso comercial no Cais Circular, em Sydney Cove, mas nada ganhei com sua
estrutura indiferente. A imagem acocorada com cabeça de choco, corpo de
dragão, asas escamosas e pedestal coberto de hieróglifos foi conservada
no Museu do Hyde Park; e a estudei bem e longamente, considerando-a
coisa de incomum e nefasta habilidade, e do mesmo mistério absoluto, a
mesma antiguidade terrível e sobrenatural estranheza de material que
notara no espécime menor de Legrasse. Geólogos, disse-me o curador,
haviam-na considerado um enigma abominável; porque garantiam que o mundo
não possuía uma pedra como aquela. Então lembrei, num calafrio, daquilo
que o velho Castro dissera a Legrasse sobre os Grandes Antigos primevos:
``Eles vieram das estrelas, e trouxeram Suas imagens Consigo''.

Abalado por uma revolução mental que jamais experimentara, estava
resolvido a visitar o imediato Johansen em Oslo. Navegando a partir de
Londres, reembarquei de uma vez para a capital norueguesa; e em um dia
de outono saltei nos cais impecáveis à sombra do Egeberg.\footnote{Castelo
  próximo da região portuária em Oslo, mandado construir por Einar
  Westye Egeberg (1851--1940). Também o nome de uma geleira na
  Antárctica.} O endereço de Johansen, vim a descobrir, ficava na velha
cidade do rei Harold Haardrada, que manteve vivo o nome de Oslo durante
os séculos em que a cidade expandida se mascarava como
``Christiana''.\footnote{Grafado ``Christiana'' no original inglês de
  Lovecraft. Em 1624 a cidade conhecida como Oslo foi renomeada
  \emph{Christiania} em homenagem ao rei Christian \versal{IV}, após um incêndio
  devastador. Em 1877 a grafia foi trocada por \emph{Kristania} e, em
  1925, a cidade mais ampla incorporou a antiga região de Oslo, e foi
  renomeada assim.} Fiz o breve trajeto de táxi, e com o coração aos
saltos bati à porta de um edifício elegante e antigo, com fachada de
estuque. Uma mulher atendeu, de rosto triste e vestida de preto, e me vi
doído de decepção quando me disse em inglês hesitante que Gustaf
Johansen já não existia.

Não sobrevivera ao seu retorno, sua esposa me disse, pois os
acontecimentos no mar em 1925 o haviam destruído. Ele não lhe contou
mais do que havia dito ao público, mas deixara um longo manuscrito ---
sobre ``assuntos técnicos'', em suas palavras --- redigido em inglês
evidentemente para protegê-la do perigo de uma leitura casual. Durante
uma caminhada por certa travessa estreita junto das docas de Gothenburg,
um embrulho de papéis caindo da janela de um sótão o havia derrubado.
Dois marinheiros lascares\footnote{Nativo das Índias Orientais.}
imediatamente o ajudaram a se levantar, mas morreu antes que a
ambulância chegasse. Os médicos não puderam encontrar uma causa adequada
para o óbito, e atribuíram-no a problema cardíaco e a uma constituição
debilitada.

Senti então revolver nas minhas entranhas aquele terror escuro que não
vai me deixar até que também chegue a minha hora; ``acidentalmente'', ou
de outro modo. Persuadindo a viúva de que minha conexão com os
``assuntos técnicos'' era autorização suficiente para ter o manuscrito,
levei comigo o documento e comecei a lê-lo no barco para Londres. Era
uma coisa simplória, incoerente --- a tentativa de diário \emph{a
posteriori} por um marinheiro ingênuo --- e lutava por relembrar
dia-a-dia aquela última viagem nefanda. Sequer posso tentar
transcrevê-lo \emph{ipsis litteris} em toda sua nebulosidade e
redundância, mas farei um resumo suficiente para mostrar por que o som
da água contra as laterais do navio se tornou tão insuportável para mim
que tive de tapar os ouvidos com algodão.

Johansen, graças a Deus, não entendeu muito, apesar de ter visto a
cidade e a Coisa, mas eu jamais dormirei tranquilo novamente enquanto
pensar sobre os horrores que espreitam sem cessar por trás da vida, em
tempo e espaço, e naquelas profanas blasfêmias vindas de estrelas
antigas e que sonham sob o mar, conhecidas e protegidas por um culto de
pesadelo, ansioso e pronto para soltá-las no mundo assim que um outro
terremoto lance sua monstruosa cidade de pedra de volta ao sol e ao ar.

A viagem de Johansen começara assim como havia dito ao vice-almirantado.
O \emph{Emma}, com lastro, zarpou de Auckland no dia 20 de fevereiro, e
sofreu o impacto total da tempestade gerada pelo terremoto que deve ter
lançado ao alto, do fundo do mar, os horrores que povoavam os sonhos dos
homens. Retomado o controle, o barco seguia bem quando parou por causa
do \emph{Alert}, a 22 de maio, e pude sentir o desgosto do imediato ao
escrever sobre o bombardeio e o naufrágio; daqueles amorenados
pervertidos do culto, no \emph{Alert,} fala com horror significativo.
Havia uma qualidade conspicuamente abominável neles, que fez sua
destruição soar quase como um dever, e Johansen mostra ingênua surpresa
diante da acusação de desumanidade levantada contra o seu grupo nos
procedimentos do tribunal de inquérito. Então, levados pela curiosidade
a respeito do vapor capturado, sob o comando de Johansen, os homens
avistam um grande pilar de pedra avultando das águas, e na latitude sul,
47º~9′, longitude oeste, 126º~43′, surge uma linha costeira feita de
lama misturada a lodo e de construções ciclópicas cobertas por alga, o
que não podia ser senão a substância tangível do supremo horror
terrestre --- o pesadelo da cidade-cadáver de R'lyeh, edificada em eras
imensuráveis além da História pelas formas repugnantes e vastas que
escorreram das estrelas negras. Lá repousam o grande Cthulhu e suas
hordas, escondidos nas criptas verdes, viscosas, enviando por fim, após
ciclos incalculáveis, os pensamentos que espalharam medo nos sonhos dos
sensitivos, e chamaram imperativamente os fiéis a vir em peregrinação
para liberar e restaurar. Johansen nada suspeitava disso, mas Deus sabe
que logo veria o bastante!

Suponho que apenas o topo de uma montanha, a horrenda cidadela coroada
de um monolito, onde o grande Cthulhu estava sepultado, tenha de fato
emergido das águas. Quando penso na \emph{extensão} do que possa estar
sendo incubado naquelas profundezas, quase tenho o desejo de me matar na
mesma hora. Johansen e seus homens se viram pasmados pela majestade
cósmica dessa Babilônia gotejante de demônios antigos, e devem ter
adivinhado sem auxílio algum que aquilo não poderia ser deste nem de
qualquer outro planeta saudável. O pasmo frente ao tamanho inacreditável
dos blocos rochosos esverdeados, frente à altura vertiginosa do grande
monolito esculpido, frente à alarmante identidade entre as estátuas
colossais, os baixos-relevos e a esquisita imagem achada no altar do
\emph{Alert}, é fortemente visível em cada linha da descrição apavorada
do imediato.

Sem saber como é o futurismo, Johansen alcançou algo muito semelhante
quando falou da cidade; pois ao invés de descrever uma construção ou uma
estrutura definida qualquer, ele se dedica apenas a impressões gerais de
ângulos vastos e superfícies de pedra --- superfícies grandes demais
para pertencer a qualquer coisa correta ou própria deste mundo, e ímpias
em suas imagens horríveis e hieróglifos. Menciono o que diz de
\emph{ângulos} porque sugere algo que Wilcox me contara de seus sonhos
pavorosos. Ele me dissera que a \emph{geometria} do lugar-de-sonho que
via era anormal, não-euclidiana, e odiosamente recendente a esferas e
dimensões separadas da nossa. E agora um marujo de poucas letras sentia
a mesma coisa ao contemplar a terrível realidade.

Johansen e seus homens desembarcaram em um barranco lamacento e
enviesado naquela monstruosa Acrópole, e escalaram às deslizadas os
titânicos blocos viscosos que não poderiam jamais ser uma escada para
mortais. Até mesmo o sol no firmamento parecia distorcido quando visto
através dos miasmas polarizantes exalando-se daquela perversão
encharcada de mar, e um suspense e uma ameaça turvos espreitavam
maliciosamente naqueles ângulos de elusiva insanidade de pedra talhada,
onde um outro relance mostrava côncavo o que fora antes convexo.

Algo muito semelhante ao pavor tomou todos os exploradores antes que se
visse qualquer coisa de mais definido do que rocha e lodo e algas. Todos
teriam fugido se não temessem o desprezo dos companheiros, e davam
buscas sem empenho --- e em vão, como se provou --- atrás de algum souvenir que pudessem carregar consigo.

Foi Rodriguez,\footnote{Mantém-se a grafia do nome como Lovecraft o
  escreveu, à espanhola.} o português, quem escalou o pé do monolito e
gritou anunciando o que descobrira. Os restantes o seguiram, e olhavam
com curiosidade para a imensa porta entalhada com o baixo-relevo agora
já familiar do molusco-dragão. Era, disse Johansen, como uma enorme
porta de celeiro; e todos sentiram que fosse porta por causa do lintel
ornado, do umbral e dos batentes à volta, embora não pudessem decidir se
jazia reta como a de um alçapão, ou oblíqua como porta externa de porão.
Como Wilcox teria dito, a geometria do lugar era toda equívoca. Não se
podia ter certeza se o mar e o chão eram horizontais, daí a posição
relativa de todo o resto parecer espectralmente variável.

Briden empurrou a pedra em diversos locais, sem resultado. E então
Donovan apalpou cuidadoso em torno da beirada, apertando, assim, cada
ponto em separado. Escalou interminavelmente ao longo do portal --- isto
é, pode-se dizer \emph{escalar} se a coisa afinal de contas não era
horizontal ---, e os homens sideravam sobre como diabos uma porta podia
ser tão vasta. Então, aos poucos e suavemente, o painel de um acre
começou a ceder para dentro no topo, e eles notaram que era balanceado.

Donovan deslizou, ou de algum modo impeliu-se ao longo do batente, ou
para baixo, juntando-se a seus companheiros, e todos observaram o
estranho recuo do monstruoso portal entalhado. Nessa fantasia de
distorção prismática a coisa se movia de um modo anômalo, diagonal,
fazendo com que as regras da matéria e da perspectiva parecessem todas
abaladas.

A abertura era negra, de uma escuridão quase material. A tenebrosidade
tinha realmente uma \emph{qualidade positiva}, pois obscurecia certas
partes dos muros internos que se teriam revelado, e que se expeliam como
fumaça de seu aprisionamento de eras, visivelmente obscurecendo o sol,
recolhendo-se furtivo no céu encolhido e minguante diante do bater de
asas membranosas. O odor exalado das funduras agora reabertas era
intolerável, e em pouco Hawkins, de ouvidos alerta, pensou ouvir um som
repulsivo, espirrado lá do fundo. Todos ouviram, e todos ainda ouviam
quando, diante dos olhos, a Coisa alastrou-se salivando e, no tatear,
espremeu Sua verde imensidade gelatinosa através do umbral negro, em
direção ao ar maculado daquela cidade venenosa de loucura.

A caligrafia do pobre Johansen quase se desfez quando escreveu isso; dos
seis homens que nunca alcançaram o navio julga que dois pereceram do
puro susto daquele instante amaldiçoado. A Coisa não pode ser descrita
--- não há língua para abismos de tamanha demência ululante e imemorial,
tão insólitas contradições de toda matéria, força, ordem cósmica. Uma
montanha caminhava ou se arrastava. Deus! Quem se surpreende que ao
redor do mundo um grande arquiteto tenha enlouquecido, e que o pobre
Wilcox tenha desandado em febre naquele instante telepático? A Coisa dos
ídolos, a cria verde e viscosa das estrelas, despertara para clamar seu
direito. As estrelas estavam certas de novo, e o que um culto
antiquíssimo falhara em conseguir de propósito, um bando de marinheiros
inocentes havia feito por acidente. Após vigintilhões de anos o grande
Cthulhu estava livre uma vez mais, e faminto por deleite.

Três homens foram varridos pelas garras flácidas antes que pudessem se
mover: que Deus lhes dê paz, se é que há paz neste universo. Eram eles:
Donovan, Guerrera e Ångstrom. Parker escorregou na direção do barco,
enquanto os outros três mergulhavam desesperados pelas visões infinitas
de rocha incrustada de verde, e Johansen jura que ele foi engolido por
um ângulo na estrutura que não deveria estar ali; um ângulo agudo, mas
que se comportava como obtuso. Portanto, apenas Briden e Johansen
alcançaram o barco, e se apressaram em direção ao \emph{Alert} enquanto
a montanhosa monstruosidade defluía pelas pedras limosas e hesitava
debatendo-se à margem das águas.

O vapor não havia se apagado por completo, a despeito de terem ido todos
para a praia; e pôr o \emph{Alert} em curso custou o trabalho de alguns
momentos de pressa febril, de um lado para o outro, entre o leme e as
máquinas. Lentamente, em meio aos horrores distorcidos daquela cena
indescritível, começou a sacudir as águas letais; enquanto, na estrutura
daquela praia bestial que não era desta terra, a Coisa titânica vinda
das estrelas babava e bramia como Polifemo amaldiçoando o barco de
Odisseu em fuga. Então, mais ousado do que o famoso ciclope, o grande
Cthulhu deslizou untuoso para dentro d'água e se pôs a persegui-los
erguendo as ondas com vastas braçadas de potência cósmica. Briden olhou
para trás e ficou louco, gargalhando estridente, e seguiu gargalhando a
intervalos até que a morte o encontrou uma noite na cabine, enquanto
Johansen perambulava em delírio.

Mas Johansen não havia definhado, ainda. Sabendo que a Coisa certamente
apanharia o \emph{Alert} antes que o vapor estivesse no máximo, tomou
uma resolução desesperada; e, ligando as máquinas em velocidade máxima,
saiu como um raio pelo convés e inverteu a roda do leme. Formou-se um
poderoso turbilhão espumante naquela salmoura perniciosa, e enquanto o
vapor fumava cada vez mais alto, o bravo norueguês levou sua nave a um
choque frontal contra o predador gelatinoso que se erguia acima da
escuma impura como a popa de um galeão demoníaco. A medonha cabeça de
molusco com seus tentáculos contorcidos se agigantou diante do gurupés
do barco robusto, mas Johansen seguiu implacável. Houve um estouro como
o de uma bexiga que explodisse, uma repulsiva limosidade de peixe-lua
fendido, um fedor como o de mil sepulcros abertos, e um som que o
cronista não ousaria pôr no papel. Por um instante o navio foi aviltado
com uma nuvem verde, acre e cegante, e depois apenas um ebulir venenoso
atrás da popa; onde --- Deus nas alturas! --- a dispersa plasticidade
daquela inominável cria celeste enevoadamente se \emph{recombinava} em
sua odiosa forma original, ao passo que sua distância se ampliava a cada
segundo, pois o \emph{Alert} ganhava o ímpeto de seu vapor crescente.

Isso foi tudo. Depois Johansen apenas cismava a respeito do ídolo na
cabine e cuidava das questões de alimentação para si e para o
gargalhante maníaco a seu lado. Ele não tentou navegar após a fuga
ousada, pois aquela reação subtraíra algo de sua alma. Chegou então a
tempestade de 2 de abril, e as nuvens se fecharam sobre sua consciência:
uma sensação de vórtice espectral pelos líquidos golfos do infinito, de
viagens vertiginosas através de universos oscilantes na cauda de um
cometa, e de mergulhos histéricos a partir das profundezas da lua e da
lua novamente às profundezas, tudo abalado por um coro escarnecedor de
deuses ancestrais distorcidos e burlescos, e de maliciosos diabretes do
Tártaro, verdes e com asas de morcego.

Em meio a esse sonho chegou o resgate
--- o \emph{Vigilant}, e a corte
do vice-Almirantado, as ruas de Dunedin, e a longa viagem de volta para
a velha casa junto do Egberg. Ele não poderia dizer nada --- pensariam
que enlouqueceu. Escreveria sobre o que havia conhecido antes que viesse
a morte, mas sua esposa não poderia suspeitar. A morte seria uma bênção
se pudesse ao menos apagar todas as suas memórias.

Esse foi o documento que li, e que pus agora na caixa de latão ao lado
do baixo-relevo e dos papéis do professor Angell. Com ela irá este meu
registro --- o teste de minha própria sanidade, onde se acha recomposto
o que espero jamais se recomponha de novo. Encarei tudo aquilo que o
universo guarda de horror, e mesmo os céus da primavera e as flores do
verão nada mais podem ser que veneno para mim. Mas não creio que minha
vida seja longa. Assim como meu tio se foi, como o pobre Johansen se
foi, também eu irei. Eu sei demais, e o culto ainda vive.

Cthulhu ainda vive também, eu suponho, novamente naquele precipício de
pedra que o escudou desde que o sol era jovem. Sua maldita cidade está
outra vez nas profundezas, pois o \emph{Vigilant} navegou sobre o local
após a tempestade de abril; mas seus ministros em terra ainda uivam e
saltam e matam à volta de monolitos encabeçados por ídolos em lugares
desertos. Deve ter ficado preso nos escolhos do naufrágio quando no
fundo de seu abismo negro, ou o mundo já agora gritaria de terror e
tremor. Quem sabe qual será o final? O que se ergueu deve afundar, e o
que afundou ainda voltará a se erguer. A abominação aguarda e sonha nas
profundezas, e a ruína se espalha pelas instáveis cidades dos homens.
Virá a hora --- mas não devo e não posso pensar! Farei uma prece para
que, se não sobreviver a este manuscrito, meus executores testamentários
ponham a cautela antes da audácia e que isto não chegue a ainda outros
olhos.
