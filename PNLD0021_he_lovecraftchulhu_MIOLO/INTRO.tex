\chapterspecial{Introdução}{De homens e monstros}{Dirceu Villa}
\hedramarkboth{Introdução}{}

Nessa antologia estão reunidos dois dos mais representativos contos do escritor estadunidense H.\,P.\,Lovecraft: ``O chamado de Cthulhu'' e ``A cor que caiu do espaço''.
No primeiro somos introduzidos no universo dos Grandes Antigos, personagens incontornáveis da mitologia que o escritor criou. Na segunda obra aproximamo"-nos de uma temática cara à ficção científica: a invasão alienígena.

Além de serem duas das melhores obras de ficção científica já escritas em todos os tempos, tornando"-se clássicos da literatura de terror, esses contos apresentam em comum certa exploração da natureza e dos limites humanos que beiram a especulação filosófica.
Nos dois encontramos figuras advindas de outras dimensões, que se tornam a expressão máxima do horror justamente por serem inapreensíveis pela mente e pela percepção humanas.
Ao explorar temas típicos da ficção, como alienígenas e monstros das profundezas do mar, Lovecraft soube imprimir"-lhes traços peculiares e característicos de seu universo, que o consagraram como um dos maiores escritores do século \textsc{xx}.
Após uma breve exposição do contexto histórico em que sua obra foi produzida, passamos por uma definição de fantasia para, enfim, discutir mais a fundo a construção dessas duas obras"-primas.  

\section*{Um mundo estranho}

Os \versal{EUA} surgiriam no ano da independência de 1776, mesmo ano, aliás
curiosamente, de surgimento da incompreensível sociedade secreta dos
\emph{Illuminati} --- criada pelo advogado bávaro Adam Weishaupt
(1748--1830) ---, com a qual compartilham laços maçons,\footnote{E os
  lemas \emph{Annuit coeptis}, o ``Tem aprovado'', e o \emph{Novus ordo
  seclorum}, ``Nova ordem dos séculos'', em que a palavra ``ordem'' não
  deve passar despercebida. A maçonaria também desempenhou um papel
  fundamental na Revolução Francesa, porque os pedreiros"-livres tinham
  de ser ocultos, dado que o \emph{Ancien Régime} não via com bons olhos
  o sistema burguês"-industrial crescente. E como escreve Walter Isaacson
  sobre um dos mais importantes agentes da independência estadunidense
  (e da Constituição do país), Benjamin Franklin: ``Franklin se tornou
  um maçom fiel. Em 1732 ajudou a esboçar o regimento interno da loja da
  Filadélfia, tornando"-se Grão"-Mestre dois anos mais tarde e imprimindo
  sua constituição'', \emph{in}: \textsc{isaacson}, Walter. \emph{Benjamin
  Franklin: An American Life}. New York: Simon \& Schuster, 2003, p.\,106.}
como vemos no símbolo maçônico da pirâmide com o olho que tudo vê,
impresso no dólar, dinheiro estadunidense, símbolo do poder na
sociedade; mas precisamente um \emph{novo tipo} de poder, o que faz os
bastidores assumirem o centro de irradiação, uma vez que a democracia se
convenciona e passa a ser usada apenas como negociação com as massas,
com as decisões declaradamente se fazendo a portas fechadas, longe dos
olhos de quem poderia não compreender a frieza do que chamam estratégia,
ou de quem poderia se horrorizar com certos métodos decisórios.

Os \versal{EUA} do \versal{XIX} são portanto os \versal{EUA} do esforço de se criar uma nação não
apenas autônoma, mas nova sob todos os aspectos, uma nação que
combinasse ciência (não em abstrato, mas como \emph{aplicação técnica}),
um modelo de democracia,\footnote{Democracia já foi muitas coisas
  diferentes --- algumas bem pouco democráticas --- desde seu nascimento
  registrado, em Atenas. Algumas de suas versões aceitavam escravidão; a
  maior parte de seus modelos excluía as mulheres, por exemplo.} um
zelo pelo dinheiro segundo a ética protestante weberiana,\footnote{A
  ética puritana, originada no calvinismo da predeterminação dos
  destinos e da moralidade individual, propõe uma vida de trabalho
  persistente e de circunspecta obediência aos deveres. Um dos deveres
  fundamentais é com o dinheiro, porque os deveres são todos do mundo
  temporal, como o é trabalhar. Não ter dívidas, ser de caráter aplicado
  e sistemático, profissional racional da divisão do trabalho e
  empenhado no \emph{lucro}, visto não apenas como um bem da
  coletividade, mas também como um \emph{dever individual} daquele a
  quem Deus aponta esse caminho, estabelecem a famosa definição do
  economista e jurista alemão Max Weber (1864--1920) em sua obra
  fundamental \emph{A ética protestante e o espírito do capitalismo}
  (1904--1905).} e, sobretudo, essa estranha relação, de sombras, com o
poder. O ponto fundamental é que se foi criando uma sociedade --- desde
a divisão do conhecimento e do trabalho, o uso das máquinas da Revolução
Industrial, a administração do crédito pessoal por bancos e a categoria
de administração da \emph{res publica} por políticos que se distanciam
de modo progressivo da esfera efetivamente pública das ações --- em que
o indivíduo é afastado dos meios da sua existência, e da existência
comum com os outros, por camadas e camadas que efetuam, \emph{sem que
ele saiba quais e para quê}, suas decisões, uma variação moderna do
afastamento exercido pela antiga sociedade estamental.

As sombras são o ponto psicossocial dessa literatura de horror (até
freudiano, como se verá), pois há o poder publicamente proposto, e há o
poder \emph{de fato}; se sempre foi assim,\footnote{Basta lembrarmos de
  uma das frases famosas de Elizabeth Tudor (1533--1603), a rainha
  Elizabeth \versal{I}: \emph{We princes, I tell you, are set on stages}, ``Nós,
  príncipes, somos postos em palcos'', cristalizando a noção, de resto
  elementar, de que a política monta uma grande fanfarra publicamente e
  faz o \emph{sinal de Harpócrates} para o que urde, aí de fato, nos
  bastidores.} amplifica"-se cada vez mais a oposição entre
\emph{aparência pública} e \emph{prática reservada}, sobretudo porque é
na última que se decidem os rumos político"-econômico"-sociais e até mesmo
culturais. O curto"-circuito não é percebido de modo geral pela população
--- que costuma obedecer à ordem vigente sem muito ruído ---, mas é
nesse escuríssimo armário de esqueletos da sociedade onde vive a
imaginação daqueles dois estadunidenses, Poe e Lovecraft. Lovecraft em
particular: seu fascínio pelo distúrbio da consciência, pela ciência
ficcional frankensteiniana, pela interferência alienígena, pela agressão
simbólica sobre a psique e pela insegurança existencial generalizada são
pontos de intensa vibração de uma \emph{angst} que não apenas não
envelheceu como está mais viva do que nunca.

O cerne da literatura de Lovecraft está no fato de que \emph{desconfia},
que intui que as maiores forças deste mundo operam nas sombras:\footnote{Para
  citar de passagem um número de exemplos na obra de Lovecraft que o
  confirmam: os ratos estão dentro das paredes, as abominações
  pré"-humanas vivem sob o oceano, a própria morte parece apenas um
  segredo ocultado que é preciso sacar da escuridão, a música inumana
  vem de um terrível diálogo com as sombras, etc.} sua prática é a de
notar o aberrante, mesmo de exagerá"-lo para efeito educativo (e que,
como deformação extrema, parte da alegoria e tende à caricatura). Quem
pensa que o faz pelo motivo trivial de tentar assustar seu público, ou
porque sua psicologia literária, como a de Poe, fosse imatura, se engana
--- sobre os dois. Não se trata de fantasia por irrealidade ou
imaturidade, mas de uma fina percepção adoentada por uma sociedade que
castiga essas percepções. Se ambos surgissem na Grécia de por volta do
século \versal{V} a.C. é bem provável que tivessem escrito tragédias para o
apreço do professor Aristóteles, ou que, surgindo no meio do século \versal{XVI}
na Inglaterra escrevessem peças macabras e sanguinolentas à espanhola,
como Thomas Kyd fez. Nascendo onde e quando nasceram, escreveram o que
se chama, sempre com algum desdém criticamente supercilioso,
\emph{literatura fantástica}.


\section*{Literatura fantástica: contra\break a definição desdenhosa}

\emph{Fantástico}, em sua origem etimológica, é, nos diz Aristóteles,
\emph{phos}, ou ``luz'', em grego.\footnote{Sobre esse assunto em
  particular, ler o importante artigo de Jorge Sallum, ``Phantasía e
  retórica estóica'', publicado na revista digital \emph{Germina
  Literatura} em março de 2008.} As antigas teorias sobre
\emph{phantasía}\footnote{\textsc{garin}, Eugenio. ``\emph{Phantasia} e
  \emph{imaginatio} fra Ficino e Pomponazzi'', \emph{in}:
  \emph{Phantasia"-Imaginatio}, atti del \versal{V} Colloquio internazionale del
  Lessico Intellettuale Europeo, Roma 9--11 gennaio 1986, a cura di M.
  Fattori, M.\,L.\,Bianchi, Roma, Edizioni dell'Ateneo, 1988.} vinham do
campo filosófico estóico que discutia as bases do que hoje chamamos, sem
filosofia alguma, \emph{realidade}: em grego distinguia"-se entre
\emph{phantásmata} (o que a mente encontra no mundo) e
\emph{phantastikón} (o que a mente concebe por si, sem a necessidade de
achar aquilo no mundo),\footnote{De modo que \emph{fantasma} e
  \emph{fantástico} têm origem etimológica comum, o que não é casual nem
  mesmo se se quiser pontuar tudo o que, depois, separou semanticamente
  as duas palavras.} duas palavras importantes para o propósito
filosófico e seríssimo de explicar realidade, imaginação e sonho. Hoje,
quando o máximo que se lê é pouco mais do que uma centena de dígitos,
filosofia obviamente não é possível; quando era, os estóicos surgiram
com uma sagacidade notável, propondo que deus, como \emph{logos}, está
em tudo, e, estando em tudo, essa fagulha criadora ativa no
\emph{phantastikón} uma concreção do campo imaginativo, uma condensação
das quase inapreensíveis \emph{lógoi spermatikói}, ou ideias seminais,
aquilo que, seguindo Platão, chamaríamos o \emph{mundo das ideias,} e
que Carl Gustav Jung bem mais tarde apropriaria sob o nome de
\emph{inconsciente coletivo}.\footnote{Há também uma conexão com a
  antropologia difusionista --- o conceito de \emph{Kulturkreise}, ou
  ``círculos de cultura'', de Leo Frobenius, no começo do século \versal{XX}.}

Fantástico, nesse sentido, serviu também, na Antiguidade, para a
circunscrição dos sonhos premonitórios na \emph{Onirocrítica} --- não
foi Sigmund Freud quem inventou a interpretação dos sonhos ---, como
definiu Artemidoro de Dáldis (século \versal{II} d.C.): sonhos com a força de
intuir o futuro;\footnote{Ou, nas palavras de Artemidoro: ``ficção
  multiforme da alma que anuncia os bens e os males futuros'',
  \emph{in}: \textsc{daldis}, Artemidoro de. \emph{El libro de la interpretación
  de los sueños}. Edición de M.\,Carmen Barrigón Fuentes y Jesús M.\,Nieto
  Ibañez. Madrid: Ediciones Akal, 1999, p.\,70. Os sonhos premonitórios
  desempenham um papel particularmente interessante em ``The Call of
  Cthulhu'', sobretudo se os virmos à luz da antiga onirocrítica.} ou
de rasgar uma delicada fenda dimensional, espaciotemporal, como
imaginaríamos depois da Relatividade de Einstein, e também com a parte
da ficção"-científica atual que escreve sobre o papel do Grande Colisor
de Hádrons, o gigantesco túnel para colisão atômica metido num buraco da
fronteira entre França e Suíça.

O poder de plasmar a realidade não foi aludido apenas pelos antigos ou
por descendentes do poderoso conhecimento oculto no Renascimento, como a
noção de \emph{disegno interno} faz ver na escultura italiana:\footnote{Como
  explica Federigo Zuccaro (1540--1609) em seu tratado \emph{L'idea dei
  pittori, scultori ed architetti}, o \emph{disegno interno} (1607) é
  \emph{luce dell'intelletto}, e um \emph{oggetto immateriale}.
  Michelangelo Buonarroti (1475--1564), como outros, assinalava a
  preexistência da figura já na pedra, que ele apenas liberava.} para
todos aqueles que admiram a arte cinematográfica de David
Lynch,\footnote{Todos os filmes de David Lynch, como os de Stanley
  Kubrick, são códigos para mecanismos de imaginação e para a relação
  complexa entre os conceitos de mente, realidade e manipulação
  ritualística, pelo poder, das forças imaginárias.} foi exatamente o
que Lynch fez no oitavo episódio da recentíssima terceira temporada de
\emph{Twin Peaks} (2017). Assistindo ao nascimento de uma nova categoria
de horror humano na explosão da primeira bomba atômica no Novo México,
em 1945, a Señorita Dido e o Bombeiro --- seres metafísicos em seu
não lugar metafísico que lembra um cinema --- geram a partir de intensa
luz uma esfera dourada, então lançada na direção da Terra por um
aparelho cujo funcionamento, aliás, não é diferente do da máquina de
metafísica amorosa inventada por Marcel Duchamp em seu \emph{Grand
Verre}, ou \emph{La mariée mise à nu par ses célibataires, même}
(1915--1923).

Apesar do desdém da crítica (que vem sendo revertido lentamente nos
últimos anos), o fantástico na literatura pode fazer recuar suas
nobilíssimas raízes até as fábulas gregas de Esopo,\footnote{Falo apenas
  das raízes ocidentais. Naturalmente, o Oriente está também repleto
  dessas raízes do fantástico.} que punham os animais a falar; aos
espetaculares feitos de transformação nas \emph{Metamorfoses} de Ovídio
(que Augusto de Campos já emparelhou à narrativa cinematográfica
muitíssimo \emph{avant la lettre}),\footnote{``Metamorfoses das
  Metamorfoses'', \emph{in}: \textsc{campos}, Augusto de. \emph{Verso Reverso
  Controverso}. São Paulo: Perspectiva, p.\,191.} depois emulados por
Dante na \emph{Commedia},\footnote{Em particular no Canto \versal{XXV} do Inferno,
  quando Dante proclama sua vitória sobre as metamorfoses ovidianas por
  ter concebido a transformação de ladrões em répteis num processo que
  jamais se conclui em qualquer uma das duas formas cambiantes.} com
seus espantosos monstros teológicos descritos em requintes gráficos; ou
às muitas viagens à Lua, de Luciano de Samósata a Cyrano de Bérgerac (o
autor, não o personagem derivado dele por Rostand); mas penso que um
ponto de inflexão se encontra no poema de John Milton (1608--1674),
\emph{Paradise Lost} (Paraíso perdido, 1667).

Lá, no verso 13 do Livro \versal{I}, Milton define seu longo poema como
\emph{adventurous song}, ou ``canção aventuresca''. Não é uma definição
casual e sem importância: imitando tanto a épica greco"-latina como o
poema teológico de Dante, seu poema nasce confuso e sem gênero claro.
\emph{Sem gênero}, até o século \versal{XVIII}, significa uma anomalia em que as
onipresentes regras retóricas de composição patinam, e é nesse patinar
que o poema de Milton, não sendo épico nem teológico, se torna
aventuroso, ou fantástico.\footnote{Os ferimentos das armas comuns de
  anjo contra anjo numa batalha se fecham com a velocidade que
  aprendemos a ver em um dos heróis mutantes da Marvel Comics, o
  \emph{X"-Men} Wolverine. Daí que Deus presenteie Miguel com uma espada
  especial, para matar anjos, que rasga a pele de Lúcifer, para terror
  deste: \emph{Then Satan first knew pain} (``Então Satã enfim conheceu
  a dor'', Livro \versal{VI}, v.\,326), com a ferida aberta de onde manava um
  ``humor nectáreo'', fluido sanguíneo, que é, anota Milton com zelo
  etnográfico, o que ``espíritos celestiais'' devem sangrar.}

O que era, em Homero ou Virgílio, ação demonstrável de virtudes civis,
claramente calcadas em aspectos mitológicos ordenados para essa
exemplaridade civilizacional, se torna ação aventuresca em Milton; o que
era teologia em Dante, estipulando uma \emph{ordo} (ordem) baseada no
conceito de \emph{rectitudo} (retidão), catolicamente definido, e onde o
mundo dito pagão se torna alegoria de perversidade antinatural (Cérbero,
o cão de três cabeças, fora da natureza; o Minotauro, aberração de duas
origens diversas; os Centauros, emenda de homem e cavalo) entendida como
a multiplicidade falsa e mortal contra a verdade eterna e imutável de
Deus, torna"-se em Milton a disputa muito dúbia de Lúcifer com o próprio
Deus, na qual episódios como a entrada secreta da Serpente no Paraíso,
ou o combate que Lúcifer dá aos anjos com canhões de fabricação
infernal, nada têm da gravidade dos gêneros antigos, mas propõem pela
primeira vez o que se leria como a ligeireza do registro fantasioso, que
não é efetivamente crível em nenhum nível de apreensão do
texto,\footnote{Isto é, depende daquela operação definida por Coleridge
  como a \emph{suspension of disbelief}, ou a ``suspensão da
  descrença''. Nas \emph{Lyrical Ballads} que compôs com Wordsworth, a
  parte de escrever sobre ``personagens sobrenaturais'' lhe havia
  cabido, mas não de qualquer jeito, ``de modo a transferir de nossa
  natureza íntima um interesse humano e uma semelhança de verdade
  suficiente para reclamar a essas sombras da imaginação aquela
  voluntária suspensão da descrença por um momento, que constitui a fé
  poética''. \emph{Biographia Literaria}, Chap. \versal{XIV}, \emph{in}:
  \textsc{coleridge}, Samuel Taylor. \emph{The Major Works}. Edited with an
  introduction and notes by H.\,J.\,Jackson. Oxford: Oxford University
  Press, 2008, p.\,314.} autonomizando a imaginação do irreal dentro de
uma aventura \emph{jamais dita em prosa ou rima}.

Como se percebe pela minha última frase acima, isso já havia ocorrido
antes de Milton --- e terá influído em sua própria escrita --- no
\emph{Orlando furioso} (1516--1532) de Ludovico Ariosto, poema em que
pessoas voam nas costas de um grifo, em que as espadas de heróis matam
magicamente centenas de inimigos, etc. Ariosto dá sentido ao conceito
posterior de \emph{fantasia}, como de \emph{aventura fantástica}, e que
antes já compareciam mesmo na invocação do poema de farsa épica, o
\emph{Baldo} (1517) de Teofilo Folengo (1491--1544): \emph{Phantasia mihi
plus quam phantastica venit (\ldots{}) cantare}, ou ``Minha fantasia, bem
mais que fantástica, vem (\ldots{}) cantar'', no estilo que ficou conhecido,
já da proposição mesma de Folengo, como \emph{macarrônico}, no molho
espesso de latim e italiano misturados.

A proposta mesma de Ariosto era a de contar os feitos de fúria de
Orlando, ``fúria'', ali, entendida como em sua raiz latina,
\emph{furor}, ``loucura''. As ligações da \emph{loucura} com o
\emph{sonho} (e a poesia) são as que lemos em todo o passado da arte
ocidental: estão por exemplo nos autos do processo inquisitorial contra
o pintor Paolo Veronese (1528--1588), que se defendeu da acusação de
heresia pronunciando sua inocência por fazer entender à Inquisição
veneziana que ``nós, pintores, utilizamos a mesma licença que poetas e
loucos''; estão, também, em \emph{Midsummer Night's Dream} (Sonho de uma
noite de verão, 1596), peça de sonho de William Shakespeare (1564--1616),
e em lugares filosóficos da literatura numerosos demais para se listar
aqui, e todos ainda sem demarcação de limite estilístico, que viria
depois.

Por isso é importante relembrar ao menos o poema da ``Rima do velho
marinheiro'', de Samuel Taylor Coleridge (1772--1834), que Lovecraft leu
ainda criança: um conviva numa festa de casamento é abordado pelo velho
marinheiro do título, homem condenado a narrar para sempre o infortúnio
seu e de sua equipagem, porque durante a viagem ele transgrediu o tabu
de matar um albatroz, desencadeando, assim, a maldição sobrenatural dos
mares sobre seu navio.

As visões daqueles seres flutuantes e angélicos, a tripulação morta"-viva
e fantasma que se ergue para completar a viagem fatal, os brilhos na
escura água do oceano e os encantos do horror, além da música mágica dos
versos, têm muito da nova exploração de campos imaginativos, e uma fonte
erudita: as discussões antigas para a delimitação entre \emph{phantasía}
e \emph{imaginatio}, fundamentais para aqueles que por exemplo viveram o
que chamamos o \emph{Renascimento}, pois, platônicos como eram, aquelas
duas coisas precisavam ser compreendidas e controladas para produzir
efeitos saudáveis, pois entendiam, com seu mestre grego, o \emph{terror
sagrado} da imaginação, que afetava a realidade de modo direto; quero
dizer, como acima o disse, o efeito de \emph{plasmar} a realidade (a
velha teoria estóica do \emph{pneuma}). Tinham receio: queriam conhecer
para usar com propriedade esse poder.

Não tinham esse receio os homens do fim do \versal{XVIII} e de todo o \versal{XIX} (a
partir do \versal{XX} ninguém mais se preocupa com o assunto, e quer apenas
dominar seu público com uma imaginação poderosa que o submeta): para
eles havia chegado a hora de abrir essa caixa de Pandora, porque toda
``superstição'' havia sido vencida no século das Luzes. Os chamados
subgêneros de \emph{ficção científica}, \emph{horror}, \emph{fantasia
medieval}, e mesmo a \emph{literatura policial} são produtos de meados
do século \versal{XIX} e começo do \versal{XX}:\footnote{Alguns dos nomes de pioneiros
  fundamentais nesses subgêneros (alguns, contemporâneos de Lovecraft):
  Jules Verne (1828--1905), Arthur Conan Doyle (1859--1930), H.\,G.\,Wells
  (1866--1946), Agatha Christie (1890--1976).} nesses casos, o subgênero
passa a criar um grupo de leitores de gosto específico, interessado
naquelas narrativas imaginativas em parte como mero \emph{escapismo} do
dia a dia massacrante do capitalismo industrial, em parte como
\emph{estímulo perceptivo"-intelectual}, e nesse caso há mesmo a
designação de \emph{ficção especulativa} para a fantasia, o novo lar da
composição alegórica após a crise da retórica antiga como produtora de
práticas letradas.

A alegoria serve à ficção científica para propor \emph{futuros
avançados} que espelhem a \emph{sombra do presente}, e situar uma
hipótese extrema como crítica dos modos de vida, condicionados já desde
a Revolução Industrial por uma desumanizada \emph{técnica};\footnote{Basta
  citar alguns nomes e livros para se constatar os mecanismos: Ray
  Bradbury com \emph{Fahrenheit 451}, Arthur C.\,Clarke com \emph{2001: A
  Space Odyssey} (em colaboração com Stanley Kubrick), Philip K.\,Dick
  com \emph{Do Androids Dream of Electric Sheep?} (romance de onde veio
  o filme \emph{Blade Runner}, de Ridley Scott, em 1982), entre outros,
  que também se desmembra na chamada \emph{ficção distópica}, como
  achamos em \emph{Brave New World}, de Aldous Huxley, \emph{1984}, de
  George Orwell, \emph{The Lathe of Heaven}, de Ursula Le Guin e
  \emph{The Handmaid's Tale}, de Margaret Atwood, entre muitos outros e
  outras.} serve ao horror para hiperdimensionar com exemplaridade os
disparates da vida social, para trazer ao palco a camada escura e
escondida das pulsões humanas; serve à literatura policial para mostrar
que a leitura do mundo, nos predicados do que é a vida, só se pode fazer
aplicando"-lhe argúcia e malícia; e serviu à fantasia dita
\emph{medieval} para reconstituir a narrativa civilizacional da épica,
só que agora também com as tintas emprestadas do romance de formação, o
\emph{Bildungsroman}, no qual um ou mais personagens enfrentam a jornada
de transformação na vida, em geral de uma doce alienação juvenil para o
peso do conhecimento do mundo na maturidade, como se vê particularmente
na jornada de Frodo em \emph{The Lord of the Rings} (1954--1955), de J.\,R.\,R.\,Tolkien (1892--1973).\footnote{É o que se pode ver também mais
  recentemente na série ainda incompleta de livros \emph{A Song of Ice
  and Fire} (1996--), de George R.\,R.\,Martin (1948), que logo se tornou
  a série televisiva \emph{A Game of Thrones}. A aproximação permanece
  válida mesmo que guardadas as diferenças grandes entre os
  empreendimentos de escrita de Tolkien e Martin. No caso de Tolkien, a
  experiência na guerra também constitui um forte condicionamento para a
  narrativa.}

Mas o tipo de fantasia de Lovecraft, ainda que se instale de modo mais
genérico dentro do horror, trouxe para o horror desenvolvimentos
inesperados, que depois influíram em toda a cultura de massas,\footnote{Já
  expus acima parte dessa maciça influência, mas podemos pensar também
  em um filme como \emph{Alien} (1979), de Ridley Scott, e, no cinema,
  seria ainda necessário destacar o diretor mexicano Guillermo del Toro
  (1964), uma vez que é provável não haver maior entusiasta de Lovecraft
  entre vencedores do Oscar; há John Carpenter (1948), especialmente por
  sua refilmagem de \emph{The Thing} (1982) com as deformidades de seu
  transmorfo alienígena, e há também, entre muitos outros, Clive Barker
  (1952), em sua obra escrita e cinematográfica, incluindo a parte de
  sua obra escrita em um domínio com larga influência lovecraftiana, as
  \versal{HQ}s; por fim, a chamada \emph{cultura nerd} em peso lhe é devota, uma
  cultura que há poucos anos chegou ao \emph{mainstream} da indústria
  cultural.} e se traduzem numa confluência entre horror, ficção
científica e uma exploração psicanalítica das perturbações da percepção.

\section*{A linguagem para o horror}

A linguagem que lemos neste ``O chamado de Cthulhu'', mas também em toda
a obra de Lovecraft, é rigorosamente característica de seu autor; a
respeito disso, não obstante, resta observar que, dada a gigantesca
influência que surtiu não apenas em seu gênero de escrita, mas em
inúmeras manifestações da indústria de entretenimento, é assim um estilo
incorporado a uma variedade de outras coisas que, na aparência, sequer
teriam a ver uma com a outra. A influência de sua linguagem é maciça, e
esse é ainda outro motivo para considerá"-la em específico por um
momento.

Lovecraft é por vezes chamado \emph{erudito}: ele não é. Seu
conhecimento é errático e para nas informações essenciais (ou mesmo nas
ideias pré"-concebidas) em vários aspectos.\footnote{O vodu, por exemplo,
  utilizado no conto de Cthulhu, vem da impressão muitíssimo
  superficial, caricata, ideológica e preconceituosa que a ocupação dos
  \versal{EUA} no Haiti (de 1915 a 1934) trouxe ao público. Um livro em
  particular, ligeiramente posterior ao conto de Cthulhu, \emph{The
  Magic Island} (1929), do jornalista, explorador e ocultista William
  Seabrook (1884--1945), popularizou uma noção de \emph{zumbi} nos \versal{EUA}
  (esse que mais tarde se tornaria a criatura morta"-viva e canibal de
  George Romero e o monstro mais popular do horror recente) e as noções
  extremas das práticas vodu haitianas, uma vez que Seabrook narra seu
  livro a partir de uma sensacionalista experiência pessoal. Filmes como
  \emph{White Zombie} (1932), de Victor Halperin, foram baseados no
  livro de Seabrook.} Há em sua escrita um uso abusivo de adjetivos e
advérbios com a tentativa de hiperbolizar ou intensificar a descrição;
por vezes, esses usos montam um sobre o outro, entupindo o texto. Não se
trata de um autor lutando para dizer o indizível, como se poderia
pensar, porque mesmo onde não está em jogo o indizível seu estilo é
redundante, sentencioso, repleto de advérbios sonoramente semelhantes e
de dupla adjetivação frequente:\footnote{Encontra"-se esse tipo de escrita
  adjetival e hiperbólica em numerosos exemplos. Ofereço um da minha
  tradução: ``Fúria animal e licenciosidade orgíaca aqui se misturavam,
  em alturas demoníacas, com êxtases de uivos e guinchados que irrompiam
  e reverberavam pelos bosques noturnos como tempestades pestilentes,
  vindas dos abismos do inferno''; sequências de advérbios se acumulam
  em menos de uma página, como \emph{hurriedly}, \emph{curiously},
  \emph{virtually}, \emph{really}, \emph{especially}, \emph{deeply},
  \emph{apparently}, \emph{wholly}, todos tomados do meio para o fim da
  página 183 da edição dos \emph{Tales}, da Library of America (2005),
  em posições por vezes tão próximas que fazem eco um ao outro. Há
  muitos mais no texto completo, e deixo à discrição da leitora e do
  leitor percebê"-lo.} por não encontrar o recurso exato, Lovecraft
acumula, tentando dobrar a percepção de seus leitores com a hipérbole.

Busca um tipo de grandiloquência de tipo grave, na qual por vezes
derrapa, mas seu talento inequívoco de linguagem se mostra em muitos
lugares: por exemplo, sempre que relata aspectos científicos, ou com
linguagem de demarcação específica em seu uso, que é quando Lovecraft se
torna minucioso e objetivo (quase maquinal) na anotação do jargão
utilizado, sejam manchetes de jornal, relatos de sociedades
antropológicas, registros de equipagem de navio, ou linguagem de boletim
policial.

Pedante, fascina por instituir em parte esse tipo de linguagem que se
juntaria ao patrimônio vocabular da narrativa gótica e de ficção
científica, e, em parte, por aderir aos estilemas mais exagerados,
superficiais e abundantes do gênero do horror, que em muitos aspectos
repetem e muitas vezes diluem procedimentos acháveis em Mary Shelley,
Edgar Allan Poe e Bram Stoker. Devemos também lembrar que a educação de
Lovecraft foi até meados do segundo grau (alega ter ido à Universidade
Brown, mas não se formou), e isso parece ter sido sentido como uma
mácula e uma vergonha, compensadas por um grande orgulho de seu
intelecto, demonstrado ostensivamente em sua escrita ficcional.

Seu intelecto era claramente amplo, sua curiosidade, onívora, mas as
lacunas de uma formação incompleta que cedeu, em algum ponto, ao mero
exercício repetitivo do gosto lhe deixou vícios notáveis de estilo, e
ideias fixas. São esses vícios e essas ideias fixas, no entanto, que o
fizeram notório como escritor e o particularizaram em uma vasta
quantidade de praticantes dos gêneros populares da escrita fantástica de
seu período. Assim como o antecessor e modelo Poe, o orgulho, os medos,
as frustrações e os ódios lhe foram combustível inventivo, e lhe deram
nervura estilística particular, especialmente para criar monstros
inesquecíveis, como o deste conto.

\section*{O Grande Cthulhu}

\emph{Cthulhu} é a fantasmagoria mais popular do muito popular
Lovecraft, e o centro de sua série sobre os Grandes Antigos, as
gigantescas e incompreensíveis criaturas anteriores a esta Terra, tal
como se a conhece; alguns chamarão ``criação de um mito'', ou de uma
``mitologia'', mas é sobretudo impreciso e hiperbólico dizê"-lo: um mito
não se cria, um mito eclode no tecido da realidade por processos
sedimentares dentro de uma cultura, e toma séculos de desenvolvimento
coletivo e anônimo, como leitura daquela mesma realidade.\footnote{Joseph
  Campbell, especialista em mitologia, diria, por exemplo, no início de
  sua entrevista sobre o tópico do mito com Bill Moyers: ``Esses pedaços
  de informação dos tempos antigos, que têm a ver com temas que basearam
  a vida humana, construíram civilizações e deram forma a religiões por
  milênios, têm a ver com problemas interiores, mistérios interiores,
  umbrais internos de passagem (\ldots{})''. \textsc{campbell}, Joseph (with Bill
  Moyers). \emph{The Power of Myth}. New York: Anchor Books, 1991, p.\,2.}

Os Grandes Antigos derivam dos compêndios ancestrais de demonologia, e
grimórios, onde frequentemente se anotam informações detalhadíssimas
sobre aquelas criaturas, como se se tratasse de registro legal em
cartório: características físicas, manifestações propícias, elementos
associados, nomes, ação, sinais identificáveis e até mesmo assinaturas.
O mais famoso --- e penso mesmo que até mais acessível de todos esses,
por não codificar nenhuma doutrina --- é o \emph{Dictionnaire Infernale}
(primeira edição em 1818), de Jacques Albin Simon Collin de Plancy
(1793--1881), ocultista francês.

Compêndio montado com substanciosos verbetes de gosto antropológico na
escrita, o \emph{Dictionnaire Infernale} de Plancy reúne um repertório
enorme de criaturas em diversas culturas (\emph{répertoire universel},
diz o subtítulo), com ilustrações que são quase icônicas na imaginação
coletiva, de tanto que foram reproduzidas. A galeria dos Grandes Antigos
claramente se inspira naquela meticulosa coleção, também pela forma
proposta aos demônios na Antiguidade, que --- como disse acima ao
mencionar Dante e suas aberrações infernais --- os constituía pela
mistura antinatural de formas e espécies.

A invenção de Cthulhu é algo central, e não apenas para Lovecraft: de
suas criaturas e ambientes de sonho --- ou talvez melhor \emph{pesadelo}
--- é a mais famosa, e figura junto com o próprio Lovecraft em inúmeras
imagens de divulgação, como sua marca registrada. Vazou para a cultura
pop, e há desde estatuetas fidedignas à descrição no conto até
pequeninos e doces bichos de pelúcia estampados com seu nome. Alguns até
ouviriam mentalmente, de imediato, as memoráveis primeiras notas de
``The Call of Ktulu'', do álbum \emph{Ride the Lightning} (1984), do
Metallica, ao escutar o nome da criatura.\footnote{Dentre os estilos
  musicais, o metal é obviamente o meio mais lovecraftiano. O Iron
  Maiden, por exemplo, estampou os célebres versos do árabe insano de
  Lovecraft na pedra tumular de sua mascote, Eddie, na capa do álbum
  gravado ao vivo, \emph{Live After Death} (1985).}

E a criatura foi um modo para Lovecraft cristalizar numa imagem um tipo
específico de terror, que em geral se chama ``cósmico'', mas um cósmico,
penso, muito íntimo e literário; do literário veremos agora, do íntimo,
adiante. Em Cthulhu estão infusas as influências monstruosas daqueles
compêndios e grimórios, mas também um poema em particular. Se já havia
assinalado que o poema de Coleridge era base imaginativa para Lovecraft
de modo geral, um outro de Lord Tennyson (1809--1892) em particular lhe
deu um bom tanto de linguagem grandíloqua e de atmosfera, para nem dizer
aspectos da própria criatura.\footnote{Quem propôs a relação foi o
  Reverendo Robert\,M.\,Price, em seu livro \emph{H.\,P.\,Lovecraft and the
  Cthulhu Mythos}, de 1990.} O poema é ``O Kraken'', que traduzo
abaixo:

\begin{quote}\noindent
Sob os trovões nas altas profundezas,\\
Fundo no profundo abismo do mar,\\
Sem sonho, o sono antigo sem surpresa\\
Dorme o Kraken: a tênue luz solar\\
Dança em seu corpo sombrio, e acima\\
Flutuam esponjas grandes, milenares;\\
Ao longe, a luz doentia anima\\
Polvos gigantescos, que aos milhares\\
Saem de incríveis grutas e cavernas\\
Com longos braços no verde dormente,\\
Que lá esteve e estará por eras;\\
No sono, vermes do mar são sua presa\\
Até que o fogo aqueça as profundezas;\\
E homens e anjos verão, finalmente,\\
Erguida num urro de morte, a fera.\footnote{``\versal{THE KRAKEN}// Below the
  thunders of the upper deep;/ Far, far beneath in the abysmal sea,/ His
  ancient, dreamless, uninvaded sleep / The Kraken sleepeth: faintest
  sunlights flee/ About his shadowy sides: above him swell/ Huge sponges
  of millennial growth and height;/ And far away into the sickly
  light,/ From many a wondrous grot and secret cell/ Unnumber'd and
  enormous polypi/ Winnow with giant arms the slumbering green./ There
  hath he lain for ages, and will lie/ Battening upon huge sea worms in
  his sleep,/ Until the latter fire shall heat the deep;/ Then once by man
  and angels to be seen,/ In roaring he shall rise and on the surface
  die'', \emph{in}: \textsc{hodder}, Karen (Introduction, bibliography and head
  notes).\emph{The Works of Alfred Lord Tennyson}. Ware: Wordsworth
  Editions Ltd., 2008, p.\,46.}
\end{quote}

Nele estão as vertiginosas direções das ``altas profundezas'', o monstro
que dorme no fundo do mar, o verdor de sua pele, os efeitos da
tempestade, a adjetivação que diz ``doentia'', ``sombrio'', as dimensões
inqualificáveis do corpo descomunal, e, naturalmente, o próprio Kraken:
monstro marinho e cefalópode da mitologia escandinava, o polvo gigante
que assombrava as antigas sagas em verso se metamorfoseou algum tanto no
cefalópode antropóide Cthulhu. Lovecraft pôde encontrar, naquele monstro
antigo que remontava, um código de seus próprios horrores, mas que
funcionou especificamente porque o verdadeiro mergulho no medo de um é o
mergulho no medo de todos.

\emph{Watchmen} (1986--1987), a \emph{graphic novel} de Alan Moore e Dave
Gibbons, utilizou um efeito lovecraftiano que sublinha o que dizia
anteriormente sobre a força da fantasia, da concreção do campo
imaginativo e dos códigos dos nossos medos. A dada altura, um monstro
viscoso e tentacular se manifesta destruindo parte de Nova York. O
monstro havia sido o resultado de uma \emph{engenharia}, cuja peça
fundamental fora o cérebro clonado de um humano sensitivo: ``O sensitivo
era a chave (\ldots{}) o cérebro era uma caixa de ressonância psíquica (\ldots{})
Codificamos muitas informações naquele sinal. Informações terríveis. As
descrições de Max Shea de um mundo alienígena, as imagens de Hira Manish
e os sons de Linette Paley\ldots{} Além dos que morreram imediatamente pelo
choque, muitos ficaram loucos pelo súbito dilúvio de sensações
grotescas''.\footnote{\textsc{moore}, Alan; \textsc{gibbons}, Dave. \emph{Watchmen}.
  Burbank: \versal{DC} Comics, 2014 (primeira edição, 1986), pp.\,391--392.}

O texto de Moore sobre o impacto da criatura explodindo sobre a cidade
mostra que a arma alienígena fora composta pela informação extraída de
artistas da \emph{palavra}, das \emph{imagens} e dos \emph{sons},
amplificadas por uma mente sensitiva; precisamente, aliás, como
Lovecraft já propunha, seja em `` O chamado de Cthulhu'' com o jovem
escultor Wilcox, seja em ``A música de Erich Zann'' com a música do
próprio Zann, e, diríamos, com as palavras de Lovecraft \emph{lui"-même}.
Moore, inventando sua criatura cthulhiana, fornece aos leitores atentos
uma definição do complexo imaginativo que os escritores põem a andar
como criaturas entre nós, devolvendo de maneira sintética as impressões
coletivas que recebem do mundo. E é aqui que achamos a oportunidade de
ir às impressões íntimas de Lovecraft, codificadas em sua ficção.

\section*{Um mundo invertido}

\emph{My world is Providence}, afirmou Lovecraft. E poderíamos dizer,
talvez com mais propriedade do que disse de si, que seu mundo era
sobretudo sua mãe. Se de Providence imaginou a vizinha Arkham\footnote{O
  Asilo Arkham, famoso por reunir os criminosos insanos da corte de
  vilões de Batman, da \versal{DC} Comics, vem diretamente de Lovecraft, que o
  inventou em ``The Thing on the Doorstep'' (1933, publicado em
  \emph{Weird Tales} em 1937): ``É verdade que pus seis balas na cabeça
  de meu melhor amigo, mas espero ainda demonstrar com este relato que
  não sou um assassino. De início serei chamado insano --- mais insano
  do que o homem contra quem disparei em sua cela no Sanatório Arkham'',
  \emph{in}: \textsc{lovecraft}, H.\,P. \emph{Tales} (Peter Straub, editor). New
  York: The Library of America, 2005, p.\,692 (tradução minha).} ---
cidade de que até mesmo nos fornece um mapa (e que muitos consideram uma
versão de Salem, em Massachusetts), onde pôs sua inventada Universidade
Miskatonics --- as condições íntimas de sua percepção foram nutridas
numa relação difícil e estranha com sua mãe. Em minha mente, Lovecraft é
uma mistura de dois personagens: o Norman Bates de \emph{Psicose} com o
Walter Kovacs\footnote{Assim como a mente fascista de Rorschach se
  nutria da publicação (ficcional) de extrema"-direita \emph{The New
  Frontiersman}, Lovecraft publicou o jornal \emph{The Conservative}.
  Para se ter uma noção da coisa, traduzo um trecho do editorial de
  julho de 1915: ``Fora do domínio da pura literatura, \emph{The
  Conservative} será sempre um defensor entusiasta da abstinência total
  e da proibição; do militarismo moderado e saudável, em contraposição à
  pregação de paz, perigosa e não patriótica; do Pã"-Saxonismo, ou a
  dominação de ingleses e de raças afins sobre as divisões inferiores da
  humanidade; e do governo constitucional representativo, em oposição
  aos perniciosos e falsos esquemas da anarquia e do socialismo'',
  \emph{in}: \textsc{joshi}, S.\,T. \emph{A Dreamer and a Visionary: Lovecraft and
  His Time}. Liverpool: Liverpool University Press, 2001, p.\,86.} de
\emph{Watchmen}, mas cuja percepção se voltou para a composição de
narrativas ao invés de práticas violentas. Ele visivelmente idolatrava a
mãe, e viveu dependente dela até boa parte de sua juventude adulta; é
provável que tenha sido virgem até depois dela morrer; realmente
notável, por exemplo, é que, passado um mês da morte, encontra sua
primeira namorada, Sofia Haft Greene --- com quem aliás se casaria três
anos depois.\footnote{Em entrevista a R.\,Alain Everts, Greene teria dito
  que Lovecraft fora virgem até o casamento (aos 34 anos de idade) e
  nunca tomara iniciativa no sexo, embora respondesse se ela o
  procurasse. \emph{In}: \textsc{joshi}, S.\,T. Idem, p.\,202.} A força que a mãe
exerceu sobre Lovecraft é semelhante à daquelas figuras ficcionais
citadas: um misto de fascínio e medo. Via sua mãe como alguém de
elegância superior, mas, ao mesmo tempo, ela cometia crueldades bizarras
com o garoto, como dizer a outras pessoas que seu filho era tão horrendo
(\emph{hideous}) que preferia mantê"-lo escondido de todos.\footnote{\textsc{joshi}, S.\,T. \emph{Idem}, p.\,67, transcrevendo relato de Clara Hess, de seu
  encontro com Susan Lovecraft, mãe do autor.}

Lovecraft acabou por muito intimamente transformar em monstro, suspeita
e horror toda a experiência daquela figura estranha, doce e autoritária,
elegante e rude: introjetou o autoritarismo e o desgosto pelo lado
externo da vida e suas manifestações de prazer --- repelia o
\emph{corpo}, e o \emph{sexo}, de forma estrutural ---, além de
desenvolver preconceitos assombrosos. É importante notar que
muitos\footnote{S.\,T.\,Joshi (1958), provavelmente o melhor biógrafo de
  Lovecraft até o momento, é um dos que minimizam seus preconceitos. Em
  um artigo no seu blog pessoal, no qual ataca um livro sobre Lovecraft
  (e ele tem razão sobre aquele ser um mau livro), escreve: ``Os ataques
  sobre o racismo de Lovecraft são um fenômeno extremamente recente, e
  parecem ser alimentados por certos escritores determinados a se
  agarrar a esse aspecto da vida de Lovecraft para derrubá"-lo um pouco
  na estima da crítica'', logo após escrever que críticos ásperos ao
  autor no passado, como Edmund Wilson, haviam encontrado muitos motivos
  de crítica a suas obras, mas que ``o racismo não era um deles'', e
  atribui o recente dilema a um ``clima hiper"-politizado, hoje''. Não se
  trata de derrubar Lovecraft, nem de hiper"-politização, mas de ser
  inequívoco o fato de que o racismo está imbricado no que escreveu.
  Louis Ferdinand Céline, por exemplo, era racista e nazista, mas
  ninguém discute a qualidade de seus romances nem sua importância para
  a literatura francesa do século \versal{XX}.} repetem, como justificativa para
os variados e espessos preconceitos de Lovecraft, o corolário de que não
se pode esperar que um homem supere as limitações ideológicas de sua
época: isso, em primeiro lugar, não é verdade, porque muitos as
superaram ao longo do que chamamos História com esse distintivo ``h''
maiúsculo, e mesmo a impressão de que o preconceito fosse no passado a
norma é no mínimo discutível; segundo, porque mesmo para o típico
preconceito de época, de meados do século \versal{XX}, Lovecraft era
\emph{particularmente} preconceituoso. Paula Guran define de modo direto
e exato:

\begin{quote}
Devemos todos reconhecer o quanto as crenças pessoais de H.\,P.\,Lovecraft
ligam"-se às suas obras. Lovecraft --- como evidenciam sua ficção,
sua poesia, seus ensaios e suas cartas --- era racista, xenófobo e
antissemita. Ele pode ou não ter odiado mulheres (misoginia), mas de
fato parece tê"-las temido (ginofobia). Sua aversão à sexualidade e à
fisicalidade iam além do tipo puritano.

Os preconceitos do autor têm sido frequentemente afastados como
``típicos'' de um homem de seu tempo. Sim, Lovecraft viveu numa época em
que o racismo era mais aberto e segregação racial era lei, mas o
preconceito de Lovecraft parece, no mínimo, algo mais evidente do que o
da maioria de seus contemporâneos.\footnote{\textsc{guran}, Paula (ed.).
  \emph{The Mammoth Book of Cthulhu: New Lovecraftian Fiction}. London:
  Little, Brown Book Club, 2016, ``Introduction'', n.\,p.}
\end{quote}

Neste ``O chamado de Cthulhu'', por exemplo, chega mesmo a chamar
``primitivos'', com especial desdém, aos descendentes dos irmãos
Lafitte, piratas franceses que se estabeleceram na Louisiana; insiste no
fato de que os adoradores do culto antigo de Cthulhu eram
\emph{half"-breed, mongrels, mixed"-blooded}, \emph{hybrids},
\emph{swarthy}, isto é, ``mestiços'', ``viralatas'', ``pardos'',
``híbridos'', ``morenos'', nos termos mais variados e pejorativos que
encontra, e para a constatação indubitável de suas inferioridades
física, moral e intelectual. São \emph{aberrants}, aberrações, meio
animalizados e, como tais, suscetíveis aos apelos das criaturas
abissais, perversas.\footnote{O fato de a narrativa ser de primeira
  pessoa, e de essa pessoa ser o personagem fictício Francis Wayland
  Thurston não alivia a situação, porque a discriminação é marca
  consistente da literatura de Lovecraft (incluindo suas cartas
  pessoais, nas quais demonstra os mesmos preconceitos), e porque, mesmo
  dentro do conto, não se propõem motivos para se desconfiar da
  autoridade de seu narrador. E os efeitos da história dependem
  justamente da combinação daqueles elementos.}

O horror que Lovecraft, como pessoa, sentia por toda a diferença, se
transfigura no material de sua percepção talentosa do medo: Lovecraft
temia as mulheres, temia as sensibilidades afinadas com a mudança (daí o
sonhar"-se alguém de uma linhagem nobre e antiga), temia as multidões de
pessoas mestiças ou de etnias diferentes, temia as culturas que não eram
a sua, e que não compreendia nem desejava compreender: temia, em suma,
tudo o que é a mobilidade inevitável da vida, ou a vida ela"-mesma. E
temia com um fascínio.

Criou, portanto, um modelo ficcional no qual pudesse defender sua psique
disso que via como uma derrocada do gênero humano, uma desordem sob um
segredo, segredo que acordaria antigos horrores ferozes e dormentes,
imortais, ódio que tomasse formas gigantescas e cuja umidade, cuja
textura reluzente fosse alegoria de uma sexualidade em retrocesso,
deformada em monstro,\footnote{A parte final de Cthulhu é basicamente o
  enfrentamento entre um tipo de pênis e uma gigantesca vagina, da qual
  foge aterrorizado até não poder evitar o contato viscoso, mas com
  tentativa de destruição, aniquilamento e repulsa. David Cronenberg
  (1943), diretor que há alguns anos era o maior expoente do
  \emph{horror físico}, ou \emph{biológico} (com tintas lovecraftianas,
  também) em filmes como \emph{Shivers} (Calafrios, 1975),
  \emph{Scanners} (1981), \emph{Videodrome} (1983) e \emph{The Fly} (A
  mosca, 1986), entre outros, tem um filme particularmente perturbador
  chamado \emph{The Brood} (Os filhos do medo, 1979), no qual enfoca as
  crias assassinas do caso patológico de uma mãe que as expele como
  cânceres do corpo, a partir de grossuras de uma infestação de ódio na
  carne. Uma metáfora brutal da sexualidade e da ligação ambígua e
  umbilical da mãe com seus filhos, e vice"-versa.} assim como sua
estranheza inumana espelhava o abismo intransponível que sentia em
relação às etnias diferentes ou misturadas de sua experiência
estadunidense.

O monstro das antigas narrativas épico"-civilizacionais em verso não era,
como por vezes possa parecer à leitura casual, indicativo da infância
mental do mundo, assim supersticioso: o monstro é um amálgama, um
condensado, uma composição de tudo o que é desafio civilizacional sem a
estrutura abstrata e analítica (e necessariamente posterior) de, por
exemplo, um romance de formação, mencionado acima. A estrutura do
monstro, como a do mito, é \emph{aglutinante,} sintética: busca captar o
perigo, o terror e a missão civilizacional em um grande complexo que,
alegoricamente, reúna as características do que o herói civilizacional
(também um complexo, mas de aspectos virtuosos) terá de enfrentar para
dar ao seu povo uma plataforma moral, social, cultural e política a
partir da qual projetar sua experiência.

No caso de Lovecraft, que pertenceu a um país e a uma cultura já
estabilizados em suas práticas, o regresso ao mito tem duas frentes:
uma, psicossocial, como também Freud estabelece psicanaliticamente em
\emph{Das Unheimliche} (O inquietante, 1919, ou O
estranho"-familiar)\footnote{O ensaio de Freud mostra precisamente esse
  \emph{inquietante} que faz o \emph{não familiar} a partir daquilo que
  é \emph{familiar}. Quando se o aplica no caso de Lovecraft é
  necessariamente por notar algo que explora a desproporção entre as
  origens de sua percepção e o tamanho monstruoso que adquirem essas
  coisas na invenção de seu \emph{inverso}, o de personagens abissais,
  criaturas entrevistas, relanceadas, indiciadas nos resquícios de sua
  presença ominosa no mundo.} e que tem seu registro literário mais
complexo, por exemplo, na obra de Franz Kafka, em especial em \emph{Die
Verwandlung} (A metamorfose, 1915); outra, pessoal, como código
``mitológico'' para suas impressões e opiniões sobre o mundo. E é por
isso que o mito revisitado é, quase sem exceção, um retrocesso: um mito,
como nasce em qualquer agrupamento humano, vem de uma experiência
coletiva, organiza o conhecimento e o dissemina por suas qualidades
sintéticas; quando o mito é trazido por circunstâncias pessoais, fazendo
o caminho inverso para dentro da psique coletiva, é a marca do desejo de
poder e de controle que não encontrou um circuito viável dentro da
experiência, revertido em monstro, fazendo vibrar aquela porção arcaica
e brutal da humanidade.

Todos sabem que o maior medo é o medo do desconhecido: e como Lovecraft
temia quase tudo o que é vida, seu notório talento pôde dar ao mundo um
negativo da experiência, ou, como se diria em \emph{Stranger Things} ---
a série famosa também por seus monstros lovecraftianos --- \emph{the
upside down}, o ``mundo invertido''.

\section*{O espaço como horror}

Das idiossincrasias do autor também surge este ``A cor que caiu do espaço''.
Bastante consciente da divisão qualitativa naquilo que
se pode chamar \emph{literatura fantástica}, Lovecraft assinala, em carta de 1934 a um Mr.\,Nelson --- já na última fase de sua escrita (viria
a morrer em 1937) --- que uma coisa é Edgar Allan Poe, Ambrose
Bierce (1842--1913) e Lord Dunsany (1878--1957), etc., autores do que
chama ``auto"-expressão'', e outra aquilo que define como algo ``composto
artificialmente para atender certas demandas do leitor superficial \&
convencional''.

Enquadrava nessa segunda categoria a vulgaridade de muitos escritores de ficção
científica (\emph{pulp}, como era o caso do período) quando confrontados
com um problema singular: como representar entidades extraterrestres? A
resposta corriqueira --- e que continua a ser corriqueira, basta ver um
filme recente de muito sucesso, \emph{Arrival} (\emph{A chegada}, 2016,
de Denis Villeneuve)\footnote{No filme, as criaturas têm a forma
  lovecraftiana de um dos Antigos, e parecem octópodes, com longos
  tentáculos que até mesmo expelem tinta negra. Quase nada escapa dessa
  regra. Nos filmes, excetuam"-se apenas alguns, como \emph{Invasion of
  the Body Snatchers} (Os Invasores de Corpos, Philip Kaufman, 1978),
  que teve um antecedente em 1956 e vem da história ``The Body
  Snatchers'', de Jack Finney, serializada pela Colliers Magazine em
  1954, ou \emph{The Thing} (O Enigma de Outro Mundo, John Carpenter,
  1982), também um tipo de refilmagem de \emph{The Thing from Another
  World}, de 1951, e que tem igualmente como base um \emph{pulp} de
  1938, ``Who Goes There?'', escrito por John W. Campbell Jr. e
  publicado pela revista \emph{Astounding Tales}. Lovecraft, em 1927,
  antecede a ambas, e podemos dizer que, enquanto os \emph{invasores de
  corpos} assimilavam"-se inicialmente a formas como que de plantas, e a
  \emph{coisa} era também um tipo de parasita alienígena que assumia
  corpos terrestres e se reconfigurava neles, os alienígenas de
  Lovecraft são muito mais imaginativos porque entidades indefiníveis.
  Lovecraft supera o esquema alegórico dessas outras ótimas narrativas,
  porque não propõe uma visão a ser lida como índice para outra coisa na
  realidade: o conto é, em si mesmo, a proposição.} --- é fazê"-los
humanoides, ou o resultado de algum tipo de evolução alternativa a
partir de criaturas encontráveis em nossa natureza terrestre.

Não seria apenas um caso flagrante de falta de imaginação --- ou de
restrições acabrunhantes de um referencial composto apenas por
experiências deste planeta e da percepção humana ---, mas uma falha
elementar de \emph{filosofia}: seria mais inteligente supor que algo que
viesse de fora da nossa experiência, tanto física como conceitual, seria
muito provavelmente tão distante daquilo que conhecemos e
\emph{reconhecemos} como vida, que provocaria terror já pela total
surpresa e pelo esforço de integrarmos aquilo a qualquer coisa redutível
ao conhecido.

Um exemplo de como costumamos proceder diante do desconhecido: na
\emph{Naturalis Historia} (História Natural) de Plínio, o velho (24--79
d.C.), é muito interessante observar que, nas descrições de animais e
plantas com que não teve contato visual, o naturalista compõe então uma
criatura arranjada de partes daquilo que é conhecido, resultando em um
tipo de híbrido \emph{imaginário} de algo \emph{real}, e daí temos
mariposas que sugam sangue, temos as \emph{lacrimis arborum} (lágrimas
de árvores),\footnote{Pline l'Ancien, \emph{Histoire Naturelle} (texte
  établi, traduit et commenté par A. Ernout et Dr. R. Pépin), Paris,
  Société d'Édition ``Les Belles Letters'', 1947, p.\,33.} as abelhas que
sofrem de uma tristeza paralisante, o uso do canino direito do lobo em
operações mágicas,\footnote{\emph{Idem}, p.\,81.} ou, supondo que, pela
proximidade da saliva e do sangue nos felinos, mesmo os domésticos, eles
seriam \emph{inuitat ad rabiem} (ou ``inclinados à
ferocidade'').\footnote{\emph{Ibidem}, p.\,83.} O rinoceronte, por
exemplo, tem um caminho desses na história da aproximação humana com
suas peculiaridades. Plínio o descreve bastante fielmente:

\begin{quote}
um animal que tem um chifre só, projetando"-se do focinho; tem sido visto
frequentemente desde então. Também ele é um inimigo natural do elefante.
Prepara"-se para o combate afiando seu chifre contra as rochas; e ao
lutar dirige"-o sobretudo contra a barriga de seu adversário, pois sabe
que é a parte mais sensível. Os dois animais são de comprimento igual,
mas as pernas do rinoceronte são muito mais curtas: sua pele tem a cor
de madeira"-de"-buxeiro.\footnote{\emph{The Natural History of Pliny the
  Elder} (translated by John Bostock and H. T. Riley), London, Henry G.
  Bohn, 1858, Book \textsc{viii}, Chapter \textsc{xxxix}.}
\end{quote}

Afirma, logo depois, que um de seus grandes inimigos naturais (além do
elefante) é o dragão, e as notas recentes ao texto supõem que com
\emph{dragão} Plínio referia algum tipo de serpente, pelo motivo talvez
decepcionante, mas até razoável, de que dragões não terão existido.

A respeito de elefantes, seria ainda curioso fazer notar como aquelas
compilações da autoridade antiga em história natural, os bestiários,
procediam seus catálogos descritivos de fauna. Em particular, um
bestiário do século \textsc{xiii}, da Bodleyan Library de Oxford, propõe a tromba
do elefante da seguinte maneira: ``Seu nariz é chamado tromba porque ele
o utiliza para levar comida à boca; a tromba é como uma cobra, e é
protegida por uma fortificação de marfim'',\footnote{\emph{Bestiary}
  (Oxford M. S. Bodley 764, with an introduction by Richard Barber),
  London, The Folio Society, 1992, p.\,40.} no qual explicitamente
recorre a ousadas combinações de imagens para tornar visível seu animal
fantástico (e real) para quem lê.

A história prosseguiria na xilogravura de Albrecht Dürer (1471--1528),
representando, em 1515, um rinoceronte do qual tinha apenas uma
descrição de alguém que vira um exemplar indiano do perissodáctilo,
enviado como um presente do papa Leão \textsc{x} a Lisboa, para o rei Manuel\,\textsc{i}. E
então Dürer o desenha como que com chapas de armadura revestindo o corpo
colossal, fazendo"-o em parte verdadeiro, em parte imaginário; e, no
século \textsc{xx}, Salvador Dalí (1904--1989) o utilizaria, já com pleno
conhecimento do que é um rinoceronte em 1954, para representá"-lo com
rendas cobrindo sua bizarríssima couraça.

Tudo o que está acima são exercícios de figurar o desconhecido de modo a
torná"-lo familiar, passível de composição, imaginável; ou, por outro
lado, com o objetivo de distorcer o real para fazê"-lo comportar a
dimensão mais ampla da hipótese.

O que pretendo dizer com esse excurso da história natural é que
Lovecraft é o primeiro de que tenho notícia a romper com esse sistema de
aproximação do desconhecido: em ``A cor que caiu do espaço'', o que faz
com engenho incomparável é propor um tipo de forma de vida indefinível
sob nossos padrões de observação, algo que, para os personagens humanos
que entram em contato com ela (e os leitores), é um horror completo e
quase uma abstração. Fazê"-lo exigiu algumas referências, uma curiosidade
muito inteligente e o labor intenso para escapar daquilo que poderia, de
outra forma, ser apenas um truque barato para manipular os medos de seu
público. Lovecraft fez, portanto, muito mais: nos fez uma proposição
hipotética, pretendeu expandir a percepção.

\section*{A invenção do incomunicável}

``The Colour Out of Space'' é um conto publicado em setembro de 1927
pela revista \emph{pulp}\footnote{Para uma síntese da história das
  revistas e do tipo de narrativa \emph{pulp}, ver p.\,\pageref{pulp}.} Amazing Stories. Lovecraft o escreveu quando
trabalhava no ensaio ``Supernatural Horror in Literature'' (Horror
sobrenatural em literatura), e interessa considerar o resultado de sua
exploração intelectual do tema na narrativa notável que produz, geminada
ao ensaio, porque é bastante claro que algumas questões do ensaio --- em
especial aquelas nas quais Lovecraft efetivamente abria novos rimos para
a ficção especulativa --- influíram na escrita desse texto, ou ganharam
nele sua demonstração ficcional.

Na opinião do próprio Lovecraft, esse conto era uma de suas narrativas
favoritas, e não por acaso: terá percebido que concebera algo singular,
algo que ninguém antes fora capaz de registrar, e que, acrescento,
ninguém mais registrou tão bem em ficção. Se chamamos esse texto
\emph{ficção científica,} nós o empobrecemos, não porque ficção
científica seja algo menor, mas porque nele há mais do que uma hipótese
científica, ou uma alegoria do presente lançada em um futuro
prospectivo. Poderíamos dizer que é tanto científico quanto filosófico,
além de, naturalmente, ser uma das histórias mais horripilantes já
concebidas.\footnote{\emph{Colour Out of Space} (2019), o filme recente,
  em parte consegue se aproximar do aspecto chocante, por exemplo, na
  cena do sótão (a cena do celeiro é quase uma citação a \emph{The
  Thing}, filme de 1982 dirigido por John Carpenter, que, por sua vez,
  buscava ser lovecraftiano), mas falha em instilar o permanente
  desassossego, e perde muito da sutileza engenhosa de construção de
  atmosfera.}

Um dos pontos fundamentais para a consideração do que se passa no conto
tem uma âncora factual igualmente extraordinária, o caso das ``garotas
radiativas'': jovens que haviam conquistado empregos bem pagos para
executar pinturas pequenas, de precisão, em diversos produtos (como os
aros dos mostradores de relógios), utilizando um material então na moda,
o elemento \emph{rádio} --- brilhante por sua radio luminescência --- em
seus pincéis, que elas lambiam para dar às cerdas melhor ponta, como
instruíam seus superiores propondo o lema \emph{lip, dip, paint} (lambe,
mergulha, pinta).

Elas não sabiam ainda da alta toxicidade do rádio, e acabavam seus
turnos com o rosto e o cabelo cintilantes, quando saíam à rua, o que
causava fascínio não apenas nelas, e chegou mesmo a significar certo
\emph{status} social. Mas, pouco depois, a ação da radiatividade fez com
que começassem a desenvolver cânceres, que surgiam antes na mandíbula (a
chamada ``mandíbula radiativa'') e chegavam até os ossos, passando
espantosamente a emitir radiação de dentro para fora.

Sofreram preconceito de toda parte, quando mesmo colegas não acreditavam
em suas versões de que o rádio as estava matando, e supunham que morriam
de sífilis: uma condenação moral de puro preconceito. Os processos
contra as companhias empregadoras demonstraram que eram expostas a um
perigo potencial desconhecido para elas, mas descobriu"-se que os
técnicos já se protegiam durante a manipulação. Seu caso, aviltante e
demorado, acabou servindo para fazer avançar juridicamente os direitos
trabalhistas sobre as doenças ocupacionais, e para expor o perigo de um
material então comercializado sem precaução.

Mas o caso também teria, como se supõe, alimentado a imaginação atenta
de Lovecraft: os ossos se fraturavam e decaíam facilmente, a necrose se
instalava nas mandíbulas, o que se iniciava com sangramento das gengivas
e acabava com o desfigurar do rosto por tumores e ossatura porosa, como
descrito e diagnosticado por um dentista ainda em 1924. Essa força
invisível, que ao mesmo tempo operava modificações completas no corpo
--- que perdia as cores, também --- até levar à morte, sugeria não
apenas mais uma hipótese sorrateira do desconhecido, mas renovava aquele
princípio frankensteiniano dos limites da ciência, quando é impossível
determinar a natureza e o efeito do que há num universo vasto demais
para ser apreendido.

Podemos notar distintamente o efeito dessa história macabra --- colhida
no mundo que apenas adentrava o uso arriscadíssimo e logo militarizado
da radiação --- em ``A cor que caiu do espaço'', no qual todas as
características horríveis da ação radiativa se observam, incluindo a
devastação de lugares que se tornam desertos de poeira cinzenta; mas
seria também necessário lembrar que não se trata nem precisamente de
radiação, e ainda menos de um \emph{ataque} extraterrestre: sequer se
define algum objetivo discernível no efeito daquilo que se sentia como
uma presença. A desproporção do evento é a desproporção entre os tipos
de entidades confrontados pelo incidente.

Mas Lovecraft ainda teria outros pontos de referência para sua
composição.

\section*{As pedras de raio}

Em ``The Colour Out of Space'' tem"-se, portanto, uma forma de vida mal
percebida, seja por pequenos fazendeiros locais ou por experimentados
cientistas, que sequer poderiam estar certos de que o que presenciavam
era de fato uma forma de vida.

A cor do título é uma sagacidade inteiramente literária, por ser
possível apenas em literatura explorar algo que, do ponto de vista dos
sentidos (e do próprio intelecto), é intangível. Lovecraft desenvolve
seu artesanato numa linha impossível às outras artes, basta ver que, na
adaptação de 2019 de seu conto para o cinema --- \emph{Colour Out of
Space}, de Richard Stanley, com Nicolas Cage ---, a cor indefinível do
conto, inexistente no nosso planeta, é convertida, pela necessidade de
representação visual, em \emph{magenta} ou \emph{fúcsia,} porque tal cor
não existe na natureza em uma única faixa de ondas do nosso espectro
visual. Conquanto o engenho se justifique, a eficácia de sua invenção é
insignificante se posta perto de sua origem literária lovecraftiana:
lendo o conto se percebe entre os personagens o desconcerto completo, a
perplexidade diante da cor inexistente, indescritível, quase nem cor,
que já por isso interfere modificando a consciência de quem entra em
contato com a estranheza absoluta.

E esse é outro ponto fundamental da história: Lovecraft implica uma
ambivalência de aspecto sobrenatural na alteração de estrutura de tudo
aquilo com que a emanação alienígena entra em contato. Pode"-se entender
a coisa de modo próprio ou figurado: tanto pode ser algo que altere
fundamentalmente as relações do novo ambiente, por condições
físico"-químicas inteiramente estranhas para este mundo, quanto pode ser
a proposição de que alterar a percepção e a consciência significa
alterar a própria realidade. Assim, nada mais funcionaria: surgem
frequências de ondas incompatíveis com o espectro visível, distorções de
toda espécie, auditivas, tácteis, irregularidades genéticas, abominações
físicas revertendo toda ordem conhecida, afetando de modo catastrófico
percepção, intelecção e biologia.

Essa proposição ficcional arregimenta grandes forças também por outro
motivo: como propus acima, Lovecraft não o faz pelo simples motivo de
compor engenhosa fantasia, mas pretende seriamente desafiar, com as
noções especuladas em seu conto, o que descreve em seu texto sobre
``Horror sobrenatural em literatura'' como o modo desdenhoso da
``sofisticação materialista'', apegada aos ``eventos externos'', que
constrói uma narrativa didática com algum nível de ``otimismo
sorridente'', ao que ele contrapõe ``as visões amplificadas'' das
novidades incipientes da ciência, como ``química intra"-atômica,
astrofísica avançada, doutrinas da relatividade, e sondagens da biologia
e do pensamento humano''.

É notável em seus dois textos esse aspecto duplo, feito de \emph{ficção}
e de \emph{convicção}: sua escrita tem, guardadas as proporções, o
fervor daquele antigo dispositivo, a visão, que encontramos em poetas
visionários como William Blake ou Arthur Rimbaud. Lovecraft é sério a
respeito do que compõe, não pretende apenas oferecer uma distração
moderadamente estimulante, mas está apresentando, na verdade, um arranjo
prospectivo e perceptivo, o que deveria ser o princípio estruturante de
toda a chamada \emph{literatura especulativa}.

Nesse ponto interessa considerar ainda outro influxo que serviu para dar
base a ``A cor que caiu do espaço'', o livro de Charles Fort
(1874--1932), \emph{Book of the Damned} (Livro dos Malditos, 1919), que,
apesar do título espetacular, busca na verdade combater um tipo de visão
convencional que Fort acreditava estar tornando a ciência em uma espécie
de religião por excluir dados e informações que não pudessem ser
atendidos pelo sistema científico. Os ``malditos'', assim, são os dados
e as informações que reúne no livro, referências, relatos, fatos e
histórias que constituiriam um grupo de fenômenos ou incidentes,
ignorados ou minimizados em suas singularidades, por uma ciência que,
segundo Fort, vinha se tornando um bloco de concordância tão sólido
quanto qualquer fé.

Em um ponto encontramos algo que terá chamado a atenção de
Lovecraft\footnote{Lovecraft não apenas é associado a Fort por pontos de
  conexão do imaginário de ambos, mas também por tê"-lo citado em ``The
  Whisperer in Darkness'' (1931).} naquele livro peculiaríssimo, o ponto
onde Fort discorre sobre as \emph{thunderstones}, ``pedras de raio'' ou
``pedras de Thor'', que folcloricamente tinham o poder de atrair raios,
e muitas delas tinham origem em meteoritos. Quando Fort, em busca de
padrões, está registrando ocorrências de quedas de aerólitos durante
tempestades, alterações nas formas das nuvens, e muito do que é
atribuído à ``ignorância do camponês'', ele escreve:

\begin{quote}
Ou --- estaríamos a caminho de explicar as ``pedras de raio''. Parece"-me
que, particularmente notável, se confirma a aceitação geral de que a
nossa não passa de uma existência intermediária, na qual nada é
fundamental, ou nada é definitivo para se tomar como um padrão positivo
a partir do qual julgar. Camponeses acreditam em meteoritos. Cientistas
excluem os meteoritos. Camponeses acreditam em ``pedras de raio''.
Cientistas excluem as ``pedras de raio''. É inútil argumentar que os
camponeses estão lá no campo, e os cientistas estão fechados em
laboratórios e salas de conferência. Não podemos tomar como base real
que, quanto aos fenômenos com os quais têm maior familiaridade,
camponeses têm mais probabilidade de estar certos do que os cientistas:
uma pletora de falácias biológicas e meteorológicas dos camponeses se
ergue contra nós.\footnote{Fort, Charles. \emph{Book of the Damned}. San
  Diego, The Book Tree, 2006 (original edition, 1919), pp.\,75--6.}
\end{quote}

Cito apenas o trecho que me parece mais contundente --- se comparado com
a narrativa de Lovecraft ---, mas há ainda tantas outras instâncias nas
quais o autor de ``A cor que caiu do espaço'' poderia ter achado
interesse vivo. Nesse caso que mencionei, porque no texto de Fort se
põem em contraste os modos habituais de se considerar \emph{a ignorância
do camponês} e \emph{o conhecimento do cientista}, o que nos leva de
imediato à cena, no conto de Lovecraft, da queda do estranho meteorito
nas terras de Nahum Gardner, o camponês e sua família que receberão as
visitas de cientistas da Universidade Miskatonic, em Arkham, cidade
grande e próxima; os mesmos cientistas que desistirão de tentar entender
o fenômeno, diante das frustrantes condições, aliás indescritíveis,
daquele mineral extraterreno, e dos ``causos'' de interior que ouvem de
pessoas simples.

Mas os efeitos daquele objeto prosseguirão modificando os camponeses em
contato com ele, e, assim, há uma clara oposição entre aqueles que estão
``no campo'', tendo de lidar com os eventos diretamente, e os que estão
em ``laboratórios e salas de conferência'', e podem deixar o assunto
assim que sentirem que não irão avançar no entendimento. Sente"-se a
hipótese do trecho de Fort sendo desenvolvida ficcionalmente pela
imaginação de Lovecraft, combinando as notícias aterradoras das garotas
radiativas e seus cânceres deformantes (pelo elemento radiativo e sua
luminescência), e as estranhas ocorrências meteoríticas levantadas sob o
título momentoso, ``maldito'', de Fort.

S.T. Joshi, estudioso e mais importante biógrafo de Lovecraft, escreve
em seu livro \emph{A Subtler Magick}: \emph{The Writings and Philosophy of H.\,P. Lovecraft} (1996)
acrescentando um detalhe que também situa, com outro dado histórico, o
momento de criação do conto: Joshi menciona o Quabbin Reservoir,
reservatório cuja construção foi anunciada em 1926 (embora só ficasse
pronto em 1939) e que ``se localiza exatamente na área central de
Massachusetts, onde o conto transcorre, e que envolvia o abandono e a
submersão de povoados inteiros na região'',\footnote{Joshi, S.T. \emph{A
  Subtler Magick}: \emph{The Writings and Philosophy of H.\,P. Lovecraft,}
  Gillette, Wildside Press, 1996, p.\,135.} precisamente como acontece na
narrativa, também com a visita do narrador, o perito que está
supervisionando a região antes da chegada das máquinas.

Talvez por isso, quero dizer, pelo material extensivo tomado de fatos à
volta de Lovecraft --- reunidos por uma concentração inventiva que não
só os animou, mas também pôde propor uma questão filosófica de
representação, que vimos ocupar sua inteligência com a redação do ensaio
sobre o horror sobrenatural na ficção --- ``A cor que caiu do espaço''
tenha se tornado o conto por excelência da marca mais significativa que
Lovecraft deixou. Visão ainda hoje sem rivais, a construção da narrativa
diz a que veio já no primeiro parágrafo, em um artesanato delicado de
linguagem compondo visualmente o lugar em nossas imaginações: é claro,
Lovecraft sabia que urdia algo especial.

\section*{Um pouco de linguagem, e uma paródia}

Se a linguagem urdida do primeiro parágrafo começa a nos introduzir numa
atmosfera, cuja invenção é fundamental para poder ir extraindo seus
efeitos na leitura, será possível também perceber que Lovecraft
pretendeu, da mesma maneira, aplicar"-se na confecção de ainda outro tipo
específico de abordagem linguística.

Há algo como uma paródia desse famoso conto no filme que juntou Stephen
King e o grande diretor de \emph{Night of the Living Dead} (A noite dos
mortos"-vivos, 1968), George Romero: \emph{Creepshow} (1982) é um
cultuado filme de terror, estruturado em vários episódios que contam
histórias diversas e independentes, como uma \textsc{hq}. No episódio ``The
Lonesome Death of Jordy Verill'' (A morte solitária de Jordy Verill),
camponês caricato (o próprio Stephen King) que vive sozinho em uma casa
rústica, vê a queda de um meteorito em sua propriedade. O meteorito se
abre e um brilho verde radiativo se expande dele: ao tocá"-lo, o pobre
Verill começa a ter o corpo tomado de espessa vegetação, como tudo a seu
redor.

King certamente pensava em Lovecraft, seu conterrâneo da Nova Inglaterra
(King é originário de Portland, no Maine) e escritor admirado, ao
escrever o conto ``Weeds'' (1976), conto que depois resultaria no
episódio paródico de \emph{Creepshow}. Há mesmo a caricatura extrema da
linguagem interiorana de seu personagem, o que também é uma piscadela
para quem tenha lido ``A cor que caiu do espaço'' e notado o cuidado com
que Lovecraft registra sua hipótese da fala simples de Nahum Gardner (e
de Ammi Pierce) quando finalmente se lhe dá voz, já perto do fim. Lá
também se nota o empenho da escrita lovecraftiana em compor aquele
triste momento, no qual o pobre homem tenta, sem muito sucesso,
comunicar a seu amigo algo da experiência indefinível que sua família,
seus animais, sua terra e ele mesmo enfrentaram.

É um conto singular, o favorito de Lovecraft, e o deste seu tradutor e
prefaciador. Boa leitura.

