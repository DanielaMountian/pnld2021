\SVN $Id: PRETAS.tex 7792 2010-11-24 19:16:35Z oliveira $
\begin{resumopage}

\item[Johan August Strindberg] (Estocolmo, 1849---\textit{id.}, 1912) 
foi escritor, dramaturgo, pintor e fotógrafo sueco. Após concluir
seus estudos, dedica"-se à carreira de professor, ao mesmo tempo em que
estuda medicina. Mais tarde tenta lançar"-se como ator, mas em 1870 vai
estudar na universidade de Uppsala, onde começa a escrever. Dois anos
mais tarde interrompe os estudos por razões financeiras; passa a
trabalhar no jornal \textit{Dagens Nyheter} e, a seguir, na
Kungliga Bibliotek --- a Biblioteca Nacional da Suécia.
Em 1879, a publicação do livro \textit{Röda Rummet} (A sala vermelha) e
a encenação da peça \textit{Mäster Olof} trazem"-lhe o reconhecimento
merecido. Em 1882, o aparecimento de \textit{Det Nya Riket} (O novo
reino) --- obra de cunho realista, repleta de críticas às instituições
sociais vigentes na época --- rende"-lhe tantas críticas que o autor vê"-se
obrigado a deixar seu país natal. Strindberg muda"-se com a primeira
mulher, Siri von Essen, e os filhos para Paris e então para a Suíça. No
exterior, escreve uma parcela significativa de sua obra, ao mesmo tempo
em que luta contra graves problemas psicológicos. Em 1897, após
divorciar"-se de Frida Uhl, sua segunda esposa, a condição mental de
Strindberg --- já delicada na época --- deteriora"-se ainda mais. Em um
período de crise profunda, atormentado pela paranoia e por surtos
psicóticos, escreve o romance \textit{Inferno}. Após tornar à Suécia em
1897, casa"-se pela terceira vez, em 1901, com a atriz Harriet Bosse,
que lhe dá a filha Anne"-Marie. Nesse período, as leituras e crenças
pessoais de Strindberg influenciam seu estilo, que passa do realismo ao
expressionismo. Rechaçado pela Academia Sueca, que até 
hoje concede o Prêmio Nobel, Strindberg foi agraciado com uma 
distinção sem precedentes: o Prêmio Anti"-Nobel, uma arrecadação 
pública de dinheiro promovida por seus conterrâneos. 
No fim da vida, o autor instala"-se na Blå Tornet --- a “Torre azul” 
onde hoje funciona o museu em sua memória.
Strindberg morreu no dia 14 de maio de 1912, deixando como legado uma
vasta produção de grande valor literário, que inclui as peças
\textit{Senhorita Júlia}, \textit{A dança da morte}, \textit{O pai},
\textit{A caminho de Damasco} e \textit{A sonata espectral}, além dos
romances \textit{Inferno}, \textit{O~filho da criada}, \textit{Defesa
de um louco} e \textit{Gente de Hemsö}. 

\item[Gente de Hemsö] é considerado uma das obras"-primas de August Strindberg. 
Escrito em sua maior parte quando o autor se encontrava no exílio autoimposto, 
foi publicado pela primeira vez em 1887, pela editora Bonniers, de Estocolmo.  
Estrondoso sucesso desde sua aparição, este romance foi concebido, nas palavras do próprio 
Strindberg, para reconquistar seu público depois de uma fase marcada pela polêmica e pelo 
ostracismo literário.  A obra traça um quadro da natureza física e humana dos arquipélagos suecos, 
berço cultural da Suécia: poucos escritos são tão característicos daquele país 
escandinavo. Fino retrato psicológico de diversas personagens cativantes,  
\textit{Gente de Hemsö} alia humor e lirismo, ocupando um lugar ímpar 
em meio à obra posterior de Strindberg, carregada de tensões e conflitos 
psicológicos.  Adaptado para teatro, cinema e \textsc{tv}, traduzido para diversos idiomas, 
este romance permanece até hoje como uma das obras mais queridas do povo sueco. 
Primeira tradução integral para o português, direta do sueco, esta edição recupera as partes 
expurgadas na primeira edição e contou com o valioso auxílio do professor Per Stam, 
diretor do Projeto Strindberg, mantido pela Universidade de Estocolmo, e responsável pela nova 
edição integral da obra completa de Strindberg na Suécia. 

\item[Carlos Rabelo] traduziu a peça \textit{Camaradagem},
de Strindberg, adaptada por Eduardo Tolentino, para o Grupo Tapa (prêmio de
melhor espetáculo da \textsc{apca}, 2006), e \textit{O amor é tão simples}, de
Lars Norèn. Para a Coleção de Bolso Hedra, traduziu direto do sueco
\textit{Sagas}, de August Strindberg.

\item[Leon Rabelo] traduziu do sueco a peça \textit{Outono e Inverno}, de Lars Norèn, 
e trabalha atualmente em diversos intercâmbios culturais entre a Suécia e o Brasil.

\item[Svenska Kulturrådet] A tradução desta obra foi selecionada e agraciada com o apoio financeiro do Svenska Kulturrådet 
(Conselho Nacional de Cultura da Suécia).


\end{resumopage}

