\SVN $Id: TEXTO.tex 12441 2013-07-31 13:28:53Z iuri $

\chapter[Em que Carlsson assume seu posto\ldots]{Em que Carlsson\break assume seu posto\subtitulo{e é
considerado um falastrão}}
 
\hedramarkboth{Em que Carlsson assume seu posto}{Strindberg}

\textsc{Ele chegou} feito uma tormenta numa tarde de abril e tinha um cantil de
Höganäs\footnote{ Espécie de aguardente regional (\textit{brännvin}, em sueco),  
no caso, da cidade de Höganäs. [Todas as notas são dos tradutores, exceto quando indicadas.]} 
numa cinta pendurada ao pescoço. Clara e Lotten tinham ido buscá"-lo de
barco no ancoradouro de Dalarö; mas levou uma eternidade para que partissem de
lá. Elas tinham que ir à venda comprar um barril de alcatrão, à farmácia buscar
pomada para o porco, e também foram ao correio apanhar um selo, e depois descer
até a casa de Fia Lövström, no Kroken, para emprestar o galo, em troca de meia
libra de corda trançada para as redes de arrastão, e por fim pararam na
estalagem onde Carlsson lhes ofereceu café com biscoitos. Embarcaram por fim,
mas Carlsson queria conduzir, ainda que não soubesse como, pois nunca vira um barco de vela
redonda, chegando mesmo a gritar para alçarem a genoa, uma vela que ali nem existia.

No posto da alfândega, os timoneiros e zeladores riram da manobra, enquanto o
barco ia à deriva, de vento em popa, rumo a Saltsäcken.

--- Ei, você! Tem um buraco no casco! --- gritou através do vento um aprendiz de
timoneiro. --- Arruma isso aí! --- e enquanto Carlsson procurava pelo buraco, Clara
deu"-lhe um empurrão e tomou o leme, ao passo que Lotten, usando os remos conseguiu
pôr o barco no vento certo, que agora deslizava a contento rumo a Aspösund.

Carlsson era baixinho e espadaúdo, da província de Värmland, com olhos azuis e
nariz adunco como um gancho. Por mais enérgico, brincalhão e curioso que fosse,
não sabia nada de navegação; fora chamado a Hemsö para cuidar da lavoura e
dos animais, já que ninguém se metia com isso desde que o velho Flod partira dessa
para melhor, deixando a viúva sozinha com a fazenda.

Quando Carlsson quis assuntar as moças sobre fatos e circunstâncias, ele
obteve respostas típicas de um ilhéu:

--- Sei não, senhor\ldots{} Posso dizer nada disso, não\ldots{} Disso tenho
certeza não\ldots{}

Dali não saía nada!

O barco chapinhava na água entre ilhotas e rochedos, enquanto a pardela cacarejava
atrás das rochas e os tetrazes baritonavam na floresta de abetos; cruzaram
enseadas e corredeiras enquanto a escuridão caía e as estrelas desfilavam para o
alto. O barco agora os carregava sobre águas largas, onde ao longe piscava o
farol de Huvudskar. Passavam às vezes por uma boia, às vezes por balizas
brancas, semelhantes a fantasmas; ainda brilhavam restos de neve feito lençóis quarando, 
e da água negra emergiam boias cegas roçando a quilha do barco quando
este passava sobre elas; uma gaivota sonolenta se assustou no seu penhasco,
despertando vida nas andorinhas"-do"-mar e gaivinas que fizeram um alarido dos
diabos, e bem longe, onde as estrelas encontravam o mar, avistava"-se o olho
vermelho e outro verde de um grande barco a vapor, que ia arrastando uma fileira
de globos luminosos projetados de suas escotilhas e salões.

Tudo era novo para Carlsson e sobre tudo ele perguntava; agora ele obtinha
respostas, tantas que teve a impressão de ter chegado a uma terra estranha.

 ``Ele veio lá da terra'', o que naquele lugar equivalia ao que se diz na cidade
 de alguém que vem da roça.

Ondulando a água, o barco adentrou por um estreito a sotavento, obrigando"-lhes
a remar. Logo adiante entraram em outro estreito onde viram uma casa iluminada
por entre amieiros e pinheiros.

--- Chegamos --- disse Clara, enquanto o barco avançava numa pequena
enseada, onde uma estreita passagem se abria no junco, que farfalhou nas bordas
do barco, acordando um lúcio que nadava, divagando ao redor de uma armadilha.

Um vira"-lata latiu e dentro da casa se avistou a luz de uma lamparina.

Enquanto isso, o barco foi atracado no ancoradouro e começaram a descarregá"-lo.
A vela foi enrolada na verga, retiraram o mastro e enrolaram o estai em torno
dos pinos. O barril de alcatrão foi rolado a terra, e as tigelas, cantis, cestos
e trouxas logo estavam sobre o ancoradouro.

Carlsson olhava ao seu redor em meio à penumbra, vendo somente coisas novas e
diferentes. Ao final do ancoradouro havia um viveiro de peixes com suas manivelas, e
no parapeito do ancoradouro apinhavam"-se boias, cabos, fateixas, chumbadas,
cordas, linhas de pesca, anzóis, e sobre o piso de tábuas achavam"-se barricas de arenque,
gamelas, tinas, selhas, cubas, caixas; e mais à frente um casebre repleto de
iscas penduradas e aves empalhadas para servir de chamariz de caça:
êideres"-edredão, mergansos, mergansos"-de"-poupa, patos"-fusco, patos"-olho"-de"-ouro,
e debaixo do telheiro, em suportes, velas e mastros, remos, croques, toletes,
calhas, varas e arpões. Sobre o chão, armações para redes de arenque, grandes
como os maiores vitrais de igreja, redes para pescar linguados com presilhas,
que serviam para se prender ao braço, redes para perca, novinhas e brancas como
rendas de trenó; e, seguindo em linha reta do ancoradouro, um caminho que
parecia uma aleia de casa senhorial, com duas fileiras de forquilhas em cada
lado, onde penduravam grandes redes de arrastão. Por esse caminho vinha se
aproximando a luz de uma lamparina, jogando seu brilho na areia, onde cintilavam
cascas de mexilhão e guelras secas de peixe, e nas redes de cerco reluziam as
sobras das escamas de arenque, feito geada em teias de aranha. A lamparina
iluminava também o rosto de uma senhora, que parecia curtido pelo vento, e um
par de olhinhos amigáveis, murchados à beira do fogão. E à frente da senhora
vinha o cachorro, um animal de pelo espesso que podia enfrentar tanto terra
quanto mar.

--- Ah, dou graças que vocês enfim voltaram --- saudou"-lhes a senhora. ---
Trouxeram o moço?

--- Sim, patroa! Aqui estamos nós, e este é Carlsson, como pode ver --- respondeu
Clara.

A senhora enxugou a mão direita no avental e estendeu"-a ao capataz.

--- Pois seja bem"-vindo, Carlsson, espero que passe bem entre nós. Meninas, vocês
trouxeram o café e o açúcar, as velas estão guardadas na cabana? Pois então
entrem para comer alguma coisa.

E o grupo subiu em procissão, Carlsson calado, curioso, esperando saber como sua
vida seria naquele novo lugar.

Dentro da casa, a lenha ardia dentro do fogão e uma mesa dobrável, branca,
estava forrada com um pano limpo; sobre a mesa havia uma garrafa de aguardente
estreitada ao meio lembrando uma ampulheta, e ao redor taças de porcelana de
Gustavsberg, pintadas com rosas e miosótis; uma bisnaga de pão fresco e
torradas, uma manteigueira, açucareiro e uma jarrinha com creme completava o
serviço, o que a Carlsson pareceu coisa de gente rica e que ele não esperava
encontrar naquele fim de mundo. A casa em si não era nada mal, quando ele a
olhava de soslaio, à luz das chamas do fogão, que somada à luz da vela de
sebo do candelabro de latão iluminava o polimento um pouco desbotado da
escrivaninha de mogno, espelhava"-se na madeira envernizada do relógio de parede
e no pêndulo, brilhava nos ornamentos prateados dos canos cinzelados das
espingardas e frisava as letras douradas das lombadas dos sermonários, 
hinários, almanaques e manuais de agronomia.

--- Vamos entrar, Carlsson --- convidou a senhora, e Carlsson, que era um filho dos
novos tempos, não se fez de rogado, entrando sem demora para se sentar no sofá
de madeira, enquanto as moças tomavam conta de sua bagagem, que foi parar na
cozinha, do outro lado do vestíbulo.

A senhora desmontou a chaleira e introduziu nela um filtro de pele curtida de
peixe; tampou e aferventou a água, enquanto reiterava o convite a Carlsson,
dessa vez acrescentando que ele se sentasse à mesa.

O capataz sentou"-se todo cheio de dedos, reparando no ambiente para saber como
velejaria por aquelas águas; ele tinha se decidido a causar boa impressão,
mas como ainda não sabia se a patroa tolerava conversa fiada, não se
aventurou a abrir logo a matraca até que soubesse bem seu rumo.

--- Essa escrivaninha é uma raridade! --- disse enquanto passava os dedos nas
flores de metal.

--- Hum! Só não tem muita coisa dentro dela --- respondeu"-lhe a senhora.

--- Como não! --- bajulou Carlsson, pondo o dedo mindinho na fechadura. --- Deve
haver um dinheirão aí dentro.

--- Antes, havia aí ao menos alguns trocados, quando a compramos num leilão, mas
perdi o Flod, e Gusten foi fazer o tiro"-de"-guerra, e assim ninguém mais tomou
conta da propriedade. E ainda se meteram a construir a casa nova, sem nenhuma
necessidade, e tudo foi indo de mal a pior. Olha aqui o açúcar, Carlsson, tome
uma xícara de café.

--- Eu primeiro? --- perguntou cerimonioso o capataz.

--- Sim, como não há mais ninguém aqui --- respondeu"-lhe a senhora. --- O meu filho,
que Deus o abençoe, está sempre pelas ilhas caçando e ainda por cima leva o Norman
com ele, e dessa maneira tudo aqui fica por fazer. Só porque apanham lá algumas
aves, deixam o rebanho e a pesca ao deus"-dará; pois veja Carlsson, é por isso
que o senhor foi chamado, para pôr tudo em ordem; e vai ter que manter
certa superioridade e ficar de olho nos meninos. Aceita um biscoito, Carlsson?

--- Sim, veja, patroa, se eu tiver que manter certa superioridade, para que todos me
escutem, então deve haver também ordem, e para isso precisarei do seu apoio,
porque sei como são os rapazes quando nos colocamos em pé de igualdade com eles
--- retrucou Carlsson com confiança, já em terreno seguro. --- Quanto à navegação,
não vou me intrometer, porque desconheço o assunto, mas em terra, estou em
casa, e é aí que meus conselhos podem lhe ser úteis.

--- Vamos acertar isso amanhã, que é domingo; quando teremos tempo para conversar
à luz do dia. Carlsson, mais um cafezinho antes de dormir.

A senhora serviu"-lhe um pouco de café, e Carlsson pegou a garrafa em forma de
ampulheta e completou bem um quarto de sua xícara. E depois de ter bebido um
gole, sentiu"-se inclinado a reanimar a conversa, que o deixara profundamente
satisfeito. Mas a senhora tinha se levantado para mexer no fogão, as moças
corriam de lá para cá, e o vira"-lata latiu no quintal, desviando a atenção para fora.

--- São os meninos que voltaram --- disse a patroa.

Do lado de fora, ouviram"-se vozes e o ruído das solas de metal sobre as pedras
da encosta, e por entre as balsaminas da janela e a luz do luar Carlsson avistou
dois vultos de homem com espingardas a tiracolo e bornal sobre as costas.

O cachorro latia no vestíbulo, e logo a porta da casa se abriu. O filho entrou
a passadas largas, com botas de marinheiro e camisão; e com o orgulho
imbatível de caçador bem"-sucedido jogou na mesa posta o bornal e uma fieira de
patos selvagens.

--- Boa"-noite, mãe, aqui está a carne! --- saudou, sem ainda ter percebido o
recém"-chegado.

--- Boa"-noite, Gusten. Vocês demoraram --- disse a senhora, lançando um
 olhar involuntário  de satisfação sobre os esplêndidos êideres com suas plumagens
negras feito carvão e brancas como giz, peito manchado de rosa e o pescoço
verde"-mar. --- Vejo que vocês fizeram uma boa caçada. Esse é o Carlsson, que
estávamos esperando.

O filho lançou"-lhe um olhar desconfiado com seus olhos pequenos e afiados,
sombreados por cílios ruivos, e mudou logo de expressão, de expansivo para
tímido.

--- Boa"-noite, Carlsson --- disse ele um tanto seco e acanhado.

--- Boa"-noite --- respondeu o capataz usando um tom ameno, pronto a se tornar
mais altivo assim que tivesse avaliado melhor o rapaz.

Gusten sentou"-se à cabeceira da mesa com o cotovelo apoiado no peitoril da
janela e aceitou uma xícara de café da mãe, misturando"-o logo com um pouco de
aguardente, e bebeu, enquanto observava discretamente Carlsson, que pegara as
aves e as examinava.

--- Essas aqui são bem vistosas --- disse Carlsson, apertando"-lhes o peito para
sentir se estavam gordas. --- Posso ver que se trata de um caçador habilidoso,
pois as balas estão nos lugares certos.

Gusten respondeu"-lhe com um sorriso maroto, percebendo naquele instante que o
capataz nada entendia de caça, elogiando um tiro que acertara as aves na penugem,
tornando"-as inúteis como chamarizes.

Carlsson, no entanto, continuava a falar destemidamente, elogiando os bornais de
pele de foca, louvando as espingardas e diminuindo"-se o mais que podia, tão
ignorante das coisas do mar como de fato era, e mais um pouco.

--- E o que foi feito do Norman? --- perguntou a senhora, que começava a ficar
sonolenta.

--- Ele está guardando as coisas na cabana --- respondeu Gusten --- mas chegará logo.

--- Rundqvist já foi se deitar, e está mesmo na hora, Carlsson deve estar cansado
da viagem. Venha que vou lhe mostrar onde vai dormir.

Carlsson queria mesmo era ficar mais um pouco e esvaziar a garrafa, mas a
indicação fora direta demais para que ele ousasse se opor. A patroa o conduziu
até a cozinha, e voltou logo para o seu filho, que imediatamente retomou seu ar
desenvolto.

--- O que você achou dele? --- perguntou"-lhe a senhora. --- Ele me pareceu confiável
e disposto.

--- Não! --- disse enfaticamente Gusten. --- Não acredite nele, mãezinha; ele só
diz bobagem, esse malandro!

--- Oh, mas que é isso; ele me pareceu honesto, apesar de falar muito.

--- Creia"-me, mãe, esse daí é um falastrão, que vai nos dar muito trabalho até
nos vermos livres dele. Mas não faz mal; ele pode trabalhar em troca de comida, mas que
fique longe de mim. Você nunca acredita no que eu digo, mas você vai ver! Você
vai ver! Depois você vai se arrepender, quando já for tarde demais. Não foi
assim com o velho Rundqvist? Ele também tinha a fala mansa, mas o corpo era
ainda mais mole, e tivemos que aturá"-lo, agora sabemos que vamos tê"-lo aqui até
o dia em que ele morrer. Esses trapaceiros, que tem a língua solta, só são valentes
na hora do mingau. Acredite!

--- Você é igual ao seu pai, Gusten. Nunca tem fé nas pessoas e depois pede o
impossível! Rundqvist não é nenhum homem do mar, também veio lá da terra; mas
ele sabe fazer tanta coisa que outros aqui não sabem; e homens do mar nós não
conseguimos mais trazer, porque eles vão para a marinha, alfândega ou 
trabalhar de timoneiro, e para cá só vem camponês. Mas escute: deve"-se
aproveitar o que se tem.

--- Ninguém mais quer servir a patrão. Preferem um emprego no
governo, e aqui se junta toda a escória lá da terra. Ninguém pensa que gente
honesta viria para essas ilhas, a não ser que tenham suas razões, e por isso
eu lhe digo como antes: abra o olho!

--- Quem deve abrir o olho é você, Gusten --- replicou a senhora ---, e tomar conta
do que é seu, já que tudo aqui um dia será seu, e devia ficar em casa e não no
mar dia e noite, e ao menos não afastar as pessoas do trabalho, como você sempre
faz.

Gusten apanhou um dos êideres enquanto respondia:

--- Ah, mãe, mas você bem que gosta de um assado sobre a mesa, depois de passar o
inverno a peixe seco e carne de porco salgada, então disso você não pode
reclamar. E de resto, eu não saio para beber, e cada um tem sua diversão. Comida,
temos o quanto nos basta, um dinheirinho no banco, e a fazenda não está
arruinada; pode até pegar fogo, já que temos o seguro.

--- A fazenda não está arruinada, eu bem sei, mas todo o resto está caindo aos
pedaços; as cercas precisam ser consertadas, é preciso limpar as valas, o teto da granja
está tão podre que chove sobre os animais; não tem uma ponte sequer inteira,
os barcos estão frágeis como gravetos, as redes precisam ser remendadas e a
estrebaria coberta. E tem mais, e mais, e ainda mais; é tanta coisa por fazer que
nunca é feita. Mas agora vamos ver se tudo isso vai se resolver com  
uma ajuda a mais, e veremos se Carlsson não é o homem para isso.

--- Que ele faça então! --- rosnou Gusten, passando a mão sobre seu cabelo
curto, deixando"-o todo em pé. --- Olha aí o Norman! Entre e tome um trago, Norman!

Norman, um baixote de ombros largos, cabelos de um loiro quase branco e com um
bigode ralo, olhos azuis, entrou na casa e sentou"-se ao lado do companheiro de
caçada, depois de cumprimentar a patroa. E assim que os dois heróis tiraram seus
cachimbos de cerâmica do bolso e os encheram de fumo, começaram a se vangloriar
dos feitos no mar, bem à maneira dos caçadores, passando em revista cada tiro,
enquanto tomavam um café batizado com aguardente. As aves foram examinadas, 
onde o tiro as ferira, os cartuchos contabilizados, os erros discutidos, e 
traçavam novos planos para outras expedições.

Enquanto isso, Carlsson tinha entrado na cozinha, onde seria seu alojamento.

Era um cômodo sem forro que parecia um barquinho emborcado, à deriva, com sua
carga que parecia constituída de todas as mercadorias do mundo. No teto
enegrecido pela fumaça, penduravam"-se na viga redes e apetrechos de pescaria, e
embaixo tábuas de barco dispostas para secar; emaranhados de linho e cânhamo,
fateixas, ferrarias, réstias de cebola, velas de sebo, farnéis; numa trave dupla
havia uma longa fileira de chamarizes recém"-empalhados; em outra tinham jogado
peles de carneiro; numa terceira balançavam galochas, blusas tricotadas,
lençóis, camisas e meias; e entre as traves uma haste de ferro na qual
penduravam pão ázimo, um varal com pele de enguia, estacas com linha de pesca e
arpões.

À janela da cumeeira havia uma mesa sem pintura, e junto à parede três
catres, já feitos com lençóis limpos de pano cru.

A senhora indicara um desses catres a Carlsson, antes de se afastar com a
lamparina, deixando o recém"-chegado na penumbra, mantida apenas pela luz fraca
da brasa do fogão e um pouco de luar, que chegava ao chão recortado pela grade
da janela. Por razões de pudor não havia iluminação junto aos catres, já que as
moças também dormiam na cozinha, e Carlsson começou a se despir à luz do
anoitecer. Despiu a capa e tirou as botas, sacou do colete o relógio 
de bolso, para dar"-lhe corda junto à luz da brasa. Torcia a chave com dificuldade, 
porque o relógio só era usado aos domingos e ocasiões especiais,
quando escutou uma voz grave e rabugenta debaixo dos lençóis:

--- E não é que o danado tem um relógio também!

Carlsson estremeceu, olhou para baixo e viu à luz da brasa uma cabeleira
despenteada com um par de olhos semicerrados pendendo sobre dois braços peludos.

--- E em que isso lhe toca? --- disse para não ficar sem resposta.

--- Tocar, só o sino da igreja, mas lá eu nunca vou! --- respondeu"-lhe a cabeça.
--- Mas é um rapaz muito chique, tem até marroquim na bota!

--- Isso mesmo, e pode acrescentar que tenho galochas!

--- Jesus! Tem galochas também; então com certeza pode oferecer um trago!

--- Sem dúvida que posso lhe oferecer um traguinho --- respondeu Carlsson já
desarmado, indo buscar o cantil de Höganäs. --- Sirva"-se, e bom proveito.

Sacou a rolha, bebeu um gole e passou adiante o cantil.

--- Deus lhe abençoe, tenho certeza que essa é da boa. Saúde! E bem"-vindo à nossa
vilazinha! E agora chega de formalidades, Carlsson, você pode me chamar de
maluco Rundqvist, que é como costumam me chamar.

E daí ele se enfiou nas cobertas.

Carlsson, por sua vez, tirou a roupa e se deitou, depois de ter pendurado seu
relógio no barril de sal e de dispor suas botas bem à vista, deixando à mostra 
o revestimento de marroquim vermelho. A casa estava em silêncio e só
se ouvia a respiração pesada de Rundqvist ao lado do fogão. Carlsson ficou
acordado pensando no futuro; as palavras da patroa cravaram"-se como um prego em sua
cabeça: ele deveria manter certa superioridade em relação aos outros e pôr a
propriedade de pé. E o prego doía"-lhe e inchava tudo ao seu redor, era como se
na sua cabeça estivesse nascendo uma planta. Ele pensava na escrivaninha de
mogno, nos cabelos ruivos do filho e seus olhos desconfiados. Já se via andando
com um grande molho de chaves num anel de aço, chacoalhando"-o contra o forro da
calça; e quando alguém lhe pedia dinheiro, ele erguia a aba de seu avental de
couro, sacodia a perna direita, enfiava a mão no bolso e apalpava as chaves contra a
sua coxa; dedilhava o molho de chaves como se estivesse desemaranhando uma rede,
e ao pegar a chave menor, a da escrivaninha, ele a introduzia na fechadura, como
fizera com o dedo mindinho naquela tarde, mas a fechadura, que lhe parecera um
olho com a pupila dilatada, ficava grande e escura feito a boca de uma
espingarda, e do outro lado de seu cano aparecia um olho vermelho de peixe, que
era o filho da patroa mirando"-lhe com precisão, traiçoeiro, como quem defende seu ouro.

O rumor de passos junto à porta da cozinha despertou Carlsson de sua sonolência. Sobre o
piso, para onde os quadrados de luar tinham se mudado, passaram dois corpos em
camisolas brancas que logo pularam na cama, que rangeu bastante, como um
ancoradouro bambo onde se atraca um barco; os lençóis se moveram com
risadinhas até que o silêncio voltou a reinar.

--- Boa"-noite, meninas --- ouviu"-se a voz sonolenta de Rundqvist. --- Sejam
boazinhas e sonhem comigo.

--- Sim, como esqueceríamos disso? --- respondeu Lotten.

--- Ssch, não fale com esse sem"-vergonha --- avisou"-lhe Clara.

--- Vocês são\ldots{} tão boazinhas. Se eu ao menos pudesse ser como vocês! ---
suspirava Rundqvist. --- Santo Deus, a velhice chega e não damos conta de mais
nada, que lixo de vida. Boa"-noite para vocês, crianças, e tomem cuidado com o
Carlsson, que ele tem relógio e botas de marroquim! Carlsson, esse é feliz. A
felicidade vem, a felicidade vai embora, feliz aquele que tem mulher a toda
hora. De que estão rindo aí? Carlsson, não me daria mais um trago? Faz tanto
frio nesse canto, vem uma friagem que entra pelo fogão.

--- Não, já chega, agora quero dormir --- resmungou Carlsson, importunado em seus
sonhos com o futuro, nos quais não havia nem vinho nem mulheres, pois já se encontrava
em sua posição de ``superioridade''.

Fez"-se silêncio novamente, somente os ruídos amortecidos das histórias dos
caçadores atravessavam as duas portas, além dos lentos puxões de vento noturno
no madeirame.

Carlsson fechou os olhos e escutou, enquanto adormecia, a voz de Lotten recitar
suavemente algo que ele primeiro não entendeu, mas que aos poucos foi se
arrastando numa ladainha, da qual ele pode discernir: \textit{e não"-nos"-deixeis"-cair"-em"-tentação, 
mas"-livrai"-nos"-do"-mal, porque"-teu"-é"-o-reino"-o-poder"-e-a"-glória"-para"-sempre"-amém.} 

--- Boa"-noite, Clara. Durma bem.

Por um breve tempo se ouviu alguém ressonando da cama das meninas, já Rundqvist
roncava tão alto que parecia um serrote na madeira fazendo as janelas tremerem,
fosse por brincadeira ou a sério. Carlsson estava sonolento, e não sabia mais se
estava acordado ou dormindo, até que sentiu alguém levantar a coberta, e um
corpo rechonchudo e suado se deitar ao seu lado.

--- Sou eu, o Norman --- escutou uma voz suave, reconhecendo então o criado que
teria como companheiro de cama.

--- Ah, é o caçador que está de volta --- resmungou Rundqvist com sua voz rouca de
baixo ---, pensei que era o Carlsson, aqui, que tivesse saído para caçar no sábado à noite.

--- E você, que não pode caçar, Rundqvist, pois nem espingarda tem mais? ---
provocou Norman em resposta.

--- Não sei caçar? --- replicou o velho, para não ficar sem dar a última palavra.
--- Eu consigo acertar estorninhos com espingarda de pederneira, sim, senhor,
daqui mesmo, deitado na cama.

--- Vocês apagaram o fogo? --- interrompeu a patroa com candura, através da
porta do vestíbulo.

--- Sim! --- responderam em coro.

--- Então boa"-noite para vocês.

--- Boa"-noite, patroa.

Seguiram"-se suspiros, arfadas e arquejos, até os roncos tomarem conta do recinto.

Mas Carlsson ainda não conseguia dormir e por um bom tempo ele contou as grades
da janela, para que depois seus sonhos se realizassem.

\oneside

\chapter[Domingo de descanso e domingo de labuta\ldots]{Domingo de descanso\break e domingo de
labuta;\subtitulo{o bom pastor e as más ovelhas;\break as galinholas que tiveram o que\break mereciam
e o capataz\break que conseguiu seu quarto}} 
\hedramarkboth{Domingo de descanso e domingo de labuta}{Strindberg}

\textsc{Quando Carlsson} acordou na manhã de domingo com o cantar do galo, todas
as camas já estavam vazias e as moças já se achavam de saia ao fogão, o sol
brilhava em cheio e sua luz se espalhava pela cozinha.

Carlsson vestiu logo as calças e saiu até a ribanceira para se lavar. Lá
encontrou o jovem Norman, sentado num barril de arenque, enquanto o prestativo
Rundqvist lhe aparava o cabelo, vestido num peitoril limpo, grande como um
jornal, além de estar com suas melhores botas. Junto a uma tina de ferro, que
lhe haviam indicado como lavatório, e usando um pedaço de sabão, Carlsson
pode tomar seu banho de domingo.

Na janela da casa se via o rosto sardento de Gusten, todo ensaboado e em meio a
caretas horríveis, refletidas num pedaço de espelho, conhecido pelo nome de
``obrigação de domingo'', diante do qual ele ia passando a navalha cintilante
para frente e para trás.

--- Vão à igreja hoje? --- disse Carlsson em lugar de bom"-dia.

--- Não, nós não vamos muito à casa de Deus --- respondeu Rundqvist; --- porque só
de ida são vinte quilômetros de remo, e o mesmo tanto na volta, e não se deve
profanar o dia de descanso com trabalho desnecessário.

Lotten saiu para enxaguar as batatas, enquanto Clara ia buscar peixe salgado
no depósito de inverno, ou na vala comum, como era chamada, onde todos
os peixes menores, que morriam na rede ou não podiam mais ser conservados no
viveiro, eram salgados e misturados sem distinção e depois consumidos segundo as
necessidades correntes da casa. Lá havia ruivos brancos, ao lado de escardínios
vermelhos; filhotes de brema, ruffes, lumpos, percas, pequenos lúcios para
fritar, solhas, tencas, lotas, coregonos"-lavareda; todos com algum tipo de
imperfeição: um com a guelra rasgada, ao outro lhe faltando um olho, arpoadas no
lombo, marcas de pisada e assim por diante. Ela apanhou dois punhados de peixe,
sacudiu-lhes o sal e assim a turma foi parar na panela.

Enquanto o desjejum era preparado, Carlsson se vestiu e deu algumas
voltas de reconhecimento pela propriedade.

A casa, que na verdade eram duas casas geminadas, erguia"-se num outeiro que
ficava na extremidade sul e interna de uma enseada bastante funda, adentrando"-se
tanto na terra que não era possível avistar dali o mar aberto, podendo"-se
imaginar que se estava à beira de um laguinho em terra firme. O outeiro acabava em um vale com
pastagens, prados e campinas, cercados por florestas de bétulas, amieiros e
carvalhos. O lado norte da enseada era protegido do vento frio por um monte
coberto de abetos, e a parte sul da ilha era constituída de bosques de pinheiros,
campos de bétulas, pântanos e charnecas, tomados aqui e ali por trechos de
lavoura.

No outeiro, junto a casa, ficava a despensa, e a um trecho dali a casa maior,
feita de madeira e pintada em vermelho, bem grande e coberta de telhas, que o
velho Flod construíra para si mesmo, mas que agora estava desabitada, já
que a viúva não queria morar ali sozinha e também porque a manutenção de tantas
lareiras exauriria desnecessariamente a floresta.

Mais ao longe, em direção ao pasto, ficavam o estábulo e o celeiro; e à sombra
de um bosque de imponentes carvalhos ficavam a sauna e o porão; ao extremo sul da
campina se avistava o telhado de uma ferraria abandonada.

Abaixo, na ponta interna da enseada, havia os depósitos de pescaria e
ferramentas junto ao ancoradouro para os barcos.

Mesmo sem admirar a beleza da paisagem, o conjunto do lugar infundia satisfação
a Carlsson. A enseada abundante em peixes, as campinas planas, as lavouras
férteis de boa posição e protegidas do vento, a densa floresta, rica em lenha,
as belas árvores de madeira valiosa nos pastos, tudo prometia bons rendimentos,
faltando"-lhes apenas uma mão forte para pôr em movimento suas riquezas e extrair
seus tesouros submersos para a luz do dia.

Depois de ter elucubrado aqui e ali, foi interrompido em suas
considerações por um retumbante ``oi!'', vindo das imediações da casa e que
ecoou pela enseada e nos estreitos; logo respondido no mesmo tom pelo celeiro, o
pasto e a ferraria.

Era Clara que chamava para o café da manhã e num instante os quatro
trabalhadores estavam sentados ao redor da mesa da cozinha, onde haviam posto
batata cozida e peixe salgado, manteiga, pão de centeio e aguardente --- já que era domingo. 
A patroa andava em torno recomendando"-lhes bom apetite e vez ou
outra reparava no fogão onde cozinhavam ração para as galinhas e porcos.

Carlsson tomou lugar em uma das cabeceiras da mesa, Gusten escolhera uma das
laterais, Rundqvist a outra e Norman a cabeceira oposta, de forma que ninguém
sabia ao certo quem tinha a posição de honra, e os quatro se viam presidindo a
mesa. Carlsson, porém, parecia conduzir a conversa e acentuava suas
frases com batidinhas de garfo na mesa. Falava de agricultura e
moradia; mas Gusten não respondia ou mudava o assunto para pesca e caça, ao que
Norman se juntava, tendo Rundqvist assumido uma neutralidade maldosa, jogando
uma lenha ambígua na fogueira quando o embate parecia se acender, soprava na
brasa quando esta parecia se apagar, partia para a direita ou aguilhoava pela
esquerda, mostrando aos convivas que eram igualmente tolos e ignorantes
e somente ele tinha discernimento.

Gusten nunca respondia a Carlsson diretamente, voltando"-se para os outros
vizinhos de mesa, e Carlsson viu que dali não podia esperar nenhuma simpatia.

Norman, que era o mais jovem, procurava sempre ter algum respaldo do filho
da casa, pois parecia"-lhe mais seguro contar com sua confiança.

--- Vejam só, estão investindo em porcos, quando não há leite no estábulo, isso
não faz sentido --- proferiu Carlsson ---, e a produção de leite só melhora quando
se semeia o pasto com trevos no outono. A agricultura precisa de
circulação; tem que circular, uma coisa influi na outra.

--- É assim também na pesca, o Norman sabe --- replicou Gusten para o seu vizinho
de mesa --- porque não se pode usar as redes de arenque antes da época de apanhar
solha, e não se apanha nenhuma solha antes que o lúcio tenha acasalado. Uma
coisa como que influi na outra, enquanto um peixe é apanhado, o outro já escapa.
Não é assim, Norman?

Norman assentiu sem reserva e por via das dúvidas seguiu na mesma toada, pois
percebera que Carlsson já preparava um novo ataque:

--- Sim, quando um peixe é apanhado, o outro já se solta.

--- Quem é que se solta? --- interveio Rundqvist com gosto, enquanto Carlsson com
um rabo de escardínio entre os dentes se revirava na cadeira para puxar a
conversa novamente para o seu lado, mas tendo agora que se juntar às risadas
gerais, causadas mais pelo prazer destrutivo de deixar a lavoura de lado do que
por mero divertimento. E, animado pelo sucesso, Rundqvist começou a desdobrar com
comicidade o assunto que achara, para que mais nenhuma conversa séria tivesse
ouvidos.

Quando o desjejum terminou, a senhora convidou Carlsson e Gusten até o
estábulo e os campos, para discutirem os reparos e a distribuição das tarefas
e para que estabelecessem o que deveria ser feito para reerguerem a propriedade;
após o que todos se reuniriam na casa para a leitura do sermão.

Rundqvist se deitou no sofá ao lado do fogão e acendeu seu cachimbo, Norman
tomou o acordeom e se sentou num canto do vestíbulo, enquanto os outros iam para
o estábulo. Lá Carlsson encontrou, não sem certa satisfação, um estado que
superava suas piores expectativas. Doze vacas deitadas comiam musgo e palha, 
pois a forragem acabara. Todas as tentativas de erguê"-las se mostraram inúteis, e
depois de ele e Gusten tentarem colocá"-las de pé com a ajuda de uma tábua, as
vacas foram deixadas à própria sorte.

Carlsson sacudiu a cabeça com gravidade, como um médico que abandona o leito de um
moribundo, mas deixou para mais tarde seus bons conselhos e sugestões de
melhorias.

Todavia os dois touros estavam ainda em pior condição, pois tinham acabado de puxar
os arados da primavera, e as ovelhas só tinham para comer a casca de galhos
cujas folhas havia muito se esgotaram.

Os porcos estavam magros como cães de caça; as galinhas corriam soltas pela
estrebaria, montes de estrume estavam espalhados aqui e acolá e água escorria
sem nenhum controle.

Depois de tudo averiguado e o estado de calamidade decretado, Carlsson deixou
claro que ali não havia mais nada a fazer a não ser o abate.

--- É melhor ter seis vacas que dão leite do que doze magras! --- e, examinando os
traseiros e os úberes das vacas, escolheu com confiança seis que iriam para a
engorda e depois para o abate.

Gusten se opôs, mas Carlsson garantia e asseverava: essas vão morrer! Tão certo
quanto eu estou vivo, essas vão morrer!

Em seguida outras melhorias deveriam ser feitas. Mas antes de tudo mandariam
comprar feno seco de boa qualidade, para só depois soltar os animais na
floresta.

Quando Gusten ouviu que iriam comprar feno, fez as mais vivas objeções quanto a
gastar dinheiro com algo que poderiam conseguir ali mesmo, mas sua mãe o
silenciou dizendo que ele não entendia do assunto.

Após outros preparativos menores, deixaram o estábulo em direção aos campos.

Lá, grandes extensões estavam em pousio.

--- Não, não, não! --- dizia Carlsson inconformado, vendo uma técnica arcaica numa
terra tão boa. --- Que tolice! Ninguém mais no mundo deixa a terra descansar,
e sim a cobre de trevos, pois quando se pode ter uma safra por ano, para que ter
uma somente a cada dois?

Gusten deu seu parecer dizendo que colheitas seguidas exaurem a terra, que, assim como
as pessoas, precisa repousar, ao que Carlsson replicou com uma explanação
correta, ainda que um tanto nebulosa, de como a cobertura de trevos adubava a
terra em vez de exauri"-la, além de deixar o campo livre de ervas daninhas.

--- Isso eu nunca ouvi antes, uma cultura que aduba a terra --- disse Gusten, não
conseguindo compreender a erudita intervenção de Carlsson sobre como as ervas na
maior parte se alimentavam ``de ar''.

Em seguida foram até as valas que se encontravam inundadas pelo lençol freático,
obstruídas de mato e com vazão ruim. As plantações espalhavam"-se irregularmente,
como se tivessem sido semeadas com desleixo, e as ervas daninhas vicejavam
intactas em meio aos torrões. O campo estava abandonado e as folhas do ano
anterior cobriam e sufocavam a grama numa mistura lamacenta. As cercas estavam
quase caindo, não havia pinguelas, tudo estava tão inóspito quanto a viúva tinha
descrito para Gusten na noite anterior. Gusten, entretanto, não dava ouvidos às
eruditas considerações de Carlsson, repelindo"-as como algo desagradável que se
desenterrava do passado, temendo a quantidade de trabalho que se avizinhava e
ainda mais o dinheiro que a sua mãe teria que desembolsar.

Quando seguiram adiante até o pasto dos bezerros, Gusten ficou para trás, e ao
chegarem à floresta, ele tinha desaparecido. Sua mãe chamou por ele, mas sem
resposta.

--- Deixe"-o ir --- disse a viúva. --- Com Gusten é assim, sempre um pouco
desinteressado e cheio de repentes quando não está no mar caçando. Não repare
nele, Carlsson, não é mau rapaz. É que o pai queria algo melhor para ele do
que ser lavrador, deixando"-o livre com seus divertimentos; assim que completou
doze anos, ele ganhou um barco só para ele, bem como uma espingarda, e desde
então não há o que fazer. E agora que a pesca não vai bem, eu tenho que
tomar conta da terra, que afinal é mais seguro do que depender do mar; e
poderia ter dado bom resultado se Gusten mantivesse a ordem entre os empregados,
mas ele sempre se põe ao nível deles, e assim o trabalho não vai para frente.

--- Sim, de fato não é bom mimar os empregados --- decretou logo Carlsson --- e vou
lhe dizer uma coisa, patroa, aqui entre nós, que se é para eu ser o intendente,
então o melhor é que eu faça as refeições na sala de jantar e que eu tenha meu
próprio quarto, ou não terei nenhum respeito e não sairemos do lugar.

--- Carlsson, refeições na sala de jantar talvez não seja possível ---
respondeu"-lhe a patroa um tanto receosa, enquanto passava por sobre a cerca. ---
Hoje, os empregados não aceitam que você coma em outro lugar que não ao lado
deles na cozinha; mesmo Flod não ousou fazer isso, e Gusten nunca toleraria;
haveria reclamações quanto à ostentação e a insatisfação seria geral. Não, isso
não pode acontecer. Mas o senhor dormir em seu próprio quarto, isso já é outra
coisa, com a qual podemos concordar; de resto, todos devem achar que já há gente
demais na cozinha, e creio que Norman prefere dormir sozinho no sofá a dividi"-lo
com outro.

Para Carlsson uma meia vitória já era suficiente e ele deixou uma carta na manga
para mais tarde.

Entraram então na floresta de pinheiros, onde ainda havia restos de neve entre
duas rochas, sujos de terra e folhas caídas; os pinheiros já transudavam resina
ao sol resplandecente de abril e sobre suas raízes floresciam anêmonas azuis,
enquanto sob os arbustos de avelãs e através das folhas mortas se espalhavam as
anêmonas brancas. O musgo exalava um vapor morno; por entre os troncos das
árvores se via o brilho da relva campestre estremecendo por sobre os prados e ao
longe a enseada azulava através da leve brisa; esquilos corriam entre os galhos e
ouviam"-se as marteladas e os pios do pica"-pau.

A senhora Flod caminhava a passos leves pela trilha, por sobre folhas de abeto e
raízes, e quando Carlsson, que a seguia, viu as solas de seus sapatos se
dobrando em passos ágeis, encobertos pelos babados da saia, lembrou"-se de que
no dia anterior ela lhe parecera mais velha.

--- Patroa, a senhora anda com muita leveza --- disse Carlsson, ousando externar seus
sentimentos de primavera.

--- O senhor não deve estar falando sério, parece até que o senhor está fazendo troça de uma pobre
velha.

--- Não, eu sempre digo o que penso --- assegurou"-lhe Carlsson ---, se eu andar no
mesmo passo, ficarei suado.

--- Nós não vamos muito longe, de qualquer maneira --- respondeu a viúva, parando
para tomar ar. --- Aqui o senhor já tem uma visão da floresta, onde os animais
passam a maior parte do verão, quando não estão nos montes.

Carlsson lançou um olhar experiente sobre a floresta e viu que havia ali muitas
braças de lenha e madeira de boa qualidade por extrair.

--- Mas ela também está pessimamente manejada, em toda parte
emaranhados de galhos caídos e copas secas, o que dificulta a passagem tanto
de animais quanto de carruagem.

--- Sim, Carlsson, o senhor mesmo pode ver o estado das coisas, agora tem como
decidir e ordenar o que quiser até deixar tudo certo, disso posso estar segura,
não é, Carlsson?

--- Vou fazer minha parte, se os outros fizerem também a deles, e para isso, patroa, a senhora
terá que me ajudar --- reiterou Carlsson, sentindo em seu íntimo que não seria
fácil galgar uma posição de comando havendo na propriedade costumes tão
arraigados.

Eles retornaram a casa numa ininterrupta conversa sobre como Carlsson conseguiria
alcançar e manter sua posição de superioridade, que ele insistia com a patroa
ser a condição principal para o futuro florescimento da fazenda. Era hora de
ler o sermão, mas não se avistava nenhum dos rapazes. Os dois caçadores tinham
saído pela floresta, e Rundqvist, como de costume, devia ter se escondido num
monte ensolarado, pois era o que ele sempre fazia quando era o momento de se ouvir a palavra
de Deus. Carlsson insistiu na leitura mesmo sem público, e se abrissem a porta
da cozinha as moças poderiam participar em meio às panelas do fogão. A viúva
expressou sua preocupação de não conseguir recitar a contento o sermão e
Carlsson de imediato se encarregou da leitura.

--- Imagine! --- Ele lera tantos sermões na vida, trabalhando na casa do
advogado corregedor, que aqui não haveria de errar.

A senhora apanhou o calendário e procurou na leitura do dia o segundo domingo
depois da Páscoa, que versava sobre o bom pastor. Carlsson tirou a homilia de
Lutero da estante e se sentou numa cadeira bem no meio da sala, de modo que a
congregação pudesse vê"-lo de modo apropriado. Em seguida abriu o livro de salmos e
começou em voz forte, modulando"-a em diferentes tons, tal como escutara os
pregadores, e como já fizera antes, a declamar o texto:

--- ``Naquele tempo Jesus disse para os judeus: `Eu sou o bom pastor: o bom pastor
oferece sua vida pelas ovelhas; mas o mercenário, e não o pastor a quem
pertencem as ovelhas, ao ver o lobo se aproximando abandona as ovelhas e
foge'.''

Uma estranha sensação de responsabilidade pessoal se apoderou do leitor enquanto
proferia as palavras: ``Eu sou o bom pastor'', e olhou gravemente para fora,
através da janela, como se procurasse os dois mercenários fugitivos Rundqvist e
Norman.

A senhora Flod balançou a cabeça em triste assentimento e pôs o gato em seu
colo, como se acolhesse a ovelha desgarrada.

E Carlsson seguiu lendo com voz vibrante, como se ele mesmo o tivesse escrito:

--- ``Mas o mercenário foge.'' Sim, ele foge --- enfatizou ---, ``pois ele é
mercenário!'' --- gritou. --- ``E não cuida das ovelhas.''

 ``Eu sou o bom pastor, e conheço minhas ovelhas, e elas me conhecem'' ---
 continuou de cor, como se fossem palavras da catequese. Em seguida, diminuiu
 a intensidade da voz, cerrou os olhos, como se sofresse uma mágoa profunda pela
 maldade dos homens e suspirou, frisando bem e lançando olhares para o lado, sem
 deixar de subentender algo ardiloso, como se lamentasse denunciar pecadores
 anônimos, mas sem querer estar ele próprio por trás da acusação:

--- ``Eu possuo também outras ovelhas, que não são deste rebanho; a elas também
tenho de pastorear, e que escutem minha voz!'' --- e com um sorriso
esclarecedor, profético, esperançoso e consolador, ele sussurrou:

--- ``E será \textit{um} rebanho, e \textit{um} pastor.''

--- E \textit{um} pastor! --- ecoou a senhora, que tinha seus pensamentos em outro
lugar.

Em seguida folheou a homilia, e sorriu amarelo quando viu que o trecho era
``danado de longo'', mas tomou coragem e começou. O assunto se desviava do
propósito que ele intentara e versava mais sobre os sentidos da simbologia
cristã, logo o interesse não era tão vivo quanto o do texto anterior. Ele aumentou
a velocidade e disparava antes das viradas de páginas, molhando o dedão para
poder virar duas de cada vez sem que a patroa notasse.

Mas quando viu o final se aproximando e suspeitou que o amém estava por perto,
diminuiu o ritmo; porém já era tarde demais, pois na última virada ele
cuspira demais no dedão, apanhando três folhas de uma só vez, dando com o ``Amém'' bem no
começo da página, como se tivesse topado a cabeça numa parede. A patroa
despertou com o susto e olhou sonolenta para o relógio, ao que Carlsson repetiu
o ``Amém'' com alguns floreios de ``em nome do Pai, do Filho e do Espírito Santo
e nosso salvador Jesus Cristo''.

Para arredondar o final e se redimir pelo que tinha pulado, ele disse um
Pai"-nosso com tanto vagar e comoção, que a viúva ainda teve tempo de dar mais um
cochilo, despertando de vez antes do final.

Para não passar o constrangimento de ter que explicar o texto, Carlsson baixou
ainda a cabeça apoiando"-a na mão esquerda numa prece silenciosa, que não podia
ser interrompida.

A senhora também se sentia culpada e quis dar prova de sua atenção opinando
sobre o que tinha entendido da leitura, mas foi interrompida pela incondicional conclusão
de Carlsson que, segundo o texto e as próprias palavras do salvador, não havia
nada além de ``\textit{um} rebanho e \textit{um} pastor! Inexoravelmente
\textit{um}; \textit{um} para todos, \textit{um}, \textit{um}, \textit{um}!''.

Clara gritou chamando para o jantar, e do fundo da floresta ouviram"-se dois
gritos alegres de reconhecimento, acompanhados por tiros de espingarda. Da
chaminé da forja saiu, como se de uma barriga faminta, o grito mais original de
Rundqvist, que ninguém poderia confundir.

Logo em seguida, as ovelhas desgarradas se aproximaram a passos rápidos da
 panela, onde foram recebidas pelas suaves repreensões da patroa por terem se
ausentado; mas nenhum dos inocentes ficou sem resposta, protestando que não
ouviram ninguém chamar por eles, caso contrário teriam vindo de imediato.

Carlsson mantinha uma dignidade dominical à mesa, e Rundqvist, por sua vez, falava com
seu obscuro dialeto acerca dos mais notáveis progressos da lavoura, dando a
entender a Carlsson que já estava inteirado e colaborava com o partido de
oposição.

Depois do jantar, em que foram servidos dois êideres cozidos ao leite com grãos
de pimenta, os homens foram dormir, cada um em seu canto, mas Carlsson apanhou o
seu livro de salmos do baú e se sentou numa pedra da ribanceira, de costas para
a casa, enquanto tirava uma soneca, o que pareceu muito promissor para a patroa
numa tarde de domingo de outra forma ociosa.

Quando Carlsson julgou já ter dado credibilidade à sua devoção por tempo
suficiente, ergueu"-se e entrou na casa pedindo para ver seu quarto. A patroa
quis adiar, primeiro teriam que limpá"-lo e fazer outros arranjos, mas Carlsson
se mostrou irredutível e foi conduzido até o sótão, onde, de fato, no fundo da
armação do teto havia uma pequena peça quadrada cercada por tábuas com uma
janela na cumeeira, fechada com uma cortina de listras azuis. O
quarto era mobiliado com uma cama e uma pequena mesa junto à janela, sobre a
qual havia uma garrafa de água. Nas paredes pendia algo encoberto por lençóis
brancos, parecendo ser roupas, o que visto mais de perto se confirmou, pois 
ali despontava um colarinho de paletó em seu cabide, e acolá se estendia uma
perna de calça. Abaixo havia um monte de sapatos, de homem e mulher, um em cima
do outro, e diante da porta um enorme baú guarnecido de ferro com fechadura de
cobre batido.

Carlsson puxou a cortina e abriu a janela para deixar sair a umidade e o cheiro
de cânfora, pimenta e absinto que impregnavam o ar, pondo o gorro em cima da
mesa foi logo explicando que dormiria muito bem ali, e quando a senhora
demonstrou sua apreensão em relação ao incômodo do frio, ele se disse acostumado
a dormir no frio, uma vantagem impossível na cozinha que era tão quente.

A senhora achou que estavam se precipitando e queria ao menos remover as roupas
para que não pegassem cheiro de tabaco, mas Carlsson garantiu que ali não
fumaria e sugeriu que as roupas permanecessem: ele não se incomodava e não
queria que a patroa tivesse trabalho adicional por sua causa. Iria para a cama
à noite e faria ele mesmo a cama antes de descer, de modo que ninguém mais
entraria no quarto que ele sabia estar abarrotado de pertences aos quais a
patroa dava muito valor.

Com a patroa já convencida, Carlsson subiu sua bagagem e seu cantil de
aguardente, pendurando a camisa num prego junto à janela e pondo as botas ao
lado dos outros pares de sapato.

Depois disso, solicitou uma conversa, na qual Gusten participaria, pois agora
o trabalho seria divido e cada um ganharia seu posto.

Com certa dificuldade, Gusten foi localizado, admitindo por fim sentar"-se um
instante dentro da casa; mas ele não tomou parte nas negociações e respondia às
perguntas somente com objeções, esquivando"-se das dificuldades, e, em resumo, 
opondo"-se a tudo.

Carlsson tentou conquistá"-lo com elogios, assustá"-lo com seu conhecimento
técnico, infundir respeito por sua experiência de mais velho, mas era como nadar
contra a corrente. Por fim, todas as partes se cansaram e Gusten desapareceu
antes que dessem conta.

Nisso a tarde caiu e o sol se pôs em meio à neblina, que logo se tornou mais
densa e cobriu o céu com nuvenzinhas de plumas; o ar continuou morno. Carlsson
saiu vagando pelo campo até chegar ao pasto dos bois; seguiu perambulando
através das aveleiras em flor, que ainda não cobriam a vista por inteiro, e 
criavam um túnel que se estendia até a praia, onde comerciantes costumavam vir
comprar lenha.

Ele parou de repente e pôde ver através dos arbustos Gusten e Norman, que se
encontravam na clareira de uma encosta com espingardas em punho, atentos a tudo
em redor.

--- Silêncio, ele está chegando! --- sussurrou Gusten, alto o suficiente para que
Carlsson escutasse, e, acreditando ser ele o alvo, se escondesse nos arbustos.

Todavia uma ave voou vagarosamente sobre os jovens abetos, lerda feito uma
coruja, com as asas pendentes, e logo em seguida veio mais outra.

O canto das aves cortou o ar, e depois o \textit{pang! pang!} das duas
espingardas, das quais saiu fumaça e saraivadas como rojões.

Os galhos da bétula se dobraram e uma galinhola caiu a uma certa distância de
Carlsson.

Os caçadores correram atrás de sua presa, o que deu motivo para trocarem algumas
palavras.

--- Teve o que mereceu --- disse Norman, manuseando o peito da ave ainda morna.

--- Eu sei de alguém mais, que deveria receber o que merece! --- disse Gusten,
levado por outros pensamentos em meio à febre da caça. --- Imagina que esse
pilantra vai se instalar no quarto também!

--- É verdade? --- exclamou Norman.

--- Isso mesmo, e agora haverá ordem na propriedade! Como se nós não soubéssemos
muito melhor como organizar isto aqui. Mas é assim mesmo, vassoura nova varre
melhor, pelo menos enquanto ainda está nova; mas ele não perde por esperar, vai ver só,
o que vai lhe acontecer!

--- E você ouviu ele dizer que a plantação de trevos pega seu alimento do ar?

--- Sim, do ar; é do meu traseiro que ela pega o alimento!

E assim os dois sabichões davam risada, enquanto Carlsson rangia os dentes atrás dos
arbustos.

--- Ele que venha --- continuou Gusten ---, eu não sou do tipo que se curva diante
de um aproveitador desse! Deixe"-o vir, para ver a dureza que será! Schhh\ldots{}
Lá vem mais um!

Os caçadores recarregaram suas armas e correram de volta para a encosta,
enquanto Carlsson voltava para casa de mansinho, decidido a partir para o ataque
tão logo tivesse tomado as devidas precauções.

De noite, após ter subido ao seu quarto, baixado a cortina e acendido a luz, ele
se sentiu um pouco aflito por estar sozinho; e veio sobre ele um certo medo em
relação às pessoas que ele acabara de deixar. Até então, sempre esteve
acostumado a se sentir o dia inteiro como parte do rebanho, sempre pronto a
ser chamado, sempre tendo acesso a um interlocutor quando era ele próprio quem
queria falar. Agora fazia"-se silêncio ao seu redor, tanto que por hábito ele
esperou que alguém lhe falasse, ouvindo vozes onde não havia; e sua cabeça, que
até o momento externava todos os pensamentos em voz alta, começou a se encher de
um excesso inesperado de ideias em botão, que germinavam e eclodiam,
querendo sair de qualquer forma, deixando um mal"-estar no corpo, tanto que ele não
conseguia se entregar ao descanso do sono.

Levantou"-se e andou de meias para frente e para trás, entre a janela e a
porta do quarto apertado, enquanto repassava todas as suas preocupações com o
trabalho do dia seguinte; organizava as tarefas na cabeça, dividia"-as;
antecipava discordâncias, vencia obstáculos, e depois de uma hora assim
trabalhando, ele tinha paz e descanso em sua mente, que parecia agora detalhada
como um livro"-caixa, no qual todos os cargos estavam em seus lugares e
somados, para que num instante se pudesse entrever a situação.

Em seguida foi para a cama, e ao sentir"-se só em meio aos lençóis limpos e
frescos, sem o perigo de ser incomodado por alguém no decorrer da noite,
sentiu"-se mais dono de si mesmo, como se ele fosse uma muda que agora deitava
suas próprias raízes depois de estar pronta para ser arrancada, deixar sua
árvore"-mãe e viver sua vida por conta própria, lutar por si própria, com grande
esforço, mas também com grande alegria.

E assim ele caiu no sono rumo a mais uma segunda"-feira da vida, e mais
uma semana de trabalho.

\chapter[Em que o capataz põe o trunfo na mesa\ldots]{Em que o capataz põe\break o trunfo na mesa,
\subtitulo{torna"-se senhor do terreiro\break e derruba os jovens valentões}}
\hedramarkboth{Em que o capataz põe o trunfo na mesa}{Strindberg}

\textsc{As bremas} se acasalavam na água, o zimbro despontava, as cerejeiras
floresciam e Carlsson semeava o centeio da primavera onde se perdera plantação
na geada do outono. Ele havia abatido seis vacas, comprara feno seco para as outras seis
que, revigoradas, foram soltas na floresta; consertava e arrumava, trabalhava
por dois e tinha uma capacidade de pôr as pessoas em ação que vencia qualquer
resistência.

Nascido em uma fazenda em Värmland, de pais com pouca iniciativa, mostrou cedo
uma definitiva aversão pelo trabalho braçal, compensada por uma incrível
inventividade em se eximir dessa penosa ``consequência do pecado original''.

Movido por um desejo de ver e conhecer todas as formas de ocupação humana, o que
o levava a não permanecer por muito tempo num lugar sem ter o que fazer, tão
logo ele aprendia o que queria já procurava um novo campo de atividade, e dessa
maneira havia passado do ofício de ferreiro para a agricultura, experimentado
trabalhar como cavalariço, ficara atrás de balcão, foi jardineiro,
foguista, telheiro e por fim vendedor ambulante de livros. Durante todas essas
transformações, ele mantivera um temperamento maleável e uma habilidade de lidar
com todas as situações e todo tipo de pessoa: entendendo os seus desígnios,
lendo os seus pensamentos, adivinhando suas intenções ocultas. Enfim, possuía
uma capacidade acima das condições ao seu redor; e suas diversas habilidades o
faziam mais competente para comandar e administrar do que obedecer a quem lhe
era inferior, servindo de roda numa carruagem que ele na verdade deveria
conduzir.

Alçado à sua nova posição por um acaso, logo se deu conta de que poderia ser de
alguma utilidade, que era capaz de fazer render uma propriedade
desvalida, que em breve seria admirado por isso e, por fim, se tornaria indispensável. Ele
havia assim estabelecido um objetivo para seus esforços e a recompensa que o
esperava numa posição mais elevada era uma firme esperança e uma força que o
impelia. Era visível e inegável que ele trabalhava para os outros, mas ao
mesmo tempo forjava sua própria felicidade, e ao pôr as coisas dessa
maneira, só aparentemente estava dando seu tempo e sua força em proveito
alheio, mostrando ser mais ajuizado que a maioria das pessoas, que bem 
gostariam de agir de modo semelhante, mas não eram capazes disso.

O maior empecilho no seu caminho era o filho da casa. Ele tinha a inclinação dos
pescadores e caçadores pelo incerto, pelas surpresas, e uma decidida má vontade
contra tudo que fosse ordenado e garantido. No seu entender, o
cultivo da terra, na melhor das hipóteses, dava apenas o esperado; nada mais,
e, com frequência, bem menos. Ao se pescar com rede ou arrastão, num dia não se
apanha nada, mas no outro dez vezes mais do que o esperado. Quando se sai para
caçar pardelas pode acontecer de se acertar uma foca; esperando"-se a
metade de um dia nas ilhotas para enganar os mergansos, pode aparecer um
edredão na frente da espingarda; sempre haveria algo diverso
do que se esperava. Uma vez que viera das classes superiores, no geral
a caça ainda era considerada como um privilégio, uma ocupação mais distinta e viril que
puxar o arado ou a carroça de estrume. Isso estava tão entranhado no povo, que
não se podia convencer um moço qualquer a conduzir um par de bois, seja pela
mansidão artificial destes animais, seja pelo fascínio que desde tempos
imemoriais cercava os cavalos.

Outra pedra no caminho era Rundqvist. Ele na verdade era um velho maroto, que 
a seu modo tentara reconquistar o paraíso terrestre, livre do trabalho pesado e
bem provido de sestas após o almoço e longas bebedeiras. Tanto por alegar
saberes ocultos, quanto por uma admitida postura de zombar de tudo que fosse
sério, particularmente o trabalho mais duro, e, em situações de mais aperto,
por meio de uma fingida fraqueza espiritual e dores no corpo, acabava por
atrair a compaixão dos seus iguais, ainda mais quando isso se traduzia num copo
de café com aguardente ou meia libra de tabaco para mascar. De sorte que ele
servia para castrar carneiros e porcos, achava"-se capaz de procurar fontes de
água com varinha de vedor e fazer com que as percas caíssem na rede; curava
variados males menores nos outros, mas conservava os seus; previa bom tempo na
lua nova quando já havia chovido a metade do mês e ofertava as moedas dos
outros sob um penhasco na praia, quando os arenques teimavam em não aparecer.
Mas sabia também fazer várias maldades, dizia, como espalhar erva daninha no
campo do vizinho, fazer com que as vacas parassem de dar leite, conhecia
simpatias e coisas do gênero, o que cercava sua pessoa de certo temor, e
todos preferiam tê"-lo como amigo.

Saber de ferraria e carpintaria era o que realmente o tornara indispensável,
mas sua capacidade de fazer tudo o que chamava a atenção elevava-o como
um perigoso oponente para Carlsson, porque o que este fazia dentro
do celeiro e nos campos não dava tanto na vista.

Assim restava Norman, um bom trabalhador, que tinha que ser conquistado da
poderosa influência de Gusten e ser recuperado para o trabalho regular no campo.

Portanto, Carlsson tinha um bocado de trabalho duro pela frente e, além disso,
precisava pôr em movimento um bom número de ardis, dignos de um chefe de estado,
para poder chegar ao topo. De qualquer modo, ele era o mais esperto e por isso venceria. 

Quanto a Gusten, ele não partiu para o ataque; deixando"-o quieto depois de ter
cooptado com favores seu aliado Norman. E isso não foi muito difícil, pois
Gusten era, francamente, pouco generoso, e durante as aventuras das caçadas
Norman era relegado aos remos, não podendo jamais dar o primeiro tiro. Quando
lhe regalava com uma dose, Gusten tomava três em segredo, de modo que as
vantagens oferecidas por Carlsson no aumento do salário, um par de meias, uma
camisa e outras coisinhas, somado à sua influência crescente, que prometia mais
do que a de Gusten que afundava, logo o conduziu à deserção. Com isso o
prazer de caçar do filho da patroa se apagou um pouco, porque partir sozinho
para o mar não era tão divertido; e na falta de companhia passou a permanecer
junto aos outros no trabalho.

Rundqvist, ele domou com mais dificuldade, pois esse pangaré era ao mesmo tempo
velho e tinhoso. Logo, porém, ele o tinha cercado.

Em vez de deixar uns trocados como oferenda às intempéries, Carlsson mandou consertarem os
barcos e pôr linhas novas nas redes, e vejam, apanhavam mais arenques do que
antes; e em vez de perambular com um ramo de sorveira, à procura de novas
fontes, Carlsson tomou conta da velha fonte, limpando"-a, construindo um
parapeito de madeira e introduzindo lá uma bomba, fazendo assim com que o ramo
de sorveira fosse parar no lixo; em vez de ficar fazendo rezas e fumegar as
vacas, ele mandou escová"-las e espalhar feno seco no estábulo. Se Rundqvist
sabia forjar cravos de ferradura, Carlsson conseguia fazer pregos; se Rundqvist
entalhava um rastelo, Carlsson construía cunha e charrua.

Quando Rundqvist se viu posto para baixo e jogado para fora de sua toca, tomou
para si atividades vistosas. Começou a limpar a casa; livrava"-se das coisas que
as pessoas jogavam fora no depósito por preguiça ou desleixo durante o inverno;
tomava conta das galinhas, do gato e pôs um trinco novo na porta.

--- Rundqvist, tão bonitinho, sem dizer nada pôs um trinco novo naquela porta
velha e quebrada --- escutou Carlsson das moças na cozinha. --- Sim, ele é mesmo
bonzinho.

Carlsson, porém, estava no seu encalço como uma flecha, e assim, numa manhã o
fogão tinha sido alvejado; em outra as tinas d’água estavam pintadas de verde e
exibiam uma fita preta com coraçõezinhos brancos; a lenha foi posta sob uma
cobertura que ele havia feito acima do depósito de madeira, atrás da despensa.
Carlsson aprendera com seu inimigo que o poder estava na cozinha e conquistou"-a
em definitivo com a nova bomba na fonte, tornando"-se então imbatível.

No entanto, Rundqvist era calejado e ardiloso, e numa noite de sábado pintou
a casinha da latrina de vermelho gritante. Carlsson, que estava na espreita,
ofereceu um quarto de garrafa de aguardente para Norman e durante a noite do domingo
da Trindade a patroa escutou passos silenciosos ao redor da casa, mas estava muito
sonolenta para se levantar, vendo só na manhã seguinte que toda a casa estava
pintava de um vermelho reluzente, com os caixilhos das janelas e as biqueiras
brancas! Rundqvist já não tinha mais forças, ainda mais na sua idade,
para continuar uma luta de tal modo esgotante. Agora, todos riam do seu cômico
refinamento de começar seus embelezamentos pela latrina, e Norman, 
traiçoeiro que era, fez uma piada que ficou famosa sobre Rundqvist que era mais ou menos assim:
``Deve"-se começar pelo lado mais importante'', diz Rundqvist, ``pinte primeiro a
latrina''. Este retirou"-se, mas permaneceu de tocaia, para tentar uma vez ainda uma nova
malandragem ou negociar uma paz proveitosa.

Gusten não interferiu, via tudo e aprovou as melhorias. ``Vão sonhando'',
pensou, ``eu virei depois para tomar o que é meu!''.

Contudo, até então não houvera tempo para que o trabalho de Carlsson rendesse
frutos aparentes, já que o dinheiro recebido pelas vacas, que tinha permanecido por
dois dias na escrivaninha, causou uma impressão muito boa no final das contas, mas
logo se evaporou deixando o vazio da saudade atrás de si.

Todavia, o auge do verão vinha chegando. Carlsson tivera muito que fazer e pouco
tempo para passear. Num domingo à tarde, ele por fim saiu pelas redondezas.
Dando uma olhada ao seu redor, deu com os olhos no casarão, deserto com as cortinas
cerradas. Como era curioso, foi em sua direção e tentou empurrar a porta, que
estava aberta. Ele passou pelo vestíbulo e descobriu uma cozinha; seguiu adiante
e entrou numa sala grande, que tinha um ar realmente senhorial; cortinas brancas,
uma cama imperial com ornamentos de metal; um espelho com moldura dourada com
entalhes e um cristal lapidado com facetas --- isso ele sabia que era coisa fina
--- sofá, escrivaninha, lareira de azulejos, tudo exatamente como numa propriedade
senhorial. E do outro lado do vestíbulo, uma sala do mesmo tamanho com fogão,
mesa de jantar, sofás, relógio de parede. Ele ficou tomado de espanto e
admiração, que logo se tornou em lástima e desprezo pela falta de tino
administrativo dos proprietários, quando viu que além de tudo na casa havia
dois quartos com várias camas feitas.

--- Tsc, tsc, tsc\ldots{} --- pensou em voz alta ---, tantas camas e nenhum
veranista.

Embriagado pela ideia do lucro vindouro, ele desceu até a senhora e alertou"-a
para o desperdício de não se alugar aquela casa durante o verão.

--- Minha nossa, como se alguém viesse se hospedar aqui! --- afligiu"-se ela.

--- Como a senhora pode estar tão certa disso? Já tentaram alugar? Já
puseram um anúncio?

--- Isso é jogar dinheiro fora! --- opinou a senhora Flod. 
--- Pode até se tentar, mas aqui não vem nenhum banhista, tenho certeza ---
concluiu a velha, que cessara de acreditar em coisas boas.

Oito dias depois, chegou um distinto senhor que andou pela prado, examinando
tudo ao seu redor. Ele se aproximou e sua chegada foi recebida apenas pelo
cachorro, enquanto as pessoas guiadas pela rotina, timidez ou delicadeza,
escondiam"-se dentro da cozinha e pela casa, após terem anteriormente se
amontoado do lado de fora, de boca aberta, observando o visitante. Assim que
chegou à porta, Carlsson saiu, como se fosse o mais corajoso.

O recém"-chegado havia lido um anúncio\ldots{} ``Ah! sim, é claro, é aqui mesmo!'' E em
seguida foi conduzido à casa maior. Ele estava bastante satisfeito e
Carlsson prometia todas as melhorias, todas, apenas se o senhor se decidisse de
imediato, pois eram tantos interessados e a estação já estava bem avançada. O
visitante, que parecia estar enlevado pela beleza do lugar, apressou"-se em
fechar o negócio, e, depois de recíprocas perguntas indiscretas acerca de
finanças e condições familiares, ele se foi.

Carlsson o seguiu até a porteira e correu depois para dentro da casa, onde na
frente da senhora da casa e de seu filho mostrou sete cédulas de dez do banco
central e uma cédula de cinco de um banco privado.

--- Ah, mas é terrível tirar assim tanto dinheiro das pessoas --- resmungou a
velha.

Já Gusten achou bom; e pela primeira vez presenteou Carlsson com seu
reconhecimento público, quando este contava como havia pressionado o homem com
uma alusão a muitos interessados.

Dinheiro na mesa era o trunfo de Carlsson e ele falava em tom ainda mais alto
depois dessa proeza, na qual lhe servira sua experiência em negócios. Mas não
era apenas o dinheiro do aluguel que caíra sobre eles, choveriam outras vantagens
indiretas, que Carlsson esboçou em traços rápidos para os que o ouviam.

Poderiam vender peixe, leite, ovos, manteiga; e a lenha não se cortaria sozinha,
para não falar nas encomendas em Dalarö, pelas quais cobrariam uma coroa cada.
E poderiam pôr à venda um vitelo, um carneiro, uma galinha que já não dava mais
ovos, batatas e verduras. Ai, ai, ai --- haveria tanto para se fazer e o senhor
que viera parecia mesmo muito distinto.

No dia de São João chegaram as aguardadas galinhas dos ovos de ouro. Eram o
senhor, sua esposa, uma filha de dezesseis anos e um filho de seis, além de duas
criadas. O senhor era violinista da orquestra do rei, gozava de boa situação e
era no mais um homem de paz, já havia passado dos quarenta. Era alemão de
nascença e tinha um pouco de dificuldade para compreender os ilhéus, por isso
limitava"-se a balançar a cabeça em consentimento e dizer ``maravilha'' para tudo
que diziam, com o que logo ganhou a reputação de ser muito bondoso. A esposa
era uma mulher correta e cuidava do seu lar e de seus filhos. Com uma conduta
digna sabia manter as criadas obedientes, sem precisar fazer uma tempestade ou
ter que lançar mão de propinas.

Como Carlsson era o menos tímido e mais falante dos locais, tomou logo conta dos
visitantes, ainda mais por julgar que tinha esse direito, já que havia sido ele quem os
trouxera, e ninguém no lugar tinha a sua disponibilidade nem a simpatia para
tomar"-lhe essa função. Contudo, a chegada dos citadinos não deixou de exercer sua
influência sob os costumes e as mentes dos nativos em geral. Viam diariamente
pessoas com roupas domingueiras, para as quais todos os dias eram um feriado,
flanando, remando sem motivo, pescando sem se importar em apanhar um peixe,
nadando, fazendo música, passando o tempo sem nenhuma preocupação, como se não
houvesse trabalho no mundo. De início isso não despertou nenhuma inveja, somente
espanto pela vida poder tomar essa forma. Admiravam"-se de pessoas que conseguiam
levar uma existência tão agradável, tão calma, tão limpa e acima de tudo bela,
sem que se pudesse dizer que foram injustas ou que tivessem explorado os pobres.
Sem perceberem, de mansinho, os moradores de Hemsö se viram sonhando, lançando
demorados olhares furtivos para o casarão; se entreviam um vestido claro de
verão nos prados, paravam e deleitavam"-se diante da imagem como na frente de uma
obra de arte; se reparavam em um chapéu de palha italiano com véu, uma fita
vermelha ao redor de uma cintura esbelta, num barco na enseada entre os abetos
da floresta, ficavam silenciosos e cheios de devoção, cheios de saudade de
uma coisa que não conheciam, que não ousavam pedir, mas que os atraía.

A conversa e o burburinho da cozinha e da casa assumiram um caráter mais tranquilo;
Carlsson aparecia sempre de camisa branca, limpa, e usava um boné azul,
adquirindo aos poucos um ar de inspetor; tinha lápis no bolso da frente ou atrás
da orelha e fumava constantemente um pequeno charuto.

Gusten, por outro lado, ficou retraído, escondendo"-se ao máximo para evitar ser
objeto de comparação; no geral falava dos citadinos com acrimônia, precisava
mais do que antes se lembrar e mencionar o dinheiro no banco, além de fazer
longos desvios para passar pelo casarão e os claros vestidos.

Rundqvist permanecia com ar lúgubre e na maior parte do tempo se escondia na ferraria,
dizendo que não se importava uma pinoia, nem que a rainha tivesse chegado,
enquanto Norman achou seu chapéu de recruta, aparecendo com cinturão sobre a
camisa e dando voltas em torno da fonte, onde as criadas dos inquilinos
costumavam ir pelas manhãs e ao anoitecer.

Era pior para Clara e Lotten, que logo viram todos os homens infamemente
sucumbindo às criadas dos inquilinos, intituladas ``senhoritas'' nas
cartas que recebiam, além de usarem chapéu quando iam de barco para Dalarö.
Enquanto isso, as moças da ilha tinham que andar descalças, pois não queriam
expor suas botas à sujeira do celeiro e arruiná"-las de vez, além de estar quente
demais no campo e na cozinha para se andar calçado. Usavam seus vestidos
desbotados e, por conta do suor, da sujeira e do debulho, não podiam nem ao menos
uma vez usar fita branca, e Clara, que fez uma tentativa com mangas de
renda, se deu mal e foi logo descoberta e castigada com incessantes piadas sobre
sua audácia em competir. Mas aos domingos, recuperavam"-se da derrota
manifestando uma vontade de ir à igreja, como não se via há anos, só para
poder vestir suas melhores roupas.

Carlsson ia sempre visitar o professor, perguntando se estava tudo em ordem, e
aparecia com frequência na varanda quando alguém lá estava, para perguntar da
saúde, prever bom tempo, propor passeios, dar conselhos e dicas de pesca, e, vez
ou outra, aceitar um copo de cerveja ou conhaque, tornando"-o logo objeto de
acusações a meia"-voz de que estava filando comida.

No sábado, quando foi o dia de levar a criada da família para fazer compras em Dalarö,
surgiu uma disputa sobre quem a levaria. Carlsson resolveu a questão de pronto
a seu favor, porque a pequena, de cabelos pretos e louvada pela brancura de sua
tez, o tinha tocado profundamente; e quando a patroa se opôs que Carlsson fosse,
sendo ele o mais importante empregado e administrador da fazenda, não podendo se
ocupar de pequenos serviços, Carlsson respondeu dizendo que o professor em
pessoa pediu que ele fosse, pois cartas importantes tinham que ser levadas ao
correio. Gusten, que mesmo tentando esconder, estava desejoso de fazer o
serviço, considerou que podia muito bem levar as cartas em mãos, ao que Carlsson
explicou decidido que estava fora de questão se permitir ao dono da fazenda
cumprir obrigações de criado, o que poderia depois gerar comentários. E assim se
decidiu.

Fazer as compras em Dalarö não deixava de ter suas vantagens, o que o astuto
capataz já tinha previsto. Primeiramente, a de se poder ver o mar ao lado de uma
moça e jogar conversa fora sem ser importunado; depois vinham as refeições por
conta e as gorjetas; e em Dalarö teria a oportunidade de estabelecer boas
relações com os comerciantes ao trazer uma cliente nova, o que sempre se traduzia
em tapinhas nas costas, um trago aqui, um charuto ali, além de trazer um brilho
novo à sua reputação, por tomar conta dos assuntos do professor e andar bem
vestido em dia de semana na companhia de uma senhorita de Estocolmo.

Contudo, as idas para Dalarö só aconteciam uma vez por semana e não era um
empecilho ao bom andar do trabalho, pois Carlsson era cuidadoso o bastante para
nos dias em que estava ausente mandar os rapazes para trabalhar por empreitada,
para que abrissem tantas braças de fossos, que arassem tal parte dos campos, que
derrubassem tantas árvores, para que depois estivessem dispensados, o que eles
aceitavam de bom grado, pois dessa maneira podiam estar livres já de tardinha.
Nessas ocasiões, quando o trabalho era calculado e tudo que tinha sido feito
examinado, o lápis e o recém"-introduzido caderno de anotações passaram a ser
respeitados e Carlsson se acostumou a atuar como inspetor, conseguindo
aos poucos que o trabalho recaísse sobre os ombros dos outros. Ao mesmo tempo,
acomodou"-se no sótão como num quarto de solteiro. Desde muito que começara a
fumar lá dentro e na mesa junto à janela ele arrumara um pequeno tinteiro,
caneta, lápis e algumas folhas de papel de carta, mais um candelabro e um
cinzeiro, fazendo"-a parecer mesmo uma escrivaninha. A janela dava para a casa maior,
onde ele permanecia nos momentos de pausa, observando os movimentos dos
inquilinos e exibindo a sua capacidade de escrever. Às tardes ele abria a
janela, apoiava o cotovelo no peitoril e ficava ali soltando baforadas do seu
cachimbo ou de um toco de charuto que guardara no bolso do paletó, enquanto lia
um semanário, o que lhe dava uma aparência, visto de baixo, de ser realmente o
patrão da fazenda.

Quando escurecia, ele acendia a luz e se deitava na cama fumando. Era então que
vinham os sonhos, ou melhor dizendo, os planos, construídos sob circunstâncias
ainda não ocorridas, mas que poderiam se tornar realidade apenas com um pouco de
boa sorte.

Uma noite, quando estava assim deitado, soltando baforadas do seu
cachimbo para espantar os mosquitos, seus olhos se detiveram no lençol
branco, que encobria as roupas, e que de repente se desprendeu e caiu. Como a
sombra de uma fileira de soldados viu todo o guarda"-roupa do falecido
senhor Flod marchar ao flanco da parede: até a janela 
e de volta à porta, conforme a brisa fazia
tremeluzir a chama da lâmpada, dando"-lhe a impressão de ver o defunto em todos
os vultos que os trajes delineavam no papel de parede quadriculado. Num momento passava
de paletó azul de lã e calças de bombazina, que ele usara quando se sentava ao
leme junto ao viveiro, quando velejara até a cidade levando peixe, depois se
sentando junto ao mastro do parque da cidade bebendo \textit{toddy}\footnote{ Bebida alcoólica, servida quente, 
em cuja composição entram variados ingredientes: destilados, água, açúcar e especiarias.} 
com os comerciantes de peixe; noutro vinha vestido com sobretudo preto, calças longas e
folgadas, como o outro se vestira para as funções da igreja, casamentos,
enterros e batizados. Agora punha uma camisa de couro de carneiro escura,
que ele usara no outono e na primavera, quando puxava rede na praia; logo depois 
exibia um grande casaco de pele de foca que conservava ainda manchas da festa
do Natal, com respingos de vinho quente. Uma cinta de viagem tricotada com
fios de lã verde, amarelo e vermelho serpenteava pelo chão como uma grande cobra
do mar com a cabeça enfiada num cano de bota.

Carlsson se sentiu mais aquecido por dentro da camisa, quando se imaginou
vestido com as confortáveis peles, macias como seda, vendo"-se conduzido num
trenó sobre o gelo, com um gorro despretensioso, indo visitar os vizinhos, que
recebiam convidados para o Natal com fogueiras na praia e tiros de espingarda
para o alto, e assim, entrando na casa aquecida, retirando o sobretudo e
aparecendo de casaca, saudado pelo pastor com familiaridade e convidado a
sentar"-se no lugar mais importante, na cabeceira da mesa, enquanto os criados
esperavam à porta ou se acotovelavam à janela.

As imagens das desejadas honrarias se tornaram tão vívidas que Carlsson se pôs
de pé e antes que percebesse, já havia se metido no casaco de pele e passava a
mão sobre suas mangas; seu corpo fremia com as cócegas que a gola lhe fazia na
bochecha. Em seguida vestiu o sobretudo e o abotoou; pôs seu espelho de barbear
na cadeira e examinou como o casaco lhe caía nas costas; deixou as mãos em
frente à lapela e andou pela sala. Um sentimento de riqueza espalhou"-se das
roupas acetinadas; algo folgazão, muito sobranceiro; a aba do sobretudo abriu"-se
quando ele se sentou na ponta da cama, fazendo de conta que estava visitando
alguém.

Enquanto estava sentado desse modo, tomado por sonhos inebriantes, escutou
vozes alegres de fora, e quando prestou atenção, percebeu a voz de Ida --- que
era a bela cozinheira --- e a voz de Norman se entrelaçarem, caírem uma sob a
outra, lado a lado como se estivessem se beijando. Aquilo lhe foi como uma
aguilhoada e num instante ele tinha devolvido o sobretudo e o casaco de 
pele para seus cabides atrás do lençol. Armando"-se com um charuto 
recém-aceso, ele desceu pela escada.

Como estava tão ocupado com importantes planos para o futuro, Carlsson tinha até
então evitado se envolver com as moças, pois sabia quanto tempo era preciso 
destinar a tais afazeres, e bem consciente de que, no instante que
partisse para o combate nessa direção, não poderia estar seguro de não oferecer
um ponto vulnerável, difícil de defender, sabendo ainda que se fosse atacado nesse
terreno, seria o fim da sua reputação e autoridade.

Mas agora que estavam competindo pela aclamada beldade, que era um prêmio
considerável a se ganhar, sentiu"-se instado a mostrar as garras e cantar de
galo no terreiro. Ele foi até a ribanceira, onde o jogo já estava bem adiantado. O
que lhe pareceu irritante foi ter que se medir logo com Norman; se ao menos
fosse o Gusten; mas aquele insignificante Norman! Ah, ele ia ver!

--- Boa tarde, Ida! --- começou, ignorando seu rival, que deixou seu lugar na
cerca com má vontade, logo ocupado por Carlsson.

E assim se meteu no jogo, utilizando"-se de sua lábia superior, enquanto Ida ia
enchendo com gravetos o cesto de lenha, e deste modo Norman ficou fora da
conversa. Ida, no entanto, era inconstante como a Lua e às vezes dirigia
algumas palavras para Norman, que Carlsson tomava para si, as devolvendo de
forma ornamentada e empolada. Mas a bela, que se divertia com a disputa, pediu
para que Norman lhe cortasse alguns gravetos, e antes que o felizardo pudesse
dar a volta na cerca, Carlsson já havia pulado sobre a grade afiada, puxado seu
canivete e partido um galho seco de abeto em gravetos; e num par de minutos
encheu o cesto de gravetos, carregando"-o com o dedo mindinho direto para a
cozinha, seguido por Ida, e lá permaneceu junto à porta, estirando o corpo para
não deixar ninguém entrar ou sair. E Norman, que não conseguiu achar nada para
fazer, deu algumas voltas em torno da ribanceira, pensando com melancolia na
fácil ascensão daquele sem"-vergonha, até que achou por bem ir até a fonte e dar
vazão a suas mágoas com um \textit{schottisch}, que ele puxou do acordeom.

As lânguidas notas das palhetas do acordeom cortaram o ar pesado da tardinha,
atravessando a porta e atingindo o trono da misericórdia junto ao fogão da
cozinha, pois Ida lembrou"-se nesse instante de que precisava ir à fonte
buscar água para o professor. Carlsson foi junto, dessa vez um tanto inseguro
com o combate que se travaria num campo que lhe era completamente estranho. Para
tentar desfazer esse ``canto de sereia'', ele se pôs a carregar o cantil de
cobre de Ida, sussurrando gentilezas à moça com voz macia, na tessitura mais
acariciante e sonora que pudesse, como que pondo letra na melodia sedutora e
transformando o solo num mero acompanhamento, mas assim que chegaram à fonte,
escutou"-se a voz da patroa chamando da casa. Ela chamava Carlsson, e pelo tom de
voz apercebia"-se que era coisa urgente. Este, primeiro ficou com raiva e
pensou em não responder, mas eis que Norman foi tomado por um espírito maligno e
com voz esganiçada ele gritou:

--- Aqui, patroa! Ele já vai!

E desejando mil vezes o inferno para o falso sanfoneiro, o conquistador teve que
se arrancar dos braços do amor e deixar sua presa quase vencida para o mais
fraco, que só podia agradecer ao acaso pela sua sorte.

A patroa chamou"-o mais uma vez, e com a voz mal"-humorada Carlsson respondeu que já
estava indo o mais rápido que podia.

--- Carlsson, queira entrar, vamos tomar um cafezinho --- disse a velha
recebendo"-o no vestíbulo, cobrindo os olhos com a mão para enxergar contra a luz
suave do crepúsculo de verão para saber se ele vinha sozinho.

Carlsson, que em outras circunstâncias teria adorado tomar um cafezinho, nesse momento
não queria saber de café nem aguardente, mas não tinha como recusar.
Acompanhado pelo som provocador e triunfante da ``Marcha de Caçador de
Norrköping'', que o bem aventurado Norman puxava do fole, ele teve que entrar na
casa. A senhora Flod estava mais amável do que o normal, já Carlsson achou"-a
mais velha e feia do que o normal; e quanto mais gentil ela ficava, mais
carrancudo ele ficava, o que por fim deixou a velha quase terna.

--- Pois sim, Carlsson ---, começou ela, enquanto servia o café --- é que na semana
que vem vamos convidar as pessoas para ceifar os campos e por isso queria lhe
falar primeiro.

Nesse instante o acordeom se calou em meio aos acordes lânguidos do trio e
Carlsson ficou petrificado e ausente, mastigando algumas palavras secas sem
muito sentido:

--- Ah é, isso mesmo, semana que vem temos que ceifar!

--- Então eu queria --- continuou a senhora --- que o senhor fosse no sábado com a
Clara para convidar as pessoas; isso também para que o senhor saísse um pouco 
e fizesse umas visitas, para se fazer mais presente, o que é sempre bom.

--- Sim, mas no sábado eu não posso --- respondeu Carlsson com brusquidão ---,
porque vou até Dalarö a serviço do professor.

--- Ao menos uma vez Norman poderia fazer a encomenda --- replicou a senhora Flod
virando as costas para o capataz, evitando olhar em seu rosto.

Foi quando se escutou o acordeom soltando doces frases, entremeadas por pausas,
que pareciam se distanciar e esmorecer noite adentro, onde o passarinho
noitibó zumbia como se estivesse na roca a fiar.

Carlsson, suando o suor da morte, engoliu seu café com aguardente. Sentia pontadas
no peito, a cabeça lhe enevoava e uma fraqueza tomou conta de seus nervos.

--- Norman não, --- instou ele --- Norman não consegue cumprir com todas as
exigências do professor, e\ldots{} e\ldots{} ele não é confiável.

--- Mas eu perguntei ao professor --- interrompeu a senhora --- e ele disse não
precisar de nada neste sábado.

Estava acabado para Carlsson; a patroa o havia apanhado feito um camundongo e
não havia mais nenhum buraco por onde escapar.

Seus pensamentos estavam em outro lugar, tanto que nem podia oferecer
resistência. Isso a senhora Flod percebeu, e aproveitou a situação para enredar
o infeliz.

--- Escute, Carlsson --- disse ela ---, espero que você não se ofenda, se você me
permitir lhe dizer uma coisa, que é para o seu bem.

--- Com os diabos, patroa, a senhora pode falar o que quiser, agora não faz mais diferença ---
exclamou Carlsson, ouvindo os sons ternos do acordeom soando ao longe e
desaparecendo nos prados.

--- Só queria dizer que você devia deixar de flertar com as moças, que
no final isso só traz confusão; sim, eu sei como é, eu entendo disso; e falo para o seu
próprio bem, Carlsson. Essas moças da cidade têm sempre um batalhão de
rapazes atrás de si, que para conseguirem alguma coisa têm que fazer uma vontade aqui,
uma criancice ali\ldots{} se elas vão ao bosque com um, vão para o prado com outro, e
quando algo dá errado, escolhem aquele que é mais bobo para sustentá"-las.
Mas é assim mesmo.

--- Que seja, para o inferno com o que os moços fazem.

--- Não fique assim --- consolou a senhora Flod. --- Se um moço como o senhor pensar em se
casar, não vai correr atrás de coquetes e coisa parecida; aqui no arquipélago
tem muita moça de boa família, posso lhe dizer, e se o senhor for ajuizado e
conduzir bem seus afazeres, vai conseguir tudo que deseja, antes do que imagina.
Por isso o senhor não deve ser teimoso, mas fazer o que lhe digo, quando lhe pedi
para ir convidar as pessoas para a ceifa. Lembre"-se que eu não iria pedir para
qualquer um fazer visitas em nome da fazenda, e mesmo sabendo que meu filho vai
se opor, não me importo, porque quando tenho alguém do meu lado, dou o
meu apoio, pode acreditar.

Carlsson começou a se acalmar, pressentindo que seria vantajoso poder falar em
nome da propriedade; mas estava ainda muito irritado para querer trocar sua
paixão por uma coisa incerta e sentia necessidade de ter uma contrapartida,
antes de se deixar convencer.

--- Eu não posso ir desse jeito, sem ter roupas apropriadas --- tentou ele para
cima da patroa.

--- Não acho que o senhor esteja tão mal quanto a isso --- disse a senhora Flod
---, mas se for só isso, podemos arranjar uma solução.

Carlsson não quis continuar naquela direção, em vez disso decidiu trocar a
promessa deixada pela metade por outra, conseguindo depois de vários argumentos
que Norman, imprescindível no trabalho de afiar as foices e
na reforma do paiol, ficasse em casa, enquanto Lotten faria as compras em
Dalarö.

 \asterisc

Eram três horas da manhã num dia do começo de julho.\footnote{ Durante o 
verão escandinavo, as noites se encurtam, chegando a desaparecer 
acima do círculo polar, no fenômeno chamado "sol da meia"-noite". Portanto, 
a essa hora, onde se passa a história, o sol já raiara.} Já escapava fumaça da
chaminé, e a cafeteira estava ao fogo; todos já estavam de pé e ao longo da
ribanceira havia uma longa mesa posta para o café. Os convidados para a ceifa,
que haviam chegado na noite anterior, dormiram nos sótãos do celeiro e da
estrebaria, e doze rapazes imponentes das ilhas vizinhas, vestidos de camisa
branca e de chapéu de palha, estavam em grupos do lado de fora da casa armados de
foices e amoladeiras. Estavam Åvassan e Svinnockarn, já velhos com as costas
emborcadas de tanto remar; Aspön com sua barba de guerreiro, uma cabeça mais
alta do que todos os outros e com seus olhos profundos e tristes da solidão do
mar, de tristezas sem nome e sem queixas; Fjällångarn, anguloso e retorcido
feito um pinheiro"-anão crescido no mais longínquo dos
escolhos; Fiversätraön, magro, curtido, vívido e seco feito um filtro de café de
pele de peixe; os de Kvarnö, reputados construtores de barco; os de
Långviksskär, os melhores caçadores de foca, o camponês de Arnö com seus filhos.
E ao redor deles, as moças andavam com passos leves trajando blusas
de linho, com xales no colo, vestidos claros de algodão e com lenços na cabeça;
elas tinham trazido cada uma seu ancinho, recém"-pintados com as cores do
arco"-íris, e elas pareciam mais estar indo para uma festa do que para o
trabalho. Os velhos as abraçavam pela cintura e conversavam com familiaridade,
já os rapazes, a essa hora ainda mantinham distância, esperando o lusco"-fusco da
noite, a dança e a música para os jogos de amor. O sol já havia despertado
fazia quinze minutos, mas ainda não ultrapassara o topo dos pinheiros para
lamber o orvalho da grama; que se estendia feito um espelho, engastado pelo
pálido junco esverdeado da enseada, onde os filhotes de pato piavam em meio às
grasnadas dos mais velhos. Lá embaixo, gaivotas velejantes,
grandes, de asas largas, brancas como a neve, como os anjos de gesso da igreja, pescavam abletes;
num grande carvalho as pegas acordaram fazendo uma algazarra e bisbilhotando
sobre o tanto de pessoas que viram ao redor da casa; o cuco cantava no prado,
exasperado, furioso, pois a estação do desejo estava se findando com a iminente
aparição dos feixes de feno; o codornizão gorjeava no campo de centeio; e na
ribanceira o cachorro corria e se alegrava por reencontrar velhos conhecidos. A
luz do sol brilhava nas mangas de camisas e nas fitas de linho, espalhando"-se
sobre a mesa com o café, onde xícaras e travessas, copos e jarras tilintavam
no decorrer da refeição.

Gusten, em geral tímido, fazia as vezes de anfitrião, e sentindo"-se
seguro em meio aos amigos de seu pai, pôs Carlsson de lado, servindo ele próprio
a aguardente. Mas Carlsson já os conhecia das visitas que fizera para
convidá"-los e se sentia em casa como um agregado mais velho, não se fazendo de
rogado. Com dez anos a mais que Gusten, com uma aparência mais madura e viril,
era fácil deixar o rapaz em posição de desvantagem, que sempre seria uma criança
aos olhos dos homens que foram próximos de seu pai.

Assim tomaram café, com o sol se levantando no céu, e os veteranos se puseram em
movimento rumo à campina com foices nos ombros, seguidos pelos mais jovens e
pelo grupo de moças.

O mato estava na altura da coxa e encrespado como um pelame, tanto que Carlsson
teve que explicar para os recém"-chegados sobre o recente manejo da campina; de
como ele mandara capinar as folhas e ervas do ano anterior, como ele aplainara
as tocas de toupeira, semeara nas manchas de geada e espalhara adubo líquido
sobre a terra. Em seguida, delegou funções como um capitão à sua tropa, deu
posições de honra para os mais velhos e respeitáveis e tomou seu posto na
retaguarda para que ele não sumisse dentro da turba. E a batalha teve início: duas
dúzias de homens em camisa branca, dispostos em cunha feito cisnes migrando no
outono, com as foices varrendo de canela a canela, vindo atrás as moças com seus
ancinhos, espalhadas feito uma revoada de andorinhas, inesperadamente se
afastando, se aproximando, e se mantendo mesmo assim unidas, cada uma seguindo
um rapaz que ceifava.\EP[-1]

As foices sibilavam e as ervas orvalhadas caíam aos montes; e lado a lado jaziam
todas as flores do verão, que ousaram sair da floresta e dos campos: margarida,
cizirão, flor"-de"-fel, aljofareira, erva"-coalheira, cerefólio, cravina,
trigo"-de"-vaca, ervilhaca, petasite, trevos e todas as ervas do campo; e no ar se
espalhava um cheiro doce de mel e especiarias, abelhas e zangões fugiam em
enxames do bando assassino, as toupeiras enfiavam"-se nas entranhas da terra
quando escutavam como ribombava em seus tetos frágeis; a cobra"-d’água 
precipitou"-se amedrontada para a represa e atirou"-se num buraco feito uma
ponta de escota; acima do campo ceifado esvoaçava um par de cotovias,
cujo ninho havia sido pisado por uma sola de sapato; estorninhos seguiam a
tropa com passinhos curtos, ciscando todo tipo de bichinho que agora aparecia à
luz do sol abrasador.

A primeira batalha estendeu"-se até a borda do campo semeado e agora os
guerreiros paravam, apoiando"-se nos cabos de suas foices, admirando a obra de
devastação que tinham deixado para trás, secando o suor debaixo dos chapéus, e
se servindo de uma nova dose de fumo de mascar que traziam em caixinhas de
latão, enquanto as moças se apressavam para alcançar a linha de frente.

E então eles dispararam novamente, em meio a um mar esverdeado de flores, em ondas
matizadas ao sabor da crescente brisa da manhã, que ainda exibia uma gama de
cores brilhantes, enquanto os firmes caules das flores e suas copas despontavam
das doces ondas de ervas, protegendo"-se das lufadas de vento e se espraiando num
verde liso como um mar em calmaria.

Paira sobre o trabalho um ar de festa e competição, e prefere"-se cair por
terra a largar da foice. Carlsson consegue que a criada do professor, Ida,
seja a sua ajudante ao ancinho, e como ele segue por último na ceifa, pode 
dirigir algumas palavras a ela sem se expor ao perigo; já Norman ele o
mantém à frente, na diagonal, sob um severo controle, e logo que este ensaia
um olhar enlevado a sudeste, Carlsson mantém a foice no seu encalço e
ouve um aviso vindo de trás, mais inamistoso que bem intencionado: ``Você aí,
cuidado com a canela!''.

Às oito horas o prado ao redor da fonte já parece um campo arado, liso como uma
mão e com grama em longas fileiras; agora admiram e inspecionam a ceifa, e
Rundqvist recebe o pior julgamento, já que se pode ver onde ele tinha ceifado, 
parecendo que por ali tinham passado elfos dançando. Rundqvist porém se 
defende dizendo que não conseguiu tirar os olhos da moça que lhe coube, 
pois que não era todo dia que ele tinha uma moça correndo atrás de si. 

Nesse momento, Clara grita de cima anunciando o almoço; a garrafa de aguardente brilha
ao sol e o jarro de refresco está posto; as batatas cozidas fumegam nas
travessas, os arenques embaçam as tigelas, a manteiga está disposta, o pão
fatiado; os tragos são entornados e o almoço começa.

Carlsson recebeu elogios e está embevecido da vitória; Ida também lhe concede
seus favores e ele a corteja com uma atenção visível, mesmo porque é a mais
bela presente. A senhora Flod, que se apressa para cá e para lá com travessas e
pratos, passa amiúde próximo deles, o suficiente para atrair a atenção de Ida,
mas não a de Carlsson, até que ela o cutuca com o cotovelo e cochicha:

--- Carlsson, o senhor também é anfitrião e deve ajudar Gusten; aqui é como se
fosse a sua casa!

Carlsson só tem olhos e ouvidos para Ida e responde à patroa com uma graça. Mas
eis que chega Lina, a babá da família do professor, e lembra Ida de que ela deve
voltar e arrumar a casa. Os homens se queixam e fazem um alvoroço, já as moças
aparentam entristecer"-se apenas o suficiente.

--- Eu vou ficar sem uma moça para recolher a palha atrás de mim? --- exclama
Carlsson, afetando embaraço para ocultar sua indignação.

--- A patroa não pode fazer isso? --- responde Rundqvist, que diziam ter olhos nas
costas.

--- A patroa vai ajudar! --- gritam os rapazes em coro. --- Tragam um ancinho para a
patroa!

A senhora Flod se esquiva atrás do avental:

--- Jesus, uma velha no meio das moças! Não, de jeito nenhum, jamais! Ah, vocês
são loucos!

Quanto mais se opunha, mais a queriam na ceifa.

--- Leve a coroa --- sussurra Rundqvist a Carlsson, enquanto Norman incentiva"-o, e
Gusten se põe sombrio feito a noite.

Não havia escolha, e debaixo de risadas e gritos Carlsson corre para dentro da
casa à procura do ancinho da patroa, escondido em algum lugar do sótão; com esta
lhe gritando atrás:

--- Não, pelo amor de Deus, ele não pode ir lá e mexer nas minhas coisas! --- E
assim os dois desaparecem por entre comentários jocosos dos que permaneceram
embaixo.

--- Parece"-me --- diz Rundqvist, quebrando o silêncio que se seguiu ---, parece"-me
que eles estão demorando demais! Vá lá e veja o que aconteceu, Norman.

A retumbante aclamação geral encoraja"-o a prosseguir:

--- O que será que eles estão fazendo lá em cima? Não, isso não pode
continuar; assim eu fico preocupado, alguém pode me dizer o que está acontecendo? --- Gusten ficou com
os lábios levemente azulados, obrigando"-se a rir para acompanhar os outros.

--- Deus me perdoe --- continuou Rundqvist no mesmo tom ---, mas não aguento mais,
se me dão licença, vou lá ver o que estão fazendo.

Nesse instante, Carlsson e a patroa saem pela porta do vestíbulo portando o
ancinho que tanto procuravam. Era uma ferramenta vistosa, com dois corações
pintados, além de um ``anno 1852'', tendo sido outrora seu ancinho de noiva, que
ninguém menos que Flod havia manufaturado, e que tinha ervilhas secas dentro da
ponta do cabo, para servirem de guizo. A lembrança das alegrias do passado
infundiu na senhora Flod um novo ânimo, e sem nenhum traço de sentimentalismo
exagerado apontou para a data, e disse:

--- Não foi ontem que Flod fez esse ancinho\ldots{}

--- E que você teve sua lua"-de"-mel, patroa --- completou Svinnockarn.

--- Pode muito bem ter outra! --- disse Åvassan.

--- Duas coisas em que não me fio: leitão de seis semanas e viúva há dois anos ---
opinou Fjällångarn.

--- Quanto mais seca a palha, mais fogo ela pega! --- provocou Fiversätraön.

E cada um punha mais lenha na fogueira, e a senhora apenas sorria e se
esquivava, mantendo uma expressão boa no rosto e fazendo brincadeiras também,
porque não valia a pena se aborrecer.

E assim desceram para ceifar a turfa, onde a cárex e a cavalinha vicejavam como
uma floresta, e a água chegava até o cano das botas dos homens. As moças, no
entanto, penduraram suas meias e sapatos na cerca.

E a viúva seguia com seu ancinho, atrás de Carlsson e à frente das outras; e
inúmeras piadas caíram por sobre os dois jovenzinhos, como estavam sendo
chamados.

Veio a tarde e veio a noite. O violinista tomou seu lugar, o celeiro foi
arrumado e varrido e a madeira lambuzada de piche para não soltar farpas. E
quando o sol se pôs, começou a dança.

Carlsson abriu o baile com Ida, que trajava um vestido preto com decote
quadrado, com franja branca e colarinho à Maria Stuart, distinguindo"-se entre as
moças do lugar como uma dama invejada, despertando temor e arrepios nos mais
velhos, e desejo nos rapazes.

Carlsson era o único que sabia a nova valsa, e por isso Ida preferia dançar com
ele repetidas vezes, depois de uma tentativa frustrada de dançar uma valsa de
três passos com Norman, ao que este, derrotado pelo seu rival, teve a infeliz
ideia de apelar para o acordeom, tanto para verter as mágoas do seu coração como
para tentar uma última armadilha para aprisionar aquela ave esplêndida e
volúvel, que acreditava ter nas mãos algumas semanas antes, mas que logo mudou
de ideia e agora beijava outro. Carlsson, no entanto, achava que o
acompanhamento era desnecessário, ainda mais por ter contratado um músico de
verdade, e o acordeom pesaroso realmente não seguia o passo do ágil violino, pelo
contrário, atrasava"-lhe o compasso e desordenava a dança. Instigado
por uma boa oportunidade de embaraçar seu rival, e como frases sobre a
improcedência do acordeom pareciam estar na ponta da língua de todos ao redor,
Carlsson encheu a voz e gritou bem no meio do celeiro para o amante infeliz,
encolhido no seu canto:

--- Ei! Largue aí esse saco de couro, e vá la fora esvaziar seus ventos, já que
parece estar tão inflado deles!

A concordância geral caiu em cheio sobre o desgraçado sob a forma de risadas de
escárnio, mas a bebida já subira à cabeça de Norman, e encantado de tal maneira 
pelo decote de Ida que lhe veio uma força inaudita, ele não se sentia nem um 
pouco inclinado a recuar do desafio.

--- Ei! --- gritou imitando Carlsson, que havia deixado escapar seu sotaque,
sempre ridículo para os suecos de Uppland. --- Vamos lá para a ribanceira, pois
hoje vou dar uma coça num porco!

Carlsson não julgou a situação tão ameaçadora, a ponto de partir para os punhos,
e se manteve ainda na região mais inofensiva das provocações.

--- E que porco tão notável é esse, que está merecendo uma coça?

--- É um porco da província de Värmland, isso eu posso dizer! --- respondeu Norman.

Ferido no seu orgulho regional, e ainda procurando no último instante uma ofensa
que não lhe veio à mente, Carlsson foi para cima do inimigo, pegou"-o pelo colete e o
arrastou até a ribanceira.

As moças se posicionaram ao redor da porta para assistir à contenda e não
ocorreu a ninguém se interpor entre os dois.

Norman era baixo e atarracado, Carlsson, porém, era mais firme e maduro. Ele
tirou o paletó, a que dava muito valor, e os lutadores se atracaram; Norman com
a cabeça atrás da guarda, como aprendera com os timoneiros; mas Carlsson o
agarrou, desferindo"-lhe um chute bem feio no meio de suas pernas, e Norman se
enrolou todo feito um porco"-espinho, caindo sobre um monte de esterco.

--- Seu pilantra! --- gritou, impedido de se defender com os punhos.

Carlsson espumava de raiva, e procurando em vão um palavrão para responder,
pôs"-se de joelhos sobre o peito do oponente abatido, estapeando"-lhe, enquanto este
cuspia e mordia, sendo por fim calado com um punhado de palha na boca.

--- Agora eu vou lavar sua boca! --- gritou Carlsson, pegando um monte de feno misturado ao esterco,
esfregou"-o na cara de Norman até que o nariz deste começou a sangrar. Mas assim que Norman
ficou com a boca desimpedida, deu vazão a todo o seu ódio, jogando
todo o seu vocabulário de ofensas na cara do vencedor, que não conseguia por nada 
dobrar a língua do vencido.

A música havia silenciado, a dança se interrompera, enquanto os espectadores
faziam suas apreciações acerca dos ditos e a troca de murros, os quais eles
escutavam e observavam com a mesma indiferença de um abate ou dança de roda, se
bem que os mais velhos acharam o ataque de Carlsson não tão correto segundo os
antigos costumes de briga. Mas de repente se escutou um grito, que afastou as
pessoas e tirou todos do clima de festa:

--- Olha a faca! --- gritou alguém, não se sabendo de quem veio.

--- Faca! --- repetiram as pessoas. --- Nada de faca! Tirem a faca dele!

Os lutadores foram cercados; Norman, que conseguira puxar seu canivete, foi
desarmado e posto de pé, depois que tiraram Carlsson de cima dele.

Carlsson vestiu seu paletó e o abotoou por sobre o colete rasgado; já Norman
saiu com a camisa em farrapos que caíam até as pernas. Com o rosto machucado,
sujo, sangrento, achou por bem se afastar até detrás da casa para não ter que
mostrar sua derrota para as moças.

Com a alegre confiança de ser o vencedor e o mais forte, Carlsson voltou ao
baile, e depois de tomar um trago, reiniciou a corte de Ida, que o recebeu
calorosamente e quase com admiração.

A dança prosseguia como uma máquina de debulhar --- a penumbra caíra; a
aguardente descia rodada a rodada e a atenção ao que outro fazia e dizia
foi se tornando menos vívida. Assim Carlsson conseguiu sair do celeiro com
Ida e alcançar o caminho do pasto sem que ninguém metesse o nariz, mas logo
quando a moça havia passado por sobre a cerca e Carlsson estava para segui"-la,
escutou a voz da viúva cortando a escuridão mesmo que não conseguisse vê"-los:

--- Carlsson! Carlsson, onde você está? Venha e dance uma música com sua
ajudante!

Carlsson não respondeu e se esgueirou para dentro do pasto silencioso como uma
raposa.

A senhora, entretanto, o tinha visto e também o lenço branco de Ida, que ela
havia amarrado em redor da cintura para proteger seu vestido das mãos suadas.
Quando ela chamou mais uma vez e ficou sem resposta, ela os seguiu, passou por
cima da cerca e entrou no pasto. Abaixo das aveleiras havia um caminho, em plena
escuridão, e ela viu apenas algo branco, que se afundava nas sombras e que por fim
desapareceu no fundo do longo túnel. Ela quis correr atrás, mas nesse momento
escutou novas vozes junto à cerca, uma grave e outra mais brilhante, porém ambas
abafadas e, quando se aproximaram, sussurrantes. Gusten e Clara passaram por
sobre a cerca, que rangia sob os passos inseguros do rapaz, e sustentada por
seus braços fortes, Clara foi erguida e desceu ao pasto. A senhora se escondeu
em meio aos arbustos, enquanto passaram de braços dados, dançando e
cantarolando, beijando"-se, como ela outrora dançou, cantou e beijou. Mais uma
vez a cerca rangeu e disparando como um bezerro veio o rapaz de Kvarnö com a
moça de Fjällång, e quando esta se ergueu sobre a cerca, com as faces rosadas
pela dança e um sorriso de abandono, mostrando o branco dos dentes, ela cruzou
os braços atrás da nuca, como se quisesse se deixar cair, e soltando uma risada
quase sem ar com as narinas arquejadas, atirou"-se de braços abertos sobre o
pescoço do rapaz, que a recebeu com um longo beijo e a carregou para dentro da
escuridão.

A senhora permaneceu atrás das aveleiras e observou cada par que chegava, um após outro, indo e vindo,
e voltando de novo, como em sua juventude, e a velha chama se acendeu, escondida
sob as cinzas de seus dois anos de viuvez.

Nesse meio tempo, o violino foi se calando aos poucos, passara da meia"-noite 
e o rubor da manhã já despontava de leve sobre a floresta ao norte;
o burburinho do celeiro se tornou mais tênue e alguns gritos de viva pelo prado
indicavam que os convidados já se dispersavam; a hora da partida se aproximava
para os ceifadores. Ela tinha que fazer as despedidas. Quando ela saiu do
caminho, onde a escuridão começava a diminuir, já se divisando o verde das
folhas, pôde ver Carlsson e Ida se aproximando no alto da encosta, de mãos dadas,
como prontos para se atirarem numa polca. Temendo a vergonha de ser encontrada
ali no ``caminho do mato'', virou"-se e se apressou em passar por cima da cerca e
voltar a casa, antes que os convivas tivessem ido embora. Mas do outro lado da
cerca encontrava"-se Rundqvist, que juntou as palmas da mão quando pôde
reconhecer a viúva, e ela escondeu o rosto no avental para não mostrar sua
vergonha:

--- Não, meu Senhor Jesuzim, a patroa também esteve no pasto? Ah, eu bem que digo; sim,
não dá mais para confiar nos velhos\ldots{} --- ela não pôde escutar mais, e
disparou rumo a casa, onde a procuravam e foi recebida com gritos de viva,
apertos de mão, agradecimentos pela alegre estadia e despedidas.

Quando o silêncio voltou a reinar e os fujões foram chamados de volta
dos prados e campinas --- apesar de nem todos terem sido encontrados ---, a senhora se
recolheu, mas ficou um bom tempo acordada, esperando ouvir os passos de
Carlsson subindo a escada.

\chapter[Boatos de casamento\ldots]{Boatos de casamento\subtitulo{e a matrona é
aceita pelo ouro}}
 \hedramarkboth{Boatos de casamento}{Strindberg}

\textsc{O feno} estava guardado, o centeio e o trigo estocados; o verão chegara
ao fim e tinha sido proveitoso.

--- O danado tem sorte! --- dizia Gusten sobre Carlsson, e não faltavam motivos
para creditarem a este o progresso do bem"-estar geral.

Os arenques tinham voltado e todos os homens exceto Carlsson estavam ao mar,
quando veio a hora de a família do professor partir para a abertura da temporada
de ópera.

Carlsson também os ajudou a preparar as malas e andava com o lápis atrás da
orelha o dia inteiro; bebia cerveja na mesa da cozinha, no armário da sala, no
banco da varanda. Ganhava aqui um chapéu de palha roto, e ali um par de
mocassins gastos, um cachimbo, uma piteira, charutos, caixas e garrafas vazias,
varas de pescar e latas de Liebig, rolhas, corda de veleiro, pregos, tudo que
não podiam levar consigo ou que fosse indesejado. Muitas eram as migalhas 
que caíam da mesa dos ricos e sua ausência seria sentida, a começar por Carlsson, que
perderia sua amada, até as galinhas e os porcos, que não teriam mais as sobras da
fina cozinha dos visitantes. Menos amarga era a tristeza das preteridas Clara e
Lotten, que apesar de terem recebido tantas xícaras de café gostoso ao
levar"-lhes leite, sentiam que sua primavera lhes seria devolvida assim que o
outono afastasse as difíceis concorrentes no mercado do amor.

E à tarde, quando o barco a vapor chegou e ancorou para buscar os hóspedes, houve
um grande alvoroço na ilha, pois nunca um barco a vapor ali se detivera.
Carlsson liderava a aproximação do barco com comandos e ordenações, enquanto o
vapor ia encostando"-se no ancoradouro. Mas aqui ele andava sobre um gelo fino,
conquanto assuntos marítimos eram"-lhe estranhos; e logo no momento mais solene,
quando a corda lhe foi jogada e ele, na presença de Ida e dos hóspedes, exibiria sua
habilidade, um maço de cabos lhe caiu por cima da cabeça, atirando seu chapéu na
água. A um só tempo teria de segurar a corda e apanhar o chapéu fugidio,
mas prendeu o pé num buraco, fez uns passos de dança e caiu sob uma chuva de
repreensões do capitão e uma saraivada de risadas escarnecedoras dos marujos da
popa. Ida virou o rosto, zangada com a atuação atrapalhada de seu herói e quase
chorando de vergonha por sua causa. Por fim ela o deixou na rampa com um breve
adeus, e quando ele quis tomá"-la pela mão e falar do próximo verão, da troca de
cartas e do endereço, a rampa balançou sobre seus pés, e ele teve que se dobrar
para frente, com o chapéu encharcado caindo"-lhe atrás da nuca, ao que o imediato
gritou"-lhe do alto do convés:

--- Vai soltar essa amarra um dia, rapaz?

Uma nova chuva de invectivas recaiu sobre o amante infeliz, antes que
conseguisse soltar o cabo. O vapor foi se afastando pelo estreito, e como um cão
cujo dono sai de viagem, Carlsson corria pela praia, pulando as pedras,
tropeçando nas raízes, para chegar a tempo no promontório, onde escondera sua
espingarda atrás de um arbusto de amieiro, para poder saudar os viajantes. Mas
ele devia ter se levantado da cama com o pé esquerdo, pois logo quando o vapor passou
a sua frente e ele dispararia para o alto, a arma falhou. Jogou a espingarda
no chão e começou a abanar seu lenço, correndo na praia e balançando seu lenço
azul de algodão, dando vivas, ofegante, mas nenhuma resposta veio do barco; nem ao
menos uma mão se ergueu, ou um lenço se viu. Ida desaparecera! Mas
ele corria sobre o cascalho, incansável e fora de si, pulando sobre a água,
disparando por entre os amieiros, e ao alcançar uma cerca atirou"-se por cima dela
e as estacas arranharam"-no. Finalmente, logo quando o barco desaparecia por
trás do promontório, deparou com os juncos da enseada; sem pensar duas vezes
correu para dentro da água, abanou mais uma vez seu lenço, soltou um último e
confuso grito de saudação. A popa do barco se enfiou por entre os pinheiros e
ele viu o chapéu do professor se despedindo e desaparecendo no promontório 
da floresta, enquanto a bandeira azul e amarela do mastro balançava ao vento, 
mais uma vez brilhando entre os amieiros; e assim o vapor desapareceu por 
completo a não ser pela grande fumaça preta, que pairava sobre 
a água como um véu de tristeza, deixando o ar escuro.

Carlsson saiu da água encharcado e a passos lentos foi buscar sua
espingarda. Observou"-a com ressentimento, como ele olharia para alguém que o
tivesse abandonado; sacudiu a cabeça, pôs um novo cartucho e disparou a arma.

Em seguida voltou para o ancoradouro. E repassou na memória a partida; como
dançara feito um palhaço sob as tábuas do ancoradouro, parecendo um boneco de
feira, escutando risadas e provocações, recordando a frieza e o constrangimento
de Ida nos olhares e no aperto de mão; ainda sentia o odor de carvão, fumaça e o
óleo da maquinaria, da fritura do restaurante e da tinta a óleo das mesas. O
barco a vapor havia parado ali no seu futuro reino e trouxera pessoas da cidade,
que o desprezavam, que num instante o derrubaram da posição que alçara e por
cujos degraus já subira um bocado, e --- aqui se fez um nó na sua garganta ---
levaram consigo sua alegria e felicidade de verão. Observou a água por um
instante, que as pás da roda tinham revolvido em uma mistura lamacenta, em cuja
superfície boiavam manchas de fuligem e espelhos de óleo, propagando as cores do
arco"-íris como uma velha vidraça de janela. Num breve espaço de tempo o monstro
havia deixado toda sorte de imundície atrás de si e conspurcado as águas claras e
verdes; tampas de cerveja, cascas de ovo, cascas de limão siciliano, tocos de
charuto, fósforos queimados, pedaços de papel com os quais as mugens e os
peixinhos brincavam; era como se todo o esgoto da cidade tivesse vindo de uma só
vez atirar sobre eles todos os seus resíduos e maldades.

Naquele instante era"-lhe insuportável pensar que, se ele estava decidido a
conquistar sua amada, precisava ir para lá, precisava entrar nas vielas e ruas
sujas, onde salários altos e paletós finos, lâmpadas de gás e vitrines de lojas,
mulheres decotadas, punhos largos e botas de cano longo, tudo que atraía a vista
se encontrava. No entanto, ele também odiava a cidade, onde seria o último, onde
seu sotaque seria motivo de riso, suas mãos rudes não poderiam fazer trabalho
delicado e onde não poderia dispor de suas variadas habilidades. E mesmo assim
precisava pensar nisso, porque Ida dissera que nunca se casaria com um
empregado, e não havia como ele se tornar patrão! Será que não mesmo?

Um vento suave veio pelo estreito, uma brisa fresca, que aumentava a cada
instante, agitando a água que começava a bater nas estacas do ancoradouro,
varrendo a fuligem e clareando o céu da tarde. Os amieiros farfalhavam, as ondas
rugiam, e despertado pelo balanço dos barcos, ele pôs a espingarda no ombro e foi
andando para casa.

O caminho seguia sob aveleiras, subindo até uma encosta; acima desta erguia"-se
um paredão de granito coberto de pinheiros onde ele ainda não havia estado.

Atraído pela curiosidade, ele subiu através de samambaias e arbustos de 
framboesa e logo chegou a uma plataforma de granito, por sobre a qual 
haviam erguido um marco marítimo. À luz do pôr do sol, a ilha se espalhava 
embaixo num único panorama, com florestas, plantações, campos, casas; 
e ao redor ilhotas, penhascos e o arquipélago que se estendia mar 
adentro. Era um belo pedaço de terra, e a água, árvores, 
pedras, tudo poderia ser dele, se quisesse estender a
mão, somente uma, e recolher a outra, que se estendia para a vaidade e a
pobreza. Não era necessário ninguém ao seu lado que o tentasse e implorasse de
joelhos diante de tal quadro, rosado pelos raios mágicos do poente; onde o azul
da água, o verde das florestas, o amarelo das plantações, o vermelho das casas
se misturavam num arco"-íris que deixaria deslumbrada uma mente menos afiada que a
de um camponês.

Furioso com a premeditada leviandade da mulher que o abandonara e que em
cinco minutos conseguira se esquecer de sua última e singela promessa de lhe dar
adeus acenando o lenço; ferido, como quem leva uma surra de vara, depois dos
arrogantes insultos daqueles palermas da cidade, inspirado pela visão da terra
fértil, das águas pesqueiras, das casas aquecidas, tomou sua decisão --- voltar
para casa, fazer uma última tentativa, ou mais duas, de dobrar aquele coração
falso, que talvez já tivesse lhe esquecido --- e tomar depois para si, só não
apelando para o roubo, tudo o que podia ser tomado.

\asterisc

Quando chegou a casa, junto à ribanceira, e viu a desolação da casa dos hóspedes,
as cortinas cerradas, palha e caixas vazias jogadas do lado de fora, deu"-lhe um
nó na garganta, como se tivesse entalado com pedaços de maçã, e depois de ter
coletado algumas lembranças dos veranistas numa sacola, esgueirou"-se o mais
silencioso possível para dentro da casa e subiu ao seu quarto. Escondeu seu 
tesouro debaixo da cama, sentou"-se à mesa, apanhou papel e caneta e
preparou"-se para escrever uma carta. A primeira página saiu de um fôlego só num
palavreado interminável, parte tirado da cachola, parte tirado das velhas
sagas de Afzelli e do cancioneiro sueco, que lera na casa de um inspetor em
Värmland e que lhe causara forte impressão:

``Ó minha querida e amada amiga!'' --- começou ele --- ``Estou sozinho no meu
quarto, Ida, e a farta que cê faz é mesmo terrível; ainda me alembro bem como
ontem quando a Idazinha veio para estas bandas, foi no tempo das colheita do
centeio da primavera e o cuco ninava os bezerro no pasto, agora é outono, e os
rapaz estão no mar pescando arenque; eu não queria preguntar muito sobre isso,
se ao partir não quis saudar do barco a vapor, Ida, já como o professor foi tão
bonzinho e educado de acenar do deque à popa, ao adentrarem o mar; as noites são
vazias qual um abismo depois da sua partida, por causa que a tristeza que é
profunda. Ao final do baile recorda"-se o que prometeu, me alembro como se
tivesse escrito, mas também eu tenho o hábito de `manter' o que prometo, o que
nem `todos' têm, mas isso tanto se me dá e não dou importância do jeito que as
pessoa são comigo, mas o que eu prometo, não esqueço, isso é certo''.

A dor da saudade tinha se apaziguado, dando lugar à amargura; e assim veio
o temor por rivais desconhecidos, pelas tentações da cidade e do salão Berns,
e convencido de sua incapacidade de evitar a queda de seu anjo, agarrou"-se aos
seus sentimentos mais nobres, e logo as velhas memórias de seu tempo de vendedor
de livros ambulante vieram à tona. Ele se tornou altissonante, severo,
conservador, um algoz vingativo, e através de sua boca um Outro (com letra
maiúscula!) falava:

``Quando penso como a Idazinha tá sozinha no labirinto da cidade, Ida, sem
uma mão que a proteja, que possa afugentar os perigo e as tentação, quando penso
em todas as oportunidade de pecado e perdição e erro, que deixa o caminho livre
e os passo ligeiro, sinto uma aferroada no meu coração, sinto que estou
faltando para com Deus e as pessoa por ter deixado ocê nas teia do
pecado, gostaria de ter sido um pai procê, e assim, Ida, podia se fiar no velho
Carlsson como a um verdadeiro pai\ldots{}''

Ao escrever as palavras ``pai'' e ``velho Carlsson'', ficou bastante
enternecido, lembrando"-se do último enterro que presenciara.

``\ldots{}um paizinho que sempre é bonzinho e com perdão no coração e nos lábio,
quem sabe por quanto tempo seria permitido ao velho Carlsson (ele já adorava a
expressão!) vagar por aí, quem viu se seus dias não foi contado como as gota
d’água do mar ou as estrela do firmamento, talvez antes que se saiba vai tombar
qual um ramo seco, e então quiçá `alguém' não acreditando quisesse desenterrá
ele da terra, mas esperemo e oremo que eu possa viver até o dia em que as flor
desponta nos campo e as rolinha canta; nesse tempo prazeroso para `muitos' que
agora se lamenta e canta como o salmista\ldots{}''

Aqui ele se esqueceu do que o salmista cantava e teve de se levantar e procurar
a Bíblia em seu baú. Mas eram mais de cem salmos à escolha, e Clara já
havia chamado para a ceia, e assim ele acabou por apanhar o primeiro que
apareceu, e completou deste modo:

 ``Tu coroas o ano da tua bondade, e as tuas veredas destilam gordura;
destilam sobre os pastos do deserto, e os outeiros cingem"-se de alegria.
Os campos cobrem"-se de rebanhos, e os vales vestem"-se de trigo; 
por isso, eles se regozijam e cantam.''\footnote{ Salmo 65:11---13, tradução 
de João Ferreira de Almeida.}

Quando releu o trecho, descobriu uma feliz alusão do que a vida no campo
representava para a cidade, e já que era esse o ponto sensível, decidiu não
tocar mais no assunto, deixando a citação parcial falar por si mesma.

Em seguida cogitou sobre o que escreveria a mais; mas sentia fome e cansaço e
não podia esconder para si mesmo que, no final das contas, tanto fazia o que
escrevesse, porque de todo jeito, Ida estaria ausente até que a primavera
retornasse.

Mas ele foi novamente aguilhoado pelo pensamento de que ela seria de outro e
com sangue frio decidiu bloquear preventivamente os canhões dos futuros
inimigos. Acrescentou então um pós"-escrito, depois de assinar “seu mais
fervoroso e fiel Carlsson”, que assim ficou: ``P.S.: Ida deve ter cuidado com o salão
Berns e o Café Blanks, pois que o professor disse que tudo quanto é moço em
Estocolmo é sabichão, e\ldots{}'' --- e achando melhor dar uma estocada nele de
qualquer maneira, pois ele iria levar o peixe para a cidade dentro de alguns
dias --- ``Norman também é sabichão'' --- e para que a afirmação tivesse efeito
retroativo, se necessário, acrescentou: --- ``desde que fez o tiro"-de"-guerra
no ano passado''.

Logo depois, desceu à cozinha para cear.

Escurecera e começara a ventar. A senhora Flod veio inquieta e se sentou à
mesa, onde Carlsson se sentara sozinho e acendera uma vela de sebo. As criadas
andavam do fogão à mesa, silenciosas e reticentes.

--- Aceite um copo de aguardente --- disse a patroa ---, vejo que o senhor está
precisando.

--- Estou mesmo, foi um trabalho danado embarcar todas as coisas --- respondeu Carlsson.

--- Pode descansar agora --- disse a senhora e foi buscar a garrafa. --- Mas que
vento terrível essa noite, vem do leste; vamos ver como os moços vão lidar com
os barcos nesse tempo.

--- Sim, nisso não posso ajudar; eu não mando no tempo --- disse Carlsson com
rispidez. --- Mas tomara que na semana que vem tenhamos bom tempo, porque estou
pensando em levar o resultado da pesca até a cidade para falar pessoalmente com
os peixeiros.

--- É o que o senhor pretende fazer?

--- Sim, creio que os meninos não estão conseguindo um bom preço pelos peixes, tem
alguma coisa errada aí, seja lá onde for.

A senhora tirava a mesa pensando que deveria haver lá outro assunto na cidade,
além do peixe.

--- Hum! --- disse ela. --- E o senhor fará a gentileza de visitar o professor,
não é?

--- Vou sim, se tiver tempo, até porque ele esqueceu uma caixa de garrafas
aqui\ldots{}

--- De todo modo, era uma gente danada de fina\ldots{} Aceita mais um trago?

--- Muito obrigado, patroa! Sim, é uma gente rara, e bem creio que virão de novo ---
ao menos pelo que ouvi de Ida.

Ele proferiu o nome dela com grande prazer, exprimindo toda a sua superioridade. A
senhora sentia por sua vez sua inferioridade, sua total desvantagem, e corou
enquanto seus olhos soltavam faíscas.

--- Eu pensei que tudo havia acabado entre o senhor e Ida --- sussurrou a senhora.

--- O que é isso? Há ainda muito pela frente --- respondeu Carlsson, que pareceu
ter fisgado algo no seu anzol.

--- Então vão se casar?

--- Pode ser, quando for a hora; mas eu iria primeiro estabelecer certas
condições.

O rosto enrugado da senhora se estremeceu e as mãos magras se agitavam na mesa
como mãos febris sob o lençol.

--- Pensa então em nos deixar? --- ousou dizer com voz trêmula e seca.

--- Um dia vou ter mesmo que sair daqui --- respondeu Carlsson. --- Cedo ou tarde
deve"-se ter seu próprio lugar, pois não se deve morrer de trabalhar para os
outros à toa.

Clara se aproximou com o mingau de farinha, e Carlsson teve uma repentina
vontade de gracejar com ela.

--- Então, Clara, vocês não têm medo do escuro, dormindo sozinhas esta noite, quando
os rapazes estão fora? Não querem que eu desça e lhes faça companhia?

--- Ah, não precisa, de jeito nenhum! --- respondeu Clara.

--- Como é? Não é preciso? O que você sabe do que eu preciso?

--- Então a Ida não foi suficiente para você? Mas andaram me dizendo que você
nem deu conta do recado!

Carlsson ficou vermelho até a raiz dos cabelos, mas na face da senhora Flod
surgiram a esperança, a curiosidade e a surpresa.

Fez"-se silêncio na cozinha por um instante. Escutava"-se como a tormenta
rugia por entre a floresta, agitando as folhas das bétulas, sacudindo as cercas,
girando o catavento e alçando a franja do telhado. Às vezes vinha uma rajada
pela chaminé insuflando o vento, liberando tantas faíscas do fogão que
Lotten tinha que proteger os olhos e a boca com a mão. Quando o vento cessava
por um instante, ouviam"-se as vagas batendo no promontório a leste. De repente
o cachorro latiu na ribanceira e o latido se distanciou, tendo o vira"-lata
corrido para saudar ou afugentar a chegada de alguém.

--- Por favor, vá e veja quem é que chega --- disse a patroa a Carlsson, que
logo se levantou.

Assim que ele atravessou a porta, viu somente uma escuridão tão espessa que parecia
poder ser cortada a faca, e o vento recebeu"-o com uma lufada que lhe deixou o
cabelo em pé como ramos de ervilhaca. Chamou pelo cachorro, mas os latidos
já estavam bem distantes no prado e já soavam animadamente em boas"-vindas.

--- Visitas a essa hora? --- disse à senhora, que também tinha se posto à porta ---
Quem será? Eu vou lá ver. Clara, acenda a lanterna e me dê meu gorro.

Ele recebeu a lanterna e saiu no prado enfrentando o vento, seguindo os latidos
até entrar no bosque de pinheiros que separava o prado da praia. Não se ouviam
mais os latidos, mas por entre o farfalhar e os estalidos das árvores, ouviam"-se
passos de bota aferrada contra as pedras da encosta, galhos rangendo, entortados
por alguém que procurava o caminho, pisadas nas poças d’água, palavrões em
resposta aos abanos de rabo do cachorro.

--- Olá! Quem vem lá? --- gritou.

--- O pastor! --- respondeu uma voz gasta, e nesse momento Carlsson viu faíscas
provocadas pelo solado de ferro contra os cascalhos, e saindo de um arbusto que
se esgueirava nas pedras um homem baixote, vestido com peles, de constituição
larga e um rosto rude, castigado pelas intempéries dos elementos, emoldurado por
um par de maltratadas suíças encanecidas e animado por dois olhos penetrantes
sob sobrancelhas que pareciam musgo.

--- Que inferno é andar pelos caminhos que vocês têm nesta ilha! --- reclamou a
modo de saudação.

--- Senhor Jesus, é o pastor, que nos visita nesse tempo de cão! --- respondeu
Carlsson respeitosamente à maldição inicial de seu consultor espiritual. --- Mas
cadê o barco?

--- É um pesqueiro, pelo que sei, e Robert o levou para o ancoradouro. Vamos para
debaixo do teto, porque o vento está chegando até os ossos. Avante!

Carlsson seguia à frente com a lanterna, seguido pelo pastor, mais o cachorro,
que fazia pequenas incursões no mato farejando um galo silvestre, que pouco
antes pulara, salvando"-se em meio às folhas.

A senhora foi ao encontro da luz da lanterna na ribanceira e quando reconheceu o
pastor, alegrou"-se e deu"-lhe as boas"-vindas.

O pastor estava levando peixe para a cidade quando a tempestade desabou, sendo
obrigado a desembarcar e passar a noite; xingava e maldizia por não poder chegar
a tempo e ver"-se livre dos peixes, agora que ``todos os desgraçados estavam no mar
apanhando tudo quanto vivia debaixo d’água''.

A senhora quis conduzi"-lo para a sala, mas ele foi direto para a
cozinha e preferiu ficar perto do fogo, onde poderia se secar. O calor e a luz
pareciam, no entanto, importunar o pastor, pois ele fazia caretas com os olhos,
como se estivesse acordando, enquanto tirou as botas molhadas de couro grosso.
Carlsson o ajudou a tirar um velho paletó cinza"-esverdeado,
revestido de pele de carneiro, e logo o sacerdote se encontrava só de camisão de
lã e de meias no canto da mesa, que a senhora limpou e dispôs para servir café.

Aquele que não conhecesse o pastor Nordström jamais adivinharia que este
ilhéu exercia uma função espiritual; de tanto que os trinta anos tomando conta das
almas do arquipélago haviam transformado o antes refinado predicante,
quando este veio ordenado de Uppsala. Seus ganhos minguados haviam"-no impelido a
tirar seu sustento do mar e da terra, e, quando isso não bastava, precisava
recorrer à boa vontade de sua congregação, que através de relações sociais,
apropriadas àquelas circunstâncias, tinha que manter. Mas a boa vontade se
mostrava mais em cafezinhos, tragos e refeições, recebidas por ocasião de suas
visitas e deste modo não somavam muito à economia do presbitério, influenciando,
em vez disso, negativamente no estado moral e físico do beneficiado. E como,
ademais, os ilhéus sabiam, após suas duras experiências nas lidas do mar, que
Deus só ajuda aquele que se ajuda primeiro, ou sendo por sua intrínseca
incredulidade em ver uma relação de efeito e causa entre a chegada de um forte
vento oriental e a confissão de Augsburg,\footnote{ Confissão de Augsburgo:
(\textit{Confessio Augustana}) principal texto normativo da 
reforma protestante, publicada em 1530. Seu principal autor 
foi Felipe Melanchton, tendo a colaboração de Martinho Lutero.}
acabavam não usando tanto a igrejinha de madeira que haviam 
construído, além disso, as idas à igreja eram 
frequentemente impedidas pela longa distância 
que se devia cruzar a remo, ou simplesmente pelos
ventos desfavoráveis, o que acabara por tornar a igreja numa espécie de mercado
popular, onde se ia encontrar os conhecidos, onde se fechavam negócios e se
ouviam notícias da corte, e o pastor se tornara a única autoridade local à qual
se podia recorrer, já que o encarregado da província vivia longe, terra adentro,
e nunca era chamado para questões legais. Tais assuntos eram resolvidos mediante
a troca de murros ou meia garrafa de aguardente.

O pastor tinha então, ao que constava, partido para a cidade com um pesqueiro
para vender peixe, que ele próprio havia conseguido no mar, e antes de chegar lá
foi impedido pelos ventos da tempestade. Com sua espingarda bem guardada num
estojo de couro, matula e livro de orações numa sacola de pele de foca, molhada
e em péssimo estado, ele agora se aproximava da luz e do calor e, depois de esfregar
os olhos, tomou seu lugar à mesa posta para o café. Não havia mais nenhum traço
de latim ou grego que se pudesse agora divisar naquela criatura à luz do fogão e
duas velas, um cruzamento entre camponês e pescador. As suas mãos outrora
brancas, que viraram páginas de livros durante toda a sua juventude, estavam
escuras e cascudas, com manchas amarelas, de água salgada e sol, rijas e
calejadas pelos remos, cabos e timão; as unhas roídas, pretas nas extremidades
do trato com a terra e com as ferramentas; a concha do ouvido repleta de cabelo
e perfurada com anéis de chumbo para evitar catarro e constipações; do bolso
lateral da camisa de lã pendia de uma trança de cabelo uma chave de relógio,
feita de um metal dourado qualquer e adornada com uma pedra semipreciosa; suas
meias de lã molhadas tinham um buraco bem no dedão, que ele sempre parecia
querer ocultar com movimentos tortuosos dos pés debaixo da mesa; a camisa estava
escurecida de suor por debaixo dos braços e o fecho da calça estava entreaberto
por falta de botões.

Tirou um cachimbo do bolso da calça, e sob o silêncio respeitoso de todos
bateu"-o contra a borda da mesa, jogando ao chão um montinho de cinzas e tabaco
azedo. Mas suas mãos estavam inseguras e não conseguiu encher o cachimbo como de
costume, complicando"-se todo e despertando assim a apreensão geral.

--- Como vai o pastor esta noite, creio que está um pouco cansado, não é mesmo?
--- foi dizendo a senhora.

O pastor ergueu a cabeça, olhou em volta e até para as vigas do teto, como se
procurasse quem estava falando.

--- Eu? --- disse ele, enchendo seu cachimbo, mas deixando cair fora um punhado de
tabaco. Sacudiu então a cabeça, dando sinais de que queria ficar em paz, e
afundou"-se em pensamentos sombrios, sem forma definida.

Carlsson, que entendera o problema, sussurrou para a patroa:

--- Ele não está sóbrio! --- e julgando que deveria ser prestativo, apanhou o bule
e encheu a xícara do pastor, posicionando a garrafa de aguardente ao seu lado,
convidando"-o com uma reverência a servir"-se à vontade.

Com um olhar demolidor, o velho levantou sua cabeça encanecida para afugentar
Carlsson, e afastando a xícara de si com repulsa, cuspiu as palavras:

--- Esta é sua casa, servo? --- e virando"-se para a senhora. --- Me dê uma xícara
de café, madame Flod.

E assim caiu num silêncio profundo, possivelmente se recordando de seus dias
de grandeza e da imoralidade crescente das pessoas.

--- Maldito servo! --- vociferou mais uma vez. --- Saia e vá ajudar o Robert!

Carlsson tentou dobrá"-lo com lisonjas, mas foi cortado imediatamente com
um ``Quem você pensa que é?'', e desapareceu pela porta.

--- Vocês têm alguém pescando? --- perguntou abruptamente para a senhora que
procurava em vão uma desculpa para o empregado, depois que o pastor se recompôs
um pouco com um gole de café.

--- Sim, Deus nos ajude --- começou a senhora. --- E foram com as redes de arrastão
também. Ninguém poderia imaginar às seis que teríamos tempestade à noite, e
conheço Gusten, ele prefere ir ao fundo a deixar a pesca amanhecer na rede.

--- Ah, bobagem, ele se vira bem! --- confortou o pastor.

--- Não diga isso, pastor! Se cai um graúdo na rede, vem muito dinheiro também,
mas peço apenas que meu menino saia ileso\ldots{}

--- Ele não seria burro de seguir adiante e puxar redes num tempo desses,
quando todo o mar está encrespado.

--- É justamente isso que se pode esperar dele; veja, ele puxou ao pai,  
é muito zeloso com o que possui, e seria capaz de dar a vida para não pôr a
perder suas redes.

--- Escute, madame, se ele é assim tão teimoso, nem o próprio diabo poderá ajudá"-lo!
No mais a pesca está boa, em Alkobbarne, jogamos seis vezes a rede e
apanhamos umas doze centenas de pescado.

--- Nossa, e estavam gordos?

--- Pode apostar! Gordos feito manteiga. Mas diga"-me, madame Flod, que conversa
é essa, que corre por aí, de que a senhora pensa em se casar de novo? É verdade?

--- Oh, Deus me livre! --- exclamou a senhora. --- Estão dizendo isso? É mesmo
terrível o que as pessoas inventam e dizem, e como correm os boatos!

--- Sim, sim, eu evito esse tipo de coisa --- retomou o pastor ---, mas é o que
estão dizendo, estão se referindo ao empregado, o que seria uma lástima para seu
menino.

--- Ah, não há nenhum perigo para o menino, e muitos já tiveram padrastos
piores.

--- Ah, é? Então é isso mesmo, posso ver. O velho corpo arde tanto que não pode
mais se segurar? A carne é fraca, hehehe!

O pastor lançou um olhar provocador sobre Clara e Lotten, para ver se elas
ficariam constrangidas. De fato, elas estavam com uma aparência bastante acanhada,
enquanto tentavam segurar o riso. O pastor voltou à carga, agora com um novo
foco:

--- Vocês, meninas, ficam aí rindo? Como se não soubessem bem do que estou falando!

--- Tenha a bondade de aceitar mais um café, pastor --- interrompeu a senhora,
agastada com o tom amoroso que a conversa começava a tomar.

--- Obrigado, madame, que gentileza a sua! Obrigado! Só mais um, talvez. Mas
preciso dormir e me pergunto se já fizeram minha cama.

Mandaram Lotten preparar a cama do pastor no quarto do sótão, depois de decidirem que
Carlsson e Robert dormiriam na cozinha.

O pastor bocejava sem parar e coçava um pé com o outro, passava a mão pelas entradas
do cabelo, como se quisesse afastar preocupações inomináveis, enquanto a cabeça
afundava em breves quicadas sobre a mesa, onde por fim o queixo se apoiou.

A vê"-lo, a senhora se aproximou dele e pôs a mão cuidadosamente sobre seu ombro,
dando"-lhe tapinhas e pedindo"-lhe com voz persuasiva:

--- Pastorzinho! Não ouviremos ao menos uma prece, antes de dormir? Pense nesta
pobre mãe e seu filho que está no mar.

--- Isso mesmo, uma breve prece! Dê"-me meu livro\ldots{} a senhora sabe onde
encontrá"-lo, dentro da matula.

A senhora apanhou a sacola de couro e trouxe um livro preto encravado com uma
cruz dourada, usado como um relicário de viagem, do qual
as senhoras idosas e os doentes costumavam receber gotas de consolo; e cheia de
enlevo espiritual, como se tivesse recebido um pedaço da igreja na sua humilde
casa, ela carregou cerimoniosamente com as duas mãos o misterioso livro feito um
pão quente, retirou com suavidade a xícara da frente do pastor, enxugou a mesa
com o avental e depositou o sagrado objeto diante de sua cabeça pesada.

--- Pastorzinho --- sussurrou a senhora enquanto o vento uivava na chaminé ---, já
está aqui o livro.

--- Bem, bem --- respondeu o pastor quase adormecido, estendendo o braço sem
levantar a cabeça, apalpando a mesa atrás da xícara e por fim atingindo 
sua asa, entornando a xícara e fazendo com que a aguardente escorresse em dois fios por
sobre a mesa envernizada.

--- Não, não, não --- lamentou a senhora protegendo o livro ---, isso não vai dar
certo; o pastor está com sono e deve ir se deitar.

No entanto o pastor já roncava com o braço apoiado sobre a mesa e o dedo médio
estendido num gesto ridículo, como se apontasse para uma direção invisível,
inalcançável naquele instante.

--- Meu Deus! Como faremos para levá"-lo para a cama? --- lamuriou a senhora para as
moças, confusa sem saber se o acordava, pois sabia como seu humor ficava
terrível quando assim faziam depois que bebia, e deixá"-lo na cozinha não era
possível por causa das moças, e tampouco dentro da casa, o que geraria boatos.
As três mulheres andavam em volta dele, feito ratinhos em volta do gato ao
qual deviam prender um guizo, mas ninguém ousava fazer nada.

Nesse meio tempo, o fogo se apagara, e a pressão do vento contra as vidraças e as
paredes irregulares chegou até o velho, que se encontrava calçado apenas com as
meias, e devia estar com frio, pois subitamente sua cabeça se levantou e com 
a boca entreaberta ele soltou três uivos, como quando a raposa entrega seu espírito,
fazendo as mulheres estremecerem.

--- Creio que espirrei --- disse o pastor, levantando"-se e indo com olhos fechados
até o sofá junto à janela, onde se afundou, estendeu as costas e com as mãos
entrelaçadas sobre o peito adormeceu num longo suspiro.

Qualquer esperança de tirá"-lo dali se foi, e Carlsson e Robert, que tinham retornado,
não ousaram encostar a mão nele.

--- Ele bate! Tenham cuidado --- esclareceu Robert. --- Deem a ele só um
travesseiro e cubram"-no que ele dorme até amanhã.

A senhora acomodou as moças em seu quarto, Robert teve que dormir num canto da
despensa e Carlsson foi para seu quarto. Apagaram"-se as luzes e se fez silêncio
na cozinha.

A senhora Flod se lembrou então que tinham esquecido de deixar um pouco de água
com o pastor e enviou"-lhe Clara com a garrafa de bronze. Ela entrou na ponta dos
pés, sem ranger a porta, mas logo retornou à sala:

--- Que safado, imaginem vocês!

--- O que foi? --- perguntou"-lhe a patroa, ansiosa de que algo pudesse ter acontecido com o
pastor.

--- A senhora precisa ver, patroa, ele queria que eu me deitasse com ele\ldots{} mas que
indecência!

--- Isso eu não posso acreditar --- foi a opinião da senhora, que não queria
desfazer a honra de ter o pastor como visita debaixo de seu teto.

--- Só sei que ele tentou me agarrar e queria coisas\ldots{}

--- Ah, deixe de falar besteiras --- cortou"-lhe a patroa, fechando a porta,
passando a chave e soprando a chama do lampião. --- Boa"-noite a todos!

Logo o sono envolveu a todos na casa, mais pesado para uns que para outros.

Na manhã seguinte, quando o galo cantou e a senhora Flod foi despertar seus
hóspedes, o pastor e Robert já haviam partido. A tormenta havia apaziguado,
frias nuvens de outono desfilavam do leste terra adentro e o céu era de um
fresco azul. Às oito horas a senhora começou sua caminhada até o promontório
oriental para ver se nenhum barco se avistaria na baía. Pelo estreito entre as
ilhotas surgia uma ou outra vela alçada, desaparecendo e se mostrando novamente. O
mar ainda estava agitado, exibindo um azul metálico, e as ilhas ao longe
pareciam miragens, suspensas no véu colorido do ar, como se tivessem levantado
voo da água e estivessem a caminho do alto como a névoa da noite. Os filhotes de
merganso, nas enseadas e cabos, fugiam para a água quando viam a águia"-rabalva
dirigir seu pesado voo sobre eles, mergulhando e vindo à tona e fugindo
novamente, revolvendo a água ao redor. Quando a senhora via as gaivotas deixando
uma ilhota e fazendo alarido, pensava: ``Lá vem um barco'', e os barcos vinham, mas
todos evitavam a ilha e seguiam para norte ou sul.

O vento frio que soprava e as nuvens brancas doíam nos olhos da senhora, que
entrou na floresta de novo, cansada de esperar; e passou a apanhar arandos
recolhendo"-os no avental, porque não conseguia ficar sem o que fazer e
precisava de algo para espairecer sua preocupação. Nada lhe era mais precioso
que seu filho, e ela não estivera tão preocupada naquela outra noite como agora, 
quando junto à cerca do pasto vira outra esperança sombria desaparecer
na escuridão. Hoje, porém, sentia ainda mais a ausência do filho, porque pressentia
que talvez logo ele a deixaria. As palavras do pastor na noite passada e as
fofocas haviam acendido um pavio e logo se ouviria o estouro: \textit{paff!} Quem iria torcer o
nariz, não estava ainda claro, mas que alguém o faria, era de se esperar.

Caminhou devagar para casa, até o outeiro do carvalho. Do ancoradouro ouvia"-se um
burburinho, e ela viu entre as folhas do carvalho como as pessoas
se moviam ao redor do casebre, conversando, negociando, averiguando,
discutindo. Algo se passara enquanto ela estava fora, mas o quê?

A preocupação acentuou a curiosidade, e ela se apressou descendo a ribanceira
para se inteirar do que acontecera. Descendo até a cerca da propriedade, viu a popa
do barco de pesca. Eles haviam, portanto, retornado depois de remar ao redor da ilha.

Norman narrava os acontecimentos com voz firme:

--- Ele afundou como uma pedra e subiu de novo; mas então foi ferido de morte bem
no meio do olho esquerdo e tudo foi rápido como uma lamparina que se apaga.

--- Oh, Senhor Jesus! Ele está morto? --- gritou a senhora e correu passando
a porteira, mas ninguém a escutou por causa de Rundqvist, que continuava a falar
da morte no barco.

--- E depois o acertamos com a fateixa e quando o gancho prendeu no lombo\ldots{}

A senhora havia contornado as redes e não conseguia mais avançar, mas viu como
se através de um espelho velado pelas redes dependuradas, como todos os
moradores da propriedade cercavam, esgueiravam"-se e agachavam"-se ao redor de um
corpo com manchas cinzentas que estava na carga do barco. Ela começou a gritar e
tentava se desvencilhar da rede que se emaranhava nos seus cabelos, enquanto as
chumbadas acertavam"-na feito um açoite.

--- O que em nome de Jesus nós apanhamos na rede dos linguados? --- soltou
Rundqvist, percebendo que algo se debatia entre os fios. --- Mas será possível, é
a patroa!

--- Ele está morto? --- gritou a senhora Flod com todas as suas forças. --- Ele está
morto?

--- Acabado como um cachorro morto!

A senhora se livrou da rede e desceu até o ancoradouro. Lá viu Gusten debruçado
sem chapéu ao longo do fundo do barco, mas se mexendo, e sob ele via"-se um
grande corpo peludo.

--- É você, mamãe? --- saudou Gusten sem se voltar. --- Olha o graúdo que
apanhamos!

A senhora arregalou os olhos, quando viu uma foca cinza e gorda, da qual Gusten
estava retirando o couro. Certamente não apanhavam focas todos os dias, a
carne era até comestível, a gordura bastava para muitas botas e o couro
valia bem umas vinte coroas; no entanto, arenques de inverno eram ainda mais
bem"-vindos e como não viu uma barbatana sequer no barco, ficou um pouco agastada,
esquecendo"-se tanto do filho que retornara quanto da foca inesperada e irrompendo em
admoestações:

--- Pois bem, mas e os arenques?

--- Não pudemos chegar até eles --- respondeu Gusten. --- E arenques pode"-se
comprar, mas foca não se apanha todo dia.

--- Você sempre diz isso, Gusten, mas que vergonha ficar fora três dias
e não voltar com um peixe sequer. O que você pensa que vamos comer durante o
inverno?

Mas ela não teve nenhum apoio, porque todos já estavam enjoados de arenques e
carne, mesmo de foca, é carne, além do que os caçadores haviam atraído todas as
atenções com suas notáveis narrações.

--- Sim --- aproveitou Carlsson para dizer, cortando para si um pedaço da presa
---, se não tivéssemos a terra, iríamos todos ficar sem comida!

Naquele dia não puderam terminar de guardar as redes, pois primeiro era preciso
cozinhar a gordura em grandes tinas de lavar roupa; e assaram e cozinharam a
carne enquanto bebiam café na cozinha. No lado sul da parede do celeiro
esticaram o couro como um sinal da vitória, falavam da caçada e a descreviam,
e todos os céticos que por ali passavam eram convidados a pôr os dedos nos buracos
da bala e escutar como a foca havia aparecido, onde ela subira na pedra, o que
Gusten dissera para Norman no último momento, quando o tiro foi desferido e como
por fim a presa se comportara no momento final, quando teve ``seu fio de vida ceifado''.

Carlsson não foi o herói desses dias, mas forjava seu aço em segredo e quando o
alvoroço da pesca por fim cessou, tomou lugar junto ao leme e partiu para a
cidade com Norman e Lotten.

\asterisc

Quando a senhora Flod foi até o ancoradouro para recebê"-los de volta da cidade,
Carlsson estava muito afetuoso e cordato, e logo a senhora percebeu que alguma
coisa acontecera.

Depois do jantar ele poderia entrar na casa e entregar o dinheiro das vendas; quando 
também se sentaria para contar os eventos. No entanto, ele teimava em começar, 
não parecendo nem um pouco inclinado a soltar o que tinha a dizer, mas a
senhora não o deixaria ir sem ao menos uma tentativa de relato de viagem.

--- Pois então, diga, Carlsson --- ela foi cavando ---, foi visitar o professor,
não é?

--- Sim, estive lá por um breve instante, é claro --- respondeu Carlsson,
visivelmente incomodado com a lembrança.

--- Ah\ldots{} Como eles estão passando?

--- Eles mandam lembranças calorosas a todos da ilha, e foram muito corteses e me
ofereceram café da manhã. O \textit{departamento} deles é bastante distinto e nós nos
entendemos bem.

--- E a comida estava boa?

--- Comemos lagostins e cogumelos; além disso, bebemos vinho do porto.

--- Mas diga, Carlsson, viu a moça também, não é?

--- Como não? --- respondeu Carlsson seguro de si.

--- E vocês se entenderam bem?

Não se entenderam de forma alguma, mas isso deixaria a senhora contente demais,
e por isso Carlsson não respondeu.

--- Eles foram tão gentis, à noite fomos ao salão Berns escutar a \textit{orquesta}, e eu
lhes ofereci \textit{cherry} e canapés. Foi, como você pode imaginar, muito agradável.

Mas a realidade não foi tão agradável assim e as coisas tinham se passado de um
jeito completamente diverso. Para começar, Carlsson havia sido recebido na cozinha
por Linda, que lhe serviu uma cerveja, já que Ida estava
ausente. Depois disso, chegou a esposa do professor o cumprimentando e dizendo a
Linda que fosse preparar lagostins para aquela noite, pois teriam visitas; e foi cuidar
de seus afazeres. Estando sós, Linda de início foi um pouco fria, mas mesmo
assim Carlsson pôde escutar dela que Ida recebera a carta dele e a lera para
todos, certa noite, quando seu noivo estava presente e eles se reuniram na
cozinha bebendo vinho do porto, enquanto Linda limpava os cogumelos. Eles
gargalharam, quase morreram de rir; o noivo leu em voz alta a carta duas vezes
com voz de pastor. O que mais causou hilaridade foi o ``velho Carlsson'' e seus
``últimos instantes'', e quando chegaram àquele ``tentações e caminho da
perdição'', o noivo --- que era carroceiro de cerveja --- sugeriu que saíssem para
cair na tentação do salão Berns, e foram até lá onde ele lhes ofereceu
\textit{cherry} e canapés.

Fosse pelo efeito de as palavras de Lina terem lhe sacudido as lembranças, que se
desordenaram, ou por desejar tão vivamente estar na pele do carroceiro de
cerveja, ele se pôs em sua privilegiada posição de anfitrião, além de ter
trocado de lugar com o desconhecido apreciador de lagostins, bebido o vinho do
porto do noivo e comido os cogumelos de Lina; fosse o que fosse, ele apresentou
o desenredar dos fatos dessa maneira para a senhora e conseguiu assim seu
intento, que era o mais importante. E quando havia acabado, sentiu"-se tranquilo
para partir ao ataque. Os rapazes estavam ao mar, Rundqvist havia se deitado, e
as moças tinham encerrado o expediente.

--- Que conversa é essa, que ronda aqui na paróquia e que agora eu escuto por
toda parte? --- começou ele.

--- O que estão dizendo agora? --- perguntou a senhora.

--- Ah, aquela velha conversa de sempre, que nós estamos pensando em nos casar.

--- Faz tempo que escutamos isso.

--- Sim, mas é mesmo um absurdo que as pessoas digam uma coisa dessas, que não
existe; não consigo entender isso de jeito nenhum --- desconversou Carlsson.

--- Claro, o que um rapaz jovem e enérgico faria com uma velha feito eu?

--- No que diz respeito à idade, não há problema. De minha parte, posso dizer que, se eu
um dia ``pensar'' em me casar, não seria com uma qualquer, que nada sabe e
ninguém viu, porque, veja bem, patroa, desejo é uma coisa e casamento é outra; o
desejo, o desejo mundano, se esvai como fumaça, e a fidelidade, o compromisso,
são como uma pitada de tabaco, quando outro pode oferecer charuto. Mas eu sou assim, 
patroa, àquela com quem eu me casar, serei fiel, e sempre fui assim, e quem
vier dizendo outra coisa, estará mentindo.

A senhora Flod afiou os ouvidos e começou a desconfiar de que naquele mato tinha coelho.

--- Mas entre o senhor e Ida, não chegaram a um entendimento? --- examinou ela.

--- Ida, sim, ela até que é boazinha, e se eu quisesse era só estalar os dedos que
a teria, mas entenda, patroa, ela não tem o caráter certo; ela é leviana e vaidosa,
e creio que ela prefere seguir caminhos errados. De resto, estou começando a
ficar velho, devo dizer, e não tenho mais vontade de correr atrás de rabo de
saia; sim, vou dizer francamente que se eu pensasse em me casar, escolheria uma
pessoa mais velha, sensata, alguém que tivesse o caráter correto; veja, não sei
bem como dizer exatamente, mas a patroa talvez me entenda mesmo assim, porque a patroa
tem sensatez, tem mesmo.

A senhora havia se inclinado sobre a mesa para escutar melhor as firulas de
Carlsson e aproveitar para dizer amém assim que ele pusesse para fora o seu sim.

--- Mas diga --- continuou ela, puxando um novo fio na meada ---, nunca pensou na
viúva de Åvassan, que está sozinha e só pensa em se casar de novo?

--- Vixe, ela? Não, essa eu conheço bem, e ela não tem o caráter certo, pois
sim, o que mais importa é o caráter. Porque dinheiro e traços exteriores, roupas
vistosas, nada disso tem valor para mim, eu não sou assim; e aquele que me
conhece bem não pode dizer o contrário.

O assunto parecia ter sido exaurido por todos os lados e alguém precisava dar a
palavra final, enquanto ainda era possível.

--- Pensou em quem, então? --- ousou dizer a senhora, dando um corajoso passo adiante.

--- Pensar, pensar! Pensa"-se uma coisa, pensa"-se outra, eu não pensei em nada
ainda, e aquele que pensa uma coisa, diz; eu me calo, para que depois ninguém 
venha me dizer que eu seduzi alguém, porque meu caráter não é assim.

Nesse momento a senhora não sabia ao certo em que pé estava, mas ela precisava
tentar mais uma vez.

--- Sim, mas, meu querido Carlsson, se você tem Ida no pensamento, não pode passar a
pensar em outra com muita seriedade.

--- Hum\ldots{} Ida, aquela raposa desavergonhada, eu não a quero, não, mesmo que a jogassem no
meu colo, não, quero alguém melhor que ela, que tenha ao menos as roupas do corpo, e
se tiver um pouco mais, não faz mal, mas eu nem reparo nisso, pois sou assim, é
o meu caráter.

Nessa altura tinham avançado e recuado tantas vezes que era perigoso ficarem
parados, se a senhora não desse um empurrãozinho.

--- Carlsson, o que diria se eu e você nos juntássemos?

Carlsson afastou tal ideia com as mãos, como se logo no primeiro instante
quisesse afugentar todas as suspeitas acerca de tal baixeza.

--- Não, isso nunca vai entrar em questão! --- assegurou solenemente. --- Nunca
mais vamos falar sobre isso, e muito menos pensar. Eu bem sei o que vão dizer, 
que eu me casei pelo ouro, mas eu não sou assim e isso não me pertence. Não,
nunca em tempo algum vamos falar disso de novo. Prometa"-me, patroa, e me dê a mão,
nunca mais vamos falar disso! Me dê a mão e vamos jurar!

Mas a senhora não quis jurar nada de mãos dadas, e queria mesmo discutir
abertamente a questão.

--- Por que não devemos falar sobre isso, que poderia acontecer? Eu
estou velha, você bem sabe, e Gusten não é maduro suficiente para tomar conta da
propriedade; preciso de alguém que fique ao meu lado e me ajude, mas entendo
que esse alguém não queira se esfalfar pelos outros e trabalhar por nada, e por
causa disso não vejo nenhum outro remédio senão nos casarmos. Deixe as pessoas
falarem, elas tagarelam do mesmo jeito, e se o senhor não tiver nada contra mim,
então não vejo o que pudesse nos impedir. O que tem contra mim, diga?

--- Não tenho nada contra a patroa, de modo algum, mas pense naquela maldita falação
sobre isso e aquilo, e também Gusten não haveria de ficar contente por nós.

--- Pff\ldots{} se você não for homem o bastante para mantê"-lo sob controle, eu posso muito bem resolver
isso. Tenho meus anos, mas não sou tão velha assim e, cá entre nós, não me importo em dizer"-lhe que\ldots{}
posso ser tão faceira quanto qualquer mulherzinha dessas, quando for preciso.

Quebrou"-se o gelo, e seguiu"-se uma maré de planos e conselhos: como dariam
a notícia a Gusten e como organizariam o casamento e tudo mais. A
conversa estendeu"-se bastante, tanto que a senhora teve de coar mais café e
buscar a garrafa de aguardente, e eles permaneceram ali noite adentro e mais um pouco.

\chapter[Briga"-se no terceiro dia do anúncio\ldots]{Briga"-se no terceiro dia do
anúncio do casamento, \subtitulo{faz"-se a comunhão e o casório,\break mas não se entra no leito nupcial}}
\hedramarkboth{Briga"-se no terceiro dia do anúncio}{Strindberg}

\textsc{Ninguém é mais} elogiado do que quando morre e ninguém é mais detratado
do que quando se casa, isso Carlsson logo pôde aprender. Gusten rugiu feito uma foca
faminta durante três dias, trovejando a plenos pulmões, enquanto Carlsson se
ausentava numa providencial viagem de trabalho. O velho Flod foi
resgatado de seu repouso terreno, contemplado, medido e considerado o melhor
homem até então criado, enquanto Carlsson foi virado pelo avesso feito roupa usada, em
cujo interior se descobrem nódoas. Lembrou"-se dele como reles
estivador de estrada de ferro e mascate de bíblias, enxotado de três empregos,
certamente fujão de um quarto, réu presumido por motivo de arruaça e briga. Tudo
isso foi jogado na cara da senhora Flod, mas ali já ardia uma nova chama. A
perspectiva de pôr fim à viuvez lhe dava novos ares e ela parecia reavivada e
com sangue novo nas faces, de modo que suportou tudo sem pestanejar.

A raiz da inimizade contra Carlsson vinha de ser ele um estranho, que através
do casamento agora abocanharia aquelas terras e águas, consideradas um bem 
comum pelos nativos. Já que a senhora estava firme no ninho e certamente
ainda viveria por muitos anos, as perspectivas do filho de chegar à posse
escasseavam. A sua posição na propriedade agora se assemelharia à de um peão,
dependente da ordenança e boa vontade do recém"-chegado capataz. Portanto não foi
de se estranhar que o novo subalterno esbravejasse o quanto podia e dirigisse
palavras rudes à mãe, ameaçando ir à Corte, abrir processo e mandar expulsar o
futuro padrasto. Pior ele ficou quando Carlsson voltou de seu pequeno passeio,
trajando o negro casaco domingueiro do finado Flod e o gorro de pele de foca
do mesmo, que ela, certa manhã nos primeiros tempos de casamento, dera ao marido de presente. O
filho nada disse, mas subornou Rundqvist a fazer uma pilhéria, de sorte que,
numa manhã, quando se sentavam para o desjejum, havia sobre o lugar de Carlsson
uma toalha ocultando um apanhado de objetos. Carlsson, que nada suspeitava,
levantou a toalha e descobriu seu assento coberto com toda a quinquilharia que
ele juntara num saco e escondera debaixo da cama em seus aposentos. Lá estavam
latas vazias de lagosta, caixas de sardinhas, vidros de cogumelo, uma garrafa de
porto, uma infinidade de rolhas, um vaso de flores trincado e muito mais.

Carlsson ficou vesgo de raiva, mas não sabia com quem trocar agressões.
Rundqvist ajudou"-o a disfarçar, vindo com a explicação de que aquilo era uma
``troça'' comum no lugar quando alguém danava a se casar. Infelizmente, Gusten
logo entrou e externou sua surpresa de que o comprador de bugigangas estava por
vir, já que ele normalmente só aparecia por volta do ano novo; e ao mesmo tempo todos
foram informados por Norman de que nenhum comprador viria, de que aquilo eram
lembranças que Carlsson guardava de Ida e com as quais Rundqvist queria se
divertir um pouco, já que tudo estava acabado entre os dois pombinhos. Dali,
seguiu"-se uma chuva de duras palavras, em consequência da qual Gusten se mandara até
a igreja e conseguira seis meses de adiamento no casamento de Carlsson, com o
argumento de que este não tinha ficha limpa. Isso foi uma pedra no sapato que
Carlsson contornou o tanto que pode, conseguindo pequenas compensações. Num
primeiro momento, ele tinha assumido sua nova posição com todas as pompas, mas
como o resultado fora ruim, adotava agora uma postura mais brincalhona,
obtendo bons resultados com todos menos Gusten, que mantinha uma obstinada
guerra oculta e não mostrava sinais de apaziguamento.

Seguiu"-se o inverno, em seu passo silencioso e monótono, com a derrubada das árvores e o corte de lenha, 
a confecção de redes e a pesca no gelo, entremeados com jogos de baralho e
café batizado, as festas natalinas e a caça de tetrazes. Veio a
primavera e as fileiras de edredões sobrevoando o mar eram um convite à caça, mas
Carlsson preferia investir todas as forças na preparação da terra. Ele queria
contar com uma boa safra, necessária para preencher os gastos que o casório
exigiria, pois a intenção era fazer uma festa de arromba, a ser lembrada
por anos a fio.

Com os pássaros migratórios, retornaram também os hóspedes de verão, e como no
ano anterior, o professor dizia sim e achava tudo uma ``maravilha'', especialmente
porque iriam celebrar bodas. Por sorte, Ida não os acompanhou desta vez. Ela saíra do
serviço em abril e diziam que logo iria se casar. Sua sucessora não era lá muito
aprazível e Carlsson tinha assuntos de sobra na cabeça para deixar"-se atrair,
além de um trunfo nas mãos que não queria perder.

O anúncio do casamento seria no meio do verão e as bodas celebradas entre a
ceifa e a colheita, quando sempre havia uma pausa nos trabalhos, tanto em
terra como no mar. Após o anúncio, percebeu"-se uma desagradável mudança de
atitude em Carlsson, que a senhora Flod foi a primeira a notar. Seguindo as
tradições, eles já viviam conjugalmente desde o começo do noivado e o noivo,
que ainda tinha o adiamento pendendo sobre sua cabeça, sabia muito bem se
comportar de acordo com as circunstâncias prescritas. Agora que o perigo estava
passando, ele começou a empinar o nariz e mostrar as garras. Mas isso não
intimidou em nada a senhora Flod, que também estava à vontade em seu novo papel,
e ela também passou a mostrar os dentes que tinha, de modo que transcorridos 
três dias do anúncio do casamento o caldo entornou. 

Toda a população da ilha, exceto por
Lotten, tinha ido à igreja para receber a comunhão. Como de hábito, tinham
escolhido o barco menor, para que, se fosse necessário remar,
houvesse o mínimo de esforço. Todos estavam apinhados lá dentro, além
da matula e meia libra de peixe que levavam para o pastor, algumas libras de vela
para o sacristão e ainda um monte de roupas de muda, velas, remos, jarros,
baldes, banquinhos e tamboretes.

Como de costume, haviam ingerido um desjejum reforçado e compartilhado tragos
sucessivos de garrafas alheias já pela manhã. Fazia calor e ninguém queria
remar, ocasionando uma pequena contenda entre os homens, pois nenhum deles queria
chegar suado à igreja. As mulheres se colocaram entre eles e quando chegaram à
enseada da igreja e escutaram os sinos, que não viam havia mais de um ano,
a briga foi deixada de lado. Era apenas a primeira chamada, e portanto tinham
bastante tempo. A senhora Flod subiu para a casa paroquial com o peixe. O pastor
ainda estava se barbeando e tinha o humor sombrio.

--- Hoje teremos gente estranha na igreja, com a visita dos moradores de Hemsö ---
foi sua saudação, enquanto limpava a lâmina com o dedo. --- Esses malditos ainda 
trazem peixe, como se eu não tivesse o mar à beira da porta --- continuou, com azedume.

Carlsson, que carregara o peixe, foi à cozinha tomar um trago.

Daí, levaram as velas para o sacristão e lá também um trago lhes foi oferecido.
Por fim, todos se reuniram na frente da igreja, onde aproveitaram para admirar
os cavalos dos grandes proprietários, ler as inscrições nas lápides e
cumprimentar os conhecidos. A senhora Flod fez uma breve visita ao túmulo do
finado senhor Flod e Carlsson saiu de cena por alguns momentos. O
campanário começou a soar e tremer, e a congregação se esgueirou para
dentro. Mas os moradores de Hemsö, desde o incêndio da velha igreja, 
não possuíam nenhum banco próprio e tiveram que ficar em pé dentro da nave.
Fazia um tremendo calor, e, desconfortáveis em meio àquele espaço solene,
suavam de puro constrangimento, parecendo uma fileira de internos num
reformatório à espera de um corretivo. Já eram onze horas quando chegaram ao salmo
que precedia o sermão e os moradores de Hemsö já tinham cruzado as pernas e trocado
de pé dezenas de vezes. O sol adentrava a igreja feito uma brasa, as testas
luziam de suor, mas, apertados feito sardinhas numa lata, não podiam se
mover para um lugar à sombra. Foi quando o sacristão escreveu o número do salmo 158
no quadro. O órgão assoviou um prelúdio e o sacristão entabulou o primeiro verso.
A congregação entrou a plenos pulmões, já ansiando pelo sermão seguinte. Mas aí
adveio o segundo verso, e o terceiro verso.

--- Não é possível, será que vão cantar todos os dezoito versos? --- sussurra
Rundqvist para Norman.

Mas era possível sim. E na porta da sacristia apareceu a cara rabugenta do
pastor Nordström, com um olhar severo e desafiador para a congregação, à qual
pretendia dar uma boa lição, já que a tinha nas mãos. E todos os dezoito versos
foram cantados e o relógio mostrava onze e meia quando o pastor se dignou a subir
ao púlpito. A essa altura eles já estavam amaciados, tão amaciados que os rostos
estavam virados para baixo como se dormissem. Mas não durou muito esse descanso,
pois de repente o pastor deu um berro que assustou os que dormitavam, fazendo"-os
levantar a cabeça sobressaltados e olhar abobados para o vizinho, como a perguntar se
havia fogo na igreja. Carlsson e a patroa haviam se posto tão à frente que
lhes era impossível qualquer retirada para a porta sem escândalo. A senhora se
limitava a chorar de cansaço e pelas botinas que lhe apertavam os pés à medida que
o calor aumentava. Às vezes, ela se virava e jogava um olhar suplicante para o noivo,
como a lhe pedir para ser carregada até o mar, mas este estava tão imerso no
culto, calçando as amplas botas de couro de cavalo do velho Flod, que apenas
retribuía a impaciência da pobre mulher com olhares severos. Enquanto isso, o
restante do grupo voltara para o fundo da nave e conseguira abrigo
debaixo da galeria do órgão, onde estava mais fresco e havia um pouco de sombra.
Lá, Gusten avistou a bomba de incêndio, onde pôde se sentar e pôr Clara em seu colo.

Rundqvist se apoiou numa pilastra e Norman ficou ao seu lado, quando começou o
sermão. Palavrório seco, sem nenhuma melodia, que durou seis quartos de hora. O
texto discorria sobre donzelas sábias e não sábias, mas como ninguém entre os
homens tomava o menor conhecimento do assunto, dormia"-se à larga, dormia"-se
sentado, pendurado, em pé. Passada meia hora, Norman cutucou Rundqvist, que
estava inclinado com a mão à frente da testa como se estivesse passando mal; o
jovem apontou com o dedão para Clara e Gusten, em cima da bomba de incêndio.
Rundqvist virou"-se lentamente, arregalou os olhos como se tivesse visto o tinhoso em
pessoa, sacudiu a cabeça e sorriu maliciosamente. É que Clara estava sentada,
com a língua para fora e os olhos cerrados, como se estivesse dormindo
e tendo pesadelos, enquanto Gusten mantinha os olhos fixos no pastor Nordström, parecendo comer
cada palavra sua, como se estivesse fazendo um esforço para ouvir a passagem do
tempo no relógio de areia. 

--- Mas que danados --- sussurrou Rundqvist, enquanto
retrocedia cuidadosamente, tateando com o pé para não fazer barulho sobre o piso. 

Norman, adivinhando a intenção de Rundqvist, esgueirou"-se feito uma enguia
para fora da igreja, por onde logo Rundqvist também seguiu, e em pouco tempo os
dois fujões estavam a caminho do barco. Lá fora, soprava uma brisa fresca do mar
e a rápida refeição que ingeriram lhes devolveu as forças. Em silêncio e com
cuidado, eles voltaram para dentro da igreja, onde viram Clara adormecida nos
braços de Gusten, que também dormia enquanto a abraçava, mas suas mãos
estavam a tal altura, que Rundqvist achou por bem abaixá"-las um pouco, no que
Gusten despertou e lançou novamente suas garras sobre a moça, como se alguém
a estivesse arrancando dele.

O sermão durou ainda meia hora, e depois houve mais meia hora de salmos
antes da comunhão. O sacramento foi recebido com sentimentos graves, e Rundqvist
chorava, mas a senhora Flod, que após a cerimônia queria se apertar num banco,
quase começou um desentendimento e foi empurrada. Acabou por passar a última meia
hora como anteriormente, ao lado do banco do sacristão, em pé sobre os saltos, com
a sensação de que o chão lhe queimava as solas do pé; e quando o pastor leu  
o proclama de casamento, ela ficou fora de si com os olhares da congregação.

Finalmente, estava tudo terminado e eles correram para a embarcação. A senhora Flod
não resistiu e arrancou as botinas após ter recebido as felicitações em frente à
igreja, carregando"-as até o barco. Chegando lá, meteu os pés na água enquanto
ralhava com Carlsson. Todos se jogaram sobre a matula, mas foi um alarido quando
descobriram que as panquecas tinham acabado. Rundqvist achou razoável a hipótese
de que tinham se esquecido de trazê"-las, enquanto Norman sugeriu que alguém as
tivesse comido durante a viagem, no que lançava uma grave suspeita sobre Carlsson.

Estavam embarcando quando Carlsson se lembrou de que tinha um barril de
alcatrão para buscar na casa paroquial. Mas aí houve tumulto. As mulheres
gritaram que não queriam alcatrão no barco por nada neste mundo, pois trajavam
seus vestidos novos, mas Carlsson buscou o barril e o acomodou na bagagem. A
briga então passou a ser sobre quem se sentaria ao lado do perigoso recipiente.

--- Mas eu me sentarei sobre o quê? --- lamentava"-se a senhora Flod.

--- Levante a saia e sente"-se sobre o traseiro --- respondeu Carlsson, que estava
consideravelmente mais à vontade, agora que já se fizera o anúncio de casamento.

--- O que você disse? --- sibilou a senhora.

--- Foi exatamente o que eu disse! Sente"-se no barco para que possamos sair
daqui!

--- Quem dá as ordens no mar, que mal lhe pergunte? --- intrometeu"-se Gusten, 
achando que a questão já lhe feria os brios. Ele se sentou aos remos, mandou
subir a vela e puxou a amarra. O barco estava sobrecarregado, e o vento
quase não soprava; o sol queimava e as cabeças fervilhavam. Avançavam como ``um
piolho sobre uma mão ensebada'' e não foi de grande ajuda os homens tomarem um
trago extra para a viagem. A paciência logo se acabou e o silêncio, que havia reinado por
um breve momento, foi quebrado por Carlsson, opinando que baixassem a vela e
pegassem os remos. Mas a isso Gusten se opôs. ``Esperem, quando chegarmos às
ilhotas, o vento pega'', ele dizia. E esperaram. Já se avistava sobre o estreito
uma faixa de azul escuro, e as ondas faziam barulho contra as ilhas exteriores.
Aproximou"-se um forte vento leste, e as velas se avivaram. Bem quando passaram
um cabo, veio uma lufada que fez o barco estremecer, se levantar e ganhar
velocidade a ponto de fazer redemoinhos atrás de si. Em sua homenagem, 
tomaram todos um trago e os ânimos melhoraram enquanto avançavam a contento. O
vento aumentava; o barco começou a adernar, mas não perdia velocidade.
Carlsson ficou com medo, agarrava"-se ao banco e pediu que diminuíssem a vela e
atracassem numa enseada. Gusten não lhe respondeu, e em vez disso puxou a escota,
fazendo entrar água no barco. Carlsson ficou possesso, levantando"-se para
colocar um remo na água. Mas a senhora Flod puxou"-o pelo casaco e o fez sentar.

--- Sente"-se no barco, homem, pelo amor de Deus! --- ela gritava.

Carlsson se sentou com o rosto lívido. Mas não ficou sentado por muito tempo,
logo deu um pulo, levantando a aba do paletó em desespero.

--- Deus tenha piedade, não é que o miserável está vazando --- ele berrou enquanto
sacudia o paletó.

--- Quem está vazando? --- perguntaram em coro.

--- O barril, diabos! Oh, Jesus Cristo! --- ouvia"-se, enquanto todos tentavam
se safar do riacho de alcatrão que corria segundo os balanços do barco.

--- Sentem"-se! --- rugiu Gusten. --- Ou vamos todos parar no fundo do mar!

 Carlsson havia se levantado novamente, bem quando vinha um novo sopro de vento.
 Rundqvist, que antevira o perigo, ergueu"-se com cuidado e deu um safanão em
 Carlsson, que caiu feito um saco. A briga parecia estar armada, forçando a senhora
 Flod a intervir numa explosão. Ela agarrou seu noivo pelo colarinho e deu"-lhe uma
 boa sacudida.

--- Que pobre coitado de homem é esse, que não sabe andar de barco! Você poderia se
portar feito gente e se sentar!

Carlsson ficou bravo, soltou"-se num puxão que acabou rasgando parte de sua gola.

--- Você quer destruir minhas roupas, bruaca! --- ele gritava, enquanto levantava
as botas para protegê"-las do alcatrão.

--- Como ousa? --- a senhora Flod faiscava de raiva. --- Seu paletó? De
quem você o ganhou, por acaso? Ser chamada de bruaca por um zé"-ninguém desses,
que não tem onde cair morto\ldots{}

--- Cale a boca! --- gritou Carlsson, atingido no ponto mais sensível. --- Ou lhe
respondo na mesma moeda!

--- Responda, que eu sei dar o troco muito bem --- retrucou a senhora Flod.

--- Eu diria que estou me contentando com carne seca enquanto poderia ter da fresca!

 Gusten, que achou já terem ido longe demais, começou a cantarolar um \textit{schottisch}, no
 qual entraram Norman e Rundqvist. A contenda amainou e passaram a atacar o
 inimigo comum, o pastor Nordström, que os deixara em pé por cinco horas e
 dezoito versos. A garrafa passou por todos, o vento tornou"-se mais estável, os
 ânimos se acalmaram e, sob o contentamento geral, o barco deslizou para dentro
 da enseada e encostou no atracadouro. 
 
 Pouco depois da viagem, passaram aos 
 preparativos das bodas, que durariam três dias e três noites. Abateram um porco
 e uma vaca, cem jarras de aguardente foram compradas, colocaram arenque no sal
 e no louro, pães foram assados, lavavam, coavam, cozinhavam, fritavam e moíam
 café. Durante todo esse tempo, Gusten andava de lado com ar suspeito. Deixava
 que trabalhassem e não se intrometia. Carlsson, por sua vez, passava a maior
 parte do tempo sentado à escrivaninha, escrevia e fazia contas; pedia
 encomendas de Dalarö e ordenava tudo que havia por fazer. 
 
 Um dia antes das bodas  Gusten acordou cedo, arrumou sua sacola, pegou a espingarda e
 partiu. A mãe acordou e perguntou aonde ia. Gusten respondeu"-lhe que ele
 tinha ganas de sair e ver se o peixe na desova teria vindo, e então se mandou.
 
 Ele tinha preparado uma matula para se ausentar por vários dias e levava
 coberta, garrafa de café e outras coisas, todas necessárias para uma visita ao
 arquipélago. Levantou logo a vela, mas em vez de se orientar para as
 enseadas, notando que o tempo já se fazia quente o suficiente para o banho nas
 praias, rumou direto para as ilhotas exteriores. 
 
 A manhã de fins de julho  estava radiante e clara, o céu era branco azulado feito 
 leite batido, e as  ilhas, ilhotas, atóis, rochedos e pedras estendiam"-se com tal suavidade na
 água que não se podia dizer se pertenciam ao céu ou à terra. Na terra, estavam
 próximos os pinheiros e amieiros e nos cabos repousavam os mergansos,
 patos"-fuscos, mergansos"-de"-poupa e gaivotas; mais adiante via"-se apenas
 pinheiros"-anão, airos e as tordas"-mergulheiras, que barulhentas feito
 papagaios enxameavam corajosamente à frente da embarcação para distrair o
 caçador de seus ninhos, cravados nas fendas das rochas. A esta altura os atóis
 já estavam mais baixos, mais desertos, havendo ali apenas um ou outro solitário
 abeto em que se penduravam casinhas de pássaro, onde edredões ou mergansos
 entesouravam seus ovos, ou crescia lá também alguma tramazeira, em cuja copa
 nuvens de pernilongos balançavam ao vento. 
 
 Mais além estava o mar liso, onde o
 moleiro voava em suas rapinas, disputando com andorinhas"-do"-mar, gaivotas e
 gaivotões e onde a águia"-rabalva era vista em seu pesado e surdo voo, às vezes
 se lançando sobre um edredão a chocar. 
 
 Era para lá, rumo ao último atol mais afastado do arquipélago, que Gusten 
 se dirigia, quase deitado sobre o leme, o cachimbo na boca, deixando se
 arrastar por uma brisa morna do sul. Às nove horas ele desembarcou em Norsten.
 Era uma minúscula ilha rochosa, de poucos alqueires com uma pequena
 depressão no centro. Apenas algumas tramazeiras desfolhadas resistiam entre as
 pedras; além de vistosos arbustos de evônimo que cresciam nas fendas, com suas
 frutinhas cor de fogo, e na baixada havia uma espessa cobertura de urzes,
 mirtilos e amoras brancas, que agora já estavam amareladas. Ao longo das
 rochas, espalhavam"-se alguns zimbros achatados, que pareciam estar se agarrando
 ao solo com as unhas para não serem levados pelo vento. 
 
 Era aqui que Gusten sentia"-se em casa; aqui ele sabia qual era o arbusto embaixo do qual 
 estaria o edredão em seu ninho, que se deixava acariciar nas costas e que lhe bicava de leve a barra da
calça. Logo ali, havia a abertura na rocha em que ele enfiaria seu cajado em
 forquilha, fazendo sair as tordas, uma das quais ele apanharia para servir no
 desjejum. Esse era o local onde os moradores de Hemsö pescavam arenque e junto
 com outro grupo de pescadores haviam construído uma casinha, onde costumavam
 passar a noite. 
 
 Foi para lá que Gusten dirigiu seus passos, pegando a chave
 debaixo do beiral do telhado e carregando para dentro suas coisas. A casinha se
 constituía apenas de um cômodo sem janelas, com catres dispostos em beliches
 uns sobre os outros, um fogão, um banco de três pernas para se sentar e uma
 mesa. 
 
 Depois de ter acomodados seus utensílios, ele subiu ao telhado, abriu a
 escotilha da chaminé e retornou para baixo. Alcançou os fósforos que
 estavam guardados sob uma viga, acendeu o fogão, onde o último visitante
 do local não se esquecera do velho costume de empilhar uma braçada de lenha para
 quem viesse depois. Colocou uma panela com batatas sobre o fogo, acrescentou
 um tanto de peixe salgado sobre as batatas e deu umas baforadas no cachimbo
 enquanto esperava. 
 
 Depois de comer e tomar uns tragos, pegou a espingarda e
 retornou ao barco, onde deixara os chamarizes para as aves. Remou e 
posicionou"-os na saída de um cabo, depois se escondeu num abrigo
 camuflado, construído com pedras e galhos de árvore. Os chamarizes boiavam sobre
 as lentas ondas que quebravam na praia, mas nenhum edredão apareceu. 
 
 A espera foi longa e por fim ele se cansou, partindo para explorar as pedras da
 praia, atrás de alguma lontra, mas só avistou cobras d’água e casas de
 marimbondo entre as esplêndidas flores de salgueirinha e aveia"-brava seca. Na
 verdade, não estava tão empenhado em conseguir coisa alguma, andava por
 andar, para não ficar em casa, pelo prazer que sentia perambulando por ali,
 onde ninguém o via nem o ouvia. 
 
 Depois do almoço, deitou"-se na casa
 para dormir e à tardinha saiu remando para tentar a sorte com o anzol de
 bacalhau. O mar agora estava imóvel como num sonho e ele via a paisagem se
 estender numa via dourada para dentro do sol poente como uma leve neblina. Fazia
 um silêncio ao seu redor como o de uma noite sem vento e podia"-se ouvir o barulho dos
 remos contra a água a milhas. As focas, que se banhavam a uma distância
 segura, levantavam suas cabeças redondas, gritavam, sopravam e mergulhavam
 novamente. 
 
 O bacalhau estava mordendo a isca e ele conseguiu puxar alguns
 barrigas"-brancas para cima, onde eles abriam suas bocarras inofensivas,
 bocejando para a água e fazendo caretas para o sol, protestando por terem sido
 içados de sua escura profundeza e balançados por sobre a amurada. 
 
 Permaneceu do lado norte do atol, mas quando a noite já ameaçava cair, ele começou a
 rumar para a terra e notou que saía fumaça da chaminé da casinha. Pensando no
 que estaria acontecendo, apressou seus passos.

--- É você, Gusten? --- ele ouviu do interior e reconheceu a voz do pastor Nordström. 

--- Mas não é que é o pastor! --- disse Gusten surpreso, vendo o velho sentado perto das brasas,
fritando arenque. --- O senhor veio sozinho para cá?

--- Sim, eu estava procurando bacalhau e fiquei mais para o lado sul, portanto
não o vi. Mas por que você não está em casa, se preparando para o casamento de
amanhã?

--- Pois é, eu não tomarei parte nesse casamento --- afirmou Gusten.

 --- Mas que história é essa, por que você não participaria? 
 
 Gusten explicou seus motivos tanto quanto pôde, de onde o pastor concluiu que ele 
 queria se ausentar de uma  cerimônia que lhe era abjeta e que com isso ele queria 
 ``dar uma lição'' naquele que lhe fizera mal.

--- Sim, mas e sua mãe? --- objetou o pastor. --- Não é uma pena ela ser
envergonhada assim?

--- Eu não penso assim --- respondeu"-lhe Gusten. --- A minha pena é bem maior. Eu
terei um pulha como padrasto e não poderei assumir a propriedade enquanto ele
estiver lá.

--- Pois sim, meu filho, mas tais coisas não podem ser mudadas, o remédio talvez
venha depois, agora você deve entrar no barco e voltar para casa. Você deve
estar presente no casamento!

--- Não, não tem jeito, já meti na cabeça que não vou --- assegurou"-lhe Gusten.

O pastor deixou o assunto e começou a comer seu arenque sobre o ferro do fogão.

--- Você não teria aí um traguinho? --- ele começou. --- É que a minha patroa tem
o hábito de trancar tudo que é mais forte e a esta altura eu não consigo mais
nada.

Sim, Gusten tinha aguardente e o pastor foi agraciado com uma dose tão boa que
ficou falante e discorreu sobre isto e aquilo, os assuntos da
comarca, tantos os internos quanto os externos. Sentados sobre as pedras no lado de
fora da casinha, eles viram o sol se pondo e o anoitecer se deitar com uma
penumbra cor de melão sobre os rochedos e a água. As gaivotas se puseram em
repouso sobre o banco de sargaço e as gralhas voltavam para o interior do
arquipélago, buscando abrigo noturno nas florestas. Estava na hora de ir para a
cama, mas antes espantaram os pernilongos de dentro da casinha, fechando a única
porta do aposento e defumando o interior com tabaco da marca Âncora Negra;
a porta foi reaberta, e a caça aos insetos se fez com galhos de tramazeiras.
Depois disso, os dois pescadores despiram"-se de seus casacos e cada qual subiu 
em seu catre forrado de musgo.

--- Dê"-me um último trago antes de dormir --- mendigou o pastor, que já tinha recebido
vários, e na beirada da cama Gusten lhe deu uma última unção antes de se deitarem.

Estava escuro dentro da casinha, apenas algumas esparsas frestas de luz
passavam pelas paredes vazadas, mas nessa penumbra os pernilongos achavam o
caminho até os sonolentos, que se viravam e se remexiam em seus catres para
escapar dos algozes.

--- Com mil diabos! --- grunhiu enfim o pastor. --- Você está dormindo, Gusten?

--- De que jeito? Dormir esta noite será impossível.

--- E o que se há de fazer?

--- Temos que levantar e fazer fogo novamente, não vejo alternativa. Se
tivéssemos um baralho, poderíamos jogar umas rodadas; você lá teria algum?

--- Eu não tenho, mas acho que sei onde o pessoal de Kvarnö esconde o deles ---
respondeu Gusten, e descendo de sua cama, abaixou"-se e procurou no chão de
terra batida embaixo do forro de musgo, achando lá um baralho gasto pelo uso. O
pastor tinha reavivado o fogo, jogado galhos de zimbro no fogão e acendido um
pedaço de vela. Gusten botou a cafeteira para ferver e achou um barril de
arenque, que foi posto entre os joelhos e serviu de mesa de jogo. Acenderam os
cachimbos; as cartas logo começaram a voar e as horas foram passando.

``Três boas'', ``passo'', ``trunfo'', ouvia"-se em meio a um ou outro palavrão,
quando um pernilongo inesperadamente deitava sua picada na nuca ou nos
braços dos jogadores.

--- Escute, Gusten --- interrompeu o pastor por fim, parecendo ter o
pensamento longe,  alheio às cartas e aos pernilongos ---, você não teria como
dar a ele uma lição, mesmo sem se ausentar do casamento? Parece"-me um tanto covarde
você correr de um sujeito desses e se você quiser enervá"-lo, eu sei um jeito.


--- E como seria isso? --- perguntou Gusten, que no final das contas achava uma
pena perder toda a comilança, ainda mais que ela se daria à custa de sua
herança.

--- Volte somente à tarde, depois da cerimônia, e diga que você se atrasou por
causa do mar. Já será um bom desaforo, e depois nós o botamos bêbado a ponto
dele não conseguir se deitar na cama nupcial, e aí damos um jeito para que os
rapazes façam pilhérias sobre o caso. Já seria o bastante, não?

Gusten pareceu gostar da ideia e esmoreceu ao pensar que ficaria três dias
sozinho no atol sendo devorado pelos pernilongos todas as noites; sobretudo por
ele de fato querer estar junto a todos e experimentar todas as iguarias que
vira serem preparadas. Um plano foi logo concebido pelo pastor, portanto, e
Gusten o poria em prática. Contentes consigo mesmos, finalmente foram se
deitar quando a primeira luz do dia já se intrometia por entre as frestas e os
pernilongos tinham se cansado de sua dança noturna.

\asterisc

 Naquela tarde, Carlsson ouvira de outros pescadores de arenque que tanto Gusten
 quanto o pastor haviam sido vistos rumando para Norsten e concluiu, bem
 acertadamente, que alguma safadeza estava por vir. Havia criado uma forte
 repulsa ao pastor, por ter adiado seu casamento por seis meses e
 por este não se cansar de lhe mostrar constante desconfiança. Carlsson
 rastejara aos pés dele, tinha se esfregado contra ele, cheio de adulações, mas sem sucesso. Se
 os dois estavam no mesmo salão, o pastor sempre virava suas costas largas para
 Carlsson, nunca ouvia o que este tinha a dizer e citava sempre histórias ambíguas que
 bem poderiam ser aplicadas sobre o caso presente. Quando Carlsson ficou
 sabendo que o pastor se encontrara com Gusten no atol, presumiu que aquilo
 fora uma reunião com propósito determinado e em vez de esperar a realização da
 decisão tomada lá, que ele suspeitava tratar de sua pessoa, elaborou um
 plano para enfrentar os conspiradores e vencê"-los. O contramestre da guarda do
 litoral estava coincidentemente de licença em Hemsö, onde ocuparia a função de
 encarregado das bebidas e assuntos ligados à comemoração, sendo conhecidos e
 apreciados seus talentos para organizar danças e festas. Carlsson concluiu
 acertadamente que podia contar com seus préstimos para pregar uma peça no
 pastor. Este tinha barrado o contramestre Rapp em sua crisma por seu grande
 apreço pelas meninas e um ano de atraso nesse quesito lhe causara
 constrangimentos no serviço. Agora, os dois inimigos da fé se reuniam sobre um
 café batizado e urdiram ali um plano para aprontar uma boa ao pastor. Tudo
 consistiria, nada mais nada menos, em que o pastor fosse impiedosamente
 embriagado, acompanhado de circunstâncias peculiares, as quais seriam
 especificadas em seu devido tempo e ocasião.

 Estavam assim plantadas as minas dos dois lados, cabendo à sorte decidir qual 
 delas surtiria maior efeito. E veio o dia das bodas. Todos acordaram cansados e
 irritadiços após tanta complicação e como os primeiros convidados chegaram
 antes da hora, pois as vias marítimas nunca são pontuais, ninguém foi 
 recebê"-los. E assim foram deixados, vagando a esmo pela propriedade como se
 ninguém os estivesse esperando. A noiva ainda não estava vestida e o
 noivo corria de lá para cá em mangas de camisa, enxugando copos, abrindo
 garrafas, colocando velas nos castiçais. A casa maior tinha sido lavada e
 espargida com folhas frescas, mas, para isso, todos os móveis tinham sido
 retirados e estavam do lado de fora, parecendo que ali haveria um leilão. No
 pátio, haviam erguido um mastro sobre o qual içaram a bandeira da
 alfândega, emprestada para a solenidade pelo inspetor. Sobre a porta,
 penduraram uma guirlanda e uma coroa de folhas de arando vermelho e
 margaridas, estando os dois pórticos enfeitados com vistosos ramos de bétulas.

 Nas janelas estavam enfileiradas garrafas com etiquetas coloridas, que luziam à
grande distância, dando a aparência de uma loja de bebidas, pois a Carlsson agradavam 
esses arranjos rebuscados.

 O ponche dourado brilhava como raios de sol através de seu vidro verde e o
 púrpura do conhaque ardia feito brasa incandescente; as tampas prateadas de
 estanho, que cobriam as rolhas, faiscavam como moedas de quarto de coroa, de
 tal modo que os mais ousados entre os jovens camponeses se acercavam de queixo
 caído, como se estivessem diante de uma vitrine de loja e já pudessem antecipar os 
deliciosos sabores, estalando a língua.

De cada lado da porta, feito peças de artilharia guarnecendo a entrada,
encontravam"-se dois tonéis de sessenta litros cada, um contendo aguardente e o
outro refresco; e atrás destes, empilhadas feito canhões, duzentas
garrafas de cerveja. A vista era magnífica e marcial, andando o contramestre
Rapp pelo local feito um chefe de polícia, com o saca"-rolhas pendurado num
cinturão, organizando o arsenal bélico sob seu comando. Ele enfeitara os
tonéis com ramos de pinheiro, colocando"-os em boa posição e introduzindo neles
torneirinhas de metal, à medida que ia brandindo seu martelinho de tanoeiro
como se fosse um escovilhão de artilharia. De tempos em tempos, batia de leve 
nos tonéis para assegurar que se encontravam cheios.
Impecavelmente trajado em uniforme de gala, com blusão azul de golas baixas,
calças brancas e chapéu de couro envernizado, e mesmo sem portar arma, por motivo de segurança,
ele infundia grande respeito nos camponeses mais jovens, pois, além de
encarregado das bebidas, era responsável pela ordem, por impedir a baderna, expulsar se
necessário e intervir se ocorresse alguma briga. Os rapazes mais abastados fingiam
desprezá"-lo, mas na verdade o invejavam, pois tudo que almejavam era poder andar
de uniforme e servir à Coroa, impedindo"-os apenas o medo que tinham dos castigos
corporais e dos irascíveis fuzileiros. 

Na cozinha, duas panelas cheias de café
descansavam sobre o fogão, à medida que várias moendas de manivela, 
emprestadas para a ocasião, giravam e grunhiam; torrões de açúcar eram
despedaçados com machadinha e pãezinhos doces se amontoavam junto à janela. As
moças iam e vinham correndo da despensa, que estava abarrotada de cozidos,
frituras e sacas de pão fresco. O tempo todo, as tranças soltas da noiva eram
vistas na janela, quando esta, ainda de camisola, colocava a cabeça para fora
gritando ordens para Lotten ou Clara. 

Avistava"-se agora uma vela após a outra entrando
pela enseada, fazendo elegantes volteios em torno do ancoradouro e 
atracando sob uma salva de tiros. Mas o povo ainda não se atrevia a subir até a
casa, perambulando em grupos pelos arredores. 

Uma feliz coincidência havia feito com que a esposa do professor e seus filhos tivessem
que se ausentar, para comparecerem a um aniversário no continente, e apenas o
professor se encontrava ali. De bom grado ele aceitara participar da recepção,
oferecendo seu salão para a cerimônia e o gramado debaixo dos carvalhos para o
café e a ceia. 

Ali se encontravam agora tábuas colocadas sobre andaimes e
barris, adaptados para servirem de banco ao longo das mesas, já
forradas com toalhas e sobre as quais já se enfileiravam xícaras de café. 

Nos arredores da casa, pequenos agrupamentos começaram a se
formar; Rundqvist, o cabelo alisado com gordura, barbeado, vestido num casaco
preto, se encarregara de receber as visitas com observações espirituosas, e
Norman, que fora incumbido com Rapp de tomar conta dos tiros de saudação,
que em sua maior parte seriam realizados com cartuchos de dinamite, estava atrás
da casa ensaiando, mesmo que em menor escala, com uma garrucha de matar passarinho. Em
compensação, ele tivera que deixar guardado seu acordeom, que hoje estava
proibido. Tinham chamado o rabequeiro mais afiado da região, o alfaiate de Fifang,
para avivar a festa e tratava"-se de um senhor altamente sensível a qualquer
intromissão em sua arte. 

E finalmente chegou o pastor, imbuído de um espírito brincalhão
e pronto a fazer troça com os noivos, como era a tradição. Ele foi recebido por
Carlsson na entrada com os votos de boas"-vindas.

--- Então, será que teremos batizado logo a seguir? --- foi a saudação do pastor
Nordström.

--- Que diabos, por que a pressa? --- respondeu o noivo sem nenhum
constrangimento.

--- Você está seguro disso? --- provocava o pastor, no que os camponeses davam
risadas. --- Eu já celebrei bodas, batizei e dei o puerpério no mesmo dia, mas
se tratava de gente disposta que já queria se garantir. Falando sério, como está
a noiva?

--- Hum, desta vez acho que você não vai precisar de tudo isso, mas nunca se sabe
quando as coisas acontecem --- respondeu Carlsson, conduzindo o pastor
para acomodá"-lo entre a esposa do sacristão e a viúva de Åvassan, que 
foram entretidas com conversas sobre o tempo e as condições de pesca.

O professor desceu, trajando fraque e gravata branca, além de cartola preta. O
pastor o agarrou imediatamente, tratando"-o como a um igual e entabulando
conversa, as senhoras ouvindo de olhos e ouvidos aguçados, certas de que o
professor devia ser um homem imensamente erudito.

--- Pois então --- começou o pastor --- como o professor tem passado o inverno na
Escandinávia?

--- É \textit{lamentáfel}, eu peidar muitos vezes! --- respondeu"-lhe o
professor.\footnote{ No sotaque carregado do alemão, a expressão “ter frio”,
\textit{fryser}, soa como \textit{fiser}, que significa “peidar”.} 
As senhoras arregalaram os olhos a mais não poder e se entreolharam.

--- Mas o que o professor está dizendo?! --- surpreendeu"-se o pastor Nordström.
--- Durante o verão, até que se tem um pouco de indisposição. Mas no inverno? Nunca
ouvi falar. Agora, é certo que as condições climáticas influenciam cada
barriga de um jeito diferente.

--- Meu esposa?\footnote{ No sotaque alemão, “barriga”, 
\textit{mage}, soa como ``esposa'', \textit{maka}.} Sim, ela é \textit{moito} bonito! 

--- Sim, sim, Deus abençoe, ainda não tive a honra de ser apresentado à senhora sua
esposa, mas eu certamente acredito, sim, eu até ouvi dizer que, enfim\ldots{}

--- Meu \textit{Teus}, eu estar dizer que eu peidar tanto, tanto\ldots{}

--- Sim, pois não, eu entendo, eu entendo\ldots{} a palavra talvez não seja a mais
apropriada, embora faça parte do idioma\ldots{}

--- Não, Senhor \textit{Teus}, pastor \textit{Nordstrem}, eu não querer falar que eu peidar, mas que
eu \textit{peidar!} --- gritava o professor.

--- Ah, entendi! O professor passa frio. Sim, agora eu percebo! O inverno aqui é
de fato muito rigoroso.

Carlsson entrou e declarou que tudo estava pronto e que só estavam à procura de
Gusten para começar. ``Onde está Gusten?'', gritava"-se agora nos arredores e se
repetia até o celeiro. Ninguém respondia. Ninguém o havia visto.

--- Ah, mas eu acho que sei onde ele está! --- informou"-lhes Carlsson.

--- Onde será? --- respondeu o pastor Nordström com voz provocativa, de maneira a
ser percebido por Carlsson.

--- Um passarinho me contou que ele foi visto lá em Norsten, e acompanhado por um
espírito de porco que o fez embebedar"-se, dá para acreditar?

--- Bem, nesse caso não vale a pena esperar por ele --- disse o pastor ---, já que ele se
meteu em má companhia. De qualquer modo, é ruim da parte dele não ficar em casa,
onde ele tem bons exemplos e pessoas de bem para guiá"-lo. Mas o que diz o noivo?
Vamos em frente ou esperamos?

A noiva escutava tudo e apesar de estar muito entristecida, sua opinião era
que se começasse, porque o café estava servido e logo ficaria frio. Portanto,
iniciaram as festividades com as cargas de dinamite nas encostas; o rabequeiro
preparava o arco e torcia as cravelhas, o pastor envergou a sua capa, os
padrinhos tomaram a dianteira e o pastor conduziu a noiva, vestida de seda
preta, véu branco e grinalda de mirto, tão apertada em suas vestes que elas
ressaltavam aquilo que devia ser escondido. E assim subiram para a residência do
professor, sob os gemidos do violino e os estrondos nas pedras.

A senhora Flod lançou um último olhar ansioso ao seu redor, na esperança de ver
o filho pródigo, e quando passaram pela porta, o pastor teve literalmente que arrastá"-la. 
Mas acabaram entrando; as visitas estavam de pé, enfileiradas contra 
a parede, como se estivessem de guarda numa execução, e os noivos
tomaram seus lugares na frente de duas cadeiras viradas, cobertas por um tapete
de Bruxelas. O pastor tinha sacado seu livro de preces e ajeitava a gola com os dedos,
limpando a garganta para começar, quando a noiva botou a mão sobre seu braço e
pediu para ele esperar. Se aguardassem alguns momentos, Gusten certamente apareceria.

Fez"-se um pesado silêncio na casa, ouvindo"-se apenas alguma bota a ranger e o roçar
dos vestidos armados, que depois de alguns instantes cessaram. As pessoas se
olhavam, ficaram constrangidas, pigarreavam e voltavam para o silêncio. Por fim
o pastor, que tinha todos os olhares voltados para si, disse:

--- Não, vamos começar, a espera já está se prolongando demais. Se não
veio até agora, não vem mais.

E ele começou a ler: ``Irmãos em Cristo! O matrimônio foi por Deus
instituído\ldots{}''. Já um bom tempo se passara, as senhoras cheiravam sua
lavanda e choravam, quando de repente se ouviu um estrondo no exterior da casa e
o tilintar de vidro se quebrando. Todos aguçaram os ouvidos, mas não se deixaram 
distrair, exceto Carlsson, que se mexeu preocupadamente e olhou de relance
pela janela. Novamente, seguiram"-se os \textit{bang! bang! bang!} Como quando
desarrolham garrafas de champanhe, e os meninos à porta começaram a rir com as
mãos tapando a boca. A confusão se acalmou por uns instantes e justo quando o
pastor perguntava ao noivo: ``Diante de Deus, onisciente e perante esta
congregação, eu lhe pergunto, Johannes Edvard Carlsson, você aceita Anna
Eva Flod como sua mulher, para amá"-la na alegria e na tristeza?''. Então se ouviu,
em vez da resposta, uma nova saraivada de rolhas de garrafa, vidro se quebrando,
e o vira"-lata, que latia freneticamente.

--- Quem está abrindo garrafas lá fora e perturbando este ato sagrado? --- rugiu
furioso o pastor Nordström.

--- Era o que eu gostaria de perguntar! --- Carlsson conseguiu dizer, já não
podendo se segurar de curiosidade e preocupação. --- É Rapp fazendo
pilhérias?

--- Ei, você, está me acusando de quê? --- reagiu Rapp com veemência, que estava à
porta e se sentiu ofendido com a suspeita.

\textit{Bang! Bang! Bang!} Os estouros não cessavam.

--- Pois em nome de Deus, saia então e veja se não acontece um acidente --- gritou
o pastor ---, continuamos depois. --- Alguns dos convidados saíram apressados,
outros se agruparam na janela.

É a cerveja! --- gritou alguém.

--- A \textit{cervessa}, a \textit{cervessa} está explodindo! --- exclamava o professor, agitado.

Mas como foram deixar as cervejas no sol? Como metralhadoras, as garrafas
explodiam e jorravam onde estavam empilhadas, espalhando espuma pelo chão.
A noiva ficou nervosa com a interrupção inesperada da cerimônia, que não podia
ser bom sinal; caçoaram do noivo pela má organização e este esteve a um passo de
brigar com o contramestre, sobre o qual queria jogar a culpa; o pastor esbravejou por
ter a cerimônia religiosa perturbada por garrafas de cerveja, mas lá fora os
meninos bebiam o que restava nas garrafas e tiveram a sorte de, no trabalho de
resgate, achar algumas quase cheias, das quais só a rolha tinha voado. Quando o
alvoroço finalmente passou, reuniram"-se de novo na casa, agora com menos
fervor, e, seguindo a pergunta do pastor para o noivo, a cerimônia continuou sem
mais incidentes, a não ser por algumas risadas mal"-contidas dos rapazes no
vestíbulo.

Choveram votos de felicidade sobre o novo casal e tão logo podiam, todos
deixaram a casa, fartos de suor, lágrimas, meias suadas, lavanda e buquês de
flores murchas. Em fileiras cerradas, rumaram para a mesa do café. Carlsson
tomou lugar entre o professor e o pastor, mas a noiva não tinha tempo para se
sentar, correndo de um lado para o outro, supervisionando o serviço. O sol brilhava na
tarde de julho e debaixo dos carvalhos espalhava"-se o som das brincadeiras e
conversas. Aguardente jorrava nas xícaras de café, repetiam"-se à vontade as
rodadas, e na cabeceira junto ao noivo foi oferecido ponche, o que não foi visto
com maus olhos pelos camponeses e seus filhos. Tratava"-se de uma bebida que não se
tomava todos os dias e o pastor aceitou"-a de bom grado em sua xícara.

 Hoje, o pastor estava estranhamente suave com Carlsson e bebia repetidas vezes à sua
 saúde, elogiando"-o e mostrando"-lhe a maior deferência, não se esquecendo também
 do professor, cuja apresentação lhe causara grande prazer, visto que 
 raramente se encontrava com um homem de erudição. Mas foi difícil acharem
 assunto, pois música não era o seu forte, além de o professor, por educação, tentar
 levar a conversa para as áreas do pastor, das quais este na verdade queria
 escapar. A dificuldade de se entenderem também contribuiu para que uma maior
 aproximação fosse impossível, além de que o professor, habituado a se
 expressar musicalmente, raramente era prolixo.

--- Vão muitos pessoas no igreja? --- perguntou, para puxar assunto com o pastor.

--- Não, eu não diria isso, exceto quando se celebra a comunhão. E nós nunca
contaremos com a presença do professor? --- perguntou o pastor.

--- Não, eu nunca comungar porque eu não poder.

--- Não pode? Como assim?

--- Eu ter indigestão com o hóstia! --- respondeu o professor, com um riso malicioso.

O pastor Nordström, mesmo não sendo exageradamente sensível, julgou o dito um tanto rude
para um senhor tão fino, e deixando"-o, virou"-se para azucrinar um pouco o noivo.

--- Pois então, Carlsson, você já foi vendedor ambulante? E o que você vendia?

--- O livro sagrado, exatamente como o senhor pastor --- riu"-lhe Carlsson em
resposta.

--- Bem, então mal você não fazia. Mas já ouviram essa, rapazes? --- 
voltando"-se para os demais. --- Já ouviram falar do vendedor ambulante que agora roda
por aí tentando ensinar os camponeses a fazerem filhos?

--- Hahahaha! --- as gargalhadas choveram de velhos e moços, enquanto as mulheres
viravam o rosto para rir disfarçadamente.

--- Vejam só que danado, querendo ensinar o padre"-nosso ao vigário!

--- Não posso acreditar nisso! --- exclamou Rundqvist com um ar
ingênuo e dissimulado. --- Seria como debulhar dentro do paiol e 
guardar o centeio do lado de fora.

Aproximou"-se da cabeceira o rabequeiro, bastante incomodado por
não ter atraído ainda nenhuma atenção. Consideravelmente 
encorajado pelo café batizado, ele quis discorrer sobre música com o professor.

--- Sua licença, senhor músico camerista --- ele saudou enquanto dedilhava seu
violino ---, olhe, temos uns pormenores em comum, pois a mim também me apetece
tocar, mas da minha maneira, é claro.

--- O diabo que te carregue, alfaiate, não seja impertinente! --- ralhou Carlsson. 

--- Sim, sim, eu peço desculpas --- respondeu este. --- Não que Carlsson
tenha alguma coisa com isso, mas sinta aqui esse violino, senhor músico da
câmara, sinta"-o e me diga se não é dos bons; comprei"-o no Hischen e ele me
custou dez contos bem pagos.

Sorridente, o professor pinçou a rabeca e disse amavelmente:

--- Moito beleza! Bonita mesmo!

--- Pois sim, isso que é conversar com gente entendida, aí se ouve uma 
opinião de verdade; falar de arte com estes aqui --- sua intenção era sussurrar, mas a voz
se recusou à nuance e ele acabou falando em alto e bom som ---, estes caipiras de
merda\ldots{}

--- Deem um chute no traseiro desse alfaiate! --- gritaram em coro. --- Não vai
ficar bêbado, alfaiate, senão não teremos dança!

--- Escute, Rapp, fique de olho no rabequeiro, não o deixe beber mais! --- disse Carlsson.

--- E por acaso eu não fui convidado também para as bebidas, você está dando uma
de sovina, seu ladrão?

O pastor interveio:

--- Sente"-se, Fredrik, e se acalme, ou você vai acabar apanhando.

Mas o rabequeiro queria por tudo discorrer sobre sua arte, e para ilustrar suas
afirmações sobre a excelência do violino, começou a tirar"-lhe uns trinados.

--- Escute, senhor camerista, escute esses baixos; soam justos como num pequeno
órgão\ldots{}

--- Calem a boca desse alfaiate!

Houve um rebuliço por entre as mesas e o alarido aumentou. Alguém gritou:
``Gusten está aqui!'' ``Onde? Onde?''. Clara informou que o viu perto do depósito de
lenha.

--- Avisem"-me quando ele entrar --- pediu o pastor ---, mas não antes,
entendeu bem?

As taças de \textit{toddy} tinham sido dispostas e Rapp abriu as garrafas de conhaque.

--- Não estão indo depressa demais com as bebidas? --- disse o pastor, na defensiva. Mas
na opinião de Carlsson as coisas estavam correndo conforme o esperado.

Rapp andava de mansinho e instigava todos a brindar com o pastor, que logo tinha
esvaziado seu primeiro \textit{toddy} e se viu obrigado a preparar o segundo.

O pastor logo começou a revolver os olhos e a mexer abobadamente a boca. Ele 
observava o tanto que podia a fisionomia de Carlsson, tentando ver se este 
já estava no ponto. Mas não lhe era fácil perceber, e por isso ele se 
contentava em continuar a brindar com anfitrião. Foi então que Clara entrou, exclamando:

--- Ele já entrou, pastor! Ele já entrou!

--- Mas que diabos você está dizendo? Quem entrou?! --- o pastor já nem se lembrava
de quem estavam falando.

--- Clara, quem foi que entrou? --- perguntaram todos.

--- Gusten, ora!

O pastor se levantou, desceu até a casa e buscou Gusten, conduzindo"-o até a
mesa, tímido e atrapalhado. Fez com que o saudassem com uma rodada de ponche e
gritos de viva. Gusten então brindou com Carlsson e disse"-lhe um breve ``boa
sorte''. Carlsson ficou emocionado e virou a taça, explicando que era uma grande
alegria rever o enteado, apesar de atrasado, e que ele sabia de dois
velhos corações que se aqueciam ao vê"-lo, apesar de haver se atrasado.

--- E acreditem"-me --- ele concluiu ---, aquele que souber tratar bem ao velho
Carlsson, sempre terá tudo dele.

Gusten não ficou particularmente comovido, mas exortou Carlsson a 
beberem uma taça a sós. O crepúsculo se aproximava, os pernilongos
dançavam e as pessoas zumbiam; as taças tilintavam, as risadas ecoavam e já se
ouviam gritinhos afoitos aqui e ali entre os arbustos, interrompidos por risadas
e gritos de vivas, chamados e tiros de festim no morno céu de verão. As mesas
foram limpas, pois agora iam servir a ceia; Rapp pendurava nos galhos 
do carvalho as lanternas coloridas que ele emprestara do professor. Norman corria
com pilhas de pratos, enquanto Rundqvist se abaixava para misturar aguardente no
refresco; as moças traziam potes de manteiga, montes de arenque sobre tábuas de
cortar, panquecas empilhadas, bandejas com almôndegas. E quando estava tudo
posto, o noivo bateu palmas:

--- Sirvam"-se, sirvam"-se com um pouco de comida! 

--- Mas onde está o pastor? --- objetaram as senhoras. --- Sem o pastor não podemos
começar!

 --- E o professor? Onde ele foi parar? Assim não há jeito! ---
 chamaram e procuraram, sem resposta. Os convivas agrupavam"-se em torno das
 mesas como cachorros famintos e com os olhos faiscantes, prontos para o ataque,
 mas nenhuma mão se mexeu e o silêncio imperava. 
 
--- Estou matutando aqui, será  que o pastor não está fazendo 
 uma pequena visita à casinha? --- ouviu"-se a voz inocente de Rundqvist.

 Sem mais delongas, Carlsson desceu para examinar a latrina, e justamente, de
 portas abertas, estavam sentados o pastor e o professor, cada qual com um jornal
 na mão, entretidos numa animada conversa. A lamparina estava posta no chão e
 projetava uma luz de ribalta sobre os dois entronados, que Carlsson não quis
 incomodar, contagiado pela sacralidade do lugar e por não querer
 atrapalhar o natural exercício das mais prementes necessidades.

--- Não --- balbuciava o pastor ---, uma vez por semana, veja, irmão --- ele agora
estava irmanado na bebida ---, uma vez por semana, esse é o meu regime. Nem mais,
nem menos!

--- Sim, sim, é moito bom, mas eu\ldots{}

--- \textit{Uma vez por semana}, é o que eu digo, e nadinha a mais! É o que diz
o tratado de Hufeland, e esse é o meu regime, meu irmão.

A conversa arriscava ser longa, e Carlsson se viu obrigado a interrompê"-la.

--- Peço vossa licença, senhores, mas a ceia está esfriando!

--- É você, Carlsson! Ah, sim! Comecem vocês, que já estamos chegando!

--- É que estão todos esperando e com o devido respeito à
circunstância, mas os senhores talvez pudessem se apressar um pouco!

--- Já estamos indo, estamos indo! Vá indo na frente!

Com satisfação, Carlsson percebeu que o seu oponente estava um tanto ``chumbado'' e
retornou com notícias tranquilizadoras de que ele  se preparava
e logo estaria presente. Instantes depois, viu"-se uma lamparina tateando seu
caminho pelo descampado, seguido por dois vultos cambaleantes que se aproximavam
das mesas postas. O rosto pálido do pastor logo foi discernido à cabeceira da
mesa e a noiva se aproximou para recepcioná"-lo com o cesto de pães, dando fim à
espera constrangedora. Mas Carlsson tinha outros planos, e com uma faca de mesa
ele bateu na travessa de almôndegas e gritou para que todos ouvissem: 

--- Silêncio, minha gente, o pastor quer dizer algumas palavras!

O pastor olhou para Carlsson, parecendo não entender onde estava, viu que tinha
algo brilhante na mão e se lembrou vagamente de que no último Natal ele fizera um
discurso segurando uma jarra de prata; portanto, ele ergueu a lamparina e
proclamou:

--- Meus amigos, temos hoje uma bela festa a comemorar.

Lançou um olhar para Carlsson, para que este lhe desse algumas informações
sobre o motivo da festa e sua natureza, pois já se encontrava totalmente confuso,
tendo lhe fugido da mente a estação do ano, o lugar, as causas e o sentido
de tudo aquilo. Mas a expressão sorridente de Carlsson não lhe oferecia solução
alguma. O pobre homem perscrutou o ambiente atrás do fio da meada. Viu as
lanterninhas suspensas no carvalho e teve a nebulosa imagem de uma imensa
árvore de Natal, seguindo daí por essa pista.

--- Essa alegre celebração de luz --- ele conseguiu dizer --- em tempos onde o sol
está subordinado ao frio, e à neve --- ele avistara o pano branco da mesa como um
campo de neve se estendendo ao infinito ---, meus amigos, quando as primeiras
neves se deitam feito um manto sobre a lama do outono\ldots{} esperem, acho que
vocês estão brincando comigo\ldots{}! Uáááááá! Ele se virou e encurvou as costas 
para vomitar.

--- O pastor apanhou um resfriado! --- disse Carlsson. --- Ele quer se deitar! Por
favor, minhas senhorias, podem começar! --- Eles não esperaram que ele repetisse a ordem,
pulando sobre as travessas, abandonando o pastor à própria sorte.

Tinham"-lhe oferecido pouso para a noite no sótão do professor, mas
para dar provas de que estava sóbrio, rejeitou, sob pena de murros,
qualquer tentativa de ajuda. E com a lamparina à altura dos joelhos, encurvado
como se estivesse procurando agulhas na relva orvalhada, tomou seu curso em
direção a uma janela iluminada. Quando, porém, chegou à cancela, errou a
distância e deu uma topada tão forte contra o mourão que a lamparina se
espatifou, apagando"-se. 

A escuridão o envolveu como um saco e ele
afundou de joelhos, mas a luz da janela ainda estava à sua frente, como um
farol, e ele seguiu adiante com a estranha sensação de que os joelhos de sua
calça preta se molhavam a cada passo e que os próprios joelhos lhe doíam, como
se estivessem batendo contra pedras. 

Topou por fim com um grande volume,
redondo e de superfície úmida; tateando em volta, toca em algo espinhoso, sua mão
encontra algo semelhante a uma torneira; e no mesmo instante ele ouve o som de um
líquido a jorrar e percebe que está se molhando. Assustado com a ideia de que
tivesse entrado na água, estende a mão atrás do suposto mastro, ao que percebe
a claridade de um portal, arrasta"-se para dentro, sente um degrau de
escada contra os joelhos, e ouve a voz de uma empregada gritando: ``Jesus, o
refresco!''. Movido por um obscuro sentimento de culpa, ele sobe a escada
engatinhando, os nós dos seus dedos esbarram numa chave, consegue abrir uma
porta, cai para dentro de um quarto e vê um grande leito, arrumado para
dois, tem força suficiente apenas para puxar a coberta, mergulhando com botas e tudo
entre os lençóis, escondendo"-se, pois se sente perseguido pelos gritos do
andar de baixo, e sente que está para morrer, ou desmaiando, ou se afogando e as
pessoas gritando pelo refresco! Momentaneamente ele desperta, reanimado, é
içado da água, de volta à vida e à mesa de Natal, para logo em seguida ser
soprado como uma vela, se apagando, morrendo, afundando, e se molhando todo.

Enquanto isso, a ceia se desenrolava sob os carvalhos, regada a tanta
cerveja e aguardente que ninguém sentiu falta do pastor, e quando tinham
devorado toda a comida, a ponto de se ver os fundos dos pratos e das travessas,
desceram todos a casa para dançar. 

A noiva queria mandar alguma iguaria para o
pastor comer em seus aposentos, mas Carlsson a convenceu de que este certamente
queria ficar em paz e que não deviam importuná"-lo, além do mais, isso só serviria para
envergonhá"-lo ainda mais. E encerrou"-se o assunto. 

Gusten havia silenciosamente desertado de seu aliado quando percebeu que este fora vencido,
ocupando"-se de seus próprios prazeres e afogando a inimizade e o despeito
no esquecimento e na embriaguez. 

A dança girava feito um moinho e o rabequeiro, sentado junto ao fogão, 
castigava a rabeca com seu arco; nas janelas abertas, as costas suadas 
se expunham ao frescor da noite. Nos arredores, os mais velhos estavam
sentados, divertindo"-se com os rojões, fumando, bebendo e troçando na
penumbra, no suave lume do fogão, que irrompia pelas vidraças, e na luz
que irradiava do salão de dança.

Nas campinas e nas encostas, os pares andavam no orvalho da grama sob o 
tênue brilho do céu estrelado, até que, no perfume da palha e ao som das
cigarras, apagassem os fogos acesos pelo calor da casa, pelo forte espírito do
vinho de centeio e pelos balançantes passos da música. 

As horas da meia"-noite
passaram em dança e o céu clareava no oriente; as estrelas sumiam para dentro
do firmamento e a constelação da Ursa Maior se virava de cabeça para baixo.
Ouviam"-se patos grasnar entre os juncos e a pálida enseada já espelhava o
claro rubor da manhã entre os vultos sombrios dos amieiros, que pareciam
invertidos na água a se estender até o fundo. Mas isso só durou um
instante; da costa vieram subindo algumas nuvens pelo céu e a noite voltou a reinar. 

Veio um alegre chamado da cozinha: ``O vinho quente! O vinho quente!''. E em
filas os homens trouxeram uma panela que flamejava e vertia uma luz azul em
torno de si, enquanto o rabequeiro executava uma marcha.

--- Vamos subir com uma primeira taça para o pastor! --- gritou Carlsson, na esperança
de coroar sua obra, e a sugestão foi recebida com gritos de viva. A procissão se
pôs em marcha em direção à casa do professor e com passos cambaleantes subiram a
escada. A chave estava na fechadura do quarto e eles o adentraram, não sem
um certo temor de serem recebidos a golpes e briga. Lá dentro, tudo estava silencioso
e à luz tremeluzente do fogo viram que a cama estava vazia e arrumada. Um
presságio sombrio de que algo terrível ocorrera se apossou de Carlsson, mas ele não externou
seus pensamentos e dissipou a surpresa e as adivinhações com uma explicação.
Lembrava"-se agora de o pastor ter dito que para escapar dos pernilongos
decidira dormir no celeiro. E como não se podia carregar fogo para o paiol, a
iniciativa esmoreceu, descendo todos para o terreiro para continuarem as libações.

Carlsson nomeou Gusten às pressas como anfitrião substituto, puxou Rapp de lado
e confidenciou"-lhe suas tenebrosas suspeitas. Sorrateiros, os dois cúmplices
se esgueiraram pela escada até a câmara nupcial, empunhando fósforos e um toco
de vela.

Quando abriram a porta, era tamanho o fedor que lhes bateu de frente que quase 
foram jogados para trás, dando"-lhes uma amostra do que viria a seguir.

Rapp acendeu o toco de vela e Carlsson, assim que deitou os olhos sobre o leito nupcial, 
teve superadas as suas piores expectativas. Sobre o travesseiro bordado havia uma cabeça
hirsuta como a de um cachorro molhado, de boca escancarada.

--- Com mil diabos! --- gemeu Carlsson. --- Nunca achei que esse miserável se
comportaria de maneira tão porca. Deus tenha piedade! Não tirou nem as botas, o canalha.

Aqui precisavam de boas ideias. Como tirariam de lá o ressacado, sem 
despertá"-lo, sem que o povo soubesse e, sobretudo, sem que a noiva o percebesse?

--- Vamos ter que tirá"-lo pela janela! --- explicou Rapp.

--- Usemos uma roldana e aí o levamos até a água! Apague a luz e vamos até o
celeiro atrás das ferramentas.

Trancaram a porta e tiraram a chave; logo depois, os dois vingadores deram a
volta na casa em direção ao celeiro. Carlsson xingava e rogava pragas,
dizendo que se ao menos conseguissem carregá"-lo para fora, poderiam expô"-lo ao
ridículo. Por sorte, a armação em tesoura ainda estava lá, depois de
terem abatido uma vaca para a festa. Após terem desarmado as suas partes e
recolhido os barrotes e as cordas, carregaram tudo escondido pelos mesmos caminhos
até chegarem à janela do pastor. Rapp buscou uma escada, armou
novamente a tesoura e fixou"-a com uma tábua na cumeeira. Depois, fez
uma argola, fixou os barrotes e encaixou a roldana. Daí entrou no
dormitório, enquanto Carlsson esperava embaixo com um gancho de barco. Depois de
Rapp ter se esforçado por uns instantes dentro do quarto, ofegando e fungando,
Carlsson viu sua cabeça saindo pela janela e ouviu a ordem: puxe! Carlsson
obedeceu e logo apareceu um vulto negro pela janela.

--- Puxe para valer! --- ordenou"-lhe Rapp, e Carlsson se esforçou tanto quanto
pôde. Debaixo da tesoura pendia agora o corpo inerte do pastor, 
incrivelmente parecido ao de um enforcado.

--- Puxe! --- comandou Rapp, mas nesse instante se ouviu um som como de um galão
de refresco perfurado e Carlsson sentiu cair algo sobre a sua cabeça e seus
ombros.

--- Jesus, ele está vomitando! --- gritou o noivo, vendo seu paletó preto
arruinado e suas mechas de cabelo salpicadas, que Rapp havia ajeitado
cuidadosamente com o grampo quente.

--- Desça com ele! --- ordenou Rapp novamente. --- Vamos lá! --- mas Carlsson já havia
dado corda e o pastor aterrou feito um saco de batatas no meio das urtigas, porém sem
soltar um pio. Num piscar de olhos, o contramestre saiu pela janela,
guardou a escada junto com a tesoura e agora o pastor era arrastado na direção
do ancoradouro. Chegando à beira d'água, Carlsson exclamou:

--- Agora, seu safado, você vai tomar um banho!

O lugar não era fundo, mas bastante lodoso, pois ano após ano eram despejados ali todos os restos
de pescaria. Rapp agarrou firme na corda que tinha amarrado em torno da cintura
do dorminhoco e o jogou na água. Nesse instante, o pastor despertou, berrando 
como um porco no abate.

--- Puxe! --- ordenou Rapp, que percebera que o povo da festa tinha ouvido e já
se aproximava. Mas Carlsson abaixou"-se e chafurdou o pastor na lama,
esfregando com as mãos suas roupas pretas, até que todo vestígio do
acidente que ocorrera no leito nupcial estivesse encoberto.

--- O que está acontecendo aí embaixo? O que foi? --- gritavam os homens em correria para lá.

--- O pastor caiu na água! Eia! --- respondeu Rapp enquanto içava para fora
da água o pastor que berrava. O povo se ajuntou todo. Carlsson fazia o papel do
nobre salvador da vida e bom samaritano, benzendo"-se e invocando os céus em seu
dialeto natal, que ele sempre usava quando queria dar uma de comovido e sincero.

--- Ocês acredita que eu cheguei aqui de pura coincidência e aí ouço algo 
debatendo e guinchando na água, que eu primeiro assuntei ser uma foca; e aí vejo que
é o nosso próprio pastorzinho. ``Ó Deus do cé'', eu gritei para o contramestre,
``não é que o pastor Nordström em pessoa está batendo suas asinhas ali''. E aí eu
digo ao Rapp: ``Rapp, vai lá correndo e pegue um cabo!'' E lá se foi Rapp atrás
do cabo. E quando a gente conseguiu fazer uma argola em torno do bucho dele, ele
danou a gritar como se estivesse sendo estripado. Vejam só o estado dele!

De fato, o pastor se encontrava num estado lastimável; os rapazes olhavam seu
guia espiritual com um misto de aversão e incurável deferência, e queriam tirá"-lo
dali o quanto antes. Improvisaram uma maca sobre dois remos e nela
colocaram o pastor, que foi erguido sobre oito fortes ombros e carregado até o
celeiro, onde trocariam suas vestes.

O rabequeiro, completamente bêbado, pensou que se tratava de uma
brincadeira e se animou, executando a canção ``Abram alas, abram alas, para o
cortejo do velho Smitten!''. Vários meninos saíram dos arbustos para engrossar a fileira, 
e o professor, redescobrindo sua juventude perdida, tomou a dianteira 
enquanto cantava. Já Norman, sem poder resistir por mais tempo, acabou sucumbindo
aos seus impulsos musicais e sacou seu acordeom.

--- Está fedendo \textit{moito}! --- foi a observação do professor, que se aproximara dos
respingos da maca; os rapazes tapavam o nariz. Nesse instante, o pastor se
mexeu e sobre suas cabeças veio uma nova enxurrada.

--- Ele passar mal! --- gritou o professor.

--- Cuidado, ele está vomitando --- avisou"-lhes Carlsson, mas tarde demais.

Quando chegaram a casa, as mulheres acudiram e viram o estado do pastor, 
compadecidas e tomadas de dó pelo desmaiado. A senhora Flod saiu em busca de uma
coberta para encobrir sua miséria, apesar das objeções de Carlsson, e mandou
também aquecer água e emprestar roupas de baixo, além de outras peças do vestuário
do professor. Quando chegaram ao celeiro, deitaram o doente, como o chamavam
--- porque ninguém cometeria o sacrilégio de dizer que o pastor estava bêbado ---, sobre
palha seca. Rundqvist apareceu com as ventosas e queria fazer uma sangria, mas
foi logo rechaçado, e quando não teve sua vontade atendida, pediu para que ao
menos pudesse ler uma reza junto ao doente, porque sabia uma boa para ovelhas com
edema, mas nem ele nem nenhum dos rapazes conseguiu chegar perto do pastor.

Carlsson subiu ao seu quarto, desta vez sozinho, para eliminar os vestígios de sua
humilhação. Quando ele entrou e viu a extensão do estrago no leito emporcalhado,
foi tomado por um momento de cansaço, exaurido como estava pelos esforços dos
últimos dias e noites, e imaginou como seria diferente com Ida se o
relacionamento deles tivesse perdurado. Chegou junto à janela e olhou para
a enseada com um olhar melancólico. As nuvens se abriram e a névoa se
estendia em lençóis sobre a água; o sol se levantava e seus raios penetravam no quarto,
iluminando o rosto pálido e os olhos marejados de Carlsson, que se apertavam, como
se resistissem às lágrimas que brotavam. Seu cabelo estava espalhado em
tufos molhados sobre a testa, sua gravata branca estava manchada, o paletó
pendia"-lhe solto. 

O calor do sol parecia lhe dar calafrios e passando a
mão sobre a testa ele se voltou para o quarto. 

``Mas que coisa horrível!'', disse a si mesmo, e arrancando"-se 
de sua letargia, começou a tirar os lençóis da cama.

\chapter[Mudanças de condição e opinião\ldots]{Mudanças de\break condição e opinião; 
\subtitulo{a agricultura declina,\break a mineração floresce}}
\hedramarkboth{Mudanças de condição e opinião}{Strindberg}

\textsc{Carlsson} não era homem de se deixar abater além da conta por
impressões ruins, ele possuía a tenacidade necessária para resistir às adversidades,
sacudir a poeira e dar a volta por cima. Alcançara a posição de
proprietário através de seus conhecimentos e sua disposição prestativa; e o fato
de a senhora Flod tê"-lo tomado como marido era lucro tanto para ela quanto para
ele, essa era sua opinião. Entretanto, passada a euforia das bodas,
o entusiasmo de Carlsson também diminuiu, seguro como estava agora de
seus direitos matrimoniais e de herança, já que podiam esperar uma criança em
alguns meses. Ele abandonara o sonho de se tornar um cavalheiro, pois vira que
isso lhe era impossível; em vez disso, ele agora queria chegar à posição de
grande proprietário. Trajava uma primorosa jaqueta de lã, à qual acrescentava um
grande avental de couro e botas impermeáveis. Passava muito de seu tempo à
escrivaninha, seu lugar favorito. Era lá que lia os jornais, mas escrevia e
calculava menos do que anteriormente, passando a supervisionar o trabalho com um
cachimbo na boca e a mostrar um decrescente interesse pela agricultura.

--- A agricultura está em baixa --- ele dizia ---, eu li nos jornais. É mais barato
comprar os grão de que se precisa!

--- Antigamente você dizia o contrário --- comentava Gusten, 
atento a tudo o que o padrasto dizia e fazia, limitando"-se a uma sonolenta
submissão a este sem, no entanto, aceitar o papel de filho daquele que ele ainda 
considerava um intruso.

--- Os tempos mudam e nós também! Agradeço a Deus por cada dia em que me
torno mais sábio! --- respondia Carlsson. 

Ele agora frequentava a igreja aos domingos, participava 
das questões públicas e foi eleito para o
conselho comunitário. Através deste, foi se aproximando do pastor até chegar o
grande dia em que pôde tratá"-lo coloquialmente. 
Essa era uma de suas maiores ambições e por
um ano ele não se cansou de repetir a todos na propriedade o que ele tinha dito
e o que o pastor Nordström lhe respondera.

--- ``Escute, meu caro Nordström'', eu disse, ``desta vez você tem que me dar
razão!'' E então Nordström respondeu: ``Carlsson'', ele me disse, ``embora você seja 
um homem inteligente e perspicaz, deve parar de ser cabeça"-dura''.

Advieram"-lhe inúmeros encargos comunitários, entre os quais a inspeção de
incêndios, o que implicava poder viajar à custa da comarca e beber
café batizado na casa dos conhecidos. Até as eleições do parlamento, que
aconteciam lá longe, no continente, acabavam ensejando pequenas negociatas e desdobramentos, 
cujas reverberações se deixavam sentir ali no arquipélago. Durante as eleições, além de duas
outras vezes ao ano, o barão vinha no barco a vapor com sua comitiva de caça, e
aí se pagava cinquenta coroas pelo direito de caça por alguns dias, ponche e
conhaque jorravam por dias e noites e despedia"-se dos caçadores com a bem
fundada opinião de que se tratava de gente mui distinta.

Desse modo, Carlsson ascendeu e tornou"-se uma sumidade no lugar: uma
autoridade de altas percepções sobre as coisas que os outros não entendiam. Mas
havia ainda um ponto vulnerável, que às vezes ele sentia: Carlsson era do
continente e não um verdadeiro homem do mar.

Para eliminar essa última falta de distinção, começou a tratar cada vez mais
dos assuntos do mar, mostrando grande interesse por tudo o que lhe dizia a
respeito. Limpava sua espingarda e saía para a caça; participava das puxadas de
arrastão e da colocação das redes de arenque, aventurava"-se em passeios a
vela mais longos.

--- A agricultura está caindo e ``nós'' temos que incrementar a pesca --- 
respondia à mulher, quando ela se preocupava com o desleixo da 
plantação e do gado. --- A pesca acima de tudo! A pesca para o pescador e a terra
para o agricultor! --- proclamava, agora de modo irresistível, depois que
aprendera com o professor da escola, no conselho da igreja, a colocar suas
palavras de modo ``parlementarista''.

Se lhes faltava algum recurso, a ordem era desmatar para obter lenha.

--- A floresta deve ser desbastada para que as árvores tenham espaço para crescer! Não sou eu
quem o diz, assim ensinam as normas racionais da propriedade moderna --- e se não
era Carlsson quem o dizia, quem eram eles para saber de alguma coisa!

Rundqvist passou a ser o encarregado da terra, Clara do gado. Nas mãos de
Rundqvist, a plantação virou um matagal, enquanto ele tirava sonecas até o
almoço na encosta dos brejos, e sonecas até a janta entre os arbustos; se as
vacas não davam leite, ele fazia encantamentos para elas.

Gusten fazia"-se ao mar ainda mais do que antes e reatou a velha parceria de caça
com Norman. O interesse, que anteriormente pusera todos os braços em
movimento, havia esmaecido; trabalhar para outrem não empolgava e por isso tudo
andava a seu próprio e vagaroso passo. 

Quando veio o outono, alguns meses após as
bodas, houve um acontecimento que pareceu uma súbita ventania sobre a embarcação
de Carlsson, recém"-saída do estaleiro de velas plenas. Aconteceu de a esposa
perder a criança, que veio prematura e natimorta. As circunstâncias, além
disso, foram preocupantes e o médico deu claras ordens que o caso estava
encerrado: mais nenhuma criança!

Foi um golpe duro para Carlsson, pois, em relação ao futuro, ele agora só tinha
perspectivas desvantajosas. Como, ainda por cima, a mulher esteve convalescente
por um bom tempo após o aborto espontâneo, essa mudança de posição se lhe
ameaçava vir muito antes do que o esperado. Tratava"-se, portanto, de empregar bem o
tempo, fazer"-se amigo do injusto destino, poupar o que se tinha e pensar no amanhã. 

Estes pensamentos injetaram um novo ânimo em Carlsson. A lavoura devia ser
retomada o mais rápido possível; o porquê disso não interessava a ninguém. Cortassem
lenha, pois iriam construir uma nova casa; o motivo não lhes
dizia respeito; a gana de caçar devia ser extirpada o quanto antes em Norman, que
mais uma vez foi separado de seu amigo; e Rundqvist foi capturado e insuflado
com vantagens à vista. Aravam, plantavam, pescavam e cortavam lenha, mas os deveres
comunitários de Carlsson foram postos de lado. 

Ao mesmo tempo, Carlsson investiu em sua vida doméstica; ficava sentado 
com a patroa e às vezes lia para ela alguma
passagem das Sagradas Escrituras ou do livro de salmos; ele implorava para o seu
coração e apelava para seus sentimentos mais nobres, sem conseguir explicar
exatamente onde queria chegar. A senhora Flod gostava da companhia e alguém para
conversar, portanto dava valor a essas pequenas distinções sem dar a elas 
qualquer eventual significado mórbido.

Certa tarde de inverno, quando a enseada estava fechada, os canais intransitáveis
e havia quatorze dias que não podiam se deslocar, visitar um vizinho ou
receber carta e jornal; quando a solidão e a neve pesavam sobre as
mentes e o dia curto permitia apenas um trabalho insignificante, o grupo havia
se juntado na cozinha e Gusten estava com eles. O fogo ardia no fogão e os rapazes
estavam a confeccionar redes; as meninas estavam fiando e Rundqvist estava
aplainando cabos de pá. A neve caíra durante todo o dia e já estava encobrindo
as janelas, fazendo a cozinha se parecer com uma catacumba e obrigando um dos
homens a sair a cada quarto de hora para remover a neve na frente da porta, para
que não ficassem presos e assim pudessem chegar ao estábulo para tirar o
leite das vacas e dar a elas a ração da noite.

Era a vez de Gusten remover a neve da saída; preparou"-se para sair com um impermeável de mar sobre o
casaco e o gorro de lontra. Ele empurrou a porta, contra
a qual a neve se acumulava, e entrou dentro da nevasca. O ar estava escuro, os
flocos de neve eram cinza como mariposas, grandes como penas de galinha e
caíam ininterruptamente, amontoando"-se uns sobre os outros,
primeiro de leve, depois pesadamente, empilhando"-se em camadas cada vez mais altas. 
Já estavam quase cobrindo uma boa parte da parede da casa e apenas no canto superior das
janelas passava o brilho da luz que vinha do interior. Gusten reparou que os
aposentos de Carlsson e sua mãe estavam iluminados. Uma súbita curiosidade
induziu"-o a remover a camada de cima da neve, criando uma escotilha; subindo
no monte de neve, ele podia olhar para dentro do quarto. Como de costume,
Carlsson estava sentado à escrivaninha e tinha diante de si um documento, com um
grande selo azul impresso no alto, parecendo a inscrição de uma nota de banco;
com a caneta erguida em sua mão, ele falava algo para a mulher, em pé ao seu
lado, e parecia prestes a entregar"-lhe a caneta para uma assinatura. Gusten colou
a orelha no vidro, mas por conta da janela dupla, ouviu apenas um murmúrio.
Queria muito saber o que se passava, intuindo que aquilo lhe dizia respeito
e sabendo que assuntos importantes eram resolvidos quando se assinava documentos
com selo.

Ele abriu a porta de entrada com cuidado, tirou seus sapatos forrados e subiu a
escada sem ruído, até chegar ao mezanino. Lá se deitou no chão e com a
cabeça inclinada por sobre a porta embaixo, ele podia ouvir o que diziam dentro
dos aposentos da mãe.

--- Anna Eva --- dizia Carlsson, num tom entre vendedor de bíblias e funcionário
da comarca ---, a vida é curta e a morte \textit{pode} nos chegar, quando menos se espera. 
Nós \textit{devemos} pensar no que virá, se for amanhã ou
depois, isso é \textit{absolutamente} indiferente! Portanto é melhor você assinar
agora do que mais tarde.

A senhora Flod não gostava de ouvir falar da morte, mas como Carlsson por meses
não falara em outra coisa, ela agora só tinha uma fraca resistência a oferecer.

--- Está certo, Carlsson, mas para mim não me é indiferente se eu morrer
hoje ou daqui a dez anos, e eu quero viver muito ainda.

--- Mas, mulher, eu não estou dizendo que você \textit{vai} morrer, eu só disse
que nós \textit{podemos} morrer, e se acontecer agora ou daqui a dez anos, tanto
faz, porque ocorrerá de qualquer modo. Portanto, assine isso aí.

--- Sim, mas é isso que eu não entendo --- a mulher ainda lhe resistia como se a
morte estivesse vindo por ela ---, isso não significaria\ldots{}

--- Sim, mas é indiferente, pois acontecerá de qualquer maneira! Talvez não seja
assim! Da minha parte, não sei! Assine mesmo assim!

Foi como colocar uma corda ao redor do pescoço dela e puxar, quando Carlsson
disse o seu ``Da minha parte, eu não sei!'', a pobre mulher não conseguia mais e
já estava cedendo.

--- Mas qual é o seu propósito em tudo isso? --- ela perguntou cansada e
aborrecida com a longa discussão.

--- Anna Eva, você deve pensar nos seus descendentes, esse é o primeiro dever de
um ser humano, e por isso você deve assinar.

Neste instante, Clara abriu a porta da cozinha e chamou por Gusten, que não
querendo revelar sua posição permaneceu calado, embora perdesse um trecho do que
se passava nos aposentos. Clara voltou para dentro e Gusten desceu a escada,
parando diante da porta a tempo de ouvir as palavras finais de Carlsson, o que o
fez concluir que a assinatura tinha sido realizada e o testamento sacramentado.
Quando Gusten entrou novamente na cozinha, os outros notaram que havia algo
diferente nele. Ele proferia frases obscuras sobre como atiraria numa raposa que
ele ouvira gritar; que era preferível ir ao mar a ficar em casa sendo comido
pelos piolhos; que um pouco de arsênico misturado na ração dos cavalos podia
animá"-los, mas também os matar, se a dose fosse excessiva. Ao contrário,
Carlsson estava de humor esplêndido à mesa do jantar, querendo se inteirar sobre
os planos de trabalho de Gusten e suas metas de caça. Pediu que
trouxessem a garrafa, e deixando seu conteúdo jorrar, disse: ``os
minutos nos são preciosos; deixem"-nos, pois, comer e beber, que amanhã não
estaremos mais aqui! Saúde!''. Gusten ficou acordado por um bom tempo naquela
noite e muitos pensamentos sombrios e planos diabólicos cruzaram sua mente; mas ele
não tinha força de vontade, não conseguia mudar as circunstâncias conforme o seu intento,
transformar seus pensamentos em ação; ao contrário, quando tinha pensado sobre o assunto,
dava"-o por realizado.

 Após dormir algumas horas e sonhar sobre coisas diversas, 
 despertou lépido como sempre, deixando as coisas como estavam, confiando que o
 amanhã a Deus pertencia, que a justiça não tardaria e outros pensamento afins.

 A primavera voltou, as andorinhas reparavam seus ninhos e o professor havia retornado.

Ao longo dos anos, Carlsson tinha cultivado um jardim em torno de sua casa,
plantado lilases, cujas mudas ele trouxera do jardim paroquial, árvores e arbustos 
frutíferos, além de cobrir as passagens com cascalho e construído
caramanchões. A propriedade vinha assumindo um ar senhorial. Ninguém podia
negar que o forasteiro trouxera conforto e bem"-estar consigo, que ele cuidara do
gado e das plantações, que ele remendara casa e cercas; até o peixe ele conseguira
vender por melhor preço na cidade, além de firmar acordos com o barco a vapor, a
fim de evitar as longas e onerosas viagens para o continente. Agora que
estava um pouco mais cansado e relaxara no seu trabalho, além de se dedicar mais
à construção de sua própria casa, começaram as reclamações. ``Trabalhem vocês
um pouco, para verem o tanto que é bom!'', respondia Carlsson. ``Cada um por si e
Deus por todos!''

Ele agora já cobrira de telhado a sua casa, plantara um jardim,
traçara as alamedas e removera o entulho. E construíra sua casa com uma certa
distinção, a ponto de ofuscar as outras a sua volta. No andar de baixo, só
havia dois quartos e a cozinha, mas mesmo assim, parecia mais imponente que
as outras casas do lugar, não se sabia bem a causa. Talvez fosse o pé"-direito
mais alto e a beirada do telhado que despontava mais longe das paredes; ou 
as cruzetas sobre os caibros, ou a varanda, que ele
erguera à frente da entrada, com degraus de acesso. Não eram luxos, mas
conferiam a casa um certo ar de mansão. A casa era vermelho"-ferrugem, com as pontas
das vigas pintadas de preto e ladrilhados; as vigas das janelas eram
brancas e a varanda, que era suspensa por quatro pilares, era pintada em azul.
Ademais, tivera a sabedoria de escolher bem o local ao pé do monte, de modo
que dois velhos carvalhos ficassem pareados em frente, feito o começo de uma
aleia ou parque. E quando se sentava na varanda, tinha"-se a melhor vista: a
enseada com os juncais, a verde campina da fonte e uma baixada que atravessava o
pasto dos bezerros, de onde se podia avistar as embarcações no canal.

Gusten acompanhava aquilo com ar zangado, desejando que tudo não existisse 
e vendo"-o como se observa um marimbondo, que faz sua casa debaixo do
teto e que se gostaria de afugentar, antes que ele lá coloque seus ovos e talvez
fique de vez com sua descendência. Mas ele não tinha forças para se opor, e por
isso não saía da inércia.

A senhora Flod, ainda doente, achava que tudo estava bem, pressentindo o
transtorno que adviria se ela partisse dessa para melhor. Não achava ruim que
seu marido --- pois afinal era isso o que ele era --- tivesse um lugar para se
abrigar e não fosse enxotado feito um coitado. Ela não entendia das questões
legais, mas intuía que nem tudo devia estar correto quanto às escrituras da
propriedade, o regime de herança e o testamento. Mas que isso viesse depois,
desde que ela não precisasse pensar sobre isso agora, que fosse no dia em que
Gusten quisesse se casar. Tais pensamentos já deviam estar na cabeça dele,
pois ele estava mudado, andando de lá para cá com ar estranho. 

Numa tarde, em fins de maio, quando Carlsson estava em sua nova cozinha, ocupando"-se com a
alvenaria do fogão, Clara veio e chamou por ele: 

--- Carlsson, Carlsson, o professor está aqui e trouxe consigo um senhor das 
Alemanhas que lhe deseja falar!

Carlsson tirou seu avental, enxugou as mãos e se arrumou para recebê"-los,
curioso sobre o motivo da visita inusitada. Saindo na varanda, ele topou com o
professor, que trazia consigo um senhor de olhar penetrante e longas barbas negras.

--- O Diretor Diethoff quer falar com você, Carlsson --- disse o professor,
indicando seu acompanhante com um gesto. Carlsson limpou o banco da varanda com
a mão e convidou"-os a sentar. O diretor não teria tempo para se sentar,
perguntando de maneira direta se a Ilhota do Centeio estava à venda. Carlsson
indagou sobre o motivo, pois a ilhota contava apenas com três alqueires, na
maior parte pedra com uma pequena mata de pinheiros, servindo apenas como
pastagem para as ovelhas.

--- O motivo é industrial --- disse o diretor, perguntando qual seria o preço.

Carlsson ficou sem saber o que dizer e pediu tempo para pensar, até que pudesse
desvendar o que dava à ilhota um valor inesperado. Mas essa não era a intenção
do diretor, repetindo a pergunta e pondo a mão sobre o bolso do paletó, onde 
um volume avantajado indicava não haver pouco dinheiro!

--- Ora, ela não deve ser assim tão cara --- disse Carlsson por fim ---, mas eu
preciso antes falar com a minha senhora e seu filho.

Ele desceu para a casa antiga; quedou"-se lá por um momento e voltou novamente.
Agora parecia acanhado e tinha dificuldade de proferir as palavras.

--- Diga o senhor mesmo, sr.~diretor, quanto gostaria de dar --- ele
finalmente disse.

Não, isso o diretor não queria dizer.

--- Se eu dissesse cinco, o senhor não acharia muito? --- Carlsson deixou 
escapar, com o coração na mão e suando frio.

O Diretor Diethoff abriu o paletó, sacou a carteira e entregou"-lhe dez notas de
cem coroas.

--- Aqui a senhor tem um sinal de mil, as quatro restantes eu entregar para a
senhor no outono. Está certo?

Carlsson quase que entregou o jogo; mas conseguiu controlar seus impulsos e
respondeu calmamente que estava tudo certo, pois na verdade ele havia pedido
quinhentas coroas, e não \textit{cinco mil}. Logo depois, os três desceram para
a senhora Flod e seu filho assinarem o contrato e dar recibo da compra.
Carlsson piscava e fazia sinais para os dois companheiros, para que entrassem no
jogo, mas eles não entendiam nada.

Finalmente, a senhora Flod botou seus óculos e leu, após já ter assinado.

--- Cinco mil! --- ela gritou. --- Meu Deus do céu, mas Carlsson tinha dito
quinhentos.

--- Não foi o que eu disse! --- respondeu"-lhe Carlsson. --- Você ouviu errado, Anna
Eva. Eu não disse cinco mil, Gusten? --- ele dava tantas piscadelas que até o
diretor percebeu.

--- Sim, foi bem a minha impressão! --- Gusten emendou como pôde. As
formalidades estavam encerradas e o diretor agora lhes declarou sua intenção de
começar na ilhota uma mina de feldspato. Ninguém ali tinha a menor ideia do que
era feldspato, e ninguém tinha se dado conta de que estavam sobre um tesouro
escondido, a não ser por Carlsson, é claro, que dizia agora ter tido alguns
pensamentos nessa direção, mas que por falta de capital estes não puderam ser
desenvolvidos. O diretor lhes explicou que feldspato era um mineral vermelho que
era muito usado na fabricação de porcelana. Dentro de oito dias, a casa do
gerente, que já estava encomendada na carpintaria, estaria erigida, e dentro de
quatorze dias o galpão de madeira para os operários estaria em seu lugar e
ao cabo de mais trinta o trabalho estaria em pleno andamento. E então partiu.

Essa chuva de ouro lhes veio de maneira tão inesperada, que mal tiveram
tempo para pensar sobre todas as consequências. Mil coroas sobre a mesa,
quatro mil no outono, por uma ilhota sem valor; quase não podiam acreditar. E
por isso passaram a tarde inteira em harmonia, especulando sobre as vantagens
que tirariam dessa inesperada bonança. Naturalmente, venderiam peixe e outros
produtos para todos aqueles trabalhadores e o gerente, lenha também, não havia
dúvida; e por certo o diretor em pessoa viria, talvez com a família, para
ali passar os verões; o que lhes possibilitaria um aumento da parte do
professor, e talvez Carlsson pudesse alugar sua nova casa para alguém e tudo
se arranjaria da melhor maneira possível para todos. Carlsson guardou pessoalmente o dinheiro na
escrivaninha, ficando ali sentado ao seu lado a metade da noite fazendo contas.

\asterisc

Durante a semana seguinte, Carlsson fez várias encomendas em Dalarö,
retornando com carpinteiros e pintores, ao que fazia pequenas recepções na
varanda, onde colocara uma mesa em que ficava sentado para beber conhaque e
fumar cachimbo. De lá, ele supervisionava o trabalho que agora avançava a
passos largos. Logo, todos os quartos estavam cobertos de tapetes, até a
cozinha, onde também foi instalado um fogão da marca Bolinder; as janelas foram
providas de postigos verdes, que podiam ser avistados de longe, a varanda foi
pintada de branco e rosa, além de ganhar uma cortina listrada de azul e branco
na parte ensolarada. Em volta da casa e do jardim estendia"-se uma treliça,
pintada de cinza com arremates brancos. O pessoal ficava parado por longos instantes, 
boquiabertos com tanta maravilha, enquanto Gusten mantinha"-se à parte,
atrás de uma esquina ou arbusto cerrado, raramente aceitando qualquer convite da
parte de Carlsson.

Era um dos sonhos de Carlsson, sonhado em noites de vigília, poder sentar feito
o professor numa varanda, recostado com indulgência, bebericar conhaque numa
taça de pé alto, admirar a vista e fumar um cachimbo --- na verdade charuto, mas
este ainda lhe era demasiado forte. E assim ele estava numa manhã, oito dias
depois, quando ouviu o sopro do barco a vapor no estreito próximo da Ilhota do
Centeio. ``Estão chegando'', pensou, e como patrão do lugar, quis ser
gentil e ir ao seu encontro. Entrou na casa e vestiu"-se apropriadamente,
mandou chamar Rundqvist e Norman para que o acompanhassem até a Ilhota do
Centeio e recepcionassem os senhores visitantes. Após meia hora, a canoa
largava do ancoradouro com Carlsson ao leme. Os rapazes foram lembrados de
remarem no mesmo ritmo, para mostrar que ali vinha gente direita. Quando deram a
volta no último pontão e o estreito se abriu à sua frente, delimitado pela Ilha
Grande de um lado, e pela Ilhota do Centeio no outro, uma vista magnífica se
ofereceu a eles. Ancorado no estreito, estava um barco a vapor, ornado de
bandeiras e sinaleiros, e entre este e a terra, corriam pequenas canoas com
marujos vestidos de jaquetas azuis e brancas. Sobre a praia rochosa, que reluzia
dos veios vermelhos de feldspato, estava um grupo de senhores e, não longe deles,
uma banda musical, cujos instrumentos de metal faziam um belo contraste com os
escuros pinheiros.

Nossos remadores de Hemsö se perguntavam qual o motivo daquilo tudo, passando a
canoa próximo da rocha para que pudessem ver e ouvir. Um, dois, três, bem quando
atracavam ao lado do agrupamento, ouviu"-se um forte sopro no ar, como se mil e
duzentos edredões tivessem levantado voo ao mesmo tempo, e aí um estrondo, que
parecia vir de dentro da rocha, e depois o som de algo se rachando, parecendo
que a própria ilhota tinha se partido ao meio.

--- Com mil diabos! --- foi tudo o que Carlsson conseguiu falar, pois no instante
seguinte caiu uma chuva de pedras na água ao redor da canoa, seguida de areia e
granizo de pedrinhas. Ouviram uma voz do alto da rocha; falando de grandes
empreendimentos e atividades econômicas, trabalho acumulado e aí algo em língua
estrangeira, que o povo de Hemsö não entendeu. Rundqvist pensou se tratar de um
sermão e tirou o chapéu, mas Carlsson entendeu que era o diretor a falar.

--- Sim, meus senhores --- concluía o diretor ---, temos muita pedra diante de nós
e eu termino por desejar que elas todas se transformem em pão!

--- Bravo!

A banda tocou uma marcha. Os senhores desceram à praia, todos carregando
pequenas pedras nas mãos, que eles manuseavam entre risos e brincadeiras.

--- Ei, vocês, o que fazem aí na canoa? --- gritou para o povo de Hemsö um senhor
que trajava o uniforme da marinha, enquanto eles se debruçavam sobre os remos.

Eles não sabiam bem o que responder, pois não tinham achado que era perigoso
ficar ali apreciando o espetáculo.

--- Mas se não é o patrão em pessoa! --- esclareceu o diretor Diethoff, que
chegara até eles. --- Trata"-se de nosso anfitrião aqui --- disse ele, apresentando
Carlsson aos outros. --- Venha tomar o desjejum conosco!

Carlsson não acreditou em seus ouvidos, mas convenceu"-se logo de que o convite era
para valer, e logo ele estava no convés do vapor, sentado a uma mesa que nunca
vira igual. Primeiro, ele quis fazer cerimônia, mas os senhores foram tão gentis
que nem deixaram Carlsson tirar seu avental de couro. Enquanto isso, Rundqvist e
Norman foram servidos junto com os funcionários, na proa.

Carlsson jamais imaginara um paraíso mais bem"-aventurado. Havia comida que ele nem
sabia o nome e que derretia na boca feito mel, comida que ardia na garganta como
se fosse aguardente, comida em todas as cores; e seis taças diante dele e de
cada um dos outros comensais; e bebia"-se de vinhos cujos aromas eram como de
florais ou como um beijo de uma moça, vinhos que ardiam no nariz, que faziam
cócegas e que faziam rir. E durante todo o tempo, a música soprava tão belamente
que fazia comichões na base do nariz, como quando se quer chorar, ou provocava
calafrios na fronte e sensações tão boas no corpo inteiro que se pensava que se ia morrer. 

E quando tudo terminou, o diretor dirigiu"-se ao anfitrião do
lugar, elogiando"-o por ter honrado sua palavra e não ter abandonado as
modernizações pelos ganhos incertos em outras áreas, onde a carestia andava de
braços dados com o luxo. E aí brindaram com ele. Carlsson não sabia quando era
para rir ou ficar sério, mas ele via os senhores rindo quando aparentemente
tinham dito coisas muito sérias, e aí ele ria junto. 

Após o desjejum, foram oferecidos café e cigarros e todos se levantaram da mesa. Carlsson, com a
nobreza de um homem feliz, foi até a proa para conferir se os rapazes tinham algo para
comer, quando o diretor o convidou para o seu camarote um instante. 

Quando entraram, o diretor Diethoff lhe fez a proposta de que Carlsson, para consolidar
sua posição e poder se apresentar com mais autoridade junto aos trabalhadores na
ilhota, se necessário fosse, adquirisse algumas ações da empresa. 

``Bem, veja, eu não entendo nada dessas coisas'', ponderou Carlsson, que possuía tino comercial o
suficiente para saber que não se fechava negócios após uma bebedeira. Mas o
diretor não o largou e, meia hora depois, Carlsson possuía quarenta ações, a cem
coroas cada, da Feldspato Eagle Ltda., além da solene promessa de ser apontado
como auditor adjunto --- Carlsson lhe pediu para escrever o termo num papel!
Quanto ao desembolso, não se falou nada; seria ``pouco a pouco'' e ``por
conta''. Daí beberam todos café, conhaque, ponche e água com gás, de modo que o
relógio mostrava seis horas quando Carlsson finalmente embarcou. À saída,
fizeram"-lhe uma guarda de honra, o que ele não entendeu, apertando a mão de cada
um dos marujos que estava à escada e convidando"-os a fazer uma visitinha quando
por lá passassem novamente. Com suas quarenta ações e cupons correspondentes,
ele se deixou remar para casa, sentado ao leme com um charuto na boca e uma
garrafa trançada de palha, cheia de ponche, entre os joelhos.

Quando pôs os pés em casa, inchado de tanta bem"-aventurança, ofereceu ponche até
ao pessoal da cozinha, mostrando as ações que pareciam enormes cédulas de banco;
queria mandar chamar ali o professor, enfrentando as resistências dos outros com
o argumento de que ele agora era auditor adjunto e, portanto, tão grã"-fino quanto um
músico alemão, que este na verdade não era um acadêmico e, por isso, nem
professor de verdade. Ele tinha planos grandiloquentes; queria fundar uma
importante empresa salgadora de arenques no arquipélago, trazer tanoeiros da
Inglaterra, fretar navios com sal diretamente da Espanha! Na mesma toada, ele
falava da atividade primária da lavoura, seus representantes e futuro,
expressando seus temores e esperanças. Eles beberam seu ponche, envolvidos pela
fumaça do charuto e as brilhantes miragens sobre o esplendoroso futuro de Hemsö.

Carlsson estava nos píncaros e teve acessos de vertigem. A lavoura foi deixada
de lado e as visitas à Ilhota do Centeio ficaram cada vez mais constantes. Travou 
amizade com o gerente e ficava sentado na varanda deste tomando conhaque
e água gaseificada, enquanto viam os operários quebrar as pedras em busca dos
veios de quartzo, atividade necessária para embarcarem o conteúdo inteiro da
ilhota em vários carregamentos. O gerente era um antigo mineiro e sabia pôr"-se 
habilmente em bons termos com o acionista e auditor adjunto, além de ter uma
precisa noção de quanto tempo o negócio duraria ali. A instalação da mineradora
teve suas pequenas influências sobre o bem"-estar físico e moral dos moradores de
Hemsö e a presença de trinta operários solteiros começara a mostrar seus
efeitos. A calma já não existia mais. Explodia"-se e detonava"-se dias inteiros
na rocha; barcos a vapor entravam com barulho no estreito; iates atracavam,
expelindo massas de gente de suas entranhas. À tarde, os operários apareciam
perto das casas, dando voltas no poço e no celeiro; gracejavam com as
moças, improvisavam bailes, bebiam com os rapazes e, vez ou outra, aprontavam
confusão. O povo passava as noites em claro e de dia não prestavam para nada,
dormiam sobre os pastos e não se aguentavam de pé ao fogão. De tempos em tempos,
recebiam a visita do gerente. Serviam"-lhe logo um café, e como não podiam lhe
oferecer aguardente, havia sempre que se ter um pouco de conhaque na casa. Ao
mesmo tempo, podiam vender peixe e manteiga, e o dinheiro entrava aos montes,
portanto viviam bem e a carne era mais frequente nos pratos.

Carlsson estava mais gordo e andava constantemente ressacado, porém sem perder o
controle. O verão passou como uma festa ininterrupta para ele, que dividia seu
tempo entre assuntos da comarca, mineração e embelezamentos da casa. O outono
chegou e ele tinha estado fora fazendo inspeção de incêndio por oito dias,
retornando a casa uma manhã bem cedo, quando foi recebido pela esposa com a
preocupante notícia de que devia ter ocorrido algo na Ilhota do Centeio. É que
já havia quatro dias que lá se fizera silêncio; nenhum estrondo se ouvira e nem o
assovio dos vapores. O povo da ilha andava ocupado com a debulha e por
isso ninguém pudera ir por conta própria. Do gerente não se tivera mais notícia,
e os trabalhadores tinham sumido da ilha à tarde. Portanto, devia ter acontecido
qualquer coisa. Para averiguar a situação, Carlsson mandou ``selar'', como ele
dizia toda vez que se mandava remar até a mina. A canoa agora estava pintada de
branco com uma faixa azul, e, para lhe dar um ar mais patronal, quando ele se
sentava ao leme, mandara confeccionar um assento estofado de pano, podendo se
sentar ereto enquanto manobrava, além de ter adestrado Rundqvist e Norman para
remarem em compasso, causando uma bela impressão quando vinha chegando. A viagem
foi rápida, instigada pela curiosidade e a apreensão, e quando chegaram à altura
da Ilhota do Centeio, eles se espantaram com o abandono que ali havia. Reinava
um silêncio sepulcral e não se via vivalma. Desembarcaram e
subiram por entre o cascalho até a mina. A casa do gerente não estava mais lá;
todos os instrumentos e ferramentas tinham sido removidos, estando apenas o galpão
no lugar, mas completamente esvaziado, os objetos removíveis tinham todos
sido levados: portas, janelas, bancos, dormitórios.

--- Dá a impressão de que picaram a mula! --- conjecturou Rundqvist. 

--- É o que parece! --- concluiu Carlsson e mandou ``selar'' novamente, mas agora para ir até Dalarö,
onde ele supunha terem lhe deixado uma carta no correio. E exatamente ali
estava uma longa carta do diretor, explicando que a empresa encerrara suas
atividades devido ao pequeno aproveitamento do recurso natural. E conquanto ao
montante de quatro mil que a Carlsson lhe fazia jus, este se equivalia às
quarenta ações que ele subscrevera, mas ainda não desembolsara, não havendo,
portanto, nenhuma questão pendente entre a companhia, o supracitado Carlsson e
sua consorte. ``Portanto, um prejuízo de quatro mil'', pensou Carlsson. Mas foi bom
enquanto durou. E seguindo sua natureza de pássaro marítimo, apesar de ser da
terra, ele se chacoalhou e estava novamente seco, e ainda mais seco ele se
sentiu quando leu no \textit{post scriptum} que todo o material deixado para
trás pertencia a Hemsö, se eles o quisessem.

Não obstante, Carlsson voltou um pouco cabisbaixo dessa viagem, despojado de um
monte de dinheiro e seu venerável título. Gusten quis fazer do caso um cavalo de
batalha e ir à forra, mas Carlsson passou uma régua sobre toda a questão:

--- Tsc tsc, como se isso valesse a pena. Não vale o incômodo.

No dia seguinte, ele estava em plena atividade com seus três homens e uma balsa,
para recuperarem tábuas e telhas da Ilhota do Centeio. E antes que se dessem
conta, ele tinha erguido uma cabana de verão, com quarto e cozinha, situada num
local, perto do estreito, de que ninguém se dera conta e que dava vista
para a vila e a enseada. O verão estava findando, junto com seus airosos sonhos.
O inverno estava à porta e o ar se punha mais pesado, os sonhos se embotavam 
e a realidade apresentava nova feição, clara para alguns, ameaçadora para outros.

\chapter[Os sonhos de Carlsson se realizam\ldots]{Os sonhos de\break Carlsson se realizam; \subtitulo{a
escrivaninha é mantida sob estrita\break vigilância, mas o inspetor põe um\break ponto final na questão}}
\hedramarkboth{Os sonhos de Carlsson se realizam}{Strindberg}

\textsc{O casamento} de Carlsson, mesmo recente, não era exatamente o que
se chama de feliz. A mulher já contava com certa idade, embora ainda não fosse
um fardo, enquanto Carlsson se encontrava às portas da idade perigosa. 
Com quarenta anos recém"-completados, ele até o momento se ocupara mais com o ganha"-pão e o
progresso, tendo que deixar passar a moça que havia amado. Agora que o objetivo
fora atingido e um futuro tranquilo se lhe avistava, a carne começava a cobrar a
sua parte. O chamado era mais forte que de costume, pois tivera um ano tranquilo
e talvez, também, por ter até então se reprimido mais do que deveria.
Sentado no calor da cozinha, seus pensamentos começavam a bailar, e seus olhos
se habituaram a seguir o corpo jovem de Clara, quando esta ia e vinha pela casa.
O olhar aos poucos ia parando, repousando em descanso, fazia depois mais alguns
rodopios, voava para lá, voltava para cá. Por fim, a menina estava lá, dentro do
olho de Carlsson, onde quer que ele estivesse, ele a via. Mas havia mais alguém que também
estava vendo, não a Clara, mas os olhos que a seguiam, e quanto mais via, mais
percebia, e veio como que uma inflamação sobre estes olhos, que ardiam e
escorriam. Estavam há alguns dias da véspera de Natal. A escuridão viera, mas a
lua se levantara e iluminava por sobre pinheiros cobertos de neve, sobre o
espelho da enseada e o chão branco. Um vento impiedoso do norte varria a neve
seca à sua frente. Dentro da cozinha, Clara colocava lenha dentro do forno,
enquanto Lotten trabalhava a massa. Carlsson estava em seu canto, à
escrivaninha, fumando seu cachimbo e ronronando feito um gato no calor; seus
olhos passeavam, se aquecendo em deleite quando paravam sobre os braços alvos de
Clara, que apareciam debaixo de suas mangas de linho.

--- Você vai ordenhar, antes de nós varrermos? --- perguntou Lotten.

--- Sim, é o que eu vou fazer --- respondeu"-lhe Clara, vestindo um casaco de pele
de ovelha, após ter colocado de lado o ancinho e a vassoura. Ela acendeu a
lanterna e saiu pela porta. Atrás dela, Carlsson se levantou e a seguiu.

Após uns instantes, a senhora Flod saiu de seus aposentos e perguntou por
Carlsson.

--- Ele foi ao estábulo, atrás de Clara --- respondeu Lotten.

Sem querer ouvir mais, a senhora Flod também pegou uma lanterna e saiu.

Lá fora, havia um vento cortante, mas ela não se incomodou em retornar e buscar
um agasalho, afinal era um trajeto curto. Estava escorregadio nos arredores e a
neve fazia pequenos redemoinhos, parecendo pó de farinha, mas ela logo chegou  ao
estábulo, entrando perto dos cochos, onde estava mais quente. Lá ela ficou
parada, ouvindo que alguém cochichava perto do cercado das ovelhas. Na tênue luz
do luar, que passava pelas teias de aranha e moinha de palha nas janelas, ela
via as vacas virarem as cabeças em sua direção, olhando"-a com seus olhos que no
escuro fosforesciam verdes. O banquinho de ordenha estava lá, bem como o tarro.
Mas não era isso que ela queria ver, era outra coisa, algo que ela teria dado
tudo para não ver; algo que a atraía feito uma degola e que a amedrontava
mortalmente.

Ela passou sobre os montes de feno, avançando pelo estábulo e chegando até as
ovelhas. Estava escuro e silencioso; a lanterna estava apagada, mas sua lamparina
ainda fumegava. As ovelhas se moveram, um tanto inquietas, e mastigavam o feno com barulho. 
Não, não era isso que ela queria ver. Ela adentrou ainda mais e chegou às galinhas, que
estavam empoleiradas em seus poleiros e cacarejavam baixinho, como se há pouco
tivessem sido despertadas. A porta estava aberta e ela saiu novamente ao luar.
Dois pares de pegadas, umas maiores do que as outras, tinham deixado rastros na
neve; elas projetavam sombras azuis e levavam até a porteira da campina, que
estava aberta. Ela as seguiu, como se estivesse sendo arrastada por alguém, as
pegadas no solo eram como uma corrente à qual ela estava presa e pela qual era
puxada de um ponto desconhecido para além da porteira. E a corrente a puxava e
puxava, levando"-a para dentro da campina, pela mesma porteira, por debaixo dos
mesmos arbustos de avelã, onde ela outrora, num momento tenebroso, tinha passado
parte de uma noite de verão e festa, que ela não queria recordar. Agora, os
arbustos de avelã estavam pelados e exibiam apenas seus pequenos botões de flor,
parecidos com lagartas de couve, enquanto os carvalhos chacoalhavam suas folhas
marrons e secas ao vento, tão finas que se podia ver as estrelas e o céu negro
esverdeado através delas. E quanto mais seguia a corrente, mais ela a puxava;
serpenteando por entre os pinheiros, que sacudiam sua neve sobre os cabelos
dela, grisalhos e raleados, quando ela esbarrava neles, jogando também neve sobre
sua malha de lã quadriculada, deixando"-a cair sobre seu pescoço e costas,
gelando e molhando"-a.

Mais e mais, ela adentrava no bosque, onde os tetrazes voavam de seu repouso noturno,
assustando"-a. Em seguida, cruzou um pântano onde os torrões cediam
quando pisados e as cercas rangiam quando transpostas.

Dois em dois seguiam os passos, uns menores e os outros maiores, lado a lado, às
vezes uns pisavam sobre os outros, passavam uns no entorno dos outros, como se
estivessem em dança. Seguiam sobre um campo seco, onde a neve tinha sido varrida
pelo vento, sobre pedreiras e brejos, por amontoados de estacas e ao lado de
árvores derrubadas pelo vento.

Ela não sabia há quanto tempo caminhava, mas sua cabeça estava gelada e suas
mãos dormentes, ela punha suas mãos dentro da roupa e soprava"-as de vez em
quando. Queria retornar, mas já era tarde e agora tanto fazia voltar como
seguir adiante. Um pouco mais à frente, ela chegou a um bosque de choupos, cujas
folhas tremiam como se castigadas pelo vento frio do norte, e assim ela
chegou a um valado. O luar esparramava"-se com nitidez, e ela teve agora a
certeza de que eles tinham se sentado ali. Ela discerniu a impressão do vestido de
Clara, do casaco de pele de ovelha. Fora ali, então! Seus joelhos tremiam, tinha
frio como se o sangue tivesse se transformado em gelo, ardia como se houvesse
água fervente em suas veias. Ela se sentou sobre o valado, chorou, gritou, ficou
subitamente calma, levantou"-se e o transpôs. À sua frente, a enseada estava
espelhada e negra, e ela avistou as luzes da casa e uma
luz dentro do estábulo. O vento soprava feito uma faca e ela o sentia atravessar
suas costas, agarrando seus cabelos e congelando suas narinas; quase correndo,
ela alcançou a água congelada e avançou sobre a placa instável, ouvia o
junco assoviando ao vento e se partindo sob seus pés, tropeçou e caiu sobre
uma boia que estava presa no gelo, mas se levantou e correu novamente, como se a
morte estivesse atrás dela e queimasse suas costas. Ela estava chegando à margem
oposta quando seu pé atravessou o gelo, que naquele lugar cobrira a correnteza
da água, feito uma janela de vidro sobre o fundo lodoso, agora tilintando e
quebrando sobre seu peso. Sentiu o frio subir por suas pernas, mas não teve
coragem de gritar, pois poderiam vir e perguntar por onde ela andara. Tossindo
como se o seu peito fosse estourar, ela saiu arrastando"-se da vala, subindo
penosamente a encosta até a casa, até sua cama, onde ela se deitou e pediu a Lotten
que acendesse um fogo na lareira e fervesse um bule com chá de sabugueiro; e
assim ficou prostrada. Ela deixou que a despissem e que a cobrissem de
cobertas e peles de ovelha, que colocassem lenha na fogueira, mas mesmo assim,
sentia frio. Finalmente, ela mandou chamar Gusten, que estava na cozinha.

--- Está doente, mãe? --- perguntou ele com sua calma corriqueira.

--- Estou nas últimas --- respondeu"-lhe arquejante a mãe ---, desta eu não
escaparei. Feche a porta e vá até a escrivaninha. A chave está atrás do chifre
com pólvora, você sabe bem onde! --- Gusten ficou pálido, mas obedeceu. --- Abra a
portinhola; puxe a terceira gaveta do lado esquerdo e pegue o grande envelope de
papel. Assim\ldots{} coloque"-o no fogo.

 Gusten obedeceu, e logo o envelope estava incandescente, retorcendo"-se e
 se transformando em cinzas.

--- Feche a portinhola, meu filho, e tranque"-a! Guarde consigo a chave! Sente
agora ao meu lado e me escute, pois amanhã já não estarei mais falando --- Gusten
se sentou, chorando um pouco, pois agora sabia que a coisa era séria.

--- Quando eu cerrar os olhos, você irá pegar o sinete de seu pai, que está com
você, e selar todas as fechaduras, até chegarem aqui os homens da lei.

 --- E Carlsson? --- perguntou titubeante o filho.

 --- Ele já tem a parte dele; essa certamente ninguém consegue lhe tirar, mas ele não terá
 nada além; e se você conseguir comprá"-la de volta, faça"-o! Que Deus lhe proteja, 
 Gusten; você poderia ter estado mais presente no meu casamento, mas 
 deve ter tido suas razões. E, veja lá, quando eu me for, seja razoável. Nada de 
 caixão estofado com placa de prata; escolha um daqueles amarelos, que se acham
 fácil na cidade; e não vá me chamar uma multidão; já um dobre de sino, disso
 eu gostaria, e se o pastor quiser me encomendar com algumas palavrinhas, ele
 será bem"-vindo; dê a ele o cachimbo com a boquilha de prata que era de seu pai,
 e um quarto de ovelha para a mulher dele; e depois, Gusten, crie juízo e
 se case; escolha uma moça de quem você goste e não saia do lado dela, mas encontre uma
 que combine com você, se ela tiver dinheiro não faz mal, mas não vá escolher
 alguém abaixo do seu nível; essas vão lhe exaurir feito carrapatos; lembre"-se
 que ``crianças iguais brincam melhor''. Se você quiser ler um pouco
 para mim, talvez eu consiga dormir.

 A porta se abriu e Carlsson entrou de mansinho; ele estava silencioso, mas
 ainda confiante.

--- Você está doente, Anna Eva? --- ele perguntou secamente. --- Vamos chamar o
médico.

--- Não é necessário --- respondeu a mulher e se virou para a parede.

Carlsson adivinhou a situação e quis se reconciliar.

--- Você está zangada comigo, Anna Eva? Ora, não vale a pena você se zangar por
uma coisa à toa! Você quer que eu leia o livro para você?

--- Não é preciso! --- foi tudo que a mulher respondeu.

Carlsson, que notara que não havia mais nada a fazer e que não gostava de
desperdiçar trabalho, deixou a situação como estava e sentou"-se no sofá de madeira.
Já que estava tudo resolvido e a mulher não queria negociar, não havia nada mais
a acrescentar; quanto a Gusten, eles certamente se entenderiam depois. Ninguém
pensou em chamar um médico, as pessoas do lugar estavam acostumadas a encarar a
morte, além do mais, todas as comunicações com a terra estavam cortadas. Por dois
dias e duas noites, Carlsson e Gusten ficaram assim, observando o quarto e
observando"-se um ao outro, quando um adormecia na cadeira ou no sofá, o outro
dava um cochilo com um olho aberto. Tão logo um se mexia, o outro despertava. Na
manhã da véspera do Natal, a senhora Carlsson faleceu. Para Gusten, foi como
se só naquele momento se cortasse o cordão umbilical, como se ele tivesse sido
arrancado do ventre materno e agora fosse um homem independente. Após ter
cerrado os olhos da mãe e colocado o livro de salmos debaixo de seu queixo, para
que a boca não se abrisse, ele acendeu uma vela na presença de Carlsson, pegou o
sinete com a cera e selou a escrivaninha. Os sentimentos reprimidos acordaram;
Carlsson aproximou"-se e posicionou"-se de costas para a escrivaninha.

--- Mas o que você está fazendo, rapaz? --- foi o que pôde dizer.

--- Eu não sou mais um rapaz --- respondeu Gusten ---, eu sou o senhor de Hemsö, e
você é apenas o agregado.

--- É o que veremos! --- exclamou Carlsson.

Gusten tirou a espingarda da parede; levantou o gatilho de modo a ver a
munição, dedilhou o cano e berrou pela primeira vez em sua vida:

--- Para fora! Ou lhe dou um tiro!

--- Você está me ameaçando?

--- Sim, não há nenhuma testemunha! --- respondeu Gusten, que parecia ter estado
com gente da lei recentemente.

Era uma mensagem clara e Carlsson a entendeu bem.

--- Espere só quando chegar a hora da partilha --- ele disse e saiu para a cozinha.

Foi um Natal lúgubre. Uma defunta dentro de casa e nenhuma possibilidade de
mandar avisos ou encomendar o caixão; a neve caía sem parar, as águas congeladas
da enseada não eram firmes o suficiente para aguentar o peso de uma pessoa, nem
permitiam que se navegasse por elas; botar uma embarcação na água era
impossível, a água misturada com o gelo não permitia que se remasse, andasse ou
velejasse. Carlsson e o senhor Flod, como Gusten queria ser chamado agora,
fingiam que o outro não existia, sentavam"-se à mesa sem trocar uma palavra. A
casa estava uma desordem; ninguém dava as ordens e cada um achava que o outro
iria fazê"-lo, deixando tudo como estava. O dia de Natal veio cinzento, nublado e
com mais neve. Ir à igreja era tão impossível quanto a qualquer outro
lugar; por isso Carlsson leu um sermão na cozinha. Todos sentiam a presença da
defunta e nenhuma alegria de Natal lhes era possível. A comida foi preparada com
desleixo, nada ficou pronto quando devia e todos estavam insatisfeitos. Havia
algo pesado no ar, tanto dentro quanto fora, e como a patroa estava na sala,
todos ficaram na cozinha. Parecia um acampamento, e quando não estavam bebendo
ou comendo, dormiam sobre o sofá, sobre a cama; tirar o baralho ou o acordeom
nem lhes passou pela cabeça.

O dia após o Natal veio e se foi com o mesmo pesar, com o mesmo fastio; mas
a paciência do senhor Flod por fim se esgotara. Ciente de que mais atraso só iria causar
um sofrimento ainda maior, pois o corpo já começara a se transformar, ele levou Rundqvist 
consigo até a carpintaria, onde juntos fizeram um caixão e o pintaram de amarelo. Com o 
que acharam na casa, envolveram a defunta. E assim veio o quinto
dia. Como o tempo não dava sinais de melhora, e parecendo que teriam de esperar
ainda mais quatorze dias, tomaram a decisão de levá"-la à igreja a qualquer preço para 
que fosse enterrada. Com essa finalidade, colocaram a maior barca na
água e os homens se prepararam para uma viagem pelo gelo, munidos de trenós,
picaretas, machados e cordas. Cedo no sexto dia, eles partiram para a
perigosíssima viagem. Às vezes, havia um talho de correnteza no gelo, e aí
aproveitavam para remar; ora chegavam a uma enseada totalmente congelada, era a
vez de colocar a barca sobre os trenós para puxar e empurrá"-la; o pior era a
lama de água com gelo, onde os remos apenas subiam e desciam, sem que a barca
avançasse mais que algumas polegadas por vez. Às vezes preferiam andar à sua
frente e abrir uma vala com as picaretas e os machados, mas ai daquele que
errasse e pisasse em falso, onde a correnteza da água tinha comido a crosta de
gelo até se tornar uma fina membrana.

A tarde já caía e não houve tempo de comer ou beber,
restando"-lhes ainda a última enseada para transpor. De onde a olhavam, ela se
descortinava como uma grande planície enevoada, com pequenas protuberâncias
redondas aqui e ali, que eram os rochedos cobertos de neve. O céu estava azulado, enegrecido a
leste, e prometia mais neve; as gralhas voavam rumo à terra buscando abrigo
para a noite; às vezes ouviam estrondos no gelo, como se este estivesse
começando a derreter, e ouviam ao longe o grito das focas no mar. A enseada
estava aberta para o mar na parte leste, mas não se via nenhuma passagem navegável
que entrasse por ela. Estranhamente, tinham a impressão de ouvir os chamados dos
tetrazes na faixa de água, mas como não tinham ouvido nenhuma notícia da terra por
quatorze dias, não sabiam se os faróis estavam apagados. De qualquer maneira,
era o que supunham, por estarem entre o Natal e o Ano Novo.

--- Isto aqui não vai adiante! --- pronunciou"-se Carlsson, que até então fora o
mais calado.

--- Sim, vamos chegar lá --- disse o senhor Flod e colocou o ombro contra o trenó
---, mas vamos parar no Rochedo das Gaivotas para comer algo.

Seguiram no rumo daquele rochedo, que ficava no meio do estreito.
Entretanto, este se encontrava mais longe do que pensavam e alterava sua aparência
à medida que se aproximavam, ficando finalmente à distância de um cabo.

--- Vala à frente! --- gritou Norman, que fazia as vezes de batedor ---, fiquem à esquerda!

Os trenós foram desviados à esquerda, e depois ainda mais para a esquerda, e
finalmente tinham dado a volta no rochedo. Ele tinha se separado do gelo, talvez
pelo último calor do sol, talvez pelas correntezas da água, e parecia
inatingível por todos os lados, ao menos de trenó. A noite estava caindo, a
situação começava a ficar desesperadora, quando Flod, que comandava a ação,
desenvolveu uma estratégia. A barca deveria primeiro deslizar, depois ser
empurrada para dentro da vala de água e neste exato momento, todos eles deveriam
jogar"-se dentro dela e pegar nos remos. Assim foi decidido, e assim foi feito.

--- Um, dois, três! --- comandou Flod. A barca tomou impulso, soltou"-se dos
trenós, mas deu uma guinada que acabou lançando o caixão ao mar.

No desespero que se seguiu, enquanto Norman e Rundqvist se safaram entrando na
barca, Flod e Carlsson, que estavam atrás, esqueceram de pular para dentro dela
e ficaram parados sobre a borda do gelo. O caixão não estava selado e se encheu
de água, afundando antes que alguém conseguisse fazer outra coisa que não pensar
na própria pele.

--- Nós dois teremos que concluir o caminho para a igreja hoje mesmo! --- ordenou
Flod, que naquele dia mostrava grande determinação, mas pouco sentido prático.
Carlsson veio com objeções, mas Gusten lhe perguntou se ele achava melhor
passarem a noite inteira sobre o gelo, e a isso ele não tinha o que responder,
vendo que não havia qualquer esperança de alcançarem o rochedo. A essa altura,
Rundqvist e Norman tinham chegado à terra, gritando para os companheiros que os
seguissem, mas Flod apenas lhes respondeu acenando adeus e apontando para o
sul, onde estava a igreja.

Carlsson e Flod andaram um longo percurso em silêncio; Gusten com a ``lança de
gelo'', que ele metia no gelo para provar se este os suportaria; Carlsson o
seguia com a gola do casaco levantada e de ar soturno, após o súbito e trágico
fim que sua esposa teve e pelo qual certamente o culpariam.

Após uma caminhada de meia"-hora, Gusten parou para respirar e aproveitou
para olhar ao redor dos rochedos e do litoral para se orientar.

--- Diabos, não é que estávamos errados! --- ele murmurou. --- Aquilo não era o
Rochedo das Gaivotas, o rochedo está aqui --- ele disse apontando para o leste.
--- Lá está o abeto de Gillöga.

Numa ilha alongada, em direção à terra, ele apontava para um abeto, que tinha
sido deixado solitário num alto desmatado e que com seus dois galhos
remanescentes parecia um telégrafo óptico, servindo de sinal orientador tanto
marítimo quanto terrestre.

--- E ali está Träslkär.

Ele falava para si mesmo e sacudia a cabeça.

Carlsson ficou amedrontado, pois ali ele não estava em casa, tendo uma confiança
ilimitada nos conhecimentos de Gusten. Este, no entanto, apenas se recompôs e
mudou o curso mais para o sul. Enquanto isso, a penumbra estava caindo. A neve
ainda clareava alguma coisa, permitindo"-lhes a orientação. Eles não falavam uma
palavra, Carlsson se mantendo o mais perto que podia dos passos de seu guia. De
repente, este parou como se ouvisse algo. Carlsson, destreinado,  não captava
nada, mas pareceu a Gusten ter ouvido um fraco rumor do leste, onde
uma barreira de nuvens, mais pesadas e escuras que as do sul, tinha aparecido.
Eles ficaram parados por um instante, até Carlsson perceber um ruído surdo e
o rumor de pancadas, que se aproximavam.

--- O que é isso? --- perguntou, se encolhendo ao lado de Gusten.

--- É o mar! --- este lhe respondeu. --- Dentro de meia"-hora teremos o vento leste
com a neve; se o pior acontecer, o gelo pode se mover e rachar. E aí nem o diabo
saberá o que será de nós. Vamos apressar o passo!

Ele começou a correr de leve e Carlsson o seguiu; a neve soprava entre seus pés
enquanto o ruído parecia lhes seguir.

 --- Estamos perdidos! --- gritou Gusten e parou, ele apontava para uma luz que
 brilhava atrás de um rochedo a sudeste. --- O farol está aceso pois o mar está aberto!

Carlsson não entendeu o perigo, mas percebia que algo muito grave se
passava, visto que agora o próprio Gusten estava amedrontado.

 Agora, o vento leste os achara e eles podiam ver a uma pequena distância a
 parede de neve se aproximando como um escudo sombrio. Em poucos segundos, 
 estavam cercados por uma nevasca que caía tão pesadamente que mais parecia ser preta feito
 carvão. Tudo escureceu ao redor deles e a luz do farol, que há pouco lhes
 indicava o caminho com seu brilho diáfano, agora sumira completamente.

 Gusten apertou o passo e Carlsson tentava segui"-lo de perto;
 mas era corpulento e não conseguia manter o ritmo. Esbaforido, ele pediu a
 Gusten para irem mais devagar, mas este não tinha vontade alguma de se sacrificar,
 correndo pela própria vida. Carlsson o agarrava pela aba do casaco, pedindo
 encarecidamente que o outro não o abandonasse, prometeu o que podia e o
 que não podia, jurava por tudo o que era mais sagrado, mas nada adiantava.

--- Cada um por si e Deus por todos! --- respondeu"-lhe Gusten, pedindo a Carlsson
que se mantivesse alguns passos atrás, porque o gelo poderia se
partir com o peso dos dois. E era isso que ele parecia fazer, pois atrás deles,
vinha cada vez mais forte um ruído de algo se partindo, e o que era pior, eles
ouviam o barulho da água, agora tão nitidamente que podiam identificar as
batidas das ondas contra os rochedos e o gelo, junto com as gaivotas despertadas
pelo que parecia ser uma presa inesperada.

Carlsson arquejava e ofegava; a distância entre ele e Gusten aumentou e após um
tempo ele se viu sozinho, correndo na escuridão. Parou para procurar as
pegadas do outro, não as viu; chamou, mas sem resposta. Era a solidão, o escuro,
o frio e a água que vinham com a morte. Avivado pelo pavor, ele retomou o passo acelerado
e correu até ver os flocos de neve caírem para trás, mesmo
indo na mesma direção que ele. Novamente, ele gritou pelo companheiro.

--- Siga a direção do vento, assim você chega à terra firme no oeste! --- ele
ouviu uma voz fugidia gritando dentro do escuro, e tudo voltou ao que era.
Mas, a essa altura, Carlsson já não tinha mais forças para correr. Exaurido, ele
reduziu a velocidade até andar, passo a passo, sem poder oferecer mais
resistência, ouvindo a aproximação do mar atrás de si, retumbante, resfolegando,
em soluços, como se ele houvesse saído à noite para caçar suas vítimas.

\asterisc

 O pastor Nordström tinha ido se deitar às oito horas da noite, ele lera um pouco
 do jornal da paróquia e depois disso caiu num sono profundo. Às onze horas, ele
 sentiu sua mulher lhe cutucando e ouviu um chamado.

--- Erik! Erik! --- ele ouvia dentro do sono.

--- Mas o que foi, você não consegue ficar quieta? --- ele grunhiu entre o sono e a vigília.

--- Quieta?! Você por acaso acha que eu não sei ficar quieta? --- temendo entrar
numa complicada discussão, o pastor logo assegurou que ela sabia, sim, ficar
quieta, acendeu um fósforo e perguntou o que se passava.

--- Há alguém chamando na frente da casa! Você não está ouvindo?

O pastor aguçou os ouvidos e colocou os óculos para escutar melhor.

--- Sim, por minha alma, é verdade! Quem será?

--- Vá lá e veja! --- respondeu"-lhe a esposa e deu mais um cutucão no velho. O
pastor vestiu as ceroulas e o casaco de pele, meteu os pés nos tamancos, desceu
a espingarda da parede e colocou"-lhe um cartucho, ajuntou a pólvora e foi ao
encontro do desconhecido.

--- Quem está aí? --- ele gritou.

--- Flod! --- respondeu uma voz grave, por trás dos arbustos de lilases.

--- Que diabos aconteceu para você chegar uma hora dessas! Sua mãe está com o pé
na cova?

--- É pior do que isso! --- disse Gusten com voz triste. --- Nós a perdemos!

--- Vocês a perderam?

--- Sim, nós a perdemos no mar.

--- Mas, pelo amor de Deus, entre, não fique aí no frio!

À luz da lamparina, Gusten parecia um espectro, afinal ele ainda não tinha
comido ou bebido naquele dia e tivera que correr mais rápido que o vento leste.
Depois de ter contado para o pastor, num fôlego só, tudo que se passara, este
foi até sua mulher e obteve, após alguns segundos de tormenta, a chave de um
certo armário na cozinha, para onde ele levou seu hóspede náufrago. E logo
Gusten estava sentado à grande mesa da cozinha, enquanto o pastor servia o
esfomeado com aguardente, toucinho, linguiça e pão.

Enquanto isso, deliberavam o que se podia fazer pelos outros. Sair pela noite e
mobilizar a vila era desperdício de esforços; acender fogos na orla era
perigoso, pois poderia enganar as embarcações, isso se a luz conseguisse
atravessar a nevasca.

Os rapazes na ilhota não lhes preocupavam tanto, pior devia ser o destino de
Carlsson. Gusten estava certo de que o gelo sobre a enseada se rompera e que
Carlsson a essa altura estava liquidado, emitindo o parecer de que Carlsson
``pagara por seus atos''.

--- Escute, Gusten --- objetou o pastor ---, eu acho que você está sendo injusto com
Carlsson, e não entendo o que você quer dizer com ``seus atos''. Como estava a
propriedade, quando ele lá começou? Ele não fez crescer a sua fortuna? Ele não
trouxe hóspedes de veraneio e construiu uma nova casa para você? E que ele tenha
se casado com a viúva, ora, ela não o queria também? Que ele a convenceu a
fazer um testamento, o que há de mal nisso? Já se ela aceitou fazê"-lo, isso é
algo pelo qual ela deveria responder. Carlsson era um sujeito de iniciativa, ele
fez tudo aquilo que você gostaria de ter feito, mas não conseguiu! Hein? Por
acaso você não quer que eu lhe recomende junto à viúva de Åvassan, com seus oito
mil de renda? Vê, Gusten, você não pode ser severo; há outros pontos de vista
sobre as pessoas além do seu!

--- Sim, sim, pode ser, mas ele tirou a vida da minha mãe, e isso eu nunca esquecerei.

--- Ah, bobagem, isso você terá esquecido tão logo entrar na cama de sua própria
mulher; e que Carlsson a tenha matado, isso não está nada provado. Se a velha
tivesse se agasalhado, antes de sair à noite, ela não apanharia o resfriado. E
ela não devia ter se espantado tanto por ter ele, um jovem rapaz, ficado de
gracejos com a moça. De qualquer modo, a coisa parece estar resolvida e nós
teremos que esperar até amanhã cedo para ver o que pode ser feito. É domingo e o
povo virá à igreja mesmo sem ser chamado. Agora vá se deitar e fique calmo, e
pense nisso que ``o pão de um é a morte do outro''.

\asterisc

Na manhã seguinte, quando a congregação já se reunia, o pastor Nordström veio
marchando com Flod ao seu lado. Em vez de entrar, ele se dirigiu ao grupo que
parecia já saber o que havia ocorrido. O pastor cancelou a missa e exortou a
todos os homens a se juntarem com os barcos no ancoradouro da igreja o mais
rápido que pudessem para salvar os necessitados. Nas fileiras de trás, ouvia"-se
algum resmungo de que o forasteiro não era benquisto no conselho da comarca e que
não se devia ficar sem o serviço religioso.

--- Conversa fiada! --- respondeu"-lhes o pastor. --- Como se vocês estivessem ansiosos
para ouvir a minha ladainha\ldots{} eu os conheço bem. Hein? O que você me
diz, senhor Åvassan, o senhor que é tão letrado nas escrituras, que consegue perceber
quando eu chego ao fundo dos assuntos eclesiásticos?

Um riso silencioso tomou conta do grupo e as resistências caíram pela metade.

--- Além do mais, teremos domingo novamente daqui a uma semana; venham e tragam
suas velhas consigo, que eu prometo passar um sabão nelas que durará meio
trimestre. Estamos de acordo agora e podemos tirar o jumento do poço?\footnote{
``E disse"-lhes: `Qual será de vós o que, caindo"-lhe num poço, em dia de sábado, 
o jumento ou o boi, o não tire logo?'.'' Lucas 14:5, tradução de João Ferreira de Almeida.}

--- Sim! Sim! --- murmuraram todos, como se tivessem sido absolvidos da
profanação do domingo.

E assim se dissolveram para ir às suas casas, trocar de roupa e se lançar ao mar.

A nevasca tinha cessado, o vento soprara para o norte e o tempo estava frio e
claro. A maré da enseada estava alta, com a água escura e azulada em torno
do branco ofuscante das ilhotas, quando uma dezena de barcos deixou o
ancoradouro da igreja. Os homens tinham vestido seus casacos de pele, capuzes de
pele de foca, levavam machados e fateixas. Velejar não se cogitava, todos
pegaram nos remos. O pastor estava com Gusten, no barco mais à frente, remado
por quatro dos ilhéus mais robustos, e levavam consigo o contramestre Rapp, como
espia e piqueteiro do gelo.

Estavam todos sérios, mas não demasiadamente tristes; uma vida a mais ou a menos
não contava muito para o mar. As ondas estavam fortes e a água que
entrava nos barcos congelava em poucos instantes, obrigando"-os a quebrá"-la e a
jogá"-la fora. Às vezes, um sinaleiro de gelo vinha boiando, arranhava os
cascos, passava debaixo dos barcos e reaparecia do outro lado; às vezes 
levavam junco congelado, folhas, gravetos, que tinham se desprendido das margens.

O pastor levava seu binóculo e olhava para o lado de Trälskär, onde os rapazes
de Hemsö estavam presos. Às vezes, ele também jogava um olhar sem esperança para
o lado da enseada, onde provavelmente Carlsson estava afogado, às vezes
procurando por pistas nos blocos de gelo soltos, o sinal de uma pegada, de
peças de roupa ou o próprio corpo. Mas sem resultado.

Após um par de horas eles se aproximaram a remo da ilhota. Rundqvist e Norman
tinham avistado a frota de salvamento já há um bom tempo e acenderam fogos na margem. 
E quando os barcos atracaram, eles demonstraram mais
curiosidade do que emoção, pois nenhum dos dois passara real perigo de vida.

--- Não enquanto se está pisando sobre chão firme! --- era a opinião de Rundqvist.

Como o dia seria curto, começaram logo o resgate da embarcação e, pouco depois,
o arrastão pelos restos mortais da senhora Flod.

Rundqvist sabia apontar o local exato onde ela estava, pois ele vira o
fogo"-fátuo subir na água. Jogaram sonda após sonda, mas sem conseguir mais do
que algas emaranhadas com ostras e lama. Continuaram durante toda a manhã até a
hora do almoço, mas sem resultado. Os homens já estavam com ar cansado e
cabisbaixos. Alguns já estavam sobre a terra para tomarem um trago, comer pão com
manteiga ou fazer café, quando Gusten finalmente declarou que não restava mais o
que fazer, que ele era da opinião de que a correnteza tinha levado o caixão para o alto"-mar.

Como ninguém queria ficar lá até que o corpo boiasse e, a rigor, aquilo não
dizia respeito a nenhum deles, sentiu"-se um certo alívio na deliberação,
podendo"-se assim interromper os trabalhos sem se mostrar insensível à desgraça dos outros.

E para que terminassem com alguma dignidade a missão malograda, o pastor se
aproximou de Flod e lhe perguntou se ele queria que fizessem uma pequena
cerimônia para a sua mãe. O pastor tinha levado consigo o livro sagrado e todos
ali conheciam ao menos algum salmo de cor. Gusten aceitou agradecido a proposta
e comunicou"-a para a congregação.

O sol já estava no final de seu curto percurso e as ilhotas se banhavam rosadas
em seus raios derradeiros, quando os homens se agruparam na margem para tomar
parte no cerimonial fúnebre improvisado. O pastor entrou num dos barcos, seguido
por Gusten, posicionou"-se na popa e ergueu o livro de salmos; descobriu sua cabeça
e levantou um lenço na mão esquerda, sendo seguido pelos homens, que 
tiraram seus gorros.

--- Vamos pegar o 432, ``À morte seguirei'',\footnote{ Do livro 
de salmos sueco; salmo composto por Brorson e Wallin.} vocês o sabem 
de cor? --- perguntou"-lhes o pastor.

--- Sim! --- foi a resposta que veio da margem.

O canto se elevou trêmulo, primeiro de frio, depois de emoção pela cerimônia
inaudita e as comoventes notas do velho salmo, que já acompanhara a tantos para
o eterno repouso.

As últimas notas ressoaram sobre as águas, contra as ilhas, e atravessaram o ar gélido. 
Seguiu"-se uma pausa, onde se ouvia apenas o vento norte nos ramos dos
abetos, o  movimento leve da água contra as pedras, o grito das gaivotas e o raspar
dos cascos contra a terra. O pastor virou seu rosto envelhecido e
rugoso para a enseada e o sol brilhou sobre sua calva, na qual os
tufos embranquecidos de cabelos se encrespavam ao vento feito tufos de líquen
de um velho pinheiro.

--- Do pó vieste e ao pó retornarás. Jesus Cristo, nosso salvador, irá te
ressuscitar no dia do Juízo Final! Oremos! --- disse com sua voz grave que lutava
contra o vento e a água para ser ouvida.

E assim a cerimônia fúnebre foi concluída com um Pai"-nosso. Após a bênção, o
pastor levantou sua mão sobre as águas para um derradeiro adeus.

Vestiram os gorros. Gusten tomou a mão do pastor na sua e lhe agradeceu, mas
parecia ter ainda algo no coração para dizer.

--- Senhor pastor, não posso deixar de pensar\ldots{} talvez fosse bom que também
Carlsson tivesse alguma palavrinha de recomendação.

--- Tudo foi dito para os dois, meu filho! Mas é um bonito gesto pensar nele,
apesar de tudo --- disse o pastor, que parecia mais comovido do que gostaria.

O sol já se punha e só lhes restava separar"-se e tentar chegar em casa o mais
rápido possível.

Os homens queriam ainda mostrar uma última deferência ao novo senhor Flod, e
após as despedidas, todos entraram em seus barcos e o escoltaram por uma
parte do caminho, alinhando os barcos como quando lançavam as redes;
depois o saudaram com os remos e gritaram adeus.

Era uma homenagem ao luto, mas também ao jovem que ingressava agora para as
fileiras dos homens e suas responsabilidades. Junto ao leme, o novo senhor de
Hemsö ordenou a seus homens que remassem para casa, para dali conduzir seu próprio
barco pelos estreitos de vento e as verdes enseadas da vida.


