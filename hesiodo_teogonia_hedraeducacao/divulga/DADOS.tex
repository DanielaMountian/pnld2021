\titulo{teogonia} % Em minúsculas!!!
\autor{Hesíodo}  % Apenas sobrenome, se for o caso. Verificar capa.% 
\organizador{Tradução e organização}{Christian Werner}   %Conferir se é apenas {Organização}; {Organização e tradução} ou apenas {Tradução}%
\isbn{978-85-7715-319-0}
\preco{22}   % Ex.: 14. Não usar ",00"%
\pag{108}   % Número de páginas
\release{\textit{Teogonia} (em grego \textit{theogonia}: \textit{theos} $=$ deus +
\textit{genea} $=$ origem) é um poema de 1022 versos hexâmetros datílicos que
descreve a origem e a genealogia dos deuses. Muito do que sabemos sobre os
antigos mitos gregos é graças a esse poema que, pela narração em primeira
pessoa do próprio poeta, sistematiza e organiza as histórias da criação do
mundo e do nascimento dos deuses, com ênfase especial a Zeus e às suas façanhas
até chegar ao poder. A invocação das Musas, filhas da Memória, pelo aedo
Hesíodo é o que lhe dá o conhecimento das coisas passadas e presentes e a
possibilidade de cantar em celebração da imortalidade dos deuses; e é a partir
daí que são narradas as peripécias que constituem o surgimento do universo e de
seus deuses primordiais.
Em nova tradução de Christian Werner, esta edição bilíngue do clássico grego conta com introdução e notas explicativas.

\textbf{Sobre o autor:}
\textsc{Christian Werner} é professor livre-docente de língua e literatura
grega na Faculdade de Letras da Universidade de São Paulo (\textsc{usp}).
Publicou, entre outros, \textit{Duas tragédias gregas: Hécuba e Troianas}
(Martins Fontes, 2004).}
