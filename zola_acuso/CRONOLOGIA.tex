\chapter{Cronologia}
\hedramarkboth{cronologia}{cronologia}

\section{1894}
\begin{itemize}
\setlength\itemsep{-1mm}
\item[20-25/set] O Serviço de Informações do Exército (a ``seção de
estatística''), dirigido pelo coronel Sandherr, intercepta uma carta --- o
``documento'' (\textit{bordereau}), como ficará conhecida ---, dirigida ao
Schwartzkoppen, adido militar alemão a serviço em Paris. A pedido do general
Mercier, ministro da Guerra, é aberta uma investigação nos gabinetes do
Estado-Maior. 

\item[6/out] As suspeitas se voltam para o capitão Alfred Dreyfus, oficial estagiário do Estado-Maior.

\item[9/out] O documento é submetido ao perito Gobert para análise. 

\item[13/out] Gobert apresenta um relatório dúbio. Uma nova perícia é
solicitada a Bertillon. Este, após a perícia, conclui a culpabilidade de
Dreyfus.

\item[15/out] Convocado pelo Ministério da Guerra, Alfred Dreyfus é preso, após
um rápido interrogatório, pelo comandante Du Paty de Clam, encarregado do
inquérito. Dreyfus é encarcerado na prisão do Cherche-Midi.

\item[18/out] Início dos interrogatórios de Dreyfus por Du Paty.

\item[23/out] Relatório oficial de Bertillon, encaminhado ao chefe de polícia
Lépine.

\item[29/out] Os peritos Charavay e Teysonnières ratificam a análise de
Bertillon.

\item[31/out] Du Paty encaminha ao Ministério de Guerra o relatório do
inquérito, do qual ele é o encarregado: não emite nenhuma conclusão precisa.

\item[1/nov] O jornal antissemita \textit{La Libre Parole} menciona pela
primeira vez o nome de Alfred Dreyfus.

\item[3/nov] O general Saussier ordena a abertura da instrução judicial,
confiada ao comandante d'Ormescheville.

\item[19/dez] Início do processo de Alfred Dreyfus perante o Conselho de
Guerra, na prisão do Cherche-Midi. O acusado é defendido por madame Démange. O
julgamento ocorre em audiência privada. 

\item[20/dez] Na segunda audiência do processo, o comandante Henry,
representando o Serviço de Informações, declara sua convicção da culpabilidade
de Dreyfus.

\item[22/dez] Por unanimidade, os juízes do Conselho de Guerra condenam Dreyfus
à deportação e à prisão perpétua em uma fortaleza. De maneira ilegal, tomaram
conhecimento, antes de suas deliberações, de um ``dossiê secreto'' enviado pelo
general Mercier e cujo conteúdo incrimina o acusado.
\end{itemize}

\section{1895}
\begin{itemize}
\setlength\itemsep{-1mm}
\item[5/jan] Degradação pública do capitão Alfred Dreyfus no pátio da Escola
Militar. Os jornais vespertinos mencionam uma suposta confissão do condenado ao
capitão Lebrun-Renaud.

\item[17/jan] Félix Faure é eleito presidente da república. À noite, partida de
Dreyfus para o local de detenção.

\item[7/fev] Mathieu Dreyfus, irmão do condenado, encontra Scheurer-Kestner,
vice-presidente do Senado (e como ele, de origem alsaciana). Pede sua ajuda,
sem êxito. 

\item[12/mar] O navio que transporta Alfred Dreyfus chega a Guiana.

\item[13/abril] O condenado é transferido para a Ilha do Diabo.

\item[1/jul] O comandante Picquart é nomeado chefe do Serviço de Informações,
substituindo o coronel Sandherr, doente.
\end{itemize}


\section{1896}
\begin{itemize} 
\setlength\itemsep{-1mm}
\item[1-2/mar] O Serviço de Informações entra em posse de um telegrama
(conhecido depois como ``\textit{petit-bleu}'') escrito por Schwartzkoppen e
endereçado ao comandante Esterhazy. Picquart decide abrir inquérito sobre
Esterhazy. Ele é imediatamente persuadido de que este último é o verdadeiro
autor do ``documento''.

\item[5/out] Picquart comunica ao general de Boisdeffre suas suspeitas sobre
Esterhazy. O chefe do Estado-Maior recomenda-lhe prudência.

\item[3/set] Picquart presta contas ao general Gonse (vice-chefe do
Estado-Maior) do resultado do inquérito sobre Esterhazy.

\item[14/set] Um artigo publicado no \textit{L'Éclair}, com a pretensão de
apresentar uma ``prova irrefutável'' da culpabilidade de Dreyfus, faz alusão
pela primeira vez à apresentação de um documento secreto aos juízes do Conselho
de Guerra.

\item[16/set] Lucie Dreyfus dirige-se à Câmara para solicitar a revisão do
processo de seu marido.

\item[27/out] Os superiores de Picquart decidem afastá-lo de Paris, e assinam
uma ordem de missão enviando-o para inspecionar a fronteira Leste
(Alsácia-Lorena).  

\item[31/out] Henry redige, ou manda redigir, um documento que incrimina
diretamente Dreyfus (uma correspondência que teria sido trocada entre
Panizzardi e Schwartzkoppen, os adidos militares italiano e alemão): é o
documento que será conhecido mais tarde como ``\textit{faux Henry}'', (``falso
Henry'').  

\item[6/nov] Bernard-Lazare publica em Bruxelas seu primeiro livreto sobre o
caso, intitulado: \textit{Um erro judicial. A verdade sobre o caso Dreyfus}.
Nas semanas seguintes, ele se esforça para convencer várias personalidades do
mundo das letras ou da política, dentre os quais Zola. Suas diligências não dão
resultado. 

\item[10/nov] \textit{Le Matin} publica um fac-símile do documento. 

\item[16/nov] Picquart (que teve de abandonar a chefia do Serviço de
Informações) deixa Paris para realizar sua missão no Leste (Alsácia-Lorena).
Publicação em Paris (pela Stock) da segunda edição do livreto de
Bernard-Lazare. 

\item[26/dez] Gonse escreve a Picquart informando que sua missão será estendida
e que ele deve partir para a Tunísia.
\end{itemize}


\section{1897}
\begin{itemize}
\setlength\itemsep{-1mm}
\item[6/jan] Gonse escreve a Picquart destinando-o ``provisoriamente''  ao 4º
Regimento de artilheiros argelinos (situado em Sousse, na Tunísia.)

\item[19/jan] Chegada de Picquart a Tunísia.

\item[29/jun] De licença, em Paris, Picquart confia ao advogado Leblois, em
sigilo, tudo aquilo que descobriu; temendo que lhe preparem alguma maquinação,
confere a seu advogado um ``mandato geral de defesa''.   

\item[13/jul] Leblois conta o que sabe a Scheurer-Kestner. Este, ainda que
comprometido com a confidencialidade, decide lançar uma campanha para a
reabilitação de Dreyfus. 

\item[17/ago] Esterhazy recebe licença por ``enfermidade temporária''.

\item[15/out] O general Billot, ministro da Guerra, ordena o prolongamento da
missão de Picquart até a fronteira tripolitana, onde alguns distúrbios foram
apontados.   

\item[16/out] Em uma reunião no Ministério da Guerra, da qual participam Gonse,
Henry e Du Paty, decide-se informar Esterhazy das acusações que ele terá de
responder.

\item[17/out] Esterhazy recebe uma carta anônima prevenindo-o disso (assinada
``Espérance''). 

\item[19/out] Publicação do primeiro número do \textit{L'Aurore}, diário
fundado por Ernest Vaughan. Georges Clemenceau, Urbain Gohier, Bernard-Lazare
fazem parte do corpo editorial. 

\item[23/out] Encontro secreto no parque de Montsouris entre Esterhazy e os
emissários do Ministério (Du Paty de Clam, Henry e Gribelin).

\item[27/out] Segundo encontro secreto, perto da ponte Alexander \textsc{iii}:
encontro de Esterhazy com a ``mulher com o rosto velado''. 

\item[29/out] Scheurer-Kestner é recebido por Félix Faure no Eliseu.

\item[30/out] Scheurer-Kestner tem uma longa conversa com Billot (amigo seu de
longa data): este lhe promete abrir inquérito, mas pede um prazo de 15 dias. 

\item[3/nov] Scheurer-Kestner vai a casa de Méline, presidente do Conselho.

\item[5/nov] Gabriel Monod afirma no \textit{Le Temps} (datado do dia 6) que
Dreyfus é vítima de um erro judicial.

\item[6/nov] Um ano após sua primeira tentativa, Bernard-Lazare faz outra
visita a Zola. Está prestes a publicar seu segundo memorando \textit{Um erro
judicial. O caso Dreyfus} (Stock).

\item[8-10/nov] Sucessivas visitas de Louis Leblois, acabam expondo os
documentos do caso Dreyfus.

\item[10/nov] Dois telegramas misteriosos (assinados ``Spéranza'' e
``Blanche'') são enviados a Picquart, trazendo à baila a sua responsabilidade
no caso.
 
\item[11/nov] Mathieu Dreyfus, que soube, graças ao banqueiro Castro, a
verdadeira identidade do autor do documento, procura Scheurer-Kestner, que lhe
confirma a informação.

\item[12/nov] Reunião na casa de Scheurer-Kestner. Estiveram presentes Mathieu
Dreyfus, Leblois e Emmanuel Arène.

\item[13/nov] Nova reunião na casa de Scheurer-Kestner. Zola, convidado do
vice-presidente do Senado, comparece em companhia de Leblois, do escritor
Marcel Prévost e do advogado Sarrut. 

\item[15/nov] Em carta aberta a um ``amigo'' (Arthur Ranc), publicado no
\textit{Le Temps} (datado do dia 16), Scheurer-Kestner afirma a inocência de
Dreyfus. Convocado pelo governo, Schwartzkoppen (nomeado comandante do 2º
Regimento de granadeiros da Guarda) deixa Paris.

\item[16/nov] Os jornais matutinos publicam uma carta, escrita na véspera, de
Mathieu Dreyfus ao ministro da Guerra, denunciando Esterhazy como o autor do
documento.

\item[17/nov] Logo após a denúncia de Mathieu Dreyfus, é aberto um inquérito
sobre as atividades de Esterhazy. Confiado ao general de Pellieux, o inquérito,
a princípio, (17-20 de novembro) administrativo, logo ganha dimensão judiciária
(21 de novembro a 3 de dezembro). 

\item[25/nov] Aparece no \textit{Figaro} o primeiro artigo de Zola em favor da
causa de Dreyfus (``M.~Scheurer-Kestner'').

\item[26/nov] Picquart, convocado da Tunísia, chega a Paris e é interrogado no
inquérito conduzido por Pellieux.

\item[28/nov] O \textit{Figaro}  publica trechos da carta de Esterhazy a sua
amante, madame de Boulancy (donde a famosa ``carta de Ulhan'').

\item[1/dez] Zola continua sua campanha no \textit{Figaro} (``O sindicato'').

\item[3/dez] Pellieux envia ao general Saussier as conclusões de seu inquérito.

\item[4/dez] Um novo inquérito sobre Esterhazy é confiado ao comandante Ravary.
Interpelação a Câmara sobre o caso. Méline se defende dizendo ``que não existe
[o caso]'', ``que não pode haver caso Dreyfus''.

\item[5/dez] Zola publica no \textit{Figaro } ''\textit{Procès-verbal}''
(``Averbação'').

\item[7/dez] No Senado, interpelação de Scheurer-Kestner contra o governo.  

\item[14/dez] Zola publica \textit{Lettre à la Jeunesse} (``Carta à
juventude''). 

\item[26/dez] Após avaliarem o documento, os três peritos, Belhomme, Varinard e
Couard, enviam suas conclusões a Ravary: eles declaram que o documento não é
obra de Esterhazy. 
\end{itemize}

\pagebreak
\section{1898}
\begin{itemize}
\setlength\itemsep{-1mm}
\item[1/jan] Ravary envia ao general Saussier um relatório do inquérito
concluindo impronúncia. Este último, apesar disso, decide pelo julgamento de
Esterhazy.  

\item[4/nov] Queixa de Picquart contra os autores dos telegramas de 10 de
novembro. O juiz Bertulus é encarregado da instrução.

\item[7/jan] Zola, \textit{Lettre à la France}. \textit{Le Siècle} publica um
ato de acusação contra Alfred Dreyfus (relatório d'Ormescheville). No mesmo
momento, Bernard-Lazare publica seu terceiro memorando, \textit{Comme on
condamne un innocent} (``Como se condena um inocente''). 

\item[10/jan] Início do processo de Esterhazy no Cherche-Midi. A audiência
privada é instituída depois de algumas horas.

\item[11/jan] O Conselho de Guerra absolve Esterhazy por unanimidade.  

\item[13/jan] Publicação de \textit{Eu acuso! } no \textit{L'Aurore}. Na
Câmara, Albert de Mun interpela o governo. No Senado, Scheurer-Kestner não é
reconduzido a suas funções de vice-presidente. No início da manhã, Picquart é
conduzido a Mont-Valérien, onde é detido em regime de isolamento.   

\item[14/jan] Início da publicação no \textit{L'Aurore}, de um primeiro
``protesto'' assinado por um grupo de intelectuais, solicitando a revisão do
processo de Dreyfus. Segue-se um segundo protesto no dia 16 de janeiro.

\item[18/jan]  Queixa de Billot contra Zola e o \textit{L'Aurore}.

\item[19/jan] É lançado o Manifesto do Proletariado pelos deputados
socialistas, hostil a todo engajamento na batalha do caso Dreyfus. Jaurès, que
o assina, muda de posição rapidamente. 

\item[20/jan] Zola e Perrenx (gerente do \textit{L'Aurore}) recebem uma
intimação.

\item[21/jan] Considerando-se difamados por \textit{Eu acuso!}, Belhomme,
Varinard e Couard intimam Zola perante o tribunal correcional.

\item[7/fev] Abertura do processo de Zola no tribunal do júri do Sena. 

\item[11-12/fev] Depoimento de Picquart (5ª e 6ª audiências). 

\item[17/fev] De Pellieux faz alusão ao ``faux Henry'' (10ª audiência).

\item[18/fev] Intervenção cominatória do general de Boisdeffre (11ª audiência). 

\item[21/fev] Requisitório do advogado geral Van Cassel. ``Declaração ao júri''
de Zola. Início da defesa de Labori.

\item[23/fev] Zola é condenado por difamação à pena máxima (um ano de prisão e
multa de três mil francos). 

\item[24-25/fev] Após duas reuniões na casa de Trarieux e na de
Scheurer-Kestner, toma-se a decisão de fundar a Liga para a defesa dos direitos
do homem e do cidadão. 

\item[26/fev] Publicação de um decreto para processo de reforma por ``falta
grave em serviço''. Zola interpõe seu recurso de cassação. 

\item[1/mar] O romance de Zola, \textit{Paris}, chega às livrarias (ed.
Fasquelle).

\item[5/mar] Duelo entre Picquart e Henry.

\item[9/mar] Zola é processado pelos três peritos, Belhomme, Varinard e Couard,
perante a 9ª câmara correcional.  

\item[15/mar] \textit{Le Siècle} anuncia a formação de um comitê para a
concessão de uma medalha de honra a Zola (medalha Charpentier).  

\item[2/abr] A Câmara criminal da corte de cassação anula por vício de forma o
julgamento do dia 23 de abril.
 
\item[8/abr] Os juízes do Conselho de Gerra que absolveram Esterhazy se
apresentam como parte civil contra Zola.

\item[8-22/maio] Eleições legislativas: Jaurès e Reinach perdem a cadeira de
deputado; Drumont é eleito em Argel. 

\item[23/maio] Processo de Zola a Versailles perante o júri de Seine-et-Oise.
Labori apela à corte de cassação, que suspende os trabalhos. Artigo calunioso
de Judet, no \textit{Petit Journal}, contra o pai do romancista. 

\item[24/maio] Zola registra queixa de difamação contra Judet.  

\item[26/maio] Eleição para Academia Francesa. Zola não obtém nenhum voto
(mesmo resultado de 8 de dezembro, em uma eleição ulterior).  

\item[4/jun] Assembleia geral constitutiva da Liga de defesa dos direitos do
homem e do cidadão. 

\item[15/jun] Méline é demitido do ministério. 

\item[16/jun] A Corte de cassação rejeita o recurso de Labori do dia 23 de
maio.  \item[28/jun]  Constituição do gabinete Brisson. O ministro da Guerra é
Godefrois Cavaignac. 

\item[7/jul] Cavaignac afirma, em um discurso na Câmara, possuir provas
irrefutáveis da culpabilidade de Dreyfus; destaca, em particular, três
documentos extraídos do ``dossiê secreto''.

\item[9/jul] Retomada do processo de difamação intentado pelos três peritos.
Zola é condenado a quinze dias de prisão com \textit{\textit{sursis}} e multado
em dois mil francos, mais cinco mil francos a título de perdas e danos para
cada um dos peritos. Em uma carta endereçada ao presidente do conselho,
Picquart questiona a validade das provas adiantadas por Cavaignac.   

\item[12/jul] Queixa de Cavaignac contra Picquart e Leblois. Prisão e
encarceramento e Esterhazy, por iniciativa do juiz Bertulus.

\item[13/jul] Prisão e encarceramento de Picquart (logo depois da queixa
registrada contra ele por Cavaignac).

\item[16/jul] Zola publica no \textit{L'Aurore} uma carta aberta endereçada a
Henri Brisson.

\item[18/jul] Retomada do processo de Zola no tribunal de Versailles. Condenado
de novo, ao fim da audiência, o romancista parte na mesma noite para a
Inglaterra para evitar que a prisão seja declarada e executada.

\item[19/jul] Labori faz uma apelação no processo dos peritos. Zola se instala
em Londres no Grosvenor Hotel. 

\item[21/jul] Início de uma série de artigos de Reinach, no \textit{Le Siècle},
sobre ``os falsários'' (dirigidos em particular contra Du Paty e Esterhazy).

\item[22/jul] Instalação de Zola no Oatlands Park Hotel (entre Weybridge e
Walton-on-Thame).

\item[26/jul] A ordem da Legião de Honra de Zola é revogada. 

\item[1/ago] Zola se instala na residência de Penn, perto de Weybridge.

\item[3/ago] Labori consegue a condenação de Ernest Judet e do \textit{Petit
Journal} por difamação na 9ª Câmara de correção.

\item[5/ago] A Corte de cassação rejeita o recurso de Zola e do
\textit{L'Aurore } contra a decisão no julgamento de Versailles em 18 de julho. 

\item[8/ago] Início do interrogatórios de Picquart pelo juiz de instrução
Fabre.

\item[10/ago] A Câmara de apelação aumenta as penas no processo movido pelos
peritos; Zola é condenado a um mês de prisão sem \textit{sursis} e multado em
dois mil francos, mais 10 mil francos a título de perdas e danos para cada um
dos peritos. Início de uma série de artigos de Jaurès no \textit{La Petite
République} (entitulados de ``As provas'').   

\item[11/ago] Chegada de Jeanne Rozerot e das crianças à Penn. Permanecem na
Inglaterra até 15 de outubro.

\item[12/ago] Deferindo requerimento do procurador Feuilloley, a Câmara de
acusação declara impronúncia em favor de Esterhazy, que é posto em liberdade. 

\item[13/ago] Examinando o dossiê secreto, o capitão Cuignet, adido no gabinete
de Cavaignac, descobre o ``faux Henry''.

\item[25/ago] O juiz Albert Fabre (encarregado da instrução aberta pela queixa
de Cavaignac) transfere Picquart e Leblois para o tribunal correcional.  
 
\item[27/ago] Instalação de Zola a Summerfield (Addlestone). O conselho de
inquérito encarregado de instituir as acusações feitas contra Esterhazy emite
parecer favorável à sua reforma. 

\item[30/ago] O coronel Henry confessa sua fraude à Cavaignac; ele é preso e
conduzido a Mont-Valérien. O general de Boisdeffre pede desligamento de suas
funções.

\item[31/ago] Suicídio de Henry na sua cela no Mont-Valérien.

\item[3/set ] Cavaignac se demite do governo.

\item[4/set] Esterhazy deixa a França.

\item[5/set] O general Zurlinden se torna ministro da Guerra.

\item[17/set] Demissão de Zurlinden. O general Chanoine o sucede.

\item[20/set] Zurlinden (reintegrado em suas antigas funções de governante
militar de Paris) ordena a instrução contra Picquart.

\item[21/set] Picquart e Leblois perante 8ª Câmara correcional: o processo é
encaminhado.

\item[22/set] Picquart é encarcerado na prisão de Cherche-Midi.

\item[23/set] A pedido dos três peritos, apreensão dos bens pelo oficial de
justiça no domicílio de Zola, em Paris. Uma nova apreensão será efetuada dia 29
de setembro.

\item[10/out] Instalação de Zola no Bailey's Hotel (Kensington, Londres).

\item[11/out] Leilão dos bens, rue de Bruxelles. Uma mesa é comprada por 32 mil
francos por Eugène Fasquelle, o que encerra o procedimento.

\item[15/out] Zola se instala no Queen's Hotel (Upper Norwood, no subúrbio de
Londres).

\item[25/out] Cai o gabinete de Brisson. Chegada de Alexandrine em Norwood: ela
permanecerá até 5 de dezembro.

\item[27/out] A câmara criminal da Corte de cassação começa o exame da demanda
de revisão.

\item[29/out] Ela defere a demanda e abre inquérito.

\item[31/out] Formação do gabinete de  Dupuy.

\item[7/nov] Em artigo escrito no \textit{Le Siècle}, Reinach acusa diretamente
Henry de cumplicidade com Esterhazy ; ele voltará a essa acusação no número de
6 dezembro, colocando-se à disposição da justiça e da viúva de Henry.

\item[8/nov] Início do inquérito na câmara criminal.

\item[19/nov] Aparece o primeiro fascículo de \textit{Dessous de l'Affaire
Dreyfus}, (``Os subterrâneos do caso Dreyfus'') de Esterhazy.

\item[24/nov] Picquart é enviado por Zurlinden perante o 2º Conselho de guerra
de Paris.

\item[25/nov] No \textit{Le Siècle} e no \textit{L'Aurore}, publicação da
primeira lista de assinaturas em homenagem a Picquart.

\item[14/dez] O \textit{Libre Parole} abre uma subscrição em favor de viúva do
coronel Henry, Berthe Henry, para que ela possa processar Reinach. A subscrição
é encerrada em 15 de janeiro de  1899.

\item[22/dez] Início da segunda estada de Alexandrine em Norwood.

\item[31/dez] Apelo em favor da fundação da Liga da Pátria francesa.
\end{itemize}

\section{1899}
\begin{itemize}
\setlength\itemsep{-1mm}
\item[3/jan] Georges Clemenceau visita Zola em Norwood.

\item[6/jan] O presidente da câmara civil da Corte de cassação, Jules Quesnay
de Beaurepaire, que acusou a câmara criminal de parcialidade com Picquart,
solicita um inquérito.

\item[8/jan] Não obtendo nenhuma resposta do primeiro presidente da Corte,
Charles Mazeau pede demissão.

\item[12/jan] O ministro da Justiça, Georges Lebret, encarrega Mazeau de abrir
inquérito sobre os fatos apontados por Quesnay de Beaurepaire.

\item[19/jan] Reunião constitutiva da Liga da pátria francesa.

\item[21/jan] Novo leilão no domicílio de Zola, em Paris. Eugène Fasquelle
compra um espelho e uma mesa por 2500 francos.

\item[27/jan] Processo entre Reinach e Berthe Henry, perante o tribunal do
júri; Labori, que defende Reinach, recorre da cassação.

\item[30/jan] Lebret submete à Câmara dos deputados um projeto de lei
atribuindo às três câmaras reunidas da Corte de cassação o poder de decisão
sobre todos os casos de revisão.

\item[9/fev]  A Câmara Criminal encerra o inquérito sobre a revisão do
processo.

\item[10/fev] A Câmara dos Deputados vota projeto de lei apresentado por
Lebret.

\item[16/fev] Morte de Félix Faure, adversário ferrenho da revisão.

\item[18/fev] Emile Loubet é eleito Presidente da República.

\item[23/fev] Funerais de Félix Faure. Tentativa de sublevação nacionalista,
suscitada por Déroulède, fracassa.

\item[27/fev] Alexandrine deixa Norwood e volta a Paris. 

\item[10/mar] Jaurès visita Zola, em Norwood.

\item[21/mar] Primeira sessão plenária da corte de cassação, todas as câmaras
reunidas sob a presidência de Charles Mazeau. Este designa como relator Alexis
Ballot-Beaupré, sucessor de Quesnay de Beaurepaire na presidência da câmara
civil.

\item[27-28/mar] Visita de Labori a Norwood.

\item[29/mar] Chegada a Inglaterra de Jeanne e as crianças, para as férias da
Páscoa: durante esse período, Zola se hospeda com eles no Crystal Palace Royal
Hotel.
 
\item[31/mar] O \textit{Figaro} começa a publicação da averbação do inquérito
realizado pela Câmara Criminal.

\item[8/abr] Em Londres, encontro de Zola e o senador Ludovic Trarieux.

\item[24/abr] Início das audiências na Corte de cassação: comparecimento do
capitão Freystaetter (um dos sete juízes militares do processo de dezembro de
1894), que, depois do suicídio de Henry convenceu-se da inocência de Dreyfus.

\item[29/abr] Última audiência na Corte de cassação para ouvir Cuignet e Du
Paty de Clam.

\item[15/mai] Início do folhetim \textit{Fécondité}, no \textit{L'Aurore.}

\item[16/mai] Nova visita de Clemenceau a Zola.

\item[29/mai] A corte de cassação se reúne para escutar o relatório de
Ballot-Beaupré.

\item[1/jun] Du Paty de Clam é preso.

\item[3/jun] A corte de cassação anula o julgamento de 1894. Dreyfus é levado
de novo ao Conselho de Guerra.

\item[5/jun] Dreyfus é informado da sentença de revisão. Zola chega a França;
levam a ele, na rue de Bruxelles, a notificação da sentença pronunciada em
Versailles, no dia 18 de julho. ``Justiça'' é publicado no \textit{ L'Aurore.}

\item[9/jun] Zola se opõe à sentença de Versailles. Picquart sai da prisão.
Dreyfus deixa a Ilha do Diabo, a bordo do cruzador Sfax.

\item[12/jun] Queda do ministério de Dupuy.

\item[22/jun] Formação do ministério de Waldeck-Rousseau, dito ``da defesa
republicana''. A pasta da Guerra é confiada ao general Galliffet, incumbido de
fazer o exército aceitar a revisão do processo.

\item[1/jul] Dreyfus chega a França, e é isolado na prisão militar de Rennes.

\item[18/jul] Le Matin publica um longo relato de Esterhazy, no qual ele
explica seu papel no caso: ele reconhece ser o autor do documento, mais diz
tê-lo escrito ``ditado'', obedecendo a ordens superiores.
 
\item[7/ago] Abertura do processo de Alfred Dreyfus perante o Conselho de
Guerra de Rennes. O acusado é defendido por Edgar Démange e Fernand Labori.

\item[12/ago] Testemunho do general Mercier. Em Paris, a polícia prende
numerosas personalidades monarquistas e nacionalistas, entre os quais Paul
Déroulède. Jules Guérin, diretor do jornal \textit{L'Antijuif}, se entrincheira
na rue de Chabrol, no sítio do Grand-Occident de France (episódio do ``fort
Chabrol'').

\item[14/ago] Em Rennes, atentado contra a vida de Labori.

\item[31/ago] Zola toma conhecimento de que foi citado em Versailles, dia 23 de
novembro.

\item[8/set] Defesa de Edgar Démange (Labori renuncia a seu direito de falar).

\item[9/set] Conclusão do processo de Rennes (25ª audiência). Dreyfus é
novamente declarado culpado, mais com ``circunstâncias atenuantes''.

\item[12/set] Zola publica ``Le cinquième acte'' (``O quinquagésimo ato''), no
\textit{L'Aurore.}

\item[19/set] Dreyfus é perdoado pelo presidente da república, Emile Loubet.
Morte de Scheurer-Kestner.

\item[21/set] O general de Galliffet envia ao exército uma nota na qual
declara: ``O incidente está encerrado''.

\item[22/set] Zola publica sua ``Carta a Madame Alfred Dreyfus'', no
\textit{L'Aurore}.

\item[4/out] Conclusão do folhetim \textit{Fécondité} no \textit{L'Aurore}: o
romance chega às livrarias em 12 de outubro (ed. Fasquelle).

\item[23/nov] O processo de Versailles é adiado por tempo indeterminado.
\end{itemize}


\section{1900}
\begin{itemize}
\setlength\itemsep{-1mm}
\item[1/mar] O Senado aprecia projeto de lei de anistia para todos os
envolvidos no caso Dreyfus.

\item[14/mar] Zola é ouvido pela Comissão do Senado, no debate sobre a lei de
anistia.

\item[29/mar] Zola publica sua ``Carta ao Senado'', no \textit{L'Aurore.}

\item[18/dez] Após vários meses de discussões, a Câmara dos Deputados vota a
lei de anistia.

\item[22/dez] Zola publica uma carta aberta ao Presidente da República, Emile
Loubet, no \textit{L'Aurore}.

\item[24/dez] O Senado vota a lei de anistia.
\end{itemize}

\section{1901}
\begin{itemize}
\setlength\itemsep{-1mm}
\item[1/fev/1901] Zola publica \textit{La Vérité en marche}, reunião de seus
artigos escritos durante o caso Dreyfus.
\end{itemize}

\section{1902}
\begin{itemize}
\setlength\itemsep{-1mm}
\item[29/set] Morte de Zola, em seu domicílo em Paris, na rue de Bruxelles.

\item[5/out] Funerais, no cemitério de Montmartre. Discurso de Anatole France :
``Invejemo-lo, sua vocação e seu coração lhe deram o destino mais grandioso:
ele foi um momento da consciência humana''.
\end{itemize}

\section{1903}
\begin{itemize}
\setlength\itemsep{-1mm}
\item[26/nov] Alfred Dreyfus escreve ao ministro da Justiça para solicitar a
revisão do processo de Rennes.
\end{itemize}

\pagebreak
\section{1904}
\begin{itemize}
\setlength\itemsep{-1mm}
\item[5/mar] A Câmara Criminal da corte de cassação defere a sua demanda, e
ordena um inquérito.

\item[19/nov] Fim do inquérito na Câmara Criminal.
\end{itemize}

\section{1906}
\begin{itemize}
\setlength\itemsep{-1mm}
\item[12/jul] A corte de cassação e todas as cortes reunidas anulam o
julgamento do Conselho de Guerra de Rennes, e declaram que a condenação de
Dreyfus foi pronunciada ``injustamente''.

\item[13/jul] A Câmara vota a lei de reintegração de Dreyfus ao exército (com a
patente de chefe de batalhão) bem como a de Picquart (com a patente de general
de brigada). No mesmo dia, aprova um outro projeto de lei solicitando a
transferência das cinzas de Zola ao Panthéon. 

\item[21/jul] Alfred Dreyfus é sagrado cavaleiro da Legião de Honra.

\item[25/out] Clemenceau se torna presidente do Conselho. Ele nomeia Picquart
para o Ministério da Guerra.
\end{itemize}

\section{1908}
\begin{itemize}
\setlength\itemsep{-1mm}
\item[4/jun] Cerimônia oficial marcando a transferência das cinzas de Zola no
Panthéon. Um jornalista chamado Louis Grégori faz dois disparos contra Alfred
Dreyfus, ferindo-o no braço.
\end{itemize}
