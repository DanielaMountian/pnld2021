\SVN $Id: PRETAS.tex 8310 2011-02-11 19:15:44Z oliveira $
\begin{resumopage}

\item[Émile Zola] (Paris, 1840---id.~1902), romancista, ativista político e
crítico francês, figura ao lado de Balzac como um dos mais importantes e
profícuos escritores do século \textsc{xix}. Sua obra, fortemente marcada pelo
realismo social, muitas vezes de caráter documental, além do acabamento
estético, possui o mérito de ter refletido as mudanças que as revoluções
industrial e econômica introduziram na França. Militante e um dos mentores da
escola naturalista, Zola conheceu de perto a pobreza. Foi colega de classe  de
Paul Cézanne, em Aix-en-Provence, onde passou a infância. Em 1859, abandona
os estudos após tentar duas vezes o exame para bacharel e vive dois anos de
grande penúria. Em 1870, inicia a redação de uma extensa série de romances, o
ciclo dos Rougon-Macquart (1871--1893). Deles destacaram-se \textit{Naná
}(1879), retrato franco sobre a prostituição, e \textit{Germinal} (1885),
relato chocante sobre as condições desumanas impostas aos mineiros.  Falece em
circunstâncias misteriosas, vítima de envenenamento por monóxido de carbono em
decorrência de uma obstrução na chaminé de sua lareira.

\item[Rui Barbosa] (Salvador, 1849---Petrópolis, 1923), escritor, jornalista,
jurista, diplomata, tradutor e político brasileiro, foi um dos mais destacados
e influentes estadistas que o Brasil já teve. Um dos fundadores da Academia
Brasileira de Letras, liderou em 1907 a delegação brasileira na Conferência de
Paz, em Haia, Holanda. Lá, sua participação brilhante lhe valeu renome
internacional e a alcunha de “Águia de Haia”. Em 1893, por combater o golpe que
levou Floriano Peixoto à presidência, teve de se exilar, primeiro em  Buenos
Aires, depois em Lisboa e, por fim, em Londres, onde permanece até 1895, e de
onde contribuía para a imprensa brasileira com uma série de artigos mais tarde
publicados sob o título de \textit{Cartas de Inglaterra}(1896). Concorreu duas
vezes à presidência do Brasil, e em 1921 foi eleito juiz da Corte Internacional
de Justiça. Sua obra, rica e extensa, abrange vários campos do saber, e incluem
os \textit{Comentários à Constituição Federal Brasileira}, \textit{O Elogio de
Castro Alves} (1881), \textit{Visita à Terra Natal} (1893), \textit{Discursos e
Conferências} (1907) e \textit{Oração aos Moços} (1920).

\item[Eu acuso!] foi publicado em 13 de janeiro de 1898 no jornal
\textit{L’Aurore}, como uma carta aberta dirigida ao então presidente Félix
Faure, causando grande e imediata repercussão. Trata da denúncia contra
oficiais do exército francês, que ocultaram a verdade no tumultuado caso em que
Alfred Dreyfus, oficial francês de origem judaica, foi acusado injustamente de
traição e espionagem. A reação não tarda e Zola é processado por difamação e
condenado. Quando não havia mais esperanças de que seu recurso fosse aceito, é
obrigado a fugir para a Inglaterra, em julho de 1899, de onde só retornaria um
ano depois, quando o caso foi reaberto.

\item[O processo do capitão Dreyfus,] de Rui Barbosa, data de janeiro de 1895,
portanto, três anos antes da carta aberta de Zola. O próprio Dreyfus, em suas
memórias, reconhece ter sido de Rui a primeira voz a se levantar publicamente
em sua defesa. A partir da Inglaterra, onde se encontrava exilado, Rui analisa,
com a eloquência que lhe é característica, os desdobramentos do caso, pontuando
as diferenças de tratamento que os jornais franceses e ingleses deram ao
\textit{affaire}.

\item[Ricardo Lísias] é escritor e doutor em Literatura Brasileira pela
Universidade de São Paulo. Publicou \textit{Capuz} (Hedra, 2001), \textit{Dos
nervos} (Hedra, 2004),  \textit{Anna O.~e outras novelas} (Globo, 2007), entre
outros títulos.

\end{resumopage}

