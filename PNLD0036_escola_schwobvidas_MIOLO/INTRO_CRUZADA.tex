%\chapter[Introdução, por Marcelo Jacques de Moraes ]{Introdução}
%\hedramarkboth{introdução}{marcelo jacques de moraes}

\chapterspecial{Introdução}{}{Marcelo Jacques de Moraes}

\noindent\textsc{Em meio} ao desespero ocasionado por uma súbita dispersão de sua volúpia amorosa,
o poeta latino Lucrécio, vagando, como por acaso, entre livros, se vê
subitamente diante do tratado de Epicuro. “De imediato”, prossegue o narrador
da pequena biografia do autor de \textit{De Rerum Natura} imaginada por Marcel
Schwob em suas \textit{Vidas imaginárias}, ele “compreendeu a variedade das
coisas deste mundo, e o quanto é vão se esforçar rumo às ideias”. 

Essa passagem pode ajudar-nos a compreender o grande desafio da literatura para
este ficcionista, ensaísta e tradutor nascido em 1867 em Chaville, região de
Champagne, e que, embora tenha ocupado, ao final do século~\versal{XIX}, um lugar de
referência nos círculos literários parisienses, é hoje um escritor praticamente
desconhecido --- não apenas entre nós, no Brasil, mas também na França, seu país
natal, onde há pouco volta a despertar interesse. Reconhecido por seus
contemporâneos pela enorme erudição, pelo “prodigioso conhecimento de tudo”,
conforme resumiu o amigo Henri de Régnier, Schwob
acreditava, como descobriria o epicurista latino nessa breve narrativa, que as
palavras “se entretecem com os átomos do mundo”. E que, longe de
tornar manifesto o sentido último e transcendente dos acontecimentos, ou seja,
longe de traçar-lhes uma dimensão simbólica e convertê-los, assim, em ideia
geral --- como, por exemplo, no fragmento em questão, a ideia de um amor fusional
que unisse definitivamente os amantes ---, as palavras devem, como a morte --- e a
literatura ---, permitir, ou talvez, melhor dizendo, urdir a “alforria” da “turba
turbulenta” de átomos que constituem, finalmente, a vida “rumo a mil outros
movimentos vãos”. 

Tal perspectiva é claramente apresentada pelo próprio escritor em seu prefácio
às \textit{Vidas imaginárias}, no qual ele afirma nos seguintes termos suas
concepções estéticas: “A arte é contrária às  ideias
gerais, descreve apenas o individual, deseja apenas o
único. Não classifica; desclassifica”.  “O ideal do biógrafo”, escreve
Schwob mais adiante, “seria diferenciar ao infinito o aspecto de dois filósofos
que tivessem inventado mais ou menos a mesma metafísica”.  Ou seja, em
vez de reduzir os acontecimentos de uma vida a uma unidade geral, por meio da
qual sua inteligibilidade se revelaria, fixando-os e determinando, por assim
dizer, a identidade do biografado, o biógrafo acumula elipticamente fatos
estranhos --- “movimentos vãos” --- sem encadeamento causal aparente, sem que, no
mais das vezes, a narração de cenas e a descrição de detalhes das vidas
imaginadas formem, em sua justaposição, uma lógica totalizadora. Como se cada
“aspecto” de uma vida --- e da história em geral --- pudesse, justamente,
“diferenciar-se ao infinito”, numa produção vertiginosa de conjecturas a que o
leitor não consegue ficar imune: a ele, também, diante desses textos, só resta
seguir conjecturando sobre seu sentido. 

Parece-me que é sob esse prisma que devemos considerar a tensão entre ficção e
história que atravessa as duas obras que constituem este volume, originalmente
publicadas em 1896: \textit{A cruzada das crianças} e as já referidas
\textit{Vidas imaginárias}. Vale esclarecer que Schwob jamais inventa
inteiramente as narrativas que as compõem, trata-se de reescrever e
reinterpretar textos mais ou menos antigos, relatos e documentos recolhidos em
sua vasta cultura, que aliava a tradição judaica de que era herdeiro --- ele
provinha de uma família de intelectuais judeus --- às literaturas grega, latina e
anglo-saxônica, assimiladas com voracidade ao longo de sua relativamente breve
vida --- o escritor morreu prematuramente em 1905, de problemas pulmonares. 

No caso da \textit{Cruzada}, Schwob parte de crônicas medievais constituídas no
século~\versal{XIII} sobre um grupo de crianças que teriam vindo de toda a Alemanha e de
toda a França para se reunir em torno de um jovem profeta que as conduziria a
Jerusalém. Deus abriria então o Mediterrâneo para que elas passassem, evocando
a saída dos judeus do Egito. No entanto, essas crianças teriam sido raptadas e
vendidas no mercado de escravos. A lenda, que alimentou uma abundante
literatura popular da época, era baseada em fatos históricos, esclarecidos
apenas na segunda metade do século~\versal{XX}, em especial pelos trabalhos de Georges
Duby e Philippe Ariès, que mostraram que elas nasciam do uso repetido, nos
relatos, da palavra \textit{pueri}, que significa de fato \textit{crianças} em
latim clássico, mas que também remetia a pessoas que se encontrassem em
situação de miséria ou de servidão, o que implica que tais cruzadas foram
provavelmente compostas por camponeses marginalizados pelas transformações
sociais e econômicas do século~\versal{XIII}. 

No entanto, independentemente de sua verdade histórica, o que interessa
especialmente a Schwob nos relatos que compõem sua \textit{Cruzada} é aquela
“diferenciação ao infinito” do acontecimento, realizada aqui por meio da
multiplicação de testemunhos do mesmo fato atribuídos a personagens reais e
imaginários. Schwob se inspiraria, como evoca Jorge Luis Borges em prólogo à
edição espanhola de 1949, no método de Robert Browning em seu poema narrativo
\textit{The Ring and the Book} (1868), no qual a história de um crime se revela
 por meio de diversos pontos de vista independentes. Na \textit{Cruzada},
temos, por exemplo, o ressentimento do leproso condenado à danação, e que poupa
a criança que não o teme: 

\begin{quote}
Minha monstruosa brancura semelha para ela a de seu
Senhor. [\ldots{}] E eu lhe disse: --- Vai em paz rumo ao teu Senhor branco, e
dize-lhe que ele me esqueceu.
\end{quote}

E a experiência da dúvida do papa
Inocêncio~\versal{II}, cujo relato, marcado por uma angústia que em certos momentos
parece beirar o cinismo, dirige-se a Deus: 

\begin{quote}
Essas crianças perecerão. Não
façais com que haja sob Inocêncio um novo massacre dos Inocentes. [\ldots{}] A vida
pregressa faz com que hesitem nossas resoluções. Não vi milagre algum.
Ilumina-me. [\ldots{}] Orienta-me, pois eu não sei. Senhor, são teus pequenos
inocentes. E eu, Inocêncio, não sei, não sei.
\end{quote} 

Temos a inocência
crédula das criancinhas, que nada detém:

\begin{quote}
Nós três, Nicolas que não sabe falar,
Alain e Denis, saímos pelas estradas rumo a Jerusalém. Faz tempo que estamos
andando. Foram vozes brancas que nos chamaram na noite. Chamavam todas as
criancinhas. [\ldots{}] Em toda parte há densas florestas, e rios, e montanhas, e
espinhos. Mas em toda parte as vozes estarão conosco.
\end{quote} 

E a condescendência altiva do calândar fiel a Maomé, que se serve do episódio para
confirmar sua própria fé:

\begin{quote}
E se não tivessem afortunadamente caído nas mãos
dos Crentes, teriam sido apanhadas pelos Adoradores do Fogo e acorrentadas em
porões profundos. E aqueles malditos teriam-nas ofertado em sacrifício ao seu
devorador e detestável ídolo. Louvado seja nosso Deus que faz bem tudo aquilo
que faz e protege até mesmo aqueles que não o confessam.
\end{quote} 

Embora distintas, e refletindo, justamente por isso, a rede complexa e escarpada
em que se trama a decantação da aventura da história no imaginário da
humanidade, essas vozes se reúnem, por outro lado, pelo timbre de solidão,
desespero e impotência com que selam a irracionalidade dessa aventura, em que
parece haver apenas vítimas.

Em \textit{Vidas imaginárias}, Schwob parte, também, de fontes históricas e
literárias bastante eruditas, mas desta vez, já o sabemos, para imaginar a vida
de seus protagonistas, todos indivíduos bem reais, mais ou menos celebrados
pela história --- do filósofo Empédocles ou do pintor Uccello, que o leitor
reconhecerá de imediato, à escrava Septima ou ao soldado Alain o Gentil,
certamente mais obscuros ---, e que são por vezes, para surpresa do leitor,
coadjuvados por  personagens bem mais notórios do que eles --- assim, por
exemplo, os percalços do ator Gabriel Spenser servem antes para evocar o famoso
dramaturgo Ben Jonson (1572---1637), contemporâneo de Shakespeare, a matrona impudica Clódia
é pretexto para um retrato de Cícero, a vida do poeta Cecco Angiolieri tem como
contraponto a de Dante, o juiz Nicolas Loyseleur nos põe diante de Joana d’Arc\ldots{} 

O essencial, contudo, é que a matéria narrada, constituída essencialmente de
informações verificáveis, não vem produzir aquele “efeito de real” a que se
referia Roland Barthes aludindo à função, notável especialmente no romance
realista do século~\versal{XIX}, de legitimação histórica da narrativa exercida pela
inserção, como que casual, de pormenores, de pequenos detalhes aparentemente
insignificantes para o enredo. Estes, na verdade, servem antes de tudo, para
Schwob, como “brechas singulares e inimitáveis” que permitem “descrever um
homem em todas as suas anomalias”. Pois não lhe interessa a história por
aquilo que podemos compartilhar com este ou aquele personagem, e que nos
levaria a uma identificação com ele, mas, ao contrário, por aquilo que de cada
um deles permanece incompartilhável, e que o restitui, assim, à sua
singularidade mais plena. Como o próprio escritor antecipa em seu prefácio:

\begin{quote}
Assim como Sócrates, Tales poderia ter dito \textit{Conhece-te a ti mesmo}, mas não esfregaria a perna do mesmo jeito, na prisão, antes de tomar a cicuta.
As ideias dos grandes homens são patrimônio comum da humanidade: cada um deles
só possuiu de fato as próprias esquisitices.
\end{quote}

Em suma, Schwob não vê a literatura como lugar de eleição e de partilha, com uma
comunidade leitora, de um “patrimônio comum da humanidade”. Mas como um
horizonte em que as insignificâncias da vida dos homens --- suas “esquisitices”
---, solicitadas pela imaginação do escritor, podem ostentar-se em toda a sua
misteriosa efemeridade, estimulando, a seu turno, a imaginação do leitor. 

Um exemplo que explicita ironicamente tal perspectiva pode ser encontrado em
``Sr.~Burke e Sr.~Hare, assassinos'', última e perturbadora narrativa da
série de \textit{Vidas imaginárias} --- e espécie de “\textit{mise en abyme}
final da arte do biógrafo”, como assinala o crítico Alexandre Gefen. Aqui o
narrador de Schwob apresenta a vida dos famosos assassinos seriais irlandeses
William Burke e William Hare, que mataram 17 pessoas em Edimburgo, na Escócia,
vendendo seus corpos para o Dr.~Robert Knox, anatomista e professor da
Faculdade de Medicina local. Como o califa das \textit{Mil e uma noites}, conta
o narrador, Burke “desejou misteriosas aventuras, sendo curioso por relatos
desconhecidos e pessoas estrangeiras”. Assim, saía pelas ruas ao anoitecer e
convidava “ao acaso” um “transeunte desconhecido” para ir ao aposento de Hare,
onde os dois o incitavam a contar-lhes “os fatos mais surpreendentes de sua
existência”. No entanto, a despeito do interesse “insaciável” de Burke pelo
relato do estranho, os dois cúmplices, antes que nascesse o dia, asfixiavam-no
lentamente, enquanto “sonhavam, imóveis, com o final da história que nunca
escutavam”, e “[concluindo] assim uma quantidade de histórias que o mundo
jamais conhecerá”. A seguir “exploravam o mistério”: examinavam cuidadosamente
tudo o que encontravam com o cadáver, antes de vendê-lo para dissecação. 


O leitor não poderá deixar aí de deparar-se com a desmedida da tarefa do
biógrafo, e a dimensão intrinsecamente inconclusiva --- e, sobretudo, imaginativa
--- de seu projeto de dar a ver ao mundo uma existência acabada. E de observar a
ironia de que o resultado desse processo de esgotamento concentrado de uma vida
--- o cadáver, justamente --- seja em seguida vendido a um anatomista, aquele que,
à procura da normalidade comum a todos os homens, o reduz a um aglomerado de
órgãos, ironia que talvez se estenda, de um lado, a certa função normativa que
a visada científica, ciosa de regularidade e de universalidade, exigia cada vez
mais do trabalho do historiador, e, de outro lado, à própria posição
(pseudo)exigente do leitor consumidor contemporâneo de Schwob, que, como o
próprio Burke, havia passado a esperar das narrativas algo além de “relatos
eternamente similares da experiência humana”.

Daí, em contrapartida, o desenvolvimento, nos tempos modernos, do “sentimento do
individual” a que se refere Schwob em seu prefácio a \textit{Vidas}, e
que ele, marcado pelo simbolismo de seus contemporâneos, sugere mais do que
explicita, inoculando-o em nós, leitores, através da “milagrosa mutação [\ldots{}]
da semelhança em diversidade”, alcançada estilisticamente por meio das
elipses, condensações, saltos, interrupções que marcam suas narrativas, tanto
ao nível da estrutura quanto do próprio encadeamento das frases, muitas vezes
justapostas sem um princípio de ligação ou de causalidade aparente. O que faz
justamente com que, vez por outra, ergamos os olhos do livro para devanear com
alguma alusão de passagem ou para especular sobre as inextricáveis intrigas da
História. Qual o sentido, por exemplo, de o incendiário Heróstrato ter morrido
no dia do nascimento de Alexandre da Macedônia? 

Antes de liberar o leitor para entregar-se ao prazer da leitura e às suas
próprias especulações, façamos um retrato sumário da vida do escritor --- retrato
nada schwobiano, como se haverá de notar ao longo da leitura do livro. De
família culta --- Isaac-Georges Schwob, seu pai, foi proprietário de jornais,
frequentou os parnasianos e foi amigo de Flaubert, e Léon Cahun, seu tio
materno, escreveu inúmeros livros e foi diretor da célebre Biblioteca Mazarine,
em Paris ---, Schwob teve sólida formação intelectual. Depois da infância,
passada no interior da França, muda-se para Paris, onde mora com o tio. Estuda
no Liceu Louis-le-Grand, frequenta a École Pratique des Hautes Études, conclui
o curso de letras na Sorbonne. Seu espectro de leituras e de interesses
intelectuais sempre foi imenso. Tudo o fascina, de Jules Verne, uma de suas
leituras preferidas desde a infância, a Luciano de Samósata, que traduz aos 16
anos; de Schopenhauer, estudado com dedicação desde a adolescência, a Saussure,
cujo curso sobre a fonética indo-europeia frequenta com entusiasmo na Sorbonne;
de François Villon, a partir de cuja obra escreveria um estudo sobre a gíria
dos ladrões, \textit{Le Jargon des Coquillards} (O jargão dos
ladrões), a Robert Louis Stevenson, cuja obra traduziu para o francês e a quem
dedicou sua primeira obra de ficção, \textit{C\oe ur double} (Coração
duplo). Colabora com jornais importantes da época --- \textit{L'Écho} e
\textit{L'Événement}, por exemplo ---, onde encontra diversos escritores e
intelectuais ativos, como Cattulle Mendès, Jean Lorrain, Guy de Maupassant,
Rémy de Goncourt e Maurice Barrès. É íntimo de Paul Claudel, Oscar Wilde e
Alfred Jarry, que lhe dedicaria \textit{Ubu Rei}. Estimula e ajuda a promover
jovens escritores pouco conhecidos, entre os quais Paul Verlaine e Jules
Renard. Traduz Defoe, Meredith, Whitman. Seus livros são recebidos
calorosamente por seus contemporâneos. Mallarmé declara-se fascinado pelo
\textit{Livro de Monelle}, por exemplo. Morre aos 37 anos, após longa luta
contra a doença.

A despeito do interesse que uma exposição menos sucinta da rica trajetória
biográfica do autor pudesse oferecer, preferimos nos deter aqui e estimular o
leitor a passar de imediato ao que importa: à obra de Marcel Schwob. Pois como
bem dizia Borges --- que refere, aliás, as \textit{Vidas imaginárias} como uma
das fontes mais significativas de sua \textit{História universal da infâmia} ---,
a biografia de um grande escritor é antes de tudo sua própria obra. É nela que
se situa o centro de seu labirinto. Vamos, portanto, a ela.






