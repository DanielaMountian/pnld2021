\chapter{Introdução}

A presente antologia reúne três das mais representativas obras literárias do escritor francês Marcel Schwob.
Além de apresentar ao público brasileiro três obras"-primas desse literato ainda pouco conhecido no país, essa reunião desvela os principais temas e características que marcaram toda a produção literária de Schwob: a mistura entre a história e a ficção, a confabulação sobre outros tempos históricos, o jogo de duplos e ilusões, o experimentalismo entre diversos gêneros literários e o viés poético de sua escrita em prosa.

A primeira narrativa, \textit{A cruzada das crianças} (1896), tem como ponto de partida as crônicas medievais do século \versal{XIII} 
sobre um grupo de crianças alemãs e francesas que teriam se reunido em torno de um jovem profeta para
marchar rumo a Jerusalém. A narrativa é composta de oito relatos que trazem
pontos de vista independentes sobre o acontecimento.
Também em \textit{Vidas imaginárias} (1896) temos narrativas cujos protagonistas são personagens históricos,
mais ou menos conhecidos. Schwob reconstitui à sua maneira a trajetória de
filósofos, escritores, escravos, soldados, piratas e criminosos, seja a partir
de biografias já existentes ou documentação histórica, seja a
partir de fontes literárias.

Por fim temos \textit{O livro de Monelle} (1894), obra única, imune a classificações. Originalmente, os dezessete 
contos que formam a obra foram publicados entre 1892 e 1894 no \textit{L'Écho de Paris}, sendo reunidos num volume 
por Léon Chailley em 1894. O livro se organiza sob a forma de um tríptico, no qual Schwob cria uma mistura sutil de
gêneros: conto, poema em prosa e texto profético sob forma de versículos. Cada
uma de suas partes se organiza, por sua vez, em fragmentos independentes que se
diferenciam não apenas por sua forma, mas também por seus temas, unidos, no
entanto, por um fio condutor: Monelle, misterioso personagem feminino. \textit{O livro
de Monelle} é um livro de luto que fascinou, à época de sua
primeira publicação, nomes como Mallarmé e Anatole France. Uma das obras mais
conhecidas de Marcel Schwob, foi várias vezes adaptado para
teatro e rádio.

Nas duas primeiras narrativas, percebe"-se a influência que o interesse histórico exerceu sobre a produção ficcional do autor. Indo morar com seu tio, um orientalista e erudito francês, ainda na adolescência, desde jovem Schwob despertou uma curiosidade pelas línguas antigas, como sânscrito, grego e latim, e pela história da civilização.
O escritor, então, vai colher em pequenos acontecimentos excêntricos do passado, como uma cruzada conduzida por crianças ou a vida dos ilustres desconhecidos da história, a inspiração para as suas obras. Na última narrativa, o leitor pode perceber a influência mais direta de sua biografia na produção ficcional. Ao que tudo indica, Monelle representava Louise, jovem que Schwob conheceu e amou em finais da década de 1890 e que morreu de tuberculose aos vinte e cinco anos.

Além de desvelarem a influência de sua biografia em sua arte, essas três narrativas conectam"-se pelos procedimentos estéticos que utilizam, marcando uma prosa viva, moderna e inventiva. Vale notar, por fim, que as três seguem uma estrutura parecida. Apesar de serem livro de contos, suas narrativas curtas se encadeiam em um mesmo eixo temático, formando cada obra um universo autônomo e interligado, em que a sequência de contos transmitem a sensação de conjunto de um romance.

\part{\textsc{vidas imaginárias}}
%\hedramarkboth{vidas imaginárias}{marcel schwob}

\chapter{Prefácio}


A ciência histórica nos deixa na incerteza acerca dos indivíduos.
Revela-nos somente os pontos em que eles foram vinculados a ações
universais. Ela nos ensina que Napoleão estava adoentado no dia de
Waterloo, que a excessiva atividade intelectual de Newton deve ser
atribuída à absoluta continência de seu temperamento, que Alexandre estava
ébrio quando matou Klitos e que a fístula de Luís \versal{XIV} pode ter motivado
certas decisões suas. Pascal discorre sobre o nariz de Cleópatra, caso
houvesse sido mais curto, ou sobre um grão de areia na uretra de Cromwell.
Estes fatos individuais todos só têm valor porque modificaram os
acontecimentos ou poderiam ter alterado sua sequência. São causas reais ou
possíveis. Há que deixá-las para os eruditos.

A arte é contrária às ideias universais, descreve apenas o
individual, deseja apenas o único. Não classifica; desclassifica. Pelo
tanto que nos interessam, nossas ideias universais podem até ser
similares àquelas vigentes no planeta Marte, e três linhas
cruzadas formam um triângulo em qualquer ponto do universo. Agora, reparem
na folha de uma árvore, com suas nervuras caprichosas, seus matizes
variados pela sombra e pelo sol, o inchamento devido a uma gota d’água
caída, a picada deixada por um inseto, o rastro prateado do pequeno
caracol, a primeira douradura mortal que assinala o outono; procurem outra
folha exatamente igual em todas as grandes florestas da terra: eu lhes
lanço o desafio. Não existe uma ciência do tegumento de um folíolo, dos
filamentos de uma célula, da curvatura de uma veia, da mania de um hábito,
das arestas de um caráter. Que tal homem tivesse o nariz torto, um olho
mais alto que o outro, a articulação do braço nodosa; que tivesse o hábito
de comer a tal hora carne branca de frango, que preferisse o malvasia ao
Château-Margaux, eis o que não tem paralelo no mundo. Assim como Sócrates,
Tales poderia ter dito \textit{γνῶθι σεαυτόν},\footnote{ Em grego, g\textit{n\=othi seauton}, ou
seja: ``Conhece-te a ti mesmo''.} mas não esfregaria
a perna do mesmo jeito, na prisão, antes de tomar a cicuta. As ideias dos
grandes homens são patrimônio comum da humanidade: cada um deles só
possuiu de fato as próprias esquisitices. O livro que descrevesse um homem
em todas as suas anomalias seria uma obra de arte, qual estampa japonesa
em que se vê eternamente a imagem de uma pequena lagarta vislumbrada certa
vez em determinada hora do dia.

As histórias se quedam caladas a respeito dessas coisas. Na fantástica
coleção de materiais fornecida pelos testemunhos, poucas são as brechas
singulares e inimitáveis. Os biógrafos antigos, sobretudo, são sovinas.
Considerando tão somente a vida pública e a gramática, dos grandes homens
nos transmitiram os discursos e títulos de seus livros. O próprio
Aristófanes foi quem nos deu a alegria de descobrir que ele era calvo, e
não fosse o nariz achatado de Sócrates ter servido para comparações
literárias, não fosse seu hábito de andar descalço ter sido parte de seu
sistema filosófico de desprezo pelo corpo, dele só conservaríamos os
interrogatórios sobre a moral. Os mexericos de Suetônio não passam de
polêmicas raivosas. O gênio bom de Plutarco fez dele, às vezes, um
artista; mas não soube entender a essência de sua arte, já que concebeu
“paralelos” --- como se dois homens adequadamente descritos nos seus mínimos
detalhes pudessem se parecer! Vemos-nos reduzidos a consultar Atenaios,
Aulo Gélio, os escoliastas e Diógenes Laércio, o qual acreditava ter
escrito uma espécie de história da filosofia.

O sentimento do individual se desenvolveu com mais força nos tempos
modernos. A obra de Boswell seria perfeita, não tivesse ele achado
necessário citar a correspondência de Johnson e fazer digressões sobre
seus livros. As \textit{Vidas de homens eminentes} de Aubrey\footnote{
\textit{Lives of Eminent Men}, de John Aubrey (1626--1697), arqueólogo,
antiquário e escritor inglês, também autor de \textit{Brief Lives}.} 
são mais satisfatórias. Aubrey
possuía, sem dúvida alguma, o instinto da biografia. É lamentável que o
estilo deste excelente antiquário não estivesse à altura de sua concepção!
Seu livro poderia ter sido a eterna recreação dos espíritos sensatos.
Aubrey nunca julgou necessário estabelecer uma relação entre detalhes
individuais e ideias universais. Bastava-lhe que outros tivessem indicado
à celebridade os homens pelos quais se interessava. Ignoramos, no mais das
vezes, se se trata de um matemático, de um homem de Estado, de um poeta ou
de um relojoeiro. Mas cada um deles tem seu traço único a distingui-lo,
para sempre, entre os homens.

O pintor Hokusai esperava alcançar, aos cento e dez anos, o ideal
de sua arte. Neste momento, dizia, seriam vivos todo ponto, toda linha
traçados por seu pincel. Por vivos, entenda-se individuais. Não há nada
mais parecido do que pontos e linhas: a geometria se funda sobre este
postulado. A arte perfeita de Hokusai exigia que nada fosse mais distinto.
Assim, o ideal do biógrafo seria diferenciar ao infinito o aspecto de dois
filósofos que tivessem inventado mais ou menos a mesma metafísica. Eis por
que Aubrey, que se atém exclusivamente aos homens, não alcança a
perfeição, não tendo sabido efetuar a milagrosa mutação, almejada por
Hokusai, da semelhança em diversidade. Mas Aubrey não chegou aos cento e
dez anos. Merece, contudo, todo o respeito, e tinha consciência
do alcance de seu livro. “Recordo'', diz ele, em seu prefácio a Anthony
Wood, ``de um dito do general Lambert --- \textit{that the best of men are but men at
the best} --- de que irão encontrar diversos exemplos nesta tosca e
apressada coletânea. De modo que esses arcanos só deverão ser trazidos à
luz dentro de cerca de trinta anos. Convém, com efeito, que autor e
personagens (tal como as nêsperas) já  tenham apodrecido”.

É possível descobrir nos antecessores de Aubrey alguns rudimentos
de sua arte. Assim, conta Diógenes Laércio que Aristóteles andava com uma
bolsa de couro cheia de óleo quente sobre o estômago, e que encontraram em
sua casa, depois de sua morte, uma quantidade de vasos de barro. Nunca
saberemos o que fazia Aristóteles com aquela cerâmica toda. E este
mistério é tão prazeroso como as conjeturas em que nos deixa Boswell sobre
o uso que Johnson fazia das cascas secas de laranja que costumava levar
nos bolsos. Nesse ponto, Diógenes Laércio quase se alça ao sublime do
inimitável Boswell. São estes, porém, raros prazeres. Ao passo que Aubrey
os fornece a cada linha. Milton, diz ele, “pronunciava com aspereza a
letra R”. Spenser “era um homem baixo, usava cabelos curtos, um estreito
cabeção e punhos estreitos”.

Barclay “vivia na Inglaterra em alguma época \textit{tempore} R.~Jacobi.
Era então um homem idoso, de barba branca, e usava um chapéu com pluma, o
que escandalizava certas pessoas severas”. Erasmo “não gostava de peixe,
embora nascido numa cidade peixeira”. Quanto a Bacon, “nenhum de seus
criados ousava se apresentar diante dele sem estar calçando botas de couro
espanhol; pois ele imediatamente sentia o cheiro do couro de terneiro, que
muito lhe desagradava”. O doutor Fuller “tinha a cabeça tão imersa no
trabalho que, ao passear e meditar antes do jantar, comia um pão de dois
soldos sem perceber”. Sobre William Davenant, faz a seguinte observação:
“Estive no seu enterro; o caixão era de nogueira. O sr.~John Denham
garantiu que era o caixão mais bonito que já vira”. Escreveu
acerca de Ben Jonson: “Ouvi do sr.~Lacy, o ator, que ele costumava usar
um casaco igual ao de um cocheiro, com fendas sob as axilas”. Eis o que o
impressiona em William Prynne: “Sua maneira de trabalhar era a seguinte.
Vestia um comprido gorro pespontado que lhe caía no mínimo duas, três
polegadas sobre os olhos e lhe servia de abajur para protegê-los da luz, e
a mais ou menos cada três horas seu criado tinha de lhe trazer pão e uma
bilha de cerveja para revigorar-lhe o espírito; de modo que ele
trabalhava, bebia e mascava seu pedaço de pão, entretendo-se assim até à
noite, quando comia um bom jantar”. Hobbes “foi se tornando muito calvo na
velhice; dentro de casa, porém, costumava estudar com a cabeça descoberta,
e afirmava nunca sentir frio, embora seu maior tormento fosse impedir que
as moscas pousassem em sua calva”. Ele nada nos diz sobre a
 Oceana de John Harrington, mas conta que o autor “\versal{A.D.}~1660 
 foi mandado como prisioneiro à Torre, onde o mantiveram,
e em seguida a Portsey Castle. Sua estada nessas prisões (sendo ele um
cavalheiro de espírito elevado e cabeça quente) foi a causa procatártica
de seu delírio ou loucura, a qual não foi furiosa --- já que conversava de
modo bastante razoável e era uma companhia bem agradável; mas acometeu-o a
fantasia de que seu suor se transformava em moscas, às vezes em abelhas,
\textit{ad cetera sobrius}; e mandou construir uma casinha versátil de
madeira no jardim do sr.~Hart (defronte ao St. James’s Park) para fazer a
experiência. Voltava-a para o sol e sentava-se diante dela; depois mandava
trazer suas caudas de raposa para espantar e massacrar todas as moscas e
abelhas que nela se encontrassem; em seguida fechava os caixilhos. Ora,
ele só realizava esta experiência na estação quente, de modo que algumas
moscas se dissimulavam nas frestas e dobras dos drapejados. Ao cabo de um
quarto de hora, talvez, o calor enxotava uma mosca, duas, ou mais para
fora da toca. Ele então exclamava: “Não estão vendo que, claramente, elas
saem de mim?”.

Eis só o que ele nos conta sobre Meriton. “Seu verdadeiro nome
era Head. O sr.~Bovey o conhecia muito bem. Nascido em\ldots{} Era
livreiro na Little Britain.\footnote{ Uma rua da City de Londres.} 
Vivera entre os ciganos. Tinha um ar tratante com aqueles
seus olhos debochados. Era capaz de se transmutar em qualquer forma. Foi
duas, três vezes à falência. Foi afinal livreiro, já próximo do fim.
Ganhava a vida com seus rabiscos. Pagavam-lhe vinte shillings por
folha. Escreveu vários livros: \textit{The English Rogue}, \textit{The Art of
Wheadling} etc. Afogou-se quando ia a Plymouth por mar, em cerca
de 1676, com aproximadamente 50 anos de idade.”


Por fim, há que citar sua biografia de Descartes:

``Sr. Renatus Des Cartes

Nobilis Gallus, Perroni Dominus, summus Mathematicus et
Philosophus, natus Turonum, pridie Calendas Aprile 1596. Denatus Holmioe,
Calendis Februarii, 1650. (Deparo com esta inscrição sob seu
retrato por C.V.~Dalen.) Como ele ocupou o tempo na juventude e por que
método tornou-se tão sábio, isso ele conta ao mundo em seu tratado
intitulado \textit{Do método}. A Sociedade de Jesus vangloria-se de que
cabe a esta ordem a honra de sua educação. Viveu vários anos em Egmont
(perto de Haia), de onde datou vários de seus livros. Era um homem
demasiado sábio para se estorvar com uma mulher; mas, sendo homem, tinha
desejos e apetites de homem; mantinha, então, uma bela mulher de boa
condição, que ele amava e da qual teve alguns filhos (acho que dois ou
três). Muito surpreendente seria se, oriundos de tal pai, não tivessem
recebido uma bela educação. Era tão eminentemente sábio que todos os
sábios iam visitá-lo e muitos lhe rogavam que mostrasse seus\ldots{}
instrumentos (naquela época a ciência matemática era fortemente ligada ao
conhecimento dos instrumentos e, como dizia o sr.~H.S., à prática dos
tornos). Ele então puxava uma gavetinha embaixo da mesa e lhes mostrava um
compasso com uma perna quebrada; e, como régua, usava uma folha de papel
dobrada ao meio”.

Está claro que Aubrey teve perfeita consciência de seu trabalho. Não pensem
que ele desconheceu o valor das ideias filosóficas de Descartes ou Hobbes.
Só não era isso que o interessava. Ele nos diz com muita propriedade que o
próprio Descartes expôs seu método ao mundo. Não ignora que Harvey
descobriu a circulação do sangue; mas prefere registrar que este grande
homem passava suas insônias perambulando de camisão, tinha uma caligrafia
ruim, e que os mais famosos médicos de Londres não dariam nem seis tostões
por uma de suas receitas. Está convencido de que nos disse tudo sobre
Francis Bacon ao explicar que este tinha olhos vivos e delicados, cor de
avelã, iguais aos olhos da víbora. Não é, porém, tão grande artista quanto
Holbein. Não sabe fixar um indivíduo para a eternidade através de seus
traços particulares sobre um fundo de parecença com o ideal. Ele dá vida a
um olho, ao nariz, à perna, aos muxoxos de seus modelos: não sabe animar a
figura. O velho Hokusai percebia muito bem a necessidade de tornar
individual aquilo que há de mais genérico. Aubrey não teve a mesma
penetração. Se o livro de Boswell coubesse em dez páginas, seria a obra de
arte esperada. O bom senso do doutor Johnson compõe-se dos mais banais
lugares-comuns; expresso com essa estranha violência que Boswell soube
retratar, possui uma qualidade única no mundo. Só que este pesado catálogo
se parece com os próprios dicionários do doutor; seria possível dele tirar
uma Scientia Johnsoniana, com índice. Boswell não teve a coragem estética de escolher.

A arte do biógrafo consiste justamente na escolha. Não lhe cabe a
preocupação de ser verdadeiro; ele deve criar em meio a um caos de traços
humanos. Diz Leibniz que, para fazer o mundo, Deus escolheu o melhor
dentre os possíveis. O biógrafo, qual divindade inferior, sabe escolher
entre os possíveis humanos aquele que é único. Não deve se enganar sobre a
arte, como Deus não se enganou sobre a bondade. O instinto de ambos
precisa ser infalível. Pacientes demiurgos reuniram para o biógrafo
ideias, fatos, movimentos de fisionomias. Sua obra se encontra nas
crônicas, nas memórias, nas correspondências e nos escólios. Em meio a
esse grosseiro conjunto, o biógrafo seleciona o material para compor uma
forma que não se pareça com nenhuma outra. Não precisa ser igual àquela
criada outrora por um deus superior, desde que seja única, como toda criação.

Infelizmente, os biógrafos em geral julgaram ser historiadores. E
nos privaram assim de retratos admiráveis. Presumiram que só a vida dos
grandes homens nos poderia interessar. A arte é alheia a tais
considerações. Aos olhos do pintor, o retrato, por Cranach, de um homem
desconhecido, tem tanto valor quanto o retrato de Erasmo. Não é pelo nome
de Erasmo que este quadro é inimitável. A arte do biógrafo seria dar igual \label{vida}
valor à vida de um pobre ator e à vida de Shakespeare. Um baixo instinto é
que nos leva a reparar com prazer no encurtamento do esternomastoide no
busto de Alexandre, ou a mecha sobre a testa no retrato de Napoleão. Mais
misterioso é o sorriso de Mona Lisa, de quem nada sabemos (pode ser o
rosto de um homem). Uma careta desenhada por Hokusai nos leva a mais
fundas meditações. Se fôssemos experimentar a arte em que exceleram
Boswell e Aubrey, decerto não nos caberia descrever em minúcias o maior
homem de seu tempo, nem registrar a característica dos mais célebres do
passado, e sim, contar com igual cuidado as existências únicas
dos homens, quer tenham sido divinos, medíocres ou criminosos.


\chapter{Empédocles (deus presumido)}

Ninguém sabe do seu nascimento nem como veio para a terra. Apareceu próximo
às margens douradas do rio Acragas, na bela cidade de Agrigento, pouco
depois do tempo em que Xerxes mandou açoitar o mar com
correntes.\footnote{ Khshayarsha \versal{I} (c.~519--465 a.C.), 
rei da Pérsia, conhecido pelos romanos como Xerxes. Certa vez,
quando o mar destruiu uma ponte pela qual pretendia passar, mandou
açoitá-lo e, com grossas correntes, prendeu várias embarcações e sobre
elas passou com suas tropas.} A tradição reporta apenas que seu
avô se chamava Empédocles: ninguém o conheceu. Deve-se decerto deduzir daí
que era filho de si mesmo, como convém a um Deus. Mas seus discípulos
asseguram que antes de percorrer em sua glória os campos da Sicília, ele
já passara quatro existências em nosso mundo, tendo sido planta, peixe,
pássaro e donzela. Usava um manto de púrpura sobre o qual caíam seus
longos cabelos; tinha na cabeça uma faixa dourada, nos pés, sandálias de
bronze, e levava grinaldas trançadas de louro e lã.

Pela imposição das mãos curava os doentes e declamava versos, ao modo
homérico, com inflexões pomposas, em pé sobre um carro e rosto voltado
para o céu. O povo o seguia em multidão e se prosternava diante dele para
escutar seus poemas. Sob o céu puro que alumia os trigais, de toda parte
vinham os homens até Empédocles, os braços carregados de oferendas. Ele os
mantinha boquiabertos cantando para eles a abóbada divina, feita de
cristal, a massa de fogo que nomeamos sol, e o amor que, qual vasta
esfera, tudo contém.

Todos os seres, dizia, são meros pedaços desconjuntados desta esfera de
amor em que o ódio se insinuou. E o que chamamos de amor é o desejo de nos
unir e nos fundir e confundir, tal como éramos outrora, no seio do deus
globular que a discórdia veio romper. Ele invocava o dia em que a esfera
divina se enfunaria, depois de todas as transformações das almas. Pois
esse mundo que conhecemos é obra do ódio, e sua dissolução será obra do
amor. Assim cantava ele pelas cidades e campos; e suas sandálias de bronze
trazidas da Lacônia tilintavam em seus pés, e à sua frente soavam os
címbalos. Entretanto jorrava, da goela do Etna, uma coluna de negra fumaça
a lançar sua sombra sobre a Sicília.

Qual rei do céu, Empédocles andava envolto em púrpura e cingido de ouro,
enquanto os pitagoristas se arrastavam em suas túnicas finas de linho, com
calçados feitos de papiro. Diziam que ele sabia fazer sumir a remela,
dissolver os tumores, e tirar as dores dos membros; rogavam-lhe que
fizesse cessar chuvas e tormentas; ele conjurou as tempestades de um
círculo de colinas; em Selinonta, expulsou a febre desaguando dois rios no
leito de um terceiro; e os habitantes de Selinonta o adoraram e
ergueram-lhe um templo, e cunharam medalhas em que sua imagem figurava
face a face com a imagem de Apolo.

Outros afirmam que ele foi adivinho, instruído pelos mágicos da Pérsia, que
dominava a necromancia e a ciência das ervas que enlouquecem. Certo dia em
que ceava em casa de Anquitos, um homem furioso, gládio em riste, irrompeu
sala adentro. Empédocles se levantou, estendeu os braços e cantou os
versos de Homero sobre o nepentes que dá insensibilidade. E logo a força
do nepentes tomou conta do furioso, e ele quedou-se parado, o gládio no
ar, esquecido de tudo, como quem bebeu o doce veneno mesclado ao vinho
espumante de uma cratera.\footnote{ Espécie de jarro, em forma de ânfora, usado pelos gregos para servir água e vinho.}

Os doentes vinham até ele fora das cidades e o cercava uma multidão de
miseráveis. Vieram mulheres se juntar ao seu séquito. Beijavam as bordas
de seu manto precioso. Uma delas se chamava Panteia, filha de um nobre de
Agrigento. Estava para ser consagrada a Ártemis, mas fugiu para longe da
fria estátua da deusa e devotou sua virgindade a Empédocles. Não se viram
as marcas de seu amor, pois Empédocles preservava uma insensibilidade
divina. Não proferia palavras senão em métrica épica, e em dialeto da
Jônia, embora o povo e seus fiéis só empregassem o dórico. Seus gestos
todos eram sagrados. Quando se acercava dos homens, era para abençoá-los
ou curá-los. A maior parte do tempo, permanecia em silêncio. Nenhum dos
que o seguiam jamais conseguiu surpreendê-lo em seu sono. Só o viram
majestoso.

Panteia se vestia de ouro e de lã fina. Seus cabelos eram arrumados à rica
moda de Agrigento, onde a vida transcorria mansamente. Tinha os seios
sustentados por um \textit{strophium} vermelho, e era perfumada a sola de
suas sandálias. No mais, era bela e esguia de corpo, e de cor mui
desejável. Não há como afirmar que Empédocles a amava, mas ele sentiu pena
dela. Com efeito, o vento vindo da Ásia engendrou a peste nos campos
sicilianos. Muitos homens foram tocados pelos dedos negros do flagelo. Até
os cadáveres dos bichos juncavam a orla dos prados, e avistavam-se aqui e
ali ovelhas despeladas, mortas, com a goela aberta para o céu, as costelas
salientes. E Panteia foi ficando lânguida por causa da doença. Caiu aos
pés de Empédocles e já não respirava. Os que a cercavam soergueram seus
membros enrijados e os banharam com vinhos e aromas. Desataram o
\textit{strophium} vermelho que prendia seus seios jovens, e enrolaram-na
em tiras. E sua boca entreaberta foi firmada com uma atadura e seus olhos
cavos já não miravam a luz.

Empédocles olhou para ela, soltou o diadema de ouro que lhe cingia a
fronte, e o impôs sobre ela. Depositou sobre seus seios a grinalda de
louro profético, cantou versos desconhecidos sobre a migração das almas, e
por três vezes ordenou-lhe que se levantasse e andasse. A multidão estava
tomada de terror. Ao terceiro chamado, Panteia emergiu do reino das
sombras, e seu corpo se animou e se pôs de pé, todo enrolado nas faixas
funerárias. E o povo viu que Empédocles era um evocador dos mortos.

Pisianato, pai de Panteia, veio adorar o novo deus. Estenderam-se mesas sob
as árvores de seus campos, para lhe oferecer libações. Ao lado de
Empédocles, escravos seguravam grandes tochas. Os arautos proclamaram, tal
como nos mistérios, o silêncio solene. Súbito, durante a terceira vigília,
apagaram-se as tochas e a noite envolveu os adoradores. Ouviu-se uma voz
forte que chamou: “Empédocles!”. Quando fez-se a luz, Empédocles havia
sumido. Os homens não tornaram a vê-lo.

Um escravo apavorado contou que vira um risco vermelho sulcando as trevas
para os lados do cume do Etna. Os fiéis galgaram as encostas estéreis da
montanha ao morno clarão da aurora. A cratera do vulcão vomitava um feixe
de chamas. Encontraram, na borda porosa de lava que circunda o abismo
ardente, uma sandália de bronze lavrada pelo fogo.

\chapter{Eróstrato incendiário}

A cidade de Éfeso, onde nasceu Heróstrato, se espraiava pela embocadura do
Caistro, com seus dois portos fluviais, até os cais de Panorma de onde se
avistava, sobre o mar de cor profunda, a linha brumosa de Samos. Era rica
em ouro e tecidos, rosas e lãs, desde que os magnesianos, com seus cães de
guerra e seus escravos lançadores de dardos, tinham sido vencidos às
margens do Meandro, desde que a magnífica Mileto fora arruinada pelos
persas. Era uma cidade indolente, em que as cortesãs eram festejadas no
templo de Afrodite Hetaira. Os efésios usavam túnicas amorginas,\footnote{
Túnicas confeccionadas na ilha de Amorgos, no Mar Egeu, muito
transparentes e de cor sempre vermelha.} transparentes, vestes
cor de violeta, de púrpura e de açafrão de linho fiado na roca, sarapides
cor de maçã amarela e brancas e rosas, tecidos do Egito da cor do jacinto,
com os fulgores do fogo e os matizes moventes do mar, e
calasiris\footnote{ Heródoto, em sua \textit{História} (\versal{II}, \versal{LXXXI}),
descreve o calasiris como sendo uma “veste de linho com franjas em torno
das pernas” usada pelos egípcios.} da Pérsia, de um pano denso,
leve, salpicado, sobre fundo escarlate, de grãos de ouro apurado no
crisol.

Entre o Monte Prion e uma alta falésia escarpada, avistava-se, à beira do
Caistro, o grande templo de Ártemis. Levara cento e vinte anos para ser
construído. Rígidas pinturas ornavam as câmaras interiores, cujo teto era
de ébano e cipreste. As pesadas colunas que o sustinham tinham sido
besuntadas com mínio. O salão da deusa era pequeno e oval. No meio,
erguia-se uma prodigiosa pedra negra, cônica e reluzente, marcada de
douraduras lunares, que não era senão Ártemis. Também o altar triangular
era talhado em pedra negra. Outras mesas, feitas de lajes negras,
continham furos regulares por onde escorria o sangue das vítimas. Nas
paredes pendiam largas lâminas de aço, encabadas de ouro, que serviam para
cortar gargantas, e o piso polido estava juncado de tiras ensanguentadas.
A grande pedra escura tinha duas mamas duras e pontudas. Assim era a
Ártemis de Éfeso. Sua divindade se perdia na noite das tumbas egípcias, e
tinha de ser adorada de acordo com os ritos persas. Possuía um tesouro
encerrado numa espécie de colmeia pintada de verde, cuja porta piramidal
era guarnecida com pregos de bronze. Ali, em meio aos anéis, às grandes
moedas e aos rubis, jazia o manuscrito de Heráclito, que proclamara o
reinado do fogo. O próprio filósofo o depositara na base da pirâmide,
enquanto a construíam.

A mãe de Heróstrato era violenta e orgulhosa. Nunca se soube quem era seu
pai. Heróstrato, mais tarde, declarou ser filho do fogo. Seu corpo tinha a
marca, sob o mamilo esquerdo, de uma meia-lua que pareceu se inflamar
quando o torturaram. As mulheres que ajudaram em seu parto anunciaram que
ele era consagrado a Ártemis. Foi colérico e permaneceu virgem. Tinha o
rosto corroído por linhas escuras e o tom de sua pele puxava para o preto.
Desde criança, gostava de ficar sob a alta falésia, perto do Artemision.
Olhava passarem as procissões de oferendas. Por ser ignorada a sua raça,
não pôde tornar-se sacerdote da deusa à qual se julgava votado. O colégio
sacerdotal teve de lhe proibir várias vezes a entrada no
\textit{naos},\footnote{ Espaço do templo reservado à estátua da divindade.} 
onde ele contava erguer o tecido precioso e pesado que velava
Ártemis. Concebeu ódio por isso e jurou violar o segredo.

O nome Heróstrato era, a seu ver, a nenhum outro igualável, e sua própria
pessoa lhe parecia superior à humanidade inteira. Desejava a glória. De
início se aliou aos filósofos que ensinavam a doutrina de Heráclito: eles,
porém, desconheciam sua parte secreta, pois que estava encerrada na
pequena cela piramidal do tesouro de Ártemis. Heróstrato só conjecturou
sobre a opinião do mestre. Endureceu-se no desprezo pelas riquezas que o
cercavam. Era extrema a sua repulsa pelo amor das cortesãs. Julgavam que
ele reservava sua virgindade para a deusa. Mas Ártemis não teve pena dele.
Parecia perigoso ao Conselho da Gerúsia,\footnote{ Conselho de anciãos, um
dos órgãos de governo de Esparta.} que vigiava o templo. O
sátrapa\footnote{ Governador de província na Pérsia antiga.}
permitiu que o exilassem nos subúrbios. Viveu na encosta do Koressos, num
subterrâneo escavado pelos antigos. Dali espreitava, à noite, as lâmpadas
sagradas do Artemision. Há quem suponha que persas iniciados vieram
entreter-se com ele. É mais provável, porém, que seu destino lhe tenha
sido revelado de repente.

Ele de fato confessou, durante a tortura, que compreendera subitamente o
sentido da expressão de Heráclito, \textit{o caminho do alto}, e por que
ensinava o filósofo que a melhor alma é aquela mais seca e mais inflamada.
Assegurou que sua alma era, neste sentido, a mais perfeita, e que quisera
proclamá-lo. Não deu outro motivo para o seu gesto além da paixão pela
glória e da alegria de ouvir pronunciarem o seu nome. Disse que só seu
reino teria sido absoluto, já que não tinha pai conhecido e que Heróstrato
seria coroado por Heróstrato, que era filho de sua obra, e que sua obra
era a essência do mundo: que assim teria sido, a um só tempo, rei,
filósofo e deus, único entre os homens.

No ano de 356, na noite de 21 de julho, não tendo a lua subido ao céu e
tendo o desejo de Heróstrato adquirido inusitada força, resolveu ele
violar a câmara secreta de Ártemis. Esgueirou-se então pela sinuosidade da
montanha até a margem do Caistro e galgou os degraus do templo. Os guardas
dos sacerdotes dormiam junto às santas lâmpadas. Heróstrato pegou uma
delas e penetrou no \textit{naos}.

Um forte cheiro de óleo de nardo se exalava. As negras arestas do teto de
ébano resplandeciam. O oval da câmara era dividido pela cortina tecida em
fios de ouro e púrpura que ocultava a deusa. Heróstrato arrancou-a,
ofegando de volúpia. Sua lâmpada alumiou o terrível cone com mamilos
eretos. Heróstrato agarrou-os com ambas as mãos e beijou com avidez a
pedra divina. Depois deu-lhe a volta, e avistou a pirâmide verde onde se
achava o tesouro. Segurou os pregos de bronze da portinhola e descerrou-a.
Mergulhou os dedos nas joias virgens. Mas pegou apenas o rolo de papiro em
que Heráclito inscrevera seus versos. Ao clarão da lâmpada sagrada,
leu-os, e tudo conheceu.

De chofre, exclamou: “Fogo, fogo!”.

Puxou a cortina de Ártemis e aproximou a mecha acesa da barra inferior. O
pano, de início, ardeu devagar; depois, devido aos vapores do óleo
perfumado que o impregnava, a chama ergueu-se, azulada, para os lambris de
ébano. O terrível cone refletiu o incêndio.

O fogo enrolou-se nos capitéis das colunas, rastejou pelas abóbadas. As
placas de ouro votadas à poderosa Ártemis caíram, uma por uma, das
suspensões para as lajes num estardalhaço metálico. Então, o feixe
fulgurante rebentou no telhado e iluminou a falésia. As telhas de bronze
ruíram. Heróstrato erguia-se em meio ao clarão, clamando seu nome noite
adentro.

O Artemision inteiro virou um amontoado vermelho no centro das trevas. Os
guardas agarraram o criminoso. Amordaçaram-no para que parasse de gritar o
próprio nome. Foi jogado nos porões, amarrado, durante o incêndio.

Artaxerxes despachou, no ato, ordem para torturá-lo. Ele só aceitou
confessar o já mencionado. As doze cidades da Jônia proibiram, sob pena de
morte, que se revelasse o nome de Heróstrato às idades por vir. Os
rumores, porém, fizeram-no chegar até nós. Na mesma noite em que
Heróstrato incendiava o templo de Éfeso, nascia Alexandre, rei da
Macedônia.

\chapter{Crates cínico}

Nasceu em Tebas, foi discípulo de Diógenes e também conheceu Alexandre. Seu
pai, Ascondas, era rico e deixou-lhe duzentos talentos. Certo dia, ao
assistir uma tragédia de Eurípedes, sentiu-se inspirado com a aparição de
Télefo, rei da Mísia, vestindo molambos de mendigo e levando uma cesta na
mão. Levantou-se no teatro e anunciou com voz forte que distribuiria a
quem quisesse os duzentos talentos de sua herança, e que doravante lhe
bastariam as roupas de Télefo. Os tebanos puseram-se a rir e se amontoaram
em frente à sua casa; ele, contudo, ria mais do que eles. Jogou-lhes seu
dinheiro e seus móveis pelas janelas, apanhou um manto de linho e um
alforje, e então se foi.

Chegando em Atenas, perambulou pelas ruas e descansou recostado nas
muralhas, em meio aos excrementos. Pôs em prática tudo aquilo que Diógenes
lhe aconselhava. O barril lhe pareceu supérfluo.\footnote{ Conta-se que
Diógenes, filósofo grego do século~\versal{III}~a.C. que fizera voto de pobreza,
dormia num barril. O barril, porém, é uma invenção gaulesa, e o mais certo
é Diógenes ter usado um vasilhame de barro para abrigar seu sono das intempéries.} 
Na opinião de Crates, o homem não era nenhum
caracol, nem um bernardo-eremita. Vivia inteiramente nu em meio ao lixo, e
juntava cascas de pão, azeitonas podres e espinhas de peixe seco para
encher seu alforje. Dizia ele que o alforje era uma cidade ampla e
opulenta onde não se via parasitas ou cortesãs, e que produzia o
suficiente em tomilho, alho, figos e pão para o seu rei. Assim, Crates
levava sua pátria nas costas e dela se alimentava.

Não se envolvia nos assuntos públicos, nem sequer para escarnecer, e não
procurava insultar os reis. Não aprovava aquele dito de Diógenes que,
certo dia, depois de exclamar: “Homens, aproximem-se!”, bateu com o bordão
naqueles que vieram, dizendo: “Eu chamei homens, não excrementos”. Crates
foi brando com os homens. Não fazia caso de nada. As chagas lhe eram
familiares. Seu grande pesar era não ter um corpo flexível o bastante para
poder lambê-las, como fazem os cães. Também deplorava a necessidade de
consumir alimentos sólidos e beber água. Achava que o homem deveria
bastar-se a si mesmo, sem qualquer auxílio externo. Pelo menos não buscava
água para se lavar. Contentava-se em esfregar o corpo nas muralhas quando
a sujeira o perturbava, tendo observado que assim faziam os asnos.
Raramente falava nos deuses, e não se preocupava com eles: pouco se lhe
dava que houvesse deuses ou não, e sabia que não podiam fazer nada por
ele. Acusava-os, aliás, de terem propositalmente tornado os homens
infelizes, ao voltarem o rosto deles para o céu e privando-os da faculdade
que tem a maioria dos animais, que andam de quatro. Já que os deuses
decretaram que é preciso comer para viver, pensava Crates, deveriam voltar
o rosto dos homens para a terra, onde crescem as raízes: não temos como
nos saciar de ar ou de estrelas.

A vida não lhe foi generosa. Contraiu remela, de tanto expor os olhos à
acre poeira do Ático. Uma doença de pele desconhecida cobriu-o de tumores.
Coçou-se com as unhas que nunca aparava e reparou que tirava daí um
benefício dobrado, desgastando-as e, ao mesmo tempo, sentindo alívio. Seus
longos cabelos ficaram como um feltro espesso, e ele os dispôs no topo da
cabeça para proteger-se da chuva e do sol.

Quando Alexandre veio ter com ele, não dirigiu-lhe palavras mordazes,
considerando-o entre os demais espectadores sem fazer distinção entre o
rei e a multidão. Crates não nutria nenhuma opinião sobre os poderosos.
Importavam-lhe tão pouco como lhe importavam os deuses. Somente com os
homens se ocupava, e com a maneira de passar a existência com a maior
simplicidade possível. As objurgações de Diógenes o faziam rir, assim como
sua pretensão de reformar os costumes. Crates julgava-se infinitamente
acima de cuidados tão vulgares. Modificando a máxima inscrita no frontão
do templo de Delfos, dizia: “Vive a ti mesmo”. A ideia de qualquer
conhecimento parecia-lhe absurda. Estudava somente as relações de seu
corpo com o que lhe era necessário, tratando de reduzi-las tanto quanto
possível. Diógenes mordia como os cães, mas Crates vivia como os cães.

Teve um discípulo cujo nome era Metrocles. Era um jovem rico de Maroneia.
Sua irmã, Hipárquia, bela e nobre, enamorou-se de Crates. Consta que foi
apaixonada por ele e foi ao seu encontro. O fato parece impossível, mas é
certo. Nada a dissuadiu, nem a sujeira do cínico, nem sua pobreza
absoluta, nem o horror de sua vida pública. Ele alertou-a de que vivia à
maneira dos cães, pelas ruas, e catava ossos nos montes de lixo.
Preveniu-a de que nada em sua vida em comum seria ocultado e que a
possuiria publicamente, quando lhe desse vontade, como fazem os cães com
as cadelas. Hipárquia imaginava tudo isso. Seus pais tentaram detê-la: ela
ameaçou se matar. Eles tiveram pena. Ela então deixou o burgo de Maroneia,
toda nua, cabelos soltos, coberta tão somente por um pano velho, e viveu
com Crates, vestida de igual maneira. Dizem que ele dela teve um filho,
Pasicles; mas não há nada certo a esse respeito.

Esta Hipárquia foi, ao que dizem, boa para com os pobres, e compadecida;
afagava os doentes com as mãos; lambia sem repulsa alguma as feridas
sangrentas dos que sofriam, persuadida de que eram para ela o que as
ovelhas são para as ovelhas, o que os cães são para os cães. Se fazia
frio, Crates e Hipárquia dormiam bem junto dos pobres, e tratavam de lhes
comunicar o calor de seus corpos. Prestavam-lhes o auxílio silente que os
animais prestam uns aos outros. Não tinham qualquer preferência por
ninguém que deles se acercasse. Bastava-lhes que fossem homens.

Eis tudo o que chegou até nós acerca da mulher de Crates; não sabemos
quando, nem como morreu. Seu irmão Metrocles admirava Crates e o imitou.
Mais não tinha tranquilidade. Sua saúde era perturbada por contínuas
flatulências que ele não conseguia conter. Entrou em desespero e resolveu
morrer. Crates soube de seu tormento, e quis consolá-lo. Comeu uma medida
de tremoços e foi se encontrar com Metrocles. Perguntou-lhe se era a
vergonha de sua enfermidade que o afligia a tal ponto. Metrocles confessou
que não podia suportar esta desgraça. Então Crates, todo inchado de
tremoços, soltou gases na presença de seu discípulo, e afirmou-lhe que a
natureza submetia todos os homens ao mesmo mal. Censurou-o então por ter
tido vergonha dos outros e lhe ofereceu seu próprio exemplo. Então soltou
mais uns gases, tomou Metrocles pela mão e o levou consigo.

Viveram assim juntos os dois por muito tempo nas ruas de Atenas, sem dúvida,
com Hipárquia. Falavam-se muito pouco. Não tinham vergonha de nada. Embora
remexessem nos mesmos montes de lixo, os cães pareciam respeitá-los. Cabe
imaginar que, tivessem sido premidos pela fome, teriam brigado a dentadas.
Os biógrafos, porém, não relataram nada do gênero. Sabemos que Crates
morreu velho; que por fim foi ficando sempre no mesmo lugar, deitado no
alpendre de um armazém de Pireu em que os marinheiros guardavam as cargas
do porto; que parou de perambular em busca de carne para roer, não quis
nem mais estender o braço, e que foi encontrado, certo dia, ressecado pela
fome.

\chapter{Septima encantatriz}

Septima foi escrava debaixo do sol africano, na cidade de Adrumeto. E sua
mãe, Amoena, foi escrava, e a mãe dela foi escrava, e todas foram belas e
obscuras, e os deuses infernais lhes revelaram os filtros de amor e de
morte. A cidade de Adrumeto era branca e as pedras da casa em que Septima
vivia eram de um rosa trêmulo. E a areia da praia era juncada de conchas
roladas pelo mar morno desde a terra do Egito, ali onde as sete bocas do
Nilo vertem sete vasos de cores variadas. Na casa litorânea em que Septima
vivia, ouvia-se morrer a ondulação prateada do Mediterrâneo e, a seus pés,
um leque de linhas azuis reluzentes se desdobrava até rente ao céu. As
palmas das mãos de Septima eram avermelhadas de ouro, e a extremidade de
seus dedos era maquiada; seus lábios cheiravam a mirra e suas pálpebras
untadas estremeciam suavemente. Assim andava ela pela estrada dos
subúrbios, levando à casa dos servos uma cesta de pães macios.

Septima enamorou-se de um jovem livre, Sextílio, filho de Dionísia. Mas não
é permitido serem amadas aquelas que conhecem os mistérios subterrâneos:
pois são submetidas ao adversário do amor, que se chama Anteros. E assim
como Eros dirige o brilho dos olhos e afia a ponta das flechas, Anteros
desvia os olhares e atenua a agrura das setas. É um deus benfazejo que
reside entre os mortos. Não é cruel, como o outro. Possui o nepentes que
traz o esquecimento. E, sabendo que o amor é a pior das dores terrestres,
odeia e cura o amor. É impotente, contudo, para expulsar Eros de um
coração ocupado. Toma, então, o outro coração. Assim luta Anteros contra
Eros. Eis por que Sextílio não pôde amar Septima. Tão logo Eros levou sua
tocha ao seio da iniciada, Anteros, irritado, apoderou-se daquele que ela
queria amar.

Septima conheceu o poder de Anteros nos olhos abaixados de Sextílio. E
quando o tremor purpúreo se apossou do ar da noite, ela saiu pelo caminho
que vai de Adrumeto até o mar. É um caminho sossegado em que os namorados
tomam vinho de tâmara, recostados nas muralhas polidas dos túmulos. A
brisa oriental sopra seu aroma sobre a necrópole. A jovem lua, velada
ainda, ali vem vagar, incerta. Muitos mortos embalsamados se quedam ao
redor de Adrumeto em suas sepulturas. E ali dormia Foinissa, irmã de
Septima, escrava como ela, que morrera aos dezesseis anos antes que homem
algum tivesse inspirado seu cheiro. A tumba de Foinissa era estreita como
seu corpo. A pedra abraçava seus seios cingidos com tiras. Bem junto de
sua fronte baixa uma comprida laje detinha seu olhar vazio. Seus lábios
enegrecidos ainda exalavam o vapor dos aromas com que a tinham embebido.
Em sua mão recatada brilhava um anel de ouro verde incrustado com dois
rubis turvos e pálidos. Meditava eternamente, em seu sonho estéril, sobre
as coisas que não conhecera.

Sob a virgem alvura da lua nova, Septima deitou-se junto à tumba estreita
da irmã, sobre a terra boa. Chorou e esfregou o rosto na grinalda
esculpida. E aproximou a boca do conduto pelo qual se vertem as libações,
e sua paixão se exalou:

--- Ó minha irmã --- disse ela ---, sai do teu sono para me escutar. A lamparina que \label{irma}
aclara as primeiras horas dos mortos se extinguiu. Deixaste escapar de
tuas mãos a ampola colorida de vidro que te havíamos dado. O fio do teu
colar se rompeu e os grãos dourados se espalharam em volta do teu pescoço.
Mais nada do que é nosso é teu e aquele que tem um gavião na
cabeça\footnote{ Referência a Seker, deus egípcio da luz e protetor dos mortos.} 
agora te possui. Escuta-me, pois tens o poder de levar
minhas palavras. Vai até a cela, sabes qual, e roga a Anteros. Roga à
deusa Hathor.\footnote{ Deusa egípcia das mulheres, do amor e da alegria.} 
Roga àquele cujo cadáver decepado o mar levou até Biblos dentro
de um baú.\footnote{ Referência à história de Osíris, deus egípcio da
vegetação e da vida além-morte.} Tem pena, minha irmã, de uma dor
desconhecida. Pelas sete estrelas dos magos da Caldeia, eu te conjuro.
Pelos poderes infernais que se invocam em Cartago, Iaô, Abriaô, Salbaal,
Bathbaal, recebe minha encantação. Faze com que Sextílio, filho de
Dionísia, se consuma de amor por mim, Septima, filha de nossa mãe Amoena.
Que ele arda na noite; que me procure junto à tua tumba, ó Foinissa! Ou
então, poderosa, leva-nos aos dois para a morada das trevas. Roga a
Anteros que resfrie nosso sopro, já que ele proíbe a Eros de acendê-lo.
Perfumada morta, acolhe a libação de minha voz. \textit{Achrammachalala!}

A virgem envolta em faixas soergueu-se em seguida e penetrou dentro da
terra, dentes à mostra.

E Septima, envergonhada, correu por entre os sarcófagos. Até a segunda
vigília, permaneceu na companhia dos mortos. Espreitou a lua fugidia.
Ofereceu o colo à mordida salgada do vento marinho. Acariciaram-na as
primeiras douraduras do dia. Depois voltou para Adrumeto, e sua longa
camisola azul flutuava atrás dela.

Entretanto, Foinissa, tesa, vagava nos circuitos infernais. E aquele que tem
um gavião na cabeça não acolheu seu lamento. E a deusa Hathor permaneceu
deitada em sua peanha pintada. E Foinissa não pôde encontrar Anteros, pois
não conhecia o desejo. Em seu coração fenecido, porém, sentiu a piedade
que os mortos têm pelos vivos. Então, na segunda noite, na hora em que os
cadáveres se soltam para cumprir as encantações, ela moveu seus pés atados
pelas ruas de Adrumeto.

Sextílio estremecia ritmicamente com os suspiros do sono, o rosto voltado
para o teto de seu quarto, sulcado de losangos. E Foinissa, morta, envolta \label{viscera}
em faixas fragrantes, sentou-se junto dele. E ela não tinha cérebro nem
vísceras; mas seu coração ressecado fora recolocado em seu peito. E nesse
momento Eros lutou contra Anteros, e se apoderou do coração embalsamado de
Foinissa. Ela imediatamente desejou o corpo de Sextílio, que ele estivesse
deitado entre ela e sua irmã Septima na casa tenebrosa.

Foinissa pôs os lábios pintados sobre a boca viva de Sextílio, e a vida se
esvaiu dele como bolha. Ela em seguida foi até a cela de escrava de
Septima, e tomou-a pela mão. E Septima, adormecida, cedeu à mão da irmã. E
o beijo de Foinissa e o abraço de Foinissa fizeram morrer, quase à mesma
hora da noite, Septima e Sextílio. Tal foi o fúnebre desfecho da luta
entre Eros e Anteros; e os poderes infernais acolheram, ao mesmo tempo,
uma escrava e um homem livre.

Sextílio jaz na necrópole de Adrumeto, entre Septima, a encantadora, e
Foinissa, sua irmã virgem. O texto da encantação está inscrito na placa de
chumbo, enrolada e perfurada com um prego, que a encantatriz inseriu no
conduto das libações da tumba de sua irmã.

\chapter{Lucrécio poeta}

Lucrécio surgiu numa grande família que havia se retirado para longe da
vida civil. Seus primeiros dias receberam a sombra do pórtico preto de uma
casa alta erguida na montanha. O átrio era severo, e os escravos,
silentes. Foi rodeado, desde a infância, de desprezo pela política e pelos
homens. O nobre Memio, que tinha a sua idade, sujeitou-se, na floresta,
aos jogos que Lucrécio lhe impôs. Juntos se surpreenderam com as rugas das
velhas árvores e espreitaram o estremecer das folhas ao sol, como um véu
víride de luz juncado de manchas douradas. Observaram seguidamente os
lombos raiados dos leitões selvagens que farejavam o solo. Atravessaram
jorros frementes de abelhas e bandos moventes de formigas em marcha. E
chegaram um dia, ao desembocar de uma mata, numa clareira cercada de
antigos sobreiros, tão estreitamente assentados que seu círculo escavava
no céu um poço de azul. O repouso daquele refúgio era infinito. Tinha-se a
impressão de estar numa ampla estrada clara seguindo para o alto do ar
divino. Lucrécio ali foi tocado pela benção dos espaços calmos.

Com Memio, deixou o templo sereno da mata e foi para Roma estudar
eloquência. O antigo fidalgo que governava a casa alta lhe deu um
professor grego e ordenou-lhe que só voltasse depois de dominar a arte de
desprezar as ações humanas. Lucrécio não tornou a vê-lo. Morreu solitário,
execrando o tumulto da sociedade. Ao regressar, Lucrécio trouxe para a
casa alta vazia, para o átrio severo e os escravos silentes, uma mulher
africana, bela, bárbara e perversa. Memio retornara para a casa de seus
pais. Lucrécio vira as facções sangrentas, as guerras partidárias e a
corrupção política. Estava apaixonado.

E de início sua vida foi encantada. Nas tapeçarias das muralhas, a mulher
africana apoiava o volume cacheado de sua cabeleira. Seu corpo todo
esposava com vagar os leitos de repouso. Ela cingia as crateras cheias de
vinho espumante com seus braços carregados de esmeraldas translúcidas.
Tinha um estranho jeito de levantar um dedo e balançar a fronte. Seus
sorrisos vinham de uma fonte funda e sombria como os rios da África. Ao
invés de fiar a lã, esmiuçava-a pacientemente em flocos pequenos que
esvoaçavam à sua volta.

Lucrécio desejava ardentemente fundir-se àquele belo corpo. Abraçava seus
seios metálicos e prendia a boca em seus lábios de um roxo escuro. As
palavras de amor passaram de um para o outro, foram suspiradas,
fizeram-nos rir, e se gastaram. Eles tocaram o véu opaco e flexível que
separa os amantes. A volúpia ganhou fúria e desejou mudar de pessoa.
Chegou ao ponto de agudez extrema em que se espalha ao redor da carne sem
penetrar nas entranhas. A africana se retraiu em seu coração estrangeiro.
Lucrécio desesperou-se por não poder realizar o amor. A mulher tornou-se
altiva, triste e silenciosa, igual ao átrio e aos escravos. Lucrécio vagou
pela sala dos livros.

Foi então que ele abriu o rolo em que um escriba havia copiado o tratado de
Epicuro.

Compreendeu de imediato a variedade das coisas deste mundo, e o quanto é
vão se esforçar rumo às ideias. O universo pareceu se assemelhar aos
floquinhos de lã que os dedos da africana espalhavam pelas salas. Os
cachos de abelhas e as colunas de formigas e o tecido movente das folhas
passavam a ser agrupamentos de agrupamentos de átomos. E em seu corpo
inteiro sentiu um povo invisível e discorde, ávido por separar-se. E os
olhares pareciam raios mais sutilmente carnudos, e a imagem da bela
bárbara, um mosaico colorido e harmonioso, e ele percebeu que era triste e
vão o fim do movimento desta infinidade toda. Contemplou, igual às facções
sangrentas de Roma com suas tropas de clientes armados e insultuosos, o
remoinho de tropas de átomos tintos com o mesmo sangue e disputando entre
si obscura supremacia. E viu que a dissolução da morte não era senão a
alforria desta turba turbulenta disparando rumo a mil outros movimentos
vãos.

Ora, tão logo se instruiu assim por meio do rolo de papiro, em que se
entreteciam as palavras gregas com os átomos do mundo, Lucrécio saiu para
a floresta transpondo o pórtico preto da casa alta dos ancestrais. E
avistou o lombo dos leitões raiados, que continuavam com o focinho
apontado para o chão. Depois, atravessou o mato e se viu de súbito no meio
do templo sereno da floresta, e seus olhos mergulharam no poço azul do
céu. Foi ali que ele colocou seu repouso.

Dali contemplou a imensidão fervilhante do universo; todas as pedras,
plantas, árvores, animais, todos os homens, com suas cores, paixões,
instrumentos, e a história daquelas coisas diversas, e seu nascimento, e
suas doenças, e sua morte. E em meio à morte total e necessária, ele
vislumbrou claramente a morte única da africana, e chorou.

Sabia que o choro tem origem num movimento específico das pequenas
glândulas que ficam sob as pálpebras e são agitadas por um cortejo de
átomos vindos do coração, quando este mesmo coração é tocado, por sua vez,
pela sucessão de imagens coloridas que se desprendem do corpo de uma
mulher amada. Sabia que o amor é causado pela simples expansão de átomos
que querem se unir a outros átomos. Sabia que a tristeza causada pela
morte não passa da pior das ilusões terrestres, pois a morta deixara de
ser infeliz e sofrer, enquanto aquele que a pranteava se afligia com os
próprios males e pensava sombriamente na própria morte. Sabia que não
resta de nós nenhum duplo simulacro para verter lágrimas sobre o próprio
cadáver estendido aos seus pés. Mas, conhecendo perfeitamente a tristeza e
o amor e a morte, e sabendo que são imagens vãs quando as contemplamos
desde o calmo espaço em que é preciso encerrar-se, seguiu chorando, e
desejando o amor, e temendo a morte.

Eis por que, ao regressar à casa alta e sombria dos ancestrais,
aproximou-se da bela africana, que cozinhava uma beberagem num pote de
metal sobre um braseiro. Pois ela, por sua vez, refletira, e seus
pensamentos haviam remontado à misteriosa fonte de seu sorriso. Lucrécio
contemplou a beberagem ainda fervente. Ela foi clareando aos poucos e
ficando igual a um céu turvo e verde. E a bela africana balançou a fronte
e ergueu um dedo. Então Lucrécio bebeu do filtro. E em seguida sua razão
sumiu, e ele olvidou as palavras gregas do rolo de papiro. E pela primeira
vez, estando louco, conheceu o amor; e durante a noite, por ter sido
envenenado, conheceu a morte.

\chapter{Clódia matrona impudica}

Era filha de Ápio Claudio Pulcro, cônsul. Ainda em tenra idade,
distinguiu-se dos irmãos pelo brilho flagrante de seus olhos. Tércia, sua
irmã mais velha, se casou cedo; a mais moça cedeu por inteiro a todos os
seus caprichos. Os irmãos, Ápio e Caio, já eram ávidos pelas rãs de couro
e os carros de nozes que confeccionavam para eles; mais tarde, foram
ávidos por sestércios. Mas Clódio, belo e feminino, foi companheiro das
irmãs. Clódia as persuadia, com seus olhares ardentes, a vesti-lo com uma
túnica de mangas, cobri-lo com um bonezinho de fios de ouro e atar-lhe no
peito um cinto flexível; então elas cobriam-no com um véu da cor do fogo e
o levavam até os quartinhos onde ele se deitava na cama com elas três. Foi
Clódia a sua preferida, mas ele tomou também a virgindade de Tércia e da
caçula.

Tinha Clódia dezoito anos, quando seu pai morreu. Ela permaneceu na casa do
Monte Palatino. Ápio, seu irmão, governava a propriedade, e Caio se
preparava para a vida pública. Clódio, ainda imberbe e delicado, dormia
entre as irmãs, que tinham ambas por nome Clódia. Começaram, às
escondidas, a ir aos banhos com ele. Davam um quarto de asse\footnote{
Unidade monetária romana, anterior ao sestércio.} aos escravos
fortes que as massageavam, depois mandavam que devolvessem a moeda.
Clódio, em sua presença, era tratado como as irmãs. Tais foram os prazeres
deles antes do casamento.

A mais moça desposou Lúculo, que a levou para a Ásia, onde guerreava contra
Mitrídates. Clódia tomou por marido seu primo Metelo, tosco homem de bem.
Naqueles tempos de tumultos, foi um espírito conservador e estreito.
Clódia não suportava sua brutalidade rústica. Já sonhava coisas novas para
seu amado Clódio. César começava a tomar conta dos espíritos; Clódia achou
que era preciso suprimi-lo. Mandou buscar Cícero por intermédio de
Pompônio Ático. Era uma companhia zombeteira e galante. Junto dela
encontravam-se Licínio Calvo, o jovem Curio, alcunhado “a menina”, Sexto
Clódio, que era seu moço de recados, Inácio e seu bando, Catulo de Verona
e Célio Rufo, que era apaixonado por ela. Metelo, pesadamente sentado, não
dizia palavra. Comentavam os escândalos acerca de César e Mamurra. Então
Metelo, nomeado procônsul, se foi para a Gália cisalpina. Ficou Clódia
sozinha em Roma com sua cunhada Múcia. Cícero foi totalmente enfeitiçado
por seus grandes olhos cintilantes. Refletiu que podia repudiar Terência,
sua mulher, e supôs que Clódia deixaria Metelo. Mas Terência descobriu
tudo e aterrorizou o marido. Cícero, medroso, renunciou aos seus desejos.
Terência exigiu mais, e teve Cícero de romper com Clódio.

O irmão de Clódia, entretanto, se ocupava. Fazia amor com Pompeia, mulher   \EP[1]%
de César. Na noite da festa da Boa Deusa, só haveria mulheres na casa de
César, que era pretor. Somente Pompeia ofereceria o sacrifício. Clódio
vestiu-se, como sua irmã costumara fantasiá-lo, de tocadora de cítara, e
entrou em casa de Pompeia. Foi reconhecido por uma escrava. A mãe de
Pompeia deu o alarme e o escândalo virou público. Clódio tentou se
defender e jurou que estava, naquela hora, em casa de Cícero. Terência
obrigou o marido a negar: Cícero depôs contra Clódio.

Clódio, com isso, estava perdido para o partido nobre. Sua irmã
recentemente passara dos trinta. Estava mais ardorosa que nunca. Teve a
ideia de fazer com que Clódio fosse adotado por um plebeu, de modo que
pudesse se tornar tribuno do povo. Metelo, que estava de volta, adivinhou
seus planos e escarneceu. Nesse momento, já não tendo Clódio em seus
braços, deixou-se amar por Catulo. O marido Metelo lhes parecia
detestável. Sua mulher resolveu se livrar dele. Certo dia, quando ele
voltou do Senado, cansado, ela lhe deu de beber. Metelo caiu morto no
átrio. Clódia agora estava livre. Deixou a casa do marido e foi, ligeira,
enclausurar-se com Clódio no Monte Palatino. Sua irmã fugiu da casa de
Lúculo e voltou para junto deles. Retomaram sua vida a três e exerceram
seu ódio.

Clódio, agora plebeu, logo foi designado tribuno do povo. A despeito de sua
graça feminina, tinha voz forte e mordaz. Conseguiu que Cícero fosse
exilado; mandou derrubar sua casa diante de seus próprios olhos, e jurou
de ruína e morte todos os seus amigos. César era procônsul na Gália e nada
podia. Cícero, porém, angariou influências através de Pompeia e foi
chamado de volta no ano seguinte. A fúria do jovem tribuno foi extrema.
Atacou violentamente Milo, amigo de Cícero, que começava a disputar o
consulado. Emboscado na noite, tentou matá-lo, derrubando seus escravos
que traziam as tochas. A popularidade de Clódio decrescia. Quadrinhas
obscenas eram cantadas sobre Clódio e Clódia. Cícero os denunciou num
discurso violento: Clódia nele era tratada de Medeia e Climenestra. A
raiva dos dois irmãos acabou por explodir. Clódio tentou incendiar a casa
de Milo, e foi abatido nas trevas por escravos guardiões.

Clódia então se desesperou. Conquistara e rejeitara Catulo, depois Célio
Rufo, depois Inácio, cujos amigos a tinham levado para as baixas tabernas:
mas só amara a seu irmão Clódio. Por ele é que envenenara o marido. Por
ele é que atraíra e seduzira bandos de incendiários. Com sua morte, o
objeto de sua vida lhe veio a faltar. Ainda era bela e ardente. Tinha uma
casa de campo na estrada de Óstia, jardins junto ao Tibre e em Baia. Lá se
refugiou. Procurou distrair-se dançando lascivamente com mulheres. Não foi
suficiente. Sua mente estava ocupada pelas luxúrias de Clódio, que ela
ainda via infantil e imberbe. Lembrava-se de como ele caíra, outrora, em
mãos de piratas da Cilícia, que tinham usado seu corpo tenro. Uma certa
taberna também lhe voltava à memória, onde estivera com ele. O frontão da
porta era todo manchado de carvão, e os homens que ali bebiam espalhavam
um cheiro forte, e tinham o peito peludo.

Roma então voltou a atraí-la. Vagou, nas primeiras vigílias, pelas
encruzilhadas e becos estreitos. Ainda era igual a insolência cintilante
dos seus olhos. Nada podia extingui-la, e ela tentou de tudo, inclusive se
expor à chuva e deitar-se na lama. Foi desde os banhos até as celas de
pedra; conheceu os porões onde os escravos jogavam dados, os subsolos onde
se embriagavam cozinheiros e carreteiros. Esperou por transeuntes nas ruas
ladrilhadas. Pereceu ao amanhecer de uma noite sufocante pelo estranho
revide de um antigo hábito seu. Um pisoeiro lhe pagara um quarto de asse;
a fim de retomá-lo, espreitou-a na alameda ao crepúsculo da aurora, e
estrangulou-a. Então jogou seu cadáver, com os olhos arregalados, na água
amarela do Tibre.

\chapter{Petrônio Romancista}

Nasceu num tempo em que saltimbancos vestindo trajes verdes faziam leitões
treinados saltarem círculos de fogo, em que porteiros barbudos, de túnicas
cor de cereja, debulhavam ervilhas em travessas de prata diante dos
elegantes mosaicos à entrada das \textit{vilas}, em que os alforriados,
cheios de sestércios, disputavam cargos municipais nas cidades de
província, em que os cantadores entoavam poemas épicos à sobremesa, em que
a linguagem era recheada de palavras do ergástulo e redundâncias infladas
vindas da Ásia.

Sua infância transcorreu em meio a tais elegâncias. Não usava duas vezes
uma roupa de lã de Tiro. A prataria caída no átrio era varrida junto com
os detritos. As refeições se compunham de coisas delicadas e inesperadas,
e os cozinheiros variavam sem cessar a apresentação da comida. Que ninguém se admirasse se, 
ao abrir um ovo, encontrasse dentro dele uma toutinegra, nem
temesse partir uma estatueta, imitação de Praxiteles, esculpida em
\textit{foie gras}. A gipsita que selava as ânforas era diligentemente
dourada. Caixinhas de marfim indiano encerravam perfumes ardentes
destinados aos convivas. Os gomis eram furados de maneiras diversas e
cheios de águas coloridas que surpreendiam ao jorrar. Todas as peças de
vidro figuravam monstruosidades irisadas. As alças de algumas urnas,
quando pegas, rompiam-se entre os dedos e os lados eclodiam derramando
flores artificialmente pintadas. Pássaros africanos de face escarlate
parolavam em gaiolas de ouro. Por detrás dos gradeados incrustados nas
ricas paredes das muralhas, urravam muitos macacos do Egito com cara de
cão. Em preciosos receptáculos rastejavam animais delgados com rutilantes
escamas flexíveis e olhos raiados de anil.

Assim viveu Petrônio, mansamente, pensando que o próprio ar que aspirava
era perfumado para o seu uso. Quando chegou à adolescência, depois de
guardar sua primeira barba num estojo ornamentado, começou a olhar em
redor. Um escravo de nome Siro, que havia servido na arena, mostrou-lhe
coisas desconhecidas. Petrônio era pequeno, escuro e vesgo de um olho. Não
era de raça nobre. Tinha mãos de artesão e um espírito cultivado. Daí~o
prazer que sentia em burilar e traçar as palavras. Elas em nada se
pareciam com o que haviam imaginado os poetas antigos. Pois tratavam de
imitar tudo o que cercava Petrônio. E foi só mais tarde que ele teve a
imprópria ambição de compor versos.

De modo que conheceu bárbaros gladiadores e fanfarrões de encruzilhada,
homens de olhar oblíquo que parecem estar vigiando os legumes e
desengancham peças de carne, meninos de cabelo crespo levados a passear
por senadores, velhos tagarelas discorrendo nas esquinas sobre assuntos da
cidade, criados lascivos e mulheres arrivistas, vendedoras de frutas e
donos de estalagens, poetas deploráveis e criadas tratantes, sacerdotisas
suspeitas e soldados errantes. Fitava neles seu olhar vesgo e captava com
precisão seus modos e intrigas. Siro o levou aos banhos de escravos, às
alcovas de prostitutas e aos redutos subterrâneos onde figurantes do circo
se exercitavam com espadas de pau. Nas portas da cidade, entre as tumbas,
contou-lhe os casos dos homens que mudam de pele, casos que os negros, os
sírios, os taverneiros e os soldados guardiões das cruzes de suplício se
transmitiam de boca em boca.

Por volta dos trinta anos, Petrônio, ávido daquela liberdade diversa,
começou a escrever a história de escravos errantes e devassos. Reconheceu
seus costumes em meio às transformações do luxo; reconheceu suas ideias e
linguagem em meio às conversas polidas dos banquetes. Solitário, diante de
seu pergaminho, apoiado à mesa perfumada de madeira de cedro, desenhou com
a ponta de seu cálamo as aventuras de um populacho ignorado. À luz de suas
altas janelas, sob as pinturas dos lambris, imaginou as tochas fumegantes
das estalagens, e ridículos combates noturnos, molinetes com lampadários
de pau, fechaduras arrombadas a machadadas por escravos da justiça,
enxergas gordurosas apinhadas de percevejos, e reprimendas de fiscais de
bairro em ajuntamentos de pobre gente vestindo cortinas rasgadas e trapos
imundos.

Dizem que assim que concluiu os dezesseis livros de sua criação, mandou
buscar Siro, e os leu para ele, e que o escravo ria e se exclamava em voz
alta batendo palmas. Foi quando conceberam o projeto de pôr em execução as
aventuras escritas por Petrônio. Tácito relata erroneamente que ele foi
árbitro da elegância na corte de Nero, e que Tigelino, por ciúmes, mandou
ditar sua ordem de morte. Petrônio não desmaiou delicadamente numa
banheira de mármore, murmurando versinhos lascivos. Fugiu com Siro e
terminou a vida percorrendo as estradas.

Sua aparência facilitou-lhe o disfarce. Siro e Petrônio se revezaram para
carregar a pequena sacola de couro que continha seus trapos e denários.
Dormiram ao relento, junto a cruzes de sepulturas. Viram luzir tristemente
na noite as lamparinas dos monumentos fúnebres. Comeram pão azedo e
azeitonas murchas. Não se sabe se roubaram. Foram mágicos ambulantes,
charlatões do campo e companheiros de soldados vagabundos. Petrônio
desaprendeu por inteiro a arte de escrever tão logo viveu da vida que
imaginara. Tiveram jovens amigos traidores, a quem amaram, e que os
abandonaram às portas dos municípios levando até o seu último asse.
Praticaram toda devassidão com gladiadores fugidos. Foram barbeiros e
serventes dos banhos. Viveram durante meses de pães funerários que
furtavam nos sepulcros. Petrônio aterrorizava os viajantes com seu olho
baço e sua cor escura de aspecto maligno. Certa noite, desapareceu. Siro
pensou que o encontraria numa alcova imunda onde tinham conhecido uma
mulher de cabelos emaranhados. Mas um salteador bêbado lhe enfiara uma
larga lâmina no pescoço, quando estavam deitados juntos, em campo aberto,
sobre as lajes de um jazigo abandonado.

\chapter{Sufrah geomante}

Conta, por engano, a história de Aladim que o mágico africano foi
envenenado em seu palácio e seu corpo, enegrecido e gretado por força da
droga, jogado aos cães e aos gatos; verdade é que seu irmão, iludido por
aquela aparência, depois de vestir o traje de santa Fátima pediu para ser
apunhalado; o certo, porém, é que Mograbi Sufrah (pois era este o nome do
mágico) simplesmente adormeceu pela força do narcótico e escapuliu por uma
das vinte e quatro janelas do salão, enquanto Aladim beijava ternamente a
princesa.

Mal alcançara o chão, depois de comodamente deslizar por um dos canos de
ouro por onde escoava a água do grande terraço, o palácio sumiu e Sufrah
se viu sozinho em meio à areia do deserto. Não lhe restava sequer uma das
garrafas de vinho da África que fora buscar na adega a pedido da jovem
princesa. Desesperado, sentou-se sob o sol ardente e, sabendo que era
infinita a extensão de areia tórrida que o cercava, enrolou a cabeça no
manto e esperou a morte. Já não possuía nenhum talismã; não tinha perfume
algum para fazer sufumigações; nem mesmo uma vara movente a indicar-lhe
uma fonte profundamente oculta para saciar sua sede. Veio a noite, azul e
quente mas aliviando-lhe um pouco a inflamação dos olhos. Teve então a
ideia de traçar na areia uma figura de geomancia e perguntar se estava
fadado a perecer no deserto. Com os dedos marcou, compostas de pontos, as
quatro linhas maiores dispostas, à direita, sob a invocação do Fogo, da
Água, da Terra e do Ar, e à esquerda, do Sul, do Oriente, do Ocidente e do
Setentrião. E nas extremidades destas linhas, juntou os pontos pares e
ímpares, a fim de compor com eles a primeira figura. Para sua alegria viu
que se tratava da Fortuna Maior, donde se deduzia que escaparia ao perigo,
devendo a primeira figura ser colocada na primeira casa astrológica, que é
a casa daquele que pergunta. E, na casa chamada “Coração do Céu”, deparou
novamente com a figura da Fortuna Maior, o que lhe mostrou que triunfaria
e alcançaria a glória. Na oitava casa, porém, que é a casa da Morte, veio
se colocar a figura do Vermelho, que anuncia o sangue ou o fogo, o que é
um sinistro presságio. Depois de dispor as figuras nas doze casas,
escolheu entre elas duas testemunhas, sendo uma delas um juiz, a fim de
garantir a correção de seu cálculo. A figura do juiz era a da Prisão, e
soube assim que encontraria a glória, sob grande perigo, em algum lugar
fechado e secreto.

Certo de não morrer de imediato, Sufrah se pôs a refletir. Não tinha
esperança de reaver a lâmpada, transportada, junto com o palácio, para o
centro da China. Ponderou, no entanto, que nunca buscara saber quem era o
legítimo dono do talismã, o antigo possuidor do grande tesouro e do jardim
dos frutos preciosos. Uma segunda figura de geomancia, que decifrou
segundo as letras do alfabeto, revelou-lhe os caracteres S.L.M.N., que ele
traçou na areia, e a décima casa confirmou que era um rei o dono daqueles
caracteres. Sufrah percebeu de imediato que a lâmpada maravilhosa
integrara o tesouro do rei Salomão. Então, estudou com atenção todos os
signos, e a Cabeça do Dragão lhe indicou o que buscava --- pois que estava
unida pela Conjunção à Figura do Moço, que assinala as riquezas ocultas
sob a terra, e à da Prisão, na qual se lê a posição das abóbadas fechadas.

E Sufrah bateu palmas: pois a figura de geomancia mostrava que o corpo do
rei Salomão estava conservado naquela mesma terra da África, e que ele
ainda trazia no dedo seu todo-poderoso selo que dá imortalidade terrestre:
embora o rei devesse estar dormindo há miríades de anos. Alegre, Sufrah
esperou a aurora. Na semiclaridade anil, viu passar beduínos saqueadores
que, quando implorou, compadeceram-se de sua desdita, deram-lhe um pequeno
saco de tâmaras e uma cabaça cheia de água.

Sufrah pôs-se a caminho, rumo ao local indicado. Era um lugar árido e
pedregoso, entre quatro montanhas nuas erguidas feito dedos para os quatro
cantos do céu. Ali traçou um círculo e pronunciou certas palavras; e a
terra tremeu e se abriu, e revelou uma laje de mármore com um anel de
bronze. Sufrah pegou o anel e três vezes invocou o nome de Salomão. A laje
imediatamente se ergueu, e Sufrah desceu ao subterrâneo por uma escada
estreita.

Dois cães de fogo acorreram de seus nichos e cuspiram chamas entrecruzadas.
Mas Sufrah pronunciou a palavra mágica, e os cães rosnantes se esvaeceram.
Encontrou então uma porta de ferro que se abriu em silêncio assim que ele
a tocou. Seguiu por um corredor escavado no pórfiro. Candelabros de sete
braços queimavam uma luz eterna. Ao fundo do corredor havia uma sala
quadrada cujas paredes eram de jaspe. Em seu centro, um braseiro de ouro
irradiava um rico clarão. E sobre um leito feito de um só diamante
entalhado, e que lembrava um bloco de fogo frio, jazia uma forma velha, de
barba branca, a fronte cingida por uma coroa. Junto ao rei estava deitado
um gracioso corpo ressecado, com as mãos ainda estendidas para apertar as
suas; o calor dos beijos, porém, estava extinto. E, na mão pendente do rei
Salomão, Sufrah viu o brilho do grande selo.

Aproximou-se de joelhos e, rastejando até o leito, ergueu a mão enrugada,
fez deslizar o anel e o apanhou. 

Imediatamente cumpriu-se a obscura predição geomântica. O sono imortal do
rei Salomão se rompeu. Num segundo, seu corpo se esfacelou e se reduziu a
um pequeno punhado de ossos brancos e polidos que as delicadas mãos da
múmia ainda pareciam proteger. Mas Sufrah, aterrorizado pelo poder da
figura do Vermelho na casa da Morte, eructou num jorro carmesim todo o
sangue de sua vida e desabou na sonolência da imortalidade terrestre. Com
o selo do rei Salomão no dedo, deitou-se junto ao leito de diamante
preservado da corrupção por miríades de anos, no local fechado e secreto
que ele lera na figura da Prisão. A porta de ferro tornou a fechar-se
sobre o corredor de pórfiro e os cães de fogo se puseram a velar o
geomante imortal.

\chapter{Frate Dolcino herético}

Aprendeu a conhecer as coisas santas na igreja de Orto San Michele, onde
sua mãe o alçava para que ele pudesse tocar com as mãozinhas as belas
figuras de cera penduradas diante da Virgem Maria. A casa de seus pais era
contígua ao batistério. Três vezes por dia, ao amanhecer, ao meio-dia e ao
entardecer, via passar dois frades da ordem de são Francisco que
mendigavam pão e carregavam os pedaços num cesto. Não raro ele os seguia
até a porta do convento. Um deles era um monge muito velho: dizia ter sido
ordenado pelo próprio são Francisco. Prometeu ao menino ensinar-lhe a
falar com os pássaros e todos os pobres animais do campo. Não demorou e
Dolcino começou a passar os dias no convento. Cantava com os frades e sua
voz era pura. Quando o toque do sino chamava para descascar os legumes,
ajudava a limpar as ervas em torno da grande selha. O cozinheiro Roberto
lhe emprestava uma faca velha e deixava que esfregasse as tigelas com seu
toalhete. Dolcino gostava de contemplar, no refeitório, a cobertura da
lâmpada onde se viam, pintados, os doze apóstolos, com sandálias de
madeira nos pés e pequenos mantos cobrindo seus ombros.

Seu maior prazer, porém, era sair com os frades quando iam mendigar o pão
de porta em porta, e segurar o cesto coberto por um pano. Certo dia em que
andavam assim, na hora em que o sol ia alto no céu, negaram-lhes esmola em
várias casas baixas à margem do rio. O calor era forte: os frades sentiam
muita sede e muita fome. Entraram num pátio que não conheciam e Dolcino se
exclamou de surpresa ao largar o cesto no chão. Porque o pátio era
revestido de vinhas folhudas e cheio de um verde diáfano e deleitável;
saltavam por ali leopardos e vários animais de além-mar, e também se viam
moças e moços vestidos com tecidos brilhantes e tocando serenamente vielas
e cítaras. Ali a calma era profunda, a sombra espessa e perfumada. Todos
ouviam em silêncio os que cantavam e o canto era de um modo
extraordinário. Os frades nada disseram; sua fome e sede se saciaram; não
ousaram pedir nada. Só a muito custo se dispuseram a sair; mas da margem
do rio, ao olharem para trás, não viram abertura alguma na muralha.
Julgaram ter sido uma visão de necromancia, até o momento em que Dolcino
destapou o cesto. Estava repleto de pães brancos como se Jesus, com as
próprias mãos, tivesse ali multiplicado as oferendas.

Assim foi revelado a Dolcino o milagre da mendicância. Ele, porém, não
ingressou na ordem, já que de sua vocação recebera uma ideia mais alta e
singular. Os frades o levavam pelas estradas afora quando iam de um
convento para outro, de Bolonha para Modena, de Parma para Cremona, de
Pistoia para Lucca. E foi em Pisa que se sentiu impelido pela verdadeira
fé. Dormia na aresta de um muro do palácio episcopal, quando foi
despertado pelo som da buzina. Uma multidão de crianças levando ramos e
candeias acesas cercava, na praça, um homem selvagem que soprava uma
trombeta de bronze. Dolcino pensou estar vendo João Batista. O homem tinha
uma barba comprida e negra; vestia, do pescoço aos pés, um cilício escuro
marcado com ampla cruz vermelha; em volta do corpo tinha amarrada uma pele
de animal. Ele exclamou com voz terrível: \textit{Laudato et benedetto et
glorificato sia lo Patre}; e as crianças repetiram; então ele acrescentou:
\textit{sia lo Fijo}, e as crianças o imitaram; então ele acrescentou:
\textit{sia lo Spiritu Sancto}; e as crianças em seguida disseram o mesmo;
então ele cantou com elas: \textit{Alleluia, alleluia, alleluia!} Por fim,
tocou a trombeta e se pôs a pregar. Sua palavra era áspera qual vinho da
montanha --- mas atraiu Dolcino. Em todo lugar onde o frade de cilício tocou
a buzina, Dolcino foi admirá-lo, desejando sua vida. Era um ignorante
agitado de violência; não sabia latim; gritava, para conclamar à
penitência: \textit{Penitenzagite!} Mas anunciava sinistramente as
predições de Merlin, e da Sibila e do abade Joaquim, que constam no
\textit{Livro das figuras}; profetizava que o Anticristo chegara sob a
forma do imperador Frederico Barbaruiva, que sua ruína estava consumada, e
que depois dele em breve surgiriam as Sete Ordens, conforme interpretação
da Escritura. Dolcino o seguiu até Parma, onde foi inspirado a tudo
compreender.

O Anunciador precedia Aquele que estava por vir, o fundador da primeira das
Sete Ordens. Sobre a pedra erigida de Parma, de onde desde muitos anos os
podestades falavam ao povo, Dolcino proclamou a nova fé. Dizia ser
necessário vestir-se com manteletes de lona branca, como os apóstolos
pintados na cobertura da lâmpada, no refeitório dos Frades Menores.
Afirmava que ser batizado não era tudo; mas, a fim de voltar completamente
à inocência das crianças, fabricou para si um berço, mandou que o
envolvessem em cueiros e pediu o seio a uma mulher simples, que chorou de
piedade. A fim de pôr sua castidade à prova, rogou a uma burguesa que
convencesse a filha a dormir toda nua junto dele numa cama. Mendigou um
saco cheio de denários e os distribuiu aos pobres, ladrões e mulheres da
vida, declarando que não se deveria mais trabalhar, e sim viver à maneira
dos bichos do campo. Roberto, o cozinheiro do convento, fugiu para
segui-lo e alimentá-lo com uma tigela roubada aos pobres frades. As
pessoas devotas acreditaram que voltara o tempo dos Cavaleiros de Jesus
Cristo e Cavaleiros de Santa Maria, e daqueles que seguiam outrora,
errantes e arrebatados, Gerardino Secarelli. Agrupavam-se, beatos, em
torno de Dolcino e murmuravam: “Pai, pai, pai!”. Mas os Frades Menores
mandaram expulsá-lo de Parma. Uma jovem de nobre família, Margherita, saiu
em seu encalço pela porta que dava para a estrada de Piacenza. Ele a
cobriu com um saco marcado com uma cruz e levou-a consigo. Os porqueiros e
vaqueiros os observavam da orla dos campos. Muitos deixaram seus animais e
vieram ter com eles. Mulheres prisioneiras, que os homens de Cremona
tinham cruelmente mutilado cortando-lhes o nariz, imploraram e os
seguiram. Tinham o rosto envolto num pano branco; Margherita as instruiu.
Estabeleceram-se, todos, numa montanha arborizada, pouco distante de
Novare, e praticaram a vida comunitária. Dolcino não instituiu nenhuma
regra ou ordem, certo de que tal era a doutrina dos apóstolos, e de que
todas as coisas deviam se dar em caridade. Quem assim queria se alimentava
das bagas das árvores; outros mendigavam nas aldeias; outros roubavam
gado. A vida de Dolcino e Margherita foi livre debaixo do céu. Mas a gente
de Novare não quis compreender isto. Os camponeses queixavam-se dos roubos
e do escândalo. Mandaram buscar um bando de homens de armas para que
cercassem a montanha. Os Apóstolos foram escorraçados pelo povo. Quanto a
Dolcino e Margherita, foram amarrados sobre um burro, o rosto voltado para
a garupa, e levados até a praça principal de Novare. Lá foram queimados,
numa mesma fogueira, por ordem da justiça. Dolcino fez um único pedido:
que no suplício, em meio às chamas, os deixassem vestidos com seus
manteletes brancos, como os Apóstolos na cobertura da lâmpada.

\chapter{Cecco Angiolieri poeta rancoroso}

Cecco Angiolieri nasceu rancoroso em Siena, no mesmo dia que Dante
Alighieri em Florença. Seu pai, que enriquecera no comércio de lã,
alinhava-se com o império. Já desde a infância, Cecco invejava os
poderosos, desprezava-os, e resmungava discursos. Muitos nobres já não
queriam se submeter ao papa. Os gibelinos, entretanto, haviam cedido. Mas
entre os próprios guelfos, havia os Brancos e os Negros. Os Brancos não se
opunham à intervenção imperial. Os Negros permaneciam fiéis à Igreja, a
Roma, à Santa Sé. Cecco teve o impulso de ser Negro, talvez porque seu pai
fosse Branco.

Teve-lhe rancor quase desde o primeiro sopro. Aos quinze anos, reivindicou
sua parte da fortuna como se o velho Angiolieri já tivesse morrido.
Irritou-se com a recusa e deixou a casa paterna. Desde então nunca cessou
de queixar-se aos passantes e aos céus. Chegou em Florença pela estrada
principal. Lá os Brancos ainda reinavam, mesmo depois que os gibelinos
foram expulsos. Cecco mendigou seu pão, delatou a dureza do pai, e
alojou-se por fim no casebre de um sapateiro, o qual tinha uma filha.
Chamava-se Becchina e Cecco pensou que a amava.

O sapateiro era um homem simples, amigo da Virgem, cujas medalhas usava, e
convencido de que sua devoção lhe dava o direito de moldar os sapatos em
couro ruim. Conversava com Cecco sobre a santa teologia e a excelência da
graça, à luz de uma candeia de resina, antes da hora de ir deitar.
Becchina lavava a louça, e seus cabelos estavam sempre embaraçados.
Zombava de Cecco porque ele tinha a boca torta.

Por esta época, começou a se espalhar em Florença o boato sobre o amor
demasiado que sentira Dante degli Alighieri por Beatriz, filha de Folco
Ricovero de Portinari. Os que eram letrados sabiam de cor as canções que
ele lhe dedicara. Cecco ouviu-as recitar e as criticou com vigor.

--- Ó Cecco --- disse Becchina ---, tu que zombas deste Dante não saberias escrever
versos tão belos para mim.

--- Veremos --- disse Angiolieri escarnecendo.

Primeiro, compôs um soneto em que criticava a métrica e o sentido das
canções de Dante. Em seguida fez versos para Becchina, que não sabia
lê-los, e caía na gargalhada quando Cecco os declamava para ela, porque
não suportava as caretas apaixonadas de sua boca.

Cecco era pobre e desprovido feito uma pedra de igreja. Amava com furor a
mãe de Deus, o que tornava o sapateiro indulgente. Ambos frequentavam
alguns eclesiásticos miseráveis, a soldo dos Negros. Esperavam muito de
Cecco, que parecia iluminado, mas não havia dinheiro para lhe dar. Assim,
apesar de sua fé louvável, o sapateiro teve de casar Becchina com um gordo
vizinho, Barberino, vendedor de óleo. “E o óleo pode ser santo!”, disse
piedosamente o sapateiro, desculpando-se, a Cecco Angiolieri. O casamento
se realizou por volta da mesma época em que Beatriz desposou Simone de
Barde. Cecco imitou a dor de Dante.

Becchina, porém, não morreu. Em 9 de junho de 1291, Dante desenhava numa
tabuleta, e era o primeiro aniversário da morte de Beatriz. Aconteceu de
ele figurar um anjo cujo rosto se assemelhava ao rosto de sua
bem-amada. Onze dias depois, em 20 de junho, Cecco Angiolieri (estando
Barberino ocupado na feira de óleos) obteve de Becchina o favor de
beijá-la na boca, e compôs um ardente soneto. O rancor nem por isso se fez
menor em seu coração. Queria ouro junto com seu amor. Não logrou
consegui-lo com os usurários. Esperou obtê-lo de seu pai e partiu para
Siena. Mas o velho Angiolieri recusou ao filho até mesmo um copo de vinho
magro, e deixou-o sentado na estrada, em frente à casa.

Cecco avistara na sala um saco de florins recém-cunhados. Era a renda de
Arcidosso e Montegiovi. Estava morrendo de fome e sede; sua túnica estava
rasgada, e sua blusa, fuliginosa. Regressou, poeirento, a Florença, e
Barberino o enxotou de sua oficina, por causa de seus farrapos.

Cecco voltou, à noite, ao casebre do sapateiro, que encontrou cantando, à
fumaça de sua candeia, suave canção para Maria.

Os dois se abraçaram e choraram copiosamente. Depois do hino, Cecco contou
ao sapateiro do rancor terrível e desesperado que nutria pelo pai, um
velhote que ameaçava viver tanto quanto o Judeu-errante Botadeo. Um padre
que vinha entrando para conferir as necessidades do povo convenceu-o a
aguardar sua libertação em estado monástico. Levou Cecco até uma abadia,
onde lhe deram uma cela e uma batina velha. O prior lhe impôs o nome de
irmão Henrique. No coro, durante os cantos noturnos, passava a mão nas
lajes frias e despojadas como ele. A raiva lhe apertava a garganta quando
pensava na riqueza de seu pai; tinha a impressão de que o mar secaria
antes que o pai morresse. Sentia-se tão destituído que em certos momentos
até pensou em ser servente de cozinha. “Está aí uma coisa'', refletiu, ``a que
se pode aspirar.”

Em outros momentos, experimentou a loucura do orgulho: “Se eu fosse o fogo'',
pensava, ``incendiaria o mundo; se eu fosse o vento, sopraria um furacão
sobre ele; se eu fosse a água, o afogaria num dilúvio; se eu fosse Deus,
eu o afundaria no espaço; se eu fosse papa, não haveria mais paz sob o
sol; se eu fosse o Imperador, cortaria cabeças à toa; se eu fosse a Morte,
iria me encontrar com meu pai\ldots{} se eu fosse Cecco\ldots{} é esta a minha
esperança\ldots{}” Ele era, porém, \textit{frate Arrigo}. Em seguida retornou
ao seu rancor. Conseguiu um exemplar das canções para Beatriz e as
comparou pacientemente com os versos que escrevera para Becchina. Um monge
errante lhe contou que Dante se referia a ele com desdém. Procurou meios
de se vingar. A superioridade dos sonetos para Becchina lhe parecia
evidente. As canções para Bice (usava o seu nome vulgar) eram abstratas e
brancas; as suas eram cheias de força e de cor. Primeiro, enviou versos de
insulto a Dante; depois, cogitou denunciá-lo ao bom rei Carlos, conde de
Provença. Por fim, já que ninguém dava atenção aos seus poemas ou cartas,
quedou-se impotente. Cansou-se afinal de alimentar seu rancor na inação,
despiu a batina, tornou a vestir sua blusa sem presilha, seu gibão
surrado, seu capuz desbotado de chuva, e voltou a esmolar a assistência
dos frades devotos que trabalhavam para os Negros.

Uma grande alegria o aguardava. Dante fora exilado: em Florença só restavam
partidos obscuros. O sapateiro murmurava humildemente para a Virgem o
triunfo próximo dos Negros. Cecco Angiolieri, em sua volúpia, esqueceu-se
de Becchina. Vagueou pelas sarjetas, comeu nacos de pão duro, correu a pé
atrás dos enviados da Igreja que iam a Roma e voltavam para Florença.
Perceberam que ele poderia ser útil. Corso Donati, violento chefe dos
Negros, de volta a Florença, e poderoso, empregou-o junto com outros. Na
noite de 10 de junho de 1304, uma turba de cozinheiros, tintureiros,
ferreiros, padres e mendigos invadiu o bairro nobre de Florença onde
ficavam as belas casas dos Brancos. Cecco Angiolieri brandia a tocha
resinosa do sapateiro, que seguia à distância, admirando os desígnios
celestes. A tudo incendiaram e Cecco queimou o revestimento das sacadas
dos Cavalcanti, que haviam sido amigos de Dante. Naquela noite, saciou com
fogo sua sede de rancor. No dia seguinte, enviou a Dante, o “Lombardo”,
versos de insulto, na corte de Verona. No mesmo dia, tornou-se Cecco
Angiolieri como desejava havia tantos anos: morrera seu pai, tão velho
como Enoc ou Elias.

Cecco correu para Siena, arrombou os cofres e mergulhou as mãos nos sacos
de florins novos, cem vezes repetindo a si mesmo que já não era o pobre
irmão Henrique, e sim um nobre, senhor de Arcidosso e Montegiovi, mais
rico que Dante e melhor poeta. Então refletiu que era um pecador e que
desejara a morte do pai. Arrependeu-se. Rabiscou, no ato, um soneto
pedindo ao Papa uma cruzada contra todos que insultassem os próprios pais.
Ansioso por confessar-se, voltou às pressas para Florença, abraçou o
sapateiro, suplicou-lhe que intercedesse junto a Maria.

Foi depressa à loja de ceras santas e comprou um círio dos grandes. O
sapateiro acendeu-o com unção. Ambos choraram e rezaram a Nossa Senhora.
Até horas tardias, escutou-se a voz serena do sapateiro entoando louvores,
feliz com seu brandão e enxugando as lágrimas do amigo.

\chapter{Paolo Uccello pintor}

Chamava-se na verdade Paolo di Dono; mas os florentinos deram-lhe o nome de
Uccelli, ou Paulo dos Pássaros, devido aos inúmeros pássaros figurados e
bichos pintados que enchiam sua casa: pois era pobre demais para alimentar
animais ou adquirir aqueles que não conhecia. Dizem até que, em Pádua,
pintou um afresco dos quatro elementos, e deu ao ar, como atributo, a
imagem do camaleão. Mas nunca havia visto nenhum, de modo que representou
um camelo barrigudo de goela aberta. (Ora, explica Vasari, o camaleão se
parece com um lagartinho seco, ao passo que o camelo é um bicho grande
desengonçado.) Pois Uccello não se importava com a realidade das coisas, e
sim com sua multiplicidade e com a infinitude das linhas; de modo que
criou campos azuis, e cidades vermelhas, e cavaleiros vestidos com
armaduras negras montando cavalos de ébano com a boca inflamada, e lanças
assestadas feito raios de luz para todos os pontos do céu. E ele tinha o
hábito de desenhar \textit{mazocchi}, que são círculos de madeira cobertos
por um pano que se põe sobre a cabeça, de maneira a que as dobras do
tecido solto emoldurem o rosto inteiro. Uccello desenhou alguns pontudos,
outros quadrados, outros multifacetados, dispostos em pirâmides e cones,
conforme todos os aspectos perspectiva, de modo que encontrava um mundo de
combinações nas dobras do \textit{mazocchio}. E o escultor Donatello lhe
dizia: “Ah! Paolo, você troca a substância pela sombra!”.

O Pássaro, porém, continuava sua obra paciente, e ajuntava círculos, e
dividia ângulos, e examinava todas as criaturas sob todos os seus
aspectos, e perguntava a interpretação dos problemas de Euclides ao seu
amigo, o matemático Giovanni Manetti; depois se confinava e cobria seus
pergaminhos e madeiras de pontos e curvas. Dedicou-se com constância ao
estudo da arquitetura, no que se fez ajudar por Filippo Brunelleschi; mas
não era com a intenção de construir. Limitava-se a reparar nas direções
das linhas, dos alicerces às cornijas, e na convergência das retas em suas
intersecções, e no modo como as abóbadas se curvavam nos fechos, e no
escorço em forma de leque das vigas de teto que pareciam se unir na
extremidade dos vastos salões. Representava também todos os animais e seus
movimentos, e os gestos humanos, a fim de reduzi-los a linhas simples.

Depois, qual o alquimista se debruçando sobre as misturas de metais e
órgãos e espreitando sua fusão em seu forno para encontrar o ouro, Uccello
vertia todas as formas no cadinho das formas. Ele as juntava, e combinava,
e fundia, a fim de obter sua transmutação naquela forma simples de que as
outras todas dependem. Eis por que viveu Paolo Uccello como um alquimista
dentro de sua casinha. Pensou que pudesse transmudar todas as linhas num
só aspecto ideal. Quis conceber o universo criado tal qual se refletia no
olho de Deus, que vê jorrar todas as figuras de dentro de um centro
complexo. Nas cercanias viviam Ghiberti, della Robbia, Brunelleschi,
Donatello, cada um deles orgulhoso e mestre em sua arte, escarnecendo o
pobre Uccello e sua loucura pela perspectiva, lamentando sua casa cheia de
aranhas, vazia de alimentos; Uccello, porém, era mais orgulhoso ainda. A
cada nova combinação de linhas, esperava ter descoberto a maneira de
criar. Seu intento não estava na imitação, e sim na capacidade de \label{intento}
desenvolver soberanamente todas as coisas, e a estranha série de capelos
com dobras lhe parecia mais reveladora que as magníficas figuras de
mármore do grande Donatello.

Assim vivia o Pássaro, e sua cabeça pensativa ficava envolta em sua capa; e
não atentava nem para o que comia nem para o que bebia, e era, em tudo,
igual a um eremita. De modo que, num prado, próximo a um círculo de
antigas pedras cravadas em meio à relva, avistou certo dia uma moça que
ria, a cabeça cingida por uma grinalda. Usava um longo vestido delicado,
preso à cintura por uma fita clara, e seus movimentos eram flexíveis como
as hastes que ela curvava ao passar. Seu nome era Selvaggia, e ela sorriu
para Uccello. Ele notou a flexão de seu sorriso. E, quando ela olhou para
ele, viu todas as pequenas linhas de seus cílios, e os círculos de suas
pupilas, e a curva de suas pálpebras, e os entrelaços sutis de seus
cabelos e, em pensamento, atribuiu à grinalda que lhe cingia a fronte uma
porção de posições. Mas Selvaggia nada percebeu, pois tinha apenas treze
anos. Pegou Uccello pela mão e o amou. Era filha de um tintureiro de
Florença, e sua mãe havia morrido. Outra mulher entrara em sua casa, e
havia surrado Selvaggia. Uccello levou-a consigo.

Selvaggia passava o dia inteiro de cócoras frente à muralha em que Uccello
traçava as formas universais. Nunca compreendeu por que ele preferia
contemplar linhas retas e arqueadas a olhar a doce figura que se erguia
para ele. Quando escurecia e Brunelleschi ou Manetti vinham estudar com
Uccello, ela adormecia, passada meia-noite, aos pés das retas
entrecruzadas, no círculo de sombra que se espraiava sob a lâmpada. Pela
manhã, acordava antes de Uccello, e se alegrava por estar rodeada de
pássaros pintados e bichos coloridos. Uccello desenhou seus lábios, e seus
olhos, e seus cabelos, e suas mãos, e retratou todas as poses de seu
corpo; mas não fez seu retrato, como faziam os outros pintores que amavam
uma mulher. Pois o Pássaro desconhecia a alegria de se ater a um
indivíduo; não se quedava num lugar só: queria pairar, em seu voo, sobre
todos os lugares. E as formas das poses de Selvaggia foram jogadas no
cadinho das formas, com todos os movimentos dos bichos, e as linhas das
plantas e pedras, e os raios da luz, e as ondulações dos vapores
terrestres e ondas do mar. E sem se lembrar de Selvaggia, Uccello parecia
quedar-se eternamente debruçado sobre o cadinho das formas.

Entretanto, não havia o que comer na casa de Uccello. Selvaggia não se
atrevia a dizê-lo a Donatello e aos outros. Ficou calada, e morreu.
Uccello representou o enrijecer de seu corpo, e a união de suas mãozinhas
magras, e a linha de seus pobres olhos fechados. Não soube que estava
morta, como nunca soubera que estava viva. Mas jogou aquelas formas novas
em meio a todas as outras que havia acumulado.

O Pássaro foi envelhecendo, e ninguém mais entendia seus quadros. Via-se
apenas uma confusão de curvas. Já não se reconheciam neles a terra, as
plantas, os animais, ou os homens. Trabalhava desde muitos anos em sua
obra suprema, que ele ocultava a todos os olhares. Essa obra deveria
abarcar todas as suas pesquisas, das quais seria, a seu ver, a imagem. Era
são Tomé incrédulo, experimentando a chaga do Cristo. Uccello tinha
oitenta anos quando concluiu seu quadro. Mandou chamar Donatello, e
piedosamente o descobriu diante dele. E Donatello exclamou: “Ó Paolo,
torne a cobrir seu quadro!”. O Pássaro interrogou o grande escultor: mas
este não quis dizer mais nada. Soube Uccello, desse modo, que havia
cumprido o milagre. Donatello, porém, vislumbrara apenas um emaranhado de
linhas.

E alguns anos mais tarde, encontraram Paolo \mbox{Uccello} morto de exaustão em
seu catre. Seu rosto estava resplandecente de rugas. Seus olhos fitavam o
mistério revelado. Segurava na mão estreitamente fechada um pequeno rolo
de pergaminho coberto de entrelaços que iam do centro à circunferência e
voltavam da circunferência ao centro.

\chapter{Nicolas Loyseleur juiz}

Nasceu no dia da Assunção, e foi devoto da Virgem. Tinha o hábito de
invocá-la em toda circunstância da vida e não podia escutar seu nome sem
que seus olhos se enchessem de lágrimas. Após ter estudado num pequeno
sótão da rua Saint-Jacques sob a férula de um magro clérigo, na companhia
de três crianças que rezingavam o Donato\footnote{ Referência à
\textit{Ars grammatica} de Élio Donato, gramático latino do século \versal{IV}.} 
e os salmos da Penitência, estudou laboriosamente a Lógica de
Occam.\footnote{ William de Ockham, ou Guilherme de Occam (1285--1347),
filósofo escolástico inglês da ordem dos franciscanos.} Cedo
tornou-se, assim, bacharel e mestre em artes. As veneráveis pessoas que o
instruíam nele observaram uma grande doçura e uma unção encantadora. Tinha
lábios polpudos dos quais deslizavam as palavras na adoração. Tão logo
obteve seu bacharelado em teologia, a Igreja lhe prestou atenção. Oficiou
de início na diocese do bispo de Beauvais, que reconheceu suas qualidades
e se serviu dele para alertar os ingleses diante de Chartres das várias
movimentações dos capitães franceses. Por volta de seus 35 anos,
fizeram-no cônego da catedral de Rouen. Lá, foi bom amigo de Jean
Bruillot, cônego e chantre, com o qual salmodiava belas litanias em honra
de Maria.

Admoestava, às vezes, Nicole Coppequesne, que pertencia ao seu cabido, por
sua lamentável predileção por santa Anastácia. Nicole Coppequesne não se
cansava de admirar que uma moça tão recatada tivesse encantado um prefeito
romano a ponto de o deixar, numa cozinha, apaixonado pelas marmitas e
caldeirões que ele abraçava com fervor; tanto assim que, com o rosto todo
enegrecido, passou a se parecer com um demônio. Mas Nicolas Loyseleur lhe
mostrava que muito maior foi o poder de Maria quando devolveu à vida um
monge afogado. Um monge lúbrico, mas que nunca se furtara a reverenciar a
Virgem. Certa noite, levantando-se para ir cumprir suas más ações, teve o
cuidado, ao passar diante do altar de Nossa Senhora, de fazer uma
genuflexão e saudá-la. Naquela mesma noite, sua lubricidade o levou a
afogar-se no rio. Mas os demônios não deram conta de carregá-lo e quando,
no dia seguinte, os monges tiraram seu corpo da água, tornou a abrir os
olhos, reanimado pela graciosa Maria. “Ah! esta devoção é um precioso
remédio'', suspirava o cônego, ``e alguém venerável e discreto como você,
Coppequesne, deveria sacrificar-lhe o amor por Anastácia.”

A gentileza persuasiva de Nicolas Loyseleur não foi esquecida pelo bispo de
Beauvais quando este começou a instruir, em Rouen, o processo de Joana, a
Lorena. Nicolas vestiu roupas curtas, laicas e, a tonsura disfarçada por
um capuz, fez-se introduzir na pequena cela circular, debaixo de uma
escada, onde estava trancada a prisioneira.

--- Joana, menina --- disse ele, postando-se na sombra ---, creio é que santa
Catarina que me traz até você.

--- Quem é o senhor, em nome de Deus? --- inquiriu Joana.

--- Um pobre sapateiro de Greu --- disse Nicolas ---, da nossa terra, ai, tão
sofrida; e os \textit{Godon}\footnote{ Godon, corruptela da blasfêmia
inglesa \textit{God damn me}, termo pejorativo, hoje em desuso, com que
eram popularmente denominados na França os invasores ingleses durante a
Guerra dos Cem Anos.} pegaram a mim como pegaram a você, minha
filha. Oxalá fosse você do céu! Eu a conheço bem, vá; e muitas e muitas
vezes a avistei quando vinha orar para a santíssima Mãe de Deus na igreja
de Sainte-Marie de Bermont. E, tal como você, não raro assisti às missas
do nosso bom cura, Guilherme Front. E você se recorda, ai, de Jean Moreau
e Jean Barre de Neufchâteau? São meus camaradas.

Então Joana chorou.

--- Joana, menina, confie em mim --- disse Nicolas. --- Ordenaram-me clérigo quando
eu era criança. E, veja, cá está a tonsura. Confesse-se, minha filha,
confesse-se com toda a liberdade, pois sou amigo de nosso gracioso rei
Carlos.

--- De bom grado me confessarei ao senhor, meu amigo --- disse a boa Joana.

Ora, uma abertura fora praticada na muralha; e do lado de fora, sob um
degrau da escada, Guilherme Manchon e Bois-Guillaume redigiam a minuta da
confissão. E Nicolas Loyseleur dizia:

--- Joana, menina, persista em suas palavras, e tenha coragem, os ingleses
não ousarão lhe fazer mal algum.

No dia seguinte, Joana compareceu perante os juízes. Nicolas Loyseleur
postara-se com um tabelião no vão de uma janela, atrás de uma cortina de
sarja, a fim de só mandar lavrar as imputações e deixar em branco as
justificativas. Os outros dois escrivães, porém, reclamaram. Quando
Nicolas reapareceu na sala, fez pequenos sinais a Joana para que ela não
demonstrasse surpresa, e assistiu, severo, ao interrogatório.

Dia 9 de maio opinou, na ampla torre do castelo, que eram urgentes os
suplícios.

Dia 12 de maio, os juízes se reuniram em casa do bispo de Beauvais para
deliberar acerca da utilidade de se submeter Joana à tortura. Guilherme
Erart julgava não ser preciso, já havendo, sem tortura, matéria
suficiente. Mestre Nicolas Loyseleur afirmou que lhe parecia adequado,
para a salvação de sua alma, ela ser submetida à tortura; mas seu conselho
não prevaleceu.

Dia 24 de maio, Joana foi levada ao cemitério de Saint-Ouen, onde a fizeram
subir num cadafalso de gesso. Lá deparou, ao seu lado, com Nicolas
Loyseleur falando-lhe ao ouvido, enquanto Guilherme Erart lhe fazia a
pregação. Quando ameaçada com o fogo, empalideceu; enquanto a amparava, o
cônego piscou para os juízes e disse: “Ela há de abjurar”. Ele conduziu
sua mão para que marcasse com uma cruz e um círculo o pergaminho que lhe
estenderam. Depois, acompanhou-a até uma portinhola rebaixada e lhe afagou
os dedos:

--- Joana, minha menina --- disse ele ---, você fez hoje um belo trabalho, benza-lhe
Deus; você salvou sua alma. Confie em mim, Joana, porque se você quiser,
será libertada. Receba suas roupas de mulher; faça o que lhe ordenarem; de
outro modo estará correndo risco de vida. E se fizer o que lhe digo, há de
ser salva, há de receber muito bem e mal nenhum; mas estará em poder da
Igreja.

Naquele mesmo dia, depois do jantar, foi visitá-la em sua nova prisão. Era
um quarto mediano do castelo, a que se chegava por oito degraus. Nicolas
sentou-se na cama, junto à qual havia uma tora de madeira atada a uma
corrente de ferro.

--- Joana, menina --- disse ele ---, veja como Deus e Nossa Senhora lhe concederam
neste dia uma grande misericórdia, ao acolhê-la na graça e misericórdia de
nossa Santa Madre Igreja; terá de obedecer com muita humildade às
sentenças e prescrições dos juízes e pessoas eclesiásticas, abandonar suas
antigas fantasias e não retornar a elas, pois do contrário a Igreja irá
abandoná-la para sempre. Tome, aqui estão roupas direitas de mulher
recatada; Joana, menina, cuide bem delas; e mande depressa raspar este
cabelo cortado em forma de rotunda.

Quatro dias depois, Nicolas esgueirou-se à noite até o quarto de Joana e
roubou a blusa e a saia que lhe dera. Ao ser informado de que ela retomara
suas roupas de homem:

--- É uma pena --- disse ---, ela é relapsa e profundamente decaída no mal.

E na capela do arcebispado, repetiu as palavras do doutor Gilles de
Duremort:

--- A nós, juízes, só resta declarar Joana herética e abandoná-la à justiça
secular, rogando-lhe que aja suavemente com ela.

Antes que a conduzissem ao triste cemitério, foi exortá-la em companhia de
Jean Toutmouillé.

--- Ó Joana, menina --- disse ele ---, não esconda mais a verdade; você não deve
pensar agora senão na salvação de sua alma. Acredite em mim, minha filha:
dentro em pouco, na assembleia, você irá se humilhar e fazer, de joelhos,
sua confissão pública. Que seja pública, Joana, humilde e pública, para a
salvação de sua alma.

E Joana rogou que ele a lembrasse de fazê-lo, temendo não ousar diante de
tanta gente.

Ele ficou para vê-la queimar. Foi então que sua devoção à Virgem se
manifestou de forma visível. Tão logo escutou os apelos de Joana a santa
Maria, desatou num pranto sentido. A tal ponto o comovia o nome de Nossa
Senhora. Os soldados ingleses, pensando que ele chorava de pena,
esbofetearam-no e perseguiram-no de espada em riste. Não fosse o conde de
Warwick estender a mão sobre ele, degolavam-no. A custo conseguiu montar
num cavalo do conde, e fugiu.

Vagueou dias sem fim pelas estradas da França, não ousando voltar para a
Normandia e temendo a gente do rei. Chegou, enfim, a Basileia. Sobre a
ponte de madeira, entre as casas pontudas, cobertas com telhas de ogivas
traçadas, e as guaritas azuis e amarelas, súbito sentiu-se ofuscado diante
da luz do Reno; julgou estar se afogando, como o monge lúbrico, em meio à
água verde que remoinhava em seus olhos; a palavra Maria ficou presa em
sua garganta, e ele morreu num soluço.

\chapter{Katherine, a rendeira mulher amante}

Nasceu em meados do século quinze, na rua de la Parcheminerie, próxima à
rua Saint-Jacques, num inverno em que fez tanto frio que os lobos correram
em Paris sobre a neve. Recolheu-a uma velha mulher, de nariz vermelho sob
o capuz, que a criou. E de início ela brincou sob os pórticos com
Perrenette, Guillemette, Ysabeau e Jehanneton, que usavam pequenas cotas e
mergulhavam nas valetas as mãozinhas vermelhas para apanhar pedaços de
gelo. Também observavam os que lesavam os transeuntes no jogo de tabuleiro
chamado Saint-Merry. E, nos alpendres, espiavam as tripas dentro das
selhas, e as compridas salsichas balançantes e os enormes ganchos de ferro
em que os açougueiros penduram as peças de carne. Nas proximidades de
Saint-Benoît le Bétourné,\footnote{ Capela construída no século \versal{VI} em
Paris, demolida em 1854 para dar lugar à universidade da Sorbonne.} 
onde ficam os \textit{scriptorium}, escutavam o ranger das penas, e
ao entardecer, pelas frestas dos ateliês, sopravam as candeias na cara dos
clérigos. Na Ponte Pequena, escarneciam das peixeiras e escapuliam ligeiro
para a praça Maubert, escondiam-se nas esquinas da rua des Trois-Portes;
então, sentadas à beira da fonte, tagarelavam até as brumas da noite.

Assim transcorreu a adolescência de Katherine, até que a velha lhe
ensinasse a sentar-se diante de uma almofada de rendeira e entrecruzar
pacientemente os fios de todos os carretéis. Mais tarde, fez da renda o
seu ofício, tendo Jehanneton se tornado capuzeira, Perrenette, lavadeira,
Ysabeau, luveira, e Gillemette, a mais ditosa, salsicheira, com um
rostinho carmesim que reluzia como se o tivessem esfregado com sangue
fresco de porco. Quanto aos que brincavam no bairro de Saint-Merry,
enveredaram por outros caminhos; uns estudavam no Monte
Sainte-Geneviève,\footnote{ No Monte Sainte-Geneviève, situado no Quartier
Latin de Paris, construíram-se ao longo de dois mil anos de história
inúmeros edifícios, igrejas, abadias, universidades, entre os quais o
Collège Sainte-Barbe, fundado em 1460 (fechado em 1999).} outros
jogavam baralho na taberna Trou-Perrette, outros brindavam com vinho de
Aunis na Pomme de Pin e outros tinham altercações no hotel da Gorda
Margot, e eram vistos à hora do meio-dia na porta da taberna da rua dos
Fèves, e à hora da meia-noite, saíam pela porta da rua dos Judeus. Quanto
a Katherine, entrelaçava os fios de suas rendas, e nos entardeceres de
verão tomava o sereno no banco da igreja, onde era permitido rir e
tagarelar.

Katherine usava uma blusa de linho cru e uma sobreveste de cor verde; tinha
loucura por adornos, e não havia nada que detestasse tanto como o barrete
que assinala as moças que não são de nobre linhagem. Gostava igualmente
das moedas de prata, brancas e, mais que nada, dos escudos de ouro. Foi o
que a levou a se relacionar com Casin Cholet, sargento a pé do Châtelet;
ele ganhava pouco com seu ofício. Ceou diversas vezes em sua companhia na
estalagem da Mula, defronte à igreja des Mathurins; e, depois de cear,
Casin Cholet ia apanhar galinhas no reverso dos fossos de Paris. Trazia-as
debaixo de seu amplo tabardo e as vendia muito bem para Machecroue, viúva
de Arnoul, bonita vendedora de aves da porta do Petit-Châtelet.

Não demorou e Katherine largou seu ofício de rendeira: pois a velha de
nariz vermelho apodrecia no Ossuário dos Inocentes. Casin Cholet encontrou
para a amiga uma quartinho subterrâneo, perto da Trois-Pucelles, e ali
vinha visitá-la ao entardecer. Ele não a proibia de se exibir à janela, os
olhos escurecidos com carvão, as faces untadas de branco de chumbo; e
todos os jarros, xícaras e pratos de frutas nos quais Katherine oferecia
de comer e beber a todos que pagassem bem haviam sido roubados no Chaire,
ou no Cygnes, ou no hotel do Plat d’Étain. Casin Cholet sumiu certo dia
depois de empenhar o vestido e o semicorpete de Katherine na
Trois-Lavandières. Disseram seus amigos à rendeira que ele fora espancado
no fundo de uma charrete e expulso de Paris, a mando do preboste, pela
porta Baudoyer. Nunca mais tornou a vê-lo; e sozinha, sem ânimo para
ganhar dinheiro, tornou-se mulher amante, residindo em todo lugar.

De início, esperava à porta das estalagens; e aqueles que a conheciam a
levavam para trás dos muros, debaixo do Châtelet ou junto ao colégio de
Navarra; depois, quando o frio se fez demasiado, uma velha complacente
deixou que entrasse nos banhos, onde a patroa lhe deu abrigo. Ali viveu
num quarto de pedra atapetado de juncos verdes. Mantiveram seu nome de
Katherine, a rendeira, embora ali não fizesse rendas. Davam-lhe, às vezes,
a liberdade de passear pelas ruas, com a condição de estar de volta à hora
em que as pessoas costumam ir aos banhos. E Katherine vagava frente aos
ateliês da luveira e da capuzeira, e não raro se demorou a invejar o rosto
sanguíneo da salsicheira, rindo em meio a suas carnes de porco. Depois
voltava aos banhos públicos, que a patroa alumiava ao crepúsculo com
candeias que ardiam vermelhas e derretiam pesadamente por detrás das
negras vidraças.

Katherine por fim se cansou de viver encerrada num quarto quadrado; fugiu
para as estradas. E, desde então, deixou de ser parisiense, ou rendeira;
foi como aquelas que rondam os arredores das cidades francesas, sentadas
nas pedras dos cemitérios, para dar prazer aos que passam. Essas meninas
não têm outro nome se não aquele que combina com sua imagem, e Katherine
recebeu o nome de Focinho. Andava pelos prados e, ao entardecer, ficava à
espreita à beira dos caminhos, e se avistava o seu vulto branco entre as
amoreiras das sebes. Focinho aprendeu a suportar o medo noturno em meio
aos mortos, quando seus pés tiritavam ao roçar os túmulos. Já não havia
moedas de prata, nem escudos de ouro; vivia pobremente de pão e queijo, e
de uma tigela d’água. Teve amigos miseráveis que lhe sussurravam de longe:
“Focinho! Focinho!”, e os amou.

A tristeza maior era escutar os sinos de igrejas e capelas; pois Focinho
recordava as noites de junho em que se sentava, de sobreveste verde, nos
bancos dos pórticos santos. Era no tempo em que invejava os atavios das
donzelas; já não lhe restava agora nem barrete, nem capuz. Cabeça
descoberta, esperava por seu pão, encostada a uma laje dura. E sentia
falta, em meio à noite do cemitério, das candeias vermelhas dos banhos
públicos, e dos juncos verdes do quarto quadrado em lugar da lama pastosa
em que seus pés afundavam.

Certa noite, um rufião se passando por homem de guerra cortou o pescoço de
Focinho para lhe tomar seu cinturão. Dentro dele, porém, não encontrou
bolsa nenhuma.

\chapter{Alain o gentil soldado}

Serviu o rei Carlos \versal{VII}, como arqueiro, desde os doze anos, tendo sido
raptado por homens de guerra na plana região da Normandia. Eis a maneira
como foi raptado. Enquanto eram incendiadas as granjas, laceradas a golpes
de adaga as pernas dos lavradores, e jogadas as meninas sobre camas de
lona, quebradas, o pequeno Alain se acaçapara numa velha pipa de vinho à
entrada do lagar. Os homens de guerra derrubaram a pipa e encontraram um
rapazinho. Carregaram-no junto com sua blusa e saiote. O capitão mandou
que lhe dessem um pequeno gibão de couro e um velho capuz remanescente da
batalha de Saint-Jacques. Perrin Godin o ensinou a atirar com o arco e a
acertar a seta da besta no alvo. Foi de Bordeaux para Angoulême e de
Poitou para Bourges, viu Saint-Pourçain, onde ficava o rei, cruzou as
fronteiras da Lorena, visitou Toul, voltou para a Picardia, entrou em
Flandres, cruzou Saint-Quentin, quebrou para a Normandia, e durante vinte
e três anos percorreu a França em companhia armada, vindo a conhecer o
inglês Jehan Poule-Cras, que lhe ensinou como jurar por \textit{Godon},
Chiquerello, o lombardo, que lhe ensinou a curar o fogo de Santo Antônio,
e a jovem Ydre de Laon, que lhe mostrou como baixar as bragas.

Em Ponteau de Mer, seu companheiro Bernard d’Anglades o convenceu a se
afastar da ordenança real, garantindo que os dois viveriam à larga
enganando os tolos com dados viciados, ditos “canhestros”. Assim fizeram,
sem se desfazerem de seus aparatos, e junto aos muros do cemitério,
fingiam jogar sobre um tamboril roubado. Um mau sargento do vicariato,
Pierre Empongnart, fez com que lhe mostrassem as sutilezas de seu jogo e
avisou que não tardariam a ser pegos; e que então deveriam descaradamente
jurar que eram clérigos, de modo a se safar dos homens do rei e clamar
pela justiça da Igreja e, para tanto, tinham de raspar rente o topo da
cabeça e deitar fora prontamente, em caso de necessidade, suas golas
rasgadas e mangas de cor. Ele próprio os tonsurou com tesouras consagradas
e os fez resmonear os sete Salmos e o versículo \textit{Dominus pars}.
Então, foi cada um para o seu lado, Bernard com Bietrix a chaveira, e Alain
com Lorenette, a cirieira.

Como Lorenette desejasse uma sobreveste de tecido verde, Alain espiou a
taberna do Cavalo Branco, em Lisieux, onde beberam um cântaro de vinho. À
noite, voltou ao jardim, com sua azagaia abriu um buraco na parede e
entrou na sala, onde encontrou sete tigelas de estanho, um capuz vermelho
e uma vergasta de ouro. Jaquet o Grande, comerciante de roupas usadas de
Lisieux, trocou-os proveitosamente por uma sobreveste como Lorenette
desejava.

Em Bayeux, Lorenette morou numa casinha pintada, onde segundo se dizia,
ficavam os banhos das mulheres, e a patroa apenas riu quando Alain o
Gentil quis retomá-la. Acompanhou-o até a porta, candeia em punho e na
outra mão uma pedra graúda, perguntando se ele acaso queria que ela lhe
esfregasse a pedra na cara para ver se ele fazia careta. Alain fugiu,
derrubando a candeia, arrancando da mão da mulher o que lhe pareceu ser
uma vergasta preciosa: mas era de cobre sobredourado, com uma grande pedra
rosa falsa.

Então Alain saiu vagamundeando, e em Maubusson, na estalagem do Papegaut,
topou com Karandas, seu companheiro de armas, comendo tripas com outro
homem chamado Jehan o Pequeno. Karandas ainda carregava sua partasana, e
Jehan Pequeno levava suas agulhetas numa bolsa pendurada no cinto. O
mordente do cinto era de prata fina. Depois de beber, deliberaram os três
ir pelo bosque até Senlis. Ao entardecer, puseram-se a caminho, e estavam
em plena floresta, sem luz, quando Alain o Gentil começou a manquejar.
Jehan o Pequeno ia na frente. E no escuro, Alain arremessou-lhe a azagaia
nas escápulas, enquanto Karandas lhe abateu a partasana na cabeça. Caiu de
bruços e Alain, se escarranchando sobre ele, abriu-lhe de ponta a ponta a
garganta com a adaga. Encheram então seu pescoço com folhas secas, para
que não ficasse uma poça de sangue no caminho. A lua surgiu numa clareira:
Alain cortou o mordente do cinto, e desatou as agulhetas da bolsa na qual
havia dezesseis leões de ouro e trinta e seis \textit{patards}.\footnote{
Leão: moeda vigente durante o reinado de Francisco \versal{I}, que trazia a efígie
deste animal. \textit{Patard}: moedinha com valor de dois soldos que
circulava na França do século \versal{XI}.} Guardou os leões e, com a azagaia em
riste jogou para Karandas, como paga, a bolsa com as moedinhas. Ali se
separaram, no meio da clareira, Karandas praguejando algo sobre o sangue
de Deus.

Alain o Gentil não se atreveu a ir a Senlis e voltou, por meio de atalhos,
à cidade de Rouen. Ao despertar de uma noite de sono sob uma sebe florida,
viu-se cercado por homens cavaleiros que lhe ataram as mãos e o conduziram
às prisões. Próximo à entrada, esgueirou-se atrás da garupa de um cavalo e
correu para a igreja de Saint-Patrice, refugiando-se junto ao altar-mor.
Os sargentos não podiam transpor o pórtico. Alain, em imunidade, andou
livremente pela nave e pelos coros, avistou belos cálices de rico metal e
galhetas boas de fundir. Na noite seguinte, teve a companhia de Denisot e
Marignon, larápios como ele. Marignon tinha uma orelha cortada. Não tinham
o que comer. Invejaram os camundongos vadios que se aninhavam entre as
lajes e engordavam rilhando migalhas do pão sagrado. Na terceira noite,
tiveram de sair, com a fome entre os dentes. Foram apanhados pelos homens
de justiça, e Alain, que se declarou clérigo, esqueceu-se de arrancar fora
as mangas verdes.

Logo pediu para ir ao retrete, descosturou o gibão e enfiou as mangas na
imundície; mas os homens do cárcere alertaram o preboste. Um barbeiro veio
raspar por completo a cabeça de Alain o Gentil, para apagar sua tonsura.
Os juízes se riram do pobre latim de seus salmos. Em vão jurou ter sido,
com um sopapo, crismado por um bispo aos dez anos de idade: não logrou
recitar o \textit{pater nostre}. Foi posto na tortura como leigo, no
cavalete pequeno, e depois no grande. No fogo das cozinhas da prisão,
declarou seus crimes, os membros desvairados pelo estiramento das cordas e
a garganta rebentada. O lugar-tenente do preboste pronunciou a sentença.
Foi amarrado à carreta, arrastado ao cadafalso e enforcado. Seu corpo
crestou-se ao sol. O carrasco pegou seu gibão, suas mangas descosturadas,
e um belo capuz de pano fino, forrado com pele de esquilo, que ele havia
roubado numa boa estalagem.

\chapter{Gabriel Spenser ator}

Sua mãe foi uma rameira, chamada Flum, que mantinha uma pequena sala
subterrânea nos fundos de Rotten-row, em Picked-hatch. Um capitão com os
dedos pesados de anéis de cobre, e dois galantes, vestidos com gibões
folgados, vinham visitá-la depois do jantar. Ela alojava três moçoilas,
cujos nomes eram Poll, Doll e Moll e não suportavam o cheiro do fumo.
Assim, subiam seguidamente para a cama, e cavalheiros educados as
acompanhavam depois de tê-las feito beber um copo de vinho da Espanha
morno a fim de dissipar o fumo dos cachimbos. O pequeno Gabriel ficava
agachado sob pano da lareira, vendo assarem as batatas jogadas nos canecos
de cerveja. Vinham também atores, das mais diversas aparências. Não
ousavam aparecer nas grandes tabernas frequentadas pelas companhias
subsidiadas. Alguns falavam em estilo de fanfarronada; outros
tartamudeavam como idiotas. Afagavam Gabriel, que com eles aprendeu versos
esparsos de tragédias e rústicas brincadeiras de palco. Ganhou um pedaço
de pano carmesim, de franja desdourada, uma máscara de veludo e um velho
punhal de pau. Assim se pavoneava sozinho frente ao átrio, brandindo um
tição como se fora um archote, e Flum, sua mãe, balançava o queixo triplo
de admiração por seu filho precoce.

Os atores o levaram ao Rideau Vert, no Shoreditch, onde ele estremeceu ante
os acessos de raiva do pequeno ator que escumava aos berros no papel de
Jeronymo. Também estava ali o velho rei Leir, com sua barba branca
rasgada, ajoelhando-se para pedir perdão à sua filha Cordélia; um palhaço
imitava as loucuras de Tarleton, e outro, envolto num lençol, aterrorizava
o príncipe Amlet. Sir John Oldcastle fazia todo o mundo rir com seu
barrigão, principalmente quando enlaçava a cintura da hospedeira, que lhe
permitia amassar a ponta de sua touca e deslizar os dedos na bolsinha de
tarlatana que trazia presa ao cinto. O Louco cantava canções que o Bobo
nunca entendia, e a todo instante um palhaço de touca de algodão apontava
a cabeça por um rasgo da cortina, no fundo do tablado, e fazia caretas.
Havia ainda um malabarista com macacos e um homem vestido de mulher que,
na ideia de Gabriel, parecia-se com Flum, sua mãe. No fim da peça, os
sacristãos lhe serviam um vinho zurrapa e gritavam que iam levá-lo para
Bridewell.\footnote{ Palácio construído como residência de Henrique~\versal{VIII}, 
doado em 1553 à cidade de Londres para servir de abrigo a crianças de
rua e mulheres desordeiras. Em 1556, foi transformado em prisão.}

Estava Gabriel com quinze anos quando os atores do Rideau Vert repararam
que ele era bonito e delicado e poderia fazer os papéis de mulheres e
moças. Flum penteava seus cabelos pretos jogados para trás; tinha a pele
muito fina, olhos grandes, sobrancelhas altas, e Flum furara-lhe as
orelhas para pendurar dois pares de pérolas falsas. Ele então ingressou na
companhia do duque de Nottingham, e confeccionaram para ele vestidos de
tafetá e damasco com lantejoulas, de pano de prata e pano de ouro,
corpetes laçados e perucas de cânhamo com cachos compridos. Ensinaram-no a
pintar-se na sala dos ensaios. De início, corava ao subir no palco; depois
fazia trejeitos respondendo aos galanteios. Poll, Doll e Moll, trazidas
por Flum, muito atarefada, declararam com risadas que estava uma perfeita
mulher e quiseram desenlaçar-lhe o corpete após a peça. Levaram-no de
volta a Picked-hatch, e sua mãe pediu que usasse um de seus vestidos para
mostrá-lo ao capitão, o qual, por pilhéria, lhe disse mil galanteios e fez
que lhe enfiava no dedo um anel grosseiro, sobredourado, com um falso rubi
incrustado.

Os melhores companheiros de Gabriel Spenser eram William Bird, Edward Juby
e os dois Jeffes. Estes resolveram, certo verão, se apresentar com atores
errantes nos burgos do interior. Viajavam num carro coberto com uma lona,
no qual dormiam à noite. Certa noite, na estrada de Hammersmith, viram
surgir de um barranco um homem apontando o cano de uma pistola.

--- Seu dinheiro! --- disse ele. --- Sou Gamaliel Ratsey, salteador pela graça de
Deus, e não gosto de esperar.

Ao que os dois Jeffes responderam, gemendo:

--- Não temos dinheiro, Vossa Graça, só essas lantejoulas de cobre e essas
peças de bijuteria pintadas, e somos pobres atores errantes tal como Vossa
Senhoria.

--- Atores? --- exclamou Gamaliel Ratsey. --- Está aí algo admirável. Não sou
saqueador nem patife, e sou amigo do espetáculo. Não fosse um certo
respeito que nutro pelo velho Derrick,\footnote{ Thomas Derrick, conhecido
carrasco inglês do século~\versal{XVII}, que executou mais de três mil pessoas.
Inventou um sistema de polias que modernizou a antiga forca, à qual seu
nome ficou desde então associado.} que saberia me arrastar até a
escada e me fazer bambolear a cabeça, nunca sairia da beira do rio, e das
alegres tabernas onde vocês, cavalheiros, costumam exibir tanto espírito.
Sejam então bem-vindos. A noite está linda. Montem seu tablado e encenem
para mim seu melhor espetáculo. Gamaliel Ratsey vai assisti-los. Isto não
é comum. Poderão contar o fato por aí.

--- Vamos precisar de luz --- disseram timidamente os dois Jeffes.

--- Luz? --- disse nobremente Gamaliel --- Estão me falando em luz? Sou aqui o rei
Gamaliel, como Elizabeth é rainha na cidade. E vos tratarei como rei. Aqui
estão quarenta shillings.

Os atores desceram, tremendo.

--- Para deleite de Vossa Majestade --- disse Bird ---, o que devemos encenar?

Gamaliel refletiu, e olhou para Gabriel.

--- Ora --- disse ele ---, uma linda peça para esta donzela, e bem melancólica. Ela
deve ficar encantadora de Ofélia. Há flores de dedaleira aqui perto;
legítimos dedos da morte. Amlet, é isso que quero. Gosto dos humores desta
composição. Se eu não fosse Gamaliel, de bom grado faria o papel de Amlet.
Vamos, e não errem seus passes de esgrima, meus excelentes troianos, meus
valentes coríntios!

Acenderam as lanternas. Gamaliel assistiu o drama com atenção. Ao final,
disse a Gabriel Spenser:

--- Bela Ofélia, dispenso-a do cumprimento. Podem partir, atores do rei
Gamaliel. Sua Majestade está satisfeita.

Em seguida se esvaneceu na escuridão.

Quando, chegada a aurora, o carro se punha em movimento, viram-no mais uma
vez a barrar o caminho, pistola em punho.

--- Gamaliel Ratsey, salteador --- disse ele ---, veio recuperar os quarenta
shillings do rei Gamaliel. Vamos, depressa. Obrigado pelo espetáculo. Os
humores de Amlet de fato me agradam infinitamente. Bela Ofélia, aceite
minhas saudações.

Os dois Jeffes, que guardavam o dinheiro, devolveram-no por força. Gamaliel
saudou e partiu a galope.

Depois dessa aventura, a trupe voltou para Londres. Contaram que um ladrão
quase raptara Ofélia de vestido e peruca. Uma mulher chamada Pat King, que
vinha seguindo ao Rideau Vert, declarou que isto não a surpreendia. Era
corpulenta, tinha cintura redonda. Flum a convidou para que conhecesse
Gabriel. Ela o achou bonitinho e o beijou carinhosamente. Depois, voltou
várias vezes. Pat era amiga de um tijoleiro, entediado com seu
ofício, que nutria a ambição de atuar no Rideau Vert. Chamava-se Ben
Jonson, e tinha muito orgulho de sua educação, sendo letrado e tendo
algumas noções de latim. Era um homem alto e forte, com marcas de
escrófulas e cujo olho direito era mais alto que o esquerdo. Tinha a voz
forte e troante. Aquele colosso tinha sido soldado nos Países Baixos.
Seguiu Pat King, agarrou Gabriel pela pele do pescoço e o arrastou até os
campos de Hoxton, onde o pobre Gabriel teve de enfrentá-lo, espada na mão.
Flum lhe passara secretamente uma lâmina dez polegadas maior. A lâmina
penetrou no braço de Ben Jonson. Gabriel teve o pulmão transpassado.
Morreu ali na relva. Flum correu a buscar os oficiais de polícia. Levaram
Ben Jonson, blasfemando, para Newgate. Flum tinha esperança de que ele
fosse enforcado. Mas ele recitou os salmos em latim, mostrou que era
letrado, e marcaram-lhe simplesmente a mão com ferro em brasa.

\chapter{Pocahontas princesa}

Pocahontas era filha do rei Powhatan, que se assentava num trono feito à
maneira de um leito, vestindo uma longa túnica costurada com peles de
ratinhos, com as caudas todas pendentes. Foi criada numa casa forrada com
esteiras, entre sacerdotes e mulheres que tinham a cabeça e os ombros
pintados de vermelho vivo e a entretinham com argolas de cobre e sininhos
de serpentes. Namontak, um servo fiel, velava pela princesa e organizava
suas brincadeiras. Levavam-na às vezes até a floresta, junto ao grande rio
Rappahanok, e trinta virgens nuas dançavam para distraí-la. Eram pintadas
com cores diversas e cingidas com folhas verdes, levavam na cabeça chifres
de bode, na cintura uma pele de lontra e, brandindo clavas, pulavam ao
redor de uma fogueira crepitante. Finda a dança, dispersavam as chamas e
reconduziam a princesa para casa à luz dos tições.

No ano de 1607, a terra de Pocahontas foi perturbada pelos europeus.
Fidalgos desbancados, escroques e caçadores de ouro atracaram no rio de
Potomac e erigiram cabanas de madeira. Deram às cabanas o nome de
Jamestown, e chamaram Virgínia a sua colônia. A Virgínia não passava,
naqueles anos, de um pequeno e mísero forte construído na baía de
Chesapeake, no meio dos domínios do grande rei Powhatan. Os colonos
elegeram para presidente o capitão John Smith, que outrora se aventurara
até a terra dos turcos. Perambulavam pelas rochas e viviam de mariscos e
do pouco trigo rijo que conseguiam angariar no tráfico com os indígenas.

De início, foram recebidos com grande cerimônia. Um sacerdote selvagem,
cabelos presos cingidos por uma coroa de pelo de gamo tinta de vermelho e
aberta como uma rosa, veio tocar diante deles uma flauta de junco. Seu
corpo era pintado de carmim, seu rosto de azul; tinha a pele salpicada de
lantejoulas de prata nativa. Assim, semblante impassível, sentou-se numa
esteira e fumou seu cachimbo de tabaco.

A seguir, homens se perfilaram em formação quadrada, pintados de preto,
vermelho e branco, alguns de meias-cores, cantando e dançando diante de
Oki, seu ídolo, feito de peles de serpentes enchidas com musgo e ornadas
com correntes de cobre.

Mas, poucos dias depois, quando explorava o rio numa canoa, o capitão Smith
foi subitamente assaltado e amarrado. Levaram-no em meio a gritos
terríveis para uma casa comprida onde foi vigiado por quarenta selvagens.
Os sacerdotes, com seus olhos pintados de vermelho e corpos pretos
rasgados por grandes riscos brancos, circundaram duas vezes o fogo da casa
de guarda com um rastro de farinha e grãos de trigo. John Smith foi então
levado até a choupana do rei. Powhatan vestia seu traje de peles e os que
o rodeavam tinham os cabelos ornados com penugem de pássaros. Uma mulher
trouxe água ao capitão para lavar-lhe as mãos, e outra enxugou-as com um
chumaço de penas. Dois gigantes vermelhos, entretanto, depositaram duas
pedras chatas aos pés de Powhatan. O rei ergueu a mão, indicando que John
Smith seria deitado naquelas pedras, e sua cabeça, esmagada a golpes de
clava.

Pocahontas tinha apenas doze anos e apontava timidamente o rosto entre os
conselheiros lambuzados. Ela gemeu, correu para o capitão e encostou a
cabeça em sua face. John Smith tinha vinte e nove anos. Usava grandes
bigodes retos, a barba em leque, e seu perfil era aquilino. Disseram-lhe
que o nome da menina do rei, que lhe salvava a vida, era Pocahontas. Mas
esse não era seu nome verdadeiro. O rei Powhatan concluiu a paz com John
Smith e o deixou ir em liberdade.

Um ano mais tarde, o capitão Smith acampava com sua tropa na floresta
fluvial. Era uma noite densa; uma chuva penetrante abafava qualquer ruído.
Súbito, Pocahontas tocou o ombro do capitão. Atravessara, sozinha, as
trevas terríveis da mata. Sussurrou-lhe que seu pai tencionava atacar os
ingleses e matá-los quando estivessem jantando. Rogou-lhe que fugisse,
caso quisesse viver. O capitão lhe ofereceu fitas e vidrarias; ela, porém,
chorou e respondeu que não ousava. E escapuliu, sozinha, floresta adentro.

No ano seguinte, o capitão Smith caiu em desgraça junto aos colonos e, em
1609, foi embarcado para a Inglaterra. Lá, escreveu livros sobre a
Virgínia, nos quais explicava a situação dos colonos e relatava suas
aventuras. Por volta de 1612, um certo capitão Argall, indo fazer comércio
com os Potomac (que eram o povo do rei Powhatan), raptou de surpresa a
princesa Pocahontas e prendeu-a em seu navio como refém. O rei, seu pai,
se indignou; mas ela não lhe foi devolvida. Languesceu, assim,
prisioneira, até o dia em que um fidalgo bem apessoado, John Rolfe,
apaixonou-se por ela e a desposou. O casamento realizou-se em abril de
1613. Dizem que Pocahontas confessou seu amor a um de seus irmãos, que foi
visitá-la. Ela chegou no mês de junho de 1616 à Inglaterra, onde se fez,
entre as pessoas da sociedade, uma grande curiosidade por visitá-la. A boa
rainha Ana acolheu-a ternamente e mandou gravar um retrato seu.

O capitão John Smith, que estava para voltar à Virgínia, veio prestar-lhe
cumprimentos antes de embarcar. Não a via desde 1608. Ela estava com vinte
e dois anos. Quando ele entrou, ela virou a cabeça e escondeu o rosto, sem
responder nem ao marido, nem aos amigos, e se quedou sozinha durante duas,
três horas. Então mandou chamar o capitão. Levantou os olhos e disse-lhe:

--- O senhor prometeu a Powhatan que aquilo que fosse seu seria dele, e ele
fez o mesmo; quando estrangeiro em sua pátria, o senhor o chamava de
\textit{pai}; sendo estrangeira na sua, assim é que vou chamá-lo.

O capitão Smith escusou-se, alegando que ela era filha de um rei.

Ela prosseguiu:

--- O senhor não temeu ir à terra de meu pai, e o assustou, a ele e a toda a
sua gente, exceto a mim: vai temer, então, que eu aqui o chame de
\textit{meu pai}? Eu lhe direi \textit{meu pai} e o senhor me dirá
\textit{minha filha}, e serei para sempre da mesma pátria que o senhor\ldots{}
Lá, haviam me dito que o senhor estava morto\ldots{}

E segredou a John Smith que seu nome era Matoaka. Os índios, temendo que se
apossassem dela por algum malefício, tinham fornecido aos estrangeiros o
falso nome de Pocahontas.

John Smith partiu para a Virgínia e nunca tornou a ver Matoaka. Ela adoeceu
em Gravesend no início do ano seguinte, empalideceu e morreu. Ainda não
tinha vinte e três anos.

No exergo de seu retrato consta a inscrição: \textit{Matoaka alias Rebecca
filia potentissimi principis Powhatani imperatoris Virginae}. Nele a pobre
Matoaka usava um chapéu de feltro alto, com duas grinaldas de pérolas; um
cabeção de renda engomada, e segurava um leque de plumas. Tinha o rosto
emagrecido, as faces encovadas e imensos olhos doces.

\chapter{Cyril Tourneur poeta trágico}

Cyril Tourneur nasceu da união entre um deus desconhecido e uma prostituta.
Tem-se a prova de sua origem divina no ateísmo heroico sob o qual
sucumbiu. Sua mãe transmitiu-lhe o instinto da revolução e da luxúria, o
medo da morte, o frêmito da volúpia e o ódio pelos reis; do pai, herdou o
gosto de se coroar, o orgulho de reinar e a alegria de criar; ambos lhe
deram o amor pela noite, pela luz vermelha e pelo sangue.

Ignora-se a data de seu nascimento; mas ele surgiu num dia escuro, num ano
pestilencial.

Nenhuma proteção celeste zelou pela moça apaixonada que ficou grávida de um
deus, já que seu corpo apareceu manchado de peste alguns dias antes de ela
parir, e a porta de sua casinha foi marcada com uma cruz vermelha. Cyril
Tourneur veio ao mundo ao som do sino do sepultador dos mortos; e assim
como seu pai sumira no céu comum aos deuses, uma carreta verde transportou
sua mãe para a fossa comum dos homens. Contam que as trevas eram tão
profundas que o sepultador teve de alumiar a entrada da casa pestilenta
com uma tocha de resina; outro cronista afirma que a névoa sobre o Tâmisa
(em que mergulhava a base da casa) se raiou de escarlate, e que da goela
do sino de chamada irrompeu a voz dos cinocéfalos; por fim, parece não
haver dúvida de que uma estrela flamejante e furiosa manifestou-se acima
do triângulo do teto, toda de raios fuliginosos, tortos, mal atados, e que
o menino recém-nascido mostrou-lhe o punho por uma lucarna enquanto ela
sacudia sobre ele seus cachos informes de fogo. Assim ingressou Cyril
Tourneur na vasta concavidade da noite ciméria.\footnote{ Referência aos
versos de Homero (\textit{Odisseia} \versal{XI}, 14): ``Nessa paragem
se encontra a cidade dos homens cimérios, que se acham sempre envolvidos
por nuvens e brumas espessas''.}

Não há como descobrir o que ele fez ou pensou até a idade de trinta anos,
quais foram os sintomas de sua divindade latente, como se convenceu da
própria realeza. Um registro obscuro e assustado traz a lista de suas
blasfêmias. Afirmava que Moisés não passara de um malabarista e que um
certo Erioto era mais hábil que ele. Que o princípio primeiro da religião
não era mais que manter os homens no terror. Que Cristo merecia mais a
morte que Barrabás, muito embora Barrabás fosse ladrão e assassino. Que se
acaso empreendesse escrever uma nova religião, ele a fundamentaria em
método mais excelente e admirável, e que o Novo Testamento tinha um estilo
repulsivo. Que tinha tanto direito de cunhar moedas quanto a rainha da
Inglaterra, e que conhecia um certo Poole, prisioneiro em Newgate, muito
entendido na mescla dos metais, com cuja ajuda tencionava um dia amoedar o
ouro com sua própria imagem. Uma alma pia riscou, no pergaminho, outras
afirmações mais terríveis.

Tais palavras, porém, foram recolhidas por uma pessoa vulgar. Os gestos de
Cyril Tourneur indicam um ateísmo mais vingativo. É representado vestindo
uma longa túnica preta, trazendo na cabeça uma gloriosa coroa de doze
estrelas, com o pé sobre o globo celeste, erguendo o globo terrestre na
mão direita. Ele percorria as ruas nas noites de peste e tempestade. Era
pálido como os círios consagrados e seus olhos luziam baçamente como os
incensórios. Afirmam alguns que tinha marcado no flanco direito um sinete
extraordinário; mas não foi possível verificá-lo após sua morte, pois
ninguém viu seus despojos.

Tomou por amante uma prostituta do Bankside, que frequentava as ruas de
beira d’água, e amou-a exclusivamente. Era ela muito jovem, e seu
semblante, inocente e loiro. Os rubores nele surgiam como chamas
vacilantes. Cyril Tourneur deu-lhe o nome de Rosamunda, e teve dela uma
filha que amou. Rosamunda morreu tragicamente, depois de ser notada por um
príncipe. Sabe-se que bebeu em taça transparente um veneno cor de
esmeralda.

Foi então que na alma de Cyril a vingança veio mesclar-se ao orgulho.
Noturno, ele percorreu o passeio público, ao longo de todo o cortejo real,
chacoalhando na mão uma tocha inflamada, de modo a iluminar o príncipe
envenenador. O ódio a toda autoridade subiu-lhe até a boca e as mãos.
Tornou-se espião de estrada, não para roubar, e sim para assassinar os
reis. Os príncipes que desapareceram naqueles tempos foram iluminados pela
tocha de Cyril Tourneur e mortos por ele.

Emboscava-se nos caminhos da rainha, próximo aos poços de saibro e fornos
de cal. Escolhia sua vítima em meio à tropa, oferecia-se para alumiar seus
passos entre os atoleiros, conduzia-a até a boca do poço, apagava a tocha
e empurrava. Chovia o saibro depois da queda. Em seguida, debruçado na
borda, Cyril derrubava duas pedras enormes para esmagar os gritos. E pelo
resto da noite velava o cadáver que se consumia na cal, próximo ao forno
vermelho escuro.

Quando saciou seu ódio aos reis, Cyril Tourneur foi tomado pelo ódio aos
deuses. O impulso divino que trazia em si o incitou a criar. Refletiu que
poderia fundar uma geração de seu próprio sangue, e propagar-se feito deus
sobre a terra. Olhou para a sua filha, achou-a virgem e desejável. Para
cumprir seu intento à face dos céus, não encontrou lugar mais
significativo que um cemitério. Jurou afrontar a morte e criar uma nova
humanidade em meio à destruição determinada por ordens divinas. Cercado de
velhos ossos, quis gerar ossos jovens. Cyril Tourneur possuiu sua filha
sobre o tampo de um ossuário.

O final de sua vida se perde num refulgir obscuro. Não se sabe por que mãos
nos foi transmitida a \textit{Tragédia do ateu} e a \textit{Tragédia do
vingador}. Uma tradição afirma que o orgulho de Cyril Tourneur cresceu
ainda mais. Mandou erguer um trono em seu jardim sombrio, e costumava
sentar-se ali, coroado de ouro, sob o relâmpago. Muitos foram os que o
viram e fugiram, aterrorizados pelas longas faíscas azuladas que
esvoaçavam sobre sua cabeça. Lia um manuscrito dos poemas de Empédocles,
que ninguém mais viu desde então. Expressou amiúde sua admiração pela
morte de Empédocles. E o ano em que ele desapareceu foi novamente
pestilento. O povo de Londres se refugiara nas barcas ancoradas no meio do
Tâmisa. Um meteoro assustador deslocou-se sob a lua. Era um globo de fogo
branco, movido por sinistra rotação. Rumou para a casa de Cyril Tourneur,
que pareceu pintada por reflexos metálicos. O homem vestido de preto e
coroado de ouro aguardava, em seu trono, a vinda do meteoro. Houve, como
antes das batalhas teatrais, um melancólico alarme de trombetas. Cyril
Tourneur foi envolto num clarão de sangue rosa volatilizado. Como no
teatro, trombetas erguidas na noite soaram uma fúnebre fanfarra. Assim foi
Cyril Tourneur lançado a um deus desconhecido no taciturno remoinho do
céu.

\chapter{William Phips pescador de tesouros}

William Phips nasceu em 1651 perto da embocadura do rio Kennebec, entre as
florestas fluviais onde os construtores de navios vinham buscar sua
madeira. Numa pobre aldeia do Maine ele sonhou, pela primeira vez, uma
aventurosa fortuna, ao contemplar a feitura das tábuas marinhas. A incerta
claridade do oceano que bate na Nova Inglaterra trouxe o cintilar do ouro
afogado e da prata abafada sob as areias. Acreditou na riqueza do mar e
desejou obtê-la. Aprendeu a construir barcos, angariou alguma fortuna e se
foi para Boston. Tão forte era a sua fé que repetia: “Ainda hei de
comandar uma nau do Rei e ter uma bela casa de tijolos em Boston, na
Avenida Verde”.

Naquele tempo jaziam no fundo do Atlântico muitos galeões espanhóis
carregados de ouro. Tal boato enchia a alma de William Phips. Soube que
uma nau enorme afundara próximo ao Porto de la Plata; reuniu tudo o que
possuía e partiu para Londres a fim de equipar um navio. Assediou o
Almirantado com petições e memorandos. Deram-lhe o \textit{Rose d’Alger},
que portava dezoito canhões e, em 1687, fez-se à vela rumo ao
desconhecido. Tinha trinta e seis anos.

Noventa e cinco homens partiam a bordo do \textit{Rose d’Alger}, entre os
quais um mestre, Adderley, de Providence. Quando souberam que Phips rumava
para Hispaniola,\footnote{ Ilha do Caribe em que se situam a República do
Haiti e a República Dominicana.} não cabiam em si de alegria.
Pois Hispaniola era a ilha dos piratas, e o \textit{Rose d’Alger} parecia
ser um bom navio. Num primeiro momento, num lugar arenoso do arquipélago,
reuniram-se em assembleia para se sagrarem cavalheiros da fortuna. Phips,
à proa do \textit{Rose d’Alger}, espiava o mar. Havia, porém uma avaria na
carena. Enquanto o carpinteiro a consertava, escutou o complô. Acorreu à
cabine do capitão. Phips ordenou-lhe que carregasse os canhões, apontou-os
para a tripulação sublevada em terra, deixou todos os homens “rebeldes”
naquele retiro deserto, e tornou a partir com alguns marujos dedicados. O
mestre de Providence, Adderley, voltou a nado para o \textit{Rose
d’Alger}. Chegaram na Hispaniola por um mar calmo, sob um sol ardente.
Phips indagou por todas as praias a respeito da nau que soçobrara mais de
meio século atrás, à vista do Porto de la Plata. Um velho espanhol ainda
se lembrava e apontou para o recife. Era um escolho comprido, arredondado,
cujos declives sumiam na água clara até o mais fundo tremor. Adderley,
debruçado sobre o filerete, ria ao fitar os redemoinhos miúdos das ondas.
O \textit{Rose d’Alger} deu lentamente a volta no recife e os homens todos
examinaram em vão o mar transparente. Phips batia o pé no castelo de proa,
entre as dragas e os ganchos. O \textit{Rose d’Alger} deu mais uma vez a
volta no recife, e sempre o solo parecia igual, com sulcos concêntricos de
areia úmida e buquês de algas inclinadas tremulando na correnteza. Quando
o \textit{Rose d’Alger} iniciou sua terceira volta o sol afundou e o mar
se fez negro.

Então ficou fosforescente. “Ali estão os tesouros!”, exclamava Adderley na
escuridão, dedo estendido para o dourado turvo das ondas. Mas enquanto o
\textit{Rose d’Alger} ainda percorria o mesmo orbe, ergueu-se a aurora
quente sobre o oceano calmo e claro. E durante oito dias seguiu singrando
assim. Os olhos dos homens se anuviavam de tanto perscrutar a limpidez do
mar. Phips já não tinha provisões. Era preciso partir. A ordem foi dada, e
o \textit{Rose d’Alger} encetou a viragem. Foi quando Adderley vislumbrou,
num flanco do recife, uma linda alga branca que vacilava, e a quis para
si. Um índio mergulhou e a arrancou. Trouxe-a, pendendo na vertical. Era
muito pesada, e suas raízes retorcidas pareciam envolver um calhau.
Adderley sopesou-a e bateu as raízes no tombadilho para livrá-la de seu
peso. Algo reluzente rolou sob o sol. Phips deu um grito. Era um lingote
de prata que devia valer umas trezentas libras. Adderley balançava
estupidamente a alga branca. Imediatamente mergulharam todos os índios. Em
poucas horas, o convés ficou coberto de sacos duros, petrificados,
incrustados de calcário e revestidos de conchinhas. Foram estripados com
escopros e martelos; e pelos rasgões surgiam lingotes de ouro e prata, e
moedas de oito.\footnote{ Moeda de oito, ou \textit{peso duro}: moeda de
prata com valor de oito reais, muito difundida no século~\versal{XVIII} durante a
colonização espanhola.} “Deus seja louvado!'', exclamou Phips, ``nossa
fortuna está feita!” O tesouro valia trezentas mil libras esterlinas.
Adderley só repetia: “E tudo isso saiu da raiz de uma alguinha branca!”.
Morreu louco nas Bermudas, alguns dias depois, balbuciando essas palavras.

Phips transportou o seu tesouro. O rei da Inglaterra o tornou sir William
Phips, e nomeou-o High Sheriff em Boston. Lá ele concretizou sua quimera e
mandou construir uma linda casa de tijolos vermelhos na Avenida Verde.
Tornou-se um homem importante. Foi quem comandou a campanha contra as
possessões francesas, e arrebatou a Acádia ao sr.~de Meneval e ao
cavalheiro de Villebon. O rei nomeou-o governador de Massachussets,
capitão geral do Maine e da Nova Escócia. Seus cofres estavam abarrotados
de ouro. Ele empreendeu o ataque ao Quebec, depois de levantar todo o
dinheiro disponível em Boston. O empreendimento falhou e a colônia saiu
arruinada. Então Phips emitiu papel-moeda. A fim de elevar seu valor,
trocou por este papel todo o seu ouro líquido. Mas a sorte havia virado. A
cotação do papel caiu. Phips perdeu tudo, ficou pobre, endividado, e seus
inimigos o espreitavam. Sua prosperidade só tinha durado oito anos.
Partiu, miserável, para Londres e ao desembarcar foi detido pela quantia
de vinte mil libras, a pedido de Dudley e Brenton. Os sargentos o levaram
para a prisão de Fleet.

Sir William Phips foi encarcerado numa cela desnuda. Só lhe restava o
lingote de prata que lhe trouxera a glória, o lingote da alga branca.
Estava exaurido de febre e desespero. A morte o apanhou pela garganta. Ele
se debateu. Mesmo nessa hora, ainda era assombrado por seu sonho de
tesouros. O galeão do governador espanhol Bobadilla, carregado de ouro e
prata, naufragara próximo às Bahamas. Phips mandou buscar o diretor da
prisão. A febre e a esperança furiosa tinham-no descarnado. Mostrou ao
diretor o lingote de prata em sua mão ressecada e murmurou num estertor:

--- Deixe-me mergulhar; veja, esse é um dos lingotes de Bo-ba-dil-la.

Então expirou. O lingote da alga branca pagou o seu caixão.

\chapter{Capitão Kid pirata}

Não há um consenso sobre por que motivo foi dado a esse pirata o nome do
cabrito (\textit{Kid}). O ato pelo qual Guilherme~\versal{III}, rei da Inglaterra,
investiu-o em suas funções na galera \textit{Aventura}, em 1695, começa
com as palavras: “Ao nosso leal e bem-amado capitão William Kid,
comandante etc. Saudações”. O certo é que, desde então, tratou-se de um
nome de guerra. Dizem alguns que, sendo elegante e refinado, ele tinha o
hábito de sempre usar, no combate e na manobra, delicadas luvas de cabrito
com barra de renda de Flandres; outros afirmam que, em suas piores
matanças, exclamava: “Eu, que sou doce e bom feito um cabrito
recém-nascido”; outros ainda garantem que ele guardava o ouro e as joias
em sacos muito flexíveis, feitos de pele de cabra nova, e que tal uso
datava do dia em que pilhou uma nau carregada de mercúrio com que encheu
mil bolsas de couro, ainda hoje enterradas na encosta de uma pequena
colina nas Ilhas Barbados. Basta saber que seu pavilhão de seda preta
trazia bordadas uma caveira e uma cabeça de cabrito, e assim também era
gravado seu sinete. Todos os que buscam os muitos tesouros que ele
escondeu nas costas dos continentes da Ásia e da América levam à sua
frente um cabritinho preto, que supostamente deveria gemer no lugar onde o
capitão enterrou seu butim; mas nenhum deles conseguiu. O próprio Barba
Negra, que tinha informações por Gabriel Loff, um antigo marujo de Kid,
não encontrou nas dunas, sobre as quais se ergue hoje Fort Providence,
mais que gotas esparsas de mercúrio ressudando na areia. E são inúteis
todas essas buscas, tendo o capitão Kid afirmado que seus esconderijos
seriam eternamente ignorados por causa do “homem da selha ensanguentada”.
Kid foi, de fato, assombrado a vida inteira por este homem, e os tesouros
de Kid, após sua morte, têm sido assombrados e defendidos por ele.

Lorde Bellamont, governador de Barbados, exasperado pelo imenso butim dos
piratas nas Índias Ocidentais, equipou a galera \textit{Aventura} e obteve
do rei, para o capitão Kid, a patente de comandante. Kid de havia muito
invejava o célebre Ireland que pilhava todos os comboios; prometeu a lorde
Bellamont tomar sua chalupa e trazê-lo com seus companheiros para serem
executados. A \textit{Aventura} levava trinta canhões e cinquenta homens.
Kid aportou primeiramente em Madeira e abasteceu-se de vinho; depois, em
Bonavista, para carregar sal; por fim, em Santiago, onde fez provisão
completa. E de lá fez-se à vela rumo à entrada do Mar Vermelho onde, no
Golfo Pérsico, há um lugar numa pequena ilha chamado Chave de Bab.

Foi lá que o capitão Kid reuniu seus companheiros e mandou que içassem o
pavilhão negro com a caveira. Juraram todos, sobre o machado, obediência
absoluta ao regulamento dos piratas. Cada homem tinha direito a voto, e
igual direito aos mantimentos frescos e licores fortes. Eram proibidos os
jogos de baralho e dados. As luzes e candeias deviam ser apagadas às oito
horas da noite. Se um homem quisesse beber mais até mais tarde, que
bebesse no tombadilho, no escuro, a céu aberto. A equipe não receberia
mulheres nem rapazes. Quem introduzisse um deles sob disfarce seria punido
de morte. Os canhões, pistolas e facões tinham de ser conservados e
polidos. As diferenças seriam resolvidas em terra, com sabre e pistola. O
capitão e o quartel-mestre teriam direito a duas partes; o mestre, o
contramestre e o artilheiro, a uma e meia; os outros oficiais a uma e um
quarto. Descanso para os músicos no dia de Sabá.

O primeiro navio com que cruzaram era holandês, comandado pelo
\textit{Schipper} Mitchel. Kid içou o pavilhão francês e deu caça. O navio
exibiu em seguida as cores francesas; com o quê o pirata interpelou em
francês. O \textit{Schipper} tinha um francês a bordo, que respondeu. Kid
perguntou-lhe se tinha um passaporte. O francês disse que sim: “Pois em
virtude de seu passaporte, respondeu Kid, por Deus, considero-o capitão
deste navio”. E imediatamente mandou enforcá-lo na verga. Então, um por
um, mandou vir os holandeses. Interrogou-os e, fingindo não entender
flamengo, ordenou a cada prisioneiro: “Francês --- a prancha!”. Amarraram uma
prancha no botaló. Os holandeses todos, nus, correram para cima dela ante
a ponta do facão do contramestre, e saltaram no mar.

Nisso, Moor, o artilheiro do capitão Kid, ergueu a voz: “Capitão'', gritou,
``por que está matando estes homens?”. Moor estava bêbado. O capitão se virou
e, empunhando uma selha, bateu-lhe na cabeça. Moor tombou, com o crânio
rachado. O capitão Kid mandou lavar a selha, em que se grudavam cabelos
com sangue coagulado. Nunca mais homem nenhum da tripulação quis mergulhar
o lambaz naquela selha. Deixaram-na amarrada no filerete.

Daquele dia em diante, o capitão Kid foi assombrado pelo homem da selha.
Quando tomou a nau moura \textit{Queda}, tripulada por indianos e
armênios, com dez mil libras de ouro, no momento da partilha do butim o
homem da selha ensanguentada estava sentado sobre os ducados. Kid o
avistou, e praguejou. Desceu para sua cabine e emborcou uma caneca de
aguardente. Em seguida, de volta ao tombadilho, mandou jogar a velha selha
ao mar. Ao abordarem a rica nau mercante \textit{Mocco}, não encontravam
com o que medir as partes de ouro em pó do capitão. “Uma selha cheinha”,
disse uma voz por trás do ombro de Kid. Ele açoitou o ar com seu facão e
enxugou os lábios, que espumavam. Então mandou enforcar os armênios. Os
homens da tripulação não pareciam ter ouvido nada. Quando Kid atacou o
\textit{Hirondelle}, deitou-se em seu beliche depois da partilha. Ao
acordar, sentiu-se banhado em suor e chamou um marujo pedindo o necessário
para lavar-se. O homem lhe trouxe água numa bacia de estanho. Kid o
encarou e berrou: “São esses os modos de um cavalheiro de fortuna?
Miserável! Trouxeste uma selha cheia de sangue!”. O marujo escapuliu. Kid
mandou desembarcá-lo e o abandonou com um fuzil, uma garrafa de pólvora e
uma garrafa de água. Não teve outro motivo para enterrar seu butim em
diferentes lugares ermos, entre as areias, se não sua certeza de que toda
noite o artilheiro assassinado vinha com sua selha esvaziar o paiol do
ouro e jogar as riquezas ao mar.

Kid foi apanhado ao largo de Nova York. Lorde Bellamont o despachou para
Londres. Foi condenado ao patíbulo. Enforcaram-no no cais da Execução, com
sua roupa vermelha e suas luvas. No momento em que o carrasco lhe passava
o capuz preto sobre os olhos, o capitão Kid se debateu e gritou: “Santo
Deus! Eu sabia que ele ainda ia me enfiar a selha na cabeça!” O cadáver
pretejado ficou enganchado nas correntes por mais de vinte anos.

\chapter{Walter Kennedy pirata iletrado}

O capitão Kennedy era irlandês e não sabia ler nem escrever. Chegou ao
posto de tenente, sob o comando do grande Roberts, pelo talento que tinha
para a tortura. Dominava perfeitamente a arte de torcer uma mecha em volta
da testa do prisioneiro até que os olhos lhe saltassem para fora, ou de
afagar-lhe o rosto com folhas de palmeira inflamadas. Sua reputação
consagrou-se no julgamento de Darby Mullin, suspeito de traição, realizado
a bordo do \textit{Corsário}. Os juízes sentaram-se junto à bitácula do
timoneiro, frente a uma tigela grande de ponche, cachimbos e fumo; então
teve início a sessão. Estavam para votar a sentença, quando um dos juízes
propôs fumarem mais um cachimbo antes da deliberação. Então Kennedy se
levantou, tirou o cachimbo da boca, cuspiu, e falou nestes termos:

--- Santo Deus! Senhores e cavalheiros de fortuna, o diabo que me carregue se
não enforcarmos Darby Mullin, meu velho companheiro. Darby é um bom moço,
santo Deus! Dane-se quem disser o contrário, e somos cavalheiros, que
diabo! Remamos juntos, Santo Deus! E, raios, gosto dele, de coração!
Senhores e cavalheiros de fortuna, eu o conheço bem; é um verdadeiro
velhaco; se viver, nunca que irá se arrepender; o diabo que me carregue se
ele se arrepender, não é, Darby, meu velho? Vamos enforcá-lo, Santo Deus! 
E se me permitem os ilustres presentes, vou beber um bom trago à sua
saúde.

Tal discurso pareceu admirável e digno das mais nobres preleções militares
relatadas pelos mais antigos. Roberts ficou encantado. Desse dia em
diante, Kennedy descobriu a ambição. Ao largo de Barbados, tendo Roberts
se perdido numa chalupa ao perseguir uma nau portuguesa, Kennedy obrigou
seu companheiros a elegê-lo capitão do \textit{Corsário}, e fez-se à vela
por conta própria. Afundaram e pilharam inúmeros bergantins e galeras,
carregados de açúcar e fumo do Brasil, sem contar o pó de ouro e os sacos
cheios de dobrões e moedas de oito. Sua bandeira era de seda preta, com
uma caveira, uma ampulheta, dois ossos cruzados e, embaixo, um coração
encimado por um dardo do qual escorriam três gotas de sangue. Assim
equipados, cruzaram com uma pacífica chalupa da Virgínia, cujo capitão era
um piedoso quacre chamado Knot. Aquele homem de Deus não trazia a bordo
nem rum, nem pistolas, nem sabres, nem facão; vestia um comprido traje
preto e usava um chapéu de abas largas da mesma cor.

--- Santo Deus! --- disse o capitão Kennedy --- É um \textit{bon-vivant}, e alegre;
é disso que eu gosto; não vamos fazer mal ao meu amigo, o senhor capitão
Knot, que se veste de forma tão divertida.

O sr.~Knot inclinou-se, com silenciosos trejeitos.

--- Amém --- disse o sr.~Knot. --- Assim seja.

Os piratas presentearam o sr.~Knot. Ofereceram-lhe trinta moedas de ouro,
dez rolos de fumo do Brasil e saquinhos de esmeraldas. O sr.~Knot recebeu
muito bem as moedas de ouro, as pedras preciosas e o fumo.

--- São presentes que é lícito aceitar, para fazer deles um uso piedoso. Ah!
Quisera o céu que nossos amigos que singram os mares fossem todos animados
por tais sentimentos! O Senhor aceita todas as restituições. São, por
assim dizer, os membros do terneiro, e as partes do ídolo Dagon\footnote{
Antigo deus semita da fertilidade e da pesca.} que estão, meus
amigos, lhe oferecendo em sacrifício. Dagon ainda reina nestes países
profanos, e seu ouro traz más tentações.

--- Que Dagon o quê --- disse Kennedy ---, cale a boca, santo deus! Receba o que
lhe dão e tome um trago.

E o sr.~Knot se inclinou serenamente. Recusou, porém, sua dose de rum.

--- Senhores, meus amigos --- disse ele\ldots{}

--- Cavalheiros de fortuna, santo deus! --- exclamou Kennedy.

--- Senhores cavalheiros, meus amigos --- retomou o sr.~Knot ---, os licores fortes
são, por assim dizer, aguilhões de tentação que nossa carne fraca não
saberia suportar. Vocês, meus amigos\ldots{}

--- Cavalheiros de fortuna, santo deus! --- exclamou Kennedy.

--- Vocês, meus amigos e afortunados cavalheiros --- retomou o sr.~Knot ---,
tarimbados que são por longas provações contra o Tentador, possivelmente,
provavelmente eu diria, não sofram seus inconvenientes; mas seus amigos
ficariam incomodados, gravemente incomodados\ldots{}

--- Incomodados o diabo! --- disse Kennedy. --- Este homem fala admiravelmente, mas
eu bebo melhor. Ele vai nos levar à Carolina para ver seus excelentes
amigos, os quais decerto possuem outros membros desse terneiro de que ele
falou. Não é, sr.~capitão Dagon?

--- Assim seja --- disse o quacre ---, mas Knot é o meu nome.

E inclinou-se novamente. As grandes abas do seu chapéu tremulavam ao vento.

O \textit{Corsário} jogou a âncora numa angra preferida pelo homem de Deus.
Prometeu trazer seus amigos e de fato voltou, naquela mesma noite, com uma
companhia de soldados enviados pelo sr.~Spotswood, governador da Carolina.
O homem de Deus jurou aos seus amigos, os cavalheiros de fortuna, que isso
era apenas para impedir que introduzissem naquelas regiões profanas seus
tentadores licores. E quando os piratas foram detidos:

--- Ah! Meus amigos --- disse o sr.~Knot ---, aceitem todas as mortificações, tal
como eu fiz.

--- Santo deus! Mortificação é a palavra --- praguejou Kennedy.

Foi posto a ferros a bordo de um navio para ser julgado em Londres. Foi
recebido pelo tribunal de Old Baley. Assinou com cruz todos os seus
interrogatórios, colocando a a mesma marca que traçava em seus recibos de
apreensão. Seu derradeiro discurso foi pronunciado no cais da Execução,
onde a brisa do mar balançava os cadáveres de antigos cavalheiros de
fortuna enforcados nas próprias correntes.

--- Santo Deus! É muita honra --- disse Kennedy ---, fitando os enforcados. Vão me
pendurar ao lado do capitão Kid. Já não tem mais os olhos, mas deve ser
ele. Só ele para usar tão rico traje de pano carmim. Kid sempre foi um
homem elegante. E escrevia! Conhecia as letras, diacho! Uma mão tão
bonita! Com licença, capitão. (Saudou o corpo ressecado com traje
carmesim). Mas ele também foi cavalheiro de fortuna.

\chapter{Major Stede Bonnet pirata por propensão}

O major Stede Bonnet era um cavalheiro aposentado do exército que vivia em
suas plantações, na ilha de Barbados, por volta de 1715. Suas lavouras de
cana-de-açúcar e café propiciavam-lhe uma renda, e fumava com prazer o
fumo que ele mesmo cultivava. Tendo sido casado, não fora feliz nessa
união e dizia-se que a mulher lhe perturbara o juízo. Com efeito, sua
mania só se manifestou depois dos quarenta e, de início, seus vizinhos e
criados prestaram-se a ela inocentemente.

A mania do Major Stede Bonnet era a seguinte. Sempre que tinha
oportunidade, depreciava a tática terrestre e louvava a da marinha. Só
sabia falar em Avery, Charles Vane, Benjamim Hornigold e Edward Teach.
Eram, segundo ele, navegadores intrépidos e homens empreendedores.
Singravam, naquele tempo o Mar das Antilhas. Se acontecia de alguém os
chamar de piratas perante o major, ele exclamava:

--- Louvado seja Deus, então, por permitir que esses piratas, como o senhor
diz, deem exemplo da vida honesta e comunitária que levavam nossos
ancestrais. Não existiam então possuidores de riquezas, nem guardiões de
mulheres, nem escravos para produzir o açúcar, o algodão e o índigo; um
deus generoso dispensava todas as coisas e cada um recebia a sua parte.
Eis por que admiro ao extremo os homens livres que dividem seus bens entre
si e juntos levam uma vida de companheiros de fortuna.

Ao percorrer suas plantações, costumava o major bater no ombro de um
trabalhador:

--- Não seria melhor, imbecil, estivar num navio de guerra ou brigantina os
fardos da miserável planta sobre cujas mudas aqui vertes teu suor?

Quase toda noite, o major reunia seus serviçais nos alpendrados e lia para
eles à luz da candeia, enquanto as moscas de cor ruidavam em volta, os
grandes feitos dos piratas da Hispaniola e da Ilha da Tartaruga. Pois
alguns folhetos alertavam as aldeias e fazendas sobre as suas rapinas. 

--- Excelente Vane! --- exclamava o major. --- Bravo Hornigold, legítima cornucópia
repleta de ouro! Sublime Avery, que carregas as joias do grande mongol e
rei de Madagascar! Admirável Teach, que soubeste governar sucessivamente
quatorze mulheres e livrar-se delas, e tiveste a ideia de entregar toda
noite a última (tem apenas dezesseis anos) a teus melhores companheiros
(por pura generosidade, grandeza de alma e ciência do mundo) na tua boa
ilha de Okerecok! Oh, feliz daquele que seguisse os teus caminhos, que
contigo bebesse seu rum, Barba Negra, mestre da \textit{Revanche da Rainha
Ana}!

Discursos esses que os criados do major escutavam com espanto e em
silêncio; e as palavras do major só eram interrompidas pelo leve ruído
abafado das lagartixas, à medida que caíam do telhado, o pavor relaxando
as ventosas de suas patas. Então o major, protegendo a candeia com a mão,
traçava com a bengala, em meio às folhas de fumo, as manobras navais
daqueles grandes capitães, e ameaçava com a \textit{lei de Moisés} (assim
se referem os piratas à bastonada de quarenta golpes) quem não
compreendesse a fineza das evoluções táticas próprias da pirataria.

Por fim, o major Stede Bonnet não conseguiu resistir; e, comprando uma
velha chalupa de dez peças de canhão, equipou-a com tudo quanto convinha à
pirataria, tal como facões, arcabuzes, escadas, pranchas, arpéus,
machados, Bíblias (para prestar juramento), pipas de rum, lanternas,
fuligem para escurecer o rosto, pez, mechas para queimar entre os dedos
dos ricos mercadores e uma quantidade de bandeiras pretas com caveira
branca, dois fêmures cruzados e o nome da nau: a \textit{Revanche}. Então,
mandou de repente subir a bordo setenta de seus criados e fez-se ao mar, à
noite, rumo ao oeste, passando rente a São Vicente, para dobrar o Yucatan
e piratear todas as costas até Savannah (onde jamais chegou).

O major Stede Bonnet nada conhecia das coisas do mar. Começou então a
perder a cabeça entre a bússola e o astrolábio, confundindo artemão com
artilharia, mezena com dezena, botaló com bota-sela, ouvido de caronada
com ouvido de canhão, escotilha com escovilhão, mandando estivar ao invés
de estingar, tão perturbado, em suma, pelo tumulto dos termos
desconhecidos e o movimento inusitado do mar, que até pensou em voltar à
terra de Barbados, não fosse o glorioso desejo de içar a bandeira preta à
vista da primeira nau tê-lo mantido em seu intento. Não embarcara nenhum
mantimento, contando com a pilhagem. Na primeira noite, porém, não
avistaram o menor clarão de algum navio de guerra. O major Stede Bonnet
resolveu então que era preciso atacar uma aldeia.

Alinhando todos os seus homens no tombadilho, distribuiu-lhes facões novos
e exortou-os à maior ferocidade; depois mandou trazer uma selha de fuligem
com a qual escureceu o próprio rosto, ordenando que o imitassem, o que
fizeram, não sem alegria.

Julgando, por fim, segundo lembrava, que convinha estimular a tripulação
com alguma bebida habitual dos piratas, mandou servir a cada homem uma
pinta de rum misturado com pólvora (na falta de vinho, o ingrediente comum
na pirataria). Os criados do major obedeceram; mas, contrariamente ao
esperado, o rosto deles não se inflamou de furor. Avançaram com certa
compostura para bombordo e estibordo e, debruçando os rostos negros por
sobre os fileretes, ofereceram a mistura ao mar celerado. Depois disso,
estando a \textit{Revanche} meio encalhada na costa de São Vicente,
desembarcaram titubeando.

A hora era matinal, e as fisionomias surpresas dos aldeões não suscitavam a
ira. O próprio coração do major não estava com disposição para gritos. De
modo que adquiriu dignamente arroz e legumes secos com porco salgado, que
pagou (à maneira pirata e com grande nobreza, segundo lhe pareceu) com
duas barricas de rum e um calabre velho. Depois disso, lograram os homens,
a muito custo, desencalhar a \textit{Revanche}; e o major Stede Bonnet,
inflado com sua primeira conquista, se fez novamente ao mar.

Navegou o dia inteiro e a noite toda, sem saber que vento o levava. Na
aurora do segundo dia, adormecido junto da bitácula do timoneiro,
desconfortável com seu facão e bacamarte, o major Stede Bonnet foi
acordado por um grito:

--- Ô da chalupa!

E avistou, a umas cento e vinte braças, o botaló de uma nau a
balançar-se. Na proa estava um homem barbudíssimo. Uma bandeirinha preta
flutuava no mastro.

--- Içar o pavilhão de morte! --- exclamou o major Stede Bonnet.

E, lembrando que sua patente era de exército em terra, decidiu no ato
adotar outro nome, seguindo ilustres exemplos. De modo que respondeu sem
qualquer demora:

--- Chalupa \textit{Revanche}, comandada por mim, capitão Thomas, com meus
companheiros de fortuna.

Com o quê o homem barbudo pôs-se a rir:

--- Um feliz encontro, companheiro --- disse ele. --- Podemos navegar juntos. E
venham tomar um trago de rum a bordo da \textit{Revanche da Rainha Ana}.

O major Stede Bonnet imediatamente percebeu que aquele era o capitão Teach,
o Barba Negra, o mais famoso entre todos que admirava. Sua alegria, porém,
foi menor do que teria imaginado. Teve a sensação de que iria perder sua
liberdade de pirata. Taciturno, passou a bordo do navio de Teach, o qual o
recebeu com muita cordialidade, um copo na mão.

--- Companheiro --- disse Barba Negra ---, você me agrada muitíssimo. Mas navega de
modo imprudente. E, se quiser um conselho, capitão Thomas, fique aqui na
nossa boa nau, e eu mando esse bravo homem muito experiente, que se chama
Richards, conduzir sua chalupa; e na nau de Barba Negra terá todo o vagar
para desfrutar livremente a vida dos cavalheiros de fortuna.

O major Stede Bonnet não ousou recusar. Desemba\-raçaram-no de seu facão e do
bacamarte. Prestou juramento sobre o machado (pois Barba Negra não
suportava a visão da Bíblia) e foi definida sua ração de biscoito e rum,
assim como a parte que lhe caberia em pilhagens futuras. O major não tinha
ideia de que a vida dos piratas fosse tão regulamentada. Suportou a fúria
de Barba Negra e os terrores da navegação. Tendo saído de Barbados como
fidalgo para ser pirata segundo sua fantasia, foi forçado a tornar-se um
pirata de fato na \textit{Revanche da Rainha Ana}.

Levou aquela vida por três meses, durante os quais auxiliou seu chefe em
treze pilhagens, e então deu jeito de voltar à sua própria chalupa, a
\textit{Revanche}, sob o comando de Richards. No que se mostrou prudente
já que, na noite seguinte, Barba Negra foi atacado à entrada de sua ilha
de Okerecok pelo tenente Maynard, vindo de Bathtown. Barba Negra foi morto
no combate, e o tenente ordenou que lhe cortassem a cabeça e a amarrassem
no alto de seu gurupés; e assim foi feito.

Entretanto, o pobre capitão Thomas fugiu para a Carolina do Sul e ainda
navegou várias semanas. O governador de Charlestown, avisado de sua
passagem, encarregou o coronel Reth de detê-lo na ilha de Sullivans. O
capitão Thomas deixou-se prender. Foi levado a Charlestown em grande pompa
sob seu nome de major Stede Bonnet, que reassumiu assim que pôde. Foi
deixado no cárcere até 10 de novembro de 1718, quando compareceu perante a
corte do vice-almirantado. O chefe de justiça Nicolas Trot o condenou à
morte com este mui belo discurso:

--- Major Stede Bonnet, o senhor foi julgado por duas acusações de pirataria,
mas sabe que pilhou pelo menos treze naus. De modo que poderia ser acusado
de mais onze delitos; mas dois serão suficientes (disse Nicolas Trot), por
serem eles contrários à lei divina, a qual ordena: \textit{Não furtarás}
(Êxodo 20, 15), e o apóstolo são Paulo declara expressamente que
\textit{os ladrões não herdarão o Reino de Deus} (\versal{I} Coríntios 6, 10). Mas
o senhor é, além disso, culpado de homicídio: e os homicidas (disse
Nicolas Trot) \textit{terão como quinhão o tanque ardente de fogo e
enxofre, a segunda morte} (Apocalipse 21, 8). E quem poderá (disse Nicolas
Trot) \textit{habitar com as labaredas eternas}? (Isaías 33, 14). Ah!
Major Stede Bonnet, tenho pleno motivo para temer que os princípios da
religião com que imbuíram sua juventude (disse Nicolas Trot) estejam muito
corrompidos por sua vida ruim e sua demasiada aplicação à literatura e à
vã filosofia do nosso tempo; pois \textit{se seu prazer tivesse sido a lei
do Eterno} (disse Nicolas Trot) \textit{e} \textit{sobre ela meditasse
dia e noite} (Salmos 1, 2), teria encontrado na\textit{ palavra de Deus
uma lâmpada para os seus pés e uma luz para os seus caminhos} (Salmos 119, 105). 
Contudo, assim não fez. Só resta-lhe então fiar-se no
\textit{Cordeiro de Deus} (disse Nicolas Trot) \textit{que tira o pecado
do mundo} (João 1, 29), \textit{que veio salvar o que estava perdido}
(Mateus 18, 11) e prometeu \textit{não jogar fora aquele que vier a ele}
(João 6, 37). De modo que se quiser retornar a ele, ainda que tarde (disse
Nicolas Trot), como os operários da undécima hora da parábola dos
vinhateiros (Mateus 20, 6-9), ele ainda poderá recebê-lo. Entretanto, esta
corte pronuncia (disse Nicolas Trot) que será conduzido ao local da
execução, onde será enforcado pelo pescoço até a morte.

Tendo escutado, compungido, o discurso do chefe de justiça Nicolas Trot,
foi o major Stede Bonnet enforcado naquele mesmo dia em Charlestown, como
ladrão e pirata.

\chapter[Sr.\,Burke e sr.\,Hare assassinos \bigskip]{Sr.\,Burke e sr.\,Hare assassinos}

Alçou-se o sr.~William Burke da mais baixa condição a uma fama eterna.
Nasceu na Irlanda e começou como sapateiro. Durante vários anos exerceu
este ofício em Edimburgo, onde tornou-se amigo do sr.~Hare, sobre o qual
teve grande influência. Na colaboração entre o sr.~Burke e o sr.~Hare, não
resta dúvida de que a força inventiva e simplificadora pertenceu ao sr.
Burke. Seus dois nomes, porém, permanecem tão inseparáveis na arte como os
nomes de Beaumont e Fletcher. Viveram juntos, trabalharam juntos e foram
presos juntos. O sr.~Hare nunca protestou contra a preferência popular que
se ateve em especial à pessoa do sr.~Burke. Tão completo desprendimento
não foi recompensado. Foi o sr.~Burke quem legou seu nome ao procedimento
peculiar que tornou conhecidos os dois colaboradores. O monossílabo
\textit{burke}\footnote{ O termo \textit{burking} entrou para a língua
inglesa para designar morte por sufocamento mediante forte pressão no peito.} 
ainda estará vivo na boca dos homens muito tempo depois de
a pessoa de Hare cair no esquecimento que injustamente se estende sobre os
trabalhadores obscuros.

O sr.~Burke parece ter colocado em sua obra a fantasia feérica da ilha
verde onde nascera. Sua alma deve ter se impregnado dos relatos do
folclore. Há no que ele fez como que um remoto sabor das \textit{Mil e uma
noites}. Qual o califa errante nos jardins noturnos de Bagdá, desejou
misteriosas aventuras, sendo curioso por relatos desconhecidos e pessoas
estrangeiras. Qual o alto escravo negro armado de pesada cimitarra, não
achou mais digno desfecho para sua volúpia que a morte dos outros. Sua
originalidade anglo-saxônica, porém, consistiu no fato de ele saber tirar
o mais prático proveito dos divagares de sua imaginação celta. Quando seu
gozo artístico terminava, o que fazia o escravo negro, pois não, com
aqueles a quem cortara a cabeça? Com uma barbárie bem árabe, decepava-os
em pedaços e os conservava, salgados, num porão. Que proveito tirava
disto? Nenhum. O sr.~Burke foi infinitamente superior.

O sr.~Hare, de certo modo, serviu-lhe de Dinarzade.\footnote{ Irmã de
Sherazade, heroína das \textit{Mil e uma noites}, a quem ajudou em seu
plano de manter o interesse do sultão narrando um conto a cada dia. Cabia
a Dinarzade acordar a irmã diariamente, antes do nascer do sol, para
contar suas histórias.} A inventividade do sr.~Burke, ao que
parece, era especialmente estimulada pela presença do amigo. A ilusão dos
sonhos de ambos permitiu que usassem um pardieiro para abrigar pomposas
visões. O sr.~Hare morava num pequeno cômodo no sexto andar de um edifício
densamente habitado de Edimburgo. Um sofá, uma caixa grande e alguns
objetos de higiene compunham decerto a sua mobília. Numa mesinha, uma
garrafa de uísque com três copos. O sr.~Burke recebia, via de regra, só
uma pessoa de cada vez, e nunca a mesma. Sua tática consistia em convidar,
ao cair da tarde, um transeunte desconhecido. Andava pelas ruas examinando
as fisionomias que despertavam sua curiosidade. Às vezes, escolhia ao
acaso. Dirigia-se ao desconhecido com a polidez de um
Harun al-Rashid.\footnote{ Harun al-Rashid (766--869), quinto califa da
dinastia abássida, protetor das letras e das artes, retratado com sua
corte nas \textit{Mil e uma noites}. Foi sob seu reinado, num período de
grande prosperidade, o apogeu do império islâmico.} O
desconhecido galgava os seis andares até o quartinho do sr.~Hare. Este lhe
cedia o sofá; para beber, oferecia-lhe uísque escocês. O sr.~Burke o
indagava sobre os fatos mais surpreendentes de sua existência. Era um
ouvinte insaciável, o sr.~Burke. O relato sempre era interrompido pelo sr.~Hare, 
antes do raiar do dia. A forma como a interrompia era
invariavelmente igual, e bastante imperativa. Para interromper o relato,
o sr.~Hare costumava passar por trás do sofá
e colocar as duas mãos sobre a boca do narrador. Nisso, vinha o sr.~Burke
sentar-se sobre o seu peito. Sonhavam ambos, imóveis nesta posição, com o
final da história que nunca escutavam. Desta maneira, o sr.~Burke e o sr.
Hare concluíram uma quantidade de histórias que o mundo jamais conhecerá.

Quando a narrativa cessava definitivamente, junto com o sopro do narrador,
o sr.~Burke e o sr.~Hare passavam a explorar o mistério. Despiam o
desconhecido, admiravam suas joias, contavam seu dinheiro, liam suas
cartas. Algumas correspondências não eram destituídas de interesse. Depois
punham o corpo para esfriar na caixa grande do sr.~Hare. E era quando o
sr.~Burke demonstrava a força prática de sua mente.

Era importante que o cadáver estivesse fresco, mas não morno, para o prazer
da aventura ser desfrutado até o final.

Naqueles primeiros anos do século, os médicos estudavam a anatomia com
paixão; mas, devido aos princípios da religião, tinham muita dificuldade
em conseguir sujeitos para dissecar. O sr.~Burke, espírito esclarecido que
era, percebera aquela lacuna da ciência. Não se sabe como chegou a se
aliar a um sábio e venerável médico, o doutor Knox, que lecionava na
faculdade de Edimburgo. O sr.~Burke talvez tivesse assistido a algumas
aulas públicas, embora sua imaginação antes devesse incliná-lo para
interesses artísticos. O certo é que prometeu ao doutor Knox fazer o
possível para ajudá-lo. O doutor Knox, por sua vez, comprometeu-se em
remunerar-lhe o esforço. A tarifa ia decrescendo dos corpos de jovens para
os corpos de velhos. Estes últimos só parcamente interessavam ao doutor
Knox. Era essa também a opinião do sr.~Burke --- pois que os velhos em geral
tinham menos imaginação. O doutor Knox se tornou famoso entre os colegas
em razão de seus conhecimentos em anatomia. O sr.~Burke e o sr.~Hare
desfrutaram da vida como diletantes. Cabe, sem dúvida, datar desta época o
período clássico de sua existência.

Pois a todo-poderosa genialidade do sr.~Burke logo o impulsionou para além
das normas e regras de uma tragédia em que havia sempre um relato e um
confidente. O sr.~Burke, sozinho (seria pueril invocar alguma influência
do sr.~Hare), evoluiu para uma espécie de romantismo. O cenário do
quartinho do sr.~Hare já não lhe bastava, inventou um procedimento noturno
no meio da névoa. Os inúmeros imitadores do sr.~Burke embaçaram um pouco a
originalidade do seu estilo. Eis aqui a autêntica tradição do mestre.

A fecunda imaginação do sr.~Burke se cansara dos relatos eternamente
similares da experiência humana. Nunca nenhum resultado correspondia à sua
expectativa. Chegou ao ponto de só se interessar pelo aspecto real, para
ele sempre variado, da morte. Concentrou o drama inteiro no desenlace. Já
não lhe importava a qualidade dos atores. Conformou-se ao acaso. O único
acessório do teatro do sr.~Burke era uma máscara de pano forrada de
alcatrão. O sr.~Burke saía pelas noites de bruma, segurando a máscara na
mão. Ia acompanhado pelo sr.~Hare. Esperava o primeiro transeunte, ia
caminhando na sua frente e então, voltando-se, apertava-lhe súbita e
firmemente a máscara de alcatrão sobre o rosto. O sr.~Burke e o sr.~Hare
seguravam imediatamente, cada um de um lado, os braços do ator. A máscara
de pano com alcatrão trazia a genial simplificação de sufocar, a um só
tempo, os gritos e o sopro. Além disto, era trágica. A névoa esfumava os
movimentos da atuação. Alguns atores pareciam fazer o papel de um bêbado.
Terminada a cena, o sr.~Burke e o sr.~Hare tomavam um \textit{cab},
destituíam a personagem; o sr.~Hare cuidava das vestimentas e o sr.~Burke
levava um cadáver fresco e limpo para o doutor Knox.

Aqui, em desacordo com a maioria dos biógrafos, deixarei o sr.~Burke e o
sr.~Hare em sua auréola de glória. Por que destruir tão belo efeito
artístico conduzindo-os languidamente até o termo de sua carreira,
revelando suas fraquezas e decepções? Não há por que vê-los de outro modo
senão com sua máscara na mão, vagando nas noites de névoa. Pois o final de
suas vidas foi banal e similar a tantos outros. Dizem que um deles foi
enforcado, e que o doutor Knox teve de deixar a faculdade de Edimburgo. O
sr.~Burke não deixou nenhuma outra obra.


\part{\textsc{a cruzada das crianças}}
%\hedramarkboth{A cruzada das crianças}{marcel schwob}

\chapter*{}
\thispagestyle{empty}
{\itshape
Circa idem tempus pueri sine rectore sine duce de universis omnium regionum villis et civitatibus 
versus transmarinas partes avidis gressibus cucurrerunt, et dum quaereretur ab ipsis quo currerent, 
responderunt: Versus Jherusalem, quaerere terram sanctam\ldots{} Adhuc quo devenerint ignoratur. 
Sed plurimi redierunt, a quibus dum quaereretur causa cursus, dixerunt se nescire. Nudae etiam mulieres 
circa idem tempus nihil loquentes per villas et civitates cucurrerunt\ldots{}}\footnote{ Em latim, no original: ``Por volta da mesma época, 
desprovidas de guia ou de líder,  crianças puseram-se a correr, a passos ávidos, de todas as vilas e cidades de cada região rumo às terras 
de além-mar, e, quando lhes perguntaram para onde corriam, responderam: ‘Para Jerusalém, em busca da Terra Santa\ldots{}’. 
Até agora não se sabe aonde chegaram. Muitas, porém, voltaram e, quando lhes perguntaram o motivo da corrida, disseram não saber. 
Também mulheres nuas, por volta da mesma época, sem nada falar, puseram-se a correr por vilas e cidades''. \textit{Tradução de Adriano Scatolin.}}


\chapter{Relato do Goliardo}
Eu, pobre goliardo, miserável clérigo errando pelos bosques e estradas a
mendigar, em nome de Nosso Senhor, meu pão de cada dia, presenciei uma
cena piedosa e ouvi as palavras das criancinhas. Sei que minha vida não é
muito santa e que cedi às tentações sob as tílias do caminho. Os irmãos
que me oferecem vinho percebem que sou pouco habituado a bebê-lo. Mas não
pertenço à seita dos que mutilam. Há malvados que furam os olhos dos
pequenos, e lhes serram as pernas e atam as mãos, a fim de exibi-los e
implorar compaixão. Eis porque me assustei ao ver todas aquelas crianças.
Nosso Senhor, sem dúvida, irá defendê-las. Falo a esmo, pois estou pleno
de alegria. Estou rindo da primavera e daquilo que vi. Meu espírito não é
muito forte. Recebi a prima tonsura com a idade de dez anos, e esqueci as
palavras latinas. Sou igual ao gafanhoto: pois vou saltando, aqui, ali, e
vou zumbindo, e às vezes abro asas coloridas, e minha cabeça miúda é
transparente e vazia. Dizem que são João, no deserto, se alimentava de
gafanhotos. Seria preciso comer muitos deles. Mas são João não era um
homem igual a nós.

Sou repleto de adoração por são João, pois ele era errante e pronunciava
palavras sem sequência. Parece-me que deveriam ser mais suaves. A
primavera também, esse ano, está suave. Nunca houve tantas flores brancas
e rosa. Os prados estão recém-lavados. Por toda parte cintila o sangue de
Nosso Senhor pelas sebes. Nosso Senhor Jesus é cor de lírio, mas seu
sangue é encarnado. Por quê? Não sei. Deve estar em algum pergaminho. Se
eu fosse um experto em letras, teria pergaminho, e nele escreveria. Assim
comeria muito bem toda noite. Iria nos conventos rezar pelos irmãos mortos
e inscreveria seus nomes no meu rolo. Transportaria meu rolo dos mortos de
uma abadia para outra. É uma coisa que agrada a nossos irmãos. Mas
desconheço os nomes de meus irmãos mortos. Quem sabe Nosso Senhor tampouco
se interesse em sabê-lo. Aquelas crianças todas me pareceram sem nome. E é
certo que Nosso Senhor Jesus tem preferência por elas. Ocupavam a estrada
qual enxame de abelhas brancas. Não sei de onde vinham. Eram peregrinos
bem pequenos. Traziam cajados de aveleira e bétula. Traziam a cruz ao
ombro; e aquelas cruzes todas eram de cores diversas. Vi algumas verdes,
decerto confeccionadas com folhas cosidas. São crianças selvagens e
ignorantes. Erram rumo a não sei quê. Têm fé em Jerusalém. Acho que
Jerusalém é longe, e Nosso Senhor deve estar mais perto de nós. Elas não
alcançarão Jerusalém. Mas Jerusalém as alcançará. Como a mim. A finalidade
de todas as coisas santas está na alegria. Nosso Senhor está aqui, neste
espinho avermelhado, e na minha boca, e em minha pobre palavra. Pois penso
nele e seu sepulcro está em meu pensamento. Amém. Vou deitar-me aqui ao
sol. É um lugar santo. Os pés de Nosso Senhor santificaram todos os
lugares. Vou dormir. Faça Jesus dormir à noite todas aquelas criancinhas
brancas carregando uma cruz. Em verdade, eu digo a ele. Tenho muito sono.
Eu digo a ele, em verdade, pois ele talvez não as tenha visto, e ele deve
velar pelas criancinhas. A hora do meio-dia pesa sobre mim. Todas as
coisas são brancas. Assim seja. Amém.

\chapter{Relato do Leproso}

Se quiser compreender o que vou lhe falar, saiba que tenho a cabeça coberta
com um capuz branco e ando chacoalhando um malho de madeira dura. Já não
sei que rosto é o meu, mas tenho medo de minhas mãos. Correm diante de mim
como bichos escamosos e lívidos. Gostaria de cortá-las fora. Tenho
vergonha daquilo que tocam. Parecem-me desalentar os frutos vermelhos que
colho e as pobres raízes que arranco nelas figuram fenecer. \textit{Domine
ceterorum libera me!} O Salvador não expiou meu pecado descorado. Fiquei
esquecido até a ressurreição. Como o sapo, selado ao frio da lua numa
pedra escura, seguirei encerrado em minha ganga hedionda quando os outros
se erguerem com seus corpos claros. \textit{Domine ceterorum, fac me
liberum: leprosus sum}.  Sou solitário e tenho horror. Só meus
dentes mantiveram sua brancura natural. Os bichos se assustam, e minha
alma quisera fugir. A luz do dia se afasta de mim. Faz mil e duzentos e
doze anos que seu Salvador os salvou, e não teve dó de mim. Não me tocou a
lança sangrenta que o perfurou. O sangue do Senhor dos outros talvez me
tivesse curado. Penso amiúde no sangue: eu poderia morder com meus dentes;
são cândidos. Já que Ele não o quis me dar, tenho a avidez de tomar aquele
que lhe pertence. Eis porque espreitei as crianças que desciam da terra de
Vendôme rumo àquela floresta do Loire. Traziam cruzes e eram-Lhe
submissas. Seus corpos eram Seu corpo e Ele não me fez parte de seu corpo.
Sou na terra rodeado por pálida danação. Espiei para chupar sangue
inocente no pescoço de uma de Suas crianças. \textit{Et caro nova fiet in
die irae.} No dia de terror, minha carne será nova. E atrás das outras
andava uma criança tenra de cabelos vermelhos. Fixei-me nela; saltei de
súbito; tomei-lhe a boca em minhas mãos horríveis. Vestia apenas uma blusa
grosseira; seus pés estavam descalços e seus olhos se quedaram plácidos. E
considerou-me sem surpresa. Então, sabendo que ela não gritaria, tive o
desejo de ainda ouvir uma voz humana e retirei minhas mãos de sua boca, e
ela não enxugou a boca. E seus olhos pareciam distantes.

--- Quem és tu? --- disse eu.

--- Johannes, o Teutão --- respondeu ele. E suas palavras eram salutares e
límpidas.

--- Aonde vais? --- disse eu ainda.

E ele respondeu:

--- A Jerusalém, conquistar a Terra Santa.

Então pus-me a rir, e perguntei-lhe:

--- Onde é Jerusalém?

E ele respondeu:

--- Não sei.

E eu disse ainda:

--- O que é Jerusalém?

E ele respondeu:

--- É Nosso Senhor.

Então, pus-me a rir novamente e perguntei:

--- O que é o teu Senhor?

E ele me disse:

--- Não sei; ele é branco.

E aquela palavra lançou-me no furor e abri os dentes sob o capuz e me
inclinei para o seu pescoço tenro e ele não recuou, e eu lhe disse:

--- Por que não tens medo de mim?

E ele disse:

--- Por que teria medo de ti, homem branco?

Então um grande pranto me agitou, e estendi-me no solo, e beijei a terra
com meus lábios terríveis, e gritei:

--- Porque sou leproso!

E a criança teuta me considerou, e disse limpidamente:

--- Não sei.

Ela não teve medo de mim! Não teve medo de mim! Minha monstruosa brancura
semelha para ela a de seu Senhor. E peguei um punhado de capim e
enxuguei-lhe a boca e as mãos. E eu lhe disse:

--- Vai em paz rumo ao teu Senhor branco, e dize-lhe que ele me esqueceu.

E a criança me olhou sem dizer nada. Acompanhei-a para fora da escuridão
desta floresta. Ela andava sem tremer. Vi sumir seu cabelo vermelho, ao
longe, no sol. \textit{Domine infantium, libera me!}  Que o som
do meu malho de madeira te alcance, como o som puro dos sinos! Mestre
daqueles que não sabem, liberta-me!

\chapter{Relato do Papa Inocêncio \textsc{iii}}

Longe do incenso e das casulas posso, muito facilmente, falar com Deus
nesta sala desdourada do meu palácio.  É aqui que  venho pensar  em minha
velhice, sem ser amparado ao andar. Durante a missa, meu coração se enleva
e meu corpo se enrijece; o cintilar do vinho sagrado enche meus olhos e
meu pensamento lubrifica-se com os óleos preciosos; mas neste local
solitário de minha basílica, posso vergar-me ao meu cansaço terrestre.
\textit{Ecce homo!} Pois o Senhor não deve realmente ouvir a voz de seus
padres em meio à pompa dos mandamentos e bulas; e decerto nem são de seu
agrado a púrpura, as joias, ou as pinturas; mas nesta pequena cela talvez
sinta pena de meu balbucio imperfeito. Senhor, estou muito velho, e eis-me
vestido de branco diante de ti, e meu nome é Inocêncio, e tu sabes que
nada sei. Perdoa-me meu papado, pois foi instituído e eu o suporto. Não
fui eu quem ordenou as honrarias. Prefiro ver o teu sol por esta vidraça
redonda do que nos magníficos reflexos de meus vitrais. Deixa-me gemer
como um idoso qualquer e voltar para ti este rosto enrugado e pálido que
ergo a muito custo das vagas da noite eterna. Os anéis escorregam em meus
dedos mirrados, como se esvaem os derradeiros dias de minha vida.

Meu Deus! Sou aqui teu vigário, e a ti estendo minha mão cova, cheia do
vinho puro de tua fé. Há grandes crimes. Há crimes muito grandes. Podemos
dar-lhes a absolvição. Há grandes heresias. Há heresias muito grandes.
Devemos puni-las impiedosamente. Nessa hora em que me ajoelho, branco,
nessa cela branca desdourada, padeço de uma angústia profunda, Senhor, sem
saber se os crimes e heresias pertencem ao pomposo domínio de meu papado
ou ao pequeno círculo de luz em que um homem velho simplesmente une as
mãos. E também, estou inquieto no tocante ao teu sepulcro. Segue cercado
de infiéis. Não soubemos retomá-lo. Ninguém dirigiu tua cruz para a Terra
Santa; mas estamos imersos no torpor. Os cavalheiros depuseram suas armas
e os reis já não sabem comandar. E eu, Senhor, eu me acuso e bato no
peito: sou fraco e velho demais.

Agora, Senhor, escuta o murmúrio trêmulo que se ergue desta pequena cela de
minha basílica, e aconselha-me. Meus servidores trouxeram-me estranhas
notícias desde o país de Flandres e Alemanha até as cidades de Marselha e
Gênova. Seitas desconhecidas vão surgir. Eles viram correr pelas cidades
mulheres nuas que não falavam. Aquelas mudas impudicas designavam o céu.
Vários loucos pregaram a ruína nas praças. Os eremitas e clérigos errantes
estão cheios de rumores. E não sei por que sortilégio mais de sete mil
crianças foram atraídas para fora dos lares. São sete mil na estrada,
carregando cruz e cajado. Não têm o que comer; não têm armas; são
incapazes e nos envergonham. São ignorantes de toda religião verdadeira.
Meus servidores interrogaram-nas. Respondem que vão a Jerusalém conquistar
a Terra Santa. Meus servidores disseram-lhes que não podiam atravessar o
mar. Responderam que o mar se abriria e ressequiria para deixá-las passar.
Os bons pais, pios e sensatos, esforçam-se por retê-las. Elas rebentam as
fechaduras à noite e transpõem as muralhas. Muitas são filhas de nobres e
cortesãs. É lamentável. Senhor, esses inocentes todos serão entregues ao
naufrágio e aos adoradores de Maomé. Vejo o sultão de Bagdá a espreitá-los
do seu palácio. Temo que os marinheiros se apoderem de seus corpos para
vendê-los.

Senhor, permiti que eu vos fale segundo as fórmulas da religião. Essa
cruzada das crianças não é uma obra pia. Não poderá obter o Sepulcro para
os cristãos. Aumenta o número de vagabundos que erram às raias da fé
autorizada. Nossos padres não podem protegê-la. É de crer que o Maligno
possui essas pobres criaturas. Seguem em bando para o precipício, como os
porcos da montanha.\footnote{ Episódio do Evangelho de Marcos 5, 13. 
[Todas as notas são da tradutora, exceto quando indicadas.]} O
Maligno de bom grado apossa-se das crianças, Senhor, como sabeis. Outrora,
assumiu a forma de um caçador de ratos, para atrair nas notas da música de
sua flauta todos os pequenos da cidade de Hamelin. Uns dizem que os
infelizes se afogaram no rio Weser; outros, que ele os encerrou no flanco
de uma montanha. Receai que Satã conduza todas as nossas crianças rumo aos
suplícios dos que não possuem nossa fé. Sabeis, Senhor, que não é bom que
a crença se renove. Tão logo ela surgiu numa sarça ardente, vós a
mandastes enclausurar num tabernáculo. E quando ela escapou dos vossos
lábios no Gólgota, ordenastes que fosse encerrada nos cibórios e
ostensórios. Esses pequenos profetas hão de abalar o edifício de vossa
Igreja. Há que proibi-los. Em detrimento de vossos consagrados, que
gastaram a vosso serviço as alvas e estolas, que resistiram duramente às
tentações para conquistar-vos, é que ireis receber aqueles que não sabem o
que fazem? Devemos deixar vir a vós as criancinhas, mas pela estrada de
vossa fé. Senhor, falo-vos segundo vossas instituições. Essas crianças
perecerão. Não façais com que haja sob Inocêncio um novo massacre dos
Inocentes.\footnote{ Crianças trucidadas a mando de Herodes, segundo o
Evangelho de Mateus 2, 16.}

Perdoa-me agora, meu Deus, o haver-te pedido conselho sob a tiara. O tremor
da velhice me está retomando. Olha minhas pobres mãos. Sou um homem muito
idoso. Minha fé já não é a dos pequeninos. O ouro das paredes desta cela
está gasto pelo tempo. Elas são brancas. O círculo de teu sol é branco.
Minha veste também é branca, e meu coração ressequido é puro. Tenho dito
segundo tua regra. Há crimes. Há crimes muito grandes. Há heresias. Há
heresias muito grandes. Minha cabeça oscila de fraqueza: talvez não se
deva punir, nem absolver. A vida pregressa faz com que hesitem nossas
resoluções. Não vi milagre algum. Ilumina-me. Será um milagre? Que sinal
lhes deste? Terão chegado os tempos? Queres que um homem muito velho, como
eu, seja igual em sua brancura às tuas cândidas criancinhas? Sete mil!
Mesmo sendo sua fé ignorante, irás punir a ignorância de sete mil
inocentes? Eu também sou Inocente.\footnote{ Jogo de
palavras com \textit{Innocent}, que significa \textit{Inocêncio}, mas
também \textit{inocente}.}  Senhor, sou inocente como elas. Não
me punas em minha extrema velhice. Os longos anos me ensinaram que esse
bando de crianças não \textit{pode} conseguir. Entretanto, Senhor, será um
milagre? Minha cela se queda em sossego, como em outras meditações. Sei
que não é preciso implorar para que te manifestes; mas eu, do alto de
minha imensa velhice, do alto de teu papado, eu te suplico. Orienta-me,
pois eu não sei. Senhor, são teus pequenos inocentes. E eu, Inocêncio, não
sei, não sei.

\chapter{Relato de três criancinhas}

Nós três, Nicolas que não sabe falar, Alain e Denis, saímos pelas estradas
rumo a Jerusalém. Faz tempo que estamos andando. Foram vozes brancas que
nos chamaram na noite. Chamavam todas as criancinhas. Eram como as vozes
dos pássaros mortos durante o inverno. E a princípio vimos muitos
pássaros, coitados, estendidos na terra enregelada, muitos passarinhos com
a garganta vermelha. Vimos depois as primeiras flores e as primeiras
folhas e com elas trançamos cruzes. Cantamos diante das aldeias, como
costumávamos fazer no ano novo. E todas as crianças corriam para nós. E
avançamos como uma tropa. Havia homens que nos maldiziam, por
desconhecerem o Senhor. Havia mulheres que nos seguravam pelo braço e nos
interrogavam, e cobriam nosso rosto de beijos. E houve também boas almas
que nos trouxeram tigelas de madeira, leite morno e frutas. E todo o mundo
tinha dó de nós. Pois eles não sabem para onde vamos e eles não escutaram
as vozes.

Por sobre a terra há densas florestas, e rios, e montanhas, e sendas cheias
de espinheiros. E no fim da terra encontra-se o mar que iremos cruzar em
breve. E no fim do mar encontra-se Jerusalém. Não temos governantes nem
guias. Mas para nós todas as estradas são boas. Embora não saiba falar,
Nicolas caminha como nós, Alain e Denis, e as terras são todas iguais, e
igualmente perigosas para as crianças. Em toda parte há densas florestas,
e rios, e montanhas, e espinhos. Mas em toda parte as vozes estarão
conosco. Há entre nós uma criança chamada Eustáquio, que nasceu com os
olhos fechados. Anda com os braços estendidos e sorri. Não vemos nada que
ele não veja. Uma menina é quem o guia e carrega a sua cruz. Chama-se
Allys. Nunca fala e nunca chora: mantém os olhos fixos nos pés de
Eustáquio, a fim de ampará-lo quando ele tropeça. Nós gostamos deles dois.
Eustáquio não vai poder ver as santas lâmpadas do sepulcro. Mas Allys lhe
tomará as mãos, para fazer com que ele toque as lajes do túmulo.

Ah! Como são belas as coisas da terra! Não nos lembramos de nada, porque
nunca aprendemos nada. Vimos, contudo, velhas árvores e rochas vermelhas.
Às vezes passamos dentro de longas trevas. Às vezes andamos até a noite em
prados claros. Gritamos o nome de Jesus nos ouvidos de Nicolas, e ele o
conhece bem. Mas não sabe dizê-lo. Ele se alegra conosco com aquilo que
vemos. Pois seus lábios podem se abrir para a alegria, e ele nos afaga os
ombros. E eles, assim, não são infelizes: pois Allys vela por Eustáquio, e
nós, Alain e Denis, velamos por Nicolas.

Diziam que nos bosques nos depararíamos com ogros e lobisomens. Mentira.
Ninguém nos assustou; ninguém nos fez mal algum. Os solitários e os
doentes vêm nos olhar e as velhas acendem, para nós, luzes nas cabanas.
Mandam tocar por nós os sinos das igrejas. Os camponeses erguem-se de
sobre os sulcos para espiar-nos. Os bichos também olham para nós e não
fogem. E desde que estamos andando, o sol se tornou mais quente, e já não
colhemos as mesmas flores. Mas todas as hastes podem trançar-se com a
mesma forma, e nossas cruzes estão sempre viçosas. Assim temos grande
esperança, e logo veremos o mar azul. E no fim do mar azul está Jerusalém.
E o Senhor deixará vir até seu túmulo todas as criancinhas. E as vozes
brancas estarão alegres na noite.

\chapter{Relato de François Longuejoue, escrevente}

Hoje, décimo-quinto dia do mês de setembro, do ano mil duzentos e doze
subsequente à incarnação de Nosso Senhor, compareceram à oficina de meu
mestre Hugues Ferré várias crianças querendo atravessar o mar para ir ver
o Santo Sepulcro. E já que o citado Ferré não possui naves mercantes
suficientes no porto de Marselha, ordenou-me que requeresse mestre
Guillaume Porc, a fim de inteirar o número. Os mestres Hugues Ferré e
Guillaume Porc transportarão as naves até a Terra Santa pelo amor de Nosso
Senhor J.C. Estão atualmente espalhadas ao redor da cidade de Marselha
mais de sete mil crianças, algumas das quais falam línguas bárbaras. E os
senhores almotacéis, com razão temendo a escassez, reuniram-se na câmara
municipal onde, após deliberar, mandaram chamar nossos ditos mestres a fim
de exortá-los e suplicar que expedissem as naves com grande diligência. O
mar não se encontra por ora muito favorável, por causa dos equinócios, mas
há que considerar que tal afluência poderia ser perigosa para nossa boa
cidade, tanto mais por essas crianças todas se encontrarem famintas em
razão da distância da estrada e por não saberem o que fazem. Mandei
convocar marinheiros no porto e equipar as naves. À hora das vésperas,
poderão puxá-las para a água. A multidão das crianças não está na cidade,
mas elas percorrem a praia juntando conchas como emblemas de viagem e, ao
que dizem, se espantam com as estrelas do mar, achando que caíram vivas do
céu a fim de lhes indicar a estrada do senhor. E sobre esse evento
extraordinário, eis o que tenho a dizer: primeiramente, que seria
desejável os mestres Hugues Ferré e Guillaume Porc conduzirem prontamente
para fora de nossa cidade essa turbulência estrangeira; segundo, que o
inverno foi bastante árduo, daí a terra estar pobre esse ano, como bem
sabem os senhores comerciantes; terceiro, que a Igreja não foi
absolutamente avisada do desígnio dessa horda que vem do norte, e que não
se envolverá na loucura de um exército pueril (\textit{turba infantium}).
E convém louvar os mestres Hugues Ferré e Guillaume Porc, tanto pelo amor
que nutrem por nossa boa cidade como por sua submissão a Nosso Senhor,
enviando suas naves e comboiando-as nesse tempo de equinócio, e com grande
risco de serem atacados pelos infiéis que pirateiam nosso mar em seus
faluchos de Argel e Bejaia.

\chapter{Relato do Calândar}

Glória a Deus!  Louvado seja o Profeta, que me permitiu ser pobre e errar
pelas cidades invocando o Senhor! Três vezes abençoados sejam os santos
companheiros de Mohammed que instituíram a ordem divina a que pertenço!
Pois me assemelho a Ele quando foi enxotado a pedradas da cidade infame
cujo nome não quero pronunciar, e refugiou-se num vinhedo onde um escravo
cristão teve pena dele, e lhe ofereceu uvas, e foi tocado pelas palavras
da fé no declínio do dia. Deus é grande! Atravessei as cidades de Mossul,
e Bagdá, e Basra, e conheci Sala-ed-Din (Deus tenha sua alma) e o sultão
Seïf-ed-Din, seu irmão, e contemplei o Comendador dos Crentes. Vivo muito
bem do pouco de arroz que mendigo e da água que vertem em minha cabaça.
Cultivo a pureza de meu corpo. Mas a pureza maior reside na alma. Está
escrito que o Profeta, antes de sua missão, adormeceu profundamente no
chão. E dois homens brancos desceram à direita e à esquerda de seu corpo e
ali se quedaram. E o homem branco da esquerda fendeu-lhe o peito com uma
faca de ouro, e dali tirou-lhe o coração, do qual espremeu o sangue negro.
E o homem branco da direita fendeu-lhe o ventre com uma faca de ouro, e
dali tirou as vísceras, que purificou. E repuseram as entranhas no lugar, e
doravante o Profeta estava puro para anunciar a fé. Trata-se de uma pureza
sobre-humana que pertence principalmente aos seres angélicos. As crianças,
contudo, são puras também. Tal foi a pureza que a adivinha desejou
engendrar quando percebeu o brilho em volta da cabeça do pai de Mohammed e
tentou unir-se a ele. Mas o pai do Profeta uniu-se à sua mulher Aminah, e
o brilho sumiu de sua fronte, e a adivinha assim conheceu que Aminah
acabava de conceber um ser puro. Glória a Deus que purifica! Aqui, sob o
pórtico deste bazar, posso descansar e saudarei os passantes. Há ricos
mercadores de tecidos e joias que ficam de cócoras. Está ali um caftã que
bem vale uns mil dinares. Eu, não preciso de dinheiro, e sou livre como um
cão. Glória a Deus! Estou lembrando, agora que estou à sombra, do começo
do meu discurso. Em primeiro lugar, falo de Deus, afora o qual não há
Deus, e do nosso Santo Profeta, que revelou a fé, pois é a origem de todos
os pensamentos, quer saiam pela boca, quer tenham sido traçados com o
auxílio do cálamo. Em segundo lugar, considero a pureza com que Deus dotou
os santos e anjos. Em terceiro lugar, reflito sobre a pureza das crianças.
De fato, acabo de ver um grande número de crianças cristãs que foram
compradas pelo Comendador dos Crentes. Eu as vi na estrada principal.
Andavam como um rebanho de ovelhas. Dizem que vieram da terra do Egito, e
que os navios dos francos as desembarcaram por lá. Satã as possuía e elas
procuravam atravessar o mar para ir a Jerusalém. Glória a Deus. Não foi
permitido que tamanha crueldade se cumprisse. Pois aquelas pobres crianças
teriam morrido no caminho, não tendo auxílio, nem víveres. Elas são
completamente inocentes. E ao vê-las atirei-me ao chão e bati com a fronte
no chão, louvando o Senhor em voz alta. Eis agora como era a aparência das
crianças. Estavam vestidas de branco, e usavam cruzes costuradas na roupa.
Não afiguravam saber onde estavam, e não pareciam aflitas. Mantêm
constantemente os olhos ao longe. Reparei numa delas que era cega e que
uma menina segurava pela mão. Muitos têm cabelos ruivos e olhos verdes.
São francos que pertencem ao imperador de Roma. Adoram erroneamente o
profeta Jesus. O engano desses francos é manifesto. Para começar, está
provado pelos livros e pelos milagres que não há palavra senão a de Maomé.
Depois, Deus permite que diariamente o glorifiquemos e esmolemos nossa
vida, e ordena a seus fiéis que protejam nossa ordem. Finalmente, negou
clarividência àquelas crianças que partiram de um país distante, tentadas
por Ibis, e não se manifestou para adverti-las. E se não tivessem
afortunadamente caído nas mãos dos Crentes, teriam sido apanhadas pelos
Adoradores do Fogo\footnote{ Adeptos do antigo zoroastrismo, religião
persa.} e acorrentadas em porões profundos. E aqueles malditos teriam-nas
ofertado em sacrifício ao seu devorador e detestável ídolo. Louvado seja
nosso Deus que faz bem tudo aquilo que faz e protege até mesmo aqueles que
não o confessam. Deus é grande! Irei agora pedir meu quinhão de arroz na
loja daquele ourives e proclamar meu desprezo às riquezas. Se a Deus
aprouver, todas as crianças serão salvas pela fé.

\chapter{Relato da Pequena Allys}

Já não consigo andar direito, porque estamos num país ardente, para onde
dois homens maus de Marselha nos trouxeram. E antes, fomos chacoalhados no
mar num dia negro, em meio aos fogos do céu. Mas meu pequeno Eustáquio não
tinha susto, porque não enxergava nada e eu segurava-lhe ambas as mãos.
Gosto muito dele, vim para cá por causa dele. Pois não sei para onde
vamos. Faz tanto tempo que partimos. Os outros nos falavam na cidade de
Jerusalém, que está no fim do mar, e de Nosso Senhor que estaria lá para
nos receber. E Eustáquio conhecia bem Nosso Senhor Jesus, mas não sabia o
que era Jerusalém, nem uma cidade, nem o mar. Ele fugiu para obedecer a
vozes, ele as ouvia toda noite. Ele as ouvia à noite por causa do
silêncio, já que não distingue a noite do dia. E ele me inquiria sobre
aquelas vozes, mas eu não podia dizer nada. Não sei nada, e só lamento por
causa de Eustáquio. Caminhávamos junto a Nicolas, e Alain, e Denis; mas
eles subiram num outro navio, e nenhum navio estava mais lá quando o sol
reapareceu. Ai, que será feito deles? Vamos tornar a encontrá-los quando
chegarmos junto de Nosso Senhor. Ainda está muito longe. Falam de um
grande rei que nos mandou buscar, e mantém em seu poder a cidade de
Jerusalém. Nesta terra tudo é branco, as casas e as roupas, e o rosto das
mulheres é coberto por um véu. O pobre Eustáquio não pode ver esta
brancura, mas falo nela, e ele se alegra. Pois diz que é o sinal do fim. O
Senhor Jesus é branco. A pequena Allys está muito lassa, mas segura
Eustáquio pela mão a fim de que ele não caia, e não tem tempo de pensar na
própria fadiga. À noite descansaremos e Allys, como de costume, vai dormir
junto de Eustáquio e, se as vozes não nos tiverem abandonado, tentará
ouvi-las na noite clara. E vai segurar Eustáquio pela mão até o final
branco da grande viagem, pois é preciso que ela lhe mostre o Senhor. E o
Senhor seguramente terá compaixão da paciência de Eustáquio, e permitirá
que Eustáquio o veja. E talvez então Eustáquio veja a pequena Allys.

\chapter{Relato do Papa Gregório \textsc{ix}}

Eis o mar\footnote{ \textit{Mer}, em francês, é palavra feminina, e
homófona de \textit{mère} (mãe).} devorador, que parece inocente e azul.
Suaves são suas dobras e é orlado de branco, qual traje divino. É um céu
líquido e seus astros estão vivos. Medito sobre ele, deste trono de rochas
para onde mandei que me trouxessem de minha liteira. Ele realmente está no
meio das terras da cristandade. Recebe a água sagrada em que o Anunciador
lavou o pecado. Às suas margens inclinaram-se todas as santas figuras e
ele abalançou suas imagens transparentes. Grande ungido misterioso, que
não tem fluxo nem refluxo, acalantador de anil, inserido no anel terrestre
qual joia fluida, interrogo-te com os olhos. Ó mar Mediterrâneo,
devolve-me minhas crianças! Por que as tomaste?

Não cheguei a conhecê-las. Minha velhice não foi afagada por seu hálito
fresco. Não vieram suplicar-me com suas ternas bocas entreabertas.
Sozinhas, qual pequenos vagabundos, plenas de uma fé furiosa e cega,
lançaram-se rumo à terra prometida e foram aniquiladas. Da Alemanha e de
Flandres, e da França e da Saboia e da Lombardia, vieram para as tuas
águas pérfidas, mar santo, zumbindo indistintas palavras de adoração.
Foram até a cidade de Marselha; foram até a cidade de Gênova. E as
carregaste em naves no teu largo dorso cristado de espuma; e te
revolveste, e estendeste para elas teus braços glaucos, e ficaste com
elas. E as outras, tu as traíste, levando-as para os infiéis; e agora
suspiram nos palácios do Oriente, cativas dos adoradores de Maomé.

Um orgulhoso rei da Ásia, outrora, mandou-te vergastar e
acorrentar.\footnote{ Xerxes \versal{I}, rei da Pérsia entre 486 e 465~a.C.} Ó mar
Mediterrâneo! Quem irá perdoar-te? És tristemente culpado. É a ti que
acuso, somente a ti, falsamente límpido e claro, miragem ruim do céu;
convoco-te à justiça perante o trono do Altíssimo, do qual derivam todas
as coisas criadas. Mar consagrado, que fizeste com nossas crianças? Ergue
para Ele teu rosto cerúleo; estende para Ele teus dedos estremecentes de
bolhas; agita teu incontável riso purpúreo; faze falar teu murmúrio e
presta-Lhe contas.

Caladas todas as tuas bocas brancas que vêm expirar a meus pés na praia,
não dizes nada. Há em meu palácio de Roma uma antiga cela desdourada, que
a idade tornou cândida como alva. O pontífice Inocêncio costumava
recolher-se ali. Afirmam que ele ali meditou longamente sobre as crianças
e sua fé, e que pediu ao Senhor um sinal. Aqui, do alto desse trono de
rochas, em meio ao ar livre, declaro que este pontífice Inocêncio possuía
ele próprio uma fé de criança, e que meneou em vão seus cabelos cansados.
Estou bem mais velho que Inocêncio; sou o mais velho de todos os vigários
que o Senhor colocou cá embaixo, e só agora começo a entender. Deus não se
manifesta. Acaso acudiu seu filho no Jardim das Oliveiras? Não o abandonou
em sua angústia suprema? Oh, a loucura pueril de invocar seu auxílio! Todo
mal e provação residem apenas em nós. Ele tem perfeita confiança na obra
afeiçoada por suas mãos. E tu traíste sua confiança. Mar divino, não te
espantes com minha linguagem. Todas as coisas são iguais perante o Senhor.
A soberba razão dos homens, pelo padrão do infinito, não vale mais que o
olhinho raiado de um dos teus animais. Deus concede igual quinhão ao grão
de areia e ao imperador. O ouro amadurece na mina tão impecavelmente como
o monge reflete no monastério. Todas as partes do mundo são identicamente
culpadas quando não seguem as linhas da bondade; pois Dele procedem. A
seus olhos não há pedras, plantas, animais, ou homens, mas criações. Vejo
todas essas cabeças alvorejadas saltando por sobre tuas ondas e
fundindo-se em tuas águas; surgem por um só segundo sob a luz do sol, e
podem ser danadas ou eleitas. A extrema velhice instrui o orgulho e
ilumina a religião. Sinto igual compaixão por essa conchinha de nácar e
por mim mesmo.

Eis porque te acuso, mar devorador, que tragaste minhas criancinhas. Lembra
do rei asiático por quem foste punido. Mas não se tratava de um rei
centenário. Não suportara anos bastante. Não podia compreender as coisas
do universo. Portanto, não te punirei. Pois meu lamento e teu murmúrio
viriam morrer ao mesmo tempo aos pés do Altíssimo, como vem morrer aos
meus pés o sussurro de tuas gotículas. Ó mar Mediterrâneo! Eu te perdoo e
te absolvo. Dou-te a mui santa absolvição. Vai e não tornes a pecar. Sou
tão culpado quando tu de faltas que ignoro. Sem cessar te confessas na
praia com teus mil lábios gementes, e eu me confesso a ti, grande mar
sagrado, com meus lábios fenecidos. Confessamo-nos um ao outro. Absolve-me
e eu te absolvo. Retornemos à ignorância e à candura. Assim seja.

Que farei eu sobre a terra? Haverá um monumento expiatório, um monumento
pela fé que ignora. As eras vindouras devem conhecer nossa piedade e não
desesperar. Deus levou para si as criancinhas cruzadas, pelo santo pecado
do mar; inocentes foram massacrados; os corpos dos inocentes encontrarão
abrigo. Sete naves soçobraram no recife do Recluso; erguerei naquela ilha
uma igreja dos Novos Inocentes e nela instituirei doze prebendados. E tu
me devolverás os corpos de minhas crianças, mar inocente e consagrado; tu
os trarás para as praias da ilha; e os prebendados os depositarão das
criptas do templo; e acenderão, acima deles, lamparinas eternas em que
arderão santos óleos, e mostrarão aos viajantes pios todas as pequenas
ossadas brancas estendidas na noite.


