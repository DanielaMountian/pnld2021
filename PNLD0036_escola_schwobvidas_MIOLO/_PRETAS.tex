\textbf{Marcel Schwob} (Chaville, 1867--Paris, 1905) foi ficcionista, ensaísta e tradutor francês. 
Com formação intelectual erudita, ocupou lugar de destaque nos meios literários parisienses
nos anos 1890, tendo convivido intimamente com escritores como Paul Claudel,
Guy de Maupassant, Jules Renard e Alfred Jarry, entre outros. Traduziu autores
latinos como Luciano de Samósata, Catulo e Petrônio, mas tinha especial
predileção por escritores de língua inglesa, como Defoe, Stevenson, Meredith e
Whitman. Entre suas obras mais importantes estão \textit{Cœur double}
(Coração duplo, 1891), \textit{Le Roi au masque d’or} (O rei
da máscara de ouro, 1892), \textit{Le Livre de Monelle} (O livro de
Monelle, 1894), \textit{La Croisade des enfants} (A cruzada das
crianças, 1896) e \textit{Vies imaginaires} (Vidas imaginárias, 1896).

       

\textbf{Dorothée de Bruchard} é graduada em Letras pela Universidade Federal
de Santa Catarina (\textsc{ufsc}), e mestre em Literatura Comparada pela University of
Nottingham, Inglaterra. Entre 1993 e 2001, dirigiu a Editora Paraula, dedicada à
publicação de clássicos em edições bilíngues. Atualmente é tradutora e responsável pelo 
site e publicações do Escritório do Livro (www.escritoriodolivro.com.br).

\textbf{Claudia Borges Faveri} é professora de Literatura Francesa e de Teoria e
História da Tradução da Universidade Federal de Santa Catarina (\versal{UFSC}). Com
doutorado em Letras pela Universidade de Nice Sophia Antipolis --- França e
pós-doutorado em Literatura pela Universidade Federal de Minas Gerais,
dedica-se a pesquisas na área de Teoria e História da Tradução e Literatura
Francesa Traduzida.


