\chapter{Vida e obra de Marcel Schwob}

\begin{flushright}
\textsc{claudia borges de faveri}
\end{flushright}\bigskip

\section{Sobre o autor}

\noindent\textsc{Nascido} em uma culta família de ascendência judia em 1867, na pequena
Chaville, perto de Paris, Marcel Schwob cresceu em um ambiente poliglota,
convivendo desde cedo com governantas inglesas e preceptores alemães. Precoce,
o pequeno Marcel, já aos dez anos, domina perfeitamente o inglês e o alemão,
mas se interessa também por gramática, história e os contos de Edgar Allan Poe,
que lê no original em inglês. Passa parte de sua infância em Nantes,
importante cidade do oeste da França, onde seu pai, George Schwob, dirige o
jornal \textit{Le phare de la Loire}. Com onze anos, leitor ávido, publica seu
primeiro artigo sobre um romance de Jules Verne no jornal de seu pai.

Adolescente, Schwob segue para Paris para continuar seus estudos e é
acolhido por seu tio Léon Cahun (1841--1900), orientalista e erudito francês,
além de autor de romances de aventura. Ele é também o diretor da Biblioteca
Mazarine, a mais antiga biblioteca pública francesa, formada a partir do acervo
pessoal do cardeal Mazarin (1602--1661). A convivência com o tio é determinante
para a vocação literária do jovem Marcel. De erudição notável, Léon Cahun
transmite a seu sobrinho o gosto pela história, pela etnologia e pela
arquivística. A paleografia grega e o sânscrito também lhe vêm através do
convívio com o tio que o faz ler, pela primeira vez --- suprema descoberta --- Villon e Rabelais.

A década de 1880 consolida o erudito, o literato e o homem fascinado pelo
submundo, os marginais e o insólito. São os anos de sua formação e um Schwob de
várias faces emerge desse mundo de leituras e descobertas, aquele que trará a
seus futuros livros a peculiar atmosfera que fascina e intriga seus leitores. O
estudo das línguas, antigas e modernas, apaixona-o: sânscrito, grego e latim,
inglês e alemão. Também a filologia e a linguística: segue os cursos de
Ferdinand de Saussure e Michel Bréal na \textit{École des Hautes Études}.
Começa a traduzir: por ora, Thomas de Quincey e Catulo. História antiga e
filosofia ocupam da mesma forma lugar de destaque no seu vasto universo de
interesse: lê Schopenhauer, Aristóteles e Espinosa.

%\section{O obscuro, o fantástico, a invenção}

No fim da década, Schwob, que, desde sua chegada a Paris, alguns anos antes,
contribui regularmente para o jornal de seu pai, intensifica sua atividade como
jornalista, sem no entanto deixar de lado suas pesquisas e estudos. Seu
primeiro livro publicado é um texto linguístico, \textit{Étude sur l’argot
français} (1889), escrito em colaboração com seu amigo Georges Guieysse. Schwob
é fascinado pelo estudo das gírias, línguas que, segundo ele, nada têm de
espontâneo, e são, ao contrário, artificiais e codificadas. Paralelamente aos
estudos filológicos, sua carreira de jornalista literário e editor se
consolida. São inúmeras as contribuições para os mais variados jornais e
revistas da época, como \textit{L’Écho de Paris}, \textit{La Lanterne} e
\textit{L’Événement}. Torna-se um nome importante nos círculos literários da
época. Profundo conhecedor de literatura inglesa, foi responsável pela
introdução na França de nomes como Daniel Defoe, Oscar Wilde, Thomas de
Quincey, Georges Meredith e Robert-Louis Stevenson. Alguns ele traduziu, como
Defoe, de Quincey e Stevenson, além de Shakespeare.

Schwob foi também um grande descobridor de talentos: Walt Whitman, Paul
Claudel, defendeu o teatro de Ibsen e a escultura de Camille Claudel. Alfred
Jarry dedicou-lhe o \textit{Ubu rei} e Paul Valéry sua \textit{Introdução ao
método de Leonardo Da Vinci}. Redescobriu François Villon, a cuja vida e obra
dedicou longos estudos, era fascinado pela personalidade do poeta marginal que
escapou duas vezes à forca e desapareceu sem deixar vestígios. Deixou
inacabado, ao morrer, o projeto de um grande livro sobre o poeta do
\textit{Testamento}.

Marcel Schwob teve uma breve existência: morre em 1905 aos trinta e sete
anos, após uma doença cruel que mina seus últimos dez anos de vida. Sua
carreira literária é ainda mais curta, pois compõe toda sua obra de ficção no
espaço de cinco anos, entre 1891 e 1896. Ele foi uma figura central no mundo
literário francês do fim do século \textsc{xix}, mas incompreendido por seus
contemporâneos, em virtude da nova visão estética que trazia.

\section{Sobre a obra}

\subsection{O livro de Monelle}

Os contos que compõem \textit{O livro de Monelle} foram publicados,
em revista, entre 1892 e 1894 e, depois, no verão de 1894, reunidos por Marcel
Schwob para formar este livro único, uma de suas obras mais conhecidas, várias
vezes adaptada para teatro e rádio e aplaudida pela crítica a cada reedição.

Obra de originalidade estética radical, \textit{O livro de Monelle} é imune
a classificações. São narrativas curtas, que se diferenciam não só por sua
forma, mas também por seus temas, unidas, no entanto, por um fio condutor
mais ou menos evidente: Monelle, misterioso personagem feminino, figura
sublimada de Louise, sua pequena “Vise”, como Schwob costumava chamar a
jovem simples, frágil e doente que ele conheceu e amou entre 1890 e 1893. Ela
vivia em condições miseráveis, prostituía-se para sobreviver e morreu aos vinte
e cinco anos de tuberculose, deixando seu amante inconsolável durante meses.

Apesar de seus esforços, Schwob não conseguiu salvá-la e a perda dolorosa
vai lhe inspirar \textit{O livro de Monelle}, um livro de luto que fascinou, à
época de sua primeira publicação, nomes como Mallarmé e Anatole France.
Segundo Pierre Champion, biógrafo de Schwob, em Monelle, o escritor colocou
toda a ternura que ele trazia escondida em si mesmo, em um livro que fala da
revelação do conhecimento pelo amor.

A obra é um tríptico heterogêneo no qual vêm se misturar conto, poema em
prosa e texto profético sob forma de versículos. E cada uma de suas partes se
organiza, por sua vez, em fragmentos independentes. Assim, a primeira parte,
``Palavras de Monelle'', é constituída por dois textos escritos e
publicados em separado. Ao primeiro deles, publicado em 1893, Schwob deu o nome
de ``As pequenas prostitutas'', fazendo de Monelle uma descendente direta
das criações de Thomas de Quincey e de Dostoiévski, as pequenas Anne e Nelly,
prostitutas como ela. Mas quem é Monelle? E qual a origem desse estranho nome?
Ao que tudo indica Monelle seria a encarnação literária de Louise, talvez então, daí,
o nome: \textit{L} de Louise, minha Louise, \textit{Mon L}. Mas no percurso de luto
que é \textit{O livro de Monelle}, Schwob empreende não somente a busca da cura para
sua dor, nela mergulhando. Ele vai também ao encontro de si mesmo e de sua arte.
Monelle representaria, então, seu alter ego feminino, ou em termos junguianos,
sua ânima: \textit{mon elle}, ou seja, \textit{meu ela}.

O segundo texto, que integra esta primeira parte, não conheceu publicação
anterior e se constitui em um longo discurso da personagem Monelle, no qual o
narrador pouco intervém. Pela boca de Monelle, Schwob anuncia sua filosofia e
mística pessoal, numa dialética poética que integra o eterno ao fugidio, a
criação à destruição e faz do momento o lugar de eleição para a vida e a
criação. As ``Palavras de Monelle'' parecem trazer, paradoxalmente, a
contestação solene do próprio ato de criação, ao mesmo tempo em que multiplicam
as sentenças e preceitos que o restabelecem e organizam. Desnecessário dizer o
quanto o texto de Schwob torna-se, por este procedimento, ambíguo, misterioso e
desafiador.

A segunda parte, ``As irmãs de Monelle'', é uma sequência de pequenos
contos, nos quais Schwob desfia as perversões e fantasias de onze meninas
prisioneiras de seus inconscientes. São textos estreitamente relacionados ao
conto maravilhoso, quer por sua temática, quer por elementos que eles atualizam
(espaço simbólico, erva mágica, espelho, princesas). Dentre eles, duas
reescrituras de Perrault: \textit{O Barba-Azul} e \textit{Cinderela}. O
universo feminino infantil que Schwob cria e recria a partir da tradição do
conto maravilhoso, em exercício de apropriação e refração tão a seu gosto,
coloca em cena as irmãs de Monelle, aquelas que vagam pelo mundo, “não tendo
ainda se encontrado”. 

A terceira parte, chamada ``Monelle'', é composta por seis curtas
narrativas, todas dedicadas, como seu nome indica, à personagem central do
livro. Aqui, a figura de Monelle reaparece, como na
primeira parte do livro, como a idealização de uma guia, aquela que vai conduzir
o narrador no seu itinerário iniciático às portas do “reino branco”, do qual é
a mensageira. Para isso será preciso renunciar a um falacioso reino negro, no
qual o narrador se perde e se atarda e, também, a um desejado reino vermelho,
que ele constrói com sua arte. Monelle aparece nesta terceira e última parte
como um personagem profundamente ambíguo, portador de uma mensagem de redenção,
mas também de morte e inconsciência. Uma libertadora que, no entanto, evolui em
uma cena sempre opressora, fria e escura.

Schwob parece, deliberadamente, confundir o leitor, sobretudo nas últimas
páginas da narrativa, quando aparece Louvette, espécie de dublê de Monelle, mas
francamente mais positiva e realista. Com Louvette o narrador empreende a parte
final de sua peregrinação, ainda em busca de Monelle, mas, aqui, toda a
narração tende à representação de uma busca que, subitamente, parece querer
afastar-se do poder sedutor e niilista que o atraía até então. É o momento da
liberação, quando o narrador, fiel a si mesmo, foge com a pequena Louvette, que
“se lembrou, e escolheu amar e sofrer”.

É também o momento em que Schwob parece (ironicamente?) escolher a realidade
e voltar ao convívio dos homens. Monelle, obra antissimbolista? Ou, talvez,
suprema ironia de um autor que, em sua breve vida, fez da ambiguidade sua
profissão de fé?

\subsection{A cruzada das crianças}

A \textit{Cruzada das crianças} é uma história curta que, embora possa ser lida
``numa sentada só'', apresenta inúmeras referências, imagens e
intertextualidades, exigindo, no fundo, uma leitura atenta, a fim de que
não se percam esses elementos.

Importa lembrar que Marcel Schwob é filólogo e, por isso, para ele, as
fontes primárias (os manuscritos ou os documentos que garantem a
veracidade dos fatos), são o ponto de partida para qualquer trabalho de
cunho histórico. Em outras palavras, o que torna a obra \textit{As cruzadas das
crianças} tão espetacular é a capacidade de seu autor dar, a fatos
históricos rigorosamente pesquisados, o caráter de obra literária, cujo
objetivo não é simplesmente nos informar um acontecimento, mas, a partir
dele, provocar emoção estética. Assim, Schwob entrelaça história e
ficção, num vaivém entre personagens reais e personagens fictícios.

A história é composta de relatos feitos por vários personagens: uns que
são observadores (um goliardo, um leproso, dois Papas, François
Longuejoue, escrevente, e um muçulmano), outros que são participantes da
expedição (Nicolas, Alain, Denis e a pequena Allys).

Em todos esses relatos, chama a atenção a presença comum e
constantemente reiterada de dois elementos: a ignorância (``nada sei'')
e a cor branca, ambos símbolos de pureza e inocência. Essa é a condição
para se chegar a Deus.

Outro fator importante de mencionar é o fato de os relatos serem feitos
no presente, conferindo à narrativa um caráter de veracidade. Os
personagens falam para o público leitor e para ninguém em particular. Na
verdade, é mais do que isso: os personagens falam para si mesmos, para
Deus, ou, como no caso do Papa Gregório \textsc{ix}, para o mar. É introspectivo
e externo, tem penetração psicológica, ao mesmo tempo em que vê
criticamente o fanatismo da sociedade medieval, por meio de personagens
diversos.

A seguir, apresenta-se a análise de alguns desses relatos, como forma de
convite a que se analisem os demais, deixando aqui algumas provocações:
como um goliardo enxerga esse evento, fruto de uma fé inabalável e de um
extremo fanatismo? Qual o recorte que o muçulmano faz de um evento que
se propõe à ``libertação'' da Terra Santa dos próprios muçulmanos? Qual
o efeito do contato do leproso com as crianças? O Papa Gregório \textsc{ix} tem o
mesmo posicionamento que seu antecessor, Inocêncio \textsc{iii}? Como a Igreja vê
essa Cruzada? E, por fim, pode-se dizer que cada um dos
personagens-relatores tem uma visão diferente do que seja infância?

\subsection{Vidas imaginárias}

Mais uma vez temos aqui um escritor que transita com maestria entre
ficção e realidade, preenchendo, através da arte, as lacunas que a
realidade nos apresenta. São relatos biográficos curtos de personagens
reais, embora muitas vezes desconhecidos, apresentados em ordem
cronológica. Seu ponto de partida, assim como na outra obra que compõe o
mesmo volume, são documentos verídicos, sobre os quais vai
acrescentando, com esmero, elementos ficcionais.

Recusando-se à biografia histórica, Schwob inaugura uma biografia
ficcional ou bioficção. Segundo o autor, há uma diferença entre relatar
uma vida e relatar os acontecimentos de uma vida, o que faz com que ``a
arte do biógrafo'' seja ``dar igual valor à vida de um pobre ator e à
vida de Shakespeare''.\footnote{Nessa edição, página \pageref{vida}.}

O poema Perguntas de um operário que lê, de Bertold Brecht, traz os
seguintes versos iniciais:

\begin{verse}
Quem construiu Tebas, a das sete portas?\\
Nos livros vem o nome dos reis,\\
Mas foram os reis que transportaram as pedras?\\
\end{verse}

O poema questiona justamente a história oficial, fundada nos grandes
feitos de grandes personagens, deixando à margem milhões de pessoas que,
de fato, construíram ou colaboraram efetivamente para que esses grandes
feitos acontecessem.

A obra \textit{Vidas imaginárias} aborda exatamente esse assunto. Nela, uma
galeria de personagens esquecidas pela história aparece em seus feitos
particulares, numa prosa fluida e rica de conotações. Esta é a razão de
as histórias serem curtas: Schwob objetiva com isso acentuar fatos
concretos que individualizam o personagem.

Das vidas narradas, algumas pertencem a personalidades famosas; outras,
a pessoas a quem hoje desconhecemos: Empédocles (filósofo
pré-socrático), Heróstrato (responsável pela destruição do templo de
Ártemis), Crates (filósofo grego), Septima encantatriz (escrava
africana), Lucrécio (poeta e filósofo romano), Clódia (filha de Claudius
Pulcher, político romano), Petrônio (escritor romano, autor da obra
Satiricon), Sufrah (mágico oriental), Frade Dolcino (frade condenado
pela Inquisição), Cecco Angiolieri (poeta), Paolo Uccello (pintor
renascentista), Nicolas Loyseleur (religioso, juiz), Katherine, a
rendeira (mulher amante), Alain (soldado do rei Carlos \textsc{viii}), Gabriel
Spencer (ator elisabetano), Pocahontas (indígena da etnia powhatan que
se casou com o inglês John Rolfe), Cyril Tourneur (dramaturgo inglês de
fins do século \textsc{xvi}), William Phips (caça-tesouros inglês), Capitão Kid
(pirata escocês), Walter Kennedy (pirata inglês), Stede Bonnet (pirata
barbadiano), Sr.\,Burke e Sr.\,Hare (assassinos).

Para Schwob nada de novo poderia ser criado, pois tudo já tinha sido
dito e esquecido. Por isso se esmera nas formas, pois, segundo ele, o
estilo é a única coisa que um artista pode fazer para a arte. Assim, não
importa quem seja o biografado: qualquer indivíduo é digno de ser
personagem de suas histórias, pois a beleza da vida está na arte de
contá-la.

Como diz na vida imaginária que construiu para Paolo Ucello pintor,
``seu intento não estava na imitação, e sim na capacidade de desenvolver
soberanamente todas as coisas''\footnote{Nessa edição, página \pageref{intento}.}

Por essa razão, em seu texto há uma clara elaboração da escrita, que
pode ser observada no uso dos tempos verbais, no cruzamento do uso nas
pessoas do discurso e, sobretudo, na profusão de figuras de linguagem
(tão próprias da escola literária a que Marcel Schwob pertence, o
Simbolismo).

Na vida imaginária de Septima encantatriz, por exemplo, notamos a
confluência de marcas formais de 3ª e 1ª pessoa: a voz do narrador (o
biógrafo) e a voz de Septima (a biografada).

Ferida de amor, Septima invoca sua irmã morta, rogando-lhe que faça
Sextílio se apaixonar por ela. Nesse momento, o narrador dá voz à
Septima: ``--- Ó minha irmã --- disse ela ---, sai do teu sono para me escutar.''\footnote{Nessa edição, página \pageref{irma}.}

Essa interferência da primeira pessoa em um gênero que é inteiramente
narrado em terceira pessoa permite uma interpretação com base na
linguística de Ferdinand de Saussure. A terceira pessoa não está no
mesmo plano da primeira e da segunda pessoas: enquanto essas são ``de
fato'' pessoas do discurso, participando ativamente dele, aquela está
ausente, distante. Narrar em terceira pessoa significa usar a língua;
narrar em primeira significa valer-se da fala. Para Schwob, biografia e
história se relacionam com a língua (que é pública e objetiva), enquanto
a vida e a arte se relacionariam com a fala (privada e subjetiva).

Ainda nessa vida imaginária, é de se notar como o autor cria uma
atmosfera fantástica e misteriosa, reforçada por momentos repulsivos em
que não poupa detalhes desagradáveis para descrever Foinissa, a irmã
morta de Septima: ``E Foinissa, morta, envolta em faixas fragrantes, sentou-se junto dele.
E ela não tinha cérebro nem vísceras; mas seu coração ressecado fora
recolocado em seu peito.''\footnote{Nessa edição, página \pageref{viscera}.}


\section{Sobre o gênero}

Sua obra literária caracteriza-se por uma negação das fronteiras
tradicionais entre os gêneros, entretecendo crônica histórica e ficção, conto e
poema em prosa, mesclados de aforismos. Sua poética desponta como uma inovação
radical na literatura do fim do século \textsc{xix} e preconiza uma escrita que vai se
desenvolver plenamente ao longo do século seguinte. Schwob rejeita a narrativa
naturalista, para ele mero inventário descritivo que banira a imaginação e a
invenção do seio da literatura. O romance, encarnação da estética naturalista,
não serve a seus propósitos estéticos. Schwob prefere, então, o conto, a novela
ou, ainda, fragmentos de prosa poética. Suas criações são agrupadas segundo um
sentido de unidade, geralmente dado e explicado em seus prefácios, e compõem o
que o autor denominou de \textit{romance impressionista}. À concepção
tradicional de intriga e cronologia Schwob opõe narrativas curtas de formas
inovadoras que apresentam não mais uma realidade catalogável, mas o trajeto de
uma experiência. O romance não é mais uma narração conclusiva, linear e
equilibrada, mas um enigma.

Em 1891, seu primeiro livro de ficção, \textit{C\oe ur double}, agrupa textos
publicados, antes, esparsamente, na revista \textit{L’Écho de Paris} e
alguns poucos inéditos. O livro é dedicado a Robert Louis-Stevenson. Schwob,
ainda jovem autor, deixa transparecer as características e temas que vão se
desenvolver ao longo de sua trajetória criativa: o mistério, a duplicidade, a
ironia e a compaixão, agindo e fazendo agir personagens marginais, foras da
lei, fantásticos e assustadores.

Em seguida, em 1892, será a vez de \textit{Le Roi au masque d’or}. Assim
como \textit{C\oe ur double}, o livro é uma antologia de contos fantásticos que
reúne textos publicados em revistas. \textit{Le Roi
au masque d’or} conserva o mesmo tom do anterior, embora inequivocamente mais
influenciado pelo simbolismo. As várias dedicatórias traçam o mapa do mundo
literário da época e das relações de Schwob com este universo: Anatole France,
Paul Claudel, Edmond de Goncourt, Oscar Wilde, Georges Courteline, entre outros.

Em 1893 Schwob publica a intrigante e erudita antologia de poemas em prosa,
\textit{Mimes}, que, em seu gosto pela mistificação, apresenta como uma
tradução do grego antigo. Neste breve volume, Schwob imita o poeta grego
Herondas, do século \textsc{iii} a.C., recriando a Grécia antiga em pequenos poemas em
prosa, forma literária que ele empregará novamente em suas obras seguintes.
\textit{Mimes}, pela ruptura formal que representa, é a passagem de Schwob a uma 
maturidade criativa que
dará seus frutos em \textit{O livro de Monelle}, \textit{Vidas imaginárias} e 
\textit{A cruzada das crianças}.
%\footnote{ Ver \textit{A cruzada das crianças/ Vidas imaginárias}.
%Tradução de Dorothée de Bruchard. São Paulo: Hedra, 2011.}

No verão de 1894 é publicado \textit{O livro de Monelle},
espécie de evangelho místico de tom niilista, povoado por menininhas
enigmáticas. Nele, Schwob sublima a dor da perda de Louise, sua
“pequena Vise”, jovem operária que foi sua amante por quase três anos.
\textit{O livro de Monelle} é a obra mais conhecida de Schwob, várias vezes
adaptada para o teatro, inspirando artistas plásticos, músicos e desenhistas.
\textit{O livro de Monelle} não é apenas conhecido por ser o mais simbolista
dos livros de Schwob, mas sobretudo por constituir o exemplo mais claro de
\textit{romance impressionista}: nenhuma trama ou cenário, mas depurada
evocação poética de um luto.

Dois anos mais tarde, em 1896, Schwob publica \textit{Vidas imaginárias} e
\textit{A cruzada das crianças} e com estes dois livros preciosos
encerra sua breve carreira de ficcionista. Com \textit{Vidas imaginárias} Schwob
inaugura um gênero de biografia, alheio à história, uma obra de
arte que deve retratar o particular de cada indivíduo. É o próprio Schwob que
afirma, no prefácio que escreveu para o livro, que a arte “opõe-se às ideias
gerais, descreve apenas o individual, deseja somente o que é único”.\footnote{ Schwob, Marcel. 
``L’art de la biographie''. In \textsc{schwob}, Marcel. \textit{\OE uvres}.
Paris: Les Belles Lettres, 2002, pp.~629--634.} Nas vinte e duas vidas
retratadas neste livro, Schwob mescla realidade e imaginação, personagens
ilustres dividem o espaço com anônimos. Qualquer indivíduo, real ou não,
ilustre ou não, pode tornar-se o protagonista dessas vidas imaginadas e imaginárias.

Em \textit{A cruzada das crianças}, também de 1896,
Schwob volta, em sua última obra, a temas e maneiras que lhe são caros:
recriação da Idade Média, a erudição como meio de reescritura, o tema da
infância, utilizando fatos supostamente reais para criar um longo poema em
prosa, no qual as diversas vozes narrativas são o arauto de um estilo mais
depurado e límpido. Tinha então vinte e nove anos e já estava doente, não
escreveu mais nenhuma ficção, com exceção do conto “L'étoile de bois”, cuja
escrita ele diz tê-lo exaurido. Esse texto excepcionalmente imagético é o adeus de Schwob
à ficção. Nele a floresta e
os elementos são os personagens que acompanham o pequeno Alain na busca por sua
estrela, que “seul Dieu sait allumer”.\footnote{ Schwob, Marcel. ``L’étoile de
bois''. In \textsc{schwob}, Marcel. \textit{\OE uvres}. Paris: Les Belles Lettres, 2002, pp.~465--480.}
Durante os dez anos seguintes, até sua morte, Schwob dedicou-se a
pesquisas, estudos, prefácios, teatro e traduções.

Apesar de Schwob ser considerado um autor para iniciados, sua obra conquista
incessantemente novos leitores com o passar dos anos. Ele é, ainda em nossos
dias, um autor secreto, pouco conhecido mesmo na França, mas a influência de
sua obra, apesar do relativo eclipse, é considerável. Ela se estende
dos surrealistas a Jorge Luis Borges, que reconhece \textit{Vidas imaginárias}
como uma das inúmeras fontes de inspiração para a sua \textit{História
universal da infâmia}.

A maneira de ler Schwob se modificou com o passar dos anos, assim sua obra
passou a despertar um novo interesse. Seus textos, povoados de magos, heréticos
e aventureiros, espelhos e máscaras, brincam com o leitor, deixando-lhe falsas
pistas, pois, servindo-se de sua erudição espantosa, Schwob mistifica, faz
passar o real pelo imaginário e o imaginário pelo real. Frequentemente visto
como o escritor da dualidade e do enigma, nele conviviam o artista e o erudito,
o homem de análise e o criador imaginativo. Mais do que um simbolista,
precursor dos surrealistas, ou decadente \textit{fin-de-siècle}, Schwob é
um autor que modificou a maneira de pensar a literatura. O eterno
recriar, o real e o imaginário entrelaçados, a multiplicação infindável de
reminiscências literárias --- características vistas, por muitos em seu tempo,
como falta de originalidade ou talento --- é o que, em nossos dias, o singulariza
e faz de Schwob um autor profundamente moderno.