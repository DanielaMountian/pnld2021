\part{\textsc{robinson crusoe}}

\chapter*{}
\thispagestyle{empty}


\vfill
A vida e as estranhas e notáveis aventuras de Robinson Crusoe,
marinheiro, natural de York; que, tendo sido arrojado de seu navio em
naufrágio, em decorrência do qual todos morreram, exceto ele próprio,
viveu por 28 anos apartado de toda gente em ilha deserta na costa da
América, próxima à foz do caudal rio da nação orinoque; com relação de
como ele foi por fim resgatado de maneira desusada por piratas.

\hfill{}Escrito por ele mesmo.

\chapter{O prefácio}

Se a relação dos trabalhos de qualquer homem particular no mundo é digna
de ser dada ao público, sem quanto a desabone em publicada, o Editor
desta história pensa ser este o caso.

As maravilhas da vida deste homem excedem (assim pensa este Editor) tudo
quanto exista; sendo raramente capaz que a vida de um homem agasalhe tão
grande variedade.

A história é contada com modéstia, com gravidade, e com útil aplicação
religiosa dos acontecimentos àquelas coisas às quais os sábios sempre as
reservam, \emph{sc.}, à instrução das gentes pelo exemplo e à
confirmação e honra da sabedoria da Providência em toda a nossa
abundância de circunstâncias, manifestem"-se elas como for.

O editor crê que se trata de honesta história de fatos; não se
verificando nela qualquer feição de fingimento; e que o bom uso dela,
sendo este objeto de disputa, quer para divertir, quer para instruir, é
um só; e que como tal, sem mais palavras, sua publicação será de grande
serventia ao mundo.

\asterisc

\chapter{Robinson Crusoe}

Nasci no ano de 1632, na cidade de York e em boa família, ainda que não
daquela região, sendo meu pai um estrangeiro de Bremen que primeiramente
se estabelecera em Hull. No comércio amealhou boa fortuna e, deixando o
negócio, passou a viver em York, onde se casara com minha mãe, nascida
Robinson, que era próspera família do lugar, donde fui chamado Robinson
Kreutznaer; por uma corruptela comum na Inglaterra, no entanto, ficamos
conhecidos por Crusoe, e desse modo passamos a nos valer desse nome,
pelo qual meus companheiros sempre a mim se referiram.

Eu tive dois irmãos mais velhos, um dos quais tenente"-coronel de um
regimento de terra inglês instalado em Flandres, que estivera sob a
autoridade do afamado coronel Lockhart e fora morto em batalha contra os
espanhóis próximo a Dunquerque. A sorte que teve meu segundo irmão eu
nunca soube, não mais do que meu pai e mãe souberam a sorte que eu mesmo
tive.

Sendo o terceiro filho da família, sem cultivo para ofícios quaisquer,
minha cabeça começou de cedo a se preencher de pensamentos de correr
mundo. Meu pai, já muito idoso, dera"-me um bom cabedal de instrução,
tanto quanto a educação doméstica e uma escola de província tinham a
oferecer, e destinou"-me às leis; mas eu não desejava outra coisa senão
cursar a carreira do mar, e tal inclinação colocava"-me tão fortemente
contra a vontade, ou melhor, as ordens de meu pai, e contra todas as
súplicas e argumentos de minha mãe e outros amigos, que parecia haver
algo de fatal nas inclinações de minha natureza, que me dirigia sem
desvio aos rigores com que a vida me acometeria.

Meu pai, homem sábio e grave, deu"-me sérios e excelentes conselhos
contra o que antevira ser meu destino. Chamou"-me certa manhã a seus
aposentos, aos quais se recolhia em razão da gota, e discorreu com muito
vigor sobre o tema. Perguntou"-me que razão tinha eu, senão uma simples
tendência à errância, para deixar a casa de meu pai e minha terra natal,
onde seria apresentado a bons círculos, teria possibilidade de amealhar
fortuna própria com desvelo e indústria e conheceria vida tranquila e
próspera. Disse"-me ele que competia apenas a homens de triste sina, de
um lado, e de superior e ambiciosa fortuna, por outro, seguir ao
estrangeiro em aventura para conhecer a ascensão mediante o perigo e a
fama mediante trabalhos de natureza estranha à boa ordem da vida; que
essas coisas ou estavam muito acima, ou muito abaixo de minha condição;
que o meu Estado era o mediano, ou o que se podia chamar de casta
superior do povo miúdo; e que tal Estado ele entendia ser, por longa
experiência, o melhor do mundo, o mais adequado à felicidade humana, e
jamais exposto às tristezas e provações, às penas e aos sofrimentos da
porção mecânica da humanidade, e tampouco submetido aos embaraços do
orgulho, da luxúria, da ambição e da inveja da porção superior da
humanidade. E disse"-me ele que julgava feliz essa condição pelo
seguinte: que era o estado da vida que todos invejavam; que os reis
amiúde lamentavam as tristes consequências de terem nascido para as
grandes coisas e desejavam ter sido colocados no meio dos dois extremos,
entre o Ordinário e o Maior; e que o sábio Salomão dava testemunho de
ser essa a verdadeira medida da felicidade quando rogava que não lhe
fosse dada nem a pobreza, nem a riqueza.

Ele pediu que fizesse em conta e em mente sempre tivesse que as
calamidades da vida são do quinhão dos que pertencem às castas superior
e inferior da humanidade; e que as gentes que no meio estão as conhecem
em menor número; isto é, que elas não estão sujeitas a tantas
intempéries e inquietações, de corpo e mente, quanto os que infligem
tais infortúnios a si mesmos em natural consequência de seu modo de
vida, seja por levarem uma vida de vícios, extravagâncias e luxúria,
seja por duras penas, a indigência e carestia; que ao estado médio da
vida convinham todos os tipos de virtude e alegria; que a paz e a
abundância eram as aias de uma fortuna mediana; que a temperança, a
moderação, a tranquilidade, a saúde, o convívio, todos os divertimentos
agradáveis e todos os aprazimentos desejáveis eram as bênçãos que
acompanham o estado médio da vida; que desse modo os homens viviam seus
dias sem alarde e obstáculo e os deixavam em conforto, sem os
constrangimentos do trabalho das mãos ou da cabeça, sem a necessidade de
escravizar"-se ao pão de cada dia, sem a intranquilidade de difíceis
circunstâncias, que roubam a paz à alma e o descanso ao corpo, sem os
exacerbamentos da paixão da inveja, ou da secreta e ardente volúpia da
ambição pelas grandes coisas; mas em condições pacíficas, vogando
calmamente pelo mundo e experimentando com sensatez e sem amargura o mel
da vida; sentindo que são felizes e aprendendo pela experiência diária a
reconhecer lucidamente que o são.

Ele instou"-me em seguida, com pulso e afeto, que não agisse como um
garoto, lançando"-me a tribulações que a natureza e a posição da vida em
que nascera pareciam ter tirado de meu caminho; argumentando que eu não
tinha necessidade de ganhar meu pão; que ele cuidaria de meu sustento e
usaria de seus meios para me oferecer a posição que acabara de
recomendar"-me; e que, se eu não conhecia tranquilidade e felicidade no
mundo, era unicamente em razão de meu destino ou erro; e que não tinha
ele culpa alguma nisso, desobrigando"-se dessa forma de me admoestar
contra as medidas que, sabia ele, seriam minha ruína. Em suma: que, do
mesmo modo que me seria muito gentil se eu ficasse e me estabelecesse em
casa como era de sua vontade, não recairia sobre si a responsabilidade
por meus infortúnios ao oferecer"-me qualquer incentivo para partir;
dizendo, à guisa de conclusão, que eu tinha meu irmão mais velho por
exemplo, diante do qual ele se valera dos mesmos e persuasivos
argumentos para que não fosse à guerra nos Países Baixos, sem ter sido
capaz, porém, de fazer valer sua vontade, uma vez que os jovens desejos
de meu irmão solicitavam que ele ingressasse no exército, onde foi
morto; e que, embora jamais fosse deixar de rezar por mim, arriscava"-se
a dizer que, se eu desse aquele passo insensato, eu não teria as bençãos
de Deus; e que, quando não houvesse quem me assistisse em minha
recuperação, eu teria tempo bastante para refletir sobre ter
negligenciado seu bom conselho.

Vi nessa parte final do discurso de meu pai, que foi sobremodo
profético, embora eu suponha que meu pai não soubesse que assim o fosse;
pois sim: vi que lágrimas corriam em profusão por seu rosto,
especialmente quando tratou de meu irmão morto; e que, ao falar que eu
teria tempo para me arrepender quando ninguém houvesse que me
assistisse, estava ele tão comovido que embargou o discurso e disse"-me
que sentia tamanha dor no coração que só lhe restava calar.

Suas palavras moveram"-me profundamente, como não podia deixar de ser; e
eu decidi não pensar mais em viajar e estabelecer"-me em casa segundo o
desejo de meu pai. Mas ai!, em poucos dias tudo se desfez; e, em suma,
para evitar maior insistência de meu pai, poucas semanas depois eu
decidi fugir para bem longe dele. No entanto, eu não agi tão
destemperadamente quanto solicitava o calor da decisão, e fui a minha
mãe em um momento em que a percebi um pouco mais feliz que o normal e
comuniquei"-lhe que tão inclinados estavam meus pensamentos a ver o mundo
que não havia o que tivesse força o bastante de convencer"-me de realizar
outra coisa, e que meu pai antes me dera seu consentimento do que me
forçara a partir sem ele; e que eu tinha dezoito anos, o que significava
ser tarde demais para iniciar"-me no aprendizado de um ofício ou
empregar"-me como escrevente de um advogado; e que eu tinha certeza de
que, se assim o fizesse, jamais concluiria o tempo requerido pela
formação e certamente fugiria de meus mestres antes de cumpri"-lo e
fugiria ao mar; e que, se ela intercedesse por mim diante de meu pai
para que me fosse dada a permissão de realizar apenas uma viagem, se eu
retornasse para casa e não gostasse, eu não nunca mais tornaria a
fazê"-lo e prometia por diligência dobrada recuperar o tempo perdido.

Essas palavras fizeram"-lhe grande alvoroço. Respondeu"-me ela que sabia
que de nada valeria conversar com meu pai sobre um assunto como aquele;
que ele sabia perfeitamente quais seriam meus prejuízos para dar seu
consentimento a qualquer coisa que me trouxesse dor; que se perguntava
como podia eu pensar em fazer uma coisa daquelas depois de ter uma tal
conversa com meu pai e de ouvir palavras tão afetuosas e gentis quanto
as que ela sabia que ele usara; e que, em suma, não haveria quem me
assistisse caso viesse a me arruinar; e eu podia estar certo de que eles
nunca me dariam seu consentimento para tanto. Que, de sua parte, ela
jamais seria responsável por minha destruição; e que eu jamais poderia
dizer que minha mãe anuía, enquanto meu pai não.

Embora minha mãe tivesse se recusado a advogar em minha causa ante meu
pai, vim a saber depois que ela lhe deu relação de minhas palavras, e
que meu pai, depois de demonstrar grande preocupação, disse"-lhe, com um
suspiro: ``O menino podia ser feliz se quedasse em casa; se ele seguir
em viagem, será a criatura mais infeliz deste mundo; não posso dar meu
consentimento.''

Apenas um ano mais tarde eu viria a partir; embora, nesse ínterim, me
fizesse obstinadamente surdo a quaisquer propostas de me acomodar em
algum ofício, e amiúde discutisse com meu pai e minha mãe acerca de sua
tão aferrada oposição ao que sabiam ser minhas inclinações. Estando,
porém, certa feita em Hull, aonde vez por outra ia, sem qualquer
intenção de fugir àquela altura; pois bem: lá estando em companhia de um
amigo em vésperas de viajar a Londres no navio do pai e sendo seduzido a
segui"-lo com o mais encantador dos argumentos da marinhagem, a saber,
que nada me custaria a passagem, não consultei meu pai ou minha mãe,
tampouco lhes comuniquei a decisão; e, deixando que soubessem ao acaso,
sem as bênçãos de Deus ou de meu pai, sem qualquer consideração das
circunstâncias e consequências, e, como Deus bem sabe, em má hora,
embarquei em um navio rumo a Londres a primeiro de setembro de 1651;
nunca, creio eu, as desditas de um jovem aventureiro começaram tão cedo
ou tanto se alongaram quanto as minhas. Mal tendo o vaso deixado o
estuário do Humber, o vento começou a soprar e o mar a se encrespar com
grande violência; e, uma vez que nunca antes estivera no mar, senti
indescritível mal estar no corpo e terror na alma. Passei, então, a
refletir seriamente sobre o que fizera, e sobre quão justo era o
julgamento de Deus que a mim se impunha por ter deixado a casa de meu
pai com tamanha vileza e abandonado meu dever; vieram"-me à memória todos
os bons conselhos de meus pais, as lágrimas de meu pai e as súplicas de
minha mãe; e minha consciência, ainda intocada pelo cúmulo de provações
que a partir dali lhe sobreviriam, repreendia"-me pelo desdém ao
conselho, e a violação de minhas obrigações para com Deus e meu pai.

Enquanto isso a tempestade aumentou, e o mar, no qual jamais havia
estado, encapelou"-se terrivelmente, embora em nada se assemelhasse ao
que veria muitas vezes depois; não, nem ao que vi alguns dias depois;
foi, no entanto, o bastante para fazer"-me grande impressão, a mim que
era tão somente um jovem marinheiro e nunca soubera o que fosse de tal
ofício. Parecia"-me que soçobraríamos sob cada onda que se erguia; e que
de cada mergulho do navio nos abismos que no mar se abriam jamais
emergiríamos; nessa agonia de pensamentos, fiz muitas promessas e tomei
muitas decisões; que se fosse da vontade de Deus poupar"-me a vida nessa
viagem e eu tornasse a colocar meus pés em terra firme, eu rumaria sem
desvios à casa de meu pai e nunca mais embarcaria em um navio enquanto
vivesse; e que eu seria fiel a seus bons conselhos e nunca mais
incorreria em padecimentos como aqueles. Era com clareza que então eu
via quão justas eram suas observações acerca do estado mediano da vida,
e quão tranquila e confortavelmente ele vivera todos os seus dias,
jamais exposto a tempestades no mar e atribulações em terra; e eu decidi
que retornaria, como um verdadeiro filho pródigo arrependido, à casa de
meu pai.

Esses pensamentos sábios e ponderados perseveraram enquanto durou a
tempestade e, em verdade, algum tempo depois; mas no dia seguinte o
vento cedeu, o mar se acalmou, e comecei a me habituar a ele; embora
tenha permanecido bastante taciturno por todo o dia, e um pouco enjoado
também. Acerca do entardecer, porém, o céu desanuviou"-se, o vento
cessou, e uma agradável noite se sucedeu; o sol que se pusera num céu
sem nuvens assim o encontrou no amanhecer; e iluminando um mar sem
encrespações, sob um vento suave ou quase nenhum, compôs a vista mais
deleitosa que eu jamais vira.

Eu dormira bem à noite, e não estava mais enjoado, mas bastante alegre,
olhando com admiração ao mar que, no dia anterior, se mostrara tão duro
e proceloso, e que em pouco tempo podia se revelar tão calmo e
aprazível. E então, temendo que minhas boas resoluções conservassem seu
efeito, meu amigo, que me incitara a viajar consigo, veio até mim;
``Bem, Bob'', disse"-me ele, dando"-me um tapa no ombro, ``como você está
depois de tudo? Vá me dizer que você não se assustou na noite passada,
quando fez aquele ventinho?'' ``Ventinho?'', exclamei eu; ``foi uma
tempestade terrível!'' ``Uma tempestade? Pobre coitado!'', ele
respondeu; ``você chama aquilo de tempestade? Ora, não foi nada; dê"-nos
um bom navio e mar aberto, e sequer sentimos um ventinho de nada
daqueles; mas você é marinheiro de água doce, Bob. Venha, vamos preparar
uma jarra de ponche e esquecer tudo isso; está vendo que delícia de
tempo?'' Em suma desse triste episódio de minha história, fizemos à
maneira de todos os marinheiros: preparamos o ponche, e com ele me
embebedei; e na licenciosidade de uma só noite todo o arrependimento,
todas as minhas reflexões acerca de meus atos passados e todas as minhas
resoluções para o futuro se afogaram. Em uma palavra, enquanto o mar
recobrava a placidez em sua superfície, e o arrefecimento da tempestade
restaurava a calma, desfez"-se a agitação dos meus pensamentos,
esqueceram"-se meus medos e apreensões quanto a ser tragado pelo mar, a
torrente de meus primeiros desejos outra vez irrompeu, e se me apagaram
os votos e promessas feitos na hora da dificuldade. Encontrei, sim,
espaço de tempo para a reflexão; e mais sérios pensamentos
esforçaram"-se, por assim dizer, em por vezes retornar; mas eu os
afastava, e deles me esquivava como que de uma perturbação, e
dedicando"-me à bebida e à companhia, logo dominei as recorrentes
investidas daqueles achaques, pois assim os chamava; e em cinco ou seis
dias celebrei uma vitória sobre a consciência tão completa quanto a
almejada por qualquer jovem que não desejasse ser por ela incomodado.
Mas eu estava prestes a passar por uma nova provação; e a Providência,
como em casos como esse, decidiu esgotar"-me as desculpas; pois se eu não
compreendi sua primeira manifestação como um salvamento, a seguinte
haveria de ser tamanha que o mais calejado e envilecido infeliz dentre
nós confessaria o perigo e a clemência de que fomos objeto.

Ao sexto dia de travessia chegamos à enseada de Yarmouth; com vento
contrário e tempo ameno, fizéramos poucos progressos desde a tempestade.
Fomos então obrigados a lançar âncora e ali permanecemos, com o vento
ainda contrário, a saber, sudoeste, por mais sete ou oito dias, durante
os quais muitos navios de Newcastle passaram pelos mesmos canais, estes
um porto comum onde navios podem aguardar vento que os direcione ao rio.

Teríamos permanecido pouco tempo ali e enfrentado a subida do rio, não
fosse o vento muito fraco; que depois de quatro ou cinco dias passou a
soprar bravamente. Sendo a enseada de Yarmouth, contudo, sabidamente um
ótimo porto, de boa ancoragem, e nossas amarras bastante fortes, nossos
marujos permaneceram bastante à vontade e nem um pouco apreensivos
quanto ao perigo, passando o tempo entre divertimento e descanso,
segundo os usos da vida no mar; mas no oitavo dia, pela manhã, o vento
aumentou, e tivemos de empenhar muita força para fazer descer os
mastaréus, e a tudo prender e arrumar para que o navio pudesse seguir o
mais tranquilamente possível. Perto do meio dia o mar encapelou"-se, e o
navio empinou e encheu"-se d'água, e nós pensamos a certas alturas que a
âncora havia se perdido; ao que nosso capitão ordenou que se lançasse a
âncora de arrasto, de forma que flutuávamos com duas âncoras, e os cabos
soltos em toda a sua extensão.

Foi então que sobreveio terrível tempestade; e eu comecei a ver o terror
e o espanto nos rostos da marinhagem. O capitão, embora atento à
conservação do navio, ao entrar e sair de sua cabine perto de mim
murmurava para si mesmo, ``Senhor, misericórdia! Acabaremos todos
náufragos! Acabaremos todos mortos!'', e coisas assim. Enquanto tinham
lugar essas primeiras dificuldades, permaneci preso a grande estupor,
jazendo imóvel em minha cabine, que ficava na antecâmara. Não sou capaz
de descrever meu estado de espírito. Não pude recobrar a primeira
penitência, que aparentemente havia desprezado e contra a qual me havia
armado: pensava que o medo da morte havia passado e que nada se podia
igualar à primeira experiência; mas quando o capitão em pessoa passou ao
meu lado, como o mencionei, e disse que naufragaríamos todos, fui tomado
de grande horror. Deixei minha cabine e olhei para o convés; e um
panorama de tamanha devastação eu jamais vira: o mar crescia em
montanhas e engolia"-nos em suas ondas a cada poucos minutos; quando pude
olhar às cercanias, nada vi senão asperezas; dois navios ancorados
próximos ao nosso haviam cortado os mastros ao pé do convés, bastante
carregados que iam; e bradavam nossos homens que um navio que flutuava a
obra de uma milha avante havia ido a pique. Outros dois navios, tendo
perdido suas âncoras, eram arrastados da enseada em direção ao mar
aberto, à deriva e sem um mastro de pé. Os vasos leves suportavam
melhor, uma vez que não jogavam tanto; mas dois ou três deles correram
próximos ao nosso dando ao vento somente as cevadeiras.

Acerca do anoitecer, imediato e contramestre imploraram a nosso capitão
que permissão lhes fosse dada para cortar o mastro de traquete, ao que o
último resistia; mas depois de o contramestre argumentar que, se não o
fizessem, o navio poderia ir dar ao fundo, ele consentiu; e quando
cortaram o mastro de traquete, o mastro principal ficou tão solto e
trabalhava tanto o navio que foram obrigados a cortá"-lo também e a
esvaziar o convés.

Pode"-se julgar em que condição me encontrava ante tudo isso, eu que era
apenas um jovem marinheiro, e que ficara aterrorizado com muito menos.
Se eu posso, contudo, expressar os pensamentos que me ocorriam então, já
passados tantos anos, eu levava comigo um terror dez vezes maior em face
de minhas convicções anteriores, e de tê"-las abandonado ante as
resoluções que viciosamente tomara, do que diante da própria morte; e
este, acrescido ao pavor a que me lançava a tempestade, colocou"-me em
condição que não sou capaz de descrever. O pior, porém, estava por vir;
a fúria em que a tempestade prosseguia era tamanha que os próprios
marinheiros reconheceram jamais ter presenciado outra igual. Tínhamos um
bom navio, mas ele estava bastante lastreado e jogava muito no mar, de
forma que os marinheiros vez por outra bradavam que ele iria a pique. A
sorte sorria"-me em um aspecto, que não sabia o que significava ``ir a
pique'', até que perguntei. Contudo, a tempestade era tão violenta que
cheguei a ponto de assistir à rara cena de o capitão, o contramestre e
alguns outros com maior entendimento do que se passava rezarem, à espera
do momento em que o navio afundaria. No meio da noite, e sob todo o
resto de nossas preocupações, um marinheiro enviado ao porão com a
tarefa de examiná"-lo informou"-nos que havia um vazamento; outro disse
que a água no compartimento chegava a quatro pés de altura. Assim, todos
os marinheiros foram convocados à bomba. Ao ouvir a ordem, pensei que
meu coração cessaria de bater, e eu caí para trás, dando ao chão ao lado
da minha cama, dentro da cabine. Contudo, os marinheiros me ergueram e
me disseram que eu, que não era capaz de fazer o que fosse antes, tinha
tão plena capacidade de bombear quanto qualquer outro; ao que me
recompus e fui à bomba e trabalhei com bastante empenho. Entrementes, o
capitão, ao avistar algumas barcas carregadas de carvão, que não sendo
capazes de confrontar a tempestade eram obrigadas a escapar ao mar
aberto e passavam por nós, ordenou que se disparasse um canhão em sinal
de apuro. Eu, que não sabia o que significava isso, pensei que o navio
tinha se partido ou que alguma outra coisa horrível tivesse acontecido.
Em suma, fiquei tão assustado que desmaiei. Como estávamos em um momento
em que todos tinham de cuidar de sua própria vida, ninguém deu atenção a
mim e a meu destino; sucedeu, sim, que outro homem acorreu à bomba e,
rolando"-me a um canto com o pé, deixou"-me caído, julgando que estivesse
morto; e levou um bom tempo até que eu recobrasse a consciência.

O trabalho não cessou; mas com o porão fazendo água, o navio parecia não
tardar a ir a pique; e embora a tempestade sugerisse leve arrefecimento,
era impossível prosseguir viagem e navegar naquelas condições até
alcançar um porto; de forma que o capitão continuou a disparar seus
canhões a demandar ajuda; e um navio leve que flutuara à nossa frente
destacou bravamente um bote em nosso auxílio. Não foi senão por sorte
que o bote aproximou"-se de nosso navio; mas não éramos capazes de
subir"-lhe a bordo, tampouco o bote de tocar nosso costado; até que aos
marinheiros, que com enorme esforço remavam e empenhavam as próprias
vidas para o salvamento das nossas, os nossos homens lançaram uma corda
por sobre a popa com uma boia a ela amarrada, e então largaram a corda a
grande distância, que eles, depois de muito esforço e acidentes,
conseguiram segurar; e nós os rebocamos a nossa popa e então embarcamos
todos. Uma vez a bordo do bote, pareceu"-nos descabido, a eles e a nós,
remar de encontro a seu próprio navio; assim, todos concordamos em
deixá"-lo à vontade das ondas, apenas remando em direção à costa tanto
quanto pudéssemos; e nosso capitão lhes prometeu que, se o bote sofresse
avarias em sua travessia até a costa, que o repararia para seu capitão;
e dessa forma, em parte à deriva, em parte à força dos remos, nosso bote
rumou ao norte, dando à costa nas imediações de Winterton Ness.

Não havia se passado mais de um quarto de hora desde que abandonáramos
nosso barco até que o vimos ir"-se ao fundo, e então compreendi pela
primeira vez o que queriam dizer com um navio ir a pique. Preciso
reconhecer que mal tinha olhos para ver quando me disseram os
marinheiros que o navio afundava; pois a partir do momento que me
embarcaram no bote, pois não posso dizer com justeza que embarquei, meu
coração como que morrera dentro de meu peito, em parte pelo medo, em
parte pelo terror da mente e dos pensamentos acerca do que me esperava.

Enquanto estivemos nessas condições, com os marujos ainda a trabalhar
nos remos para levar o bote em direção à costa, pudemos ver, pois éramos
capazes de avistar a praia sempre que nosso bote tocava o alto das
ondas, grande número de pessoas acorrendo à faixa de areia para
ajudar"-nos quando ali déssemos; mas percorremos lentamente o caminho até
a costa; e tampouco fomos capazes de alcançá"-la até que, passado o farol
em Winterton, a praia aponta a oeste na direção de Cromer, e a terra
cortou um pouco a violência do vento. Aqui a travessia encontrou fim, e
não obstante a dificuldade, que foi muita, desembarcamos todos a salvo
na praia e a pé rumamos a Yarmouth, onde, como marinheiros acometidos de
grande infortúnio, recebemos tratamento humano tanto dos magistrados da
cidade, que colocaram a nossa disposição boas acomodações, quanto dos
mercadores particulares e proprietários de navios, que nos deram
dinheiro bastante para que viajássemos de volta a Londres ou a Hull,
como nos aprouvesse.

Tivesse eu tido então o bom pensamento de retornar a Hull e para casa,
eu teria sido feliz, e meu pai, emblema da parábola de nosso abençoado
Senhor, teria até mesmo sacrificado o bezerro cevado em meu favor; pois
escutando que o navio em que eu partira se perdera na enseada de
Yarmouth, só depois de bom tempo teve certeza de que eu não havia me
afogado.

Minha má sina, porém, arrastava"-me então com obstinação a que nada podia
resistir; e embora não poucas vezes meu bom senso e mais sóbrio juízo
tenham aos brados me apontado o caminho de casa, estava além de meus
poderes fazê"-lo. Não sei que nome dar a isso, nem sugerirei que se trata
de uma lei secreta que a tudo governa, que nos apressa a tornarmo"-nos
instrumentos de nossa própria destruição, muito embora esta se nos
apresente às claras, e corramos em sua direção de olhos abertos. Decerto
nada, exceto uma tal inevitável e irrefreável desventura, a qual me era
impossível escapar, poderia ter me conduzido adiante contra as sóbrias
ponderações e argumentos de meus mais recônditos pensamentos, e contra
dois tão evidentes sinais, como os que encontrei em minha primeira
viagem.

Meu camarada, que antes havia cuidado para que calejasse, e que era
filho do capitão, mostrava"-se agora menos intrépido do que eu. Em nosso
primeiro encontro em Yarmouth, que não se deu antes de passados dois ou
três dias de nossa chegada, pois ficamos separados na cidade em
distintas acomodações; pois bem, em nosso primeiro encontro, seu tom
pareceu"-me outro; e, com feições bastante melancólicas e balançando a
cabeça, perguntou"-me como eu ia; e dizendo a seu pai quem eu era, e como
havia embarcado naquela viagem unicamente com a tenção de
experimentar"-me, pois tinha a vontade de viajar ao estrangeiro, seu pai,
dirigindo"-se a mim num tom muito grave e aflito, disse, ``Meu jovem,
nunca mais vá ao mar; tome o acontecido como sinal claro e à vista de
que não é marinheiro.'' ``Ora, senhor'', disse eu, ``você não vai mais
ao mar?'' ``Esse é outro caso'', disse ele; ``é minha vocação, e
portanto meu dever; mas tendo você feito essa viagem almejando a
experiência, atente à mostra que lhe ofereceram os Céus do que o espera,
caso persista. Talvez isso tudo nos tenha acometido por causa sua, como
um Jonas no navio a Társis. Diga"-me'', prosseguiu ele, ``quem é você; e
por que foi ao mar?'' Em resposta, contei"-lhe um pouco de minha
história; ao fim da qual ele irrompeu em estranho paroxismo: ``O que fiz
eu'', disse ele ``para que um tal pobre infeliz embarcasse em meu navio?
Não piso no mesmo convés que você nem mesmo por mil libras.'' Era em
verdade, como disse, uma explosão de suas paixões ainda agitadas pelo
sentimento da perda, e ia além do que a autoridade lhe permitia. De
qualquer maneira, depois ele conversou muito gravemente comigo,
aconselhando"-me vivamente que retornasse a meu pai e não provocasse a
Providência com a minha ruína; dizendo que eu podia ver a mão visível
dos Céus contra mim. ``E, jovem'', disse ele, ``fie"-se nisso: se você
não regressar, só encontrará desastre e decepção aonde quer que vá, até
que as palavras de seu pai se cumpram.''

Separamo"-nos logo em seguida; pois fiz pouco do que me havia dito, e não
o vi mais; e que caminho tomou, não sei dizer. Quanto a mim, tendo algum
dinheiro no bolso, viajei a Londres por terra; e ali, bem como na
estrada, muitos embates tive comigo mesmo quando ao rumo que devia tomar
em minha vida e se devia ir para casa ou para o mar.

Quanto a ir para casa, a infâmia se opôs às melhores moções que se me
ofereciam aos pensamentos; ocorrendo"-me de pronto que me tornaria objeto
do riso de meus vizinhos e sentiria vergonha de encontrar não só meu pai
e minha mãe, como qualquer pessoa; donde muitas vezes vim a observar
desde então quão incongruente e irracional é a têmpera ordinária da
humanidade, em especial da juventude, em face da razão que lhe deve
guiar os passos em casos que tais, a saber, que não se envergonham de
pecar, mas de arrepender"-se; envergonham"-se não da ação em razão da
qual, com justiça, se devem julgar estúpidos, mas de refreá"-la, o que
faria com que fossem vistas como pessoas dotadas de sabedoria.

Nesse ponto da vida, entretanto, permaneci algum tempo, incerto de que
decisões tomar, e de que curso seguir. Persistia uma irresistível
relutância de regressar a minha casa; e ao me demorar, as lembranças das
dificuldades por que passei se desfizeram; e enquanto arrefeciam, o
tênue impulso do retorno que conservava em meus desejos foi"-se com elas,
até que por fim deixei de lado tais pensamentos e saí em demanda de nova
viagem.

A má influência que me levou para longe da casa de meu pai, que me
precipitou à descabida e tresloucada ideia de fazer fortuna, e que com
tamanha tenacidade se imprimiu em meus pensamentos que quedei mouco a
quaisquer bons conselhos e pedidos e mesmo às ordens de meu pai; pois
bem, essa mesma influência, tenha sido qual for, apresentou aos meus
olhos a mais desventurada de todas as empresas; e assim embarquei em um
navio com destino à costa de África; ou, como a chamavam vulgarmente
nossos marinheiros, uma Viagem à Guiné.

Foi grande infortúnio meu que em todas essas aventuras não embarquei
como marinheiro; pois, não obstante tivesse trabalhado um pouco mais
duramente do que o normal, teria ao mesmo tempo aprendido o dever e o
ofício de um homem do convés e em tempo teria me qualificado imediato ou
tenente, se não capitão. Mas como sempre foi meu destino escolher o
pior, assim o fiz; e tendo dinheiro no bolso e boas roupas a me vestir,
eu sempre pude embarcar na condição de cavalheiro; e assim eu não tinha
quaisquer tarefas no navio, nem aprendia a fazer qualquer coisa.

Foi sorte minha travar contato com boas companhias em Londres, o que nem
sempre acontece a jovens desgarrados e desorientados como eu era; visto
que o diabo geralmente não perde a oportunidade de colocar"-lhes
armadilhas bem cedo na vida. Esse não foi meu caso, porém; pois vim a
conhecer o capitão de um navio que havia estado na costa da Guiné; e
que, tendo conhecido bom sucesso naquelas bandas, decidira a elas
voltar. Esse capitão, em apreço a minha conversação, que não era de todo
desagradável, e ao vir a saber que tinha ânsias de conhecer o mundo,
disse"-me que, se eu o acompanhasse em viagem, não haveria despesas; e
que faria parte de seu rancho e o auxiliaria; e que, se estivesse em
condições de levar comigo mercancias de algum tipo, poderia desfrutar de
todas as vantagens de seu comércio; ao qual talvez houvesse alguma
compensação.

Aceitei a oferta; e cultivando amizade próxima com esse capitão, que era
homem honesto e de simples trato, segui em viagem ao seu lado; levando
comigo pequena fazenda, que, graças à honestidade desinteressada do meu
amigo, fiz aumentar consideravelmente; pois levava obra de quarenta
libras em quinquilharias, como me aconselhara o capitão a comprá"-las.
Essas quarenta libras eu reunira com a ajuda de parentes com os quais me
correspondera; e que, creio eu, conseguiram fazer com que meu pai, ou ao
menos minha mãe, contribuíssem com aquela que era minha primeira
entrepresa.

Dentre todas as minhas aventuras, essa foi a única em que posso dizer
ter sido bem sucedido, e isso devo à integridade e honestidade de meu
amigo, o capitão; ao lado de quem adquiri bons conhecimentos dos
cálculos e leis da navegação, aprendi como manter um registro da rota de
viagem, realizar observações, e, em suma, aprender algumas coisas que
eram necessárias ao entendimento de um navegante; pois, tanto quanto lhe
aprazia ensinar"-me, tomei gosto em aprender; e, em poucas palavras, essa
viagem me transformou em navegador e mercador; pois trouxe comigo de
minha aventura cinco libras e nove onças de ouro em pó, que me renderam
em Londres, em meu retorno, quase trezentas libras; e isso encheu"-me
daquelas aspirações que desde então consumaram minha ruína.

Mesmo nessa viagem conheci contrariedades; em particular, a de um enjoo
permanente, tendo sucumbido a violenta calentura decorrente do calor
excessivo do clima; uma vez que nosso principal negócio estava na costa,
entre obra de dez graus norte e a própria Equinocial.

Pretendi, então, tornar"-me um mercador da Guiné; e tendo meu amigo, para
meu grande pesar, morrido logo após seu regresso, decidi seguir na mesma
viagem e embarquei no mesmo navio com aquele que fora seu imediato na
viagem anterior e herdara o comando do navio. Foi a mais infeliz viagem
já feita por um homem; pois embora não levasse comigo mais de cem libras
dos meus recém"-amealhados ganhos, de modo que me restavam duzentas, que
deixara ao cuidados da viúva de meu amigo, que me era muito justa, ainda
assim incorri em terríveis infortúnios ao longo da viagem; e o primeiro
foi que nosso navio, que atravessando na proximidade das Canárias, ou
melhor, entre as ditas ilhas e a costa africana, foi surpreendido, no
raiar do dia, por um navio pirata turco vindo de Salé, que nos perseguiu
a toda vela. Em nossa fuga, enfunamos todo o velame que nossas vergas
podiam estender, ou nossos mastros conseguiam suportar; mas observando
que os corsários se vizinhavam, e certamente nos alcançariam em poucas
horas, preparamo"-nos para lutar, com nosso navio dispondo de doze
canhões, e o pirata dezoito. Por volta das três da tarde ele logrou seu
intento; e tendo por equívoco se alinhado a nosso quartel de popa, em
vez de postar"-se de través a nossa ré, como planejava, levamos oito de
nossos canhões para a defesa daquele ponto; ao que disparamos sobre ele
uma canhonada que então o fez afastar"-se, depois de devolver nossa
descarga, disparando também a munição leve dos duzentos homens que
levava a bordo. Não tivemos, porém, nenhum homem ferido, todos
permanecendo vivos. Ele preparou"-se para mais uma vez nos atacar, e nós
para defendermo"-nos; mas abordando em seguida nosso navio pelo quartel
oposto, desembarcou sessenta homens sobre nossas amuradas, que
imediatamente procederam com a destruição de nosso convés e cordoalha.
Nós os rechaçamos com artilharia leve, piques curtos, caixas de pólvora
e outros, e os varremos de nossos conveses duas vezes. Contudo, para
encurtar essa parte melancólica de nossa história, nosso navio sendo
avariado, e três de nossos homens mortos e oito feridos, fomos obrigados
à rendição e levados todos prisioneiros para Salé, um porto pertencente
ao mouro.

O tratamento que ali recebi não foi tão terrível quanto previ, nem fui
levado pela terra dentro à corte do imperador, como o mais de nossos
homens; mas fui detido pelo capitão do corsário como seu butim
particular e feito seu escravo, sendo jovem e ágil, e adequado a seus
negócios. Foi"-me devastadora a surpreendente mudança de mercador a
escravo miserável; e não raro buscava na memória as palavras proféticas
que meu pai dirigira a mim, segundo as quais só conheceria dissabores e
não teria quem me amparasse; e tais palavras eu pensava se terem
realizado tão completamente que nada podia ser pior; e que a mão dos
céus por fim se abatera sobre mim, e eu viria a conhecer morte sem
redenção. Mas não! Essa era apenas uma amostra dos infortúnios que
experimentaria, como se verá ao longo desta história.

Tendo meu novo amo, ou senhor, me levado a sua casa, tinha a esperança
de que ele me levasse consigo quando ao mar voltasse, acreditando que em
algum momento seria seu destino ser capturado por alguma nau de guerra
portuguesa ou espanhola; e que então eu seria libertado. Logo
desesperei, contudo; pois quando foi ao mar, ele deixou"-me em terra para
que cuidasse de seu jardinzinho e executasse a penosa lida dos escravos
da casa; e quando regressou da travessia, ordenou que fosse viver na
cabine do navio para fazer"-lhe guarda.

Ali, não tinha pensamentos para outra coisa senão minha fuga, e os
métodos que pudesse adotar para tanto, sem que chegasse a qualquer
solução minimamente plausível; nada se apresentando que tornasse sua
suposição razoável; pois não tinha a quem comunicá"-la e que comigo
embarcasse; nenhum companheiro de cativeiro, nenhum inglês, irlandês ou
escocês, mas unicamente eu; de modo que por dois anos, embora amiúde
encontrasse algum consolo no imaginar, jamais tive a menor perspectiva
de colocá"-la em prática.

Passados obra de dois anos, apresentou"-se estranha circunstância que
reavivou o velho pensamento de levar a cabo alguma tentativa de
libertar"-me. Com meu amo em casa permanecendo mais tempo do que a praxe
sem esquipar seu navio, para o que, como escutara, faltava"-lhe dinheiro,
ele tinha em costume, uma ou duas vezes por semana, às vezes mais,
quando bom tempo fazia, tomar do bote do navio e sair à pesca; e como
sempre levava consigo a mim e a um jovem ``maresco'', ou mourisco, para
remar o bote, e nós o deixávamos muito satisfeito, e eu provei"-me
bastante habilidoso no ofício; de forma que às vezes ele me mandava com
um mouro, alguém de sua família, e o jovem ``Maresco'', como o chamavam,
para pescar"-lhe um peixe para a ceia.

Aconteceu certa feira que, saindo para pescar numa manhã de grande
calmaria, uma neblina surgiu tão densa que, embora não estivéssemos a
meia légua da praia, perdemo"-la de vista; e ao remar sem saber para onde
ou em que direção, esforçamo"-nos o dia inteiro e toda a noite seguinte;
e quando veio a manhã descobrimos que havíamos rumado ao mar em vez de
remar em direção à praia; e que estávamos a pelo menos duas léguas de
terra. De todo modo, retornamos à costa sem problemas, ainda que ao
custo de muitos trabalhos e algum perigo; pois o vento fez"-se forte pela
manhã; apenas sentindo os apertos da fome.

Nosso amo, porém, acautelado por esse acidente, decidiu ser mais
precavido consigo mesmo no futuro; e tendo tomado o escaler de nosso
navio inglês, decidiu que não sairia mais para pescar sem agulha de
marear e mantimentos; e assim ordenou que o mestre"-carpinteiro de seu
navio, também um escravo inglês, construísse um pequeno camarote, ou
cabine, no meio do escaler, como o de um lanchão, com lugar de onde se
pudesse governar o leme e içar a vela principal; e uma cabine logo
atrás, para um ou dois marinheiros ali ficarem e manobrarem as velas. O
escaler vogava com uma vela que chamamos de bujarrona, cuja retranca
ficava sobre a cabine, esta muito baixa e apertada, com um compartimento
para si e mais um ou dois escravos e mesa para o rancho, com pequenas
travas para prender garrafas da beberagem que julgasse adequada ao
consumo; e sobretudo pão, arroz e café.

Saíamos com frequência com esse barco para pescar; e como eu era o mais
habilidoso na pesca, ele jamais partia sem mim. Aconteceu que ele
convidou para sair nesse barco, por divertimento ou pesca, alguns mouros
de distinção daquele lugar, e aos quais dedicara extraordinária
preparação, embarcando de um dia para o outro um inopinado estoque de
provisões; e ordenando"-me que transportasse e preparasse com chumbo e
pólvora três fuzis que estavam a bordo de seu navio, pois eles
pretendiam também caçar aves.

Preparei tudo tal como ele me orientara e aguardei a manhã seguinte com
o barco, este inteiro lavado, com suas insígnias e flâmulas içadas, e
tudo quanto acomodasse seus convidados; quando, por fim, meu amo subiu a
bordo sozinho e disse"-me que seus convivas haviam postergado o passeio
em razão de negócio que exigia cuidados, e ordenou que eu, com o homem e
o garoto, como sempre, partíssemos com o barco e lhes pescássemos alguns
peixes, pois seus amigos jantariam em sua casa, e exigiu que tão logo os
tivesse em posse, que os levasse a sua casa; ao que iniciei os
aprestamentos para tanto.

Nesse instante, meus antigos pensamentos de libertação acenderam"-se em
minha mente, pois então percebi que teria uma pequena embarcação à minha
disposição; e tendo meu senhor partido, corri com o necessário não a uma
pescaria, mas a uma viagem; embora não soubesse, nem mesmo tivesse
chegado a pensar, que rumo tomar; pois meu desejo era para qualquer
lugar, desde que longe dali.

Meu primeiro ardil foi tratar com o dito Mouro como precisasse de
mantimentos para nossa subsistência a bordo; pois lhe disse que não
podíamos supor que comeríamos do peixe de nosso amo; ao que concordou;
trazendo ao escaler uma cesta enorme de seus pães abiscoitados ou
bolachas e três ânforas de água fresca; e eu sabia onde a caixa de
garrafas se encontrava, e que era evidente, por suas formas, provir de
algum butim inglês; e a levei discretamente para dentro do barco
enquanto o Mouro se encontrava em terra, como se já estivesse ali antes
para nosso amo; surripiei também um imenso bloco de cera de abelha,
pesando obra de cinquenta libras, além de um pedaço de barbante ou
cordel, uma machadinha, uma serra e um martelo, todos os quais de grande
utilidade posteriormente, em especial a cera para a fabricação de velas.
Preguei"-lhe ainda outro engano, no qual ele também caiu: seu nome era
Ismael e atendia também por Mule ou Moile; então o chamei, ``Mule, as
armas de nosso amo estão no navio; não poderias buscar um pouco de
pólvora e chumbo, talvez cacemos alguns \emph{alcamis} (uma ave
semelhante ao nosso maçarico) para nós, e eu sei que ele guarda a
munição no navio''; ao que ele respondeu afirmativamente, e assim trouxe
uma grande bolsa de couro com obra de uma libra e meia de pólvora,
talvez mais; e outra com chumbo, obra de cinco ou seis libras, as quais
deixou no barco. Entrementes, eu havia encontrado um pouco de pólvora de
meu amo na cabine principal, com a qual enchi uma das garrafas da caixa,
um vasilhame grande e quase vazio cujo conteúdo abriguei em outro; e
assim cercado de tudo quanto fosse necessário, remamos para longe do
porto para pescar. O forte construído na entrada do porto nos conhecia e
não nos deu atenção; e não estávamos distantes uma milha do porto quando
enfunamos vela e procedemos com a pesca. Soprava um vento nor"-noroeste,
o que contrariava minha vontade; pois fosse ele sul, estava certo de que
alcançaríamos a costa espanhola, ao menos a baía de Cadiz; mas minha
decisão estava tomada; e soprasse o vento de onde quer que fosse,
fugiria daquele lugar horrendo em que me encontrava, deixando o resto ao
destino.

Depois de termos pescado algum tempo sem efeito, pois quando os peixes
mordiam minha isca, eu não os puxava, para que o Mouro não os visse;
disse eu a ele, ``Nada vamos conseguir dessa forma; não vamos atender o
pedido de nosso amo; precisamos ir para mais longe''. Sem perceber o
engano, o Mouro concordou, e estando à proa, içou as velas; e uma vez
que tinha o leme em mãos, levei o barco obra de uma milha adiante e o
manobrei como se fosse pescar; e deixando o leme ao cuidado do garoto,
dei um passo ao ponto em que se encontrava o Mouro e, fingindo abaixar
para buscar algo que estivesse atrás dele, surpreendi"-o passando meu
braço sob sua virilha e lançando"-o ao mar por sobre a amurada; ao que
ele emergiu quase de pronto, pois nadava como cortiça, chamou por mim e
implorou para ser trazido a bordo, e disse"-me que viajaria o mundo
comigo; e ele nadava com velocidade tal no encalço do barco que me
haveria alcançado muito rapidamente, uma vez que o vento era pouco; ao
que fui à cabine e, tomando uma das espingardas, fiz"-lhe pontaria e
disse que não o havia ferido e que, uma vez que não se agitasse,
tampouco o faria. ``Mas'', prossegui eu, ``tu nadas bem o bastante para
alcançar a costa, e o mar está calmo; pois então nada até a praia, como
tão bem sabes, e eu não lhe farei mal; mas se te aproximares do barco,
atravesso"-te a cabeça a chumbo, porque estou decidido a conquistar a
minha liberdade''; assim, ele fez meia"-volta e nadou no rumo da praia, e
não tenho dúvida de que a tocou com facilidade, excelente nadador que
era.

Teria muito me satisfeito levar o Mouro comigo e afogar o garoto, mas
não podia me arriscar a confiar nele. Quando ele se afastou, voltei"-me
ao garoto, que chamavam de Xuri, e lhe disse, ``Xuri, se fores fiel a
mim, farei de ti um grande homem; mas se não tocares teu rosto em sinal
de fidelidade a mim, isto é, se não jurares por Maomé e a barba de seu
pai, eu te lanço também ao mar''. O garoto sorriu e falou tão
inocentemente que desconfiar de sua franqueza eu não podia; e jurou ser
fiel a mim e aventurar"-se mundo afora comigo.

Enquanto estive à vista do Mouro que nadava, conduzi o barco mar
adentro, alongando"-me a barlavento para que pudessem concluir que seguia
em direção à boca do Estreito (como o teria feito a princípio qualquer
um que estivesse em seu juízo), pois quem poderia supor que navegaríamos
ao sul à verdadeira costa dos Bárbaros, onde nações inteiras de negros
decerto nos cercariam com canoas e nos destruiriam; onde jamais nos
seria possível desembarcar em terra sem que fôssemos devorados por
bestas selvagens ou ainda mais cruéis selvagens de casta humana.

Mas à medida que anoitecia, mudei a direção e apontei a sudeste,
tendendo o curso um pouco a leste para não perder de alcance a costa; e
com bons ventos e um mar tranquilo, avancei de tal forma que creio que,
no dia seguinte, às três da tarde, quando pela primeira vez avistei
terra firme, não estava a menos de cento e cinquenta milhas ao sul de
Salé; muito além dos domínios do imperador do Marrocos ou de qualquer
outro rei, pois gente não havia.

Era, no entanto, tamanho o medo que contraíra dos Mouros, e tão
terríveis os temores que tinha de cair"-lhes prisioneiro, que não parei,
nem me aproximei da praia, nem ancorei; com o vento a fazer bonança,
assim naveguei por cinco dias; e então, com o vento dando sul, concluí
também que, se algum de nossos navios estivesse em perseguição a mim, a
essas alturas eles também haviam desistido; por isso, aventurei"-me a ir
à costa e lancei âncora na foz de um riacho; não tendo ideia de onde
estava, tampouco conhecendo a latitude, a região, a nação ou o rio. Não
vi, nem desejei ver pessoas; a coisa fundamental que queria era água
fresca. Chegamos a esse riacho à noite, e decidimos nadar à praia assim
que caísse a noite e explorar o lugar; mas assim que escureceu, ouvimos
barulhos tão terríveis de latidos, rugidos e uivos de criaturas
selvagens, não sabíamos de que tipo, que o pobre garoto estava a ponto
de morrer de medo e me implorou que não fosse à praia até que
amanhecesse; ``Então não vamos, Xuri; mas pode ser que durante o dia
encontremos homens que nos serão tão ruins quanto esses leões.'' ``Mas
então a gente faz chumbo neles'', disse Xury, rindo, ``bota eles pra
correr''; tendo Xuri aprendido o inglês nas conversas entre nós
escravos. De qualquer modo, fiquei feliz em ver o garoto tão alegre e
dei"-lheum trago (do caixa de garrafas de nosso amo) para animá"-lo. O
conselho de Xuri era bom, afinal, e eu o acatei; e lançamos nossa
pequena âncora e permanecemos imóveis a noite inteira; digo ``imóveis''
pois tampouco dormimos! Duas ou três horas depois vimos enormes e
gigantescas criaturas (não sabíamos que nome tinham) de vários tipos,
chegando à praia e correndo para a água, chafurdando e se banhando pelo
prazer de se refrescarem; e seus uivos e berros eram medonhos como nunca
havia escutado.

Xuri estava horrivelmente assustado, e também eu; mas ficamos os dois
ainda mais assustados quando ouvimos uma dessas enormes criaturas vir a
nado de encontro ao barco; não sendo capazes de vê"-la, mas inferindo por
seu arfar se tratar de imensa e monstruosa besta fera. Xuri disse ser um
leão, e bem o podia, porquanto eu sabia; donde o pobre Xuri pediu aos
gritos que erguêssemos âncora e remássemos para longe; ``não, Xuri'',
disse eu, ``podemos prender o cabo da âncora à boia e nos afastar mar
adentro; eles não podem seguir"-nos até muito longe''. Mal o havia dito,
quando percebi a criatura (ou o que quer que fosse) estava a uma
distância de dois remos do barco, o que me causou espanto; assim, fui
imediatamente à porta da cabine e, tomando de minha espingarda,
disparei"-a contra ela; ao que ela de pronto fez meia"-volta e nadou à
praia.

Mas é impossível descrever os sons horríveis, os berros e uivos
hediondos que vieram fosse de beira"-mar, fosse de terra dentro, ante o
estampido ou repercussão da descarga da arma, coisa que, tenho motivos
para crer, aquelas criaturas nunca tinham ouvido antes. Convenci"-me,
assim, de que não havia como tocar a praia à noite naquela costa. Era
questão também ir à praia durante o dia; pois tornar"-se presa de
qualquer um dos bárbaros era tão ruim quanto sê"-lo de leões e tigres; ao
menos estávamos igualmente apreensivos do perigo de ambos.

A despeito do que fosse, estávamos obrigados a dar à praia num ou noutro
sítio para fazer aguada, pois dela já não nos restava nem um púcaro no
barco; sendo a questão quando ou aonde chegar: Disse"-me Xuri que, se o
deixasse ir à praia com uma das ânforas, descobriria se havia água e
traria um pouco para mim. Perguntei"-lhe por que iria, por que não
deveria eu ir, e ele ficar no barco? O garoto respondeu com tanta
afeição que esta fez com que o amasse para sempre. Respondeu"-me ele:
``Se o selvagem vier, ele me devora, e vais embora''. ``Então, Xuri'',
disse eu, ``nós dois iremos, e se os homens selvagens vierem, nós os
mataremos, eles não comerão nenhum de nós''. Dito isso, dei a Xuri um
pedaço de bolacha para comer e um trago da caixa de garrafas de nosso
amo, da qual tratei antes; e levamos o barco ao mais próximo possível da
praia, tanto quanto julgamos seguro, e completamos a travessia a pé em
direção à praia, carregando não mais do que nossas armas e duas ânforas
para a água.

Não queria perder de vista o barco, temendo a chegada de canoas com
selvagens descendo o rio; mas o garoto, avistando um terreno plano a
obra de uma milha terra dentro, insinuou"-se até lá. Não demorou para que
viesse correndo em minha direção. Pensei que viesse perseguido de algum
selvagem, ou assustado de alguma besta fera, e fui ao seu encontro para
ajudá"-lo; mas quando me aproximei, vi que trazia algo pendurado ao
ombro, uma criatura que alvejara, semelhante a uma lebre, porém de cor
diversa e pernas mais longas, o que nos contentou bastante, pois tinha
carne muito saborosa; mas alegria maior estava em o pobre Xuri contar"-me
que havia encontrado água doce e nenhum selvagem.

Descobrimos mais tarde que não precisávamos de tanto trabalho para fazer
aguada; pois subindo um pouco o rio onde estávamos, encontramos água
fresca quando vazou a maré, que não corria muito ao interior; por isso,
enchemos nossas ânforas, regalamo"-nos da lebre que ele matara e
preparamo"-nos para seguir caminho, sem termos dado com vestígios de
criatura humana naquelas bandas.

Como antes já havia feito uma viagem àquela costa, sabia muito bem que
as ilhas Canárias e também as do Cabo Verde ficavam não muito distantes
do litoral. Mas como não dispunha de instrumentos de aferição para
descobrir em que grau estávamos, e não sabia com precisão ou ao menos me
lembrava em que latitude se encontravam, não sabia onde procurá"-las ou
em que momento navegar em direção a mar aberto para lhes ir de encontro;
doutro modo, facilmente poderia encontrá"-las. Donde minha esperança era
que, se ao longo da costa vogasse até chegar à região em que faziam
negócio os ingleses, encontraria algum de seus navios em sua costumeira
prática comercial, o qual nos resgataria e embarcaria.

Tanto quanto podia estimar, o país onde então estava devia de ser o que,
entre os domínios do Imperador de Marrocos e os dos Negros, permanece
deserto e ermo, exceto pela alimária selvagem; tendo os negros o evadido
e se ido para sul por temor aos mouros, e não achando os mouros que
valesse a pena ocupá"-lo por sua esterilidade; e, de fato, ambos
enjeitando"-o pelo prodigioso número de tigres, leões, leopardos e outras
feras criaturas que ali tem abrigo; de modo que os mouros dele se valem
apenas para a caça, para onde vão como um exército, dois ou três mil
homens de cada vez; e, verdadeiramente, percorridas quase cem milhas
naquela costa, víamos nada além de um país deserto e ermo pelo dia; e
pela noite ouvíamos somente os uivos e rugidos dos brutos animais.

Por algumas vezes durante o dia pensei haver vista do pico de Tenerife,
isto é, a assomada do monte Tenerife, nas Canárias, e levava grande
desejo de me aventurar na esperança de alcançá"-lo; mas tendo duas vezes
tentado, fui contrariado pelos ventos, indo o mar alto demais para meu
pequeno barco; resolvi, então, conservar meu primeiro desígnio e seguir
no correr da costa.

Depois de termos deixado aquele lugar, várias vezes fui obrigado a
lançar âncora em demanda d'água fresca; e uma vez em particular, cedo da
manhã, largamos âncora sob um promontório que, não obstante pequeno, era
muito alto; e com a maré começando a baixar, restava"-nos um tanto a
percorrer. Xuri, mais atento do que eu ao que nos cercava, chamou"-me
delicadamente e disse que teria sido melhor permanecermos longe da
costa, pois, assim o disse, ``Vê ali o monstro terrível que dorme ao
lado da pedra''. Olhei aonde Xuri apontava e vi que, de fato, se tratava
de animal feroz, pois era um terrível e imenso leão que jazia no flanco
da praia, à sombra de uma pedra da colina que pairava um pouco acima de
si. ``Xuri'', disse eu, ``vai até a praia e o mata''. Xuri parecia
assustado e respondeu: ``Mato eu? Ele me come numa boca!'', com o que
ele quis dizer ``um bocado''. Eu, no entanto, nada mais disse ao garoto,
pedindo"-lhe apenas que ficasse quieto, e tomei de nossa maior arma,
quase do calibre de um mosquete, e a muni com boa carga de pólvora e
duas metralhas e a deixei; carregando outra com dois balotes e uma
terceira (pois tínhamos três peças) com cinco. Pus nele com todo cuidado
a mira da primeira peça para acertar"-lhe a cabeça; jazia ele, porém, com
a pata um pouco acima do focinho, de tal forma que o chumbo o acertou
perto do joelho e rebentou"-lhe o osso. Ele se ergueu de pronto num
rugido, mas atinando com a pata quebrada, veio ao chão; e então subiu
nas três que ainda tinha e bramiu de forma medonha como nunca havia
escutado. Espantou"-me um pouco não lhe ter atingido a cabeça; de
qualquer maneira, travei da segunda peça sem demora e, embora começasse
a se movimentar, fiz"-lhe fogo novamente. Acertei"-lhe a cabeça e tive o
prazer de vê"-lo tombar, sem fazer muito alvoroço, mas lutando pela vida.
Xuri tomou coragem, então, e pediu"-me que o deixasse ir à praia.
``Vai'', disse eu; e então o garoto saltou na água e, levando numa das
mãos uma arma pequena, nadou até a praia com a outra e, aproximando"-se
da criatura, encostou"-lhe o cano da peça no ouvido e fez nova descarga,
com a qual o despachou de vez.

Era caça, com bem o reconhecíamos; comida, porém, não era; e lamentei
muito a perda de três cargas de pólvora e munição num animal que de nada
nos servia. Disse Xuri, no entanto, que desejava parte dele; para tanto,
veio a bordo e pediu"-me a machadinha. ``Para quê, Xuri?'', perguntei"-lhe
eu. ``Quer cortar a cabeça'', respondeu"-me ele. Xuri, porém, não foi
capaz de lhe cortar a cabeça; mas arrancou"-lhe uma pata, que trouxe
consigo e era monstruosa.

Ocorreu"-me, contudo, que a pele dele pudesse ter, de um jeito ou de
outro, algum valor para nós; e decidi, se possível, esfolá"-lo.
Pusemo"-nos, então, Xuri e eu, a trabalhar; mas Xuri era melhor na
tarefa, pois eu mesmo pouco destro era em fazê"-lo. A tarefa tomou"-nos o
dia inteiro, mas lhe arrancamos a pele por fim e a estendemos em cima de
nossa cabine. O sol secou"-a por completo em dois dias, e ela, depois,
serviu"-me para deitar sobre ela.

Tendo partido daquelas bandas, navegamos ao sul continuamente, sem
desvios, por dez ou doze dias, fazendo uso parcimonioso de nossos
mantimentos, que começavam a minguar muitíssimo, e não nos arrojando à
praia mais do que o necessário à obtenção de água doce. Meu objetivo era
chegar ao rio Gâmbia ou Senegal, ou seja, a qualquer ponto da península
de Cabo Verde, onde eu tinha a esperança de achar algum navio europeu e,
em não o achar, não sabia que outro rumo havia de tomar, senão sair em
demanda das ilhas ou ali perecer entre os negros. Eu sabia que todos os
navios da Europa com destino à costa da Guiné ou ao Brasil ou às Índias
Orientais passavam pelo dito cabo ou pelas ilhas; em suma, entreguei
toda a minha sorte a esse único ponto, de achar alguma embarcação ou
perecer.

Passados dez dias da decisão já mencionada, comecei a ver que era terra
habitada; e em dois ou três sítios, enquanto navegávamos, vimos gentes
paradas na praia a olhar"-nos; também pudemos perceber que eram muito
pretas e que nada vestiam que lhes reparasse e cobrisse as carnes.
Senti"-me disposto em certa ocasião a ir a terra e travar contato; mas
Xuri, que bom conselheiro era, disse"-me ``Não ir, não ir''. Não
obstante, naveguei para mais acerca da praia para conversar com elas, e
as vi correr um bocado ao longo da costa atrás de mim; observei que não
portavam armas, exceto um deles, que trazia bastão longo e fino que Xuri
disse ser uma lança, e que eram capazes de arrojar a grande distância e
com boa pontaria; por isso mantive"-me à distância, mas vali"-me de sinais
para dialogar com eles o melhor que pude; e, em particular, usei de
sinais que sugeriam comida. Eles sinalizaram que parasse o barco, pois
me apanhariam um pouco de carne; ao que recolhi minha vela e ali
permaneci, enquanto dois deles correram terra dentro e em menos de meia
hora retornaram e trouxeram consigo duas peças de carne seca e um pouco
de grãos, dos que eram produto de sua terra; mas não sabíamos o que era
um ou outro; aceitamos de bom grado, mas não sabíamos como chegar a
eles, pois eu não queria aventurar"-me em terra a seu encontro, e eles
tinham igual medo de nós; por fim, tomaram eles uma decisão segura para
todos, pois levaram os presentes à praia e ali os deixaram e se foram e
ficaram muito longe até que nós os buscamos e trouxemos a bordo, e então
eles se achegaram de nós novamente.

Fizemos"-lhes sinais de agradecimento, pois nada tínhamos com que os
regalar em retribuição; mas a oportunidade de expressar"-lhes gratidão
ofereceu"-se maravilhosamente naquele exato instante; pois enquanto
demorávamos à beira"-mar, vieram das montanhas em direção à praia duas
criaturas imensas, uma perseguindo a outra em grande fúria (como o
compreendemos); se era macho a perseguir fêmea, se era divertimento ou
rixa, não nos era dado a saber; tampouco conseguíamos discernir se era
acontecimento ordinário ou extraordinário, mas creio que fosse o último;
porque, em primeiro lugar, criaturas bravias como aquelas raramente
aparecem senão à noite; e, em segundo lugar, porque notamos aquelas
gentes muito assustadas, sobretudo as mulheres. O homem que tinha a
lança ou o dardo permaneceu onde estava, os demais fugiram; no entanto,
como as duas criaturas acorreram diretamente à água, não ameaçando
arrojar"-se contra qualquer dos negros, mas mergulhando no mar e nadando,
como buscassem distração; por fim, uma delas começou a vizinhar"-se de
nosso barco mais do que eu esperava; eu estava, no entanto, preparado
para ela, pois fui expedito em carregar minha arma e pedi a Xuri que
fizesse o mesmo com as outras duas. Tão logo o animal pôs"-se ao meu
alcance, descarreguei e acertei"-lhe a cabeça; ele de pronto afundou,
para surgir imediatamente e se debater como lutasse pela vida; e tanto
lutava que tentou regressar à praia; mas entre o ferimento, que era
mortal, e o afogamento na água, ele morreu pouco antes de chegar à
terra.

É impossível dar conta da admiração dessas pobres criaturas ante o
estampido e o fogo da minha arma; algumas estiveram a ponto de morrer de
medo, e ao chão foram como se tivessem se finado do próprio terror que
sentiram; mas quando viram a besta morta e na água afundada, e eu lhes
fiz sinais para que chegassem à praia, os negros animaram"-se de coragem
e assim o fizeram; e começaram a procurá"-la. Dei"-lhe vista pelo sangue
com que manchava a água; e, com a ajuda de uma corda, que em torno dela
lancei e entreguei aos negros para que a rebocassem, eles a arrastaram à
praia, e viram ser um leopardo muito notável, de pelagem pintada e
extraordinária beleza; e os negros ergueram as mãos assombrados a pensar
no que eu utilizara para a haver matado.

A outra criatura, assustada com o clarão do fogo e o estrondo da arma,
nadou em direção à praia e correu às montanhas de onde viera; sem que eu
fosse capaz de precisar, àquela distância, o que era. Logo achei que os
negros queriam comer a carne da criatura, ao que desejei fazer com que a
aceitassem como oferecimento de minha parte; e ao sinalizar"-lhes que
podiam levá"-la, mostraram"-se muito gratos. Imediatamente se puseram à
obra; e, embora não tivessem faca, com um pedaço de madeira afiado
esfolaram"-lhe a pele mais depressa e rapidamente do que o teríamos feito
com uma faca. Eles me ofereceram um pouco da carne, a qual eu recusei,
dando a entender que era de minha vontade que com ela ficassem; no
entanto, apontei à pele, que deixaram que levasse sem dela fazer grande
caso, trazendo"-me muito mais de seu mantimento, que, embora eu não
soubesse o que fosse, mesmo assim aceitei. Fiz então um gesto para eles
pedindo água e estendi"-lhes uma de minhas ânforas, virando"-a de ponta
cabeça para mostrar"-lhes que estava vazia e desejava enchê"-la; ao que de
pronto chamaram alguns amigos seus, e assim vieram duas mulheres, que
trouxeram um enorme jarro feito de barro, queimado ao sol, como eu
supunha; e esse jarro eles deixaram na praia, como antes o haviam feito,
e até ele enviei Xuri com minhas ânforas, e ele encheu todas as três. As
mulheres estavam tão nuas quanto os homens.

Estava eu então suprido de raízes e grãos, embora não achasse o que
fossem, e água; e, deixando os bons negros, naveguei por obra de onze
dias mais, sem qualquer circunstância que me aproximasse da costa, até
que avistei terra que se estendia por grande extensão mar adentro, à
distância de quatro ou cinco léguas; e estando o mar muito aprazível,
mantive"-me bem distante da costa para tocar o cabo. Dobrando"-o, por fim,
alongado obra de duas léguas de terra, vi com clareza que havia mais
terras do outro lado, no rumo do mar; e concluí, então, como certo
realmente era, que aquele era o Cabo Verde, e aquelas eram suas ilhas,
chamadas daí Ilhas de Cabo Verde. Elas estavam, contudo, a grande
distância, e eu não era capaz de tomar resolução do melhor a ser feito;
pois se me assaltasse ventania, talvez não chegasse a nenhum dos dois
lugares.

Nesse dilema, recolhido a meus pensamentos, entrei na cabine e
sentei"-me, deixando a Xuri o leme, quando de súbito o garoto gritou,
``Senhor, senhor, um navio a vela!'', e a ponto de perder o siso,
julgando que por necessário fossem alguns dos navios de seu amo enviados
em nossa perseguição, enquanto sabia eu que já íamos muito distantes
deles. Saltei para fora da cabine e imediatamente vi não só o navio,
como também o que era, isto é, que se tratava de nau portuguesa e, como
eu pensava, com destino à costa da Guiné em demanda de negros. Ao
observar sua rota, no entanto, convenci"-me logo de que tomava caminho
diverso e não fazia tenção de vizinhar"-se da costa; ao que apontei mar
adentro o máximo que pude, decidido a travar contato com eles, se
possível.

Apesar de navegar a toda a vela, percebi que não seria capaz de
cruzar"-lhes o caminho, e que haveriam de desaparecer antes mesmo que eu
lhes pudesse chamar; mas depois de esticar a vela ao máximo e me ver em
vias de desesperar, eles, ao que parece, avistaram"-me com a ajuda de
suas lunetas, supondo ser um bote europeu pertencente a algum navio que
se perdera; assim, amainaram velas para permitir que eu os alcançasse.
Animei"-me com isso, e como trazia a insígnia do meu amo a bordo, fiz
dela bandeirola para sinalar dificuldades, e disparei uma arma. Eles
deram com ambos; pois me disseram que viram a fumaça, embora não
tivessem escutado a arma. Diante desses sinais, eles muito gentilmente
retornaram e puseram"-se a minha espera; e em obra de três horas os
alcancei.

Perguntaram"-me eles de onde eu vinha em português, espanhol e francês,
mas eu não entendia nenhuma dessas línguas; finalmente, um marinheiro
escocês que estava a bordo dirigiu"-se a mim, e respondi"-lhe que era
inglês, que havia escapado ao cativeiro dos mouros em Salé; e eles então
me convidaram a embarcar, e muito gentilmente me acolheram, a mim e a
tudo que me pertencia.

Foi alegria inexprimível, como se pode crer, ter sido dessa maneira
salvo, pois assim eu o compreendia, da condição tão cruel e desesperada
em que estava; e imediatamente ofereci tudo quanto tinha ao capitão do
navio em troca de meu salvamento; mas respondeu"-me ele generosamente que
nada aceitaria de mim e que tudo o que tivesse me seria entregue em
segurança quando chegasse aos Brasis. ``Pois'', disse"-me ele, ``eu
salvei"-lhe a vida senão como salvaria a minha própria; e pode ser o caso
que, em algum momento, seja meu destino ser salvo da mesma forma. Além
disso'', prosseguiu ele, ``quando o levar aos Brasis, que tão distantes
são do seu país, se lhe tirar as posses, você padecerá da fome ali; e
então eu terei tirado a vida que dei. Não, não'', concluiu ele: ``Eu
levarei o \emph{señor} inglês por amor ao próximo; e o que é seu terá
uso na compra de seu mantimento naquelas terras e de sua passagem de
volta para casa''.

E o capitão foi assim caridoso na proposta como justo na execução, pois
deu ordem expressa aos marinheiros de não tocar em qualquer coisa que
minha fosse; assim, tomou tudo em sua própria posse; entregando"-me, em
troca, um inventário preciso do que colocava sob sua proteção, para que
o pudesse reaver, incluindo até mesmo as três ânforas.

Meu barco, por sua vez, era muito bom; o que ele reconheceu, dizendo que
o compraria para uso em seu próprio navio; perguntando"-me que preço lhe
dava. Respondi"-lhe que ele havia sido tão generoso em tudo que não podia
oferecer valor pelo barco, deixando que ele próprio o fizesse; ao que
ele disse que assinaria letra de câmbio em valor de oitenta moedas de
\emph{peso fuerte,} a serem pagas quando chegásssemos; e que, caso
alguém mais desse pelo barco, ele cobriria a oferta; e ele me ofereceu
mais sessenta moedas de \emph{peso fuerte} por meu garoto Xuri, as quais
hesitei em aceitar; não porque não estivesse disposto a deixá"-lo ao
capitão, mas porque me causava incômodo vender a liberdade do menino,
que tão prestimoso e fiel havia sido a mim para que eu conquistasse a
minha. No entanto, quando lhe dei notícia do motivo, ele o julgou justo
e comprometeu"-se, em troca, que daria ao garoto carta na qual se
obrigava a alforriá"-lo em dez anos, desde que ele se tornasse cristão.
Xuri respondeu que estava de acordo em permanecer com o capitão, e assim
deixei que o capitão com ele ficasse.

Fizemos boa viagem aos Brasis, e cheguei à Baía de Todos os Santos em
obra de vinte e dois dias. Dessa maneira, era novamente salvo da mais
desafortunada das todas as condições de vida; e o que fazer em seguida
era questão a ser respondida.

Nunca me foge à memória a generosidade com que fui tratado pelo capitão;
que não aceitara qualquer pagamento pela passagem, deu"-me vinte ducados
pela pele do leopardo e quarenta pela pele do leão que eu trazia no
barco e cuidou que tudo o que tivesse a bordo do navio me fosse entregue
sem subtrações; comprando o que me dispus a vender, a saber, a caixa de
garrafas, duas de minhas armas e o que restara da cera de abelha, visto
que do mais havia feito velas; e conseguindo, em suma, obra de duzentas
e vinte moedas de peso fuerte por toda a carga; quantia com a qual
desembarquei na costa dos Brasis.

Havia não muito tempo que estava ali quando fui recomendado à casa de
homem bom e honesto como ele, proprietário de um dito ``engenho'', pois
assim o chamam, isto é, uma terra para o cultivo e uma casa de produção
de açúcar. Vivi com ele algum tempo e me familiarizei com as maneiras
que conheciam de plantar a cana e produzir o açúcar; e testemunhando a
vida que bem viviam os senhores dessas fazendas e a rapidez com que
granjeavam riqueza, decidi que, se conseguisse licença para ali me
estabelecer, tornar"-me"-ia senhor entre eles, resolvido entrementes a
descobrir maneira de fazer com que remetessem o dinheiro que deixara em
Londres. Com esse objetivo, obtendo uma espécie de carta de
naturalização, tomei posse das sesmarias todas que meu dinheiro pode
amealhar e constituí plano para minha plantação e colônia, plano que
estava à altura do dinheiro que esperava receber da Inglaterra.

Tinha eu um vizinho, português de Lisboa, porém nascido de pais
ingleses, cujo nome era Wells, que vivia em circunstâncias em nada à
minha dessemelhantes. A ele chamo de vizinho, pois o engenho dele
confinava com o meu e desfrutávamos de boa sociedade. Tinha pouco
dinheiro, assim como ele; e por obra de dois anos plantamos senão para
subsistir. Veio o crescimento, porém, e nossas terras se organizaram; de
forma que, no terceiro ano, fizemos plantação de tabaco em pouca
quantidade e preparamos ambos as glebas para plantio de canas em
abundância no ano seguinte; mas carecíamos os dois de ajuda; e percebia
então, mais do que nunca, o erro que cometera ao me separar de meu
garoto Xuri.

Mas, ai!, não é de se admirar que eu tenha cometido erro sem conserto; e
não havia remédio senão continuar; eu tinha negócio bastante alheio ao
meu gênio e diretamente contrário à vida que desejava, pela qual
abandonara a casa de meu pai e fizera ouvidos moucos a seu bom conselho.
Quer dizer: eu estava entrando no estado mediano, ou casta superior do
povo miúdo, à qual meu pai antes me aconselhara; e na qual, se decidisse
permanecer, poderia perfeitamente tê"-lo feito em casa e jamais ter
corrido o mundo em trabalhos como fizera; e não foram poucas as vezes em
que o disse a mim mesmo, a saber, que poderia antes ter conquistado
aquilo na Inglaterra, entre meus amigos, a ter percorrido cinco mil
milhas de mares e agrestes entre estranhos e selvagens e me estabelecer
num lugar do mundo onde jamais receberia notícias de quem tivesse o
menor conhecimento a meu respeito.

Costumava, então, refletir sobre minha condição com grande pesar. Não
tinha com quem conversar, senão vez por outra com esse vizinho; e
negócio nenhum, exceto o das mãos; e costumava dizer que vivia como
fosse um náufrago em ilha deserta, que a ninguém tivesse senão a si
mesmo. Mas quão justo é, e o quanto todos os homens devem refletir sobre
isso, que, quando comparam suas condições presentes a outras piores, os
Céus podem obrigá"-los a fazer a troca e convencê"-los mediante a
experiência de sua felicidade pregressa; quero dizer, quanta justiça há
em que a vida de fato solitária em que pensava, vivida em uma ilha de
pura solidão, haveria de ser meu quinhão, eu que tanto e tão
injustamente a comparava com a vida que então levava, na qual, caso
tivesse permanecido, muito possivelmente me haveria tornado senhor de
copiosa fortuna e prosperidade.

Encontrava"-me já adequado a meus limitados recursos para a manutenção do
engenho quando meu amável amigo, o capitão do navio que me resgatara no
mar, retornou, pois o navio ali permaneceu por obra de três meses
recebendo carga para a viagem; quando, ao falar"-lhe sobre a modesta
quantia que deixara em Londres, ofereceu"-me ele este conselho amigável e
sincero, a saber, ``\emph{Señor} inglês'', pois assim sempre me chamava,
``se você me fornecer cartas de apresentação e uma procuração em meu
nome, com orientações à pessoa que guarda seu dinheiro em Londres para
enviar seus bens a Lisboa, às pessoas que eu designar e sob a forma de
artigos nestas terras procurados, trago"-lhe em meu retorno o que desse
processo resultar, se a vontade de Deus assim for; mas, como os assuntos
humanos estão todos sujeitos a mudanças e desastres, preferiria que você
solicitasse apenas cem libras esterlinas, as quais são, segundo o disse,
metade da quantia que tem, e correria o risco apenas com uma metade;
para que, caso seja seguro, você possa solicitar o restante sob o mesmo
método; e, caso venha a ocorrer algum acidente, ainda tenha a outra
metade à qual recorrer a seu mantimento.''

Era conselho tão são, e tão amigável parecia, que não pude senão me
convencer de que era a melhor resolução a se tomar; e, assim, preparei
cartas para a senhora com quem havia deixado meu dinheiro e uma
procuração ao capitão português, como era de sua vontade.

Escrevi à viúva do capitão inglês relação minuciosa de todos os meus
trabalhos, meu cativeiro e fuga, e de como dera com o capitão português
no mar e da humanidade de sua pessoa e em que condição encontrava"-me
então, com todas as demais instruções necessárias ao meu suprimento; e
quando esse honesto capitão chegou a Lisboa, encontrou meios, por alguns
comerciantes ingleses ali estabelecidos, de enviar não apenas o que era
solicitado, como relação minuciosa de minha história, a um mercador de
Londres, que finalmente foi de encontro a ela; ao que ela não apenas
entregou o dinheiro, mas do próprio bolso enviou ao capitão português um
presente muito bonito por sua humanidade e caridade para comigo.

O mercador de Londres, despendendo as cem libras em mercadorias
inglesas, como o capitão havia solicitado, enviou"-lhas diretamente em
Lisboa, e ele me as trouxe todas, sem qualquer prejuízo, aos Brasis;
dentre as quais, sem minha orientação (pois era inexperiente demais em
meu ofício para nessas coisas), ele cuidara para que houvesse toda sorte
de ferramentas, ferragens e trastes necessários ao engenho e que me
foram de grande utilidade.

Quando chegaram os suprimentos, pensei ter sido agraciado de grande
felicidade, tendo me espantado o bom sucesso de tudo; e meu leal
procurador, o capitão, utilizara das cinco libras que minha amiga
enviara"-lhe à guisa de recompensa na contratação de um criado, para
serviço de seis anos, sem aceitar qualquer retribuição, senão um pouco
de tabaco, que o obriguei a aceitar, sendo ele de minha própria
produção.

E não só isso: sendo os artigos todos de fabricação inglesa, como
tecidos, lãs, baetas e coisas particularmente valiosas e desejáveis na
região, encontrei meios de vendê"-los com grande lucro; de maneira que
amealhei obra de quatro vezes o valor inicial da carga e me vi
infinitamente à frente de meu pobre vizinho, quero dizer, nos progressos
de meu engenho; pois a primeira coisa que fiz foi adquirir um escravo
negro e um criado europeu --- quero dizer, outro além do que o capitão me
trouxe de Lisboa.

Mas, como a prosperidade mal empregada por vezes produz os meios de
nossa maior adversidade, assim se deu comigo. Obtive no ano seguinte
muito bom sucesso em minha plantação: produzi cinquenta grandes rolos de
tabaco do meu próprio solo, mais do que havia destinado à troca por bens
de primeira necessidade com meus vizinhos; e esses cinquenta rolos, cada
qual pesando mais de cem \emph{pesos}, foram bem curados e estocados à
espera do retorno da frota de Lisboa: e, assim, vendo o aumento dos
negócios e da riqueza, minha cabeça fez"-se cheia de projetos e empresas
para além do meu alcance; tal como muitas vezes se constitui, em
verdade, a ruína das melhores cabeças nos negócios.

Tivesse eu permanecido na situação em que me achava, espaço haveria para
o acontecimento de todas as alegrias pelas quais meu pai tão
sinceramente me recomendara uma vida tranquila e retirada, e das quais
dissera ele com tanta sabedoria que o estado mediano estava repleto; mas
outras coisas sucederam, e eu me tornaria o agente voluntário de todo o
meu dissabor; e, em particular, faria crescer minha culpa e dobrar
minhas reflexões sobre mim mesmo, sobre as quais em minhas futuras
desventuras teria tempo bastante para me debruçar; sendo todos esses
infortúnios causados por minha aparente obstinação em aderir à minha
tola inclinação de vagar pelo mundo e de levá"-la a cabo, em contradição
com as mais lúcidas noções de cuidar de meu próprio bem em justa e
simples demanda daquelas condições de vida que a natureza e a
Providência concorriam em me oferecer e cumprir com meu dever.

Como já havia se passado em meu afastamento de meus pais, não me sentia
satisfeito e necessitava partir e abandonar a feliz ventura de tornar"-me
um homem rico e próspero em meu novo engenho para tão somente perseguir
um incontinente e excessivo desejo de elevar"-me mais rápido do que a
natureza da coisa admitia; e assim sucedeu que mais uma vez me arrojei
ao mais profundo abismo da desgraça humana em que um homem jamais esteve
ou que fosse compatível com sua pura e mera sobrevivência.

Para chegar em justa sequência aos pormenores dessa parte de minha
história; é preciso que se leve em conta que, já tendo vivido então
quase quatro anos nos Brasis e começando a crescer e prosperar em meu
engenho, eu não somente adquirira conhecimento da língua, como havia
contraído contatos e amizades entre meus companheiros de fazenda e
também entre os comerciantes de São Salvador, que era o nosso porto; e
que, em minhas conversas com eles, frequentemente lhes havia falado
sobre minhas duas viagens à costa da Guiné; e sobre a maneira de
negociar com os negros de lá e como era fácil a compra naquelas bandas,
em troca de ninharias como miçangas, brinquedos, facas, tesouras,
machadinhas, pedaços de vidro e similares, não apenas de pó de ouro,
grãos da Guiné, presas de elefantes, etc., mas de negros para o trabalho
dos Brasis em copioso número.

Eles sempre ouviram com grande atenção o que dizia sobre esses assuntos,
mas em especial a parte relacionada à compra de negros, que não era
comércio comumente praticado naqueles tempos; mas que, na medida em que
era, se explorava mediante os \emph{asientos}, ou licença dos reis da
Espanha e Portugal, sob registro público; de maneira que poucos negros
chegavam aos Brasis, e estes eram excessivamente caros.

Sucedeu que, estando em companhia de uns comerciantes e senhores de
engenho que conhecia, e tratando muito seriamente desses assuntos, três
deles vieram a mim na manhã seguinte e disseram que muito haviam
ponderado sobre o que lhes havia dito na noite anterior e vinham
fazer"-me proposta secreta; e, tendo exigido sigilo, confidenciaram"-me
que faziam tenção de esquipar navio que viajasse à Guiné; que todos
tinham engenhos, assim como eu, e que padeciam de limitações pela falta
de braços ao trabalho; que, como se tratava de negócio que não podia ser
realizado, porque não podiam vender publicamente os negros quando
trazidos, tinham por vontade fazer apenas uma travessia, carregar os
negros à costa privadamente e dividi"-los entre seus próprios engenhos;
e, em suma, a questão era se eu faria as vezes de comissário no navio
para administrar a questão comercial na costa da Guiné; ao que me
ofereciam uma parte igual de negros, sem despesa de qualquer quantia.

É preciso dizer que seria uma justa proposta, tivesse sido feita a
alguém que não estivesse constrangido aos cuidados de um engenho em vias
de se fazer muito valioso e lucrativo. Para mim, no entanto, que estava
assim estabelecido e situado, e nada tinha a fazer senão dar seguimento
por mais três ou quatro anos ao que iniciara, e fazer o pedido de
remessa das cem libras restantes da Inglaterra; e que naquele momento, e
com esse modesto acréscimo, reunia fortuna de não menos de três ou
quatro mil libras esterlinas, as quais também por sua vez aumentavam;
para um homem em minha posição, pensar em viagem como aquela era um
despropósito sem tamanho.

Tendo eu nascido, contudo, para ser meu próprio destruidor, foi para mim
tão impossível resistir àquela oferta quanto, antes, fora resistir a
meus primeiros desígnios vagabundos, quando o bom conselho de meu pai
encontrou ouvidos moucos. Em suma, disse"-lhes que ia de bom grado, desde
que se comprometessem a cuidar do meu engenho em minha ausência e
procedessem com ele tal como os orientasse, em caso de minha calamidade.
Todos deram palavra de que assim o fariam e assinaram cartas ou
contratos de compromisso; e eu fiz um testamento formal, versando sobre
meu engenho e bens em caso de minha morte, tornando o capitão do navio
que salvara minha vida, como já foi dito, meu herdeiro universal, mas
obrigando"-o a dispor de meus bens tal como o documento instruía; de
forma que metade da quantia apurada na venda do que me pertencia ficava
para si, e a outra metade enviada para a Inglaterra.

Em suma, pus grande cuidado na preservação de meus bens e manutenção de
meu engenho; e tivesse eu me valido da metade de um tal recato no exame
de meu próprio interesse e no juízo acerca do que deveria e não deveria
ter feito, decerto que nunca me haveria afastado de tão próspera
empresa, abandonando a promessa de circunstâncias benfazejas para ir ao
mar e correr assim todos os seus conhecidos riscos; sem falar nas razões
que tinha para esperar que particulares infortúnios recaíssem sobre mim.

Mas fui imprudente e segui descuidadamente os ditames da minha fantasia,
não os de minha razão; e assim, tendo sido o navio aparelhado, o
carregamento provido e tudo o mais feito em conformidade com o celebrado
com meus sócios na viagem, embarquei no desafortunado primeiro dia de
setembro de 1659, sendo aquela a mesma data em que oito anos antes
deixara a casa de minha família em Hull, a fim de fazer o papel de
rebelde à autoridade de meu pai e mãe, e de tolo aos meus próprios
interesses.

Nosso navio tinha capacidade para obra de cento e vinte toneladas,
carregava seis armas e quatorze homens, além do capitão, seu filho e eu;
não levávamos a bordo grande caixaria com mercadorias, senão a das
quinquilharias adequadas ao comércio com os negros, como miçangas,
pedaços de vidro, conchas e outras ninharias, em especial espelhinhos,
facas, tesouras, machadinhas, e objetos que tais.

No mesmo dia em que embarcamos, viajamos a norte, sem jamais nos alongar
da costa, com o objetivo de apontar à costa africana quando chegássemos
a dez ou doze graus de latitude norte, o que, ao que parece, era a
maneira de fazer aquela viagem naqueles tempos. O tempo esteve muito
bom, apesar do calor, em toda a nossa travessia ao norte, até que
chegamos à altura do cabo Santo Agostinho; de onde, seguindo ao alto
mar, perdemos a visão da terra firme como se navegássemos em direção à
ilha Fernando de Noronha, conservando a proa a nor"-noroeste, e aquelas
ilhas a leste; e cruzamos nesse percurso a linha equinocial em obra de
doze dias; e estávamos, segundo nossa última observação, em sete graus e
vinte e dois minutos latitude norte, quando um tornado violento, ou
furacão, fez"-nos perder o governo do navio. Começou a sudeste, passou a
noroeste e depois se fixou em nordeste; de onde soprou de maneira tão
terrível que, por doze dias seguidos, nada se pode fazer senão permitir
que o navio tomasse seu próprio rumo e conservar o vento em popa,
deixando que nos levasse para onde o destino e a fúria dos ventos o
direcionassem; e, durante esses doze dias, não preciso dizer que não
houve um só dia em que não esperasse ser tragado pelo mar; ninguém no
navio, em verdade, esperava salvar a própria vida.

No meio de tamanho desalento, vimos, além do horror da tempestade, um de
nossos marinheiros morrer de calentura, e um homem e o grumete do
capitão serem levados pelo mar; por volta do décimo segundo dia, com a
tempestade arrefecendo, o capitão fez, como pôde, observação de nossa
posição e descobriu que íamos a mais ou menos onze graus de latitude
norte, mas que havia vinte e dois graus de diferença de longitude a
oeste do Cabo Santo Agostinho; de maneira que inferiu havermos sido
arrastados à proximidade da costa da Guiana, ou parte norte do Brasil,
para além do rio Amazonas, em direção ao rio Orinoco, comumente chamado
de Grande Rio; e começou a consultar"-me sobre o curso que deveria tomar,
pois o navio fazia água e ia muito avariado, e ele estava voltando
diretamente à costa do Brasil.

Pus a isso grande objeção; e, procedendo ao seu lado com o exame das
cartas da costa da América, concluímos que não havia país habitado a que
recorrer até que chegássemos ao círculo das Ilhas Caraíbas; e decidimos
seguir rumo às ilhas Barbados, travessia que podíamos fazer, como
esperávamos, em obra de quinze dias, desde que navegássemos em mar
aberto e evitássemos assim a corrente de ar da baía ou golfo do México;
considerando que não tínhamos condições de fazer nossa viagem à costa de
África sem algum auxílio, fosse para o navio, fosse para nós mesmos.

Com esse projeto, mudamos nosso rumo e nos afastamos em sentido
oés"-noroeste para alcançar algumas de nossas ilhas inglesas, onde eu
esperava encontrar ajuda; mas nossa viagem foi determinada de outra
maneira; pois, estando na latitude de doze graus e dezoito minutos, uma
segunda tempestade nos sobreveio, arrastando"-nos com a mesma violência a
oeste, e afastando"-nos de tal maneira do caminho de todo comércio humano
que, se todas as nossas vidas tivessem sido salvas no mar, mais provável
era que morrêssemos devorados por bárbaros do que retornássemos ao nosso
país.

Em tamanho desalento, com o vento ainda soprando muito forte, um dos
nossos marinheiros de manhã cedo anunciou: ``Terra!'', e mal havíamos
saído da cabine para observar, na esperança de ver nosso paradeiro, o
navio encalhou num banco de areia e, num momento em que se encontrava
imóvel, quebrou tão grosso mar sobre ele que pensamos que todos haviam
perecido; e fomos imediatamente obrigados a nos abrigar da espuma e das
ondas do mar na alheta à popa.

Não é fácil para quem não conheceu semelhante condição descrever ou
conceber a consternação dos homens em circunstâncias tais; não sabíamos
onde estávamos, ou a que terras fôramos arrastados, se a ilha ou
continente, se a sítio habitado ou deserto; e sendo a fúria do vento
ainda grande, embora um pouco menor do que a princípio, desesperamos de
ver que o navio pouco tempo aguentaria sem se fazer em pedaços, a menos
que os ventos, por obra de milagre, mudassem naquele instante. Em suma,
permanecemos sentados olhando uns para os outros, à espera de a morte
sobrevir"-nos a todo momento, e com cada marujo cuidando de preparar"-se
ao outro mundo; pois havia pouco ou nada a fazer então; sendo nosso
conforto presente, e este todo o conforto que tínhamos, que, ao
contrário de nossa expectativa, o navio ainda não se havia partido, e o
capitão dissera que o vento abrandava.

Pois bem: embora pensássemos que o vento começava a amainar, o navio
permanecia ao seco preso e, encalhado demais para que tivéssemos
esperanças de arrancá"-lo dali, encontrávamo"-nos em circunstâncias de
fato terríveis, e nada tínhamos a fazer senão salvarmos nossas vidas da
maneira que pudéssemos; tínhamos um bote preso à popa, antes da
tempestade, mas este recebera avarias ao ser lançado primeiramente
contra o leme do navio, partindo"-se em seguida e tendo afundado ou sido
levado pelo mar, de maneira que não se podia derivar qualquer esperança
dele; tínhamos ainda um segundo bote a bordo, mas como levá"-lo ao mar
era grande dúvida; mas não havia espaço para discussão, pois a todo
minuto imaginávamos que o navio far"-se"-ia em pedaços, e alguém dissera
que já estava em verdade perdido.

No meio de tamanho infortúnio, o imediato de nossa embarcação conseguiu
segurar o bote, e com a ajuda dos demais marinheiros, conseguiu descê"-lo
pelo costado do navio; com todos nele embarcados, soltamo"-lo e, em
número de onze, entregamo"-nos à misericórdia de Deus e à fúria do
elemento; pois, embora a tempestade tivesse diminuído consideravelmente,
o mar quebrava muito encapelado na costa e bem podia ser chamado de
\emph{Den wild Zee}, como os holandeses chamam o mar numa tempestade.

E agora nosso caso tornava"-se de fato desesperador; pois se fazia
evidente a todos que o mar crescera de tal maneira que o bote não
suportaria esforço tamanho, e nosso afogamento era certo. Quanto a
esticar velas, não as tínhamos, nem se as tivéssemos poderíamos ter
feito algo com elas; por isso, trabalhamos nos remos em direção à terra,
embora com um aperto no coração, como homens a caminho do patíbulo; pois
todos sabíamos que, quando acerca da costa, nosso bote seria destruído
em mil pedaços sob a força das ondas. Entregamos, no entanto, nossas
almas a Deus da maneira mais sincera; e com o vento empurrando"-nos o à
costa, aceleramos nossa destruição com nossas próprias mãos, remando
tanto quanto podíamos em direção à terra.

O que era a costa, se de pedra ou de areia, se íngreme ou rasa, não
sabíamos. A única esperança que poderia racionalmente nos dar a menor
sombra de expectativa era dar com alguma baía ou golfo, ou com a foz de
algum rio, aonde, muito por acaso, poderíamos entrar com nosso bote e
ficar sob a proteção da terra, talvez em águas calmas. Mas não havia o
que a isso se assemelhasse; e à medida que nos avizinhávamos da costa, a
terra se nos parecia mais assustadora que o mar.

Depois de remarmos, ou melhor, ficarmos à deriva por obra de uma légua e
meia, como prevíamos, uma furiosa onda cresceu à nossa popa, com seu
rolo imenso como uma montanha, e claramente nos pediu que esperássemos
pelo golpe de misericórdia. Em suma, a onda arrastou"-nos com tamanha
fúria que o bote de uma só vez tragado; e separou"-se tanto da embarcação
quanto uns dos outros, não nos dando tempo sequer para que disséssemos
``Ó Deus!'', pois fomos todos engolfados num instante.

Não há o que descreva a confusão de pensamentos que senti quando
afundei; pois, apesar de ser bom nadador, não consegui emergir das ondas
para tomar fôlego, senão depois de a onda me levar, ou melhor, arrastar
por longo caminho até a praia, e depois de se ter esgotado e recuado,
deixando"-me quase em terra seca, mas praticamente morto da água que
engolira. Não me restava mais presença de espírito do que fôlego, pois
ao me ver mais perto de terra do que esperava, levantei"-me e tentei
caminhar em direção à terra o mais rápido que pudesse, antes que outra
onda regressasse e novamente me arrastasse; mas logo descobri que era
impossível evitá"-lo; pois vinha o mar atrás de mim tão alto quanto uma
imensa colina e com a fúria de um inimigo contra o qual não tinha meios
ou forças de me defender; minha tarefa era prender a respiração e boiar
na superfície da água, se capaz fosse; e, assim nadando, preservar minha
respiração e fazer"-me piloto de mim mesmo em direção à costa, se
possível; sendo minha maior preocupação então que a onda, que me levaria
uma boa distância rumo à praia quando arremetesse à terra, não me
levasse de volta consigo quando regressasse ao mar.

A onda que me sobreveio sepultou"-me a uma profundidade de vinte ou
trinta pés em seu próprio corpo; e eu pude sentir"-me arrastado por uma
boa distância em direção à praia com grande violência e rapidez; mas
prendi a respiração e ainda me vali das forças que ainda tinha para
nadar mais além. Estava prestes a perder o fôlego quando, ao sentir que
emergia, para o meu alívio imediato, dei com minha cabeça e mãos
arremessadas acima da superfície da água; e embora não tenha conseguido
manter"-me nessa condição por dois segundos, foi grande o alívio, donde
tomei fôlego e recobrei o espírito. As águas me mais uma vez me cobriram
algum tempo, mas não além de minhas forças; e, vendo que se haviam
esgotado e começado a recuar, pus"-me a avançar contra o regresso das
ondas e senti o chão novamente sob os pés. Fiquei parado alguns
instantes para recuperar o fôlego, e até que as águas de mim se
afastaram e corri à praia com as forças que ainda tinha. Nisso, porém,
não encontrei livramento da fúria do mar, que veio ainda outra vez em
meu encalço; e mais duas vezes fui levado pelas ondas e trazido de
volta, sendo a praia muito plana.

A última dessas duas vezes foi quase fatal; pois o mar, tendo me
arrastado como antes, deixou"-me, ou melhor, arrojou"-me contra uma rocha,
e tamanha foi a violência que quedei sem sentidos, e em verdade
desenganado quanto ao meu salvamento; pois a pancada, atingindo"-me o
flanco e o peito, fez com que perdesse todo o ar; e tivessem as ondas
retornado de pronto, teriam me afogado; mas recuperando"-me um pouco
antes do regresso das ondas e, vendo que novamente seria coberto com a
água, decidi agarrar"-me a uma rocha e, assim, segurar a respiração, se
possível, até que a onda recuasse; e assim, como as ondas já não rolavam
tão altas quanto de início, e estava eu mais próximo de terra, segurei
firme até que as ondas abrandaram e pus"-me, então, a correr mais uma
vez, o que me levou tão acerca da praia que a onda seguinte, embora
tenha me coberto, não o fez a ponto de me tragar; e na corrida seguinte,
cheguei à terra firme, onde, para meu grande conforto, subi os penedos
da costa e sentei"-me no relvado, livre de perigo e fora do alcance da
água.

Ao ver"-me em terra firme e a salvo na praia, comecei a olhar aos céus e
agradecer a Deus por minha vida ter sido poupada em circunstâncias nas
quais, havia poucos minutos, não restava qualquer esperança. Julgo
impossível dar expressão, para a vida, ao que são os êxtases e
transportes da alma quando esta encontra a salvação, pode"-se dizer, de
dentro da própria sepultura: e agora não me surpreende o costume, isto
é, quando um malfeitor, já com o nó da corda já apertado em torno do
pescoço, encontra"-se amarrado e prestes a dar adeus a este mundo e então
recebe uma comutação da pena; isto é, não me surpreende que nesse exato
instante tragam"-lhe também um cirurgião que o sangra, de maneira que o
espanto não lhe roube o espírito animal do coração e o mate:

\emph{Pois as alegrias súbitas, como a dor, de pronto arrebatam.}

Caminhei pela praia com as mãos ao alto e todo o meu ser, como é
possível dizer, transportado na contemplação de meu salvamento, fazendo
mil gestos e movimentos que não posso descrever, pensando em todos os
meus companheiros perdidos, e que não teria restado uma única alma viva
além de mim mesmo; pois, quanto a eles, nunca mais os vi, ou qualquer
sinal deles, exceto três de seus chapéus, um gorro e dois sapatos que
não eram do mesmo par.

Quis avistar o navio encalhado, mas a rebentação e a espuma do mar iam
muito altas e mal conseguia vê"-lo, tão longe estava; e pensei, Senhor!,
como havia conseguido chegar à praia?

Depois de acalmar minha mente com o quinhão confortável de minha
condição, comecei a olhar em meu entorno para examinar o lugar em que
estava e como deveria proceder a seguir; e logo vi meu conforto
arrefecer e que, em uma palavra, tivera um salvamento terrível; pois
estava molhado, não tinha roupas para vestir, nem nada de comer ou beber
que me trouxesse alívio; nem nenhuma perspectiva diante de mim senão a
de morrer de fome ou ser devorado por animais selvagens; e o que me era
particularmente aflitivo é que eu não tinha arma, nem para caçar, nem
para matar qualquer criatura para meu sustento, nem para me defender de
qualquer outra criatura que desejasse me matar para o próprio. Em uma
palavra, não tinha nada além de uma faca, um cachimbo e um pouco de
tabaco em uma caixa. Essas eram todas as minhas provisões; e isso me
levou a terríveis agonias da mente, tais que por algum tempo caminhei de
um lado para o outro como se tivesse perdido a razão. No cair da noite,
comecei a considerar, com um aperto no coração, o que seria da minha
sorte se houvesse bestas vorazes naquele lugar, pois à noite elas sempre
saem ao exterior em busca de suas presas.

O remédio que se ofereceu a meus pensamentos naquele momento foi subir
em uma árvore alta, copada como um abeto, mas espinhosa, que perto de
mim crescia, e onde decidi permanecer sentado a noite toda, e no dia
seguinte refletir sobre a morte que haveria de morrer, pois ainda não
via como sobreviver; e caminhei obra de duzentas jardas a partir da
costa com a intenção de encontrar água fresca para beber, tendo"-a
encontrado, para minha grande alegria; e, tendo bebido e posto um pouco
de tabaco na boca para evitar a fome, fui à árvore e, subindo"-a,
esforcei"-me por posicionar"-me de maneira tal que, se dormisse, não
cairia. E, tendo"-me cortado um bastão curto, como um porrete, para minha
defesa, abriguei"-me; e, exaurido que estava, ferrei no sono e dormi tão
confortavelmente quanto, creio, poucos poderiam ter feito em minha
condição, despertando revigorado como jamais pensei que pudesse me
sentir em circunstância como aquela.

Quando acordei, era dia claro, fazia bom tempo, e a tempestade amainara,
de modo que o mar não crescia e enfurecia como antes. Mas o que mais
espantou me causou foi o fato de a cheia da maré ter arrancado, durante
a noite, o navio ao banco de areia onde encalhara e o levado quase até a
rocha que mencionei atrás, onde tanto me ferira ao ser arrojado pela
força das águas contra ela; e estando a obra de uma milha do ponto da
praia onde me encontrava, e o navio parecendo imóvel, quis subir a
bordo, para que ao menos pudesse conservar algumas coisas necessárias
para o meu uso.

Quando desci de meu refúgio na árvore, olhei em volta de mim, e a
primeira coisa que avistei foi o navio, que jazia, uma vez que o vento e
o mar o arremeteram contra a praia, a obra de duas milhas à minha
direita. Eu segui tanto quanto pude por terra para chegar até ele; mas
encontrei um braço ou uma entrada de mar entre o navio e mim, com obra
de meia milha em largo; por ora, então, recuei, intentando alcançar o
navio, onde esperava encontrar algo para meu imediato mantimento.

Pouco depois do meio dia o mar estava muito calmo, e a maré tanto recuou
que pude chegar a um quarto de milha do navio; e aqui senti renovar"-se
em mim a minha dor; pois era com clareza que via que, se tivéssemo"-nos
mantido a bordo, estaríamos todos a salvo, ou seja, todos seguros em
terra, e eu não me veria em tamanha aflição, inteiramente destituído de
todo conforto e companhia como então me encontrava. Isso trouxe mais uma
vez lágrimas aos meus olhos; mas como nelas pouco alívio havia, resolvi,
se possível, chegar ao navio; ao que tirei minhas roupas, pois estava o
tempo quente ao extremo, e entrei na água; mas quando cheguei ao navio,
vi"-me em dificuldade ainda maior de não saber como embarcar; pois,
estando o navio encalhado, e alto em relação à superfície da água, nada
havia ao meu alcance em que me segurasse. Nadei em volta dele duas vezes
e, na segunda vez, dei com um pedacinho de corda, que me surpreendeu não
ter visto a princípio, pendurado da mesa de enxárcia a proa e muito
baixo, e que com muita dificuldade agarrei, e com a ajuda dessa corda
subi no castelo de proa do navio. Ali descobri que o navio fazia água e
tinha grande quantidade de água no porão, mas que estava tão arrimado ao
lado de um banco de areia dura, ou talvez terra, que sua popa se erguia
sobre tal banco, e a proa quase mergulhava na água; de forma que toda a
quadra de popa estava intacta, e tudo quanto naquela parte havia,
enxuto; pois tenha certeza de que foi minha primeira tarefa pesquisar e
ver o que se perdera e o que havia em condições de uso; e, primeiro,
descobri que todas as provisões do navio estavam secas e intocadas pela
água, e estando com muita fome, fui à despensa, enchi meus bolsos de
bolacha e pus"-me a comer enquanto fazia outras coisas, pois não tinha
tempo a perder. Também encontrei um pouco de rum na cabine do capitão,
do qual tomei generoso trago e do qual, de fato, necessitava para
recobrar o espírito com o que tinha diante de mim. Tudo o que quis
naquele momento foi um bote para aperceber"-me das muitas coisas que eu
previa que me seriam muito necessárias.

Era vão ficar parado e desejar o que não se poderia ter; e o extremo em
que me encontrava aguçou"-me a atenção; tínhamos várias vergas de
reposição, e duas ou três grandes traves de madeira e um ou dois
mastaréus sobressalentes; decidi começar a trabalhar por eles, e
lancei"-os ao mar tantos quantos cujo peso pude suportar, amarrando"-os
todos com uma corda para que não se extraviassem; feito isso, desci pelo
costado do navio e, trazendo"-os para mim, amarrei quatro deles em ambas
as extremidades tão bem quanto pude, à guisa de jangada, e coloquei na
transversal dois ou três pedaços pequenos de tábua sobre eles,
descobrindo que podia andar muito bem sobre eles, mas que não eram
capazes de suportar muito peso, uma vez que as peças eram muito leves;
pus, então, mãos à obra e, com a serra de um carpinteiro, cortei um
mastaréu sobressalente em três comprimentos e os acrescentei à minha
balsa, ao que dediquei muito trabalho e esforço. A esperança de me
suprir do necessário, porém, animava"-me a ir além do que teria sido
capaz de fazer em outras circunstâncias.

Tendo feito uma jangada forte o bastante para suportar bom peso, cuidei
a seguir das coisas com que carregá"-la, e de como preservar das ondas do
mar o que nela depositava; mas não me demorei em pensamentos sobre isso,
e logo tratei de colocar sobre ela todas as pranchas ou tábuas que pude
obter e, tendo bem ponderado acerca do que me era mais necessário,
peguei três dos baús dos marinheiros, que havia aberto e esvaziado, e os
botei em minha jangada; enchendo o primeiro deles com provisões, a
saber, pão, arroz, três queijos holandeses, cinco pedaços de carne seca
de cabra, da qual dependia muito de nosso sustento, e um pouco dos grãos
europeus que restavam e que ali estavam por causa de algumas aves que
trouxemos para o mar conosco, mas as aves estavam mortas. Havia uma
mistura de cevada e trigo, mas, para minha grande tristeza, descobri por
fim que os ratos haviam comido ou estragado tudo; quanto à bebida,
encontrei várias caixas de garrafas pertencentes ao nosso capitão, em
meio às quais havia tônicos; e, no total, obra de cinco ou seis galões
de aguardente ordinária, os quais deixei à parte, não havendo
necessidade de colocá"-los no baú, nem espaço para eles. Enquanto fazia
isso, percebi que a maré começa a fluir, sem tumulto; e tive a tristeza
de ver meu paletó, minha camisa e colete, que havia deixado na areia,
serem levados pelas águas; quanto aos meus calções, que eram de linho e
iam até os joelhos, nadei até o navio com eles e minhas meias. Isso,
porém, me levou à busca de roupas, das quais encontrei boa quantidade,
mas que não recolhi senão em número bastante para o presente uso, pois
havia outras coisas que no momento mais me interessavam, como, em
primeiro lugar, ferramentas para o trabalho em terra, e foi depois de
longa procura que encontrei o baú do carpinteiro, este, em verdade, uma
conquista muito útil e muito mais valiosa para mim do que uma carga de
ouro teria sido naquele momento. Coloquei"-o em minha jangada, inteiro
como estava, sem perder tempo para examiná"-lo, pois sabia em geral o que
ele continha.

Ocupei"-me em seguida de encontrar munição e armas. Havia duas
espingardas de caça na cabine principal e duas pistolas, todas muito
boas armas, as quais busquei primeiro, com alguns chifres de pólvora,
uma pequena carga de chumbo e duas velhas espadas enferrujadas; também
tinha notícia de três barris de pólvora no navio, mas não sabia onde
nosso artilheiro os havia guardado; com muita dificuldade eu os
encontrei, dois deles secos e utilizáveis, o terceiro molhado; e os dois
primeiros eu levei à jangada, mais as armas de fogo, e assim me vi
carregado de muito bom aviamento e comecei a pensar em como chegaria à
costa com ele, sem vela, remo ou leme; e a menor refega de vento teria
levado ao fundo toda a minha navegação.

Encontrei arrimo em três coisas, a saber: a primeira, o mar calmo e
próspero; a segunda, a maré enchente avançando sobre a praia; a
terceira, o pouco vento que soprava de mar; e assim, tendo encontrado
dois ou três remos quebrados pertencentes ao bote, e além das
ferramentas que estavam no baú, mais duas serras, um machado e um
martelo; assim fretado lancei"-me ao mar. Por uma milha ou perto disso,
minha jangada fez muito boa viagem, senão por se alongar, segundo o
percebia, um pouco do ponto em que antes encontrara pouso; donde inferi
que havia algum influxo d'água, e consequentemente tive esperança de
encontrar riacho ou rio ali que pudesse fazer de porto para o
desembarque de minhas provisões.

Como eu imaginava, avistei uma pequena abertura da terra e senti uma
forte corrente de maré que me arrastava em sua direção; e governei minha
jangada o melhor que pude, então, de guisa que permanecesse ao meio da
correnteza. Mas aqui estive a ponto de sofrer um segundo naufrágio, o
qual, tivesse ocorrido, creio que me haveria definitivamente alquebrado
o ânimo; pois, nada conhecendo da costa, minha jangada deu com uma das
extremidades num baixio, enquanto a outro permaneceu solta a flutuar; e
por muito pouco toda a minha carga não correu à ponta livre e não foi ao
fundo. Fiz grande esforço arremetendo minhas costas contra os baús para
mantê"-los em seus lugares, mas não conseguia usar de minha força para
soltá"-la, nem deixar a posição em que me encontrava; e amparando os baús
com todo o vigor, assim permaneci por quase meia hora, espaço em que a
maré enchente fez a jangada recuperar um pouco o nível e, a seguir, com
a água subindo ainda mais, ela se pôs a flutuar novamente, e eu a
empurrei com o remo que tinha, enfiando"-o no leito do canal. Mais
adiante por fim, achei"-me na foz de um riacho, e com terra de ambos os
lados e uma forte corrente ou maré movendo"-me água acima, procurei de um
lado e de outro sítio adequado ao desembarque; pois não desejava
adentrar demasiado a terra e esperava, a seu tempo, avistar algum navio
no mar. Portanto, decidi fazer abrigo tão acerca da costa quanto
pudesse.

Dei vista, então, de uma pequena enseada na margem direita do riacho, à
qual com grande custo guiei minha jangada, e então dela tanto me
aproximei que, tocando a terra com o remo, podia arribar, mas aqui mais
uma vez quase teria entregue minha carga às águas; pois sendo a margem
íngreme, ou seja, ribanceira, não havia ponto de desembarque; e se uma
das extremidades da minha jangada nela montasse, alta ficaria a ponto de
afundar a outra, o que colocaria em risco, como antes, meus aviamentos.
Tudo o que eu podia fazer era esperar até que a maré enchesse de todo,
usando o remo à maneira de âncora para conservar um dos lados da jangada
próximo à ribeira, perto de pedaço plano de chão que, esperava eu, a
água cobrisse; e assim foi. Tão logo percebi água o bastante, pois minha
jangada tinha obra de dez polegadas de calado, fi"-la avançar naquele
terreno; e ali atraquei, enfiando meus dois remos quebrados no chão, um
em cada ponta da jangada, mas em lados opostos; e assim descansei até
que a maré vazasse, e então deixei minha jangada e todo o carregamento
em porto seguro.

A tarefa seguinte era de haver vista do lugar e buscar sítio adequado em
que me abrigasse, e em que guardasse meus bens para protegê"-los do que
quer que lhes pudesse suceder; pois onde eu estava, ainda não sabia; se
fosse em continente, se fosse em ilha; habitada ou desabitada; à mercê
ou não de animália brava. Assomava uma colina a menos de uma milha de
onde eu estava, que se erguia muito íngreme e alta e que parecia encimar
outras colinas, as quais ao norte se perfilavam numa serra. Tomei uma
das espingardas, uma das pistolas e um chifre de pólvora; e assim
armado, perfiz a viagem de descobrimento ao cume da dita colina, ao qual
cheguei depois de muitos trabalhos, declarando"-se, então, minha fortuna
de grande aflição, a saber, que eu estava em uma ilha cingida de mar por
toda a parte, sem terra que se entrevisse, senão alguns penedos, estes
muito afastados; e dois ilhéus ainda menores, distantes obra de três
léguas a oeste.

Descobri também que a ilha em que eu estava não tinha cultivo e, vendo
boas razões para acreditar, era despossuída de gente, exceto pelos feros
animais, dos quais, no entanto, vi nenhum, declarando"-se, contudo, a
abundância de aves, sem que lhes soubesse as castas; tampouco se eram de
comer ou não. Quando retornava, descarreguei contra um grande pássaro
que vi empoleirado em uma árvore da banda de uma vasta mata. Penso ter
sido a primeira arma ali disparada desde a Criação do mundo; mal se
fizera o estampido, e de todas as partes da mata grande cópia de
pássaros alevantou voo, pássaros de maneiras várias, que produziram uma
confusão de grasnares e que gritaram cada qual segundo seu costumeiro
canto; nenhum dos quais, porém, de mim conhecido. Quanto à criatura que
matei, julguei ser um gênero de falcão, de cor e bico semelhantes, mas
sem as presas ou garras mais do que comuns. Sua carne era repulsiva e de
nada servia.

Satisfeito com o que descobrira, voltei à jangada e pus"-me a trabalhar
na transposição de minhas provisões e aviamentos à terra, o que me tomou
o resto do dia; e o que seria de mim à noite, eu não sabia dizer, e
tampouco onde descansaria, visto que tinha medo de me deitar ao chão,
sob o risco de algum animal selvagem devorar"-me, embora, como vim a
descobrir mais tarde, tais medos não se justificassem.

Esforcei"-me, contudo, para cercar"-me dos baús e tábuas que trouxera à
terra, com os quais fiz uma maneira de cabana para agasalhar"-me à noite;
quanto à comida, não vira ainda meios de abastecer"-me, exceto por duas
ou três criaturas semelhantes a lebres, as quais correram à floresta em
que eu matara a ave.

Passei então a pensar que ainda muitas coisas proveitosas a mim poderiam
ser buscadas no navio, especialmente parte do cordame, as velas e outros
utensílios passíveis de serem carregados à terra; e assim decidi
realizar, se possível fosse, nova viagem ao navio; e como sabia que a
primeira tempestade que soprasse haveria de fazê"-lo em pedaços, resolvi
delongar tudo o mais até que conseguisse retirar tudo o que pudesse ao
navio. Convoquei, assim, um conselho, isto é, em meus pensamentos, para
deliberar se devia ou não fazer a travessia de volta com a jangada; o
que me pareceu impraticável: decidi, então, ir como antes fora, na maré
baixa; e assim o fiz, exceto por ter me despido antes de deixar a
cabana, saindo sem mais do que uma camisa em tecido xadrez, um par de
calções de linho e um par de escarpins nos pés.

Subi a bordo do navio como antes o havia feito e preparei uma segunda
jangada; e, tendo a experiência da primeira, nem a fiz tão difícil de
manejar, nem a carreguei muito, embora ainda assim tenha levado comigo
muitas coisas de proveito; como antes, na oficina do carpinteiro
encontrei duas ou três sacolas cheias de pregos e cravos, um ótimo
macaco de rosca, uma dúzia ou duas de machadinhas e, acima de tudo,
aquela coisa muito útil chamada rebolo, ou pedra de amolar; reunidas
todas essas coisas, incluí outras pertencentes ao artilheiro,
especialmente dois ou três pés de cabra e dois barris de balas de
mosquete, sete mosquetes, outra espingarda de caça, com uma pequena
quantidade de pólvora; uma sacola grande cheia de metralhas e um grande
rolo de folhas de chumbo; mas este último era tão pesado que eu não fui
capaz de içá"-lo para que passasse a amurada do navio.

Ademais, recolhi todas as roupas masculinas que encontrei e uma vela da
gávea de traquete sobressalente, uma maca de dormir e outros artigos de
cama; e com isso carreguei minha segunda jangada, e todas essas coisas
chegaram a salvo em terra, para meu grande alívio.

Preocupava"-me, durante a minha ausência de terra, que minhas provisões
pudessem ter sido devoradas na praia; mas, quando retornei, não
encontrei vestígio de visitante, apenas uma criatura semelhante a um
gato selvagem sobre um dos baús, a qual, quando me aproximei, correu a
curta distância e então parou. Ela sentou"-se muito tranquila e
indiferente e olhou"-me diretamente no rosto, como se tivesse a pretensão
de travar contato comigo; eu apresentei"-lhe minha arma; mas, como ela
não compreendeu sua utilidade, conservou"-se absolutamente
desinteressada, nem sinalizou que se afastaria; ao que lancei a ela um
pedaço de bolacha, embora, a propósito, dele não pudesse me servir à
vontade, pois não tinha bom provimento: de qualquer modo, eu lhe dei um
pedaço, ao qual ela acorreu e cheirou, comeu e (como se tivesse gostado)
procurou por mais; ao que a agradeci, sem poder oferecer"-lhe mais; e ela
se foi.

Tendo trazido meu segundo carregamento à terra firme, embora tenha sido
obrigado a abrir os barris de pólvora e transportar seu conteúdo em
partes, por grandes e pesados que eram, passei a trabalhar no erguimento
de uma pequena tenda com a vela e alguns postes que cortei para tal fim:
e a essa tenda trouxe tudo o que sabia que se poderia estragar, fizesse
chuva ou sol; e empilhei todos os baús e barris vazios num círculo em
torno da tenda, o que fiz com a tenção de fortificá"-la contra qualquer
ataque súbito, de homem ou animal.

Assim feito, reforcei a porta da tenda com paus por dentro e um baú
vazio do lado de fora; e estendendo uma das camas no chão, deixei as
duas pistolas ao lado de minha cabeça e minha espingarda ao meu largo,
deitando"-me pela primeira vez e dormindo pacificamente a noite toda,
pois estava extenuado e preocupado; havendo pouco descansado na noite
anterior e trabalhado muitíssimo o dia inteiro para buscar e trazer à
terra tantas coisas.

Tinha, então, creio eu, o maior armazém já reunido por um só homem; mas
ainda não estava satisfeito, pois porquanto o navio ali permanecesse,
julguei ser preciso tirar dali tudo que pudesse carregar; e assim, todos
os dias, na maré baixa, subia a bordo e levava comigo uma ou outra
coisa. Na terceira visita, em particular, carreguei tudo que pude
recolher aos cabos do aparelho do navio e, com eles, todas as pequeninas
amarras e gaxetas, além de um pedaço de lona sobressalente, que servira
ao reparo ocasional do velame, e do barril de pólvora que se havia
molhado. Por fim, levei comigo todo o velame, tendo de cortá"-lo em
pedaços, porém, e carregá"-lo na maior quantidade possível a cada viagem;
visto que não se aproveitavam mais como velas, mas sim como lona.

Mas o que me serviu de remédio maior foi que, ao fim de tudo, depois de
tê"-las feito em número de cinco ou seis, e pensando não mais haver no
navio coisas dignas de meu interesse; pois bem, ao fim encontrei um
enorme barril de pão, três pipas de rum ou aguardente, uma caixa de
açúcar e um casco de farinha branca; o que muito me espantou, porque já
não mais esperava encontrar sortimentos, senão os já arruinados pela
água. Logo esvaziei o barril dos pães e os embrulhei pedaço por pedaço
em retalhos das velas que eu cortara; transportando, então, tudo em
segurança à terra.

No dia seguinte, fiz outra viagem; e então, tendo retirado ao navio tudo
quanto fosse portátil e passível de manejo, cuidei das amarras; e
cortando o grande cabo da âncora em pedaços tais que pudesse
transportá"-los, levei ainda dois cabos de atracação e uma espia à praia,
além de todas as ferragens que pude; e depois de serrar a verga da
cevadeira, a verga de mezena e o que mais estivesse ao meu alcance para
construir uma grande jangada, esta eu carreguei de todos esses artigos
pesados e parti. Mas minha boa fortuna começou então a me abandonar;
pois a jangada ia tão sobrecarregada e difícil de governar que, depois
de adentrar a pequena enseada onde havia desembarcado os meus demais
aviamentos, não sendo capaz de conduzi"-la com a destreza de antes, ela
virou, lançando a minha carga e a mim na água. Eu próprio não sofri
maior ferimento, pois estava próximo à margem; quanto à minha carga,
porém, houve grande perda, especialmente dos ferros, os quais julgava
que seriam de grande utilidade para mim. De qualquer modo, quando a maré
baixou, fui capaz de resgatar a maior parte dos pedaços do grande cabo,
bem como parte dos ferros, não obstante me tenham exigido infinito
trabalho; pois fui obrigado a mergulhar na água, exercício que me causou
enorme fadiga. Depois disso, subi todos os dias a bordo e trouxe comigo
o que era capaz de levar.

Fazia treze dias que eu estava na praia, e onze vezes havia estado a
bordo do navio; e nesse espaço carregara comigo tudo quanto um par de
mãos fosse capaz de carregar, embora creia piamente que, tivesse o tempo
permanecido firme, teria transportado o navio inteiro peça a peça.
Enquanto me preparava para a décima"-segunda subida, percebi que o vento
soprava mais forte; ainda assim, na vazante da maré segui viagem e,
embora pensasse ter vasculhado a cabine tão completamente que nada mais
pudesse se encontrar, deparei"-me com um armário com gavetas. Dentro de
uma delas, estavam algumas lâminas de barbear e um par de tesouras, além
de uma dúzia de bons garfos e facas; dentro de outra, uma quantia de
trinta e seis libras esterlinas em papel, moedas europeias e dos Brasis,
\emph{pesos fuertes} e um pouco de ouro e de prata.

Ao ver esse dinheiro, sorri. ``Ó droga!'', exclamei, ``que serventia
tendes vós? De nada valeis para mim, tampouco carregar"-vos; não comprais
uma dessas facas; não tenho onde empregar"-vos; assim, permanecei onde
estais e segui ao fundo do mar como criatura indigna de salvamento''.
Depois de refletir um pouco, porém, levei"-o comigo; e embrulhando tudo
num pedaço de lona, comecei a pensar em construir uma nova jangada; mas
enquanto me preparava para tanto, deparou"-me o céu nublado, e o vento
embraveceu; e em quinze minutos forte ventania levantou da costa.
Ocorreu"-me, então, que era inútil tentar construir uma jangada no mar
naquelas condições; e que devia, sim, partir antes da subida da maré,
caso contrário eu talvez não fosse capaz de chegar à praia. Assim,
saltei ao mar e atravessei a nado o braço de mar de entre o navio e a
praia, o que não fiz sem dificuldade, em parte pelo peso das coisas que
trazia no meu entorno; e, em parte pela aspereza das águas; pois o vento
soprava muito feroz, e antes de a maré subir por completo, desabou uma
tempestade.

Eu já havia chegado a minha pequena tenda, onde descansei tendo comigo a
salvo todos os meus bens. Aquela noite inteira foi castigada por ventos
muito fortes; e, de manhã, quando olhei para fora, não havia mais navio
à vista; causou"-me um pouco de espanto, mas me recuperei com uma
reflexão que muito me satisfez, a saber, que não havia perdido tempo,
nem arrefecido em minha diligência para dali tirar tudo quanto me
pudesse ser útil; e que, de fato, restara pouco no navio que eu pudesse
trazer, caso tivesse mais tempo.

Deixei, então, de dedicar meus pensamentos ao navio ou a qualquer coisa
que dele pudesse vir, exceto pelos destroços que porventura chegassem à
praia; como, de fato, chegaram, em fragmentos os mais variados; mas
essas coisas me foram de pouca utilidade.

Meus pensamentos ocuparam"-se, então, de minha proteção contra os
selvagens, caso viessem a aparecer, e as feras, caso na ilha houvesse; e
eu tinha muitos pensamentos sobre o método de levá"-la a cabo e a guisa
de morada a ser construída, se devia cavar buraco na terra ou erguer
tenda no chão. Em suma, decidi"-me sobre ambos os temas, e não será
impróprio oferecer relação das maneiras e a descrição de como o fiz.

Logo descobri que o sítio em que me encontrava não era adequado ao meu
assentamento, em particular por tratar"-se de terreno baixo e alagadiço,
próximo ao mar, assim como acreditava não ser saudável, sobretudo pela
falta de água doce em sua imediação; decidi, portanto, buscar lugar mais
salubre e conveniente.

Ponderei sobre muitas coisas que, haja vista minha situação,
mostravam"-se"-me adequadas: 1º, a salubridade e a água doce, acima
mencionadas; 2º, ser valhacouto ao calor do sol; 3º, a segurança diante
de predadores, fossem homens ou animais; 4º, a visão do mar, pois se
Deus colocasse qualquer navio à vista da praia, eu não perderia qualquer
oportunidade de salvamento, cuja expectativa ainda não me via inclinado
a abandonar de todo.

Em demanda de lugar próspero a tanto, avistei uma campinazinha que
ladeava um morro comprido, cuja face de frente ao terreno era íngreme
como uma parede, de modo que de cima nada me poderia cair de assalto;
enquanto em seu flanco abria"-se um côncavo, um tanto desgastado como se
fosse uma entrada ou porta de caverna, sem que houvesse uma caverna ou
verdadeiramente um caminho rocha adentro.

Na relva que havia à frente desse côncavo, decidi armar minha tenda. A
planície em questão não tinha cem passos em largo, nem chegava a se
alongar por duas vezes esse tanto, e se estendia como um gramado diante
da minha porta; contornado por todos os lados por declives muito
acidentados que chegavam à terras baixas de beira"-mar. Estava a
nor"-noroeste da colina; de modo que permanência protegido do calor o dia
inteiro, até que este viesse de sudoeste ou de sua imediação, o que,
nessas regiões, dá"-se perto do poente.

Antes de erguer minha tenda, desenhei um semicírculo diante da
concavidade da rocha, o qual tinha obra de dez passos de semirraio a
partir da rocha, e vinte de lado a lado.

Ao longo do dito semicírculo, finquei duas fileiras de estacas robustas,
enfiando"-as no chão até se declararem firmes como esteios, tendo as
pontas afiadas e medindo, desde o chão, obra de cinco pés e meio. As
duas fileiras não ficavam a uma distância de um palmo uma da outra.

Vali"-me dos pedaços do cabo de âncora que cortara no navio para
colocá"-los de comprido, uns sobre os outros, entre as duas linhas de
estacas, do chão ao cimo, cravando outras estacas nesse espaço, as quais
se apoiavam contra elas, como o contraforte de um pilar, com obra de dez
palmos de altura; e essa cerca era tão forte que nem homem, nem besta
eram capaz de penetrá"-la ou atravessá"-la pelo alto. Isso me custou muito
tempo e trabalho, especialmente para cortar as estacas na mata,
trazê"-las ao sítio e fincá"-las na terra.

A entrada ao lugar, fiz de tal modo que não se desse por meio de porta,
mas mediante uso de uma escada curta com a qual atravessava pelo alto e
a qual erguia tão logo passava ao lado de dentro; e assim me encontrava
completamente protegido e fortificado de tudo e todos, e dessa guisa
dormia em segurança durante a noite, o que de outra forma não me seria
possível; embora, como depois me pareceu, não houvesse necessidade de
tamanha prudência com os inimigos dos quais inferia o perigo.

Para o lado de dentro desse cercado ou fortaleza, carreguei toda a minha
riqueza, todos os meus aviamentos, provisões de munição e comida, dos
quais se fez a relação acima; e estendi uma grande tenda, a qual, para
agasalhar"-me das chuvas que naquela parte do ano eram muito violentas,
fiz em dobro, isto é, esticando uma vela menor do lado de dentro e outra
maior acima dela, e revestindo esta última com uma lona alcatroada que
havia guardado entre as velas.

E a partir de então não fiquei mais deitado na cama que trouxera a terra
firme, mas em uma maca, ou rede, de fato muito boa, que pertencera ao
contramestre do navio.

A essa tenda, levei todas as minhas provisões e tudo quanto pudesse se
perder com a umidade; e uma vez protegida toda a minha fazenda, fechei a
passagem, que até então deixara aberta, e desse modo atravessei, como eu
disse, de um lado para o outro com uma escada baixa.

Depois de fazer isso, comecei a abrir caminho rocha dentro e, levando
toda a terra e cascalho que arrancava à pedra através da tenda,
depositei"-a no solo interno ao cercado, à maneira de um terraço, o que
ergueu o nível do chão em três palmos; e assim produzi uma cavidade
atrás de minha tenda, a lhe fazer as vezes de porão.

Custaram"-me muitos trabalhos e dias até que tudo isso fosse
aperfeiçoado, e portanto devo voltar a algumas outras coisas que
ocuparam meus pensamentos. Enquanto realizava meus projetos de
erguimento da tenda e abertura da cavidade interna, sucedeu que uma
tempestade de chuva precipitou"-se de nuvem escura e carregada, e um raio
repentino alumiou"-se, seguido de um imenso trovão, como é natural que
ocorra. Não fiquei tão surpreso com o raio, mas sim com o clarão que,
tão ligeiro quanto a coriscada, acendeu"-se em meus pensamentos: Minha
pólvora! Desesperei ao pensar que, de uma só vez, poderia perder toda a
minha pólvora, da qual dependia, assim o pensava, não apenas para minha
defesa, mas também para o provimento de víveres. De maneira alguma me
havia preocupado com o perigo que corria; embora, caso chegasse fogo à
pólvora, eu jamais viesse a saber o que me havia atingido.

A circunstância causou"-me tamanha impressão que, passada a tempestade,
dispensei"-me de todas as minhas tarefas, meu cercado e fortaleza, e
dediquei"-me à confecção de sacos e caixas para separar a pólvora em
pequenas porções, na esperança de que, a despeito do que houvesse, o
fogo não prendesse em tudo de uma só vez; e de conservar essas porções
tão apartadas entre si que uma não incendiasse a outra. Completei a
tarefa em obra de duas semanas; e creio que minha pólvora, cujo peso
somava como duzentos e quarenta libras, dividiu"-se em não menos de cem
porções; quanto ao barril molhado, supus que não representasse perigo;
depositei"-o, então, em minha nova caverna, que, a meu gosto, chamei de
minha cozinha; e o restante, para que não molhasse, agasalhei em buracos
aqui e ali em meio às rochas, assinalando com muito cuidado onde
colocara.

Enquanto isso se cumpria, saía ao menos uma vez por dia com minha
espingarda, tanto para meu divertimento e ver se era capaz de matar
qualquer coisa digna de ser comida, quanto para familiarizar"-me, tanto
quanto pudesse, com o que a ilha dava. Em minha primeira excursão
descobri a existência de cabras na ilha, o que me trouxe grande
satisfação; esta acompanhada de um infortúnio, a saber, que eram tão
arredias, espertas e ligeiras que achava enorme dificuldade em
aproximar"-me; o que não me desanimou, pois não tinha dúvida de que em
algum momento acertaria uma delas, como logo veio a acontecer, pois,
depois de descobrir algumas de suas paragens de descanso, permanecia a
sua espera desta maneira, isto é, observei que, se me vissem nos vales
enquanto estavam sobre as rochas, fugiam apavoradas; no entanto, quando
se alimentavam nos vales, e eu estava nas pedras, não davam por mim;
donde concluí que, pela natureza de sua óptica, sua visão dirigia"-se tão
para baixo que elas não viam prontamente objetos que estivessem acima
delas. Passei, então, a seguir este método: escalava as rochas para,
primeiramente, elevar"-me acima delas, e então costumava fazer um bom
alvo. Na primeira descarga que dirigi contra essas criaturas, matei uma
cabra que tinha ao seu lado um cabritinho que mamava, o que me causou
grande pena; quando a cabra veio ao chão, o cabritinho permaneceu ao seu
lado até que me aproximei para levá"-la, e não só, como também, ao
carregá"-la sobre meus ombros, o cabritinho seguiu"-me até o cercado,
diante do qual depositei a mãe ao chão e tomei"-o nos braços, levando"-o
para dentro de minha tenda com a tenção de criá"-lo e domesticá"-lo; ele
não comia, contudo, de guisa que fui forçado a matá"-lo e a eu próprio
comê"-lo, e assim ambos me forneceram carne por um bom tempo, pois eu era
parcimonioso em minhas porções e conservava minhas provisões (o pão,
sobretudo) tanto quanto eu podia fazê"-lo.

Tendo construído minha habitação, achei absolutamente necessário me
prover de um espaço em que fizesse fogo e queimassem combustíveis: e
acerca do que então fiz, assim como das obras necessárias à ampliação de
minha caverna e dos ajustes que realizei com vistas a meu bem"-estar,
darei relação completa a seu tempo; em primeiro lugar, porém, devo
tratar brevemente de mim mesmo e de meus pensamentos sobre a vida, que,
como bem se pode supor, não eram poucos.

Era sombria a noção que fazia de minha condição, pois não havendo dado
àquela ilha sem ter sido levado a rasto, como se diz, por violenta
tempestade, que nos desviara muitíssimo da rota pretendida, e assim me
encontrando a grande distância, a saber, de centenas de léguas, da
circulação normal do comércio da humanidade, tinha grandes motivos para
crer que tais circunstâncias eram uma determinação dos Céus, segundo a
qual naquele lugar ermo e deserto haveria eu de acabar meus dias; as
lágrimas escorriam"-me abundantes pelo rosto enquanto fazia tais
reflexões; e por vezes debatia comigo mesmo as razões de a Providência
destruir tão completamente Suas criaturas e torná"-las tão absolutamente
infelizes; tão indefesas, abandonadas e arruinadas que somente alguém
destituído da luz da razão estaria em condições de agradecer por uma tal
vida.

No entanto, sempre deparava uma pronta resposta que refreava esses
pensamentos e me repreendia; em particular um dia, quando caminhava à
beira"-mar com minha espingarda em punho e muito melancólico acerca de
minha condição, a razão, por assim dizer, rebateu meus argumentos nos
seguintes termos: ``É bem verdade que estás em situação de completo
abandono; mas, lembra"-te, onde estão os demais? Não viestes vós em
número de onze no barco? Onde estão os outros dez? Por que não se
salvaram eles, e tu não te perdeste? Por que foste tu o escolhido? É
melhor estar aqui ou lá?'', apontando ao mar. Há de se ver todo mal à
luz do bem que nele assiste, e do pior que lhe faz companhia.

Recobrei, no entanto, os velhos pensamentos relativos a quão bem provido
estava para minha subsistência e qual teria sido o meu destino se o
navio, numa probabilidade de cem mil para um, não tivesse flutuado do
local onde havia encalhado para tão acerca da praia que me houve tempo
de buscar tudo que dentro dele havia? O que teria sido de mim, caso
tivesse sido forçado a viver na condição em que cheguei à praia, sem o
necessário para a vida ou para obtê"-lo e supri"-lo? E mais, disse eu em
voz alta, embora para mim mesmo: o que poderia ter feito sem uma arma,
sem sua munição, sem ferramentas para o fabrico e para o trabalho, sem o
que vestir, sem onde dormir, sem uma tenda ou qualquer forma de abrigo?
E, no entanto, tinha eu todas essas coisas em quantidade bastante e,
além disso, em tão boas condições vivia que era capaz de prover a mim
mesmo sem precisão de minha arma, quando sua munição acabasse; de modo
que tinha uma expectativa tolerável de subsistir sem carestias enquanto
vivesse, uma vez que considerei desde o primeiro momento como
confrontaria os acidentes que me pudessem assaltar e pelo tempo por vir,
não somente o tempo que sucedesse o fim das munições de minhas armas,
mas também quando minha força e saúde declinassem.

Confesso que jamais me ocorrera a possibilidade de minha munição
destruir"-se de uma só vez; digo, de minha pólvora explodir por obra de
um raio; e foi precisamente por isso que o raio e o trovão suscitaram em
mim pensamentos tão assombrosos, como os exponho agora.

E agora, estando a ponto de iniciar a melancólica relação dos quadros de
uma vida silenciosa como possivelmente nunca dantes no mundo se tenha
tido notícia, devo começar do princípio, e proceder com sua ordem de
ocorrência. Segundo a mim consta, foi no dia 30 de setembro que, nas
circunstâncias anteriormente mencionadas, pus pela primeira vez os pés
nesta ilha horrenda, quando o sol, estando para nós em seu equinócio de
outono, encontrava"-se quase acima de minha cabeça, pois recorda"-me que
estava, segundo os cálculos, na latitude de nove graus vinte e dois
minutos a norte da Linha.

Depois de lá estar obra de dez ou doze dias, ocorreu"-me que, à falta de
cadernos, pena e tinta, perderia a noção do tempo, e até mesmo a
diferença entre os dias de trabalho e os dias a serem guardados; mas,
para evitar que isso acontecesse, talhei com minha faca em letras
maiúsculas num poste, o qual transformei em enorme cruz fixada na praia
a que as águas do mar me trouxeram, \textsc{Cheguei a esta praia em 30
de set. de 1659.} Nas laterais do poste, que era quadrado, imprimia a
cada dia um talho, tendo o do sétimo dia o dobro do tamanho dos demais,
e o do primeiro dia do mês o dobro ainda do tamanho deste último; e
assim conservei meu calendário, ou o cálculo semanal, mensal e anual do
tempo.

Em seguida, devemos observar que, dentre as muitas coisas que trouxe do
navio nas várias expedições que fiz a ele, acima mencionadas, encontrei
várias coisas de menor valor, porém não menos proveitosas para mim, as
quais omiti noutro momento deste relato; como, em particular, penas,
tinta e papel, recolhidos aos pertences do capitão, do contramestre e
dos mestres artilheiro e carpinteiro; três ou quatro agulhas de marear,
ou bússolas, alguns instrumentos de cálculo, relógios de sol, lunetas,
mapas e diários de navegação, os quais reuni num só amontoado, a
despeito de querê"-los ou não; além disso, também encontrei três Bíblias
muito boas, chegadas a mim junto ao meu carregamento da Inglaterra, e
que estavam guardadas entre as minhas coisas; alguns livros portugueses
também; e entre eles dois ou três livros de oração papistas e vários
outros, todos os quais eu cuidadosamente conservei. E não posso
esquecer"-me de que levávamos no navio um cachorro e dois gatos, cuja
história admirável terei a ocasião de contar a seu momento; pois
carreguei comigo os dois gatos; enquanto o cachorro, este saltou do
navio e nadou à praia em minha direção no dia seguinte à minha primeira
expedição, sendo"-me leal servidor por muitos anos. Não havia o que ele
não me pudesse buscar, tampouco companhia que não me pudesse fazer;
faltava"-me apenas que ele conversasse comigo, o que era impossível. Como
dito anteriormente, encontrei penas, tinta e papel e os conservei ao
máximo, e hei de mostrar que, enquanto a tinta durou, mantive notações
muito precisas, o que, uma vez acabada, tornou"-se impossível, pois não
era capaz de produzi"-la por qualquer meio que pudesse conceber.

E isso trouxe"-me ao pensamento as muitas outras coisas que desejava,
apesar de tudo o que havia reunido; e a tinta era uma delas, assim como
uma pá, uma picareta e uma enxada, para cavar ou remover a terra,
agulhas, alfinetes e linha; e quanto a itens de mesa e banho, não tardei
a aprender a passar bem sem eles.

Com a falta de ferramentas sucedeu que todos trabalhos se realizavam
morosamente, tomando"-me quase um ano terminar por inteiro minha modesta
paliçada ou habitação cercada. As estacas, ou esteios, tão pesados
quanto os aguentava carregar, tomaram"-me um longo tempo cortando"-os e
talhando"-os, e ainda mais levando"-os a minha casa; de guisa que, por
vezes, necessitava de dois dias inteiros para cortar e transportar um só
poste, e um terceiro para fincá"-lo no chão; para cujo fim empreguei de
início um pesado tronco, preferindo noutro momento um dos pés de cabra,
o qual, diferentemente do que pensara, tornou a fixação desses postes ou
esteios muito morosa e exaustiva.

Mas que necessidade tinha eu de me preocupar com a morosidade de
qualquer coisa que tivesse de fazer, uma vez que dispunha de todo o
tempo do mundo para sua execução, nem tinha qualquer outro emprego, caso
o dito terminasse ou, pelo menos, que eu pudesse antever, senão o de
perambular pela ilha em busca de comida, o que fazia mais ou menos todo
dia.

Comecei então a refletir seriamente acerca de minha condição e as
circunstâncias a que me reduzia, e elaborei por escrito meu estado de
coisas, não tanto para deixá"-lo aos que viessem depois de mim, pois era
provável que eu tivesse apenas poucos herdeiros, mas para liberar meus
pensamentos de me debruçar sobre tais questões diariamente e, assim,
afligir"-me a mente; e à medida que a razão passou a dominar"-me o
desalento, tive condições de encontrar algum consolo, e a confrontar o
bem com o mal, de modo que pudesse dispor de algo que diferenciasse meu
caso de outros piores; e relacionei com muita imparcialidade, como
devedor e credor, os confortos de que desfrutava ante os infortúnios de
que padecia, assim:

\noindent\resizebox{\textwidth}{!}{
\begin{tabular}{c|c}
\textbf{\textsc{mal}} & \textbf{\textsc{bem}} \\ \hline
\begin{tabular}[c]{@{}c@{}}Estou perdido em uma terrível ilha\\ deserta, sem qualquer esperança\\ de salvamento.\end{tabular} & \begin{tabular}[c]{@{}c@{}}Mas estou vivo, não afogado como\\ toda a companhia do navio.\end{tabular} \\ \hline
\begin{tabular}[c]{@{}c@{}}Encontro-me isolado e desviado,\\ por assim dizer, do mundo inteiro,\\ para minha infelicidade.\end{tabular} & \begin{tabular}[c]{@{}c@{}}Mas também fui desviado de\\ toda a companhia do navio e\\ poupado da morte; e aquele\\ que milagrosamente me\\ salvou da morte pode\\ libertar-me desta condição.\end{tabular} \\ \hline
\begin{tabular}[c]{@{}c@{}}Estou apartado da humanidade,\\ solitário, banido do convívio humano.\end{tabular} & \begin{tabular}[c]{@{}c@{}}Mas não estou morrendo de fome\\ e definhando em lugar infértil sem\\ qualquer provimento.\end{tabular} \\ \hline
Não tenho roupas que me abriguem & \begin{tabular}[c]{@{}c@{}}Mas estou em clima quente, onde, caso\\ tivesse roupas, mal seria capaz de vesti-las.\end{tabular} \\ \hline
\begin{tabular}[c]{@{}c@{}}Não tenho defesas, nem meios de\\ resistência, a qualquer violência de\\ homem ou animália.\end{tabular} & \begin{tabular}[c]{@{}c@{}}Mas estou perdido em uma ilha onde\\ não avisto bestas feras que me firam,\\ como as vi na costa da África. E se\\ meu navio tivesse afundado lá?\end{tabular} \\ \hline
\begin{tabular}[c]{@{}c@{}}Não disponho de ouvidos\\ e braços que me amparem\end{tabular} & \begin{tabular}[c]{@{}c@{}}Mas Deus maravilhosamente permitiu\\ que o navio encalhasse próximo à costa,\\ de modo que posso obter tudo quanto\\ seja necessário ao sustento de minhas\\ necessidades ou que me permita\\ suprir-me pelo tempo que viver.\end{tabular}
\end{tabular}
}

\bigskip



No geral, eis aqui um testemunho cabal de que difícil seria no mundo
encontrar tão desafortunado destino, porém no que tinha de negativo ou
positivo sempre havia algo digno de gratidão; e que isso sirva de
depoimento da experiência da mais infeliz condição neste mundo, a saber,
que sempre podemos encontrar algo que nos conforte e que, na descrição
do bem e do mal, se registre no cômputo como crédito.

Tendo, então, permitido que meus pensamentos encontrassem algo de bom em
minha condição; e desistindo de observar o mar, com a intenção de
avistar algum navio; digo, tendo desistido dessas coisas, passei a me
dedicar ao ajustamento de minha maneira de viver e ao desembaraço de
todo o possível.

Já descrevi minha habitação, que era uma tenda sob o flanco de um
penedo, cercada por robusta paliçada de esteios e amarras, a qual
prefiro doravante chamar de muro, pois junto dela ergui uma guisa de
muro com torrões de turfa, com obra de cinco palmos de espessura, do
lado de fora; e passado algum tempo, penso que um ano e meio, levantei
caibros que inclinados se apoiavam na rocha, cobrindo"-os de sapê ou
galhos de árvore e de tudo de que pudesse ter proveito para prevenir"-me
da chuva, que caía muito violenta em certas épocas do ano.

Já fiz relação de como trouxe todos os meus bens à tenda e à caverna que
na rocha abrira atrás dela; mas devo também dizer que, a princípio, tudo
fazia um amontoado confuso de coisas que, não estando em ordem, ocupavam
todo o lugar, e eu não tinha espaço para me virar; encetei, assim, a
ampliação de minha caverna e dependências rocha adentro; pois que se
tratava de rocha arenosa, que facilmente cedia ao trabalho que nela se
empregava; e quando me vi bastante guarnecido em relação às bestas
predadoras, passei ao aprofundamento do flanco direito da pedra; e, em
seguida, tornando mais uma vez à direita, cavei com afinco e fiz"-me uma
porta para sair de minha paliçada ou fortificação.

Tal trabalho não apenas permitia"-me ir e vir, fazendo as vezes de
passagem de fundos para minha tenda e armazém, como abria espaço para a
arrumação de meus bens.

Feito isso, passei a me aplicar a coisas que se faziam necessárias à
medida que delas sentia grande falta, como, em particular, uma cadeira e
uma mesa, pois sem elas eu não era capaz de desfrutar dos poucos
confortos que tinha no mundo. Sem uma mesa não conseguia escrever ou
comer e tampouco ter prazer em muito que fazia.

Pus"-me a trabalhar em ambas; e neste ponto, creio ser importante
observar que, sendo a razão substância e origem da matemática, desde que
se determine e analise tudo por meio da razão e se produza o juízo mais
racional das coisas, todo homem é capaz de assenhorear"-se, cedo ou
tarde, de qualquer arte mecânica. Eu nunca havia manejado uma ferramenta
em minha vida; e, no entanto, com o tempo, com diligência, esforço e
engenho, constatei, por fim, que nada me faltava que eu não pudesse
fazer, desde que tivesse ferramentas. Mesmo assim, fiz uma abundância de
coisas sem ferramentas, e algumas delas com não mais do que uma enxó e
uma machadinha, coisas que talvez nunca dantes tenham sido feitas dessa
maneira, e com grande dificuldade. Se, por exemplo, minha vontade era
ter uma tábua, não havia outro jeito senão cortar uma árvore, colocá"-la
sobre uma das pontas diante de mim e talhá"-la de um lado e de outro, até
que ficasse fina com uma tábua, para depois alisá"-la com a enxó. É certo
que, com esse método, eu conseguia produzir de uma árvore inteira uma
única tábua; mas quanto a isso, o único remédio era a resignação, que
igualmente se aplicava à prodigiosa quantidade de tempo e trabalho que a
fabricação de uma prancha ou tábua me tomava; mas meu tempo ou trabalho
valiam pouco, e assim ficavam bem empregados, de uma forma ou de outra.

Enfim, fabriquei uma mesa e uma cadeira, como observei acima, em
primeiro lugar; e fi"-las a partir dos pedaços de tábua que havia trazido
do navio com a jangada. Mas, quando trabalhei na produção de tábuas, tal
como acima, estas resultaram"-me compridas prateleiras, de dez palmos em
largo, instaladas umas sobre as outras ao longo de uma das paredes de
minha caverna, para sobre elas dispor todas as minhas ferramentas,
pregos e ferragens; e, em suma, dar à maioria das coisas seus devidos
lugares, para que assim pudesse delas facilmente dispor; e cravei ferros
na rocha para pendurar minhas armas e todas as coisas que pudessem ser
assim guardadas.

Desse modo, estando minha caverna diante dos olhos, sua aparência era a
de um armazém geral de tudo quanto fosse necessário; e tinha eu tudo tão
à mão, que havia em grande prazer ver toda a minha fazenda disposta em
tal ordem e, sobretudo, ver quão grande era meu armazém de coisas
fundamentais.

E foi então que comecei a manter um diário de meus empreendimentos;
pois, de fato, a princípio era muito o meu desassossego, e não apenas
quanto aos trabalhos, mas também pela excessiva perturbação a que
estavam entregues meus pensamentos; e meu diário resultaria repleto de
muitas coisas melancólicas; por exemplo, teria nele dito o seguinte:
``\emph{30 de set.}: Depois de chegado à praia e sido poupado do
afogamento, em vez de agradecer a Deus por meu salvamento, tendo antes
vomitado grande quantidade de água do mar que engolira e me recuperado
um pouco, corri pela praia de punhos em riste e batendo em minha cabeça
e rosto, lamentando aos brados meu infortúnio e gritando, `É meu fim! É
meu fim!', até a exaustão, quando fui forçado a deitar"-me ao chão e
repousar, mas sem ousar dormir por medo de ser devorado.''

Alguns dias depois, e depois de ter subido a bordo do navio e de lá ter
tirado tudo quanto pude carregar, tive em grande ânsia subir ao alto de
um monte e observar o mar na esperança de avistar um navio, e então à
enorme distância mostrou"-se o que me pareceu uma vela, o que me animou
muitíssimo, mas depois de mirá"-la fixamente, a ponto de quase cegar,
perdi"-a, ao que me sentei e chorei como criança, fazendo assim meu
destempero aumentar meu infortúnio.

Tendo, contudo, superado essas coisas em alguma medida e organizado os
itens domésticos e a moradia, fabricado uma mesa e uma cadeira, e
organizado o mais da forma que me aprazia, comecei a escrever um diário,
do qual lhes ofereço cópia (embora todos esses particulares sejam mais
uma vez relacionados) do que pude registrar, pois, em não havendo mais
tinta, fui obrigado a abandoná"-lo.

\chapter{O diário}

\emph{30 de setembro de 1659.} Eu, um pobre e desafortunado homem de
nome Robinson Crusoe, tendo naufragado durante terrível tempestade em
alto"-mar, fui ter à costa desta desditosa e horrenda ilha, a qual
batizei ``A Ilha do Desespero'', com a companhia do navio inteira
perdida, exceto por mim, que quase morri.

O restante daquele dia, passei atormentado das lúgubres circunstâncias
em que me encontrava, a saber, que não tinha comida, casa, roupas,
armas, nem lugar que me agasalhasse; e, desesperado de qualquer remédio,
não vi mais que morte em meu redor, em vias que estava tanto de ser
devorado por bestas vorazes, como de ser assassinado por bárbaros ou
morrer de fome à falta de comida. No cair da noite, dormi em uma árvore,
temendo as criaturas selvagens; mas ferrei no sono, apesar de chover sem
cessar.

\emph{1º de outubro.} Vi pela manhã, para minha grande surpresa, que o
navio se soltara com a cheia da maré e flutuara na direção da costa para
mais acerca da ilha, o que me trouxe certo alívio, por um lado, pois ao
vê"-lo inteiro e não reduzido a destroços, esperava eu subir a bordo, se
o vento amainasse, e obter comida e artigos de primeira necessidade,
mas, por outro lado, renovou minha tristeza pela perda de meus
companheiros; imaginando que, tivéssemos ficado todos a bordo,
poderíamos ter salvo o navio ou, pelo menos, que não teriam todos se
afogado, como se deu; e que, tivessem os marinheiros se salvado, talvez
tivéssemos podido construir, a partir dos destroços do navio, barco que
nos levasse a alguma outra parte do mundo. Passei grande parte desse dia
enchendo"-me de dúvidas quanto a essas coisas; mas por fim, vendo o navio
praticamente fora d'água, caminhei pela areia vizinhando"-me tanto quanto
possível e depois nadei a bordo. Nesse dia também chovia, embora sem
vento.

\emph{De 1º de outubro a 24 de outubro}. Todos esses dias decorreram
entre várias excursões ao navio, para que dele carregasse tudo o que
pudesse extrair; o que levava à terra a bordo de jangadas sempre na
cheia da maré. Dias de muita chuva também, embora com alguns intervalos
de tempo bom. Segundo podia depreender, era estação das chuvas.

\emph{20 de outubro.} Minha jangada virou, e se foi toda a fazenda que
nela carregava; mas, estando em águas rasas, e sendo minha carga
sobretudo pesada, recuperei muitas das coisas perdidas quando a maré
baixou.

\emph{25 de outubro.} Choveu a noite toda e o dia inteiro, com refegas
de vento; nesse espaço o navio partiu"-se em pedaços, o vento se fez mais
fero, e dele o que se viu novamente foram destroços, e apenas na maré
baixa. Passei o dia cobrindo e protegendo os aviamentos que salvara para
que a chuva não os estragasse.

\emph{26 de outubro.} Caminhei pela praia quase todo o dia, em demanda
de sítio em que pudesse fixar minha habitação, estando muito preocupado
em me proteger de qualquer ataque noturno, fosse de animais selvagens,
fosse de homens; ao que acerca do anoitecer encontrei lugar apropriado,
debaixo de um penedo, e demarquei um semicírculo para meu acampamento,
que resolvi fortalecer com uma obra, muro ou fortificação, feita de
estacas duplas, forradas com cabos e coberto por fora de torrões de
turfa.

Entre os dias 26 e 30, trabalhei arduamente no transporte das minhas
provisões para minha nova habitação, embora em parte chovesse muito
forte.

No dia 31, pela manhã, saí com minha espingarda pela ilha à caça de
comida e com tenção de descobrir a região, quando matei uma cabra, tendo
sua cria me seguido até minha casa, e a qual depois tive de matar, pois
não se alimentava.

\emph{1º de novembro.} Montei minha tenda sob um penedo e ali passei
minha primeira noite, fazendo"-a tão ampla quanto possível, com esteios
cravados no chão para balançar minha rede.

\emph{2 de novembro.} Arrumei todos os meus baús e tábuas e os pedaços
de verga que com que fiz minhas jangadas, e com eles ergui uma cerca em
meu entorno, um pouco para dentro do local que demarcara para minha
fortificação.

\emph{3 de novembro.} Saí com minha espingarda e matei duas aves, que se
assemelhavam a patos e forneceram muito boa carne. À tarde trabalhei
para fabricar"-me uma mesa.

\emph{4 de novembro.} Hoje de manhã procedi com a organização de meus
horários de trabalho, de sair com minha espingarda, e meu tempo de
satisfazer o sono e o lazer; a saber, toda manhã saía com minha
espingarda por duas ou três horas, desde que não estivesse chovendo;
depois, tomava meu tempo com trabalho até onze horas, comendo em seguida
o que houvesse para meu sustento; e do meio dia às duas eu me deitava
para dormir, sendo o clima excessivamente quente; e depois, no
entardecer, voltava ao trabalho. A parte deste dia e do seguinte
dedicadas ao trabalho foram totalmente empregadas no fabrico de minha
mesa, pois eu ainda era um trabalhador muito sofrível, embora o tempo e
a necessidade tenham por fim feito de mim um artesão completamente à
vontade com o seu ofício, assim como creio que qualquer um poderia
sê"-lo.

\emph{5 de novembro.} No dia de hoje saí com minha espingarda e meu
cachorro e matei um gato selvagem, de pele muito macia, mas carne
inútil. Esfolava e preservava a pele de toda criatura que matava.
Voltando à beira"-mar, vi aves marinhas de muitas castas que desconhecia;
e quedei espantado e quase assustado com duas ou três focas que, no
instante em que as olhei, sem saber o que eram, entraram no mar e me
escaparam.

\emph{6 de novembro.} Depois do meu passeio matinal, prossegui no
trabalho com a minha mesa e o concluí, embora não tenha ficado do meu
agrado; não demorou muito para que aprendesse a fazer"-lhe reparos.

\emph{7 de novembro.} O bom tempo começou a firmar"-se. Os dias 7, 8, 9,
10 e parte do dia 12 (porque o dia 11 era domingo), eu empreguei
inteiramente no fabrico de uma cadeira, e com muito esforço a concluí de
guisa a imprimir"-lhe forma aceitável, mas que muito me desagradou; mesmo
enquanto nela trabalhava, eu a desfiz em pedaços muitas vezes.
\emph{Nota}, logo deixei de guardar os domingos, pois, em omitindo sua
marcação no poste, perdi"-lhes a conta.

\emph{13 de novembro.} Hoje choveu, trazendo à terra e a mim bastante
refrigério; mas a chuva veio acompanhada por trovões e relâmpagos
terríveis, que por causa de minha pólvora transiram"-me de medo; e tão
logo ela cessou, decidi separar meu provimento de pólvora no maior
número possível de porções, para que não corresse perigo.

\emph{14, 15, 16 de novembro.} Tomei esses três dias fabricando
bauzinhos, ou caixas, capazes de conter pólvora em obra de uma libra, ou
duas no máximo; e lhes tendo depositado a pólvora, guardei"-os em lugares
os mais seguros, tão remotos uns dos outros quanto possível. Num desses
três dias, matei um pássaro de bom porte e carne saborosa, mas cujo nome
não conhecia.

\emph{17 de novembro.} Nesse dia, comecei a cavar o penedo atrás de
minha tenda, para abrir espaço a minha maior conveniência. \emph{Nota},
havia três coisas que eu muito queria para essa obra, a saber, uma
picareta, uma pá e um carrinho de mão ou cesto; por isso, suspendi a
tarefa e me pus a pensar em como suprir essa necessidade e fabricar
ferramentas. Os pés de cabra fizeram as vezes de picareta; que, embora
pesados, eram adequados o suficiente; mas havia a pá, sem a qual, de tão
absolutamente necessária que era, eu era capaz de realizar nada; embora
não soubesse que maneira de pá devia fazer.

\emph{18 de novembro.} No dia seguinte, tendo saído em demanda na mata,
dei com uma árvore, ou a ela semelhável, da madeira a que nos Brasis
chamam pau"-ferro por sua excessiva dureza, da qual, com muito trabalho e
quase embotando meu machado, cortei um pedaço; levando"-o também para
casa, mas com não pouca dificuldade, pois era muito pesado.

A dureza excessiva da madeira, e o fato de não ter outra maneira,
exigiu"-me bastante tempo na manufatura da ferramenta, pois foi pouco a
pouco que terminei por talhá"-la nas formas de uma pá, com o cabo em
formato exatamente igual ao nosso na Inglaterra; porém, sem que a parte
larga tivesse ferro que lhe calçasse, ela não duraria muito; tendo, não
obstante, servido bem aos usos em que tive a ocasião de empregá"-la; mas
creio que nunca uma pá tenha sido feita a essa maneira ou tenha exigido
tanto tempo em sua fabricação.

Havia ainda outras necessidades a suprir, pois queria um cesto ou um
carrinho de mão. Um cesto, eu não era capaz de fabricar de jeito nenhum,
sem dispor de coisas como galhos de salgueiro, que se dobrassem para
fazer artigos de vime, ou pelo menos sem os ter descoberto; e quanto a
um carrinho de mão, achei"-me capaz de produzi"-lo inteiro, exceto pela
roda, que não sabia como fazer, nem como vir a sabê"-lo; ademais, não
havia jeito de fazer os encaixes para o fuso ou eixo da roda, de guisa
que desisti; tendo feito uma maneira de balde para deitar a terra fora
da caverna, como o que usam os trabalhadores que carregam argamassa a
serviço dos pedreiros.

Fazê"-lo não foi tão dificultoso quanto fazer a pá; mas o balde, a pá e a
vã tentativa de construir um carrinho de mão tomaram"-me não menos do que
quatro dias; quero dizer contínuos, senão por meu passeio matinal com
minha espingarda, que raramente deixava de fazer, assim como muito
raramente dele não resultava algo digno de comer.

\emph{23 de novembro.} Com minha outra tarefa então suspensa, à espera
de que fabricasse as ditas ferramentas; tendo"-as terminado, dei"-lhe
continuidade a ela, e trabalhando todos os dias, conforme minha força e
tempo permitiam, levei dezoito dias inteiros ampliando e aprofundando
minha caverna, para que pudesse guardar meus bens comodamente.

\emph{Nota}, durante todo esse tempo, trabalhei para tornar esse cômodo
ou caverna espaçoso o bastante para servir"-me de armazém, cozinha, sala
de jantar e despensa; quanto a meu apartamento, este permaneceu na
tenda; senão que, por vezes, na estação chuvosa do ano, tanto chovia que
não conseguia conservar"-me seco; o que noutro tempo fez com que cobrisse
todo o meu espaço interno à paliçada com compridos postes em maneira de
vigas, os quais apoiei na rocha, e o carregasse com as folhas largas das
árvores de lá, como se faz com a palha nas casas inglesas.

\emph{10 de dezembro.} Passava a crer que minha caverna ou câmara havia
terminado, quando de repente (parece que o espaço resultara demasiado
amplo) desceu do teto e de uma das paredes grande quantidade de terra; e
tanta que, em suma, fiquei assustado, e não sem razão, pois estivesse eu
debaixo dela, jamais viria a necessitar de um coveiro. Esse desastre
resultou em muito trabalho a fazer, pois era preciso carregar para fora
toda a terra que caíra; e, o que era de maior importância, tinha o teto
a arrimar, para ter certeza de que não mais cairia.

\emph{11 de dezembro.} Encetei o trabalho em tais tarefas e fixei dois
pontaletes ou pilares do chão ao teto, com dois pedaços de tábua
atravessados sobre cada um, o que terminei no dia seguinte; e erguendo
mais pilares com tábuas, em obra de uma semana tinha o teto seguro e os
pilares dispostos um ao lado dos outros fizeram vezes de divisórias para
uma parte da casa.

\emph{17 de dezembro.} Deste dia ao dia 20, coloquei prateleiras e
cravei ferros nos pilares para pendurar tudo o que neles se pudesse
assim dispor; e então comecei a pôr em boa ordem o lado de dentro.

\emph{20 de dezembro.} Carreguei então tudo para dentro da caverna, e
comecei a mobiliar minha casa, e com alguns pedaços de tábuas montei uma
maneira de armário para a preservação de meus mantimentos; mas as tábuas
começaram a me faltar muitíssimo; também fiz outra mesa.

\emph{24 de dezembro.} Muita chuva a noite toda e o dia todo. Não saí.

\emph{25 de dezembro.} Chuva o dia inteiro.

\emph{26 de dezembro.} Sem chuva, e a terra muito mais fresca e
aprazível do que antes.

\emph{27 de dezembro.} Matei um cabrito jovem e feri outro para que o
pudesse prender, e o levei para casa amarrado a uma corda; quando
cheguei, amarrei sua pata com uma tala, pois estava quebrada.
\emph{N.B.}: Cuidei tão bem dele que ele sobreviveu, e a pata melhorou e
ficou tão forte quanto antes; mas por alimentá"-lo tanto tempo, ele
amansou e passou a comer da relva da minha porta e não ali passou a
viver. Assim me ocorreu pela primeira vez a ideia de criar animais
mansos, para que deles pudesse ter comida quando minha pólvora e minhas
balas acabassem.

\emph{28, 29 e 30 de dezembro.} Muito calor e brisa nenhuma, de modo que
não houve excursão pela ilha, exceto no fim da tarde, à caça de comida.
Nesse tempo, cuidei em colocar todas as minhas coisas em ordem.

\emph{1º de janeiro.} Ainda muito quente, mas saí pela manhã e no fim da
tarde com a minha espingarda e repousei pelo meio do dia. Naquele
entardecer, alongando"-me ainda mais nos vales que vizinhavam o centro da
ilha, descobri que havia muitas cabras, embora demasiado arredias, sendo
muito difícil aproximar"-me delas; assim, decidi tentar trazer meu
cachorro para caçá"-las.

\emph{2 de janeiro.} Assim, saí com meu cachorro no dia seguinte e
coloquei"-o à caça das cabras, mas não logrei no que intentava, pois
todas confrontaram o cachorro, e ele percebeu muito bem o perigo, pois
não se aproximou delas.

\emph{3 de janeiro.} Iniciei os trabalhos em minha cerca ou muro; a
qual, ainda temendo ser atacado por alguém, decidi fazer muito grossa e
forte.

\emph{N.B.} Tendo sido o muro descrito antes, omito propositalmente o
que se escreveu no diário; basta observar que não levei menos tempo do
que de entre 2 de janeiro a 14 de abril; tempo que dispendi trabalhando
e terminando a obra, assim como aperfeiçoando"-a, embora não tivesse mais
do que vinte e quatro jardas de extensão, formando um semicírculo de um
ponto a outro da rocha, da qual distava oito jardas, a contar do centro
da caverna.

Em todo esse tempo trabalhei muito, com as chuvas obstando"-me muitos
dias e até semanas inteiras; mas julgava que nunca estaria perfeitamente
seguro até que o muro estivesse completo; e mal se pode crer nos
indescritíveis esforços consumidos por tudo isso, especialmente o
transporte dos postes pela mata e o processo de enfiá"-los no terreno,
pois eu os fiz muito maiores do que o necessário.

Concluído o muro, ao qual acresci a proteção de uma parede de torrões de
turfa erguido imediatamente a sua frente, convenci"-me que, em caso que
alguém à praia arribasse, de lá nada acharia que se assemelhasse a uma
habitação; e foi muito bom que o tenha feito dessa forma, como se poderá
observar a seguir, em ocasião muito notável.

Durante esse tempo, fiz minhas excursões de caça à mata todos os dias
que a chuva o permitiu; e nessas caminhadas não raro fiz descobertas de
uma ou outra coisa que me beneficiavam; em especial, encontrei uma casta
de pombo selvagem que constrói seus ninhos não nas árvores, à guisa do
torcaz, mas como o doméstico, nos buracos das rochas; e, levando comigo
alguns ainda jovens, tomei por empresa amansá"-los; quando cresceram,
porém, voaram para longe, o que talvez se tenha dado sobretudo por falta
de alimento, pois nada lhes tinha a oferecer; no entanto, com frequência
eu encontrava seus ninhos e recolhia os mais jovens, que forneciam muito
boa carne.

E então, em dando boa ordem a meus assuntos domésticos, vi"-me em
necessidade de muitas coisas cuja fabricação a princípio achei ser
impossível, como, de fato, em alguns casos era. Por exemplo, eu nunca
seria capaz de fabricar um barril com cinta. Eu dispunha de alguns
poucos, como já se disse, mas nunca alcancei a destreza de fabricar um,
embora tenha gasto muitas semanas em torno disso; eu não conseguia
colocar as tampas, nem unir as aduelas com precisão entre si de maneira
que a água permanecesse retida, e assim desisti dessa tarefa.

Da mesma maneira, tinha as velas em grande falta; por isso, tão logo
caía a noite, o que ocorria por volta de sete horas, eu era obrigado a
ir para a cama. Lembrei"-me, no entanto, do pedaço de cera de abelha com
que fiz velas em minha aventura africana; mas não tinha meu dispor o que
se assemelhasse; o remédio que encontrei foi separar o sebo das cabras
que matava e, com um pratinho de barro que assei ao sol, ao qual
acrescentei um pavio de cânhamo, fazer um candeeiro para mim; e isso me
deu luz, embora não clara e constante como a de uma vela; no meio de
todos os meus trabalhos, ocorreu que, escarafunchando em minhas coisas,
deparei"-me com um saco não grande que, como suspeitara anteriormente,
havia estado cheio de grãos para a alimentação de aves, não desta
viagem, mas de expedição anterior, como suponho, quando o navio veio de
Lisboa, os quais haviam sido inteiramente devorados pelos ratos, e eu
nada encontrei no saco além de cascas e pó; mas disposto a conservá"-lo
para algum outro uso, penso que para o abrigo da pólvora, quando a
dividi por medo dos raios, ou utilidade do gênero, eu sacudi as cascas
para debaixo da pedra de um dos lados de minha fortificação.

Foi um pouco antes das grandes chuvas, há pouco mencionadas, que joguei
essas cascas e poeira fora, sem lhes dar atenção, nem me lembrando de
que havia abandonado alguma coisa ali, quando, obra de um mês depois ou
acerca disso, vi uns talos verdes brotando do chão, que eu imaginei
serem de alguma planta que não tivesse visto; mas qual não foi meu
espanto e absoluta admiração quando, passado um pouco mais de tempo, vi
obra de uma dúzia de espigas despontarem, da mais perfeita cevada verde
de procedência europeia, ou melhor, inglesa.

É impossível expressar o espanto e a confusão dos meus pensamentos
naquele momento; até então, minhas ações não haviam tido qualquer
fundamento religioso; é justo dizer, sim, que meus pensamentos
carregavam poucas noções religiosas, nem conservavam qualquer sentido do
que me acometera que não fosse o acaso, ou, como levianamente dizemos, a
vontade de Deus; sem nem mesmo investigar os propósitos da Providência
nessas coisas, ou seu mandamento no governo dos acontecimentos no mundo;
mas depois de ver a cevada crescer ali, em clima que eu sabia não ser
adequado aos grãos e, especialmente, sem saber como ele dado ali,
assaltou"-me estranha admiração, e comecei a imaginar que Deus havia
feito aquele grão crescer ali por obra de milagre, sem o auxílio de
qualquer semente semeada, e que sua razão de ser era puramente o meu
sustento naquele lugar selvagem e desolado.

Senti a comoção daquilo em meu coração e lágrimas brotaram em meus
olhos, e comecei a persignar"-me, pois tal prodígio da natureza
acontecera por minha causa; e não me causou menos estranheza que, perto
dele, por toda a extensão da lateral da pedra, surgiam outros talos
dispersos, que se provaram talos de arroz, e que eu conhecia, pois os
tinha visto cultivados em África, quando eu lá havia estado em terra
firme.

Eu não apenas pensei que se tratavam de puros produtos da Providência ao
meu sustento, como, não duvidando de que houvesse mais naquelas terras,
saí por toda a ilha onde antes eu estivera e procurei em todos os cantos
e debaixo de cada rocha em busca de mais, porém sem bom sucesso; por
fim, veio"-me aos pensamentos que havia despejado um saco de comida de
galinha naquele lugar; e então o maravilhamento começou a se desfazer;
assim como devo confessar que minha gratidão religiosa à Providência de
Deus igualmente arrefeceu, ao descobrir que tudo aquilo não era mais que
o ordinário; embora eu tivesse de ser tão grato por tal Providência, tão
estranha e inesperada, quanto se tivesse sido um milagre: pois de fato
se tratava de obra da Providência em meu favor que ela tivesse
determinado ou ordenado que uma dúzia daqueles grãos tivessem
permanecido intocados, (quando os ratos destruíram todo o resto), como
se tivessem caído do céu; do mesmo modo que também advinha de sua
agência que eu os jogasse naquele lugar específico, onde, estando à
sombra de elevada rocha, eles brotaram prontamente; enquanto que, se os
tivesse jogado em outro lugar, naquela hora, eles teriam sido queimados
e destruídos.

Guardei com cuidado, disso não se tenha dúvida, as espigas desses grãos
quando se apresentaram à colheita, que ocorreu por volta de fins de
junho; e, debulhando cada espiga, decidi semeá"-los todos novamente, na
esperança de que logo tivesse quantidade bastante que me suprisse de
pão; mas não foi senão no quarto ano que eu pude me permitir comer o
menor grão desses cereais e, mesmo assim, com moderação, como direi
posteriormente, a seu momento; pois perdi tudo o que semeei na primeira
colheita por não observar o tempo correto; porque eu os semeei pouco
antes do tempo de seca, de guisa que eles nunca brotaram, ao menos não
como deveriam: do que tratarei a seu tempo.

Além dessa cevada, havia, como acima, vinte ou trinta talos de arroz,
que eu preservava com o mesmo cuidado e cujo uso era semelhante ou ao
mesmo objetivo, a saber, fazer pão, ou ainda comida; pois eu descobri
maneiras de cozinhá"-lo sem assar, embora tenha feito isso depois de
algum tempo. Mas retorno ao meu diário.

Dei"-me a duros e excessivos trabalhos nesses três ou quatro meses para
terminar meu muro; e no dia 14 de abril eu o fechei, projetando a
entrada não através de porta, mas por travessia do muro com o auxílio de
uma escada, para que não houvesse qualquer evidência de minha habitação
se vista do lado de fora.

\emph{16 de abril.} Terminei a escada; então subi com a escada ao alto
do muro, ergui"-a detrás de mim e a joguei para dentro. Isso representou
um encerramento total para mim; pois do lado de dentro eu tinha espaço
bastante, e nada podia chegar a mim vindo de fora, a menos que primeiro
pudesse galgar minha parede.

No dia exatamente seguinte ao término do muro, vi o meu trabalho quase
ser totalmente destruído de uma só vez, e eu próprio morto. Eis o caso:
enquanto eu me ocupava do lado de dentro, atrás da minha tenda, logo à
entrada de minha caverna, causou"-me horrível susto uma coisa de fato das
mais terríveis; pois de repente uma porção de terra soltou"-se do teto da
minha caverna e da beira da pedra acima da minha cabeça; e dois dos
pilares com que havia escorado a caverna racharam medonhamente, de guisa
que fiquei muito amedrontado, mas nada me ocorria que fosse de fato a
causa, apenas pensando que o teto da minha caverna havia cedido, como
ocorrera anteriormente; e temendo quedar enterrado nela, acorri a minha
escada e, sem pensar que ali estivesse a salvo tampouco, galguei meu
muro temendo os pedaços da rocha que esperava que rolassem sobre mim.
Num instante, tinha os pés no chão firme, mas vi tratar"-se claramente de
um terrível terremoto, tendo o sítio em que estava tremido três vezes a
obra de intervalos de oito minutos, com três tremores tais que seriam
capazes de derrubar o mais sólido edifício que supostamente existisse na
terra; e um grande pedaço do cimo de uma rocha que ficava a uma meia
milha de mim, próxima ao mar, rolou com um estrondo como nunca ouvira em
toda a minha vida. Percebi também que o próprio mar fora posto em
violenta agitação; e acredito que os tremores foram mais fortes debaixo
d'água do que na ilha.

Fiquei tão admirado da coisa em si, nunca tendo experimentado algo
semelhante, ou conversado com quem o tivesse vivido, que permaneci morto
ou estupefato; e o movimento da terra fez meu estômago enjoar como o de
quem fosse lançado ao mar; mas o barulho da queda da rocha me despertou,
por assim dizer, e tirando"-me do estupor em que me encontrava, encheu"-me
de horror, e eu só pensava no penedo inteiro desabando sobre minha tenda
e todos os meus itens domésticos e enterrando tudo de uma vez; e isso
fez minha alma desfalecer dentro de mim uma segunda vez.

Depois de terminado o terceiro tremor, tendo permanecido insensível por
algum tempo, comecei a recobrar o espírito; e, no entanto, não tinha
coragem o suficiente para atravessar meu muro novamente, temendo ser
enterrado vivo, e ficando, então, sentado no chão, muito abatido e
desconsolado, sem saber o que fazer. Tudo isso sem que eu tivesse o
menor pensamento religioso; nada além do comum ``Senhor, tende piedade
de mim!''; e quando tudo se acabou, este também se foi.

Enquanto estive sentado, vi o céu ficar escuro e nublado, como se fosse
chover. Logo depois, o vento começou a enfurecer"-se, de maneira que em
menos de meia hora se formou um furacão terrível; de repente, o mar
cobriu"-se de espuma branca; as águas arrojaram"-se sobre a praia, as
árvores foram arrancadas pelas raízes, e era uma tempestade terrível; e
isso durou obra de três horas, começando a arrefecer depois; e em mais
duas horas, tudo estava bastante calmo e caiu uma chuva muito forte.

Tudo isso enquanto eu estava sentado no chão, aterrorizado e abatido;
quando de repente me ocorreu que esses ventos e chuvas eram as
consequências do terremoto, que já havia então acabado, e eu poderia me
aventurar na minha caverna novamente. Com esse pensamento, senti meu
espírito reviver; e com a chuva também auxiliando meu convencimento,
entrei e permaneci sob minha tenda, mas a chuva era tão violenta que
minha tenda estava a ponto de ceder; e fui forçado a entrar na minha
caverna, embora com muito medo e desconforto, por medo de vê"-la cair
sobre minha cabeça.

Essa chuva violenta forçou"-me a um novo trabalho, a saber, abrir um
buraco em minha nova fortificação, à maneira de um ralo, para deixar a
água sair, a qual de outro modo teria inundado minha caverna. Depois de
estar por algum tempo em minha caverna e não sentir mais tremores de
terra, comecei a ficar mais calmo; e para auxiliar em meu ânimo, que de
fato bastante me faltava, fui ao meu pequeno armazém e tomei um trago de
rum; o que fiz com moderação naquele momento e sempre, sabendo que mais
dele não teria quando acabasse.

Continuou chovendo a noite toda e grande parte do dia seguinte, de modo
que não pude seguir em excursão na ilha; mas, com minha mente mais
calma, comecei a pensar no que era melhor fazer, concluindo que, se a
ilha estava sujeita a tais terremotos, não havia condições de viver em
uma caverna, o que me levou a ponderar sobre a construção de uma
cabaninha em local aberto, que pudesse cercar de um muro, como fizera
aqui, protegendo"-me de animais selvagens ou homens; pois concluí que, se
ficasse onde estava, certamente acabaria cedo ou tarde enterrado vivo.

Com esses pensamentos, decidi remover minha tenda do local em que
estava, o qual ficava logo abaixo do precipício da colina; e que, em
caso de novo tremor, certamente cairia sobre minha tenda; e passei os
dois dias seguintes, sendo estes 19 e 20 de abril, planejando onde e
como remover minha habitação.

O medo de ser engolido vivo pela rocha era tal que não conseguia ter um
sono tranquilo, mas a apreensão de permanecer do lado de fora, sem a
proteção de um muro, em nada lhe era dessemelhante; ademais, olhando em
volta e vendo como tudo estava arrumado, e quão agradável era meu
refúgio, e quão seguro do perigo era, tinha muita hesitação em partir.

Entrementes, ocorreu"-me que fazer isso exigiria muito tempo, e que eu
devia me contentar em iniciar o trabalho onde eu estava, até estabelecer
um acampamento para mim e torná"-lo seguro de forma a mudar"-me para lá.
Então, com esta resolução, acalmei"-me por algum tempo e decidi trabalhar
com a maior ligeireza na construção de um muro em círculo, como antes,
com estacas e cabos etc., e montar minha tenda quando o tivesse
concluído; e arriscar"-me permanecer onde estava até terminá"-lo e
deixá"-lo pronto para minha mudança. Este foi o dia 21.

\emph{22 de abril.} Na manhã seguinte, comecei a pensar nos meios de
colocar essa decisão em prática, mas tinha grande falta de minhas
ferramentas. Eu dispunha de três machados grandes e machadinhas em bom
número (pois as levávamos para a troca com os aborígenes); mas com os
muitos trabalhos de cortar e rachar madeira dura e nodosa, todos estavam
embotados e cheios de mossa; e embora houvesse uma pedra de amolar, não
tinha condições de girá"-la e lhes afiar as lâminas. Isso me custou tanta
reflexão quanto a um estadista que tratasse de importante questão
política, ou a um juiz que estivesse diante de decidir pela vida ou a
morte de um homem. Por fim, construí uma roda movida a corda, com o
intuito de girá"-la com o pé e manter as duas mãos livres. \emph{Nota},
Eu nunca tinha visto nada como isso na Inglaterra, ou ao menos não dei
atenção a como era feito, embora, como desde então observei, seja muito
comum lá; além disso, minha pedra de amolar era muito grande e pesada.
Essa máquina me custou uma semana inteira de trabalho para trazê"-la à
perfeição.

\emph{28 e 29 de abril.} Usei esses dois dias inteiros para afiar minhas
ferramentas, revelando"-se muito boa a máquina com que girei minha pedra
de amolar.

\emph{30 de abril.} Tendo percebido que minhas provisões de pão estavam
baixo já havia algum tempo, fiz então um levantamento do que tinha e
reduzi minha ração a um pedaço de bolacha por dia, o que me desanimou
muito.

\emph{1º de maio.} Pela manhã, olhando ao mar, com a maré baixa, avistei
algo na praia maior do que o normal, e parecia um barril; quando me
aproximei, encontrei um barrilete e dois ou três pedaços dos destroços
do navio, os quais haviam sido levados à costa pelo furacão; e olhando
ao próprio navio naufragado, ele pareceu"-me estar mais elevado na
superfície da água do que de costume. Examinei o barrilete chegado à
praia e logo descobri que se tratava de pólvora; mas esta se havia
molhado e estava empelotada e dura como pedra; não obstante, eu o rolei
para mais adiante na praia e caminhei pelas areias para chegar o mais
próximo possível dos destroços do navio e procurar mais.

Quando me aproximei do navio, achei"-lhe a posição estranhamente
modificada. O castelo de proa, que antes estava enterrado na areia,
erguera"-se ao menos seis pés do chão, e a popa, que se encontrava
destroçada e fora separada do resto do navio pela força do mar logo
depois de eu tê"-la deixado, quando a vasculhava, havia sido como que
levantado e deitado; e a areia fora lançada tão alto a ré daquela banda
que, enquanto antes havia um grande braço de mar diante dela, em razão
do qual eu não conseguia me aproximar um quarto de milha dos destroços
sem nadar, agora eu podia caminhar até ela quando a maré baixava; a
princípio fiquei surpreso, mas logo concluí que aquilo havia de ser obra
do terremoto; e da mesma maneira que sua violência fizera o navio
rasgar"-se ainda mais, muitas coisas chegavam diariamente à costa
desprendidas pelo mar e roladas gradualmente à terra pelos ventos e pela
água.

Isso desviou por completo meus pensamentos do projeto de mudar"-me do
lugar que habitava; e passei a ocupar"-me, especialmente naquele dia, de
encontrar algum ponto por onde pudesse entrar no navio; mas descobri que
nada que tal se podia esperar, pois todo o seu interior estava repleto
de areia. No entanto, como eu aprendi a não me desesperar em relação ao
que quer que fosse, decidi separar todos os pedaços do navio que
pudesse, concluindo que tudo o que conseguisse ali seria de algum
proveito para mim.

\emph{3 de maio.} Iniciei a tarefa com a minha serra e cortei um pedaço
de viga, que, segundo pensei, mantinha unido algum ponto do convés
superior ou tombadilho, e quando eu o serrei, limpei tudo que pude da
areia que se alojara na parte que se mantinha mais elevada; mas com a
maré prestes a encher, fui obrigado a suspender o trabalho por ora.

\emph{4 de maio.} Saí à pesca, mas não pesquei nenhum peixe que ousasse
comer, até que me cansei da empreitada, mas quando estava a ponto de ir
embora, cacei um filhote de golfinho. Eu havia preparado uma longa linha
a partir do cordame das vergas, mas não tinha anzol; mesmo assim, eu
frequentemente pescava peixe bastante, tanto quanto cuidasse comer; e eu
os secava ao sol e os comia secos.

\emph{5 de maio.} Trabalhei nos destroços; cortei outra viga em pedaços
e trouxe três grandes tábuas de abeto dos conveses, as quais amarrei
juntas e fiz flutuar até a praia quando veio a maré.

\emph{6 de maio.} Trabalhei nos destroços; tirei vários parafusos e
outros pedaços de ferraria; empenhei"-me muitíssimo, regressei muito
cansado e tive pensamentos de desistir.

\emph{7 de maio.} Fui novamente aos destroços, não com a tenção de
trabalhar; constatei, porém, que os destroços haviam cedido ao próprio
peso, com as vigas se rompendo; e que vários pedaços do navio pareciam
soltos, e que o interior do porão ia tão aberto que eu podia ver dentro
dele, mas quase inteiramente cheio de água e areia.

\emph{8 de maio.} Fui aos destroços e levava um pé de cabra para
arrancar pedaços do convés, que então já estava bem livre de areia e
água. Arranquei duas tábuas e as também levei para a praia com a cheia
da maré. Deixei o pé de cabra nos destroços para o dia seguinte.

\emph{9 de maio.} Fui aos destroços, e com o pé de cabra abri caminho
convés dentro, e toquei vários barris e os soltei com a ferramenta, mas
não consegui abri"-los; senti também o rolo de folhas de chumbo inglês, e
consegui mexê"-lo, mas era pesado demais para removê"-lo.

\emph{10-14 de maio.} Fui todos os dias aos destroços; e retirei muitos
pedaços de madeira, tábuas ou pranchas, e duzentas ou trezentas libras
de ferro.

\emph{15 de maio.} Levei duas machadinhas para tentar arrancar algum
pedaço do rolo de chumbo, apoiando a lâmina de uma delas no rolo e
desferindo em sua parte traseira um golpe com o outro; mas como ele
estava obra de dois palmos dentro d'água, não tive bom sucesso.

\emph{16 de maio.} Havia ventado bastante durante a noite, e os
destroços pareciam mais frágeis pela força d'água, mas passei tanto
tempo na mata em busca de pombos para comer, que a maré me impediu de ir
aos destroços naquele dia.

\emph{17 de maio.} Vi alguns pedaços dos destroços na praia a grande
distância, obra de duas milhas de mim, mas resolvi ver o que eram e
descobri ser um pedaço da proa, mas pesado demais para que eu o levasse.

\emph{24 de maio.} Todos os últimos dias até hoje, trabalhei nos
destroços; e com trabalho árduo soltei tanto algumas coisas com o pé de
cabra, que a primeira maré cheia trouxe à praia vários barris e dois dos
baús dos marinheiros; mas com o vento hoje soprando da costa dia, nada
veio senão pedaços de madeira e um barrilete que trazia um pouco de
carne de porco dos Brasis; mas a água salgada e a areia a haviam
estragado.

Segui nessa tarefa todos os dias até \emph{15 de junho}, senão o tempo
necessário para caçar, ao qual reservei, durante esse período de
trabalho, o intervalo em que a maré estivesse alta, para que eu pudesse
estar pronto na vazante; e a essa altura eu já possuía madeira, tábuas e
ferragens suficientes para construir um bom barco, se soubesse como
fazê"-lo; e também consegui, em várias vezes e várias partes, quase cem
\emph{pesos} de folha de chumbo.

\emph{16 de junho.} Descendo à beira"-mar, encontrei uma grande
tartaruga. Essa foi a primeira que eu vi; o que, ao que parece, era um
infortúnio só meu, não qualquer problema de escassez ou defeito do
lugar; porque, caso eu estivesse do outro lado da ilha, poderia ter
encontrado centenas delas todos os dias, como descobri mais tarde; mas
talvez tivesse pago muito caro por isso.

\emph{17 de junho.} Passei o dia cozinhando a tartaruga. Eu encontrei
dentro dela sessenta ovos; e a carne dela foi para mim, então, a mais
saborosa e agradável que já havia provado na vida, sem ter me alimentado
de outra carne, desde que pusera os pés naquele lugar horrível, além de
cabras e aves.

\emph{18 de junho.} Choveu o dia todo e fiquei do lado de dentro. Achei
a chuva fria, e eu próprio um tanto gelado, o que sabia não ser comum
naquela latitude.

\emph{19 de junho.} Muito doente e tremendo, como se fizesse frio.

\emph{20 de junho.} Sem descanso a noite toda, dores de cabeça violentas
e febre.

\emph{21 de junho.} Muito doente e muito apreensivo e temeroso em
relação a minha triste condição, de estar doente e sem ajuda. Rezei a
\textsc{Deus} pela primeira vez desde a tempestade na costa de Hull, mas
mal era capaz de entender o que dizia ou sua razão; estando meus
pensamentos em absoluta confusão.

\emph{22 de junho.} Um pouco melhor; mas ainda sob terrível apreensão
ante a doença.

\emph{23 de junho.} Novamente me sentindo muito mal, com calafrios e dor
de cabeça violenta.

\emph{24 de junho.} Muito melhor.

\emph{25 de junho.} Maleita muito violenta, com o ataque perdurando sete
horas; crises de frio e calor, com forte suor em seguida.

\emph{26 de junho.} Melhor; e não tendo comida para comer, peguei minha
arma, mas me senti muito fraco; mesmo assim, matei uma cabra e, com
muita dificuldade a levei para casa, e assei um pouco dela, e comi; eu a
teria ensopado com muita alegria e feito um caldo, mas não tinha
caldeirão.

\emph{27 de junho.} A febre foi mais uma vez tão violenta que fiquei
deitado o dia todo, sem comer ou beber. Tinha certeza de que morreria de
sede; mas estava tão fraco que não tinha forças para me levantar ou
buscar água para beber. Orei novamente a Deus, mas tinha os pensamentos
confusos; e quando não estavam, era de tudo tão ignorante que não sabia
o que dizer; deitado, apenas clamava, ``Senhor, olhai para mim! Senhor,
tende pena de mim! Senhor, tende piedade de mim!'' Imagino não ter feito
outra coisa por duas ou três horas, até que o ataque cessou, e eu caí no
sono, sem acordar até tarde da noite; quando acordei, senti"-me muito
revigorado, mas fraco e com muita sede; no entanto, como não tinha água
em minha habitação, fui forçado a ali permanecer até a manhã seguinte e
dormi novamente. Nesse segundo sono, tive este sonho terrível.

Imaginava estar sentado no chão, do lado de fora do meu muro, onde
fiquei quando a tempestade soprou após o terremoto, e que via um homem
descer de uma grande nuvem negra, em uma chama brilhante de fogo, e
então ele tocava o chão. Ele inteiro brilhava como uma chama, de modo
que eu mal conseguia olhar para ele; seu semblante era assustador para
além do que posso descrever, não há como retratá"-lo; e quando ele tocou
o chão com os pés, era como se a terra tremesse, como antes no
terremoto, e todo o ar parecia, para meu assombro, como se estivesse
repleto de relâmpagos de fogo.

Logo que ele desceu sobre a terra, ele avançou em minha direção, com uma
lança longa ou uma arma na mão, para me matar; e quando ele chegou a um
terreno elevado, a certa distância, ele falou comigo, ou ouvi uma voz
tão terrível que é impossível expressar o terror que ela me causava.
Tudo o que posso dizer que entendi foi o seguinte: ``Se ao veres todas
essas coisas não te arrependeste, agora morrerás''; ao que, pronunciadas
essas palavras, ele levantou a lança que estava em sua mão para me
matar.

Ninguém que venha algum dia a ler este relato espera que eu seja capaz
de descrever os horrores da minha alma diante dessa visão terrível,
pois, muito embora fosse um sonho, eu sonhei com esses horrores; também
não é possível colocar em palavras a impressão que permaneceu em minha
mente quando acordei e descobri que fora apenas um sonho.

Ai! eu não tinha conhecimento do divino. O que recebera da boa instrução
de meu pai já se havia perdido, então, por uma ininterrupta sequência de
oito anos de vicissitudes de convés e diálogos constantes com homens que
a mim não eram dessemelhantes, dissolutos e profanos no mais elevado
grau. Não me lembro de ter tido, durante todo aquele período, um
pensamento que me fizesse ou olhar aos céus na direção de Deus, ou para
dentro de mim mesmo, em reflexão acerca de minha conduta; lembro, sim,
de certo idiotismo da alma, sem anseio do bom, ou consciência do mal,
que me dominava por completo; e, assim, eu era a mais embotada, vulgar e
corrupta criatura que se podia imaginar em meio à marinhagem, sem ter a
mínima ideia, fosse do temor a Deus, quando em perigo, ou da gratidão a
Deus, quando em salvamentos.

Ao dar relação do que já se faz passado em minha história, será mais
fácil dar"-lhe credito se a ela se ajuntar o comentário de que, ao longo
de toda a variedade de infortúnios que até este dia me assaltaram, eu
nunca, sequer uma vez, pensei tratar"-se da mão de Deus, ou que fosse um
justo castigo por ter pecado; ao que correspondiam meu comportamento
rebelde contra meu pai e meus pecados presentes, que eram muitos; ou
mesmo um castigo pelo curso geral de minha vida de vícios. Quando me
encontrava em minha desesperada expedição pelas costas desertas de
África, nunca cheguei a pensar no que seria de mim, nem a desejar uma só
vez que Deus me guiasse até onde devesse chegar, ou que me protegesse do
perigo que julgava me cercar, fosse ele de criaturas vorazes, fosse do
bárbaro cruel. Mas eu era tão somente insciente de um Deus ou de uma
Providência, meus atos eram os de um simples bruto governado pelos
princípios da natureza; ou quando muito pelos ditames do senso comum.

Quando fui salvo e recolhido no mar pelo capitão português, bem recebido
e tratado com honra e justiça, bem como com caridade, eu não conhecia a
menor gratidão em meus pensamentos. Quando, de novo, naufraguei e me
encontrei arruinado, sob o risco de afogar"-me nesta ilha, o remorso ia
longe de mim, tampouco via tudo aquilo como um castigo. Apenas disse a
mim mesmo muitas vezes que eu era um cão desgraçado, nascido para
conhecer unicamente o infortúnio.

É verdade, quando eu cheguei àquela praia e vi toda a tripulação de meu
navio afogada e eu somente vivo, surpreendi"-me com a espécie de êxtase e
os transportes da alma que, com o auxílio da graça de Deus, se teriam
elevado ao mais genuíno agradecimento; mas tudo acabou onde começou, em
um simples acesso de alegria ou, como posso dizer, na satisfação de
estar vivo, sem a mínima reflexão sobre a distinta bondade da mão que me
poupara e me destinara a ser poupado quando todos os demais haviam sido
aniquilados, nem uma investigação acerca das razões de a Providência ter
sido misericordiosa comigo. Precisamente como o tipo vulgar de alegria
que os marinheiros geralmente têm, depois de terem chegado a salvo à
praia depois de um naufrágio, e que eles afogam logo em seguida num
trago de ponche, esquecendo"-se quase tão logo o perigo se afasta; e todo
o resto da minha vida eu vivi dessa forma.

Mesmo quando, posteriormente, sob ponderada consideração, fiz"-me
consciente de minha condição, de como fora abandonado àquele lugar
abominável, inalcançável à espécie humana, sem qualquer esperança de
conforto ou redenção, tão logo vi um só futuro, o de não padecer e
morrer de fome, todo o sentido de meu sofrimento se desfez; e eu comecei
a me acalmar, a aplicar"-me às tarefas adequadas à minha preservação e
provimento, e se me tornou bem remota a possibilidade de sofrer com
minha condição como um castigo dos céus, ou como a mão de Deus
abatendo"-se sobre mim; tais pensamentos muito raramente passavam"-me pela
cabeça.

O crescimento dos grãos, como sugerido em meu diário, teve de início um
pouco de influência sobre mim, e começou a afetar"-me seriamente à medida
que pensei que trouxesse consigo algo de milagroso; mas logo que parte
desse pensamento se desfez, toda a impressão que surgira dele também se
apagou, como já sugeri.

Mesmo o terremoto, embora nada pudesse ser mais terrível em sua
natureza, ou que mais imediatamente remetesse ao poder invisível que
sozinho guia todas as coisas a ele semelhantes; assim que o primeiro
terror se foi, com ele partiu a impressão que deixara. Eu não tinha mais
ideia de Deus ou de seus castigos, nem menos do papel de sua mão na
presente aflição de minhas circunstâncias, do que se me encontrasse na
mais próspera condição da vida.

Mas ali, quando comecei a adoecer, e uma ideia desanuviada dos mistérios
da morte colocou"-se diante de mim; quando meu ânimo passou a ceder sob o
fardo de uma forte confusão, e a natureza se exauriu com a violência da
febre; a consciência, que por tanto tempo permanecera em sono profundo,
começou a despertar, e eu comecei a repreender"-me por minha vida
passada, na qual, por uma inaudita tendência ao vício, eu havia tão
evidentemente provocado a justiça de Deus que este colocou"-me sob
inauditos abalos, abatendo"-se sobre mim com seu poder daquela maneira
tão vingativa.

Senti a opressão dessas reflexões durante o segundo ou terceiro dia de
minha perturbação; e em meio a sua violência, fosse da febre, fosse das
terríveis reprimendas que me vinham da consciência, elas arrancaram"-me
palavras que tinham algo de prece a Deus, embora não pudesse dizer que
fosse prece acompanhada de esperanças e anseios, senão voz de puro medo
e terror. Meus pensamentos embaralhavam"-se, as culpas eram imensas em
minha mente, e o horror de morrer em tão desditosa condição nublava meu
juízo ante sua mera sugestão; e em meio a tais tribulações da alma não
sabia o que minha língua queria dizer. Antes eram clamores, como
``Senhor, que miserável criatura sou! Se eu cair doente, é certo que
morrerei por falta de ajuda; ai, o que será de mim!'' As lágrimas,
então, brotavam de meus olhos, e não fui capaz de dizer o que quer que
fosse por algum tempo.

Nesse ínterim, lembrei"-me do bom conselho de meu pai, e com ele sua
predição, que mencionei no início de minha história; a saber, que se eu
desse esse passo estúpido, Deus não me abençoaria, e eu teria tempo dali
em diante para refletir sobre ter negligenciado seu conselho quando não
houvesse quem me assistisse em minha recuperação. ``Ora'', disse eu,
``as palavras de meu pai se fizeram reais; a justiça de Deus se abate
sobre mim, e não tenho quem me ouça ou socorra. Rejeitei a voz da
Providência, que misericordiosamente me havia posto em uma posição ou
estado da vida em que eu poderia ter tido uma vida feliz e tranquila;
mas não o consegui ver, tampouco fui capaz de aprender com os meus pais
a reconhecer a bênção presente naquilo tudo. Abandonei"-os à tristeza de
lamentar minha loucura, e agora estou abandonado à tristeza de lamentar
as consequências disso; fiz mau uso de seu auxílio e assistência, que me
teriam dado boa condição no mundo e tornado tudo simples; tinha, pelo
contrário, dificuldades a superar, enormes demais para a própria
natureza suportar, e nenhuma assistência, auxílio, conforto, conselho.''
Foi então que bradei, ``Senhor, sede meu auxílio, pois passo por grande
aflição''. Essa foi minha primeira prece, se assim posso chamá"-la, e a
fiz por muitos anos. Mas retorno a meu diário.

\emph{28 de junho.} De alguma forma revigorado pelo sono, e com a crise
arrefecida, levantei"-me; e embora fossem imensos o medo e o terror que o
sonho me havia trazido, imaginei que o ataque de febre se repetiria no
dia seguinte; de guisa que era o momento de sair à caça do que pudesse
me restaurar e alimentar enquanto estivesse doente; e a primeira coisa
que fiz foi encher um garrafão d'água e colocá"-lo sobre a mesa, ao
alcance de minha cama; e para desfazer a disposição febril ou resfriante
daquela água, misturei a ela um quarto de pinta de rum; em seguida,
separei um pedaço de carne de cabra e a assei nos carvões, mas não pude
comer muito; caminhei um pouco, mas me sentia muito fraco e de modo
geral muito triste e de coração apertado ante a ideia de minha triste
condição, temendo o retorno de minha moléstia no dia seguinte; à noite,
jantei três dos ovos de tartaruga, que assei nas brasas e comi, como se
diz, na casca, e, que eu me recorde, essa foi a primeira refeição na
vida à qual pedi a bênção de Deus.

Depois de comer, tentei andar, mas me vi tão fraco que mal era capaz de
carregar a espingarda, pois sem ela jamais saía; caminhei, porém, um
trecho curto e me sentei no chão, com os olhos ao mar, que estava bem
diante de mim, bastante calmo. Sentado ali, vi ocorrerem"-me alguns
pensamentos:

Que são o mar e a terra, dos quais tantas coisas havia visto? De onde
vieram? E o que sou eu, e todas as demais criaturas, as selvagens e as
domésticas, as humanas e as bestiais, de onde viemos?

Certamente somos todos feitos por um poder secreto, que formou a terra e
o mar, o céu e o ar. E quem é?

Seguiu"-se naturalmente, então, que é Deus quem fez todas as coisas. Pois
bem: o argumento toma um estranho rumo, pois se foi Deus que fez todas
as coisas, ele guia e governa tudo, e todas as coisas que o concernem;
pois o poder que poderia fazer todas as coisas deve decerto ter o poder
de guiá"-las e conduzi"-las.

Se assim era, nada pode acontecer no grande circuito de suas obras,
mesmo sem seu conhecimento ou determinação.

E se nada acontece sem o seu conhecimento, ele sabe que eu estou aqui, e
estou nesta terrível condição; e se nada acontece sem sua determinação,
ele determinou que tudo isso se abatesse sobre mim.

Nada ocorreu aos meus pensamentos que contradissesse qualquer uma dessas
conclusões; e daí que me ocorreu, com força ainda maior, que Deus
necessariamente havia determinado que todas aquelas coisas me
acometessem; que eu havia sido trazido a essas circunstâncias miseráveis
por seu instituto, tendo ele o único poder, não só sobre mim, mas sobre
tudo que no mundo se passava; donde se seguiram as perguntas:

Por que Deus fez isso comigo? O que eu fiz para ser assim tratado?

Fui interrompido nessas perguntas por minha consciência, como se tivesse
blasfemado, e parecia"-me que ela falava a mim como uma voz: ``\textsc{M
i s e r á v e l}! Perguntas sobre o que fizeste? Olha para trás, para a
tua vida horrível e mal vivida e pergunta a ti mesmo o que não fizeste!
Pergunta, por que não conheceste a destruição há muito tempo? Por que
não te afogaste nas águas da enseada de Yarmouth; por que não foste
morto em batalha quando o navio foi dominado pela nau de guerra de Salé;
por que não foste devorado pelas bestas selvagens na costa de África; ou
não te afogaste \textsc{a q u i}, quando toda a tripulação pereceu,
exceto por ti? Perguntas `o que eu fiz?'''

Essas reflexões emudeceram"-me, deixaram"-me estupefato, sem uma palavra a
dizer; e não fui capaz de responder a mim mesmo, e levantei"-me,
melancólico e entristecido, caminhei de volta a meu refúgio e subi o
muro como se estivesse indo para a cama; mas meus pensamentos haviam
sido perturbados com imenso pesar, e eu não tinha vontade de dormir;
então, sentei"-me em minha cadeira e acendi meu candeeiro, pois começava
a ficar escuro: à medida, então, que a apreensão quanto ao retorno de
minha enfermidade me amedrontava, ocorreu"-me que os brasileiros não
usavam outra droga que não o tabaco para quase todas as suas moléstias,
e eu tinha um pedaço de rolo de tabaco quase curado em um dos baús e
outro que ainda estava verde, não de todo curado.

Recorri a ele, guiado por Deus, sem dúvida; pois nesse baú encontrei
cura para o corpo e para a alma. Abri o baú e encontrei o que procurava,
o tabaco; e como os poucos livros que resgatara ali também estavam,
peguei uma das Bíblias que mencionei atrás, e que até esse momento eu
não tinha encontrado tempo ou inclinação para abrir; e trouxe ambos, o
tabaco e a Bíblia, comigo à mesa.

O uso que devia fazer do tabaco em minha enfermidade, ou se ele faria
bem para ela ou não, eu não sabia; mas eu tentei muitos experimentos com
ele, como estivesse decidido a acertar de uma forma ou de outra.
Primeiro peguei um pedaço de folha e o masquei; o que, em verdade, de
pronto quase me deixou tonto, estando o tabaco verde e forte, e não
estando eu muito acostumado a ele. Então peguei mais um pouco e o
mergulhei por uma ou duas horas em um pouco de rum e decidi tomar um
trago dele antes de me deitar; e por fim, queimei"-o em pouca quantidade
em uma panela de brasas e mantive tanto quanto pude, fosse pelo calor,
fosse pelo sufocamento, meu nariz sobre os fumos que subiam.

No espaço dessa operação, abri a Bíblia e comecei a lê"-la; mas o tabaco
havia perturbado demasiado meus pensamentos para conseguir ler, ao menos
naquele momento; tendo, porém, aberto o livro despretensiosamente, as
primeiras palavras que se me depararam foram estas: ``E invoca"-me no dia
da angústia; eu te livrarei, e tu me glorificarás.''\footnote{Salmos
  50:15 {[}\textsc{n.\,e.}{]}}

As palavras casavam muito bem com a minha situação e causaram alguma
impressão em meus pensamentos no momento em que foram lidas, embora não
tanto quanto causaram posteriormente; pois quanto a conhecer livramento,
a palavra não era palpável, por assim dizer, para mim; sendo coisa tão
remota, tão impossível em meu entendimento das coisas, que eu comecei a
dizer como os filhos de Israel, quando lhes fora prometido carne para
comer, ``Acaso Deus pode preparar"-nos uma mesa no deserto?''\footnote{Salmos
  78:19 {[}\textsc{n.\,e.}{]}} Então eu comecei a me perguntar, ``Poderia Deus me
libertar desse lugar?'', e como muitos anos se passaram sem essa
esperança se manifestar, essa dúvida por muito tempo preponderou;
contudo, aquelas palavras causaram grande impressão em mim, e refleti
sobre elas muitas vezes. Ficou tarde, e o tabaco tinha, como disse,
deixado minha cabeça tão lesa que me vi inclinado a dormir; de maneira
que deixei meu candeeiro ardendo na caverna, temendo sentir falta de
qualquer coisa à noite, e fui para a cama. Mas antes de me deitar, fiz o
que nunca tinha feito em minha vida, ajoelhei"-me e rezei a Deus para que
cumprisse sua promessa, que se eu o invocasse em dia de angústia, ele me
daria livramento; e depois de terminar minha prece, imperfeita e
trôpega, bebi o rum em que havia mergulhado o tabaco, que estava tão
forte e impregnado das folhas que mal era capaz de tomá"-lo; e
imediatamente após isso, fui para a cama. Percebi de pronto que ele
subiu"-me à cabeça violentamente; e então caí num sono profundo, não
acordando até que, pelo sol, fossem três da tarde do dia seguinte; ou
melhor, dessa hora em diante, tenho para mim que talvez tenha dormido
todo o dia e noite seguintes, até que acordei quase às três horas do dia
posterior a esse; pois de outro modo não sei como poderia ter perdido um
dia em minha conta dos dias da semana, como alguns anos depois me
pareceu que perdera; pois se eu o tivesse perdido por ter cruzado e
recruzado a Linha,\footnote{A personagem se refere aqui à linha do
  Equador. A referência é equivocada: perdem"-se e ganham"-se dias e horas
  na travessia de longitudes, não latitudes.} eu teria perdido mais de
um dia; mas certamente perdi um dia na minha contagem, e nunca soube de
que maneira.

Tenha se dado de uma maneira ou de outra, de qualquer forma, quando eu
acordei, senti"-me absolutamente revigorado, e de espírito vivaz e
alegre; e quando me levantei, estava mais forte do que no dia anterior,
e meu estômago melhor, pois estava com fome; e, em suma, não tive
acessos no dia seguinte e segui melhorando. Esse foi o dia 29.

No dia 30 me senti bem, é claro, e saí com minha espingarda, mas não
quis alongar"-me de minha casa; matei uma ou duas aves marinhas e outra
semelhante a um ganso selvagem, e as levei comigo, mas não me vi muito
animado de comê"-las; então, comi mais dos ovos de tartaruga, que eram
muito saborosos. Nessa noite, voltei a tomar do remédio, que supus ter
feito bem a mim no dia anterior, isto é, o tabaco mergulhado no rum; só
não bebi tanto quanto antes, nem masquei as folhas ou mantive a cabeça
aos fumos; não me senti tão bem no dia seguinte, contudo, que era o
primeiro de julho, quanto imaginei que pudesse me sentir; pois eu tive
um pouco de acesso de calafrios, mas não de grande monta.

\emph{2 de julho.} Voltei às três guisas de uso do remédio, mediquei"-me
como de início e dobrei o trago que bebi.

\emph{3 de julho.} Os acessos sumiram definitivamente, embora não tenha
recobrado o todo de minhas forças senão muitas semanas depois. Enquanto
recuperava dessa maneira a saúde, meus pensamentos voltavam"-se sempre à
Escritura, ``Eu te livrarei'', mas a impossibilidade de meu livramento
não me deixava e fazia"-se obstáculo à esperança; ao desanimar ante tais
pensamentos, porém, ocorreu"-me que tanto ansiava meu livramento da maior
das aflições que pouca atenção eu dava ao livramento que havia recebido;
e eu era como que levado a me fazer perguntas como, a saber, ``Não havia
eu sido livrado, e maravilhosamente, da doença? E da mais terrível
condição que podia haver, e também tão assustadora para mim? E que lição
eu havia disso tirado? Tinha eu feito o que me cabia? Deus me havia dado
livramento, mas eu não o havia glorificado; ou seja, eu não atribuía,
nem era agradecido ao que me ocorrera como um livramento; e como eu
podia esperar livramento maior?''

Isso tocou demasiado meu coração; e imediatamente me ajoelhei e agradeci
em voz alta a Deus por minha recuperação de minha enfermidade.

\emph{4 de julho.} De manhã peguei
a Bíblia; e começando pelo Novo Testamento, pus"-me a lê"-lo com seriedade
e estabeleci a mim mesmo sua leitura um pouco toda manhã e toda noite;
não me atendo ao número de divisões, mas a tanto quanto meus pensamentos
me movessem. Não demorou muito, depois de eu ter me empenhado nessa
tarefa, para que sentisse meu coração mais profunda e sinceramente
afetado pelos vícios de minha vida pregressa. A impressão do meu sonho
reviveu; e as palavras ``Se ao veres todas essas coisas não te
arrependeste'' atravessavam com severidade todos os meus pensamentos. Eu
estava implorando seriamente a Deus que me desse arrependimento, quando
aconteceu, providencialmente, que, lendo as Escrituras, deparei"-me com
as seguintes palavras, ``Deus com a sua destra o elevou a Príncipe e
Salvador, para dar o arrependimento e a remissão dos
pecados''.\footnote{Atos 5:31} Deixei o livro; e com o meu coração, bem
como minhas mãos erguidas aos céus, transportado em êxtase e felicidade,
bradei, ``Jesus, filho de Davi! Jesus, Príncipe e Salvador elevado!
Dá"-me arrependimento!''

Essa foi a primeira vez em toda a minha vida que havia sido capaz de
dizer, no verdadeiro sentido da palavra, que rezei; pois rezei com um
entendimento de minha condição e uma ideia de esperança digna das
Escrituras, fundada no encorajamento da palavra de Deus; e a partir de
então, eu posso dizer que passei a ter a esperança de que Deus me ouviu.

Agora eu comecei a interpretar as palavras acima mencionadas, ``E
invoca"-me no dia da angústia; eu te livrarei'', num sentido diferente do
que jamais fizera; pois naqueles tempos eu não tinha outra ideia do que
se chamava livramento, senão de livrar"-me ao cativeiro em que eu me
encontrava; pois embora caminhasse sem impedimentos pela ilha, a ilha em
verdade me era uma prisão, e no pior sentido da palavra; mas agora eu
aprendi a compreender a palavra de outra forma. Eu observava minha vida
em retrospecto com tamanho horror, e meus pecados pareciam"-me tão
terríveis, que minha alma não buscava em Deus outra coisa senão o
livramento do fardo de culpa que se alevantava ameaçador contra toda a
minha consolação. Quanto à minha vida solitária, ela nada significava.
Eu já não rezava para o meu livramento dela, nem pensava a seu respeito;
nada mais importava em comparação a isto que aqui acrescento à maneira
de sugestão a quem me leia, a saber, que sempre que se chega a um
verdadeiro sentido das coisas, o livramento do pecado se revela uma
bênção muito maior do que o livramento da aflição.

Mas, feito o acréscimo, retorno ao meu diário.

Minha condição começava agora a ser, embora não menos desditosa no que
toca à minha maneira de viver, ao menos muito mais tranquila em minha
mente; e estando os meus pensamentos orientados a coisas de natureza
mais elevada por uma constante leitura das Escrituras e as preces a
Deus, eu guardava em meu íntimo grande conforto, tal como até então não
havia conhecido; também, com minha saúde e força de volta, trabalhava
vivamente para prover"-me de tudo que quisesse e tornar minha maneira de
viver tão desimpedida de contratempos quanto pudesse.

Do dia 4 ao dia 14 de julho, cuidei sobretudo de sair em caminhada
munido de minha espingarda, sempre um pouco mais longe por vez, como um
homem que recupera forças depois de um acesso de doença; pois mal se
pode imaginar o quanto padecera e a que a doença me reduzira. O
medicamento de que fizera uso era de todo novo e talvez nunca tivesse
curado a maleita antes; tampouco o recomendo a quem seja, por causa
desse experimento; e embora tenha desbastado o acesso, contribuiu
demasiado para me enfraquecer; pois tive frequentes convulsões de meus
nervos e membros por algum tempo.

Disso tudo tirei também e em particular a seguinte lição, que permanecer
ao ar livre na estação chuvosa era coisa mais perniciosa para minha
saúde, sobretudo naquelas chuvas que vinham seguidas de tempestades e
furacões de vento; visto que as chuvas que vinham na estação de seca
eram quase sempre acompanhadas de tais tempestades, eu descobri que essa
era uma chuva muito mais perigosa do que a chuva que caía em setembro e
outubro.

Eu já estava naquela ilha infeliz havia obra de dez meses, parecendo"-me
que privado de toda a possibilidade de livramento de uma tal condição; e
tendo a firme certeza de que forma humana nenhuma jamais havia posto os
pés naquele lugar. Tendo feito minha habitação completamente segura
segundo o entendia, tive grande desejo de fazer um maior descobrimento
da ilha para ver que outros produtos poderia encontrar, dos quais nada
sabia.

Foi no dia 15 de julho que comecei a fazer uma investigação mais detida
da ilha em si: primeiro, subi o riacho, a partir de onde, supunha, eu
havia fundeado com minhas jangadas; descobri depois de duas milhas
correnteza acima que a maré já não o enchia e que ele não passava de um
riachinho de água doce e muito limpa; mas sendo a estação seca, mal
havia água em alguns trechos dele; ao menos não o bastante para que
corresse qualquer fluxo, como se podia perceber. Nas margens desse
córrego encontrei muitas agradáveis savanas ou campinas, planas, lisas,
e cobertas de relva; e nas partes em que o terreno se elevava, próximas
às partes mais altas da ilha, onde a água, como se podia supor, nunca
inundava, encontrei muitas plantas de tabaco, verdes, e crescendo em
talos grandes e muito fortes; e havia grande diversidade de outras
plantas, das quais não tinha qualquer entendimento ou conhecimento e,
talvez, tivessem virtudes próprias, as quais eu não podia esclarecer.

Procurei por raízes de cassava ou mandioca, das quais os índios, em
quaisquer climas, fazem seu pão, mas não as encontrei. Vi grandes
plantas de aloé, mas então não sabia o que eram. Vi muitas plantas de
cana de açúcar, mas incultas, imperfeitas, carecidas de cultivo. Dei"-me,
então, por satisfeito com essas descobertas e retornei, pensando comigo
mesmo sobre o curso que poderia tomar para conhecer a virtude e a
dignidade de todas as plantas ou frutas que descobrisse, mas não
conseguia chegar a uma conclusão; pois, em suma, fizera tão poucas
investigações enquanto estivera nos Brasis que quase nada sabia das
plantas selvagens; pelo menos, pouquíssimo que pudesse ser de qualquer
serventia em meu padecimento.

No dia seguinte, o décimo sexto do mês, segui pelo mesmo caminho; e
avançando um pouco mais do que avançara no dia anterior, vi que o
córrego e as savanas chegavam ao fim, e as terras cobriam"-se de mais
árvores do que antes. Nesse lugar encontrei toda casta de frutas, em
particular melões em grande abundância sobre a terra e bagas de uvas nas
árvores. As vinhas haviam se espalhado sobre as árvores, e os cachos
encontravam"-se então em sua plenitude, muito cheios e maduros. Essa foi
uma descoberta surpreendente, e fiquei muito feliz com ela; mas a
experiência me havia ensinado que devia comer muito pouco delas;
lembrando"-me que quando estava em terra firme na Berbéria, o consumo de
bagas de uva matou muitos dos nossos ingleses que ali haviam sido feitos
escravos, ao abatê"-los com febres e disenterias. Mas eu encontrei uso
excelente dessas bagas; qual seja, o de curá"-las ou secá"-las ao sol e
conservá"-las, portanto, passas, o que pensei que fosse, como de fato
era, saudável e saboroso de se comer quando não pudesse mais colhê"-las.

Passei toda a noite ali e não retornei a minha habitação; o que, a
propósito, foi a primeira noite, digamos, que pousei longe de casa. À
noite, segui minha primeira precaução e subi em uma árvore, onde dormi
bem; e na manhã que sucedeu prossegui com meu descobrimento; viajando
obra de quatro milhas, a julgar pela extensão do vale, mantendo o rumo
norte, acompanhando uma serra que se estendia desde o sul naquela
direção.

No fim dessa caminhada, dei com uma clareira onde o terreno parecia
formar um declive a oeste; e uma pequena fonte d'água fresca, que
rebentava da banda da colina mais próxima, corria para a outra, ou seja,
para o leste; e as terras pareciam tão vivas, verdes e florescentes,
tudo parecia em tão constante verdor ou florescer de primavera que mais
parecia um jardim cultivado.

Desci um pouco para a banda daquele aprazível vale, examinando"-o com uma
espécie de prazer secreto (embora misturado a meus outros e aflitos
pensamentos), ao concluir que tudo aquilo era meu; e que eu era um rei
ou senhor irrevogável de todas aquelas terras e tinha seu direito de
posse; e se eu pudesse reclamá"-las, poderia tê"-las como patrimônio tão
completamente quanto qualquer senhor de herdade na Inglaterra. Vi
abundância de cacaueiros, laranjeiras, limeiras e limoeiros; mas todos
selvagens, e muitos poucos deles carregados de frutos, ao menos não
então. As limas verdes que colhi, porém, não só eram muito saborosas,
com também salutares; e eu misturei seu suco depois a água, resultando
refrescantes, agradáveis e saudáveis.

Vi, então, que tinha trabalho o bastante para recolhê"-las e levá"-las
para casa; e eu decidi abastecer"-me de provisões de limões, limas e uvas
para a estação das chuvas, que eu sabia que se avizinhava.

Para tanto, reuni grande quantidade de bagas em um monte num lugar, e
outro menor em outro lugar, e grande parte dos limões e limas num
terceiro; e levando um pouco de cada porção comigo, viajei de volta para
casa; decidido a regressar levando comigo uma sacola ou saco ou o que
pudesse fabricar para carregar o resto para casa. Assim, tendo passado
três dias nessa caminhada, cheguei em casa (assim devo chamar agora
minha tenda e minha caverna); mas antes que tivesse chegado, as uvas
tinham se estragado; uma vez que a sustância do fruto e peso do sumo as
tinham arrebentado e machucado, para pouco ou nada serviam; quanto aos
limões, estavam todos em bom estado, mas pude levá"-los apenas uns
poucos.

No dia seguinte, o décimo nono, eu retornei, tendo fabricado duas
pequenas sacolas para trazer para casa minha colheita; mas
surpreendeu"-me que, chegando ao meu monte de uvas, tão suculentas e
saborosas quando as colhi, as encontrei todas espalhadas, estraçalhadas
e lançadas por toda parte, aqui e ali, e comidas e devoradas em
abundância. Daí que concluí que havia criaturas selvagens nas
imediações, as quais tinham feito aquilo; mas o que eram, eu não sabia.

Contudo, ao ver que não havia como empilhá"-las, nem carregá"-las nos
sacos sem que não acabassem destruídas ou esmagadas por seu próprio
peso, decidi proceder de outra forma; pois reuni uma grande quantidade
de uvas e as pendurei nas pontas dos galhos das árvores, para que
pudessem curar e secar ao sol; e quanto às limas e limões, carreguei"-os
tantos quantos fui capaz.

Enquanto retornava dessa jornada, contemplei com grande prazer a
fertilidade daquele vale e quão aprazível o lugar se mostrava; e o
abrigo às tempestades que naquela banda das águas havia, e a floresta; e
concluí que havia escolhido como sítio para fixar minha residência, sem
sombra de dúvida, a pior região da ilha. Passei, então, a considerar a
mudança de minha habitação e procurar por lugar tão seguro quanto aquele
onde me encontrava instalado naquela farta e agradável região.

Esse pensamento ocupou por muito tempo minha cabeça, e por um período
ele me agradou bastante, tentado que estava pela aprazibilidade do
lugar; mas quando cheguei a uma reflexão mais detida, levei em conta que
estava, então, à beira"-mar, onde era ao menos plausível que algo pudesse
acontecer em meu proveito e, pelo mesmo desfortúnio que me fizera dar
ali, outros pobres coitados poderiam chegar ali; e embora fosse
vagamente provável que se passasse algo desse tipo, ainda assim,
cercar"-me de colinas e mata no centro da ilha era antecipar minha
servidão e tornar meu salvamento não apenas improvável, como impossível;
e que, assim, entendi que não devia de forma alguma fazer a mudança.

No entanto, tamanho era o encanto que sentia por aquele lugar que passei
muito do meu tempo ali durante os dias restantes do mês de julho; e,
embora, ao pensar melhor, houvesse decidido não me mudar, construí ali
uma guisa de morada e a protegi à distância com uma cerca forte, sendo
esta um tabique duplo, tão alto quanto o pudesse alcançar, de estacas
bem firmes e preenchido de galhos; e ali, muito seguro, passava por
vezes duas ou três noites seguidas, sempre atravessando por cima dele
com uma escada, como antes; de maneira que imaginei ter, então, minha
casa no campo e minha casa no litoral; e com esse trabalho segui até o
começo de agosto.

Havia eu acabado de terminar minha cerca e começado a desfrutar de meu
trabalho, quando as chuvas vieram, e forçaram"-me a permanecer em minha
primeira habitação, pois embora tivesse feito uma tenda como a outra,
com um pedaço de vela, e a tivesse estendido muito bem, eu não tinha ali
o agasalho de uma colina para proteger"-me das tempestades, nem uma
caverna atrás de mim em que me recolhesse quando as chuvas eram
copiosas.

No início de agosto, como disse, eu havia terminado minha morada e
começado a gozar a vida. Em 3 de agosto, encontrei as uvas que havia
pendurado perfeitamente secas e, de fato, o sol produziu excelentes
passas; então comecei a retirá"-las das árvores, e minha decisão foi
muito feliz, pois as chuvas que se seguiram as teriam estragado, e eu
havia perdido a melhor parte de minha comida de inverno; pois eu tinha
cerca de duzentos cachos fartos dela. Tão logo eu os havia recolhido
todos, e os transportado em sua maior parte para minha caverna, começou
a chover; e a partir daí, que era o 14 de agosto, choveu mais ou menos
todos os dias até meados de outubro; e às vezes tão violentamente que eu
não conseguia deixar minha caverna por muitos dias.

Nessa estação, surpreendeu"-me muito o aumento de minha família; eu
ficara preocupado com o desaparecimento de um dos meus gatos, que fugiu
de mim ou, como pensei, morrera, e dela não tivera notícias até que,
para minha surpresa, ela retornou em fins de agosto com três filhotes.
Isso me pareceu muito estranho porque, embora houvesse matado um gato
selvagem, como o chamei, com minha espingarda, pensei que se tratasse de
casta diversa de nossos gatos europeus; mas os filhotes eram do mesmo
tipo doméstico que a gata; e sendo meus gatos ambos do sexo feminino,
julguei tudo muito estranho. Mas a partir desses gatos passei a ser tão
incomodado por sua infestação que fui forçado a matá"-los como pragas ou
animais selvagens e a levá"-los para longe de minha casa tanto quanto
possível.

Do dia 14 ao dia 26 de agosto, a chuva foi incessante, de modo que eu
não podia sair, e cuidava para que não me molhasse demais. Nesse
confinamento, comecei a sentir privação de comida: mas saindo duas
vezes, matei uma cabra num dia, e no outro, que, foi o dia 26, encontrei
uma grande tartaruga, que me foi grande deleite, e minha comida assim se
regulou; comia um punhado de passas no dejejum; ceava um pedaço de carne
de cabra, ou de tartaruga, assada, pois, para meu grande infortúnio, não
tinha panela para ferver ou cozinhar o que fosse; e dois ou três dos
ovos de tartaruga no jantar.

Durante esse confinamento em meu abrigo para proteger"-me da chuva,
trabalhei duas ou três horas todos os dias no aumento de minha caverna,
e aos poucos avancei em um dos lados, até que cheguei à parte externa da
colina e cavei uma porta ou passagem, que confinava no lado de fora de
meu muro ou paliçada; e por ela entrava e saía. Mas eu não fiquei muito
perfeitamente à vontade com essa abertura; pois, como eu tinha cuidado
para que assim fosse, eu estivera em absoluta clausura; enquanto,
naquele momento, julgava estar exposto, e aberto a qualquer coisa que me
sobreviesse; e, no entanto, não tinha a percepção de que houvesse
qualquer coisa a temer, uma vez que a maior criatura que havia visto na
ilha era uma cabra.

\emph{30 de setembro.} Era chegado o infeliz aniversário do meu
desembarque. Contei os talhos em meu poste e verifiquei que estava
naquelas praias a trezentos e sessenta e cinco dias. Reservei o dia para
um solene jejum e exercícios espirituais, prostrando"-me ao chão na mais
séria humilhação, confessando meus pecados a Deus, reconhecendo Seus
justos juízos sobre mim e rezando para que Ele tivesse misericórdia por
mim através de Jesus Cristo; e não tendo experimentado o menor
restaurativo por doze horas, até mesmo depois de o sol se pôr, eu então
comi um bocado de bolacha e um punhado de uvas e me deitei, terminando
meu dia como o comecei.

Até aquele momento, não havia observado o dia do descanso religioso;
pois como de início não conhecia qualquer sentido de religião, depois de
algum tempo eu passei a não distinguir os dias da semana mediante um
talho mais longo do que o comum para o sábado, e assim de fato não sabia
quais eram os dias da semana; mas tendo então assinalado os dias, como o
disse atrás, verifiquei que um ano se havia completado; e assim eu o
dividira em semanas e separei todo sétimo dia como um sábado; embora
tenha descoberto ao fim de meu cômputo que perdera um dia ou dois em meu
registro.

Passado pouco tempo, minha tinta começou a me faltar, e assim me
contentei em usá"-la com mais parcimônia e a declarar apenas os
acontecimentos mais notáveis de minha vida, sem um diário
\emph{memorandum} de outras coisas.

A estação chuvosa e a estação seca começaram então a esclarecer"-se para
mim, e eu aprendi a dividi"-las dessa forma e assim a me prover conforme
a necessidade. Antes disso, contudo, paguei alto preço por minha
experiência, e o que a seguir relato foi um dos mais infelizes
experimentos que empreendi. Mencionei que havia guardado algumas espigas
de cevada e arroz, as quais para meu espanto haviam brotado, como eu
pensava, por si mesmas, e creio que havia obra de trinta talos de arroz
e vinte de cevada, e eu entendi que era tempo favorável de plantá"-las
passadas as chuvas, estando o sol em sua posição sul em relação a mim.

Assim, cavei com minha pá de madeira um pedaço de chão tão bem quanto
podia, e o dividindo em duas partes, semeei meus grãos; mas enquanto os
semeava, por acaso ocorreu"-me que eu não os semeasse todos de uma vez,
pois eu não sabia dizer quando era o tempo apropriado para tanto; e eu
semeei dois terços das sementes, conservando um punhado de cada.

Enorme foi meu alívio posteriormente por assim fazê"-lo, pois nenhum dos
grãos que semeei deu em algo; pois com os meses de seca iniciando na
sequência, e a terra não recebendo chuva depois de feita a semeadura,
não havia umidade que assistisse seu crescimento, e elas não brotaram
até que a estação de chuvas retornasse e então eles rebentaram como se
tivessem sido recentemente plantados.

Observando que minhas sementes não cresciam, o que eu facilmente supus
que ocorresse em razão da seca, sai em demanda de um pedaço de chão
úmido para uma nova tentativa, e eu cavei um pedaço de chão próximo de
minha habitação e semeei o resto de minhas sementes em fevereiro, um
pouco antes do equinócio vernal, e tendo os meses chuvosos de março e
abril para regá"-las, elas brotaram muito bem, e renderam ótima colheita;
mas tendo conservado apenas parte das sementes, e não ousando semear
tudo que tinha, restava"-me apenas uma pequena quantidade por fim, sem
que meus grãos chegassem a meio celamim de cada.

Mas com esse experimento fiz"-me senhor desse ofício e aprendi com
precisão quando se dava a estação própria da semeadura, de forma que eu
podia esperar por dois tempos de plantio e duas colheitas todo ano.

Enquanto esses grãos cresciam eu fiz uma pequena descoberta que me foi
muito útil posteriormente: tão logo se acabaram as chuvas, e o tempo
começou a firmar, o que se deu por perto de novembro, eu fiz uma visita
à minha morada no interior da ilha, onde embora eu não tivesse estado
por alguns meses, encontrei as coisas exatamente como as havia deixado.
O círculo ou dupla sebe que eu construíra não só estava inteiro e firme;
como os esteios que havia cortado de algumas árvores que cresciam na
imediação estavam cheios de brotos, assim como os salgueiros em geral
abrolham em novos ramos depois de um ano da poda de sua copa. Não podia
dizer o nome da árvore de que os cortara. Espantou"-me, assim como me
agradou, ver o crescimento dessas jovens árvores; e eu as podei e as
deixei crescer tanto quanto pude; e é difícil crer quão belas ficaram em
três anos; de guisa que, embora a sebe fizesse um círculo de obra de
vinte e cinco jardas de diâmetro, as árvores, pois assim as podia
chamar, logo o cobriram; e formaram uma sombra completa, suficiente como
abrigo para toda a estação seca.

Isso me levou a cortar mais esteios e a produzir uma sebe como essa no
semicírculo em torno de meu muro (refiro"-me ao muro de minha primeira
casa), o que fiz; e colocando as árvores ou esteios em fileira dupla,
distante obra de oito jardas de minha primeira cerca, elas então
cresceram, e formaram, antes, uma bela cobertura para minha habitação;
servindo, por fim, de proteção, como o relatarei a seu tempo.

Descobria, então, que as estações do ano podiam ser em geral divididas
não em verão e inverno, como na Europa, mas em estações de chuva e seca,
assim dispostas:

\medskip

\noindent\resizebox{\textwidth}{!}{
\begin{tabular}{|c|c|}
\hline
Meados de Fevereiro, março, meados de abril & \begin{tabular}[c]{@{}c@{}}Tempo chuvoso, com o sol\\ no equinócio ou em sua proximidade\end{tabular} \\ \hline
\begin{tabular}[c]{@{}c@{}}Meados de abril, maio, junho,\\ julho, meados de agosto\end{tabular} & \begin{tabular}[c]{@{}c@{}}Tempo de seca, com o sol então\\ a norte da linha do Equador\end{tabular} \\ \hline
\begin{tabular}[c]{@{}c@{}}Meados de agosto, setembro,\\ meados de outubro\end{tabular} & \begin{tabular}[c]{@{}c@{}}Tempo chuvoso, com o sol\\ retornando à primeira posição\end{tabular} \\ \hline
\begin{tabular}[c]{@{}c@{}}Meados de outubro, novembro, dezembro,\\ janeiro, meados de fevereiro\end{tabular} & \begin{tabular}[c]{@{}c@{}}Tempo de seca, com o sol então\\ a sul da linha do Equador\end{tabular} \\ \hline
\end{tabular}
}

\medskip

A estação das chuvas por vezes duravam mais ou menos, na medida em que
os ventos começavam a soprar, mas essa foi a observação geral que fiz.
Depois de ter descoberto por experiência os malefícios consequentes de
ficar exposto à chuva, cuidei para prover"-me antecipadamente, de modo
que não fosse obrigado a sair, e permaneci abrigado tanto quanto
possível durante os meses das chuvas.

Nesse período encontrava muitos afazeres, e muito adequados às
circunstâncias, pois tinha grandes ocasiões para muitas coisas das quais
não tinha provimento senão com trabalho duro e empenho constante; em
particular, tentei muitas vezes fabricar"-me um cesto, mas todos os
gravetos que consegui para tanto provaram"-se tão quebradiços que não
serviam para nada. Provou"-se de excelente ajuda para mim que, quando eu
era um menino, eu costumava sentir grande prazer em parar na oficina do
cesteiro na cidade em que meu pai vivia para vê"-los produzir seus
utensílios de vime; e sendo muito prestimoso, como os meninos geralmente
são, e um grande observador dos modos com os quais eles realizavam essas
coisas, e por vezes servindo de ajudante, obtive por esses meios um
conhecimento completo dos métodos de fazê"-lo, e sem carência de outra
coisa além dos materiais, quando me ocorreu que talvez os galhos daquela
árvore da qual cortei meus esteios podiam ser tão resistentes quanto os
dos salgueiros da Inglaterra, e decidi tentar.

Assim, no dia seguinte fui à minha casa de campo, como a chamava, e
cortando alguns dos galhos menores, encontrei"-os para os meus propósitos
em quantidade que me bastava; de guisa que, quando posteriormente
retornei, fui munido de uma machadinha para cortá"-los em grande número,
pois não tardei a descobrir que existiam em abundância. Estes, eu cuidei
para que secassem dentro de meu círculo ou sebe, e quando eles se
mostraram adequados ao uso, levei"-os a minha caverna; e ali, durante a
estação seguinte, empreguei"-me na fabricação, tanto quanto pude, de
muitos cestos, fosse para transportar terra, fosse para carregar ou
conservar o que fosse, quando tivesse a necessidade; e embora eu não os
tenha realizado com muito esmero, eu os fiz suficientemente úteis para
meus propósitos; assim, posteriormente, cuidei para nunca sair em
excursão sem eles; e à medida que meus cestos de vime envelheceram,
produzi outros, especialmente fortes e profundos, para guardar meus
grãos, em lugar dos sacos, quando vim a tê"-los em quantidade.

Havendo dominado essa dificuldade e dispendido enorme tempo com ela,
ocupei"-me de suprir, se possível, duas carências. Não dispunha de
vasilhas para conservar qualquer líquido, exceto dois barriletes que
estavam quase cheios de rum e algumas garrafas de vidro, algumas de
tamanho comum e outras que eram em formato de caixa, quadradas, para o
mantimento de água, aguardente etc. Não tinha também uma panela em que
pudesse ferver o que fosse, exceto por um caldeirão, o qual havia
recolhido ao navio, e era muito grande para aquilo em que desejava
empregá"-lo, que seja, preparar caldos e ferver pedaços de carne. A
segunda coisa que me empenhei em ter foi um cachimbo de tabaco, mas me
era impossível fabricá"-lo; para isso também encontrei um meio, contudo.

Empreguei"-me no enfiamento da minha segunda fileira de esteios ou
pilares e nesse trabalho com o vime todo o verão, ou estação seca,
quando outra tarefa me tomou mais tempo do que podia imaginar que
dispenderia.

Mencionei antes que eu tinha por plano passar toda a ilha em vista e que
eu subira o ribeirão até o ponto em que construíra minha habitação e até
onde eu encontrara um caminho que confinava no mar do outro lado da
ilha; agora eu estava disposto a atravessar aquela banda da costa;
assim, com minha espingarda, uma machadinha e uma quantidade maior de
pólvora e chumbo do que a habitual, com dois pedaços de bolacha e um
grande punhado de passas em meu bolso, iniciei minha caminhada. Quando
passei pelo vale onde minha casa ficava, como já foi dito, eu apanhei a
vista do mar a oeste, e sendo dia claro, avistei terra sem dificuldades;
se era uma ilha ou continente, não era capaz de dizer; mas chegava a
grande altura, estendendo"-se de oeste a oes"-sudoeste a uma grande
distância; não me parecendo alongar"-se por menos de quinze ou vinte
léguas.

Não era capaz de dizer que parte do mundo seria aquela, senão que sabia
se tratar de parte da América e, como concluí por todas minhas
observações, devia estar próxima aos domínios espanhóis, e talvez
inteiramente povoada dos bárbaros da terra, e se ali tivesse chegado,
haveria encontrado situação pior do que a presente; e portanto assenti
nas disposições da Providência, que eu comecei então a reconhecer e que,
passei eu a crer, dispunha todas as coisas para o melhor; ou seja,
aquietei meus pensamentos com essa ideia e deixei de afligir"-me com os
desejos infrutuosos de permanecer ali.

Ademais, depois de alguma reflexão a respeito, pensei que, se aquela
terra era a costa espanhola, eu certamente veria, cedo ou tarde, algum
navio em carreira, indo ou voltando, de uma maneira ou de outra; mas se
não, então aquela era a costa selvagem entre a possessão espanhola e os
Brasis, onde estão os piores selvagens; pois são antropófagos, ou
comedores de gente, e não hesitam em matar e devorar todos os corpos
humanos que lhes caem em mãos.

Feitas essas considerações, avancei em paz. Descobri que aquele lado da
ilha onde agora estava era muito mais aprazível do que o meu, tendo
encontrado campinas ou savanas agradáveis, adornadas de flores e relva,
e cheia de belas florestas. Vi papagaios em abundância, e tentei prender
um para preservá"-lo e domesticá"-lo e ensiná"-lo a conversar comigo.
Depois de muito esforço, peguei um jovem papagaio, pois o acertei com
uma vara e, tendo"-o recuperado, levei"-o para casa; mas precisei de
alguns anos para ensiná"-lo a falar; contudo, por fim o ensinei a
chamar"-me pelo nome muito naturalmente. Mas o acidente que se seguiu,
embora seja de pouca monta, mostrar"-se"-á bastante deleitoso a seu tempo.

Essa viagem me distraiu muitíssimo. Encontrei nas terras baixas lebres
(assim como as julguei ser) e raposas; mas elas diferiam enormemente de
outras com que havia me deparado, nem fui capaz de persuadir"-me de me
alimentar delas, embora as tivesse matado em grande número. Mas eu não
tinha a necessidade de me arriscar, pois comida não me faltava e de
muito boa qualidade, especialmente dessas três maneiras, a saber,
cabritos, pombos e tartarugas, que acompanhada das uvas, nem mesmo o
mercado de Leadenhall seria capaz de oferecer mesa melhor do que a
minha, guardadas as devidas proporções; e embora meu caso fosse deveras
deplorável, eu tinha grandes razões para agradecer não ter sido levado a
extremos por comida, tendo"-a em vez disso copiosa, e mesmo saborosa.

Durante essa expedição nunca atravessei mais do que duas milhas num dia,
nem algo perto disso; mas eu fazia tantos desvios e retornos para ver as
descobertas que podia fazer que chegava cansado demais ao lugar onde
decidia pernoitar; e então ou descansava em uma árvore ou me cercava de
uma fileira de estacas enfiadas no chão, retiradas de alguma árvore, de
forma que nenhuma criatura selvagem se aproximasse sem me acordar.

Tão logo eu cheguei à praia, causou"-me espanto ver que eu tinha me
estabelecido no pior lado da ilha, pois aqui, de fato, a praia estava
coberta de tartarugas em grande quantidade, enquanto do outro lado eu
havia me deparado com três em um ano e meio. Aqui também havia aves de
muitas castas e em abundância, das quais algumas me eram conhecidas,
outras não, e muitas delas ofereciam carne muito saborosa, mas seus
nomes eu não sabia dizer, exceto daquelas chamadas pinguins.

Eu poderia ter matado tantas quanto eu quisesse, mas estava muito
parcimonioso com a minha pólvora e o meu chumbo, e portanto
interessava"-me mais tombar uma cabra, se pudesse, da qual eu poderia
mais bem me alimentar; e embora houvesse muitos cabritos aqui, mais do
que do meu lado da ilha, era muito mais difícil aproximar"-me deles,
sendo a região plana e uniforme, com eles me vendo muito antes do que
quando eu estava nas colinas.

Admito que esse lado da ilha era muito mais agradável do que o meu, mas
eu não tinha a menor vontade de mudar"-me, pois uma vez que eu havia
fixado residência, esta se fez natural para mim, e todo o tempo parecia
que eu estava ali em viagem e longe de casa. Contudo, viajei pela praia
do mar ao leste suponho que doze milhas, e então fincando um poste
grande na areia à guisa de marco, concluí que poderia retornar para casa
e que a próxima viagem que fizesse seria para o outro lado da ilha, a
leste de minha morada, e assim até que eu chegasse a meu poste
novamente; do que tratarei a seu tempo.

Tomei um caminho diferente para regressar do que aquele em que vim,
pensando que eu poderia facilmente conservar toda a ilha à vista, de
forma que não perdesse de vista minha primeira morada; mas eu me
enganei, pois tendo caminhado duas ou três milhas, eu me vi descendo um
enorme vale, mas tão cercado de colinas, e estas cobertas de mata, que
eu não conseguia ver qual era o meu caminho por qualquer direção senão a
do sol, nem mesmo então, embora eu não pudesse precisar muito bem qual
era a posição do sol àquela hora do dia. Aconteceu, para meu infortúnio,
que uma neblina se fechou por três ou quatro horas do dia enquanto eu
estava no vale, e não sendo capaz de ver o sol, caminhei muito
desconfortavelmente e por fim fui obrigado a retornar à praia, procurar
por meu poste e retornar pelo mesmo caminho em que viera: e então, em
jornadas tranquilas, retornei à minha casa, com o tempo demasiado
quente, e minha espingarda, o chumbo, a machadinha e outras coisas muito
pesadas.

Nessa jornada meu cachorro surpreendeu um jovem cabrito e agarrou"-o; e
eu, correndo para prendê"-lo, assim o fiz e o salvei do cachorro com
vida. Tinha vontade de levá"-lo para casa se eu pudesse; pois eu tinha
muitas vezes pensado se não seria possível pegar uma cabra ou duas e
assim fazer uma criação de bodes e cabras domésticos, que pudessem me
suprir quando minha pólvora e chumbo tivessem se acabado. Fiz uma
coleira para essa criaturazinha e, com um cordão, que havia feito de
algum cabo de verga e sempre trazia comigo, conduzi"-o pelo caminho,
embora com alguma dificuldade, até que cheguei a minha casa de campo e
ali eu o prendi e deixei, pois estava muito impaciente para chegar a
minha casa, da qual estivera ausente por quase um mês.

Não consigo expressar a satisfação que senti de entrar em meu velho
retiro e deitar"-me em minha cama de lona. Essa breve viagem em excursão,
sem qualquer lugar fixo de repouso, me fora tão desagradável que, minha
própria casa, como a chamava comigo mesmo, era um assentamento perfeito
se comparado àquilo; e tornou tudo em meu entorno tão confortável, que
eu resolvi que nunca mais me distanciaria muito dele enquanto fosse meu
destino permanecer na ilha.

Permaneci aqui por uma semana, para descansar e regalar"-me depois de
minha longa viagem; boa parte da qual gastei na pesada tarefa de fazer
uma gaiola para o meu Loro, que começava a ser domesticado e a se
afeiçoar a mim. Então eu comecei a pensar no pobre cabrito que confinara
dentro do círculo de minha casa de campo e decidi sair e trazê"-lo para
casa e dar"-lhe um pouco de comida; assim saí e encontrei"-o onde o
deixei, pois de fato ele nunca poderia ter saído, mas estava quase morto
por falta de comida. Saí e cortei tantos galhos de árvores e arbustos
quanto os consegui encontrar, e os lancei por cima da cerca e, tendo ele
se alimentado, eu o amarrei como antes e o levei embora, mas ele ficou
tão manso por estar com fome que eu não tive necessidade de amarrá"-lo,
pois ele me seguiu como um cachorro; e como eu continuamente o
alimentei, a criatura ficou tão doce, tão gentil e apegada que a partir
de então se tornou um dos meus animais domésticos e nunca mais me deixou
depois disso.

Era chegada a estação chuvosa do equinócio outonal, e eu preservei o 30
de setembro com a mesma solenidade do anterior, sendo o aniversário do
meu desembarque na ilha, passados agora dois anos, e sem maior esperança
de salvamento do que a do primeiro dia, celebrei a data inteira em
humildes e agradecidos reconhecimentos das muitas maravilhosas mercês
com que minha condição solitária foi agraciada, e sem as quais ela
poderia ter sido infinitamente mais desditosa. Ofereci humildes e
sinceros agradecimentos a Deus ter querido desvelar"-me que era possível
que eu pudesse ser mais feliz nessa solitária condição do que eu teria
sido na liberalidade do trato com os homens e com todos os prazeres do
mundo; ao fato de ele ter suprido completamente as deficiências de minha
vida solitária; e a carência de convívio humano, por Sua presença e as
comunicações de Sua graça a minha alma; amparando"-me, apoiando"-me,
confortando"-me e incentivando"-me a confiar em Sua providência aqui, e
ter a esperança de Sua presença eterna no mundo porvir.

Era agora que eu começava sensivelmente a sentir quanto mais felicidade
havia na vida que agora vivia, com todas as suas infelizes
circunstâncias, do que a vida abominável, amaldiçoada e viciosa que
levara em toda a quadra passada de meus dias; e agora eu trocava minhas
tristezas e minhas alegrias; meus próprios desejos eram outros, outros
os objetos de minha preferência, e meus prazeres absolutamente novos em
relação ao que eram em minha chegada, ou mesmo pelos dois anos que
haviam se passado.

Antes, enquanto caminhava, fosse em minhas caçadas ou em exploração da
região, a aflição de minha alma ante minha condição se abatia
subitamente sobre mim e sentia meu coração faltar"-me ao pensar nas
matas, montanhas e desertos em que me encontrava e em como eu era um
prisioneiro, trancado pelas eternas grades e grilhões do oceano em um
agreste desabitado, sem redenção. Em meio a maior calma de meus
pensamentos, isso irrompia como uma tempestade e me fazia brandir os
punhos e chorar como criança; às vezes, era acometido disso em meio às
tarefas, e imediatamente parava e suspirava e mirava o chão por uma ou
duas horas; e isso era ainda pior, pois se eu conseguisse explodir em
lágrimas ou expressar"-me em palavras, essa angústia me deixaria, e a
dor, tendo se exaurido, diminuiria.

Mas agora eu começava a exercitar"-me com novos pensamentos: lia
diariamente a palavra de Deus e aplicava todos os confortos dela a meu
estado presente. Uma manhã, estando muito triste, abri a Bíblia nas
seguintes palavras: ``Não te deixarei, nem te desampararei''.\footnote{Josué
  1:5 {[}\textsc{n.\,e.}{]}} Imediatamente me ocorreu que essas eram palavras
dirigidas a mim: por que mais elas seriam dirigidas de tal maneira,
justamente no momento em que lamentava minha condição, como alguém
abandonado por Deus e o homem? ``Pois bem'', disse eu, ``se Deus não me
abandonou, que tristes consequências podem haver, ou de que importa, se
eu estiver abandonado do mundo inteiro, vendo por outro lado que, se eu
tivesse o mundo inteiro e perdesse a graça e a mercê de Deus, a perda
seria infinitamente maior?''

A partir desse momento, comecei a concluir em meus pensamentos que era
possível que eu fosse mais feliz nessa condição de abandono e solidão do
que era provável que eu jamais seria em qualquer outro estado particular
do mundo; e com esse pensamento eu estava a ponto de agradecer a Deus
por ter me levado àquele lugar.

Não sei o que foi, mas algo transiu minha mente ante aquele pensamento,
e não ouso colocá"-lo em palavras. ``Como podes ter te tornado um tal
hipócrita'', disse eu, em voz alta, ``a ponto de fingir ser grato por
uma condição com a qual, embora possas tentar te satisfazer, tu deverias
antes rezar do fundo do coração para ser dela libertado?'' E então eu
parei; mas embora eu não pudesse agradecer a Deus por estar ali, eu
sinceramente agradeci a Deus por me ter aberto os olhos, ainda que por
quaisquer aflitivas providências, a reconhecer a condição primeira de
minha vida e a lamentar por meus pecados e arrepender"-me. Eu nunca abri
a Bíblia, ou fechei"-a, sem que minha própria alma dentro de mim
agradecesse a Deus por ordenar meu amigo na Inglaterra, sem qualquer
pedido meu, a embrulhá"-la entre minhas coisas; e por auxiliar"-me a
salvá"-la do afundamento do navio.

Assim, e com essa disposição da mente, iniciei meu terceiro ano; e
embora não dê ao leitor o aborrecimento de relação muito detalhada de
minhas obras nesse ano, à maneira do que fiz no primeiro, em geral é
possível observar que raras vezes estive inativo; tendo dividido
regularmente meu tempo segundo os muitos empreendimentos diários que
tinha diante de mim, tais como, em primeiro lugar, meu dever para com
Deus, e a leitura das Escrituras, às quais sempre reservava algum tempo
três vezes ao dia; em segundo lugar, minhas excursões com minha
espingarda em demanda de comida, o que na maioria das vezes tomava"-me
três horas todas as manhãs, quando não chovia; em terceiro lugar, a
organização, o corte, a preservação e o cozimento do que tivesse caçado
para o meu suprimento; coisas que me tomavam boa parte do dia. Também há
de se considerar que, no meio do dia, com o sol a pino, a violência do
calor era demasiado grande para qualquer trabalho; de forma que à tarde
não tinha mais do que quatro horas para as tarefas, com a exceção de que
às vezes eu trocava as horas de caça e trabalho, e trabalhava pela manhã
e saía com minha espingarda no entardecer.

Para esse curto período disponível às tarefas desejo que se acrescente a
excessiva dificuldade do que obrava; as muitas horas em que, por falta
de ferramentas, de ajuda e destreza; tudo o que fazia esgotava meu
tempo. Por exemplo, passei quarenta e dois dias fazendo uma tábua para
uma longa prateleira que queria em minha caverna; enquanto dois
serralheiros, com suas ferramentas e uma cova de serraria, teriam
cortado seis delas da mesma árvore em meio dia.

Meu caso era que eu precisava cortar uma árvore grande, pois a tábua
precisava ser espaçosa. Levei três dias para cortar a árvore e mais dois
arrancando os galhos e reduzindo"-a a uma tora ou pedaço de madeira.
Mediante indescritível esforço de talhá"-la e apará"-la, reduzi ambos os
lados da tora a lascas até que ela se fez leve o bastante para ser
carregada; em seguida, eu a virei e tornei um de seus lados liso e plano
como uma tábua de ponta a ponta, virando esse mesmo lado para baixo e
cortando o outro lado até que eu reduzi a tábua de obra de três
polegadas em largo, e a alisei de ambos os lados. Qualquer um pode
julgar o trabalho de minhas mãos por uma tal obra; mas esforço e
paciência me fizeram levá"-lo a cabo, e muitas outras coisas. Apenas
observo isso em particular para mostrar a razão de por que tanto de meu
tempo se esvaía em tão pouco trabalho; a saber, que o que podia ser
pouco a ser feito com auxílio e ferramentas, era imensa lide e requeria
prodigioso tempo a se fazer sozinho e à mão.

Mas não obstante isso, com paciência e diligência eu realizava tudo que
minhas circunstâncias tornavam necessário que fizesse, como parecerá
pelo que se segue.

Eram agora os meses de novembro e dezembro, e eu aguardava minha
colheita de cevada e arroz. O chão que eu havia revolvido e adubado não
era grande; pois, como observei, minhas sementes de cada uma das plantas
não superavam a quantidade de meio celamim, havendo eu perdido uma
colheita inteira na estação seca. Mas agora meus grãos prometiam uma boa
safra, quando de repente percebi que corria o risco de perdê"-los todos
novamente para inimigos de variada casta, os quais me era praticamente
impossível manter afastados; tais como, em primeiro lugar, as cabras e
as criaturas selvagens que eu chamava de lebres, que, saboreando a
doçura das lâminas, nelas ficavam noite e dia, tão logo despontavam, e
as comiam tão rente que mal tinham tempo de crescer em talo.

A isso não vi remédio senão criar uma barreira em torno da plantação com
uma sebe; o que eu fiz com muita aplicação e mais, pois requeria
rapidez. Contudo, como minha terra arável era pequena, proporcional à
minha colheita, eu a consegui cercar inteiramente em obra de três
semanas; e abatendo algumas das criaturas durante o dia, colocava meu
cão para guardá"-la à noite, amarrando"-o a uma estaca à entrada, onde ele
ficava e latia a noite inteira; então em pouco tempo os inimigos
abandonaram o lugar, e os grãos cresceram bem e fortes e começaram a
amadurecer rapidamente.

Mas da mesma maneira que a animália me havia arruinado antes, quando
meus grãos não eram mais que brotos, os pássaros passaram a fazer o
mesmo quando eles estavam em espiga; pois, passando eu pelo lugar com o
intuito de ver como cresciam, vi minha seara cercada de aves, de cujas
famílias eu não sei dizer, que estavam à espreita, por assim dizer, como
que à espera de que partisse. Descarreguei nelas imediatamente, pois
sempre tinha minha espingarda comigo. Assim que disparei, ergueu"-se uma
pequena nuvem de aves, as quais não havia visto, de entre a própria
plantação.

Isso afetou"-me profundamente, pois antevi que em poucos dias elas
devorariam todas as minhas esperanças; que eu passaria fome e nunca mais
seria capaz de cultivar grãos; e eu não sabia o que fazer; contudo,
decidi não perder meus grãos, ainda que precisasse vigiá"-los dia e
noite. Em primeiro lugar, fui ao meio deles para ver que estrago já
havia sido feito, e descobri que as aves tinham destruído parte
considerável deles; mas como estavam ainda verde demais para elas, a
perda não foi grande, de guisa que o restante daria uma boa colheita,
caso o pudesse salvar.

Fiquei próximo dela para carregar minha espingarda; e então, me
afastando, pude ver sem dificuldade os salteadores acomodados em todas
as árvores ao meu redor, como apenas esperassem o momento em que tivesse
partido, e o acontecimento provou que assim era; pois enquanto caminhava
para longe, como se estivesse indo embora, tão logo fiquei fora de sua
vista elas desceram uma a uma para dentro da plantação mais uma vez.
Senti"-me tão provocado que não pude esperar com paciência até que outras
viessem, sabendo que cada grão que elas comiam significava, em
consequência, como se pode dizer, um celamim de pão a menos para mim;
mas aproximando"-me da sebe, atirei novamente e matei três delas. Era
isso que eu deseja; e assim feito, peguei"-as e tratei"-as como se tratam
os mais afamados ladrões na Inglaterra, pendurando"-as em correntes, para
o terror das demais. Era impossível imaginar que isso teria o efeito que
de fato teve; pois as aves não apenas não pousaram mais na plantação,
como, em suma, abandonaram toda aquela parte da ilha, e eu nunca mais vi
um pássaro sequer próximo daquele lugar, pelo tempo que meu espantalho
ficou ali pendurado.

Fiquei feliz por isso, tenha certeza, e por volta do fim de dezembro,
que era nossa segunda seara do ano, colhi meus grãos.

Havia em grande falta uma foice ou gadanha para ceifar as plantas, e
tudo o que me restava era fazer uma, tão bem quanto pudesse, a partir de
uma das espadas ou cutelos que conservei entre as armas retiradas do
navio. Contudo, sendo pequena minha plantação, não tive grande
dificuldade de cortá"-la; em suma, eu a colhi a minha maneira, pois não
ceifei senão as espigas, e as transportei em um grande cesto que havia
feito para tanto, e então as debulhei com minhas mãos; e no fim de toda
a minha colheita, eu descobri que do meu meio celamim de sementes eu
produzira praticamente dois alqueires de arroz e quase dois alqueires e
meio de cevada; por assim dizer, pois era apenas uma suposição, não
tendo eu como medi"-los na época.

Foi, contudo, um grande incentivo para mim, e eu antevi que, a seu
tempo, agradaria a Deus suprir"-me de pão; no entanto, encontrei"-me
novamente envolvido em perplexidades, pois não sabia nem como moer e
produzir a fécula da minha cevada, ou mesmo como limpá"-la e separá"-la;
nem, uma vez transformada em fécula, como fabricar dela o pão. Tais
coisas, além do desejo de ter grãos para armazenamento em boa quantidade
e deles conservar um suprimento constante, fizeram com que decidisse
nada saborear desses grãos e guardá"-los todos para a semeadura na
estação seguinte; e no meio tempo empregar todo o meu estudo e horas de
trabalho para lograr esse imenso trabalho de prover"-me de grãos e pão.

Podia"-se dizer, então, que eu trabalhava para ganhar o meu pão.
Constitui uma pequena maravilha, e creio que poucas pessoas se dedicaram
a pensar tanto a respeito, a estranha variedade de pequenas coisas
necessárias ao provimento, à fabricação, à cura, ao tratamento, ao
preparo e finalização desse item que é o pão.

Eu, que estava reduzido a um completo estado de natureza, assim o
constatei para o meu desânimo diário; e sempre me via mais e mais ciente
disso, mesmo depois de ter conseguido o primeiro punhado de grãos de
cevada, que, como já foi dito, surgiu inesperadamente e para o meu
espanto.

Em primeiro lugar, eu não tinha arado para revolver a terra, nem pá ou
enxada para cavá"-la. Pois bem, isso eu consegui ao fabricar"-me uma pá de
madeira, como dito anteriormente; mas isso fez meu trabalho um exercício
muito cansativo; e embora me tenha custado muitos dias fazê"-la, por
falta de um calço de ferro ela não apenas se gastou como tornou a lide
ainda mais penosa e sua realização ainda pior.

Isso eu suportei, contudo, e bastou"-me completar a tarefa com paciência
e aceitar o resultado ruim. Quando os grãos foram semeados, eu não tinha
rastelo, necessitando fazê"-lo eu mesmo, arrastando um enorme e pesado
galho de árvore para arranhar a terra, como se poderia dizer, em vez de
ará"-la ou abrir"-lhe sulcos.

Com o crescimento da plantação, já tratei das tantas coisas que tinha em
falta para cercá"-la, protegê"-la, ceifá"-la ou colhê"-la, secá"-la e
transportá"-la para casa, joeirá"-la e guardá"-la; depois, careci de um
moinho para moê"-la, de peneiras para peneirá"-la, de fermento e sal para
transformá"-la em pão e um forno para assá"-lo e ainda assim todas essas
coisas eu fiz sem instrumentos, como há de se observar; e os grãos eram
um inestimável conforto e vantagem para mim também. Tudo isso, como eu
disse, tornava tudo laborioso e muito aborrecido para mim, mas não havia
o que fazer quanto a isso; tampouco era meu tempo de todo perdido para
mim, pois tal como eu o havia dividido, certa porção dele era todo dia
destinada a esses trabalhos; e como eu decidi não usar os meus grãos
para a fabricação do pão até que tivesse quantidade maior, eu tive os
seis meses seguintes para aplicar"-me inteiramente ao trabalho e invenção
para prover"-me de instrumentos próprios à execução de todas as operações
necessárias para tornar os grãos, quando os tivesse, adequados ao meu
uso.

Mas primeiro eu devia preparar mais terra, pois eu tinha agora sementes
bastantes para semear mais de um acre de chão. Antes de fazer isso,
tomou"-me o trabalho de pelo menos uma semana a produção de uma pá, que
quando ficou pronta revelou"-se muito sofrível e muito pesada, exigindo
dupla dedicação o trabalho com ela; o qual superei, de qualquer forma,
semeando minhas sementes em dois grandes trechos planos de chão, tão
próximos da minha casa quanto os pude encontrar, segundo meu juízo, e os
cerquei com uma boa cerca, cujos esteios foram todos cortados da madeira
que havia usado antes e que sabia que brotaria, de modo que no espaço de
um ano eu sabia que teria uma cerca viva ou natural, que careceria
apenas de uns poucos reparos. Esse trabalho não me tomou menos de três
meses, porque grande parte daquele tempo era o da estação chuvosa,
quando não conseguia sair de casa.

Dentro de casa, ou seja, quando chovia e não podia sair, eu encontrava
variadas tarefas; lembrando que, enquanto trabalhava, divertia"-me
conversando com o meu papagaio e ensinando"-o a falar; e rapidamente
ensinei"-lhe a compreender o próprio nome e por fim a falá"-lo bem alto,
``\textsc{L o r o}'', que foi a primeira palavra que eu escutei na ilha
saída de qualquer boca que não a minha. É pressuposto que não fosse esse
meu trabalho; era, porém, uma boa ajuda, pois então tinha eu uma imensa
tarefa em mãos; a saber, eu havia dedicado longo estudo e de muitas
maneiras à feitura de vasos de argila, dos quais necessitava muito,
porém não sabia como fazer. Considerando o calor do clima, porém, não
tinha dúvida de que, caso encontrasse argila, poderia fabricar alguns
potes que, secos ao sol, seriam duros e fortes o bastante para suportar
o manuseio e receber qualquer coisa que fosse seca e precisasse assim
permanecer; e uma vez que isso era exigido no preparo dos grãos, da
fécula etc., que era a coisa que estava fazendo, decidi fazer uns tão
grandes quanto pudesse, e adequados simplesmente a serem vasilhames para
conservar o que quer que colocasse dentro dele.

Quedaria eu digno da pena, ou melhor, do riso do leitor, se lhe contasse
de quantas maneiras estranhas empreendi a produção dessa massa; e se lhe
falasse da fealdade, da estranheza e do horror das coisas que moldei; e
de quantas delas se desmancharam de fora para dentro e dentro para fora,
não sendo o barro bastante sólido para suportar o próprio peso; e
quantas se racharam pelo calor excessivamente violento do sol, ao serem
apressadamente expostas; e quantas se desfizeram em pedaços apenas com
sua remoção, antes e depois de secarem; e, em suma, como, depois de
grande dificuldade para encontrar o barro, cavá"-lo, temperá"-lo, trazê"-lo
para casa e moldá"-lo, não fui capaz de fazer mais do que duas coisas de
barro enormemente feias (não posso chamá"-las de vasilhames) em obra de
dois meses de trabalho.

De qualquer modo, enquanto o sol assava esses dois vasilhames, muito
secos e rijos, eu os ergui com muito cuidado e os coloquei em dois
grandes cestos de vime, que tinha fabricado para tanto, de guisa que não
quebrassem; e havendo entre o pote e a cesta um pouco de espaço, eu o
preenchi de palha de arroz e cevada; e com esses dois potes sempre
secos, pensava poder conservar meus grãos secos e talvez a fécula,
quando a cevada estivesse moída.

Embora muitos erros tenha cometido em meu projeto e execução de potes
grandes, outras coisas menores fiz com bom sucesso; tais como potinhos
redondos, pratos planos, ânforas e panelinhas e quaisquer outras coisas
em que minha mão as moldasse; e o calor do sol as assava e enrijecia de
maneira a causar espanto.

Mas tudo isso não respondia a minha finalidade, que era fabricar um
vasilhame de barro que recebesse o que fosse líquido e suportasse o
fogo; pois isso não era eu capaz de fazer. Aconteceu, depois de algum
tempo, fazendo um fogo alto para cozinhar minha comida, que ao tirá"-la
depois de tê"-la feito, encontrei o pedaço quebrado de um de meus potes
de barro no fogo, ardendo duro como pedra e vermelho como uma telha. Foi
uma feliz surpresa vê"-lo, e pensei comigo mesmo que, se ardiam
quebrados, decerto podiam ser feitos de modo a arder inteiros.

Isso me levou a estudar como produzir meu fogo de modo a fazê"-lo assar
potes. Eu não tinha ideia do que era uma fornalha como a que os oleiros
usam, nem sabia como vidrá"-los com chumbo, embora tivesse um pouco de
chumbo que se pudesse usar para esse fim; mas coloquei três panelas
grandes e dois ou três vasilhames uns sobre os outros e os cobri de
lenha, com uma pilha grande de brasa debaixo dela. Alimentei o fogo com
combustível renovado no redor da parte exterior e sobre a parte de cima,
até que vi as panelas no interior quase inteiramente em brasa e observei
que elas não tinham se quebrado de maneira alguma; e quando as vi
totalmente vermelhas, deixei"-as naquele calor por cinco ou seis horas,
até que descobri que uma delas, embora não tivesse rachado, havia
derretido e escorrido; pois a areia que estava misturada ao barro
derretera com a violência do calor, e se teria transformado em vidro se
eu a tivesse ali mantido; então eu diminuí meu fogo gradualmente até que
as panelas começaram a perder a coloração vermelha; e permanecendo
atento a elas a toda a noite, para não permitir que o fogo diminuísse
muito rápido, de manhã eu tinha a meu dispor três panelas muito boas,
não direi bonitas, e dois outros vasilhames de barro, tão assados e
rijos quanto os podia querer; e um deles perfeitamente vidrado com o
escorrimento da areia.

Depois desse experimento, não preciso dizer que não me faltaram mais
utensílios de barro; mas não posso calar que, quanto às suas formas,
deixavam muito a desejar, como se pode imaginar, não tendo eu jeito de
fazê"-las senão como as crianças moldam tortas de barro, ou como uma
mulher que nunca aprendeu a fazer crescer a massa faz tortas.

Nenhuma alegria que tivesse por objeto algo de natureza tão ordinária
comparou"-se à minha, quando descobri que havia feito uma panela de barro
capaz de suportar o fogo; e sequer tive paciência o bastante para
esperar que houvessem esfriado antes de levar um deles ao fogo novamente
com um pouco d'água para ferver uma porção de carne, o que fez
admiravelmente bem; e com um pedaço de carne de cabra fiz um ensopado
muito saboroso, embora eu precisasse de aveia e muitos outros dos
ingredientes exigidos para fazê"-lo tão bem quanto um dia eu o provara.

Minha preocupação seguinte foi a de fazer"-me um pilão de pedra para
esmagar ou bater os grãos da cevada; pois quanto à moenda, não havia
meios de chegar à perfeição do engenho com um par de mãos. Para suprir
essa falta, os recursos faltavam"-me enormemente; pois, dentre todos os
ofícios do mundo, eu era particularmente inábil ao corte de pedra;
tampouco tinha ferramentas para fazê"-lo. Passei muitos dias para
encontrar uma pedra grande de guisa que lhe pudesse abrir um buraco e
torná"-la adequada para um pilão, e não conseguia encontrar nenhuma;
exceto as que estavam na rocha sólida e que não tinha meios de cortar ou
cavar; nem em verdade eram as rochas da ilha de dureza suficiente, mas
frágeis e arenosas, incapazes de suportar o peso de um pilão pesado ou
mesmo de triturar o grão sem misturá"-lo com areia; assim, depois de
muito tempo perdido em demanda de uma pedra, eu desisti; decidindo
procurar por um bloco grande de madeira dura; que eu encontrei, em
verdade, muito mais facilmente; e obtendo um tão grande quanto tinha eu
força para picar"-lhe um buraco, eu o marquei e dei"-lhe forma externa com
o meu machado e minha machadinha; e então, com a ajuda do fogo e de uma
diligência infinita, cavei um espaço vazio nela, como os índios nos
Brasis fazem suas canoas. Depois disso, fiz um enorme e pesado pilão ou
bastão de uma madeira chamada pau"-ferro; e isso eu preparei e reservei
para receber minha colheita seguinte de grãos, que eu decidira triturar
ou moer em fécula para fazer pão.

Minha dificuldade seguinte era fazer uma peneira ou coador para fazer
passar a farinha e separá"-la da casca e do farelo; sem o que eu não via
possibilidade de fazer pão. Essa era a coisa mais difícil até mesmo de
pensar a respeito; pois é claro que eu não tinha nada próximo da coisa
necessária para fazê"-la; eu me refiro a uma tela ou tecido fino para
coar a fécula. E nesse ponto passei muitos meses atravancado; não
sabendo em verdade o que fazer. Não me restava mais algodão do que meros
trapos; eu tinha pelo de cabra, mas não sabia como fiar ou tecer; e
tivesse sabido eu como, não tinha ferramentas para fazê"-lo. O remédio
que encontrei para isso foi que, por fim, eu me lembrei de que tinha,
entre as roupas dos marinheiros que havia conservado do navio, alguns
lenços de musselina ou calicô; e com algumas dessas peças eu fiz três
peneirinhas bastante apropriadas à tarefa; e assim eu trabalhei por
alguns anos; como fiz depois, direi a seu tempo.

A parte de assar era a próxima a ser considerada, e como eu fabricaria o
pão quando viesse a ter os grãos; pois, em primeiro lugar, eu não tinha
fermento. Quanto a essa parte, não havia como suprir a carência, então
eu não me preocupei demasiado com isso. Mas quanto a um forno,
apliquei"-me bastante. Enfim, descobri um experimento para isso também,
que era: fiz uns vasilhames de barro, bastante largos mas não fundos,
aproximadamente de dois pés de diâmetro, e não mais do que um palmo de
profundidade. Eu os assei no fogo, à maneira dos outros, e os deixei à
parte; e quando eu precisava assar, eu fazia fogo alto sobre minha
lareira, que havia coberto com algumas lajotas de minha própria lavra;
as quais não podia, porém, dizer quadradas, como são as lajotas.

Quando a lenha já havia se tornado brasa ou carvão incandescente, eu a
espalhei por essa lareira de modo a cobri"-la inteiramente, e ali deixei
que ficasse até que a lareira estivesse muito quente; e então limpei
toda a brasa, coloquei meus filões, ou filão, e virando a travessa de
barro sobre eles, pus a brasa em volta do lado de fora dela, para manter
e aumentar o fogo; e assim, tão bem quanto o faria o melhor forno do
mundo, eu assei meus pães de cevada, e em pouco tempo me tornei,
ademais, um bom produtor de massas; pois fiz muitos bolos e pudins de
arroz, mas não fiz tortas, nem tinha eu com o que recheá"-las, caso as
fizesse, senão com carne de aves ou cabras.

Não é preciso dizer que todas essas coisas me tomaram boa parte do
terceiro ano de minha vida na ilha; pois há de se observar que, nos
intervalos dessas coisas, tinha minha própria colheita e propriedade a
administrar; pois fiz a colheita de meus grãos nessa estação, e os
carreguei para casa tão bem quanto pude, e os conservei na espiga, em
minhas grandes cestas, até que tive tempo de removê"-los, pois eu não
tinha chão onde debulhá"-los ou instrumento com o qual fazê"-lo.

E agora, de fato, com meu suprimento de grãos aumentando, eu realmente
precisava construir celeiros maiores; eu precisa de um lugar onde os
conservar, pois o aumento dos grãos foi tamanho que eu tinha obra de
vinte alqueires de cevada, e quantidade não dessemelhante de arroz; de
tal forma que eu comecei a usá"-lo à vontade; pois meu pão se esgotara
havia muito tempo; então decidi ver que quantidade me bastaria para um
ano inteiro, e semear apenas uma vez ao ano.

No todo, eu descobri que quarenta acres de cevada e arroz eram muito
mais do que eu podia consumir em um ano; então decidi semear a mesma
quantidade que semeara anteriormente a cada ano, na esperança de que uma
tal quantidade me provesse completamente de pão etc.

Enquanto essas coisas eram feitas, o leitor pode estar certo de que meus
pensamentos voltaram"-se muitas vezes à extensão de terra que eu vira a
partir do outro lado da ilha; e eu não me encontrava sem anseios
secretos de estar naquela praia, imaginando que, ao ver a terra
principal, e as terras desabitadas, eu pudesse encontrar um meio, de uma
forma ou de outra, de me transferir para lá e talvez encontrar
finalmente alguma escapatória.

Mas entrementes eu não levava em consideração os perigos de uma tal
empresa, e a possibilidade quedar em poder dos selvagens; e selvagens
tais que eu talvez tivesse razão de imaginá"-los ainda piores do que os
leões e tigres de África, de guisa que, se eu caísse em suas mãos,
correria risco mil vezes maior de ser morto e de talvez ser devorado;
pois ouvira histórias de que o povo da costa das Caraíbas era formado de
antropófagos, ou comedores de gente, e eu sabia pela latitude que não
estava longe daquela costa; então, supondo que não eram antropófagos,
eles ainda poderiam me matar, como muitos europeus caídos em seu poder
haviam sido, mesmo quando haviam estado em grupos de dez ou vinte; sem
falar que eu era um apenas, e pouco ou nada podia me defender; pois bem,
todas essas coisas, às quais precisava dedicar longa reflexão e que me
ocorreram meus pensamentos somente depois, não me ofereceram apreensão
de início, e minha cabeça revolveu intensamente a ideia de chegar àquela
costa.

Desejei, então, ter comigo meu menino Xuri, e o escaler com a bujarrona,
com o qual viajei mais de mil milhas pela costa de África; mas eram
pensamentos vãos; e então pensei que devia sair para examinar o bote de
nosso navio, que, como eu disse, havia sido levado a longa distância na
praia durante a tempestade quando naufragamos. Ele permanecia
praticamente no mesmo ponto aonde havia terminado a travessia, mas não
muito; e tinha virado, por força das ondas e dos ventos, quase de borco,
contra uma elevação de areia dura na praia, mas sem ter água ao seu
redor.

Se eu tivesse tido mãos para dar"-lhe conserto e à água lançá"-lo, haveria
tido o bote grande serventia, e eu poderia ter retornado aos Brasis com
ele sem maior dificuldade; mas eu poderia ter previsto que não mais
seria capaz de virá"-lo e colocá"-lo sobre seu casco do que de mover a
ilha sob meus pés; apesar disso, fui à mata e cortei alavancas e rolos e
os levei ao bote decidido a tentar o que tivesse condições de realizar;
sugerindo a mim mesmo, que se eu conseguisse tão somente virá"-lo,
poderia reparar as avarias que havia recebido, e ele voltaria a ser um
bote em perfeito estado, e eu poderia seguir com ele ao mar muito
facilmente.

Não poupei esforços nesse trabalho infrutuoso, tendo passado, julgo eu,
três ou quatro semanas em torno dele; e concluindo, por fim, impossível
erguê"-lo com minha pouca força, pus"-me a cavar a areia, para miná"-la de
modo que ele tombasse, dispondo pedaços de madeira para empurrá"-lo e
guiá"-lo na queda.

Mas quando eu fiz isso, não me vi com força bastante para colocá"-lo em
movimento novamente, ou de passar por baixo dele, muito menos movê"-lo
até a água; e no entanto, embora fosse forçado a abandonar as esperanças
do bote, as ânsias que tinha de aventurar"-me na travessia rumo ao
continente cresceram, mais do que diminuíram, à medida que os meios para
tanto me pareceram impossíveis.

Ao longo do tempo, isso me levou a pensar se não era possível
fabricar"-me uma canoa, ou piroga, à guisa das feitas pelos nativos
daqueles climas, mesmo desprovidos de utensílios adequados; ou, como eu
poderia dizer, sem ajuda, a partir do tronco de uma grande árvore
somente; e isso eu não apenas julguei ser possível, como fácil, e
agradaram"-me muitíssimo os pensamentos sobre seu fabrico; e tendo eu
muito mais conveniências para produzi"-la do que quaisquer negros e
índios; mas sem considerar os especiais inconvenientes a que estava
submetido, maiores do que os dos índios; a saber, a falta de braços que
a arrastassem à água quando pronta, sendo essa um obstáculo de superação
muito mais difícil para mim do que todas as consequências que decorriam,
a eles, da falta de ferramentas; pois de que me serviria, depois de ter
escolhido uma árvore de grande porte na mata e com muito esforço a ter
cortado, ser capaz de talhar e alisar a parte exterior do tronco, para
que tivesse a forma apropriada de um barco, e de queimar ou cortar o
lado de dentro para torná"-lo oco, de maneira a fazer dele um barco; de
que me serviria tudo isso se, ao fim e ao cabo, precisasse deixá"-lo onde
o encontrei por não ter força para lançá"-lo à água?

Pode"-se querer saber: como não pesei minimamente minhas circunstâncias
ao fabricar a canoa? Eu devia ter de pronto considerado os meios de
levá"-la ao mar; mas meus pensamentos estavam tão voltados à minha
travessia do mar que eu nunca cheguei a considerar se teria ou não
condições de levá"-la à água: e de fato, dada sua natureza, era mais
fácil para mim levá"-la por quarenta e cinco milhas de mar do que por
quarenta e cinco braças de terra, de onde ela estava, para colocá"-la na
água.

Pus"-me a trabalhar na canoa tão iludido quanto poderia estar um homem em
posse de seus sentidos. Bastava"-me o projeto, desinteressando"-me saber
se teria condições de levá"-lo a cabo; não que não me ocorresse pensar
amiúde na dificuldade de lançar o barco à água; dava fim a meus
pensamentos a esse respeito, porém, com uma estúpida resposta, ``Antes
de tudo, terminemos a canoa; tenho certeza de que encontrarei meio de
levá"-la quando estiver pronta.''

Método de maior equivocação não poderia haver; mas prevaleceu a força de
minha fantasia, e dessa guisa iniciei o trabalho. Cortei um cedro; ao
que muito me perguntei se Salomão havia conhecido um daqueles para a
edificação do Templo de Jerusalém;\footnote{1 Reis 5 {[}\textsc{n.\,e.}{]}} ele
tinha cinco pés e dez polegadas de diâmetro na parte mais próxima ao
chão e quatro pés e onze polegadas de diâmetro na outra ponta de seus
vinte e dois pés; a partir donde ele diminuía e, então, separava"-se em
galhos. Não foi sem infinita determinação que derrubei essa árvore;
tendo passado vinte dias talhando"-a e cortando"-a na base; e mais
quatorze dias obrando no desbaste dos ramos e galhos de sua imensa copa,
que eu cortei e aplainei com machado e machadinha e fadigas inauditas;
depois disso, custou"-me um mês para dar"-lhe forma e proporção de algo à
maneira de um casco de bote, que fosse apto à boa navegação. Tomou"-me
ainda obra de três meses abrir"-lhe o interior e dar"-lhe acabamento de
guisa a fazer dele uma embarcação; o que fiz, em verdade, sem o recurso
ao fogo, valendo"-me tão somente de maceta e cinzel, e a força de ásperos
trabalhos, até que a tornei uma bela piroga, grande o bastante para
carregar vinte e seis homens, e consequentemente grande o bastante para
transportar a mim e toda a minha carga.

Tendo terminado o trabalho, quedei maravilhado com o que dele resultou.
Era de fato muito maior do que qualquer canoa ou piroga feita de uma
árvore que tivesse visto em minha vida. Custara exaustivo labor, tenha
certeza; e tivesse eu a lançado à água, não tenho dúvida de que teria
partido na mais insana e improvável das travessias de que se tem
notícia.

Mas todas as minhas astúcias de levá"-la à água fracassaram; não obstante
a fadiga que me causaram. Ela estava a não mais de cem jardas da água;
mas o primeiro inconveniente era que se encontrava em um aclive em
direção ao ribeirão. Bem, sem deixar"-me dominar pela desesperança,
decidi cavar a superfície do chão de maneira a produzir um declive; o
que encetei e demandou"-me quantidade prodigiosa de trabalho; mas quem se
incomoda com a dificuldade quando tem seu salvamento em vista? Quando
assim o fiz, e a dificuldade foi superada; porém, tudo permaneceu como
estava; pois não tinha mais força para mover a canoa do que tivera
quando precisei tirar o bote do lugar.

Medi, então, a distância de chão e decidi abrir um canal ou doca para
levar a água à canoa, uma vez que não conseguia levar a canoa à água.
Dei início a esse trabalho; mas tão logo comecei e calculei a
profundidade que havia de cavada, e quão largo ele precisava ser e a
quantidade de terra a ser lançada fora, descobri que, pelo número de
mãos de que dispunha, isto é, unicamente as minhas, a tarefa levaria
obra dez ou doze anos até se completar; pois a margem confinava em uma
elevação, de maneira que na ponta mais alta o canal teria aos menos
vinte pés de profundidade; de guisa que, por fim, embora com enorme
relutância, também desisti da minha tentativa.

Doeu"-me bastante a frustração; e agora via, embora demasiado tarde, a
loucura de iniciar um projeto antes de calcular os custos; e antes que
julguemos com justeza nossas próprias forças para ultimá"-lo.

Em meio a esses trabalhos completei meu quarto ano na ilha e conservei
meu aniversário com os mesmos louvores e achando o mesmo alívio de
antes; pois, por estudo constante e diligente aplicação à palavra de
Deus e pelo auxílio de Sua graça, ganhei saberes diversos do que tinha
anteriormente. Eu cultivava ideias distintas das coisas. Eu olhava para
o mundo como coisa distante, com a qual nada tinha a ver e em relação a
qual não alimentava expectativas, nem desejos: em suma, coisa com a qual
nada tinha de fato a ver, nem jamais provavelmente voltaria a ter; de
maneira que o via como talvez o vejamos do além; a saber, como um lugar
em que havia vivido, mas que deixara; e bem podia dizer, como o
patriarca Abraão a Dives, ``está posto um grande abismo entre nós e
vós''.\footnote{Lucas 16: 26 {[}\textsc{n.\,e.}{]}}

Em primeiro lugar, ali eu me via distante de toda a maldade do mundo;
não tinha nem a concupiscência da carne, a concupiscência dos olhos e a
soberba da vida.\footnote{1 João 2: 16} Não tinha nada a cobiçar, pois
tinha tudo de que podia regalar"-me; eu era senhor de toda a herdade; ou,
se aprouvesse, podia intitular"-me rei ou imperador de todas as terras de
que tinha posse. Não havia rivais; eu não tinha concorrente, ninguém com
quem disputasse a soberania ou o poder: poderia ter produzido
carregamentos inteiros de grãos, mas não havia necessidade disso; assim,
deixava que crescessem tanto quanto os pensava bastantes para minha
situação; havia testudos ou tartarugas bastantes; mas, vez por outra,
uma era tudo que me bastava para meu proveito; eu tinha madeira capaz de
erguer uma frota de navios; e tinha uva o bastante para carregar essa
mesma frota inteira do vinho ou das passas que produzisse.

Mas o valor residia apenas no que pudesse utilizar; e bastante eu tinha
de comer e suprir minhas necessidades; assim, o que era o restante para
mim? Se eu matasse mais carne do que pudesse comer, poderia comê"-la o
cão, ou os vermes; se eu semeasse mais cevada do que pudesse comer, ela
se estragaria; as árvores que eu cortava ficavam no chão para apodrecer,
pois não podia fazer outro uso delas senão para combustível; e deste não
havia eu necessidade senão para o cozimento da minha comida.

Em suma, a natureza e a experiência das coisas diziam"-me, em justa
reflexão, que todas as boas coisas deste mundo são boas para nós na
medida de sua utilidade; e que, o que quer que acumulemos para dar aos
outros, nós desfrutamos tanto quanto podemos usar, e não mais. O mais
cobiçoso e avarento dos miseráveis deste mundo teria se curado do vício
da cobiça se em minha situação estivesse; pois eu possuía infinitamente
mais do que sabia como usar. Não havia espaço para o desejo, exceto de
coisas que não tinha, e elas eram coisas de ponta monta, embora, em
verdade, de grande utilidade para mim. Tinha, como sugeri anteriormente,
um pouco de dinheiro, bem como de ouro ou prata, por perto de trinta e
seis libras esterlinas. Ai! ali jazia aquela coisa inútil e infeliz; não
tinha uso para ela; e muitas vezes pensei comigo mesmo que teria dado um
punhado daquilo por uns cachimbos de tabaco; ou por uma moenda manual
para triturar meus grãos; sim, eu teria dado tudo pelo mais barato
punhado de sementes de nabo e cenoura da Inglaterra, ou por um punhado
de ervilhas e feijões e um frasco de tinta. Tal como eram as
circunstâncias, não tirava qualquer vantagem ou benefício delas; que
ficavam em uma gaveta, e mofavam com a umidade da caverna nas estações
chuvosas; e se eu tivesse uma gaveta cheia de diamantes, a situação
teria sido a mesma; não tendo qualquer valor para mim, pois não teriam
uso.

Eu havia procurado uma situação para minha vida que fosse mais tranquila
em si do que a princípio, e pacífica para meus pensamentos assim como
para o meu corpo. Amiúde eu me sentava para comer com gratidão e
admirava a Providência da mão de Deus, que havia então preparado minha
mesa no deserto. Havia aprendido a atentar mais ao lado bom de minha
condição, e menos ao lado ruim, e a pensar no que me aprazia mais do que
no que me faltava; e isso me trouxe por vezes uns alívios à alma tais
que não sou capaz de expressá"-los; e que registro aqui para que os
descontentes fiquem a par, os quais não são capazes de desfrutar com
satisfação o que Deus lhes deu; pois eles veem e desejam algo que Ele
não lhes deu. Todo nosso descontentamento acerca do que nos falta
parece"-me saltar da falta de gratidão pelo que temos.

Outra reflexão foi de grande utilidade para mim, e sem dúvida seria a
qualquer um que quedasse vítima de infortúnio como o meu; e essa era
comparar minha condição presente com o que esperava que, a princípio,
pudesse ser; ou melhor, com o que certamente poderia ter sido, caso a
Providência de Deus não tivesse maravilhosamente conduzido o navio a um
naufrágio em cerca da costa, onde eu não só pude tocá"-la, como podia
levar o que tirava dele à praia, para meu alívio e conforto; sem o que
teria falta de ferramentas para o trabalho, armas para a defesa e
pólvora e chumbo para a obtenção de meu mantimento.

Passava horas inteiras, posso dizer dias inteiros, representando a mim
mesmo, com as mais vivas cores, que opções teria de ação se não tivesse
buscado provimento no navio; e como minha comida ficaria restrita aos
peixes e tartarugas; e que, tendo se passado um longo tempo até que os
encontrasse, eu teria antes morrido; e que teria sobrevivido, caso não
morresse, como um completo selvagem; que se eu tivesse logrado matar
cabra ou ave, por qualquer motivo, eu não teria guisa de pelá"-los ou
abri"-los, ou separar a carne da pele e das vísceras, ou cortá"-la; e
teria arrancá"-la com meus dentes e puxá"-la com minhas garras, como uma
besta.

Essas reflexões fizeram"-me muito sensível à bondade da Providência para
comigo e muito grato por minha condição presente, com todas as suas
provações e infortúnios; e essa parte eu também não podia senão
recomendar à reflexão daqueles que estão dispostos, em sua miséria, a
dizer, ``Existe aflição maior do que a minha?'' Que eles ponderem sobre
quão piores são as situações de algumas pessoas, ou como piores poderiam
ter sido suas próprias, se a Providência tivesse julgado adequado.

Ocorreu"-me outra reflexão que também trouxe à minha mente auxílio e
esperança; e essa foi comparar minha situação atual com o que havia
merecido e tinha, portanto, motivos para esperar da mão da Providência.
Havia conhecido uma vida terrível, de todo destituída do conhecimento e
do temor de Deus. Fui bem instruído por pai e mãe; em seus primeiros
esforços, eles não me faltaram com o esforço de infundir em meus
pensamentos uma reverência religiosa a Deus e o sentido do dever que me
era exigido, bem como da natureza e finalidade do meu ser. Mas, ai!
tendo conhecido iniciação precoce na vida marítima, que de todas as
vidas é a mais destituída da temência a Deus, embora sempre os
marinheiros se vejam diante do terror; pois bem, ao iniciar"-me cedo na
vida marítima e no convívio com a marujada, todo aquele pouco senso de
religião que trazia comigo foi arrancado de mim aos risos por meus
companheiros de rancho e seu desprezo embrutecido aos perigos; e pelas
visões da morte, que se me tornaram habituais; e pela longa ausência de
toda guisa de oportunidade de conversar senão com quem fosse igual a
mim, ou de ouvir qualquer coisa que fosse boa ou tendesse a tanto.

Tão despejado eu estava de tudo que fosse virtuoso ou da menor ideia do
que eu era ou deveria ser que, nos maiores livramentos com que fui
agraciado, tais como minha fuga de Salé; meu acolhimento pelo capitão
português do navio; meu tão bom estabelecimento nos Brasis; meu
recebimento da carga da Inglaterra, e coisas que tais; nunca as palavras
``Obrigado, Senhor!'' estiveram em meus pensamentos ou em minha boca;
nem acometido da maior dificuldade cheguei a pensar em uma prece ou a
dizer ``Senhor, tende misericórdia de mim!'' ou a mencionar o nome de
Deus, a não ser invocando"-o em vão ou blasfemando.

Terríveis reflexões ocuparam minha mente por muitos meses, como já
observei, a respeito de minha vida pregressa de vício e embrutecimento;
e quando olhava em meu redor e examinava as providências particulares
que me haviam assistido desde minha chegada à ilha e como Deus me havia
sido generoso; tendo não apenas me punido menos do que minha iniquidade
havia merecido, mas tendo tão abundantemente provido; isso me deu
grandes esperanças de que meu arrependimento fora aceito e que Deus
ainda tinha misericórdia de mim.

Com esses pensamentos em mente, cultivei não só uma ideia de resignação
à vontade de Deus na presente disposição de minhas circunstâncias; mas
também de um sincero agradecimento por minha condição; e que eu, ainda
estando vivo, não devia queixar"-me, uma vez que não conhecera a punição
devida a meus pecados; que eu me beneficiava de tantas mercês, pelas
quais não havia razão de esperar naquele lugar, que eu nunca mais devia
lamentar"-me ante minha condição, mas rejubilar"-me e agradecer todos os
dias pelo pão que tinha; que nada, senão uma multidão de maravilhas,
poderia ter proporcionado; que eu deveria até mesmo concluir que não
havia morrido de fome por milagre tão grande quanto o de Elias
alimentado por corvos, ou melhor, por uma série de milagres; e que eu
dificilmente seria capaz de designar lugar mais benéfico nas bandas
desertas do mundo onde pudesse ter naufragado; lugar no qual, embora não
houvesse sociedade, o que por um lado me afligia, não havia deparado
bestas vorazes, lobos ou tigres furiosos que me ameaçassem a vida, por
outro; criaturas venenosas ou peçonhentas das quais acabasse por me
alimentar, para meu padecimento; ou quaisquer selvagens que me
assassinassem e devorassem.

Em suma, se minha vida era, de um lado, uma vida de padecimentos, por
outro era uma vida de misericórdia, e eu não precisava de nada para
transformá"-la em uma vida de conforto senão ser capaz de compreender a
bondade de Deus para comigo; e Seu cuidado para comigo nessa condição
era meu consolo diário; e depois que fiz bons avanços nessas matérias,
segui minha vida e nunca mais me fiquei triste.

Eu já estava aqui havia tanto tempo que muitas coisas que eu trouxera à
costa para meu auxílio ou haviam acabado, ou muitas haviam se estragado
ou estavam a ponto de terminar.

Minha tinta, como já o disse, havia acabado fazia algum tempo, senão por
um restinho que eu havia feito render misturando à água de pouco em
pouco até que ficou tão apagada que mal imprimia qualquer aparência de
preto no papel; tanto quanto durou, fiz uso dela para dar notícia com
minúcia dos dias do mês em que qualquer coisa de monta me acontecia e,
antes, para registrar o tempo que passara; lembro que havia uma estranha
coincidência de dias em várias providências de que fui objeto; e que, se
eu fosse supersticiosamente inclinado a observar os dias como fatais ou
afortunados, eu poderia ter tido razão de examiná"-los com grande
curiosidade.

Primeiro, observara que o dia em que eu afastei de minha família e
amigos e fugi a Hull fora posteriormente o mesmo em que fui capturado
pelo navio de guerra de Salé e feito cativo.

O dia do ano que escapei ao naufrágio daquele navio na enseada de
Yarmouth foi o mesmo em que, posteriormente, fiz minha fuga de Salé no
barco.

O dia do ano em que nasci, isto é, 30 de setembro, foi o mesmo em que,
no meu vigésimo sexto aniversário, tive minha vida milagrosamente salva,
quando fui arrojado à costa naquela ilha, de modo que minha vida de
vícios e minha vida de solidão tiveram início no mesmo dia.

O pão foi o que veio a acabar depois da tinta; ou melhor, a bolacha que
trouxera do navio e que despendi com frugalidade ao limite,
permitindo"-me apenas uma porção por dia por mais de um ano; e no entanto
fiquei praticamente sem pão por quase um ano até que conseguisse grãos
próprios, e enorme razão tinha de ser grato mesmo por tê"-lo tudo, uma
vez que sua obtenção, como atrás o disse, beirou o milagroso.

Minhas roupas também começaram a se gastar; quanto ao linho, não usara
por um bom tempo, exceto por umas camisas xadrez que encontrara nos baús
dos demais marinheiros e que eu conservei com cuidado; pois muitas vezes
eu não conseguia usar mais do que uma camisa; e era de grande ajuda que
tivesse, entre todas as roupas dos marinheiros do navio, quase trinta
camisas. Haviam restado também muitos casacões de vigília grossos, mas
eles eram quentes demais para serem usados; e embora fosse verdade que o
clima era tão quente que não havia necessidade de roupas, eu não podia
permanecer exatamente nu; não, ainda que estivesse inclinado a tanto, o
que não era fato, não podia tolerar um tal pensamento, embora estivesse
sozinho.

A razão de não conseguir ficar nu era que eu não suportava o calor do
sol tão bem dessa maneira quanto vestido; ou melhor, o próprio calor
muitas vezes criava bolhas em minha pele; enquanto, vestido de camisa, o
ar fazia algum movimento e, passando por dentro da camisa, sentia"-me
duas vezes mais fresco do que fora dela. Tampouco podia eu expor"-me ao
calor do sol sem um chapéu ou gorro; o calor do sol, atacando com a
violência que lhe era própria naquele lugar, dava"-me dor de cabeça quase
imediatamente, ao castigar"-me diretamente a cabeça sem que usasse um
desses acessórios, de modo que não o suportava; enquanto, se o vestisse,
ela de pronto acabava.

Diante do que se colocava, comecei a pensar em colocar em ordem aos
poucos os trapos a que chamava de roupas; eu tinha puído e gasto todos
os coletes que tinha, e minha tarefa era agora tentar produzir casacas a
partir daqueles casacões de vigília que tinha comigo e outros materiais
de que dispunha; assim iniciei o trabalho de costura, ou melhor, de
destruição, pois os resultados foram sofríveis. Contudo, logrei fazer
dois ou três novos coletes, os quais serviram"-me por um bom tempo;
quanto aos calções ou ceroulas, só posteriormente cheguei a um bom
resultado.

Mencionei ter conservado as peles de todas as criaturas que matei, isto
é, das de quatro patas, e as ter pendurado, espichadas com pedaços de
pau, ao sol, por meio do qual algumas delas ficaram tão duras e secas
que pouco proveito tirei delas, mas outras se tornaram bastante úteis. A
primeira coisa que fiz delas foi um grande gorro para a minha cabeça,
com o pelo do lado de fora para proteger"-me da chuva, e isso executei
tão bem que depois fiz um conjunto inteiro de roupas a partir dessas
peles; isto é, um colete e calções abertos nos joelhos, e ambos largos,
pois precisavam manter"-me fresco, não aquecido. Não posso omitir em
registro que ficaram terrivelmente malfeitos; pois se eu era um
carpinteiro ruim, saía"-me um alfaiate ainda pior. Contudo, havia neles
logrado tão bem quanto me fora possível, e quando saía, caso chovesse,
com o pelo do meu colete e do gorro voltados para fora, conservava"-me
bastante seco.

Depois disso, gastei bastante tempo e esforço para fazer uma sombrinha;
na verdade, havia esse utensílio em grande falta e tinha muita vontade
de fazer uma; eu as vira fabricadas nos Brasis, onde eram muito úteis
nos terríveis calores de lá. E eu sentia que os calores em nada diferiam
aqui dos de lá, se não fossem maiores, dada a proximidade do equinócio;
ademais, como era obrigado a passar muito tempo ao ar livre, era"-me
coisa muito útil, fosse para as chuvas, fosse para os calores. Não foram
poucas as penas para fabricá"-lo, e levou muito tempo até que eu
conseguisse produzir algo resistente; ou melhor, depois de concluir que
havia encontrado o caminho, penso que estraguei dois ou três antes de
chegar a um a contento; por fim, fiz um que respondeu razoavelmente bem;
a principal dificuldade que encontrei foi fazê"-lo fechar. Conseguia
abri"-lo; mas se não o conseguisse também fechar e trazê"-lo para dentro,
ele não portátil para mim senão em cima de minha cabeça, o que não era o
bastante. De qualquer modo, finalmente, como eu disse, cheguei a um
resultado adequado e o cobri com peles, os pelos voltados para fora, de
guisa que ele repelia a chuva como um toldo e protegia"-me tão bem do sol
que podia andar no clima mais quente com a mesma tranquilidade com que
antes caminhava no mais fresco; e quando ele não era mais necessário,
podia fechá"-lo e carregá"-lo debaixo do braço.

Vivi, assim, muito confortavelmente, com meus pensamentos inteiramente
voltados a resignar"-me ante a vontade de Deus e colocando"-me
inteiramente à mercê de Sua Providência. Isso tornou minha vida melhor
do que a sociável; pois quando começava a lamentar a falta de
conversação, perguntava"-me se conversar a um só tempo com meus próprios
pensamentos e, como espero que possa dizer, com o próprio Deus por
clamores e preces não era melhor do que gozar plenamente do contato
humano no mundo.

Pelos cinco anos que se sucederam, não posso dizer que qualquer coisa
extraordinária se tenha passado comigo; tendo vivido da mesma maneira,
com a mesma postura e no mesmo lugar que antes; e as principais coisas
nas quais eu estava empregado, além do meu trabalho anual de plantar
cevada e arroz e secar minhas passas, mantimentos dos quais conservava o
suficiente para um ano de provisões; quero dizer, além desse trabalho
anual e da caça diária, à qual saía com minha espingarda, houve o
trabalho de fabricar uma canoa, a qual eu finalmente terminei; de guisa
que, cavando para ela um canal de seis pés de vão e quatro pés de
profundidade, eu a levei ao riacho, à distância de quase meia milha.
Quanto à primeira, que era tão grande, por tê"-la feito sem pensar em que
providências tomar, e como a poderia lançar à água; sem ter sido capaz
de levá"-la à água, ou de levar a água a ela, fui obrigado a abandoná"-la
onde estava como lembrança para ensinar"-me a ser mais sábio em outra
oportunidade. Em verdade, na oportunidade seguinte, embora não
conseguisse encontrar árvore apropriada para a empreitada, senão em
lugar onde não conseguia chegar à água sem percorrer distância menor do
que, como já disse, quase meia milha; mesmo assim, por compreender por
fim a viabilidade do projeto, nunca desisti de meu empenho; e, embora
tenha gasto quase dois anos nele, nunca me indispus com o trabalho, na
esperança de finalmente ter um barco para ir ao mar.

No entanto, apesar de minha canoinha estar pronta, o tamanho dela não
respondia de todo ao projeto que tinha em vista ao produzir a primeira;
quero dizer, o de aventurar"-me em travessia à \emph{Terra Firma}, onde
ela estava com suas mais de quarenta milhas em largo; assim, a pequenez
do meu barco ajudou a colocar um fim nesse projeto, e então deixei de
pensar nele. Mas, tendo eu posse de um barco, meu próximo projeto
tornou"-se o de realizar uma viagem em redor da ilha; pois, uma vez que
estivera eu em sítio do outro lado, atravessando"-a, como já o descrevi,
por terra, as descobertas que fiz naquela breve jornada deixaram"-me
muito curioso de ver outras partes da costa; e agora que dispunha de um
barco, não pensava em nada além de navegar em redor da ilha.

Para esse propósito, de guisa que pudesse fazer tudo com discernimento e
cuidado, construí um pequeno mastro para a minha canoa e costurei a ele
uma vela a partir de pedaços de vela do navio que estavam guardados, e
os quais tinha em grande número.

Depois de esquipar meu barco de mastro e vela, e experimentá"-lo,
descobri que ele navegava muito bem; fiz, então, bauzinhos ou caixas,
postos em suas extremidades para conservar provisões, itens necessários,
munições etc., e assim mantê"-los secos, seja da chuva ou das ondas do
mar; e cortei pequeno espaço no fundo do barco, comprido e oco, no qual
eu podia deixar minha espingarda, fazendo uma aba para cobri"-la e,
assim, mantê"-la seca.

Também prendi meu guarda"-chuva na carlinga da popa, como um mastro, para
ficar sobre minha cabeça e manter o calor do sol longe de mim, como um
toldo; e assim, de vez em quando, fazia viagens curtas ao mar, mas nunca
ia muito longe nem distante do riachinho. Por fim, curioso de ver a roda
do meu modesto reino, resolvi fazer meu cruzeiro; e, assim, apercebi meu
navio para a viagem, fretando"-o de duas dúzias de filões, que talvez eu
devesse chamar biscoitos, de pão de cevada, um pote de barro cheio de
arroz cozido e seco, alimento de que me servi bastante, uma garrafinha
de rum, uma porção de carne de cabra, pólvora e chumbo para matar
outras, e dois casacões de vigília, daqueles que, como mencionei antes,
conservara dos baús dos marinheiros; dos quais um servia para forrar o
chão e outro para me cobrir à noite.

Foi no sexto dia de novembro, no sexto ano de meu reinado, ou cativeiro,
caso se prefira, que parti nessa viagem, descobrindo que seria mais
longa do que supunha; porque, embora a ilha em si não fosse muito
grande, quando cheguei à banda leste, encontrei um arrecife que se
estendia por duas léguas no mar, com alguns trechos estando acima da
linha d'água, outros abaixo dela; e além disso, um baixio, que jazia
seco por mais meia légua, de modo que fui obrigado a seguir bom trecho
rumo a mar aberto para dobrar"-lhe a ponta.

Quando os descobri, pensei em desistir de minha empresa e regressar;
pois não sabia quão longe eles me obrigariam a adentrar o mar; e, acima
de tudo, desconhecia maneira de retornaria. Fiz, então, âncora; pois
havia fabricado uma guisa de âncora com o pedaço de um gancho quebrado
que encontrei no navio.

Tendo prendido o barco, peguei minha espingarda e fui à praia, subindo
uma colina, de onde parecia ser possível observar a ponta; e vendo"-a em
toda a sua extensão, decidi aventurar"-me.

Ao avistar o mar daquela colina em que estava, observei uma corrente
forte, em verdade muitíssimo furiosa, que ia a leste e quase tocava a
ponta; e a examinei com mais cuidado, porque reconheci que poderia
correr o risco de, chegando ao extremo, ser tragado por sua força mar
adentro e não conseguir retornar à ilha; e, de fato, se eu não tivesse
subido primeiro essa colina, creio que assim teria sucedido; pois a
mesma corrente passava do outro lado da ilha, porém a maior distância da
costa; e vi que havia uma forte contracorrente na imediação da praia; e,
assim, tinha eu apenas que sair da primeira corrente, e cairia então em
uma contracorrente.

Permaneci ali, de qualquer maneira, dois dias, porque o vento soprava
com força de les"-sudeste, e sendo exatamente contrário à corrente,
produzia a rebentação próximo à ponta; de maneira que não era seguro
para mim seguir acerca demais da costa, por causa da rebentação, nem
distante demais, por causa da correnteza.

No terceiro dia, de manhã, tendo o vento diminuído da noite para o dia,
o mar estava calmo, e eu lancei"-me à empreitada; mas novamente me faço
aviso a todos os pilotos precipitados e imperitos; pois mal havia eu
tocado a ponta, estando a distância da rocha não maior do que meu
próprio bote, encontrei"-me em águas de grande profundidade e em uma
corrente não dessemelhante à eclusa de um moinho, carregando meu barco
consigo com tamanha violência que nada que eu pudesse fazer me livrava
de seu perigo; alongando"-me a grande velocidade da contracorrente, que
ia ficando à minha esquerda. Não havia vento que soprasse em meu
auxílio, e tudo o que eu podia fazer com meus remos quedava inútil; e
então antevi a possibilidade de meu perdimento no mar; pois como a
corrente estava em ambos os lados da ilha, eu sabia que em algumas
léguas seus braços se reuniriam, e então seria irremediavelmente lançado
mar adentro; vendo sequer maneira de evitá"-lo; de forma que o único
destino que tinha diante de mim era o de perecer; não à beira"-mar, pois
essa era uma ideia bastante pacífica, mas de fome. Na verdade, eu havia
encontrado uma tartaruga na praia, tão grande quanto eu era capaz de
erguê"-la, e a jogara para dentro do barco; e eu tinha uma ânfora grande
de água fresca, isto é, um de meus potes de barro; mas o que é tudo isso
quando se está sendo arrastado ao vasto oceano, onde, com certeza, não
havia costa, continente ou ilha, por pelo menos mil léguas?

E agora eu via como era simples para a Providência de Deus converter a
mais desafortunada das condições humanas em coisa ainda \emph{pior}.
Pensava agora em minha ilha solitária e deserta, e ela se me parecia o
lugar mais aprazível do mundo, e toda a felicidade que meu coração era
capaz de desejar era estar lá novamente. Estendi minhas mãos a ela com
ânsias pungentes. ``Ó deserto feliz!'', exclamava eu, ``nunca mais te
verei. Ó miserável criatura! Qual será teu destino?'' Repreendi"-me,
então, por meu temperamento ingrato, e por como havia lamentado minha
solidão; e agora o que eu daria para estar em terra firme ali de novo!
Assim, nunca vemos o verdadeiro estado de nossa condição até que este
nos seja ilustrado por seus contrários; nem sabemos como valorizar
aquilo de que desfrutamos, senão por sua falta. Mal se pode imaginar a
consternação em que me encontrava naquele momento, expulso que era da
minha amada ilha (pois assim ela me parecia) e lançado no vasto oceano,
a quase duas léguas de distância, e em grande desalento de voltar a
recuperá"-la. Entreguei"-me a um imenso esforço, no entanto, e no limite
da exaustão de minhas forças mantive tanto quanto pude a canoa apontada
ao norte, isto é, na direção da banda da correnteza em que a
contracorrente se encontrava; por volta do meio dia, quando o sol
atravessou o meridiano, pensei ter sentido uma brisa ligeira de vento em
meu rosto, chegando de su"-sueste. Isso trouxe algum alento ao meu
coração, e especialmente quando, obra de meia hora depois, bons ventos
sopraram. A essa altura, eu já ia a uma distância assustadora da ilha, e
estivesse o tempo minimamente nublado ou nevoento, teria de outro modo
não logrado; pois eu não tinha agulha de marear a bordo e não
conseguiria jamais ter navegado em direção à ilha, se uma única vez a
tivesse perdido de vista; mas como o tempo continuava aberto,
esforcei"-me para novamente erguer o mastro e esticar a vela, seguindo o
máximo possível na direção norte para sair da correnteza.

Tão logo ergui o mastro e enfunei a vela, o barco começou a se afastar;
e observei, mesmo através da transluzimento da água, que a correnteza
logo sofreria alguma alteração; pois onde a correnteza era forte, a água
se turvava; mas percebendo a água cristalina, notei que ela perdera
força; e assim avistei a leste, a cerca de meia milha de distância, que
o mar rebentava sobre umas rochas, que, como pude perceber, faziam com
que a correnteza se dividisse novamente; e como o veio principal corria
ao sul, deixando as rochas a nordeste, o outro retornava movido de volta
pelas rochas, produzindo uma forte contracorrente que se voltava para
noroeste, com um fluxo muito acentuado.

Aqueles que sabem o que é receber o perdão na escada para o cadafalso,
ou o resgate das mãos de ladrões que os vão matar, ou que souberam
extremos tais, podem imaginar qual foi meu presente espanto de
felicidade e quão alegremente coloquei meu barco no fluxo dessa
contracorrente; e com o vento aumentando, foi com grande satisfação que
lhe estiquei a vela, navegando de vento em popa e com uma maré forte, ou
contracorrente, sob mim.

Essa contracorrente levou"-me de volta obra de uma légua em direção à
ilha, mas aproximadamente duas léguas mais ao norte do que a primeira
corrente me levara a princípio; de forma que, quando já me encontrava
acerca da ilha, deparou"-me a costa norte dela, ou seja, o extremo da
ilha oposto ao do qual partira.

Havia percorrido obra de uma légua, auxiliado por esse fluxo, ou
contracorrente, quando percebi que ele perdia força e não mais me
servia. De qualquer maneira, descobri que entre essas duas grandes
correntes; a saber, a da face sul que me alongara da costa, e a norte,
que ficava obra de uma légua do outro lado; entre as duas, na esteira da
ilha, encontrei águas calmas, que não corriam em direção alguma; e ainda
contando com vento próspero, segui viagem à ilha, embora sem avançar à
larga como antes.

Por volta das quatro horas, à tarde, encontrando"-me então uma légua
distante da ilha, cheguei à ponta do recife, razão de tamanho desastre;
alongando"-se, como o descrevi anteriormente, a sul, e fazendo recuar a
corrente mais ao sul, ele produzia outra contracorrente ao norte, esta
muito forte, porém sem influência sobre a direção que tomava, que era
oeste, mas quase totalmente norte. Com uma brisa forte, no entanto,
segui com a contracorrente, apontando obliquamente a noroeste; e em obra
de uma hora vi"-me a uma milha da praia, de onde, havendo águas calmas,
não tardei a tocar terra firme.

Em terra firme, caí de joelhos e agradeci a Deus por meu salvamento,
deixando de lado todas as ideias que atribuíam minha sobrevivência a
minha canoa; e obtendo refrigério das coisas que levava, trouxe minha
embarcação para perto da praia, a uma pequena cala que vislumbrara entre
árvores, e estando exausto de tanta faina e das fadigas da travessia,
deitei"-me para dormir.

Encontrava"-me agora cercado de muitas dúvidas quanto ao caminho que
tomar com a canoa em meu regresso. Havia corrido muitos perigos e
conhecia muito bem a situação para pensar em seguir na rota que
perfizera para chegar; e o que poderia estar do outro lado (quero dizer,
do lado oeste), eu não sabia, nem tinha qualquer tenção de levar a cabo
novas empreitadas; assim, decidi na manhã seguinte navegar rumo a oeste,
margeando a costa, e investigar se não havia riacho onde pudesse ancorar
minha nau em segurança, de maneira a alcançá"-la novamente, se assim o
quisesse; depois de aproximadamente três milhas costeando a praia, dei
com uma enseada ou baía muito propícia, obra de uma milha adiante, a
qual se estreitava até que confinava em um riachinho ou regato, onde
encontrei porto muito conveniente para a minha canoa, e onde ela
permaneceu como se estivesse em um pequeno cais feito unicamente para
ela. Ali aportei, e depois de deixar meu barco a salvo e seguro, fui à
praia para compreender o que ali havia e ver onde estava.

Logo descobri que não me alongara muito do lugar onde antes havia
estado, quando viajei a pé àquela praia; e nada carregando comigo do que
levara na canoa, senão minha espingarda e meu guarda"-sol, pois fazia
muito calor, iniciei minha marcha. O caminho era bastante confortável,
se o comparava à viagem que enfrentara, e cheguei a minha boa cabana no
entardecer, onde encontrei tudo exatamente como havia deixado; pois eu a
sempre conservava em boa ordem, sendo, como disse antes, minha casa de
campo.

Saltei a sebe e deitei"-me à sombra para descansar o corpo, pois estava
muito cansado, e adormeci; mas julgue"-o você, que lê minha história:
qual não foi meu assombro quando fui tirado de meu sono por uma voz que
várias vezes me chamou pelo nome: ``Robin, Robin, Robin Crusoe: pobre
Robin Crusoe! Onde você está, Robin Crusoe? Por onde você anda? Por onde
você andou?''

Estava eu tão ferrado no sono, exaurido de tanto remar, na primeira
parte do dia, e caminhar, na segunda, que não acordei completamente; mas
cochilando, entre o sono e a vigília, pensei que em sonho alguém falasse
comigo; mas, como a voz continuasse a repetir ``Robin Crusoe, Robin
Crusoe'', comecei por fim a despertar mais inteiramente, e de pronto me
vi muito assustado, e ergui"-me sob grande consternação; mas tão logo
meus olhos se abriram, vi meu Loro pousado sobre a sebe; e imediatamente
entendi que era ele quem falava comigo; pois era naquele tom lamentoso
que eu costumava conversar com ele e ensiná"-lo; e ele havia aprendido
tudo tão perfeitamente que pousava no meu dedo e levava o bico para
perto do meu rosto e exclamava: ``Pobre Robin Crusoe! Por onde você
anda? Por onde você andou? Como você veio parar aqui?'' e outras coisas
que eu lhe havia ensinado.

De qualquer maneira, embora soubesse que era o papagaio e que não podia
ser mais ninguém, precisei de algum tempo até me recompor. Primeiro,
espantou"-me como a criatura havia dado ali; e, ademais, como ali havia
permanecido e em nenhum outro lugar; mas, bastando"-me que não se
tratasse de outro que não o bom Loro, não pensei mais no caso; e
estendendo"-lhe a mão e chamando"-o pelo nome, ``Loro'', a criatura
sociável veio até mim e pousou em meu polegar, como costumava fazer, e
prosseguiu em seu diálogo comigo: ``Pobre Robin Crusoe!'' e ``como eu
vim parar aqui?'' e ``onde eu estava?'', como estivesse felicíssimo de
me ver de novo; e então eu o levei para casa junto comigo.

Eu havia, então, me aventurado pelo mar o bastante para não pensar em
navegar por um bom tempo, e tinha trabalho o suficiente para me aquietar
por muitos dias e refletir sobre o perigo que correra. Eu ficaria muito
feliz de ter minha canoa mais uma vez comigo em minha banda da ilha; mas
eu não conhecia meios plausíveis de fazê"-lo. Quanto à banda leste da
ilha, que contornara, havia compreendido perfeitamente bem que não havia
como se aventurar por ali; sentia meu coração parar, e meu sangue correr
frio, só de pensar no que sucedera; e quanto ao outro lado da ilha, eu
não sabia o que poderia encontrar ali; mas, supondo que a corrente
corresse contra a costa a leste com a mesma força com que passava do
lado oposto, eu correria o mesmo risco de ser levado pela força das
águas na direção da ilha que correra antes, ao ser afastado dela: assim,
com esses pensamentos, resignei"-me em ficar sem barco, embora o fabrico
deste tivesse exigido o trabalho de muitos meses, além de tantos outros
para levá"-lo ao mar.

A governar desse modo meus impulsos, permaneci quase um ano; durante o
qual vivi muito sóbria e recolhidamente, como se pode inferir; e como
meus pensamentos eram muito claros quanto à minha condição e
inteiramente pacificados na resignação às disposições da Providência,
pensava viver uma vida realmente muito feliz em tudo, senão no que se
referia à sociedade.

Nesse período, encontrei aprimoramento em todas as artes mecânicas
exigidas por minhas necessidades; e julgo que, quando exigido,
mostrei"-me bom carpinteiro, especialmente se considerando o número de
ferramentas que possuía.

Além disso, cheguei a uma inesperada perfeição em meu trabalho com louça
de barro e com engenho logrei em realizá"-las com uma roda, o que achei
infinitamente mais fácil e melhor; pois fazia torneadas e bem modeladas
as coisas que antes chegavam a ser ofensivas de se ver. Mas acho que
jamais me envaideci tanto de minha própria arte, nem jamais me alegrei
tanto com qualquer coisa que descobri, quanto ao me ver capaz de
fabricar um cachimbo; e embora se tenha revelado muito feio e
desajeitado, quando ficou pronto, e fosse vermelho à maneira de meus
outros utensílios de barro, uma vez que se mostrava duro e firme e
permitia que eu tragasse a fumaça, ele serviu"-me admiravelmente bem,
pois sempre fora habituado a fumar; e havia cachimbos no navio, mas
dei"-lhes pouca atenção a princípio, não pensando que houvesse tabaco na
ilha; depois, quando vasculhei novamente o navio, não fui capaz de
encontrá"-los.

Também se aperfeiçoaram meus utensílios de vime, e produzi uma
abundância das cestas que me eram necessárias, e tão bem quanto o
engenho me sugeria; e apesar de não serem muito bonitas, ainda assim
eram muito úteis e convenientes para que nelas guardasse minhas coisas
ou com elas levasse coisas para casa. Se eu matasse uma cabra nas
cercanias, por exemplo, podia pendurá"-la em uma árvore, esfolá"-la,
curá"-la, cortá"-la em pedaços e levá"-la para casa em uma cesta; e o mesmo
se aplicava a uma tartaruga, a qual eu podia cortar, retirar os ovos e
um pedaço ou dois da carne, os quais me eram bastantes, e transportá"-los
para casa em uma cesta e deixar o resto para trás. Além disso, cestos
grandes e profundos eram os armazéns da minha cevada, que eu sempre
debulhava tão logo suas espigas secavam e curavam, conservando seus
grãos em outros cestos grandes.

Comecei então a perceber que minha pólvora diminuía consideravelmente;
essa era uma carência que me seria impossível de suprir; e pus"-me a
pensar seriamente no que fazer quando a pólvora me faltasse; isto é, o
que faria para matar qualquer cabra. Como dito em relação ao terceiro
ano de minha permanência na ilha, eu havia conservado uma cabrita e a
amansado, e eu tinha a esperança de encontrar um bode; mas não havia
maneira de fazê"-lo, até que a jovem cabrita tornou"-se uma velha cabra; e
eu jamais tive coragem de matá"-la, até que por fim morreu da idade.

Mas encontrando"-me, então, no décimo primeiro ano de minha residência e,
como já disse, observando a míngua de minha munição, pus"-me a estudar a
arte da caça e domínio das cabras, com o intuito de descobrir se não
conseguia pegá"-las vivas; e, em especial, uma cabra prenha.

Para esse fim, fabriquei armadilhas para prendê"-las; e creio que mais de
uma vez elas se deixaram pegar por tais ardis; mas a estrutura deixava a
desejar, pois eu não dispunha de arame, e sempre as encontrava
destruídas, e as iscas devoradas.

Por fim, decidi experimentar um fojo, e assim cavei no chão buracos
grandes em grande número, em sítios onde havia observado que as cabras
tinham o costume de se alimentar; e sobre esses buracos coloquei
armações de madeira também por mim construídas, com bom peso sobre elas;
e várias vezes deixei ali espigas de cevada e arroz seco sem que a
armadilha estivesse em funcionamento; e pude facilmente perceber que as
cabras haviam ali estado e comido os grãos, pois via as marcas de seus
pés. Por fim, armei três fojos em uma noite e, na manhã seguinte,
encontrei"-os todos inteiros, com as iscas devoradas e desaparecidas; o
que não pouco abateu meu ânimo. Fiz, então, mudanças em minhas
armadilhas; e, para o não incomodar com detalhes, saindo uma manhã para
inspecioná"-las, encontrei em uma delas um grande bode velho; e em outra
delas, três cabritos, um macho e duas fêmeas.

Quanto ao bode velho, eu não sabia o que fazer com ele; sendo ele tão
fero que não tinha coragem de entrar no fosso para apanhá"-lo; isto é,
tentar buscá"-lo com vida, o que era meu intento. Eu o podia ter matado,
mas não era minha vontade, tampouco correspondia ao meu objetivo; assim,
deixei"-o sair, e ele fugiu como se tivesse perdido o siso. Mas eu não
havia dado atenção a algo que depois vim a aprender; isto é, que a fome
é capaz de amansar um leão. Se eu o deixasse ficar ali três ou quatro
dias sem comida e depois levasse a ele água para beber e um pouco de
cevada, ele quedaria manso como um dos filhotes; pois eles são criaturas
muito sagazes e dóceis, quando bem tratadas.

Enfim, deixei"-o partir na ocasião, pouco sabedor que era na época: em
seguida tratei dos três filhotes, e, tomando"-os um a um, amarrei"-os
juntos com barbantes, e com alguma dificuldade levei"-os todos para casa.

Demorou um pouco de tempo até que se alimentassem; mas, servindo"-lhes um
pouco de cevada ainda fresca, eu os seduzi, e eles começaram a serenar;
e então constatei que, se eu quisesse prover"-me de carne de cabra quando
não mais dispusesse de pólvora e chumbo, só o poderia fazer mantendo
criação doméstica desses animais, quando talvez eu as tivesse em torno
de minha casa como um rebanho de ovelhas.

Mas ocorreu"-me, então, que devia manter separados os mansos dos
selvagens, caso contrário eles sempre resultariam bravos quando
crescessem; e a única maneira de fazê"-lo era conservando um pedaço de
chão bem cercado de sebe ou esteio para ali os manter, e de tal forma
que os que dentro estivessem não pudessem sair, e os que estivessem fora
não pudessem entrar.

Esse era um empreendimento imenso para um par de mãos, mas ao perceber
que havia absoluta necessidade em fazê"-lo, minha primeira empreitada foi
descobrir um pedaço de chão adequado, onde pudesse haver pastagens para
que comessem, água para que bebessem e abrigo que do sol os protegesse.

Os conhecedores desses cercados pensarão que dispunha de pouco
discernimento, mas rapidamente assinalei um sítio muito apropriado para
meus caprinos; sendo este um pedaço simples e aberto de campina, ou
savana, como nosso povo chama nas colônias ocidentais, que tinha dois ou
três regatos de água fresca e, em uma das pontas, arvoredos em
abundância. Tais conhecedores sorrirão ante minha previdência, quando
lhes disser que comecei meu cercamento desse pedaço de chão de forma que
minha sebe ou cerca tinha ao menos duas milhas. A loucura disso não
estava tanto na extensão do cercado, pois mesmo que ele tivesse dez
milhas, eu teria tempo de realizá"-lo; a questão é que não havia
considerado que, em curral tamanho, minhas cabras quedariam tão arredias
quanto se tivessem toda a ilha para si, e tão alongado seria o campo de
caça que jamais seria capaz de alcançá"-las.

Minha cerca havia sido iniciada e contava com obra de cinquenta jardas
quando fui acometido desse pensamento; e então parei de imediato e
decidi em princípio cercar terras em extensão de cento e cinquenta
jardas de comprido e cem de largo, as quais, comportando tantas cabras e
bodes quanto os pudesse ter em um espaço imediato de tempo, poderiam ser
aumentadas à medida que crescesse o rebanho.

Eu era prudente em meu proceder, e trabalhava com disposição. Levei obra
de três meses para completar a primeira parte do cercado, e até que a
tivesse completado mantive os três filhotes amarrados no melhor pedaço
de chão e os costumava alimentar tão próximos de mim quanto o possível,
para que se familiarizassem comigo; muitas vezes ia a seu encontro e os
alimentava com espigas de cevada ou punhados de arroz, que lhes oferecia
na palma de minha mão; de modo que, quando concluí o cercado, e eu os
deixei livres, eles me seguiam aonde quer que eu fosse, balindo por um
pouco de grão.

O curral respondeu a sua finalidade, e em obra de um ano e meio eu
reunia um rebanho aproximadamente doze cabras, entre filhotes e outros;
nos dois anos seguintes, cheguei a quarenta e três animais, além de
vários que peguei e matei para comer. Depois disso, acresci outros cinco
pedaços de terra para servir"-lhes de pasto, com chiqueirinhos aonde os
podia levar para prender em número que quisesse e portões que separavam
um lote do outro.

Mas isso não era tudo; pois agora eu não só tinha carne de cabra para
alimentar"-me quando me aprouvesse, como também leite; algo que de início
sequer me ocorreu e que, então, quando me veio à mente, causou"-me
agradável surpresa, pois passei a trabalhar em minha leiteria e
produzia, por vezes, cerca de um galão ou dois de leite por dia. E como
a Natureza, que provê de comida a todas as criaturas, também orienta
naturalmente seu aproveitamento, eu, que nunca ordenhara uma vaca, muito
menos uma cabra, nem testemunhara a produção de manteiga ou queijo,
muito pronta e destramente, apesar das tentativas e erros, fabriquei ao
fim e ao cabo tanto a manteiga quanto o queijo, os quais nunca mais
vieram a me faltar.

Com que misericórdia nosso Criador é capaz de tratar Suas criaturas,
mesmo naquelas condições em que elas parecem alquebradas e condenadas à
destruição. Como Ele pode tornar doce a mais amarga das providências e
nos dar motivos para louvá"-Lo em masmorras e prisões! Que mesa se
preparava para mim naquele deserto, onde de início nada via senão a
iminência da morte pela fome!

Um estoico teria sorrido ao nos ver, a mim e minha pequena família,
sentados para o almoço; havia minha Majestade, Príncipe e Senhor de toda
a ilha; sob meu poder absoluto estavam as vidas de todos os meus
súditos; os quais podia enforcar e arrastar à força de cavalos, assim
como aos quais podia oferecer e subtrair a liberdade; e não havia
rebeldes entre eles.

E então observar como, à maneira de um rei, eu jantava também sozinho,
servido por minha criadagem; Loro, como fosse o meu favorito, era a
única pessoa autorizada a falar comigo. Meu cachorro, já muito
envelhecido e senil, não encontrando companheira com a qual pudesse
multiplicar sua espécie, estava sempre à minha direita; e dois gatos, um
de um lado da mesa, e outro do outro, esperavam de vez em quando um
bocado de minha mão, como sinal de especial mercê.

Esses não eram os dois gatos que eu trouxera inicialmente à costa, os
quais se encontravam mortos e enterrados por mim mesmo perto da minha
habitação; mas, tendo um deles procriado com criatura cuja casta
desconhecia, os dois em questão eram os que eu conservara em mansidão;
enquanto os demais viviam uma vida selvagem na mata e vieram, por fim, a
se me tornar problemáticos; pois eles costumavam entrar em minha casa e
saquear"-me; até que finalmente fui obrigado a matá"-los, o que fiz com
descargas de chumbo, eliminando grande número; e assim eles
desapareceram. Cercado dessa companhia e com abundância eu vivi; não
podia dizer que algo me faltasse senão a sociedade; e desta, algum tempo
depois, eu viria a ter demais.

Inquietava"-me um tanto, como já se disse, o desejo de tornar a usar
minha canoa; embora relutasse demasiado correr novos riscos; e, assim,
por vezes eu ficava a elaborar planos de fazê"-lo atravessar a ilha; já
em outras, resignava"-me em não o reaver. Mas eu sentia um estranho
desassossego de ir àquele ponto da ilha em que, na minha última andança,
como foi dito, subi uma colina para ver como se dispunha a costa e como
eram as correntes, de forma que pudesse ver o que tinha de fazer: essa
inclinação crescia em mim todos os dias, e finalmente decidi viajar até
lá por terra, seguindo a praia. Assim o fiz; mas se alguém na Inglaterra
tivesse conhecido um homem como eu, isso lhes teria causado tanto medo
quanto muito riso; e, como eu frequentemente parava para examinar a mim
mesmo, não podia senão sorrir ante a ideia de viajar por Yorkshire com
instrumentos e vestimentas que tais. Tenho o prazer de oferecer"-lhes um
esboço da minha figura, como se segue.

Eu usava um gorro alto e disforme, feito de pele de cabra, com uma aba
caída na parte de trás, a qual servia para proteger"-me o pescoço do sol
e da chuva, nada sendo tão prejudicial naqueles climas quanto a chuva
molhando a carne sob as roupas.

Eu usava uma casaqueta de pele de cabra, com as fraldas descendo ao meio
das coxas, e um par de calções abertos à altura dos joelhos, os quais
feitos da pele de um bode velho, cujos pelos eram tão compridos de ambos
os lados que, como pantalonas, chegavam à altura de minhas canelas;
meias e sapatos, eu não os tinha, porém fabricara um par de algo que mal
era capaz de nomear, algo como borzeguins folgados em meus tornozelos,
os quais amarrava dos dois lados como polainas, não obstante seu formato
bárbaro, como de resto eram minhas roupas.

Eu tinha um cinto largo de couro de cabra, seco, que prendia com duas
tiras do mesmo material, em lugar de fivelas, e com passadores de ambos
os lados. Em vez de espada e punhal, trazia pendurados, de um lado e de
outro, uma serra e uma machadinha. Eu tinha outro cinto, não tão largo,
e preso da mesma maneira, pendurado no ombro; e em sua parte inferior,
sob o braço esquerdo, pendurava duas bolsas, ambas também fabricadas com
couro de cabra, em uma das quais guardava minha pólvora, e na outra meu
chumbo. Nas costas, levava uma cesta e, no ombro, uma espingarda, e
sobre a cabeça um guarda"-sol, grande, desajeitado e feio, produzido do
mesmo couro, mas que, afinal, era a coisa mais necessária que eu tinha,
além de minha arma. Quanto ao meu rosto, sua cor não era realmente tão
mulata quanto se poderia esperar de um homem desprovido de cuidados que
vivia a nove ou dez graus do equinócio. Minha barba, eu deixara certa
feita crescer a ponto de alcançar dois palmos; mas, tendo tesouras e
lâminas em grande quantidade, cortava"-a bem rente, exceto pelo que
crescia acima do lábio superior, que havia aparado de forma a cultivar
um grande par de suíças maometanas, como eu as havia visto ostentadas
por alguns turcos em Salé; pois os mouros, ao contrário dos turcos, não
as usavam; desses bastos bigodes, ou suíças, não direi que eram longos o
bastante para pendurar meu chapéu em suas voltas, mas eram compridos e
tinham feições monstruosas, e como tais se teriam mostrado assustadores
na Inglaterra.

Mas isso só se diz de passagem; pois, no tocante à minha figura, eram
tão poucos os que havia para me observar que ela não tinha qualquer
interesse; por isso, nada mais tenho a dizer. Nessa maneira de vestes
segui em minha nova jornada, e fiquei fora por cinco ou seis dias.
Viajei primeiro pela praia, diretamente ao local aonde fiz âncora com
minha embarcação para subir os penedos; e, como agora não tinha barco
com que me preocupar, caminhei por terra acompanhando trilha similar
para alcançar a mesma altura a que chegara antes, quando, olhando
adiante à extensa ponta dos arrecifes, os quais fora obrigado a dobrar
com meu barco, como se disse atrás: muito espanto me causou ver o mar
calmo e tranquilo, sem ondulações, sem movimento, sem corrente, não mais
do que em outros lugares.

Aquilo me deixou bastante perplexo, e decidi passar algum tempo
observando o mar, para ver se não era o movimento das marés que
ocasionara aquilo; mas então me convenci disto, a saber, que a maré
vazante de oeste, unida à corrente das águas de um grande rio que ali
desembocava, devia ser a razão dessa corrente; e que, conforme o vento
soprasse com mais força de oeste ou norte, essa corrente se aproximava
ou se afastava da costa; pois, esperando ali até perto do entardecer,
subi de novo a rocha e, depois que a maré vazante se fez, vi claramente
a corrente formar"-se como antes, apenas mais distante, ficando a meia
légua da costa; enquanto, no meu caso, ela se aproximou da costa e
arrastou"-me junto com minha canoa, o que noutro momento não teria
acontecido.

Essa observação convenceu"-me de que eu precisava apenas observar o fluxo
e refluxo da maré, de modo que poderia muito facilmente pilotar mais uma
vez minha canoa pela ilha; mas quando comecei a pensar em colocar essa
observação em prática, causou"-me tamanho terror a lembrança do perigo
que correra que não era capaz de pensar nisso com alguma serenidade;
assim, pelo contrário, tomei outra decisão, a qual era mais segura,
embora mais trabalhosa, isto é, a de construir, ou melhor, fabricar"-me
outra piroga ou canoa, e assim ter duas embarcações, uma para cada lado
da ilha.

O leitor há de compreender que eu dispunha, a esta altura, tal como as
poderia nomear, de duas fazendas na ilha; uma que era minha pequena
fortificação ou tenda, com o muro que a cingia ao pé da rocha e a
caverna na parte de trás, a qual nesse momento eu já havia ampliado em
diferentes apartamentos ou cavidades, uns dentro dos outros. Um deles,
que era o mais seco e o maior, e tinha uma porta que confinava para fora
da minha muralha ou fortificação; ou melhor, para fora a partir do ponto
em que minha muralha se unia à rocha; estava todo cheio das grandes
vasilhas de barro as quais foram objeto de relação anterior, além de
catorze ou quinze grandes cestos, cada qual com capacidade de cinco ou
seis alqueires, nos quais conservava meus mantimentos da terra,
especialmente minha cevada, em parte na espiga já despida da palha, e em
parte debulhada com minhas mãos.

Quanto ao meu muro feito, como dito, de longos esteios ou estacas
compridas, estas cresceram todas como árvores e, à época, já eram tão
grandes e espalhadas que não restava qualquer aparência de habitação por
trás delas.

Próximos dessa minha casa, mas um pouco alongados terra adentro, e em
sítios mais baixos, jaziam meus dois torrões de cevada, que eu
conservava devidamente cultivados e semeados, e que a seu tempo eram
objeto de colheita; e sempre que a ocasião me exigia mais grãos, eu
acrescia a eles mais terras contíguas e igualmente adequadas.

Além desses, eu tinha minha sede no campo, na qual agora também havia
uma plantação já razoável; pois, primeiro, tive meu modesto pavilhão,
como eu o chamava, ao qual reservava cuidados; isto é, tratava da sebe
que o cercava, a qual eu constantemente ajustava a uma boa altura, com a
escada sempre do lado de dentro; das árvores, que de início não eram
outra coisa senão suas estacas, mas agora eram muito firmes e altas; e
assim eu as preservava sempre podadas, para que pudessem se espalhar e
crescer ainda mais espessas e bravas, e produzir uma sombra mais
agradável, ao que, segundo o julgava, se prestavam muito bem. No centro
de tudo isso, eu conservava minha tenda sempre armada, sendo ela um
pedaço de vela que se esticava sobre postes, cravados no chão com esse
propósito e que nunca careciam de conserto ou renovação; e sob ela eu
fabricara meu almadraque ou colchão com as peles das criaturas que
matara e com outras coisas macias, e uma coberta colocada sobre elas, a
qual pertencera a nossas acomodações no navio e eu havia recuperado; e
um casacão de vigília que usava para me cobrir; e aqui, sempre que havia
ocasião para ausentar"-me de minha sede principal, instalava minha
habitação no campo.

Junto a ela, eu tinha currais para meu gado, isto é, para minhas cabras;
e uma vez que eu havia feito um esforço inconcebível para cercar e
cerrar esse terreno; quedava em um tamanho desassossego de verificar sua
inteireza, para que as cabras não fugissem, que nunca partia até que,
com infinito labor, fincasse por toda a parte externa da cerca pequenas
estacas, e estas tão próximas umas das outras, que mais parecia uma
paliçada do que uma sebe, mal restando espaço para atravessar a mão
entre elas; e depois, quando essas estacas cresceram, como todas nas
estações chuvosas que se seguiam, elas tornaram o cercado forte como um
muro, ou melhor, mais forte do que qualquer muro.

Com isso se atesta que eu não conhecia tempo ocioso e que não poupava
esforços para realizar o que parecesse necessário ao meu confortável
sustento; pois eu julgava que a manutenção de uma criação de animais
mansos à minha disposição era o mesmo que conservar um armazém vivo de
carne, leite, manteiga e queijo tanto quanto eu vivesse no lugar, ainda
que por quarenta anos; e que mantê"-los ao meu alcance dependia
inteiramente do aperfeiçoamento de meus cercados de guisa que eu pudesse
ter certeza de que não me faltariam; o que pelo método atrás referido
garanti de maneira tal que, quando essas pequenas estacas começaram a
crescer, eu as havia fincado tão unidas umas às outras que fui obrigado
a arrancar algumas delas.

Naquele sítio também cultivava minhas uvas, das quais dependia
principalmente para minha despensa de passas de inverno, e a cuja
conservação nunca deixei de dispensar grande cuidado, uma vez que eram
as melhores e mais agradáveis delícias de minha ração; e, de fato, não
apenas deliciosas, como medicinais, saudáveis, nutritivas e muito
refrigerantes.

Como elas estavam também a meio do caminho entre minha outra habitação e
o local onde eu havia aportado meu barco, muitas vezes eu ficava e
pernoitava em meu caminho até lá, pois costumava visitar meu barco,
mantendo todas as coisas a ele concernentes e pertencentes em muito boa
ordem. Às vezes, saía com ele à guisa de divertimento, mas não empreendi
novas viagens arriscadas, não me alongando da praia mais do que um ou
dois lanços de pedra, tão apreensivo eu ficava de ser perturbado em meu
reto entendimento por correntes, ventos, ou qualquer outro acidente. Mas
agora chego a uma nova circunstância de minha vida.

Aconteceu que um dia, por volta do meio"-dia, indo de encontro ao meu
barco, causou"-me enorme espanto ver a estampa de um pé descalço na
praia, em cuja areia era muito evidente. Era como se tivesse sido
trespassado por um raio, ou como se estivesse diante de uma aparição.
Fiz"-me todo olhos e ouvidos, mas nada fui capaz de ver ou ouvir; subi a
um terreno elevado para avistar mais longe; caminhei de uma ponta a
outra da praia, mas tudo que havia era uma única marca, nada mais, e
retornei a ela para averiguar se não havia outra ou se talvez não fosse
apenas minha imaginação; mas não havia espaço para dessemelhança, pois
ali estava a imagem de um pé, com dedos, calcanhar e todas as partes de
um pé; como havia chegado ali, eu não sabia, nem podia conjeturar; mas
depois de inúmeros pensamentos erráticos, como um homem absolutamente
perplexo e fora de seu juízo retornei à minha fortificação, não
sentindo, como dizemos, o chão debaixo dos pés, e aterrorizado à última
potência, olhando para trás a cada dois ou três passos, confundindo
arbustos e árvores e reconhecendo em qualquer toco à distância um homem;
não é possível descrever em quantas e variadas formas minha imaginação
apavorada representou"-me coisas, e quantas ideias loucas surgiam a cada
momento em minha fantasia, e que estranhos e incredíveis desatinos me
ocorreram pelo caminho.

Quando cheguei a meu castelo, pois creio que assim foi que passei a
chamá"-lo dali em diante, invadi"-o como se estivesse sob perseguição. Se
adentrei com o uso da escada, que concebera inicialmente para tal uso,
ou se atravessei o buraco na rocha que chamava de porta, não me lembro;
tampouco me recorda a manhã seguinte, pois nenhuma lebre se escondeu ou
raposa se entocou com mais terror nos pensamentos do que eu retornei a
meu retiro.

Não consegui dormir naquela noite; tão mais longe eu estava do motivo do
meu medo, maior era minha apreensão, o que é algo contrário à natureza
de tais coisas, e em especial às usanças de todas as criaturas com medo;
mas me encontrava tão aturdido com minhas próprias ideias assustadoras
da coisa que não imaginava senão aflições, mesmo agora que eu estava a
uma grande distância dali. Havia momentos em que eu pensava ser o diabo,
e a razão se unia a mim nessa suposição; pois como qualquer outra coisa
em forma humana poderia ter chegado ali? Onde estava o navio que os
trouxe? Que marcas havia de quaisquer outros passos? E como era possível
que um homem chegasse ali? Mas que Satanás tivesse se investido da forma
humana naquele lugar, onde não havia qualquer razão para tanto, senão
para deixar ali uma impressão de seu pé, e mesmo isso sem qualquer
propósito, pois ele não podia presumir que eu a encontraria; isso, por
sua vez, também me estarrecia; julgando que o diabo poderia ter
descoberto abundância de maneiras de me amedrontar para além de uma
simples pegada; e que, uma vez que eu vivia do outro lado da ilha, ele
não poderia ser tão parvo a ponto de deixar uma marca em sítio em que
havia uma mínima probabilidade de eu a encontrar, e também na areia,
visto que uma simples onda do mar sob vento forte a teria destruído
completamente. Tudo isso parecia inconsistente com a coisa em si e com
todas as ideias que costumamos fazer da sutileza do diabo.

A cópia de coisas como essa ajudou a dissuadir"-me de todas as apreensões
relacionadas a se tratar do diabo; e eu assim concluí que deveria ser
criatura mais perigosa; a saber, que deviam ser alguns dos bárbaros do
continente que avistava e que haviam errado pelo mar em suas canoas; e
que, fosse por obra das correntes que os arrastaram, fosse por obra de
ventos contrários, haviam dado à ilha e estado em terra, mas partido
novamente para o mar; tão pouco afeitos, talvez, a permanecer naquela
ilha deserta quanto eu de tê"-los aqui.

Enquanto me alongava nessas reflexões em minha mente, sentia"-me muito
grato em meus pensamentos, feliz que estava de não estar nas cercanias
naquele momento ou de não terem visto meu barco, por razão do qual
teriam concluído que o lugar era habitado, e talvez procurado por mim
ilha adentro. Quantos pensamentos terríveis atormentaram minha
imaginação sobre terem descoberto meu barco, e que gente aqui havia; e
que, nesse caso, eu certamente os veria retornar em números ainda
maiores e me devorar; e que, se acontecesse de não me encontrarem, o
mesmo não aconteceria com meu refúgio, e eles destruiriam toda a minha
cevada e levariam todo o meu rebanho de cabras, e eu por fim morreria
faminto.

Assim, meu medo expulsou toda a minha esperança religiosa, toda aquela
confiança anterior em Deus que se baseava em uma experiência maravilhosa
que eu tivera de Sua bondade; como se Aquele que me havia alimentado por
milagre até então não pudesse preservar com Seu poder as provisões que
me havia feito com Sua bondade. Culpava"-me por meu desleixo, de não ter
semeado mais cevada em um ano do que a que me bastaria até a estação
seguinte, como se nenhum acidente pudesse vir a impedir"-me de consumir
os grãos que estavam plantados no chão; e considerei essa tão justa
culpa, que decidi que teria no futuro cevada reservada para dois ou três
anos; de maneira que, acontecesse o que fosse, eu não perecesse por
falta de pão.

Por que estranhos e extremos trabalhos da Providência se cumpre a vida
do homem! Quantas diferentes forças secretas premem nossos afetos à
medida que diferentes circunstâncias se nos apresentam! Hoje amamos o
que amanhã odiamos; hoje buscamos o que amanhã evitamos; hoje desejamos
o que amanhã tememos, e que somente de imaginar nos faz tremer; isso foi
exemplificado em mim, naquele momento, da maneira mais viva que se possa
imaginar; pois eu, cuja única aflição era parecer banido da sociedade
humana, estar sozinho, cercado pelo oceano sem limites, apartado da
humanidade e condenado ao que chamava de vida silenciosa; ser como
alguém que o Céu não julgasse digno de estar entre os vivos ou de
aparecer entre Suas demais criaturas; eu a quem ter visto representante
de minha espécie teria sido o mesmo que me elevar do mundo dos mortos à
vida e a maior bênção que o próprio Céu poderia me conceder, depois da
suprema bênção da Salvação; pois bem, eu tremia agora ante a própria
apreensão de ver um homem, e estava pronto a enterrar"-me no chão diante
da mera sombra ou do silencioso sinal de um homem ter posto um pé na
ilha.

Tal é o estado dessemelhante da vida humana; e isso posteriormente,
quando já um pouco recuperado de meu primeiro espanto, me proporcionou
muitas curiosas especulações. Julguei que aquele era o estado da vida
que a infinitamente sábia e boa Providência de Deus me havia destinado;
que, não sendo capaz de prever quais seriam os propósitos da sabedoria
divina em tudo isso, não deveria contestar Sua soberania; que, na
condição de criatura Sua, tinha Ele um inquestionável direito segundo a
Criação de governar"-me e dispor de mim absolutamente, tal como julgasse
adequado; que, sendo eu criatura que O havia ofendido, também tinha Ele
o direito altíssimo de condenar"-me ao castigo que justo julgasse; e que
meu papel era submeter"-me a Sua indignação, porque eu havia pecado
contra ele.

Eu então refleti: se Deus, que não é apenas justo, mas onipotente, havia
julgado com correção punir"-me e afligir"-me, também estava a Sua mercê
salvar"-me; e dessa forma, se não julgasse Ele adequado, era meu dever
inquestionável resignar"-me absoluta e inteiramente à Sua vontade; assim
como, por outro lado, também ter nele a esperança, orar a Ele, e em
silêncio atender aos ditames e orientações de Sua providência diária.

Esses pensamentos tomaram"-me muitas horas e dias; ou melhor, semanas e
meses: e não posso agora omitir um efeito particular dessas meditações,
a saber, que certa manhã, deitado em minha cama, e pejado de pensamentos
sobre os perigos que me esperavam ante a aparição do selvagem, percebi
que estes me traziam grande desassossego, ao que me vieram aos
pensamentos aquelas palavras do Livro Sagrado: ``Invoca"-me no dia da
angústia; eu te livrarei, e tu me glorificarás''.\footnote{Salmos 50:15}

Diante disso, levantando"-me com alegria da minha cama, não senti apenas
meu coração consolado, como fui conduzido e incentivado a orar com
absoluta sinceridade a Deus por meu salvamento; e quando terminei minha
prece, peguei minha Bíblia e, abrindo"-a para ler, as primeiras palavras
que se me apresentaram foram: ``Espera no Senhor, anima"-te, e ele
fortalecerá o teu coração; espera, pois, no Senhor.''\footnote{Salmos
  27:14} É impossível colocar em palavras o conforto que a passagem me
trouxe. Em resposta, fechei o livro agradecido e não me senti mais
triste, ao menos não por aquele motivo.

Em meio a essas reflexões, apreensões e meditações, vim a considerar um
dia que tudo aquilo podia ser uma simples quimera de minha invenção; e
que o dito pé podia ser o meu próprio, que ali estampara quando em terra
havia desembarcado de minha canoa; o que me animou um pouco também, e eu
passei a persuadir"-me de que tudo se tratava de ilusão; que não era nada
além de meu pé; e por que eu não poderia ter deixado por aquele caminho
minha canoa, do mesmo modo que também por ele estava indo a seu
encontro? Novamente, porém, concluí que não tinha meios de dizer ao
certo por onde pisara e não pisara; e que se, por fim, aquela era
somente a marca de meu próprio pé, fizera eu então o papel dos tolos que
se esforçam em inventar histórias de espectros e aparições e então se
assustam com elas mais do que qualquer um.

Comecei, então, a ganhar coragem e arriscar alguns passos no exterior;
pois não saí do castelo por três dias e noites, de modo que começaram a
faltar"-me os mantimentos, visto que tinha pouco ou nada em seu interior,
senão alguns bolos de cevada e água; e também sabia que minhas cabras
precisavam ser ordenhadas, o que era geralmente um divertimento que
reservava ao entardecer: e as pobres criaturas padeciam de muita dor e
incômodo por falta da mungidura; o que, de fato, quase me levou a perder
algumas delas e a quase lhes secar o leite.

Alentando"-me, portanto, com a crença de que aquilo não era senão a
impressão de um de meus próprios pés, e que eu podia deveras dizer que
me assustara ante minha própria sombra, tornei a sair em minhas
excursões e segui para minha morada no campo para a ordenha do redil, ou
rebanho: mas a ver o pavor com que avançava, a frequência com que olhava
para trás, a ligeireza com que vez por outra abandonava meu cesto e
corria pelo salvamento de minha vida; isso levaria qualquer um a pensar
que algum mau pensamento torturava"-me a consciência, ou que o medo
recentemente me assombrara; o que, em verdade, era o caso.

No entanto, tendo assim caminhado por dois ou três dias sem que visse
qualquer coisa, comecei a ganhar coragem e a pensar que tudo aquilo não
passava de minha própria imaginação: mas não conseguiria convencer"-me
completamente da ideia até que descesse novamente à costa, visse a marca
do pé e com o meu próprio a medisse, de modo a ver se havia semelhança
ou proporção que me certificassem de que era o meu próprio pé: mas,
quando cheguei ao local, primeiramente, pareceu"-me claro que, quando
deixei minha canoa, não poderia ter estado em qualquer parte da praia
acerca dali; em segundo lugar, quando cheguei a medir a pegada com o meu
próprio pé, percebi que meu pé estava longe de ser tão grande; ambas as
coisas encheram minha cabeça de novas imaginações e provocaram"-me os
vapores novamente ao mais elevado grau, de maneira que tremi de frio
como quem tivesse caído em febre; e retornei para casa com a crença de
que um homem ou um bando haviam ali pisado em terra; ou, em suma, que a
ilha era habitada, e talvez fosse surpreendido em algum momento; e o que
fazer para garantir minha segurança, eu não sabia.

Oh, que decisões ridículas os homens tomam quando possuídos pelo medo!
Este priva"-os do uso dos remédios que a razão oferece para seu alívio. A
primeira coisa que pensei em fazer foi derrubar meus cercados e levar
todo o meu gado manso para a mata, para que o inimigo não o encontrasse
e, então, passasse a frequentar a ilha em demanda do mesmo ou
espoliá"-lo; em seguida, ocorreu"-me reabrir os sulcos de meus dois campos
de milho para que não encontrassem grãos ali e desse modo também fossem
levados a frequentar a ilha: e por fim, demolir minha morada e tenda,
para que não deparassem com vestígios de habitação e assim fossem
levados a procurar mais, a fim de descobrir as pessoas que ali viviam.

Esses foram os temas das reflexões da primeira noite depois de estar de
volta a minha casa, quando as apreensões que haviam invadido minha mente
ainda eram recentes e minha cabeça estava repleta de vapores, como dito
atrás: desse modo, o medo do perigo é dez mil vezes mais aterrorizante
do que o próprio perigo quando se manifesta aos olhos; e o fardo da
ansiedade se nos apresenta maior, em grande medida, do que o mal que nos
leva a tanto; e o pior de tudo foi que não encontrei o conforto para
esse problema na resignação que costumava praticar, a qual eu esperava
ter. Era como Saul, pensei, que reclamava não apenas que os filisteus o
perseguiam, mas que Deus o havia abandonado; pois naquele momento não
tomei o devido cuidado para recompor meus pensamentos, rogando a Deus em
minha angústia e entregando a Sua providência, como eu havia feito
antes, minha defesa e salvamento; o que, se eu tivesse feito, ao menos
me teria feito encontrar alívio mais encorajador ante esse novo
contratempo, e talvez o tivesse enfrentado com mais fortaleza.

Esse tumulto em meus pensamentos manteve"-me desperto a noite inteira;
mas de manhã adormeci; e, exaurido, digamos, pelas perplexidades de
minha mente, e esgotado em meu espírito, dormi um sono pesado e acordei
mais serenado do que jamais havia estado. E agora começava a pensar
serenamente; e, empenhando"-me em grande debate comigo mesmo, concluí que
a ilha, a qual era próspera, frutífera e não mais alongada de terra
firme do que eu havia visto, não estava tão esquecida quanto poderia
imaginar; que, embora não houvesse declarados habitantes que vivessem no
local, ainda assim, às vezes, podiam surgir barcos à distância pouca da
costa, os quais, com propósito ou talvez não mais do que movidos por
fortes ventos, podiam tocar o lugar; que eu morava ali havia então
quinze anos e ainda não havia encontrado a menor sombra ou figura de
qualquer pessoa; que, se a qualquer momento fossem levados para lá, era
provável que fossem embora o mais rápido possível, pois nunca haviam
pensado em fixar"-se ali por qualquer razão; que o máximo de que eu podia
derivar qualquer perigo era de qualquer desembarque acidental de pessoas
do continente, que, como era de se pensar, se fossem levadas para lá,
ali estariam contra sua vontade; de guisa que não permaneciam ali,
partindo dali com toda celeridade, raramente ficando uma noite na costa,
exceto no caso de não terem o auxílio das marés e da luz do dia; e que,
portanto, bastava que eu buscasse algum um refúgio seguro, caso algum
bárbaro viesse a pousar no local.

Comecei, então, a padecer de enorme arrependimento por ter cavado minha
caverna larga a ponto de abrir nela uma porta, a qual, como disse atrás,
confinava além do ponto em que minha fortificação se unia à rocha. Ao
ponderar gravemente sobre isso, resolvi construir uma segunda
fortificação, à igual maneira de um semicírculo, à distância de meu muro
precisamente onde havia plantado uma fileira dupla de árvores, obra de
doze anos antes, às quais já fiz menção; árvores essas que haviam sido
plantadas com tão pouca distância entre si que requeriam apenas algumas
estacas colocadas entre si para que resultassem mais grossas e mais
fortes, não tardando para que meu muro logo estivesse terminado.

Agora tinha eu uma dupla muralha; tendo minha parede externa sido
encorpada com pedaços de madeira, cabos velhos e tudo que eu consegui
pensar para torná"-la forte; havendo nela sete buraquinhos, todos com o
diâmetro de um braço meu. À parede interna, eu acrescentei obra de dez
pés de espessura, o que fiz trazendo continuamente terra da minha
caverna e depositando"-a ao pé da parede e caminhando sobre ela; e
através dos sete buracos, logrei ajeitar os mosquetes, sobre os quais
disse atrás que havia resgatado em número de sete de dentro do navio na
costa; dispondo"-os à maneira de canhões, e os prendendo a estruturas que
os sustentavam como as carretas das bocas de fogo, de maneira que eu
pudesse disparar todos os sete no espaço de dois minutos. Esse muro, eu
levei obra de um mês para completar, e não me senti seguro até que o
terminasse.

Quando isso foi feito, cravei por todo o chão do lado de fora de minha
muralha, a uma longa distância de ambos os lados, estacas ou paus de uma
madeira semelhante à do salgueiro, que descobri tão aptas a crescer
quanto firmes quando enfiadas na terra, e em tal número que creio que
talvez as tenha plantado em número de vinte mil, deixando um espaço
tamanho entre eles e minha parede que eu pudesse avistar um inimigo sem
que eles encontrassem abrigo entre as árvores jovens, caso tentassem se
aproximar de meu muro externo.

Assim, em dois anos, tive um bosque cerrado, e em cinco ou seis anos,
uma floresta diante da minha habitação, crescendo tão monstruosamente
grossa e forte que era em verdade perfeitamente intransitável: e homem
nenhum, de qualquer espécie, jamais poderia imaginar que havia algo além
dela, muito menos habitação. Quanto à maneira de entrar e sair que
estabelecera, visto que não havia deixado abertura, coloquei duas
escadas, estando a primeira apoiada em uma reentrância baixa na rocha,
na qual havia espaço para uma segunda; de guisa que, quando eram
retiradas as duas escadas, nenhum homem vivo era capaz de se aproximar
sem fazer mal a si próprio; e, caso conseguisse ultrapassar o obstáculo
da pedra, ainda se veria diante da minha muralha externa.

Assim, tomei todas as medidas que a prudência humana podia sugerir em
nome de minha própria preservação; e ver"-se"-á por fim que elas tinham
justa razão de ser; embora eu nada vislumbrasse naquele momento, nada
mais do que unicamente o medo me sugeria.

Enquanto isso tinha lugar, eu não descuidava de todo meus outros
afazeres; sendo grande a preocupação que tinha com meu pequeno redil de
cabras, o qual não somente me garantia mantimento em qualquer
circunstância, passando a ser bastante sem que necessidade houvesse de
recurso à pólvora e ao chumbo, como também eximia a fadiga da caça às
bravas; e de modo algum desejava perder o benefício que me rendiam e ter
de criá"-las todas novamente.

Com esse propósito, após longa reflexão, consegui pensar em duas
maneiras de conservá"-las: uma era encontrar outro lugar conveniente para
a escavação de uma cova no chão, com o intuito de guardá"-las nesse lugar
todas as noites; e o outro era cercar dois ou três pedacinhos de terra a
boa distância uns dos outros, e ocultos tanto quanto fosse possível, em
cada qual eu pudesse manter não mais que meia dúzia de cabritos; de
forma que, se algum desfortúnio assaltasse o rebanho em geral, eu
pudesse voltar a criá"-los com menor dificuldade e em menor tempo: e,
embora isso exigisse tempo e trabalho tamanhos, julguei que fosse o mais
acertado a fazer.

Assim, passei algum tempo em demanda das partes mais retiradas da ilha;
e cheguei a uma delas, tão escondida quanto meu coração podia ter
desejado: era um pedacinho de terra alagadiça num val da mata espessa,
onde, como o disse atrás, eu quase me perdera enquanto tentava voltar de
minha excursão à parte oriental da ilha. Ali encontrei uma clareira de
três acres aproximadamente, tão cercada de floresta que esta quase
formava um cercado natural; ao menos, não requeria tanto trabalho
realizá"-lo, como outros trechos de terreno em que tanto havia obrado.

Pus"-me imediatamente a trabalhar nesse pedaço de chão; e em menos de um
mês eu o havia cerrado de tal maneira que minha criação, ou rebanho,
como o quiserem chamar, que não se encontrava tão arredia quanto a
princípio deveria estar, viu"-se ali bem protegida. Sem mais demora,
levei dez cabras e dois bodes para esse curral; e, quando eles lá já se
encontravam, continuei a produzir melhorias na cerca até que a tornasse
tão segura quanto a outra; o que, no entanto, fiz com mais
tranquilidade, tomando"-me bem mais tempo.

Todo esse trabalho despendido derivava unicamente de meu medo ante a
pegada do pé de um homem; pois como até aquele momento eu não havia
visto qualquer criatura humana aproximar"-se da ilha, e dois anos haviam
decorrido sob tal apreensão, a qual tornou minha vida muito menos
confortável do que antes fora; como bem pode ser imaginado por qualquer
um que saiba o que é viver à sombra do \emph{medo do homem}; e isso devo
observar, com igual pesar, que a intranquilidade de minha mente também
teve grande efeito sobre a porção religiosa de meus pensamentos; pois o
pavor e o terror de cair nas mãos de selvagens e antropófagos tanto se
abatiam sobre meu espírito que raramente me encontrava com o devido
espírito para dedicar"-me ao meu Criador; ao menos, não com a calma e a
resignação da alma que eu costumava ter; orando a Deus como sob grande
aflição e pressão da mente, cercado de perigo e esperando todas as
noites ser assassinado e devorado antes do amanhecer; e, com o devido
testemunho de minha experiência, digo que uma alma repleta de paz,
gratidão, amor e afeto dispõe de têmpera mais adequada à prece do que
uma cheia de terror e a inquietude: e que, sob o pavor da destruição
iminente, um homem não está mais apto a uma realização reconfortante do
dever de orar a Deus do que ao arrependimento quando enfermo e de cama;
pois esses afligimentos afetam a mente, como outros o corpo; e o
tormento da mente há de necessariamente causar incapacitação tão grande
quanto o do corpo, e até muito maior; pois a prece a Deus é em verdade
um ato da mente, não do corpo.

Sigamos. Depois de proteger parte de meus bens viventes, percorri a ilha
inteira em busca de outro lugar recluso em que pudesse cercar um novo
curral; e assim, caminhando mais em direção à ponta oeste da ilha do que
jamais havia feito, e observando o mar, julguei ter visto um barco a uma
grande distância da praia. Eu havia encontrado uma ou duas lunetas no
baú de um dos marinheiros, o qual havia retirado do navio, mas não as
tinha comigo; e o barco ia tão longe na costa que eu não sabia dizer de
que se tratava; embora o tivesse observado no limite do que meus olhos
alcançavam; se era um barco ou não, eu não sabia ao certo; e quando
desci da colina, não pude ver mais nada, de modo que então desisti;
decidindo apenas não sair mais sem uma das lunetas no bolso.

Quando desci a colina, chegando à beira"-mar em ponto onde, na verdade,
nunca havia estado antes, constatei que encontrar na ilha a pegada do pé
de um homem não era coisa tão inaudita quanto imaginava; e que era
providência especial que eu tivesse sido lançado à banda da ilha que os
bárbaros não frequentavam; pois teria facilmente descoberto que nada era
mais comum às canoas vindas de terra firme, quando ocorria que
estivessem um pouco distantes de sua costa, do que se dirigir àquele
lado da ilha em busca de porto: da mesma forma, como eles amiúde se
encontravam e lutavam em suas canoas, os vencedores, tendo tomado
prisioneiros, levavam"-nos para essa costa, onde, segundo seus horrendos
costumes, sendo todos comedores de gente, eles os matavam e devoravam;
do que tratarei a seguir.

Enquanto eu descia a colina rumo à praia, como disse acima, estando a
sudoeste da ilha, fui acometido de grande perplexidade e estupor; e não
tenho palavras para expressar o horror que senti em minha mente ao ver
espalhada na linha da areia uma abundância de crânios, mãos, pés e
outros ossos de corpos humanos; e observei em particular um lugar onde
se havia acendido o fogo e um círculo fora cavado na terra, como uma
cova, onde se supunha que os desditosos bárbaros sentavam"-se para seus
banquetes bestiais com os corpos de seus semelhantes.

Fiquei tão consternado com a vista dessas coisas que por algum tempo não
fui capaz sequer de aplicar a razão ao perigo que então me ameaçava;
sendo qualquer apreensão de minha parte soterrada por meus pensamentos
ante aquele infernal assombro de brutalidade selvagem, aquele horror
diante da degeneração da natureza humana; do qual, embora já tivesse
tido recorrentes notícias, jamais houvera visão tão próxima; por fim,
desviando o rosto ao formidável espetáculo, senti o estômago enjoar, e
eu estava a ponto de desmaiar quando a natureza se encarregou de
despachar a desordem do meu estômago; e tendo vomitado com violência
incomum, senti"-me um pouco aliviado, mas não aguentei ficar no local por
um instante a mais; tornei então a subir a colina a toda velocidade e
caminhei de volta à minha própria habitação.

Quando me afastara um pouco daquela parte da ilha, parei por algum tempo
como que pasmo; e então, recuperando"-me, voltei os olhos aos céus com
toda a paixão da minha alma e, com rios de lágrimas nos olhos, agradeci
a Deus, que me dera o primeiro quinhão em uma parte do mundo onde eu me
encontrava apartado de criaturas terríveis como aquelas; e que, embora
eu considerasse minha condição presente muito desventurada, ela ainda
havia me proporcionado tantos confortos que ainda tinha mais a agradecer
do que a lamentar; e, acima de tudo, que eu tinha, mesmo nessa condição
desventurada, recebido o consolo do conhecimento Dele e a esperança de
Sua bênção; que era uma alegria mais do que suficientemente equivalente
a todo o infortúnio de que padecera ou podia padecer.

Com esse espírito de gratidão, retornei a meu castelo e comecei a ficar
tranquilo quanto à firmeza de minha condição, e mais do que jamais
estivera; pois constatei que aquele infelizes nunca iam à ilha em busca
do que pudessem obter; talvez não em demanda, carência ou esperança do
que houvesse; e, sem dúvida, tendo frequentemente estado em sua porção
de mata sem encontrar o que os satisfizesse. Eu sabia que estava na ilha
havia quase dezoito anos, e nunca vira o menor sinal de pegada de
criatura humana ali antes; e eu podia ali permanecer outros dezoito anos
tão inteiramente encoberto quanto estava então, desde que não me
revelasse a eles, coisa que não tinha em qualquer razão desejar; sendo
meu único propósito manter"-me totalmente oculto onde estava, a menos que
eu encontrasse melhor casta de criatura do que os bárbaros comedores de
gente para me dar a conhecer.

No entanto, eu sentia tanta repulsa aos desfortunados selvagens de que
tenho falado, e ao miserável e bestial costume de se devorarem e comerem
uns aos outros, que continuei apreensivo e triste, e por dois anos não
me desgarrei de meu círculo; e quando digo meu próprio círculo, quero
dizer com isso minhas três fazendas, a saber, meu castelo, minha sede no
campo, que chamava de morada, e meu curral na mata: nem procurava
floresta adentro outra utilidade que não fosse um espaço para minhas
cabras; pois a repulsa que a natureza me dava a esses desgraçados
infernais era tamanha que eu temia tanto vê"-los quanto temia ver o
próprio diabo; também não saí em demanda da minha barca durante esse
tempo; e comecei a pensar em me fazer outra; pois eu não conseguia
pensar em nenhuma outra tentativa de conduzi"-la em redor da ilha para
perto de mim, temendo encontrar algumas dessas criaturas no mar; nas
mãos das quais, caso assim fosse, eu sabia qual seria meu fim.

O tempo, no entanto, e a satisfação que tive por não correr o risco de
ser descoberto por aquela gente começaram a aliviar a inquietação que
sentia por elas; e voltei a viver da mesma maneira serena de antes,
apenas com essa diferença, que usava de mais cautela e mantinha os olhos
mais atentos em cerca de mim do que antes, temendo o infortúnio de ser
visto por um deles; e, particularmente, tinha mais cautela no uso de
minha espingarda, temendo que algum deles, estando na ilha, pudesse
ouvi"-la; e foi, portanto, uma providência muito boa para mim que eu
tivesse me provido de uma criação de cabras mansas e que não precisava
mais caçar pela mata ou atirar nelas; e se eu prendi alguma depois
disso, foi por ardis e armadilhas, como fizera antes; de modo que,
durante os dois anos seguintes, creio que não disparei minha arma uma
única vez, embora nunca tenha saído sem ela; e, além do mais, como eu
recolhera três pistolas no navio, sempre as carregava comigo, ou pelo
menos duas delas, enfiando"-as no cinto de pele de cabra; também removi a
ferrugem de um dos cutelos que eu tinha encontrado no navio, e me fiz um
cinto para pendurá"-lo também; de modo que me tornei então um sujeito
que, em se olhar, causava medo, quando saía pela ilha, caso se adicione
à descrição anterior de mim mesmo o particular de duas pistolas e um
cutelo largo que pendurei ao flanco com um cinto, porém sem a bainha.

Com as coisas assim decorrendo por algum tempo, parecia, como disse
atrás, excetuando as ditas precauções, que recobrava o meu modo quieto e
pacífico de viver; e todas essas coisas tendiam a mostrar"-me cada vez
mais quão distante estava minha condição de ser desafortunada, em
comparação com outras; ou melhor, com muitos outros tipos da vida que,
pela vontade de Deus, poderiam ter sido minha sorte. Isso me levou a
refletir sobre quão poucos queixumes haveria em meio à humanidade, em
qualquer condição da vida, se as pessoas preferissem comparar sua
condição com outras piores, a fim de agradecer, a aproximá"-la de outras
melhores, em auxílio a seus murmúrios e reclamações.

Como na minha condição atual não havia em verdade muitas coisas de que
carecesse; e observei que o desassossego em que havia vivido por obra
dos desditosos selvagens, e as preocupações concernentes a minha própria
preservação, haviam embotado meu esforço de invenção em benefício de meu
bem"-estar; e eu havia abandonado um bom propósito, sobre o qual já havia
muito ponderado; isto é, o de tentar transformar parte da minha cevada
em malte e fabricar um pouco de cerveja. Essa era realmente uma quimera,
e muitas vezes eu me censurei por sua ingenuidade: pois eu notava que
seria impossível produzir muitas das coisas necessárias à fabricação da
cerveja; como, em primeiro lugar, barris em que a conservassem, os quais
eram coisa que, como foi dito antes, nunca fui capaz de construir,
embora tenha consumido não só muitos dias, como semanas e meses em
tentativas, porém sem qualquer bom sucesso. Em segundo lugar, não tinha
lúpulo para conservá"-la, levedura para fermentá"-la, caldeirão de ferro
para fervê"-la; e, não obstante tudo o que me faltava, creio eu que, não
tivesse sido acometido de tanto, isto é, dos medos e terrores em que os
selvagens me haviam enleado, eu teria tomado para mim a empresa e talvez
o lograsse; pois, uma vez que decidia começar alguma coisa, raramente
desistia sem levá"-la a efeito.

Mas minha invenção voltava"-se a outras coisas; pois, noite e dia, eu não
conseguia pensar em outra coisa, senão em meios de destruir alguns dos
monstros em seu aprazimento cruel e sangrento e, se possível, salvar a
vítima que até lá levassem para assassinar. Para dar relação de todos os
artifícios que projetei, ou melhor, obsessivamente elaborei em meus
pensamentos com o intuito de aniquilar essas criaturas, ou ao menos
assustá"-las, de maneira que não mais viajassem até ali, seria necessário
um tomo maior do que o previsto para esta obra; todos, porém,
infrutuosos, pois nada poderia ser levado a cabo, a menos que eu mesmo
estivesse lá para realizá"-lo: e o que um homem apenas poderia fazer
entre eles, quando talvez houvesse vinte ou trinta deles reunidos com
seus dardos, ou seus arcos e flechas, com os quais poderiam acertar o
alvo tanto quanto eu com minha arma?

Às vezes eu premeditava cavar um buraco debaixo do lugar onde eles
faziam fogo, e depositar ali cinco ou seis libras de pólvora, de guisa
que, quando acendessem o fogo, ali se incendiassem, explodindo também
tudo quanto próximo estivesse: mas havendo, então, meu depósito de
pólvora reduzido a um barril, e não pretendendo desperdiçá"-la com os
bárbaros; assim como não tinha como fazê"-la queimar no momento preciso
que os espantasse, não servindo a mais do que, na melhor das hipóteses,
produzir um estouro aos seus ouvidos e assustá"-los, sem efeito bastante
para que abandonassem o lugar: dessa forma, não dei prosseguimento a
esse plano; cogitei, então, emboscá"-los, escondendo"-me em algum lugar
conveniente com minhas três armas, todas as quais municiadas de dupla
carga; e disparar em meio a sua cerimônia sangrenta, quando os mataria
ou feriria em dois ou três por descarga; e atacando"-os em seguida com
minhas três pistolas e minha espada, não tinha dúvida de que, fossem
eles vinte, eu os mataria todos. Essa fantasia nutriu meus pensamentos
por algumas semanas, e ela tanto me animara que cheguei a sonhar muitas
vezes com ela; e, em outras noites, que apenas disparava contra eles.

Nesse plano, tanto empenhei minha imaginação que cheguei a empregar
vários dias em demanda de sítios em que apropriadamente me ocultasse,
como disse, de forma a observá"-los; e muitas vezes visitei o próprio
local, que então se me havia tornado conhecido; mas enquanto minha mente
se enchia assim do desejo de uma sangrenta vingança, na qual passasse à
espada, como o posso dizer, vinte ou trinta deles, o horror que guardava
naquele lugar e os sinais dos sanguinários bárbaros que se devoravam uns
aos outros, arrefeciam em mim a crueldade.

Encontrei, por fim, um lugar na encosta, o qual julguei oportuno para
nele esperar em segurança, até que avistasse os barcos dos selvagens que
se avizinhassem; e no qual, antes mesmo de chegarem à praia, podia me
encobertar um denso arvoredo, tendo uma de suas árvores um buraco grande
o suficiente para me abrigar inteiramente; e ali me era possível
permanecer e observar todos os seus feitos crueis, e mirar com efeito
suas cabeças, quando muito próximas estivessem umas das outras, de sorte
que fosse quase impossível desperdiçar a munição; ou que, em caso de
erro, deixasse três ou quatro deles feridos ao primeiro tiro.

Nesse lugar, decidi fazer valer meu plano; e, assim, preparei dois
mosquetes e a espingarda com que caçava minhas aves. Os dois mosquetes,
eu carreguei com um par de lingotes cada e quatro ou cinco balas
menores, do tamanho das balas de pistola; e a espingarda que carreguei
com um punhado de chumbo grosso; também carreguei minhas pistolas com
cerca de quatro balas cada; e, nessa condição, bem fornido de munição
para uma segunda e terceira cargas, preparei"-me para a expedição.

Depois de ter assim definido o plano do que almejava e, na minha
imaginação, colocá"-lo em prática, fiz contínuas expedições matinais ao
cimo da colina, o qual se alongava do meu castelo, como eu o chamava,
obra de três milhas ou mais, com a tenção de conseguir avistar algum
barco ao mar, o qual se achegasse à praia ou apontasse próximo em sua
direção; mas logo me aborreci dessa difícil tarefa, da qual, passados
dois ou três meses de vigília constante, jamais retornei com descoberta
qualquer; não tendo havido, durante todo esse tempo, a menor evidência,
não apenas na costa ou perto dela, como em todo o oceano, até onde meus
olhos ou a luneta pudessem alcançar em todas as direções.

Meus planos conservaram sua firmeza por todo o período em que realizei
as excursões diárias à colina para a observação da chegada dos bárbaros;
e por todo o período minha disposição permaneceu apta à furiosa execução
de vinte ou trinta selvagens nus a serem mortos por uma ofensa sobre
cujo respeito eu não havia até então meditado, para além de minhas
paixões terem sido de pronto incendiadas pelo horror que concebi ante o
costume monstruoso da gente daquela terra, que aparentemente havia sido
condenada pela Providência em sua sábia disposição do mundo a não ter
outro guia senão o de suas abomináveis e torpes paixões; e,
consequentemente, foram deixados, e talvez o tivessem sido desde o
início dos tempos, a realizar coisas tão horríveis e sujeitados a
costumes tão hediondos, aos quais nada além da natureza, esta
inteiramente abandonada pelo Céu e movida por alguma degeneração
infernal, os poderia ter feito incorrer. Mas naquele momento, quando,
como eu disse, me encontrei cansado das longas e infrutíferas jornadas
que fizera todas as manhãs, meus pensamentos sobre o que faria começaram
a mudar; e com mais frieza e serenidade passei a me questionar sobre o
que estava acerca de fazer; que autoridade ou vocação tinha eu para
instruir"-me juiz e carrasco desses homens como se fossem criminosos,
homens os quais Deus julgara adequado por tantas eras que permanecessem
impunes e fossem, pode"-se o dizer, os executores de seus julgamentos
entre si; muitas foram as vezes em que me perguntei até que ponto essas
pessoas cometiam um crime diante de mim; que direito tinha eu de me
envolver nas disputas sobre o sangue uns dos outros, que tão
promiscuamente derramavam; e como sei o que Deus mesmo julga nesse caso
específico? É certo que esse costume não configura crime para essa
gente; suas consciências não o reprovam, nem a luz de seu entendimento o
censura; eles não sabem que é uma ofensa a Deus, e então a cometem
desafiando a Justiça Divina, como fazemos em quase todos os pecados que
cometemos. Eles não julgam que matar um cativo oriundo da guerra seja
mais criminoso do que matar um boi; nem que comer carne humana seja
diferente do que para nossa gente é comer carne de carneiro.

Tendo ponderado brevemente sobre essas coisas, seguiu"-se necessariamente
que sobre elas eu estava errado; que essas gentes não eram assassinas,
no sentido que antes as havia condenado em meus pensamentos; não mais do
que os cristãos eram assassinos, os quais frequentemente matam os
prisioneiros feitos em batalha; ou de forma ainda mais contumaz, em
muitas circunstâncias, passam à espada batalhões inteiros de homens sem
oferecer misericórdia, ainda que estes tenham deposto as armas e se
rendido.

Em segundo lugar, ocorreu"-me que, embora o tratamento que dessem uns aos
outros fosse tão brutal e desumano, em verdade ele não era de qualquer
consequência para mim. Que aquelas pessoas não me haviam ferido; e que,
se tentassem me capturar, ou eu achasse necessário atacá"-las em nome de
minha preservação imediata, haveria alguma razão de ser; mas que,
estando eu fora de seu alcance, sem que eles tivessem qualquer noção de
minha existência e, consequentemente, nenhum interesse em mim; assim,
não era justo que eu os atacasse. Que a dita prática justificava a
conduta dos espanhóis em todas as barbaridades que perpetravam na
América, onde haviam destruído aquelas gentes aos milhões; as quais,
embora fossem idólatras e bárbaras, e tivessem vários rituais sangrentos
e bárbaros em seus costumes, como sacrificar corpos humanos a seus
ídolos, no tocante aos espanhóis, eram gentes muito inocentes; e que
eliminá"-las é visto com grande repulsa e horror mesmo pelos próprios
espanhóis nos dias de hoje; e por todas as outras nações cristãs da
Europa como absoluta carnificina, uma crueldade sangrenta e abominável,
injustificável para Deus ou para o homem; e de tal maneira que a própria
alcunha \emph{espanhol} é tida por todas as pessoas de humanidade ou
compaixão cristã com medo e terror; como se o reino da Espanha fosse
particularmente eminente pela produção de uma raça de homens desprovidos
dos princípios da misericórdia, ou da mais profunda piedade para com os
desafortunados, que a todos pertence e é considerada uma marca dos mais
generosos pendores na mente.

Essas considerações me levaram a uma interrupção, e a quase suspensão
completa; e pouco a pouco comecei a desviar"-me do meu plano e a concluir
que havia tomado providências equivocadas em minha decisão de atacar os
selvagens; e que não era de minha conta meter"-me com eles, a menos que
primeiro me atacassem; e que, sim, era da minha conta, se possível,
impedi"-lo; mas que, se eu fosse descoberto e atacado por eles, então
sabia como proceder.

Por outro lado, argumentei comigo mesmo que, em verdade, aquela não era
a maneira de salvar"-me, mas de arruinar"-me e destruir"-me por completo;
pois necessidade havia de que os matasse a todos, não só os que
estivessem em terra naquele momento, mas também os que chegassem a terra
posteriormente, pois se um deles somente escapasse para contar a sua
gente o que lhes havia acometido, eles retornariam aos milhares para
vingar a morte de seus companheiros, e assim eu só teria a mim
proporcionado uma morte certa, o que não tinha eu, então, tenção de
fazer.

No geral, concluí que não deveria, quer por princípio, quer por
estratagema, de guisa alguma meter"-me com aquela gente: que havia
necessidade, sim, de esconder"-me deles por todos os meios possíveis e de
não deixar o menor vestígio que a eles permitisse supor que na ilha
havia ser vivente, ou melhor, de forma humana.

A religião uniu"-se à prudência; e eu me convenci então, de muitas
maneiras, de que feria enormemente minha fé ao inventar todos aqueles
meus planos sangrentos de destruição de criaturas inocentes, isto é, no
que se referia a mim. Quanto aos crimes dos quais eram culpadas, nenhuma
relação eu tinha com eles; eram próprios de sua nação, e devia eu
deixá"-los à justiça de Deus, que é quem governa as nações, e sabe,
mediante punições nacionais, fazer retribuição justa por ofensas
nacionais; e levar o julgamento público àqueles cujas ofensas são
públicas, da maneira que Lhe convenha.

Isso me pareceu tão claro que fiquei bastante aliviado em não ter me
permitido fazer algo que, como então o via, não seria menor pecado do
que o assassinato voluntário, caso o tivesse cometido; e dirigi meus
mais humildes agradecimentos de joelhos a Deus, que assim me livrou da
culpa do sangue; pedindo a Ele que me concedesse a proteção de sua
Providência, para que eu não caísse nas mãos dos bárbaros, e tampouco
que eu lhes pusesse as mãos, a menos que os Céus me enviassem sinais
claros para que o fizesse, em defesa de minha própria vida.

Assim ajuizado permaneci mais ou menos um ano; e tanto afastara o desejo
de uma circunstância em que atacasse aqueles infelizes que durante todo
esse tempo não subi a encosta para avistá"-los, nem para saber se algum
deles na praia se encontrava, de forma a não ser tentado a recobrar
qualquer um de meus planos contra eles, nem levado a atacá"-los por força
de uma boa oportunidade; tendo feito somente a remoção de minha canoa,
que estava do outro lado da ilha, e que conduzi a seu extremo leste,
onde lhe dei fundo em uma pequena enseada de sob algumas rochas altas, e
onde eu sabia que, por causa das correntes, os selvagens não se
aventurariam, ao menos não com seus barcos, por qualquer motivo.

Com minha canoa, trouxe todas as coisas que lá havia deixado e que a ela
pertenciam, embora não fossem necessárias à simples navegação ali; a
saber, um mastro e uma vela que fabricara para ela e algo à semelhança
de uma âncora, mas que, em verdade, não podia ser chamada de âncora ou
fateixa; sendo, no entanto, o melhor que pude produzir para tal efeito.
Todas essas coisas eu removi, de forma que não restasse o menor vestígio
de descoberta, ou nenhuma evidência de barco ou habitação humana na
ilha.

Além disso, mantive"-me, como já se disse, mais recolhido do que nunca, e
raras foram as ocasiões em que deixei minha caverna, senão para o
trabalho de todo dia, a saber, de ordenha de minhas cabras e governo de
meu pequeno redil na mata; que havia levantado noutra parte da ilha e
estava fora de perigo; pois certo era que essa gente selvagem que por
vezes visitava a ilha jamais viera com quaisquer pensamentos de
descobrir alguma coisa; e consequentemente nunca se afastavam da costa;
e não tenho dúvida de que elas possam ter estado várias vezes em terra,
depois de minhas apreensões a seu respeito terem me acautelado tanto
quanto antes; e de fato, pensava retrospectivamente com algum horror
sobre o que teria sido de mim se tivesse com eles me deparado e sido
descoberto antes daquilo, quando desprotegido e desarmado, exceto por
uma espingarda, e esta pouco municiada; caminhando eu por toda a parte,
sempre inquirindo o que pudesse me servir; que espanto teria sido o meu,
se quando eu descobri a pegada de um homem, em vez disso eu tivesse
visto vinte ou trinta selvagens, e todos eles a minha caça, não havendo
eu possibilidade de fuga ante a rapidez da corrida dos bárbaros.

Era grande a tristeza que essas reflexões suscitavam em minha alma, e
trazia aborrecimento à minha mente pensar no que deveria ter feito e em
como eu quedaria impotente ante sua força, e em como me faltaria
presença de espírito para fazer o que pudesse fazer; e muito menos o que
então, depois de tamanhas deliberações e preparações, poderia estar apto
a realizar. De fato, depois de pensar seriamente nessas coisas, via"-me
acometido de forte melancolia, a qual, às vezes, muito se dilatava; mas
ao fim tudo se resolvia na gratidão àquela Providência que me havia
livrado de tantos perigos invisíveis e me afastado de infortúnios nos
quais eu não poderia ter sido agente de meu salvamento; pois não tinha a
menor noção de tais perigos, ou a mínima suposição de sua possibilidade.

Isso renovou uma meditação que muitas vezes havia surgido em meus
pensamentos em tempos anteriores, quando comecei a ver as disposições
misericordiosas de Deus nos perigos que atravessamos nesta vida. Quão
maravilhosamente somos salvos, e nada sabemos; que força se manifesta
quando, em uma dúvida ou hesitação (uma perplexidade, assim chamamos)
quanto a seguir um caminho ou outro, um alumbramento misterioso nos
aponta um rumo, embora tivesse tenção de enveredar por outro: ou melhor,
quando os sentidos, as disposições e talvez os negócios nos movem a
tomar um rumo, e uma estranha impressão sobre a mente, pois
desconhecemos de que fonte emana e por cujo poder se manifesta,
interpõe"-se e força"-nos a outro; de maneira que posteriormente parecerá
que, se tivéssemos ido pelo caminho que a princípio tomaríamos, teríamos
conhecido a ruína e a desolação. A partir dessas e de muitas meditações
a elas semelhantes, estabeleci noutro momento uma regra, esta qual seja,
que sempre que deparasse com os ditos alumbramentos misteriosos, ou
impressões da mente, para fazer ou não fazer qualquer coisa que se me
apresentasse, ou para seguir por um ou outro caminho, nunca deixaria de
obedecer ao que em mistério me fosse dito; embora outro motivo para isso
não conhecesse, além de uma impressão ou sugestão que despertasse em
minha mente. Eu poderia dar muitos exemplos do sucesso dessa conduta no
decorrer de minha vida, e especialmente na parte final da minha
permanência naquela desventurada ilha; além de muitas ocasiões em que
muito provavelmente eu o teria notado, caso eu o tivesse visto então com
os mesmos olhos com que agora via. Mas nunca é tarde para se fazer
sábio; e não posso deixar de aconselhar a todos os homens de juízo,
cujas vidas são apresentadas a notáveis circunstâncias, como as minhas,
ou mesmo que não sejam tão notáveis, a não menosprezar os misteriosos
lampejos da Providência, venham eles da invisível inteligência que seja,
a qual não discutirei, e talvez não possa explicar; mas certamente são
prova do entendimento dos espíritos, e das secretas conversações entre o
encarnado e o desencarnado; e prova que não se pode contestar; da qual
terei ocasião de dar exemplos notáveis no restante de minha residência
solitária naquele lugar de abandono.

Creio que o leitor não julgará estranho se disser que tais inquietações,
os constantes perigos em que vivi e o desassossego que então sentia,
deram fim a toda a invenção e a todos os engenhos que havia pensado para
meu futuro mantimento e conveniência. Ocupava"-me, então, mais dos
cuidados com minha segurança do que com o provimento de comida.
Privava"-me de pregar um prego ou cortar madeira, por medo de que fosse
ouvido o barulho que eu fizesse; e, pelo mesmo motivo, ainda mais de
disparar minha arma; e, acima de tudo, causava"-me imenso desconforto
acender o fogo, temendo que me traísse a fumaça, que é visível a uma
grande distância durante o dia. Por esse motivo, deixei de realizar as
tarefas que exigiam lume, como assar o barro de panelas e cachimbos,
etc., em minha nova morada na mata, onde, depois de algum tempo, e para
meu indescritível alívio, descobri uma caverna natural na terra, a qual
se estendia muitíssimo chão adentro e onde, penso eu, nenhum selvagem,
caso a sua entrada chegasse, bravo seria a ponto de se aventurar; nem,
de fato, qualquer outro homem, senão alguém que, como eu, nada desejava
mais que um refúgio seguro.

A boca dessa caverna ficava ao fundo de uma grande rocha, onde por mero
acidente (assim eu o entenderia, se não visse copiosa razão de atribuir
todas essas coisas à Providência) eu cortava alguns galhos grossos de
árvore para fazer carvão; mas antes de prosseguir, devo expor a razão de
eu ter produzido esse carvão, que foi a seguinte:

Tinha medo de fazer fumaça em redor de minha casa, como já foi dito; e,
no entanto, não poderia viver ali sem assar meu pão, cozinhar minha
carne etc. então, logrei queimar um pouco de madeira sob a turfa, como
já havia visto na Inglaterra, até que se tornasse carvão seco, isto é, o
carvão de madeira; o qual, depois de apagar o fogo, resguardei para
carregá"-lo para casa e, assim, levar a cabo as demais tarefas às quais
havia falta de fogo, sem que temesse a fumaça.

Para encurtar: enquanto eu cortava um pouco de madeira, percebi que,
atrás dos ramos muito espessos de um arbusto, ou galharia ao rés do
chão, havia uma espécie de espaço vazio, o qual tive vontade de
investigar; e, alcançando com dificuldade a boca da caverna, descobri
que era muito larga, isto é, que era grande de forma a permitir que eu e
talvez mais outra pessoa nela ficássemos de pé: mas devo confessar que,
mal havia entrado, apressei"-me a sair, quando, prosseguindo em indagação
do lugar, o qual era absolutamente escuro, vislumbrei os dois grandes
olhos brilhantes de alguma criatura, se demônio ou homem, não o pude
saber, que reluziam como duas estrelas, refletindo a luz fraca
proveniente da boca da caverna, que incidia diretamente sobre eles.

No entanto, depois de algum intervalo de tempo, recobrei a razão e
declarei"-me mil vezes tolo, dizendo a mim mesmo que um homem que teme o
diabo não estava apto a viver vinte anos sozinho numa ilha; e que era
muito provável que não houvesse nada mais assustador naquela caverna do
que eu mesmo; e assim, com o ânimo refeito, tomei de uma acha de fogo e
entrei novamente, com a madeira acesa na mão; e não havia dado três
passos quando me vi tão assustado quanto antes; pois ouvi um gemido
muito alto, qual o de um homem que padecesse de dor, e este seguido de
ruídos sucessivos, como se fossem palavras que não se dissessem
inteiras, e depois um novo gemido profundo. Dei um passo para trás e, em
verdade, tamanho foi o assombro que suei frio, e tivesse eu um chapéu na
cabeça, não duvido que meu cabelo o tivesse erguido. Mesmo assim, fiz
das tripas coração, e depois de alentar"-me com estarem o poder e a
presença de Deus em toda parte, e prontos a me proteger, mais uma vez
dei um passo à frente e, à luz da tocha, que segurava um pouco acima da
cabeça, vi deitado ao chão um bode monstruoso e assustador,
despedindo"-se desta vida, como dizemos, e com a respiração ofegante e
morrendo da velhice.

Cutuquei"-o para ver se conseguia tirá"-lo de lá, e ele tentou se
levantar, mas não conseguiu se erguer; e pensei comigo mesmo que ele bem
poderia permanecer ali; pois se ele havia me assustado, certamente
assustaria qualquer um dos selvagens, caso tivesse coragem bastante para
entrar ali enquanto alguma vida nele houvesse.

Recuperado de minha consternação, pus"-me a olhar no meu redor, quando
descobri que a caverna era muito pequena, isto é, poderia ter obra de
doze pés de altura, mas sem qualquer forma, nem redonda, nem quadrada,
não tendo conhecido mãos que se empenhassem em abri"-la, senão as da mera
natureza. Observei também que havia um lugar em sua ponta mais distante
que se abria para além, mas era tão baixo que exigia que eu me
arrastasse de gatinhas para entrar nele, e aonde dava, eu não sabia;
então, não dispondo de vela, desisti por ora, resolvido porém a voltar
no dia seguinte com velas e um acendedor, que havia feito do fecho de um
dos mosquetes e um pouco de fogo"-grego na caçoleta da escorva.

No dia seguinte, portanto, rumei à caverna municiado de seis velas
grandes que eu próprio fabricara; pois fazia, então, muito boas velas
com o sebo das cabras; e, entrando nesse lugar baixo, fui obrigado a
arrastar"-me de gatinhas, como disse atrás, quase dez jardas; o que, a
propósito, julguei um grande risco, considerando que não sabia quão
longe poderia ir, nem o que encontraria além. Quando atravessei o
estreito, vi o teto elevar"-se obra de vinte pés; mas nunca havia estado
diante de tão gloriosa visão naquela ilha quanto a das laterais e do
teto daquela abóbada ou caverna; as paredes refletiam minhas duas velas
em cem mil cintilações; e o que na rocha havia, quer fossem diamantes,
quer fosse qualquer outra pedra preciosa ou, como suspeitava ser, ouro,
eu não sabia dizer.

O lugar em que estava era uma furna, ou gruta, muito aprazível, como se
poderia esperar, embora perfeitamente escura; o chão era seco e plano,
coberto de uma espécie de cascalho solto, de modo que não se via
criatura peçonhenta ou nauseabunda, nem havia umidade ou água nas
laterais ou no teto. A única dificuldade era a entrada, que, no entanto,
sendo aquele um local seguro, e um refúgio tal qual eu o queria, concluí
ser uma conveniência; de maneira que me alegrou muitíssimo a descoberta
e decidi levar sem demora algumas das coisas que eu mais ansiava para
esse lugar; em particular, minha provisão de pólvora e todas as minhas
armas sobressalentes, a saber, duas espingardas, pois eu tinha três no
total, e três mosquetes, pois deles tinha oito ao todo; de forma que
mantive cinco apenas em meu castelo, que permaneciam a postos como bocas
de fogo em meu cercado mais exterior; e também podiam ser levados a
qualquer excursão.

Na ocasião do transporte de minha munição à caverna, aproveitei o ensejo
para abrir o barril de pólvora que tirei do mar e estava encharcado; e
descobri que a água havia penetrado perto de três ou quatro polegadas na
pólvora por todos os lados, que empelotando e endurecendo, tinha seu
interior preservado como uma amêndoa dentro da casca, de guisa que eu
dispunha de quase sessenta libras de muito boa pólvora no centro do
barril; e essa foi uma descoberta muito feliz para mim naquele momento;
e assim transportei tudo para lá, sem conservar mais do que duas ou três
libras de pólvora na minha fortaleza, por medo de qualquer investida dos
selvagens; também trasladei para a caverna todo o chumbo que ainda
tinha.

Em minha fantasia, surgia eu então como um dos antigos gigantes, os
quais se dizia habitarem as cavernas e grutas, onde ninguém os podia
alcançar; pois enquanto ali estive permaneci convencido de que, mesmo se
quinhentos selvagens saíssem a caçar"-me, nunca seriam capazes de
descobrir"-me; ou ali, se o fizessem, não teriam a audácia de atacar"-me.

O bode velho que eu encontrei em vias de perecer morreu na boca da
caverna no dia seguinte à minha descoberta; e encontrei mais facilidade
em cavar uma boa cova ali, e empurrá"-lo para dentro e cobri"-lo de terra
do que arrastá"-lo para fora; assim, enterrei"-o ali, de forma a poupar
meu nariz do odor.

Eu estava então em meu vigésimo terceiro ano de residência na ilha e
sentia"-me tão naturalizado à terra e em paz com as maneiras de viver
que, pudesse ter desfrutado da certeza de que nenhum selvagem ali
chegaria para tirar"-me a tranquilidade, teria me contentado com a
rendição e minha permanência ali pelo resto dos meus dias, até o último
instante, quando me deitaria e morreria, como o velho bode na caverna.
Tinha também pequenas distrações e divertimentos, que permitiam que o
tempo passasse de forma muito mais agradável do que outrora; em primeiro
lugar, ensinara a meu Loro a falar, como disse antes; e ele o aprendeu
muito fluentemente, e falava de uma forma clara e articulada que me era
muito agradável; e ele viveu comigo por não menos que vinte e seis anos.
Quanto tempo mais ele tinha a viver, eu não sei, embora digam nos Brasis
que vivem cem anos; talvez meu pobre Loro ainda viva, chamando até hoje
seu ``Pobre Robin Crusoe''; e desejo a nenhum inglês a desventura de dar
à ilha e ouvi"-lo; pois caso isso aconteça, ele talvez julgasse ser o
demônio. Meu cachorro foi para mim um companheiro agradável e carinhoso
por não menos que dezesseis anos, tendo morrido da mera velhice; quanto
aos meus gatos, eles multiplicaram"-se a tal ponto, como dito atrás, que
logo fui obrigado a matá"-los em grande número, para impedir que
devorassem a mim e a tudo o que tinha; mas, por fim, quando os dois
velhos gatos que trouxera comigo morreram, e depois de algum tempo
afastando"-os de mim, sem que lhes deixasse qualquer provisão, todos
foram para o meio do mato, onde se tornaram selvagens, exceto dois ou
três deles, os quais preservei mansos e dos quais muito gostava, e cujas
crias, quando as tinham, eu sempre afogava; e estes faziam parte da
minha família. Além disso, eu sempre conservei dois ou três cabritos à
roda da casa, os quais ensinei a comer da minha mão; e eu tive ainda
dois papagaios mais, os quais falavam muito bem, chamando"-me pelo meu
nome, porém jamais como o meu primeiro; em verdade, tampouco gastei com
qualquer um deles o tempo que reservara a ele. Tinha também várias aves
marinhas mansas, cujo nome eu não conhecia, as quais havia capturado na
praia e cujas asas havia cortado nas pontas; e com os esteios que
espetara diante da muralha de minha fortaleza já crescidos e erguendo"-se
em bom bosque, todas essas aves viviam entre tais árvores baixas e se
criavam ali, o que me era muito agradável; de maneira que, como o tenho
dito, comecei a ficar muito contente com a vida que levava,
excetuando"-se não poder me ver a salvo do pavor dos selvagens.

Mas as coisas tomaram outro rumo; e talvez não escape a todas as pessoas
que estiverem diante de minha história depreender essa justa observação;
a saber, com que frequência, no decorrer de nossas vidas, o mal que mais
tentamos evitar, o qual, quando a ele sucumbimos, é o mais terrível para
nós, amiúde se revela o próprio meio ou porta de nosso salvamento, sendo
possível apenas por seu expediente que nos levantemos da aflição em que
caímos. Eu poderia dar muitos exemplos disso no decurso de minha vida de
agruras tamanhas; mas em nada essa lição fez"-se mais admirável do que
nas circunstâncias dos meus últimos anos de residência solitária naquela
ilha.

Estava no mês de dezembro, como disse atrás, em meu vigésimo terceiro
ano de residência na ilha; e sendo aquele o solstício do sul, pois a ele
não podia dar o nome de inverno, era o momento preciso de minha
colheita, que exigia que eu me alongasse com muita frequência às
plantações; e foi assim que, saindo de manhã cedo, antes mesmo de haver
plena luz do dia, espantou"-me avistar um brilho de fogueira na praia, a
obra de duas milhas de distância na direção dos confins da ilha, onde
observara a presença de selvagens, como já foi dito; o lume, porém,
estando não do outro lado, mas, para minha grande consternação, do meu
lado da ilha.

Fiquei em verdade muito espantado com o que vi e parei na imediação de
meu arvoredo, sem coragem de prosseguir, temendo ser surpreendido; e,
mesmo assim, senti conflagrados meus pensamentos, pois me causava
inquietação que, se os selvagens, ao vaguear pela ilha, encontrassem
minha cevada, fosse crescida, fosse cortada, ou qualquer um de meus
trabalhos ou benfeitorias, eles concluiriam imediatamente que havia
gente no lugar e não descansariam até que me descobrissem. Nessa grande
aflição, retornei diretamente ao meu castelo, puxei a escada atrás de
mim e cuidei para que tudo parecesse o mais inabitado e natural
possível.

Então me preparei do lado de dentro, colocando"-me em posição de defesa.
Carreguei todas as minhas bocas de fogo, como chamava, a saber, meus
mosquetes, armados em minha nova fortificação, e todas as minhas
pistolas, e decidi que me defenderia até o último suspiro, sem
esquecer"-me de invocar a Proteção divina e orar com ardor a Deus por meu
salvamento contra os bárbaros; e permanecendo assim por obra de duas
horas; quedando, por fim, muito impaciente com a falta de notícias do
exterior, visto que não dispunha de espiões que pudesse despachar.

Depois de ficar ali por um pouco mais de tempo, pensando no que devia
fazer nesse caso, não consegui mais suportar a angústia de nada saber; e
assim, encostando minha escada no flanco da colina, onde havia rebordo
plano, como dito atrás, e em seguida puxando"-a atrás de mim para mais
uma vez a botar de pé e montar, subi ao cimo da colina; e travando de
minha luneta, que levara para tanto, pus"-me de bruços no chão e comecei
a examinar o lugar; descobri, então, que eram nove os selvagens, e que
estes estavam nus em redor de uma pequena fogueira, a qual haviam
acendido sem intenção de aquecer"-se, uma vez que não necessitavam disso,
sendo o clima extremamente quente; mas, como supunha, com o intuito de
preparar a nefasta ração de carne humana que haviam trazido consigo, não
se sabendo determinar se viva ou morta.

Eles tinham duas canoas, que haviam arrastado à praia; e estando a maré
baixa, pareceu"-me que esperavam o retorno da cheia para partir. Não é
fácil imaginar o estado de perturbação em que essa visão me colocava,
principalmente ao vê"-los chegados ao meu lado da ilha, e tão perto de
mim; mas quando me ocorreu, ao observá"-los, que sua chegada talvez
sempre coincidisse com a maré baixa, passei a ter pensamentos mais
tranquilos, feliz de poder sair com segurança durante toda a maré cheia,
caso não tivessem antes chegado à costa; e tendo feito essa observação,
passei a sair para o meu trabalho de colheita sem preocupações.

Minhas expectativas foram confirmadas; pois assim que a maré correu a
oeste, eu os vi embarcar na canoa e remar para longe; devia ter
registrado que, por uma hora ou mais antes de partirem, eles dançaram, e
com minha luneta eu pude facilmente discernir suas posturas e gestos;
não pude senão notar, por minha mais aguda observação, que estavam
completamente nus e não tinham o que minimamente os cobrisse; mas, se
eram homens ou mulheres, eu não era capaz de dizer.

Tão logo os vi embarcar e partir, coloquei duas espingardas nos ombros,
duas pistolas no cinto, e o cutelo largo, sem bainha, ao flanco, e com
toda a presteza de que era capaz fui à colina de onde fizera seu
primeiro descobrimento; e tendo lá chegado, o que não me tomou menos de
duas horas (pois não conseguia imprimir boa velocidade, tão carregado
que estava de armas), observei que havia ainda outras três canoas dos
selvagens naquele lugar; e, olhando para além, vi que estavam todas
juntas no mar, com destino à terra firme.

O que tinha diante de mim era horrível de se ver, especialmente quando,
descendo a encosta, dei com as evidências tremendas que os nefastos
trabalhos por eles executados haviam sido deixado para trás; a saber, o
sangue, os ossos e parte da carne dos corpos humanos comidos e devorados
por aqueles malditos com prazer e folgança; tendo sido tamanha a minha
indignação ante o que via que passei a premeditar a aniquilação dos
próximos com quem ali me deparasse, a despeito de quem e quantos fossem.

Pareceu"-me claro que as visitas que faziam à ilha não eram muito
frequentes, pois somente um ano e meio depois outros deles vieram à
costa; com o que quero dizer que por todo esse tempo não encontrei
pegadas ou vestígios deles; pois nas estações chuvosas eles decerto não
viajavam ao exterior, ou ao menos não tão longe; no entanto, foi com
grande desconforto que vivi todo esse período, em razão do constante
temor de ser por eles surpreendido; donde observo que a expectativa do
mal é mais amarga do que o sofrimento, especialmente se não há maneira
de dirimir tal expectativa ou temor.

Durante todo esse período, vi"-me tomado de humores assassinos e empenhei
a maior parte do tempo, que deveria ter sido mais bem empregado, na
elaboração de estratégias de cerco e ataque contra eles, quando
novamente os visse; e em especial para o caso de estarem divididos em
dois grupos, como ocorrera na última ocasião; não pensando que, se eu
matasse um deles, a saber, de dez ou doze, eu ainda teria de matar outro
no dia, na semana ou no mês seguinte, e ainda outros e outros e
consecutivamente \emph{ad infinitum}, até que por fim não seria menos
assassino do que eles eram enquanto devoradores de homens; e talvez
muito mais.

Passava, então, meus dias em grande aflição e agitação de espírito, na
expectativa de que cedo ou tarde caísse nas mãos daquelas criaturas
impiedosas; e se, em algum momento, aventurei"-me distante de meu
castelo, não foi sem observar o meu entorno com grande cuidado e
cautela; e então descobri, para meu grande conforto e satisfação, como
fora providente em criar um rebanho ou redil de cabras mansas; pois de
forma alguma ousaria disparar minha arma, especialmente perto das bandas
da ilha em que costumavam ter, temendo alvoroçar os selvagens; e se eles
então fugissem de mim, eu tinha certeza de que os veria novamente,
chegando com talvez outras duzentas ou trezentas canoas, em poucos dias,
e nesse caso eu sabia o que me esperava.

No entanto, vivi ainda um ano e três meses até que desse com qualquer
vestígio dos selvagens, e então mais uma vez os encontrei, como direi a
frente. É verdade que possam ter lá estado uma ou duas vezes naquele
ínterim; mas ou ali não se estenderam, ou pelo menos deles não tive
notícia; mas no mês de maio, tanto quanto pude calcular, do meu vigésimo
quarto ano, tive um encontro muito estranho com eles; o qual a seu tempo
relatarei.

A perturbação da minha mente por esse espaço de quinze ou dezesseis
meses foi grande; tinha um sono intranquilo, sonhava sonhos
assustadores, não raro despertava sobressaltado durante a noite. No
decorrer do dia, meus pensamentos eram acometidos de imensa preocupação;
e à noite meus sonhos versavam muitas vezes sobre eu matar os selvagens,
e sobre razões que justificassem meu ato; mas deixando tudo isso por
ora, julgo que eram meados de maio, o décimo sexto dia, como o
registrava meu pobre calendário talhado no poste; pois ainda assinalava
ali os dias e meses; pois bem, foi em 16 de maio que soprou um forte
vendaval, o dia todo, acompanhado de tormenta de raios e trovões; ao
qual se seguiu noite áspera. Não me recorda o momento preciso, mas foi
enquanto lia a Bíblia, mergulhado em pensamentos muito sérios sobre
minha condição presente, que fui surpreendido por um estampido de
canhão, como que disparado do mar.

O espanto que me sobreveio era de natureza completamente distinta dos
que até então experimentara; pois sugeria a meus pensamentos ideias de
outra espécie. Ergui"-me de um salto; e, num instante, coloquei de pé
minha escada no meio da rocha, puxando"-a comigo para novamente a armar e
montar; chegando ao cume da colina no exato instante em que um clarão
anunciava um segundo disparo, que em pouco menos de um minuto eu ouvi; e
pelo som, soube que vinha das bandas do mar em que minha canoa fora
arrastada.

De imediato julguei que se tratava de navio em perigo, que, navegando em
companhia de um ou vários navios, disparara as canhonadas em sinal de
dificuldade e com o intuito de obter ajuda; e ocorreu"-me naquele
instante pensar que, embora não os pudesse socorrer, talvez eles a mim o
pudessem, de maneira que reuni toda a madeira seca que pude recolher
sobre a colina e, erguendo"-a em uma boa pilha, ateei"-lhe fogo. A madeira
estava seca e queimou livremente; e embora soprasse vento forte, queimou
sem dificuldade; e tinha certeza de que, havendo ali algo como um navio,
sua tripulação necessariamente o tinha avistado, como sem dúvida foi o
caso; pois assim que meu fogo se ergueu, ouvi um novo disparo, e em
seguida muitos outros, todos do mesmo lugar; eu fiz meu fogo queimar a
noite inteira, até o amanhecer; e quando já fazia dia claro, avistei
algo no mar, a uma grande distância, a leste da ilha; vela ou casco, não
era capaz de discernir, nem mesmo com minha luneta; sendo grande a
distância, e restando ainda um pouco de neblina, ao menos àquela
longitude no mar.

Observei"-o de tempo em tempo ao longo de todo o dia, e logo percebi que
não se movia; de onde concluí que se tratava de navio ancorado; e,
desejoso que estava, como se pode supor, de auxílio, tomei minha arma e
corri em direção à banda sul da ilha, às rochas aonde eu havia sido
arrastado pela correnteza; e, ali chegando, com o tempo já perfeitamente
aberto, pude ver claramente, para minha grande tristeza, os destroços de
um navio, à noite arrojados contra os recifes submersos em que vim a dar
com minha canoa; os quais, servindo de obstáculo à violência da
correnteza, formavam uma espécie de contrafluxo, ou redemoinho, foram a
razão de meu salvamento da mais inditosa e desesperada circunstância em
que já estive em toda a minha vida.

Assim, o que é a segurança para um homem leva a um outro a destruição;
pois parece que aqueles homens, a despeito de quem fossem, não dispunham
de conhecimento do lugar e, com os recifes completamente submersos,
foram contra eles arrojados durante a noite, com o vento soprando forte
de leste e les"-nordeste. Se tivessem avistado a ilha, como devo
necessariamente supor que não avistaram, decerto que teriam lutado para
chegar a salvo à praia com a ajuda de seu bote; mas os disparos dos
canhões em busca de ajuda, sobretudo ao terem avistado minha fogueira,
como eu imaginara, encheram"-me de pensamentos. Em primeiro lugar,
imaginei que, ao avistar meu lume, pudessem ter embarcado no bote e se
empenhado em chegar à costa; mas que, com o mar encapelado, talvez
tivessem naufragado; noutros momentos, imaginei que pudessem ter perdido
o bote antes, como poderia ter ocorrido de diversas formas;
particularmente em razão da invasão do convés pelas águas, o que muitas
vezes obriga a tripulação a destruir ou fazer o bote em pedaços, e por
vezes lançá"-lo ao mar com suas próprias mãos; outras vezes, imaginei que
viajassem em companhia de outros navios, ou de um comboio, o qual, ante
o pedido de socorro, recolheu e levou a tripulação apurada; em outras
oportunidades, ocorreu"-me que todos haviam descido ao mar no bote e,
arrastados pela mesma correnteza que me tragara anteriormente, foram
arrastados ao grande oceano, onde nada havia além de padecimentos e
morte; e que àquelas alturas talvez estivessem ante a fome e a
necessidade de terem de comer uns aos outros.

Como tudo isso, na melhor das hipóteses, não passava de conjeturas, na
condição em que estava, não podia fazer mais do que refletir sobre os
padecimentos daqueles pobres homens e sentir pena deles, o que de minha
parte tinha este bom efeito; a saber, que me dava ainda mais razões de
ser grato a Deus, que me provera de tamanha felicidade e conforto em meu
desterro; e que, das companhias de dois navios que haviam naufragado
naquelas bandas do mundo, nenhuma outra vida poupara, senão a minha.
Mais uma vez aprendia a reconhecer que é muito raro que a Providência de
Deus nos abandone a uma condição de padecimento e desespero tais que não
encontremos algo digno de gratidão, e outras pessoas em circunstâncias
piores do que a nossa.

Esse era o caso daqueles homens, em relação aos quais não conseguia ver
meios de supor que tivessem sobrevivido; não havia justificativa
racional que amparasse meu anseio ou esperança de que nem todos
houvessem perecido; a não ser que tivessem sido resgatados por outro
navio do comboio; e essa não passava de uma possibilidade remota, pois
não via o menor sinal ou índice de que algo assim tivesse acontecido.

Não conheço empenho possível de palavras que explique o estranho desejo
ou ânsia que senti em minha alma ante esta visão, que algumas vezes
assim se me mostrou; que, ai!, uma ou duas almas; ou que mesmo uma alma
apenas tivesse conhecido salvamento e, assim, chegasse até mim, de forma
que eu pudesse ter um companheiro ou semelhante que conversasse comigo e
com quem eu pudesse falar! Em todo o tempo que vivera em solidão, nunca
havia sentido desejo tão urgente e violento de travar contato com meus
semelhantes, nem pesar tão profundo por faltar"-me esse contato.

Existem certos mecanismos secretos nas paixões que, quando colocados em
movimento por algum objeto que se veja, ou mesmo que se faça presente
aos pensamentos por força da imaginação, suscitam a impetuosidade da
emoção e levam a alma a tão violenta e obsessiva adoração do objeto que
sua ausência se torna insuportável.

Assim era esse profundo desejo de que um único homem tivesse sido
poupado! Que tivesse sido apenas um, que tivesse sido apenas um!,
repetia eu incontáveis vezes; e sentia minhas emoções tão comovidas por
essas palavras que minhas mãos se fechavam em punho, premindo os dedos
contra as palmas com tanta força que, caso eu estivesse segurando
qualquer coisa mais frágil, decerto que a teria esmagado sem o querer; e
sentia os dentes baterem e travarem em minha boca com tamanha violência
que somente algum tempo depois eu era capaz de separá"-los.

Que os naturalistas expliquem essas coisas, e suas razões e maneiras;
tudo o que posso lhes dizer é a descrição do fato, que muito me espantou
quando ocorreu, dado que não conhecia sua procedência; era, sem dúvida,
o efeito de desejos ardentes e de fortes ideias formadas em minha mente,
ante o conforto que a conversação com um dos meus semelhantes cristãos
teria representado para mim.

Mas não quis o meu destino, ou o destino deles, que assim fosse; pois
até o último ano de minha residência na ilha, nunca vim a saber se
alguém havia escapado àquele navio; conhecendo apenas a dor, alguns dias
depois, de ver o cadáver de um rapaz afogado dar à praia nos confins da
ilha próximos ao naufrágio. Vestia apenas o colete de marujo, um par de
calções abertos à altura do joelho e uma camisa de linho azul; mas nada
trazia que me permitisse adivinhar a nação; tinha nos bolsos duas moedas
de \emph{peso fuerte} e um cachimbo; este último tinha dez vezes mais
valor que os primeiros.

O tempo se acalmou, e eu senti grande desejo de ir com minha canoa aos
destroços do naufrágio; não tendo dúvidas de que encontraria algo de
útil a bordo, mas movido sobretudo pela possibilidade de ainda haver
alguma criatura viva a bordo, cuja vida eu poderia salvar e que poderia,
por sua vez, trazer grande alívio à minha; e esse pensamento agarrou"-se
ao meu coração de guisa tal que não encontraria sossego, noite e dia,
até que fosse com minha canoa aos destroços; e entregando o que demais
houvesse à Providência de Deus, tinha para mim que a impressão em minha
mente era tamanha que não podia resistir a ela, que certamente procedia
de algum mandamento insondável e que eu quedaria em falta comigo mesmo
se não fosse.

Sob o poder dessa impressão, apressei"-me de volta a meu castelo,
preparei tudo para a minha viagem, peguei uma boa quantidade de pão, um
pote grande de água fresca, uma agulha de marear, ou bússola, para a
travessia, uma garrafa de rum, pois ainda tinha uma ótima reserva, e uma
cesta de passas; e, assim, cercando"-me de tudo o necessário, fui a minha
canoa, esvaziei"-a da água que nela havia, levei"-a ao mar, fretei"-a com
meus mantimentos e em seguida retornei à casa para buscar outras coisas;
consistindo a segunda carga em um grande saco de arroz, o guarda"-sol
para ter sombra sobre minha cabeça, outro pote grande de água e obra de
duas dúzias mais de filõezinhos ou bolos de cevada, além de uma garrafa
de leite de cabra e uma peça de queijo; o que levei com grande esforço
até minha canoa; e orando a Deus pela boa condução de minha viagem,
parti e, remando a canoa ao longo da costa, cheguei finalmente à ponta
da ilha, a nordeste, cabendo"-me então decidir se devia me lançar ao
oceano e me aventurar ou não. Atentei às rápidas correntezas que se
estendiam de ambos os lados da ilha, ao longe, e que muito eu temia,
pela lembrança do perigo em que outrora havia incorrido, e senti
faltar"-me a coragem; pois antevi que, se eu adentrasse qualquer uma das
correntes, havia o risco de ser arrastado mar adentro, a ponto de perder
a ilha de meu alcance e visão; e que, então, sendo pequena a minha
canoa, caso soprassem ventos mais fortes, eu inevitavelmente quedaria à
deriva.

Esses pensamentos trouxeram tamanha opressão a minha mente que comecei a
desistir de minha empresa; e, tendo levado a canoa a um riachinho na
praia, desembarquei e sentei"-me em uma pequena elevação do terreno,
muito pensativo e inquieto, entre o medo e o desejo de iniciar a
jornada; e enquanto refletia, pude perceber que a maré subia e a cheia
se aproximava, o que por muitas horas tornaria minha viagem
impraticável; ao que me ocorreu que eu deveria subir à maior elevação
que pudesse encontrar e observar, se possível, como agiam as marés ou
correntes no momento da cheia, para que eu pudesse ajuizar se, no caso
de ser arrastado por uma delas mar adentro, não poderia ser trazido de
volta pela outra com igual rapidez. Tão logo o pensei, divisei uma
pequena colina, da qual se podia observar o mar de ambos os lados e que
proporcionava visão clara das correntes ou movimentos da maré e do
caminho que tomaria em meu retorno. Dali, descobri que, se a vazante se
estendia à proximidade da ponta sul da ilha, a cheia conduzia à costa
norte; donde me bastava apontar ao lado norte da ilha em meu retorno que
tudo correria bem.

Animado por essa observação, decidi na manhã seguinte partir com a
vazante da maré; e tendo pernoitado dentro da canoa sob um casacão de
vigília, já mencionado, iniciei viagem. Primeiro, adentrei o mar tomando
o curso norte, até que comecei a sentir a influência da correnteza que
se deslocava a leste e me levou a uma boa velocidade; sem, contudo, o
fazer com a violência da corrente do lado sul, que antes me havia feito
perder o governo da embarcação; mas mantendo o rumo sob forte controle
com o uso de um remo, segui em boa velocidade na direção dos destroços,
chegando a seu costado em menos de duas horas.

Era uma visão horrível de se ver; a nau, que por sua estrutura era
espanhola, estava bem presa, enfiada entre duas rochas; toda a popa e
quartel haviam sido destruído pelo mar; e a violência com que castelo de
proa fora arrojado contra os recifes, nos quais estava preso, havia
deitado os mastros grande e de mezena; isto é, os havia rompido; mas o
gurupés estava firme, assim como a estrutura da proa; e quando me
aproximei do navio, um cachorro apareceu sobre a amurada e ao me ver
chegar começou a uivar e chorar; e assim que o chamei, ele pulou ao mar
para vir ao meu encontro, e eu o coloquei a bordo da canoa, quase que
estava morto de fome e sede; e ofereci"-lhe um filão de pão, que ele
devorou como um lobo faminto que passara fome por quinze dias na neve;
em seguida, dei"-lhe um pouco de água fresca, que, se eu o deixasse à
vontade, ele teria bebido até rebentar.

Depois disso, subi a bordo; mas a primeira visão com que me deparei foi
a de dois homens afogados, na cozinha ou castelo de proa do navio,
abraçados um ao outro; donde concluí, como é de fato provável, que
quando o navio encalhou, estando em meio a uma tempestade, o mar
quebrava tão alto e ininterruptamente sobre o convés que os homens não
foram capazes de suportá"-lo e sufocaram com a constante invasão da água
tal como estivessem debaixo d'água. Além do cachorro, nada mais havia no
navio que tivesse vida; nem qualquer mercadoria que eu pudesse ver, além
da que a água destruíra. Havia alguns barris de bebida, não sabia dizer
se de vinho ou conhaque, que estava mais fundo no porão e que, com o
recuo da água, eu conseguia ver; mas eles eram grandes demais para serem
tirados dali. Vi vários baús, que, creio eu, pertenciam a alguns dos
marinheiros; e levei dois deles a minha canoa, sem examinar o que neles
havia.

Se a popa do navio tivesse sido reparada, e a proa solta, estou
convencido de que poderia ter feito uma boa viagem; pois, diante do que
encontrei naqueles dois baús, tinha razões para supor que o navio levava
grande quantidade de riqueza a bordo; e, se posso inferir do caminho que
ele percorria, talvez tivesse partido de Buenos Aires, ou do Rio da
Prata, na parte sul da América, além dos Brasis, com destino a Havana,
no Golfo do México, e de lá, quiçá, rumo à Espanha. Havia nele grande
tesouro, sem dúvida, mas sem qualquer serventia a quem quer que fosse
naquele momento; e o que havia se dado do resto da tripulação, eu então
desconhecia.

Encontrei, além desses baús, um barrilete cheio de bebida, contendo obra
de vinte galões, que com muita dificuldade levei para minha canoa. Havia
vários mosquetes na cabine e um grande chifre de pólvora, que trazia
perto de quatro libras; quanto aos mosquetes, não havia razão para
pegá"-los, então os deixei, mas levei a pólvora, assim como uma pá de
fogueira e pinças para a lenha, que eu queria enormemente, e também duas
chaleirinhas de latão, uma panela de cobre para fazer chocolate e uma
grelha; e com essa carga e o cachorro eu parti, já começando a cheia da
maré; e na mesma noite, obra de uma hora depois do anoitecer, retornei à
ilha, esgotado e exaurido.

Repousei naquela noite na canoa e, pela manhã, decidi abrigar o que
havia conseguido em minha nova caverna, e não levar para o meu castelo.
Depois de alimentado, dispus toda a minha carga em terra e comecei a
examiná"-la com cuidado. O barril de bebida, descobri ser uma espécie de
rum, mas não da casta que nos Brasis havia; e, para dizer o simples, não
era bom; mas quando abri os baús, encontrei muitas coisas de grande
utilidade; em um deles, por exemplo, encontrei em uma bonita frasqueira,
com garrafas de tipo extraordinário, todas cheias de licores muito finos
e deliciosos; as garrafas continham obra de três quartilhos cada e
tinham tampa de prata. Encontrei dois potes de compotas, ou doces de
frutas, muito saborosos, e tão bem fechados que a água salgada não as
estragara; e mais outros dois que não tiveram a mesma sorte. Encontrei
algumas camisas muito boas e que me foram muito benvindas; e obra de uma
dúzia e meia de lenços de linho branco e outros coloridos para usar no
pescoço; os primeiros também muito benvindos, sendo extremamente
refrescantes para secar o rosto em dias quentes. Além disso, quando
cheguei à gaveta do baú, encontrei três grandes sacos de \emph{pesos
fuertes}, com cerca de mil e cem peças no total; e em um deles,
embrulhados em papel, seis dobrões de ouro e algumas barras ou cunhas de
ouro; os quais, no todo, penso que pesassem perto de uma libra.

No outro baú havia algumas roupas, mas de pouco valor; e que, pelo que
trazia, devia ter pertencido ao mestre"-artilheiro; embora nele não
houvesse pólvora comum, mas duas libras de uma pólvora de um tipo
refinado, conservada em três frasquinhos com a utilidade, imagino, de
carregar armas de caça quando necessário. No geral, a viagem me rendeu
poucas coisas de utilidade; pois, quanto ao dinheiro, não tinha uso para
ele; não diferindo para mim da poeira que trazia sob os pés; e eu o
teria dado todo em troca de três ou quatro pares de sapatos e meias
ingleses, que eram coisas que tinha em grande falta e não usava havia
muitos anos. Em verdade, havia conseguido dois pares de sapato, que
havia tirado dos pés de dois homens afogados que encontrei no navio, e
mais outros dois pares em um dos baús, todos muito benvindos, mas que
não eram como os nossos sapatos ingleses, fossem os de passeio ou de
serviço; sendo o que chamamos de escarpins, mais do que sapatos.
Encontrei no baú desse marinheiro cerca de cinquenta moedas de
\emph{peso fuerte} em reais espanhóis, mas nenhum ouro; supus que este
pertencera a um homem mais pobre que o outro, que parecia ter pertencido
a algum oficial.

De qualquer forma, trasladei todo esse dinheiro para minha caverna no
castelo e o arrumei, como fizera com o dinheiro que trouxera de nosso
próprio navio; mas foi uma pena, como dito atrás, não ter alcançado a
outra parte do navio; pois creio que teria carregado minha canoa várias
vezes com dinheiro; o qual, no caso de escapar algum dia à Inglaterra,
poderia ter conservado ali em boa segurança até que eu pudesse retornar
para buscá"-lo.

Tendo então desembarcado todas as minhas coisas na praia e as destinado
a local protegido, voltei à minha canoa e remei"-a sem nunca me afastar
da costa até seu antigo porto, onde a fundeei, e botei"-me em marcha à
minha antiga habitação, que encontrei tranquila e em ordem; a partir
desse momento, passei a sentir"-me mais em paz e a viver à guisa antiga e
cuidar dos meus assuntos domésticos; e por um tempo vivi muito bem e
tranquilamente, apenas um pouco mais atento do que era meu costume,
observando o entorno com mais frequência e sem realizar tantas excursões
ao interior da ilha; e se por acaso me alongasse com mais liberdade,
sempre tinha por rumo a banda de leste, aonde estava eu seguro de que os
selvagens nunca chegavam, e aonde eu podia ir sem tantas precauções, nem
tão carregado de armas e munição, como sempre era o caso quando ia à
banda de oeste.

Nessas condições vivi por outros dois anos; mas minha desafortunada
cabeça, sempre pronta a me lembrar de que nasceu para trazer
padecimentos ao meu corpo, esteve todo esse tempo ocupada de maquinações
e projetos relacionados a como, caso fosse possível, poderia eu escapar
daquela ilha; noutros momentos, meu desejo era o de realizar outra
viagem aos destroços do naufrágio, embora a razão me dissesse que nada
ali restava que fosse digno de um tamanho risco; ou ainda de vaguear
aqui ou ali; e acredito verdadeiramente que, se eu tivesse então o barco
com que parti de Salé, teria buscado aventura em mar aberto, apontando a
qualquer direção e sem saber para onde.

Sou, por tudo que cerca minha história, um \emph{memento} àqueles que
foram tocados pela praga geral da humanidade, da qual, tanto quanto sei,
deriva metade de suas mazelas; a saber, a de não bastar ao homem a
condição a que Deus e a natureza o destinaram; pois, sem considerar aqui
meu estado primeiro, e os excelentes conselhos de meu pai, aos quais
minha oposição foi, tal como posso chamá"-lo, meu \textsc{Pecado
Original}; meus erros subsequentes, de mesma espécie, haviam servido de
meio para que chegasse a essa triste situação; pois se a Providência,
que com sua benevolência me havia assentado como plantador nos Brasis,
tivesse me abençoado com desejos comedidos, e eu tivesse me contentado
com um progresso paulatino, é muito provável que tivesse chegado a ser
por aquele período, isto é, \emph{o período em que vivi na ilha}, um dos
mais importantes plantadores dos Brasis; ou melhor, estou convencido,
pelos melhoramentos que implementei durante o breve tempo em que lá
vivi, e pelo acrescimento que provavelmente teria conhecido, caso lá
tivesse permanecido, de que talvez tivesse atingido a valia de cem mil
\emph{moidores}, ou reais portugueses de ouro; e que razão tinha eu para
deixar uma fortuna estabelecida, uma plantação bem abastecida, sempre
melhor e maior, para fazer mercancia de negros na Guiné, quando
paciência e tempo tanto teriam aumentado nossa riqueza ali, que os
poderíamos ter comprado à nossa porta daqueles cujo negócio é
capturá"-los? E que nos custassem mais caro, pois a despesa que se teria
poupado não valia um risco tão grande.

Mas, como demonstra o destino das cabeças jovens em geral, a reflexão
acerca da insensatez é exercício que demanda idade ou uma experiência
conquistada a duras penas, e comigo não foi diferente; no entanto, o
erro deitou tão profundas raízes em meu temperamento que eu não era
capaz de me contentar com minha situação, e assim me dedicava
incessantemente ao estudo detido de meios e possibilidades de fuga
daquele lugar; e para que eu possa, com maior prazer para o leitor,
expor o que ainda resta de minha história, talvez não seja impróprio dar
relação de minhas primeiras ideias acerca do estúpido plano de minha
fuga; e apresentar"-lhe como agi e sobre que fundamentos.

O correto seria que eu estivesse, então, recolhido a meu castelo,
passada minha jornada recente ao naufrágio, com minha fragata fundeada e
protegida sob as águas, como sempre, e minha condição restaurada ao que
antes fora: em verdade, eu tinha mais riqueza então do que havia tido
antes, mas de forma alguma era mais rico; pois não tinha como dar"-lhe
mais uso do que os índios do Peru antes da chegada dos espanhóis ali.

Era uma noite da estação chuvosa em março, no vigésimo quarto ano de
minha permanência naquela ilha de solidão; eu estava deitado na minha
cama, ou rede, acordado, muito bem de saúde, sem dores, sem aflições ou
inquietações do corpo; e tampouco da mente, ou não mais do que o comum;
mas me era impossível fechar os olhos, quero dizer, para dormir; não,
sequer uma piscadela a noite toda, que se passou como se diz a seguir:

É impossível, senão inútil, estabelecer a inumerável multidão de
pensamentos que àquelas alturas da noite atravessavam"-me em um turbilhão
aquela larga avenida do cérebro a que chamam memória. Percorri toda a
história da minha vida sob forma de miniatura, ou, como a posso referir,
abreviada, até a minha chegada à ilha, e também daquela parte da minha
vida que se inicia com minha residência ali. Em minhas reflexões sobre
minha situação desde que pisara pela primeira vez aquela praia,
comparava a feliz condição dos meus assuntos em meus primeiros anos com
a vida de intranquilidade, medo e preocupação, que se iniciava com meu
reconhecimento de uma marca de pé na areia; não que eu desacreditasse
que os selvagens haviam frequentado a ilha por todo aquele período, pois
em verdade era possível que tivessem estado várias centenas de vezes em
terra; a questão era que isso jamais chegara a meu conhecimento, de
forma que desconhecia apreensões acerca de sua presença; minha
satisfação era perfeita, embora o perigo fosse o mesmo; e eu era tão
feliz em meu desconhecimento do perigo que era como se nunca tivesse de
fato sido a ele exposto. Essa noção suscitou em meus pensamentos
reflexões muito propícias, e em particular esta: quão infinitamente
benévola é a Providência, que em sua governança da humanidade forneceu
limites tão estreitos à visão e ao conhecimento que temos das coisas; de
maneira que, embora andemos em meio a tantos milhares de perigos, cuja
visão, se nos fosse revelada, nos poderia perturbar a mente e subtrair o
ânimo, somos conservados calmos e serenos sob a dissimulação da verdade
das coisas, nada sabendo dos perigos que nos cercam.

Depois de algum tempo entretido em tais pensamentos, pus"-me a refletir
seriamente sobre o real perigo que me cercara por tantos anos naquela
mesma ilha; e sobre como havia andado com absoluta segurança e com toda
a tranquilidade possível; mesmo quando talvez nada além do cimo de uma
colina, de uma grande árvore ou da aproximação casual da noite me
separasse da pior espécie de aniquilação, a saber, a de cair sob o poder
de selvagens comedores de gente, ou antropófagos, que me teriam
capturado com o mesmo interesse com que eu capturava uma cabra ou
tartaruga; e que me matariam e devorariam com a mesma inocência com que
eu matava e devorava pombos ou tarambolas. Eu não faria justiça a mim
mesmo se dissesse que não era sinceramente grato ao meu grande
Preservador, a cuja proteção singular atribuía, com grande humildade,
todos esses salvamentos de mim desconhecidos, e sem o qual teria
inevitavelmente caído em mãos impiedosas.

Findos esses pensamentos, minha cabeça ocupou"-se por algum tempo de
sopesar a natureza daquelas desafortunadas criaturas, isto é, os
selvagens; e a razão de o sábio governante de todas as coisas ter
entregue algumas de suas criaturas a tal desumanidade; ou melhor, a uma
condição ainda mais vil, inferior à própria bestialidade, à medida que
devoram seus próprios semelhantes; mas como tudo isso resultou em
especulações (então) infrutíferas, ocorreu"-me inquirir acerca da porção
do mundo habitada por aqueles desgraçados; e quão distante da costa
ficava o ponto de que partiam; o que buscavam ao se alongarem tanto de
onde viviam; que maneira de barcos tinham; e por que eu não poderia
organizar minha vida e minhas tarefas de forma a ter condições de ir até
lá assim como eles tinham condições de vir a mim.

Eu não me dava ao trabalho de pensar em como agir, caso fosse até lá; e
no que seria de mim se caísse nas mãos dos selvagens; ou em como
escaparia deles se me atacassem; assim como não pensava em como poderia
chegar à costa e não ser atacado por alguns ou muitos deles, sem
qualquer possibilidade de salvamento; ou ainda, caso não fosse
capturado, como encontraria mantimento, ou até que ponto desviar minha
rota; pois bem, nenhum desses pensamentos chegou a me ocorrer; mas minha
mente não se ocupava de outra coisa senão de minha travessia ao
continente a bordo de minha canoa; e julgava minha condição presente a
mais desventurada possível, restando apenas a morte como pior destino do
que ela; e pensava que, se eu tocasse a costa do continente, talvez
encontrasse assistência, ou poderia seguir nunca distante da praia, como
fiz na costa africana, até dar em alguma região habitada onde houvesse
socorro; ou, por fim, me deparasse com algum navio cristão que pudesse
me resgatar: e o pior que me poderia acometer era a morte, que acabaria
de uma só vez com todos esses dissabores. Peço que se tenha em vista que
tudo isso era fruto de perturbações dos pensamentos, de humores
intranquilos, postos em desespero por prolongadas atribulações e as
expectativas frustradas nos destroços do navio do qual subira a bordo e
onde estivera tão perto de conseguir o que tanto desejava, a saber,
alguém com quem conversar e de quem obter algum conhecimento do lugar em
que estava e dos meios plausíveis de meu salvamento; sim, esses
pensamentos abalaram"-me imensamente; toda a paz que havia conquistado a
partir de minha resignação à Providência de Deus e de minha paciente
espera da manifestação dos desígnios dos Céus parecia ter chegado ao
fim; era como se eu não tivesse o poder de dirigir meus pensamentos a
outra coisa senão o plano de uma viagem ao continente, que caía sobre
mim com tamanha força e ímpeto do desejo que melhor seria não conhecer
resistência.

Depois de duas horas ou mais envolvido nesses pensamentos, tão violenta
era minha agitação que sentia meu próprio sangue ferver, e meu pulso
bater tão rápido e forte como se eu estivesse ardendo em febre,
unicamente derivada da extraordinária comoção de minha mente; e a
natureza, como se eu me encontrasse esgotado e exaurido do meu próprio
pensar, lançou"-me em sono profundo; poder"-se"-ia pensar que tudo não
passou de sonho, mas não sonhara com nada semelhante; sonhei, sim, que
saindo uma manhã de meu castelo, como de costume, avistava em cerca da
costa duas canoas e onze selvagens, que chegavam à terra trazendo
consigo outro selvagem com o intuito de matá"-lo e comê"-lo; quando, de
repente, o selvagem que seria morto escapou e fugiu para salvar sua
vida; e pensava em meu sono que ele vinha se esconder no pequeno bosque
cerrado que cercava minha fortificação; e vendo"-o sozinho, sem que os
outros o procurassem ali, chamava"-lhe a atenção e, sorrindo, o atraía: e
ele se ajoelhava diante de mim, parecendo implorar por socorro; ao que
lhe mostrava minha escada, fazendo"-o subir e trazendo"-o para minha
caverna; e ele se tornava meu serviçal; e assim que me via em posse
desse homem, dizia a mim mesmo: Agora posso me arriscar na travessia ao
continente, pois esse sujeito me servirá de piloto e me dirá o que fazer
e aonde ir para procurar mantimentos e aonde não ir, sob o perigo de ser
devorado; e em que lugares me aventurar e quais evitar; e acordei com
esse pensamento; e estava sob tão indescritível impressão de felicidade
ante a perspectiva sonhada da fuga, que a frustração que senti ao
recobrar a razão e descobrir que tudo não passava de um sonho relevou"-se
igualmente extrema em sentido contrário e converteu"-se em imensa
tristeza.

Diante disso, porém, cheguei à conclusão de que só seria capaz de
encetar um esforço de fuga se tivesse um selvagem em meu poder: e, se
possível, que fosse um dos prisioneiros a quem houvessem condenado à
morte, levado à ilha para ser devorado; mas esses pensamentos faziam"-se
acompanhar desta dificuldade: que era impossível levar o plano a cabo
sem que atacasse toda uma caravana desses comedores de gente e os
matasse a todos; e isso não apenas figurava gesto de absoluto desespero,
com grande risco de revés; como, por outro lado, havia ponderado
longamente sobre toda a sua legitimidade; e meu coração tremia ante a
ideia de tamanho derramamento de sangue, embora tivesse por finalidade
meu salvamento. Não preciso repetir os contra"-argumentos que me haviam
ocorrido, isto é, os mesmos atrás arrolados; mas embora tivesse então
novas razões a oferecer, a saber, que aqueles eram homens inimigos à
minha vida, e que me haviam de devorar se o pudessem; que era questão
urgente de sobrevivência libertar"-me daquela morte em vida, e que agia
em minha própria defesa, caso viessem eles a me atacar, e ainda outras;
pois bem, embora esses argumentos surgissem como justificativa do ato, a
ideia de derramar sangue humano em nome de meu salvamento me era
repulsiva, e de tal forma que por um bom tempo não consegui admiti"-la.

De qualquer forma, depois de muito debater em meu íntimo, e depois de
grandes perplexidades em torno do tema, pois todos esses argumentos, de
uma maneira ou de outra, permaneceram em luta em meus pensamentos por um
longo período, o imperioso e urgente desejo de libertação prevaleceu
sobre tudo o mais; e decidi ter um desses selvagens sob meu poder, sem
medir esforços para tanto. O passo seguinte era elaborar um plano, o
que, em verdade, era muito difícil de fazer; e não sendo capaz de
encontrar meios possíveis para a realização de meu projeto, decidi
montar vigília para saber de sua chegada à praia, e deixar que a ocasião
determinasse o resto; agindo ao sabor da oportunidade, fosse ela o que
fosse.

Com essas decisões tomadas, passei a montar vigília, o que fazia tanto
quanto podia e, em verdade, fiz tanto que acabei por quedar muito
aborrecido; pois esperei por mais de um ano e meio; saindo quase
diariamente às bandas de oeste e à ponta sudoeste da ilha com o intuito
de avistar canoas, sem que nenhuma aparecesse. Foi grande o desalento e
não pouco o incômodo dele derivado, embora não possa dizer que este, a
exemplo do que ocorrera anteriormente, tenha embotado meu desejo pela
coisa; pois quanto mais parecia demorar, mais desejoso eu ficava; em
suma, os antigos cuidados de evitar ser visto pelos selvagens davam
lugar à minha ânsia de atacá"-los.

Além disso, imaginava"-me capaz de assenhorar de um, ou ainda dois ou
três selvagens, caso os tivesse, de forma que se tornassem integralmente
meus escravos e realizassem o que lhes ordenasse, sem que eu corresse
qualquer risco de que me atacassem. Esse pensamento me cativou por um
bom tempo, mas nada se me apresentou; e todas as minhas fantasias e
planos reduziram"-se a nada, pois nenhum selvagem se aproximava da ilha.

Obra de um ano e meio depois de ter cultivado essas ideias, refletindo
bastante sobre elas sem que jamais chegassem a qualquer realização, por
falta de ocasião para as colocar em prática, espantou"-me avistar certa
manhã não uma, mas cinco canoas, todas arrastadas à areia, do meu lado
da ilha, e sem que as pudesse ver as pessoas a que pertenciam, todas
desembarcadas. O número desfez todos os meus planos; pois diante de
tantas canoas, e sabendo que estas sempre vinham de quatro ou seis
tripulantes, ou às vezes mais, não sabia o que pensar, nem que medidas
tomar para atacar sozinho vinte ou trinta homens; recolhi"-me então ao
meu castelo, onde permaneci confuso e incomodado, mas realizando, mesmo
assim, todos os preparativos anteriormente pensados e conservando"-me
pronto para a ação, caso alguma coisa se apresentasse. Tendo esperado um
bom tempo, sempre de ouvidos a postos para reconhecer algum ruído, e por
fim quedando muito impaciente, deixei minhas armas ao pé da escada e
subi ao cimo da colina, seguindo as duas passagens de sempre;
postando"-me de forma, contudo, que minha cabeça não surgisse acima da
colina, e eles não pudessem me avistar de maneira alguma; foi quando
observei, com a ajuda de minha luneta, que não eram menos de trinta
homens; que tinham fogo aceso, e carne preparada. Como a tinham
cozinhado, eu não sabia, assim como não sabia o que era; mas todos
dançavam em torno do fogo com gestos e volteios bárbaros, à sua maneira.

Ao observá"-los, pude ver através de minha luneta dois pobres coitados
arrastados dos barcos, onde, ao que parecia, haviam sido deixados para,
posteriormente, serem mortos. Percebi que um deles foi abatido de
pronto, derrubado, suponho, com porrete ou espada de madeira, pois assim
o faziam; e dois ou três outros começaram imediatamente a trabalhar no
corpo, abrindo"-o para seu preparo, enquanto a outra vítima era deixada
por sua conta, à espera de que pudessem tratar de sua sorte. Naquele
exato momento, a Natureza infundiu nesse pobre coitado, que se viu
ligeiramente livre, uma esperança de vida, e ele se afastou e correu com
incrível rapidez pelas areias em minha direção, isto é, em direção à
banda da costa onde minha habitação estava.

Fiquei terrivelmente assustado (devo reconhecer) ao vê"-lo correr em
minha direção; e especialmente quando, como eu pensava, o vi perseguido
por todo o grupo, esperando, então, que parte do meu sonho estivesse
ganhando vida, e ele certamente buscasse abrigo em meu bosque; mas não
podia contar de forma alguma com o resto de meu sonho, isto é, no qual
os outros selvagens não o perseguiam até lá, nem o encontravam. Mantive
o posto, portanto, e senti que recobrava o ânimo quando vi que não mais
do que três homens seguiam em seu rasto; e ainda mais quando notei que
ele os superava em muito na corrida, avançando no terreno de tal maneira
que, se pudesse aguentar meia hora, facilmente escaparia.

Havia entre eles e meu castelo o riacho que mencionei com frequência na
primeira parte da minha história, quando desembarquei meus carregamentos
do naufrágio; e ficou claro para mim que ele o devia superar a nado,
caso contrário seria recapturado ali; mas quando o selvagem em fuga ali
chegou, fez pouco do perigo, embora a maré estivesse alta, e mergulhou,
atravessou"-o com não mais do que trinta braçadas, veio à terra e correu
com força e rapidez extraordinárias. Quando os outros três deram com o
riacho, pude ver que um deles não sabia nadar, e que, permanecendo em
uma margem, ficou a observar a outra, sem ir além, e recuando lentamente
em seguida; o que, a propósito, provou"-se muito bom para ele no geral.

Observei que os dois que nadavam não tinha a metade da rapidez daquele
que fugia. Foi quando constatei com muita força, em verdade, uma força
irresistível, que aquele era o momento de obter um servo, ou talvez um
companheiro, ou assistente; e que era claramente convocado pela
Providência de Deus a salvar a vida daquela pobre criatura; sem demora e
com expedição desci as escadas, peguei minhas duas armas, que estavam
ambas ao pé da escada, como já foi dito; e subindo novamente e com a
mesma pressa ao cume da colina, cruzei"-o na direção do mar; e havendo um
atalho colina abaixo, desci de forma a me postar a meio caminho entre os
perseguidores e o perseguido; e chamei aos gritos o fugitivo, que,
olhando para trás, mostrou"-se talvez tão assustado comigo quanto estava
com eles; mas acenei para que voltasse; e, nesse ínterim, avancei
lentamente em direção aos dois que o perseguiam; e então, indo de
encontro ao que vinha à frente, derrubei"-o com uma coronhada; temendo
chamar a atenção dos demais com um disparo, embora àquela distância a
descarga não tivesse sido facilmente ouvida, e sem a visão da fumaça
tampouco seriam capazes de entender o que era. Tendo derrubado o
primeiro, o outro que vinha atrás parou, como se estivesse com medo, e
eu avancei em sua direção; mas, ao me aproximar, percebi que ele trazia
um arco e flecha e o armava tendo"-me por alvo; assim, fui obrigado a
atirar nele primeiro, o que fiz, matando"-o no primeiro tiro. O pobre
fugitivo parou; e embora visse os dois inimigos quedos e mortos, como o
pensava, estava tão assustado com o fogo e o estalo de minha arma que
ficou imóvel, sem avançar nem retroceder, não obstante parecesse mais
inclinado a fugir do que a se aproximar; de maneira que eu mais uma vez
o chamei e fiz sinais para que se achegasse, o que ele compreendeu sem
dificuldade, avançando um pouco e parando, e depois mais um pouco e
novamente parando; e pude perceber que ele tremia, como se tivesse sido
feito prisioneiro e estivesse prestes a morrer, como seus dois inimigos.
Acenei novamente para que se aproximasse, e o incentivei de todas as
maneiras possíveis; e ele se aproximava cada vez mais, ajoelhando"-se a
cada dez ou doze passos em sinal de reconhecimento por lhe ter salvado a
vida. Dirigi"-lhe um sorriso e um olhar amigo, e acenei para que ele se
achegasse ainda mais; até que se aproximou definitivamente; e então mais
uma vez se ajoelhou, beijou o chão e encostou a cabeça no chão e,
pegando"-me o pé, colocou"-o sobre sua cabeça; isto, ao que parece, o
fazia em sinal de jurar ser meu escravo para sempre; eu o levantei e o
tratei bem e o encorajei de todas as maneiras possíveis. Mas havia mais
trabalho a fazer, pois percebi que o selvagem que havia derrubado não
estava morto, mas apenas atordoado pelo golpe, e começando a voltar a
si; então, apontei em sua direção e mostrei"-lhe que o selvagem não
estava morto; ao que eleme dirigiu algumas palavras, e embora eu não as
pudesse entender, ainda assim as julguei agradáveis de se ouvir; pois
eram o primeiro som da voz de um homem que eu ouvia, exceto pela minha
própria voz, havia mais de vinte e cinco anos. Mas não havia tempo para
tais reflexões; o selvagem que fora derrubado recuperou"-se a ponto de se
sentar no chão, e percebi que meu selvagem começou a sentir medo; mas
quando atentei a isso, apontei minha outra arma ao homem, como se fosse
atirar; ao que meu Selvagem, pois assim o chamo agora, fez um movimento
para emprestar"-lhe meu cutelo, que trazia ao flanco preso ao meu cinto
sem bainha, e assim o fiz. Mal o teve em mãos, ele correu ao inimigo e,
de um só golpe, cortou"-lhe a cabeça com precisão e limpeza de fazer
inveja a qualquer carrasco alemão; o que achei muito estranho para
alguém que, tinha eu razões para crer, jamais havia visto uma espada em
sua vida, senão as de madeira que eram do uso de sua gente; no entanto,
como vim a saber depois, eles aparentemente fazem suas espadas de pau
tão afiadas e pesadas, e de uma madeira tão dura, que chegam a cortar
cabeças com elas, assim como braços, e também de um só golpe. Ao
fazê"-lo, ele veio rindo para mim em sinal de triunfo e trouxe"-me a
espada novamente, e com uma cópia de gestos que não compreendi, deixou"-a
ao lado da cabeça do selvagem, que havia matado diante de mim.

Mas o que mais o surpreendeu foi saber como eu matei o primeiro índio; e
assim, apontando em sua direção, fez sinais para que eu o deixasse ir
até ele; e eu dei"-lhe permissão, tão bem quanto pude; e quando lá
chegou, ficou parado, como que espantado, olhando para o morto,
virando"-o primeiro de um lado, depois do outro; examinando o ferimento
que a bala havia produzido, que parecia estar somente em seu peito, onde
o disparo havia aberto um buraco, sem que grande quantidade de sangue
houvesse escorrido; mas ele sangrara por dentro, pois estava muito
morto. Ele pegou o arco e as flechas e voltou; eu me virei, então, para
seguir no caminho de volta e pedi que ele viesse comigo, fazendo sinais
para que entende que outros ainda poderiam aparecer.

Em resposta, ele sugeriu com gestos que os enterraria com areia, para
que não fossem vistos pelos demais, caso viessem; e então respondi com
sinais para ele o fizesse; ele pôs mãos à obra; e num instante abrira um
buraco na areia grande o bastante para enterrar o primeiro, arrastou"-o
para dentro e o cobriu; e assim fez com o outro; creio que tenha
enterrado os dois em um quarto de hora; em seguida, chamei"-o, e o levei
não ao meu castelo, mas à minha caverna, na parte mais distante da ilha;
e assim não permiti que meu sonho se realizasse nesse ponto, isto é, que
ele procurasse abrigo em meu bosque.

Ali eu lhe dei pão e um punhado de passas para comer, e um gole d'água,
da qual julguei que ele estava em grande necessidade, por causa da fuga:
e o tendo refrescado, fiz sinais para que ele se deitasse e dormisse,
apontando"-lhe um lugar onde havia disposto bastante palha de arroz e um
cobertor sobre ela, que por vezes eu mesmo usava para dormir; e então a
pobre criatura se deitou e dormiu.

Era um homem de aparência bela e aprazível e de compleição perfeita,
membros direitos e fortes, não muito grandes; alto e bem torneado, e,
como pude inferir, contando seus vinte e seis anos de idade. O semblante
inspirava tranquilidade, não tinha feições bestiais e ameaçadoras;
parecia trazer algo de muito viril em seu rosto, não obstante os traços
portassem toda a delicadeza e doçura do europeu, em especial quando
sorria. Tinha os cabelos pretos e longos, não encaracolados como lã; a
testa muito alta e larga, e grande vivacidade e brilho nos olhos. Sua
pele não era de todo negra, mas muito morena; e, no entanto, não de um
moreno desagradável, amarelado e nauseabundo, como o dos brasileiros,
dos virginianos e outros nativos da América, mas de uma espécie de oliva
escuro e reluzente, bastante agradável de se ver, embora não muito fácil
de descrever. O rosto era redondo e gordo; o nariz pequeno, não chato,
como os negros; uma boca muito boa, de lábios finos e bonitos dentes,
direitos e brancos como o marfim. Depois de ter cochilado, mais do que
dormir, por não mais de meia hora, ele se levantou e saiu da caverna a
minha procura; pois eu estivera ordenhando as cabras que conservava em
um curral não distante dali; e quando me avistou, veio correndo em minha
direção e voltou a estender"-se ao chão com sinais de uma disposição
humilde e grata, o que demonstrava com gestos bastante estranhos, a
saber: deitando a cabeça no chão, próxima ao meu pé, e colocando meu
outro pé sobre sua cabeça, como havia feito antes; e depois disso,
fazendo todos os sinais possíveis de sujeição, servidão e submissão,
para que soubesse que ele me serviria enquanto vivesse. Eu o entendi em
grande parte e fiz com que compreendesse que estava muito satisfeito. Em
pouco tempo, comecei a conversar com ele; e ensinei"-o a conversar
comigo; e em primeiro lugar o fiz aprender que seu nome seria
Sexta"-Feira, que foi o dia em que lhe salvei a vida; e assim o fiz para
conservar a memória do tempo; e ensinei"-o a dizer Senhor, e fi"-lo
aprender que aquele seria o meu nome; também o ensinei a dizer Sim e Não
e a compreender o significado de cada um; e ofereci"-lhe um pouco de
leite em um pote de barro e fiz com que me visse tomando o leite e
molhando nele meu pão, para que assim aprendesse; e dei"-lhe um filão de
pão para que fizesse o mesmo, ao que ele rapidamente obedeceu, e fez
sinais de que era bom.

Fiquei ali com ele a noite toda; mas assim que raiou o dia, pedi que ele
viesse comigo e fiz com que entendesse que lhe daria algumas roupas; ao
qual se mostrou muito feliz, pois estava completamente nu; e quando
passamos pelo local onde ele havia enterrado os dois homens, ele o
apontou exatamente e mostrou as marcas que havia feito para
encontrá"-los, sugerindo por meio de sinais que deveríamos desenterrá"-los
e comê"-los; e diante disso, demonstrei muita irritação e expressei"-lhe
minha aversão a isso, sinalizando que poderia vomitar só de pensar
naquilo, e acenei com a mão para que se afastasse, o que fez
imediatamente e com grande submissão. Eu então o levei ao alto da
colina, para ver se seus inimigos haviam partido; e, pegando minha
luneta, olhei e vi com clareza o lugar onde haviam estado, mas sem sua
presença ou a de suas canoas; de maneira que era evidente que eles
haviam partido e deixado os dois companheiros para trás, sem
procurá"-los.

Mas eu não fiquei satisfeito com essa descoberta; e, tendo agora mais
coragem e, consequentemente, mais curiosidade, segui com meu
Sexta"-Feira, que levava na mão a espada que lhe havia entregue e nas
costas o arco e as flechas, os quais julguei que ele poderia usar com
destreza, e além disso uma das espingardas para meu uso, enquanto eu
levava duas comigo; e dali marchamos ao lugar onde as criaturas haviam
estado; pois eu pretendia obter mais informações a seu respeito. Quando
cheguei ao local, senti o sangue gelar em minhas veias e meu coração vir
à boca ante o horror do espetáculo; de fato, era uma visão terrível, ao
menos para mim, pois a Sexta"-Feira parecia nada dizer. O lugar estava
coberto de ossos humanos, o chão tingido de sangue e grandes postas de
carne abandonadas por toda a parte, meio comidas, rasgadas e
chamuscadas; e, em suma, todos os índices do banquete que ali tivera
lugar em celebração à vitória sobre seus inimigos. Vi três crânios,
cinco mãos e os ossos de três ou quatro pernas e pés, e abundância de
outras partes do corpo; e Sexta"-Feira fez"-me entender que eles haviam
trazido quatro prisioneiros para seu aprazimento; que três deles foram
devorados e que ele, apontando para si mesmo, era o quarto; que uma
grande batalha entre eles e o rei vizinho, do qual ele havia sido
súdito, e que haviam sido tomado prisioneiros em grande cópia e levados
a diferentes lugares por aqueles que os haviam capturado em luta a fim
de banquetearem, como o haviam feito os desgraçados que tinham dado
naquela praia com os seus presos.

Ordenei que Sexta"-Feira reunisse todos os crânios, ossos, carne e tudo
quanto restasse, e com eles fizesse uma pilha, à qual ateei fogo e fiz
arder até que se reduzisse a cinzas. Percebi que Sexta"-Feira ainda tinha
em grande desejo experimentar da carne e que ainda era em sua natureza
um antropófago, ou comedor de gente; mas eu demonstrei tamanha aversão
unicamente à ideia e ao menor indício de sua realização que ele não
ousou expressá"-lo: pois eu tinha de alguma forma o feito compreender que
o mataria se ele o fizesse.

Assim que fizemos tudo isso, retornamos ao nosso castelo; e ali me pus a
trabalhar para meu Sexta"-Feira; e antes de tudo, dei"-lhe calções de
linho, que busquei no baú do pobre mestre"-artilheiro encontrado nos
destroços do naufrágio, como dito atrás, e que, com um pequeno ajuste,
serviram"-lhe muito bem; em seguida, fiz para ele um colete de couro de
cabra, tão bem quanto minha habilidade permitia; pois já era naquele
tempo um alfaiate razoavelmente bom; e dei"-lhe um gorro que fiz de pele
de lebre, muito apropriado e bastante elegante; e assim por ora o vesti,
e relativamente bem, e ele fico muito satisfeito de se ver quase tão bem
quanto seu mestre; verdade seja dita, a princípio ele sentiu"-se
desajeitado com as roupas: vestir os calções lhe era muito penoso, e as
mangas do colete feriam"-lhe os ombros e a parte interna dos braços; mas
depois que lhe abri um pouco a costura nos pontos que o machucavam, e
depois de ele ter se habituado aos trajes, passou a tê"-los em grande
estima.

No dia seguinte, depois de ir a minha cabana com ele, comecei a pensar
no lugar em que o alojaria, de maneira a lhe servir bem e ser"-me cômodo;
e assim armei uma pequena tenda no espaço de entre minhas duas
fortificações, do lado de dentro do último muro e do lado de fora do
primeiro; e havendo ali uma passagem, ou porta, para a minha caverna,
montei um batente de porta formal e fabriquei uma porta de tábuas e a
levantei na passagem, já na parte interna da passagem; e, fazendo com
que a porta se abrisse por dentro, trancava"-a à noite, levando também
minhas escadas para dentro; e assim Sexta"-Feira não conseguia passar ao
interior de meu primeiro muro sem fazer barulho a ponto de
necessariamente me despertar; pois meu primeiro muro tinha então uma
cobertura cerrada, de compridos postes que cobriam toda a minha tenda e
que, inclinados, apoiavam"-se na parede de pedra, cruzados por galhos
menores no lugar de ripas, que depois cobri com palha de arroz em grossa
camada, tão forte quanto a que se faz de junco; e no buraco ou espaço
que havia para entrar ou sair pela escada, eu coloquei uma espécie de
alçapão que, se fosse forçado do lado de fora, não abriria, mas cairia,
fazendo grande barulho; e quanto às armas, eu as levava todas as noites
para o lado de dentro.

Mas eu não precisava de forma alguma dessa precaução; pois jamais se
conheceu criado mais fiel, amoroso e sincero do que Sexta"-Feira: sem
maus humores, más vontades ou interesses, sempre cioso do dever e com
ele comprometido; seu afeto se ligava a mim como o de uma criança ao
pai; e ouso dizer que ele teria sacrificado sua vida para salvar a minha
em qualquer ocasião; as muitas demonstrações que me ofereceu disso
dirimiram quaisquer dúvidas, e logo me convenci de que não precisava me
valer de precauções quanto a minha segurança por sua causa.

Isso me ofereceu não poucas ocasiões de observar, e com maravilha, que,
embora tenha sido da vontade de Deus, em Sua Providência e governo do
que obrara com Suas mãos, subtrair a tão grande parte do mundo de Suas
criaturas os melhores costumes aos quais estão adaptadas suas faculdades
e os poderes de suas almas, Ele lhes concedeu os mesmos poderes, a mesma
razão, as mesmas afeições, os mesmos sentimentos de bondade e dever, as
mesmas paixões e ressentimento do mal cometido; o mesmo senso de
gratidão, sinceridade, fidelidade e todas as capacidades de fazer e
receber o bem que conferiu a nós; e que, quando lhe apraz oferecer"-lhes
ocasião para seu exercício, essas criaturas encontram"-se tão preparadas,
ou melhor, ainda mais preparadas do que nós a sua justa aplicação aos
bons usos para os quais nos foram dadas; e esse pensamento me fazia por
vezes bastante melancólico, ao constatar, à medida que se me
apresentavam as várias circunstâncias, que mau uso fazemos de todas
elas, ainda que tenhamos esses poderes iluminados pelo grande lume de
instrução, pelo espírito de Deus e o conhecimento de Sua palavra,
acrescidos ao nosso entendimento; e pensar nas razões de Deus de furtar
ao mesmo conhecimento salvador tantos milhões de almas que, se eu
pudesse julgar por aquele pobre selvagem, fariam deles muito melhor uso
do que nós.

A partir daí, fui por vezes levado a ponto de invadir a soberania da
Providência e, por assim dizer, denunciar a justiça de uma tão
arbitrária ordem das coisas, que escondia essa Luz de alguns e a
revelava a outros esperando, não obstante, igual dever de ambos; mas eu
me continha e barrava meus pensamentos com esta conclusão, a saber:
primeiro, que não sabíamos por que Luz e Lei essas criaturas seriam
condenadas; mas que, por ser Deus por necessidade, e pela natureza de
Seu ser, infinitamente santo e justo, de guisa a não ser possível que
todas essas criaturas tivessem sido condenadas a seu abandono, tudo se
devia ao pecado contra essa Luz, que, como dizem as Escrituras, era para
si mesmo também lei,\footnote{Romanos 2:14 {[}\textsc{n.\,t.}{]}.} e mediante as
regras que suas consciências reconhecessem justas, embora o fundamento
se não tive nos sido revelado; e segundo, que, uma vez que todos somos o
barro na mão do oleiro,\footnote{Jeremias 18:6 {[}\textsc{n.\,t.}{]}.} nenhum vaso
poderia dizer"-lhe, ``Por que me formaste assim?''

Voltando ao meu novo companheiro: fiquei muito satisfeito com ele e
tomei para mim a tarefa de ensinar"-lhe tudo quanto fosse adequado para
que se me tornasse útil, solícito e habilidoso; mas especialmente para
que falasse e me entendesse quando eu falasse; e ele foi o aluno mais
dedicado de que já tive notícia; e, em particular, mostrava"-se tão
alegre, tão constantemente empenhado e satisfeito quando conseguia me
entender, ou quando fazia com que eu o entendesse, que muito me aprazia
conversar com ele; e minha vida então começou a ser tão tranquila que
passei a pensar comigo que, desde que estivesse a salvo de outros
selvagens, não me importava de permanecer ali enquanto vivesse.

Dois ou três dias depois de ter retornado ao meu castelo, concluí que,
para acabar com os horrendos hábitos alimentares de Sexta"-Feira e de
seus prazeres estomacais de comedor de gente, eu devia fazer com que
experimentasse outra carne; levando"-o então comigo em excursão certa
manhã pela mata, com o intuito, em verdade, de abater um bode de meu
próprio redil, levá"-lo para casa e assá"-lo; mas, no caminho, deparei"-me
com uma cabra descansando na sombra e dois filhotes acomodados ao seu
lado. Fiz com que Sexta"-Feira parasse. ``Espere'', disse eu, ``fique
parado''; e fiz sinais para que não se mexesse; imediatamente travei de
minha espingarda, atirei e matei um dos filhotes. A pobre criatura, que
já me havia visto matar um selvagem, seu inimigo, à distância, sem
saber, nem poder imaginar, como eu o fizera, ficou sensivelmente
estarrecida, tremeu e agitou"-se e pareceu tão espantada que pensei
desmaiaria. Ele não viu o filhote em que atirei, nem percebeu que eu o
havia matado, mas abriu o colete para verificar se não havia sido
ferido; e, como vim a descobrir, pensou então que eu estava decidido a
matá"-lo; pois ele veio e ajoelhou"-se diante de mim, e abraçando meus
joelhos disse muitas coisas que eu não compreendi; mas podia facilmente
depreender que implorava para que não o matasse.

Logo descobri uma maneira de convencê"-lo de que não lhe faria mal; e,
pegando"-o pela mão, ri e apontei ao filhote que havia matado,
gesticulando para que corresse e o buscasse, o que ele fez; e enquanto
ele refletia e examinava a criatura para ver como havia sido morta,
carreguei minha arma novamente, e logo depois avistei uma grande ave,
semelhante a um falcão, empoleirada em cima de uma árvore ao alcance de
um disparo; então, para fazer com que Sexta"-Feira entendesse um pouco o
que eu faria, chamei"-o novamente, apontei ao pássaro, que na verdade era
um papagaio, embora pensasse que fosse um falcão; ou seja, apontando ao
papagaio, e à minha arma, e ao chão sob o papagaio, para que visse que
eu o faria cair, fiz com que entendesse que atiraria e mataria aquele
pássaro; e assim, atirei e pedi que olhasse, e imediatamente ele viu o
papagaio cair, ao que mais uma vez demonstrou medo, apesar de tudo o que
lhe disse; e descobri que ele estava espantado sobretudo por não ter
visto eu colocar nada na arma, julgando que nela houvesse uma reserva
mágica de morte e destruição, capaz de matar homem, animal, pássaro ou
qualquer coisa próxima ou distante; e o assombro que isso suscitou nele
foi de tal monta que só se desfez muito tempo depois; e acredito que, se
o permitisse, ele nos teria idolatrado, a mim e à espingarda. Quanto à
arma em si, ele não a ousou tocar por vários dias; mas quando estava
sozinho conversava e falava com ela, como se ela lhe respondesse;
pedindo"-lhe, como soube depois dele, que ela não o matasse.

Pois bem: depois de acalmar"-se um pouco, ordenei com um gesto que ele
corresse e buscasse o pássaro que alvejara, o que ele fez, embora tenha
levado algum tempo; pois, não estando completamente morto, o papagaio
afastara"-se um tanto do local onde caíra: ele o encontrou, porém, e o
trouxe para mim; e como havia percebido sua ignorância quanto à arma,
aproveitei o ensejo para novamente a carregar sem que ele o visse, de
sorte que eu estivesse preparado para qualquer outro alvo que se pudesse
apresentar; mas nada se me ofereceu naquele momento: então, carreguei o
filhote de bode para casa e, na mesma noite, esfolei"-o e cortei"-o da
melhor maneira; e, dispondo de uma panela para tal fim, fervi ou cozi um
pouco da carne e preparei um caldo que resultou muito saboroso; e depois
de ter comido um pouco, ofereci outro tanto ao meu companheiro, que
parecia muito feliz com a oferta e a saboreou bastante; o que se lhe
mostrava mais estranho, porém, era ver"-me usar de sal na carne. Com
gestos, ele sugeriu"-me que o sal não era coisa boa de se comer; e,
levando um pouco à própria boca, mostrou"-se como que nauseado e
engasgado e cuspiu, lavando a boca em seguida com água fresca; de minha
parte, coloquei um pouco de carne sem sal na boca e fingi cuspir e
engasgar por falta do mesmo, tão rápido quanto ele o havia feito com o
sal; mas de nada serviu; ele nunca gostou de usar sal na carne ou no
caldo; ao menos, não por um bom tempo, e posteriormente sempre em
quantidade muito pequena.

Depois de alimentá"-lo de caldo e carne cozida, resolvi oferecer"-lhe um
banquete no dia seguinte e assei uma posta da carne do bode; e assim o
fiz pendurando"-o diante do fogo num pedaço de barbante, como havia visto
ser costume de muitas pessoas na Inglaterra, fincando dois esteios, um
de cada lado do fogo, e sobre ambos um terceiro, na horizontal, ao qual
se amarra o barbante, fazendo a carne girar o tempo todo. Sexta"-Feira
admirou"-se muitíssimo do preparo; mas quando veio a provar a carne, ele
usou de tantos meneios para dar expressão do quanto havia gostado que
não fui capaz de compreendê"-lo; e por fim me deu a entender, tão bem
quanto pôde, que nunca mais comeria da carne de um homem, o que folguei
muito em saber.

No dia seguinte, coloquei"-o a trabalhar no bater e peneirar dos grãos
tal como eu costumava fazer, do que já tratei noutro momento; e ele não
tardou a compreender como fazê"-lo e passou a fazer tão bem quanto eu,
especialmente depois de ver a razão do trabalho, isto é, o fazimento da
massa do pão e depois sua assadura; e em pouco tempo Sexta"-Feira
mostrou"-se apto a fazer todo o trabalho em meu lugar e tão bem quanto eu
o fazia.

Passei, então, a considerar que, havendo duas bocas para alimentar em
vez de uma, era preciso dar mais terreno a minha plantação e semear uma
quantidade de grão maior do que a de costume; e demarquei, assim, um
pedaço maior de terra e comecei a cercá"-lo como o fizera antes, no que
Sexta"-Feira trabalhou não só com muita disposição e diligência, mas
também com muita satisfação: e eu lhe disse a razão de fazê"-lo; que era
para a cevada e, com ela, fazer mais pão, porque ele agora vivia comigo,
e dessa forma teríamos o bastante para ambos. Ele demonstrou
entendimento disso e me deu a saber que julgava que eu vinha trabalhando
muito mais por sua presença do que o faria por mim mesmo; e que ele
trabalharia por mim com mais afinco, se eu lhe dissesse o que fazer.

Aquele foi o ano mais agradável que vivi na ilha. Sexta"-Feira começou a
falar muito bem e a compreender os nomes de quase tudo o que porventura
lhe pedisse, e de todos os lugares aonde tinha de enviá"-lo, e conversava
bastante comigo; de maneira que, em suma, tornei a usar minha língua,
para o que, de fato, eu tinha antes muito pouca ocasião; com o que quero
dizer, \emph{para falar}. Além do prazer de conversar com ele, o próprio
sujeito me trazia por si só contentamento; sua honestidade simples e
genuína ficava mais evidente para mim a cada dia; e de fato comecei a
amar a criatura; crendo eu que, de sua parte, ele me amou mais do que
lhe fora possível amar qualquer coisa antes.

Certa feita, tive a ideia de perguntar"-lhe se tinha qualquer tenção
desejosa de retornar a sua terra; e, tendo lhe ensinado inglês tão bem
que ele se tornara capaz de responder"-me quase qualquer pergunta,
perguntei"-lhe se a nação a que pertencia nunca vencia batalhas; ao que
ele sorriu e disse ``Sim, sim, sempre melhor''; isto é, querendo dizer
que sempre levavam a melhor na luta; e assim iniciamos o seguinte
diálogo: ``Vocês semprem lutam melhor; como você foi capturado,
Sexta"-Feira?''

\emph{Sexta"-feira.} Minha nação derrota bastante, mesmo assim.

\emph{Senhor.} Como derrota? Se sua nação os derrota, como você foi
capturado?

\emph{Sexta"-Feira.} Eles é muito mais que a minha nação no lugar que eu
vem; eles leva um, dois, três e eu: minha nação derrota mais eles no
outro lugar, mas eu não lá; lá minha nação leva um, dois, milhares.

\emph{Senhor.} Mas por que seu lado não o resgatou das mãos de seus
inimigos?

\emph{Sexta"-feira.} Eles corre, um, dois, três e eu, e vai embora canoa;
minha nação não tem canoa agora.

\emph{Senhor.} Bem, Sexta"-Feira, e o que sua nação faz com os homens que
levam? Ela os leva presos e os come, como eles?

\emph{Sexta"-feira.} Come também; come tudo.

\emph{Senhor.} Para onde eles são levados?

\emph{Sexta"-feira.} Para outro lugar, onde eles acha melhor.

\emph{Senhor.} Eles vêm aqui?

\emph{Sexta"-feira.} Sim, sim, vem; vem para outro lugar.

\emph{Senhor.} Você já esteve aqui com eles?

\emph{Sexta"-feira.} Sim, esteve (\emph{aponta ao lado noroeste da ilha,
que, ao que parece, era o lado deles}).

Com isso, entendi que Sexta"-Feira estivera entre os selvagens que
costumavam vir à praia nas bandas mais distantes da ilha, nas mesmas
ocasiões de comer homens em que ele viera preso; e algum tempo depois,
quando tomei coragem de levá"-lo àquele sítio, sendo o mesmo que
mencionei anteriormente, ele conhecia o local e disse"-me que estivera lá
uma vez quando comeram vinte homens, duas mulheres e um criança; ele não
sabia contar até vinte em inglês, mas o fez enfileirando o mesmo número
de pedras e gesticulando para que eu as contasse.

Dei relação ao caso por isto: porque, depois do diálogo que travei com
ele, perguntei"-lhe a que distância ficava a costa de nossa ilha; e se as
canoas não naufragavam com frequência; e ele me disse que não havia
perigo, tampouco naufrágios; e que depois de adentrar um tanto o mar,
havia uma corrente e um vento que sempre se comportavam de um jeito pela
manhã e de outro à tarde.

Quanto a isso, entendi que isso não fosse mais do que os conjuntos da
maré, a vazante e a cheia; mas depois compreendi que elas eram
ocasionadas pelo grande fluir e refluir do grande rio Orinoco; na foz ou
golfo de cujo rio, como vim a descobrir noutra ocasião, nossa ilha
repousava; e que a terra que eu julgava estar a oeste e noroeste era a
grande ilha de Trinidad, a norte da foz do rio. Fiz a Sexta"-Feira mil
perguntas sobre a terra, as gentes, o mar, a costa, e as nações
próximas; e ele me disse tudo o que sabia com a maior franqueza
imaginável. Perguntei"-lhe os nomes das várias nações da sua gente, mas
não compreendi outro nome além de \emph{caribs}; donde eu entendi sem
maior dificuldade que se tratava do povo caraíba, que nossos mapas
colocam em terras de América e que se espalha da foz do rio Orinoco até
a Guiana e para além desta a Santa Marta. Ele me disse que para muito
além da lua, isto é, para além do ponto em que a lua se punha, que devia
ser a oeste de seu país, habitavam homens brancos e barbudos, como eu, e
apontou para minhas compridas suíças, que mencionei antes; e que eles
haviam matado \emph{muito home}, sendo essas suas palavras: ao que
entendi que ele se referia aos espanhóis, cujas crueldades na América se
haviam estendido por todo o país e eram lembradas por todas as nações de
geração em geração.

Perguntei"-lhe se ele podia me dizer como eu poderia deixar a ilha e
chegar aos homens brancos; ao que respondeu que podia, sim, sim, que eu
conseguiria ir \emph{em dois canoa}; não compreendi o que ele quis dizer
com isso, nem conseguir fazer com que me descrevesse o que entendia por
\emph{dois canoa}, até que com enorme esforço inferi que ele
provavelmente se referia a um enorme navio, grande como \emph{dois
canoa}.

Essa parte das palavras de Sexta"-Feira me deixou muito contente; e, a
partir desse momento, cultivei alguma esperança de que em algum momento
encontraria uma oportunidade de escapar daquele lugar, e que aquele
pobre selvagem poderia servir de meio para tanto.

Durante o longo período em que Sexta"-Feira já vivia comigo, e no qual
começara a conversar comigo e a me compreender, eu não vi necessidade de
estabelecer uma base de conhecimento religioso em sua mente;
especificamente, perguntei"-lhe certa feita quem o havia criado? A pobre
criatura não me entendeu, pois pensou que eu lhe perguntara quem era seu
pai; mas usei de outra ideia, e perguntei"-lhe quem havia feito o mar, o
chão em que andávamos e as colinas e matas; e ele me disse que havia
sido um Benamuque, que vivia além de todas as coisas; não conseguindo
descrever o que fosse dessa grande pessoa, senão que era muito velho;
muito mais velho, disse ele, do que o mar ou a terra, que a lua ou as
estrelas. Perguntei"-lhe, então, se essa pessoa antiga havia feito todas
as coisas, por que todas as coisas não o adoravam; ao que ele pareceu
muito sério e, com um olhar de perfeita inocência, respondeu: ``Todas as
coisas dizem Ó para ele''. Perguntei"-lhe se as pessoas que morrem em seu
país partiam para algum lugar; ao que disse que sim; que todas iam a
Benamuque; e então eu lhe perguntei se os que eles comiam também iam ter
com ele. Ele disse que sim.

A partir daí, comecei a instruí"-lo no conhecimento do verdadeiro Deus; e
disse"-lhe que o grande Criador de todas as coisas vivia lá em cima,
apontando ao céu; que ele governava o mundo segundo o mesmo Poder e
Providência com os quais o havia criado; que ele era onipotente, e podia
fazer tudo por nós e dar"-nos e tirar"-nos tudo; e assim, aos poucos,
abri"-lhe os olhos. Ele ouviu com grande atenção e recebeu com satisfação
a ideia de que Jesus Cristo fora enviado para nos redimir; e o modo com
que fazíamos nossas orações a Deus, e o fato de ele nos poder ouvir,
mesmo no céu. Um dia ele me disse que, se nosso Deus nos conseguia ouvir
de além do sol, ele certamente era um Deus maior do que o seu Benamuque,
que não vivia tão distante e mesmo assim não os conseguia ouvir até que
subissem às grandes montanhas onde ele habitava para falar com ele; e eu
lhe perguntei se ele fora até lá para falar com ele, e ele respondeu que
não, que nunca os jovens iam, apenas os velhos, os quais chamou de
``Ouocaque''; isto é, como explicou ao ser solicitado, seus religiosos
ou clérigos; e que eles saíam para dizer ``Ó'' (assim chamava suas
preces) a Benamuque; retornando depois com o que Benamuque lhes havia
dito. Dessa maneira, percebi que há sacerdócio mesmo entre os mais cegos
e ignorantes pagãos do mundo; e a postura de fazer da religião um
segredo, a fim de preservar a veneração do povo ao clero, não é
exclusividade romana, mas talvez presente em todas as religiões do
mundo, mesmo entre os mais brutais e bárbaros Selvagens.

Cuidei de esclarecer essa fraude a Sexta"-Feira; e disse"-lhe que o
pretexto dos velhos de seu povo de subirem às montanhas para dizer Ó a
seu deus Benamuque era falso; e as supostas palavras que traziam de lá
consigo, ainda mais falsas; que, no caso de terem se deparado ali com
qualquer resposta ou tratado com qualquer um, decerto que teria sido com
um espírito maligno; e então iniciei uma longa conversa com ele sobre o
diabo, sua origem, sua rebelião contra Deus, sua inimizade com o homem,
a razão disso, seu estabelecimento nos recantos sombrios do mundo para
ser adorado no lugar de Deus, e como Deus; e sobre os muitos artifícios
de que se valeu para iludir a humanidade e levá"-la à ruína; e sobre como
ele tinha acesso secreto a nossas paixões e afetos, por meio do qual
adaptava suas armadilhas de tal guisa às nossas inclinações que nos
tornávamos nossos próprios tentadores e caminhávamos à ruína com nossas
próprias pernas.

Descobri que não era tão fácil imprimir noções corretas em sua mente
sobre o diabo, como fora sobre o ser de um Deus. A natureza me havia
auxiliado em todos os meus argumentos para demonstrar o bem, até mesmo a
necessidade de uma grande Primeira Causa, de um poder governante sobre
todas as coisas; a de uma Providência secreta que a tudo dava rumo e da
equidade e justiça de prestar homenagem àquele que nos criou, e de
coisas que tais; mas não havia nada desse tipo que demonstrasse a noção
de espírito maligno; de sua origem, seu ser, sua natureza e, acima de
tudo, sua inclinação a praticar o mal e a seduzir"-nos de maneira a
praticá"-lo; e a pobre criatura me intrigou de tal maneira, por uma
pergunta meramente natural e inocente, que mal soube o que lhe dizer. Eu
havia tratado muito com ele sobre o poder de Deus, sua onipotência, sua
natureza aversa ao pecado, seu ser como fogo que consome\footnote{Deuteronômio
  4:24 {[}\textsc{n.\,e.}{]}} para os praticantes do mal; sobre como, ao nos ter
criado a todos, tinha o poder de destruir a nós e a todo o mundo em um
instante; e ele me ouviu com muita seriedade o tempo todo.

Depois disso, contei"-lhe como o diabo era inimigo de Deus no coração dos
homens, e usava de toda a sua malícia e astúcia para ludibriar os bons
desígnios da Providência e arruinar o reino de Cristo na terra, e coisas
assim. ``Bem'', diz Sexta"-Feira, ``mas você dizer que Deus ser tão
forte, tão grande; ele não ser muito forte, muito mais forte que o
diabo?'' ``Sim, sim'', disse eu, ``Sexta"-Feira; Deus é mais forte que o
diabo; Deus está acima do diabo, e, por isso, oramos a Deus para
esmagá"-lo debaixo dos nossos pés e nos permitir resistir às tentações
dele e apagar seus dardos inflamados.''\footnote{Romanos 16:20 e Efésios
  6:16 {[}\textsc{n.\,e.}{]}} ``Mas'', retrucou ele, ``se Deus ser muito mais
forte, muito mais que o diabo, por que Deus não matar o diabo e assim
não deixar mais ele fazer o mal?''

A pergunta me causou estranha surpresa; e, apesar de tudo, embora já
tivesse eu idade, era um professor inexperiente e pouco qualificado para
ser um casuísta ou solucionador de dificuldades da consciência; e de
pronto não soube o que lhe dizer; e então fingi não o ter ouvido e
perguntei"-lhe o que ele havia dito; mas ele queria muito uma resposta
para se ter esquecido da pergunta e a repetiu com as mesmas palavras
desajeitadas acima. A essa altura, eu já me havia recuperado um pouco e
disse: ``Deus por fim o punirá severamente; a ele é reservado o juízo e
há de ser lançado na cova sem fundo para habitar no fogo eterno.''
Sexta"-Feira não ficou satisfeito e respondeu repetindo minhas palavras:
``\textsc{Reservado, por fim,} eu não entender; mas por que não matar o
diabo agora; não matar tempo antes?'' ``Você também pode me perguntar'',
respondi eu, ``por que Deus não mata a você ou a mim, quando fazemos
coisas más que o ofendem; somos poupados para nos arrepender e ser
perdoados''; ao que ele refletiu um pouco sobre isso. ``Bem, bem'',
disse ele, muito carinhosamente, ``muito bem; então você, eu, diabo,
todos pecador, todos poupado, todos arrependido, Deus perdoa tudo''.
Neste ponto, ele me deixou estupefato; e isso foi para mim um testemunho
de como ideias mais básicas inerentes à natureza, embora guiem criaturas
dotadas de razão ao conhecimento de um Deus e a um justo culto ou
respeito ao ser supremo de Deus como consequência de nossa natureza; não
obstante tais ideias, nada senão a revelação divina pode formar o
conhecimento de Jesus Cristo e de uma libertação que nos é dada por seu
intermédio; de um Mediador da Nova Aliança e de um Intercessor no
escabelo do Trono de Deus;\footnote{Cf. Hebreus 9:11 a 10:22 {[}\textsc{n.\,e.}{]}}
pois bem: nada além de uma revelação do céu pode formar o conhecimento
dessas coisas na alma; e que, portanto, o Evangelho de nosso Senhor e
Salvador Jesus Cristo; quero dizer, a Palavra de Deus e o Espírito de
Deus, prometido para ser guia e santificador de seu povo, são os
instrutores absolutamente necessários das almas dos homens ao
conhecimento salvador de Deus e aos meios da Salvação.

Ditas essas coisas, encerrei o referido diálogo entre mim e meu criado,
levantando"-me às pressas, como tivesse sido acometido da repentina
necessidade de sair; em seguida, ordenando"-lhe que se alongasse de casa
em demanda de qualquer coisa, orei contritamente a Deus que me desse
capacidade de instruir o pobre selvagem no caminho da salvação; que
assistisse com seu Espírito o coração da pobre e ignorante criatura a
receber a luz do conhecimento de Deus em Cristo, reconciliando"-o consigo
mesmo; e que me guiasse a tratar com ele a partir da Palavra de Deus, de
guisa que sua consciência pudesse ser convencida, seus olhos abertos, e
sua alma salva. Quando ele retornou, iniciei longo discurso sobre o
problema da redenção do homem pelo Salvador do mundo e a doutrina do
Evangelho que os céus pregam, a saber, de arrependimento perante Deus e
fé em nosso abençoado Senhor Jesus. Assim, tão bem quanto pude, eu lhe
expliquei a razão de nosso Redentor assumir para si não a natureza dos
anjos, mas a semente de Abraão; e como, por essa razão, os anjos caídos
não tiveram sua parte na redenção; e que ele veio somente às ovelhas
perdidas da casa de Israel,\footnote{Mateus 10:6 {[}\textsc{n.\,e.}{]}} e coisas
do gênero.

Eu tinha, como Deus bem sabia, mais sinceridade do que conhecimento em
todos os métodos de que me vali para a instrução dessa pobre criatura e
devo reconhecer o que acredito ser visto de todos os que agem sob o
mesmo princípio, a saber, que, enquanto lhe esclarecia as matérias, eu
em verdade me informava e instruía de muitas coisas que não sabia ou
sobre as quais não havia ponderado anteriormente; mas que ocorriam
naturalmente a minha mente enquanto as procurava de guisa a dar
conhecimento ao pobre selvagem; e mais entusiasmo senti em minha
investigação das coisas com esse propósito do que jamais sentira até
então; de sorte que, a despeito de ter ou não se tornado melhor por meu
intermédio, fato é que tinha eu grandes motivos de ser grato por seu
advento; minha tristeza se fizera mais leve; minha morada, infinitamente
mais confortável; e quando olhava para isso à luz da vida solitária em
que estivera eu aprisionado, eu não apenas me comovia a mirar ao céu,
como procurasse a mão que me havia conduzido até lá, como me tornara
instrumento de sua Providência para o salvamento da vida e, tanto quanto
sabia, da alma de um pobre selvagem, e instrução deste no verdadeiro
conhecimento da religião e da doutrina cristã, a fim de que conhecesse
Jesus Cristo, em quem a vida é eterna; Pois bem: quando refletia sobre
todas essas coisas, sentia uma alegria secreta percorrer todas as partes
da minha alma, e muitas vezes me alegrava por ter sido levado àquele
lugar, que tantas vezes pensei ter sido o maior dentre todos os
padecimentos que me poderiam haver acometido.

Nessa grata disposição permaneci por todo o meu tempo restante; e o
diálogo em que Sexta"-Feira e eu empregávamos nossas horas era tal que os
três anos em que vivemos juntos foram perfeita e inteiramente felizes,
\emph{se algo como a felicidade completa se dá a conhecer no mundo
sublunar}. O Selvagem tornara"-se um bom cristão, um cristão muito melhor
que eu era; embora eu tenha motivos para crer, e agradeço a Deus por
isso, que éramos igualmente penitentes, recuperados e confortados; tendo
a Palavra de Deus para ler, e não estando mais distantes da instrução de
seu Espírito quanto estaríamos na Inglaterra.

Ao ler as Escrituras, sempre cuidei que chegasse a seu entendimento,
tanto quanto era capaz de fazê"-lo, o significado do que lia; e ele, pela
seriedade de suas perguntas e questionamentos, tornou"-me, como contei
atrás, um estudioso muito mais zeloso do conhecimento das Escrituras do
que jamais teria sido por minha mera leitura pessoal. Outra coisa que
também não posso deixar de observar aqui, derivada da experiência nessa
quadra solitária de minha vida, é isto: que graça infinita e
inexprimível é que o conhecimento de Deus e a doutrina da salvação por
Cristo Jesus estejam tão claramente expostos na Palavra de Deus e possam
ser tão facilmente recebidos e compreendidos que, da mesma forma que a
simples leitura da Bíblia me fez capaz de entender bastantemente o meu
dever, de me levar sem desvios à grande obra do arrependimento sincero
por meus pecados, e de me agarrar a um Salvador pela vida e salvação, a
uma reforma instituída em meus atos e à obediência a todos os
mandamentos de Deus, e isso sem qualquer professor ou instrutor (quero
dizer, humano), a mesma simples instrução serviu suficientemente ao
esclarecimento dessa criatura selvagem e a torná"-la um ser cristão tal
como poucos conheci em minha vida.

Quanto a todas as disputas, contendas, lutas e enfrentamentos que
tiveram lugar no mundo em torno da religião, quer se tratassem de
sutilezas nas doutrinas ou de maquinações do governo da Igreja, elas nos
eram de todo inúteis; assim como, segundo meu entendimento, elas o são
para o resto do mundo. Tínhamos o guia correto para o céu, a saber, a
Palavra de Deus; e tínhamos, pela graça de Deus, reconfortante
entendimento do Espírito de Deus, que nos ensinava e instruía por sua
palavra e nos conduzia a toda a verdade e nos tornava aptos e obedientes
à instrução de sua palavra; e, caso estivesse a nosso alcance, não vejo
que utilidade haveria para nós um maior conhecimento das questões de
contenda religiosa que tantos problemas causaram ao mundo; mas devo
prosseguir com a relação das coisas históricas, cada qual segundo sua
ordem.

Depois de Sexta"-Feira e eu nos termos tornado mais conhecidos um do
outro, e de ele ter passado a entender quase tudo o que lhe dizia e
falar com bastante desembaraço, embora num inglês para mim ruim, fi"-lo
conhecer minha própria história, ou pelo menos quanto dela se
relacionava à minha chegada àquele lugar, como vivera ali e por quanto
tempo; e iniciei"-o nos mistérios, pois assim ele os compreendia, da
pólvora e da bala, e o ensinei a atirar; e dei"-lhe uma faca, com a qual
ficou maravilhosamente encantado; e lhe fiz um cinto, com uma alça nele
pendurado, do mesmo tipo que usamos na Inglaterra para o porte de
adagas; mas para a alça, em lugar de uma adaga, dei"-lhe um machado, que
não apenas era uma boa arma em algumas circunstâncias, como bastante
útil em outras ocasiões.

Descrevi"-lhe a terra de Europa, em particular a de Inglaterra, de onde
viera; nossas maneiras de viver, de adorar a Deus, de nos comportarmos
uns com os outros e de fazermos comércio em navios em todas as partes do
mundo; e dei"-lhe relação do afundamento do navio em que estava e
mostrei"-lhe, tão perto quanto podia chegar, o local onde haviam estado
seus destroços; que já haviam sido feitos em pedaços e levados pelas
águas.

Mostrei"-lhe o que restava de nosso bote, que se perdeu enquanto
escapamos e que não tinha força para mover então, mas que já se
encontrava quase destroçado; e ao ver o bote, Sexta"-Feira se pôs um bom
tempo pensativo, e nada disse; mas quando perguntei"-lhe em que pensava,
ele disse por fim: ``Eu ver bote assim chegar em lugar na minha nação''.

Por um bom tempo não o compreendi; mas, por fim, quando examinei mais a
fundo o que ele havia dito, depreendi que um bote, tal como aquele,
havia dado à costa da terra onde ele habitava; isto é, como ele
explicou, que fora arrastado até lá pelas intempéries do tempo. Eu
imaginei então que algum navio europeu tivesse naufragado nas imediações
de sua costa, e o bote houvesse se desprendido da embarcação e tido em
terra; mas fui tão desatento que não pensei em homens que, sobrevivendo
a um naufrágio, tivesse rumado para lá, e muito menos de onde poderiam
ter vindo; assim, só lhe pedi uma descrição do bote.

Sexta"-Feira fez uma boa descrição do bote; mas fez com que eu mais bem o
compreendesse quando acrescentou, com alguma emoção, ``Nós salvar homes
branco da água''. Então eu perguntei se havia no bote algum \emph{homes
branco}, como ele os chamava; ao que ele respondeu que havia, que ``o
bote cheio homes branco''; e quando quis saber quantos, ele os contou
nos dedos em número de dezessete. Perguntei"-lhe então o que havia
acontecido com eles; e ele me disse: ``Eles viver, eles morar minha
nação''.

Isso trouxe novos pensamentos à minha cabeça; pois então imaginei que
aqueles pudessem ser os homens pertencentes ao navio que se perdera na
proximidade de minha ilha, como eu a chamava agora; e que, depois de o
navio arremeter sobre os recifes, e eles compreenderem a fatalidade que
os vitimava, salvaram"-se em seu bote e deram naquela praia selvagem
entre seus habitantes.

Diante disso, perguntei"-lhe mais criticamente o que se havia passado com
eles. Ele me assegurou que ainda viviam lá; que lá habitavam havia perto
de quatro anos; que os selvagens não os atacaram e lhes forneceram
mantimento. Perguntei"-lhe a razão de eles não os terem matado e comigo;
ao que ele respondeu, ``Não, eles fizeram irmão com eles''; isto é, tal
como eu o compreendi, uma trégua; e então acrescentou: ``Eles comer home
só quando lutar guerra''; isto é, eles comem apenas os homens que vêm
para lutar contra eles e são capturados em batalha.

Foi depois desse tempo considerável que, estando no cimo da colina, nas
bandas de leste da ilha, de onde, como disse atrás, em um dia claro, fiz
o avistamento de terra firme, ou o continente de América; Sexta"-Feira,
estando o tempo bastante ameno, mira muito seriamente ao continente e,
transportado de espanto, põe"-se a pular e dançar e me chama, pois eu
estava a alguma distância dele. Perguntei"-lhe: ``O que se passa?'' ``Oh,
alegria!'', diz ele; ``Oh, alegria! Ali minha terra, minha nação!''

Notei uma extraordinária sensação de felicidade em seu rosto, e seus
olhos brilhavam, e seu semblante revelava uma estranha ânsia, como
cultivasse em seus pensamentos o desejo de retornar a sua terra; e essa
minha observação suscitou"-me pensamentos vários, os quais me deixaram
não tão à vontade com meu Sexta"-Feira quanto estava; pois não tinha
dúvida de que, se Sexta"-Feira pudesse retornar à sua nação, ele
abandonaria toda a sua religião, assim como todas as suas obrigações
para comigo; e estaria inclinado a dar a seus compatriotas um relato a
meu respeito e talvez retornar, com cem ou duzentos deles, e me devorar
num banquete no qual poderia encontrar a satisfação que costumava sentir
nos de seus inimigos capturados em guerra.

Mas fiz muito mau juízo da pobre e honesta criatura, pelo que lamentei
muito depois. No entanto, tendo minhas suspeitas aumentado e durado
algumas semanas, mostrei"-me um pouco mais prevenido e não tão próximo e
gentil como antes; no que também fazia mau juízo, uma vez que a honesta
e grata criatura não tinha pensamentos outros, senão os que consistiam
nos melhores princípios, fossem os de cristão religioso, fossem os de
amigo agradecido, como depois se revelou para minha absoluta satisfação.

Enquanto minha desconfiança durou, tenha certeza de que o pressionei com
perguntas diárias com o intuito de descobrir algum de seus novos
pensamentos, os quais tinha para mim que os cultivava; mas tudo o que me
disse era tão honesto e inocente que nada encontrei com que alimentasse
minhas suspeitas; e apesar de toda a minha inquietação, ele por fim
voltou a ganhar minha confiança; tampouco ele percebeu que eu estava
desconfortável e, portanto, eu não podia suspeitar de que me enganava.

Um dia, subindo a mesma colina, mas com o mar coberto de névoa, de forma
que não podíamos ver o continente, chamei"-o e disse: ``Sexta"-Feira, você
não deseja estar em seu próprio país, em sua própria nação?'' ``Sim'',
disse ele, ``ficar feliz por estar em minha nação.'' ``O que você faria
lá?'', perguntei eu. ``Você tornaria a ser selvagem, comer carne de
homem e se comportar como o bárbaro que era?'' Ele pareceu preocupado e,
balançando a cabeça, disse: ``Não, não, Sexta"-Feira dizer eles para
viver bem; para orar Deus; dizer comer pão de grão, carne de gado,
leite; home não.'' ``Ora'', disse"-lhe eu, ``eles vão matar você.'' Ele
se mostrou sério ante minha afirmação e disse: ``Não, não, eles não
matar, eles desejar aprender amor'', com o que quis dizer que estariam
dispostos a aprender; e acrescentou que eles aprenderam muito dos homens
barbudos que chegaram no bote; ao que perguntei se ele iria voltar; e
ele sorriu e me disse que não conseguiria nadar tão longe. Eu lhe disse
que faria uma canoa para ele; ao que ele respondeu que iria desde que eu
fosse com ele. ``Eu ir?'', disse eu; ``ora, eles vão me devorar se eu
chegar lá?'' ``Não, não'', disse ele, ``eu fazer eles não comer você; eu
fazer eles amar muito''; com o que quis dizer que ele lhes contaria como
eu havia matado seus inimigos e lhe salvado a vida, e assim ele faria
com que me amassem. Então ele me contou, tão bem quanto era capaz, da
gentileza com que eram tratados os dezessete homens brancos, ou
barbudos, como ele os chamava, que haviam dado naquela costa em
dificuldade.

A partir de então, confesso, tive em grande vontade aventurar"-me e ver a
possibilidade de me juntar àqueles homens barbudos, que sem dúvida eram
espanhóis e portugueses; e com a certeza de que, se eu lá chegasse,
poderíamos encontrar algum meio de escapar dali, estando no continente,
e unidos em grande contingente; melhor do que eu, sozinho e sem
assistência, partindo de uma ilha a quarenta milhas da costa. Então,
depois de alguns dias, levei Sexta"-Feira mais uma vez ao trabalho,
valendo"-me de conversação, e disse a ele que lhe daria um bote para
retornar à sua nação; e, assim, levei"-o a minha fragata, que deixava do
outro lado da ilha, e depois de esvaziá"-la da água, porque eu a
conservava debaixo dela, trouxe"-a à superfície, apresentei"-lha, e nela
embarcamos.

Descobri que ele era muito hábil na condução da canoa, capaz de
navegá"-la tão prestesmente quanto eu; então, quando ele embarcou, eu lhe
perguntei: ``Pois então, Sexta"-Feira, iremos à sua nação?'' Senti que a
pergunta o havia aborrecido bastante; aparentemente porque achava a
canoa pequena demais para ir tão longe. Eu então lhe disse que tinha uma
maior; e, no dia seguinte, fui ao local onde estava a primeira canoa que
havia fabricado, mas que não conseguira levar à água. Ele disse que era
bastante grande; mas como não havia dispendido cuidados a ela, e lá ela
permanecera vinte e dois ou vinte e três anos, o sol a rachara e secara,
estando em certa medida podre. Sexta"-Feira me disse que uma canoa como
aquela serviria muito bem e carregaria ``muito mantimento, bebida,
pão'', sendo esse seu modo de falar.

Minha disposição geral, à época, de partir ao continente com meu
selvagem era tamanha que lhe disse que faríamos uma canoa tão grande
quanto aquela e que ele retornaria com ela. Ele não respondeu uma
palavra, demonstrando"-se muito grave e triste; e tendo"-lhe, então,
perguntado de que padecia, ele retornou a pergunta: ``Você muito bravo
com Sexta"-Feira, o que fez eu?'' Disse"-lhe que não entendia o que queria
dizer, pois não estava bravo com ele. ``Não raiva!'', disse ele,
repetindo as palavras várias vezes; ``Por que mandar Sexta"-Feira para
minha nação?'' ``Ora, Sexta"-Feira'', respondi eu, ``você não disse que
desejava estar lá?'' ``Sim, sim'', diz ele, ``deseja os dois lá; não
Sexta"-Feira lá sem Senhor''. Em suma, ele não tinha a intenção de partir
sem mim. ``Eu lá, Sexta"-Feira?'', retorqui eu; ``mas o que farei lá?''
Ele retrucou de imediato. ``Você fazer muito bem'', respondeu ele;
``ensinar home selvagem ser bom, sóbrio, home manso; dizer eles conhecer
Deus, orar Deus e viver vida nova.'' ``Ai, Sexta"-Feira!'', exclamei eu:
``não sabes o que dizes; sou apenas um homem ignorante.'' ``Não, não'',
disse ele, ``você ensinar bem Sexta"-Feira, você ensinar bem nação.''
``Não, não, Sexta"-Feira'', respondi eu, ``você irá sem mim e me deixará
aqui vivendo sozinho, como era antes.'' Ele pareceu novamente confuso
com essas últimas palavras, e correndo a uma das machadinhas que ele
costumava usar, pegou"-a e a entregou a mim sem demora. ``O que devo
fazer com isso?'', perguntei"-lhe. ``Você pegar, matar Sexta"-Feira'',
disse ele. ``Por que devo matar você?'', quis saber, ao que ele me
respondeu no mesmo instante, ``Por que você mandar Sexta"-Feira embora?
Pegar, matar Sexta"-Feira, não mandar Sexta"-Feira embora.'' Isso ele
disse com tamanha sinceridade que vi lágrimas brotarem em seus olhos. Em
suma, declarou"-se ali para mim um tão profundo e firme sentimento, que
lhe disse então, e muitas vezes depois, que nunca o mandaria embora se
ele estivesse disposto a ficar comigo.

No geral, depreendendo de suas palavras um sólido afeto por mim, e que
nada o poderia separar de mim, concluí que todo o fundamento de seu
desejo de retornar a sua província residia em seu ardente amor por seu
povo, e as esperanças que tinha de que eu lhe fizesse bem; algo que, não
tendo sido concebido por mim, não tinha a menor ideia, intenção ou
desejo de empreender. Contudo, encontrava eu ainda forte inclinação para
tentar a escapada, a qual baseava nas palavras de Sexta"-Feira, isto é,
nos dezessete homens barbudos que lá viviam; e, portanto, sem mais
demora, saí com Sexta"-Feira para encontrar uma árvore grande, apropriada
ao corte e à fabricação de uma grande piroga ou canoa para empresa da
viagem. Havia árvores bastantes na ilha para a construção de uma pequena
frota, não de pirogas ou canoas, mas mesmo de boas e grandes
embarcações; mas a principal coisa a que prestei atenção foi encontrar
uma tão próxima à água que a pudéssemos lançar quando estivesse pronta,
evitando o erro que cometi na primeira oportunidade.

Por fim, Sexta"-Feira escolheu uma árvore, pois descobri que ele sabia
muito melhor do que eu que casta de madeira era mais adequada para a
tarefa; e tampouco sei dizer até hoje qual o nome da casta de árvore que
cortamos, exceto por se assemelhar bastante com a árvore a que chamamos
taiúva, ou algo entre ela e uma espécie de pau"-brasil, pois era da mesma
cor e cheiro. Sexta"-Feira queria abrir na árvore a cavidade ou oco com
fogo para assim fazer um barco, mas eu mostrei"-lhe como abri"-lo com
ferramentas; o que, depois que eu ter"-lhe demonstrado como fazer, ele
executou com muita destreza; e em obra de um mês de duro obrar, a
concluímos, tendo"-a feito muito elegante e conveniente; especialmente
quando, com nossos machados, os quais lhe ensinei como manejar, talhamos
e cortamos a parte exterior nas verdadeiras formas de um barco; depois
disso, porém, levamos praticamente quinze dias para levá"-la à água
rolando"-a sobre grandes troncos, como avançássemos polegada por
polegada; mas quando lá chegou, teria levado vinte homens com muita
facilidade.

Quando posta na água, fiquei impressionado ao ver com que destreza e
rapidez meu Sexta"-Feira tinha o poder de governá"-la, manobrá"-la e
remá"-la, não obstante seu tamanho; e assim perguntei se ele faria a
travessia e se nós conseguiríamos navegar com ela; ao que ele respondeu,
``Conseguir, atravessar muito bem mesmo vento grande.'' Eu tinha, porém,
um segundo projeto, do qual ele nada sabia, isto é, fazer um mastro e
uma vela e equipá"-la com âncora e cabo. Quanto a um mastro, era muito
fácil consegui"-lo; escolhi um cedro jovem e reto, que encontrei perto do
local e que havia em abundância na ilha, e pus Sexta"-Feira a cortá"-lo,
dando"-lhe instruções sobre como modelá"-lo e adequá"-lo. Quanto à vela,
essa era incumbência minha; eu sabia ter velhas velas, ou melhor,
pedaços de velas velhas, em número o suficiente; mas como eu as tinha
havia, àquelas alturas, vinte e seis anos, e não fora muito cuidadoso em
sua conservação, sem que imaginasse que algum dia lhes pudesse dar uso,
como a situação sugeria, estava certo de que haviam apodrecido; o que se
confirmou em grande medida, exceto por dois pedaços que me pareceram em
muito bom estado e com eles me coloquei a trabalhar; o que sucedeu não
sem enorme esforço, e uma costura desajeitada e dificultosa, disso
tenham certeza, dada a carência de agulhas; do que resultou uma coisa
horrível de três pontas, como a que na Inglaterra chamamos costela de
carneiro, ou bujarrona, a levar a retranca na base e uma espicha curta
no topo, assim como em geral os escaleres que acompanham nossos navios e
eu tão bem sabia governar, pois não era dessemelhante a que tinha no
barco em que escapei da Berbéria, como relacionado na primeira parte da
minha história.

Fiquei obra de dois meses realizando esta última tarefa, a saber,
amarrar e ajustar meus mastros e velas; pois eu a conclui muito
completamente, fabricando um pequeno estai, e uma vela de traquete para
o auxílio do aparelho caso tivéssemos de virar a barlavento; e, além de
tudo isso, fixei um leme na popa para o governo do barco; e embora eu
fosse um um construtor naval desajeitado, conhecia a utilidade e até a
necessidade de uma peça como aquela e me esforcei tanto para fazê"-la
que, enfim, logrei; ainda que, considerando os muitos projetos ruins que
executei e fracassaram, creio que me tenha custado quase tanto trabalho
quanto fabricar o barco.

Com tudo isso feito, ensinei ao meu Sexta"-Feira o que competia à
navegação do barco; pois embora soubesse muito bem remar uma canoa, nada
sabia do concernente a uma vela e um leme; e ele quedou muito espantado
quando me viu conduzir o barco com o leme, e como a vela cambava e
enfunava de um jeito ou de outro à medida que mudávamos o curso de
navegação; pois bem, quando ele viu isso, ficou como que assombrado e
estupefato; no entanto, com um pouco de prática, tudo se lhe tornou
conhecido, e ele se fez um marinheiro experto, exceto pela agulha de
marear, ou bússola, que não consegui fazer com que compreendesse bem.
Por outro lado, como havia muito pouco tempo nublado, e raramente ou
nunca havia nevoeiro naquelas bandas, havia menos necessidade de um tal
instrumento, uma vez que as estrelas sempre eram visíveis à noite, e a
costa durante o dia, exceto nas estações chuvosas; quando não havia como
se arriscar em excursão, fosse por terra, fosse por mar.

Eu estava então no vigésimo sétimo ano do meu cativeiro naquele lugar;
embora os três últimos anos em que eu tive aquela criatura comigo
devessem ser deixados de fora, tendo minha estada se feito outra do que
todo o resto do tempo havia sido. Conservei o aniversário de meu
desembarque ali com a mesma gratidão a Deus por suas misericórdias como
no início; e se eu havia tido tanta razão para reconhecê"-lo a princípio,
naquele momento tinha ainda mais, haja vista os novos testemunhos do
zelo da Providência para comigo e as grandes esperanças de encontrar um
salvamento efetivo e expedito; pois eu cultivava a impressão
inquebrantável em meus pensamentos de que meu salvamento era próximo e
que eu não conheceria um novo ano naquele lugar; sem descuidar, no
entanto, de minha lida, abrindo o chão, plantando, erguendo cercas como
de costume; colhendo e curando minhas uvas e as demais coisas
necessárias de outrora.

Enquanto isso, era chegada a estação das chuvas, quando me recolhia ao
espaço da casa mais do que em outros momentos; e por isso havíamos
guardado nosso novo veleiro com toda a segurança, levando"-o riacho
acima, onde, como foi dito atrás, chegava do navio com minhas jangadas;
e arrastando"-o para a margem, na marca mais alta da água, coloquei
Sexta"-Feira a cavar uma pequena doca, grande o bastante para que nela
fosse preso e profunda o suficiente para provê"-lo de água com que
flutuasse; e então, quando a maré baixou, fizemos uma barragem forte na
saída, de guisa que a água não entrasse; e assim ele permaneceu em
terreno seco, em relação à maré do mar; e para impedir a chuva,
cobrimo"-lo de muitos galhos, formando uma camada tão densa que nossa
embarcação se encontrava tão agasalhada quanto estivesse numa casa; e
assim esperamos pelos meses de novembro e dezembro, nos quais planejei
empreender minha aventura.

Com a chegada da estação seca, os pensamentos acerca do meu intento
retornaram com o bom tempo e eu passei a me preparar diariamente para a
viagem; e a primeira coisa que fiz foi reservar uma certa quantidade de
provisões que nos servissem de mantimento durante a viagem; e pretendia,
no espaço de uma semana ou quinze dias, romper a barragem e lançar nosso
barco. Certa manhã, enquanto me ocupava de alguma tarefa desse tipo,
chamei Sexta"-Feira e pedi"-lhe que fosse à beira"-mar e ver se conseguia
encontrar uma tartaruga, que caçávamos todas as semanas para o
provimento de ovos e carne. Fazia pouco que Sexta"-Feira partira quando
retornou correndo e praticamente saltou meu muro ou cerca externa, como
quem não tocasse o chão, nem degraus em que pisava; e antes que tivesse
tempo de falar, ele gritou: ``Ó senhor! Ó senhor! Ó triste! Ruim!''
``Que houve, Sexta"-Feira?'', perguntei"-lhe. ``Ali lá'', disse ele, ``um,
dois, três canoas; um, dois, três!''; ao que concluí, dada a maneira de
dizer, que eram seis; tendo descoberto, depois de inquiri"-lo, que eram
apenas três. ``Não tenha medo, Sexta"-Feira'', disse eu, e procurei
animá"-lo tanto quanto me era possível; no entanto, o coitado estava
muito assustado, pois tudo o que lhe passava pela cabeça era que estavam
à sua procura e o fariam em pedaços e devorariam; e o coitado tremia
tanto que já não sabia o que fazer com ele; e então procurei acalmá"-lo
na medida do possível e disse"-lhe que eu corria tanto perigo quanto ele,
e que eles me comeriam igualmente. ``Mas Sexta"-Feira'', disse eu,
``precisamos combatê"-los. Você consegue lutar, Sexta"-Feira?'' ``Eu
atirar'', respondeu ele, ``mas eles muitos, muitos''. ``Não importa'',
retorqui; ``nossas armas assustarão os que não matarmos''; e então lhe
perguntei se, caso eu garantisse que o defenderia, que ele também o
faria por mim e ficaria do meu lado e faria tudo o que lhe pedisse. Ele
disse: ``Se o Senhor decidir eu morrer, eu morrer.'' Então fui
buscar"-lhe um generoso trago de rum e dei a ele; pois eu havia sido tão
cuidadoso com o rum que ainda me restava muito; e tendo"-o bebido, fi"-lo
tomar as duas espingardas, que sempre levávamos conosco, e carregá"-las
de chumbo grosso, como o de pequenas balas de pistola; e então peguei
quatro mosquetes e os carreguei com dois lingotes e cinco pequenas balas
cada; e minhas duas pistolas eu carreguei com punhado de balas cada; e
pendurei minha grande espada desembainhada ao meu flanco, como sempre, e
entreguei a Sexta"-Feira a machadinha.

Assim preparado, travei de minha luneta e subi ao cimo da colina para
ver o que era capaz de descobrir do alto; e soube sem demora, com meu
instrumento, que se contavam vinte e um selvagens, três prisioneiros e
três canoas; e que tudo se tratava de um triunfante banquete ao qual se
reservavam aqueles três corpos humanos (um banquete bárbaro, de fato),
mas nada que fugisse a seus costumes.

Observei também que eles haviam desembarcado não onde o haviam feito, na
ocasião da fuga de Sexta"-Feira, mas perto do meu riacho, em um ponto
onde a praia era baixa e uma mata cerrada quase confinava com o mar; e
isso, mais a repulsa que sentia ao objetivo nefasto que aqueles
desgraçados tinham em vista, encheu"-me de tamanha indignação que fui a
Sexta"-Feira e disse"-lhe que estava decidido a ir até eles e matá"-los
todos; e perguntei"-lhe, ``Você ficará do meu lado?'' Ele já havia
superado o susto e, um pouco restituído de coragem pelo trago de rum que
lhe dera, mostrou"-se animado e me disse, como antes, que morreria se eu
assim o quisesse.

Nesse acesso de fúria, peguei das armas que havia carregado e, como
antes, as dividi entre nós; entregando a Sexta"-Feira uma pistola para
que a pendurasse no cinto, e três espingardas no ombro; enquanto portava
eu mesmo uma pistola e as outras três armas; e assim preparados saímos
em marcha; e levei comigo uma garrafinha de rum no bolso e dei a
Sexta"-Feira uma sacola grande em que havia mais pólvora e chumbo; e
quanto à ação, ordenei que ficasse perto de mim e não se movesse, nem
atirasse ou fizesse qualquer coisa até que o pedisse; e que,
entrementes, não dissesse uma palavra; e assim perfizemos um circuito à
direita de quase uma milha, cruzando o riacho e adentrando a mata; de
maneira que me posicionasse à distância de uma descarga antes que fosse
por eles descoberto, o que, segundo via em minha luneta, não era difícil
de acontecer.

Enquanto perfazia o caminho, via retornarem antigos pensamentos e
comecei a moderar minha decisão; não quero dizer que tenha sentido medo
quanto a seu número, pois, uma vez que eram tão somente pobres coitados
nus e desarmados, em verdade eu lhes era superior; de fato, embora
estivesse só; mas me ocorreu pensar: por qual desígnio, em que
circunstância, ou antes, que necessidade tinha eu de mergulhar as mãos
em sangue, atacar gentes que, a mim, nada haviam feito, nem intenção
tinham de fazê"-lo, e que, no que dizia respeito a mim, eram inocentes, e
cujos costumes bárbaros eram seu verdadeiro desastre, havendo nelas, a
exemplo do observado em outras nações daquela parte do mundo, sinal de
que Deus as havia abandonado àquela aviltante estupidez e a hábitos
crueis; sem que me tivesse instado a ser juiz de suas ações, muito menos
carrasco de Sua justiça. Do mesmo modo, ocorreu"-me que, se Ele o
julgasse adequado, segundo Seu juízo, tomaria a causa para Si e,
lançando sua vingança à nação, puni"-los"-ia na condição de povo por
crimes concernentes a ele como um todo; e que assim, entrementes, aquilo
não era de minha conta; mas que se justifica no caso de Sexta"-Feira, que
lhes era inimigo declarado e vivia em estado de Guerra declarada contra
aquelas gentes propriamente; e que era lícito que ele os atacasse, mas
eu não podia dizer o mesmo com relação a mim. Essas coisas agitaram"-se
tão fortemente em meus pensamentos durante todo o caminho que decidi que
só me colocaria próxima delas para observar"-lhes o banquete bárbaro e
que agiria como fosse da vontade de Deus; e que, a não ser que se me
revelasse exortação mais clara do que a que até aquele momento conhecia,
não me envolveria com eles.

Assim decidido, adentrei a mata e, com toda cautela e silêncio
possíveis, e Sexta"-Feira seguindo logo atrás em meu encalço, caminhei
até chegar aos limites do arvoredo, do lado que lhes ficava próximo; com
apenas aquele extremo de floresta entre eles e mim; e então chamei
discretamente Sexta"-Feira e, apontando"-lhe uma árvore de largo tronco
naquele fim de mata, ordenei que fosse até lá e me dissesse se podia ver
claramente o que faziam eles ali; ao que ele me atendeu e retornou sem
demora com a notícia de que se lhes podia ver claramente daquele ponto;
e que todos estavam à roda do fogo servindo"-se da carne de um de seus
prisioneiros; e que outro os aguardava, preso a pouca distância do grupo
na faixa de areia, e que seria morto a seguir; e isso me incendiou a
alma dentro de mim, pois Sexta"-Feira disse que não se tratava de homem
de sua nação; mas de um dos homens barbados dos quais havia feito
relação e que haviam chegado a suas terras de barco. A simples menção ao
homem branco barbado me encheu de horror; e indo até a árvore, avistei
claramente, usando de minha luneta, um homem branco, deitado na areia da
praia com mãos e pés atados com capim, ou algo como junco; e que era um
homem europeu e estava vestido.

Havia outra árvore, e arbustos cerrados logo adiante dela, obra de
cinquenta jardas mais próxima do grupo do que o lugar onde eu estava; e,
fazendo um pequeno contorno por dentro da mata, percebi que poderia
alcançá"-la sem ser descoberto e que, dali, ficara a uma distância de
meio disparo deles; e assim contive meus humores, embora estivesse
enfurecido ao mais alto grau, e recuando uns vinte passos, segui por
trás de alguns arbustos, que se estendiam por todo o caminho até chegar
à outra árvore; e então me postei em uma pequena elevação do terreno,
que me dava uma perspectiva completa deles, a uma distância de obra de
oitenta jardas.

Eu não tinha um instante a perder; pois dezenove dos horríveis
desgraçados estavam sentados no chão, quase amontoados uns sobre os
outros; enquanto dois haviam sido destacados para ir ao abate do pobre
cristão e trazê"-lo, depois, ao fogo, talvez membro a membro, e eles
estavam agachados desatando as amarras dos pés da vítima; e então eu me
voltei a Sexta"-Feira e disse ``Sexta"-Feira, faça o que eu mandar''; ao
que Sexta"-Feira respondeu afirmativamente; e em seguida eu disse,
``Então, Sexta"-Feira, faça exatamente o que me vir fazer''; e assim eu
deitei ao chão um dos mosquetes e a espingarda, o que fez Sexta"-Feira
com as suas armas; e com o outro mosquete, fiz mira nos canibais,
orientando a Sexta"-Feira que fizesse o mesmo; e então perguntei se ele
estava preparado, ao que ele respondeu que estava; e assim ordenei que
ele atirasse, e nós atiramos ao mesmo tempo.

Sexta"-Feira fez mira muito melhor do que eu, de maneira que, da banda em
que ele atirou, dois dos selvagens caíram mortos e mais três feridos;
enquanto eu, do meu lado, matei um e feri dois. Foi grande a
consternação causada, como se pode crer; e todos os que não foram
feridos puseram"-se de pé num salto, mas sem saber de pronto para onde
correr ou para onde olhar, pois não sabiam de onde vinha seu extermínio.
Sexta"-Feira conservou"-se atento a mim, para que pudesse, como lhe
ordenara, observar o que eu fazia; e dessarte tão logo o primeiro
disparo foi executado, lancei a arma ao chão e tomei da espingarda, o
que Sexta"-Feira repetiu; ele me viu apontá"-la, o que ele também fez; e
eu lhe perguntei, ``Preparado, Sexta"-Feira?'', ao que ele respondeu que
sim; ``Então fogo, em nome de Deus!'', e com isso disparei mais uma vez
contra os desgraçados ainda transidos de espanto, o que Sexta"-Feira
também fez; e como nossas armas estavam carregadas do que chamo de
chumbo grosso, ou pequenas balas de pistola, apenas dois deles tombaram;
mas tantos estavam feridos que corriam de um lado para o outro gritando
e berrando como loucos, todos cobertos de sangue, e a maioria
miseravelmente ferida; ao que mais três caíram logo depois, ainda não de
todo mortos.

Abandonando as armas descarregadas e travando do mosquete ainda
guarnecido, pedi que Sexta"-Feira me seguisse; o que fez ele com muito
coragem; ao que corri para fora da mata e me revelei, com Sexta"-Feira
bem próximo de mim; e tão logo percebi que me haviam visto, gritei o
mais alto que pude, ordenando que Sexta"-Feira fizesse o mesmo; e
correndo com toda a ligeireza, que, a propósito, não era tão ligeira,
tão carregado que estava de armas, dirigi"-me sem desvio à pobre vítima,
deitada como dito atrás no areal, ou praia, entre o local onde haviam
estado os selvagens e o mar; uma vez que os dois magarefes que se
preparavam para abatê"-lo, surpreendidos com nossa primeira descarga,
haviam fugido varados de medo para beira"-mar e saltado para dentro de
uma canoa, no que foram acompanhados de outros três; ao que me voltei a
Sexta"-Feira e ordenei que avançasse e os alvejasse; o que ele
compreendeu imediatamente e, correndo obra de quarenta jardas para assim
cercar"-se deles, disparou, e pensei que ele os havia matado todos; pois
os vi tombar uns em cima dos outros dentro da canoa; e apesar de dois
deles terem se erguido quase de pronto, dois deles haviam morrido e um
terceiro, que permanecera deitado no fundo do barco como se estivesse
morto, quedara ferido.

Enquanto Sexta"-Feira disparava contra os selvagens, eu puxei minha faca
e cortei os caniços que amarravam a pobre vítima; e, soltando"-lhe mãos e
pés, levantei"-o e perguntei na língua dos portugueses o que era ele; ao
que ele respondeu em latim, ``\emph{christianus}''; mas estava tão fraco
e frágil que mal conseguia ficar de pé ou falar. Tirei, então, minha
garrafa do bolso e ofereci"-lhe, fazendo sinais de que devia beber, o que
ele fez; e dei"-lhe um pedaço de pão, que ele comeu; e em seguida o
inquiri sobre sua nação; ao que ele respondeu, ``\emph{español}''; e
estando um pouco recuperado, esforçou"-se em expressar, mediante todos os
sinais que foi capaz de fazer, o quanto estava em dívida comigo por seu
salvamento. ``\emph{Señor}'', disse eu, com todo o espanhol que pude
reunir, ``conversaremos depois; agora, devemos lutar; e se ainda lhe
restar alguma força, tome desta pistola e espada e prenda"-as a si''; as
quais ele aceitou com muita gratidão; e tão logo teve as armas em mãos,
elas como que lhe infundiram novo vigor; e ele atacou seus assassinos
como uma fúria e fez duas delas em pedaços num piscar de olhos; pois a
verdade é que, tendo tudo aquilo os surpreendido enormemente, as pobres
criaturas ficaram tão assustadas com o barulho de nossas bocas de fogo
que caíam unicamente de espanto e medo; e não tinham mais forças de
encetar sua fuga do que suas carnes de resistir a nossos disparos; e
assim se deu com os cinco contra os quais Sexta"-Feira fez descarga na
canoa; pois se três deles caíram com o chumbo que receberam, os dois
restantes caíram de pavor.

Conservei minha arma na mão, sem atirar, mas com o intuito de manter"-me
pronto para tanto; pois dera ao espanhol minha pistola e espada; e assim
chamei Sexta"-Feira e ordenei que ele corresse à árvore de onde fizemos a
primeira descarga, e pegasse as armas que ali haviam ficado, o que fez
com grande presteza; e em seguida, dando"-lhe meu mosquete, ocupei"-me de
recarregar todo o arsenal e ordenei que recorressem a mim quando
necessitassem. Enquanto municiava as armas, teve lugar um fero embate
entre o espanhol e um dos selvagens, que avançou sobre ele com uma de
suas grandes espadas de madeira, a mesma arma que o teria matado, não
tivesse eu o evitado. O espanhol, homem intrépido e valente, porém
acometido de fraqueza, lutou longamente contra o índio e abriu"-lhe duas
grandes feridas na cabeça; mas o selvagem, sendo homem de força e vigor,
aproximou"-se dele, derrubou"-o (estando ele fraco), e estava a ponto de
arrancar"-lhe a espada da mão; quando o espanhol, embora no chão,
sabiamente largando a espada, sacou a pistola do cinto e disparou contra
o selvagem, que morreu ali mesmo; antes que eu, que corria para
ajudá"-lo, tivesse me aproximado.

Sexta"-Feira, agora deixado à vontade para agir como lhe aprouvesse,
perseguiu os desgraçados fugitivos armado unicamente de sua machadinha:
e com ela despachou todos os três que, como dito atrás, haviam quedado
feridos a princípio, e todos os demais de que pôde dar conta; e o
espanhol, vindo a mim em busca de uma arma, recebeu uma das espingardas,
com que perseguiu dois dos selvagens, ferindo a ambos; no entanto,
estando ele fraco para correr, os dois escaparam"-lhe rumo à floresta,
por onde Sexta"-Feira seguiu"-lhes no encalço, matando um deles; sendo o
outro rápido demais para ele; e este, embora estivesse ferido, mergulhou
no mar e nadou com as forças que tinha de encontro aos dois que haviam
partido na canoa; de maneira que foram três, um dos quais ferido, sem
que soubéssemos se morrera ou não, os que escaparam de nossas mãos de um
total de vinte e um. A relação dos demais mostra que:

3 foram mortos por nosso primeiro disparo, quando postados na árvore;

2 foram mortos na descarga que se seguiu;

2 foram mortos por Sexta"-Feira na canoa;

2 foram mortos pelo mesmo, dos quais um primeiro fora ferido;

1 foi morto por Sexta"-Feira na mata;

3 foram mortos pelo espanhol;

4 morreram, aqui e ali, em decorrência dos ferimentos, ou foram mortos
por Sexta"-Feira durante perseguição;

4 escaparam na canoa, dentre os quais um ferido, senão morto;

21 no total.

Os que estavam na canoa esforçaram"-se muito para escapar ao alcance das
armas; e embora Sexta"-Feira tenha feito dois ou três disparos contra
eles, não creio que os tenha acertado em nenhum deles. Sexta"-Feira quis
que eu pegasse uma de suas canoas e saísse em perseguição aos
sobreviventes; e em verdade fiquei preocupado com sua fuga, temendo que,
ao levar as novas a sua gente, talvez retornassem com duzentas ou
trezentas canoas e nos aniquilassem por sua mera maioria; e assim
consenti em caçá"-los pelo mar e, correndo a uma de suas canoas, saltei
para dentro dela e ordenei que Sexta"-Feira fizesse o mesmo: mas quando
entrei na canoa, causou"-me espanto encontrar outra pobre criatura ali
deitada, de pés e mãos atados à espera do abate, como o espanhol
estivera, e quase morta de medo, sem saber o que se passava; pois ele
não havia sido capaz de olhar por cima da lateral da canoa, estando ele
amarrado com tanta força no pescoço e nos calcanhares e por tanto tempo
que, de fato, restava"-lhe pouca vida.

Cortei imediatamente os caniços ou juncos trançados que o prendiam e o
teria ajudado a se levantar; mas ele não tinha forças para se levantar
ou falar, limitando"-se a gemer da forma mais lamentável, acreditando
ainda, ao que parecia, que só fora desamarrado para ser morto.

Quando Sexta"-Feira foi a seu encontro, ordenei que conversasse com ele
lhe explicasse sobre seu salvamento; e tirando minha garrafa do bolso,
fiz com que desse ao pobre coitado um trago, e este, aliado à notícia de
seu salvamento, o reanimou, e ele sentou"-se na canoa; mas quando
Sexta"-Feira começou a ouvi"-lo falar e fitou"-lhe o rosto, teria levado
qualquer um às lágrimas ver como Sexta"-Feira o beijou e abraçou,
permanecendo ao seu lado e
chorando, rindo, gritando,
pulando, dançando, cantando, e então voltando a chorar, cerrando as mãos
em punho e batendo em seu próprio rosto e cabeça; e então tornando a
cantar e pular novamente como criatura assaltada de alguma perturbação.
Levei algum tempo até que conseguisse fazê"-lo falar comigo ou me dizer o
que se passava; mas quando voltou um pouco a si, disse"-me que era seu
pai.

Não é fácil para mim expressar como fiquei comovido de assistir ao
êxtase e afeição filial manifestos nesse pobre selvagem ante a visão do
pai e seu salvamento da morte; nem mesmo posso descrever metade das
extravagâncias a que seu amor deu vazão, pois que entrou e saiu da canoa
incontáveis vezes: e quando entrava, sentava"-se ao lado do pai, abria os
braços e trazia a cabeça do pai ao peito, segurando"-a contra si até por
meia hora; então ele pegava seus braços e tornozelos, que estavam
dormentes e rígidos por causa das amarras, e os esfregava e friccionava;
ao que eu, compreendendo o que fazia, ofereci"-lhe um pouco de rum da
garrafa para que lhe ungisse os membros, o que lhes fez muito bem.

Esse caso pôs fim à nossa perseguição à canoa que levava os outros
selvagens, os quais agora se encontravam quase fora da vista de nossos
olhos; e foi uma felicidade não termos ido ao mar, pois duas horas
depois, antes que pudessem ter completado um quarto do caminho, começou
a soprar um forte vendaval, que seguiu soprando com muita força noite
adentro, de noroeste, isto é, o que os fazia remar contra o vento, de
forma que não acreditava que sua canoa lhe pudesse sobreviver ou que
eles viessem a tocar novamente a costa.

Mas voltando à Sexta"-Feira; estava ele tão entretido com seu pai que não
tive coragem de afastá"-lo de sua presença; quando, por fim, julguei que
ele pudesse deixá"-lo por um momento, chamei"-o, e ele veio pulando e
rindo, e muitíssimo feliz; e então lhe perguntei se ele havia dado pão
ao pai, ao que balançou a cabeça e respondeu, ``Não, cachorro feio comer
tudo''. Assim, dei"-lhe um naco de pão, que guardava em uma bolsinha
usada com esse propósito, e um trago de rum, não para que bebesse, mas
para que levasse ao pai; e tendo no bolso dois ou três punhados de
passas, entreguei"-lhe uma porção a ser oferecida ao pai. Mal serviu ao
pai as passas, observei"-o sair do barco e correr para longe, como que
enfeitiçado, pois tinha os pés mais rápidos que eu já vi na vida; e pois
bem: ele correu com tal ligeireza que, por assim dizer, num piscar de
olhos o perdi de vista; e embora eu o chamasse e gritasse por seu nome,
foi inútil, pois não houve o que o parasse, e num quarto de hora eu o vi
retornar, embora não tão rápido quanto partira; e, à medida que se
aproximava, percebi que sua lentidão se dava em razão de trazer ele algo
nas mãos.

Quando chegou próximo de mim, descobri que havia ele estado em casa em
demanda de uma ânfora ou jarro de barro para levar um pouco de água
fresca ao pai, e que ele trouxera mais dois filões de pão; dando"-me o
pão, mas levando a água ao pai; embora, estando eu também com muita
sede, tenha bebido dela um bocadinho. A água reanimou seu pai mais do
que todo o rum ou bebida que lhe tenha oferecido, pois ele estava a
ponto de desmaiar de sede.

Tendo seu pai bebido da água, chamei"-o para saber se sobrara um pouco;
ao que ele respondeu que sim; e então pedi que a desse ao pobre
espanhol, que dela necessitava tanto quanto seu pai; e mandei um dos
filões que Sexta"-Feira havia trazido também a ele, que estava de fato
muito enfraquecido e repousava sobre um terreno gramado à sombra de uma
árvore; e cujos membros também se encontravam muito rijos e inchados por
causa das amarras grosseiras com que fora preso. E quando o vi tomar da
água e comer do pão que Sexta"-Feira lhe servira, fui a seu encontro e
dei"-lhe um punhado de passas; e ele fitou meu rosto com todos os sinais
de gratidão e reconhecimento que se podiam mostrar em um semblante; mas
estava tão fraco, não obstante seu empenho na luta, que não conseguia
ficar em pé; o que tentou fazer duas ou três vezes, porém sem bom
sucesso, tão inchados e doloridos estavam seus tornozelos; e então pedi
que descansasse, e que Sexta"-Feira lhe esfregasse os tornozelos e os
banhasse com rum, tal como fizera com seu pai.

Observei que, a cada dois minutos, ou talvez menos, durante todo o tempo
em que esteve ali, a pobre criatura carinhosa virava a cabeça para ver
se o pai estava no mesmo lugar e posição em que o deixara; até que,
então, o pai não pôde ser visto; ao que ele se pôs de pé num instante e,
sem dizer uma palavra, correu a seu encontro com a rapidez que lhe era
própria, com a qual mal se percebia seus pés tocarem o chão à medida que
avançava; e quando chegou, descobriu apenas que havia o pai se deitado
para aliviar os membros; e retornando Sexta"-Feira até mim sem demora, eu
pedi ao espanhol que deixasse Sexta"-Feira ajudá"-lo a chegar à canoa, e
então ele o levaria para nossa morada, onde ele estaria sob meus
cuidados. Mas Sexta"-Feira, vigoroso e forte como era, colocou o espanhol
nas costas e levou"-o à canoa, depositando"-o delicadamente sobre a
lateral ou amurada da canoa, com os pés para dentro da embarcação; para
então, erguendo"-o completamente, acomodá"-lo ao lado de seu pai; feito
isso, deixou a embarcação mais uma vez para empurrá"-la ao mar e, em
seguida, conduzi"-la a remo ao longo da costa, o que fez mais rápido do
que eu era capaz de andar, ainda que o vento soprasse também muito
forte; e, assim, ele levou os dois a salvo ao nosso ribeirão e,
deixando"-os na canoa, correu para buscar a outra embarcação. Quando
passou por mim, chamei"-o e perguntei para onde ia; ao que me respondeu,
``Vai buscar mais barco'', e correu como o vento; pois é certo que
nenhum homem ou cavalo era mais rápido; e ele chegou com a outra canoa
ao riacho quase ao mesmo tempo que o alcancei por terra; por fim,
atravessou"-me e foi ajudar nossos novos hóspedes a saírem do barco,
tendo"-o feito; porém, nenhum dos dois era capaz de andar; de maneira que
o pobre Sexta"-Feira não soube o que fazer.

Para remediar o problema, pus"-me a trabalhar em meu pensamento e,
chamando Sexta"-Feira e pedindo"-lhe que os deixassem acomodados na margem
enquanto vinha até mim, logo fiz uma espécie de padiola para pousá"-los,
e Sexta"-Feira e eu os carregamos ambos sobre ela entre nós; porém,
quando chegamos ao nosso muro ou fortaleza, deparamo"-nos com um problema
ainda maior; pois era impossível atravessá"-la com os dois, e eu não
desejava arrebentá"-la; assim, coloquei"-me novamente a trabalhar; e
Sexta"-Feira e eu, em obra de duas horas, fizemos uma bela tenda, coberta
com as velhas lonas das velas e, por cima destas, galhos de árvores,
erguida no espaço externo de nosso muro mais avançado, entre ele e o
jovem bosque que eu plantei; e ali fizemos para eles duas camas de
coisas de que dispunha, isto é, de uma boa palha de arroz, com
cobertores colocados sobre ela para que neles se deitassem, além de
outros dois que lhes foram dados para que se cobrissem.

Minha ilha encontrava"-se, então, povoada, e eu me julgava muito rico em
súditos; e era uma reflexão muito feliz que me ocorria, o quão parecido
com um rei eu me tornara. Em primeiro lugar, toda a ilha era de minha
propriedade única, de forma que eu tinha sobre ela um indiscutível
direito de posse. Em segundo lugar, meu povo estava inteiramente sujeito
ao meu poder; eu era senhor e legislador absolutos; todos deviam suas
vidas a mim e estavam prontos a dar suas vidas por mim, havendo ocasião
para tanto. Era digno de nota, também, que tínhamos somente três
súditos, e os três de religiões diferentes; sendo Sexta"-Feira
protestante, seu pai, um pagão e antropófago, e o espanhol, papista; no
entanto, concedia a liberdade de consciência em meus domínios; o que
aqui digo de passagem.

Tendo garantido a segurança de meus dois prisioneiros resgatados e lhes
dado abrigo e um lugar de pouso, pus"-me a pensar em preparar"-lhes algum
mantimento; e, assim, ordenei que Sexta"-Feira buscasse um cabrito, isto
é, um animal ainda jovem, em meu redil particular, o qual seria morto; e
quando cortei os quartos traseiros e os piquei em pedacinhos, pedi que
Sexta"-Feira os fervesse e cozinhasse, e ao ensopado acrescentei arroz e
cevada; e fiz dessa forma, garanto"-vos, um prato muito saboroso de carne
e caldo; e tendo cozinhado ao ar livre, pois não acendia o lume na parte
de dentro de meu muro interno, tudo trasladei para a nova tenda; e tendo
preparado uma mesa ali para os hóspedes, sentei"-me e comi de meu próprio
jantar também com eles, e, tanto quanto pude, animei"-os e incentivei"-os;
para o que tive Sexta"-Feira como meu intérprete, especialmente para seu
pai, mas também para o espanhol; pois o espanhol falava muito bem a
língua dos selvagens.

Depois que termos ceado, ou melhor, jantado, ordenei que Sexta"-Feira
pegasse uma das canoas e fosse buscar nossos mosquetes e outras armas de
fogo que, por falta de tempo, havíamos deixado no local de batalha; e no
dia seguinte ordenei"-lhe que sepultasse os cadáveres dos selvagens, que
jaziam ao sol e não tardariam a tornar"-se ofensivos; assim como
orientei"-o a enterrar os horríveis restos do festim bárbaro, que sabia
serem muitos, porque não podia pensar em fazer eu mesmo; isto é, não
suportaria vê"-los, caso tomasse aquele caminho; e todas essas coisas
Sexta"-Feira executou à perfeição, tendo apagado as evidências da
passagem dos selvagens naquele lugar; de maneira que, quando retornei ao
sítio, mal pude saber onde tudo havia se passado, senão pelo extremo da
mata que apontava ao local.

Comecei, então, a travar ligeira conversação com meus dois novos
súditos; e, primeiro, fiz com que Sexta"-Feira perguntasse a seu pai o
que pensava ele da fuga dos selvagens naquela canoa, e se poderíamos
esperar seu retorno com poderio grande demais para resistirmos; e seu
primeiro comentário foi que os selvagens na canoa certamente não haviam
sobrevivido à tempestade que se abateu na noite em que partiram; tendo,
pelo contrário, naufragado, ou ainda sido arrastados ao sul, na direção
de praias em que haveriam de ter conhecido a aniquilação, da mesma forma
com que certamente teriam se afogado, no caso de terem se perdido em mar
aberto; mas, quanto ao que fariam no caso de terem tocado a praia sem
sustos, disse ele que não sabia; mas que era sua opinião que estavam
eles tão assustados ante a maneira com que haviam sido atacados, com o
estrondo e o fogo, que acreditava que os sobreviventes contariam a seu
povo que todos haviam sido mortos por trovões e relâmpagos, não pelas
mãos de homens; e que os dois que haviam ali surgido, a saber,
Sexta"-Feira e eu, éramos dois espíritos celestes, ou Fúrias, que haviam
descido dos céus para destruí"-los, e não homens armados; e disso,
afirmou ele, ele estava certo; pois que os escutara gritando essas
coisas uns aos outros em sua língua; e que lhes era impossível conceber
que um homem pudesse lançar fogo, trovejar e matar à distância sem que
erguesse a mão, como havia sido feito: e quanto a essas coisas o velho
selvagem tinha razão; pois, como vim a saber posteriormente por outras
vias, os selvagens nunca mais tentaram viajar à ilha depois daquilo;
tendo quedado tão consternados com os relatos oferecidos por aqueles
quatro homens (pois aparentemente eles haviam sobrevivido ao mar) que
acreditaram que os que chegassem à praia daquela ilha enfeitiçada seriam
aniquilados pelo fogo dos deuses.

Isso, entretanto, eu não sabia; e, portanto, permaneci sob contínua
apreensão por um bom tempo e me mantive sempre em guarda, com todo o meu
exército: pois, como éramos então em quatro, teria confrontado uma
centena de selvagens em campo aberto na ocasião que se colocasse.

Em pouco tempo, porém, sem que mais canoas aparecessem, o medo de tal
chegada se desfez; e voltei a tomar em consideração meus pensamentos
prévios acerca uma viagem ao continente; contando com os augúrios do pai
de Sexta"-Feira, de que sob seus auspícios poderia estar certo de ser bem
recebido por sua gente, caso eu fosse.

Mas os meus pensamentos interromperam"-se momentaneamente depois de uma
conversa séria que travei com o espanhol, e de ter compreendido que
havia outros dezesseis homens, entre seus conterrâneos e portugueses,
que tendo naufragado e ali encontrado refúgio, viviam em boa paz com os
selvagens, porém em grande dificuldade para suprir suas necessidades e,
em verdade, sobreviver. Inquiri"-lhe todos os detalhes de sua viagem; e
vim a saber que se tratava de navio espanhol em travessia do Rio da
Prata a Havana, com instruções para nessa localidade deixar sua carga,
que consistia sobretudo de couro e prata, e trazer em seu retorno tantas
mercadorias europeias quantas lá encontrasse; que tinha cinco
marinheiros portugueses a bordo, estes salvos de outro afundamento; que
cinco de seus próprios homens se afogaram quando a primeira embarcação
se perdeu; e que os sobreviventes enfrentaram infinitos trabalhos e
perigos e chegaram, esfaimados, à costa dos selvagens, onde esperavam
ser devorados a cada momento.

Disse"-me que levavam consigo algumas armas, mas eram absolutamente
inúteis, pois não tinham pólvora ou chumbo, tendo o mar poupado apenas
uma pequena porção de pólvora, que usaram tão logo pisaram em terra
firme para se abastecerem de alguma comida.

Perguntei"-lhe o que julgava ele que sucederia com os náufragos ali, e se
planejavam fuga. Disse ele que haviam deliberado muito sobre o assunto;
mas que, não tendo embarcação ou ferramentas para construí"-la, nem
provisões de qualquer tipo, suas reuniões sempre terminavam em lágrimas
e desesperação.

Perguntei"-lhe como ele entendia que os náufragos receberiam uma proposta
de fuga de minha parte; e se, uma vez que todos estivessem aqui, isso
não poderia ser feito. Disse"-lhe com absoluta franqueza que temia
sobretudo a traição e o mau trato, caso colocasse minha vida em suas
mãos; pois que a gratidão não é virtude inerente à natureza do homem;
nem os homens costumam ajustar o tratamento que despendem aos favores
que recebem, mas às vantagens que esperam obter. Disse"-lhe que me seria
um terrível golpe ser o instrumento de seu salvamento e, depois,
transformado em prisioneiro na Nova Espanha, onde um inglês certamente
conheceria o sacrifício, a despeito da necessidade ou acidente que o
levasse àquelas bandas; e que preferia ser entregue aos selvagens e
devorado vivo a cair nas garras impiedosas dos padres e ser conduizido à
Inquisição. Acrescentei ainda que, do contrário, estava persuadido de
que, uma vez todos estivessem aqui, seríamos em número bastante para
construir uma barca capaz de levar"-nos todos, fosse aos Brasis ao sul,
fosse às ilhas ou costa espanhola ao norte; mas se, em retribuição,
eles, quando colocasse armas em suas mãos, levassem"-me contra minha
vontade a seu próprio povo, um tal tratamento não estaria à altura de
minha bondade para com eles; e ademais tornaria minha situação ainda
pior do que a anterior.

Ele respondeu, com muita retidão e honestidade, que a condição dos
referidos homens era tão miserável e que disso tanto padeciam que,
acreditava ele, teriam por abominável a ideia de dedicar mau trato a
quem quer que contribuísse a seu salvamento; e que, caso fosse de minha
vontade, ele iria até eles acompanhado do velho e trataria do assunto
com eles, retornando com a resposta que lhe dessem; e que exporia as
condições que deveriam ser por eles aceitas sob juramento, a saber, de
que deveriam submeter"-se absolutamente a minha autoridade de comandante
e capitão; e que deveriam jurar, pelos Santos Sacramentos e pelas
Sagradas Escrituras, a me serem fiéis e seguir viagem ao país cristão
que fosse de minha vontade, e a nenhum outro; e que permanecessem total
e absolutamente sob minhas ordens até que desembarcassem a salvo em tal
país; e que firmassem um contrato para tanto.

Disse"-me ele, então, que em primeiro lugar jurava para mim que jamais se
apartaria de mim enquanto vivesse, senão quando recebesse de mim ordem
para fazê"-lo; e que permaneceria ao meu lado até a última gota de seu
sangue, caso um entre seus conterrâneos ferisse minimamente as juras
firmadas.

Disse"-me ele também que eram todos homens de boa civilidade e honestos,
e que se encontravam em grande dificuldade, não dispondo de armas ou
roupas, nem de comida, estando dessarte à mercê e discrição dos
selvagens; sem qualquer esperança de retornar ao próprio país; e que
tinha certeza que, se lhes desse eu cabo de tamanhas aflições, eles
viveriam e morreriam por mim.

Dadas as garantias, decidi arriscar"-me a salvá"-los, se possível, e
enviar"-lhes o velho selvagem e o dito espanhol para que com eles
selassem o acordo; mas quando tudo havia se aprontado e os dois eram
prestes a partir, o próprio espanhol opôs objeção tão cheia de
prudência, por um lado, e de lealdade, por outro, que folguei muito em
ouvi"-lo; e por seu bom conselho, adiou"-se o salvamento de seus camaradas
em pelo menos um ano. Eis o caso:

Ele vivia conosco já havia obra de um mês; espaço durante o qual lhe
permiti que observasse de que maneira, e com o auxílio da Providência,
provera"-me de mantimentos; e houve ele ciência das provisões de milho e
arroz que tinha em armazém; o que, embora fosse mais do que suficiente
para mim, não era bastante, não sem bom governo, para minha família, que
agora contava de quatro almas; e menos seria se viessem seus
compatriotas, que eram, segundo dissera o espanhol, em número de
quatorze, ainda vivos; de maneira que pouco teríamos que abastecesse
nosso navio, se o construíssemos, para uma viagem a qualquer uma das
colônias cristãs da América; disse"-me ele, então, que julgava mais
aconselhável permitir que ele e os outros dois preparassem e cultivassem
um pouco mais de terra, tanto quanto se pudesse semear com as sementes
excedentes que tivesse; e que deveríamos esperar pela colheita seguinte,
para que tivéssemos grãos em quantidade para suprimento de seus
conterrâneos quando chegassem; pois a necessidade podia tentá"-los à
discórdia e à ideia de que não conheciam ali salvamento, mas tão somente
a troca de uma dificuldade por outra. ``Você sabe'', lembrou ele, ``que
os filhos de Israel, embora a princípio se tenham regozijado com a
libertação do Egito, por fim se rebelaram contra o próprio Deus, que
lhes deu livramento, quando careceram de pão no deserto''.

Foi tão propício seu recato, e tão bom seu conselho, que só pude me
sentir muito agradecido por sua proposta, e feliz com sua lealdade;
então começamos a cavar e revolver a terra, nós quatro, tanto quanto nos
permitiam as ferramentas de madeira; e no espaço de um mês, ao final do
qual era tempo de semeadura, havíamos limpado e podado tanta terra que
semeamos vinte e dois alqueires de cevada e dezesseis potes de arroz,
que era, em suma, todas as sementes de que dispúnhamos: de fato, mal
havíamos reservado o bastante que nos alimentasse pelos seis meses de
espera pela colheita; isto é, contando a partir do momento em que
separamos nossas sementes para a semeadura; pois não era necessário
tanto tempo de germinação naquela terra.

Vivendo então em sociedade de bom tamanho, com números bastantes para
dirimir o medo dos selvagens, caso viessem, a menos que fossem
demasiados, íamos à vontade por toda a ilha, sempre que achávamos
ocasião; e uma vez que tínhamos nossa fuga ou salvamento em nossas
mentes, era impossível, \emph{ao menos para mim}, deixar de pensar nos
meios de levá"-lo a cabo; assim, marquei árvores que achei adequadas ao
nosso trabalho, e pedi que Sexta"-Feira e seu pai as cortassem; e então
ordenei que o Espanhol, a quem transmiti meus pensamentos sobre o
assunto, vigiasse e comandasse a obra. Mostrei"-lhes com que incansáveis
trabalhos eu talhara uma árvore grande até que fosse transformada em
tábuas comuns, e ordenei que fizessem o mesmo, até que fizeram obra de
uma dúzia de tábuas grandes de bom carvalho, com quase dois pés em largo
e trinta e cinco de comprido, e de duas a quatro polegadas de espessura;
pode"-se imaginar o prodigioso esforço exigido.

Entrementes, consegui aumentar tanto quanto pude meu pequeno redil, ou
rebanho, de cabras; e com esse propósito fiz Sexta"-Feira e o Espanhol
saírem à mata num dia, o que tornamos a fazer, eu e Sexta"-Feira, no dia
seguinte (pois trabalhávamos em turnos); e desse modo obtivemos perto de
vinte cabritos a serem criados com os demais; pois sempre que abatíamos
a mãe, salvávamos os filhotes e os incluíamos em nosso curral. Ademais,
chegando a época de cura das uvas, ordenei que se pendurasse ao sol uma
quantidade tão prodigiosa de cachos que, creio eu, se estivéssemos em
Alicante, onde as passas são secas ao sol, teríamos enchido sessenta ou
oitenta barris; e estas, com nosso pão, compunham grande parte de nossa
ração, e um muito bom viver, asseguro"-lhes, pois são grande nutrimento.

Era chegado o tempo da colheita, e nossos grãos estavam em ótima
condição; e embora não tivesse sido a mais abundante safra que tivesse
visto na ilha, bastava a nossos fins; pois de vinte e dois alqueires de
cevada carregamos e debulhamos acima de duzentos e vinte alqueires; e o
arroz em proporção similar; o que dava quantidade bastante para nosso
mantimento até a próxima colheita, ainda que todos os dezesseis
espanhóis tivessem tocado minha costa; ou, caso estivéssemos preparados
para uma viagem, teria abastecido com abundância nosso navio de modo que
pudéssemos viajar a toda parte do mundo, isto é, da América.

Depois de então guardar e proteger nosso armazém de grãos, começamos a
trabalhar para produzir mais utensílios de vime, a saber, grandes
cestos, nos quais os conservamos; e o Espanhol foi muito destro e hábil
nessa tarefa, e muitas vezes me censurou por não ter me valido desse
material para a confecção de artigos de defesa; mas não vi necessidade
disso.

E assim, dispondo de um suprimento completo de comida para todos os
convidados que aguardava, dei permissão ao Espanhol para que fosse ao
continente e visse o que se poderia fazer daqueles que deixara para
trás. Dei"-lhe ordens estritas de não trazer à ilha homem algum que,
antes de tudo, não tivesse jurado na presença de si e do velho selvagem
que jamais colocaria em perigo deliberado a pessoa que tivera a bondade
de mandá"-los buscar com a intenção de libertá"-los; mas que ficaria ao
seu lado e o defenderia contra quaisquer tentativas de dolo, sempre que
se apresentassem, permanecendo inteiramente sob seu comando; e que tal
jura se fizesse por escrito e firmada de próprio punho. Como fariam
isso, se eu sabia que não tinham pena ou tinteiro, era uma pergunta que
nunca nos havíamos feito.

Assim instruídos, o Espanhol e o velho selvagem, pai de Sexta"-Feira,
partiram em uma das canoas nas quais podiam dizer que haviam chegado, ou
melhor, haviam sido trazidos, na condição de prisioneiros a serem
devorados pelos selvagens.

Dei a cada um deles um mosquete com pederneira, e obra de oito cargas de
pólvora e chumbo, pedindo que fizessem uso parcimonioso de ambas, e não
as usassem senão em circunstâncias urgentes.

Esses foram esforços felizes, sendo as primeiras providências tomadas
por mim em vinte e sete anos e alguns dias almejando meu livramento.
Dei"-lhes provisões de pão e passas bastantes para seu mantimento por
muitos dias e bastantes para todos os espanhóis por um espaço de oito
dias; e desejando"-lhes uma boa viagem, eu os vi partir, combinando com
eles um sinal que deveriam arvorar de forma a fazê"-lo visível em ocasião
de seu retorno, pelo qual os poderia reconhecer, à distância, antes de
chegarem à costa, quando retornassem.

Partiram com vento propício em um dia de lua cheia do mês de outubro,
segundo meus registros; mas, quanto ao cômputo preciso dos dias, uma vez
que o perdi, nunca mais pude recuperá"-lo; nem mesmo mantive contagem
correta e pontual do número de anos; embora, como ficou provado em exame
posterior do que registrara, descobri que fizera um cálculo correto dos
anos.

Passados oito dias em que esperava ambos, deu"-se estranho imprevisto, do
qual talvez não tenha conhecido semelhante na História. Estava eu
ferrado em sono em minha cabana certa manhã, quando Sexta"-Feira veio em
desabalada corrida até mim e gritou: ``Senhor, senhor, eles chegar,
chegar!''

Levantei de um salto e, sem medir o perigo, saí, tão logo me pus
vestido, atravessando meu bosque, que a essa altura, aliás, ia
tornando"-se mata densa; isto é, sem medir o perigo, saí sem minhas
armas, o que não estava em meu uso fazer; e espantei"-me quando, ao
voltar"-me ao mar, vi um barco a obra de légua e meia de distância, vindo
à praia, com a bujarrona, ou costela de carneiro, como a chamam,
desfraldada sob vento muito propício: também observei, então, que não
vinham das bandas do continente, mas do extremo sul da ilha; ao que
chamei Sexta"-Feira e pedi"-lhe que não se alongasse de mim, pois que
aquelas não eram as pessoas por quem esperávamos, de forma que não
sabíamos se vinham em paz ou se eram inimigas.

Em seguida, fui buscar minha luneta para ver o que podia inferir de seu
aparecimento; e tendo retirado a escada, subi à cumeeira da colina, como
costumava fazer quando quedava apreensivo com alguma coisa, e para
observar com mais discernimento sem ser descoberto.

Mal havia posto os pés na colina quando meus olhos avistaram claramente
um navio lançando âncora, a obra de duas léguas e meia de distância de
mim, sul"-sudeste, mas a não mais de uma légua e meia da praia;
tratando"-se, pelo que podia ver, de navio inglês, e o bote, um escaler
da mesma nação.

Não posso expressar a confusão em que estava; pois apesar da alegria
indescritível de avistar um navio, tripulado, como tinha eu razões de
crer, por meus próprios conterrâneos e, portanto, amigos, sentia
pairarem sobre mim algumas misteriosas dúvidas, as quais não sabia dizer
de onde vinham, ordenando que me mantivesse em guarda. Em primeiro
lugar, ocorreu"-me perguntar a que negócios um navio inglês vinha dar
naquelas bandas do mundo, uma vez que não estava nos caminhos, fossem de
ida, fossem de volta, às partes do mundo onde os ingleses faziam
comércio; e eu sabia que não haviam caído tempestades que os colocassem
em perigo; de sorte que, se fossem de fato ingleses, era muito provável
que haviam ali dado sem boas intenções; e que, portanto, era melhor que
eu continuasse como estava a cair nas mãos de ladrões e assassinos.

Que nenhum homem despreze as misteriosas sugestões e avisos do perigo
que às vezes lhe são dadas em situações em que pode pensar que não
existe possibilidade de que sejam reais. Que tais sugestões e avisos nos
sejam dados, creio que poucos que tenham conhecido as coisas deste mundo
o podem negar; que sejam descobrimentos de um mundo invisível e as
comunicações de espíritos, não se pode duvidar; e se a tendência deles
parece ser advertir"-nos do perigo, por que não deveríamos supor que
advenham de algum agente amigo, não se questionando se é supremo, ou
inferior e subordinado, e que são dadas para nosso bem?

A presente questão me dá copiosa confirmação da justiça desse
raciocínio; pois se não tivesse eu sido cauteloso ante essa secreta
admoestação, tenha vindo de onde quer que seja, eu teria conhecido
grande dissabor e um estado muito pior do que o anterior, como se verá a
seguir.

Não permaneci parado por muito tempo, pois vi o escaler aproximar"-se da
costa, como se procurasse riacho que pudesse adentrar para a
conveniência do desembarque; no entanto, não tendo avançado o bastante,
não encontraram a pequena enseada onde antes fundeei minhas jangadas;
indo com o escaler à praia, a obra de meia milha de onde estava, o que
me foi bastante benfazejo; pois de outra forma eles teriam dado à terra
bem na minha porta, como posso chamá"-la, e logo eu teria sido expulso de
meu castelo e talvez saqueado de tudo que tinha.

Quando puseram os pés na praia, folguei bastante em ver que eram
ingleses, ao menos em sua maioria; um ou dois dos marinheiros, julguei
se tratarem de holandeses, o que não se confirmou; contando"-se ao todo
onze homens, três dos quais desarmados e, como pensei, amarrados; e
quando os primeiros quatro ou cinco deles saltaram a amurada, poiaram em
terra os ditos três na condição de prisioneiros; um dos quais, como pude
perceber, valendo"-se dos mais apaixonados gestos de súplica, aflição e
desespero, mesmo com certo exagero; e os outros dois por vezes
levantando as mãos e parecendo de fato aflitos, mas não tanto quanto o
primeiro.

A cena me perturbou muitíssimo, e não sabia qual podia ser seu
significado. Sexta"-Feira me chamou a atenção em inglês, tão bem quanto
pôde: ``Ó senhor! Ver homes inglês comer prisioneiro igual homes
selvagem.'' ``Ora, Sexta"-Feira'', disse eu, ``você acha que vão
comê"-los?'' ``Vai'', diz Sexta"-Feira, ``vai comer''. ``Não, não'', disse
eu, ``Sexta"-Feira; de fato, acredito que serão assassinados; mas esteja
certo de que não serão comidos.''

Nesse ínterim, não consegui fazer ideia do que se passava; mas tremia
diante do horror do que via, esperando pelo momento quando os três
prisioneiros seriam mortos; isto é, a certa altura eu vi um dos vilões
erguer um grande cutelo, como os marinheiros o chamam, ou espada, para
golpear um dos pobres prisioneiros; e esperava vê"-lo tombar a cada
instante; de maneira que sentia o sangue correr frio em minhas veias.

Desejei de todo coração a presença do Espanhol e do selvagem que fora
com ele, ou que algum meio houvesse de me aproximar deles à distância de
um disparo sem ser descoberto, para que pudesse resgatar os três homens,
pois não vi armas de fogo que tivessem entre si; mas me ocorreu outro
estratagema.

Depois de observar o tratamento acintoso que os três homens recebiam da
parte dos insolentes marinheiros, observei que estes começaram a se
dispersar pela ilha, como se quisessem conhecer"-lhe o interior. Observei
que os três outros homens também tinham liberdade para ir aonde lhes
aprouvesse; sentaram"-se, porém, todos os três no chão, muito
melancólicos, como homens em desespero.

Isso me fez lembrar de quando cheguei à praia e comecei a olhar ao meu
redor; e de como me dei por perdido; e de como observava,
assustadíssimo, o que me cercava; e das terríveis apreensões que sofri;
e de como encontrei agasalho na galhada de uma árvore, temendo ser
devorado por feras.

Assim como eu nada sabia naquela noite do provimento que receberia, com
o auxílio da tempestade e da maré, pela condução providencial do navio a
cerca de terra firme, do qual me alimentei e sustentei por tanto tempo;
aqueles três pobres homens abandonados nada sabiam de quão certas eram
sua libertação e suprimento, de quão perto estes estavam deles, e de
como se encontravam verdadeiramente a salvo, ao mesmo tempo em que se
pensavam perdidos e de tudo desesperados.

Muito pouco vemos diante de nós no mundo, e muitas são as razões que
temos de confiar com alegria no grande Criador do mundo, que jamais
deixa Suas criaturas em absoluto desamparo; pois, nas piores
circunstâncias, elas sempre têm algo por agradecer e, não raro, estão
mais próximas da libertação do que imaginam; quando não vêm a conhecer
seu salvamento por aquilo mesmo que as parecia levar à destruição.

Era maré cheia quando o grupo deu à praia; e enquanto alguns permaneciam
à guarda dos prisioneiros, e outros perambulavam para saber em que sorte
de lugar estavam, todos descuidaram da vazante da maré, de forma que a
água baixou e recuou consideravelmente, e o escaler encalhou.

Haviam deixado dois homens no escaler; os quais, como veio depois a meu
conhecimento, adormeceram, tendo bebido demasiada aguardente; no
entanto, tendo um deles acordado um pouco antes do outro e percebendo o
barco preso demais ao fundo para manobrá"-lo, chamou aos gritos os
companheiros que erravam pela ilha; ao que então todos correram ao
escaler; ia além de suas forças, no entanto, arrastá"-lo à água, sendo o
barco muito pesado, e a costa daquela banda da ilha tendo uma areia
macia e lamacenta, quase movediça.

Dada a situação, no que fizeram valer a condição de marinheiros, que são
as gentes menos dadas à previdência de que se tem notícia na humanidade,
os homens desistiram e tornaram a errar pela ilha; e ouvi um deles
gritar ao outro, como o chamasse para fora do barco: ``Ora, deixe isso
pra lá, Jack; na próxima cheia ele volta pra água''; confirmando
plenamente, portanto, a resposta a minha primeira pergunta, sobre a
nação a que pertenciam.

Durante todo esse tempo, mantive"-me muito escondido, sem nem uma única
vez ousar alongar"-me de meu castelo para além de meu ponto de
observação, próximo à cumeeira da colina: e fiquei feliz de constatar
quão bem fortificado eu estava; e eu sabia que levaria não menos de dez
horas para que o escaler viesse a flutuar novamente, ocasião em que
estaria escuro, e eu poderia ter mais liberdade de estudar"-lhes os
movimentos e ouvir o que diziam, caso em algum momento conversassem.

Nesse ínterim, preparei"-me para uma batalha, como o fizera antes; embora
com mais recato, sabendo que teria de lidar com inimigo distinto; e
ordenei a Sexta Feira, que eu fizera um excelente atirador, que se
carregasse de armas. Peguei duas espingardas e entreguei"-lhe três
mosquetes. Fazia eu uma figura muito formidável; vestido que estava de
minha terrível casaqueta de pele de cabra, com o grande gorro que
mencionei, a espada desembainhada ao flanco, duas pistolas na cintura e
as armas de cano longo, uma em cada ombro.

Era minha intenção, como disse acima, não fazer qualquer movimento até
que caísse a noite; mas aproximadamente duas horas, sendo o momento mais
quente do dia, descobri que todos haviam se dispersado pela mata e, como
pensei, se deitado para dormir. Os três pobres homens angustiados,
ansiosos demais com sua condição para dormir, tinham, no entanto,
encontrado agasalho debaixo de uma grande árvore, a não mais de um
quarto de milha de onde eu estava, e, como pensava, longe dos olhos de
todos os demais.

Com isso resolvi revelar"-me a eles e obter conhecimento da situação em
que se encontravam; imediatamente me pus em marcha, vestido como dito
acima, com Sexta"-Feira a uma boa distância atrás de mim, tão assustador
com suas armas quanto eu, mas sem ostentar a \emph{figura espectral} em
que me apresentava.

Aproximei"-me deles o quanto pude sem despertar"-lhes a atenção e, em
seguida, antes que me vissem, chamei"-os em espanhol: ``Quem são vocês,
cavalheiros?''

Eles se assustaram com o barulho, mas ficaram dez vezes mais perplexos
quando se deparam com a figura bárbara que tinham diante de si. Eles não
responderam, mas pensei ter percebido que correriam de mim, quando me
dirigi a eles em inglês. ``Cavalheiros'', disse eu, ``não temam; talvez
os senhores tenham um amigo próximo quando não esperavam tê"-lo.''
``Então ele é um enviado dos céus'', respondeu"-me um deles muito
gravemente, tirando o chapéu ao mesmo tempo em sinal de respeito; ``pois
nossa condição já se encontra além de todo o auxílio do homem.'' ``Todo
auxílio vem do céu, senhor'', retorqui eu, ``mas poderiam os senhores
colocar este estranho que aqui se apresenta a par da situação, pois
ambos parecem viver grande aflição? Vi quando vocês deram à praia; e
quando pareceram fazer súplicas aos brutos que vieram com vocês. Vi um
deles erguer a espada para matá"-los.''

O pobre homem, trêmulo, assustado, com o rosto coberto de lágrimas,
respondeu: ``Estou falando com Deus ou homem? É um homem real ou um
anjo?'' ``Não tema por isso, senhor'', disse eu; ``se Deus tivesse
enviado um anjo para aliviá"-lo, ele teria vindo em melhores trajes e
armado de maneira diversa desta em que vocês me veem; peço que abandonem
o medo; sou inglês e estou disposto a ajudá"-los; vejam que tenho apenas
um servo; temos armas e munição; diga"-nos livremente, podemos dar"-lhes
auxílio? O que se passa?''

``Nossa história, senhor'', respondeu ele, ``é muito longa para ser
contada enquanto nossos assassinos se encontram tão perto; mas, em
poucas palavras, senhor, eu era o comandante daquele navio; minha
tripulação se amotinou contra mim; com muito custo, foram convencidos a
não me assassinar e, por fim, me abandonaram, tendo ao meu lado apenas
esses dois homens, meu imediato e um passageiro, nesta praia deserta,
onde esperavam que morrêssemos, acreditando ser lugar desabitado e sem
ainda ter dele ideia clara.''

``Onde estão essas feras, seus inimigos?'', perguntei"-lhes eu; ``você
sabe para onde eles foram?'' ``Lá estão eles, senhor'', disse ele,
apontando a um bosque de árvores; ``meu coração estremece de medo de que
nos tenham visto e o ouvido falar; se assim é, certamente nos matarão a
todos.''

``Eles têm armas de fogo?'', perguntei eu. O homem respondeu que tinham
apenas duas, uma das quais haviam deixado no escaler. ``Bem, então'',
disse eu, ``deixe o resto comigo; Vejo que estão dormindo; é coisa fácil
matá"-los todos; mas devemos fazê"-los prisioneiros?'' Ele me disse que o
grupo tinha dois bandidos que nada tinham a perder e que não era seguro
demonstrar"-lhes misericórdia; mas que se estes fossem presos, acreditava
ele que os demais voltariam ao seu dever. Eu perguntei a ele quais eram.
Ele disse que não era capaz de distingui"-los àquela distância; mas que
obedeceria às minhas ordens quaisquer fossem elas. ``Bem'', disse eu,
``vamos nos afastar para longe de onde seus olhos e ouvidos alcancem, de
maneira que não despertem, e tratamos com mais vagar disso''. Então eles
recuaram de bom grado ao meu lado, até que a floresta nos escondeu.

``Senhor'', disse eu, ``caso eu me arrisque em sua libertação, você
estará disposto a aceitar duas condições minhas?'' Ele as antecipou,
dizendo que tanto ele quanto o navio, uma vez recuperados, haveriam de
permanecer sob meu comando e ordens no que dissesse respeito a tudo; e
que, se o navio não fosse recuperado, ele viveria e morreria comigo em
qualquer parte do mundo para onde eu o mandasse; ao que os outros dois
homens disseram o mesmo.

``Bem'', disse eu, ``minhas condições são apenas duas: 1, que enquanto
estiver nesta ilha comigo, o senhor não reclamará qualquer autoridade; e
que se eu lhe der armas em mãos, você as irá, em todas as ocasiões,
devolver"-me, e não causará nenhum mal a mim ou aos meus nesta ilha, e
nesse ínterim será governado por minhas ordens.

2, que se o navio for ou puder ser recuperado, você levará a mim e meu
homem à Inglaterra livres de qualquer cobrança.''

Ele me deu todas as garantias que a invenção ou fé do homem pudessem
conceber de que ele cumpriria com essas exigências absolutamente
razoáveis e, além disso, conservaria um sentimento de dívida para
comigo, por lhe ter salvo a vida, e o reconheceria em todas as ocasiões
enquanto vivesse.

``Bem'', disse eu, ``eis aqui três mosquetes para vocês, com pólvora e
bala; digam"-me então o que vocês acham que deve ser feito.'' Deu"-me ele
todos os testemunhos possíveis de sua gratidão; e se ofereceu a ser
totalmente conduzido por mim. Disse"-lhe que achava muito difícil tentar
qualquer coisa; mas que o melhor método que podia conceber era atirar em
todos de uma só vez, enquanto descansavam; e que, se qualquer um deles
por acaso não fosse morto na primeira descarga de munição e se rendesse,
poderíamos salvá"-lo e, assim, confiar totalmente à Providência de Deus a
direção das balas.

Respondeu"-me ele, com bom senso, que não os mataria, se o pudesse
evitar; mas que dois deles eram arrematados canalhas, e haviam sido os
autores de todo o motim no navio, e que, se eles escapassem, nossa
derrota era certa; pois eles retornariam a bordo e trariam toda a
companhia do navio e nos aniquilariam a todos. ``Bem, então'', disse eu,
``a necessidade legitima meu conselho, pois é a única maneira de salvar
nossas vidas''. No entanto, vendo"-o ainda cauteloso quanto ao
derramamento de sangue, disse"-lhe que eles deveriam cuidar eles mesmos
do assunto, tal como o julgassem conveniente.

Enquanto assim conversávamos, ouvimos alguns deles despertarem, e logo
em seguida vimos dois deles de pé; ao que lhe perguntei se eram eles os
líderes do motim. Respondeu"-me que não. ``Bem, então'', disse eu,
``deixem"-nos escapar; pois a Providência parece tê"-los despertado com o
propósito de que se salvassem. Portanto'', eu disse, ``se os demais lhes
escaparem, a responsabilidade é toda de vocês''.

Tomado de ânimo, então, ele empunhou o mosquete que lhe dera, trazendo a
pistola ao cinto e mais seus dois companheiros, cada qual com uma arma
em punho; e os dois homens que iam com ele, estando à frente, foram os
primeiros a descarregar, ao que um dos marinheiros que estava acordado
virou"-se e, ao vê"-los chegar, gritou aos demais; mas era tarde; pois no
instante em que vozeou em aviso, eles atiraram; quero dizer, os dois
homens, visto que o capitão sabiamente poupou a própria arma; e tão bem
eles haviam disparado contra os homens que conheciam, que um deles
quedou morto ali mesmo, e o outro muito ferido; e este, não estando
morto, pôs"-se de pé num pulo e pediu veementemente auxílio ao outro; mas
o capitão, aproximando"-se, disse"-lhe que era tarde demais para pedir
socorro, que devia dirigir seu pedido de perdão a Deus por sua vilania,
e tendo"-o dito abateu"-o com a coronha do mosquete, de maneira que ele
nunca mais falou; havia outros três na companhia, e um deles estava
levemente ferido. A esta altura eu já me fazia presente; e quando viram
o perigo que corriam, e quão inútil seria resistir, imploraram por
misericordia; ao que o capitão lhes disse que pouparia suas vidas se lhe
garantissem o arrependimento da traição que haviam perpetrado e lhe
jurassem ser fieis na recuperação do navio e, posteriormente, levá"-lo de
volta à Jamaica, que era de onde vinham. Eles lhe deram todas as
demonstrações de sinceridade que se poderiam desejar; e ele estava
disposto a lhes dar fé e poupar"-lhes as vidas, ao que não me opunha,
tendo"-lhe instado que os mantivesse de mãos e pés atados enquanto na
ilha estivessem.

Enquanto isso se passava, enviei Sexta"-Feira com o imediato do capitão
ao barco sob ordens de apreendê"-lo e trazer"-lhes os remos e as velas, o
que fizeram; e por fim, os três homens que erravam pela ilha, os quais
se encontravam, para sua felicidade, separados dos demais, retornaram ao
som da salva de tiros disparados; e vendo o capitão, que antes era seu
prisioneiro, tornado seu senhor, submeteram"-se também às amarras; e
assim nossa vitória se fez completa.

Restava que o capitão e eu expuséssemos um ao outro os pormenores de
nossas circunstâncias; e, começando eu, contei"-lhe toda a minha
história, que ele ouviu atentamente e não sem espanto, em especial ante
as maravilhas que cercaram meu apercebimento de munição e víveres; e, de
fato, sendo minha história uma fiada de sucessos de grande admiração,
ela tocou"-lhe profundamente; e quando ponderou a partir dela sobre sua
própria situação, e sobre como eu parecia ter sido ali poupado com o
propósito de salvar"-lhe a vida, as lágrimas cobriram"-lhe o rosto e ele
calou.

Encerrada a conversação, levei"-o, assim como seus dois homens, à minha
morada, conduzindo"-os exatamente ao ponto de onde saí, a saber, o alto
da casa, na qual os revigorei com os mantimentos que tinha e lhes
mostrei todos os instrumentos que havia fabricado durante os anos de
minha longuíssima permanência naquele lugar.

Tudo o que lhes mostrei, tudo que lhes disse, foi recebido com absoluto
assombro; mas, acima de tudo, o capitão admirou"-se de minha
fortificação, e da perfeição com que escondi meu retiro com um
bosquezinho, o qual, então contando quase vinte anos de seus plantio,
com suas árvores crescendo muito mais depressa do que na Inglaterra,
tornara"-se mata tão cerrada que era intransponível por onde quer que
nele se tentasse entrar, exceto pela banda em que conservara minha
passagem sinuosa; e eu lhe disse que aquele era meu castelo e minha
residência, mas que tinha sede no campo, como a maioria dos príncipes,
aonde me recolher quando dada a ocasião, e a qual lhe apresentaria
noutro momento; uma vez que devíamos tratar, então, de como recuperar o
navio. Ele concordou comigo quanto a isso, mas me disse que não sabia
quais medidas tomar; pois ainda havia vinte e seis marinheiros a bordo
que, tendo"-se associado em maldita conjuração, pela qual a lei lhes
cobraria a vida, a encrudesceriam por não ter mais a perder; e com ela
iriam às últimas consequências, sabendo que, se fossem subjugados,
seriam levados à forca tão logo desembarcassem na Inglaterra, ou em
qualquer uma das colônias inglesas, e que, portanto, não havia maneira
de atacá"-los com um número tão reduzido quanto o nosso.

Meditei algum tempo sobre o que ele havia dito e concluí que se tratava
de juízo muito racional e que, portanto, algo deveria ser pensado sem
demora, de maneira a atrair os homens a bordo para alguma armadilha para
surpreendê"-los e evitar que desembarcassem sobre nós, e nos destruíssem;
ocorrendo"-me, então, que em pouco tempo a tripulação do navio, ao se
perguntar o que teria acontecido com seus companheiros e com o escaler,
certamente daria à costa a sua procura, e que então, talvez viessem em
seu outro bote armados e fossem fortes demais para nós; o que ele julgou
possível.

Diante disso, eu lhe disse que a primeira coisa que tínhamos a fazer era
abrir buracos no casco do escaler que estava na praia, para que não o
carregassem; e destituí"-lo de tudo que o fizesse útil, de sorte a
tornar"-se imprestável à navegação; e, assim, subimos a bordo,
tomamo"-lhes as armas restantes a bordo, e tudo quanto lá encontramos, a
saber, uma garrafa de aguardente e outra de rum, alguns biscoitos, um
chifre de pólvora e um pão grande de açúcar em um pedaço de vela;
pesando o pão de açúcar cinco ou seis libras; todas essas coisas sendo a
mim muito bem"-vindas, principalmente a aguardente e o açúcar, que já não
tinha havia muitos anos.

Depois de carregadas todas essas coisas para a costa (os remos, o
mastro, a vela e o leme do escaler foram carregados antes, como dito
atrás), abrimo"-lhe um enorme buraco na popa, de maneira que, mesmo se
dessem à praia com força o bastante para destruir"-nos, não conseguiriam
levar o barco.

Em verdade, não tinha muita confiança de que fôssemos capazes de
recuperar o navio; mas era minha opinião que, caso partissem eles sem o
escaler, eu não teria muita dificuldade para torná"-lo novamente apto a
levar"-nos às ilhas à sotavento e encontrar nossos amigos, os espanhóis,
no caminho; pois eu ainda os tinha em meus pensamentos.

Enquanto assim estávamos levando a cabo nossos planos, e tínhamos antes
de tudo, à força bruta, arrastado o barco a uma altura tal da praia que
o mar não o tocasse na alta da maré; e, além disso, havíamos aberto um
buraco em seu casco, grande demais para ser prestes consertado, e por
fim meditávamos sobre o que deveríamos fazer; escutamos um disparo de
canhão vindo do navio e vimos um galhardete sendo arvorado em sinal para
que o escaler retornasse a bordo; mas este permaneceu imóvel; e outros
disparos foram feitos e outros sinais erguidos em direção ao escaler.

Por fim, quando todos os seus sinais e disparos quedaram infrutuosos, e
eles descobriram que o barco não deixava a praia, observamo"-los com a
ajuda de minha luneta içar outro bote e remá"-lo em direção à costa; e
notamos, ao se aproximarem, que não eram menos de dez homens nele, e que
traziam consigo armas de fogo.

Estando o navio a obra de duas léguas da praia, tínhamos deles uma
perspectiva completa enquanto vinham, e uma clara visão até mesmo de
seus rostos; visto que a maré os havia carregado um pouco a leste do
outro barco, eles tiveram de remar contra a corrente para chegar ao
mesmo ponto onde havia fundeado, e estava, o escaler.

Assim, portanto, tínhamos uma perspectiva completa deles, e o capitão
reconhecia a figura e sabia o caráter de cada um dos homens no bote, dos
quais, disse ele, havia três companheiros muito honestos, que, tinha ele
certeza, haviam sido levados àquela conspiração pelos demais, sendo
dominados pela força e pelo medo.

Quanto ao contramestre, que parecia ser o oficial comandante entre eles,
e todos os demais, eram tão destemperados dentre os membros da
tripulação e, sem dúvida, nada tinham a perder em sua aventura; e ele
estava terrivelmente apreensivo de que fossem fortes demais para nós.

Sorri para ele e disse"-lhe que homens em nossas circunstâncias estavam
além da ação do medo; que, ao perceber que praticamente todas as
condições em que poderíamos estar seriam melhores do que a em que nos
encontrávamos, era de se esperar que a consequência, fosse morte ou a
vida, certamente seria uma libertação; e perguntei"-lhe o que pensava ele
das circunstâncias de minha vida e se não valia a pena me arriscar em
nome de minha libertação. ``E onde, senhor'', disse eu, ``está a sua
crença de que fui preservado aqui com o propósito de salvar"-lhe a vida,
que lhe deu ânimo há pouco? De minha parte'', prossegui, ``parece haver
apenas uma coisa errada em toda a situação''. ``E o que é?'',
perguntou"-me ele. ``Ora'', respondi eu, ``é que há, como você diz, três
ou quatro homens honestos entre eles, os quais deveriam ser poupados;
estivessem todos na banda pérfida da tripulação, eu concluiria que a
providência de Deus os teria assinalado para entregá"-los a seu
discernimento; pois acredite que todo homem que der à praia é nosso e há
de morrer ou viver segundo seu comportamento para conosco.''

Ao falar"-lhe essas coisas com voz elevada e semblante alegre, creio que
lhe infundi grande coragem; por isso nos entregamos de corpo e alma à
ação, tendo considerado, à primeira aparição do bote vindo do navio,
separar nossos prisioneiros; e, tendo"-os, assim, prendido sem falha.

Dois deles, em relação aos quais o capitão não tinha tanta segurança,
deixei aos cuidados de Sexta"-Feira, que ao lado de um dos três homens
libertados os levou à minha caverna, onde permaneceram bastante
distantes e fora do perigo de serem ouvidos ou descobertos, ou de
encontrar caminho para fora da floresta, caso conseguissem se libertar;
e ali ficaram amarrados, não sem provisões; e eu lhes prometi que, se
ali continuassem em silêncio, lhes daria a liberdade em um ou dois dias;
mas que, se tentassem escapar, seriam mortos sem misericórdia. De sua
parte, prometeram suportar fielmente o confinamento com paciência, e
ficaram muito gratos por terem recebido tão bom tratamento a ponto de
receberem provisões e lume; pois Sexta"-Feira deu"-lhes velas de nossa
própria fabricação para seu conforto; e sem que soubessem fez guarda à
boca do local.

Os outros prisioneiros foram agraciados com melhor tratamento; de fato,
dois deles foram mantidos presos, pois o capitão não lhes tinha
confiança; mas outros dois, alistei ao meu serviço por recomendação do
capitão, e por seu solene comprometimento de viver e morrer conosco; e,
assim, com estes e os outros três homens honrados, éramos sete ao todo,
e bem armados; e não tive dúvidas de que saberíamos lidar bem com os dez
que chegavam, visto que o capitão dissera que também havia três ou
quatro homens honestos entre eles.

Assim que deram ao local onde estava o escaler, os assim chegados
conduziram o bote à praia e desembarcaram todos, puxando o barco atrás
de si, o que me deixou bastante satisfeito, pois temia que preferissem
ter fundeado o bote a alguma distância da costa, deixando alguns homens
para protegê"-lo; de sorte que não seríamos capazes de tomar o barco.

Já em terra firme, a primeira coisa que fizeram foi correr em direção ao
outro barco; e não era difícil observar que lhes era grande o espanto em
encontrá"-lo subtraído, como dito atrás, de tudo o que antes nele havia,
e com um grande buraco na popa.

Depois de um pouco de reflexão sobre o que viam, emitiram dois ou três
gritos a plenos pulmões, tencionando que seus companheiros atentassem a
sua chegada; mas foram em vão; então, formaram um círculo e dispararam
uma salva de tiros com suas pistolas, a qual de fato ouvimos e cujos
ecos se fizeram ouvir na mata; o resultado, porém, foi o mesmo, uma vez
que os presos na caverna, estávamos nós certos, não os puderam ouvir; e
os que estavam sob nossa guarda, embora os tivessem ouvido bastante bem,
não ousaram dar"-lhes resposta.

Fora tamanho o assombro ante o que viram, que, conforme depois nos
relataram, decidiram retornar todos a bordo do navio e informar os que
ali haviam permanecido que a tripulação do escaler fora inteira
assassinada, e seu casco furado; assim, arrastaram o bote de volta à
água e embarcaram todos.

Foi grande o espanto do capitão, e mesmo a perplexidade ante o que via;
pois acreditou ele que os marinheiros embarcariam no navio para partir,
dando seus companheiros por mortos; de forma que perderia, então, a
embarcação que, a princípio, tinha a esperança de recuperar; mas ele não
tardaria para ficar assustado pela razão oposta.

Não se passara muito tempo desde que haviam partido no bote, quando
percebemos que voltavam à praia; porém, com este novo proceder, em torno
da qual pareciam todos ter deliberado; a saber, o de deixar três homens
no bote, enquanto os demais iam à praia e adentravam a mata à procura
dos companheiros.

A novidade nos causou grande decepção, pois já não sabíamos o que fazer,
visto que capturar aqueles sete homens em terra não nos seria vantagem
se deixássemos o bote escapar; pois os remanescentes remariam ao navio,
e então os demais decerto ponderariam e zarpariam, tornando a
recuperação do navio impossível.

Não havia, porém, remédio senão esperar e ver o que o desenrolar dos
acontecimentos proporcionaria, e assim o fizemos; e os sete homens
chegaram à praia, e os três que permaneceram no bote levaram"-no a uma
boa distância da costa e fundearam para esperá"-los; de guisa que nos era
impossível chegar aos que estavam no bote.

Os que deram à praia mantiveram"-se juntos, indo ao alto da pequena
colina sob a qual ficava minha habitação; e os podíamos ver claramente,
embora eles não nos pudessem avistar; e teríamos estado muito contentes
se eles tivessem se aproximado de nós, para que pudéssemos disparar
contra eles, ou que tivessem se afastado, para que pudéssemos sair de
onde estávamos.

Mas quando chegaram ao alto da colina, onde puderam avistar uma longa
trilha aos vales e bosques, que ficavam a nordeste, e onde a ilha ficava
mais baixa, foi enorme o alarido que produziram, até que cansaram; e
temendo, ao que parecia, alongarem"-se da costa em excursão, ou distantes
uns dos outros, eles se sentaram um ao lado do outro sob uma árvore para
pensar no que fariam. Tivessem eles julgado adequado dormir ali, como
outros deles o haviam feito, muitos nos teriam sido os trabalhos
poupados; mas estavam por demais apreensivos do perigo para
arriscarem"-se a dormir, embora não soubessem qual era o perigo a ser
temido.

Ante a deliberação em que se encontravam, o capitão me fez proposta
muito sensata; a saber, que talvez todos disparassem uma nova salva, com
o intuito de se fazerem ouvir por seus companheiros, e que então
deveríamos avançar sobre eles nesse preciso instante, quando suas armas
se encontrassem todas descarregadas; e eles certamente se renderiam, e
nós os capturaríamos sem derramamento de sangue. A proposta me agradava,
desde que a levássemos a cabo quando bastante próximos para atacá"-los
sem que recarregassem suas armas.

Mas a dita salva não aconteceu; e permanecemos parados por um longo
tempo, bastante indecisos sobre o curso a seguir; por fim, disse"-lhes
que em minha opinião nada seria feito até a noite; e que se, então, eles
não retornassem ao bote, talvez pudéssemos encontrar um meio de nos
posicionar entre eles e a costa e, assim, usar de algum artifício contra
os que permaneciam no mar para atraí"-los à praia.

Esperamos muito tempo, ainda que muito impacientes, por sua partida; e
ficamos muito intranquilos quando, depois de longa deliberação, todos se
levantaram e puseram"-se em marcha rumo ao mar; movidos, aparentemente,
por tão terrível temor dos perigos do lugar que haviam decidido retornar
ao navio, dar os companheiros como perdidos e prosseguir, então, com a
viagem planejada.

Assim que percebi caminhavam em direção à praia, imaginei o que de fato
se mostrou ser, que davam a busca por encerrada e retornavam; e o
capitão, assim que lhe confidenciei meus pensamentos, entregou"-se a
grande aflição; mas logo pensei em um plano para trazê"-los de volta, o
qual correspondeu perfeitamente a meus fins.

Ordenei que Sexta"-Feira e o imediato do capitão atravessassem o riacho a
oeste, em direção ao local onde os selvagens desembarcaram quando
Sexta"-Feira foi resgatado; e que, assim que chegassem a um pequeno
aclive, a não mais de meia milha de distância, vozeassem tão alto quanto
pudessem, e esperassem até que os marinheiros os escutassem; que tão
logo escutassem a resposta dos marinheiros, tornassem a gritar; e,
então, mantendo"-se fora de vista, circulassem, sempre devolvendo seus
chamados para, então, os atraírem o mais longe possível mata adentro, e
então retornassem até mim pelo caminho que os indicara.

Eles subiam no bote quando Sexta"-Feira e o imediato gritaram; e logo que
os ouviram, os marinheiros responderam e correram ao longo da costa para
o oeste, em direção à voz que ouviram, quando se depararam com o riacho,
cuja água subia, de maneira que não podiam prosseguir, e então chamaram
pelo bote para que os atravessasse, como eu esperava.

Uma vez que chegaram à outra margem, observei que o bote seguiu um bom
trecho ribeira acima; e encontrando como que um ancoradouro, um dos três
homens desembarcou para acompanhar os demais, de sorte que apenas dois
permaneceram no barco, que fora amarrado ao toco de uma pequena árvore.

Era tudo que eu desejava; e deixando que Sexta"-Feira e o imediato do
capitão cuidassem do que lhes havia sido pedido, encarreguei"-me dos
demais homens; e, cruzando o riacho sem chamar a atenção, surpreendemos
os dois homens antes que percebessem; um deles deitado na margem, e o
outro no barco; sendo que o primeiro, dormitando em terra, tão logo fez
o gesto de erguer, recebeu o capitão, que ia à frente e saltou"-lhe em
cima, derrubando"-o para, em seguida, gritar ao homem no barco que se
rendesse, caso contrário seria um homem morto.

Não foram precisos muitos argumentos para persuadir o homem sozinho a
bordo do bote, quando se viu diante de cinco outros, e tendo seu
companheiro posto fora de combate; ademais, este era, ao que parecia, um
dos três que não estavam tão empenhados no motim quanto o restante da
tripulação e, portanto, não só foi facilmente persuadido a se render,
como, depois, a se unir a nós, o que fez com muita sinceridade.

Nesse ínterim, Sexta"-Feira e o imediato do capitão cuidaram muito bem do
que lhes fora solicitado em relação aos demais, atraindo"-os, com
chamados e respostas, de uma colina a outra, e de um bosque a outro, até
que não só os cansaram profundamente, como os deixaram num ponto do
qual, estavam certos, não conseguiriam retornar ao bote antes de
escurecer; e, de fato, eles próprios também estavam muito cansados
quando nos encontraram.

Restava"-nos, então, apenas fazer vigília, na escuridão, para
surpreendê"-los e impor"-lhes nosso domínio.

Eles chegaram ao bote apenas muitas horas depois de Sexta"-Feira retornar
a mim; e podíamos ouvir os que vinham à frente, muito antes de se darem
à vista, chamando os da retaguarda para que se apressassem; e também
podíamos ouvir em resposta os queixumes de quão cansados estavam e de
que mais rápido não podiam caminhar; o que recebemos como uma boa
notícia.

Por fim, chegaram ao bote: mas é impossível expressar sua confusão
quando se depararam com a embarcação presa e encalhada, a maré vazante,
e os dois homens ausentes; e podíamos ouvi"-los chamar uns aos outros em
grande consternação, dizendo entre si que estavam em uma ilha encantada;
e que ou ela era habitada, e eles seriam todos assassinados, ou nela
havia demônios e espíritos, e todos acabariam por ser levados e
aniquilados.

Eles gritaram de novo e muitas vezes chamaram seus dois companheiros
pelos nomes; mas não obtiveram resposta. Depois de algum tempo, podíamos
vê"-los, sob a pouca luz que havia, indo de um lado para o outro, as mãos
agarradas umas às outras como fazem os homens em desespero; e por vezes
iam e sentavam"-se no barco para descansar, para logo retornar à margem e
perambular sem rumo como antes.

Meus homens queriam que lhes desse permissão para atacá"-los na
escuridão; mas eu preferia avançar em uma melhor posição, de maneira a
poupá"-los e matá"-los em menor número possível; e, sobretudo, a não
arriscar a vida de nossos homens, pois sabia que os outros estavam muito
bem armados. Decidi esperar para ver se o grupo não se separaria; e,
desse modo, para dominá"-los, aproximei minha emboscada e ordenei que
Sexta"-Feira e o capitão se arrastassem para o mais perto que pudessem do
lugar em que estavam sem despertar"-lhes a atenção e que, então, se
postassem o mais próximo possível antes de atirar.

Não estavam havia muito nessa posição quando o contramestre, que era o
principal líder do motim e então se mostrava o mais abatido e desanimado
dentre todos, caminhou em sua direção, acompanhado de outros dois
marinheiros; e tão ansioso o capitão estava para colocar as mãos naquele
arrematado patife que mal conseguiu esperar que ele se aproximasse para
atacá"-lo; uma vez que só conseguiam ouvir sua voz; mas quando por fim se
encontravam perto, o capitão e Sexta"-Feira colocaram"-se de pé e
dispararam.

O contramestre morreu no local; a descarga acertou ainda um dos outros
dois homens, que tombou bem ao seu lado, embora só tenha morrido uma ou
duas horas depois; e o terceiro fugiu.

Ao barulho dos disparos, avancei imediatamente com todo o meu exército,
que agora contava oito homens, a saber, eu mesmo, o
\emph{Generalíssimo}; Sexta"-Feira, meu tenente"-general; o capitão e seus
dois homens; e os três prisioneiros de guerra aos quais havíamos
confiado armas.

Quando os encontramos, já fazia escuro, de sorte que não puderam atentar
a nosso número; e ordenei que o homem que havíamos deixado no barco, e
que agora tinha parte em nosso grupo, os chamasse pelos nomes, de guisa
que eu pudesse tentar trazê"-los a uma conversação e, então, celebrar um
acordo, o que ocorreu exatamente como desejávamos: pois, de fato, não
era difícil pensar que, dada a condição em que se encontravam, estariam
eles muito dispostos a capitular; e assim ele chamou um deles tão alto
quanto pôde: ``Tom Smith! Tom Smith!''; ao que este respondeu
imediatamente: ``Quem é? Robinson?'', pois parecia ter reconhecido a voz
que o chamava; e esta lhe disse, ``Sim, sim; pelo amor de Deus, Tom
Smith, mãos ao alto e renda"-se, ou vocês todos serão agora mesmo
mortos''.

``A quem devemos nos render? Onde eles estão?'', quis saber Smith;
``Aqui estão eles'', foi a resposta, ``aqui está nosso capitão e
cinquenta homens com ele, e eles o estão caçando há duas horas; o
contramestre está morto; Will Fry está ferido, e eu fui feito
prisioneiro; e se vocês não se renderem, todos morrerão''.

``Eles vão nos dar clemência, então'', perguntou Tom Smith, ``e nós nos
renderemos''; ``Vou perguntar, se vocês se renderem'', disse Robinson; e
então ele perguntou ao capitão, e o próprio capitão anunciou"-se:
``Smith, você conhece minha voz; se vocês depuserem suas armas
imediatamente e se renderem, terão as vidas poupadas; todos menos Will
Atkins.''

Diante disso, Will Atkins bradou: ``Pelo amor de Deus, misericórdia; o
que eu fiz, capitão? Todos agiram tão mau quanto eu''; o que, a
propósito, não era verdade; pois parece que esse Will Atkins foi o
primeiro homem a colocar as mãos no capitão quando se amotinaram, e o
tratou barbaramente, amarrando"-lhe as mãos e usando de palavras
injuriosas; no entanto, disse"-lhe o capitão que deveria depor as armas
segundo o que a consciência lhe ditasse e que confiasse na clemência do
governador; com o que ele se referia a mim, pois todos me chamavam de
governador.

Em suma, todos depuseram as armas e pediram clemência; e despachei o
homem que havia negociado com eles e outros dois para que os amarrassem
todos; e então meu grande exército de cinquenta homens, que, incluindo
aqueles três homens, contava apenas oito, avançou e apreendeu a todos e
a seu barco; descontados a mim e mais um, que nos mantivemos às
escondidas por razões de estado.

Nosso próximo trabalho era o de consertar o bote e pensar em um plano de
captura do navio: e quanto ao capitão, agora ele estava à vontade para
travar diálogo com seus homens; ele protestou contra a vilania das
práticas com que o vitimaram, e tratou longamente da maldade de suas
intenções e de como certamente ela por fim os levaria a dissabores e
aflições, e talvez à forca.

Todos pareciam muito arrependidos e imploravam por suas vidas; quanto a
isso, respondeu"-lhes que não eram seus prisioneiros, mas do comandante
da ilha; que eles haviam pensado que o abandonariam em uma ilha deserta
e desabitada; mas que havia sido desígnio de Deus que tivessem dado em
praias habitadas e que seu governador era um inglês; e que ele os
poderia enforcar todos ali, se fosse essa sua vontade; mas que, uma vez
que ele lhes havia concedido clemência, supunha ele que os enviaria à
Inglaterra para ali serem colocados sob os devidos ritos legais, com
exceção feita a Atkins, a quem aconselhava, segundo ordens do
governador, preparar"-se para a morte, pois seria enforcado de manhã.

Embora tudo não passasse de ficção sua, atingiu"-se o efeito almejado;
pois Atkins caiu de joelhos implorando ao capitão que intercedesse junto
ao governador por sua vida; e todos os demais lhe suplicaram, pelo amor
de Deus, que não fossem enviados à Inglaterra.

Ocorreu"-me então que havia chegado o tempo de nossa libertação e que
seria muito fácil fazer com que aqueles homens se empenhassem na
recaptura do navio; assim, recolhido à escuridão para que não vissem a
figura do governador que tinham, solicitei a presença do capitão; quando
chamei, estando eu a boa distância, um dos homens, a quem cabia repetir
minhas palavras ao capitão e dizer, ``Capitão, o comandante chama''; ao
que o capitão logo respondeu, ``Diga a Sua Excelência que já irei a seu
encontro''. Isso bastou para enganá"-los; pois todos acreditaram que o
comandante estava próximo, com seus cinquenta homens.

Quando o capitão veio até mim, contei"-lhe meu projeto de apreensão do
navio, do qual ele gostou muito, e decidi colocá"-lo em execução na manhã
seguinte.

Mas, a fim de executá"-lo com mais arte e certeza de bom sucesso,
disse"-lhe que era preciso que dividíssemos os prisioneiros, e que ele
pegasse Atkins e mais dois dos piores e os enviasse em correntes à
caverna em que os outros estavam. Ficaram encarregados da tarefa
Sexta"-Feira e os dois homens que vieram a terra com o capitão.

Eles os levaram à caverna como a uma prisão: e era, sem dúvida, um lugar
medonho, especialmente a homens em suas condições.

Os outros eu mandei à minha cabana, como a chamava, e da qual dei
descrição completa; e como era ela cercada, e eles estavam presos, o
lugar mostrava"-se bastante seguro, considerando que demonstravam bom
comportamento.

Ao encontro destes despachei pela manhã o capitão, para que travasse
conversação com eles; e, em uma palavra, que os ouvisse e julgasse, e
então me dissesse se entendia serem eles dignos de confiança ou não para
subir a bordo e dar assalto ao navio. Ele tratou com os presos sobre as
ofensas a ele dirigidas; sobre a condição em que se encontravam; e
disse"-lhes que, embora o governador lhes tivesse dado clemência, quanto
à presente ação, caso fossem enviados à Inglaterra, todos quedariam
decerto expostos no patíbulo; mas que, uma vez que se unissem àquela tão
justa manobra de recuperação do navio, o governador se comprometeria em
interceder em seu favor.

Qualquer um pode imaginar quão prontamente tal proposta seria aceita
pelos homens em sua condição; prostraram"-se de joelhos diante do capitão
e prometeram, com as mais profundas juras, que lhe seriam fiéis até a
última gota de sangue e gratos pela vida e que seguiriam com ele por
todo o mundo e o teriam como um pai enquanto vivessem.

``Bem'', respondeu o capitão, ``devo retornar e relatar ao governador o
que os senhores disseram e ver o que posso fazer para que ele dê seu
consentimento''; e então ele me trouxe um relato da disposição em que os
encontrara, e que de fato acreditava que eles nos seriam fiéis.

No entanto, para que pudéssemos ficar muito seguros, pedi"-lhe que
retornasse e destacasse cinco deles e lhes dissesse que compreendessem
que não os escolhia pela falta de homens, mas porque deles faria seus
auxiliadores, e que o governador conservaria sob custódia os outros
dois, além dos três que se encontravam no castelo (minha caverna)
encarcerados, como reféns da lealdade daqueles cinco; e que, caso se
revelassem infiéis na execução do plano, os cinco reféns seriam
acorrentados vivos a postes na praia e lá seriam abandonados.

Isso pareceu"-lhes duro e os convenceu de que o governador falava sério;
de qualquer modo, não havia alternativa para eles, senão aceitar; pois
cabia agora aos prisioneiros, tanto quanto ao capitão, persuadir os
outros cinco a cumprirem seu dever.

Nossa força estava assim organizada para a expedição: 1, o capitão, seu
imediato e o passageiro; 2, os dois prisioneiros da primeira chegada,
recomendados pelo capitão, aos quais dei liberdade e confiei armas; 3,
os outros dois que até então mantivera amarrados em minha cabana, mas
que haviam sido soltos por moção do capitão; 4, os cinco finalmente
liberados; de guisa que eram doze ao todo, além de cinco que mantivemos
presos na caverna, na condição de reféns.

Perguntei ao capitão se ele estava disposto a dar assalto a bordo do
navio com aquele grupo de homens; pois, quanto a mim e a Sexta"-Feira,
não julgava que devêssemos sair em excursão com sete homens deixados
para trás; e não nos custava pouco esforço mantê"-los presos e
fornecer"-lhes alimento.

Quanto aos cinco da caverna, decidi mantê"-los presos, mas Sexta"-Feira ia
duas vezes ao dia a seu encontro, para suprir"-lhes do que fosse
necessário; e fiz com que os outros dois carregassem os mantimentos a
certa distância, de onde Sexta"-Feira deveria levá"-los.

Quando me apresentei aos dois reféns, estava ao lado do capitão, que
lhes disse que eu era a pessoa designada pelo governador para cuidar
deles; e que era do agrado do governador que não fossem a parte alguma,
senão sob minha orientação; e que, se o fizessem, seriam levados ao
castelo e postos a ferros: de sorte que, como jamais permitimos que me
vissem como governador, agora lhes aparecia como pessoa diversa, e
falava do governador, da guarnição, do castelo e assim por diante em
todas as ocasiões.

Não restavam dificuldades a obstar o caminho do capitão, apenas o
trabalho de aparelhar e esquipar seus dois barcos e dar conserto a um
deles; e ele nomeou o antigo passageiro capitão de um deles, ao qual
destacou quatro dos homens; e ele e seu imediato, além da companhia de
cinco outros homens, tripularam o outro; e eles organizaram muito bem
suas tarefas, pois chegaram ao navio por volta da meia"-noite. Assim que
chegaram a cerca do navio, ordenou que Robinson chamasse os marinheiros
de vigília e lhes dissesse que haviam recuperado os homens e o barco,
mas que demorara muito para que os encontrassem e assim por diante;
mantendo a conversa até que tocassem o casco do navio; quando o capitão
e o imediato invadiram"-no armados, derrubaram de pronto o segundo
imediato e o mestre"-carpinteiro com a coronha de seus mosquetes,
recebendo o leal apoio de seus homens; e prenderam todos os demais que
estavam no convés principal e no tombadilho, fechando os alçapões para
manter nas cobertas inferiores os que nelas se encontravam, quando o
outro bote e sua tripulação, invadindo a embarcação pelas enxárcias de
proa, dominaram o castelo de proa; e a escotilha que dava acesso à
cozinha, aprisionando os três homens que lá se encontraram.

Feito isso, e com tudo sob o devido controle no convés, o capitão
ordenou ao imediato, com três homens, que invadissem a cabine de popa,
onde descansava o novo capitão rebelde, que, ao escutar o alarido,
levantara"-se e, com dois homens e um grumete, empunharam armas; e quando
o imediato, com um pé de cabra, arrombou a porta, o capitão rebelde e
seus homens dispararam ferozmente contra os nossos, ferindo o imediato
com uma bala de mosquete, que lhe quebrou o braço, e mais dois dos
homens, sem contudo matá"-los.

O imediato, clamando por ajuda, correu, de qualquer modo, para dentro da
cabine, embora ferido, e com sua pistola atirou na cabeça do novo
capitão, entrando a bala por sua boca e saindo novamente pela parte de
trás de um de seus ouvidos, o que fez tombar para todo e sempre: ao que
os demais amotinados cederam, e o navio foi por fim recuperado, sem mais
vidas perdidas.

Assim que a situação no convés do navio foi posta sob controle, o
capitão ordenou que fossem disparados sete tiros de canhão, que eram o
sinal combinado por nós para que eu fosse informado do bom sucesso do
assalto, pelo qual, tenha a certeza, fiquei muito contente, tendo feito
vigília na praia até quase duas horas da manhã.

Depois de inequivocamente ouvir o sinal, tratei de dormir; e tendo sido
um dia de grande esforço, ferrei no sono, até que fui surpreendido pelo
disparo de uma arma; e colocando"-me de pé num pulo, ouvi um homem
chamar"-me pelo nome de ``Governador! Governador!'' e logo reconheci a
voz do capitão; quando, subindo até o alto da colina, lá ele estava e,
apontando ao navio, me abraçou, ``Meu querido amigo e libertador'',
disse ele, ``aí está o seu navio; pois ele é todo seu, assim como nós e
tudo o que a ele pertença''. Olhei em direção ao navio, que flutuava a
pouco mais de meia milha da costa; pois eles levantaram âncora assim que
o dominaram e, fazendo bom tempo, fundearam"-no bem próximo à foz do
pequeno riacho; e com a cheia da maré, o capitão trouxe o bote perto do
lugar que servira de ancoradouro para minhas jangadas, e desembarcou bem
na minha porta.

A princípio, vi"-me a ponto de desmaiar de surpresa; pois, afinal, meu
salvamento estava a meu alcance, com todas as conveniências, e um grande
navio pronto para me levar para onde eu quisesse ir. No primeiro
instante, emudeci; e não tivesse ele me tomado em seus braços, e eu me
agarrado a ele, teria ido ao chão.

Ele percebeu minha estupefação e imediatamente sacou uma garrafa do
bolso e me deu um trago de cordial, que trouxera consigo para tal fim.
Depois de o tomar, sentei"-me no chão; e embora tenha me recobrado,
precisei de algum tempo até me encontrar em condição de travar
conversação com ele.

Em todo esse tempo, o pobre homem encontrava"-se em tão grande felicidade
quanto eu, mas não igualmente assaltado de espanto; e disse"-me um sem
número de coisas gentis, de maneira a acalmar"-me e trazer"-me à
consciência; mas tal era a inundação de alegria em meu peito, que pôs
todo o meu espírito em grande perplexidade; este, por fim, irrompeu em
lágrimas, e logo recuperei a fala.

Em seguida, foi minha vez de abraçá"-lo como meu libertador, e nos
regozijamos os dois; e eu disse a ele que o via como um homem enviado
dos Céus para o meu salvamento, e que toda a aventura parecia ser uma
enfiada de prodígios; que coisas como essas eram os testemunhos que
tínhamos de uma mão secreta da Providência a governar o mundo, e uma
evidência de que os olhos de um Poder infinito eram capazes de
perscrutar até o mais recôndito canto do mundo e enviar auxilio aos
desafortunados sempre que Lhe fosse sua vontade.

Não esqueci de elevar meu coração aos Céus em gratidão; e que coração
poderia abster"-se de abençoá"-Lo, Ele que não apenas proveu
milagrosamente um homem em tamanho desterro e desolação, mas a Quem
sempre se deve reconhecer a procedência de toda salvação.

Depois de termos conversado um pouco, disse"-me o capitão que me trouxera
pequenos revigorantes, os quais me eram oferecidos pelo navio e que os
desgraçados que por tanto tempo se haviam assenhorado dele não tinham
pilhado; dito isto, chamou seus homens no bote e ordenou que trouxessem
à terra as coisas destinadas ao governador; e, de fato, foi um presente
tal que mais parecia que eu não seguiria viagem com eles, como se ainda
fosse morar na ilha, e eles estivessem em vias de partir sem mim.

Primeiro, ele me trouxe uma caixa de frascos cheios de excelente
cordial, seis grandes garrafas de vinho Madeira; cada qual com dois
quartos; duas libras de um excelente tabaco; doze bons pedaços de carne
seca de boi, e seis de carne seca de porco, com um saco de ervilhas e
obra de cem pedaços de bolacha.

Ele também me trouxe uma caixa de açúcar, uma caixa de farinha, um saco
cheio de limões e duas garrafas de suco de lima verde e uma abundância
de outras coisas; mas além disso, e o que me foi mil vezes mais útil,
trouxe"-me seis camisas novas e limpas, seis lenços de pescoço muito
bons, dois pares de luvas, um par de sapatos, um chapéu e um par de
meias, e um traje completo de seu próprio uso, apenas um pouco gasto;
isto é, vestiu"-me dos pés a cabeça.

Foi um presente muito gentil e adequado, como se pode imaginar, para
alguém em minhas circunstâncias; mas nunca se me deparara algo tão
desagradável, desajeitado e desconfortável quanto me foi vestir pela
primeira vez essas roupas.

Terminadas essas cerimônias, e depois de todas essas boas coisas terem
sido levadas a minha habitação, começamos a deliberar sobre o que se
deveria fazer com os prisioneiros que tínhamos; pois era assunto de
importante reflexão se podíamos ou não nos arriscar a levá"-los conosco,
especialmente dois deles, que sabíamos serem incorrigíveis e intratáveis
ao último grau; e o capitão disse que sabia se tratarem de tão
arrematados canalhas que não havia como emendá"-los, e se fosse o caso de
levá"-los consigo, que fosse a ferros, como malfeitores, a serem
entregues à justiça na primeira colônia inglesa a que chegássemos; e
percebi que o próprio capitão estava muito preocupado com a questão.

Diante disso, disse"-lhe que, se ele assim o desejasse, estaria
comprometido a colocar diante de mim os dois homens de que ele falava
para que eles declarassem por vontade própria seu desejo de permanecer
na ilha. ``Eu ficaria muito feliz com isso'', diz o capitão, ``de todo o
coração''.

``Bem'', disse eu, ``vou mandar buscá"-los e tratar com eles em seu
lugar''; e assim, ordenei a Sexta"-Feira e os dois reféns, que se
encontravam então desimpedidos, tendo seus camaradas cumprido com a
promessa; isto é, ordenei que fossem à caverna e transportassem os cinco
homens, presos como estavam, à cabana e lá os mantivem até minha
chegada.

Depois de algum tempo, cheguei lá vestido de minhas novas roupas; e mais
uma vez fui chamado de governador; e todos estando reunidos, e tendo o
capitão ao meu lado, ordenei que os homens fossem trazidos à minha
presença e disse"-lhes que recebera relação completa das vilezas de seu
comportamento para com o capitão e de como haviam fugido com o navio e
preparavam"-se para cometer mais roubos, mas que a Providência os havia
enredado em seus próprios atos, e que haviam caído na cova que tinham
aberto para outros.

Dei"-lhes ao conhecimento que, sob ordens minhas, o navio havia sido
recuperado; e que estava pronto para zarpar; e que eles logo teriam
condições de ver que recompensa havia recebido seu novo capitão pela
vilania cometida, e que sem tardar o veriam pendurado no lais de verga.

Que, quanto a eles, desejava eu saber que razões me davam para que não
os executasse como piratas presos em flagrante delito, uma vez que não
podiam eles duvidar que estava eu investido de autoridade para fazê"-lo.

Um deles tomou a palavra em nome dos demais; que nada tinham a dizer
senão isto, que quando foram capturados, o capitão lhes prometera
clemência, e eles humildemente imploravam por minha misericordia; mas
disse"-lhes que não sabia que misericórdia ter com eles; pois quanto a
mim, estava decidido a deixar a ilha com todos os meus homens, e o
capitão me concedera passagem para seguir à Inglaterra; e que quanto ao
capitão, ele não poderia transportá"-los à Inglaterra senão como
prisioneiros a ferros, a serem julgados por motim e sequestro do navio;
delitos dos quais a consequência, havia mister que soubessem, era o
patíbulo; de guisa que eu não sabia dizer o que lhes seria melhor, senão
que quisessem tentar sua sorte na ilha; e que se assim o desejassem,
era"-me indiferente concedê"-lo; uma vez que tinha liberdade de sair da
ilha, estava inclinado a poupar"-lhes a vida, caso pensassem ser capazes
de ali viver.

Eles pareceram muito gratos por isso e disseram que preferiam a aventura
de ficar ali a serem levados à Inglaterra para serem enforcados; e assim
me desobriguei do assunto.

O capitão, por sua vez, parecia criar obstáculos, como se não se
atrevesse a deixá"-los ali; ao que me mostrei um tanto aborrecido com o
capitão, dizendo"-lhe que eram meus prisioneiros, não dele; e que, uma
vez que lhes havia oferecido tantos favores, teria de fazer valer minha
palavra; e que se ele não julgasse conveniente consentir, eu os deixaria
em liberdade, tal como os encontrei; e se não fosse de seu agrado, ele
os poderia reaver, se pudesse.

Diante disso, eles demonstraram bastante gratidão; e, assim, coloquei"-os
em liberdade e ordenei"-lhes que se recolhessem à floresta, ao lugar de
onde vieram, e eu lhes deixaria armas de fogo, munição e orientações
sobre como viver bem, se julgassem adequado.

Depois disso, preparei"-me para embarcar no navio; mas disse ao capitão
que permaneceria em terra naquela noite para preparar minhas coisas, e
pedi que ele subisse a bordo naquele ínterim e conservasse a ordem dos
assuntos do convés e enviasse um bote à terra para me buscar no dia
seguinte; ordenando"-lhe, em todo caso, que fizesse com que o novo
capitão, que fora morto, fosse pendurado na verga para que os homens que
permaneceriam na ilha o vissem.

Com a partida do capitão, mandei chamar os homens a minha habitação e
expus"-lhe um longo arrazoado sobre suas circunstâncias; e lhes disse que
julgava terem feito boa escolha; e que, se o capitão os tivesse levado
embora, decerto seriam enforcados. Mostrei"-lhes o novo capitão,
pendurado no lais de verga do navio, e lhes disse que não tinham nada
menos a esperar.

Quando todos declararam sua vontade de permanecer, disse"-lhes que os
inteiraria da história de minha vida ali e os instruiria a torná"-la mais
fácil; e assim lhes contei toda a história do lugar e de minha chegada;
e mostrei"-lhes minhas fortificações, e a maneira como fabricava o pão,
plantava a cevada, curava as uvas; e, em suma, tudo quanto fosse
necessário para lhes tornar fácil a vida; e contei"-lhes também a
história dos dezessete espanhóis que eram aguardados; para os quais
deixei uma carta, e os fiz prometer que lhes dariam bom tratamento.

Deixei"-lhes minhas armas de fogo, a saber, cinco mosquetes, três
espingardas e três espadas; sobrava"-me ainda mais de um barril e meio de
pólvora; pois, passados um ou dois anos, pouco a usei e pouco a
desperdicei. Detalhei a eles como criava as cabras e os instruí em
maneiras de ordenhá"-las e engordá"-las e de fabricar manteiga e queijo.

Em suma, contei"-lhes cada parte de minha própria história; e disse"-lhes
que intercederia junto ao capitão para que lhes deixasse outros dois
barris de pólvora e algumas sementes para a horta, as quais me haveriam
ajudado muitíssimo; além disso, dei"-lhes o saco de ervilhas que o
capitão me trouxera como presente e sugeri que cuidassem de semeá"-las e
multiplicá"-las.

Feito tudo isso, deixei"-os no dia seguinte e embarquei no navio.
Preparamo"-nos para zarpar prontamente, mas não levantamos âncora naquela
noite. Na manhã seguinte, bem cedo, dois dos cinco homens chegaram ao
costado do navio a nado; e queixando"-se dos outros três com as mais
tristes palavras, imploraram para serem admitidos a bordo, pelo amor de
Deus, pois seriam mortos; dirigindo suas súplicas ao capitão para que os
permitisse o embarque, embora ele os fosse enforcar sem demora.

Quanto a esse caso, o capitão fingiu não ter poder sem mim; e apenas
depois de algumas dificuldades, e de suas solenes promessas de que não
cometeriam outros crimes, eles foram trazidos a bordo; e, em seguida,
violentamente açoitados, tendo sal e vinagre aplicados às feridas;
depois do que se mostraram companheiros muito honestos e tranquilos.

Algum tempo depois desse episódio, ordenou"-se que fosse descido o bote,
com a maré alta, para que se levasssem as coisas prometidas aos homens;
ao que o capitão, por minha intercessão, pediu que se acrescentassem
seus baús e roupas; os quais eles receberam, demonstrando"-se muito
gratos. Eu também lhes infundi esperança, dizendo"-lhes que, se estivesse
em meu poder enviar qualquer navio para recebê"-los, eu não os
esqueceria.

Ao deixar a ilha, levei a bordo, como relíquias, o ótimo gorro de pele
de cabra que costurara, meu guarda"-chuva e meu papagaio; também não me
esqueci de trazer comigo o dinheiro que mencionei atrás, que por tanto
tempo havia permanecido comigo sem utilidade que se encontrava
enferrujado ou manchado, mal lembrando a prata antes de ser um pouco
esfregado e manuseado; bem como o dinheiro que resgatara nos destroços
do navio espanhol.

E assim deixei a ilha a 19 de dezembro, conforme constatei pelo diário
de bordo do navio, ano de 1686, depois de ter nela permanecido por vinte
e oito anos, dois meses e dezenove dias; conhecendo a salvação desse
segundo cativeiro no mesmo dia do mês em que outrora escapei no
\emph{barco"-longo}, como o chamam os espanhóis, aos mouros de Salé.

Nessa embarcação, após uma longa viagem, cheguei à Inglaterra a 11 de
junho, ano de 1687, estando trinta e cinco anos ausente.

Quando cheguei à Inglaterra, era um perfeito estranho para toda a gente,
como se ali jamais tivesse sido conhecido. Minha benfeitora e fiel
depositária, com quem deixei meu dinheiro em custódia, estava viva,
porém conhecera grandes padecimentos; tornou"-se viúva por uma segunda
vez e vivia em grande dificuldade. Tranquilizei"-a quanto ao que me
devia, garantindo"-lhe que não lhe causaria problemas; e, pelo contrário,
em gratidão por seu cuidado e fidelidade para comigo, conceder"-lhe"-ia
tanto conforto quanto o permitisse minha pequena poupança, que, naquela
época, de fato me permitiria fazer muito pouco por ela; mas
assegurei"-lhe que jamais me esqueceria de sua bondade para comigo; e não
a esqueci quando amealhei o bastante para ajudá"-la, como se verá a seu
momento.

Depois segui para Yorkshire; mas meu pai estava morto, e minha mãe e
toda a família extinta, exceto por duas irmãs e dois dos filhos de um de
meus irmãos; e como há muito havia sido dado como morto, nenhuma
provisão fora feita para mim; de sorte que, em suma, nada encontrei que
pudesse me trazer algum auxílio ou conforto; e o pouco dinheiro que
tinha não seria de grande ajuda para que me estabelecesse no mundo.

Em verdade, encontrei boa e inesperada manifestação de gratidão; e assim
se deu, que o comandante do navio que tivera a felicidade de salvar, bem
como sua carga e navio, prestou tão generosa relação aos proprietários
da maneira como salvara as vidas dos marinheiros e do navio, que eles me
convidaram para encontrá"-los e alguns outros mercadores interessados, e
todos juntos me ofereceram generosos cumprimentos sobre o episódio e um
presente de quase 200 libras esterlinas.

Mas depois de fazer várias reflexões sobre as circunstâncias da minha
vida, e quão pouco isso significaria para meu restabelecimento no mundo,
resolvi ir a Lisboa para talvez obter informações sobre a condição da
minha plantação nos Brasis, e sobre o destino de meu sócio, que, tinha
eu motivos para supor, já havia alguns anos me dera por morto.

Com esse objetivo embarquei rumo a Lisboa, onde cheguei em abril
seguinte, tendo ao meu lado Sexta"-Feira, companheiro fiel em todas as
errâncias e servo leal em todas as circunstâncias.

Quando cheguei a Lisboa, encontrei, mediante investigação, e para minha
particular satisfação, o meu velho amigo, o capitão do navio, que me
resgatara no mar a cerca da costa de África; e que, já entrado em anos,
havia deixado de ir ao mar, tendo encarregado o filho, que estava longe
de ser jovem, de seu navio, ainda realizando comércio com os Brasis. O
velho não me reconheceu e, verdade seja dita, eu mesmo mal o reconheci;
mas logo me veio sua lembrança, e ele não tardou a recordar"-se de quem
eu era, quando lho disse.

Depois das efusivas expressões de afeto, da parte do velho conhecido,
perguntei, evidentemente, sobre minha plantação e meu sócio; e o velho
me disse que não visitava os Brasis havia obra de nove anos; mas que
estava certo de que, quando partiu, meu sócio estava vivo, mas os
administradores que havia encarregado de cuidar de meus interesses na
sociedade estavam ambos mortos; que, no entanto, acreditava que eu
ouviria bom relato das rendas e frutos da plantação; pois, ante a crença
geral de que eu havia me perdido e me afogado, meus administradores
haviam entregue a relação da produção de minha parte da plantação ao
procurador fiscal, que dela havia se apropriado, caso eu nunca viesse
reivindicá"-la: destinando um terço ao rei, e dois terços ao mosteiro de
Santo Agostinho, para serem gastos em benefício dos pobres e a conversão
dos índios à fé católica; mas que, se eu reaparecesse, ou qualquer um
por mim, para reivindicar a herança, a apropriação seria revertida;
exceto pelas rendas e frutos, ou dividendos anuais, distribuídos para
usos caritativos, que não poderiam ser restaurados: mas ele me assegurou
que o fiscal da receita das terras de el"-Rei, e o provedor, ou
intendente do mosteiro, haviam cuidado o tempo todo para que o titular,
isto é, o meu sócio, prestasse todos os anos relação fiel da produção,
da qual recebiam devidamente a minha metade.

Perguntei"-lhe se sabia até que ponto haviam crescido os dividendos da
plantação? E se julgava valer a pena cuidar do assunto? Ou se, ao dar
ali, encontraria algum obstáculo ao meu direito de posse da minha
metade?

Disse"-me ele não saber com precisão até que ponto a plantação rendera
maiores dividendos; mas disto ele sabia: que meu parceiro conhecera
grande enriquecimento no desfrute de sua parte; e que, tanto quanto se
recordava, viera a seu conhecimento que a terça parte do rei subtraída
ao que me pertencera, que, ao que parece, era destinada a outro mosteiro
ou casa religiosa, somava mais de duzentos \emph{moidores} por ano; que,
quanto a eu ser restituído pacificamente de minha parte, não havia
dúvida a respeito, estando meu sócio vivo para testemunhar meu título, e
encontrando"-se meu nome inscrito no registro do país; ele também me
contou que os sobreviventes de meus dois administradores eram pessoas
muito justas, honestas e muito ricas; e que acreditava ele que eu não só
teria a ajuda deles para me colocar na posse do que me pertencia, como
encontraria em suas mãos soma muito considerável de dinheiro em meu
nome; tratando"-se do produto da fazenda enquanto seus pais foram
responsáveis pelos cabedais, e antes que fosse entregue, como dito
acima; o que, tanto quanto se lembrava, durou obra de doze anos.

Mostrei"-me um pouco preocupado e intranquilo com esse relato, e
perguntei ao velho capitão como havia se dado que meus representantes
passaram a dispor dos meus bens, quando ele sabia que eu havia redigido
o meu testamento, nomeando o capitão português meu herdeiro universal,
etc.

Ele me disse que era verdade; mas que, como não havia evidências de
minha morte, ele não podia fazer as vezes de executor até que algum
relato de minha morte surgisse; e, além disso, não estava disposto a
interferir em algo tão remote; que era verdade que havia registrado meu
testamento e reclamado seu direito; e pudesse ele ter feito qualquer
relação de minha morte ou sobrevivência, ele teria agido por procuração
e tomado posse do engenho (assim eles chamam a casa de açúcar), e dado a
seu filho, que então estava nos Brasis, ordens para fazê"-lo.

``Mas'', disse o velho, ``tenho uma notícia para lhe dar, que talvez não
lhe seja tão aceitável quanto as demais; pois bem, acreditando que você
havia se perdido, e todos acreditando o mesmo, seu sócio e
representantes me entregaram em seu nome as prestações de contas dos
dividendos dos primeiros seis ou oito anos, e eu as recebi; mas havendo
naquela época grandes despendimentos para o aumento das instalações, a
construção de um engenho e a compra de escravos, os lucros não se
aproximaram de tanto quanto depois se produziu; no entanto'', completou
o velho,`` vou dar"-lhe relação correta do que recebi no todo, e em que
se gastou.''

Passados alguns dias da palestra com esse velho amigo, ele me trouxe
relação da renda dos primeiros seis anos de minha plantação, assinada
por meu sócio e os comerciantes"-representantes, sempre entregue em
mercadorias, a saber, fumo em rolo e açúcar em caixas, além de rum,
melaço, etc., isto é, os produtos de um engenho; e vim a saber por essa
relação que a cada ano os dividendos aumentavam consideravelmente; mas
com o emprego de vultuosas quantias, como dito acima, a soma a princípio
era pequena; porém, o velho deu"-me a ver que era devedor em quatrocentos
e setenta \emph{moidores}, além de sessenta caixas de açúcar e quinze
rolos duplos de fumo, que se perderam com seu navio; tendo ele
naufragado em viagem de volta a Lisboa, obra de onze anos depois de eu
ter deixado a plantação.

O bom homem começou então a queixar"-se de seus infortúnios e de como
fora obrigado a usar meu dinheiro para recuperar suas perdas e comprar
uma parte em um novo navio; ``porém, meu velho amigo'', disse ele,
``você não quedará carente de suprimento para suas necessidades; e assim
que meu filho voltar, será plenamente satisfeito.''

Com isso, ele saca uma velha bolsa e me dá cento e sessenta
\emph{moidores} portugueses em ouro; e entregando"-me a escritura do
título do navio, no qual seu filho tinha ido para os Brasis e do qual
era proprietário de uma quarta parte, e seu filho de outra, ele os
coloca em minhas mãos como garantia do resto.

Comoveu"-me demasiado a honestidade e bondade do pobre homem para ser
capaz de aceitá"-lo; e lembrando"-me de tudo que fizera por mim, de como
me havia levado ao mar e tratado generosamente em todas as ocasiões, e,
em particular, do amigo sincero que se revelava então a mim, mal fui
capaz de conter o choro ante o que acabava de me dizer; assim,
perguntei"-lhe se as circunstâncias o permitiam dispor de tanto dinheiro
e se o gesto não o deixaria em dificuldades; ao que me respondeu que um
pouco de dificuldade seria inevitável; mas, de qualquer forma, aquele
era meu dinheiro, e eu podia estar em maior necessidade do que ele.

Tudo o que o bom homem dizia era cheio de afeição, e mal conseguia
conter as lágrimas ao ouvi"-lo; por fim, tomei uma centena de
\emph{moidores} e pedi"-lhe pena e tinta para assinar um recibo; e,
devolvendo"-lhe o restante, disse"-lhe que, se algum dia recuperasse a
posse do engenho, o restituiria daquela parte igualmente, como, de fato,
depois o fiz; e que, quanto à nota de compra de sua parte no navio do
filho, eu não a aceitaria de forma alguma; mas que, se precisasse do
dinheiro, reconhecia que ele era honesto o bastante para mo devolver; e
que, se viesse a receber aquilo que me dava motivos para esperar, jamais
lhe tomaria um \emph{penny} a mais.

Passado isso, o velho me inquiriu se deveria ele me cercar de algum
instrumento com que reivindicasse minha plantação; disse a ele que
pensava em o fazer pessoalmente. Disse que devia eu assim fazer, se mo
aprouvesse; mas que se não o fizesse, havia meios bastantes para
garantir o meu direito e tomar posse imediata dos dividendos para meu
uso: e como havia navios no rio de Lisboa prontos a zarpar ao Brasil,
fez com que eu declarasse meu nome em registro público, acompanhado de
declaração de próprio punho em que afirmava, sob juramento, estar eu
vivo e ser a pessoa que ocupara a terra para o cultivo da plantação já
mencionada.

Sendo tudo isso devidamente atestado por um notário, e uma procuração
lavrada, instruiu"-me ele a enviar o documento, com carta redigida por si
mesmo, a um mercador conhecido no local; e então propôs que permanecesse
com ele até que recebesse alguma resposta sobre o requerido.

Nenhuma conduta foi mais honrosa do que a resultante dessa procuração;
pois em menos de sete meses recebi grande pacote dos herdeiros de meus
administradores, os mercadores por razão dos quais fui ao mar, no qual
se encontravam as seguintes cartas e documentos em anexo:

Em primeiro lugar, havia a conta corrente dos dividendos da minha
fazenda ou plantação, desde o ano em que os seus pais haviam acertado
paga com o meu velho capitão português, o que se deu por seis anos;
sendo o saldo, ao que se indicava, de 1174 \emph{moidores} a meu favor.

Em segundo lugar, havia as contas dadas de mais quatro anos, durante os
quais preservaram os bens em seu poder, antes que o governo reclamasse
sua administração como posse de pessoa desaparecida, a que chamavam de
morte civil; e o saldo dessa, tendo aumentado o valor da plantação,
chegava a {[}trinta e oito mil oitocentos e noventa e dois cruzados{]},
isto é, obra de {[}três mil duzentos e quarenta e um{]}
\emph{moidores}.\footnote{A edição de 1719 traz em branco os espaços
  referentes aos valores recebidos por Robinson. As quantias foram
  estipuladas posteriormente pelo editor W. P. Trent {[}\textsc{n.\,e.}{]}.}

Em terceiro lugar, havia a conta do prior do mosteiro de Santo
Agostinho, que recebera dividendos por mais de catorze anos; mas não se
dando conta do que fora empenhado no hospital, declarava"-se muito
honestamente a soma de oitocentos e setenta e dois \emph{moidores} não
distribuídos, os quais reconhecia sendo meus; quanto à parte do rei,
nada foi restituído.

Havia uma carta do meu sócio, congratulando"-me com muita afeição por eu
estar vivo e trazendo registro de como a propriedade cresceu e o que
produziu em um ano; com os detalhes do número de pés quadrados ou acres
que continha; o quanto nela se plantara, e quantos escravos nela havia;
e fazendo vinte e duas cruzes representando bênçãos, disse"-me que havia
rezado o mesmo número de Ave Marias para agradecer à Virgem por eu estar
vivo; convidando"-me muito apaixonadamente para que fosse tomar posse do
que me pertencia; e entrementes pedindo que lhe desse orientação quanto
a quem seria o depositário de meus bens, caso eu não comparecesse
pessoalmente; concluindo com uma calorosa demonstração de amizade, sua e
de sua família; e enviando"-me de presente sete belas peles de leopardos,
que tinha ele, ao que parecia, recebido de África, por algum navio que
havia enviado até lá, e que, ao que parecia, fizera viagem melhor do que
a minha. Ele me enviou também cinco baús de excelentes doces e cem
moedas de ouro não cunhadas, não tão grandes quanto os \emph{moidores}
portugueses.

Pela mesma frota, meus dois representantes mercadores me enviaram mil e
duzentas caixas de açúcar, oitocentos rolos de tabaco e o restante do
cômputo em ouro.

Poderia muito bem dizer agora que, de fato, Jó conheceu fim melhor do
que seu começo.\footnote{Jó 42:12 {[}\textsc{n.\,e.}{]}} É impossível expressar o
alvoroço em meu coração quando percorri essas cartas e me vi cercado de
toda a minha riqueza; pois, visto que os navios do Brasil vêm todos em
comboios, os mesmos navios que trouxeram minhas cartas trouxeram minhas
mercadorias; e os bens já se encontravam seguros no rio antes que as
cartas e documentos chegassem às minhas mãos. Em uma palavra, empalideci
e fui acometido de grande indisposição; e, se o velho não me tivesse
acudido e buscado um cordial, creio que a repentina surpresa de alegria
teria magoado a natureza, e eu teria dado com a morte ali mesmo: antes,
a indisposição persistiu e, tendo assim permanecido por algumas horas,
um médico foi chamado, e tendo inferido algo da verdadeira causa de meu
estado, ordenou que eu fizesse uma sangria; depois da qual senti alívio
e recobrei o vigor; porém, em verdade, creio que, não tivesse conhecido
remédio pela vazão assim dada aos humores, teria morrido.

Tornara"-me, não mais que de repente, senhor de uma fortuna de mais de
cinco mil libras esterlinas em dinheiro, e detentor de uma propriedade,
como a bem poderia chamar, nos Brasis, de renda superior a mil libras
por ano, tão bem fundada quanto uma propriedade em terras inglesas: e,
em suma, encontrava"-me em circunstâncias que mal era capaz de
compreender, e tampouco sabia como conhecer tranquilidade para seu
usufruto.

A primeira coisa que fiz foi recompensar meu benfeitor original, meu bom
e velho capitão, de quem conhecera a caridade em minha aflição, a
gentileza em meus inícios e a honestidade no final. Mostrei"-lhe tudo o
que me fora enviado; disse"-lhe que, depois da Providência do Céu, que
dispôs todas as coisas, era a ele que devia minha sorte; e que então
cabia a mim recompensá"-lo, o que eu faria cem vezes mais; de guisa que
lhe devolvi, antes de tudo, os cem \emph{moidores} que dele havia
recebido; em seguida, mandei chamar um notário e fiz com que redigisse
um perdão geral ou quitação dos quatrocentos e setenta \emph{moidores},
que ele reconhecia que me eram devidos, da maneira mais plena e firme
possível. E depois, fiz com que se lavrasse uma procuração, a qual o
designava recebedor dos rendimentos anuais de meu engenho; e nomeando
meu sócio para lhe prestar contas e realizar em meu nome as pagas por
intermédio dos comboios usuais; e por uma cláusula final, fiz a ele
concessão vitalícia de cem moidores por ano, extraída dos ditos
rendimentos, e de cinquenta moidores por ano a seu filho depois dele; e
assim eu retribuí meu velho amigo.

Tinha eu, então, que pensar sobre que rumos tomar, e o que fazer com a
propriedade que a Providência me havia colocado em mãos; e, de fato, via
minha cabeça então ocupada de mais aflições do que as conhecia no meu
estado de coisas na ilha, onde não carecia de nada além do que tinha e
não tinha nada além do que necessitava; ao passo que tinha, naquele
momento, uma grande responsabilidade sobre mim, e minha tarefa era
defendê"-la. Já não havia caverna em que escondesse meu dinheiro, ou
lugar em que pudesse ficar sem fechadura ou chave, criando mofo e
manchas sem quem tivesse interesse nele; pelo contrário, não sabia onde
o meter, nem a quem confiá"-lo. Meu antigo patrono, o capitão, de fato,
era honesto, e esse era o único refúgio que tinha.

Em seguida, meu negócio nos Brasis parecia me chamar para lá; mas era
impossível pensar em ir para lá antes de resolver minhas pendências e
deixar meu patrimônio em boas mãos. A princípio, pensei em minha velha
amiga, a viúva, que sabia ser honesta e seria justa comigo; mas já tinha
idade e era pobre e, pelo que sabia, podia estar endividada; de maneira
que, em suma, não tive outra alternativa senão voltar eu mesmo para a
Inglaterra e levar meus pertences comigo.

Passaram"-se alguns meses, entretanto, até que tais assuntos estivessem
encerrados; e, portanto, uma vez que havia recompensado plenamente o
velho capitão, e para sua satisfação, que havia sido meu antigo
benfeitor, comecei a pensar na pobre viúva, cujo marido fora meu
primeiro benfeitor, e ela, enquanto esteve em seu poder sê"-lo, minha
fiel conselheira e guardiã. Assim, a primeira coisa que fiz foi
encontrar um comerciante em Lisboa que escrevesse a seu correspondente
em Londres, não meramente para levar"-lhe letras de câmbio, mas para
procurá"-la em pessoa e entregar"-lhe cem libras minhas em espécie, e que
a confortasse na pobreza dizendo"-lhe que, enquanto eu vivesse, ela
recebia novas remessas; ao mesmo tempo, enviei a minhas duas irmãs no
campo cem libras cada, estando elas, embora não em grande necessidade,
em circunstâncias não muito favoráveis; uma delas tendo contraído
matrimônio e enviuvado; e a outra conhecendo marido que não lhe era tão
bom quanto deveria ser.

Mas, entre todos os meus parentes ou conhecidos, ainda não era capaz de
designar um a quem pudesse confiar o grosso de meu patrimônio, de
maneira que pudesse ir para os Brasis e deixar o que me pertencia em
segurança; e isso me trazia grande inquietação.

Tinha a princípio a intenção de retornar aos Brasis e lá me estabelecer,
pois me sentia, por assim dizer, natural do lugar; mas conservava
escrúpulos em minha mente a respeito da religião, que sem que eu
percebesse me refreavam; e disso tratarei com mais vagar agora. De
qualquer forma, não era a religião que me impedia de retornar naquele
momento; e assim como não tivera escrúpulos de comungar aparentemente da
religião do lugar durante o tempo em que ali vivera, não o tinha então;
a questão era que, vez por outra tendo refletido mais sobre ela (do que
antes), quando me punha a pensar em viver e morrer entre eles, passei a
me arrepender de ter me confessado papista, e pensei que talvez não
fosse a melhor religião em que morrer. Como disse, porém, não era isso
que me impedia de retornar aos Brasis; senão que eu, de fato, não sabia
a quem confiar o que me pertencia; e então decidi finalmente retornar à
Inglaterra com minhas posses, onde, uma vez que chegasse, concluí que
deveria travar contatos, ou encontrar parentes que me seriam leais; e,
assim, preparei"-me para retornar à Inglaterra com as minhas posses.

A fim de preparar minha partida, primeiramente eu, estando o comboio
vindo dos Brasis prestes a zarpar, decidi dar respostas à altura das
justas e fieis prestações que recebera dali; e, assim, escrevi ao prior
do mosteiro de Santo Agostinho uma carta cheia de agradecimentos pela
justeza de sua prestação de contas e o envio dos oitocentos e setenta e
dois moidores, dos quais declinava, desejando oferecer quinhentos ao
mosteiro, e trezentos e setenta e dois para os pobres, ao discernimento
do prior; desejando que me dirigissem as preces de seus Pais Nossos e
assim por diante.

Em seguida, escrevi uma carta de agradecimento aos meus dois
administradores, com o merecido reconhecimento a tamanha justiça e
honestidade; quanto a lhes enviar algum presente, estavam eles muito
acima de qualquer circunstância que o ensejasse.

Por fim, escrevi ao meu sócio, reconhecendo sua diligência no aumento do
engenho e integridade em fazer crescer o conjunto de instalações;
dando"-lhe instruções para o futuro governo de meu quinhão, segundo os
poderes que delegara a meu antigo benfeitor, a quem desejava que ele
enviasse tudo quanto me fosse devido, até que ele tivesse mais
particularmente notícias minhas; assegurando"-lhe que tencionava não
apenas ir até ele, como estabelecer"-me ali pelo restante de minha vida.
Ao dito, acrescentei o belo presente de umas sedas italianas, que
destinava a sua esposa e duas filhas, pois o filho do capitão
informou"-me que as tinha; além de duas peças de fina gabardina inglesa,
da melhor que encontrei em Lisboa, cinco peças de baeta preta e rendas
de Flandres de bom valor.

Tendo assim dado fim a meus negócios, vendido minha carga e transformado
todos os meus pertences em boas letras de câmbio, minha dificuldade
seguinte era que caminho tomar rumo à Inglaterra: pois embora fosse
bastante acostumado ao mar, andava acometido então de uma estranha
aversão a ir à Inglaterra pelo mar; e embora não pudesse dar qualquer
explicação a isso, a dificuldade tornou"-se tamanha que, embora tivesse
despachado minha bagagem, acabei por mudar de ideia não uma, mas duas ou
três vezes.

É verdade que no mar conheci apenas dissabores, e essa pode ser uma das
razões. Que nenhum homem faça pouco dos fortes impulsos de seus próprios
pensamentos em casos de tal importância: dois dos navios que escolhera
para embarcar, quero dizer, mais particularmente do que quaisquer
outros, isto é, a ponto de ter em um deles embarcado meus pertences e,
no outro, estando acertado com o capitão; pois bem, dois desses navios
se perderam. Um foi levado por piratas e o outro naufragou no Start,
próximo a Torbay, incidente do qual houve apenas três sobreviventes; de
maneira que em ambas as embarcações teria conhecido infortúnios; e em
qual delas maior, era difícil dizer. Encontrando"-me assim atormentado em
meus pensamentos, meu velho capitão, a quem tudo comuniquei, fez fortes
apelos para que não seguisse por mar, mas que viajasse por terra até A
Corunha e cruzasse a Baía de Biscaia com destino a Rochelle, de onde se
fazia viagem tranquila e segura por terra a Paris e, dali, a Calais e
Dover; ou que rumasse a Madri e dali atravessasse a França.

Em suma, estava tão refratário à ideia de atravessar por mar, exceto de
Calais a Dover, que decidi viajar todo o caminho por terra; o que, não
tendo eu pressa, tampouco me importando os custos, foi sem sombra de
dúvida o caminho mais deleitoso; ao que concorreu meu velho capitão
ter"-me apresentado um cavalheiro inglês, filho de comerciante situado em
Lisboa, disposto a viajar comigo; reunindo"-se a nós mais dois mercadores
ingleses e dois jovens senhores portugueses, os últimos indo apenas a
Paris; de sorte que, ao todo, éramos seis, acompanhados de cinco
criados; com os dois mercadores e os dois portugueses levando consigo
apenas um criado para cada dupla, poupando dessa maneira os valores de
passagem; e eu trazendo ao meu lado um marinheiro inglês, contratado
para viajar comigo na condição de criado, e meu bom Sexta"-Feira, muito
estrangeiro para fazer as vezes de serviçal em uma viagem.

Dessa forma, parti de Lisboa; e estando nossa companhia muito bem
montada e armada, formávamos uma pequena tropa, da qual deram a mim a
honra de chamar de capitão, não só porque era o homem mais velho, como
porque tinha dois criados e era, em verdade, a razão de todo o percurso.

Não o tendo incomodado, leitor, com nenhum de meus diários marítimos,
tampouco o perturbarei com meus diários terrestres; não devo, porém,
omitir algumas aventuras que nos sucederam nessa aborrecida e difícil
jornada.

Quando chegamos a Madri, nós, que éramos todos estrangeiros em Espanha,
desejamos permanecer ali algum tempo para ver a corte de Espanha e o que
fosse digno de apreciar; mas, sendo aquela a parte final do verão,
apressamo"-nos e partimos de Madri em meados de outubro; e chegando à
borda de Navarra, causou"-nos grande apreensão, em vários vilarejos do
caminho, o relato de que tanta neve caía no lado francês das montanhas,
que vários viajantes haviam sido forçados a retornar a Pamplona, depois
de terem tentado a travessia sob imenso risco.

Quando chegamos à Pamplona propriamente, descobrimos que assim era; e a
mim, que sempre estivera acostumado a um clima quente, e a regiões onde
mal se conseguia vestir roupas, o frio era insuportável; em verdade, era
doloroso e assustador, uma vez que, quando deixáramos a Velha Castela,
dez dias antes, a temperatura, mais do que quente, era tórrida; enquanto
ali sentíamos o vento cortante das montanhas dos Pireneus, de um frio
tão severo que se mostrava intolerável, colocando nossos dedos das mãos
e dos pés sob o risco de entorpecimento e morte.

O pobre Sexta"-Feira ficou deveras assustado ao ver as montanhas cobertas
de neve e sentiu o tempo frio, que ele nunca havia visto ou sentido em
sua vida.

Para completar o quadro, quando chegamos a Pamplona, a neve caiu com
tanta violência e por tanto tempo que as pessoas diziam que o inverno
chegara antes do tempo; e as estradas, que antes eram difíceis, estavam
intransitáveis; pois, para encurtar, a neve estava em alguns lugares
espessa demais para que pudéssemos viajar; e, por não estar congelada,
como é o caso nos países do norte, não havia como seguir sem corrermos o
risco de sermos enterrados vivos a cada passo. Ficamos não menos de
vinte dias em Pamplona; quando (vendo a proximidade do inverno, e
nenhuma probabilidade de o tempo melhorar; pois, tanto quanto a memória
do homem permitia saber, era o inverno mais rigoroso que a Europa já
havia conhecido) propus que partíssemos para Fontarabia, e de lá
tomássemos um navio para Bordéus, que era uma viagem muito curta.

Mas, enquanto pensávamos sobre essa possibilidade, chegaram quatro
cavalheiros franceses que, tendo parado do lado francês dos
desfiladeiros, como nós do lado espanhol, encontraram um guia que,
percorrendo a região em cerca da divisa com Languedoc, os havia levado
através das montanhas por caminhos tais que a neve não os havia
incomodado demasiado; e onde a haviam encontrado em qualquer quantidade,
diziam estar congelada o suficiente para suportá"-los e aos cavalos.
Mandamos chamar esse guia, que nos disse que se encarregaria de nos
transportar da mesma forma, sem que corrêssemos o perigo da neve, desde
que estivéssemos armados o suficiente para que nos protegêssemos das
feras; pois, disse ele, nessas grandes neves era frequente que lobos se
mostrassem no sopé das montanhas, famintos pela falta de comida, estando
o solo coberto de neve. Dissemo"-lhe que estávamos bem preparados para
criaturas que tais, desde que nos protegesse de uma espécie de lobos de
duas pernas, que, segundo nos haviam dito, era perigo maior,
especialmente do lado francês das montanhas.

Ele nos convenceu de que não havia perigo daquele tipo no caminho que
iríamos trilhar; e assim prontamente concordamos em segui"-lo, assim como
doze outros cavalheiros com seus criados, uns franceses, outros
espanhóis, que, como disse, haviam tentado seguir viagem e quedaram
obrigados a recuar.

Assim, partimos de Pamplona com nosso guia em 15 de novembro; e fiquei
em verdade surpreso, pois, em vez de seguir adiante, ele retrocedeu
conosco pela mesma estrada que viéramos de Madri obra de vinte milhas;
quando, tendo atravessado dois rios e adentrando a planície, vimo"-nos
outra vez em clima quente, onde a região era agradável e não havia sinal
de neve; mas, de repente, virando à esquerda, aproximou"-se das montanhas
por outro caminho; e embora fosse verdade que as colinas e precipícios
pareciam terríveis, foram tantas as voltas que deu, os meandros que
percorreu, e tão sinuosos os caminhos que tomou, que sem o perceber
cruzamos os cumes das montanhas sem que a neve nos obstasse; e de
repente ele nos apresentou as agradáveis e abundantes províncias de
Languedoc e Gasconha, todas verdes e florescentes; embora estivesse a
uma grande distância, pois tínhamos caminhos difíceis pela frente.

Ficamos um pouco intranquilos, porém, quando se percebeu que nevava
havia um dia e uma noite inteiros, e tanto que não podíamos seguir
viagem; mas ele pediu que nos acalmássemos; pois em breve teríamos
atravessado tudo; e notamos que, de fato, começamos a descer todos os
dias e a avançar ao norte sempre mais; e assim, aos cuidados de nosso
guia, prosseguimos.

Faltavam cerca de duas horas para o anoitecer quando, nosso guia estando
algo à nossa frente, e não à vista, surgiram correndo de um leito seco,
adjacente à mata cerrada, três lobos monstruosos e, em seu encalço, um
urso; e dois dos lobos arrojaram"-se sobre o guia, e estivesse ele meia
milha adiante, teria sido devorado antes que pudéssemos ajudá"-lo; pois
um deles agarrou"-se ao seu cavalo, e o outro atacou o homem com tamanha
violência que ele não teve tempo, ou presença de espírito bastante, para
sacar sua pistola, pondo"-se a gritar e berrar com muita força; e estando
Sexta"-Feira ao meu lado, pedi"-lhe que avançasse e visse o que se
passava. Assim que Sexta"-Feira avistou o homem, gritou tão alto quanto
ele: ``Ó senhor! Ó senhor!'', mas, como homem de brios, cavalgou
diretamente em direção ao pobre homem, e com sua pistola atirou na
cabeça do lobo que o atacava.

Grande sorte teve o pobre homem por se tratar de Sexta"-Feira; pois,
habituado que era em seu país a tais criaturas, não sentia medo, e
aproximou"-se dela e descarregou a arma; ao passo que qualquer outro
dentre nós teria disparado a longa distância e talvez não houvesse
acertado o lobo, ou tivesse corrido o risco de ferir o homem.

Mas era o suficiente para ter aterrorizado homens mais corajosos do que
eu; e, de fato, o caso assustou toda a nossa companhia, quando, com o
barulho da pistola de Sexta"-Feira, ouvimos de ambos os lados os mais
medonhos uivos de lobos; e o som, redobrado pelo eco das montanhas,
dava"-nos a impressão de houvesse um número prodigioso deles; e talvez
não fossem de fato poucos, a ponto de não termos motivo para apreensão.

De qualquer modo, tendo Sexta"-Feira matado um dos lobos, o outro que se
agarrara ao cavalo abandonou"-o imediatamente e fugiu; tendo por
felicidade se agarrado a sua cabeça, onde as tachas do freio
prenderam"-se a seus dentes; de sorte que não lhe causou muitos
ferimentos. Mas o homem ficou muito ferido; pois a criatura furiosa o
havia mordido duas vezes, uma vez no braço e a outra um pouco acima do
joelho; e ele estava a ponto de ser lançado ao chão pela agitação do
cavalo, quando Sexta"-Feira chegou e matou o lobo.

Não é difícil supor que, ao estampido da pistola de Sexta"-Feira, todos
ajustamos o andor e cavalgamos tão rápido quanto nos permitia o chão,
que era muito acidentado, para ver o que se passava. Assim que superamos
as árvores que nos obstavam a visão, percebemos com clareza o que se
passara e como Sexta"-Feira havia livrado o pobre guia, embora não
tenhamos no instante discernido que espécie de criatura ele havia
matado.

Mas nunca uma luta foi tão renhida, nem travada de forma tão formidável,
quanto a que se seguiu entre Sexta"-Feira e o urso, que nos trouxe a
todos (embora a princípio tenhamos ficado muito assustados e temerosos
por ele) um divertimento como não se pode imaginar. Como o urso é
criatura pesada e desconforme, e não galopa como o lobo, que é leve e
ligeiro, ele tem duas qualidades que geralmente são a regra de suas
ações; primeiro, quanto aos homens, que não são sua presa propriamente;
isto é, não são sua presa propriamente, embora não possa dizer quais são
os efeitos da fome excessiva, o que era provável na ocasião, estando o
chão inteiro coberto de neve; mas quanto ao homem, ele geralmente não o
ataca, senão quando o homem o ataca primeiro; por outro lado, se no caso
de você o encontrar na mata, se você não bolir com ele, ele não bolirá
com você; cuide apenas de ser cortês com ele e dar"-lhe passagem; pois
ele é um cavalheiro caprichoso, que não se desviará nem um palmo em seu
caminho, nem por um príncipe; e, se você estiver realmente com medo, o
melhor a fazer é olhar para outro lado e não parar; porque por vezes,
caso você interrompa o passo, fique parado e olhe fixamente para ele, é
possível que ele o tome como afronta; e se você jogar ou atirar qualquer
coisa nele, ainda que seja apenas um pedaço de pau do tamanho de seu
dedo, ele se põe ofendido e deixa tudo quanto esteja a fazer para buscar
a vingança; tendo por única motivação a satisfação da mesma; e essa é
sua primeira qualidade; sendo a segunda que, uma vez ofendido, ele nunca
o deixará, noite ou dia, até que alcance sua vingança; perseguindo a uma
boa velocidade até que o alcance.

Sexta"-Feira havia livrado nosso guia e, quando nos aproximamos, ele o
ajudava a apear do cavalo, pois o homem estava ferido e assustado, e, em
verdade, mais o primeiro que o segundo; quando de repente vislumbramos o
urso saindo da mata; e era monstruoso, de longe o maior que já vi.
Ficamos todos um pouco assustados quando o vimos; mas quando Sexta"-Feira
deparou"-se com o animal, não se podia deixar de ver a alegria e a
coragem no semblante do sujeito.

``Oh! Oh! Oh!'', disse Sexta"-Feira, três vezes, apontando ao urso;
``Senhor, dar licença, eu cumprimentar ele; eu fazer o senhor
divertir.''

Espantou"-me muito ver o sujeito tão satisfeito. ``Seu estúpido'', eu
disse, ``ele vai comer você.''. ``Ele me devorar! Me devorar'', exclamou
Sexta"-Feira, duas vezes de novo; ``Eu devorar ele; eu divertir senhor;
vocês ficar aí, eu divertir vocês''; e então ele se senta, descalça num
segundo as botas e coloca uns escarpins (como chamamos os calçados
baixos que eles usam), que ele trazia no bolso, dá ao meu outro criado
seu cavalo, e com a arma sai em desabalada corrida, arrebatado como o
vento.

O urso movia"-se com vagar e não dava mostras de que avançaria sobre
alguém, até que Sexta"-Feira, aproximando"-se, chamou"-o, como se o urso
pudesse entendê"-lo. ``Aqui, aqui'', disse Sexta"-Feira, ``eu falar
você.'' Nós o seguíamos à distância; pois agora descendo a face gascã
das montanhas, adentrávamos uma imensa floresta, onde as terras eram
planas e bastante abertas, com muitas árvores espalhadas aqui e ali.

Sexta"-Feira, muito mais ligeiro do que o urso, arrojou"-se sobre ele com
rapidez e, travando de uma grande pedra, atirou"-a contra ele,
acertando"-o na cabeça; o que não teve mais efeito do que se a tivesse
lançado contra um muro; porém atendeu aos objetivos de Sexta"-Feira, pois
o finório tanto não tinha medo que tencionara com isso apenas fazer o
urso segui"-lo e nos dar algum divertimento, como disse ele.

Assim que o urso sentiu o golpe e viu Sexta"-Feira, deu meia volta e veio
a seu encontro, dando passadas diabolicamente largas e avançando a um
ritmo estranho, como o de um cavalo a meio galope; Sexta"-Feira correu a
toda brida, sugerindo em seu rumo que vinha em nossa direção em busca de
ajuda; e assim todos decidimos de imediato atirar no urso e salvá"-lo;
ainda que me enfurecesse por o ter trazido para perto de nós, quando ele
seguia em outra direção e cuidava da própria vida; mas sobretudo me
enfureci por ter ele dirigido o urso contra nós e depois fugido; e
gritei: ``Seu cão! É isso que você chama de divertimento? Venha, tome
seu cavalo, que nós vamos descarregar contra a criatura'', e tendo
escutado o que lhe disse, gritou, ``Não atirar, não atirar, quieto, que
vai divertir''; e enquanto a criatura ligeira corria dois pés a cada um
do urso, ele se virou de súbito em nossa direção e, avistando um grande
carvalho adequado a seu propósito, acenou para que o seguíssemos; e
dobrando o passo, subiu agilmente na árvore, tendo pousado a arma no
chão, a obra de cinco ou seis jardas do pé da árvore.

O urso logo chegou à árvore, e nós o acompanhamos à distância; a
primeira coisa que ele fez foi parar diante da arma e cheirá"-la; e
pôs"-se a subir a árvore, escalando"-a como um gato, embora tão
monstruosamente pesado; deixando"-me estupefato a loucura, tal qual a
pensava, de meu homem, sem ter a menor condição de ver o que pudesse me
levar ao riso; até que, vendo o urso escalar a árvore, todos cavalgamos
para perto dele.

Quando chegamos à árvore, lá estava Sexta"-Feira, indo à ponta de um
galho largo, e o urso a meio caminho dele. Assim que o urso tocou a
parte onde o galho da árvore se mostrava mais fraco, ``Ha!'' ele disse,
``agora ver ensinar urso dançar''; e então ele começou a pular e sacudir
o galho; no qual o urso começou a cambalear, no entanto ainda parado,
começando a olhar para trás, para ver como voltaria; quando, de fato,
rimos muito. Mas Sexta"-Feira não havia ainda acabado com ele; ao vê"-lo
parado, chamou"-o novamente, como se achasse que podia falar inglês: ``O
quê, não vem mais longe? Vem, vem'', e então ele deixou de balançar o
galho; e o urso, como se tivesse compreendido o que dizia ele,
aproximou"-se um pouco mais; e ele começou a pular de novo, e o urso
parou de novo.

Julgamos que era um bom momento para disparar contra sua cabeça e gritei
a Sexta"-Feira que ficasse parado, que atiraríamos contra o urso: mas ele
respondeu com veemência: ``Oh, não! Não! Não hoje!'', com o que ele
queria dizer ``agora''; de qualquer modo, para encurtar a história,
Sexta"-Feira dançou tanto, e o urso ficou tão aborrecido, que nós de fato
rimos muito, mas ainda não conseguíamos imaginar o que o sujeito faria;
pois de início entendemos que ele tencionava chacoalhar o urso até que
este caísse; e percebemos que o urso era esperto demais para isso; pois
ele nunca avançava a ponto de ser lançado árvore abaixo, mas se agarrava
ao galho com suas enormes patas, de maneira que não conseguíamos
imaginar que fim teria a história, e qual o sentido do gracejo.

Mas Sexta"-Feira rapidamente dirimiu nossas dúvidas; pois vendo que o
urso se agarrava ao galho, e que não seria convencido a ir mais longe,
``Bem, bem'', disse Sexta"-Feira, ``não vem mais perto; não ficar perto
de mim; eu ir para você''; e com isso ele foi para a extremidade menor,
que dobraria com seu peso, e suavemente se agachou e desceu preso ao
galho até quase tocar o chão e conseguir colocar"-se de pé, correndo para
sua arma, pegando"-a e colocando"-se à espera.

``Bem'', disse"-lhe eu, ``o que você vai fazer agora? Por que você não
atira?'' ``Sem atirar'', respondeu Sexta"-Feira, ``não ainda, se atirar,
não matar; ficar parado, divertir mais vocês''; e assim ele fez, como se
verá agora; pois quando o urso viu que seu inimigo tinha se afastado,
ele recuou do ganho onde estava; mas o fez lentamente, olhando para trás
a cada passo, e recuando até que tocou o tronco da árvore; e então, com
o traseiro à frente, desceu a árvore usando de suas garras e movendo uma
pata por vez, muito lentamente; a esse ponto, e um pouco antes de
conseguir apoiar as patas no chão, Sexta"-Feira aproximou"-se dele, enfiou
o cano de sua espingarda na orelha do urso e, disparando, fê"-lo cair
morto como pedra.

O finório virou"-se, então, para verificar se não ríamos; e quando ele
viu que nossos semblantes pareciam satisfeitos, ele se pôs a rir alto.
``Assim matar ursos na minha terra'', disse Sexta"-Feira. ``Como vocês os
matam assim, se vocês não têm armas de fogo?'', retorqui eu. ``Não'',
disse ele, ``mas atirar grande flecha muito comprida''.

Isso havia constituído, sem dúvida, grande divertimento para nós; mas
ainda nos encontrávamos em lugar selvagem, com nosso guia bastante
ferido e quase sem saber o que fazer; com os uivos dos lobos ainda
ressoando em minha cabeça; e de fato, senão pelos sons que antes
escutara na costa da África, dos quais já disse algo noutro momento,
nunca havia escutado coisa que tanto me enchesse de terror.

Essas coisas, e a proximidade da noite, exigiam que partíssemos; não
fosse por isso, como teria desejado Sexta"-Feira, decerto teríamos
esfolado aquela criatura monstruosa, cuja pele valia o esforço; mas
tínhamos quase três léguas pela frente, e nosso guia nos apressava;
assim, deixamo"-lo e seguimos em nossa jornada.

O solo ainda estava coberto de neve, embora não tão profunda e perigosa
como nas montanhas; e as criaturas vorazes, como ouvimos mais tarde,
desciam à floresta e à planície em busca de alimento, compelidas pela
fome, causando estragos muitos nas aldeias, onde surpreendiam as gentes
do campo, matando"-lhes ovelhas e cavalos em grande número, e pessoas
também, algumas.

Tínhamos um lugar de perigosa travessia pela frente, e nosso guia nos
disse que, se houvesse mais lobos na região, nós os encontraríamos lá; e
se tratava de pequena planície, cercada de mata por toda a parte, e com
um longo e estreito desfiladeiro, ou passo, que deveríamos superar para
vencer a floresta, e então chegar à aldeia onde nos hospedaríamos.

Estávamos a meia hora de o sol se pôr quando entramos na floresta; e não
fazia muito que anoitecera quando chegamos à planície. Nada encontramos
na primeira mata, senão que, numa pequena clareira em meio às árvores,
de extensão não maior do que cem braças, vimos cinco grandes lobos
cruzarem a estrada a toda velocidade, um após o outro, como se
estivessem perseguindo uma presa; e tendo"-a em vista; eles não deram por
nossa presença e sumiram de vista em poucos instantes.

Diante disso, nosso guia, que, a propósito, era apenas um pobre a quem
muito faltava coragem, pediu que nos mantivéssemos a postos, pois
acreditava que mais lobos estavam por vir.

Mantivemos nossas armas carregadas e os olhos vivos ao redor; mas não
avistamos outros lobos, até que atravessamos a mata, que tinha obra de
meia légua, e adentramos na planície. Assim que chegamos à planície,
tivemos oportunidade de inquirir as cercanias; e o primeiro objeto com
que nos deparamos foi um cavalo morto; isto é, um pobre cavalo que os
lobos haviam matado, e em torno do qual ao menos uma dúzia deles se
aplicava; não o devorando, como se podia observar, mas lhe beliscando os
ossos; pois já lhe haviam comido toda a carne.

Não julgamos adequado perturbá"-los em seu festim, tampouco eles nos
dedicaram muita atenção. Sexta"-Feira teria disparado contra eles, mas eu
não o toleraria, de forma alguma; pois percebia que tínhamos mais a
fazer do que nos parecia. Não havíamos percorrido metade da planície
quando começamos a ouvir uivos assustadores de lobos na mata à nossa
esquerda, e logo em seguida vimos não menos de cem deles vindo em nossa
direção, todos formando um só corpo, e a maioria em linhas tão regulares
quanto as de um exército formado por experientes oficiais. Mal conseguia
conceber como os receberíamos; mas percebi que nos reunir numa só
fileira era a única forma; e assim num instante nos colocamos em
formação; mas para que não houvesse grande intervalo entre as baterias,
ordenei que disparássemos alternadamente, homem sim, homem não; e que os
que não tivessem atirado permanecessem prestes a lhes dar uma segunda
salva, caso continuassem a avançar em nossa direção; e então que aqueles
que haviam atirado primeiro não tencionassem carregar suas espingardas
novamente, mas que permanecessem prontos, cada qual com uma pistola;
pois todos nós estávamos armados com uma espingarda e um par de pistolas
cada; e assim fomos, por este método, capazes de disparar seis salvas,
cada grupo a sua vez; ainda que, de qualquer modo, não tenha havido
necessidade de tanto; pois, ao disparar a primeira salva, o inimigo
interrompeu o avanço por completo, ficando apavorado tanto com o barulho
quanto com o fogo. Quatro deles, levando tiros na cabeça, tombaram; e
vários outros ficaram feridos e sangraram, como pudemos ver pela neve.
Percebi que haviam parado, mas não recuado imediatamente; então,
lembrando"-me de que me haviam dito que as mais ferozes criaturas ficavam
apavoradas ao som da voz de um homem, fiz com que todos gritassem o mais
alto que pudessem; ao que pude constatar que a ideia não estava de todo
errada; pois, ao nosso grito, eles se puseram a dar meia volta e se
retirar. Então ordenei que uma segunda rajada fosse disparada em sua
retaguarda, o que os colocou a galope, seguindo para a floresta.

Isso nos deu tempo para carregar outra vez nossas armas; e para que não
perdêssemos tempo, o fizemos em movimento; mas tínhamos pouco mais do
que carregado nossas espingardas e nos aprestado, quando ouvimos um
barulho terrível na mesma mata à nossa esquerda, apenas mais adiante,
vindo da mesma direção que tomávamos.

Caía a noite, e a luz começava a se fazer crepúsculo, o que tornava tudo
pior para nós; mas com o aumento do alvoroço, podíamos facilmente
perceber que eram uivos e gritos daquelas criaturas infernais; e de
repente nos deparamos com duas ou três tropas de lobos, uma à nossa
esquerda, outra em nossa retaguarda e mais uma à nossa frente, de
maneira que parecíamos estar cercados por eles: porém, como não
avançaram sobre nós, mantivemos nosso caminho adiante, tão rápido quanto
podíamos fazer nossos cavalos cavalgarem, o que, sendo a trilha muito
acidentada, era apenas um bom e duro trote; e assim, avistamos a entrada
de um bosque pelo qual deveríamos passar, no outro lado da planície; mas
ficamos muito espantados, quando ao nos aproximarmos da trilha ou passo,
vimos um número indeterminado de lobos parados logo à entrada.

De repente, de outra abertura em meio ao arvoredo, escutamos um
estampido de arma; e ao olharmos naquela direção, vimos apressar"-se um
cavalo, selado e com rédeas, em desabalada carreira, e dezesseis ou
dezessete lobos atrás de si, a toda brida. O cavalo levava vantagem
sobre eles; mas, como supúnhamos que ele não conseguiria aguentar
naquele ritmo, não tínhamos dúvida de que eles por fim o alcançariam; e
foi assim aconteceu.

Nesse momento, presenciamos uma terrível cena; pois ao cavalgar à
abertura de onde o cavalo saíra, deparamo"-nos com as carcaças de outro
cavalo e dois homens, devorados pelas famintas criaturas; e um dos
homens, sem dúvida, era o mesmo que ouvíramos disparar a arma, pois ali
jazia uma arma descarregada bem ao seu lado; enquanto o homem tinha a
cabeça e o tronco consumidos.

Foi grande o horror que nos assaltou, e não sabíamos que curso tomar;
mas as criaturas não nos deram tempo para dúvidas; pois reuniram"-se ao
nosso redor na esperança de obter uma presa; e eu tenho absoluta crença
de que havia trezentas delas. Aconteceu, para nossa grande sorte, que à
entrada da floresta, não muito distante, havia uns grandes troncos, que
haviam sido cortados no verão anterior, e suponho que lá estivessem para
frete. Levei minha pequena tropa para entre os troncos e, colocando"-nos
perfilados atrás de uma comprida árvore, aconselhei a todos que ali
apeassem e, mantendo a árvore diante de nós à guisa de anteparo,
formassem um triângulo, ou três frentes, protegendo nossos cavalos em
seu centro.

Foi o que fizemos, e foi bom; pois não se tem notícia de ataque mais
furioso do que o realizado por aquelas criaturas contra nós naquele
lugar. E elas avançaram com uma espécie de rosnado (e montaram no tronco
cortado, que como disse, era nosso anteparo), como se estivessem apenas
se arrojando contra sua presa; e essa fúria, ao que parece, fora
ocasionada sobretudo por verem os cavalos atrás de nós, os quais eram as
presas que almejavam. Ordenei aos nossos homens que atirassem como
antes, uns após os outros; e tão certeiros foram que mataram vários
lobos já na primeira salva; mas havia a necessidade de dar carga
contínua, pois eles avançavam como demônios, os que vinham atrás
empurrando os que estavam à frente.

Quando disparamos uma segunda salva de nossas espingardas, entendemos
que eles haviam parado um pouco, e eu esperava que partissem; mas foi
apenas um instante; pois outros avançaram por sua vez; e então
disparamos duas saraivadas de nossas pistolas; e creio que nessas quatro
cargas matamos dezessete ou dezoito deles, e ferimos o dobro, mas eles
retornaram.

Eu não queria gastar nossa munição tão rápido; então chamei meu criado,
não Sexta"-Feira, pois este estava mais bem empenhado; e ele, com a maior
destreza imaginável, carregou minha espingarda e a dele enquanto
estávamos em luta; mas, como disse, solicitei meu outro criado, e
entregando"-lhe um chifre de pólvora, pedi"-lhe que derramasse um rastro
dela ao longo do tronco, e que fosse comprido. Ele assim o fez, e só lhe
restou tempo para fugir, quando os lobos chegaram, a ponto de alguns
subirem a barreira; e eu, disparando uma pistola descarregada perto da
pólvora, toquei"-lhe fogo; os que estavam sobre o tronco arderam com ela,
e seis ou sete deles tombaram; ou melhor, saltaram entre nós diante da
força e medo do fogo; os quais despachamos num instante, enquanto os
restantes se viram tão assustados com a luz, que a noite, pois já estava
quase escuro, fazia ainda mais terrível, que eles recuaram um pouco.

Ao que ordenei que nossas últimas pistolas fossem disparadas em uma só
saraivada, e em seguida gritamos; e diante dos gritos os lobos correram
com rabo entre as pernas, e nós atacamos de imediato os que contávamos
quase em vinte, todos feridos e se debatendo no chão; e estes executamos
com nossas espadas, o que atendeu nossas expectativas, visto que os
gemidos e uivos que produziram foram mais bem compreendidos por seus
companheiros; de sorte que fugiram e nos deixaram.

Por fim, havíamos matado mais ou menos sessenta deles; e, fosse de dia,
teríamos matado muitos mais. Assim desocupado o campo de batalha,
pudemos avançar novamente, pois ainda tínhamos quase uma légua pela
frente. Muitas vezes ouvimos os uivos e o vozerio das feras na mata
enquanto íamos; e por vezes imaginávamos ter visto algumas delas; mas
com a neve nos confundindo os olhos, não tínhamos certeza. E assim
levamos obra de uma hora para chegar à cidade onde nos hospedaríamos,
que encontramos em grande terror e inteira armada; pois, ao que parece,
na noite anterior os lobos e alguns ursos haviam invadido a aldeia,
colocando"-os em tamanho sobressalto que foram obrigados a manter guarda
noite e dia, mas especialmente à noite, para proteger seu gado e,
sobretudo, sua gente.

Na manhã seguinte, nosso guia encontrava"-se tão enfermo, e seus membros
tão inchados com a inflamação de suas duas feridas, que ele não foi
capaz de prosseguir viagem; de forma que fomos obrigados a buscar um
novo guia e rumar para Toulouse, onde fazia clima quente, e a terra era
fértil e agradável, e sem neve, sem lobos ou coisa que o valesse; mas
quando contamos nossa história em Toulouse, disseram"-nos eles em
resposta que aquilo não era mais do que o comum na mata ao pé das
montanhas, especialmente quando havia neve no chão; mas muitos quiseram
saber que sorte de guia era aquele que nos conduzia e se aventurava a
nos levar por aqueles caminhos em estação tão severa, e se manifestaram
espantados por não termos sido devorados.

Quando lhes contamos como nos posicionamos, com os cavalos no meio, eles
ficaram pasmos e responderam que apenas por muita sorte ainda restávamos
vivos, pois fora a visão de sua presa, a saber, dos cavalos, que havia
deixado os lobos tão furiosos; e que em outros momentos eles de fato
temem as armas de fogo; mas que, estando excessivamente famintos, e
furiosos em razão da fome, a ânsia de atacar os cavalos os havia tornado
insensíveis ao perigo, e que, se não tivéssemos, pelos disparos
contínuos e, por fim, pelo estratagema do rastro de pólvora, os
dominado, não era pequena a probabilidade de termos sido estraçalhados;
enquanto que, houvéssemos nós permanecido sobre nossos cavalos e atirado
de sobre eles, as feras não teriam uma tal percepção dos cavalos como
presa com os homens neles montados do que o contrário; e disseram"-nos
que, por fim, se tivéssemos permanecido todos juntos e abandonado nossos
cavalos, eles estariam tão ansiosos para devorá"-los, que poderíamos ter
saído em segurança, especialmente com nossas armas de fogo à mão, e
sendo em grande número com éramos.

De minha parte, nunca estive tão sensível ao perigo de morte que corria;
pois, ao ver mais de trezentos demônios correndo em meio a rugidos e de
bocas abertas prontas a nos devorar, e não ter o que nos abrigasse ou
para onde correr, dei"-me por morto; e, tal como aconteceu, creio que
nunca mais pensarei em atravessar aquelas montanhas: penso que
preferiria muito mais seguir mil léguas por mar, ainda que certo de que
enfrentaria uma tempestade por semana.

Nada tenho de extraordinário a relatar em minha passagem pela França;
nada além do que outros viajantes relataram com muito mais interesse do
que eu poderia fazê"-lo. Viajei de Toulouse a Paris, e sem maiores
paradas, cheguei a Calais e desembarquei são e salvo em Dover a 14 de
janeiro, depois de enfrentar uma estação terrivelmente fria.

Eu tinha chegado agora ao destino de minhas viagens e, em pouco tempo,
todos os meus bens recém"-descobertos encontravam"-se em segurança ao meu
redor, tendo recebido devidamente pelas letras de câmbio que trouxera
comigo.

Minha principal guia e conselheira particular era minha boa e velha
viúva, que, em gratidão pelo dinheiro que eu lhe mandara, não conhecia
esforço ou dificuldade quando se tratava de me auxiliar; e confiava
tanto nela que estava absolutamente em paz quanto à segurança de meus
pertences; e, de fato, encontrava"-me muito feliz, como fora desde o
início e dali seria até o fim, sob a integridade imaculada dessa boa
dama.

E então eu passei a pensar em deixar meus bens com essa boa dama e
viajar a Lisboa e, em seguida, aos Brasis; mas um novo escrúpulo
rondou"-me a mente, e esse foi a religião; pois conservava dúvidas sobre
a religião romana, mesmo quando vivi no exterior, e em particular em meu
período de solidão; então eu sabia que não viajaria aos Brasis e muito
menos habitaria lá, a não ser que decidisse abraçar a religião católica
romana sem reservas; a não ser, por outro lado, que eu decidisse
sacrificar meus princípios e me tornar um mártir da religião e morrer na
Inquisição; então decidi permanecer em casa, e se encontrasse meios para
tanto, desfazer"-me da plantação.

Para tanto, escrevi ao meu velho amigo de Lisboa, que, em resposta,
disse"-me que poderia facilmente vendê"-la ali; mas que se eu julgasse
adequado dar"-lhe permissão para oferecê"-la em meu nome aos dois
mercadores, herdeiros de meus administradores, que viviam nos Brasis e
saberiam apreciar inteiramente seu preço, os quais viviam no próprio
lugar e, sabia eu, eram ricos; de maneira que ele acreditava que eles
gostariam de comprá"-la; e ele não duvidava de que eu conseguisse de
quatro a cinco mil moedas de \emph{peso fuerte}, ou mais.

Assim, concordei, dando"-lhe ordem de oferecê"-la a eles, e ele assim o
fez; e em obra de oito meses, com o navio retornando, ele me enviou
notícia de que eles haviam aceito a oferta e feito uma remessa de trinta
e três mil \emph{pesos fuertes} a um correspondente em Lisboa, que seria
responsável pelo pagamento.

Em troca, assinei o instrumento de venda na forma que enviaram de
Lisboa, e enviei"-o ao meu amigo, que a mim remeteu as letras de câmbio
no valor de trinta e duas mil e oitocentas moedas de \emph{peso fuerte}
em troca da propriedade; reservando"-se o pagamento vitalício de cem
\emph{moidores} por ano para ele, meu amigo, e de cinquenta
\emph{moidores} depois para seu filho, que eu lhes havia prometido,
valores os quais a plantação faria cumprir sob a forma de rendas
constituídas. E assim eu encerro a primeira parte de uma vida de fortuna
e aventura, uma vida que se cumpriu nos estranhos e extremos trabalhos
da Providência, e de uma variedade de que o mundo raramente terá
notícia; começando de maneira tola, mas encerrando com muito mais
felicidade do que qualquer outra parte dela jamais me permitiria
esperar.

Qualquer um pensaria que, nesse estado de rara boa fortuna, eu não
correria mais nenhum perigo; e de fato assim seria, se a ela tivessem
concorrido outras circunstâncias; mas eu estava acostumado a uma vida
errante, não tinha família, nem muitos parentes; nem, por mais rico que
fosse, tinha muitos conhecidos; e embora eu tivesse vendido minha
propriedade nos Brasis, não conseguia tirar aquele país da cabeça e
tencionava bater asas novamente; em especial, não pude resistir ao forte
desejo de ver a minha ilha e de saber se os pobres espanhóis ainda por
lá viviam e que tratamento os canalhas que ali deixara lhes haviam
despendido.

Minha grande amiga, a viúva, procurou dissuadir"-me veementemente, e
tanto foi persuasiva que por quase sete anos ela conseguir conter minha
vontade de partir, período durante o qual trouxe meus dois sobrinhos,
filhos de um de meus irmãos, sob meus cuidados; tendo o mais velho algo
próprio, criei"-o como cavalheiro e dispus em testamento um acréscimo a
sua propriedade; o outro eu coloquei com o capitão de um navio; e depois
de cinco anos, julgando"-o um jovem razoável, ousado e empreendedor,
coloquei"-o em um bom navio e mandei"-o ao mar; e esse jovem depois me
atraiu, ainda que velho, para outras aventuras.

Nesse ínterim, em parte me instalei aqui; pois, em primeiro lugar,
casei"-me, e não para meu infortúnio ou insatisfação, e tive três filhos,
dois meninos e uma filha; mas tendo minha esposa morrido e meu sobrinho
retornado a casa com o bom sucesso de uma viagem à Espanha, minha
inclinação a viajar ao exterior e sua insistência foram mais fortes e
embarquei em seu navio na condição de mercador particular com destino às
Índias Orientais; isso se deu no ano de 1694.

Nessa viagem visitei minha nova colônia na ilha, vi meus sucessores, os
espanhóis, ouvi toda a história de suas vidas e dos vilões que lá
deixei; e de como no início eles insultaram os pobres espanhóis; e
depois se conciliaram, discordaram, se uniram, se separaram; e de como,
por fim, os espanhóis foram obrigados a usar de violência contra eles;
de como eles foram subjugados pelos espanhóis; e com que honestidade os
espanhóis os trataram; uma história que, caso fosse contada, seria tão
cheia de variedade e acidentes formidáveis quanto a minha própria; em
particular suas batalhas contra os caraíbas, que várias vezes
desembarcaram na ilha; e quanto às melhorias que fizeram à própria ilha,
e como cinco deles realizaram uma excursão ao continente, de onde
trouxeram onze homens e cinco mulheres prisioneiros, entre os quais, em
minha chegada, encontrei obra de vinte crianças na ilha.

Ali fiquei mais ou menos vinte dias, deixando"-lhes suprimentos de todas
as coisas necessárias, e particularmente de armas, pólvora, chumbo,
roupas, ferramentas e dois trabalhadores, que trouxe da Inglaterra
comigo, a saber, um carpinteiro e um ferreiro.

Além disso, dividi as terras em partes entre eles, reservei para mim a
propriedade do todo, mas dei"-lhes as partes, respectivamente, conforme
seu assentimento; e tendo todas as coisas resolvidas com eles, e
exortado"-lhes para que não deixassem o lugar, eu parti.

De lá segui para os Brasis, de onde mandei uma barca, que lá adquiri,
com mais gente para a ilha; e nela, além de outros suprimentos, enviei
sete mulheres, tais como as julguei adequadas ao serviço, ou a serem
esposadas por quem as quisesse. Quanto aos ingleses, prometi mandar"-lhes
algumas mulheres da Inglaterra, acompanhadas de um bom frete de objetos
de primeira necessidade, desde que se dedicassem ao plantio, e assim o
cumpri posteriormente. Os sujeitos mostraram"-se muito honestos e
industriosos depois de terem sido dominados e terem suas propriedades
delimitadas. Enviei"-lhes, também, dos Brasis, cinco vacas, três delas
prenhes de bezerros, algumas ovelhas e alguns porcos, que quando voltei
estavam consideravelmente aumentados.

Mas sobre todas essas coisas, com relação de como trezentos caraíbas
vieram e os invadiram e arruinaram suas plantações, e de como eles
lutaram contra todo aquele exército duas vezes, tendo sido derrotados a
princípio, e um deles morto; e de como, por fim, uma tempestade destruiu
as canoas de seus inimigos, e eles mataram de fome ou em batalha quase
todos os demais; e de como eles renovaram e recuperaram a posse de sua
plantação, e ainda viviam na ilha.

De todas essas coisas, com alguns incidentes muito notáveis em algumas
novas aventuras minhas, por mais dez anos, talvez venha a dar relação
mais adiante.

\bigskip
\bigskip

\begin{center}
\textsc{fim}
\end{center}