\chapterspecial{Introdução}{Crusoe, o náufrago}{}


\section{A prosa ficcional do romance}

O surgimento da forma romance\footnote{A palavra, nas línguas latinas,
  designava antes narrativas em verso, como também eram reconhecidas em
  inglês (que, aliás, chama \emph{novel}, a partir do italiano
  \emph{novella}, o que chamamos \emph{romance}). E assim os vários
  romances de cavalaria, como aqueles do Rei Arthur e da Távola Redonda,
  e o \emph{Roman de la Rose}, ou os numerosos \emph{romanceros}
  espanhóis.} praticamente coincide com as descobertas do chamado Novo
Mundo: não é casual. O romance é também um novo mundo formal nas
práticas letradas, que de modo lento foi alcançando dignidade conferida,
de início, apenas ao verso. Sua estrutura misturada, a hipótese do
gênero baixo recebendo as peculiares contribuições da escrita
tratadística, da retórica persuasiva das paixões, do drama, e mesmo da
\emph{ars dictaminis} --- ou a arte de redigir cartas --- levou tempo a
se estabelecer dentro de um pensamento que anteriormente se propunha a
partir de hipóteses fixas, de gêneros bem delimitados, dentro do que se
chamou, com Aristóteles, uma \emph{poética}\footnote{Significando uma
  arte (\emph{tekhné}), como todas, da imitação.}.

Daí que Petrônio, como Rabelais e também Cervantes, tenham escrito seus
respectivos \emph{Satyricon} (c. 60 d.C.), \emph{Gargântua e Pantagruel}
(1532--1564) e \emph{Dom Quixote} (1605) utilizando a \emph{vis comica},
ou um princípio do riso, codificado aristotelicamente como a ``imitação
dos piores'', e assim no \emph{Satyricon} contam"-se aventuras rudes de
personagens baixos, em \emph{Gargântua e Pantagruel} trata"-se de
gigantes (dentro do conceito de \emph{deformidade} ou
\emph{desproporção}) e em \emph{Dom Quixote} se avança para o que
chamaríamos \emph{patético}, ou seja, nem propriamente para o riso nem
para o aspecto trágico, mas para uma medida modesta na qual todas as
desproporções da cômica loucura de seu protagonista (como a rústica
astúcia de seu escudeiro) nos consternam, preparando"-nos para o que
Machado de Assis adiante já podia dizer engenhosamente aos leitores,
abrindo o seu \emph{Memórias Póstumas de Brás Cubas} (1881), que o
fizera com ``a pena da galhofa e a tinta da melancolia''\footnote{Lembrar
  que em 1590 havia sido publicado em Veneza o \emph{Pastor Fido}, de
  Giovanni Battista Guarini (1538--1612), que se apresentava, desde a
  folha de rosto, como uma \emph{Tragicomedia Pastorale}, evidente e
  voluntária mistura dos estilos alto e baixo, criticada por Gianvicenzo
  Gravina (1664--1718) no segundo livro de seu \emph{Della Ragion
  Poetica} (1771), onde afirma, quando fala do gênero pastoral: ``o qual
  {[}Guarini{]} transportou ao campo até as cortes, aplicando àqueles
  personagens {[}pastores rústicos{]}, no seu Pastor Fido, as paixões e
  costumes das antecâmaras, e as mais artificiosas tramas dos gabinetes,
  pondo na boca dos pastores preceitos encontrados, via de regra, no
  mundo político, e na das amorosas ninfas pensamentos tão rebuscados
  que parecem saídos das escolas dos atuais declamadores e
  epigramatistas'', \emph{in}: Gravina, Vincenzo. \emph{Della Ragion
  Poetica Libri Due}, Roma, \textsc{mdccviii}, Presso Francesco Gonzaga,
  pp.200--1.}.

O século \textsc{xviii}, sobretudo francês, faria do que chamamos prosa ficcional
algo que prepararia o terreno para o surgimento posterior de alguém como
Stendhal ou Flaubert, a partir dos quais a prosa se tornaria o veículo
mais prestigiado no registro da complexidade da vida humana. De
Flaubert, a própria poesia das vanguardas de língua inglesa tomaria a
noção de \emph{mot juste}, ou a ``palavra exata'', para resgatar a
robustez de uma arte da palavra; ou, como diria Stendhal, no famoso
diagnóstico exposto em \emph{De l'amour} (Do amor, 1822):

A poesia, com suas comparações obrigatórias, sua mitologia em que o
poeta não crê, sua dignidade de estilo à maneira de Luís \textsc{xiv} e todo o
aparato de seus ornamentos ditos poéticos, está muito abaixo da prosa
quando se trata de dar uma ideia clara e precisa dos movimentos do
coração (\ldots{})\footnote{Stendhal. \emph{Do Amor} (tradução de Roberto
  Leal Ferreira), São Paulo, Martins Fontes, 1991, p.218 (dos
  \emph{Fragmentos diversos}, \textsc{xciii}, ``\textsc{o amor antigo}''). Obviamente, o
  romance já perdeu essa qualidade superior ao menos desde a década de
  1950, quando passa a ter dificuldades de renovar as forças e os
  interesses da forma, depois sobretudo do \emph{Ulysses} (1922) e do
  \emph{Finnegans Wake} (1937), de James Joyce, e do beco sem saída das
  narrativas de Samuel Beckett, que misturava o próprio Joyce à seca
  máquina de escrita de Gertrude Stein. O que não significa que não se
  escreveram romances importantes desde então. \emph{Neue Leben} (Vidas
  Novas, 1993), de Ingo Schulze, é um exemplo.}.

E Stendhal estava certo. Stendhal, aliás, recomendava como manual de
estilo o \emph{Código Civil} francês, o código napoleônico.\footnote{Stendhal
  escreve a Balzac, e diz: ``Compondo a \emph{Chartreuse}, para acertar
  o tom, eu lia de tempos em tempos algumas páginas do Código Civil'',
  ``À Honoré de Balzac, 17--28 octobre 1840'', \emph{in}: Stendhal,
  \emph{Aux âmes sensibles: Lettres choisies (1800--1842)}, Choix
  d'Emmanuel Boudot"-Lamotte (Édition établie, présentée et annotée par
  Mariella di Maio), Paris, Gallimard, 2011, p.473.}

Mas aqui nos encontramos antes de tudo isso; estamos ainda, como
concebeu Ian Watt, na ascensão do romance\footnote{Watt, Ian. \emph{A
  ascensão do romance: estudos sobre Defoe, Richardson e Fielding}
  (tradução de Hildegard Feist), São Paulo Companhia das Letras, 1990.},
quando Daniel Defoe (1660--1731) escreve este seu livro sobre o náufrago
\emph{Robinson Crusoe} (1719), em primeira pessoa, incluindo o modo
epistolar de um diário. Mas nem ele nem Henry Fielding (1707--1754)
seriam os primeiros na chamada ``ascensão do romance'', sequer em língua
inglesa: a poeta, dramaturga e, sim, romancista, Aphra Behn (1640--1689),
precede a ambos em mais de 20 anos com \emph{Oroonoko, or the Royal
Slave, a True Story} (Oroonoko, ou um Escravo Real, uma História
Verdadeira, 1688)\footnote{O romance de Behn tem até mesmo um início de
  tipo antropológico, descrevendo os costumes da terra distante, como a
  leitora e o leitor certamente se recordarão de que é o caso de um bom
  tanto de romances, incluindo até mesmo, curiosamente, \emph{Dracula}
  (1897), de Bram Stoker. Watt em seu livro afirma que Behn, como
  outros, tem um estilo não realista, e a aproxima de raízes que chama
  ``medievais'' (arrolando seu nome junto aos de Rabelais e Sidney, o
  Philip Sidney do ``romance'' pastoral \emph{A Arcádia da Condessa de
  Pembroke}, de 1570--86), o que, não sendo inteiramente falso, não é
  inteiramente verdadeiro, e eu certamente não a distinguiria de modo
  cortante de Defoe, Fielding, ou Richardson, como faz Watt: ela está
  muito mais próxima deles do que de Rabelais, Sidney ou Bunyan.},
quando o desenho daquilo que viríamos a chamar ``romance'' já está quase
inteiramente traçado.

Interessante é notar que a viagem, o relato da distância --- espacial,
como de costumes --- fornece o impulso, como líamos na premissa mesma da
antiga \emph{Odisseia}, ou mesmo nos pontos de partida de muito drama,
como podemos lembrar de \emph{A Tempestade} (1610), de William
Shakespeare, levando em conta para a sua escrita, também, como
\emph{Crusoe}, um registro de viagem; e igualmente um naufrágio, que
contribui com elementos fundamentais e formantes para a história contada
lá, com objetivos inteiramente outros.

\section{Curiosidade, viagem (o movimento da empresa)}

Assim como o ``sonho visionário'' que em passado ainda mais remoto
alimentou sistematicamente a \emph{Divina Comédia}, o \emph{Romance da
Rosa}, o \emph{Hypnerotomachia Poliphili}, e muitos outros, a viagem
traz dois aspectos particularmente interessantes para um tipo então novo
de narrativa, cujo objetivo é duplo: o espanto com cenários exóticos e
aventuras ainda não vistas, e o recurso de verossimilhança para ancorar
aquelas explorações. No caso de Behn, o instrumento ficcional é o de
dizer que se falará de uma história \emph{vista}; no de Defoe, em
primeira pessoa, de uma história \emph{vivida}.

O apelo profundo delas é esse, que havia servido já antes a Marco Polo
ou Jean de Mandeville como o interesse vivo em seus relatos e invenções:
na arte nascente do romance, servirá para abrir um novo capítulo na
história das narrativas, um que, logo no século \textsc{xviii}, trará longa lista
de viajantes da nobreza estendendo seus curiosos conhecimentos de coisas
jamais vistas a seus leitores, como o fazia o então
\emph{wunderkabinett}, ou o ``gabinete de curiosidades'', que foi um
modelo inicial de museu, e depois se tornou o modelo que temos hoje,
organizado por alguns princípios científicos que os gabinetes então não
possuíam (é, até certo ponto, interessante notar que as viagens
descritas em romances, diários e relatos, na viva mobilidade das
aventuras, depois viesse a compor a imobilidade de registro instrutivo
dos gabinetes e museus).

Não obstante, essa curiosidade por um mundo que se abria novo desde o
século das chamadas grandes navegações, o \textsc{xvi}, ganhou, no \textsc{xvii} e no
\textsc{xviii}, algumas de suas expressões igualmente novas. Nota"-se que, como
veremos adiante, Michel de Montaigne (1533--1592), no trigésimo"-primeiro
de seus \emph{Ensaios} (1580), aquele intitulado ``Dos Canibais'', já
assinalava um lugar de dignidade para os egressos desse novo mundo, os
índios tupinambás que foram levados à corte francesa a partir da chamada
``França Antártica'' de Villegaignon, isto é, o Rio de Janeiro. Esses
povos demonstravam não apenas absoluto desprezo pelas encarquilhadas
regras de uma sociedade já doente --- como era a da nobreza europeia
---, mas também em seus hábitos, particularmente o da
\emph{antropofagia}, forneciam a lição de ao menos comer os inimigos
quando estivessem mortos, como afirma Montaigne num ataque às violências
de religião. Consideraremos um pouco mais a antropofagia quando falarmos
do papel de Sexta"-Feira nesta obra.

Mas o importante a se reter no momento é que o romance começou a
notoriamente ganhar sua energia como forma, na Europa, a partir dos
relatos de viagens, da forma misturada, e de uma ascensão de classes de
comerciantes, de banqueiros e de gente de empresa que tinha, na viagem e
no impacto do outro (muitas vezes, impacto que significou extremo
preconceito, violência e genocídio) um diferencial em relação ao
estamento: era a oposição entre a imobilidade estamental e a mobilidade
daqueles que adiante fariam a Revolução Francesa, os comerciantes e
banqueiros que tiveram juntamente na Revolução Industrial o impulso
necessário a impor uma \emph{transferência de privilégios}, a do
nascimento nobre para o feroz empreendimento financeiro bem"-sucedido,
pois as assim chamadas \emph{revoluções} nada mais foram do que um
reajuste do lugar do privilégio\footnote{Temos a dúbia felicidade de
  estar em um lugar histórico interessante para observar esse aspecto,
  porque em nosso tempo acontece precisamente o mesmo: assistimos a um
  relocar dos privilégios, com a nova hiperconcentração de renda e os
  truques do poder contra as instituições e os direitos, derrubando"-os
  para remodelá"-los ao novo privilégio, tecnocrata e corporativo.}.

O pai de Defoe era comerciante de gordura e membro destacado da
Companhia Whistful de Açougueiros, e o próprio Defoe foi comerciante de
algodão, vinhos e tecidos (além de coletor de impostos e, é provável em
algum momento, espião), além de dirigir uma fábrica de tijolos: estão
ambos na posição nascente daquilo que viríamos a chamar
\emph{capitalismo}. Nota"-se explicitamente, penso, a conexão que propus
estabelecer acima.

Defoe não era um escritor \emph{per se}, e obviamente viveu antes de se
poder fazer da autoria uma profissão de modo mais geral, o que começa a
acontecer no século \textsc{xviii}, e ainda assim do modo como se sabe que
costuma viver um escritor ou uma escritora; mas mesmo antes havia quem
decidisse que a fonte principal daquilo que faria seria escrever: é
importante notar que Defoe não era um desses.

Não por acaso, sua narrativa tem muito pouco do que não é objetivo, e
seu personagem, também pela verossimilhança do contado, demonstra ser
sobretudo alguém de ação, alguém que põe no \emph{cego movimento} e na
\emph{empresa} o risco mesmo de sua vida, e isso define o personagem de
Crusoe, tomando uma ou outra característica emprestada do próprio Defoe,
e contra os avisos do pai do personagem, que lhe cobre de cautelas e
vaticínios, aconselhando sem sucesso uma vida quieta, como Polônio faz a
Laertes em \emph{Hamlet}.

\section{Naufragar}

Naufragar é fascinante, como --- diria Fernando Pessoa --- navegar é
preciso. Do poema de Leopardi (\emph{e il naufragar m'è dolce in questo
mare}, diz o último verso metafórico de ``L'infinito'') ao
\emph{blockbuster} com Tom Hanks e uma bola de basquete, o naufrágio é
sempre duplo: naufraga"-se efetivamente no mar, e naufraga"-se como
personalidade, ou coletividade, em um momento conotativo, que afunda e
isola. Para Shakespeare foi a hora, o momento mágico, em que um mal do
poder no ducado de Milão começava a ser reparado exemplar e magicamente
em sua peça final, e uma nova consciência emergia nos personagens.

E na base da história que Defoe conta (como acontecera também com
Shakespeare) estava um caso real\footnote{Para Master William foi o
  naufrágio do \emph{Sea Venture}, em 1609, contado por William
  Strachey.}: Defoe conhecia, como toda a Europa, a façanha quase
fantástica do escocês Alexander Selkirk (1676--1721), oficial da marinha
real que passou quatro anos e alguns meses numa ilha isolada do Oceano
Pacífico, após ter sobrevivido a um naufrágio, também com o auxílio do
que pôde resgatar do navio em ruínas. A verdade é que, tendo vivido uma
vida cheia de incidentes (como a Guerra da Sucessão Espanhola), e tendo
iniciado seus dias de mar como pirata, Selkirk sobrevivera a incidentes
mui desafiadores já antes. Reunindo elementos daquela jornada extrema,
Defoe montou por sua vez sua narrativa, tendo ele mesmo alguma
experiência de mar, como o prova \emph{A Tour thro' the Whole}
\emph{Island of Great Britain} (1724--1727), relato de viagens efetivas
suas e de viagens de livro, que escreveu e mandou imprimir.

A aventura de Crusoe, no entanto, se inicia com os sentidos morais das
advertências paterna e materna contra uma vida no mar, que fariam assim
compor o romance como um percurso de \emph{formação}\footnote{\emph{Robinson
  Crusoe} já foi visto como \emph{Bildungsroman}, ou ``romance de
  formação''; na minha opinião, de modo acertado. Se se dá o nascimento
  desse tipo específico de romance em \emph{Wilhelm Meisters Lehrjahre}
  (ou Os Anos de Peregrinação de Wilhelm Meister, 1795--6) de Goethe,
  assim como também se exemplifica depois com \emph{L'Éducation
  Sentimentale (}A Educação Sentimental, 1869) de Flaubert, aquele
  romance dos princípios da forma já trazia nele aspectos fundamentais
  do desenvolvimento transformador do personagem que se observa em obras
  posteriores: Crusoe passa por experiências que o modificam, passo a
  passo, de maneira substancial.}: toda vez que Crusoe entra em um navio
os infortúnios se acumulam, e o infeliz passa de naufrágio a naufrágio,
e à escravidão entre os mouros (de que foge com plano audaz, e levando
consigo um menino, Xury, que se torna seu criado e é mais tarde vendido
a um navio), à vida em um lugar desconhecido (\emph{Brazils}, estes
Brasis)\footnote{Interessante para leitores e leitoras do Brasil o fato
  de que aporta na (\emph{Bay de Todos los Santos, or All Saints Bay})
  Baía de Todos os Santos, e passa por Fernando de Noronha. Contempla,
  descreve e vive o engenho de açúcar: \emph{ingenio, as they call it
  (that is, a plantation and a sugar"-house).}} como senhor de engenho de
açúcar que, por fim, naufraga assim que põe os pés novamente em um navio
quando, desta vez, ia buscar escravos, e nesse naufrágio a habilidade de
Defoe na construção da imagem do desespero humano sob a força das ondas
que o arrastam é um dos pontos altos de sua escrita, assim como a
Providência entrevista num lance de sorte com o surgimento de mudas de
cevada inglesa junto de sua cabana.

Sem detalhar demasiado a narrativa para quem ainda a lerá, é importante
fazer notar que aquele respeito circunspecto, devido à sabedoria paterna
em favor da vida mediana, encontrará em Crusoe jovem apenas o desastre
ético de uma incontrolável vontade, \emph{immoderate}, no texto
original, em que a falta de moderação, de autocontrole é vocabulário
marcado da ética aristotélica, adaptada depois aos usos cristãos.

Para Aristóteles, a incontinência significava equívoco ético por
ignorância, porque significava a incapacidade de se ser senhor de si.
``E o homem incontinente {[}akratés{]}, sabendo que o que faz é mau, o
faz levado pela paixão'', escreve Aristóteles na \emph{Ética a
Nicômaco}\footnote{Aristóteles. \emph{Ética a Nicômaco} (tradução de
  Leonel Vallandro e Gerd Bornheim a partir da versão inglesa de W. D.
  Rosá), São Paulo, Abril Cultural (col. Os Pensadores), 1973, Livro
  \textsc{vii}, parte 1, p.357. Lição que se espalha pela tradição antiga na
  Europa, e é sempre resgatada em textos morais ou de etiqueta. Baltasar
  Gracián (1601--1658) o faria de modo notável no \emph{Oráculo manual y
  arte de prudencia} (1647), por exemplo.}. Crusoe, portanto, não sendo
mau, seria o tipo incontinente, \emph{immoderate}, e assim está
cristãmente exposto a consequências que vão além da ordem deste mundo, e
não é por outro motivo que vê, na ilha, sua própria situação como um
efeito sobrenatural de sua pertinácia, punição divina.

Prestimosos achados chegam a ele ao descobrir seu navio encalhado perto
da praia, de onde sacará muito do necessário a sua extraordinária
sobrevivência, em particular, para o que estamos considerando, duas
Bíblias, que serão mais do que seu único recurso de palavra impressa,
será sua reconexão com o deus cristão, pois nela achará, abrindo o texto
à ventura, palavras de encorajamento numa conversa íntima com a
divindade, uma vez que está perdido para o resto do mundo (com a exceção
de um cachorro, dois gatos, e mais tarde um papagaio, ``Loro'', e
habitantes autóctones e prisioneiros de outras partes próximas, como
veremos).

Deus é um silêncio eloquente, portanto: lido na ordem do mundo como ação
direta sobre suas escolhas humanas, num diálogo pelo mistério da palavra
sagrada, impressa. Crusoe se encontrará infuso nele, de tal modo que o
amor por deus, e a sensação de rejeição por ele, o agitam
permanentemente num vaivém pelo livro.

De modo mais prosaico, mas igualmente fundamental, os presentes da
tragédia pelos quais batalha para extrair do navio e levar à praia são,
como os mantimentos de um navio à época, uma mímica da civilização,
agora de importância vital para a metáfora de seu desastre. Diz ele
nesta tradução de Bruno Gambarotto: ``eu teria antes morrido; e que
teria sobrevivido, caso não morresse, como um completo selvagem''; com
aquele equipamento, e com o uso de seu engenho, o homem Crusoe
duplicará, como puder, os regimes da civilização para não regredir à
barbárie nem perder a vida (seu medo constante).

Leva meses construindo uma cabana reforçada, dividindo os mantimentos em
categorias, atento à preservação de seu mínimo espólio, e estrutura um
calendário para não se perder na sucessão de dias e noites, e para
guardar os dias santos. O personagem é homem de seu tempo, com pouca
crítica que destoe do que era comum à época, um tanto simplório
filosoficamente, e talvez comova pela verossimilhança da exposição de
suas enormes falibilidades e dos anos que leva a reelaborar sua vida.

Incidentalmente, é interessante observar que muito do que Crusoe precisa
reaprender, e sozinho, são coisas de tão elementar conhecimento
acumulado pela humanidade e, ao mesmo tempo, penso que sobretudo para
pessoas de grandes cidades hoje, tão difíceis de saber, porque hoje
contamos com uma rede de serviços a nos abastecer, sem necessidade
alguma, por exemplo, de se saber quando é propício plantar determinadas
sementes, como fazer aduelas para montar um barril que de fato retenha a
água, como reforçar muros ou telhados que resistam à chuva e a outros
imprevistos naturais. E poderíamos multiplicar os exemplos, de modo que
o livro também traz esse peculiar interesse, o de observar, na cuidadosa
descrição de Defoe, os empenhos e engenhos que por vezes custam anos a
Crusoe aperfeiçoar, algo que nossa comunidade humana já não pensa mais
em como regrar e organizar, pela atual divisão dos conhecimentos e do
trabalho.

A tinta, a pena e o papel que vêm com o que pôde retirar do navio, antes
do desaparecimento da nave após uma dura tempestade, auxiliam no
registro de seus dias na ilha, assim como o mantêm aferrado a mais esse
ponto, naquele momento, tão definidor da civilização: a capacidade de
escrever, de registrar o mundo e o pensamento.

Mas a civilização, sobretudo como disposta pela Europa nos séculos
anteriores às Grandes Navegações, encontra em Crusoe --- como em outros
livros e textos célebres do período --- um grande desafio. O grande
desafio é o outro, e o outro, nesse sentido, é algo que provoca abrasão,
é algo que põe, desde lá, uma questão definitiva sobre a centralidade da
experiência de uma cultura. E que ganha desdobramentos fundamentais para
nossa vida adiante.

\section{O diário, Sexta"-Feira e «fogo, em nome de Deus»}

Manter os dias teve também, no livro, outra importância: o famoso
Sexta"-Feira. Quero pensar que nossas atuais sensibilidades se incomodem
com o eurocentrismo do personagem (e o de seu autor), e que se manifesta
sem muito pudor: é dado o fato de que a verdade cristã do homem branco
triunfará com deus e o uso de seu engenho superior, de sua
autodeterminação empresarial. Não obstante, como em todo texto de
talento sobre esses encontros iniciais, espantados e violentos entre
duas culturas, sempre há pontos notáveis onde se pode inferir o que
futuramente levaria o século \textsc{xviii} do Iluminismo francês e voltairiano à
``tolerância'' pelo outro que, de qualquer forma, já por ser nada mais
que tolerado, não passa ainda de inferior, e, para o mundo
contemporâneo, às dificílimas lutas por direitos civis, representação,
igualdade, que ainda hoje vemos, e cujas piores características foram
revisitadas recentemente na crise europeia dos refugiados das guerras
que a própria Europa vem levando a outros lugares. Em todos os casos, há
sempre uma linha já disposta que atravessa os conflitos, criticando"-os.

Assim, quando Crusoe sai a explorar mais o espaço de sua ilha, avista
terra que supõe ser a América, e logo lhe vêm receios de antropófagos.
Mas levaria ainda algum tempo para aquela famosa cena da surpresa diante
da pegada estranha achada na areia da praia, que ocorre na metade do
livro: ``Aconteceu que um dia, por volta do meio"-dia, indo de encontro
ao meu barco, causou"-me enorme espanto ver a estampa de um pé descalço
na praia''.

As longas preparações que o pavor lhe obriga são de toda sorte antes
mesmo do efetivo contato, o que se aprofunda quando os traços da
presença se aprofundam e encontra crânios e restos humanos em uma praia,
já tendo vivido muitos anos solitário em sua ilha. E mesmo esse encontro
fantasmal se construirá em anos. Quando, ao fim de longos vinte anos na
ilha --- e alguns da descoberta da pegada --- decide atacar os
``selvagens'' de modo preventivo, os pensamentos de violência que
cultivara em seu pavor cedem ao estabelecimento provisório de proporções
um pouco à moda das reflexões que já achávamos no ensaio \textsc{xxxi} do livro \textsc{i}
dos \emph{Ensaios} de Montaigne, o ``Dos canibais'', como assinalei
acima.

Escreve Montaigne: ``não vejo nada bárbaro ou selvagem no que dizem
daqueles povos; e, na verdade, cada qual considera bárbaro o que não se
pratica em sua terra''\footnote{Montaigne, Michel de. \emph{Os Ensaios}
  (tradução de Sérgio Milliet), São Paulo, Abril Cultural (col. Os
  Pensadores), 1972, Livro \textsc{i}, \textsc{xxxi}, p.105.}; e Crusoe:

\begin{quote}
Tendo ponderado brevemente sobre essas coisas, seguiu"-se necessariamente
que sobre elas eu estava errado; que essas gentes não eram assassinas,
no sentido que antes as havia condenado em meus pensamentos; não mais do
que os cristãos eram assassinos, os quais frequentemente matam os
prisioneiros feitos em batalha; ou de forma ainda mais contumaz, em
muitas circunstâncias, passam à espada batalhões inteiros de homens sem
oferecer misericórdia, ainda que estes tenham deposto as armas e se
rendido.
\end{quote}

E, se não tem a veemência ou o refinamento do pensamento de Montaigne,
igualmente não é mais algo como o primeiro movimento de chamar
``bárbaro'' a quem não seja do mesmo costume. Em parte, proporia que a
verossimilhança de Crusoe se estabelece pelo fato de que, tendo vivido
vinte anos a sós na ilha, obrigado por questão de sobrevivência a
retornar a práticas diretas do contato com a natureza sem recursos do
conforto europeu, Crusoe, mesmo que fervorosamente convertido por seus
únicos livro e deus --- e mesmo por isso ---, pondera o sentido do
costume, quando então seus estranhos ``inimigos'' deixam de ser da arte
do diabo e se tornam, como ele diria adiante, ``inocentes''\footnote{O
  que é feito com um objetivo também claro, e britanicamente político:
  verberar em comparação os espanhóis como povo sem piedade, desumano,
  por seu massacre dos povos nativos americanos.}.

Voltarão, como seus receios frequentes, a se tornar inimigos, até que,
ainda outros anos depois, determina fazer de um ou mais seus servos, o
que até eventualmente lhe permitiria deixar a ilha (após mais de 25
anos). Um incidente lhe traz aquele que chamaria \emph{Sexta"-Feira}, por
motivo do dia em que se conheceram. A descrição do que lhe era agradável
nesse personagem denuncia que as características físicas de pele, como
de forma, de índios e negros, eram incômodas a Crusoe (como a Defoe e ao
costume europeu): Sexta"-Feira tem uma pele de cor diversa daquela que
chama ``nauseabunda'' dos brasileiros nativos, e o nariz não é achatado.
O trecho inicial dos entendimentos entre Crusoe e o nativo também
interessa a esse ponto:

\begin{quote}
em primeiro lugar o fiz aprender que seu nome seria Sexta"-Feira, que foi
o dia em que lhe salvei a vida; e assim o fiz para conservar a memória
do tempo; e ensinei"-o a dizer Senhor, e fi"-lo aprender que aquele seria
o meu nome (\ldots{})\footnote{No original: \emph{I likewise taught him to
  say Master, and then let him know that was to be my name}.}
\end{quote}

E é descrito como criado fiel, amoroso a ponto de se sacrificar pelo
amo. A passagem é bastante significativa, em especial se se relembra
quão fácil se dispôs Crusoe a ir buscar escravos para os engenhos da
Bahia, ou como apreendeu e vendeu seu criado mouro, o rapaz Xury: os
dois casos exemplificam que tanto o despeito pela humanidade alheia,
como a tolerância que aceita o ``inferior'' desde que obediente e
convertível a cristão, são exemplos daquilo que se chama
\emph{colonialismo}, para o que a superioridade indiscutível de raça,
costumes, cultura, religião ( a ``verdadeira'', do ``verdadeiro Deus'')
e predestinação era, como por vezes ainda é, o argumento que permite um
império e justifica submissão, o que se sublinha com a proposição de
Crusoe de ser o ``rei'' de sua ilha, quando chegam poucos outros
``súditos'' a quem permite liberdade de consciência --- havia entre eles
um papista, um espanhol: tolerado, mas anônimo, Crusoe o chama sempre
por seu gentílico.

Tão claro é o peso do livro de Defoe que o grande poeta caribenho, Derek
Walcott (1930--2017), escreveu poemas lidando com o personagem Crusoe.
Nascido na ilha de Santa Lúcia, nas Índias Ocidentais, Walcott se
celebrizou em particular por seu poema épico \emph{Omeros} (1990), que
lhe impulsionou ao Nobel. Andrea Mateus, professora da \textsc{uft}, escreveu em
2007 um artigo chamado ``A figura de Crusoe na poesia de Derek
Walcott'', no qual percebe e registra um jogo de engenho notável em seu
poema ``Crusoe's Journal'', um ajuste de contas poético: mesclado das
várias vozes que vão compondo elipticamente o relato do livro, em
determinado momento se ouve no poema como que a voz de Sexta"-Feira
dizendo que, como um Cristóvão (\emph{Colombo}, mas também aquele que
etimologicamente \emph{porta o Cristo}), esse novo Adão transformava os
nativos \emph{into good Fridays}, ``em dóceis Sexta"-Feira'', que lhe
recitavam louvores como papagaios, mas que mudavam para si mesmos a
língua aprendida.

O poema, que vinha regido pelo ritmo mais comum à língua inglesa, o
jambo, alterna"-se para o ritmo oposto, o troqueu, assinalando, assim, na
própria tessitura do verso, a modificação fundamental do ritmo na
linguagem absorvida. Crusoe, que impusera um nome ao nativo e
ensinara"-lhe sua língua sem aprender a daquele outro, e que também ao
espanhol não ofereceu a dignidade de ter nome em seu diário, tornou"-se
uma das figuras determinantes, às quais os países colonizados se dirigem
quando interpelam os registros do colonizador e reconstituem sua
história e seu ângulo dos fatos.

\emph{Robinson Crusoe} é, portanto, um livro direto, lido até mesmo em
adaptações infanto"-juvenis de coleções de aventura, e um livro muito
complexo. Seus ricos detalhes objetivos não nos deixarão esquecer que
também influenciou Herman Melville na escrita da obra"-prima que é
\emph{Moby Dick} (1851), e seus contornos históricos, até hoje, como se
vê acima, propõem discussão vigorosa.

\medskip

\hfill{}Boa leitura.
