\SVN $Id: APENDICE.tex 10653 2012-02-23 16:07:37Z oliveira $

\part[Apêndice: Três artigos\\ sobre o ódio racial]{Apêndice:\break Três artigos sobre\break o ódio racial}

\chapter{Nota preliminar}
\hedramarkboth{Nota Preliminar}{Marcus Vinicius Mazzari}


{\itshape
O chamado “caso de Damasco”, ensejo imediato para uma série de artigos
de Heine publicados no \emph{Augsburger Allgemeine Zeitung}, consiste
numa acusação de assassinato ritualístico levantada em 1840 contra a
comunidade judaica da capital síria. Durante vários meses, o caso
ocupou a opinião pública internacional, levando a complexas negociações
diplomáticas entre potências europeias, sobretudo a Áustria de
Metternich e a França (governada então por Louis"-Adolphe Thiers), e o
Império Otomano e seus representantes no Oriente Médio.

Já em sua época, o assunto não pôde ser inteiramente esclarecido, e
várias contradições impossibilitam ainda hoje uma reconstituição mais
precisa de seus detalhes e desdobramentos. No dia cinco de fevereiro de
1840, o frade capuchinho Tommaso da Sardegna desaparece
misteriosamente de Damasco, onde vivia num albergue francês desde o ano
de 1806. Poucos dias antes, algumas pessoas (entre as quais um
comerciante turco) presenciaram, numa praça central da cidade, um
violento desentendimento entre o frade e um vendedor de mulas
muçulmano, durante o qual aquele amaldiçoara o Islã e o vendedor
gritara sob extrema exaltação: “Esse cachorro cristão vai morrer pelas
minhas mãos”.

Apesar disso, o cônsul francês em Damasco, Ratti"-Menton, que recebe a
incumbência de apurar o caso, concentra as investigações no bairro
judeu, onde o frade teria sido visto pela última vez. A primeira vítima
dos acontecimentos é um jovem judeu, açoitado até a morte pela
população açulada após afirmar ter visto o frade, pouco antes do
desaparecimento, em outra parte da cidade. Vários judeus são detidos e
levados à prisão local assim como ao próprio consulado francês, onde
são torturados de maneira brutal. Alguns resistem à tortura, mas outros
(como um idoso de oitenta anos) sucumbem; ainda outros convertem"-se
ao islamismo ou fazem confissões disparatadas, que acabam levando a
novas prisões e torturas. Um dos torturados declara que o receptor do
sangue do frade teria sido o principal rabino de Damasco, Jacob Antini,
que é imediatamente preso e torturado. Não obtendo deste nenhuma
confissão, as autoridades muçulmanas, manipuladas pelo cônsul francês,
prendem um outro rabino, Moses Abu Afie, obrigando"-o a converter"-se
ao islamismo.

Em etapa posterior, as investigações voltam"-se às circunstâncias do
desaparecimento do criado do frade Tommaso. Um muçulmano de nome
Murad"-el"-Fallat, que trabalhara para um dos judeus incriminados,
afirma ter visto o criado (amarrado e amordaçado) na casa de outro
conceituado judeu damasceno, Meir Farchi, e cita como testemunha, pois
também estaria presente na casa, o cidadão austríaco Isaac Piccioto,
que é detido e interrogado. Piccioto, porém, apresenta um álibi
consistente, confirmado por um cidadão inglês e pelo próprio cônsul da
Inglaterra. Murad"-el"-Fallat muda então o seu depoimento e cita
outros nomes de judeus. Desencadeia"-se nova onda de prisões e
recrudesce a perseguição à população judaica de Damasco, que contava na
época cerca de trinta mil pessoas. A perseguição espalha"-se para
outras cidades, como Beirute e Esmirna, reforçada pela acusação (o
chamado “libelo de sangue”) de que os judeus, segundo a orientação de
seus escritos religiosos, necessitam de sangue cristão para a
preparação dos pães ázimos do Pessach. O cônsul francês providencia
traduções para o árabe de velhos panfletos cristãos sobre esse suposto
ritual e ordena a sua distribuição entre a população muçulmana.

Essa espiral de violência é, porém, quebrada com a corajosa intervenção
do cônsul austríaco em Damasco, Anton Laurin, que protege de modo
consequente o seu conterrâneo Isaac Piccioto e envia ao chanceler
Metternich relatos minuciosos sobre o “horripilante interrogatório de
alguns israelitas de Damasco suspeitos do assassinato do Padre Tommaso
da Sardegna e de seu criado”. Desenvolve também várias gestões junto ao
paxá e vice"-governador do Egito, Mehemet Ali, o que tem por
consequência a decretação do fim das torturas e a libertação de todos
os acusados. Mas outros detalhes da intervenção do cônsul Laurin, assim
como do engajamento igualmente corajoso e humanitário do advogado
francês Isaac Adolphe Crémieux, são apresentados nos artigos de Heine.
}

\hfill Do tradutor

\chapter{Paris, 7 de maio de 1840}

\textsc{Os jornais parisienses} de hoje trazem um relato que o cônsul austríaco
em Damasco enviou ao cônsul"-geral da Áustria em Alexandria a respeito
dos judeus damascenos, cujo martírio faz lembrar os tempos mais
tenebrosos da Idade Média. Enquanto nós, aqui na Europa, tomamos tais
contos da carochinha sobre os judeus apenas como assunto poético e nos
deleitamos com aquelas sagas horripilantemente ingênuas, com as quais
os nossos antepassados não se angustiavam pouco;\footnote{ O próprio
fragmento\textit{ O Rabi de Bacherach} constitui um exemplo de
elaboração poética desses acontecimentos --- “sagas” e “contos da
carochinha” (\textit{Märchen}) sobre judeus --- que Heine localizava
apenas em séculos remotos e, subitamente, via apresentarem"-se como
realidade contemporânea no relato do cônsul austríaco Caspar Giovanni
Merlato (1798---1882). Em maio de 1840, todavia, o \textit{Rabi} era
texto ainda inédito, mas o leitor deste artigo podia pensar em algumas
narrativas históricas do romantismo europeu, como o romance
\textit{Ivanhoé }(1820), de Walter Scott.} enquanto aqui, entre
nós, tão somente em poemas e romances se fala daquelas feiticeiras,
daqueles lobisomens e judeus que necessitam do sangue de piedosas
crianças cristãs para suas cerimônias satânicas; enquanto rimos e nos
esquecemos disso tudo, no oriente as pessoas começam a recordar"-se de
maneira muito aflitiva da velha superstição e a exibir fisionomias
demasiado sérias, fisionomias de ira sombria e de exasperado e mórbido
tormento! Enquanto isso, o carrasco tortura e no banco de suplício o
judeu confessa que, com a aproximação da festa do \textit{Pessach}, ele
precisa de sangue cristão para embeber os seus secos pães pascoais e,
para essa finalidade, teria imolado um velho capuchinho. O turco é
desatinado e despiciendo, colocando com prazer os seus bastões e
instrumentos de tortura à disposição dos cristãos, contra os judeus
acusados.\footnote{ Ensejado pelas lutas do povo grego contra a
dominação otomana (entre 1821 e 1829), Heine apresenta por vezes uma
imagem bastante negativa do “turco”.} Pois ambas as seitas lhe são
igualmente odiosas, considera a ambas como cães, chegando mesmo a
conferir"-lhes esse honroso nome; certamente fica contente quando o
\textit{Giaur} cristão lhe dá oportunidade de, com alguma aparência de
legalidade, maltratar o\textit{ Giaur} judeu.\footnote{ Giaur é a
designação turca para os não"-muçulmanos. A etimologia a dá como
derivada do termo árabe \textit{kafir}, “infiel”.} Esperai até que
o paxá fique por cima e não precise mais temer a influência armada do
europeu --- então ele prestará ouvidos ao cão circuncidado e este irá
acusar nossos irmãos cristãos, Deus sabe do quê! Hoje bigorna, amanhã
malho! --- 

 Mas para o aliado da humanidade uma coisa dessas será sempre uma
estocada no coração. Manifestações desse tipo são uma desgraça,
imprevisíveis são as suas consequências. O fanatismo é um mal
contagiante, que se dissemina sob as mais diversas formas e por fim se
desencadeia contra todos. O cônsul francês em Damasco, conde
Ratti"-Menton, tornou"-se responsável por coisas que provocam aqui um
grito generalizado de horror.\footnote{ Benoît"-Laurent Comte de
Ratti"-Menton, nascido em 1799, ingressou já em 1822 no serviço
diplomático francês. O seu primeiro posto é em Gênova, onde trabalha
por um ano (1824) como “\textit{élève"-vice"-consul}”. Ao longo de
sua carreira --- nas décadas subseqüentes Ratti"-Menton tem postos em
Nápoles, na Sicília, Grécia, Rússia, Síria, China, Índia e, por fim, no
Peru --- envolve"-se várias vezes em escândalos financeiros e políticos.
Segundo os “Arquivos do Ministério de Assuntos Estrangeiros”
(\textit{Archives du Ministière des Affaires Etrangères}),
aposenta"-se em 1861, mas não há informação sobre a data de sua morte.} 
Foi ele que inoculou no oriente a superstição ocidental e
distribuiu entre o populacho de Damasco um panfleto em que se imputa
aos judeus o assassinato de cristãos. Esse panfleto transbordante de
ódio, que o conde Menton recebeu de seus amigos ideológicos com a
orientação de difundi"-lo, foi extraído originalmente da
\textit{Bibliotheca prompta a Lucio Ferrario,}\footnote{ Trata"-se da
obra em oito volumes \textit{Prompta Bibliotheca Canonica, Juridica,
Moralis, Theologica, Ascetica, Polemica, Rubristica, Historica},
publicada em 1706 por Lucius Ferrari, provincial da ordem dos
franciscanos e conselheiro da Inquisição. A obra compendiava centenas
de resoluções e decisões da Igreja e apresentava também inúmeros
acontecimentos de cunho religioso. No verbete “Hebraeus” trazia, entre
outras histórias, a suposta tentativa empreendida sem sucesso por um
judeu em Brandenburgo de perfurar uma hóstia consagrada, o que
provocou, junto com outras acusações, a morte na fogueira de quarenta
judeus.} em que se afirma com toda determinação que os judeus
necessitam de sangue cristão para a solenidade da festa do
\textit{Pessach}. O nobre conde resguardou"-se de repetir a saga
medieval vinculada a essa afirmação, isto é, que os judeus roubam
hóstias consagradas para a mesma finalidade e as perfuram com agulhas
até que o sangue comece a escorrer --- uma infâmia que na Idade Média
vinha à luz do dia não apenas mediante testemunhas juradas, mas também
pelo fato de que uma auréola brilhante se expandia sobre a casa judia
em que uma dessas hóstias roubadas era crucificada. Não, os infiéis, os
maometanos jamais teriam acreditado em algo semelhante e assim o conde
teve de buscar refúgio, no interesse de sua missão, em histórias menos
mirabolantes. Estou dizendo no interesse de sua missão, e entrego essas
palavras à mais ampla reflexão. O senhor conde está apenas há pouco
tempo em Damasco; seis meses atrás, ele era visto aqui em Paris, a
forja de todas as confraternizações progressivas, mas também de todas
as retrógradas. --- O atual ministro dos assuntos exteriores, o senhor
Thiers, que recentemente procurou impor"-se não apenas como homem da
humanidade, mas até mesmo como um filho da Revolução, dá mostras no
tocante aos acontecimentos de Damasco de uma estranha
complacência.\footnote{ Desde a publicação de sua \textit{História da
Revolução Francesa} (\textit{Histoire de la Révolution Française}, 1823
--- 27), Louis"-Adolphe Thiers (1797---1877) procurou estilizar"-se
como “filho e herdeiro” da Revolução. Heine reporta"-se às atas de um
discurso que Thiers pronunciou na câmara em 24 de março de 1840: “Não
vos esqueçais, senhores, vós representais uma revolução [\ldots]. É
necessário amá"-la, respeitá"-la, acreditar na legitimidade de sua
meta, em sua perseverança nobre, em sua força invencível, para
representá"-la com dignidade, com confiança. De minha parte, senhores,
sou filho dessa revolução, sou o mais humilde de seus filhos\ldots\ (Risos
e murmúrios.)” \textit{Ne oubliez pas, Messieurs, vous représentez
une révolution }[\ldots].\textit{ Il faut l’aimer, la respecter, croire à
la légitimité de son but, à sa noble persévérance, à sa force
invencible, pour la représenter avec dignité, avec confiance. Pour moi,
Messieurs, je suis enfant de cette révolution, je suis le plus humble
de ses enfants}\ldots\ (Rires et Murmures.)} De acordo com a edição
de hoje do “Moniteur” um \textit{vice"-cônsul} já deve ter partido
para Damasco a fim de investigar o comportamento do \textit{cônsul}
francês nessa cidade.\footnote{ O \textit{Moniteur Universel} de sete
de maio de 1840 trazia essa informação reportada por Heine: “O governo
acaba de enviar um vice"-cônsul a Damasco com a missão de colher
informações sobre o assassinato do padre Tomás e tudo o que se vincula
a esse infeliz acontecimento”. \textit{Le Gouvernement vient
d’envoyer un vice"-consul à Damas, avec la mission de prendre des
informations sur l’assassinat du père Thomas, et sur tout ce qui
rattache à ce malheureux évènement}. } Um vice"-cônsul! Certamente
um funcionário subordinado proveniente de uma paisagem ideológica
vizinha, sem nome e sem o respaldo de independência apartidária!


\chapter{Paris, 27 de maio de 1840}

\textsc{A respeito} da sangrenta questão de Damasco, jornais do norte da
Alemanha ofereceram diversas informações, as quais --- datadas em parte
de Paris, em parte de Leipzig, mas todas provenientes da mesma pena e
elaboradas no interesse de uma certa \textit{clique} --- têm o objetivo
de confundir a opinião pública alemã.\footnote{ Heine alude a artigos
do jornalista Richard Otto Spazier (1803---1854), que escrevia como
correspondente em Paris para jornais alemães, em especial o Jornal de
Leipzig (\textit{Leipziger Allgemeine Zeitung}). Nos artigos em questão
Spazier reproduzia habilmente insinuações antissemitas que circulavam
em Paris; também defendia a postura passiva de Thiers em relação aos
acontecimentos de Damasco.} Não vamos enfocar aqui a
personalidade e nem os motivos daquele que as redigiu e
abstenhamo"-nos também de toda investigação concernente aos
acontecimentos de Damasco; vamos nos permitir apenas algumas
observações corretivas quanto àquilo que, em relação a esses mesmos
acontecimentos, foi dito sobre a imprensa e os judeus parisienses. Mas
também no âmbito dessa tarefa, o que nos move é mais o interesse da
verdade do que o das pessoas; e no que diz respeito aos judeus locais,
é bem possível que o nosso testemunho fale antes contra eles do que a
seu favor. --- Pois, verdadeiramente, nós iríamos antes enaltecer do que
repreender os judeus de Paris se eles, como anunciaram os mencionados
jornais do norte da Alemanha, tivessem demonstrado tão grande zelo em
prol dos seus desgraçados irmãos de fé em Damasco e, desse modo, não
tivessem medido esforços financeiros para salvar a honra de sua
religião difamada. Mas não é esse o caso. Os judeus na França estão
emancipados há um tempo já demasiado longo para que os vínculos étnicos
não se mostrassem agora bastante frouxos;\footnote{ A emancipação dos
judeus franceses data já de 1791, quando adquirem os direitos civis dos
demais cidadãos. As constituições de 1814 e 1830 garantiram a rabinos
remuneração do Estado francês.} esses vínculos submergiram quase
que por completo --- ou, melhor dito --- emergiram e se dissolveram na
nacionalidade francesa. Eles são exatamente tão franceses como todos os
demais e, portanto, também são acometidos por veleidades de entusiasmo
que duram 24 horas e, quando o sol está quente, até mesmo
três dias!\footnote{ Alusão à revolução parisiense de julho de 1830,
ocorrida entre os dias 27 e 29. Também em outros textos, Heine sugere
que o entusiasmo revolucionário dos franceses dependia diretamente do
tempo.} --- E isso é válido para os melhores. Muitos entre eles
ainda praticam o cerimonial judaico, o culto exterior, mas
mecanicamente, sem saber por que o fazem, apenas por costume antigo;
não há o menor vestígio de crença íntima, pois na sinagoga, tanto
quanto na igreja cristã, o ácido espirituoso da crítica voltairiana
atuou de maneira devastadora. Entre os judeus franceses, o ouro é, como
entre os demais franceses, o deus do dia e a indústria é a religião
dominante. Nesse sentido, os judeus locais poderiam ser divididos em
duas seitas: na seita da \textit{rive droite} e na seita da
\textit{rive gauche}. É que esses nomes se referem às duas linhas
ferroviárias que, uma ao longo da margem direita do Sena e a outra
estendendo"-se pela margem esquerda, levam a Versalhes, sendo
dirigidas por dois famosos rabinos das finanças, os quais divergem e
querelam entre si do mesmo modo como outrora Rabi Samai e Rabi Hillel
na antiga Babilônia.\footnote{ Os dois rabinos mencionados --- Samai (ou
Shammai), e Hillel, o Velho (ou Hillel, o Babilônico) --- viveram por
volta do início da era cristã. Figuras de alto relevo na tradição
judaica, Samai praticava uma exegese mais rigorosa da lei judaica,
enquanto o seu oponente Hillel assumia posições moderadas, tendo também
introduzido procedimentos hermenêuticos na interpretação do Antigo
Testamento. Quanto às duas linhas ferroviárias mencionadas por
Heine, a da margem direita do Sena, financiada pelo barão Rothschild,
foi inaugurada a dois de agosto de 1838. A da margem esquerda só entrou
em operação em 1841, mas Benoit Fould já havia vencido a concorrência
para a sua construção em 1837.}

 Temos de fazer justiça ao grão"-rabino da \textit{rive droite}, o
barão Rothschild, e declarar que ele deu mostra de uma simpatia mais
nobre pela casa de Israel do que o seu antagonista versado nas
escrituras, o grão"-rabino da \textit{rive gauche}, o senhor Benoit
Fould, o qual, enquanto os seus irmãos de fé eram torturados e
estrangulados na Síria sob incitamento de um cônsul francês, ostentava
a inabalável paz de espírito de um Hillel ao pronunciar alguns belos
discursos na câmara dos deputados franceses sobre a conversão dos
proventos e a taxa de desconto bancário.\footnote{ Esse juízo negativo
sobre o financista e político Benoit Fould, cuja esposa tinha vínculos
de parentesco com Heine, será abrandado num artigo posterior, datado de
três de junho de 1840. Sob o ensejo de um discurso pronunciado por
Fould na câmara dos deputados de Paris, Heine escreve então que a
“interpelação do senhor Benoit Fould dá testemunho de grande
inteligência e dignidade”.}

 O interesse que os judeus daqui demonstraram pela tragédia de Damasco
se reduz a manifestações bastante irrelevantes. O consistório
israelita reuniu"-se e deliberou à maneira morna de todas as
corporações; o único resultado dessas deliberações foi a opinião de que
as atas do processo devem ser apresentadas à opinião pública.\footnote{
Esse \textit{Consistoire central israélite de France}
foi criado em 1808 por Napoleão Bonaparte e tinha por finalidade
oferecer garantia estatal às atividades relacionadas ao culto judaico
na França. } O senhor Cremieux, o famoso advogado que a todo tempo
dedica sua generosa eloquência não apenas aos judeus, mas também aos
oprimidos de todas as confissões e de todas as doutrinas,
encarregou"-se da publicação acima mencionada; e, com exceção de uma
bela mulher e de alguns jovens eruditos, o senhor Cremieux é certamente
o único em Paris que se incumbiu efetivamente da causa de
Israel.\footnote{ A “bela mulher” a que se refere Heine é Betty
Rothschild, esposa do barão Rothschild. Isaac Adolphe Crémieux
(1796---1880), com quem Heine trava contato pessoal pouco depois da
publicação deste artigo, tornou"-se conhecido na França já em 1819, ao
defender com sucesso três jovens acusados de cantar publicamente, em
pleno período da Restauração, a \textit{Marseillaise}, o hino da
Revolução. Após o seu corajoso engajamento no caso de Damasco, resumido
por Heine, Crémieux continua dedicando"-se a causas liberais e
humanitárias. Quando em 1860 os cristãos são perseguidos no Líbano,
Crémieux propõe a fundação de um comitê judeu de ajuda humanitária para
perseguidos de todas as religiões e de todas as nacionalidades.
Defendeu também os militantes socialistas Louis Blanc e Pierre Leroux
em processos políticos.} Sob sacrifício extremo de seus interesses
pessoais, desprezando toda perfídia insidiosa, ele se contrapôs com
destemor às insinuações detestáveis e até mesmo se ofereceu a viajar
para o Egito, caso o processo dos judeus damascenos venha a ser levado
nesse país diante do tribunal do paxá Mehemet Ali.\footnote{ Mehemet
(ou Mohammed) Ali (1769---1849), vice"-governador do Egito de 1805 até
1848. Graças à sua participação oportunista nas guerras de libertação
do povo grego (primeiro aliado, depois adversário dos turcos),
conquistou a Síria, que se encontrava sob domínio otomano, em 1833.
Mehemet Ali é o fundador da dinastia que governou o Egito até 1953.}

 O autor pouco fidedigno dos relatos nas mencionadas folhas do norte da
Alemanha, também do “Jornal de Leipzig” [\textit{Leipziger Allgemeine
Zeitung}], insinua com uma pérfida observação lateral que o senhor
Cremieux publicou como anúncio, desembolsando o valor correspondente, a
réplica com que ele soube neutralizar os falsos relatos das missões
cristãs publicados nos jornais parisienses. Sabemos de fonte segura que
as redações dos jornais se declararam prontas a publicar aquela
réplica, sem cobrar taxa alguma, caso se quisesse esperar alguns dias;
e somente diante da exigência de impressão a mais rápida possível,
algumas redações calcularam os custos de uma edição suplementar ---
custos que, na verdade, não são de grande monta em se considerando o
poder financeiro do consistório israelita. O poder financeiro dos
judeus é de fato imenso, mas a experiência ensina que bem maior ainda é
a sua avareza. Um dos membros mais bem conceituados do dito consistório
--- ele é conceituado, a saber, em cerca de trinta milhões de francos ---,
o senhor W.~de Romilly, não daria talvez nem cem francos se fosse
solicitado a contribuir com uma coleta para a salvação de toda a sua
estirpe.\footnote{ O banqueiro Worms de Romilly, presidente do
consistório israelita entre 1824 e 1843, era considerado especialmente
avaro e implacável com os seus devedores.} Trata"-se de uma
invenção antiga, deplorável, mas ainda não desgastada, quando se
atribuem àquele que ergue a sua voz em defesa dos judeus os motivos
financeiros mais espúrios; estou convencido de que Israel jamais soltou
dinheiro sem que os seus dentes fossem arrancados com violência, como
no tempo dos Valois.\footnote{ A situação dos judeus na França durante
o século \textsc{xiv} caracteriza"-se por uma alternância de expulsões e
repatriações, sendo que a autorização para permanecerem no país sempre
esteve ligada a interesses econômicos. Expulsos da França em 1306, os
judeus puderam retornar em 1315; novamente banidos em 1323, foram
chamados de volta em 1359, sob o domínio da dinastia dos Valois, que se
estendeu de 1328 a 1589. No governo de Carlos \textsc{vi} (1380---1422), dessa
mesma dinastia, foram definitivamente expulsos em 1394.} Não faz
muito tempo, enquanto folheava a \textit{Histoire des Juifs} de
Basnage,\footnote{ Essa obra em quinze volumes de Jacques Basnage,
\textit{Histoire des Juifs depuis Jésus"-Christ jusqu’à présent}, é
também uma das fontes utilizadas por Heine para a redação do
\textit{Rabi de Bacherach}. Foi publicada na Holanda entre 1706 e 1716.} 
tive de rir do fundo do coração em virtude da ingenuidade com
que o autor, acusado pelos seus adversários de ter recebido dinheiro
dos judeus, defendia"-se de tais acusações; acredito em cada uma de
suas palavras quando ele acrescenta melancolicamente: “\textit{Le
peuple juif est le peuple le plus ingrat qu’ il y ait au
monde!}”\footnote{ “O povo judeu é o povo mais ingrato que há no
mundo.”} É claro que aqui e ali há exemplos de que a vaidade soube
abrir os obstinados bolsos dos judeus, mas nesses casos a sua
liberalidade se mostrou mais repulsiva do que a sua sovinice. Um antigo
fornecedor prussiano que, aludindo ao seu nome hebreu Moses (é que
Moses significa em alemão “tirado da água”, em italiano “\textit{del
mare}”), assumiu o nome mais sonoro, correspondente à forma italiana,
de um barão Delmar --- este prussiano fundou aqui há algum tempo um
instituto educacional para jovens aristocratas empobrecidas,
disponibilizando para essa finalidade mais de um milhão e meio de
francos; uma ação nobre, que no Faubourg Saint"-Germain lhe foi levada
em tão alta conta que lá mesmo as \textit{douairières} de orgulho mais
antigo e as senhoritas mais impertinentemente jovens já não troçam dele
de maneira ostensiva.\footnote{ \textit{Douairière} significa uma viúva
de alta posição social. O financista e barão Delmar --- Ferdinand Moritz
Levy Baron von Delmar (1782---1858) --- reagiu com várias ameaças à
menção de seu nome neste artigo de Heine.} Será que esse nobre da
estirpe de Davi contribuiu com apenas um centavo para uma arrecadação
em prol dos interesses dos judeus? De minha parte, posso garantir que
um outro barão tirado da água, que no nobre Faubourg faz as vezes de
\textit{gentilhomme catholique} e de grande escritor, não atuou pelos
companheiros de linhagem nem com o seu dinheiro nem com a sua
pena.\footnote{ Heine alude aqui a Ferdinand de Eckstein, que gozou de
grande prestígio na corte de Luis \textsc{xviii} (1814 a 1824) e que assumiu em
1826 a direção da revista \textit{Le Catholique}.} Nesse ponto eu
tenho de fazer uma observação, talvez a mais amarga de todas. Entre os
judeus convertidos há muitos que, por hipocrisia covarde, discursam
sobre Israel com maledicência ainda maior do que a de seus inimigos
natos. Desse mesmo modo, certos escritores costumam, para não lembrar a
própria origem, ou falar muito mal dos judeus ou não falar
absolutamente nada. É esse um fenômeno conhecido, deploravelmente
ridículo. Mas pode ser útil chamar especialmente agora a atenção do
público para esse fato, uma vez que não apenas nas mencionadas folhas
do norte da Alemanha, mas também num jornal muito mais importante se
pôde ler a insinuação de que tudo o que foi escrito em favor dos 
judeus damascenos teria fluido de fontes judaicas, como se o cônsul
austríaco em Damasco fosse judeu, como se todos os demais cônsules
nessa cidade, à exceção do francês, fossem puros judeus.\footnote{ Num
artigo publicado no Jornal de Leipzig em maio de 1840, Spazier escreve
tratar"-se de um fato comprovado que “com exceção do cônsul francês em
Damasco, acusado com tanta veemência de parcialidade, o senhor Merlato,
assim como quase todos os demais cônsules das potências européias na
Síria --- da Rússia, Dinamarca, Prússia etc. --- são igualmente 
israelitas”. No entanto, em artigo datado de quatro de junho Spazier
admite que “o senhor Merlato não é israelita, mas sim, como o senhor
Laurin, cristão e funcionário austríaco”.} Conhecemos essa
tática, nós já a experimentamos por ocasião da Jovem
Alemanha.\footnote{ A expressão “Jovem Alemanha” (\textit{Junges
Deutschland}) designa um grupo de literatos e publicistas que se
caracterizavam por posições bastante críticas à política alemã
contemporânea. O principal oponente da “Jovem Alemanha” foi Wolfgang
Menzel (1798---1873), contra quem Heine publicara em 1837 o livro
\textit{Sobre o denunciante}. Heine era o único membro desse movimento
com origem judaica, mas mesmo assim Menzel escreveu em outubro de 1835
que era necessário passar a designar a “Jovem Alemanha” de “Jovem
Palestina”.} Não, todos os cônsules de Damasco são cristãos, e
que o cônsul austríaco nessa cidade nem sequer tenha qualquer origem
judaica, isso nos garante exatamente a maneira franca e destemida com
que ele assumiu a proteção dos judeus perante o cônsul francês; --- e o
que este último é, isto o tempo irá mostrar.


\chapter{Paris, 25 de julho de 1840}

\textsc{Nos teatros} dos \textit{boulevards} parisienses a história de Bürger, o
poeta alemão, está sendo apresentada agora como tragédia. Nós vemos
então como ele, escrevendo a “Leonore” sob o luar, vai cantando:
\textit{Hurra! les morts vont vite --- mon amour, crains"-tu les
morts?}\footnote{ No dia 11 de julho de 1840 estreou em Paris, no
\textit{Gymnase Dramatique}, a peça \textit{Lénore}, de
Jean"-August"-Jules Loiseleur (1816---1900), que mistura o assunto da
célebre balada “Lenore” (1773) com a biografia de seu autor, o poeta
Gottfried August Bürger (1747---1794). O refrão citado diz: “Hurra! os
mortos vão rápido --- meu amor, tens medo dos mortos?” (No original
alemão: \textit{Hurra! die Toten reiten schnell! / Graut Liebchen auch
vor Toten}?)} Trata"-se efetivamente de um bom refrão e vamos
antepô"-lo ao nosso artigo de hoje, relacionando"-o de forma a mais
íntima com o ministério francês. --- De longe o cadáver do gigante de
Santa Helena vem marchando e se aproximando cada vez mais
ameaçadoramente, em alguns dias abrem"-se os túmulos também aqui em
Paris e os ossos insatisfeitos dos heróis de julho vêm para fora e
dirigem"-se para a praça da Bastilha, aquele lugar aterrorizante em
que os fantasmas de \textit{anno} de 89 continuam a rondar\ldots\footnote{
Em julho de 1840, François"-Ferdinand d{\textquotesingle}Orléans,
príncipe de Joinville, parte para a ilha de Santa Helena com a
finalidade de repatriar os restos mortais de Napoleão Bonaparte. No dia
28 deste mesmo mês, os restos mortais de 504 revolucionários que
tombaram durante o levante de julho de 1830 são transladados para a
\textit{Place de la Bastille} e sepultados solenemente junto ao seu
pedestal da \textit{Colonne de Juillet}.} \textit{Les morts vont
vite --- mon amour, crains"-tu les morts?}

 Estamos, de fato, muito atemorizados com a iminência dos dias de julho
que, neste ano, serão comemorados de maneira especialmente pomposa ---
mas também pela última vez, como se acredita; não é todo ano que o
governo pode tomar sobre os ombros um fardo tão terrível. Nesses dias a
excitação será tão mais intensa quanto mais familiares e afins forem os
sons que se propagam da Espanha para cá, quanto mais ásperos os
detalhes do levante de Barcelona, onde os assim chamados miseráveis se
arrojaram até o mais grosseiro insulto contra a majestade.\footnote{
Heine alude a desdobramentos da guerra civil espanhola, que eclode logo
após a morte do rei Fernando \textsc{vii}, em setembro de 1833. Entram em choque
então, na luta pela sucessão, os absolutistas, liderados por Dom Carlos
(irmão do rei) e os moderados, em torno da rainha Maria Cristina,
regente durante a minoridade de sua filha Isabel. Com o apoio de tropas
inglesas, os moderados conseguem prevalecer, mas focos da resistência
carlista se verificam até 1840 na Catalunha e Aragão. Medidas
impopulares anunciadas pela rainha regente em julho de 1840 levam ao
levante de Barcelona, sobre o qual o \textit{Moniteur} parisiense
trazia o seguinte relato, reproduzido no \textit{Jornal de Augsburgo},
para o qual escrevia Heine: “Na noite de 18 de julho eclodiu uma
revolta sangrenta e o poder militar, que se subtraiu à autoridade dos
ministros, nada fez para impedir a desordem. [\ldots] Barcelona
encontra"-se na mais extrema agitação. A rainha regente foi ofendida
com toda grosseria”.}

 Enquanto no ocidente a guerra de sucessão termina e a verdadeira guerra
revolucionária se inicia, no oriente os assuntos vão se emaranhando num
novelo inextricável. A revolta na Síria coloca o ministério francês no
mais extremo embaraço.\footnote{ Notícias de que Mehemet Ali decretaria
um recrutamento obrigatório na Síria levaram as facções cristãs dos
maronitas (e uma parcela dos drusos) à sublevação a que alude Heine. Os
maronitas, estreitamente ligados à igreja católica e influenciados pela
política francesa, constituíam o único grupo étnico no mundo islâmico
com autorização para portar armas e reivindicavam a criação de um
estado cristão.} Por um lado ele quer, com toda a sua influência,
apoiar o poderio do paxá do Egito; por outro lado, ele não pode
desautorar inteiramente os maronitas, os cristãos do monte Líbano que
plantaram a bandeira da indignação. Pois esta bandeira é a tricolor
francesa, por meio da qual os rebeldes querem se apresentar como
pertencentes à França, acreditando ainda que ela apóia Mehemet Ali
apenas nas aparências, mas em segredo atiça os cristãos sírios contra o
domínio egípcio. Até que ponto eles estão autorizados a supor tal
coisa? Será que a situação é mesmo como se afirma, isto é, será que
alguns dirigentes do partido católico tramaram, sem conhecimento prévio
do governo francês, uma insurreição dos maronitas contra o paxá, na
esperança de que se poderia então, com o enfraquecimento dos turcos,
implantar um reino cristão após a expulsão dos egípcios da Síria? Essa
tentativa inoportuna e, ao mesmo tempo, tão piedosa irá desencadear
muita desgraça por lá. Mehemet Ali indignou"-se a tal ponto com a
deflagração da revolta síria  que passou a agir como um animal
selvagem, não intencionando nada menos do que o aniquilamento de todos
os cristãos no monte Líbano. Tão somente as considerações do
cônsul"-geral da Áustria conseguiram demovê"-lo desse plano desumano,
e é a esse homem magnânimo que milhares de cristãos devem a vida,
enquanto o paxá tem ainda mais o que lhe agradecer: esse cônsul salvou
o seu nome da vergonha eterna.\footnote{ Como se depreende desta
passagem, a política de defesa dos direitos humanos praticada pelo
cônsul austríaco Anton von Laurin (1789--1869) o leva a intervir
tanto a favor dos judeus como dos cristãos.} Mehemet Ali não é
insensível ao prestígio de que goza no mundo civilizado e o senhor von
Laurin desarmou a sua ira pintando"-lhe de maneira muito especial as
antipatias que ele faria recair sobre si, em toda a Europa, com o
assassinato dos maronitas, para prejuízo extremo de seu poder e de sua
glória.

Desse modo, o velho sistema de extermínio dos povos vai sendo banido aos
poucos do oriente por meio da influência europeia. Também os demais
direitos do indivíduo quanto à própria existência vão alcançando por lá
reconhecimento cada vez maior e notadamente as crueldades da tortura
irão ceder lugar a um processo criminal mais brando. É a sangrenta
história de Damasco que irá gerar esse último resultado e, nesse
sentido, a viagem do senhor Cremieux à Alexandria deverá ser registrada
nos anais da humanidade como significativo evento. Este famoso e culto
jurista, que se encontra entre os homens mais celebrados da França e
sobre quem eu tracei anteriormente algumas considerações, já deu início
à sua peregrinação verdadeiramente devota, acompanhado da esposa que
quis compartilhar todos os perigos que ameaçam o seu marido. Que esses
perigos --- que talvez pretendam apenas amedrontá"-lo e desviá"-lo do
nobre caminho que começou a trilhar --- sejam tão pequenos como aqueles
que os engendram! Este advogado dos judeus está de fato defendendo ao
mesmo tempo a causa de toda a humanidade. Não se trata de nada menos do
que introduzir também no oriente o sistema europeu em processos
criminais. O processo contra os judeus damascenos começou com a
tortura; não chegou ao fim, já que um súdito austríaco foi
denunciado e o cônsul desse país se contrapôs à sevícia daquele. O
processo deve agora ser novamente instruído, sem os suplícios de praxe,
sem aqueles instrumentos de tortura que arrancam dos acusados as
declarações mais disparatadas e intimidam as testemunhas. O
cônsul"-geral em Alexandria está movendo céus e terras para sabotar
essa nova instrução do processo; pois sobre o comportamento do cônsul
francês em Damasco poderia recair então luz demasiado forte e a imagem
da França na Síria seria abalada pela ignomínia do seu representante. E
a França tem planos de longo alcance com esse país, planos que datam
ainda do tempo das cruzadas, que não foram abandonados nem sequer pela
Revolução, planos que mais tarde estiveram na mira de Napoleão e que
mesmo o senhor Thiers tem em mente. Os cristãos sírios esperam dos
franceses a sua libertação e estes, por mais que se apresentem em casa
como livres"-pensadores, gostam todavia de passar por devotos
protetores da fé católica no oriente, onde lisonjeiam a zelosa
dedicação dos monges. É assim que explicamos a nós mesmos os motivos
pelos quais não apenas o senhor Cochelet em Alexandria, mas até mesmo o
nosso presidente de conselho, o filho da Revolução em Paris, toma o
cônsul de Damasco sob a sua proteção.\footnote{ Como observa
ironicamente Heine, o cônsul"-geral francês em Alexandria,
Adrien"-Louis Cochelet (1788---1858) sempre interveio junto a Mehemet
Ali, paxá e vice"-governador do Egito, no sentido de proteger as ações
do cônsul Ratti"-Menton no tocante ao caso de Damasco. A ironia de
Heine volta"-se novamente à autoestilização de Thiers como “filho e
herdeiro” da Revolução.} Na verdade, não se trata agora das altas
virtudes de um Ratti"-Menton ou das características ruins dos judeus
damascenos --- talvez não haja aqui nenhuma grande diferença, e assim
como aquele é demasiado irrelevante para o nosso ódio, os últimos
também o seriam para a nossa predileção --- , mas se trata de sancionar a
abolição da tortura no oriente por meio de um exemplo clamoroso. Por
isso, os cônsules das grandes potências européias, notadamente da
Áustria e da Inglaterra, encaminharam junto ao paxá do Egito uma nova
instrução do processo dos judeus damascenos, mas sem admissão de
tortura. E a esses cônsules talvez possa proporcionar alguma alegria
maliciosa o fato de que exatamente o senhor Cochelet --- o cônsul
francês, o representante da Revolução e seu filho\footnote{ Heine
atribui agora ao cônsul"-geral da França em Alexandria, Adrien"-Louis
Cochelet, o mesmo epíteto (“filho da Revolução”) dispensado
anteriormente ao ministro dos assuntos exteriores, Louis"-Adolphe
Thiers.} --- oponha"-se a essa nova instrução do processo e tome
assim o partido da tortura.



