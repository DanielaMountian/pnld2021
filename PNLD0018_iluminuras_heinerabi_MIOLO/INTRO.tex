\hyphenation{Nietzsche Poe-ta poe-ta Loewen-thal}

\chapter{Heine e o judaísmo}
\hedramarkboth{Heine e o judaísmo}{Marcus Mazzari}

\begin{flushright}
\textsc{marcus mazzari}
\end{flushright}

\section{Sobre o autor}

\noindent\textsc{A grande admiração} que Machado de Assis tinha pela prosa de Heine vem à
tona em sua primeira crônica de 1894, a qual se abre justamente com um
adeus ao ano velho: “\textit{Sombre quatre"-vingt"-treize!}” Era
lamentável que Heine, especula o cronista, não estivesse ainda vivo
para comentar o advento do anarquismo, pois o faria melhor do que
ninguém: “Mas Heine, que veio ao mundo no próprio dia 1º de janeiro de
1800, bem podia ter vivido até 1899, e contar tudo o que se passou no
século, com a sua pena mestra de \textit{humour}\ldots\ Oh! página
imortal!”

Machado comete um erro quanto à data de nascimento do autor celebrado,
mas a culpa não é sua e sim do próprio Heine, que sempre procurou
estilizar"-se como um dos primeiros homens do século que Victor Hugo
chamaria mais tarde  de “grande e forte”. Numa nota autobiográfica que
sobrepõe a “verdade poética” à estrita realidade factual, escreveu: “Em
volta de meu berço brincaram os últimos raios lunares do século \textsc{xviii} e
a primeira aurora do século \textsc{xix}”.

Tendo um incêndio destruído documentos e certidões da família Heine,
declarações como essa fizeram vigorar por largo tempo, entre estudiosos
e biógrafos do poeta, muita controvérsia em torno da efetiva data de
seu nascimento. Hoje, contudo, é largamente aceito (mas não
indiscutível) que o menino Harry Heine (o nome Heinrich só se
oficializou com o seu batismo na igreja protestante em 1825) tenha
vindo ao mundo no dia 13 de dezembro de 1797, numa tradicional família
judaica de Düsseldorf, cidade renana que de 1795 a 1813 --- período,
portanto, que compreende os seus anos de infância --- esteve sob domínio
napoleônico, o que muito favoreceu a integração social dos judeus.

Contudo, a verdade poética de Heine quer que o seu berço bifronte tenha
sido iluminado pelo último luar do século \textsc{xviii}: assim podemos conceber
o poeta que reivindicou para si o título de “derradeiro romântico”, o
autor do \textit{Livro das canções} (1821), um dos maiores sucessos
líricos de todos os tempos, o poeta tão amado por Schubert, Schumann,
Mendelssohn, Brahms e tantos outros compositores, mas também por
Friedrich Nietzsche, que em seu livro \textit{Ecce homo} declara ter
recebido de Heine o “mais elevado conceito de poeta lírico”, afirmando
ainda procurar em vão, ao longo de toda a história da poesia, por uma
música “igualmente doce e apaixonada”. Esse “derradeiro romântico” é
também o autor da canção “Loreley” que, musicada em 1831 por Friedrich
Silcher, ganhou incomparável popularidade, tornando"-se tão célebre
que experimentou a questionável honra de ser incluída numa antologia
lírica nacional"-socialista, mas com o adendo “autor desconhecido”.

Mas Heine apresenta"-se também como um dos primeiros a ser banhado pela
luz do século \textsc{xix} e podemos pensar aqui no poeta que, legítimo
precursor de um Brecht ou Maiakóvski, denunciou com insuperável ironia
o tráfico escravo em seu “O Navio Negreiro”, poema do qual diz o
Conselheiro Aires, no último romance de Machado, ter perpetuado a nódoa
que marca o nome do Brasil. Ou podemos pensar ainda, entre vários
outros de seus \textit{Zeitgedichte} (poemas dedicados a assuntos
contemporâneos), no poema “Os tecelões da Silésia”, extraordinário
panfleto revolucionário que deve sua gênese a uma sublevação operária
ocorrida em junho de 1844 e que, com o seu vigoroso ritmo imitativo do
trabalho dos operários junto ao tear mecânico, converteu"-se no hino
que abria toda sessão da Liga dos Comunistas em Londres, como escrevia
a Heine em 1847 um de seus líderes. E, nesse mesmo ano, o tradutor
francês dos “Tecelões” (“Les Tisserands”) observava numa nota
que essa canção se tornara a Marselhesa dos trabalhadores alemães:
\textit{``Cette chanson est devenue la Marseillaise des ouvriers
allemands''}.

Não foi, porém, apenas em versos que Heine se revelou lúcido
contemporâneo do século cuja aurora teria coincidido com a sua chegada
ao mundo, como pretende a mencionada nota autobiográfica.
Estabelecendo"-se na capital francesa em 1831 num exílio de início
voluntário (logo depois, contudo, o retorno à Alemanha não lhe é mais
possível), Heine torna"-se correspondente de um influente diário
liberal de Augsburgo (\textit{Augsburger Allgemeine Zeitung}), e assim
nasce um dos mais extraordinários publicistas de todos os tempos --- nas
palavras do crítico Marcel Reich"-Ranicki, “o mais significativo
jornalista entre os poetas alemães e o mais famoso poeta entre os
jornalistas do mundo todo”. Seus artigos lêem"-se ainda hoje como
verdadeiras obras"-primas da prosa jornalística e grandes temas do
século \textsc{xix} são abordados com incomparável argúcia, mas também sempre
com muita verve e o \textit{humour} ressaltado por Machado em sua
crônica. E se é verdade que Heine não tratou do anarquismo moderno em
sua especificidade, o advento do comunismo torna"-se objeto de uma
aguda reflexão no prefácio que escreveu ao volume \textit{Lutetia} (ou
\textit{Lutezia}, na grafia original do autor: nome latino de
Paris), o qual enfeixa artigos escritos entre fevereiro de 1840 e maio
de 1844: ao mesmo tempo que saúda a entrada em cena do “colosso”
comunista, Heine --- que, aliás, em 1844 teve estreita convivência com o
jovem Marx --- exprime as mais profundas inquietações e alerta para
deformações que sete décadas mais tarde o stalinismo começaria a
converter em realidade. Digno de nota é também um artigo de outubro de
1832, inserido na série “Situações Francesas” (\textit{Französische
Zustände}), em que o autor delineia relações entre escravidão e
liberalismo, e procura demonstrar que este, a rigor, não existiria
plenamente nem sequer nos centros mais avançados. Comentando escritos
do conde Moltke sobre liberdade de comércio, nos quais se exprimem as
posições mais liberais, Heine observa: 
\begin{hedraquote}
Há uma coisa curiosa com esses
nobres! Mesmo os melhores entre eles não conseguem desvincular"-se dos
seus interesses de nascimento. Na maioria dos casos eles podem pensar
liberalmente, talvez de modo ainda mais independente e liberal do que os
plebeus, talvez mais do que estes amar a liberdade e sacrificar"-se
por ela --- mas não são nada receptivos em relação à igualdade civil. No
fundo nenhum ser humano é inteiramente liberal, apenas a humanidade o
pode ser, pois um indivíduo possui a porção de liberalismo que falta ao
outro e, desse modo, as pessoas complementam"-se em sua totalidade da
maneira mais eficaz. O conde possui certamente a mais firme convicção
de que o tráfico negreiro é algo ilegal e abominável, e vota
seguramente em favor de sua abolição. Já Mynheer van der Null, um
mercador de escravos que eu conheci sob as arvorezinhas em Roterdã,\footnote{
Heine alude jocosamente à rua \textit{Boompjes}, “arvorezinhas” em
holandês.} está, ao contrário, inteiramente convencido de que
tráfico negreiro é plenamente natural e decente, ao passo que
prerrogativas de nascimento, privilégios hereditários, nobreza,
constituem algo injusto e absurdo, que todos os homens honestos têm a
obrigação de abolir.
\end{hedraquote}

Outro tema que adquire expressivo relevo nos artigos heinianos diz
respeito ao antissemitismo e fanatismo religioso. Ensejo imediato
para essa série de artigos redigidos em 1840 foi um \textit{pogrom}
deflagrado em Damasco pelo cônsul francês Ratti"-Menton. Analisando
friamente cada detalhe da ação do diplomata, assim como os subsequentes
desdobramentos na capital síria e as reações dúbias do governo francês,
Heine começa a perceber os novos contornos que o antigo ódio cristão
aos judeus vai assumindo e, desse modo, a intuir sua transição ---
conforme observou Günter Grass --- para a “organizada loucura racial do
antissemitismo”. Mas os acontecimentos de Damasco também inauguram
uma nova etapa na intrincada história do relacionamento de Heine com o
judaísmo, estimulando"-o ao mesmo tempo a retomar um projeto romanesco
que havia abandonado cerca de quinze anos atrás: \textit{O Rabi de
Bacherach}, a narrativa que esta edição oferece,
seguida de três artigos que,
dedicados ao \textit{pogrom} damasceno, descortinam uma vista para essa
extraordinária prosa jornalística.

\section{Sobre a obra}

Heine concebeu o projeto de um romance histórico que fosse ao mesmo
tempo uma enciclopédia épica do destino judaico na sociedade feudal em
abril ou maio de 1824. O jovem literato encontrava"-se então sob o
influxo de uma estada em Berlim marcada por intensa participação nas
atividades promovidas pela “Associação para a Cultura e Ciências dos
Judeus” (\textit{Verein für Kultur und Wissenschaften der Juden}). Tal
associação fora fundada em 1819 por intelectuais de origem judaica que
se empenhavam em fazer frente à escalada de antissemitismo que se
verificava na Alemanha nesse período dominado pela política da
Restauração --- um exemplo é a revogação na Prússia, imediatamente após a
derrota de Napoleão, do chamado Edito da Tolerância, promulgado em 1812
e que facultava aos judeus acesso à carreira universitária e demais
cargos públicos. O principal objetivo dos intelectuais engajados nesse
projeto era, assim, desfazer preconceitos e promover uma convivência
tolerante entre as culturas alemã e judaica: “favorecer o advento de
uma época em que ninguém mais pergunte na Europa quem é judeu e quem é
cristão”, segundo a formulação de um de seus fundadores, o jurista
Eduard Gans.

Embora Heine se definisse por essa época como “indiferentista”,
adversário de toda religião positiva (e avesso, portanto, tanto ao
judaísmo quanto ao cristianismo), ele identifica"-se com as propostas
da associação berlinense, assume tarefas práticas enquanto professor de
alemão, francês e história, e promete ao historiador Leopold Zunz,
editor da \textit{Revista para a Ciência do Judaísmo}, um ensaio sobre
a “grande dor judaica”. Ao seu amigo Moses Moser escreve em agosto de
1823: “Confesso que me posicionarei entusiasticamente em prol dos
judeus e de sua equiparação social; e em tempos ruins, que serão
inevitáveis, o populacho germânico ouvirá a minha voz, de tal modo que
ecoará nos palácios e cervejarias alemãs”.

Entretanto, após o seu retorno à cidade de Göttingen, onde fazia o curso
de jurisprudência, decide dar um tratamento literário ao tema proposto
e, movido por essa intenção, mergulha no estudo de obras referentes ao
judaísmo. Numa carta datada de 25 de outubro de 1824, comunica a Moser
ter avançado significativamente na redação do \textit{Rabi de
Bacherach}, e acrescenta: 
\begin{hedraquote}
Mas ele ficará bastante extenso, será
certamente um grosso volume, e com indescritível amor venho acalentando
essa obra em meu peito. Pois ela nasceu, de fato, apenas do amor e não
da vã sede de glória. Pelo contrário, se eu fosse dar ouvidos à opinião
geral, eu não a escreveria de modo algum. Estou vendo de antemão
quantas portas vou fechar com isso e as hostilidades que provocarei.
Mas, exatamente por nascer do amor, haverá de ser um livro imortal, uma
eterna chama na catedral de Deus, e não uma luz de teatro crepitante e
fugaz. Apaguei desse livro muita coisa já escrita, e somente agora
logrei dominar o todo.
\end{hedraquote}

No entanto, a partir do patamar atingido no final de 1824, os esforços
de Heine no sentido de desbastar o vasto material histórico pesquisado
e integrá"-lo na narrativa não obtêm mais nenhum avanço significativo.
O seu fôlego épico não se revela propriamente de longo alcance, e
sinais de exaurimento não demoram muito para manifestar"-se. E  às
dificuldades que enfrenta enquanto narrador vêm somar"-se ainda fatos
exteriores, como a dissolução, em meados de 1825, da “Associação para a
Cultura e Ciências dos Judeus” e, logo em seguida, sua conversão um
tanto pragmática ao protestantismo (a certidão de batismo é
caracterizada por ele como seu “\textit{Entreebillet} para a cultura
europeia”). Mesmo assim, as cartas enviadas a Moses Moser pelos meses
seguintes continuam a atestar um entusiasmo inquebrantável pelo seu
romance histórico: em julho de 1825, Heine diz estar convicto de que
somente ele poderá escrever essa obra que deverá ser considerada como
“fonte” por todos os séculos vindouros.

As intenções, porém, permanecem distantes da concretização, e de 1826 em
diante Heine vai se distanciando cada vez mais daquele projeto inicial
de escrever uma obra imortal sobre “a grande dor judaica”. Por um lado,
ele encontra na prosa ágil e espirituosa de seus \textit{Quadros de
viagem} um meio mais eficiente de expressão literária; por outro lado,
as transformações que se dão com suas concepções políticas e sociais
têm implicações também para sua postura perante o judaísmo. Sem deixar
em momento algum de solidarizar"-se com os judeus discriminados, ele
passa a enxergar a questão, de maneira crescente, num contexto mais
amplo de opressores e oprimidos, e sua solução é inserida no movimento
de emancipação da humanidade. No vigésimo nono capítulo da “Viagem de
Munique a Gênova”, publicada na terceira parte dos \textit{Quadros de
viagem}, ele declara ser a emancipação a grande tarefa da época: 

\begin{hedraquote}
Não apenas a dos irlandeses, gregos, judeus de Frankfurt, negros da América
Central e demais povos oprimidos, mas sim a emancipação do mundo
inteiro, especialmente da Europa, que atingiu a maioridade e se
desprende dos férreos laços dos privilegiados, da aristocracia. E, por
mais sutis que sejam as cadeias de silogismos forjadas por alguns
filósofos renegados da liberdade para nos provar que milhões de seres
humanos foram criados como animais de carga para alguns milhares de
cavaleiros privilegiados, eles não nos convencerão disso enquanto não
demonstrarem, como diz Voltaire, que aqueles vieram ao mundo com selas
sobre as costas e os últimos com esporas nos pés.
\end{hedraquote}

Nova inflexão no relacionamento de Heine com o judaísmo adveio, conforme
mencionado, com o \textit{pogrom} desencadeado pelo cônsul francês em
1840. Volta então a concentrar"-se na especificidade do
antissemitismo, redige a série de artigos sobre os acontecimentos de
Damasco, retorna ao manuscrito do \textit{Rabi}, acrescenta"-lhe alguns
novos trechos e se decide finalmente, dezesseis anos após ter iniciado
o projeto, a publicá"-lo como “fragmento de romance” no quarto e
último volume de seu \textit{Salon}, uma série que criara em 1834 para
dar vazão a escritos esparsos (e intitulada segundo o salão do Louvre
destinado a exposições de artistas contemporâneos).

Assim veio a lume essa obra que, pode"-se dizer, representou um
verdadeiro calvário para o narrador Heine. Pois outra adversidade
sofrida pelo “pobre Rabi” (como algumas vezes o chamou) foi o incêndio
ocorrido na casa de sua mãe, em Hamburgo, no ano de 1833, o qual
destruiu vários manuscritos que ali deixara guardados antes de emigrar
para a França. Nunca se soube exatamente em que medida o \textit{Rabi}
foi atingido pelo incêndio, mas é possível afirmar com segurança que o
texto jamais esteve concluso, devendo"-se, portanto, relativizar a
explicação que o autor oferece para a interrupção abrupta da narrativa.
A crítica é também unânime em reconhecer que o terceiro capítulo, que
tanto em seu teor quanto no aspecto estilístico se diferencia
nitidamente dos dois anteriores, foi escrito (ou, ao menos,
reelaborado) somente em 1840, para conferir certo arredondamento ao
texto a ser publicado.

Desse modo, se já no segundo capítulo cristãos e judeus de Frankfurt são
retratados com muita ironia, no terceiro essa tendência se reforça
sobremaneira, e o leitor poderá encontrar também indícios mais
sugestivos da direção que a narrativa possivelmente tomaria.
Evidencia"-se que a figura do cavaleiro espanhol, a quem “os sombrios
nazarenos obcecados pelo sofrimento” são tão repulsivos quanto “os
ressequidos hebreus sem alegria”, foi concebida de modo a encarnar
motivos que deveriam ser desdobrados na segunda parte do romance,
provavelmente por um período de sete anos (recorrente na narrativa, em
consonância com a tradição bíblica) e em terras espanholas ou talvez
americanas: para essa direção aponta uma observação de Heine referente
à coincidência, em 1492, entre a expulsão dos judeus da Espanha e a
descoberta da América. Além disso, o diálogo travado, nesse terceiro
capítulo, entre Don Isaac Abarbanel e Rabi Abraão remete a questões com
as quais Heine, próximo então a um ateísmo mesclado com elementos
panteístas, defrontava"-se teoricamente, por exemplo, a contraposição
entre sensualismo e espiritualismo --- ou, nos termos do próprio autor,
entre um “helenismo” voltado aos prazeres e impressões sensuais e um
ascético “nazarenismo” (\textit{Nazarenertum}).

\textit{O Rabi de Bacherach} permite assim não apenas conhecer mais de
perto as mudanças por que passou a postura de seu autor diante do
judaísmo (e da religião em geral) ao longo desses dezesseis anos, como
também abre acesso privilegiado para a compreensão de um aspecto
fulcral de toda a história alemã. A vastidão e a aspereza do tema
impediram Heine de concluir o projeto épico concebido na juventude, mas
mesmo enquanto fragmento o \textit{Rabi} possui caráter autônomo e
desde a sua publicação nunca deixou de inquietar os leitores.

%\subsection{repercussão da obra}

Em 1894, o dramaturgo e diretor teatral Karl Weiser realizou uma
adaptação da narrativa para o Teatro da Corte de Weimar e em 1913 um
autor de nome Max Viola deu"-lhe continuidade e a levou a um desfecho
mirabolante, atando de maneira forçada todos os fios que Heine deixara
soltos. De grande valor artístico é o ciclo de 16 litogravuras com que
Max Liebermann, aos 75 anos de idade, ilustrou uma edição do
\textit{Rabi} publicada em 1923, num dos momentos mais felizes de
integração entre artes gráficas e texto literário. Quanto às
interpretações teóricas suscitadas pelo fragmento heiniano, merecem
especial menção o trabalho de doutorado apresentado em 1907 pelo futuro
autor de romances históricos Lion Feuchtwanger, o qual, a exemplo do
herói da narrativa (assim como do próprio Heine), também passaria pela
experiência do exílio, e o prefácio, ao mesmo tempo erudito e engajado,
escrito por Erich Loewenthal, para uma edição do \textit{Rabi} que
ainda pôde ser publicada em Berlim no ano de 1937: vítima do
\textit{pogrom} nacional"-socialista, esse importante crítico
(conceituado nome na filologia sobre Heine e Platão) perderia a vida em
1944 no campo de extermínio de Auschwitz.

Expressiva homenagem coube ainda ao \textit{Rabi de Bacherach} por
intermédio do romancista alemão do pós"-guerra Günter Grass. Num texto
de 1979, “Como dizer aos nossos filhos?”, o autor do célebre romance
\textit{O tambor de lata} submete o fragmento de Heine a uma leitura ao
mesmo tempo política e literária, comentando ainda a tentativa que
empreendera na juventude no sentido de dar"-lhe prosseguimento e
incorporar ao desfecho a experiência histórica do holocausto
nacional"-socialista --- isto é, levar Rabi Abraão e sua mulher Sara não
para a Frankfurt do final da Idade Média, mas para a dos anos 1930.
Embora fracassada, a tentativa teve significado
fundamental para sua formação enquanto romancista: 

\begin{hedraquote}
Sem que eu tivesse encontrado qualquer ponto de sustentação, ficara 
obcecado pela ideia de concluir o fragmento de Heine \textit{O Rabi de Bacherach}. A ironia
romântica de Heine instigou"-me a assumir uma posição de contraponto.
Seu fracasso diante de material tão amplo fez"-me ambicioso. Hoje eu
sei que, sem o desvio pela Bacherach de Heine, eu não teria encontrado
o acesso à história dos judeus de Danzig. 
\end{hedraquote}

Essa observação refere"-se
mais diretamente à gênese do livro \textit{Do diário de um caracol}
(1972); considerando"-se, porém, o papel que personagens judias e o
tema do antissemitismo desempenham no \textit{tambor de lata} (1959)
e \textit{Anos de cão} (1963), adquire relevância também para os seus
dois primeiros romances, que compõem, ao lado da novela \textit{Gato e
Rato }(1961), a chamada Trilogia de Danzig.

Em “Como dizer aos nossos filhos?”, Grass opera, sobre o fundamento de
sua visão não linear da História, um entrelaçamento das questões que
Heine procurou abordar no \textit{Rabi }com o fenômeno do
nacional"-socialismo e desdobramentos políticos que se prolongaram até
os anos 1970 na República Federal da Alemanha. Recusando"-se sempre a
dissociar o presente do passado, Grass enfatiza, como um dos
pressupostos para os crimes nazistas, a omissão de milhões de pessoas
(sobretudo das igrejas católica e protestante, a cristandade
clericalmente organizada) e, ao mesmo tempo, empenha"-se em desvelar a
função mistificadora exercida pelos modernos meios de comunicação de
massa, como em séries televisivas (\textit{Holocausto}, 1979) de
sucesso tão retumbante quanto questionável: “esclarecimento em massa” é
visto assim, em seu caráter superficial e apelativo, enquanto reflexo
moderno de “extermínio em massa”.

Tomando como ponto de partida o fragmento de Heine, Grass explicita
também pontos fundamentais de sua estética narrativa, entre os quais o
empenho em contrapor"-se a concepções de História que em última
instância eximem os indivíduos de suas verdadeiras responsabilidades e
rotinizam a barbárie. Pois, enquanto resistência à “força niveladora da
transitoriedade”, a concepção de tempo histórico que subjaz aos seus
livros caracterizar"-se"-ia pela fusão de passado, presente e
futuro: \textit{Vergegenkunft}, conforme se exprime num neologismo que
condensa os substantivos \textit{Vergangenheit} (“passado”),
\textit{Gegenwart} (“presente”) e \textit{Zukunft} (“futuro”).

“Queria ensinar às crianças que toda história que se passa hoje na
Alemanha já começou há séculos, que essas histórias alemãs com as suas
sempre renovadas atribuições de culpa não envelhecem, não podem
cessar”, diz Grass no texto em questão, e essa intenção “pedagógica”
converte"-se em procedimento literário na narrativa \textit{O encontro
em Telgte} (1979), construída em torno de uma reunião fictícia, em
1647, entre escritores barrocos de língua alemã, empenhados em discutir
a situação política e cultural da nação devastada pela Guerra dos
Trinta Anos. Traçando paralelos com uma iniciativa semelhante, mas
agora real, empreendida três séculos mais tarde por um grupo de
escritores alemães (o chamado Grupo 47), a narrativa começa da seguinte
forma: “Ontem será o que amanhã foi. Nossas histórias alemãs de hoje
não precisam ter acontecido agora. Esta começou há mais de trezentos
anos. Outras histórias também. De tão longe vem toda história que se
passa na Alemanha”.
Para a constituição dessa perspectiva histórica fundamentada na fusão
de temporalidades não terá deixado de contribuir a longa convivência de
Grass com os escritos de Heine, em especial as reiteradas leituras do
fragmento \textit{O Rabi de Bacherach}, em que o autor remonta às
desvairadas perseguições medievais aos judeus para tocar nas raízes do
antissemitismo de seu tempo. 


\section{Sobre o gênero} %%%

O romance, em relação a outros gêneros em prosa, caracteriza"-se pela pluralidade dramática,

\begin{quote}
uma série de dramas, conflitos ou
células dramáticas. Em princípio, não há limite para os núcleos
dramáticos que podem compor a ação de um romance. Ao ficcionista,
cabe selecionar os que possuem a virtualidade de se organizar harmonicamente.
E essa escolha é o grande obstáculo que se lhe 
depara, dado que infinitas
possibilidades lhe são oferecidas ao
simples golpe de vista lançada sobre os acontecimentos diários.
A imaginação, com transfundi-los e transcendê-los, faz o resto,
avultando ainda mais o número de
caminhos revelados à sua intuição.\footnote{\textsc{moisés}, Massaud. \textit{A criação literária}. São Paulo: Cultrix, 2006, p.\,172.}
\end{quote}

Como desdobramento desse núcleo dramático plural, o crítico Massaud Moisés
vê na estrutura do romance um encadeamento de conteúdos, em que ``no fim de cada episódio, procura deixar sementes de mistério ou conflito para manter aceso o interesse do leitor. É raro que esvazie o recheio dramático duma célula antes de prosseguir, pois frustraria a curiosidade do leitor''\footnote{Ibidem, p.\,114.}

No caso de Heine, enquanto narrador e poeta mas, sobretudo, enquanto jornalista tão admirado por Machado de Assis, sua prosa insere"-se inequivocamente na tradição
iluminista, encarnando como talvez nenhum outro escritor alemão, para
citar uma observação de Grass, “o esplendor e a miséria do Iluminismo
europeu” --- e também vale lembrar nesse contexto que o próprio Heine se
define, numa carta de setembro de 1855 (portanto, na fase mais
excruciante de sua longa enfermidade e a cinco meses da morte em 17 de
fevereiro do ano seguinte), como “um pobre rouxinol alemão que fez o
seu ninho na peruca de Monsieur de Voltaire”.

Ainda mais do que em seu romance,
nos três artigos enfeixados neste volume o leitor verá articular"-se
uma resistência ao fanatismo religioso que lembra efetivamente, em
vários aspectos, a postura voltairiana no \textit{Dictionnaire
philosophique}, conforme a mostra em especial o verbete
“\textit{Fanatisme}”. A luta de Heine contra o fanatismo religioso mas
também político, contra toda sorte de racismo, preconceitos e
dogmatismos vem sempre acompanhada pela mensagem de tolerância e se
alinha, segundo a sugestão da citada passagem dos \textit{Quadros de
viagem}, no amplo movimento para a emancipação dos povos oprimidos.
Todavia, essa mesma mensagem de tolerância que perpassa os artigos de
Heine adverte para a possibilidade funesta de um povo que se encontra
hoje sob opressão converter"-se amanhã em opressor: isso significaria
apenas intensificar a espiral de violência e, consequentemente, o
círculo diabólico que no primeiro dos artigos sobre o \textit{pogrom}
de Damasco vem condensado no ditado “hoje bigorna, amanhã malho!”

 O grande compromisso de Heinrich Heine foi manifestamente com o projeto
iluminista de emancipação e esse engajamento --- latente na ficção
histórica, mais explícito na prosa jornalística --- pode ser observado
tanto no fragmento romanesco \textit{O Rabi de Bacherach} como nos
admiráveis artigos reunidos no volume \textit{Lutetia}. Para a mesma
direção apontam também os \textit{Quadros de viagem}, obra em que a
aspiração heiniana por uma humanidade livre e emancipada encontra, numa
dicção ainda impregnada de tons românticos, a seguinte formulação:

\begin{hedraquote}
Realmente não sei se mereço se um dia as pessoas vierem a adornar o
meu caixão com uma coroa de louros. A poesia, por mais que eu a tenha
amado, sempre foi para mim apenas uma diversão sagrada ou um
bem"-fadado meio para atingir fins celestiais. Nunca dei muito valor à
glória literária, e pouco me importa se as minhas canções serão
elogiadas ou criticadas. Contudo, uma espada devereis colocar em meu
caixão, pois fui um brioso soldado na luta de libertação da
humanidade.
\end{hedraquote}
