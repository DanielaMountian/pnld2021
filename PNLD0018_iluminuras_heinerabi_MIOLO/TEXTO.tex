
\hyphenation{Feucht-wanger Nieder-heim-bach Höllen-thier}

\chapter{Introdução}


\textit{O Rabi de Bacherach} foi concebido inicialmente como projeto de romance histórico,
no início de 1824, durante um período de bastante
contato com a história e a cultura do povo judeu. O ensejo imediato
para a narrativa foi a sua intenção de contrapor"-se à escalada de
antissemitismo que se manifestava então na Alemanha e promover o
intercâmbio e o diálogo entre as culturas alemã e judaica. O enredo
de \textit{O Rabi de Bacherach} é situado no final do século \textsc{xv}, mas o
narrador remonta também a séculos anteriores para tocar nas raízes
históricas do antissemitismo na Alemanha. Apesar de seu entusiasmo
pelo projeto, Heine não conseguiu vencer a amplitude e a aspereza do
assunto, publicando a narrativa, como “fragmento de romance”, em 1840
no 4\oi\,volume da série intitulada \textit{Salon}. Entretanto, mesmo em seu
caráter fragmentário, a obra constitui expressivo exemplo da
arte narrativa de Heine e é o documento mais elucidativo de sua
flutuante relação com o judaísmo.

Além de seu romance histórico, o tradutor Marcus Mazzari organizou para a presente edição uma seleta de três artigos que enfocam a questão do
fanatismo religioso, aprofundando questões que já vinham sendo tratadas em seu romance.
Esses textos foram extraídos do volume \textit{Lutetia}, em
que Heine enfeixou 61 artigos escritos em Paris entre fevereiro de 1840
e maio de 1844, bem como quatro artigos dos anos 1843 a 1846. Esse
volume, que constitui verdadeira obra"-prima da prosa jornalística,
foi publicado em alemão em 1854, com o subtítulo \textit{Relatos sobre
arte, política e vida social}, que recebeu, poucos meses depois, uma edição
francesa (\textit{Lutèce. Lettres sur la Vie politique, artistique et
sociale en France}), para a qual Heine escreve o célebre prefácio em
que comenta o advento do comunismo.

Disposto dessa maneira, o presente volume apresenta uma ampla percepção da relação de Heine com o judaísmo, além de desvelar ao leitor o contexto da cultura judaica na Europa e a relação que os intelectuais mantinham com essa cultura no século \textsc{xix}.
Ainda mais do que em seu romance,
nos três artigos enfeixados neste volume o leitor verá articular"-se
uma resistência ao fanatismo religioso que lembra efetivamente, em
vários aspectos, o iluminismo voltairiano do \textit{Dictionnaire
philosophique}, em especial no verbete
“\textit{Fanatisme}”. A luta de Heine contra o fanatismo religioso mas
também político, contra toda sorte de racismo, preconceitos e
dogmatismos vem sempre acompanhada pela mensagem de tolerância e se
alinha no amplo movimento para a emancipação dos povos oprimidos.
Todavia, essa mesma mensagem de tolerância que perpassa os artigos de
Heine adverte para a possibilidade funesta de um povo que se encontra
hoje sob opressão converter"-se amanhã em opressor: isso significaria
apenas intensificar a espiral de violência e, consequentemente, o
círculo diabólico que no primeiro dos artigos sobre o \textit{pogrom}
de Damasco vem condensado no ditado “hoje bigorna, amanhã malho!”

 O grande compromisso de Heinrich Heine foi manifestamente com o projeto
iluminista de emancipação e esse engajamento --- latente na ficção
histórica, mais explícito na prosa jornalística --- pode ser observado
tanto no fragmento romanesco \textit{O Rabi de Bacherach} como nos
admiráveis artigos reunidos no volume \textit{Lutetia}. Para a mesma
direção apontam também os \textit{Quadros de viagem}, obra em que a
aspiração heiniana por uma humanidade livre e emancipada encontra, numa
dicção ainda impregnada de tons românticos, a seguinte formulação:

\begin{hedraquote}
Realmente não sei se mereço se um dia as pessoas vierem a adornar o
meu caixão com uma coroa de louros. A poesia, por mais que eu a tenha
amado, sempre foi para mim apenas uma diversão sagrada ou um
bem"-fadado meio para atingir fins celestiais. Nunca dei muito valor à
glória literária, e pouco me importa se as minhas canções serão
elogiadas ou criticadas. Contudo, uma espada devereis colocar em meu
caixão, pois fui um brioso soldado na luta de libertação da
humanidade.
\end{hedraquote}

\bigskip

\hfill{\itshape\small Consultoria de Marcus Mazzari}

\chapter*{}
\thispagestyle{empty}

\vspace*{4cm}
\begin{verse}
O autor, saudando serenamente,\\
dedica a lenda do Rabi de Bacherach\\
ao seu querido amigo\\
\qquad\qquad\qquad\qquad\textsc{heinrich laube}\footnote[*]{
Heinrich Heine Laube (1806---1884): escritor e publicista pertencente à
chamada ``Jovem Alemanha'', movimento que entre os anos de 1830 e 1850
exerceu forte oposição à política conservadora, e mesmo repressiva, em
vigor nos estados alemães. Heine tencionava dedicar"-lhe o memorial
\textit{Ludwig Börne, }publicado em 1840, mas depois transferiu a
dedicatória ao \textit{Rabi}.}

\end{verse}

\chapter[Primeiro capítulo]{Primeiro capítulo}


\vspace{2.5cm}

\textsc{Abaixo} da região renana,\footnote{ Esta tradução foi
realizada a partir do texto estabelecido no volume 5 da edição
histórico"-crítica das obras de Heine --- edição patrocinada pela cidade de
Düsseldorf, em colaboração com o Instituto Heinrich Heine: ``Der Rabbi von
Bacherach'', in \textit{Historisch"-kritische Gesamtausgabe der Werke}, Band 5
(organizado por Manfred Windfuhr).  Hamburgo, Hoffmann und Campe, 1994.}
onde as margens da torrente perdem a feição
sorridente, montanha e rochedos, com suas impávidas fortalezas em
escombros, assumem maior aspereza e se alça uma magnificência mais
selvagem e grave --- nesse lugar, como numa horripilante saga de tempos
imemoriais, fica a sombria, a antiquíssima cidade de
Bacherach.\footnote{ Heine
retirou a grafia Bacherach, em lugar de Bacharach, da obra de Johann
Jacob Schudt \textit{Jüdische Merckwürdigkeiten }[Curiosidades
judaicas], publicada entre 1714 e 1721. Entre as muitas fontes que
pesquisou para redação do \textit{Rabi }estão: Jaques Basnage:
\textit{Histoire de la religion des juifs depuis Jésus Christ
jusq'à présent.} Roterdã, 1707 (quinze volumes).
Christian Friedrich Bischoff: \textit{Dissertatio
historico"-philologica, de origine, vita, atqve scriptis Don Isaaci
Abrabanielis}. Altdorf, 1708.  Tilemann Elhen: \textit{Fasti
Limpurgenses. }\textit{Das ist: Eine wohlbeschriebene Chronick Von der
Stadt und den Herren zu Limpurg auff der Lahn} [\textit{Limburger
Chronik}]. Wetzlar, 1720.   Anton Kirchner: \textit{{Geschichte der
Stadt Frankfurt am Main}}. Frankfurt a. M., 1807"-1810. Achilles
Augustus von Lersner:\textit{{Der Weit"-berühmten Freyen Reichs"- und
Handels"-Stadt Frankcfurt am Mayn}} \textsc{chronica}. Frankfurt am Main, 1734.
 Juan"-Antonio Llorente: \textit{Histoire critique de
l’inquisition d’Espagne}. Paris, 1818 (2ª\,edição). Aloys Schreiber:
\textit{{Handbuch für Reisende am Rhein}}. Heidelberg, 1822.
Johann Jacob Schudt: \textit{{Jüdische Merckwürdigkeiten}}. Frankfurt a.
M. e Leipzig, 1714"-1721. Niklas Vogt: \textit{{Rheinische
Geschichten und Sagen}}. Frankfurt a. M., 1817.  \textit{Die
Pesach"-Hagada, oder Erzählung von Israël’s Auszug aus Egypten}.
Leipzig, 1839 (em idioma hebraico e alemão).}
 Todavia, nem sempre foram assim podres e deterioradas essas muralhas
com suas ameias desprovidas de merlões e suas atalaias cegas, em cujas
frestas silva o vento e aninham"-se os pardais; nem sempre imperou
nessas vielas lamacentas e miseravelmente feias, as quais se avistam
através do portão arruinado, silêncio tão desolador, interrompido
apenas de quando em quando pelos gritos de crianças, berros de mulheres
e mugidos de vacas. Orgulhosas e rijas foram outrora essas muralhas, e
por essas vielas circulava então uma vida pujante e livre --- poder e
luxo, prazer e sofrimento, muito amor e muito ódio. Bacherach
incluía"-se nessa época entre aqueles municípios fundados pelos
romanos durante o seu domínio junto ao Reno; e embora os tempos
subsequentes tivessem sido muito tempestuosos e Bacherach viesse assim
a cair sob a dominação da dinastia dos Hohenstaufen e, posteriormente,
dos Wittelsbach, os seus habitantes souberam no entanto, segundo o
exemplo de outras cidades renanas, conservar uma vida comunitária
bastante livre. Consistia esta numa associação de corporações
particulares, sendo que a corporação dos antigos patrícios e a das
ligas dos artesãos, que por sua vez se subdividiam de acordo com os
diferentes ofícios, lutavam de ambos os lados pelo monopólio do poder:
dessa forma, quando se voltavam para fora dos muros da cidade, como
proteção e defesa perante a rapinosa nobreza vizinha, originava"-se
uma estreita união entre as corporações todas; internamente, contudo,
por força de interesses conflitantes, obstinavam"-se em permanente
cisão. E, por isso, havia pouca convivência entre as corporações de
Bacherach, muita desconfiança e, com frequência, até mesmo violentas
irrupções das paixões. O alcaide em exercício tinha a sua sede na
elevada fortaleza de Sareck e, como o seu falcão, arrojava"-se para
baixo quando o chamavam --- e, muitas vezes, mesmo sem ser chamado. O
clero dominava na obscuridade mediante o obscurecimento do espírito.
Uma corporação das mais isoladas e impotentes, que ia sendo
gradativamente excluída dos direitos civis, era a pequena comunidade
judia que já no tempo dos romanos se estabelecera em Bacherach e mais
tarde, durante a grande perseguição aos judeus, acolheu em seu seio
grupos inteiros de fugitivos que professavam a mesma fé.

A grande perseguição aos judeus começou com as cruzadas e recrudesceu da
maneira mais feroz por volta da metade do século \textsc{xiv}, ao final da
grande peste que, como toda desgraça pública, teria sido provocada
pelos judeus, uma vez que se afirmava terem eles desencadeado a ira de
Deus e envenenado as fontes com ajuda dos leprosos. O populacho
açulado, sobretudo as hordas de flagelantes --- mulheres e homens seminus
que, em busca de expiação, percorriam a região do Reno e o restante
território do sul da Alemanha supliciando os próprios corpos e entoando
uma desvairada canção em louvor da Virgem Maria ---, assassinaram então
muitos milhares de judeus, mas também os torturavam ou batizavam à
força. Uma outra acusação que já nos primeiros tempos, e depois ao
longo de toda a Idade Média até o início do século passado,
custou"-lhes muito sangue e angústia, consistia no conto pueril e
fantasioso, mas repetido à exaustão em crônicas e lendas, segundo o
qual os judeus roubavam hóstias consagradas e as perfuravam depois com
facas até que o sangue começasse a
escorrer;\footnote{ Referência à chamada ``profanação de hóstias'', acusação que na Idade
Média se levantava frequentemente contra os judeus. Também era assim
que se explicavam manchas avermelhadas que determinados fungos faziam
aparecer nas hóstias.}
dizia ainda tal conto da carochinha que eles imolavam crianças cristãs
durante a comemoração do Pessach, com a finalidade de utilizar o sangue
em suas cerimônias religiosas noturnas. Nesse dia de festa, os judeus,
já bastante odiados por causa de sua fé, de suas riquezas e de seus
livros contábeis, encontravam"-se inteiramente nas mãos de seus
inimigos; e com extrema facilidade podiam estes provocar sua desgraça,
bastando para isso espalhar o boato de um tal infanticídio --- talvez até
mesmo introduzissem sorrateiramente um ensanguentado cadáver de criança
na casa proscrita de um judeu para depois, durante a madrugada,
investir de surpresa contra a família judia congregada em oração,
quando então se assassinava, saqueava e batizava, e grandes milagres
aconteciam graças à criança encontrada morta, a qual a Igreja por fim
chegava até mesmo a canonizar. São Werner é um desses santos e em sua
homenagem foi instituída em Oberwesel aquela suntuosa abadia que hoje
representa uma das mais belas ruínas às margens do Reno e que, com a
magnificência gótica de seus vitrais compridos e ogivais, de suas
colunas que se elevam orgulhosas e de seus entalhes em pedra, tanto nos
encanta quando passamos por ali num belo e verdejante dia de verão e
desconhecemos a sua
origem.\footnote{ Aloys
Schreiber relata em seu \textit{Handbuch für Reisende am Rhein} [Manual
para viajantes do Reno] a lenda de São Werner, um rapaz de quatorze
anos que teria sido assassinado por judeus no ano de 1287. A acusação
provocou o assassinato de 2000 judeus, incontáveis outros pagaram
vultosas quantias para preservar a vida. Werner foi logo canonizado e
se converteu no santo padroeiro de quatro igrejas renanas. O seu nome,
contudo, não consta mais do calendário católico, pois investigações
históricas demonstraram a inconsistência da acusação.}
 Em homenagem a esse santo foram erigidas ainda três outras grandes
igrejas às margens do Reno, e incontáveis judeus foram mortos ou
atormentados. Isso aconteceu no ano de 1287 e também em Bacherach, onde
se construiu uma dessas igrejas de São Werner, desabaram então sobre os
judeus muitas aflições e muita miséria. Porém, pelos dois séculos
seguintes eles foram poupados de tais acessos da cólera popular, embora
sempre estivessem suficientemente expostos a hostilidades e ameaças.

Contudo, quanto mais os afligia o ódio de fora, tanto mais estreita e
íntima tornava"-se a convivência doméstica, tanto mais profundos a
devoção e o temor dos judeus de Bacherach perante Deus. Um modelo de
procedimento agradável a Deus era o rabino local, chamado Rabi Abraão,
um homem ainda jovem, mas amplamente famoso em virtude de sua erudição.
Ele havia nascido nessa cidade, e seu pai, que ali fora igualmente
rabino, ordenou"-lhe como sua última vontade que se dedicasse à mesma
função e jamais abandonasse Bacherach, a não ser sob perigo de vida.
Essa ordem e uma estante com livros raros foi tudo o que lhe legou o
pai, o qual vivera na pobreza e apenas para suas atividades de escriba.
No entanto, Rabi Abraão era um homem muito rico; casado com a filha
única de seu falecido tio paterno, que exercera a profissão de
joalheiro, herdou suas imensas riquezas. Alguns intrigantes na
comunidade insinuavam que o Rabi teria desposado sua mulher justamente
por causa do dinheiro. Mas todas as mulheres contradiziam tal coisa e
sabiam contar velhas histórias: como o Rabi, antes de sua viagem a
Espanha, já estava apaixonado por Sara --- na verdade, chamavam"-na a
bela Sara --- e como Sara teve de esperar sete anos até que o Rabi
regressasse da Espanha, uma vez que, valendo"-se do costume do anel
nupcial, ele a desposara contra a vontade do pai e mesmo sem a anuência
da própria Sara. É que todo judeu pode fazer de uma moça judia sua
esposa legal se conseguir enfiar"-lhe um anel no dedo e, ao mesmo
tempo, pronunciar as palavras: ``Tomo"-te por minha mulher segundo os
costumes de Moisés e
Israel!''\footnote{ Heine
retirou essa fórmula da obra de Johann Schudt \textit{Jüdische
Merckwürdigkeiten }[Curiosidades judaicas], onde ela aparece como
momento culminante numa série de cerimônias anteriores e posteriores,
todas minuciosamente descritas. Para os propósitos da narrativa, Heine
reduz, portanto, todo o ritual do matrimônio a esse único ato da
passagem da aliança.}
 À menção de Espanha, os intrigantes costumavam sorrir de maneira muito
peculiar, e isso certamente acontecia por causa de um obscuro boato,
segundo o qual Rabi Abraão teria de fato, durante sua estada na Alta
Escola de Toledo, se consagrado fervorosamente aos estudos da Lei
Divina, mas que também imitara costumes cristãos e incorporara
concepções dos livre"-pensadores, a exemplo daqueles judeus espanhóis
que se encontravam então num extraordinário nível cultural. No íntimo
da alma, porém, aqueles intrigantes acreditavam muito pouco na
veracidade do boato insinuado. Pois desde o seu retorno de Espanha, o
estilo de vida do Rabi era sobremaneira puro, piedoso e severo;
praticava os rituais mais insignificantes com atemorizado escrúpulo,
costumava jejuar todas as segundas e quintas"-feiras, somente no Sabá
ou nos outros dias de festa experimentava carne e vinho, seus dias
transcorriam em orações e estudos: durante o dia comentava a Lei Divina
no círculo dos alunos, atraídos a Bacherach pela fama de seu nome, e à
noite contemplava as estrelas no céu ou os olhos da bela Sara. O
matrimônio do Rabi era estéril; ao seu redor, contudo, não faltavam
vida e movimentação. O grande salão de sua casa, situada ao lado da
sinagoga, ficava aberto para uso de toda a comunidade: aqui as pessoas
saíam e entravam sem cerimônia, faziam orações ligeiras, ou vinham em
busca de novidades, ou realizavam assembleias em situações críticas;
aqui as crianças brincavam na manhã do Sabá, enquanto na sinagoga era
lido o trecho semanal; aqui as pessoas se reuniam em cortejos nupciais
e fúnebres, desentendiam"-se e reconciliavam"-se; aquele que
estivesse com frio encontrava aqui um fogão aquecido e o faminto, uma
mesa posta. Além disso, circulava em torno do Rabi uma multidão de
parentes, de irmãs e de irmãos de fé com suas mulheres e filhos, assim
como tios e tias comuns a ambos, a ele e a sua mulher: uma extensa
parentela cujos membros viam no Rabi a figura principal da família,
frequentavam sua casa desde manhã cedinho até tarde da noite e nos dias
de festa costumavam se reunir ao seu redor para cear juntos. Tais
refeições comunitárias na casa do rabino aconteciam de forma muito
especial por ocasião da comemoração anual do Pessach, uma antiquíssima
e maravilhosa festa que ainda hoje os judeus do mundo inteiro, em
eterna memória de sua libertação do cativeiro egípcio, celebram, na
véspera do décimo quarto dia do mês de
Nissan,\footnote{ Originalmente o Pessach consistia numa cerimônia de oferendas celebrada
por pastores e transformou"-se mais tarde na principal festa judaica
de recordação do cativeiro egípcio e do retorno à terra dos
antepassados. Nissan é o chamado mês da primavera, o primeiro no
ano judaico e estende"-se aproximadamente do final de março a final de
abril (Êxodo, 12). À festa do Pessach corresponde no cristianismo,
tanto na extensão temporal como em importantes traços ritualísticos, a
semana santa, desde o domingo de ramos até a Páscoa.}
da seguinte forma:

Tão logo chega a noite, a dona da casa acende as velas, estende a toalha
sobre a mesa, dispõe no seu centro três pães ázimos, de formato
achatado, cobre"-os com um guardanapo e sobre essa elevação coloca
seis pequenas bandejas contendo alimentos simbólicos, a saber, um ovo,
alface, raiz"-forte, um osso de cordeiro e mingau amarronzado feito de
uvas passas, canela e nozes. O chefe da casa senta"-se a essa mesa com
todos os parentes e companheiros e lê a estes trechos de um livro cheio
de aventuras chamado Hagadá, cujo conteúdo consiste numa insólita
miscelânea de sagas dos antepassados, histórias miraculosas do Egito,
narrativas curiosas, questões controversas, orações e canções solenes.
Uma grande refeição noturna insere"-se no centro dessa comemoração, e
até mesmo durante a leitura em voz alta as pessoas experimentam, nos
momentos determinados, pequenas porções dos pratos simbólicos, comendo
então pedacinhos de pão ázimo e bebendo de quatro taças de vinho tinto.
Melancolicamente sereno, grave e lúdico, feérico e misterioso é o
caráter dessa solenidade noturna, e o tom tradicionalmente melodioso
com que a Hagadá é lida pelo chefe da casa e de quando em quando
repetida em coro pelos ouvintes, soa de maneira tão arrepiante e
íntima, acalentando com doçura tão maternal e ao mesmo tempo
despertando com tamanho arrebatamento, que mesmo aqueles judeus que
desde muito tempo se extraviaram da fé dos antepassados e partiram em
busca de honrarias e prazeres alheios, são abalados no mais fundo do
coração quando os velhos e familiares acordes do Pessach penetram
casualmente em seus ouvidos.

Rabi Abraão encontrava"-se certa vez no grande salão de sua casa,
celebrando com seus parentes, alunos e demais convidados o ritual
noturno da festa do Pessach. No aposento tudo estava mais
resplandecente do que de costume: sobre a mesa estendia"-se uma toalha
de seda bordada em tons coloridos e cujas franjas douradas roçavam o
chão; os pratinhos com alimentos simbólicos brilhavam de maneira
aconchegante, e assim também as longas taças de vinho em relevo com
cenas da História Sagrada. Os homens estavam com os seus casacos
escuros, seus chapéus de abas largas, igualmente escuros, e traziam em
volta do pescoço gorjeiras brancas; as mulheres, em suas vestes de
tecido lombardo, que cintilavam maravilhosamente, ostentavam na cabeça
e no pescoço joias em ouro e pérolas; e o prateado candeeiro do Sabá
derramava sua luz solene sobre os rostos enlevados dos velhos e dos
jovens. Sentado sobre as almofadas em veludo purpúreo numa poltrona
mais elevada do que as outras e recostado, como ordena a tradição, Rabi
Abraão lia e entoava a Hagadá, enquanto o animado coro o acompanhava ou
respondia nos momentos prescritos. O Rabi trajava igualmente solenes
roupas escuras, seus traços fisionômicos, de contorno nobre, algo
severo, mostravam"-se mais suaves do que de costume, os lábios
desabrochavam num sorriso da barba marrom, como se quisessem narrar
fatos propícios, e os seus olhos transbordavam de venturosos
pressentimentos e recordações. A bela Sara, sentada ao seu lado numa
poltrona de veludo igualmente elevada, não ostentava, como anfitriã,
nenhuma de sua joias; apenas um linho alvo envolvia o seu corpo esbelto
e seu semblante piedoso. Esse semblante era comoventemente belo,
porquanto a beleza das judias costuma ser em geral de natureza
particularmente comovente: a consciência da profunda miséria, da amarga
humilhação e dos tristes perigos em que vivem seus parentes e amigos
espalha por sobre os delicados traços de seus rostos uma certa
interioridade sofredora e uma atenta angústia amorosa, as quais
encantam nosso coração de maneira especial. Assim apresentava"-se
nesse dia a bela Sara e fitava continuamente os olhos do marido; de
quando em quando olhava também para a Hagadá aberta diante de si, um
belo livro encadernado como pergaminho em ouro e veludo; esse exemplar,
com suas antiquíssimas manchas de vinho, era também uma herança dos
tempos de seu avô, e lá estavam aquelas muitas imagens, pintadas com
tanto colorido e vigor, que Sara, já como pequena menina, contemplava
tão prazerosamente na noite do Pessach, e que representavam as mais
variadas histórias bíblicas: como Abraão estilhaça com o martelo os
ídolos em pedra de seu pai; como os anjos acercam"-se de Abraão; como
Moisés golpeia mortalmente o
\textit{Mitzri};\footnote{
Isto é, o egípcio (\textit{Mitzri}, em hebraico) que Moisés mata por
estar maltratando um hebreu (Êxodo, 2---12).}
como o Faraó se senta luxuosamente no trono; como os sapos não o
deixam em paz nem sequer à mesa; como ele, graças a Deus, se afoga;
como os filhos de Israel atravessam cautelosamente o mar Vermelho e
como estes --- com seus bois, ovelhas e vacas --- quedam"-se boquiabertos
diante do monte Sinai; e, depois, também o rei Davi, tangendo a harpa
e, por fim, Jerusalém, com as torres e ameias de seu templo, sendo
iluminada pelo brilho do sol!

A segunda taça já havia sido servida, as fisionomias e as vozes iam
tornando"-se cada vez mais límpidas, e então o Rabi --- levantando um
dos pães ázimos com alegres saudações --- leu as seguintes palavras da
Hagadá: ``Vede! Este é o alimento que os nossos antepassados provaram no
Egito! Todo aquele que tiver fome, que venha e o prove! Todo aquele que
estiver triste, que venha e compartilhe de nossa alegria no Pessach! No
presente ano comemoramos a festa aqui, mas no ano vindouro na terra de
Israel! No presente ano nós a comemoramos ainda como escravos, mas no
ano vindouro, como filhos da liberdade!''

Nesse instante abriu"-se a porta do salão e entraram dois homens
grandes e pálidos, envoltos em capotes muito largos, e um deles falou:
``A paz esteja convosco, somos companheiros de fé em viagem e desejamos
comemorar convosco a festa do Pessach''. E o Rabi respondeu, de modo
ligeiro e amável: ``Convosco esteja a paz, sentai"-vos perto de mim''.
Os dois forasteiros sentaram"-se de imediato à mesa e o Rabi
prosseguiu na leitura em voz alta. Às vezes, enquanto os outros ainda
estavam no ato da repetição, ele lançava palavras carinhosas a sua
mulher; e, aludindo a um velho gracejo segundo o qual todo chefe de
família judeu se tem por rei nessa noite, disse a Sara: ``Alegra"-te,
minha rainha!'' Mas ela respondeu, sorrindo melancolicamente: ``É que nos
falta o príncipe!'', e com isso ela se referia ao filho da casa que,
como o exige uma passagem da Hagadá, deve interrogar seu pai, com
palavras prescritas, a respeito do significado da
festa.\footnote{ A
pergunta a ser feita pelo ``filho da casa'' (ou pelo participante mais
jovem da cerimônia) é a seguinte: ``Por que esta noite é diferente de
todas as outras noites?'' A resposta inicia"-se com as palavras: ``Em
todas as outras noites nós provamos alimento fermentado e não
fermentado, mas nesta noite apenas e tão somente não fermentado\ldots''}
 O Rabi não respondeu nada, limitando"-se apenas a apontar para uma
imagem sobremaneira graciosa, que acabara de ser aberta na Hagadá e na
qual se viam os três anjos acercando"-se de Abraão para anunciar que
lhe nasceria um filho de sua esposa Sara; esta, entretanto, postada à
entrada da tenda com sua astúcia feminina, escuta sorrateiramente a
conversa. Esse gesto silencioso derramou tonalidades rubras sobre as
faces da bela mulher, que baixou os olhos, mas depois voltou a fitar
carinhosamente o marido, que prosseguia na história 
maravilhosa: como certa vez na cidade de Bene Beraque, Rabi Joshua,
Rabi Eliezer, Rabi Elazar, Rabi Akiba e Rabi Tarfon conversaram a noite
inteira, recostados em suas cadeiras, sobre a retirada dos filhos de
Israel do Egito, até que os seus discípulos vieram anunciar"-lhes que
já era dia e que a grande oração matutina estava sendo lida na
sinagoga.

E enquanto a bela Sara ia ouvindo com devoção, os olhos sempre fixos no
marido, ela percebeu que sua fisionomia foi repentinamente desfigurada
por horrível rigidez, o sangue como que desapareceu de seus lábios e
faces e os olhos saltaram como farpas de gelo; --- mas, quase que no
mesmo instante, ela viu seus traços assumirem de novo a mesma
tranquilidade e alegria anteriores, lábios e faces recobrarem a cor, os
olhos se movimentarem satisfeitos; viu, também, como uma disposição
frenética, no mais totalmente estranha ao marido, apoderou"-se de todo
o seu ser. A bela Sara assustou"-se como jamais se assustara em toda
sua vida, e um pavor gélido levantou"-se em seu íntimo --- não tanto em
virtude das manifestações de horrorosa rigidez que por um momento
vislumbrara no rosto do marido como, muito mais, por causa do
contentamento posterior deste, que gradativamente ia se transformando
em tresloucada animação. O Rabi brincava com o seu barrete
empurrando"-o de uma orelha à outra, gracejava puxando e enrolando os
cachos de sua barba, cantava o texto da Hagadá como se fosse uma
modinha e, ao enumerar as pragas do Egito --- ocasião em que se deve
mergulhar várias vezes o indicador na taça de vinho e lançar ao chão as
gotas pendentes --- salpicou as moças com vinho tinto e assim surgiram
muitas gargalhadas e queixas por causa das gargantilhas manchadas. Toda
essa jovialidade efervescente e convulsiva do marido tornava"-se cada
vez mais estranha à bela Sara; angustiada por inominável apreensão, ela
olhava para o movimentado burburinho formado por todas aquelas pessoas
que oscilavam rejubilantes para lá e para cá, mordiscavam os delgados
pães do Pessach, ou bebericavam das taças de vinho, ou tagarelavam
entre si, ou ainda cantavam em voz alta --- todos sobremaneira
satisfeitos.

Chegou então o momento em que o jantar é servido. Todos se levantaram
para se lavar e a bela Sara foi buscar o grande lavatório prateado e
luxuosamente adornado com relevos trabalhados a ouro, oferecendo"-o 
sucessivamente a cada um dos convivas em cujas mãos se despejava água.
Enquanto prestava esse serviço também ao Rabi, este lhe piscou
significativamente com os olhos e esgueirou"-se porta afora. A bela
Sara foi em seu encalço; o Rabi agarrou incontinenti a mão de sua
mulher e, arrastando"-a com pressa pelas escuras vielas de Bacherach,
transpôs o portão e, sempre apressado, alcançou a estrada que conduz
até Bingen acompanhando o curso do Reno.

Era uma dessas noites primaveris que, embora bastante cálidas e
estreladas, enchem nossa alma de insólitos tremores. As flores exalavam
um odor funéreo; maliciosos e atemorizados ao mesmo tempo, os pássaros
chilreavam; a lua projetava traiçoeiros reflexos amarelos sobre a
torrente envolta num murmúrio sombrio; os elevados rochedos da margem
assemelhavam"-se a cabeças de gigantes oscilando ameaçadoramente; a
sentinela da fortaleza de Strahleck soprava em seu melancólico
instrumento, e em meio a tudo isso soou o pequeno sino fúnebre da
igreja de São Werner, estridente e zeloso. A bela Sara trazia na mão
direita o lavatório prateado, pela esquerda o Rabi ainda a segurava e
ela sentia como os seus dedos estavam gélidos e como o seu braço
tremia. Mas ela o seguia em silêncio, talvez porque já estivesse
acostumada desde muito tempo a obedecer cega e incondicionalmente ao
marido, talvez também porque os seus lábios estivessem cerrados por um
medo íntimo.

Abaixo da fortaleza de Sonneck, defronte a Lorch, mais ou menos onde
fica hoje a pequena aldeia de Niederrheinbach, há um maciço rochoso que
se debruça recurvado sobre a margem do Reno. O Rabi galgou esse rochedo
com sua mulher, olhou em todas as direções e depois cravou os olhos nas
estrelas. Trêmula e trespassada por calafrios advindos de mortais
angústias, a bela Sara quedava"-se ao lado do marido e contemplava seu
pálido rosto que, iluminado fantasmagoricamente pelo luar,
contraía"-se de quando em quando em expressão de sofrimento, terror,
devoção e ira. Mas quando o Rabi, num gesto brusco, arrebatou de sua
mão o lavatório prateado e o lançou ao Reno, levantando"-se um som
abafado, ela não pôde suportar por mais tempo aquela angústia horrorosa
e, exclamando ``\textit{Schadai}
misericordioso!'',\footnote{
Em seus estudos de história judaica, Heine anotou uma série de rituais
de invocação, especialmente no contexto da Cabala. Por meio de
oferendas e palavras mágicas procurava"-se aplacar forças naturais,
numa espécie de cerimônia mágica de defesa. \textit{Schadai} significa
em hebraico ``Todo"-poderoso'' e sua utilização remonta, segundo Lion
Feuchtwanger, a antigos escritos judaico"-alemães. Contudo, a oferenda
do lavatório prateado é comentada por Feuchtwanger como ato ``não
judaico'', lembrando antes a tradição grega, como se manifesta, por
exemplo, na balada de Schiller ``O anel de Polícrates''.}
caiu aos pés do marido, suplicando"-lhe que desvendasse por fim o
obscuro enigma.

O Rabi, privado de voz, moveu várias vezes os lábios mudos e disse por
fim: ``Estás vendo o anjo da morte? Lá embaixo, ele paira sobre
Bacherach! Mas nós escapamos à sua espada. Louvado seja o Senhor!'' E
com voz ainda trêmula de aversão íntima, contou que, enquanto declamava
a Hagadá, recostado na poltrona e de ânimo sereno, olhou casualmente
debaixo da mesa e lá avistou a seus pés um ensanguentado cadáver de
criança. ``Percebi então'' --- acrescentou o Rabi --- ``que as nossas duas
visitas tardias não eram da comunidade de Israel, mas sim do
ajuntamento dos ímpios, que combinaram introduzir sorrateiramente
aquele cadáver em nossa casa para nos acusar de infanticídio e incitar
o povo a nos saquear e assassinar. Não pude deixar transparecer que
tinha percebido essa obra das trevas, pois desse modo teria
tão"-somente apressado minha desgraça, e apenas a astúcia nos salvou a
ambos. Louvado seja o Senhor! Nada temas, bela Sara, também os nossos
amigos e parentes serão salvos. Os malignos só estavam sedentos de meu
sangue; eu escapei de suas mãos e eles contentar"-se"-ão com a minha
prata e o meu ouro. Vem comigo, bela Sara. Vamos para uma outra terra e
deixemos a desgraça para trás; e para que ela não mais nos persiga,
lancei"-lhe como conciliação o último de meus pertences, a bacia
prateada. O Deus de nossos antepassados não nos abandonará. --- Desçamos,
bela Sara, estás cansada. Lá embaixo está o silencioso Wilhelm junto ao
seu barco; ele nos conduzirá Reno acima''.

Sem emitir som algum e com os membros como que alquebrados, a bela Sara
deixou"-se cair nos braços do Rabi, que lentamente a levou para a
margem abaixo. Lá estava o silencioso Wilhelm --- um rapaz surdo"-mudo,
mas extraordinariamente belo, o qual exercia a pesca e mantinha o seu
barco ancorado nesse local, para sustento de sua velha mãe de criação,
uma vizinha do Rabi. Mas foi como se ele tivesse adivinhado de imediato
a intenção do Rabi; sim, parecia mesmo que estava esperando por ele. Em
torno de seus lábios cerrados desenhou"-se a mais amorosa compaixão,
seus grandes olhos azuis repousaram profundamente significativos sobre
a bela Sara e, desvelando"-se em cuidados, ele a conduziu ao barco.

O olhar do rapaz mudo despertou a bela Sara de seu torpor; ela sentiu de
repente que tudo aquilo que o marido lhe contara não era um mero sonho
e torrentes de amargas lágrimas escorreram por suas faces, agora tão
alvas como suas vestes. Sentada no centro do barco, era uma dolorosa
imagem marmórea; postados ao seu lado, o marido e o silencioso Wilhelm
iam remando energicamente.

Seja pelos golpes uniformes do remo, seja pelo balanço da embarcação, ou
pelo odor daquelas margens montanhosas onde viceja a alegria, mesmo a
pessoa mais angustiada é serenada de maneira especial ao deslizar
suavemente, numa noite primaveril e num leve barco, sobre a querida e
límpida correnteza renana. Verdadeiramente, o velho e bondoso pai Reno
não pode suportar que os seus filhos chorem; aplacando as lágrimas, ele
os acalenta em seus braços fiéis e conta"-lhes suas mais belas
histórias encantadas e promete"-lhes suas riquezas mais douradas,
talvez até mesmo o submerso e antiquíssimo tesouro dos Nibelungos.
Também as lágrimas da bela Sara escorriam cada vez mais suaves, suas
dores mais atrozes eram levadas pelas ondas sussurrantes, a noite
perdia o seu aspecto tenebroso e as montanhas pátrias saudavam acenando
com o mais carinhoso adeus. Antes de todas, porém, saudou"-a
intimamente sua montanha predileta, a
Kedrich,\footnote{ Kedrich
ou Kädrich (em grafia mais correta) fica à margem direita do Reno, nas
imediações de Bacharach. A lenda que assoma fragmentariamente na
imaginação de Sara, envolvendo a moça raptada, os anões e o intrépido
cavaleiro, foi retirada do manual de Schreiber sobre o Reno. Dessa
mesma obra provém também a lenda associada ao Vale dos Rumores
(\textit{Wisperthale}, como escreve Heine).}
e, iluminada por estranho luar, era como se lá em cima estivesse
novamente uma mocinha com os braços medrosamente estendidos, como se
uma multidão de ágeis anõezinhos saísse rastejando de suas frestas
rochosas e como se um cavaleiro viesse a pleno galope vencendo a
montanha. E no íntimo da bela Sara era como se ela fosse novamente uma
pequena menina e estivesse sentada no colo de sua tia de Lorch e esta
lhe contasse a história do intrépido cavaleiro que liberta a pobre
mocinha raptada pelos anões; e ainda outras histórias verdadeiras, a do
estranho Vale dos Rumores, que ficava mais adiante e onde os pássaros
conversavam racionalmente entre si; e a da terra das broas natalinas,
para onde vão as crianças obedientes; histórias de princesas
enfeitiçadas, árvores cantantes, castelos de vidro, pontes douradas,
ninfas sorridentes\ldots\ Mas, em meio a todos esses contos maravilhosos
que, brilhantes e melodiosos, começavam a viver, a bela Sara ouviu a
voz de seu pai, que ralhava irritado com a pobre tia porque enfiava
tantas tolices na cabeça da menina! Em seguida se lhe representou que a
sentavam num banquinho diante da poltrona aveludada de seu pai e este
acariciava com aquela mão macia o longo cabelo da filha, sorria
satisfeito com os olhos e balançava prazerosamente para lá e para cá,
envolto em seu largo camisolão de sabá feito de seda azul\ldots\ Certamente
devia ser sabá, pois a toalha florida estava estendida sobre a mesa,
todos os utensílios no cômodo tinham sido areados e resplandeciam como
espelhos, o administrador da comunidade, sentado com sua barba branca
ao lado do pai de Sara, mastigava uvas passas e falava em hebraico;
também o pequeno Abraão entrou com um livro de tamanho descomunal e
modestamente pediu permissão ao tio para comentar um trecho das
Sagradas Escrituras, pois queria que o próprio tio se convencesse de
que ele tinha estudado muito na semana passada e merecia agora muitos
elogios e doces\ldots\ Então, o pequeno menino colocou o livro sobre o
braço da poltrona e comentou a história de Jacó e Raquel: como Jacó
ergueu a voz e caiu em prantos quando avistou sua priminha Raquel pela
primeira vez; como Jacó falou"-lhe confidencialmente junto à fonte;
como depois teve de servir sete anos por Raquel e como esses anos lhe
passaram tão depressa; e como ele desposou por fim Raquel e sempre e
sempre a amou\ldots\ Subitamente, a bela Sara lembrou"-se ainda de que seu
pai exclamara em tom divertido: ``Será que também tu não irás querer
desposar tua prima Sara?'', e a isso respondeu seriamente o pequeno
Abraão: ``É o que desejo, e ela terá de esperar sete anos''. Envoltas em
luz crepuscular, essas imagens foram passando pela alma da bela mulher
--- ela se via brincando infantilmente com o pequeno primo, que agora era
tão grande e se tornara seu marido, na cabana de
folhagens;\footnote{ Essa
cabana, enfeitada com ramos, folhas e frutas, está associada a uma
espécie de festa da colheita que se comemora em setembro ou outubro
como lembrança da proteção divina durante a peregrinação pelo deserto
(Festa dos Tabernáculos ou Festa das Cabanas; \textit{Sucot}, em
hebraico). No início do segundo capítulo o narrador dirá que as ``belas
murtas'' e ramos utilizados em Bacherach vinham da região de
Sachensausen em Frankfurt. (Sachsenhausen é também o nome de um campo
de concentração construído nas imediações de Berlim em 1936; a
coincidência dos nomes é um dos elementos que levaram Günter Grass ao
projeto, comentado na introdução a este volume, de transpor a história
de Rabi Abraão e Sara para os anos trinta do século passado.)}
e como eles se deleitavam com as tapeçarias coloridas, os espelhos, as
flores e maçãs banhadas a ouro; como o pequeno Abraão sempre lhe fizera
carinhos, até que pouco a pouco foi crescendo e se tornando rabugento,
tornando"-se por fim tão grande e tão rabugento\ldots\ E finalmente, ela
está sozinha em casa num sábado à noite, o luar claro penetra pela
janela de seu quarto; de repente a porta se escancara e o seu primo
Abraão entra como um raio, em trajes de viagem e pálido como a morte. E
ele agarra sua mão, enfia um anel de ouro em seu dedo e pronuncia
solenemente: ``Com isso, tomo"-te por minha esposa, segundo os costumes
de Moisés e de Israel! Mas agora'' --- acrescenta trêmulo --- ``agora preciso
partir para a Espanha. Adeus, sete anos terás de esperar por mim!'' E
ele parte precipitado; chorando, a bela Sara conta tudo isso a seu
pai\ldots\ Ele se encoleriza e vocifera: ``Vai cortar o cabelo, pois és uma
mulher casada!''\footnote{ Pelos costumes judaicos mais rigorosos, uma mulher 
casada deveria pelo menos trazer os cabelos cobertos.}
 --- e ele quer sair ao encalço de Abraão para arrancar"-lhe uma carta de
separação, mas este já transpôs todas as montanhas. O pai volta
silencioso para casa e enquanto a bela Sara o ajuda a descalçar as
botas de cavalgar e pondera suavemente que em sete anos Abraão estará
de volta, ele amaldiçoa: ``Por sete anos tereis de mendigar'', e pouco
depois ele morre.

E assim as velhas histórias foram desfilando diante da bela Sara como um
ligeiro jogo de sombras. As imagens também se entrelaçavam
estranhamente e de quando em quando espreitavam rostos barbudos, em
parte familiares, em parte desconhecidos, e surgiam enormes flores com
folhagens fabulosamente desenvolvidas. Era também como se o Reno
murmurasse as melodias da Hagadá e as imagens desta assomassem das
águas, em tamanho natural e desfiguradas, imagens insensatas: o
patriarca Abraão esfacela amedrontado as figuras de ídolos que a cada
vez se recompõem rapidamente; o Mitzri defende"-se encarniçadamente do
enfurecido Moisés; o Monte Sinai relampeja e arde em chamas; o
Rei"-Faraó nada no Mar Vermelho com a pontiaguda coroa de ouro na
boca, presa entre os dentes; sapos com fisionomias humanas vão nadando
atrás dele; as ondas espumam e rumorejam, e uma escura mão de gigante
levanta"-se ameaçadoramente da torrente.

Era a assim chamada Torre dos Ratos de
Hatto,\footnote{ As histórias em torno dessa torre construída no século 
\textsc{xiii} em uma ilha próxima à cidade de Bingen (e que hoje funciona
como farol para a navegação) são também minuciosamente relatadas por Schreiber. Segundo
uma das lendas, o desapiedado arcebispo da Mogúncia (Hatto) teria
negado ajuda aos pobres durante um período de fome e, por isso, os
ratos o devoraram vivo nessa torre.}
e o barco acabara justamente de passar pelo redemoinho de Bingen. Isso
arrancou a bela Sara de seus devaneios e ela fixou os olhos nas
montanhas da margem, em cujos cumes brilhava a iluminação dos castelos
enquanto mais abaixo vagava a névoa da madrugada, iluminada pelo luar.
Mas, de repente, ela acreditou vislumbrar ali os seus amigos e
parentes, que com terrível rapidez desfilavam ao longo do Reno, com
rostos cadavéricos e envoltos em tremulantes camisas mortuárias. A
vista se lhe escureceu, uma torrente gelada derramou"-se por sua alma
e, como em sonho, ela ainda ouviu como o Rabi lhe entoava a oração
noturna, com aquela lentidão angustiada que se deve observar em casos
de moribundos; e imersa em devaneios ela ainda balbuciou as palavras:
``Dez mil à sua direita, dez mil à sua esquerda; para proteger o rei dos
temores
noturnos\ldots''\footnote{ Nessas palavras balbuciadas por Sara, Heine associa livremente duas
passagens da oração noturna judaica. Uma delas é tomada ao ``Cântico dos
Cânticos'': ``É a liteira de Salomão!/ Sessenta soldados a escoltam,/
soldados seletos de todo Israel./ São todos treinados na espada,/
provados em muitas batalhas./ Vêm todos cingidos de espada,/ temendo
surpresas noturnas'' (3---7,8). A outra passagem da oração noturna diz,
na versão apresentada por Feuchtwanger: ``Que à direita daquele que
dorme monte guarda o arcanjo Miguel e à esquerda, o arcanjo Gabriel''.}


E então dissiparam"-se repentinamente todos os horrores e toda aquela
escuridão insinuante; a sombria cortina foi rasgada no céu e nas
alturas surgiu a cidade sagrada de Jerusalém, com suas torres e
portais. Em dourado esplendor brilhou o templo; no pátio a bela Sara
avistou o pai, em seu camisolão amarelo de sabá e rindo satisfeito com
os olhos; das janelas circulares do templo saudaram"-na alegremente
todos os seus amigos e parentes; no santuário ajoelhava"-se o piedoso
rei Davi com o manto purpúreo e a coroa reluzente, fazendo ressoar
ternamente o seu canto e a melodia da harpa --- e esboçando um sorriso de
bem"-aventurança, a bela Sara adormeceu.

\chapter{Segundo capítulo}

\textsc{Quando} a bela Sara abriu os olhos, foi levemente ofuscada pelos raios
solares. Assomaram as altas torres de uma grande cidade, e o silencioso
Wilhelm, em pé e com o remo na mão, ia conduzindo o barco em meio a
divertido tumulto formado por muitos navios embandeirados com todas as
cores, cujas tripulações contemplavam ociosas a movimentação abaixo ou
estavam inteiramente ocupadas com a descarga de caixas, rolos e barris
que eram transportados às margens em embarcações menores. E as
ininterruptas advertências dos barqueiros, a gritaria dos comerciantes
postados na margem, os berros dos funcionários alfandegários que, em
suas casacas vermelhas, com seus bastões brancos e rostos alvos, iam
saltando de navio em navio --- tudo isso produzia um barulho atordoante.

``Sim, bela Sara'' --- disse o Rabi a sua mulher, sorrindo serenamente ---
``esta aqui é a livre e imperial Frankfurt"-sobre"-o"-Meno, cidade de
comércio mundialmente famosa, e este rio sobre o qual estamos navegando
é exatamente o Meno. E ali adiante, aquelas casas risonhas cercadas por
colinas verdes, é a região de Sachsenhausen, de onde o paralítico
Gumpertz, na época da festa das cabanas, nos traz as belas murtas. Aqui
vês a sólida ponte do Meno, com os seus treze arcos, e todas essas
pessoas, todos esses veículos e cavalos passam com segurança sobre ela;
e lá no meio está a casinha da qual dizia Mühmele Täubchen ser habitada
por um judeu batizado, o qual paga seis moedas de prata a todo aquele
que lhe traz uma ratazana morta, auxiliando desse modo a comunidade
judaica a cumprir a obrigação de entregar anualmente cinco mil caudas
de ratazana ao conselho da
cidade''.\footnote{ Esta
punição coletiva, relatada em detalhes por Anton Kirchner em sua
\textit{Geschichte der Stadt Frankfurt am Main} [História da cidade
Frankfurt sobre o Meno], foi imposta em decorrência da participação de
um judeu disfarçado em um torneio cavaleiresco. Heine, no entanto,
aumenta o valor da recompensa por cauda de ratazana e fixa livremente o
seu número em cinco mil.}


A bela Sara foi compelida a rir alto dessa guerra que os judeus de
Frankfurt são obrigados a travar com as ratazanas; a diáfana luz solar
e o mundo novo e colorido que se levantava à sua frente haviam
dissipado de sua alma todo o horror e o medo da noite anterior, e
quando o marido e o silencioso Wilhelm a desembarcaram na margem, ela
sentiu"-se invadida por alegre segurança. Mas o silencioso Wilhelm,
com os seus belos olhos de um azul profundo e uma expressão dividida
entre sofrimento e serenidade, contemplou"-lhe longamente o rosto; em
seguida, dirigiu ainda um olhar significativo ao Rabi, saltou de volta
ao barco e pouco depois já havia desaparecido.

``O silencioso Wilhelm tem de fato muitas semelhanças com o meu falecido
irmão'', observou a bela Sara. ``Os anjos são todos muito parecidos'' ---
respondeu ligeiramente o Rabi e, tomando sua mulher pela mão, foi
conduzindo"-a por entre o burburinho humano da margem, onde --- por ser
a época da feira pascoal --- estavam armadas muitas barracas de madeira.
Quando entraram na cidade pela escura porta do Meno, depararam"-se com
uma movimentação não menos barulhenta. Ali, numa rua estreita,
levantava"-se uma loja comercial ao lado da outra e as casas
mostravam"-se, como por toda a parte em Frankfurt, especialmente
voltadas para o comércio: no térreo não havia janelas, mas apenas
portas abobadadas, todas abertas de forma a permitir que a visão
alcançasse até o fundo da casa e assim os transeuntes podiam apreciar
nitidamente as mercadorias expostas. Quão admirada ficou a bela Sara
com a quantidade de objetos preciosos e seu esplendor jamais visto! Lá
estavam os venezianos, que ofereciam todo o luxo do oriente e da
Itália, e a bela Sara ficou como que enfeitiçada ao avistar os adornos
e joias empilhados, gorros e corpetes coloridos, braceletes e colares
dourados --- todos aqueles enfeites reluzentes que as mulheres tanto
gostam de admirar e mais ainda de ostentar. Os tecidos de veludo e
seda, ricamente bordados, pareciam querer falar com a bela Sara e fazer
cintilar de novo em sua memória as coisas mais maravilhosas; e em seu
íntimo era como se ela fosse novamente uma pequena menina e Mühmele
Täubchen tivesse cumprido a promessa de levá"-la à feira de Frankfurt
e, de repente, estivesse diante dos belos vestidos de que tanto ouvira
falar. Com secreta alegria, ela já pensava no que levaria para
Bacherach, qual de suas duas priminhas, a pequena Blümchen ou a pequena
Vögelchen, iria gostar mais do cinto em seda azul; pensava também se as
calcinhas verdes serviriam no pequeno Gottschalk, --- subitamente, porém,
disse a si mesma: ``Oh, meu Deus! Se nesses anos as crianças já
cresceram e ontem foram assassinadas!'' Contraiu"-se violentamente e as
imagens da noite já ameaçavam levantar"-se dentro dela com todo o seu
horror; mas os vestidos bordados a ouro piscavam"-lhe com os seus
milhares de olhos vivazes, expulsando de sua mente os pensamentos
sombrios. E, ao levantar a vista para o rosto do marido, este se
mostrava desanuviado e expressava sua habitual suavidade austera.
``Fecha os olhos, bela Sara'' --- disse o Rabi, e continuou a conduzir sua
mulher em meio à multidão de pessoas.

E que agitação mais rutilante! Eram, sobretudo, os comerciantes que
barganhavam entre si em voz alta, falavam sozinhos enquanto faziam
contas com os dedos, ou então estavam voltando para as respectivas
hospedarias, seguidos por ajudantes carregados com quantidades imensas
de pacotes e que iam trotando a passos curtos. Outros rostos deixavam
transparecer que tinham sido atraídos até ali apenas pela curiosidade.
Pela manta vermelha e pelo colar dourado podia"-se reconhecer o
corpulento membro do conselho. O gibão negro, luxuosamente confortável,
denunciava o venerável e orgulhoso patrício. O morrião de ferro, o
gibão de couro amarelo e as esporas ruidosas anunciavam o pesado
escudeiro. Sob uma touca de veludo negro, que descia pontiaguda até a
altura da testa, ocultava"-se um róseo rosto feminino, e os moços
solteiros, farejando"-a como cães de caça, passavam por perfeitos
janotas com seus alegres barretes emplumados, seus tilintantes sapatos
de bico fino e comprido, suas roupas em seda bicolores: sendo o lado
direito verde e o lado esquerdo vermelho, ou um lado listrado como o
arco"-íris e o outro com estampas coloridas, os tolos rapazes pareciam
estar cindidos ao meio. Arrastados pela torrente humana, o Rabi e sua
mulher chegaram ao Romano. Este nome --- que designa a ampla praça do
mercado, cercada por casas com frontões elevados --- provém de uma
gigantesca casa chamada ``Ao Romano'', comprada mais tarde pelo corpo de
magistrados e convertida em conselho da cidade. Nesse edifício
elegia"-se o imperador da Alemanha e ali em frente celebravam"-se com
frequência nobres torneios entre cavaleiros. O rei Maximiliano, que
tinha paixão por coisas desse tipo, encontrava"-se então em Frankfurt
e em sua homenagem organizara"-se alguns dias antes um grande duelo
diante do Romano.\footnote{
Fontes históricas consultadas por Heine indicam que Maximiliano \textsc{i}, rei
desde o ano de 1486 e coroado imperador alemão em 1493, esteve em
Frankfurt em 1489, todavia em meados de junho e não na época da Páscoa.
De qualquer modo, Heine estabelece assim um ponto de referência para a
localização temporal da narrativa.}
Junto às liças de madeira, que estavam sendo desmontadas pelos
carpinteiros, havia ainda muitos ociosos que contavam como o duque de
Braunschweig e o margrave de Brandemburgo investiram, no dia anterior,
um contra o outro ao som de tambores e trompetes; como o senhor Walter,
o Vilão, arrancara o Cavaleiro do Urso com tal violência da sela que
estilhaços de lança voaram pelo ar; contavam, ainda, como o alto e
loiro rei Max, apreciando o espetáculo do balcão e rodeado por sua
corte, esfregara as mãos de contentamento. As tapeçarias douradas ainda
guarneciam o encosto do balcão e das pontiagudas janelas do conselho.
Também as outras casas da praça do mercado apresentavam um aspecto
solene e estavam enfeitadas com brasões, especialmente a casa Limburg,
em cujo estandarte se via uma donzela ostentando um gavião no braço,
enquanto à sua frente um macaco segura um espelho. No balcão dessa
casa, reunidos em alegre conversação, muitos cavaleiros e damas
contemplavam o povo que lá embaixo deslizava como que dividido em
procissões e grupos fantásticos. Que quantidade de ociosos se
acotovelando ali para saciar a curiosidade! Ali as pessoas riam,
queixavam"-se, furtavam e se beliscavam mutuamente nas ancas,
festejavam; e, em meio a tudo isso, ressoava estridente a trombeta do
médico que --- sobre uma plataforma, envolto numa capa vermelha e
acompanhado de seu macaco e de seu bobo --- anunciava de maneira muito
peculiar suas habilidades, enaltecia suas tinturas e pomadas milagrosas
ou então contemplava com expressão compenetrada o vidro com urina que
alguma velha lhe mostrava, ou mesmo se preparava para arrancar o molar
de um pobre camponês. Dois mestres da esgrima, esvoaçantes em suas
faixas e cintos coloridos e fazendo zunir seus floretes,
encontravam"-se ali como que por acaso e lançavam"-se um contra o
outro tomados por uma cólera fingida; após um longo duelo,
declaravam"-se ambos imbatíveis e recolhiam algumas moedas. Com
tambores e pífaros passou marchando a recém"-fundada corporação dos
caçadores. Seguiu"-se a esta, conduzida pelo carcereiro que empunhava
uma bandeira vermelha, uma tropa de moças itinerantes: vinham da casa
de mulheres ``Ao Asno'', na cidade de Würzburgo, e dirigiam"-se ao Vale
das Rosas, onde a digníssima autoridade lhes designara alojamento para
o período da feira. ``Fecha os olhos, bela Sara'', disse o Rabi. Pois
aquelas raparigas estranhas e sumariamente vestidas, entre as quais
havia algumas muito bonitas, comportavam"-se de maneira a mais
obscena, desnudavam o peito alvo e insolente, provocavam os transeuntes
com palavras indecorosas, rodopiavam no ar os compridos bastões que
levavam em suas andanças e, ao descerem a rua na direção do Portal de
Santa Catarina, cavalgando agora os bastões como se fossem cavalinhos
de pau, passaram a entoar a canção das bruxas com voz estridente:

\begin{verse}
``Cadê o cabrão, aquela besta infernal?\\
Cadê o cabrão? E se não vem o cabrão,\\
Então vamos cavalgar, vamos cavalgar,\\
Então vamos cavalgar no bastão!''\footnote{ No
original essa ``canção das bruxas'' diz: ``\textit{{Wo ist der Bock, das
Höllenthier?/ Wo ist der Bock? Und fehlt der Bock,/ So reiten
wir, so reiten wir,/ So reiten wir auf dem Stock!}}'' Não parece haver um
modelo direto para essa canção, mas Heine faz ecoar aqui versos
pronunciados por bruxos e bruxas na ``Noite de Valpúrgis'' do
\textit{Fausto} goethiano, por exemplo: ``Levar"-te, hoje, a vassoura
pode,/ Leva a forquilha e leva o bode;/ Quem hoje não puder subir,/
Perdido está para o porvir'' (versos 4.000---4.003). Citado segundo a
tradução de Jenny Klabin Segall: \textit{Fausto }--- \textit{Primeira
Parte}. Editora 34, 2004.}
\end{verse}

Essa ladainha, que ainda se pôde ouvir de longe, dissolveu"-se por fim
nos acordes arrastados de uma procissão religiosa que se aproximava.
Era um triste cortejo de monges calvos e descalços, os quais conduziam
velas acesas, estandartes com imagens de santos ou ainda grandes
crucifixos prateados. À frente, portando os incensórios fumegantes,
marchavam rapazes com casacas vermelhas e brancas. No meio do cortejo,
sob um suntuoso baldaquino, viam"-se eclesiásticos trajando alvas
finamente rendadas ou estolas de seda em várias cores, e um deles
levava nas mãos um vaso dourado, que brilhava como o sol; e quando o
cortejo alcançou o nicho sagrado, situado a um canto da praça, aquele
levantou o vaso, ao mesmo tempo que cantava e declamava palavras em
latim\ldots\ Nesse instante soou um pequeno sino e todo o povo em redor
emudeceu, pôs"-se de joelhos e persignou"-se. O Rabi, no entanto,
disse a sua mulher: ``Fecha os olhos, bela Sara!'' --- e rapidamente a
puxou para uma estreita viela lateral, depois passaram por um labirinto
de ruas apertadas e sinuosas até que por fim atravessaram aquela praça
erma e abandonada que separava o novo bairro judeu do resto da cidade.

Antigamente os judeus viviam entre a catedral e a margem do Meno, mais
precisamente da ponte até a Fonte dos Andrajos e da Balança da Farinha
até São Bartolomeu. Mas os sacerdotes católicos conseguiram uma bula
papal que vedou aos judeus viver em tal proximidade da igreja
principal, e assim o conselho dos magistrados designou"-lhes um lugar
no Fosso da Lã, onde eles edificaram o atual bairro judeu. Este foi
cercado por muralhas e os portões reforçados com correntes de ferro, a
fim de resguardá"-los do assédio do populacho. Pois ali os judeus
viviam igualmente sob opressão e medo, e a memória de aflições antigas
estava mais vívida do que nos dias de hoje. No ano de 1240, o populacho
desvairado provocou um grande banho de sangue entre eles, o qual se
chamou a primeira batalha dos judeus; e no ano de 1349, quando os
flagelantes --- em suas andanças pelo país --- incendiaram a cidade e
colocaram a culpa nos judeus, estes foram em grande parte assassinados
pelo povo açulado ou encontraram a morte nas chamas de suas próprias
casas ­--- e a isso se chamou a segunda batalha dos
judeus.\footnote{ No ano
de 1349 o movimento de flagelantes sofreu, em decorrência da peste
negra, significativo aumento na Alemanha. Sobre as duas datas
mencionadas por Heine nesse segmento da narrativa, escreve Kirchner:
``Para os judeus, os anos de 1240 e 1349 permanecerão sempre como uma
triste lembrança. Mas também antes e depois dessas cenas de terror, a
sua vida mostra"-se insegura, incerta a sua situação, instável o seu
patrimônio. A cada ruído, eles se contraem de medo, a cada boato,
aguçam os ouvidos com angústia''.}
 Depois disso, eles voltaram a ser ameaçados com batalhas desse tipo, e
nos períodos de agitação interna em Frankfurt, especialmente por
ocasião de uma disputa entre o conselho e as corporações, o populacho
cristão esteve várias vezes a ponto de invadir o bairro judeu. Este
tinha dois portões, que eram fechados por fora durante os feriados
católicos e por dentro nos feriados judeus, e na frente da cada portão
havia um posto de vigia com soldados da cidade.

Quando o Rabi chegou com sua mulher ao portão do bairro judeu, os
lansquenetes --- como se podia ver através das janelas abertas --- estavam
deitados sobre as tarimbas da torre de vigia; do lado de fora, defronte
ao portão, o tocador de tambor improvisava em pleno sol com seu enorme
instrumento. Era uma figura pesada e vultosa. Gibão e calças em tecido
amarelo reluzente, com enchimento nos braços e nas ancas, pontilhado de
cima a baixo com pequenas pregas vermelhas, como que a mostrar
incontáveis línguas humanas. Peito e costas encouraçados com um estofo
negro, do qual pendia o tambor; sobre a cabeça havia um gorro achatado,
redondo e negro. O rosto era igualmente achatado e redondo, de uma cor
amarelo"-alaranjada, salpicado por pequenas úlceras vermelhas e
contraído em expressão de sorriso e bocejo. Ali sentado, esse sujeito
ia tamborilando a melodia da canção entoada outrora pelos flagelantes
durante a batalha dos judeus, e com a sua voz de cerveja, em tom
áspero, ia gargarejando as palavras:

\begin{verse} 
``A nossa amada Senhora\\
Saiu pelo orvalho da manhã,\\
Kyrie eleison!''\footnote{
Os versos ``gargarejados'' por essa personagem foram retirados de um
texto reproduzido na chamada \textit{Limburger Chronik }[Crônica de
Limburg]. Trata"-se de uma canção em louvor de Maria que se inicia do
seguinte modo: ``Saiu a nossa senhora, Kyrieleison./ Pelo orvalho da
manhã, Aleluia./ Encontrou"-a então um rapaz, Kyrieleison/ Sua barba
já estava despontando, Aleluia./ Bendita sê tu, Maria''.}
\end{verse} 

``Hans, esta é uma melodia ruim!'' --- exclamou uma voz atrás do portão
cerrado do bairro judeu --- ``Hans, é também uma canção ruim, não combina
com o tambor, não combina com nada e muito menos na missa e na manhã de
Páscoa; canção ruim, canção perigosa, Hans, pequeno Hans, pequenino
Hans do tambor, eu sou uma pessoa sozinha e se tu gostas de mim, se
gostas do Stern, do comprido Stern, do comprido Nasenstern, então
para!''

Estas palavras foram pronunciadas pela pessoa que não se via em parte
com rapidez angustiada e em parte com lentidão suspirante, num tom em
que a brandura arrastada se alternava bruscamente com a dureza rouca,
tal como observamos nos tísicos. Mas o tocador de tambor permaneceu
impassível e, desdobrando em seu instrumento a melodia anterior,
continuou a cantar:

\begin{verse} 
``Veio então um pequeno rapaz\\
Cuja barba já estava despontando,\\
Aleluia''.
\end{verse} 

``Hans'' --- exclamou novamente a voz da pessoa acima mencionada --- ``Hans, eu
sou uma pessoa sozinha, e esta é uma canção perigosa, e eu não gosto de
ouvi"-la e tenho os meus motivos e se tu gostas de mim, então canta
outra coisa e amanhã vamos beber\ldots''

Ao ouvir a palavra ``beber'', aquele Hans deixou de tamborilar e cantar,
dizendo então num tom que afetava honradez: ``Que o diabo carregue os
judeus, mas tu, caro Nasenstern, tu és meu amigo, eu te protejo e se
continuarmos a beber juntos, ainda vou converter"-te. Quero ser o teu
padrinho; se fores batizado, irás para o céu, e se tiveres talento e
empenhar"-te em aprender comigo, poderás até tornar"-te um tocador de
tambor. Sim, Nasenstern, tu ainda podes chegar longe; e se formos beber
juntos amanhã, o meu tambor haverá de enfiar todo o catecismo em tua
cabeça\ldots\ Mas agora abre o portão, pois aqui estão dois estranhos que
desejam entrar''.

``Abrir o portão?'' --- gritou aquele Nasenstern, e quase que a voz não lhe
saiu. ``Não dá para ser assim tão depressa, caro Hans, pois nunca se
sabe, nunca se sabe\ldots\ e eu sou uma pessoa sozinha. O Veitel
Rindskopf\footnote{ Na
segunda metade de 1815, Heine realizou um estágio em um banco de
Frankfurt cujo proprietário também se chamava Rindskopf (literalmente,
cabeça de gado). Outros nomes judaicos que aparecem na narrativa foram
igualmente inspirados por pessoas que Heine conheceu durante o período
passado em Frankfurt.}
 traz a chave consigo e no momento ele está parado ali no canto,
murmurando a oração dos dezoito parágrafos e, portanto, não é possível
interrompê"-lo.\footnote{
Trata"-se de uma oração central do culto judaico, executada em pé e na
direção de Jerusalém. A designação hebraica ``dezoito'' refere"-se ao
número das bênçãos contidas na oração.}
 Jäkel, o Tolo, também está aqui, mas agora está esvaziando a bexiga. Eu
sou uma pessoa sozinha!''

``Que o diabo carregue os judeus!'', disse o Hans do tambor e, rindo alto
dessa sua frase zombeteira, dirigiu"-se à torre de vigia e foi
deitar"-se também sobre a tarimba.

Enquanto o Rabi se quedava com sua mulher diante do grande portão
fechado, levantou"-se atrás deste uma voz rangente, fanhosa, algo
escarnecedora: ``Deixa um pouco de estrugir, meu pequeno Stern; pega a
chave no bolso do casaquinho do Rindskopf ou pega o teu nariz e abre o
portão com ele. Já faz tempo que as pessoas estão esperando''.

``As pessoas?'' --- gritou atemorizado aquele homem que era chamado de
Nasenstern --- ``eu achava que era um só\ldots\ eu te peço, Tolo, caro Jäkel
Tolo, dá uma olhada lá fora para ver quem são''.

Abriu"-se então uma janelinha gradeada, embutida no portão maciço, e
surgiu um gorro amarelo, de duas pontas, e debaixo dele, burlescamente
enfeitado, o rosto habituado a gracejar de Jäkel, o Tolo. No mesmo
instante a abertura da janela se fechou e lá de dentro ouviu"-se troar
uma voz irritada: ``Abre, abre; lá fora estão apenas um homem e uma
mulher''.

``Um homem e uma mulher!'' --- gemeu aquele Nasenstern --- ``E quando o portão
estiver aberto, a mulher se desfaz do casaco e é também um homem e
então são dois homens, e aqui estamos apenas nós três!''

``Não sejas um coelho medroso'' --- revidou Jäkel, o Tolo --- ``Sê destemido e
mostra coragem!''

``Coragem!'' --- exclamou Nasenstern, contrariado e rindo amargamente ---
``Coelho! Coelho é uma comparação infeliz. Coelho é um animal impuro.
Coragem! Não foi por causa da coragem que me colocaram aqui, mas sim
por causa da prudência. Se vierem muitas pessoas, a minha obrigação é
gritar. Mas eu próprio não posso detê"-las. Meu braço é fraco, tenho
uma fontanela\footnote{
Não se trata da ``moleira'' própria de bebês, mas sim de um antigo
procedimento medicinal: uma ferida mantida artificialmente aberta para
a eliminação de substâncias consideradas nocivas ao corpo.}
 e sou uma pessoa sozinha. Se atirarem em mim, sou um homem morto. Então
o rico Mendel Reiss se senta à mesa no sabá, limpa da boca o molho de
uvas passas, acaricia a barriga e diz talvez: ‘O comprido Nasenstern
era de fato um sujeitinho valente; se não fosse ele, o portão teria
sido arrombado; ele se deixou matar por nós, era mesmo um sujeitinho
valente e é pena que esteja morto\ldots{}'\,''

Nisso, a voz foi ficando cada vez mais branda e chorosa, mas de repente
assumiu um tom apressado, quase enfurecido: ``Coragem! E para que o rico
Mendel Reiss possa limpar o molho de passas da boca, acariciar a
barriga e chamar"-me de valente sujeitinho, eu devo deixar"-me matar?
Coragem! Destemido! O pequeno Strauss era destemido e ontem foi ver o
duelo no Romano, achando que ninguém o reconheceria por trajar um
casaco violeta --- de veludo, três florins o côvado, com cauda de raposa,
todo bordado a ouro, luxuosíssimo ---, e eles bateram tanto no casaco
violeta até que desbotou e também suas costas ficaram tão violetas que
agora nem mais parecem costas de gente. Coragem! O curvo Leser era
destemido, chamou o nosso vil alcaide de vilão e eles o dependuraram
pelos pés entre dois cachorros, enquanto o Hans do tambor
tamborilava.\footnote{ O
judeu que ousasse ofender ou simplesmente criticar uma autoridade
sofria via de regra a mais infame das execuções, que era justamente ser
dependurado pelos pés ao lado de cachorros. Heine se informou a
respeito dessa prática em suas fontes, sobretudo no livro de Lersner
sobre a história de Frankfurt.}
 Coragem! Não sejas um coelho! Entre tantos cães o coelho está perdido,
eu sou uma pessoa sozinha e realmente tenho medo''.

``Vamos, desembucha de uma vez!'' --- exclamou Jäkel, o Tolo.

``Realmente eu tenho medo!'' --- repetiu suspirando Nasenstern --- ``eu sei que
o medo está nas veias e que eu o herdei de minha saudosa mãe\ldots''

``Sim, sim!'', interrompeu"-o Jäkel, o Tolo, --- ``e tua mãe o herdou do pai
dela e este, por sua vez, do pai dele e, assim, todos os teus
antepassados herdaram o medo um do outro, até chegar ao fundador de tua
estirpe, o qual foi lutar contra os fariseus sob o comando do rei Saul
e foi o primeiro a desembestar em fuga. --- Mas olha lá, o pequeno
Rindskopf já está quase pronto, acaba de se inclinar pela quarta vez,
saltita como uma pulga pronunciando três vezes a palavra santo e agora
vai enfiando cuidadosamente a mão no bolso\ldots''

E, de fato, as chaves tilintaram, abriu"-se rangendo uma ala do portão
e o Rabi adentrou com sua mulher a rua dos judeus, inteiramente
deserta. Mas aquele que abrira o portão, um pequeno homem com expressão
carrancuda e bonachona ao mesmo tempo, balançava a cabeça como alguém
que não deseja ser perturbado em seus devaneios; e depois de ter
fechado meticulosamente o portão, arrastou"-se sem dizer palavra
alguma para um canto atrás, mas sem deixar de murmurar suas orações.
Menos calado era Jäkel, o Tolo, um homenzinho atarracado, de pernas um
tanto tortas, com uma cara avermelhada e risonha, e uma mão
descomunalmente carnuda, a qual assomou das compridas mangas de sua
jaqueta com estampas coloridas para dar as boas"-vindas. Atrás dele
mostrava"-se --- ou, antes, escondia"-se --- uma figura comprida e magra,
o pescoço fino envolto numa gargantilha branca de preciosa batista, o
rosto afilado e pálido adornado de maneira estranha por um nariz
incrivelmente comprido, que se movimentava curioso e atemorizado para
lá e para cá.

``Por Deus, bem"-vindos neste dia de festa!'' --- exclamou Jäkel, o Tolo ---
``e não vos admireis de que a rua esteja agora tão deserta e silenciosa.
No momento, toda a nossa gente está na sinagoga e vós chegastes a tempo
de ouvir lá a história do sacrifício de Isaac. Eu a conheço, é uma
história interessante, e se já não a tivesse ouvido trinta e três
vezes, eu a ouviria novamente de bom grado neste ano. É também uma
história importante, pois se Abraão tivesse realmente imolado a Isaac,
e não ao cabrito, então haveria agora neste mundo mais cabritos e menos
judeus''. E com trejeitos incrivelmente engraçados, esse Jäkel começou a
cantar a seguinte canção da Hagadá:

``Um cabritinho, um cabritinho, comprado por paizinho, que deu por ele
duas moedinhas --- um cabritinho, um cabritinho! 

Veio um gatinho e comeu o cabritinho, comprado por paizinho, que deu por
ele duas moedinhas --- um cabritinho, um cabritinho! 

Veio um cãozinho e mordeu o gatinho, que comeu o cabritinho, comprado
por paizinho, que deu por ele duas moedinhas --- um cabritinho, um
cabritinho! 

Veio uma varinha e surrou o cãozinho, que mordeu o gatinho, que comeu o
cabritinho, comprado por paizinho, que deu por ele duas moedinhas --- um
cabritinho, um cabritinho! 

Veio um foguinho e queimou a varinha que surrou o cãozinho, que mordeu o
gatinho, que comeu o cabritinho, comprado por paizinho, que deu por ele
duas moedinhas --- um cabritinho, um cabritinho!

Veio uma aguinha e apagou o foguinho, que queimou a varinha, que surrou
o cãozinho, que mordeu o gatinho, que comeu o cabritinho, comprado por
paizinho, que deu por ele duas moedinhas --- um cabritinho, um
cabritinho!

Veio um boizinho e bebeu a aguinha, que apagou o foguinho, que queimou a
varinha, que surrou o cãozinho, que mordeu o gatinho, que comeu o
cabritinho, comprado por paizinho, que deu por ele duas moedinhas --- um
cabritinho, um cabritinho!

Veio um carniceirinho e abateu o boizinho, que bebeu a aguinha, que
apagou o foguinho, que queimou a varinha, que surrou o cãozinho, que
mordeu o gatinho, que comeu o cabritinho, comprado por paizinho, que
deu por ele duas moedinhas --- um cabritinho, um cabritinho!

Veio um anjinho da morte e abateu o carniceirinho, que abateu o
boizinho, que bebeu a aguinha, que apagou o foguinho, que queimou a
varinha, que surrou o cãozinho, que mordeu o gatinho, que comeu o
cabritinho, comprado por paizinho, que deu por ele duas moedinhas --- um
cabritinho, um
cabritinho!''\footnote{
Essa canção, \textit{Chad Gadya} em hebraico, encontra"-se no final da
Hagadá de Pessach e tem por base um texto da Idade Média que desdobra
uma passagem do profeta Jeremias (30---16): ``Mas todos os que te
devoravam serão devorados, todos os teus adversários irão para o
cativeiro, os que te despojavam serão despojados, e todos os que te
saqueavam serão saqueados''. A narrativa de Heine deixa de fora a
estrofe final, na qual Deus destrói por fim o anjinho da morte,
justamente para introduzir nessa lacuna o comentário de Jäkel, o Tolo,
sobre a vingança final. Numa versão diversa, essa canção do cabritinho
foi recolhida por Clemens Brentano em sua coletânea \textit{A trompa
mágica do menino} (\textit{Des Knaben Wunderhorn}, 1808), o mais
importante cancioneiro do romantismo alemão.}


``Sim, bela senhora'' --- acrescentou o que cantara --- ``ainda virá o dia em
que o anjo da morte abaterá o carniceiro, e todo o nosso sangue se
derramará sobre
Edom;\footnote{ Povo
associado à descendência de Esaú (Gênesis, 36---1) que se estabeleceu
ao Sul do mar Morto. Com as campanhas do rei Davi contra os edomitas
iniciou"-se uma relação de extrema hostilidade entre os dois povos.
Desse modo, o termo ``Edom'' passou a designar metonimicamente todos os
inimigos dos judeus.}
 pois o nosso Deus é um Deus da vingança\ldots''

Mas, de repente, sacudindo enérgico a seriedade que involuntariamente o
acometera, Jäkel, o Tolo, voltou ao seu estilo farsesco e prosseguiu
com sua retumbante voz de bufão: ``Não temais, bela senhora, o
Nasenstern não vos fará nenhum mal. Ele só é perigoso para a velha
Schnapper"-Elle. Ela apaixonou"-se pelo nariz dele --- mas também é um
amor merecido. Seu nariz é tão belo como a torre que mira em direção a
Damasco, e tão sublime quanto os cedros do Líbano. Por fora, ele brilha
como ouro reluzente ou xarope e, por dentro, é pura música e
graciosidade. Floresce no verão, enregela"-se no inverno e tanto no
verão quanto no inverno é acariciado pelas alvas mãos da
Schnapper"-Elle. Sim, a Schnapper"-Elle está apaixonada pelo
Nasenstern, loucamente apaixonada. Ela cuida dele, alimenta"-o e tão
logo ele esteja suficientemente cevado, ela o desposará; e, pela idade
que tem, a Schnapper"-Elle mostra"-se ainda bastante jovem, e aquele
que chegar aqui em Frankfurt daqui a trezentos anos não poderá ver o
céu por causa de tantos Nasenstern!''

``Vós sois Jäkel, o Tolo'' --- exclamou sorridente o Rabi --- ``percebo"-o por
vossas palavras. Várias vezes ouvi falarem a vosso respeito.''

``Sim, pois é'' --- retrucou aquele com modéstia brincalhona --- ``sim, sim, é
isso o que faz a fama. Frequentemente a pessoa é considerada por toda
parte como um tolo ainda maior do que a própria pessoa imagina. Mas eu
me esforço muito para ser um tolo, e dou saltos e me balanço todo para
que os chocalhos tilintem. Para outros, a coisa é bem mais fácil\ldots\ Mas
dizei"-me, Rabi, por que viajais em dia de festa?''

``Minha justificativa'' --- atalhou o que fora perguntado --- ``está no Talmude
e diz: O perigo sobrepõe"-se ao sabá''.

``Perigo!'' --- gritou repentinamente o comprido Nasenstern, agitando"-se
como que tomado por um medo mortal --- ``Perigo! Perigo! Hans do tambor,
toca o tambor, tamborila! Perigo! Perigo! Hans, o tambor\ldots''

Lá de fora, porém, exclamou o Hans do tambor com sua grossa voz de
cerveja: ``Por todos os sacramentos e trovões! O diabo carregue os
judeus! Hoje já é a terceira vez que tu me acordas, Nasenstern! Não me
faças ficar furioso! Se eu me enfurecer, torno"-me o satanás em pessoa
e então --- tão certo como sou um cristão ---, então pego o arcabuz,
meto"-o pela abertura do portão, começo a atirar e que cada um proteja
o seu nariz!''

``Não atire! Não atire! Eu sou uma pessoa sozinha'', choramingava o
amedrontado Nasenstern, ao mesmo tempo que comprimia o rosto contra o
muro próximo --- e permaneceu nessa posição, tremendo e rezando baixinho.

``Mas dizei, contai o que aconteceu'' --- exclamou então Jäkel, o Tolo, com
toda a curiosidade apressada que já por aquela época era característica
dos judeus de Frankfurt.

O Rabi, porém, apartou"-se dele e se pôs a subir a rua dos judeus com
sua mulher. ``Vê, bela Sara'' --- disse ele suspirando --- ``é essa a proteção
que tem Israel! Por fora, falsos amigos guardam os seus portões e, por
dentro, os seus guardiões são a tolice e o temor!''

Lentamente os dois foram percorrendo a comprida e deserta rua, onde
apenas de vez em quando se via alguma florescente menina esticar a
cabeça para fora da janela, enquanto o sol se refletia solene e
alegremente sobre as reluzentes vidraças. É que por aquela época as
casas do bairro judeu eram ainda novas e simpáticas, também mais baixas
do que hoje em dia, sendo que só mais tarde --- quando os judeus se
multiplicaram bastante em Frankfurt e, contudo, não puderam aumentar
seu território --- eles passaram a construir um andar sobre o outro,
espremendo"-se como sardinhas e, desse modo, definhando de corpo e
alma. A parte do bairro judeu que ficou em pé depois do grande incêndio
e que porta o nome de Rua
Velha\footnote{ Esse
``grande'' incêndio ocorreu nos dias 14 e 15 de 1711 e destruiu por
completo o bairro dos judeus em Frankfurt. De suas leituras das obras
de Schudt e Lersner sobre a história da cidade, Heine fez vários
excertos a respeito desse incêndio.}
 --- aquelas casas altas e escuras, onde um povo úmido, sempre sorrindo
compulsivamente, não cessa de barganhar --- é um horripilante monumento
da Idade Média. A sinagoga mais antiga já não existe mais; ela era
menos espaçosa do que a atual, construída mais tarde, após a comunidade
ter acolhido os judeus expulsos de Nuremberg. Aquela ficava mais ao
norte. O Rabi não precisou perguntar pela sua localização. Já à
distância distinguiu o barulho confuso de muitas vozes. No pátio da
Casa do Senhor separou"-se de sua mulher. Depois de ter lavado as mãos
na fonte ali situada, adentrou a parte inferior da sinagoga, onde os
homens oram; a bela Sara, ao contrário, tomou uma escada e chegou acima
à ala das mulheres.

Essa ala superior era uma espécie de galeria com três fileiras de
assentos de madeira pintados de vermelho"-pardo; em sua parte
superior, o espaldar de tais assentos era guarnecido por uma tábua
pênsil, a qual se podia dobrar facilmente para ser utilizada como apoio
para os livros religiosos. Ali as mulheres tagarelavam sentadas uma ao
lado da outra, ou ficavam em pé e oravam com fervor; às vezes,
acercavam"-se curiosas da enorme grade que perfilava o lado oriental e
de onde, através de balaústres finos e verdes, podiam avistar a ala
inferior da sinagoga. Nesse lugar, atrás de elevados púlpitos
individuais, ficavam os homens em seus casacos escuros, as barbas
pontiagudas e espichadas sobre as gargantilhas brancas, as cabeças
cobertas com solidéus mais ou menos envoltas por um pano quadrangular,
em lã ou seda branca, às vezes também enfeitado com galões dourados e
revestido das franjas prescritas pela Lei. As paredes da sinagoga eram
uniformemente caiadas e ali não se via nenhum outro enfeite além da
dourada grade de ferro em volta do tablado quadrangular onde são lidos
os textos da Lei, e da arca sagrada --- uma caixa de fina confecção,
aparentemente sustentada por colunas de mármore com suntuosos capitéis,
cujas flores e folhagens se ramificavam graciosamente, e cerrada por
uma cortina de veludo azul, sobre a qual havia uma inscrição piedosa
bordada com lantejoulas, pérolas e pedras preciosas. Além disso, pendia
do teto a prateada lâmpada da memória, e levantava"-se ainda um
tablado rodeado por grades, em cuja área --- em meio aos mais variados
objetos sagrados --- via"-se o candelabro de sete braços, o menorá. E,
diante deste, o rosto voltado na direção da arca, estava o chantre,
cujo canto era acompanhado de maneira instrumental pelas vozes de seus
dois ajudantes, o baixo e o soprano. É que os judeus baniram de sua
igreja toda música genuinamente instrumental, presumindo que o louvor a
Deus é mais edificante quando se levanta do fervoroso peito humano do
que de frios tubos de órgão. A bela Sara se alegrou como uma criança
quando o chantre, um excelente tenor, elevou a voz para entoar as
antiquíssimas e graves melodias que ela tão bem conhecia; e enquanto
essas melodias iam florescendo envoltas em fresca e insuspeitável
graça, o baixo rugia em contraponto os tons profundos e obscuros,
enquanto o soprano trinava nos intervalos com sua doce e delicada voz.
Na sinagoga de Bacherach a bela Sara jamais havia ouvido um canto
semelhante, pois ali o presidente da comunidade, Davi Levi,
desempenhava a função do chantre, e quando aquele homem trêmulo e já
entrado em anos, com sua voz queixosa e débil, aplicava"-se a trinar
como uma menina e, nesse esforço monstruoso, balançava febrilmente o
braço que pendia frouxo, então uma tal cena suscitava antes o riso do
que a devoção.

Uma pia sensação de bem"-estar, mesclada com curiosidade feminina, levou
a bela Sara a aproximar"-se da mencionada grade, de onde ela pôde
contemplar a ala inferior, a assim chamada escola dos
homens.\footnote{ ``Escola''
era um termo mais popular e tem a sua origem na concepção da sinagoga
como local de ensinamento e aprendizagem.}
 Jamais havia visto companheiros de fé reunidos em número comparável ao
que avistava abaixo, e no íntimo do coração foi sentindo uma crescente
alegria por se ver cercada por tantas pessoas que lhe eram tão próximas
na origem comum, na maneira de pensar e nos sofrimentos. Entretanto,
muito mais ainda se comoveu a alma da mulher quando três anciãos,
transbordando veneração, postaram"-se diante da arca sagrada,
descerraram a cortina reluzente e com muitos cuidados retiraram aquele
livro que Deus escrevera com a própria mão sagrada e para cuja
conservação os judeus tanto suportaram, tanta miséria e tanto ódio,
ignomínia e morte --- um martírio milenar. Esse livro, um grande rolo de
pergaminho, estava envolto como uma criança principesca num pequeno
manto de veludo vermelho, bordado em várias cores; na parte superior do
pergaminho, afixadas em ambos os rolos de madeira, havia duas caixinhas
prateadas nas quais chocalhavam e tilintavam graciosamente vários
sininhos e pequenas granadas; e na frente, presas por correntinhas
prateadas, pendiam medalhas de ouro com pedras preciosas coloridas. O
chantre tomou o livro como se fosse uma criança de verdade --- uma
criança por cuja causa foi preciso passar por imensos sofrimentos e
pela qual se tem tanto mais amor ---, acalentou"-o em seus braços,
bailou com ele para lá e para cá, apertou"-o contra o peito e,
arrebatado por esse contato, elevou a voz para entoar uma tão jubilosa
e pia canção de agradecimento que, para a bela Sara, foi como se as
colunas da arca sagrada começassem a florescer, como se as maravilhosas
flores e folhagens dos capitéis se pusessem a crescer, como se os tons
do soprano se transformassem em puros rouxinóis, a abóbada da sinagoga
fosse arrombada pela poderosa voz do baixo e do céu azul descesse então
o rejúbilo divino. Era um belo salmo. A comunidade repetia em coro os
versos finais, e o chantre, com o livro sagrado nas mãos, foi
caminhando lentamente em direção do tablado situado no centro da
sinagoga, ao mesmo tempo que homens e meninos acorriam para beijar ou
mesmo apenas tocar o invólucro de veludo. Uma vez no tablado, o pequeno
manto de veludo foi retirado do livro sagrado, assim como as faixas com
inscrições coloridas que também o envolviam, e do rolo de pergaminho
agora aberto, naquele tom melodioso que por ocasião da festa do Pessach
é modulado de maneira muito especial, o chantre leu a edificante
história da tentação de Abraão.

A bela Sara afastou"-se humildemente da grade e uma corpulenta mulher
de meia"-idade, com mil enfeites e aparentando uma benevolência um
tanto afetada, proporcionou"-lhe com gestos silenciosos acompanhar as
orações em seu livro. Essa mulher não era certamente muito versada nas
escrituras, pois enquanto lia as orações em tom de murmúrio --- como
costumam fazer as mulheres, uma vez que não lhes é permitido cantar em
voz alta --- a bela Sara percebeu que ela pronunciava muitas palavras por
pura adivinhação e que passava por cima de não poucos trechos. Mas,
depois de certo tempo, os límpidos olhos da boa mulher ergueram"-se
lenta e sofregamente, um sorriso liso esparramou"-se pelo seu rosto,
que parecia feito de porcelana vermelha e branca, e num tom que de tão
distinto parecia querer derreter"-se, disse à bela Sara: ``Ele canta
muito bem. Mas na Holanda ouvi cantarem ainda muito melhor. A senhora é
de fora e talvez não saiba que o chantre é de Worms e que as pessoas
querem mantê"-lo aqui se ele contentar"-se com quatrocentos florins
por ano. É um homem amável e as suas mãos são como alabastro. Eu dou
muito valor a uma mão bonita. Uma bela mão dignifica a pessoa toda!''

E nisso a boa mulher colocou vaidosamente a própria mão, que de fato era
ainda bonita, sobre o espaldar do púlpito e, dando a entender com um
gracioso movimento de cabeça que não queria ser interrompida em suas
palavras, acrescentou: ``O pequeno soprano é ainda uma criança e já
parece acabado. O baixo é feio por demais, e o nosso Stern disse, certa
vez, com muito espírito: ‘O baixo é um tolo ainda maior do que se
costuma exigir de um baixo!’ Todos os três tomam as refeições em minha
cozinha e talvez a senhora desconheça que eu sou a Elle Schnapper''.

A bela Sara agradeceu por essa informação, ensejando assim que aquela
Schnapper"-Elle lhe contasse pormenorizadamente muitas coisas: como
ela estivera outrora em Amsterdã e lá, em virtude de sua beleza, tivera
de enfrentar muitas armadilhas; como ela viera a Frankfurt três dias
antes de Pentecostes e se casara com o Schnapper; como este veio por
fim a falecer, dizendo"-lhe no leito de morte as coisas mais
comoventes; contou"-lhe ainda como era difícil para uma chefe de
cozinha conservar as mãos. Às vezes lançava um olhar desdenhoso para o
lado, visando provavelmente algumas jovens zombeteiras que examinavam
sua roupa. E, de fato, tais vestes eram bastante curiosas: um amplo e
fofo casaco de cetim branco sobre o qual estavam bordadas em cores
vivas todas as espécies da arca de Noé; um gibão de tecido dourado,
parecendo uma couraça; mangas em veludo vermelho com pregas amarelas;
sobre a cabeça uma touca incrivelmente alta e, em volta do pescoço, uma
onipotente gargantilha de linho rígido branco, assim como uma corrente
prateada da qual ficava dependurada, à altura do peito, toda espécie de
penduricalhos, camafeus e outras raridades, entre outras coisas uma
enorme reprodução da cidade de Amsterdã. Mas a vestimenta das outras
mulheres não era menos curiosa e consistia numa mistura de modas de
diferentes épocas; mais de uma daquelas mulherzinhas cobertas de ouro e
diamantes se assemelhava a uma joalheria ambulante. É certo que já
naquela época a lei prescrevia aos judeus de Frankfurt uma determinada
vestimenta e, para se diferenciarem dos cristãos, os homens deviam
ostentar anéis amarelos em seus casacos, e as mulheres, véus
azul"-listrados em suas
toucas.\footnote{ Na Idade
Média, tanto a igreja quanto as autoridade seculares impuseram várias
prescrições de vestuário para os judeus, com a finalidade de
diferenciá"-los nitidamente dos cristãos. Já há documentos a respeito
datados de 1215, mas as fontes consultadas por Heine enfatizam o ano de
1452, quando tais prescrições se tornaram muito mais rigorosas em
decorrência de um sínodo que estigmatizou os judeus como ``inimigos da
cruz de Cristo''.}
 Todavia, no bairro judeu esse decreto da autoridade era pouco
considerado e ali, principalmente nos dias de festa e ainda mais na
sinagoga, as mulheres buscavam competir entre si para ver quem
ostentava mais luxo --- e isso em parte para serem invejadas e em parte,
também, para exibirem a prosperidade e a solidez financeira de seus
maridos.

Enquanto trechos da lei mosaica são lidos na parte inferior da sinagoga,
a devoção costuma relaxar um pouco. Alguns se colocam mais à vontade e
se sentam, ou também cochicham com o vizinho sobre assuntos mundanos,
ou saem ao pátio para respirar ar fresco. Nessas alturas, os meninos
tomam a liberdade de visitar as mães na ala das mulheres e ali a
devoção certamente já sofreu retrocessos ainda maiores: ali elas
tagarelam, mexericam, riem e, como acontece por toda parte, as mulheres
mais jovens troçam das mais velhas e estas por sua vez se queixam da
leviandade da juventude e da decadência dos tempos. Mas, assim como na
parte inferior da sinagoga de Frankfurt havia o chantre, que puxava o
canto, assim também havia na ala superior aquela que puxava a conversa.
Esta se chamava Hündchen Reiss, uma mulher vulgar e esverdeada, que
farejava toda e qualquer desgraça e sempre trazia uma história
escandalosa na ponta da língua. O alvo habitual de suas maledicências
era a pobre Schnapper"-Elle; Hündchen Reiss sabia macaquear de maneira
muito engraçada a distinção forçada nos gestos da outra, assim como o
decoro lânguido com que acolhia as reverências zombeteiras da
juventude.

``Sabíeis'' --- disse então Hündchen Reiss --- ``sabíeis que ontem a
Schnapper"-Elle disse o seguinte: ‘Se eu não fosse bonita, inteligente
e amada, eu não gostaria de estar neste mundo’?''

Então as risadas abafadas foram se tornando cada vez mais audíveis, e a
Schnapper"-Elle, percebendo logo atrás que aquilo acontecia às suas
custas, levantou os olhos com ar de desprezo e partiu velejando como um
orgulhoso navio de luxo para um lugar mais distante. Vögele Ochs, uma
mulher redonda e um tanto estúpida, observou com compaixão que a
Schnapper"-Elle era de fato vaidosa e limitada, mas que tinha um
coração bondoso e fazia o bem às pessoas necessitadas.

``Especialmente ao Nasenstern'' --- sibilou a Hündchen Reiss. E todas as que
conheciam aquela terna relação riram desbragadamente.

``Sabíeis'' --- acrescentou Hündchen maliciosamente --- ``que agora o
Nasenstern também está dormindo na casa da Schnapper"-Elle\ldots\ Mas,
vede que coisa, lá embaixo a Süschen Flörsheim está usando o colar que
o Daniel Fläsch deixou empenhado com o marido dela. Como a Fläsch está
irritada\ldots\ Agora ela está conversando com a Flörsheim\ldots\ E com que
amabilidade elas se cumprimentam! E, no entanto, se odeiam como Madiã e
Moab!\footnote{ A
rivalidade entre os madianitas e os moabitas, dois povos de origem
semítica, era proverbial entre os judeus.}
 Como trocam sorrisos afetuosos! Não vos devoreis de tanto carinho!
Quero ouvir essa conversa''.

E então, como um animal à espreita, Hündchen esgueirou"-se até o local
em que se encontravam as mulheres e ouviu como elas se queixavam por
terem se matado de trabalhar naquela semana a fim de pôr ordem na casa
e arear os utensílios da cozinha, coisa que deve ser feita antes da
festa do Pessach, para que nenhuma migalha de pão fermentado fique
grudada nas panelas. As duas falavam também das dificuldades que
enfrentaram para assar os pães ázimos. A Fläsch tinha queixas ainda
mais específicas: ela tivera de passar por tantos aborrecimentos no
salão do forno da comunidade, pois pelo resultado do sorteio ela só
pôde assar os seus pães nos últimos dias, às vésperas da festa, e ainda
por cima no final da tarde; a velha Hanne havia preparado mal a massa,
com os seus rolos de amassar as criadas a deixaram demasiado fina, a
metade dos pães queimou no forno e, além disso, caiu uma chuva tão
torrencial que não parou de gotejar pelo telhado de madeira do salão do
forno --- e assim, molhadas e cansadas, tiveram de trabalhar
exaustivamente madrugada adentro.

``E a senhora, querida Flörsheim'' --- acrescentou a Fläsch com uma
amabilidade condescendente, que de forma alguma era autêntica ---, ``a
senhora também teve uma parcela de culpa nisso tudo, pois não me mandou
ninguém que pudesse ajudar no forno''.

``Ah, perdão!'' --- retrucou a outra ---, ``o meu pessoal esteve tão ocupado,
as mercadorias para a feira precisavam ser empacotadas, nós temos agora
tanto o que fazer, meu marido\ldots''

``Eu sei'' --- e a Fläsch cortou"-lhe a palavra num tom frio e seco --- ``eu
sei, muito que fazer, muitos penhores e bons negócios, e colares\ldots''

Uma palavra venenosa estava prestes a escapar dos lábios daquela que
falava e a Flörsheim já se mostrava vermelha como um caranguejo, quando
de repente a Hündchen Reiss se pôs a gritar: ``Meu Deus! A forasteira
está caída, morrendo\ldots\ Água! Água !''

A bela Sara havia perdido a consciência, estava pálida como a morte, e
um bando de mulheres se acotovelava ao redor dela, gemendo e correndo
com afobamento para cima e para baixo. Uma sustinha"-lhe a cabeça, a
outra segurava"-a pelo braço, algumas velhas aspergiam"-na com a água
daqueles copinhos que ficavam pendurados atrás dos bancos e servem para
lavar as mãos, no caso de estas tocarem casualmente o próprio
corpo.\footnote{ Na
verdade, a lavagem das mãos era prescrita apenas quando tocavam em algo
considerado impuro.}
 Outras seguravam rente ao nariz da mulher desmaiada um velho limão
incrustado com pedacinhos de cravos, limão esse que provinha ainda do
último dia de jejum e cujo aroma era inalado para fortalecer os nervos.
Extenuada e respirando profundamente, a bela Sara abriu os olhos e
agradeceu com olhares silenciosos os cuidados bondosos. Naquele
instante, porém, a oração dos dezoito parágrafos foi solenemente
entoada na parte inferior da sinagoga e, dado que ninguém pode
perdê"-la, aquelas mulheres alvoroçadas correram de volta aos seus
lugares para orar conforme está prescrito, ou seja, em pé e com o rosto
voltado ao oriente, que é o ponto cardeal onde fica Jerusalém. Vögele
Ochs, Schnapper"-Elle e Hündchen Reiss foram as que mais tempo ficaram
junto à bela Sara: as duas primeiras prestando"-lhe a ajuda mais
solícita e a última indagando sucessivas vezes pelos motivos que a
levaram a desmaiar tão repentinamente.

Mas o desmaio da bela Sara tinha uma causa muito especial, a saber: é
costume na sinagoga que alguém que tenha escapado de um grande perigo
deva, depois da leitura dos trechos da Lei, apresentar"-se perante
todos e agradecer à Providência Divina pela sua salvação. E quando o
Rabi levantou"-se na ala inferior para proferir o agradecimento, a
bela Sara reconheceu a voz do marido e percebeu como sua entoação foi
gradativamente se convertendo no sombrio murmúrio da oração para os
mortos; ela ouviu os nomes das pessoas queridas e dos familiares,
acompanhados ainda por aquela palavra de bênção que se confere aos
mortos. E a última esperança desapareceu da alma da bela Sara; e a sua
alma foi dilacerada pela certeza de que as pessoas queridas e os
familiares haviam sido realmente assassinados, que sua pequena sobrinha
estava morta, que também suas priminhas Blümchen e Vögelchen estavam
mortas, que também o pequeno Gottschalk estava morto --- todos
assassinados, todos mortos! E a dor causada pela consciência desse fato
foi tamanha que ela própria teria morrido se um desmaio benfazejo não
tivesse se derramado pelos seus sentidos.



\chapter{Terceiro capítulo}

\textsc{Quando} a bela Sara, após o término do ofício divino, desceu ao pátio da
sinagoga, o Rabi já estava lá esperando por ela. Saudou"-a com
semblante sereno e a conduziu à rua, onde o silêncio anterior já havia
desaparecido completamente e se podia observar agora uma fervilhante e
ruidosa multidão. Um formigueiro de capas negras e barbas; reluzentes
mulheres esvoaçando como escaravelhos dourados; meninos em roupas novas
conduzindo os livros litúrgicos dos mais velhos; e as meninas menores,
que não tinham permissão de ir à sinagoga, deixavam agora suas casas,
corriam saltitando ao encontro dos pais e inclinavam os cachinhos
diante destes para receber a bênção: todos alegres e contentes,
percorrendo a rua de cima a baixo na deliciosa expectativa de um bom
almoço, cujo cheiro suave, de pôr água na boca, já se levantava das
panelas escuras e marcadas a giz, que as criadas sorridentes iam buscar
no grande fogão comunitário.

Nesse tumulto todo, chamava especialmente a atenção a figura de um
cavaleiro espanhol, em cujos jovens traços fisionômicos pairava aquela
encantadora palidez que as mulheres atribuem de costume a uma
infelicidade amorosa e os homens, ao contrário, a um amor feliz. Embora
bamboleando com indiferença, o seu andar tinha uma graciosidade
calculada; as penas de seu barrete movimentavam"-se mais com o
elegante balanço da cabeça do que pelo sopro do vento; suas esporas
douradas tilintavam mais do que seria necessário, e assim também a
bainha de sua espada, que ele parecia estar empunhando e cujo cabo
resplandecia em meio à branca capa de cavaleiro que, embora aparentasse
envolver com displicência seus membros esguios, denunciava o mais
cuidadoso pregueado. De quando em quando --- em parte com curiosidade, em
parte com maneiras de entendido --- aproximava"-se das mulheres que
passavam, fixava um olhar tranquilo em seus rostos, demorava"-se na
contemplação quando as fisionomias valiam a pena, dizia também algumas
ligeiras lisonjas a mais de uma meiga criança e continuava a caminhar
despreocupadamente, sem esperar o efeito de suas palavras. Já desde
algum tempo ele vinha rondando a bela Sara, sendo rechaçado a cada vez
pelo seu olhar imperioso ou pelo sorriso enigmático do marido, até que
por fim, sacudindo orgulhosamente de si todo acanhamento desajeitado,
postou"-se atrevido no caminho dos dois e com garrida segurança fez o
seguinte discurso, em tom docemente galante: 

``\textit{Señora}, eu juro! Escutai, \textit{señora!} Juro pelas rosas
das duas Castelas, pelos jacintos aragoneses e pelas flores de romã da
Andaluzia! Pelo sol que ilumina a Espanha inteira com todas as suas
flores, cebolas, ervilhas, florestas, montanhas, mulas, cabritos e
velhos cristãos! Pela abóboda celeste, onde o sol não passa de um tufo
dourado! Por Deus, que do alto dessa abóbada medita dia e noite na
criação das mais graciosas figuras femininas\ldots\ Eu juro,
\textit{señora,} sois a mais bela mulher que já vi em terras alemãs e
se estiverdes disposta a aceitar meus serviços, então vos peço que me
concedeis a graça, a honra e a permissão de nomear"-me vosso cavaleiro
e trazer vossas cores no escárnio e na dignidade''.

Uma expressão de dor enrubesceu o semblante da bela Sara e, com um olhar
tanto mais pungente quanto mais suaves os olhos, num tom tanto mais
aniquilador quanto mais mansa e trêmula a voz, respondeu a mulher
profundamente ferida:

``Nobre Senhor! Se quiserdes ser meu cavaleiro, tereis de lutar contra
povos inteiros, e nessa luta há pouca gratidão e ainda menos honra a
conquistar! E se quiserdes mesmo ostentar minhas cores, então tereis de
pregar anéis amarelos em vossa capa ou cingi"-la com uma faixa
listrada de azul: pois essas são as minhas cores, as cores de minha
casa --- dessa casa tão miserável que se chama Israel e é escarnecida nas
ruas pelos filhos da fortuna!''

Um repentino rubor cobriu as faces do espanhol, um infinito
constrangimento se apoderou de seus traços e, quase gaguejando, ele
disse por fim:

``\textit{Señora}\ldots\ Vós me entendestes mal\ldots\ uma brincadeira
inocente\ldots\ mas, por Deus, não quis escarnecer de Israel\ldots\ eu próprio
descendo da casa de Israel\ldots\ meu avô era judeu, talvez até mesmo meu
pai\ldots''

``E com toda certeza, \textit{señor}, o vosso tio é judeu'' ---
interrompeu"-o bruscamente o Rabi, que até aqui tinha observado
tranquilamente a cena, e acrescentou com um olhar alegre e zombeteiro:
``e eu mesmo posso garantir que \textit{Don} Isaac Abarbanel, sobrinho
do grande rabi, é um rebento do melhor sangue de Israel, quando não da
própria estirpe real de
Davi''.\footnote{ Com a
expressão ``o grande rabi'' a personagem de Heine refere"-se a Isaac
(Jizchak) Abarbanel (1437---1508), erudito sefardita e estadista
bastante conceituado em sua época. Como conselheiro financeiro,
colocou"-se a serviço dos regentes de Portugal, Castela, Nápoles e
Veneza, tendo redigido também importantes comentários sobre o Antigo
Testamento e escritos messiânicos. Após fracassar na tentativa de
impedir a expulsão dos judeus da Espanha (1492), ele próprio abandonou
esse país para dedicar"-se aos estudos. Seu terceiro filho, chamado
Samuel, converteu"-se ao cristianismo, mas a existência de um
sobrinho, como apresenta a ficção de Heine, não é historicamente
comprovada.}


Sob a capa do espanhol tilintou então a bainha da espada, suas faces
empalideceram até a extrema lividez, desenhando"-se nos lábios uma
renhida luta entre sarcasmo e dor, os olhos parecendo expedir um ódio
mortal --- e num tom de voz completamente alterado, gélido e cortante,
ele disse:

``\textit{Señor} Rabi! Vós me conheceis. Pois bem, então sabeis também
quem sou eu. E se a raposa sabe que eu pertenço à linhagem do leão,
então ela irá se precaver e não vai querer arriscar a pele provocando
minha cólera! Pois como haveria a raposa de julgar o leão? Somente
aquele que sente como o leão pode compreender suas
fraquezas\ldots''\footnote{ A família Abarbanel pretendia descender em 
linha direta do próprio rei Davi e, por isso, a metáfora do leão alude 
a feitos relatados pelo jovem Davi ao rei Saul, entre os quais ter perseguido e matado leões
que tentavam arrebatar ovelhas de seu pai (Primeiro Samuel, 17---34).
Sagrado rei, Davi tomou o leão como um de seus emblemas.}


``Oh, eu as compreendo muito bem'' --- respondeu o Rabi, e uma melancólica
gravidade cobriu sua fronte ---, ``eu compreendo muito bem que o leão, por
orgulho, se desfaça de sua pele principesca e se disfarce com a
colorida couraça de um crocodilo só porque agora é moda ser um
crocodilo manhoso, astuto e voraz! O que não vão fazer os animais
menores se o leão está se negando a si próprio? Mas, toma cuidado,
\textit{Don }Isaac, tu não foste feito para o elemento do crocodilo. A
água --- tu sabes muito bem do que estou falando --- é a tua desgraça e
nela irás
soçobrar.\footnote{ Com
essas palavras, Rabi Abraão não apenas se refere ao episódio do
quase"-afogamento de seu velho conhecido Isaac nas águas do Tejo (que
já Plínio, em sua \textit{Naturalis historia}, descreve como um rio
aurífero), mas alude também à água do batismo, como leve censura à
conversão daquele ao cristianismo.}
 O teu reino não está na água; nesse elemento, a mais fraca truta pode
prosperar melhor do que o rei da selva. Tu te lembras ainda de como os
redemoinhos do Tejo queriam engolir"-te\ldots''

Irrompendo subitamente em sonora gargalhada, \textit{Don} Isaac abraçou
fortemente o Rabi, cobriu sua boca de beijos e, assustando os judeus
que por ali circulavam com os saltos inflamados que dava com suas
esporas tilintantes, exclamou por fim em seu tom naturalmente afável e
alegre:

``Verdadeiramente, tu és Abraão de Bacherach! E foi não apenas uma coisa
engraçada como, sobretudo, um ato de amizade quando daquela vez, na
ponte de Alcântara em Toledo, tu te jogaste à água, agarraste pelo
topete este teu amigo que bebe bem melhor do que sabe nadar e o puxaste
ao seco! Eu já estava prestes a fazer investigações bastante profundas:
se realmente existem pepitas de ouro no fundo do Tejo e se os romanos
tiveram razão ao chamá"-lo de rio dourado. Asseguro"-te que ainda
hoje sinto calafrios com a simples lembrança daquela brincadeira
aquática''.

E, proferindo essas palavras, o espanhol se agitou como que sacudindo de
si gotículas de água. O semblante do Rabi expressava, porém, o maior
contentamento. Apertou repetidamente a mão de seu amigo e dizia a cada
vez: ``Como fico feliz!''

``Eu também fico muito feliz'' --- disse o outro --- ``há sete anos que não nos
vemos; quando nos despedimos, eu ainda era um jovem fedelho, enquanto
tu\ldots\ tu já eras tão determinado e sério\ldots\ Mas, o que se deu com a
bela \textit{doña} que te custava naquela época tantos suspiros ---
suspiros bem rimados, que acompanhavas com o alaúde?''

``Fala baixo! A \textit{doña} está nos ouvindo, é a minha mulher, e tu
mesmo lhe rendeste hoje uma prova de teu gosto e de teu talento
poético''.

Não sem um resto daquele constrangimento anterior, o espanhol
cumprimentou a bela mulher e esta lamentava agora, com amável brandura,
ter turbado um amigo de seu marido com manifestações de desalento.

``Ah, \textit{señora}'' --- respondeu \textit{Don} Isaac --- ``aquele que
apanhou uma rosa com mão desajeitada não pode queixar"-se dos espinhos
que o feriram! Quando a estrela da tarde se reflete reluzente e dourada
na torrente azul\ldots''

``Pelo amor de Deus, eu te peço!'' --- interrompeu"-o o Rabi --- ``Para com
isso\ldots\ Se tivermos de esperar até que a estrela da tarde se reflita
reluzente e dourada na torrente azul, minha mulher morre de fome; desde
ontem ela não come nada e passou nesse tempo por tantas calamidades e
fadigas''.

``Então vou conduzir"-vos à melhor cozinha de Israel'' --- disse
\textit{Don} Isaac --- ``vou conduzir"-vos à casa de minha amiga
Schnapper"-Elle, que fica aqui perto. Já estou sentindo o seu delicado
cheiro, isto é, o cheiro da cozinha. Oh, se tu soubesses, Abraão, como
esse cheiro me invoca! Desde que estou nesta cidade é ele que me atrai
com tanta frequência às tendas de Jacó. O trato com o povo de Deus não
é de resto nenhuma paixão minha; e, na verdade, não venho à rua dos
judeus para orar, mas sim para comer\ldots''

``Tu nunca nos amaste, \textit{Don} Isaac\ldots''

``É verdade'' --- continuou o espanhol --- ``eu amo a vossa cozinha muito mais
que vossa fé; a esta falta o tempero certo. E a vós próprios, jamais
consegui digerir"-vos de todo. Mesmo em vossos melhores tempos, mesmo
sob o governo de meu antepassado Davi --- que fora rei de Judá e Israel ---
eu não teria suportado permanecer entre vós e, numa bela manhã, teria
certamente escapulido da fortaleza de Sião e emigrado para a Fenícia ou
para a Babilônia, onde a alegria de viver fervilhava no templo dos
deuses\ldots''

``Isaac, estás blasfemando contra o Deus único'' --- murmurou sombriamente o
Rabi --- ``és muito pior do que um cristão, tu és um pagão, um
idólatra\ldots''

``Sim, eu sou um pagão, e os sombrios nazarenos obcecados pelo sofrimento
me são tão repulsivos quanto os ressequidos hebreus sem alegria. Que a
nossa querida senhora de Sídon, a sagrada
Astarte,\footnote{ Astarte
(ou Astartétia) é a deusa fenícia da fertilidade e a ela se vinculavam
cultos de caráter fortemente sensual. Nas importantes cidades de Sídon
e Tiro havia vários templos dedicados a Astarte e na antiga Babilônia
era a divindade que presidia a prostituição praticada nos templos. O
próprio Salomão, ao chegar na velhice, prestou"-lhe culto junto com as
suas muitas mulheres estrangeiras (``moabitas, amonitas, edomitas,
sidônias e heteias''), o que provocou a ira de Iahweh (Primeiro Reis,
11, 1---13).} me perdoe quando ajoelho e oro diante da dolorosa mãe do
Crucificado\ldots\ Só o meu joelho e a minha língua cultuam a morte, meu
coração permanece fiel à vida!\ldots''

``Mas não faças esta cara'' --- continuou o espanhol ao perceber quão pouco
edificante seu discurso soava ao rabino. --- ``Não me olhes com
repugnância. O meu nariz não cometeu apostasia. Certa vez, quando o
acaso me trouxe a esta rua na hora do almoço e, emanando das cozinhas
dos judeus, penetraram em meu nariz esses cheiros tão familiares,
sobreveio"-me aquela nostalgia que os nossos antepassados sentiam ao
se recordarem das panelas de carne do
Egito;\footnote{ Alusão às
queixas que os hebreus, em determinado momento da caminhada no deserto,
passaram a fazer a Moisés e Aarão: ``Antes fôssemos mortos pela mão de
Iahweh na terra do Egito, quando estávamos sentados junto à panela de
carne e comíamos pão com fartura'' (Êxodo, 16---3). }
 apetitosas lembranças da juventude levantaram"-se em meu íntimo; em
meu espírito, vi novamente aquelas carpas com o escuro molho de passas
que minha tia sabia preparar de maneira tão edificante na noite de
sexta"-feira; revi aquela carne de carneiro refogada com alho e
raiz"-forte, capaz de ressuscitar os mortos; e aquelas almôndegas
boiando misticamente na sopa\ldots\ e minha alma derreteu"-se como os
trinados de um rouxinol apaixonado e desde então venho comendo na
cozinha de minha amiga, \textit{doña} Schnapper"-Elle!''

Entrementes, eles chegaram a essa cozinha; a própria Schnapper"-Elle
estava à porta de sua casa, saudando amavelmente os visitantes da feira
que entravam famintos. Atrás dela, com a cabeça apoiada em seu ombro,
estava o comprido Nasenstern, que examinava curioso e amedrontado os
que iam chegando. Com exagerada \textit{grandezza}, \textit{Don} Isaac
acercou"-se de nossa estalajadeira, a qual correspondia às profundas
reverências do espanhol zombeteiro com infinitas mesuras. Em seguida,
ele tirou sua luva direita, enrolou a mão descoberta com a ponta de sua
capa, tomou a mão de Schnapper"-Elle, acariciou"-a lentamente com os
pelos de sua barba aparada e disse:

``\textit{Señora}! Vossos olhos competem com o ardor do sol! Contudo,
embora os ovos mais duros se tornem quanto mais tempo são cozidos, o
meu coração não obstante tanto mais amolece quanto mais o cozem os
raios flamejantes de vossos olhos! Da gema de meu coração levanta"-se
adejando o alado deus Amor e busca um aconchegante ninho em vosso
peito\ldots\ Este peito, \textit{señora,} com o que devo compará"-lo? Em
toda a criação não há uma única flor, um único fruto que se lhe iguale!
É uma planta única em sua espécie! Embora a tempestade desfolhe as
rosinhas mais delicadas, vosso peito é uma rosa de inverno que resiste
a todas as intempéries! Embora o azedo limão, quanto mais envelhece,
mais amarelo e enrugado se torne, vosso peito compete todavia com a cor
e a delicadeza do mais doce ananás! Oh, \textit{señora}, e se a cidade
de Amsterdã é tão bela como me contaste ontem, anteontem e todos os
dias, o terreno sobre o qual ela repousa é mil vezes mais belo\ldots''

O cavaleiro pronunciou essas últimas palavras com fingido acanhamento,
olhando de soslaio, languidamente, para a grande imagem pendurada no
pescoço de Schnapper"-Elle; Nasenstern dirigiu"-lhe também um olhar
perscrutador, de cima para baixo, e aquele peito decantado iniciou um
movimento tão turbulento que a cidade de Amsterdã balançou para lá e
para cá.

``Ah!'' --- suspirou Schnapper"-Elle --- ``a virtude vale mais do que a
beleza. De que me adianta a beleza? Minha juventude vai passando e
desde a morte do Schnapper --- ele ao menos tinha belas mãos --- de que me
serve a beleza?''

Nisso ela tornou a suspirar e como um eco, quase inaudível, o Nasenstern
suspirou atrás dela.

``De que vos serve a beleza!'' --- exclamou \textit{Don} Isaac --- ``Oh,
\textit{doña} Schnapper"-Elle, não pequeis contra a generosidade da
natureza prodigiosa! Não desdenheis de seus mais elevados dons! Ela se
vingaria terrivelmente. Estes olhos que insuflam vida se tornariam
baços e mortiços, estes lábios graciosos se embotariam até a extrema
insipidez, este corpo casto e sedento de amor se transformaria num
pesado barril de sebo, a cidade de Amsterdã viria então a repousar
sobre um pântano podre\ldots''

E assim foi descrevendo, passo a passo, a efetiva aparência da
Schnapper"-Elle, de tal modo que a pobre mulher, cada vez mais
angustiada, procurava furtar"-se ao inquietante discurso do cavaleiro.
Nessa situação, ficou duplamente feliz ao avistar a bela Sara e poder
informar"-se circunstancialmente se ela se recuperara por completo do
desmaio. Com isso, enveredou por uma animada conversa em que
desenvolveu tanto a sua falsa distinção quanto a sua autêntica bondade
de coração; e com mais prolixidade do que inteligência, contou aquela
história fatal de como ela própria quase desmaiara de medo quando
chegou, a bordo de uma
barcaça,\footnote{
\textit{Trekschuite}, no original: espécie de embarcação que homens ou
cavalos puxavam por cabos a partir das margens. Na época em que se
passa a narrativa era um meio de transporte muito usado nos canais
holandeses.}
a Amsterdã --- cidade da qual não conhecia absolutamente nada ---, e o
patife do carregador de malas a conduziu, não a uma hospedaria decente,
mas sim a uma despudorada casa de mulheres, coisa que ela logo percebeu
pelas muitas bebedeiras e insinuações imorais\ldots\ e, como dito, ela
teria realmente desmaiado se ao longo das seis semanas que passou
naquela casa ardilosa ela tivesse ousado, por um instante sequer,
fechar os olhos\ldots''

``Por causa de minha virtude'' --- acrescentou ainda --- ``não pude ousar tal
coisa. E tudo isso me aconteceu por causa da minha beleza! Mas a beleza
passa e a virtude permanece''.

\textit{Don} Isaac já estava a ponto de elucidar criticamente os
detalhes dessa história quando, felizmente, o vesgo Aaron Hirschkuh, da
cidade de
Homburg"-sobre"-o"-Lahn,\footnote{ Lahn, com 245 quilômetros de extensão, é um afluente do 
Reno. Uma vez que \textit{Hirschkuh} significa em alemão a fêmea do veado
(\textit{Hirsch}), Heine reforça a caracterização zoomórfica atribuindo		\EP[-1.2]
ao vesgo Aaron a designação para boca (\textit{Maul}) empregada apenas
para animais (algo análogo a ``focinho'', \textit{Schnauze} em alemão).}
surgiu à porta com o guardanapo branco na bocarra e reclamou irritado
porque a sopa já fora servida havia algum tempo, os fregueses se
encontravam todos à mesa e somente a estalajadeira estava faltando. --- ---
---

\bigskip

[\textit{O final e os capítulos subsequentes perderam"-se sem culpa do
autor.}]

