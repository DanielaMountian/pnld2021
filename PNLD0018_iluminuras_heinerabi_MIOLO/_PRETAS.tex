\textbf{Heinrich Heine} (Düsseldorf, 1797--Paris, 1856) é um dos maiores nomes
da literatura alemã. Seus primeiros poemas são publicados já
em 1817, num jornal de Hamburgo. Em 1824 surge a coletânea lírica
\textit{Trinta e três poemas}, na qual se inclui a canção “Loreley”,
uma das mais célebres de toda a literatura alemã. Neste mesmo ano faz
uma viagem a pé pela região do Harz (norte da Alemanha), e em seguida
visita Goethe em Weimar. Dois anos mais tarde publica a narrativa
\textit{Viagem pelo Harz}, elaboração poética das observações,
experiências e reflexões feitas durante a caminhada. Em outubro de 1827
vem a lume novo volume lírico, \textit{Livro das canções}, acolhido
entusiasticamente pela juventude alemã, e que se torna, ao longo dos
anos, uma fonte de inspiração para vários compositores de
\textit{Lieder} (canções).
Em 1831, após concluir a quarta e última parte de seus \textit{Quadros
de viagem}, emigra para Paris, de onde passa a enviar artigos para um
influente jornal liberal alemão (\textit{Augsburger Allgemeine
Zeitung}). Com suas obras, artigos e intervenções busca promover o
intercâmbio cultural e a aproximação entre a França e a Alemanha, como
atestam as seguintes palavras de Balzac: “Heine representa em Paris o
espírito e a poesia da Alemanha, assim como encarna na Alemanha a
crítica francesa mais viva e espirituosa”. Depois de 12 anos de
ausência, retorna à Alemanha para visitar sua mãe em Hamburgo. Elabora
as impressões de viagem no longo poema \textit{Alemanha, um conto de
inverno}, obra considerada por muitos como a maior sátira da literatura moderna.
Em 1840, o governo francês concede"-lhe uma pensão no valor de 400 francos
mensais, o equivalente ao salário de um professor universitário bem remunerado,
mas apesar de seu prestígio na França, Heine tem a prisão decretada em vários
estados da Alemanha e suas obras são cada vez mais visadas pela censura.
Em 1848, a já abalada saúde do autor piora sensivelmente
e a partir de então se vê preso ao que chamou de sua “cripta de
colchões”, vítima de uma doença degenerativa que provoca dores atrozes,
obrigando"-o a tomar altas doses de morfina. Contudo, sua produção
literária prossegue intensa até os últimos dias de vida. Falece em 17
de fevereiro de 1856 e três dias depois é sepultado no cemitério de
Montmartre.


\textbf{Marcus Vinicius Mazzari} é professor de Teoria Literária na Universidade
de São Paulo. Traduziu para o português textos de Walter Benjamin, Bertolt
Brecht, Adelbert von Chamisso, Thomas Mann, Günter Grass, Goethe, entre outros.
Organizou uma edição bilíngue da primeira e segunda partes de \textit{Fausto},
de Goethe (Trad. J.~K.~Segall, Editora 34, 2004/2007). Publicou ainda, entre
outros trabalhos, \textit{Romance de formação em perspectiva histórica}
(Ateliê, 1999) e \textit{Die Danziger Trilogie von Günter Grass. Erzählen gegen
die Dämonisierung deutscher Geschichte} (Berlim, 1994).


