\chapter[vida e obra, \emph{por Fernanda Trindade Luciani}]{vida e obra}
\hedramarkboth{vida e obra}{fernanda trindade luciani}

\begin{flushright}
\textsc{fernanda trindade luciani}
\end{flushright}

\section{Sobre o autor}

 As circunstâncias da vinda ao Brasil do português Gabriel Soares de
Sousa, nascido provavelmente no início da década de 1540, são incertas,
mas se sabe que chegou a terras americanas em 1569. Há indícios de que,
quando decidiu desembarcar no litoral baiano, fazia parte de uma frota
de três naus, comandada por Francisco Barreto, que partiu de Lisboa em
direção a Monomotapa, hoje em Moçambique, com a finalidade de explorar
as cobiçadas minas africanas. Não são conhecidas as razões que fizeram com que Soares ali
permanecesse em vez de seguir viagem para o Oriente, que atraía na
época a maior parte dos viajantes e exploradores; 
mas é fato que se casou na colônia e fixou moradia em uma região ao sul do
Recôncavo baiano, onde ergueu um engenho de açúcar no rio Juquiriçá.
Tornando"-se um dos homens principais da capitania, proprietário de
terras, casas na cidade, bois e escravos, chegou a ocupar o cargo de
vereador da Câmara da Bahia.\footnote{ Segundo a Carta Régia de 16 de
novembro de 1581, apresentada em Câmara em 19 de maio de 1582, citada
por Varnhagen, Gabriel Soares de Sousa assinou, como vereador, o Auto
de aclamação e juramento de fidelidade, prestado pelo Senado da Câmara
da Bahia a Filipe \textsc{ii}, em 25 de maio de 1582 (Cf. Francisco Adolfo de
Varnhagen, \textit{História Geral do Brasil}, edição crítica de Capistrano de 
Abreu e Rodolfo Garcia. 3ª ed. São Paulo: Melhoramentos, [s/d], tomo \textsc{i}, p.~468).}
Infelizmente, há pouca notícia a respeito da vida de Gabriel Soares na
colônia ou das condições em que colheu as informações sobre a costa do
Estado do Brasil e a capitania da Bahia de Todos os Santos. Os dados
biográficos sobre o autor têm sido reiterados, com apenas alguns
acréscimos, desde a pesquisa realizada por Varnhagen, em razão da
pequena documentação disponível. Um dos poucos testemunhos que fornece
alguma informação sobre a vida de Gabriel Soares é o testamento que ele
escreveu em 1584, já com a intenção de deixar a colônia em viagem à corte
real.\footnote{ Seu testamento, lavrado em 21 de agosto de 1584, veio a
público pela primeira vez em 1866, graças ao historiador Mello Moraes,
que o fez copiar do documento original conservado no Livro Velho do
Tombo do Mosteiro de São Bento, na Bahia; e está publicado nesta edição
(Cf. Alexandre José Mello Moraes, \textit{Brasil Histórico.} Rio de
Janeiro: 2ª série, 1, 1866, pp. 248 e 251--52.).}

Não se sabe por que esse respeitável senhor de engenho do Recôncavo
deixou o Brasil justamente naquele momento. É conhecido apenas que, tendo
herdado alguns mapas e pedras preciosas de seu irmão, João Coelho de
Sousa, que falecera no Brasil em uma expedição pelo sertão à procura de
diamantes, esmeraldas e ouro, Gabriel Soares partiu para a Espanha com
a intenção de solicitar ao rei Filipe~\textsc{ii} (\textsc{i} de Portugal)\footnote{ Em 4
de agosto de 1578, D.~Sebastião faleceu na batalha de Alcácer"-Quibir,
no Marrocos, em luta contra os mouros pela disputa da região. Era
o fim um período de conquistas lusas iniciadas desde a Tomada
de Ceuta em 1415. O rei, morrendo ainda jovem, não deixara herdeiros, e
seu tio"-avô, o Cardeal D.~Henrique, acabou por assumir o trono
português, governando até 1580, quando veio a falecer. Surgira, então,
uma grave questão sucessória, e entre os candidatos ao trono estava o
neto legítimo de D.~Manuel~\textsc{i}, Filipe~\textsc{ii}, rei da Espanha. Este assumiu o
trono português, sob o título de Filipe~\textsc{i} de Portugal, permanecendo no
governo dos dois reinos até 1598, ano de sua morte.} apoio a sua empreitada de exploração das terras coloniais para além do rio São Francisco, em busca de riquezas
minerais. 

%\section{Dos manuscritos ao Tratado descritivo}

Em uma pequena apresentação que precedia o \textit{Roteiro geral} e o
\textit{Memorial}, dedicada a Sua Majestade, sob o título “Descrição do
que se contém neste caderno”, Soares de Sousa justifica os manuscritos a
Filipe 	\textsc{ii}. Nesse breve texto, conhecido como “Proêmio”, por assim ter
sido nomeado por Varnhagen, o autor explica, como leal vassalo, que sua
“pretensão é manifestar a grandeza, fertilidade e outras grandes partes
que têm a Bahia de Todos os Santos e demais Estados do Brasil”, levando
ao conhecimento do rei as condições em que se encontrava a colônia
portuguesa, cujas terras tinham sido deixadas em estado de abandono
pelos monarcas anteriores, justamente pela pouca notícia que dali lhes
chegava. Continua sua justificativa argumentando que “a el"-rei nosso
senhor convém, e ao bem do seu serviço, que lhe mostre, por estas
lembranças, os grandes merecimentos deste seu Estado, as qualidades e
estranhezas dele, etc., para que lhe ponha os olhos e bafeje com seu
poder”. Gabriel Soares estava defendendo que apenas quando detentora de
tal conhecimento é que a Coroa poderia proteger aquelas possessões e
explorá"-las adequadamente.\footnote{ Esse texto foi publicado pela
primeira vez por Varnhagen, precedendo o “Roteiro geral” e o
“Memorial”, na edição de 1851 (Cf. Gabriel Soares de Sousa,
\textit{Tratado Descritivo do Brasil em 1587}. Edição castigada
pelo estudo e exame de muitos códices manuscritos existentes no Brasil,
em Portugal, Espanha e França e acrescentada de alguns comentários à
obra feitos por Francisco Adolfo de Varnhagen. Rio de Janeiro: Tip.
Universal de Laemmert, 1851) e posteriormente na edição espanhola,
organizada por Cláudio Gans (Cf. Gabriel Soares de Sousa, \textit{Derrotero general de la costa
del Brasil y Memorial de las Grandezas de Bahia}. Madri: Cultura
Hispânica, 1958). O documento encontra"-se publicado na presente edição
sob o título “Descrição do que se contém neste caderno”.}

Em 1590, após pelo menos quatro anos de espera, as solicitações de
Gabriel Soares de Sousa foram atendidas por Filipe \textsc{ii}. Entre as
principais concessões reais estavam o título de capitão"-mor e
governador da conquista do Rio São Francisco, o direito de nomear seu
sucessor em caso de falecimento e a permissão para prover por três anos
todos os ofícios de justiça e de fazenda nas terras que fossem por ele
ocupadas. Além disso, o rei incentivou esse empreendimento colonial por
meio da distribuição de honras e mercês aos primeiros participantes da
expedição e da permissão ao governador"-geral do Estado do Brasil, D.
Francisco de Sousa,\footnote{ D. Francisco de Sousa (c.1540--1611),
fidalgo português, residia na corte filipina, ocasião que o fez
conhecer Gabriel Soares de Sousa, ao ser nomeado pelo rei Filipe 	\textsc{ii}
como sétimo governador"-geral do Brasil, cargo que passou a exercer, na
Bahia, a partir de 1591. Onze anos depois, retornou à Corte, onde
iniciou negociações para voltar ao Brasil em busca de metais e pedras
preciosas nas capitanias ao sul da Bahia. Permaneceu no reino até 1609,
quando foi novamente instituído de um importante cargo na colônia, o de
governador das capitanias do sul (Francisco de Assis Carvalho Franco.
\textit{Dicionário de Bandeirantes e Sertanistas do Brasil}. São Paulo:
Itatiaia, 1986, pp. 399--400).} para que cedesse duzentos índios
flecheiros à empreitada.\footnote{ Os alvarás que concedem a Gabriel
Soares de Sousa essas honras e mercês estão publicados em\textit{
Pauliceae Lusitana Monumenta Historica}. Lisboa: Real Gabinete Portugal
de Leitura do Rio de Janeiro, 1956, tomo I, p. 407 e ss.} Em posse
desses privilégios reais, Gabriel Soares organizou uma armada com cerca
de 360 homens para retornar ao Brasil em uma urca
flamenga fretada pela Fazenda Real. Partiu de Lisboa em sete de abril
de 1591, mas, antes que alcançasse seu objetivo final, a embarcação
naufragou na enseada de Vazabarris,\footnote{ Esse topônimo é derivado
da expressão portuguesa “dar em vaza"-barris”, que significava perder"-se
sem esperanças de salvação, pois aquela era uma região em que ocorriam
frequentes naufrágios.} no litoral sergipano, o que levou à morte de
alguns tripulantes e à perda de parte do armamento. O nomeado
capitão"-mor e governador da conquista, contudo, não desistiu e caminhou
até a Bahia com os sobreviventes para lá reorganizar a expedição com
auxílio de D.~Francisco de Sousa. Naquele mesmo ano de 1591 partiu para
o sertão em direção à foz do Rio São Francisco em busca das tão
sonhadas minas, mas a ele sucederia o mesmo que a seu irmão anos antes,
vindo a falecer no início da viagem. As circunstâncias de sua morte não
são claras, se causada por doença ou por vingança de índios
aprisionados; o que se sabe é que em seguida à fatalidade seus ossos
foram levados para a Capela do Mosteiro de São Bento na cidade de
Salvador. Acredita"-se que seu corpo tenha sido sepultado no lugar onde
ainda hoje se encontra, no interior da capela, uma lápide com a
inscrição ``aqui jaz um pecador''.\footnote{ Em seu
testamento, Gabriel Soares de Sousa pedia que fosse enterrado na 
capela"-mor do Mosteiro de São Bento e que sobre sua sepultura fosse colocada
aquela mesma inscrição.}

 Parece, portanto, que os manuscritos \textit{Roteiro geral} e
\textit{Memorial}, endereçados a um dos mais influentes ministros de
Filipe 	\textsc{ii}, D.~Cristóvão de Moura, para que chegassem ao conhecimento do
rei, surtiram o efeito esperado por Gabriel Soares, que conseguiu as
extraordinárias concessões reais para realizar sua expedição na
colônia. E, apesar de suas ricas informações (é provavelmente a fonte
documental mais completa a respeito do primeiro século de colonização
do Brasil), não admira que esses textos não tenham sido publicados até
o século \textsc{xix}. Ainda que não fossem impressos, os escritos sobre o
ultramar dos séculos \textsc{xvi} e \textsc{xvii} não permaneceram completamente
desconhecidos dos leitores contemporâneos nem das gerações posteriores.
Muitos textos correlatos e coevos aos manuscritos de Gabriel Soares
tiveram a mesma sorte, desde a Carta de Pero Vaz de Caminha a D.~Manuel~\textsc{i}, 
datada de 1500, que só veio a ser publicada em 1817;\footnote{ A
carta de Pero Vaz ao rei permaneceu desconhecida por mais de dois
séculos, conservada no Arquivo Nacional da Torre do Tombo, em Lisboa.
Foi encontrada pelo secretário de Estado português José de Seabra da
Silva, em 1773, noticiada pelo historiador espanhol Juan Bautista Muñoz
e publicada pela primeira vez pelo padre Manuel Ayres de Cazal em sua
\textit{Corografia Brasílica} (Manuel Ayres de Cazal,		
\textit{Corografia Brazílica, ou Relação Histórico"-Geográfica do Reino
do Brasil}. Rio de Janeiro: Impressão Régia, 1817).} passando pelo
\textit{Diário de Navegação}, de Pero Lopes de Sousa, divulgado por
Varnhagen em edição de 1839;\footnote{ Cf. Pero Lopes de Sousa, \textit{Diário da navegação da armada que 
foi à terra do Brasil em 1530 sob a capitania"-mor de Martim Afonso de Sousa,
escrito por seu irmão Pero Lopes de Sousa}. Publicado por Francisco Adolfo de Varnhagen. 
Lisboa: Typ. da Sociedade Propagadora dos Conhecimentos Uteis, 1839.} e pelos textos do jesuíta Fernão Cardim
escritos entre 1583 e 1601, reunidos e publicados 
sob o título de \textit{Tratados da terra e gente do Brasil} apenas em
1925;\footnote{ Cf. Fernão Cardim, \textit{Tratados da terra e gente do Brasil}. 
Introdução e notas de Batista Caetano, Capistrano de Abreu e Rodolfo Garcia. 
Rio de Janeiro: J.~Leite e Cia., 1925. 
A obra reúne três manuscritos do autor: \textit{Do clima e terra do Brasil}, \textit{Do princípio e Origem dos índios do Brasil}
e \textit{Narrativa Epistolar}, este último havia sido publicado por Varnhagen em 1847 (Fernão Cardim, \textit{Narrativa Epistolar de uma
Viagem e Missão Jesuítica pela Bahia, Ilheos, Porto Seguro, Pernambuco, Espírito Santo, Rio de Janeiro, S. Vicente, (São Paulo), 
etc., desde o ano de 1583 ao de 1590}. Lisboa: Imprensa Nacional, 1847).} até os manuscritos do frei franciscano Vicente do Salvador,
\textit{História do Brasil 1500--1627}, publicados na íntegra em 1888.\footnote{ Cf. Vicente do Salvador, \textit{História do Brasil 1500-1627}.
Introdução de Capistrano de Abreu. Rio de Janeiro: Anais da Biblioteca Nacional, vol. 13, 1888.}
Uma das poucas exceções a essa prática é a \textit{História da
província de Santa Cruz}, de autoria do gramático Pero de
Magalhães Gandavo, que foi escrita e impressa em língua portuguesa no
próprio século \textsc{xvi}.\footnote{ Cf. Pero de Magalhães Gandavo,
\textit{História da Província de Santa Cruz} \& \textit{Tratado da
Terra do Brasil}. Lisboa: Officina de Antônio Gonsalves, 1576.}

\section{Sobre a obra}

\noindent{}Inéditos e anônimos por mais de dois
séculos, os manuscritos \textit{\mbox{Roteiro} geral com largas informações de
toda a costa do Brasil} e \textit{Memorial e declaração das grandezas
da Bahia de Todos os Santos, de sua fertilidade e das notáveis partes
que tem} foram publicados apenas no século \textsc{xix}, com o título 
\textit{Tratado descritivo do Brasil em 1587}, cuja autoria foi atribuída a
Gabriel Soares de Sousa. No entanto, eram textos conhecidos já no
século \textsc{xvi} por meio de cópias manuscritas que circulavam não apenas na
Península Ibérica, mas em outros países europeus, sendo citados desde
então em relatos, memórias e tratados de viajantes, religiosos e homens
que viviam na colônia ou exploravam as terras americanas. Quando
finalmente publicados, despertaram grande interesse nos estudiosos dos
primórdios da colonização do Brasil, e ainda nos dias de hoje são
considerados um valioso documento, bastante solicitado pela
historiografia. %Para uma análise introdutória dessa obra, talvez seja conveniente observar os três tempos que compõem sua trajetória: o contexto em que foi produzida, o momento de sua publicação e seu emprego no presente.

Tal documento surgiu durante o tempo em que seu autor esperava, em Madri, o despacho
real a respeito de suas solicitações. No intuito de reforçar a
importância de seu plano e demonstrar conhecimento sobre a colônia,
Soares passou a limpo em um caderno suas lembranças dos dezessete anos
em que viveu no Estado do Brasil, como relata em carta, datada de 1º de
março de 1587, que precedia esses seus escritos.\footnote{ Essa valiosa
carta, uma das peças"-chave para que Varnhagen restituísse a autoria dos
manuscritos \textit{Roteiro geral} e \textit{Memorial}, encontra"-se
publicada na presente edição, sob o título “Epístola do autor a D.
Cristóvão de Moura”.} Essa epístola, enviada ao valido do
rei, Cristóvão de Moura,\footnote{ D. Cristóvão de Moura e Távora
(1538--1613) foi o principal articulador, junto à nobreza portuguesa, da
sucessão do trono de Portugal em favor do rei espanhol após a morte de
D. Sebastião. Com a ascensão de Filipe 	\textsc{ii} ao trono, integrou o Conselho
de Portugal e recebeu, em 1594, o título de Conde de Castelo Rodrigo.}
acompanhava os dois manuscritos que viriam a compor o \textit{Tratado
descritivo}: o primeiro, \textit{Roteiro geral}, que descreve a costa
brasileira desde o Maranhão até o Rio da Prata; e o segundo,\textit{
Memorial e declaração das grandezas da Bahia,} mais longo, no qual o
autor oferece um minucioso relato das plantas, animais, rios, relevo,
povos nativos, povoações, vilas e engenhos da capitania da Bahia,
especialmente do Recôncavo.  A produção de Gabriel Soares de Sousa
compreende ainda um terceiro manuscrito, no qual dirige severas
críticas aos Jesuítas da Bahia, por conta da ganância desses religiosos
e de sua interferência nas questões referentes à mão"-de"-obra
indígena.\footnote{ Uma cópia desse terceiro documento, feita pelos
jesuítas, que responderam a cada uma das críticas de Gabriel Soares,
foi encontrada no arquivo da ordem em Roma pelo padre Serafim Leite,
que a publicou com o título “Capítulos de Gabriel Soares de Sousa”
(Cf. “Capítulos de Gabriel Soares de Sousa contra os Padres da
Companhia de Jesus que residem no Brasil”. \textit{Anais da Biblioteca
Nacional,} vol. 62, 1942, pp. 340--381).} 

Após mais de 150 anos desde sua primeira publicação
integral, os textos do \textit{Roteiro geral} e do \textit{Memorial}
disponíveis hoje em dia ainda são os que foram estabelecidos por
Varnhagen em meados do século \textsc{xix}. Diante disso, a presente edição é um
esforço de restaurar ao leitor contemporâneo um códice alternativo de
fins do século \textsc{xvi} ou início do \textsc{xvii}. Preservado na Biblioteca Guita e
José Mindlin, o manuscrito contém a cópia dos dois textos de Gabriel
Soares\textit{,} que vêm precedidos pelo traslado da carta enviada a
Cristóvão de Moura e pela breve apresentação intitulada “Descrição do
que se contém neste caderno”.  O códice traz como referência na lombada
a inscrição “Descripcion y noticias del Brasil. Sin autor. Anno 1587”,
o que sugere tratar"-se de uma cópia espanhola.

Nossa intenção, portanto, é apresentar uma obra que contemple a edição
de 1851, estabelecida por Varnhagen, a de 1879, revista por ele, mas publicada um ano após sua
morte, e o referido códice. Para tanto, procuramos incorporar as
variantes de cada um e definir os termos ou frases que nos pareceram
mais acertados pelo contexto do trecho em questão. Mantivemos, em
notas, no caso de divergências entre os textos, os excertos ou
vocábulos variantes. Além disso, o leitor notará a flutuação da grafia 
em certos vocábulos neste próprio códice, --- por exemplo, ``pitiguar'' e ``pitigoar'', 
``Pernambuco'' e ``Pernambuquo'' ---, que optamos por manter, em nome do interesse 
filológico que essas variantes possam suscitar.
 Vale ressaltar ainda que, como se observará, o
códice que nos propomos a restaurar e que certamente não foi consultado
por Varnhagen, guarda grande semelhança com o texto por ele
estabelecido. Ademais de algumas frases presentes no manuscrito que não
se encontram nas edições de 1851 e 1879 ou vice"-versa, e de algumas
palavras distintas, o que mais destoa entre as versões é a grafia dos
nomes próprios, das plantas e das populações indígenas. Isso
considerado, seria mais acertado dizer que Varnhagen, ao preparar a
edição de 1851, tomou por base um único códice, possivelmente a cópia
do acervo da Biblioteca de Évora, que ele tanto elogiou em carta ao
Instituto Histórico Brasileiro. E, como essa epístola e os comentários
do historiador a cada um dos capítulos fazem parte da trajetória
editorial do \textit{Tratado descritivo}, optamos por mantê"-los nesta edição. 

Por fim, faz"-se necessário pontuar alguns critérios utilizados no
estabelecimento do texto da presente edição. Com a intenção de
facilitar a leitura para um público contemporâneo, fizemos a
atualização ortográfica, restabelecendo os acentos e as formas verbais
segundo a ortografia contemporânea e eliminando as maiúsculas
desnecessárias; optamos pela pontuação moderna; desenvolvemos as
abreviaturas; e preferimos colocar por extenso todos os numerais.
Contudo, acreditamos ser adequado manter a grafia original dos nomes
das frutas, animais, grupos indígenas e dos patrônimos e topônimos,
assinalando em nota a grafia moderna, quando foi possível a
identificação. Respeitamos também a divisão dos parágrafos e os
intertítulos conforme ocorrem no códice.  

\section{Sobre o gênero}



%No caso do \textit{Tratado descritivo do Brasil}, o considerável
%circuito de sua distribuição e consumo, ainda que sob a forma anônima
%ou apócrifa, parece notório ao se verificar, por um lado, que cópias
%dos manuscritos são encontradas hoje em arquivos públicos e
%particulares de Portugal, Espanha, França, Inglaterra e Áustria e, por
%outro, que muitos autores fizeram referências a Gabriel Soares ou a seu
%texto, com autoria equivocada ou anônima, antes mesmo de sua primeira
%publicação. Entre eles destacam"-se Pedro de Mariz, já no próprio século
%\textsc{xvi}, Frei Vicente de Salvador, Antônio Leon Pinelo e Simão de
%Vasconcelos, no século \textsc{xvii}, Frei Antônio de Santa Maria Jaboatão, no
%século \textsc{xviii}, e Pedro Manuel Ayres de Cazal e Von Martius, no século
%\textsc{xix}.\footnote{ Cf. Pedro de Mariz, \textit{Diálogos de Vária História}.
%Coimbra: Officina de Antonio de Mariz, 1594; Frei Vicente do
%Salvador,\textit{ História do Brasil} [1627]. Rio de Janeiro: Anais da
%Biblioteca Nacional, vol.13, 1888; Antônio Leon Pinelo, \textit{Epítome
%de la Biblioteca oriental y occidental, náutica y geográfica}. Madri:
%por Juan Gonzales, 1629;  Simão de Vasconcelos, \textit{Crônica
%da Companhia de Jesus no Estado do Brasil}. Lisboa: 1663, e
%\textit{Notícias Curiosas e Necessárias das Cousas do Brasil}. Lisboa:
%por João da Costa, 1668; Frei Antonio de Santa Maria Jaboatão,
%\textit{Novo Orbe Sefárico Brasílico ou Crônica dos Frades Menores do
%Brasil}. Lisboa: Officina de Antonio Vicente da Silva, 1761; 
%Pedro Manuel Ayres de Cazal, \textit{Corografia Brazílica, ou Relação
%Histórico"-Geográfica do Reino do Brasil}. Rio de Janeiro: Impressão
%Régia, 1817; e Karl Friedrich Von Martius,
%\textit{Nova Genera et Species Plantarum
%Brasiliensium}. Munique: 1823--1832, 3 vols., e \textit{Herbarium
%Florae Brasiliensis}. Munique: 1837.} Era comum esses tratados,
%memórias e relatos circularem apenas em cópias manuscritas, podendo ser
%alterados pelos copistas ou plagiados por outros autores. Além das
%referências a seu texto em obras posteriores, as solicitações de
%Gabriel Soares e as concessões reais foram evocadas por outros súditos
%que negociaram com a Coroa expedições mineradoras no ultramar. Seu
%processo de petições e concessões passou, portanto, a servir de modelo
%ou exemplo a outros exploradores coloniais.\footnote{ O acordo entre
%Gabriel Soares de Sousa e o rei Filipe~\textsc{ii} serviu de base não apenas
%às negociações de expedições em busca de pedras e metais preciosos
%referentes ao Estado do Brasil, como aos pedidos feitos pelo 
%governador"-geral D. Francisco de Sousa, que havia acompanhado as
%solicitações de Soares e o malogro de sua empreitada na Bahia, e até
%para expedições em Angola. (Cf. Rodrigo Ricupero. \textit{Honras e Mercês.
%Poder e patrimônio nos primórdios do Brasil}. Tese de doutoramento apresentada
%na Faculdade de Filosofia, Letras e Ciências Humanas/\textsc{usp}, 2006, pp. %61--64).}
%
%\section{As mercês reais}

Entre os discursos que os descobrimentos produziram e que
tinham o rei por destinatário final, podem"-se destacar as narrativas em
estilo elevado dos grandes feitos dos vassalos no ultramar, bem como os
relatos descritivos dos recursos humanos e
naturais da colônia. Diferenças à parte, ambos os tipos de texto
perfaziam uma função político"-social comum. Por um lado, seus autores
os escreviam com o objetivo de obter privilégios reais; por outro, a
Coroa se valia desses escritos, que, junto das correspondências
coloniais, a aparelhavam para uma administração mais eficiente de suas
distantes possessões ultramarinas. Essa tecnologia de poder se
inscrevia em uma prática das monarquias ibéricas que consistia em
centralizar a distribuição de mercês por serviços prestados.\footnote{
Sobre a prática dos reinos ibéricos de distribuição de mercês, ver:
Fernanda Olival, “Um rei e um reino que viviam da mercê”, \textit{in}:
Fernanda Olival, \textit{As Ordens Militares e o Estado
Moderno. Honra, Mercê e Venalidade em Portugal (1641--1789).} Lisboa:
Estar, 2000; e António Manuel Hespanha, “La Economia de la gracia”,
\textit{in:} A. M. Hespanha, \textit{La Gracia del derecho: economia de
la cultura en la Edad Moderna.} Madri: Centro de Estudios
Constitucionales, 1993.} Isso não se restringia ao reino e, expandido
para o ultramar, viabilizava a colonização estreitando os laços entre
vassalos e rei nos planos material, simbólico e informativo.\footnote{Sobre os vínculos que eram criados entre os vassalos, que produziam conhecimento acerca do ultramar, e o rei, que
lhes concedia privilégios por esse serviço, e sobre a importância dessa prática para a manutenção do império
português, ver: Ronald Raminelli, \textit{Viagens Ultramarinas. Monarcas, vassalos e
governo à distância}. São Paulo: Alameda, 2008, p.~26 e ss.}

Durante a União Ibérica, essa lógica cobrou importância ainda maior,
pois o governo dos Habsburgo procedeu a uma reforma administrativa de
fundo que resultou na reorganização das capitanias, no envio de mais
oficiais régios à colônia e na criação de um conselho judicial na Bahia
(Tribunal da Relação) – enfim, na aproximação entre a metrópole e o
Brasil, que não estivera no centro das preocupações dos monarcas
portugueses.\footnote{ Segundo António Manuel Hespanha, teria havido
uma “modernização do sistema político português” durante o governo
filipino. Em comparação com a “portuguesa”, a forma “espanhola” de
poder era mais centralizada, livre de limitações
corporativas e, assim, mais eficaz; portanto mais “moderna”. Tais
novidades teriam alcançado também o Estado do Brasil, como argumentou
Guida Marques (Cf. A.M.~Hespanha, “O Governo dos Áustrias e a
modernização da constituição política portuguesa”, \textit{Penélope},
2, 1989, pp. 50--73; e Guida Marques, “O Estado do Brasil na União
Ibérica”, \textit{Penélope}, nº27, 2002, pp. 7--35).} Interessado na
descoberta de novas jazidas, Filipe \textsc{ii} se mostrou especialmente
generoso na concessão de mercês aos súditos que quisessem realizar
expedições na colônia e entregar ao rei informações etnográficas,
geográficas e botânicas. Essa prática de distribuição de privilégios
incentivava os súditos a esforçarem"-se por obtê"-los por meio de
guerras, descobertas, composição de relatos etc., os quais, por sua
vez, facultavam à Coroa, sem grandes custos nem investimentos iniciais,
acumular conhecimentos indispensáveis ao domínio das colônias e
garantir superioridade técnica e política sobre os demais
povos.\footnote{ Sobre a formação da colônia e a prática da Coroa em
integrar os vassalos à empresa colonial, ver: Ilana Blaj, \textit{A
Trama das Tensões}. \textit{O Processo de Mercantilização de São Paulo
Colonial (1681--1721)}. São Paulo: Humanitas, 2002; e Rodrigo Ricupero,
\textit{Formação da Elite Colonial}. São Paulo: Alameda, 2008.} Vale
lembrar que nem todos os Estados adotavam essa mesma política para
incentivar iniciativas particulares de exploração e de coleta de
informações. Diferentemente dos impérios ibéricos, por exemplo, as
Províncias Unidas dos Países Baixos e sua Companhia das Índias
Ocidentais financiavam viagens de reconhecimento de seus, e mesmo de
outros, territórios coloniais, ou encomendavam a seus funcionários
relatórios que descreviam a paisagem das terras do ultramar e suas
atividades comerciais e agrícolas.\footnote{ Os relatórios mais
completos que temos sobre as capitanias de Pernambuco, Itamaracá,
Paraíba e Rio Grande no século \textsc{xvii} foram produzidos durante o domínio
neerlandês. Mesmo antes da conquista, uma relação dos engenhos e da quantidade de açúcar produzido
naquelas capitanias foi elaborada, provavelmente em
1623, pelo judeu de origem portuguesa José Israel da Costa por
encomenda da Companhia neerlandesa (“Açúcares que fizeram os engenhos de
Pernambuco, Ilha de Itamaracá e Paraíba – ano 1623”, in: \textit{Fontes
para a História do Brasil Holandês}. Textos editados por José Antonio
Gonsalves de Mello; Organização de Leonardo Dantas Silva. 2ªed.,
Recife: Centro de Estudos Pernambucanos, 2004, vol.1, pp. 28--32).}

As concessões de Filipe \textsc{ii} a Gabriel Soares inserem"-se nesse contexto da
união das Coroas ibéricas de incentivo à expansão territorial e à
procura por veios minerais nas possessões portuguesas. É um precioso
exemplo da prática de distribuição de mercês em recompensa aos serviços
prestados ao rei. Soares, como dito, arcaria com todos os custos e
riscos da expedição, e parte de suas mercês apenas passariam a ter
validade caso o empreendimento tivesse sucesso. Contudo, afora a
expedição no alto São Francisco, os próprios manuscritos
\textit{Roteiro geral} e \textit{Memorial} também se tornaram fatores
para a aquisição de honras e mercês. Era um trunfo a mais para Gabriel
Soares, que, assim, procurava nobilitar"-se tanto pela espada quanto
pela pena.

%\section{O caráter promocional}

O início da colonização portuguesa na América não produziu abundante
documentação, nem oficial nem de particulares. Até o século \textsc{xvii}, como
se sabe, as atenções da Coroa, dos exploradores, dos comerciantes e dos
viajantes estavam voltadas para as Índias Orientais. Diante disso, o
\textit{Tratado descritivo do Brasil em 1587} se torna um dos mais
importantes documentos para os interessados e especialistas no período
colonial. Sua descrição permanece a melhor fonte de informação
sobre a Bahia do século \textsc{xvi} e vem sendo amplamente utilizada pela
historiografia desde o século \textsc{xix}, além de servir a diferentes campos
de pesquisa como a antropologia e a botânica.\footnote{ Além do próprio
Adolfo de Varnhagen, no século \textsc{xix}, Capistrano de Abreu, Paulo Prado,
Gilberto Freyre, Almeida Prado e Sérgio Buarque de Holanda, até meados
do século \textsc{xx}, debruçaram"-se sobre o texto de Gabriel Soares e fizeram
dele uma valiosa fonte em que se basearam para defender suas diferentes
visões sobre o Brasil. Recentemente, pesquisadores também têm recorrido ao
\textit{Tratado} para abordarem os mais diversos temas, como: Ronaldo
Vainfas, Stuart Schwartz, John Monteiro, Luis Felipe de Alencastro,
Ronald Raminelli e Rodrigo Ricupero (Cf. João Capistrano de Abreu,
\textit{Capítulos de História colonial}. Anotada e prefaciada por José
Honório Rodrigues. Belo Horizonte: Itatiaia; São Paulo: Publifolha,
2000 (1907); Paulo Prado, \textit{Retrato do Brasil: ensaio sobre a
tristeza brasileira}. Organizado por Carlos Augusto Calil. 8ª ed. São
Paulo: Companhia das Letras, 1997 (1928); Gilberto Freyre, \textit{Casa
Grande e Senzala}. São Paulo: Global, 2006 (1933); João Fernando de
Almeida Prado, \textit{A Bahia e as Capitanias do Centro do Brasil
(1530--1626)}. Coleção História da Formação da Sociedade Brasileira.
São Paulo: Cia.~Editora Nacional, 1945, 3 vols.; Sérgio Buarque de
Holanda, \textit{Visão do Paraíso}. São Paulo: Brasiliense, 1999
(1959); Ronaldo Vainfas, \textit{A heresia dos índios: catolicismo e
rebeldia no Brasil colonial}. São Paulo: Companhia das Letras, 1995, e
\textit{Ideologia e Escravidão}. Rio de Janeiro, Editora Vozes, 1986;
Stuart Schwartz, \textit{Segredos Internos: engenhos e escravos na
sociedade colonial.} São Paulo: Companhia das Letras, 1995; Luis Felipe de
Alencastro, \textit{O trato dos viventes: formação do Brasil no
Atlântico Sul}. São Paulo: Companhia das Letras, 2000; John M.
Monteiro, \textit{Negros da Terra}. São Paulo: Companhia das Letras,
1994, e ``As ‘Castas de Gentio’ na América Portuguesa Quinhentista:
Unidade, Diversidade e a Invenção dos Índios no Brasil'', in:
\textit{Tupis, Tapuias e Historiadores. Estudos de História Indígena e
do Indigenismo}. Tese Apresentada para o Concurso de Livre
Docência Área de Etnologia/\textsc{ifch}"-Unicamp, Campinas, 2001 (Uma
versão desse texto foi publicada pela primeira vez em inglês: “The
Heathen Castes of sixteenth"-Century Portuguese America: Unity,
Diversity, and the Invention of the Brazilian Indians”,
\textit{Hispanic American Historical Review}, 80:4, nov.~2000, pp.
697--719); Ronald Raminelli, \textit{Imagens da colonização. A
representação do índio de Caminha a Vieira}. Rio de Janeiro:
Jorge Zahar, 1996, e \textit{Viagens Ultramarinas. Monarcas, vassalos e
governo à distância}. São Paulo: Alameda, 2008; Rodrigo Ricupero,
\textit{Formação da Elite Colonial}. São Paulo: Alameda, 2008).}

O título atribuído por Varnhagen à obra de Gabriel Soares é aquele a que 
o próprio autor se refere, em duas passagens do texto, aos seus
escritos: “como se verá neste Tratado”, na introdução que precede os
manuscritos, e “o conteúdo neste Tratado”, no capítulo final do
\textit{Memorial}.  A forma literária definida na época como “tratado”
correspondia a uma dissertação, lançada em papel, a respeito de um
determinado assunto. E Soares delimita claramente, logo na apresentação
dos manuscritos, o assunto que se propõe a abordar: as riquezas e
fertilidade da Bahia de Todos os Santos e dos demais estados do Brasil.
Ao intitular o primeiro manuscrito de \textit{Roteiro geral da costa Brasílica}, o autor o inscreve em um gênero corrente no
século \textsc{xvi}, o dos “roteiros”, empregado na descrição minuciosa de uma
viagem ou de pontos e acidentes geográficos da costa de uma dada
região, incluindo a exposição dos cabos, baixos, ilhas, portos,
correntes, ventos, para guiar os navegantes. Com relação ao título do
segundo texto, \textit{Memorial e declaração das grandezas da Bahia}, 
interessa considerar duas acepções para o vocábulo “memorial”. A
primeira, mais evidente, significa um livro onde se guardam as
lembranças ou um relato de memórias; sentido este de que Gabriel Soares
se apropria ao escrever na carta a Cristóvão de Moura: “fiz [\ldots]
muitas lembranças por escrito do que me pareceu digno de notar”. A
outra acepção, não explicitada por Soares e de que acabamos de tratar,
é o papel que se dá a alguém para pedir uma mercê ou a petição para
lembrar a alguém o que se pede.\footnote{ Raphael Bluteau,
\textit{Vocabulario Portuguez \& Latino}. Coimbra: Colégio das Artes,
1712--1728, 8 vols. e 2 suplementos; e Antônio de Moraes Silva,
\textit{Diccionario da Língua Portugueza}. 2ªed. Lisboa: Typographia
Lacerdina, 1813 (1789), 2 vols.} 

%A obra de Gabriel Soares de Sousa, desde sua primeira edição no século
%\textsc{xix}, está dividida em duas partes, que correspondem aos dois
%manuscritos enviados ao valido de Filipe \textsc{ii}. A primeira parte,
%“Roteiro geral da costa brasílica”, composta por 74
%capítulos, inicia"-se com um breve histórico da conquista do território
%americano pelos portugueses, desde a chegada de Pedro Álvares Cabral
%até a instituição das capitanias hereditárias no governo do rei D. João
%\textsc{iii}, retomando os limites da colônia estabelecidos pela divisão
%espacial do Novo Mundo segundo o Tratado de Tordesilhas. A isso segue a
%exposição da hidrografia, do relevo, das povoações e vilas e dos povos
%nativos ao longo do litoral do Estado do Brasil, entre a foz do rio
%Amazonas e o Rio da Prata, vislumbrando as possibilidades de pesca,
%criação de gado, roças e lavouras, assim como de portos e abrigadas
%para embarcações. Essa descrição da paisagem de norte a sul da costa
%brasileira está organizada segundo os limites hidrográficos que
%definiam as capitanias, para as quais o autor apresenta seu donatário,
%processo de ocupação e primeiros povoadores.
%
%A uma exposição mais detalhada da capitania da Bahia é que Gabriel
%Soares dedica o “Memorial e declaração das grandezas da Bahia”, a
%segunda parte do \textit{Tratado,} mais longa e certamente mais
%importante. Nos seus 196 capítulos, o autor se
%propõe a
%\begin{hedraquote}
%tratar e explicar o que se dela [Bahia] não sabe para que venham à
%notícia de todos os ocultos desta ilustre terra, por cujos merecimentos
%deve de ser mais estimada e reverenciada do que agora é.
%\end{hedraquote}
%
%Essa segunda parte tem início com a nomeação, por D. João \textsc{iii}, de Tomé
%de Sousa a capitão e governador"-geral de todo o Estado do
%Brasil. Essa sucinta retomada histórica, que ocupa os cinco
%primeiros capítulos e compreende ainda a administração dos dois
%governadores"-gerais subsequentes, Duarte da Costa e Mem de Sá, tem o
%propósito de demonstrar as primeiras iniciativas da Coroa na América e
%de estabelecer uma comparação com a situação colonial no momento em que
%o autor está escrevendo, o que reforçaria as ideias, por ele
%defendidas, de abandono daquelas terras pelos últimos reis e da
%necessidade de intervenção real para sua defesa e prosperidade –
%note"-se que a valoração pejorativa do recente passado português atendia
%à política pró"-Castela de seu protetor, Cristóvão de Moura.  Nos
%capítulos seguintes, apresenta a cidade de Salvador e a capitania da
%Bahia, sobretudo o Recôncavo, em uma detalhada descrição dos engenhos
%(com o nome de seus proprietários e tipo de energia motriz, se água ou
%bois), da hidrografia, da flora, da fauna e das populações nativas, com
%seus costumes, crenças, alianças, guerras e trajetória espacial. Por
%fim, aborda as maneiras de fortificar a Bahia e prospectar jazidas
%minerais, tudo “como convém ao serviço de El"-rei Nosso Senhor e ao bem da terra”.
%
%Essas pormenorizadas exposições, que destacam perigos e potencialidades
%das possessões americanas, não são desinteressadas, voltadas apenas
%para o conhecimento em si mesmo. Soares de Sousa vislumbra sempre nos
%objetos relatados as possibilidades de engrandecer as riquezas da terra
%e de seus moradores e, ao mesmo tempo, da Coroa; em outras palavras, de
%concretizar o “projeto” colonial metropolitano. Pode"-se dizer que o
%princípio organizacional do texto é demonstrar em caráter promocional –
%no que devemos considerar as prováveis hipérboles em suas descrições –
%quão proveitoso seria ao rei incentivar a exploração daquelas terras e,
%assim, lhe conceder as mercês requeridas para sua expedição pelo
%sertão.  Ao descrever a hidrografia e o relevo da costa brasílica, por
%exemplo, Gabriel Soares se preocupa em detalhar as baías, rios
%navegáveis, ilhas, abrigadas e surgidouros; pormenores essenciais à
%defesa do território e a uma navegação segura para as embarcações metropolitanas.
%
%O mesmo se passa com a exposição das sementes, frutas, legumes e
%árvores, para os quais, além do tamanho, forma, cor, cheiro e
%semelhanças com outros gêneros metropolitanos, o autor atenta sempre
%para seu proveito como alimento, suas propriedades medicinais ou o
%perigo que podem ocasionar a quem os comer. Para as árvores,
%em particular, destaca a qualidade de sua madeira, apontando a
%facilidade de lavrá"-la e suas serventias na construção de embarcações,
%de estruturas de casas e de peças para os engenhos. Quanto à
%“animália”, que inclui os mamíferos, peixes, répteis, aves e insetos, o
%caminho é o mesmo. Cumpre notar que Gabriel Soares os descreve
%minuciosamente, procurando compará"-los a espécies conhecidas no
%universo cultural europeu, uma vez que tem por interlocutores agentes
%metropolitanos, a quem busca traduzir os dados coloniais como meio de
%persuadi"-los a explorarem as terras americanas.  É o caso da ave
%macucaguá: “de cor cinzenta, do tamanho de um grande pato, mas tem no
%peito mais titelas que dois galipavos, as quais são tenras como de
%perdiz, e da mesma cor [\ldots] Tem o bico pardaço, da feição da galinha”.		
%Ademais, tem o cuidado de incluir, na descrição de cada espécie,
%características benéficas ou prejudiciais aos homens. Exemplos disso
%são os venenos de cobras e insetos, as pragas destruidoras das
%plantações, bem como as qualidades e doenças associadas a certos tipos de carnes e %ovos. 
%
%O complexo panorama etnográfico sugerido por Gabriel Soares não deixa
%dúvidas de sua longa convivência com os índios, que formavam a maior
%parte da mão"-de"-obra escrava colonial até meados do século \textsc{xvii}. E, ao
%tratar dos diferentes grupos indígenas, como os Tupinambás, Tapuias,
%Aimorés, Tapanazes e Goitacazes, percebe"-se que a maneira de expor e a
%intenção ao fazê"-lo não se dão de forma distinta à adotada para a flora
%e fauna.  Soares os toma ou como possíveis aliados nas guerras contra
%os inimigos estrangeiros ou como companheiros nas expedições pelo
%sertão; ou, ainda, como valiosa reserva de trabalho para a indústria
%açucareira.\footnote{ Sobre a abordagem de Gabriel Soares de Sousa em
%relação aos grupos indígenas em seu \textit{Tratado}, ver John M.
%Monteiro, ``As ‘Castas de Gentio’ na América Portuguesa'' ,
%\textit{op. cit.}, pp. 18--24.} Todos esses exemplos reforçam o teor
%radicalmente pragmático e promocional do \textit{Tratado}, cujo
%objetivo, em última análise, consiste em destacar as potencialidades
%dos recursos naturais e das populações nativas americanas como
%fundamentais ao processo de conquista territorial e à manutenção do domínio %colonial.

%\section{Trajetória editorial}
%
%Antes da publicação completa dos manuscritos \textit{Roteiro geral} e
%\textit{Memorial}, organizada  por Francisco Adolfo de Varnhagen, que
%uniu os dois textos sob o título \textit{Tratado descritivo do Brasil
%em 1587}, alguns capítulos haviam sido impressos anônimos no periódico
%\textit{O~Patriota Brasileiro} (1830), em Paris, e na \textit{Revista
%do Instituto Histórico e Geográfico Brasileiro} (1839), no Rio de
%Janeiro,\footnote{ “Roteiro geral”, \textit{O Patriota Brasileiro},
%Paris, 1830; “Extrato de um manuscrito que se conserva na Biblioteca de
%S.M. o Imperador e que tem por título Descripção Geográfica da América
%Portuguesa”. \textit{Revista do Instituto Histórico e Geográfico
%Brasileiro}, 1839.} assim como haviam sido feitas duas edições, com o
%texto truncado e sem autoria. Uma é a organizada pelo frei brasileiro
%José Mariano da Conceição Veloso, que iniciou em Lisboa, pela
%Tipografia do Arco do Cego, uma impressão, não concluída, de um códice
%do século \textsc{xvii} intitulado \textit{Descripção Geographica da América
%Portuguesa};\footnote{ Essa edição é composta por setenta capítulos da
%primeira parte e mais setenta e quatro da segunda. Existe um exemplar
%dessa rara edição na biblioteca do Instituto Histórico e Geográfico
%Brasileiro, no Rio de Janeiro (\textit{Descripção Geographica da América
%Portuguesa}. Lisboa: Tipografia do Arco do Cego. Sob a supervisão de
%Frei J. Mariano da Conceição Veloso. Sem autoria e sem data).} e a outra
%é a edição da Academia Real das Ciências de Lisboa, que publicou, em
%1825, a partir de um manuscrito incompleto e apócrifo, a obra
%\textit{Notícia do Brasil} na ``Colecção de Notícias para a
%Geografia das Nações Ultramarinas''.\footnote{ \textit{Notícia
%do Brasil, descrição verdadeira da costa daquelle estado, que pertenceu
%à Coroa do Reyno de Portugal, sítio da Bahia de Todos os Santos}, in:
%``Colecção de Notícias para a História e Geografia das Nações
%Ultramarinas'', Tomo \textsc{iii}. Lisboa: Academia Real das
%Ciências, 1825.} Os méritos de Varnhagen, que não são pequenos,
%consistiram no estabelecimento do texto obtido pelo cotejo de cerca de
%vinte códices encontrados em arquivos europeus e na Biblioteca Nacional
%do Rio de Janeiro, nos cuidadosos comentários a respeito de cada
%capítulo e no esclarecimento da autoria da obra.\footnote{ Varnhagen
%explica que analisou cerca de vinte códices do texto de Gabriel Soares de
%Sousa: um na Biblioteca de Paris, em 1847; três na Biblioteca de Évora;
%três na Biblioteca Portuense; outros na Biblioteca das Necessidades em
%Lisboa; dois em Madri; três na Academia de Lisboa; outros três de
%“menos valor” no Rio de Janeiro; um na Torre do Tombo em Lisboa; e
%outro em Neuwied, que fora dado, na Bahia, ao príncipe Maximiliano
%(“Carta ao Instituto Histórico do Brasil”, 1º de março de 1851, \textit{Revista do %Instituto Histórico e
%Geográfico Brasileiro},
%Tomo \textsc{xiv}, 1851).}  Em 1839, o historiador, com apenas 22 anos
%de idade, após percorrer os acervos europeus que sabia guardarem algum
%códice com a cópia dos manuscritos – já que acreditara não ter
%encontrado o original – escreveu suas \textit{Reflexões Críticas},		
%também publicadas na ``Colecção de Notícias'' da Academia de Lisboa.
%Nesse texto, que marcou a trajetória editorial do \textit{Tratado},
%Varnhagen defende a legitimidade da autoria de Gabriel Soares de Sousa
%para o \textit{Roteiro geral} e o \textit{Memorial}, fixa a verdadeira
%data do texto e explica o método utilizado, os documentos analisados e
%as cópias encontradas que o fizeram chegar a tais conclusões.\footnote{
%Francisco Adolfo de Varnhagen, \textit{Reflexões criticas sobre o
%escripto do seculo \textsc{xvi} impresso com o título de Notícias do Brasil no
%tomo \textsc{iii}}\textit{\textsuperscript{} }\textit{da Coll. de Not. Ultr.
%Acompanhadas de interessantes notícias bibliographicas e importantes
%Investigações históricas}. Lisboa: Typ. da Academia, 1839.}  
%
%Em carta enviada de Madri ao Instituto Histórico e Geográfico
%Brasileiro, em 1851, Varnhagen então recomenda a publicação do
%\textit{Tratado descritivo.}\footnote{ “Carta ao Instituto Histórico do
%Brasil” [1º de março de 1851], \textsc{rihgb}, Tomo \textsc{xiv}, 1851. Carta %publicada
%na presente edição.} Naquele mesmo ano, foram feitas duas publicações
%do texto completo estabelecido e comentado pelo historiador,
%uma  na \textit{Revista} do  Instituto e outra pela
%Tipografia Universal de Laemmert.\footnote{ Gabriel Soares de Sousa, \textit{%Tratado descritivo
%do Brasil em 1587}. Edição castigada pelo estudo e exame de muitos
%códices manuscritos existentes no Brasil, em Portugal, Espanha e França
%e acrescentada de alguns comentários à obra feitos por Francisco Adolfo
%de Varnhagen. \textit{Revista do Instituto Histórico e Geográfico Brasileiro},
%Tomo \textsc{xiv}, 1851; Gabriel Soares de Sousa, \textit{Tratado descritivo do %Brasil em 1587}. Edição
%castigada pelo estudo e exame de muitos códices manuscritos existentes
%no Brasil, em Portugal, Espanha e França e acrescentada de alguns
%comentários à obra feitos por Francisco Adolfo de Varnhagen. Rio de
%Janeiro: Tip. Universal de Laemmert, 1851.} O mesmo texto, apenas
%acrescido de um “Aditamento”, no qual Varnhagen escreve uma pequena
%biografia de Gabriel Soares, foi impresso em 1879 pela Tipografia de
%João Inácio da Silva, no Rio de Janeiro.\footnote{ Gabriel Soares de
%Sousa, \textit{Tratado descritivo do Brasil em 1587}. Rio de Janeiro:
%Tip. de João Inácio da Silva, 1879.} Todas as publicações subsequentes
%se basearam no texto da edição de 1879,\footnote{ Refiro"-me aqui às
%edições de Gabriel Soares de Sousa de 1886, 1938, 1971, 1974 e 2000
%(\textit{Tratado descritivo do Brasil em 1587}. Rio de Janeiro: Tip. de
%João Inácio da Silva, Tip. Perseverança, 2ª edição do tomo \textsc{xiv} da
%\textit{Revista do Instituto Histórico e Geográfico Brasileiro}, 1886;
%\textit{Tratado descritivo do Brasil}. São Paulo: Cia Editora Nacional,
%1938; \textit{Tratado descritivo do Brasil em 1587.} São Paulo:
%Companhia Editora Nacional, Série Brasiliana, 1971;  \textit{Notícia do
%Brasil}. Comentários e notas de Varnhagen, Pirajá da Silva e Edelweiss.
%São Paulo: Ed. Patrocinada pelo Dep. de Assuntos Culturais do
%Ministério da Educação e da Cultura, 1974; \textit{Tratado descritivo
%do Brasil em 1587}. Recife: Massangana, 2000). Sobre a
%trajetória editorial da obra, ver: Gabriela Soares de Azevedo,
%\textit{Leituras, notas, impressões e revelações do Tratado descritivo
%do Brasil em 1587 de Gabriel Soares de Sousa}. Dissertação de Mestrado,
%\textsc{uerj}, 2007.} inclusive a de 1945, sob o título \textit{Notícia do
%Brasil}, organizada pelo médico brasileiro Antônio Pirajá da Silva,
%que, sem incluir os comentários de Varnhagen, realizou um cuidadoso
%trabalho, estabelecendo detalhadas notas lexicais, biográficas e
%históricas.\footnote{ Gabriel Soares de Sousa, \textit{Notícia do
%Brasil}. Introdução e notas de Manuel Antônio Pirajá da Silva. São
%Paulo: Martins, 1945, 2 vols. Vale
%compartilhar a surpresa de encontrar na Biblioteca Guita e José Mindlin
%um exemplar da edição de Gabriel Soares de 1879 que pertenceu a Pirajá
%da Silva, com \textit{marginalia} desse minucioso médico e pesquisador
%brasileiro.}  Além dessas publicações em português, existe uma edição
%em língua espanhola datada de 1958, \textit{Derrotero general de la
%costa del Brasil y Memorial de las Grandezas de Bahia,} organizada pelo
%jornalista e historiador brasileiro Cláudio Ganns a partir de um
%manuscrito em espanhol.\footnote{ Gabriel Soares de
%Sousa, \textit{Derrotero general de la costa del Brasil y Memorial de
%las Grandezas de Bahia.} Introdução de Cláudio Ganns e notas finais de
%Francisco Adolpho de Varnhagen. Madrid: Cultura Hispânica,
%1958\textbf{\textit{.}}} A oitava edição, publicada pelo Ministério da
%Educação e da Cultura do Brasil em 1974 como parte da coleção
%“Brasiliensia Documenta”, foi organizada pelo erudito paulista Edgard
%de Cerqueira Falcão, que, cotejando as edições anteriores, uniu os
%comentários de Varnhagen às notas de Pirajá da Silva e do etnólogo
%Frederico Edelweiss.\footnote{ Gabriel Soares de Sousa, \textit{Notícia
%do Brasil}. Comentários e notas de Varnhagen, Pirajá da Silva e
%Edelweiss. São Paulo: Ed. Patrocinada pelo Dep. de Assuntos Culturais
%do M.E.C., 1974.} A edição mais recente, de 2000, organizada pelo
%historiador Leonardo Dantas, foi publicada pela editora pernambucana
%Massangana, da Fundação Joaquim Nabuco, dentro da “Série
%Descobrimentos”, e retoma, sem alterações, o texto da edição de
%1879.\footnote{ Gabriel Soares de Sousa, \textit{Tratado descritivo do
%Brasil em 1587}. Recife: Editora Massangana, 2000.}
%
%\section{A presente edição}


\begin{bibliohedra}

\tit{azevedo}, Gabriela Soares de. \textit{Leituras, notas, impressões e revelações do 
Tratado descritivo do Brasil em 1587 de Gabriel Soares de Sousa}. Dissertação de Mestrado, \textsc{uerj}, 2007.

\tit{bluteau}, Raphael. \textit{Vocabulario Portuguez \& Latino}. Coimbra: Colégio das Artes, 1712--1728, 8 vols.~e 2 suplementos.

\tit{franco}, Francisco de Assis Carvalho. \textit{Dicionário de Bandeirantes e Sertanistas do Brasil}. São Paulo: Itatiaia, 1986.

\tit{ganns}, Cláudio. “O Primeiro Historiador do Brasil em Espanhol”. 
\textit{Revista do Instituto Histórico e Geográfico Brasileiro}, vol.~238, jan/março, Rio de Janeiro, 1958, pp.~144--168.

\tit{hespanha}, António Manuel. “La Economia de la gracia”, in: A.~M.~Hespanha, 
\textit{La Gracia del derecho: economia de la cultura em la Edad Moderna}. Madri: Centro de Estudios Constitucionales, 1993.

\titidem. “O Governo dos Áustrias e a modernização da 
constituição política portuguesa”. \textit{Penélope}, 2, 1989, pp.~50--73.

\tit{marques}, Guida. “O Estado do Brasil na União Ibérica”. \textit{Penélope}, nº27, 2002, pp.~7--35.

\tit{monteiro}, John M.. \textit{Negros da Terra}. São Paulo: Companhia das Letras, 1994.

\titidem. “As ‘Castas de Gentio’ na América Portuguesa quinhentista: Unidade, 
diversidade e a invenção dos índios no Brasil. In: \textit{Tupis, Tapuias e Historiadores. 
Estudos de História Indígena e do Indigenismo}. Tese Apresentada para o Concurso de Livre Docência Área de Etnologia/\textsc{ifch}"-Unicamp, Campinas, 2001.

\tit{moraes}, Alexandre José Mello. \textit{Brasil Histórico}. Rio de Janeiro, 2ª série, 1, 1866.

\tit{moraes silva}, Antônio de. \textit{Diccionario da Língua Portugueza}. 2ªed. Lisboa. Typographia Lacerdina, 1813 (1789), 2 vols.

\tit{olival}, Fernanda. “Um rei e um reino que viviam da mercê”, in: \textsc{olival}, Fernanda. 
\textit{As Ordens Militares e o Estado moderno. Honra, mercê e venalidade em Portugal (1641--1789)}. Lisboa: Estar, 2000.

\tit{pirajá da silva}, Manuel Augusto. “Notas e comentários”, in: Sousa, Gabriel Soares de.
\textit{Notícia do Brasil}. Introdução e notas de Manuel Antônio Pirajá da Silva. São Paulo: 
Ed.~Patrocinada pelo Dep.~de Assuntos Culturais do \textsc{m.e.c.}, 1974.

\tit{raminelli}, Ronaldi. \textit{Viagens ultramarinas. Monarcas, vassalos e governo à distância}. São Paulo: Alameda, 2008.

\tit{ricupero}, Rodrigo. \textit{Formação da Elite Colonial}. São Paulo: Alameda, 2008.

\titidem. \textit{Honras e Mercês. Poder e patrimônio nos primórdios do Brasil}. Tese de doutoramento apresentada na Faculdade de 
Filosofia, Letras e Ciências Humanas/\textsc{usp}, 2006.

\tit{schaub}, Jean"-Frédéric. \textit{Portugal na Monarquia Hispânica, 1580--1640}. Lisboa: Livros Horizonte, 2001.

\tit{stella}, Roseli Santaella. \textit{O Domínio Espanhol no Brasil durante a Monarquia dos Filipes}. São Paulo: Unibero, 2000.

\tit{varnhagen},  Francisco Adolpho de. \textit{Reflexões criticas sobre o escripto do 
seculo \textsc{xvi} impresso com o título de Notícias do Brasil no tomo 3º da Coll.~De Not.~Ultr.~Acompanhadas 
de interessantes notícias bibliographicas e importantes investigações históricas}. 
Lisboa: Typ.~da Academia, 1839. Rio de Janeiro: Revista do Instituto Histórico Geográfico Brasileiro, 1940.

\titidem. \textit{História Geral do Brasil}. Edição crítica de Capistrano 
de Abreu e Rodolfo Garcia. 3ª ed.~São Paulo: Melhoramentos, [s/d], \mbox{tomo \textsc{i}}.

\titidem. “Breves Comentários à precedente obra de Gabriel Soares”, 
in: Sousa, Gabriel Soares de. \textit{Tratado descritivo do Brasil em 1587}. Rio de Janeiro: 
Tip.~de João Inácio da Silva, 1879, pp.~\textsc{xiii--xxviii}.

\vspace*{2ex}

\textbf{Textos de Gabriel Soares de Sousa}

\tit{sousa}, Gabriel Soares de. \textit{Epístola do Autor a Dom Cristóvão de Moura} e \textit{Declaração do que se contém neste caderno}. 
Biblioteca Guita e José Mindlin, Códice “Descripcion y noticias del Brasil. Sin autor. Anno 1587”.

\titidem. \textit{Roteiro geral com largas informações de toda a Costa que pertence ao Estado do Brasil e a descrição 
de muitos lugares dela, especialmente da Bahia de Todos os Santos}. Biblioteca Guita e José Mindlin, 
Códice “Descripcion y noticias del Brasil. Sin autor. Anno 1587”.

\titidem. \textit{Memorial e declaração das grandezas da Bahia de Todos os Santos, de sua 
fertilidade e das notáveis partes que tem}. Biblioteca Guita e José Mindlin, 
Códice “Descripcion y noticias del Brasil. Sin autor. Anno 1587”.

\titidem. \textit{Tratado descritivo do Brasil em 1587}. Edição castigada pelo estudo e exame de muitos códices 
manuscritos existentes no Brasil, em Portugal, Espanha e França e acrescentada de alguns comentários 
à obra feitos por Francisco Adolfo de Varnhagen. Rio de Janeiro: Typographia Universal de Laemmert, 1851.

\titidem. \textit{Tratado descritivo do Brasil em 1587}. Edição castigada pelo estudo e exame de muitos códices 
manuscritos existentes no Brasil, em Portugal, Espanha e França e acrescentada de alguns comentários 
à obra feitos por Francisco Adolfo de Varnhagen. Segunda edição mais correta e acrescentada com um 
aditamento. Rio de Janeiro: Typographia de João Inácio da Silva, 1879.

\titidem. \textit{Notícia do Brasil}. Introdução e notas de Manuel Augusto Pirajá da Silva. 
São Paulo: Martins, 1945, 2 vols.

\titidem. \textit{Tratado descritivo do Brasil em 1587}. São Paulo: Companhia Editora Nacional, Série Brasiliana, 1971.

\titidem. \textit{Notícia do Brasil}. Comentários e notas de Varnhagen, Pirajá da Silva e Edelweiss. 
São Paulo: edição patrocinada pelo Dep. de Assuntos Culturais do \textsc{m.e.c.}, 1974.

\titidem. \textit{Tratado descritivo do Brasil em 1587}. Organizado por Leonardo Dantas. Recife: Massangana, 2000.

\end{bibliohedra}

