\part{\textsc{apêndices}}


\chapter[O Testamento de Gabriel Soares de Sousa]{O Testamento de\break Gabriel Soares de Sousa}
\hedramarkboth{Testamento}{Gabriel Soares de Sousa}
\vspace*{-3em}


\textit{Antes de embarcar para a
Europa, Gabriel Soares de Sousa deixou, em 10 de agosto de 1584, um
testamento, cuja abertura se deu em 10 de julho de 1592. Inscrito no
Livro do Tombo do Mosteiro de São Bento, em Salvador, o texto foi
editado pela primeira vez, em 1866, pelo historiador Alexandre José
Mello Moraes (Vide A.J.~Mello Moraes,} Brasil Histórico. \textit{Rio de
Janeiro, 2ª série, 1, 1866, pp.~248 e 251--52). Varnhagen, no
“Aditamento” à edição de 1879, publicou algumas cláusulas do documento
(Cf.~Gabriel Soares de Sousa,} Tratado Descritivo do Brasil em
1587. \textit{Rio de Janeiro, Tip. de João Inácio da Silva, 1879, pp.
\textsc{xiii--xxviii}.)}
\bigskip

\textsc{Jesus Maria,}
\medskip

Em nome do Padre e do filho e do Espírito Santo, amém. Saibam quantos
este instrumento virem que, no ano do nascimento de nosso Senhor Jesus
Cristo de mil e quinhentos e oitenta e quatro anos, aos dez dias do mês
de agosto, na cidade do Salvador, estando eu, Gabriel Soares de Souza,
de caminho para Espanha,\footnote{ Gabriel Soares lavrou seu testamento
pouco antes da viagem que fez à Espanha, onde solicitou ao rei Filipe
\textsc{ii} permissão e mercês para realizar uma expedição pelo sertão baiano em
busca de pedras e metais preciosos.} disposto, são e bem valente em
todo o meu entendimento e perfeito juízo, assim e da maneira que a Deus
em mim pospondo o pensamento em meus pecados, temendo a estreita conta
que deles hei de dar a nosso Senhor, determinei fazer este meu
testamento em o qual declaro minha derradeira vontade, o que se
cumprirá e guardará inteiramente como abaixo e ao diante vai declarado,
sem lhe pôr dúvida ou embargo algum. 

Primeiramente, encomendo minha alma agora e sempre e quando deste corpo
se apartar a nosso Senhor Jesus Cristo, a quem humilde peço perdão de
todos os meus pecados à honra das cinco chagas que ele padeceu na
árvore da Cruz Santa e à honra de todos os mistérios de sua sagrada
morte e paixão; a quem peço que não julgue minhas culpas com aquela ira
que pela graveza delas estou merecendo, senão com a grandeza de sua
divina misericórdia, em a qual ponho a esperança de minha salvação, em
o favor e a ajuda da sacratíssima Virgem Maria Nossa Senhora sua
Majestade, a quem afincadamente peço que se lembre deste seu devoto à
honra daqueles quinze mistérios que se encerram no seu santo Rosário,
de quem fui sempre devoto, ainda que o não rezasse com aquela limpeza e
devoção a que sou obrigado, mas confio na sua santa piedade que não
será isso parte para deixar de ser minha advogada, pois a ela sempre 
foi e é dos pecadores; mas, como me eu conheço por maior que todos, com
toda eficácia lhe peço me não desampare, pois sempre socorreu as
pessoas dos que por ela chamaram.

Tomo por meu advogado ao Anjo Gabriel, cujo nome tenho, do qual não fui
capaz, pois me entreguei tanto aos pecados ao qual peço, à honra e
louvor do paraíso, de que ele tanto goza, e à honra daquela Santa
embaixada que ele levou à Virgem nossa Senhora, que seja terceiro para
com ela para que ela seja diante de seu precioso filho e deste me
alcance perdão de meus pecados. Outrossim, tomo por advogado o anjo da
minha guarda para que com o favor da Virgem Madre de Deus defenda esta
alma pecadora do inimigo tentador, para que me não tente nem perturbe
na hora da morte, em a qual protesto de acabar como fiel cristão firme
e forte com a esperança que tenho nas santas chagas de Cristo, em cuja
fé protesto de morrer e viver. Tomo por advogado a nosso glorioso Padre
São Bento, de cuja ordem sou irmão, mas na vontade sou frade professo,
a quem humildemente peço me não desampare e me recolha debaixo do seu
amparo, pois tamanha vontade tenho de o servir e ajudar a aumentar sua
religião. Outrossim, tomo por advogado ao santíssimo Padre São
Francisco e ao Senhor São Domingos, de cujas ordens sou irmão há muitos
anos, ainda que ruim, pois tão mal os tenho servido, de que lhes peço
perdão e que não bastem minhas culpas para deixarem de ser meus
advogados diante de Deus; aos quais peço que eles me alcancem, que eu
possa gozar das indulgências, sacrifícios, orações, esmola de que gozam
os seus frades e irmãos, assim na morte como na vida. Outrossim, tomo
por meu advogado ao bem"-aventurado Santo Alberto, da ordem da Madre de
Deus monte do Carmo, em cuja irmandade entrei, do que me não quis nunca
aproveitar e andei sempre como ovelha perdida, mas, já que Deus me
chegou a este tempo, peço ao bem"-aventurado Santo que terce por mim 
diante desta Senhora e me alcance dela perdão dos erros passados para
que me deixe gozar, do que gozam os seus frades e irmão da sua Santa
Ordem, com o que tenho grande esperança de me salvar. Encomendo mais
minha alma ao bem"-aventurado São João Batista e a todos os santos
apóstolos, aos gloriosos mártires São Lourenço e São Sebastião e a
todos os santos e santas da Corte do Céu, aos quais peço que todos
juntos e cada um por si roguem por mim a nosso Senhor e lhe peçam
perdão de meus pecados por mim a nosso Senhor, e lhe peçam perdão de
meus pecados por mim, e me leve a sua santa glória, para que foi
criado. Donde quer que eu falecer, me enterrarão no hábito de São
Bento, e havendo ali mosteiros de sua ordem, onde me enterrarão; não
havendo maneira deste hábito, e havendo Mosteiro de São Francisco, me
enterrarão no seu hábito, e os religiosos destas duas ordens me
acompanharão, e a cada um darão de esmola cinco mil Réis, e pelo hábito
dez cruzados. 

Se Deus for servido que eu faleça nesta cidade e capitania, meu corpo
será enterrado em São Bento da dita cidade na capela"-mor, onde se me
porá uma campa\footnote{ Pedra que se coloca na superfície de
sepulturas.} com um letreiro que diga \textit{aqui jaz um pecador}, o
qual estará no meio de um escudo que se lavrará na dita campa; e, sendo
Deus servido de me levar no mar ou em Espanha, todavia se porá na dita
capela"-mor a dita campa com o dito letreiro, em a qual sepultura se
enterrará minha mulher Anna de Argollo. Acompanhará meu corpo, se
falecer nesta Cidade, o cabido,\footnote{ Conjunto de clérigos.} a quem
se dará a esmola costumada, e aos Padres de São Bento levarão de oferta
um porco e seis almudes\footnote{ Medida de capacidade equivalente, no
Brasil, a 32 litros.} de vinho e cinco Cruzados. Acompanhar"-me"-ão dois
pobres, com cada um a sua tocha ou círios\footnote{ Grande vela de
cera.} nas mãos, e darão de aluguel à confraria de onde forem um
Cruzado de cada uma e, a cada pobre, pelas levarem, dois tostões; não
dobrarão os sinos por mim e somente se farão os sinais que se fazem por
um pobre quando morre. Deixo à Casa da Santa Misericórdia desta cidade
quarenta mil Réis de esmola para se dourar o retábulo e, por me
acompanhar, cinco mil Réis. Deixo à Confraria do Santíssimo Sacramento
cinco mil Réis e à de Nossa Senhora do Rosário dois mil Réis; far"-me"-ão
no mosteiro de São Bento, quer faleça nesta capitania, quer em outra
qualquer parte, três ofícios de nove lições em três dias a fio, tanto
que eu falecer ou se souber a certeza de minha morte; em cada ofício se
dará de oferta um porco e cinco alqueires\footnote{ Medida de
capacidade, de volume variável, que equivalia a aproximadamente catorze
litros em Portugal.} de farinha; e não me farão pompa nenhuma, somente
me porão um pano preto no chão com dois bancos cobertos de preto, e, em
cada um, cinco velas acesas em cada ofício; destes me dirão cinco
missas rezadas à honra das cinco chagas de nosso Senhor Jesus Cristo,
com seus responsos sobre a sepultura. Nos outros dias seguintes, me
dirão em três dias a fio cinco missas cada dia, rezadas as primeiras
cinco à honra dos gozos de nossa Senhora e, a outro dia, as outras
cinco à honra dos cinco passos dolorosos da Madre de Deus e, ao
terceiro dia, outras cinco à honra dos cinco mistérios gloriosos da
Madre de Deus conforme a contemplação do Rosário. Medirão na mesma
casa, acabados os ofícios, mais cento e cinquenta missas rezadas e
quinze cantadas, e as cantadas darão de oferta a cada uma sua galinha e
canada\footnote{ Antiga medida para líquidos que equivalia a
aproximadamente dois litros.} de vinho, e umas e outras sairão com seu
responso sobre minha sepultura, e as missas se repartirão pela maneira
seguinte: nos primeiros cinco dias se dirão em cada dia dez missas
rezadas e uma cantada, como acima fica dito, à honra dos prazeres que
se contemplam no Rosário de Nossa Senhora; em os outros cinco dias logo
seguintes se dirão em cada dia outras dez missas rezadas e uma cantada
à honra dos cinco mistérios gloriosos da Virgem Nossa Senhora da Madre
de Deus, e, se não houver padres no dito mosteiro que bastem para dizer
estas missas juntas humildemente, peço ao Padre Dom Abade que ordene
com os padres do Colégio\footnote{ Jesuítas.} ou da Sé com que se
possam dizer estas missas juntas como tenho declarado, porque tenho
confiança na Madre de Deus que, no cabo destas missas, saia minha alma
do purgatório. Como se acabarem de dizer estas missas, como tenho
declarado, a outro dia seguinte se me diga um ofício de nove lições
como os que acima tenho declarado. Mando que se digam pela alma de meu
pai e mãe cinquenta missas rezadas, as quais se dirão como se acabarem,
as que acima tenho declarado. 

Mando que se tomará de minha fazenda valia de quinhentos Cruzados, e se
partirão por cinco moças pobres a cem cruzados por cada uma, para ajuda
de seus casamentos, o que repartirá o Padre Dom Abade com informação do
provedor da Santa Misericórdia. Eu tenho duas irmãs viúvas, uma se
chama Dona Margarida de Souza e outra Maria Velha, ambas moradoras em 
Lisboa, e não tenho herdeiro forçado, e darão a cada uma delas, de
minha fazenda, do rendimento dela, vinte mil Réis a cada uma, e,
falecendo alguma delas ou sendo falecida, darão à que ficar viva cada
ano quarenta mil Réis em sua vida, tão somente os quais lhe mandarão
por letra à Lisboa, de maneira que lhe seja paga a dita quantia.
Declaro que tenho um livro das contas que tenho com as pessoas a quem
devo, pelo qual se fará conta com as pessoas a quem estou em obrigação,
ao pé de cujo título fica assinado por mim, ao qual livro se dará
inteiro crédito, porque pelas declarações dele deixo desencarregada
minha consciência. Nesse mesmo livro, de minha razão tenho escrito o
que tenho de meu, assim de fazenda de raiz como escravos, bois de carro
e éguas e outros móveis e índios forros, e nele tenho em lembrança os
encargos em que estou, assim às pessoas que me servem e serviram, como
a outras pessoas, ao qual dará outrossim inteiro crédito, porque o fiz
só a fim de consertar minha consciência, o que não posso tratar nem
esmiuçar neste testamento pelas mudanças que o tempo faz e eu não saber
qual há de ser a derradeira hora em que o Senhor há de chamar"-me, para
a qual não achei melhor remédio que este. Depois de meu falecimento se
ordenará o inventário de minha fazenda, e se fará conta do que devo, e
se porá em ordem de se pagarem minhas dívidas para o que se venderão os
móveis de casa, bois e éguas e açúcar que se achar, e para o que restar
se concertarão meus testamenteiros com os credores para se pagarem
pelos rendimentos de minha fazenda, se disso forem contentes, o que se
há de negociar de maneira que minha alma não pene na outra vida por
isso; e, não querendo eles esperar em tal caso, se arrendará o engenho
de antemão ou se venderão as novidades dele; enquanto isto bastar, se
venderão as terras que tenho no Jaqueriça,\footnote{ Jequiriçá.} que,
com as éguas e fazenda, que valem muito, por serem muitas e boas, em
tudo farão meus herdeiros, de maneira que fique desencarregado.

Declaro por meus testamenteiros ao Reverendo Padre Frei Antonio Ventura
e a minha mulher Anna de Argollo, para que ambos façam cumprir este meu
testamento como nele se contém, e, sendo caso que ela ou por não poder
estar presente na cidade ou por suas indisposições, possa acudir e
fazer cumprir este meu testamento, que tudo o feito pelo Reverendo
Padre somente seja valioso; e, porque o tempo faz grandes mudanças que
com elas há morrer e ausentar, não podendo por algum lícito impedimento
cumprir o Reverendo Padre este meu Testamento, digo que em tal caso
seja meu testamenteiro o Reverendo Padre que lhe suceder no cargo de
Abade no dito Mosteiro de São Bento, mas, ainda que o Reverendo Frei
Antonio Ventura não seja Abade, quero que sempre ele seja meu
testamenteiro.

 Como Nosso Senhor não foi servido que eu tivesse filhos de minha
mulher, nem outros alguns, nem sobrinhos filhos de meus irmãos, nem
tenho herdeiros forçados a quem pertença minha fazenda; e, porque não
herdei de meu pai, nem de meus avôs, e adquiri por minha indústria e
trabalho e porventura alguns encargos de consciência que ora não sei; e
declaro por meu herdeiro de toda a minha fazenda ao Mosteiro de São
Bento da Cidade do Salvador da Bahia de Todos os Santos, com condição
que eu e minha mulher Ana de Argollo nos enterremos ambos na dita
capela"-mor, que ora é; e, falecendo antes que se faça a capela"-mor da
Igreja nova, passarão a nossa ossada à dita capela"-mor, onde estava a
minha sepultura com a campa no meio da capela com o letreiro que está atrás
declarado.

Serão obrigados o Abade que agora é, e o diante for, e religiosos me
dizer, em cada dia, uma missa rezada por minha alma, para enquanto o
mundo durar com o seu responso sobre a sepultura, e, cada ano, pela
semana dos santos, um ofício de nove lições. E, sendo caso de Deus se
sirva de me levar para si no mar ou em Espanha, de onde meus ossos não
podem ser trazidos a este Mosteiro, digo que, sem embargos disso, se me
ponha esta sepultura na capela"-mor dele, para lembrança de se me dizer
o responso; sob ela espera se enterrar minha mulher tão somente.
Declaro que os chãos que tenho nesta cidade, que houve de Antonio da	\EP[-1]
Fonseca, de Anna de Paiva, de Pedro Fernandes e de Braz Afonso, e a
terra que tenho valada no caminho da Vila Velha da banda do mar e da
outra banda, que foi de Antonio de Oliveira, queria que ficasse tudo a
meu quinhão, por tudo ser mui necessário para o Mosteiro, onde se podem
fazer muitas tercenas\footnote{ No original, “terisinas”. Tercenas ou
terracenas são armazéns construídos na beira de rios ou junto de cais,
para guardar cereais, armamentos e munições.} ao longo do mar pelo
alugar e, pelo caminho acima, muitos foros de casas e muitas casas ao
longo da estrada, que, tudo pelo tempo em diante, virem a render muito
para o convento; e, porque hei este testamento por acabado, pelo qual
hei por revogados todos os que dantes tenho feito, este só quero que
valha porque esta é minha derradeira vontade; o qual fiz por minha mão,
assinado por mim, \textit{Gabriel Soares de Souza}. 


\chapter[Ao Instituto Histórico do Brasil]{Ao Instituto Histórico do Brasil}
\hedramarkboth{ao instituto histórico do brasil}{f. a. de varnhagen}
\hedramarkboth{Ao Instituto Histórico}{}

\textsc{Senhores:}

Sabeis como a presente obra de Gabriel Soares, talvez a mais admirável 
de quantas em português produziu o século quinhentista, 
prestou valiosos auxílios aos escritos do padre Cazal e dos 
contemporâneos Southey, Martius e Denis, que dela fazem menção com 
elogios não equívocos. 

Sabeis também como as \textit{Reflexões críticas} que sobre essa obra 
escrevi foram as primícias que ofereci às letras, por intermédio da 
Academia das Ciências de Lisboa que se dignou, ao acolhê"-las no corpo 
de suas memórias, contar"-me nos do seu grêmio. Sabeis como aquela obra corria espúria, pseudônima e corrompida no título e na data, quando 
as \textit{Reflexões críticas} lhe restituíram genuinidade de doutrina e    
legitimidade de autor e de título, e lhe fixaram a verdadeira idade. 
Sabereis, finalmente, como nada tenho poupado para restaurar a obra, 
que por si constitui um monumento levantado pelo colono Gabriel 
Soares à civilização, colonização, letras e ciências do Brasil em 1587.

Essa restauração dei"-a por enquanto por acabada; e desde que o Sr. 
Ferdinand Denis a inculcou ao público europeu, com expressões tão 
lisonjeiras para um de vossos consócios, creio que devemos 
corresponder a elas provando nossos bons desejos, embora a realidade 
do trabalho não vá talvez corresponder à expectativa do ilustre escritor 
francês quando disse: 

\begin{hedraquote}
Ce beau livre [\ldots] a été l'objet d'une [\ldots] 
(permiti"-me, Senhores, calar o epíteto com que me quis favorecer) [\ldots] 
dissertation de M. Adolfo de Varnhagen. Le [\ldots] écrivain que nous 
venons de nommer a soumis les divers manuscrits de Gabriel Soares à un 
sérieux examen, il a vu même celui de Paris, et il est le seul qui puisse
donner aujourd'hui une édition correcte de cet admirable traité, si 
précieux pour l'empire du Brésil.
\end{hedraquote}

Sem me desvanecer com as expressões lisonjeiras que acabo de 
transcrever do benévolo e elegante escritor, não deixo de me reconhecer 
um tanto habilitado a fazer"-vos a proposta que hoje vos faço de 
imprimirdes o códice que vos ofereço.

Não há dúvida, Senhores, que foi o desejo de ver o exemplar da 
Biblioteca de Paris o que mais me levou a essa capital do mundo 
literário em 1847. Não há dúvida que, além desse códice, tive eu ocasião 
de examinar uns vinte mais. Vi três na Biblioteca Eborense, mais três na 
Portuense e outros na das Necessidades em Lisboa. Vi mais dois 
exemplares existentes em Madri; outro mais que pertenceu ao convento 
da congregação das Missões e três da Academia de Lisboa, um dos 
quais serviu para o prelo, outro se guarda no seu arquivo, e o terceiro na 
livraria conventual de Jesus. Igualmente vi três cópias de menos valor 
que há no Rio de Janeiro (uma das quais chegou a estar licenciada para a 
impressão); a avulsa da coleção de Pinheiro na Torre do Tombo, e uma 
que em Neuwied me mostrou o velho príncipe Maximiliano, a quem na 
Bahia fora dada de presente. Em Inglaterra deve seguramente existir, 
pelo menos, o códice que possuiu Southey; mas foram inúteis as buscas 
que aí fiz após ele, e no Museu Britânico nem sequer encontrei notícia 
de algum exemplar.

Nenhum daqueles códices porém é, a meu ver, o original; e 
baldados foram todos meus esforços para descobrir este, seguindo as 
indicações de Nicolau Antônio, de Barbosa, de Leon Pinelo e de seu 
adicionador Barcia. Na Biblioteca de Cristóvão de Moura, hoje existente 
em Valência e pertencente ao Príncipe Pio, posso assegurar"-vos que não 
existe ele, pois que, graças à bondosa amizade desse cavalheiro, me foi 
permitido desenganar"-me por meu próprio exame. A livraria do conde 
de Vila"-Umbrosa guarda"-se incomunicável na ilha de Malhorca, e não 
há probabilidade de que quando nela se ache ainda o códice que 
menciona Barcia, possa ele ser o original. A do conde de Vimieiro foi 
consumida pelas chamas, as quais pode muito bem ser que devorassem 
os cadernos originais do punho do nosso colono.

Graças, porém, às muitas cópias que nos restam, a uma das de 
Evora, sobretudo, creio poder dar no exemplar que vos ofereço o 
monumento de Gabriel Soares tão correto quanto se poderia esperar sem 
o original, enquanto o trabalho de outros e a discussão não o 
aperfeiçoem ainda mais, como terá de suceder.

Acerca do autor talvez que o tempo fará descobrir na Bahia mais 
notícias. Era filho de Portugal, passou à Bahia em 1570, fez 
se senhor"-de"-engenho e proprietário de roças e fazendas em um sítio 
entre o Jaguaribe e o Jequiriçá. Voltando à península, dirigiu"-se a 
Madri, onde estava no 1º de março de 1587, em que ofertou seu livro a 
Cristóvão de Moura, por meio da seguinte carta: 
\begin{hedraquote}
Obrigado de minha curiosidade, fiz, por espaço de 17 anos que 
residi no Estado do Brasil, muitas lembranças por escrito do que me 
pareceu digno de notar, as quais tirei a limpo nesta corte em este 
caderno, enquanto a dilação de meus requerimentos me deu para isso 
lugar; ao que me dispus entendendo convir ao serviço de El"-Rei Nosso 
Senhor, e compadecendo"-me da pouca notícia que nestes reinos se tem 
das grandezas e estranhezas desta província, no que anteparei algumas 
vezes, movido do conhecimento de mim mesmo, e entendendo que as 
obras que se escrevem têm mais valor que o da reputação dos autores 
delas. Como minha tenção não foi escrever história que deleitasse com 
estilo e boa linguagem, não espero tirar louvor desta escritura e breve 
relação (em que se contém o que pude alcançar da cosmografia e 
descrição deste Estado), que a V.~S.~ofereço; e me fará mercê aceitá"-la, 
como está merecendo a vontade com que a ofereço; passando pelos 
desconcertos dela, pois a confiança disso me fez suave o trabalho e 
tempo que em a escrever gastei; de cuja substância se podem fazer 
muitas lembranças à S. M. para que folgue de as ter deste seu Estado, a 
que V.~S.~faça dar a valia que lhe é devida; para que os moradores dele 
roguem a Nosso Senhor guarde a mui ilustre pessoa de V.~S.~e lhe acrescente a 
vida por muitos anos. Em Madri o 1º de março de 1587, \textit{Gabriel Soares de Sousa}. 
\end{hedraquote}

Para melhor inteligência das doutrinas do livro acompanho esta 
cópia dos comentos que vão no fim. Preferi este sistema ao das notas 
marginais inferiores, que talvez seriam para o leitor de mais 
comodidade, porque não quis interromper com a minha mesquinha 
prosa essas páginas venerandas de um escritor quinhentista. Abstive"-me 
também da tarefa, aliás enfadonha para o leitor, de acompanhar o texto 
com variantes que tenho por não"-legítimas. 

Esta obra, doze anos depois, já existia em Portugal ou por cópia ou 
em original; e em 1599 a cita e copia Pedro de Mariz na segunda edição 
de seus \textit{Diálogos.} Mais tarde, copiou dela Fr.~Vicente de Salvador, e, 
por conseguinte, o seu confrade Fr.~Antônio Jaboatão. Simão de 
Vasconcellos aproveitou do capítulo 40 da 1ª parte as suas Notícias 51 a 
55, e do capítulo 70 a Notícia 66.\footnote{ Varnhagen se refere aqui às obras de Frei 
Simão de Vasconcellos, \textit{Notícias curiosas e necessárias das cousas do Brasil} (1668); de 
Frei Antonio de Santa Maria Jaboatão, \textit{ Novo orbe sefárico brasílico} (1761); e de 
Frei Vicente do Salvador, \textit{ História do Brasil} (1627).}  

Assim, se vós o resolverdes, vai finalmente correr mundo, de um 
modo condigno, a obra de um escritor de nota. Apesar dos
grandes dotes do autor, que o escrito descobre, apesar de ser a obra tida 
em conta, como justificam as muitas cópias que dela se tiraram, mais de 
dois séculos correram sem que houvesse quem se decidisse a imprimi"-la 
na íntegra. As mesmas cópias por desgraça foram tão mal tiradas, que 
disso proveio que o nome do autor ficasse esgarrado, o título se trocasse 
e até na data se cometessem enganos!

Pesa"-nos ver nos tristes azares desse livro mais um desgraçado 
exemplo das injustiças ou antes das infelicidades humanas. Se essa obra 
se houvesse impresso pouco depois de escrita, estaria hoje tão popular o 
nome de Soares como o de Barros.\footnote{ Referência a João de Barros (c.~1496--1570). 
Ver nota 18 da Primeira Parte do texto.} Nosso autor é singelo, quase 
primitivo no estilo, mas era grande observador, e, ao ler o seu livro, vos 
custa a descobrir se ele, com estudos regulares, seria melhor geógrafo 
que historiador, melhor botânico que corógrafo, melhor etnógrafo que zoólogo.

Em 1825 realizou a tarefa da primeira edição completa a Academia 
de Lisboa; mas o códice de que teve de valer"-se foi infelizmente pouco 
fiel, e o revisor não entendido na nomenclatura das coisas da nossa terra. 
Ainda assim muito devemos a essa primeira edição; ela deu 
publicamente importância ao trabalho de Soares, e sem ela não teríamos 
tido ocasião de fazer sobre a obra os estudos que hoje nos fornecem a 
edição que proponho, a qual, mais que a mim, a deveis à corporação 
vossa coirmã, a Academia Real das Ciências de Lisboa.\\

\noindent\textit{F. A. de Varnhagen}  \hspace{\stretch{1}}\textit{Madri, 1º de março de 1851}\\


\chapter[Breves comentários]{Breves comentários \subtitulo{à precedente obra de Gabriel Soares}}
\hedramarkboth{Breves comentários}{F. A. de Varnhagen}

\section*{Introdução}
\noindent Quando em princípios de março deste ano escrevíamos em Madri a 
dedicatória que precede a presente edição da obra de Gabriel Soares, e 
lhe serve como de prefácio, não podíamos imaginar que tão cedo 
veríamos em execução a nossa proposta, e menos podíamos adivinhar 
que concorreríamos até para realizar, sendo ao chegar à corte chamado a 
desempenhar as funções do cargo de primeiro"-secretário do nosso 
Instituto Histórico, cargo a que, pelos novos estatutos, anda anexa a 
direção dos anais que há catorze anos publica esta corporação.

Animados pelo voto da maior parte dos nossos consócios, 
entregamos ao prelo o manuscrito da obra sobre que tanto tínhamos 
trabalhado, e seguimos com igual voto sua impressão, sem desfeiteá"-la 
com interrupções. E dando"-nos por incompetente para a revisão das 
provas de um livro que quase sabemos de cor, tivemos a fortuna de 
alcançar nessa parte a coadjuvação do nosso amigo e consócio o Sr. Dr. 
Silva, que se prestou a esse enfadonho trabalho com o amor do estudo 
que o distingue. Ainda assim, tal era a dificuldade da empresa, que nos 
escaparam na edição algumas ligeiras irregularidades e imperfeições que 
se levantarão na folha de erratas ou se advertirão nestes comentários que 
ora redigimos, com maior extensão do que os que havíamos escrito em 
Madri, e que mencionamos na dedicatória. É mais difícil do que parece a 
empresa de restaurar um códice antigo do qual existem, em vez do 
original, uma infinidade de cópias mais ou menos erradas em virtude de 
leituras erradas feitas por quem não entendia do que lia.

O tempo fará ainda descobrir algumas correções mais que 
necessitar esta obra, já pelo que diz respeito a nomes de lugares que hoje 
só poderão pelos habitantes deles ser bem averiguados, já por 
alguns nomes de pássaros, insetos, e principalmente de peixes não 
descritos nos livros, e só conhecidos dos caçadores, roceiros e 
pescadores.

Nos presentes comentários não repetiremos quanto dissemos nas  \textit{Reflexões críticas}, 
escritas ainda nos bancos das aulas com o tempo que 
forrávamos depois de estudar a lição.

Além de havermos em alguns pontos melhorado nossas opiniões, 
evitaremos aqui de consignar citações que pudessem julgar"-se nascidas 
do desejo de ostentar erudição; desejo que existiu em nós alguma vez, 
quando principiantes, por certo que já hoje nos não apoquenta.

Alguém quereria talvez que aproveitássemos para esta edição 
muitas notícias que, por ventura deslocadas, se encontram nas \textit{Reflexões críticas}. 
De propósito, porém, não quisemos sobrecarregar mais estes 
comentários; além de que as notícias úteis que excluímos serão 
unicamente algumas bibliográficas de obras inéditas, cuja existência 
queríamos acusar aos literatos, e esse serviço já está feito. Muitos dos 
nossos atuais comentos versarão sobre as variantes dos textos, e sobre as 
diferenças principais que houver entre a nossa edição e a da Academia 
das Ciências de Lisboa (Tomo 3º das \textit{Memórias ultramarinas}.)

Não faltará, talvez, quem censure o não havermos dado melhor 
método ao escrito de Soares acompanhando"-o de notas que facilitassem 
mais a sua leitura. Repetimos que não ousamos ingerir nossa 
mesquinha pena em meio dessas páginas venerandas sobre que já 
pesam quase três séculos. Nem sequer nelas ousamos introduzir o título 
--- \textit{Tratado descritivo do Brasil} --- que adotamos no rosto para melhor 
dar a conhecer o conteúdo da obra; pelo contrário, conservamos 
efetivamente em toda esta o título com que já ela é conhecida e citada de 
--- \textit{Roteiro geral} --- que, aliás só compete à primeira parte. O que sim 
fizemos a benefício dos leitores foi redigir um índice lacônico e claro, 
introduzindo nele, por meio de vinte títulos, a divisão filosófica da 
segunda parte, sem em nada alterar a ordem e numeração dos capítulos. 
Cremos com este índice, que será publicado em seguida destes 
comentários, ter feito ao livro de Soares um novo serviço.

O público sabe já como este livro corria anônimo, sendo que Cazal, 
Martins e outros o iam quase fazendo passar por obra de um tal 
Francisco da Cunha, quando as \textit{Reflexões críticas}, para acusar dele o 
autor, idade e título, chamaram a atenção dos literatos sobre o que 
haviam consignado: 1º) a \textit{Biblioteca Lusitana} (Tomo 2º, p. 321); 2º) a 
obra de Nicolau Antônio (Tomo 1º, \textit{Comentários},
p. 509 e Tomo 2º, p. 399); 3º) a do adicionador do americano Pinello, 
o espanhol Barcia (Tomo 2º, col. 680 e Tomo 3º, col. 1710); e 4º) o 
próprio autor, que consignou seu nome na sua obra (Parte 1ª, cap. 40 e 
Parte 2ª, caps. 29, 30, 127 e 177).

Como sobre cada um dos capítulos de Soares temos alguma 
reflexão a fazer, para não introduzirmos nova numeração e adaptarmos 
melhor os comentários à obra a que se destinam, os numeraremos 
sucessivamente, segundo os capítulos; assim, desde o 1º até o 74º, 
serão eles referentes aos respectivos capítulos da 1ª parte; os 75, 76, 77, 
etc, pertencerão aos 1º, 2º, 3º, etc, da 2ª parte; de modo que a 
numeração do capítulo desta última a que se refere o comentário será 
conhecida logo que ao número que tiver este se abater o mesmo 74. E 
vice"-versa; adicionando"-se 74 ao número do capítulo da 2ª parte, se 
terá o do comentário respectivo. Assim o índice da obra, com os seus 
títulos etc., poderá também consultar"-se como índice destes.

\section*{Comentários}
\medskip

\begin{enumerate}
\item  O princípio desta obra contém na parte histórica muitos erros, nascidos de escrever o 
autor só talvez por tradição, tantos anos depois dos sucessos que narra. A costa do 
Brasil foi avistada por Cabral aos 22 de abril, e não aos 24. A missa de posse teve lugar 
no dia 1º de maio, e a 3 já a frota ia pelo mar afora. Coelho voltou à Europa logo 
depois, e não quando já reinava D. João \textsc{iii}, o que equivalia a dizer uns vinte anos mais 
tarde. Cristóvão Jacques foi mandado por este último rei como capitão"-mor da costa, 
mas não foi o descobridor da Bahia, que estava ela descoberta mais de vinte anos antes. 
Pero Lopes passou a primeira vez ao Brasil com seu irmão Martim Afonso em 1530 e, 
por conseguinte, depois de Jacques, a respeito de quem se pode consultar a memória 
que escrevemos, intitulada \textit{As primeiras negociações diplomáticas respectivas ao Brasil.}

\item O texto da Academia de Lisboa nomeia erradamente Clemente \textsc{vii} como autor da 
bula em favor dos reis católicos, o que deve ter procedido de nota marginal, de algum 
ignorante possuidor de códice, que o copista aproveitasse.

\item Acerca das informações que dá o autor dos terrenos ao norte do Amazonas, cumpre 
advertir que essa parte da costa era então pouco frequentada pelos nossos; e, portanto, 
neste capítulo, como no que diz respeito à doutrina do 1º, o nosso A. não pode servir 
para nada de autoridade.

\item O descobrimento do Amazonas por Orellana foi em 1541; a sua vinda da Espanha em 
meados de 1545; e a expedição de Luís de Melo por 1554. A ida deste cavalheiro à Índia, 
em 1557, e seu naufrágio, em 1573. --- Consulte"-se Diogo de Couto, 
\textit{Década} 7ª, livro 5º, cap. 2º, e \textit{Década} 9ª, cap. 27; e Antônio Pinto Pereira, Parte 
2ª, p. 7 e 58.

\item  À vista da posição em que se indicam os baixios, deduz"-se que o A.
se refere à baia de São José; e, portanto, a ilha em que naufragou Aires
da Cunha deve ser a de Santa Ana, que terá a extensão que lhe dá Soares,
quando a do Medo ou do Boqueirão não tem uma légua. Macaréu é o termo verdadeiramente português para o que nós chamamos, como na 
língua dos indígenas, \textit{pororoca}. É o fenômeno chamado \textit{Hyger} e \textit{Bore} no Severn e Parret. 
Na França também o tem a Gironda, com o nome cremos que de \textit{Mascaret}. A do Amazonas 
é descrita por Condamine, e também nos \textit{Jornais} de Coimbra n. 30 e 87.

\item Este Rio Grande é o atual Parnaíba.

\item O Monte de Li, talvez assim chamado porque se parecia ao de igual nome na Ásia, 
será o de Aracati. Os atlas de Lázaro Luís e Fernão Vaz Dourado e outros antigos 
manuscritos trazem aquele nome.

\item Este nome de Cabo Corso aqui repetido vem em muitas cartas antigas e modernas; o 
que se não dá a respeito do outro do capítulo 3.

\item Neste capítulo se contém a história do castelhano feito botocudo, que se embarcou 
para a França, e deu talvez origem a unir"-se este fato ao nome de Diogo Álvares, o 
Caramuru. Veja a nossa dissertação sobre o assunto, que o Instituto se dignou premiar.

\item  É hoje sabido pelos documentos que encontramos na Torre do
Tombo, como esta capitania de Barros era mista, sendo ele donatário ao
mesmo tempo que Fernão Álvares de Andrade e Aires da Cunha de 225
léguas de costa, e não de cinquenta separadas só para ele. A expedição teve
lugar por outubro de 1535.

\item  Baer, vulgarmente chamado Barleus, chama à baía da Traição \textit{Tebiracajutiba}, 
o que corresponde talvez ao nosso \textit{Acajutibiro}, que Cazal leu
(Tomo 1º, p. 197) \textit{Acejutibiró}.

\item A respeito da colonização da Paraíba deve consultar"-se a obra especial mandada 
escrever pelo Pe.~Cristóvão de Gouveia; dela temos por autor o  Pe.~Jerônimo 
Machado.

\item \textit{Pitagoares}, diz aqui o nosso autor. Outros escrevem \textit{Pitaguaras}, o que quereria 
dizer que estes índios se sustentavam de camarões. \textit{Tabajaras} significa os habitantes das 
aldeias, e era o nome que se dava a todos os indígenas que viviam aldeados.

\item \textit{Aramama} deve ser o mesmo rio \textit{Guiramame} mencionado na \textit{Razão do Estado do Brasil}, 
obra citada por Morais no \textit{Dicionário}, e que hoje temos certeza de haver sido escrita pelo 
próprio governador D. Diogo de Menezes. \textit{Abionabiajá} há de ser a lagoa \textit{Aviyajá} citada na conhecida \textit{Jornada do Maranhão}.

\item Rio de \textit{Igaruçu} ou de \textit{Igara"-uçu} quer dizer rio da Canoa Grande ou rio da Nau. Este 
nome denuncia que o sítio era frequentado por navios europeus.

\item A doação de Duarte Coelho era de sessenta léguas de costa, e não de cinquenta.

\item Ponta de \textit{Pero Cavarim}, P. Lopes (\textit{Diário}, p. 11) disse: \textit{Percauri}. Pimentel 
escreveu (p. 215) \textit{Pero Cabarigo}, a mesma ortografia seguiu Antônio Mariz Carneiro. 
O nome era naturalmente de objeto indígena, e degenerou em outro que se poderia crer 
de algum piloto europeu.

\item As notas que o texto acadêmico admitiu a este capítulo que trata do litoral da atual 
província de Alagoas, são evidentemente estranhas a ele, pois uma até refere um fato de 
1632. --- Aqui as daremos corretas, para evitar ao leitor o trabalho de as ir ler onde 
estão: ``Neste rio Formoso, por ele acima quatro léguas, está o lugar de Serenhém. 
Foi sondá"-lo Andrés Marim, tenente de artilharia, com pilotos, o ano de 1632. A melhor entrada da 
barra é pela banda do sul, pela qual entra por sete, seis braças, e pela banda do norte entra 
por cinco e quatro; e não se há de entrar pelo meio, porque tem de fundo braça e meia. O 
porto está da banda do sul. ``Tamanduaré é uma enseada oito léguas ao sul do cabo de Santo Agostinho, e uma 
légua ao sul do rio Formoso, e duas ao norte do rio Una; desemboca nela o rio das ilhotas ou 
Mambucaba; está cercada da banda do mar com arrecifes, e uma barra de sete braças de 
fundo na boca, em baixa"-mar de águas vivas; e logo mais dentro seis, na maior parte dela 
cinco, e bem junto à terra quatro; tem bom fundo; cabem nesta enseada cem navios e mais.'' 

\item A serra da Aquetiba será talvez a que hoje se diz da \textit{Tiúba}.

\item São curiosas as informações que Soares, só por noções dos indígenas, nos transmite 
dos gentios de além do Rio de São Francisco que se ataviavam com joias de ouro. Trata"-se dos habitantes do Peru.

\item A correção da palavra indígena --- \textit{manhana} --- para significar ``espia'' se colige do 
\textit{Dicionário brasílico}, que na palavra ``vigia'' traz o significado \textit{manhane.}

\item Do nome ``Rio do Pereira'' se faz menção no famoso Atlas de Vaz Dourado, do qual 
existe na biblioteca pública de Madri um exemplar mais aprimorado ainda do que o que 
se guarda com tanto recato no arquivo chamado da \textit{Torre do Tombo} de Lisboa. O nome 
de Torre do Tombo, para que de uma vez satisfaçamos em assunto sobre que algumas 
pessoas nos têm por vezes pedido informações, veio de que o tombo e arquivos da 
Coroa portuguesa se guardavam antigamente em uma torre do Castelo de Lisboa 
(onde estavam também os paços de Alcáçova), e por isso os papéis se diziam guardados na \textit{Torre 
do Tombo}. O terremoto de 1755 destruiu a tal Torre, e o arquivo passou para as abóbadas do 
(hoje extinto) mosteiro de São Bento, onde ainda está com o antigo nome, pelo hábito.

\item No lugar onde se lê: ``Até onde chega o salgado''  expressão esta mui frequente no 
nosso autor para designar o mar, diz o texto acadêmico, quanto a nós menos 
corretamente, a \textit{salgada}.

\item O rio \textit{Itapocuru} diz"-se hoje \textit{Tapicuru}. V. \textit{Tab. Perpet. Astron.} p. 217; \textit{Paganino}, p. 
21; Mapa de José Teixeira (de 1764) etc. Parece ter sido o que nos mapas de Ruysch 
(1508), de Lázaro Luiz e Vaz Dourado se chamou de \textit{S. Jerônimo}.

\item O texto da academia não mencionava o nome Real onde, na linha 8ª, se diz: --- porque toda esta costa do rio Real etc.

\item  \textit{Jacoípe} se lê nos códices que vimos; temos porém por melhor ortografia o escrever 
\textit{Jacuhipe} ou \textit{Jacuhype}, com a \textit{Corografia brasílica,} porque o nome quer dizer o esteiro ou igarapé do jacu.

\item Pimentel, Paganino e as \textit{Tábuas perpétuas astronômicas} escreveram \textit{Tapoã}; Mariz 
Carneiro, \textit{Tapoam}; porém mais conforme à etimologia fora dizer"-se e escrever"-se 
\textit{Itapuã}: \textit{Ita}, pedra, \textit{puã}, redonda.

\item No final deste capítulo 28 se encontra a notícia que melhor se desenvolve no cap. 
2º da 2ª parte (com. 76), acerca do fato que deu lugar a ser Diogo Álvares apelidado 
de Caramuru. Cosulte"-se a dissertação que citamos (com. 9), impressa no tomo 3º da 
2ª série da \textit{Revista} do Instituto, p. 129.

\item \textit{Boipeba}, como escreve Soares, é nome mais correto do que o de \textit{Boypeda}, usado por 
Pimentel, e seguido nos roteiros ingleses. \textit{Boi"-peba} significa cobra achatada.

\item Confirmamos não haver alteração na palavra \textit{Amemoão} ao lermos \textit{Memoam} na 
viagem de Luiz Thomaz de Navarro (1808) e \textit{Mamoam} no mapa de  Baltazar da 
Silva Lisboa.

\item Deixamos o nome de \textit{Romeiro} aportuguesado, por assim o acharmos nos melhores 
códices; mas o homem chamava"-se \textit{Romero}, que é ainda hoje nome de famílias 
castelhanas.

\item Os \textit{aimorés} são talvez os \textit{puris} de hoje, raça esta que, pelas palavras que se 
conhecem de sua língua, ainda não podemos classificar entre as desta América 
Antárctica. --- Os antigos pronunciavam às vezes \textit{gaimurés}, e quando faltavam com o 
acento na última sílaba, o nome se apresentava como muito diferente do que é, 
lendo"-se \textit{gaimures}.

\item \textit{Patipe} quer dizer ``esteiro do coqueiro'' (paty). Assim, melhor se escreverá, como 
faz Cazal (tomo 2º, p. 101), \textit{patype}. O amanuense do exemplar que serviu à edição 
anterior escreveu na última sílaba um f em vez de p. Cremos piamente que sem má intenção arranjou a palavra que daí resultou.

\item  \textit{Sernambitibi} ou \textit{Sernambitiba}, segundo a etimologia, é o verdadeiro nome do rio 
que de tantos modos se tem escrito, segundo dissemos nas \textit{Reflexões críticas} (nº 26, p. 
22). Cazal (ou o escrito que o guiou) chegou a adulterar este nome, não só em 
\textit{Simão de Tyba} (\textsc{ii}, p. 71), como logo depois (\textsc{ii}, p. 78) em \textit{João de Tyba}! Estas e outras hão de 
chegar a convencer os nossos governos de que o conhecimento de um pouco da língua 
indígena é para nós pelo menos tão importante, para não escrevermos disparates, como 
o de um pouco de grego e latim. A língua guarani já está reduzida à escrita, e salva de 
perecer de todo, graças sobretudo ao \textit{Tesouro} e à \textit{Arte e Vocabulário} de Montoya. E se 
não tratarmos de reimprimir estes livros e de os estudar, um dia os vindouros o farão, e 
nos chamarão a juízo por muitos erros em que houvermos caído por nossa ignorância; e 
porventura por um pouco de filáucia em termos por línguas sábias e aristocráticas 
unicamente o grego e o latim. Veja"-se a nossa dissertação ``Sobre a necessidade do 
estudo e ensino das línguas indígenas'' no tomo 3º da \textit{Revista}, p. 53.

\item Novo exemplo dos inconvenientes de ignorar inteiramente a língua indígena nos dá 
o nome de um rio do fim deste capítulo 35, que foi interpretado \textit{Insuacoma}, 
em vez de \textit{Juhuacema}, que Luiz Thomaz Navarro escreveu Juassema. O príncipe Maximiliano de 
Neuwied, em sua \textit{Viagem} (tomo 1º, p. 295), diz \textit{Jaússema}; e o Dr. Pontes, na sua carta 
geográfica pôs \textit{Juacein}. \textit{Juacê} quer dizer sede, e \textit{eyme}, sem; de modo que o nome do rio 
significa talvez ``rio que não tem sede''  nome que está muito no gosto dos que davam 
os indígenas, que no sertão chamam a outro --- o \textit{Igareí}, rio da sede, ou sem água.

\item Deste capítulo aproveitou Cazal no tomo 2º, p. 70 e 72. A mulher do donatário 
chamava"-se Inês Fernandes e seu filho Fernão do Campo.

\item Por \textit{Jucuru} se nomeia o rio que no mapa 3º da \textit{Razão do Estado} se diz \textit{Jocoruco}, e 
numa grande carta do \textit{Depósito hidrográfico} de Madri, \textit{Jucurucu}.

\item \textit{Maruípe} é quanto a nós um erro que se repetiu nos códices. Deve"-se ler \textit{Mocuripe} 
com Pimentel (p. 239) e com Laet, numa das cartas do \textit{Novus orbis}, impresso em 1633. 
Laet, nesta obra, que depois se publicou em francês, consultou sobre o Brasil os 
escritos do paulista Manuel de Morais. Esta edição latina foi a 3ª, sendo as primeiras, 
holandesas, 1625 e 1630, de Leyden. O rio mencionado diz"-se hoje \textit{Mucuri}; e Neuwied 
(\textsc{i}, 236) escreveu \textit{Mucury}.

\item \textit{Tupiniquim} ou \textit{Tupin"-iki} quer dizer simplesmente \textit{o tupi do lado} ou  vizinho lateral; 
\textit{Tupinae} significa \textit{tupi mau}.

\item  Este capítulo 40 foi o que Vasconcelos transcreveu quase na íntegra
nas suas \textit{Notícias} (51 a 55), e que nos serviu para confirmar que ele tivera conhecimento da obra
de Soares. \textit{Aceci} há"-de ser o \textit{Guasisi} da \textit{Razão do Estado}, \textit{Aceci} 
de Brito Freire.

\item A doação da Ilha a Duarte de Lemos teve lugar em Lisboa, aos 20 de agosto de 
1540, pelos serviços que o mesmo Lemos prestara ao donatário, na defesa da capitania. 
A confirmação régia é datada de Almeirim aos 8 de janeiro de 1549 (\textit{Chanc. de D. 
João \textsc{iii}}, fol. 108 v).

\item Neste capítulo, faltam ao texto acadêmico umas cinco linhas, aliás importantes, que 
no nosso se encontram no fim do 2º parágrafo e princípio do 3º.

\item  Deve ler"-se acentuando \textit{Goarapari}, que Vasconcelos na \textit{Vida de Anchieta} 
(p. 338) escreve \textit{Goaraparim}, e a \textit{Razão do Estado}, \textit{Goaraparig}. O
texto acadêmico dizia \textit{Goarapira}. \textit{Leritibe} é adulteração de \textit{Leritiba}, que
em guarani significa ``a ostreira''.

\item Tivemos ocasião de consultar e de conservar em nossas mãos uma carta autografa de 
Pero de Góis para Martim Ferreira, de quem se faz menção neste capítulo 44; e por ela 
conhecemos que é de letra sua o texto do códice do \textit{Diário} de Pero Lopes existente na 
Ajuda, que demos à luz; e isso se confirma com o asseverar aqui Soares que Góis 
acompanhara sempre o mesmo Pero Lopes, e com ele se perdera no Rio da Prata, isto é, 
na ilha de Gorriti do porto de Montevidéu, segundo sabemos. As emendas feitas nas 
primeiras páginas do dito texto do \textit{Diário} são de letra de Martim Afonso, que hoje 
distinguimos perfeitamente. Fiquem estas advertências aqui consignadas,  enquanto 
não temos para elas melhor lugar.

\item O texto da Academia diz \textit{tapanazes}, em vez de \textit{papanazes}. Este nome ou alcunha 
derivou, quanto a nós, da \textit{Zygaena}, chamada pelos indígenas \textit{papaná} e pelos nossos 
antigos de ``peixe"-martelo''.

\item  Ainda que o autor, no capítulo precedente, havia dito que o gentio goitacá tem a 
linguagem diferente dos seus vizinhos tupiniquins, não podemos entender essa afirmativa 
muito em absoluto, à vista do que assevera agora --- de que os papanases se fazem entender 
do mesmo gentio goitacá e do tupiniquim. Isto vai conforme com a ideia sabida de que os 
invasores que dominavam o Brasil na época da colonização eram geralmente da mesma raça, 
havendo que excetuar os aimorés, que depois apareceram acossados talvez do oeste. 
Remetemos a tal respeito o leitor para o que dizemos num escrito impresso no tomo 5º da 2ª 
série da \textit{Revista} do Instituto, p. 373 e ss.

\item O texto da Academia dá 22°3/4 ou 22°45' S. à latitude da ilha de Santa, que em 
outros códigos achamos 22 1/3 ou 22°20', o que mais se aproxima da de 22°25' S., que 
hoje se lhe calcula.

\item O \textit{Cabo Frio} jaz, segundo Roussin, em 23°1'18'' S., e, segundo Livingston (1824), 
em 23°1'2'' S., do que não se estava longe no tempo do nosso autor, que o arruma em 
23°. 

\item \textit{Saquarema} se diz hoje, e não \textit{Sacorema}.

\item Conservamos a palavra \textit{Viragalhão} dos códices, pois seria adulterá"-los o 
substituí"-la pela mais correta \textit{Villegagnon}, que aliás é menos eufônica para nós. 
O ilhéu de  \textit{Jeribatuba}, que quer dizer ``do coqueiral'' (de jeribás),
é o que hoje se diz ilha dos Coqueiros.

\item Por este capítulo se confirma que a primeira fundação de uma colônia nesta baía  
do Rio de Janeiro teve lugar na Praia Vermelha; e que o saco do Botafogo se chamava 
de Francisco Velho, por pertencerem essas terras ao  talvez tronco  primitivo da 
família Velho, no Brasil.
As palavras --- que se chama da Carioca --- não se leem no texto da Academia, mas 
sim no importante códice mais antigo de Évora, e em outros.

\item  Porto de Martim Afonso era o esteiro que vai ter ao Aterrado.
Chamou"-se daquele nome, não, quanto a nós, por via do célebre capitão de
igual nome, mas sim da aldeia do principal Arariboia, que no batismo se
chamou Martim Afonso.
A descrição da enseada desta nossa baía não pode estar mais exata. Os nomes  
\textit{Inhaúma}, \textit{Sucuruí}, \textit{Baxindiba} e \textit{Macucu}  são hoje quase os mesmos. A ilha da \textit{Madeira} é a 
das Cobras.

\item Mem de Sá foi nomeado por provisão de 23 de julho de 1556. Partiu da Bahia para a 
conquista do forte de Villegagnon em 16 de janeiro de 1560. Chegou ao Rio a 21 de 
fevereiro; rendeu o inimigo a 15 de março.

\item Salvador Correia governou tanto tempo o Rio de Janeiro que a sua ilha se ficou 
chamando até hoje do \textit{Governador}. Antes tinha"-se denominado \textit{Parnapicu}, 
\textit{do Gato}, \textit{dos Maracaiás} e \textit{dos Engenhos}.

\item Apesar de todas as diligências, até hoje ainda não nos foi possível encontrar o 
manuscrito de Antônio Salema sobre a \textit{Conquista do Cabo Frio}.

\item Do texto da Academia consta que Salvador Correia foi nomeado governador por 
provisão de 10 de setembro de 1557. Isto parece verdade, mas não cremos que fosse 
escrito por Gabriel Soares, se não erudição de algum copista. Nos melhores códices 
não se encontra essa cláusula.

\item O primeiro sesmeiro da Ilha Grande foi o Dr. Vicente da Fonseca, por carta de 24 de 
janeiro de 1569.
À Ilha de São Sebastião chamavam os indígenas, segundo Hans Staden, \textit{Meyembipe}; 
e a dos Alcatrazes, \textit{Uraritan}.
O morro e ponta de \textit{Caruçu} chama"-se vulgarmente de \textit{Cairuçu} e já assim escreveram 
Vasconcelos (p. 286) e Frei Gaspar da Madre de Deus (p. 17).

\item  \textit{Tamoio} quer dizer ``avô'', ``ascendente'', ``antepassado''.  Era o nome
com que os indígenas de São Vicente designavam os desta província fluminense, 
o que comprova as nossas fortes conjecturas de que a emigração tupi marchou 
do norte para o sul. Os tamoios chamavam"-se a si tupinambás, segundo Staden; e 
aos vizinhos do sul apelidavam os \textit{temiminós}, isto é, seus netos ou descendentes.

\item  A ilha da barra do porto de São Vicente, que Soares diz parecer
moela de galinha, chama"-se ainda hoje \textit{da Moela}.
Os Esquertes de Flandres eram uma família flamenga que se estabeleceu em São 
Vicente. Um dos indivíduos chamava"-se Erasmo Esquert, segundo Pedro Taques.

\item Martim Afonso recebeu cem léguas da costa por doação, e não cinquenta; e ainda 
assim a sua capitania saiu uma das mais pequenas, em braças quadradas. Esse grande 
capitão não voltou a São Vicente, depois de ser donatário; mandou, sim, providência, 
lugar"-tenentes etc.

\item Tampouco nos consta que Pero Lopes voltasse mais ao Brasil depois de ser aqui 
donatário, e temos quase certeza que não.

\item É sem verdade que Soares afirma que não havia noutro tempo formigas em São 
Paulo. Já Anchieta dá delas conta. E São Paulo é, desgraçadamente, terra proverbial, 
quanto às tanajuras, às saúvas e às tocas de cupins.

\item Em vez de \textit{Guainá} ou antes \textit{Guaianá}, escreve Staden \textit{Wayganna}.

\item Ilha \textit{Branca} é talvez adulteração de ilha do \textit{Abrigo}, que é a mesma fronteira à ponta 
do padrão, de que no capítulo seguinte se trata.

\item O Cabo do Padrão chama"-se hoje Ponta de \textit{Itaquaruçá}. Segundo o exame que aí 
fizemos pessoalmente em janeiro de 1841, esse padrão ou padrões (pois existem três 
iguais) foram aí postos por ordem de Martim Afonso, cuja armada (segundo P. Lopes) 
se demorou 44 dias no vizinho porto de Cananea. O leitor pode consultar o que 
ponderamos a tal respeito no Tomo 5º da 2ª série da \textit{Revista} do Instituto, p. 375.

\item A baía das \textit{Seis Ilhas} é naturalmente a enseada formada pelo rio Itajaí.

\item O nome de ilha de Santa Catarina foi dado pelos castelhanos da armada de Gabeto, 
em 1526. Antes, chamavam"-lhe \textit{Ilha dos Patos}, e já lemos que os indígenas a 
denominavam \textit{Xerimerim}.

\item Diz aqui Soares que a linguagem dos Carijós é diferente da de seus vizinhos; mas 
isso não se deve entender muito restritamente, porquanto no capítulo 63 assevera que 
com eles se entendem os guaianás.

\item O nome de porto de D. Rodrigo proveio de aí ter estado o infeliz D. Rodrigo da 
Cunha, que tão tristes episódios passou nesta costa.

\item Porto da \textit{Alagoa} é o da \textit{Laguna}. Não sabemos se a adulteração veio da pena do 
autor, ou se a causou algum copista que não quis admitir em sua cópia aquelas palavras 
espanholadas.

\item Chama"-se aqui rio de Martim Afonso ao Mampituba; mas entenda"-se que não foi 
neste rio, mas sim, no pequeno Chuí, que aquele capitão naufragou, o que se deduz da 
leitura atenta do \textit{Diário} de P. Lopes. A lagoa dos Patos chamavam alguns antigos de  
\textit{Tibiquera},  ou ``dos cemitérios''  talvez em virtude de alguns dos indígenas que ainda 
hoje por ali se encontram, segundo nos assegura o Sr. Conselheiro Batista de Oliveira.

\item Nas últimas linhas deste capítulo 72 confirma Soares a geral opinião de que os 
indígenas de toda esta costa, ainda quando vivendo a grandes distâncias uns dos outros, 
``são todos uns e têm quase uma vida e costumes''.  De expressões quase idênticas se 
serve o seu contemporâneo Pedro de Magalhães Gandavo, o amigo de Camões.

\item Monte de Santo Ovídio é o conhecido cerro da baía de Montevideu, o que Pero 
Lopes quis infrutuosamente chamar de monte de São Pedro.

\item O texto da Academia arruma, com manifesto erro, o cabo das Correntes em 36º de 
latitude S.; outros textos que seguimos dão 39º; mas cremos que houve neste número 
também engano, e que Soares poria com os pilotos do tempo o cabo em 38º.

\item O texto da Academia põe a saída de Tomé de Sousa de Lisboa a 1º de fevereiro, e 
não a 2, como os mais códices.

\item  Volve Soares a ocupar"-se do célebre Caramuru, a cujo assunto parece
que dedicava certa predileção. As notícias são ainda mais minuciosas que
as que chamaram nossa atenção no com. 28.

\item O primeiro assento da povoação da cidade era próximo à barra, e, segundo a 
tradição, onde está hoje o bairro da Vitória.

\item Às sábias providências da metrópole em favor da colonização da Bahia, deveu 
talvez Portugal a conservação de todo o Brasil, segundo melhor desenvolveremos em 
outro lugar.

\item No texto da Academia se dão mais as seguintes informações acerca do governador 
D. Duarte: ``Fidalgo muito ilustre, filho de D. Álvaro da Costa, embaixador del"-rei D. 
Manuel ao imperador Carlos \textsc{v}''.  Não as admitimos, por não se acharem nos melhores 
códices.

\item A explicação de Porto Seguro até o cabo Santo Agostinho, com que se conclui o § 
1º, não se contém no texto acadêmico.

\item Ao lermos esta parte da descrição da cidade, quando aportamos na Bahia em 
princípios de maio deste ano, quase que acompanhávamos o autor passo a passo, tanta 
verdade há em sua descrição.

\item Quase no fim do capítulo, em vez de ``capelães da misericórdia ou dos \textit{engenhos} 
diz, incorretamente, o texto da Academia ``capelães da misericórdia  ou  dos  \textit{enjeitados}''. 

\item  A respeito do colégio dos padres da Companhia na Bahia parece"-nos
que o leitor levará a bem que lhe demos aqui outra descrição, ainda quando
não seja senão para lhe fazer constar a existência de um curioso livrinho
como é a obra do pe.~Fernão Cardim, que imprimimos em 1847.
Diz esse escritor, em 1585: ``Os padres têm aqui um colégio novo, quase acabado; é 
uma quadra formosa, com boa capela, livraria e alguns treze cubículos, os mais deles têm a 
janela para o mar; o edifício é todo de pedra e cal destra, que é tão boa como a pedra de 
Portugal, os cubículos são grandes, os portais de pedra, as portas de angelim forradas de 
cedro; das janelas descobrimos grande parte da Bahia, e vimos os cardumes dos peixes e 
baleias andar saltando na água, os navios estarem tão perto que quase ficam à fala; a igreja é 
capaz, bem cheia de ricos ornamentos de damasco branco e roxo, veludo verde e carmesim, 
todos com tela de ouro, tem uma cruz e turíbulo de prata etc.'' [\ldots] 
``A cerca é mui grande, bate o mar nela, por dentro se vão os padres embarcar, têm uma 
fonte perene de boa água com seu tanque, aonde vão se recrear; está cheia de árvores de 
espinhos etc. [\ldots]''

\item Corrigimos \textit{hortas} onde no fim do capítulo dizia \textit{outras} o texto acadêmico; e, 
também, segundo a lição dos melhores códices, \textit{vinte religiosos}  em vez de \textit{doze}. 

\item Também aqui seguimos os melhores códices, escrevendo duas vezes Sua Majestade, 
e não Sua Alteza.

\item Este capítulo foi bastante retocado à vista das cópias mais dignas de fé, como o 
leitor pode deduzir pela confrontação. A observação de Soares de melhorarem de sabor 
e aroma os vinhos fortes que passam a linha é hoje tão admitida como é verdade que da 
Europa se mandam vinhos a viajar através da zona tórrida, só para os beneficiar.

\item Na antepenúltima linha do 1º § do capítulo 13 dizia erradamente o primitivo texto 
``por civilidade''  em vez de ``possibilidade'',  como escrevemos.

\item Chamamos a atenção do leitor sobre a relação de 1:2:3 entre as classes dos 
defensores da Bahia em 1587, a saber: dois mil colonos europeus, quatro mil africanos, 
e seis mil índios civilizados.

\item O nosso autor, que tanto entusiasmo e predileção mostra pelo Brasil, não contente 
com o haver dito no proêmio que este Estado era ``capaz para se edificar nele um 
grande império'', repete esta sua aspiração à nossa independência e nacionalidade, 
dizendo neste capítulo que já D. João \textsc{iii}, com mais alguns anos de vida, poderá ter aqui 
edificado ``um dos mais notáveis reinos do mundo''. 
É sabida a anedota referida pelo autor dos \textit{Diálogos das grandezas do Brasil} (obra 
escrita no século seiscentos) da profecia do astrólogo que, ao chegar a Lisboa a nova do 
descobrimento da terra da Vera Cruz, vaticinou que havia ela de 
ser abrigo e amparo da metrópole. Depois da aclamação de D. João \textsc{iv} tratou a Espanha de 
lhe ceder o Brasil, e tornar a reunir a si Portugal, o que se teria realizado se a França não se 
metesse de permeio. O marquês de Pombal ideou trazer ao Pará a sede da monarquia; depois 
dele, o poeta Alvarenga convocava para o Brasil a rainha Maria \textsc{i} (\textit{Florilégio da poesia brasileira}, 
tomo 2º, p. 37) e o alferes Lisboa (em 1804) desejava que em Minas o príncipe 
D. João fosse estabelecer seu império (\textit{Florilégio}, p. 574). Estes fatos são, pelo menos, 
curiosos.

\item Na doação da ilha de Taparica, ou Itaparica, como agora se diz, se compreendia a 
de \textit{Tamarantiba}. Receberam ambas foral em 1556.

\item Onde se diz --- ``da parte do Padrão'' --- parece"-nos que houve salto de uma palavra e 
se deve entender ``da parte da ponta do Padrão''. 

\item A ilha de Maré, de que se faz aqui menção, é a mesma que inspirou o poeta baiano 
Manuel Botelho de Oliveira, que tão belamente a descreveu em sua \textit{Musa do Parnaso} 
(Lisboa, 1705, p. 127).\footnote{ Varnhagen refere"-se ao livro de poemas \textit{Música do Parnaso}, de Manuel Botelho de Oliveira, 
publicado pela primeira vez em 1705, no qual se destaca o poema ``À Ilha da Maré''.} 
Essa bonita composição foi reproduzida  no  \textit{Florilégio}, tomo 1º, p. 134.

\item O texto da Academia contém, depois da palavra ``Pirajá'', do 3º § deste capítulo, as 
seguintes linhas, que não encontramos nos melhores códices, e devemos julgar 
introduzidas por curiosos: ``Esta enseada tem na barra de fundo duas braças de preamar; 
cabem até oitenta navios de força, os quais entram descarregados e hão de sair na mesma 
forma. Tem na boca duas fortificações, uma maior de uma banda, e outra mais 
pequena de outra.''

\item O texto a que nos temos referido trazia ``Alteza'' onde outra vez admitimos ``Majestade'' 

\item Diz o mesmo texto \textit{Ponta do Toque}  em vez de \textit{P. do Toque"-Toque}, como 
sabemos que se chama.

\item \textit{Aratu} lemos num dos códices, e admitimos a lição, ao saber que havia por ali um 
engenho com tal nome; o que se não dá, segundo nos asseveram vários baianos 
entendidos com o nome \textit{Utum} do texto acadêmico. \textit{Otuim} e \textit{Utuim} se lê, porém, 
em alguns manuscritos. No texto mencionado lê"-se também \textit{Curnuibão} em vez de 
\textit{Carnaibuçu} ou \textit{Carnaybuçu}  como lemos no \textit{J. de Coimbra}, nº 86, p. 67.
No mesmo texto se lê, ainda erradamente, \textit{Sacarecanga} e \textit{Pitanga} em vez de \textit{Jacarecanga} 
e \textit{Petinga}.

\item A palavra \textit{Tayaçupina}, a que pusemos um ponto de interrogação, não  nos foi 
possível decifrar adequadamente.

\item \textit{Caípe} ou \textit{Cahype} quer dizer ``o esteiro do mato''.  Tratando do engenho de Antônio da 
Costa, lê"-se no texto da Academia, depois da frase ``que está muito bem acabado''  as 
seguintes palavras, evidentemente anacrônicas para o livro de Soares: ``Que depois 
foi de Estêvão de Brito Freire, que Deus perdoe, e fez 
outro engenho por nome São Tiago, bem no fim de Pernamerim, para a banda da freguesia 
Tamarari de água das melhores que hoje no Brasil há''.  De Itapitanga volve o autor a 
ocupar"-se no cap. 187.

\item  Notam"-se grandes variantes entre o nosso texto e o da Academia.
Além de linhas que lá faltam, leram"-se errados os bem conhecidos nomes
de \textit{Paraguaçu}, \textit{Acu}, \textit{Cajaíba} e \textit{Tamarari}. Farreirey foi erro que escapou
ainda no nosso texto: leia"-se Tareiry.

\item No mesmo texto acadêmico lê"-se Antônio Penella e Rodrigo Muniz, em vez de 
Antônio Peneda e Rodrigues Martins, como encontramos nos códices mais dignos de 
crédito.

\item Aqui temos um novo rio de \textit{Igaraçu}, o que prova que habitualmente ali 
chegavam, como fica dito (com. 15), as naus dos europeus.

\item \textit{Pujuca} é o nome que dá o nosso texto à ribeira que, entre outros, o da Academia 
escreveu \textit{Puinqua}.

\item O rio \textit{Irajuhí} é o que hoje se diz \textit{Pirajuhia}. No texto da Academia encontram"-se  
\textit{Irayaha}, o que procedeu naturalmente de má leitura do copista.

\item Jiquiriçá é o nome que hoje se dá ao rio que Soares designa por \textit{Jequeirijape}.

\item Conclui Soares com a sua minuciosa descrição de todos os recôncavos da Bahia, 
cuja extensão, sem meter os rios de água doce, avalia em 53 léguas; e nessa extensão 
conta 39 ilhas, além de dezesseis do interior dos rios. A topografia do Recôncavo ainda até 
hoje não teve melhor, nem mais exato aluno.

\item São curiosas as notas estatísticas da Bahia (em 1587), e permita"-se que as 
recapitulemos: 36 engenhos, que exportavam anualmente para cima de 120 mil arrobas 
de açúcar; 62 igrejas, entrando dezesseis freguesias e três mosteiros, e 1.400 barcos de remo.

\item Algumas variações encontrará o leitor no nosso texto, graças à confrontação de 
tantos códices. As primeiras éguas valiam a 60\$ rs. e ficaram depois a 12\$; e não eram 
a 100\$ e ficaram a 20\$; os cavalos que por negócio se levavam embarcados a 
Pernambuco eram lá pagos a 200 e 300 cruzados, e não a 20 e a 30, o que quase 
equivalia aos preços da Bahia etc.

\item No exemplar da Academia diz"-se (p. 135), acerca das plantas de soca --- ``que são 
as que rebentam e brotam das primeiras cortadas''. Foi por certo explicação de algum 
copista animado de excesso de zelo.

\item No último §, tratando"-se dos inhames trazidos das ilhas da África, vem no texto da 
Academia, em vez daquele nome, o de taiobas, que é o nome indígena, e não se 
encontra nos mais códices, mas sim, inhames.

\item  Hortaliças que já se cultivaram na Bahia em tempo de Soares, e
por este já apontadas no capítulo 36: \textit{Cucumis sativus --- Cucurbita pepo ---
C. citrulus --- Sinapis nigra --- Brassica napus --- Raphanus sativus --- Brassica
oleracea crispa --- B. o. miliciana --- Lactuca sativa --- Coriandrum sativum
--- Anethum graveolens --- A. focniculum --- Apum petroselinum --- Mentha sativa --- 
Allium cepa --- Allium sativum --- Solanum melongena --- Plantago
--- Mentha pulegium --- Sisymbrium nosturtium --- Ocimum minimum --- O. basilicum --- 
Amaranthus blitum --- Portulaca oleracea --- Cichoneum endívia
--- Lipidum sativum --- Daucus carota --- Beta vulgaris --- Spinacea oleracea etc.}

\item Não respondemos pela devida exatidão da ortografia dos nomes das espécies de 
mandioca apontadas no capítulo 37. No texto acadêmico vem diferentemente, e 
Marcgraf e Vasconcelos trazem outras denominações. O mesmo faz José Rodrigues de 
Melo, que escreveu em verso latino o melhor tratado que conhecemos acerca desta raiz 
alimentícia; este tratado em dois cantos foi traduzido pelo Sr. Santos Reis, e 
publicado na Bahia, com outras composições análogas, em um tomo, com o adequado 
título de \textit{Geórgica brasileira}.

\item A tapioca de que Soares trata era preparada um pouco diferentemente do que hoje 
se usa no comércio. Este nome e o da mandioca são puros guaranis; e foram ambos 
adotados pela Europa, como tantos outros nomes indígenas, segundo iremos vendo.

\item Não deixou Rodrigues de Melo de escrever com elegância acerca das 
propriedades venenosas do sumo da mandioca crua: 

%\poemtitle{}
\settowidth{\versewidth}{Fac procul hinc habeas armenta, omnemque volucrum}
\begin{verse}[\versewidth]
\textit{Fac procul hinc habeas armenta, omnemque volucrum \\
Atilium gentem, positos neque tangere succos \\
Permitias: namque illa quidem niveoque colore \\
Innataque trahit pecudes dulcedine captas \\
Portio: mortiferum tamen insidiosa venenum \\
Continet: et fibris ubi pestem hausere, furore \\
Huc illuc actae pecudes per prata feruntur, \\
Et gyros agitant crebos, \& c.}\\*
\end{verse}

\item  A pronunciação \textit{tipeti} ou, aportuguesadamente, \textit{tipitim}, temo"-la por
mais conforme à dos indígenas do que a de \textit{tapeti}, \textit{tapetim} etc. Moraes
adotou aquela primeira; mas esta última parece"-nos mais eufônica. \textit{Urupema}
(segundo o \textit{Dic. Bras.}, p. 27) era qualquer crivo: a ortografia de Soares é
a seguida por Moraes. Há, porém, quem escreva \textit{gurupema} (Cunha Matos), \textit{garupemba} 
(\textit{Mem. da Acad. de Lisboa}, Tomo 7º), \textit{goropema} (João Daniel, P. 5ª, p. 24), e \textit{oropema} 
(Antonil, p. 117 da 1ª ed.).

\item  Diz Rodrigues de Melo a respeito da \textit{carimã}:
\begin{verse}
\textit{Quae succo nocuit radix, feret ipsa salutem\\
Jam praelo domita elicitoque innoxia succo}. \\*
\end{verse}


\item As palavras --- ``algumas jornadas'' --- no princípio do capítulo, faltam no texto 
acadêmico.

\item É curiosa a variedade de ortografia com que se tem escrito o nome que adotamos 
dos indígenas para a planta de raiz amilácea que Pohl denominou \textit{Manihot Aypi}, 
seguindo para esta denominação da espécie da ortografia de Lery (p. 135 da edição da 
Rochelle, de 1578), do \textit{Tesoro guarani}, de Martinière (T. l.º, p. 120), que adotaram 
Denis e St. Hilaire; Vasconcelos também uma vez assim escreve (not. 140), bem que em 
geral seja nisso irregular (V. Liv. lº, not. 71, 73 e 74). --- Soares com o seu contemporâneo 
Gandavo (fl. 16 da ed. 1576) parece ter preferido a mais aportuguesada de \textit{aipim}, 
seguida por Antonil (p. 69), por Vandelli, alferes Lisboa, Rebelo (p. 110) e os viajantes 
Spix e Martius (T. 2º, p. 526). Botelho de Oliveira escreveu \textit{aypim} (\textit{Floril.}, p. 142) e 
Cazal (\textsc{i}, 115) igualmente; Marcgraf \textit{aipii}, e assim se lê no \textit{Coro das musas} (T. 1º, p. 
143), e nos dicionários portugueses, que também dão \textit{impim}. O autor do \textit{Caramuru}, (C. 
4º, est. 19) escreveu \textit{aipi}.
Esperamos que o leitor nos desculpe a digressão que fizemos sobre essa palavra, acerca 
da qual desejávamos que se assentasse em uma ortografia. Apesar da preferência que já a 
ciência deu a \textit{aipi} nós em linguagem preferiríamos, com os clássicos Gandavo e Soares, \textit{aipim}.

\item No capítulo 44 descreve Soares várias \textit{Convólvulus}, a \textit{Dioscorea sativa}, 
o \textit{Caladium sagittifolium} (Vent.) e talvez o \textit{C. Poecile} de Schott.

\item Ao \textit{Zea Mais L}. se diz no texto que chamavam os índios \textit{ubatim}; cremos que diria 
Soares \textit{abatim}, pois \textit{abaty} e \textit{avaty} encontramos em muitos autores.

\item Abbeville (fl. 229) refere que os indígenas do Maranhão chamavam às favas \textit{comandá}, 
e o pe.~Luiz Figueira na sua gramática da língua geral (p. 87 da 4ª ed.) dá o 
mesmo significado.

\item À conhecida planta leguminosa \textit{Arachis hypogoea} L. chama Soares, à portuguesa,  
\textit{amendoí} como se proviesse de amêndoa. O nome é degenerado do \textit{mandubi} ou \textit{manduí}
indígena. Abbeville escreveu (fl. 226 v.) \textit{mandou}. Na  Espanha  chamam"-lhe  
\textit{avellanas} (avelãs) \textit{americanas}.

\item No capítulo 48 trata Soares das pimentas que dão várias solaneias capsicinas do 
Brasil, das quais não se esqueceu de tratar Fingerhuth na sua monografia impressa em 
1832. Cremos que o nosso autor menciona sucessivamente o \textit{Capsicum cerasiforme, 
cordiforme, baccatum, longum} e \textit{frutescens}. --- Montoya (\textit{Arte y Bocab.}, p. 141) chama à 
pimenta \textit{quíyí}; o \textit{Dic. Bras. kyynha}; Monteiro de Carvalho, com Piso, \textit{quiya}.
\textit{Jukiray} quer dizer ``molho de sal'',  \textit{jukyra}, sal (\textit{Dic. Bras.}, p. 70), e \textit{ay} molho (idem, p. 52).
No códice da Bib. Portuense (1019/6) lê"-se mais no fim deste capítulo o seguinte:

``Há outra casta de pimenta, a que chamam \textit{cuiemirim,} por ser mais pequena que todas; 
da qual se usa como das demais e tem as mesmas qualidades, cuja árvore é pequena.
Há outra pimenta, a que chamam \textit{cuiepiá,} que na feição é mais redonda e pequena da qual se 
usa como das mais e tem as mesmas qualidades, cuja árvore não é grande.

Há outra pimenta, a que chamam \textit{cuiepupuna}, do tamanho de um gravanço muito redondo. 
Esta em verde é muito preta e depois de madura faz"-se vermelha, e queima a seis 
palmos, e dá fruta em todo o ano; todas estas pimentas são cheias por dentro de umas 
sementes brancas da feição da semente de mastruços, que queima mais que a casca, e delas 
nascem as pimenteiras quando as semeiam. 

E já que dissemos das pimentas que queimam, digamos agora das que o não fazem e 
que são muito doces, uma das quais se chama \textit{Saropó}, que é tamanha como uma avelã, a 
qual como é madura se faz vermelha, e de toda a maneira é muito doce, cuja árvore é de 
cinco a seis palmos, e dá todo o ano novidade; estas pimentas se fazem em conserva em 
açúcar. 

A outra casta, a que chamam \textit{ayo}, que é da feição de uma bolota, e do seu tamanho, a 
qual se faz vermelha como é madura, e sempre é muito doce, a qual se faz também em 
conserva em açúcar e se faz árvore grande, que em todo o ano dá fruto. 

Não é bem que se faça pouca conta da pimenta do Brasil, porque é muito boa e não 
tem outro mal que queimar mais que a da índia, e quem muito a tem em costume folga mais 
com ela, e acha"-lhe mais gostoso que à da Índia, da qual por esse respeito se gasta pouca no 
Brasil, onde os franceses vão buscar a natural da terra, porque da casca vermelha se 
aproveitam nas tintas da mesma cor, e se quando vão resgatar a essa costa acha"-se muita dela, 
estimá"-la"-iam muito mais que o pau"-brasil; e das sementes de dentro se aproveitam 
pisando"-as bem e lançando por cima das pimentas da Índia, com o que a refinam e abatem; 
ainda que se faz este benefício a esta pimenta, poderá entrar na Espanha muita soma, se Sua 
Majestade der a licença para isso; de tal massa é esta terra da Bahia, que se lhe lançarem a 
semente do cravo o dará, como noz"-moscada, que tem o sabor dela, e dá outras árvores que 
dão canela; se for à terra quem a saiba beneficiar será como a de Ceilão, de que se dirá 
adiante.''

\item  Soares dá notícia de mais espécies de anacárdios do que as conhecidas dos 
naturalistas; mas no sertão vimos nós ainda uma espécie (talvez gênero) mas cuja planta é 
rasteira. O caju oriental é descrito pelo conhecido botânico português Loureiro, na 
\textit{Flora cochinchinensis} (Ed. 1790, \textsc{ii}, 248; e Berlim, 1793, p. 304).
A palavra \textit{catinga}, no sentido de mato canasquento ou charneca de moitas e matagais é 
de origem indígena e deriva de \textit{ca} e \textit{tinga}, mato brancacento. Catinga no sentido de ``mau 
cheiro'' se não derivou desta mesma acepção, deve ser voz africana.

\item Deste capítulo 
parece deduzir"-se que já antes da introdução no Brasil das bananas 
da África e da Ásia, havia na terra pelo menos duas espécies de pacobas: grandes e 
pequenas.

\item \textit{Mamão} (\textit{Carica Papaya} L.) não é fruta indígena do Brasil; porém outro tanto não 
sucede à papaiácea \textit{jaracatiá} a que nosso Veloso chamou (\textit{Flora flum.}) \textit{Carica dodecaphylla}.

\item As árvores frutíferas indígenas com que se ocupa Soares no capítulo 52 estão hoje 
todas conhecidas e descritas pelos naturalistas. A \textit{mangaba} é a \textit{Hancornia speciosa} de 
Gomes; os \textit{araçás} pertencem, bem como as guaiabas, ao gênero \textit{Psidum}; o \textit{araticu} é uma  
\textit{Anona}; vem depois o \textit{abajeru} (Abbeville, fol. 224, escreve \textit{Ouagirou}), que parece um \textit{Chrysobalanus}; 
segue talvez a rosácea \textit{Rubus idaeus} ou \textit{occidentalis} (Veloso \textsc{v}, est. 81 
e 82); notamos depois entre outras a \textit{Byrsonima Crisophylla}, de Kunth; a \textit{Vilex Taruma} e  
\textit{Inga edulis} de Martius; a \textit{Spondias myrobalanus} de Veloso (\textit{Flora flum.} \textsc{iv}, est. 185); a  
\textit{Moronobea esculenta} de Arruda ou a \textit{Platonia excelsa} de Martius, o \textit{Caryocer Pequi} 
etc. Tudo isso, salvo engano.

\item O ambu, imbu, ombu ou umbu (que para todas as ortografias há autoridades) é a 
notável planta que o nosso Arruda (\textit{Discurso dos jardins}) denominou \textit{Spondias tuberosa}.

\item Das frutas do sertão da Bahia que Soares reúne no cap. 54 há menos 
conhecimento. Trata"-se de um \textit{Lecythis}, segue"-se talvez uma planta rizobolácea, outra 
apocínea (talvez outra \textit{Cariocarácea}), um \textit{Genipa} e o conhecido \textit{oiti}, de que Arruda fez o 
novo gênero \textit{Pleragina}. Cazal (\textsc{ii}, 60) escreve \textit{goyty}, Vasconcelos (\textsc{ii}, 87), \textit{gutti}, 
Abbeville \textit{ouity}. --- Este capítulo necessita mais estudo.

\item Para melhor se identificar o leitor com a sinonímia das primeiras remetemo"-lo ao 
exame da magnífica monografia desta família, do célebre Martius, precedendo a ele, se 
for possível, o conhecimento prático das mesmas.
Nas \textit{Reflexões críticas} enganamo"-nos a tal respeito em várias de nossas conjecturas, 
feitas sem fundamento e só quase inspiradas, como em outros lugares da seção 4ª desse 
escrito, pelo desejo de acertar.

\item  Bem conhecida é a passiflora \textit{maracujá"-açu}, com que se começa o
capítulo das ervas frutíferas. Não nos acontece outro tanto com a planta
de que se trata depois, e que nos parece alguma \textit{solanácea}. Segue um \textit{Cactus}, 
com o nome indígena por nós conhecido, logo depois um \textit{Astrocarium}
e termina o capítulo em duas plantas bem conhecidas: uma bromeliácea e
um \textit{Piper}, segundo cremos; talvez o \textit{unguiculatum} de Ruiz e Pavon. No nosso
texto se escrevem elas \textit{carautá} e \textit{nhamby}. Esta última palavra escreve Piso
e a Farmacopeia tubalense \textit{nhambi}. Quanto àquela, Vasconcelos (\textsc{ii}, not.
70) diz \textit{caragoatá}; Antonil (p. 113), \textit{caravatá}; Piso e Brotero, \textit{caraguatá};
Bluteau \textit{caragoatá} e também \textit{caraoatá}; Fr.~Antônio do Rosário \textit{carauatá} e
Morais \textit{carahuatá}; mas, hoje, mais geralmente, em quase todas as nossas
províncias, se adotou \textit{gravatá}.

\item O ananás oferece exemplo de mais uma palavra indígena nossa que passou às 
línguas da Europa, e à linguagem das ciências, depois que Thunberg formou o gênero  
\textit{Ananassa}. Vamos registrando estes fatos para decidir se para nós a língua guarani é ou 
não digna, a par da grega, de ser cultivada como língua sábia, necessária para dar 
esclarecimentos não só na etnografia e na botânica, como nos diferentes ramos da 
zoologia. Só na botânica, além do mencionado gênero \textit{Ananassa}, temos com nomes 
brasileiros os gêneros (não falando nas espécies) \textit{Andira}, \textit{Apeiba}, \textit{Jacaranda}, \textit{Icica} e \textit{Inga}.

\item A cabureíba está hoje designada como \textit{Miroxylon cabriuva}. Não sabemos qual 
espécie de copaífera é mais geral na Bahia, à qual se referia Soares. As virtudes do seu 
óleo foram já em 1694 apregoadas pelo Dr. João Ferreira da Rosa no \textit{Tratado da constituição pestilencial de Pernambuco}, p. 51 a 56.

\item \textit{Embaíba} (ou, segundo outras ortografias \textit{embaúba}, \textit{imbaíba}, \textit{ambaíba} e \textit{ambayva}) 
é a conhecida \textit{Cecropia}, árvore urticácea de cujas tolhas se alimenta a preguiça 
(animal, se entende). Quando às \textit{caraobas}, os indígenas davam este nome a várias 
plantas bignoniáceas, e não nos é fácil acertar quais delas são as duas de que se ocupa 
Soares, bem que imaginemos a primeira a da estampa 50 da \textit{Flora} de Veloso, e em tal 
caso é a que Martins classificou como \textit{Cybistas antisyphilitica}.

\item A árvore da almécega ou \textit{icica} (\textit{ygcyca} no \textit{Dic. bras.}) é do gênero que Aublet 
designou com o próprio nome guianense (e que também é nosso) de Icica. --- \textit{Corneíba} 
é a \textit{Schinus aroeira}, de Veloso; \textit{Geneúnia} é uma \textit{Cassia}, não nos é fácil saber qual; 
\textit{cuipeúna} parece um \textit{Myrtus}; seguem dois cipós leguminosos e  o  conhecido  
\textit{Rhizophora mangle} L., ou mangue"-vermelho.

\item  As plantas descritas no capítulo 61 são todas de uso comum, e por isso mui 
conhecidas; vêm a ser: a nicociana, o rícino ou mamona, a batata"-de"-purga ou jalapa 
(\textit{jeticuçu}) e a rubiácea ipecacuanha, que o nosso autor escreve \textit{pecauem}, e os antigos jesuítas  
\textit{ipecacoaya}, de onde derivou o nome poaya,  que muitos lhe dão. Ao tabaco chama Soares  \textit{petume}; segundo 
Montoya (\textit{Voc.}, p. 203), dizia"-se em guarani \textit{petyma},  ou, como traz o \textit{Dic. 
bras.}, \textit{pytyma}. Damião de Góis (\textit{Crônica de D. Manuel}, Parte \textsc{i}, cap. 56) e com ele Baltasar 
Teles (\textit{Crônica da Companhia de Jesus}, Parte \textsc{i}, Livro 3º, cap. 3º, p. 442), chamam"-lhe \textit{betum}.  
O cronista do rei D. Manuel narra como essa planta foi levada à Europa por seu 
irmão Luís de Góis, que ao depois foi jesuíta, e de quem nenhum botânico tem feito caso até 
hoje, apesar do serviço que fez, muito maior do que Nicot. As minuciosas informações sobre 
o como se fumava são hoje mui curiosa prolixidade, por isso mesmo que todos sabem o que 
é ``beber fumo'' como Soares chama ao fumar.

\item  \textit{Manyú} deve entender"-se o nome indígena do algodoeiro (\textit{Gossypium vitifolium} de 
Lam.). O \textit{Dic. bras.} diz \textit{amanyú}; o Montoya (p. 151) \textit{Amandiyú}; em Abbeville (fol. 226 v.) 
lemos \textit{amonyiou}. A \textit{Lantana aamara} é hoje conhetida por toda a parte; \textit{ubá} ou taboca é o 
\textit{Ginerium sacharoides} de Kunth; não sabemos se há 
engano na palavra \textit{jaborandi} ou na última, \textit{jaborandiba} quando nos diz o autor que o 
nome dado pelos indígenas às duas plantas era o mesmo; o último é evidentemente o 
\textit{Piper jaborandi} de Veloso. Não afiançamos a correção ortográfica em \textit{caaplam}; deveria talvez ler"-se, com Piso, \textit{caaopiá}, planta do gênero que Vandelli denominou \textit{Vismia}, em honra do seu 
amigo M. de Visme.

\item Aos fedegosos (\textit{Cassia sericea} Sw.) chamavam os jesuítas \textit{tareroguy}, de onde se 
pode ver que não haverá erro no nosso texto em \textit{tararucu}; bem que nos inclinemos mais 
à desinência em \textit{quy}, e seríamos de opinião que a preferíssemos para a nossa língua em 
todos os casos idênticos, pois até parece que os muitos \textit{u u} tornam a linguagem 
tristonha. Para reduzir as outras plantas, apesar de terem alguns nomes conhecidos, até 
na botânica, encontramos contrariedades, as quais todas só poderá aplainar algum 
naturalista que se ache na província em que o autor vivia.

\item O cedro, chamado \textit{acayaca} pelos indígenas (\textit{Dic. bras.}, p. 23), é, segundo se nos 
assegura, do gênero \textit{Cedrela}.

\item Não respondemos pela correção do nome da segunda árvore que o nosso texto 
chama de \textit{guaparaíba}, e, menos ainda, pela do da Academia, \textit{quoapaiju}; pois nem 
sabemos o que seja. Da jutaipeba propôs"-se Baltasar Lisboa a fazer um novo gênero 
com o nome de \textit{Jatahypeba valenciana}.

\item Também quis o mesmo Baltasar criar um novo gênero com o nome de \textit{Massaranduba}, 
talvez sem saber se esta sapotácea, embora no Brasil cientificamente 
desconhecida então, não pertencia a algum velho gênero. Para se classificar de novo 
na botânica é necessário ter sobretudo muita erudição dos escritos da ciência: muitos 
gêneros se contam hoje que se hão pouco a pouco ir reduzindo a espécies de outros. 
Quanto às espécies, principalmente na América, onde as fisionomias naturais têm tanta 
semelhança umas com outras, apesar das distâncias, estamos persuadidos que mais de 
metade delas se verão reduzidas a simples variedades, quando haja viajantes 
naturalistas que percorram todo este continente, e tratem de harmonizar os trabalhos 
dispersos de tantos, cada qual a querer"-se fazer célebre e aos seus protetores.  Um 
classificador de plantas deve ser exclusivamente botânico.
Segundo o nosso texto, chamavam os índios \textit{andurababapari} ao angelim, que Piso 
chama \textit{Andira ibacariba}, e Martius reduziu sob o título \textit{Andira rosea}. A palavra andira faz 
crer que alguma coisa tinham os morcegos que ver com esta árvore.
O códice acadêmico diz \textit{andurababajari}, e o coronel Carlos Julião (últ. núm. do \textit{Patriota}, p. 98) 
o teria visto em manuscrito. No \textit{Dic. bras.} (p. 12) chama"-se"-lhe \textit{pobura}. 
Arruda tinha denominado o angelim \textit{Sholemora pernambucensis}. Lamarck havia já proposto 
o gênero \textit{Andira}, de que é sinônimo e \textit{Geoffroya} de Jacquin. O jequitibá não 
sabemos que esteja reduzido. Ubiraém é naturalmente o \textit{burayén} de 
Antonil (p. 57), que o Sr. Riedel classificou como \textit{Crysophyllum buranhe}. Sepepira é a 
sicopira (assim escreve Moraes); talvez a mesma que Baltasar queria designar com o nome 
de \textit{Joannesia magestas}. Antonil (p. 51, 56) escreve sapupira, e o autor do poema  
\textit{Caramuru}, supopira. A \textit{Bowdichia major} de Martius é uma sicopira; a \textit{urucurana} do Rio de 
Janeiro foi reduzida pelo Sr. Dr. Freire Alemão a um gênero novo, a que deu o nome de 
\textit{Hyeronima alchorneoides}. Não sabemos se a da Bahia é diferente.

\item  Antonil (p. 57) escreve \textit{camassari} e Cazal \textit{camaçari}. O autor pondera mais 
adiante (cap. 191) o valor desta árvore, da qual seria fácil extrair alcatrão. Guanandi é 
talvez uma clusiácea, e poderá ser a mesma \textit{Moronobea coccinea} que encontrou Aublet 
na Guiana Francesa.

\item  Das árvores que dão embira mencionadas no capítulo 68 é mais conhecida a que 
Veloso (\textsc{ix}. est. 127) designou por \textit{Xylopia muricata}.

\item Das madeiras de lei que neste comentário cabe tratar, só nos consta que estejam 
classificadas a do pau"-ferro, e a que Soares diz ubiraúna, se é a braúna vulgar  
(\textit{Melanoxylon brauna}, de Schott). \textit{Ubira"-una} significa ``madeira  preta''  e 
\textit{ubira"-piroca} ``madeira cascuda'' ou ``escamosa''. 

\item \textit{Tatajiba} ou antes \textit{Tatajuba} (\textit{juba} significa amarelo) é a \textit{Broussonetia tinctoria} 
Mart.; Sereíba a \textit{Avicenia nilida} L.; e a terceira árvore, cujo nome não podemos 
ainda justificar é a \textit{Laguncularia rucemosa} de Gaertner.

\item À apeíba, com este mesmo nome, deu ciência um gênero, na ordem natural das  Tiliáceas. 
Aqui trata"-se da \textit{jangadeira}, ou árvore das jangadas, que Arruda apelidou \textit{A. cimbalaria}. 
Sobre as outras árvores não nos atrevemos a fazer reflexões sem mais 
conhecimento especial delas; deixamos essa tarefa para os que forem botânicos de 
profissão; o fim deste comentário é outro, e ainda quando estudássemos toda a vida das 
ciências, que abrange hoje o livro de Soares, em alguns pontos deixaríamos de ser 
juízes competentes. O nome da árvore com que começa o capítulo deveria 
etimologicamente, talvez, ser \textit{calamimbuca}, isto é ``pau de cinza''. 
\textit{Ubiragara} quer dizer ``árvore de canoas''.  --- Cremos que seja a figueira do mato ou 
gameleira (\textit{Ficus doliaria} Mart.) --- Se soubermos algum dia a língua tupi ou guarani e 
estudarmos bem os seus nomes de árvores, animais etc., acharemos que todos eles terão 
como este sua explicação das propriedades e usos dos respectivos objetivos; o que já 
advertimos com a palavra \textit{andira} no com. 140.

\item  \textit{Carunje} parece"-nos palavra adulterada. \textit{Inhuibatan} escreve J. André
Antonil (p. 57). \textit{Jacaranda} é já um gênero botânico criado por Jussieu; não
sabemos se a ele pertence o de Soares. \textit{Mocetayba} escreve o jesuíta Vasconcelos
(\textsc{ii}, 80), e \textit{messetaúba} Antonil (P. 56 e 57). \textit{Ubirataya} é talvez 
a \textit{ubiratai} ou \textit{urutai} descrita por José Barbosa de Sá (fol. 361 v.), num extenso livro 
manuscrito do século passado, obra feita no sertão quase com tantas informações filhas 
da própria observação do autor, como esta de Soares que ora comentamos. \textit{Tangapemas} 
lemos em Vasconcelos (\textsc{ii}, nº 18). Referimo"-nos, deste jesuíta, quase sempre às  
\textit{Notícias curiosas}, que tiveram terceira edição no Rio de Janeiro em 1824, num volume 
de 183 páginas de 4º.

\item \textit{Ubiratinga} quer dizer ``madeira branca''.

\item \textit{Anema} significa --- ``cheirar mal'' --- (\textit{Dic. bras.}, p. 40); de modo que \textit{ubirarema} quer 
dizer --- ``madeira fedorenta'' ---  \textit{Guararema} se lê no \textit{Patriota}
(\textsc{iii}, 4º, 8); outros dizem \textit{ibirarema}.

\item A leguminosa de que primeiro se trata, com o nome \textit{comedoy}, é naturalmente do 
gênero \textit{Ormosia}. \textit{Araticupana} (como diz o texto da Academia e vem de Moraes) 
é a  \textit{Annona palustris} L. \textit{Anhangakybaba} seria mais corretamente a tradução de ``pente 
do diabo''. \textit{Cuié"-yba}, ou ``árvore das cuias'', é a conhecida \textit{Crescentia cujete} L.
Da  \textit{jatuaíba} ou \textit{jutuaíba} trata também Barbosa de Sá, fol. 365 v.

\item O timbó"-cipó é a \textit{Paullinia pinnala} de Lineu; o cipó"-imbé o \textit{Philodendron imbê} 
de Schott (Veloso, \textit{Flora flum.}, \textsc{ix}, est. 109).

\item Tocum, segundo é sabido, é uma espécie de \textit{Astrocarium}.

\item A ave que Soares designa por águia \textit{Caburé"-açu} é, pelos indícios que nos dá, 
a  \textit{Trachypetes aquilas} de Spix. --- \textit{Nhandu} ou ema é a \textit{Struthia rhea} de Lineu. --- 
Abbeville (fol. 242) escreve \textit{Yandou}. --- O \textit{tabuiaiá}, que Baena
(\textit{Corogr.}, p. 100) diz \textit{tambuiaiá}, pela etimologia se julgaria um \textit{Anser}, pois que \textit{aiá} quer dizer 
pato; mas a descrição conforma"-se mais a que seja algum \textit{Cassicus}.

\item  O \textit{macucaguá} descrito por Soares não é o macuco vulgar; parece
antes a \textit{Perdix capoeira} de Spix, e por conseguinte não \textit{Tinamus}; Abbeville
escreve \textit{macoucaua}, e Staden \textit{mackukawa} (P. 2ª, cap. 28).
O \textit{motum} de Soares é exatamente o \textit{Crax rubirostris} de Spix (\textit{Av.} \textsc{ii}, Tab. 67). O \textit{jacu} 
por ele descrito não nos parece nenhum dos galináceos classificados no gênero \textit{Penélope}; 
cujos nomes brasílicos para as espécies \textit{jacupema}, \textit{jacutinga} etc, a ornitologia já admitiu.
\textit{Tuiuiu} é reconhecidamente o \textit{Tantalus loculator} de Lineu. Em Cayena, chamam porém  \textit{Touyouyou} à \textit{Micteria 
americana}.

\item O \textit{Canindé} de Soares é uma variedade da \textit{Aratinga luteus} de Spix (\textit{Av.} Tom. 1º 
Tab. 16). Confronte"-se também a descrição de Buffon (\textit{Hist. nat.}, Tom. 7º, p. 154 e 155, edic. 4º gr.). 
--- A arara e tucanos são bem conhecidos. --- \textit{Embagadura}, entre os 
indígenas, era o punho da espada, segundo melhor se explica no capítulo 173.

\item \textit{Uratinga} (\textit{Ouira"-tin} de Abbeville, fol. 241) é a \textit{Ardea egretta} de Lineu; 
\textit{Upeca}, \textit{Vpec}, de Abbeville (fol. 242), \textit{Ipecu} no \textit{Dic. bras.} (p. 59), é árvore do gênero \textit{Anas}.
\textit{Aguapeacoça} ou \textit{Piassoca} a \textit{Palamedea cornuta} de Lineu; 
\textit{Jabacatim} a ribeirinha que Moraes (no voc. Papa"-peixe) designou por \textit{jaguacati}. Os 
\textit{gariramas} são do gênero \textit{Tringa}. \textit{Jacuaçu} é evidentemente a \textit{Ardea soclopacea} de Gmelin, 
para a qual Viellot propôs o gênero \textit{Aramus}, havendo sido por Spix denominada \textit{Ralus ardeoides}.

\item O \textit{nhambu} é conhecidamente o \textit{Tinamus plumbeus} de Temnink. \textit{Picaçu}, \textit{parari}, 
\textit{juriti} e \textit{piquepeba} parecem as \textit{Colombinas griseola},
\textit{strepitans}, \textit{caboclo} e \textit{campestris} de Spix.

\item Papagaio é voz africana; era o nome dado em Guiné aos cinzentos, primeiros que 
se levaram a Portugal. O nome brasílico é \textit{ageru} ou \textit{ajuru} como admite Moraes (\textit{Dic. 
port.}) --- Abbeville (fol. 234) escreveu \textit{juruue}. --- Assim, \textit{ageru"-açu} 
(que outros escrevem \textit{juru"-assu}) significa papagaio
grande, e \textit{ageru"-eté} papagaio verdadeiro. O primeiro, bem como \textit{corica},
parecem do gênero \textit{Ara}. Thevet (fol. 93 das \textit{Singul.}) escreveu \textit{Aiouroub}.
Tuim será um dos \textit{Psitaculus gregarius} de Spix. --- Soares escreveu, com
Gandavo, \textit{maracaná}; outros, porém, dizem \textit{maracanã}. --- Consulte"-se Marcgraf
(p. 20); Johnston, \textit{Avi}, p. 142; Willugby, \textit{Ornithol.}, p. 74, e Brisson, 
\textit{Ornithol.}, Tom. 4º, p. 202.

\item  O capítulo 84 ocupa"-se de várias aves ribeirinhas; talvez da \textit{Ardea garzetta}
de Gmelin; da \textit{Sterna magnirostris} de Spix; de uma \textit{Procellaria};
da \textit{Micteria americana}; de alguns \textit{Ibis, Triaga} etc. --- \textit{Socori} deve ser \textit{Socó"-boi} 
ou \textit{Ardea cocoi} de Lath. --- Em vez de \textit{margui} lemos em uns códices \textit{margusi}, 
e talvez se devesse ler \textit{majuí}, que é o nome dado às andorinhas
(\textit{Dic. bras.}, p. 12). --- \textit{Pitchuão} parece que se diz no \textit{Peregrino da América} (p.
48) que era o \textit{bem"-te"-vi}; mas a descrição de \textit{pitaoão} não se conforma.

\item  \textit{Urubu} é o \textit{Vultur jota} de C. Bonaparte; \textit{cará"-cará} o \textit{Polyborus vulgaris} de Vieillot; 
\textit{toacaoam} o \textit{Astur cachinans} de Spix (Tom. 1º, Tab. 2º):
--- \textit{Urubutinga}, à vista da descrição não pode deixar de ser o \textit{Cathartes papa},
e impropriamente chamou Lineu a uma águia negra \textit{Falco urubutinga}
quando esta última palavra quer dizer ``urubu branco'';  mas igual troca
já se fez com a \textit{araraúna}. Difícil será reduzir a espécie de \textit{Falco} ou \textit{Milvius}
de que trata o autor com tão pouca explicação.

\item A primeira e terceira aves parecem \textit{Strix}. A segunda cremos que será o 
\textit{Trogon curucuí} de Lavaillant. --- Desculpe"-se a Soares ocupar"-se, a par destes, de um 
quiróptero, seu companheiro de noite. --- Quanto à ortografia dos nomes, Souza Caldas 
escreveu (\textit{Canto das aves}) \textit{Jacorutu}, e Abbeville, em francês, \textit{Joucouroutou}.

\item \textit{Uranhengatá} é o passarinho do Brasil que substitui no canto o canário e o 
pintassilgo. \textit{Gorinhatá} escrevem alguns; e Nuno Marques Pereira, no \textit{Peregrino da América} 
(Lisboa, 1760, p. 48), \textit{Guarinhatão}. Hoje diz"-se \textit{Grunhatá} (Cazal \textsc{i}, 84, e 
Rebelo, \textit{Cor. da Bahia}, 1829, p. 56). --- Parece o \textit{Icterus citrinus} de Spix. \textit{Sabiatinga} 
(que ainda hoje em algumas partes se chama sabiá"-branco) é o \textit{Turdus orpheus} de Spix. 
\textit{Tié"-piranga} é o nosso mui conhecido tié  (\textit{Tangara nigrogularis} de Spix).
--- \textit{Guainambi} é o nome indígena dos beija"-flores, que hoje 
constituem vários gêneros; e \textit{Aiaiá} o da linda colhereira que Vieillot designou como 
\textit{Platalea aiaia}. \textit{Jaçanã}, pelo nome, deve ser o gênero \textit{Parra}; e neste caso talvez a de que trata Soares 
seria encarnada por metamorfose que essa espécie sofra, como acontece aos guarás (\textit{Ibis ruber}). --- 
Segue"-se a \textit{Tangara coclistis} de Spix, e mais duas aves que também podem ser do 
mesmo gênero, se alguma não é antes \textit{Muscicapa} ou \textit{Lanius}. A última ave é da família dos 
psitácidas.

\item Os pássaros que melhor conhecemos, além do que primeiro tratou no capítulo 
anterior, e torna a ocupar"-se, são: o \textit{sabiapoca} ou \textit{sabiá"-da"-praia}, que Spix denomina  
\textit{Turdus rufiventer}, e do qual diz (p. 69 do texto) ser ``\textit{cantu melódico uti philomela 
europaea insignis}'',  e o \textit{querejua} ou \textit{crejoá} que é a \textit{Ampelis cotinga} de Lineu.

\item \textit{Nhapupé} é o \textit{Tinamus rufescens} de Temnink. A \textit{saracura} pertence ao gênero  
\textit{Rallus}; Spix descreve"-a como \textit{Galinula saracura}. \textit{Oru} é o \textit{Trogon sulphureus} de Spix, 
o \textit{anu} (que Moraes diz \textit{anum}), o \textit{Crolophaga ani} de Lineu. Segue"-se a \textit{Ardea maguari} 
de Vieill, e talvez um \textit{Tinamus}, vários \textit{Turdus}; e conclui"-se o capítulo com um 
trepador pica"-pau (\textit{Picus}), manifestamente o que Spix denominou \textit{P. albirostris}, e que, 
segundo Cuvier julga, tem analogia com o \textit{P. martius} de Lineu.

\item Ocupa"-se o autor se dar notícia geral dos ortópteros e lepidópteros. No \textit{Dic. bras.} 
(p. 42) lemos \textit{tucuna}, e em Abbeville (fol. 255 e 255v.) \textit{pananpanam} e \textit{arara}.

\item  Seguem vários himenópteros da família melífera. Da \textit{canajuba}
trata Baena (\textit{Corog.}, p. 121) e da \textit{copueruçu}, Carvalho (cap. 351) e Piso (p.
287), que também se ocupa da \textit{taturama} (p. 289).

\item Outros da família diptóptera de Latreille --- e alguns dípteros etc. --- Abbeville 
escreve (para ser lido por franceses), \textit{lururugoire} e \textit{merou} ou \textit{berou} por \textit{terigoá} e \textit{meru}, p.
167.

\item Mais dípteros, um ortóptero e um coleóptero da família dos longicórnios de 
Latreille, ou cerambicinos de Lamk.

\item \textit{Tapir"-eté} ou simplesmente \textit{tapir} era o nome que davam os indígenas ao conhecido 
paquiderme \textit{Tapir americanus}, que Buffon descreve no tomo undécimo de sua obra 
(Edic. de 4º, p. 464). --- Os castelhanos lhe chamaram \textit{ante} e \textit{danta} e os portugueses 
\textit{anta}, porque designavam a esse tempo com tal nome (derivado do arábico, que é 
semelhante) o búfalo (\textit{Bos bub alus} de Lineu) que havia na África e no sul da Europa, e 
cujas peles curtidas de cor amarela, que muito se empregavam nos vestuários e 
armaduras no século \textsc{xvi}, puderam substituir pelas do nosso tapir, com mais vantagem ao 
menos no preço. A resistência das couras de anta à estocada era proverbial. 

\item \textit{Jaguareté} ou \textit{jaguar"-verdadeiro} é a \textit{Felis onca} de Lineu.

\item Há talvez engano em supor um animal \textit{Felis} habitador dos rios ou anfíbio; no 
tamanho das presas também deve haver engano, pois não podem ser de um palmo.

\item Julgamos mais acertado não querer reduzir, sem bastante segurança, as três 
espécies de cervos de que se ocupa Soares, se bem que uma nos pareça o \textit{C. rufus} de 
Cuvier, e outra o \textit{C. tenuicornis} de Spix.

\item Ocupa"-se o autor do \textit{tamanduá"-açu} ou \textit{Myrmecophaga jubata}. Segue"-se talvez 
uma espécie de \textit{aguarachaí} ou \textit{Canis azaroe}; e depois o quati, espécie de \textit{Nasua}, o 
maracajá ou \textit{Felis tigrina} e o \textit{seriguê} ou gambá, que no Rio da Prata chamam \textit{micuré}, 
espécie do \textit{Didelphis} de Lineu. Gandavo (fl. 22v) escreveu \textit{cerigoês} e Vasconcelos (Liv. 
2º, not. 101) \textit{çarigué}. --- Ao bolso do abdome chamavam os indígenas \textit{tambeó}.

\item \textit{Jaguarecaca} (talvez antes \textit{jaguatecaca}) diz Soares ter sido o nome do conhecido  
\textit{Mephitis foeda} de Ill., que Cazal (\textsc{i}, 64) designou por \textit{Jaraticaca}.

\item Os paquidermes que se descrevem todos parecem \textit{dicotyles} e nenhum \textit{sus}. 
Deixemos a redução das espécies aos que tenham à vista bons exemplares adquiridos 
nas imediações da Bahia. --- Os nomes nos manuscritos não sofreram adulteração; mas 
hoje alguns variam em \textit{caititu}, \textit{taiatitu} e \textit{tiririca}.

\item Poucas palavras terão sofrido entre nós mais variedades na ortografia do que a da 
\textit{capivara}, que assim se pronuncia e escreve hoje quase geralmente o nome do  \textit{Hydrochoerus capibara} de Cuvier.
Os outros anfíbios não podemos determinar só pelos nomes: um pode ser a \textit{Mustela lutra brasiliensis}; os outros, 
talvez, \textit{Viverras}.

\item Chama se \textit{tatu"-açu} ao \textit{tatu"-aí} ou \textit{Dasypus unicinclus}; \textit{tatu"-bola} é o \textit{D. tricinctus}; 
os dois últimos parecem ambos da espécie \textit{D. novemcinctus}.

\item As pacas e cotias bem conhecidas são, assim do vulgo como dos naturalistas. --- 
\textit{Cotimirim}, ou antes, \textit{coatimirim} é o estimado \textit{caxinguelê}, do gênero \textit{Sciurus}.

\item O capitulo 104 dá razão de cinco animais da ordem dos quadrumanos, cada um 
de seu gênero. O guipó é \textit{Callitrix}; o guariba, \textit{Mycetes}; os saguis da Bahia, \textit{Jacchus}; os 
do Rio, \textit{Midas}; e os anhangás ou diabos são evidentemente \textit{Nocthora}.

\item Se o autor andou tão sistemático no capítulo que acabamos de comentar, não 
sucedeu assim no imediato, onde ajuntou vários animais mui diferentes: \textit{saviá} (ou 
talvez \textit{sauiá}) e seus compostos \textit{S. tinga} e \textit{S. coca}, são espécies dos gêneros \textit{Mus} e do  
\textit{Kerodon} de Neuwied. --- \textit{Aperiás} são os \textit{preás} ou \textit{Anoema cobaia} L.; \textit{Tapotim} é a \textit{Lepus 
brasiliensis} de Gmelin; e \textit{jupati} um marsupial,  provavelmente  a  denominada marmota (\textit{Didelphis murina}).

\item Para não interrompermos o pouco que nos falta da classe dos mamíferos, não nos 
deteremos com largo exame no capítulo em que Soares dá notícia de alguns répteis do 
gênero \textit{Emys}, e talvez de mais algum da família dos quelônios. O nome brasílico \textit{jabuti} 
já está também consignado nos tratados da ciência zoológica, e nos museus do 
Universo.

\item A preguiça (gênero \textit{Bradypus} de Lineu) é pelo jesuíta Vasconcelos denominada 
(Liv. 2º, nº 100) \textit{Aig}. --- \textit{Haût} dizia Thevet.

\item Não sabemos como entende Soares que \textit{jupará} ou antes \textit{jurupará} queira dizer 
noite. \textit{Juru} significa ``boca''  e noite ou escuro traduz"-se por ``pituna''.  Sabemos que 
existe ainda nas nossas províncias do norte um animal daquele nome, que se caça de 
noite, quando vem comer fruta em certas árvores, e que em algumas terras lhe chamam 
\textit{jurupari}. Este nome quase equivalia entre os indígenas ao de \textit{Anhangá}. Assim talvez o 
animal seja algum gênero \textit{Nocthora} (com. 178). O \textit{cuandu}, \textit{cuim} e \textit{queiroá} são espécies 
de \textit{Hystrix}.

\item Enceta"-se uma das ordens dos répteis com a jiboia, mui propriamente chamada  \textit{Boa constrictor}. 
Atualmente há duas delas vivas no nosso museu. Veja"-se a dissertação 
sobre ofiologia do Sr. Burlamaque na Biblioteca Guanabarense, que publica os 
trabalhos da ``Sociedade Velosiana'' (agosto de 1851).

\item São conhecidos os ofídios de que trata o capítulo. Ao último chamou Abbeville  
\textit{Tarehuboy}, e Baena (\textit{Cor. do Pará}, p. 114), \textit{tarahiraboia}.

\item Hoje diz"-se vulgarmente jararaca (\textit{Trigonocephalus jararaca} (Cuv.). A \textit{ububoca} 
ou coral, pelo nome deve ser a \textit{Elaps marcgravii} de Spix.

\item O nome de \textit{boicininga} caiu em desuso e só ficou o de cascavel (\textit{Crotalus cascavella}). 
Os quiriguanos chamavam"-lhe \textit{emboicini} ou \textit{boiquirá}; assim como, 
segundo J. Jolis (\textit{Saggio del Chaco}, p. 350), chamavam \textit{boitiapó} à que Soares diz 
\textit{boitiapoia}, mais conhecida por ``cobra"-de"-cipó''  talvez pelo uso dos indígenas de 
açoitarem com ela, pelas cadeiras, a suas mulheres quando lhes não davam filhos.
\textit{Ubojara} é naturalmente a \textit{Coecilia ibiara}, Daud, p. 63 e 64.

\item \textit{Trigonocephalus surucucu} chama Cuvier ao ofídio que em vulgar designamos 
com este último nome. --- O ubiracoá parece a \textit{Natrix punctatissima} de Spix. Os outros 
são talvez espécies de \textit{Xiphosoma}. \textit{Urapiagara} ou \textit{Guirapiagara} quer dizer ``comedora 
dos ovos dos pássaros'' 

\item Na ordem dos sáurios menciona Soares um jacaré que, como se sabe, é do gênero 
da família dos crocodilos. --- Sanambus e tijus (ou teiús) são \textit{iguanas}. \textit{Anijuacanga} 
talvez seja adulteração de \textit{teju"-acanga}.

\item Trata"-se de alguns anfíbios da família \textit{Ranidae} --- O sapo é o \textit{Pipa cururu} de Spix.  
\textit{Juí"-giá} quer dizer ``rã do gemido'', e por este nome é hoje conhecido em algumas 
províncias este batráquio.

\item Não sabemos individuar os ápteros miriápodes que Soares descreve neste 
capítulo, por nossa míngua de conhecimentos entomológicos, e falta de coleções que 
nos sirvam de guia. Piso (p. 287) escreve \textit{ambuá}.

\item Outro tanto dizemos acerca dos pirilampos ou vagalumes que devem, 
naturalmente, pertencer, como os que conhecemos, à ordem dos coleópteros. Piso 
(p. 291) disse \textit{memoá}.

\item Da classe aracnídea trata"-se no capítulo 118, bem como dos articulados do 
gênero \textit{Scorpio}, \textit{Mygala} etc.

\item Não nos foi possível encontrar coleções contendo os himenópteros tratados nos 
quatro capítulos que seguem. Abbeville (fol. 255 v.) chama \textit{Ussa"-ouue} à formiga 
saúva ou tocantera.

\item A palavra \textit{goajugoaju} parece"-nos não ter sofrido adulteração; é uma \textit{Formica destrutrix}.

\item O \textit{Dicionário} de Moraes anda falto de um acento na segunda sílaba da palavra \textit{içás}.

\item  \textit{Taciba} é em geral a palavra para dizer formiga na língua guarani.

\item \textit{Copi} ou \textit{cupim} é o conhecido \textit{Termes fetale} de Lineu (Cuvier, t. 3º, p. 443). 
Neste capítulo há no nosso texto melhoramentos de variantes importantes.

\item Abbeville (fol. 256) chama \textit{tou} ao que Soares e o pe.~Luís Figueira (\textit{Gram.}, p. 48) 
dizem \textit{tunga}, e \textit{Attun}, Hans Staden. É a \textit{nigua} dos espanhóis, e \textit{chique} dos franceses 
(Labat., \textit{Viag.}, 1724; T. 1º, p. 52 e 53).

\item O nome \textit{pirapuã} dado pelos indígenas ao cetáceo baleia pode traduzir"-se por ``peixe redondo'' ou ``peixe"-ilha''. 

\item Segundo nos informa o Sr. Maia, não consta que o espadarte frequente hoje a 
nossa costa. E se nunca a frequentou é ele de opinião que o de que Soares trata seja 
antes o \textit{Histiophorus americanus} de Cuv. O peixe monstro de que se faz menção seria 
naturalmente algum cachalote de extraordinário tamanho.

\item A ideia de homens marinhos era familiar aos índios. Gandavo (fol.~32) dá notícia 
deles, com o mesmo nome que Soares, apenas diversamente escrito, \textit{hipupiara}. O 
pe.~João Daniel, no \textit{Tesouro do Amazonas} (P. 1ª, cap. 11) também se mostra, em tal 
assunto, crédulo. --- Soares não pôde ser superior ao que terminantemente ouvia 
afirmar, e ao seu século; pois que era ideia antiga também na Europa, com as sereias 
etc. Bem conhecida é a passagem de Dante, tantas vezes citada: 

\begin{verse}
\textit{Che sotto l'acqua è gente che sospira,\\ 
E fanno pullular quest'acqua al summo.} \\*
\end{verse}
As assaltadas de que se faz menção seriam talvez obra de tubarões ou de jacarés, uma 
vez que por ali não consta haver focas.

\item Trata o cap. 128 de peixes dos gêneros \textit{Pritis}, \textit{Squalus} etc. Romeiro é o \textit{Echenes 
remora} de Lineu. Abbeville (fol. 245 v.) escreveu \textit{Araouaova}, e Thevet (\textit{Singul.}, fol. 
133, e  \textit{Cosmogr.}, fol. 967 v.), \textit{Houperou}, o que comprova a exatidão nos termos \textit{Aragoagoary} 
e \textit{Uperu} de Soares, atendida naqueles a ortografia francesa.

\item \textit{Coaraguá} ou \textit{Guarabá} (\textit{Dic. bras.}, p. 60) é conhecido cetáceo do gênero  \textit{Trichechus}.

\item O \textit{beijupirá}, sem questão o mais estimado peixe do Brasil, como assevera Soares, 
é o escomberoide antes denominado \textit{Centronotus}, e hoje classificado como 
\textit{Elacate americana (Cuv. e Val., Hist. des Pois.,} 8.334).
Olho"-de"-boi (que deve ser algum \textit{Thinnus}) diz"-se em guarani \textit{Tapir"-siçá}. Do \textit{camoropi} 
tratam Laet (p. 570), Lago (p. 62), Abbeville (fol. 224), Gandavo e Pitta (p. 42).

\item  Ainda que sejam mui nomeados os peixes que Soares reuniu no
capítulo 131, confessamos que deles só conhecemos a cavala, escomberoide
do gênero \textit{Cybium} (Cuv. e Val., \textit{Hist. des Pois.}, tom. 8º, p. 181).

\item  Melhor acertamos acerca dos peixes cartilaginosos. \textit{Panapaná}
(nome que também nos transmitem Thevet e Abbeville) é a \textit{Zygena malleus}
de Vallenciennes, gênero da família dos \textit{Squalidae}, bem como os cações. Os
bagres são \textit{Siluridae}, talvez do gênero \textit{Galeichthys} e \textit{Pimelodus}. Piso trata
deles com nomes análogos, \textit{curuí} e \textit{urutu}. \textit{Caramuru} é um ciclóstomo, talvez \textit{Petromyzon}. 
As arraias do Brasil são de vários gêneros: \textit{Raia}, \textit{Pastinaça} e
\textit{Rhinoptera}; e os nossos pescadores desta parte da costa as distinguem com
as denominações de Santa, Barboleta e Manteiga, Ticonha, Boi (a negra),
Treme"-treme etc. Jabibira é significado que se confirma no \textit{Dicionário brasílico}, p. 66.

\item  Preparemo"-nos para encontrar em um capítulo peixes muito 
dessemelhantes entre si. Vereis ao lado de algum \textit{Lobotes} (?) um \textit{Thynnus}, uma \textit{Coryphena}, 
um \textit{Scomber}, um \textit{Serranus}, um \textit{Elops}. Julgamos o roncador dos cienidas,
as agulhas dos esocidas, o peixe"-porco dos balistidas e este último mui 
provavelmente \textit{Monocanthus}. Quanto aos nomes indígenas, temos por exatos todos
os do nosso texto. \textit{Guaibi"-coara} explica a denominação que menciona
Piso (p. 56), porquanto guaibi ou guaimim (segundo escreveu o autor do
\textit{Dicionário brasílico}) quer dizer ``velha''.  \textit{Jurucuá} é, segundo Piso, o nome
das tartarugas, que Soares teve a lembrança pouco feliz de arrumar neste
capítulo.

\item  De novo atende Soares a outros peixes, como se juntos tivessem
saído de um lanço de rede. Trata"-se primeiro da \textit{Mugil albula} de Lineu,
que é dos mais abundantes da nossa costa. O peixe"-galo em questão é do gênero \textit{Argyreyosus} 
ou do \textit{Blepharis}, ou de algum dos 
outros que constituíam o Zeus de Lineu, os quais se podem compreender na famílias dos 
escomberoides. \textit{Pororé} é o nome que significa ``enxada'',  porém a enxada"-peixe, ou peixe"-enxada, é da família quetodôntida, 
e do gênero \textit{Ephippus}, quanto alcançam nossos exames. 
A coirimá ou corimá pertence ao citado gênero \textit{Mugil}. Arobori deve ser dos \textit{Clupidas}, e 
carapeba do gênero \textit{Sciena}.

\item  Jaguariçá é naturalmente da família dos ciprínidas; piraçaquê do
gênero \textit{Conger}. O bodião é peixe diferente, segundo os países. O nome
\textit{atucopã} verifica"-se pelo de \textit{oatucupá}, que se dá o \textit{Dic. bras.} (p. 62) para
a pescada. A palavra \textit{guaibi"-quati} tem o que quer que seja que ver com ``velha'' (com. 207).

\item Uramaçá ou aramaçá segundo os que seguem Marcgraf, é do gênero \textit{Pleuronectes}. 
Aimoré parece um \textit{Lophius}. O baiacu é um \textit{Tetraodon} e o piraquiroá 
um \textit{Diodon}. Estes dois peixes da família gimnodôntida servem de confirmar a 
propriedade que guardavam os guaranis em suas denominações: ao baiacu, que ainda 
hoje serve de proverbial comparação para os que imitam a rã da fábula, designarem 
eles por sapo; e piraquiroá, traduzido ao pé da letra quer dizer peixe"-ouriço ou 
peixe"-porco"-espinho, nome dado pelos pescadores. Concluiremos o que temos a dizer sobre o 
cap. 136, depois de parar algum tempo admirando Soares a descrever a \textit{Malthea Vespertilio}, 
que tão frequente é em nossas águas, com o nome de morcego"-do"-mar. Foi 
com um exemplar preparado, que tem o nosso Museu do Rio de Janeiro, e depois com 
outro que se acabava de pescar, à vista, que tivemos bem ocasião de admirar o gênio 
observador e talento descritivo de Soares. \textit{Vacupuá} é seguramente adulteração de 
\textit{Baiacu"-puá}.

\item Deixamos para os que venham a fazer \textit{ex"-professo} estudos sobre a nossa 
ictiologia, tão pouco estudada até agora, os exames que não nos é possível ultimar 
acerca da doutrina deste capítulo, além do muito que deixamos nos capítulos já 
comentados. O de que tratamos conclui com um crustáceo bem conhecido.

\item Seguem outros crustáceos. \textit{Uçá} é o \textit{Cancer uca} de Lineu ou \textit{Ocypode fossor} de Latr.

\item Mais crustáceos do gênero \textit{Cancer}, \textit{Grapsus} etc. O uso já admitiu a pronúncia 
e ortografia de \textit{siri} com preferência a todas as outras. O nosso autor dava"-lhe novo 
cunho de autoridade.

\item \textit{Leri} é o nome genérico da ostra, e ainda nos lembramos da graça que os tamoios 
acharam ao francês Lery de ter um nome como o deles. Abbeville (fol. 204) diz \textit{Rery}, e 
desta maneira de pronunciar (mais exata visto que segundo Soares os indígenas não 
tinham o L de Lei) veio Reritiba (Vasconc. not. 59).

\item Os testados de que trata Soares são conhecidamente \textit{Anodon}, \textit{Unio}, \textit{Mytilus}.

\item Descrevem"-se a \textit{Amputaria gigas} de Spix, alguns \textit{Bulimus}, \textit{Helix} etc. Os nomes 
indígenas notam"-se variantes dos do texto acadêmico, que traz o \textit{Papesi}, \textit{Oatapesi} e 
\textit{Jatetoasu} diferentes.

\item Compreende o capítulo vários equinodermos, parenquimatosos, pólipos etc.

\item São"-nos mui familiares os nomes e o gosto dos peixes lembrados no cap. 144, os 
quais se encontram nos rios do sertão; mas, sem exemplares à vista, não queremos 
arriscar opinião sobre o lugar que eles ocupam na ictiologia, sendo mui natural que pela 
maior parte estejam por classificar; ainda assim, conservamos lembrança da forma 
petromizonida dos muçus; da ciprínida das traíras; da silúrida dos tamuatás; da pérsida 
dos ocris etc.

\item Vêm de novo alguns testáceos e crustáceos: são \textit{Anodon}, \textit{Helix}, \textit{Unio} etc., de água 
doce.

\item O texto da Academia nomeava \textit{Goachamoi} o que em outros códices lemos 
\textit{Guoanhamú}; hoje dizemos \textit{ganhamu}.

\item Não havia, e insistimos ainda nesta ideia, no Brasil, nação tapuia. Esta palavra 
quer dizer ``contrário'' e os indígenas a aplicavam até aos franceses, contrários dos 
nossos, chamando"-lhes \textit{tapuytinga}, isto é, \textit{tapuia branco} (veja"-se o \textit{Dic. bras.}, Lisboa, 
1795, p. 42). Antigamente, no Brasil, como atualmente ainda no Pará, chamava"-se 
tapuia ao gentio bravo; e tapuia se iam chamando uns aos outros, os mais aos menos 
civilizados. Quando os tupis invadiram o Brasil do norte para o sul (e não do sul para o 
norte, como pretendeu Hervas e com ele Martius), chamaram \textit{tapuias} às raças que eles 
expulsaram. Os tupis, que a si se chamavam tupinambás, ou tupis abalizados, foram 
logo seguidos de outros de sua mesma raça, que se chamavam tupinambás, e deram aos 
vencidos que empurraram para o sul e para o sertão o nome de \textit{tupi"-ikis} e de \textit{tupin"-aem}, 
isto é, tupis laterais e tupis maus, como já dissemos (com. 39).

\item  O fracionamento crescente da raça túpica, que se estendia por
quase todo o Brasil, na época do descobrimento, era tal que não exageram
os que creem que, a não ter lugar a colonização europeia, a mesma raça devia
perecer assassinada por suas próprias mãos; como quase vai sucedendo nestes
matos virgens, em que temos índios bravos fazendo"-se uns aos outros crua
guerra. Sem a desunião da raça tópica nunca houvera uma nação pequena
como Portugal colonizado extensão de terra tão grande como a que vai do
Amazonas ao Prata. Os primeiros colonos seguravam"-se na terra à custa
desta desunião, protegendo sempre um dos partidos, que com essa superioridade 
ficava vencedor, e se unia aos da nova colônia, mesclando"-se com ela
em interesses, e até em relações de parentesco etc. Às vezes, chegavam a
fomentar a desunião política, o que não deve admirar quando vemos que isto ainda 
hoje é seguido, e que nações, aliás poderosas, não conquistariam muitas vezes nações 
fracas, se dentro destas não achassem partidos discordes em quem pudessem encontrar ponto 
de apoio sua alavanca terrível.

\item O nome indígena do termo da Bahia deve estar certo, porquanto os jesuítas o 
repetem, escrevendo"-o \textit{Quigrigmuré}. Cremos ser a mesma Bahia o local a que se 
quis referir Thevet (fl. 129) com o nome de Pointe de \textit{Crouestimourou}. Não andaria, 
porém, já neste nome a ideia da residência do Caramuru?

\item Neste capítulo confirma Soares que o nome dos indígenas, antes de se dividirem, 
era o de tupinambás; e que falavam geralmente a mesma língua por toda a costa, e 
tinham os mesmos costumes etc.

\item O principal cacique dos tupinambás tinha (e tem ainda) entre eles o nome de 
\textit{morubixaba}. No nosso museu há o retrato de um de Mato Grosso todo vestido de gala, 
e que no batismo se chamou (como o governador) José Saturnino.

\item A respeito da condição da mulher entre os tupinambás consulte"-se o que diz o 
pe.~Anchieta (Tom. l.º da 2ª S. da \textit{Rev. do Inst.}, p. 254). Esse escrito de Anchieta devemos 
à bondade do nosso amigo o Sr. Dr. Cunha Rivara, bibliotecário de Évora, e que tantos 
outros serviços tem prestado às letras brasileiras.

\item As axorcas usadas pelas mulheres eram denominadas como diz o nosso autor, pois 
que o confirma Abbeville escrevendo (fl. 274) \textit{tabucourá}.

\item Os primeiros apelidos derivavam, entre os tupis, segundo Soares, 1º de animais, 2º 
de peixes, 3º de árvores, 4º de mantimentos, 5º de peças de armas etc.
É o que sucede por toda a parte com a raça humana. Nos nossos mesmos nomes não 
acontece isso? Vejamos: 1º, Leões, Lobos, Coelhos, Cordeiros, Carneiros, Pacas etc.; 2º, 
Sardinhas, Lampreias, Romeiros etc.; 3º, Pinheiros, Pereiras, Titara etc.; 4º, Leites, 
Farinhas, Trigos, Cajus etc.; 5º, Lanças, Couraças etc. O que dizemos dos nossos nomes 
pode aplicar"-se aos ingleses, franceses, alemães etc.

\item \textit{Metara} era o nome indígena dos botoques da cara; às vezes tinham a forma de uma 
bolota grande; outras vezes eram como uma muleta em miniatura. É claro que, com tais 
corpos estranhos na boca e nas faces, a fala dos gentios se dificultava, ou, antes, era 
mais difícil entendê"-los, nem que tivessem a boca cheia, como diz Thevet. Quando 
tiravam o botoque saía a saliva pelo buraco, e por graça deitavam eles às vezes por ali a 
língua de fora. Temos visto botoques de mármore, de âmbar e de cristal de rocha.

\item O bicho em questão de pele peçonhenta é descrito por Soares no cap. 66, sob o 
nome de \textit{socaúna}.

\item O parentesco mais prezado deste gentio depois do de pai a filho, era o de tio 
paterno a sobrinho. Pelo sangue de mãe não havia parentesco, o que também era 
admitido entre os antigos egípcios. Os romanos também faziam grande diferença entre 
o parentesco dos tios paternos e maternos, distinguindo \textit{patruus} de \textit{avunculus}, e sendo 
aquele o segundo pai, padrinho ou preceptor nato. Assim, a ideia de \textit{fraternidade}, de 
que o Evangelho se serviu e se servem hoje os filantropos como protótipo dos 
sentimentos da piedade e caridade, não era o que grassava entre essas raças; e, na 
verdade, já desde Caim e Abel, os irmãos, por via de rivalidades cotidianas, nem 
sempre são modelos de sentimentos puros, caridosos e pios, que o cristianismo quis 
simbolizar com a fraternidade. Os tupis davam preferência ao parentesco do patruísmo, 
e diziam"-se por ventura uns aos outros, tios, como nós hoje em comunhão social nos 
dizemos irmãos. Na Espanha e Portugal, e mesmo entre nós, no sertão, ainda se chama 
\textit{tio} a qualquer homem do campo ou do mato a quem se não sabe o nome; \textit{irmão} diz"-se 
aos pobres, quando se lhes não dá esmola, e \textit{pai} ou \textit{paizinho} aos pretos, sobretudo 
quando velhos. Temos ideia de haver lido que o uso antigo de chamar"-se a gente por 
tios procede do tempo dos fenícios e dos egípcios. Sendo assim, teríamos nestes 
fatos mais um ponto de contato para a possibilidade de relações de outrora entre o Egito 
e América, acerca do que Lord Kingsborough apresentou tantas probabilidades. É certo 
que a mesma expressão \textit{tupi} quer dizer tio, segundo Montoya, e pode muito bem ser 
que o nome que hoje damos à raça, não signifique senão tios; assim, \textit{tupi"-mbá} 
significaria os tios boa gente;  \textit{tupi"-aem} os tios maus; \textit{tupi"-ikis} os tios contíguos etc. 
Os nossos africanos ainda se tratam mutuamente por tios. E talvez não só em virtude do 
uso europeu, como do dos tupis, e quem sabe se mesmo deles africanos. Não faltará 
quem ache estas nossas opiniões demasiado metafísicas; mas não são filhas de dúvidas 
que temos, e publicando"-as não fazemos mais que levá"-las ao terreiro da discussão.

\item Segundo Thevet (fl. 114 v.) para fazer o sal ferviam a água do mar até engrossá"-la 
e ficar ela em metade, e tinham então uma substância com que faziam cristalizar esta 
calda salitrosa.

\item O \textit{timbó} e o \textit{tingui} são o trovisco do Brasil. --- Quanto à criação de animais e 
pássaros domésticos, era ela anterior à colonização, porquanto já na carta de Pero Vaz 
de Caminha se lê que com isso se ocupavam os das aldeias vizinhas a Porto Seguro.

\item Recomendamos a leitura deste capítulo 160 aos que sustentam o pouco préstimo do 
nosso gentio, que por ``filantropia'' estamos deixando nos matos tragando"-se uns aos 
outros, e caçando os nossos africanos (a que chamam ``macacos do chão'') só para os 
comer!

\item O uso de comer terra e de mascar barro é coisa ainda hoje vista entre alguns 
caboclos e moleques.

\item Também chamamos a atenção sobre este capítulo. Tal é a magia da música e da 
poesia que a apreciam até os povos sepultados na maior brutalidade.

\item Quanto aqui se relata é confirmado por Lery, Thevet, Fernão Cardim e mais 
viajantes antigos. \textit{Ereiupê} era o salamaleque da raça tupi.

\item Cangoeira de fumo era nem mais nem menos do que um cigarro monstro cuja capa 
exterior se fazia de folha de palmeira, em lugar de ser de papel, ou de folha de milho ou 
do mesmo tabaco.

\item O uso de curar feridas com fogo debaixo de si foi advertido por Pero Lopes, 
quando diz que se curavam \textit{ao fumo}. O último parágrafo deste capítulo não se encontra no texto da Academia.

\item O apuro dos sentidos entre os indígenas é proverbial: e ainda nos tempos modernos 
se vê confirmado por todos os viajantes que têm visitado as cabildas errantes em nossas 
matas.

\item Em vez de \textit{tajupares} escreveu o autor do \textit{Dic. bras.} (p. 21) \textit{tejupaba}, e Abbeville 
(fol. 3 v. e 121) \textit{aiupawe}.

\item \textit{Caiçá} era o nome do tapigo, tapume silvado ou sebe, que fazia a contracerca ou 
circunvalação das tranqueiras ou palancas. É palavra que se encontra três vezes na  
\textit{Relação da tomada da Paraíba} do pe.~Jerónimo Machado.  \textit{Cazia} diz o texto acadêmico.

\item Como tipo da eloquência guerreira indígena eram consideradas as declamações do 
célebre principal Quoniambebe, de quem trataremos em outra ocasião.

\item O apelido de nascença, de que tratamos (com. 228), só servia aos indígenas 
enquanto por alguma façanha não conquistavam outro mais honroso. Pode"-se dizer que 
com este segundo nome ficavam titulares. Para memória dos novos títulos sarjavam o 
corpo de riscos indeléveis; o que era honra de que só usava quem a conquistava. Eram 
os riscos como uma farda ou condecoração, que promoveram o riso, quando trazidos 
por quem não as houvesse de direito.

\item \textit{Mazaraca} dizia aqui, em vez de \textit{muçurana}, o texto acadêmico. As relações dos 
prisioneiros com as gentias, que lhes davam por companheiras, poderiam talvez 
explicar a salvação de alguns. Deste modo encaramos o assunto do Caramuru como 
romance histórico.

\item Era para o gentio reputado vil covardia do prisioneiro o não afrontar a morte com 
arrogância, e o não exalar o último suspiro com alguma afronta contra os vencedores. 
Assim os indígenas deviam fazer triste ideia dos cristãos quando eles pediam a Deus a 
misericórdia na hora da morte, ou faziam alguma outra súplica. Foi por isso que a 
Câmara da Bahia, representando ao rei contra a ineficácia das Ordens régias de se 
levarem os mesmos indígenas por meios de brandura, disse que eles não agradeciam esses meios 
brandos, antes se enfatuavam mais com eles, imaginando que provinham do medo. --- ``Se \textsc{v}. 
Alteza quiser tomar informações por pessoas que bem conheçam a qualidade do gentio desta 
terra, achará que por mal, e não por bem, se hão de sujeitar a trazer à fé: porque tudo o que 
por amor lhes fazem atribuem é com medo e se danam com isso.'' --- O mesmo assegura 
Thevet na sua \textit{Cosmog.}, fol. 909, falando dos antigos tupinambás ou tamoios do Rio de 
Janeiro: ``Et estiment celuy là poltron, et lasche de coeur, lequel ayant le dessus de son 
ennemy, le laisse aller sans se venger, et sans le massacrer''.  É o que ainda sucede com os 
dos nossos sertões. Os bugres recebem presentes de ferrinhos que no ano seguinte enviam 
contra o benfeitor mui aguçados, nas pontas de suas flechas; ou assassinam aqueles que, 
depois de lhes fazer presentes, neles confiam. Ainda temos na ideia o horror que nos causou 
o assassinato do sertanista Barbosa e seus dois companheiros, descrito em um número 
anterior (nº 19) da \textit{Revista do Instituto}.

\item \textit{Embagadura} é o nome do punho da espada tangapema; acha"-se repetido neste \textit{Tratado}, no cap. 80.

\item \textit{Moquém} (de onde derivou o nosso verbo \textit{moquear}) é a mesma expressão que na 
América do Norte se converteu em \textit{boucan}, de onde veio bucaneiro.

\item Por este capítulo 175 vemos que entre os tupinambás da Bahia só os moços iam à 
cova dentro de talhas pintadas (\textit{igaçabas} ou \textit{camucins}); falta, pois, examinar se essas 
múmias acocoradas que se têm encontrado em talhas contêm cadáveres que se possam 
julgar de pessoas adultas.

\item Algumas particularidades narradas por Soares têm analogia com o que praticava a 
antiguidade, tanto no que respeita ao carpir os mortos, como ao desamparar ou matar os 
doentes em perigo.

\item O pequeno mui alvo de que dá notícia Soares, quanto a nós, é o caso de um albino 
na raça tupinambá. Não temos notícias de outros fatos ou exames a tal respeito.
A frequência e familiaridade com que Soares se serve já em seu tempo da palavra 
\textit{mameluco} faz"-nos crer que ela foi adotada no Brasil com analogia ao que se passava na 
Europa. Sem nos ocuparmos da etimologia dessa palavra (que é árabe, língua que não 
conhecemos) nem das acepções diferentes em que foi tomada, sabemos que nos séculos \textsc{xv} e 
\textsc{xvi} chamavam vulgarmente na Espanha, e talvez também em Portugal, mamelucos os 
filhos de cristão e moura ou de mouro e cristã. O nome brasílico para mestiço era \textit{curiboca}, 
que hoje se emprega noutra acepção.

\item  \textit{Tabuaras} dizem algumas cópias, em vez de tapuras, o que pouco
dista de tapuias. Abbeville (fol. 251 v.) é de parecer que \textit{tabaiares} quer
dizer ``grandes inimigos'';  assim será, mas não se confunda com \textit{tabajaras}, que quer 
dizer \textit{os das aldeias} ou \textit{os aldeões}.  Talvez o nome em questão se devesse ler 
antes \textit{tapurá}, e neste caso seria quase o mesmo que \textit{tubirá} ou \textit{timbirá}, que ainda hoje se dá a 
uma nação do sertão; \textit{timbirá} é nome injurioso, como ``patife'' 

\item Pelo que nos revela Soares a invasão dos Tupinaes devia ser muito numerosa, 
porquanto se diz que eles ``andavam correndo toda a costa do Brasil''  antes da vinda 
dos tupinambás.

\item Amoipiras quer dizer ``os parentes cruéis'', \textit{Amôig}, parente (\textit{Tesoro} de 
Montoya, fol. 32 v.) e \textit{Pira}, cruel (fol. 297 v.). Merece, pois, quanto a nós, menos 
crédito a etimologia de Soares de um chefe chamado Amoipira.

\item O que Soares conta da indústria dos Amoypiras é aplicável em tudo ao que praticava 
o mais gentio antes de comunicar com os europeus.
No nosso Museu da Corte e no do Instituto Histórico se guardam vários utensílios em 
tudo primitivos. As folhas dos machados eram umas cunhas de pedra esverdeada, como de 
sienito ou diorito, bem que pela dureza se deviam julgar de pórfiro. --- De pedra usavam 
também grandes bordões, como as alavancas ordinárias, que lhes serviriam de arma 
ofensiva, e a perfeição como são feitas basta para caracterizar a paciência dos artistas, que 
não usavam de metais, nem de mós.

\item  Vasconcelos (p. 146 e 148) dá notícia de outra nação de \textit{igbiraiaras},
a que os nossos chamavam bilreiros, no sul do Brasil.
Temos de novo que lastimar a credulidade do século: agora são mulheres de uma só 
teta, que pelejavam como amazonas.

\item Soares, com seu espírito penetrante, explica a verdadeira causa da vitória dos 
estrangeiros tupis contra as antigas raças que habitavam o nosso território pela desunião 
delas entre si: ``Por onde se diminuem em poder para não poderem resistir a seus 
contrários, com forças necessárias, por se fiarem muito em seu esforço e ânimo, não 
entendendo o que está tão entendido, que o esforço dos poucos não pode resistir ao 
poder dos muitos''. 

\item O nome de \textit{maracás} procedeu, talvez segundo muito bem nos lembra o nosso 
erudito amigo Sr. Joaquim Caetano da Silva, de temerem eles com a fala e imitarem, 
com isso, a bulha dos maracás.

\item Alude Soares, e só por informações gerais, a todo gentio que habitava  as terras 
das hoje províncias de Goiás, Mato Grosso e Pará.

\item Os habitantes das serras do sertão que viviam como trogloditas seriam 
naturalmente os parecis.

\item A rocha que tanta admiração causa ao autor é talvez alguma de formação 
secundária ou terciária, abundante de incrustações.

\item As pedras de alfebas são, naturalmente, produtos zoófitos. Com as \textit{formas} 
feitas de barro, sem ser louça nem telha e tijolo (se não houver erro dos copistas), queria 
talvez Soares designar os potes, cântaros etc.

\item Dá uma ideia da prosperidade da Bahia em 1587 o haver aí 240 carpinteiros e cinquenta 
tendas de ferreiros, com seus obreiros.

\item Da árvore \textit{camaçari} tratou suficientemente Soares no cap. 67. --- Cremos que até 
hoje não se tem ninguém aproveitado de sua lembrança para fabricar dela alcatrão e 
mais produtos resinosos, como a terebentina, breu e o competente ácido pirolenhoso 
ou água"-russa.

\item A palmeira, de cujas \textit{barbas} diz Soares que se faziam amarras, era a conhecida 
\textit{piaçaba}, nome que em Portugal se adotou, pronunciando"-o \textit{piaçá}.

\item \textit{Adargoeiro} é talvez a árvore africana que hoje se diz dragoeiro, que dá o 
sangue"-de"-drago; e o nome dragoeiro anda corrompido se acaso a madeira da árvore serviu 
alguma vez para adargas.

\item Soares, levado de bons desejos, acreditou na existência de minas de aço, e 
imaginou porventura que o aço se tirava em Milão da rocha, já pronto.
Quanto ao que diz do cobre nativo, não tardou que os fatos o confirmassem, a ponto 
que de junto da Cachoeira saiu um dos maiores pedaços de cobre nativo conhecidos, qual é 
o que se guarda na História Natural de Lisboa.

\item  Já dá Soares notícia que no seu tempo vinham do sertão, de mistura 
com o cristal, ``pontas \textit{oitavadas} como diamantes, \textit{lavradas pela natureza},
de muita formosura e resplandor''. Não teremos aqui a primeira notícia de diamantes no sertão da Bahia? --- Quanto às 
pedras verdes dos beiços, que se tiravam das montanhas, já delas faz menção Thevet (fol. 
121) em 1557. Cabral viu já dessas pedras em 1500, segundo Caminha.

\item  As esmeraldas descobertas no século \textsc{xvi} seriam naturalmente as
turmalinas. Thevet (\textit{France antarctique}, fol. 63) diz ter visto pedras que
se podiam julgar verdadeiras esmeraldas.
As rochas eram evidentemente de ametista ou quartzo hialino violeta, cuja abundância 
em nossos sertões é tal que foi causa de que baixassem de preço no mercado tais pedras.

\item  Soares, não contente com ter inculcado a um valido de Filipe \textsc{ii} a
grande importância do Brasil (no livro que por vezes ele denomina francamente 
de \textit{Tratado}) receoso que essa corte, onde só se atendia às riquezas
do Peru e à guerra aos hereges, não se comovesse senão por aliciantes análogos, 
conclui sua obra com asseverar: 1º, que das minas do Brasil poderiam
quase, sem trabalhos nem despesas, tirar mais riquezas do que das Índias
Ocidentais; 2º, que se não cuidavam do Brasil e os luteranos viessem a
saber o que por cá havia, não tardariam em se assenhorear da Bahia, e se	\EP[-1]
o chegassem a efetuar muito custaria botá"-los fora.

Estas duas verdades proféticas 
fariam, só por si, a reputação de um homem, ainda 
quando ele não houvesse escrito, como Soares, um \textit{Tratado} verdadeiramente enciclopédico 
do Brasil. Os holandeses vieram na América vingar"-se de Filipe \textsc{ii} e do seu Duque de 
Alba, e as minas de Minas inundaram o universo, do século passado para cá, de ouro e 
diamantes. Do homem superior que tinha entregue grande parte de seu tempo a observar, 
a meditar e a escrever, nenhum caso naturalmente se fez. O seu livro esteve quase dois 
séculos e meio sem publicar"-se, e o autor naturalmente depois da dilação (como ele diz) de 
seus requerimentos em Madri, veio a passar vida tão obscura que nem é sabido quando, nem 
onde morreu. Assim aconteceu também, e ainda outro dia, ao homem que depois de Soares 
mais notícias deu acerca do Brasil: ao modesto autor da \textit{Corografia brasílica}.
\medskip

\indent\textit{F.~Adolpho de Varnhagen}  

\hspace{\stretch{1}}\textit{Rio de Janeiro, 15 de setembro de 1851}\\
\end{enumerate}




