\chapter{Sobre o projeto \textsc{fwp} e as narrativas de escravizados}
\hedramarkboth{sobre o projeto fwp\ldots}{}

\section{Sobre o autor}

\noindent\textit{Nascidos na escravidão} não é obra de única autoria,
mas um conjunto de vozes daqueles que vivenciaram um dos maiores horrores da humanidade:
a escravidão. Entre as milhões de pessoas que vivenciaram a escravidão norte"-americana, 
muitas sobreviveram e tiveram seus relatos e depoimentos de vida colhidos através do
Projeto Federal de Escritores (\textsc{fwp}), criado em 1929 para garantir uma renda a escritores desempregados durante a Grande Depressão.

São narrativas que revelam a memória das milhares de pessoas sobreviventes ao trauma da escravidão. Ao todo, o Projeto Federal de Escritores salvou 2400 narrativas, aproximações de dentro, em perspectiva única, do que foi o escravismo sulista norte-americano, que às vésperas da abolição da escravatura contava com 4 milhões de escravizados em seus campos de trabalho. Nesses relatos o leitor vai encontrar a descrição do universo familiar dos cativos e a resistência do trabalhador, mas também a exploração do trabalho e a violência simbólica, física e até sexual cometida pelos senhores brancos. Seus testemunhos sugerem que o objetivo da escravidão negra nas Américas era análogo à utopia autoritária do capital no século \textsc{xxi}: desumanizar o ser humano até reduzi"-lo à condição inanimada e sedutora de uma mercadoria.

Das 2400 narrativas coletadas pelo \textsc{fwp}, o historiador Tâmis Parron coletou 204 para compor esse \textit{Nascidos na escravidão: depoimentos norte"-americanos}.
Formado em jornalismo e história pela Universidade de São
Paulo (\textsc{usp}), Tâmis também organizou a edição de \textit{Iracema}, de José de Alencar
(Hedra, 2006) e as \textit{Cartas a favor da escravidão} (Hedra, 2020) do mesmo autor. Atualmente, finaliza o mestrado \textit{A política da escravidão
no Império do Brasil}, e escreve em co"-autoria com Rafael de Bivar Marquese e
Márcia Berbel um livro de história comparada sobre a defesa política do sistema
escravista na monarquia brasileira e no império espanhol, no âmbito do Projeto
Temático ``Formação do Estado e da Nação: Brasil, c.~1780--1850'' (Fapesp).



\section{Sobre a obra}

Inéditas no país, essas 204 narrativas de ex-escravizados norte-americanos foram organizadas sobre temas centrais do escravismo na América: cultura negra, resistência, violência, relações familiares durante a escravidão e relações raciais, trabalho, emancipação e condições de vida.
Ao final de alguns relatos, entre os mais significativos de cada eixo temático, encontram"-se também comentários do historiador norte"-americano Paul D.\,Escott, especialista em história da escravidão na Wake Forest University, que contextualizam as declarações de um ex"-escravizado ou fornecem informações úteis sobre as realidades econômicas e sociais e as variações internas da escravidão nos Estados Unidos da América.
Além de coletar os relatos, os participantes do Projeto Federal de Escritores também tiraram fotos dos ex"-cativos, também reunidas nessa edição para enriquecer a compreensão moderna do que foi a mais cruel instituição da história humana.

A média de idade das pessoas ouvidas nessas entrevistas era, na época em
que foram realizadas, de cerca de oitenta anos. Por isso, temos de ter a
ideia de que grande parte dos entrevistados viveu a escravidão durante a
infância, ficando em cativeiro até por volta dos dez anos. Isso ocorreu
porque, em 1860, eclodiu a Guerra Civil Norte"-Americana, ou Guerra de Secessão (1860--1865), responsável pelo fim do sistema escravagista no país.
Como grande parte dos entrevistados passou a infância sob a condição de
escravizados, a obra permite observar como funcionava o regime
escravista dos Estados Unidos sob a perspectiva de testemunhas reais.

Há um dado curioso sobre o contexto em que as entrevistas que deram
origem ao livro foram realizadas: quando o entrevistador era uma pessoa
branca, muitas informações sobre a brutalidade do regime eram omitidas.
Já quando o entrevistador era uma pessoa negra, os entrevistados falavam
toda a verdade e expressavam sentimentos e opiniões sobre o passado.
Além do mais, os anos de 1930 foram marcados por uma forte segregação
racial institucionalizada nos Estados Unidos, sobretudo nos estados do
sul do país.
É preciso ter essa informação em perspectiva para se pensar como, a depender da condição da entrevista, os relatos eram matizados e amenizados, sobretudo pelo receio do entrevistado de ofender os entrevistadores brancos. Assim, os entrevistados geralmente
começavam dizendo coisas positivas sobre seus donos e a sua experiência
sob a escravidão.

Apesar dessa limitação,
a obra permite observar o contexto vivido pelos escravos durante o
século \textsc{xix} estadunidense. O país, nos anos de 1850 e 1860, antes da
eclosão da Guerra Civil, tinha cerca de quatro milhões de negros
escravizados, concentrados nos estados do sul. Esse número de cativos só
é inferior ao índice registrado no Império Romano, na Antiguidade. Os
principais produtos cultivados nos Estados Unidos, pelos senhores de
escravos, e que estavam voltados para o mercado externo, eram o algodão,
açúcar, arroz e tabaco. O algodão foi o produto de maior
relevância porque, no contexto da Primeira Revolução Industrial, era
responsável por movimentar grande parte da economia do mundo
capitalista.

Nesse sistema de \emph{plantation}, a dinâmica de trabalho de um
escravizado na fazenda de seu senhor era exaustiva. A configuração do
mundo de trabalho, porém, não era homogênea para todas as localidades em
que vigorava a escravidão naquele país: além do trabalho, é preciso ter
em mente que esses escravizados eram agentes sociais que faziam
resistência ao sistema. Portanto, a falsa ideia de docilidade dessas
pessoas deve ser combatida. A fuga era um dos principais modos de
resistência dos cativos. Na obra \emph{Nascidos na escravidão}, o relato
de Thomas Cole, no contexto da Guerra Civil, torna possível observar as
condições que permitiram que ele conseguisse fugir e adentrar nas lutas
do \emph{front} pelo Norte.

Notamos também que ele menciona os cães. Os cachorros eram um dos
artifícios usados pelos senhores de escravos na caça aos cativos que
fugiam mata adentro. Isso ocorria porque a \emph{plantation}
norte"-americana era um vasto campo aberto, de propriedade do senhor de
escravos, o qual tinha grande domínio visual do espaço. Porém, quando um
escravizado fugia em direção à mata, esse domínio se perdia e capatazes
e cachorros seguiam trilhas e pistas para capturar os cativos fugitivos.
Em uma situação de guerra, essas rotas ganhavam um significado ainda
maior como símbolos de resistência.

A estrutura da escravidão no sul dos Estados Unidos estava consolidada e
não houve um desgaste gradual ao longo do tempo, como aconteceu em
território brasileiro, por exemplo. O escravismo norte"-americano acabou
de maneira abrupta com a eclosão da Guerra Civil e com a luta das
colônias do Norte contra as colônias do Sul.

Outro dado que nos ajuda a compreender o sistema escravista dos Estados
Unidos, em comparação com o brasileiro, é o fato de que, com o fim desse
regime no ano de 1865, houve algumas políticas públicas para a inserção
dos negros escravizados na sociedade estadunidense. Ainda assim, essa
experiência durou apenas doze anos. Com o fim da dita Reconstrução, no
ano de 1877, sobreveio a ascensão do segregacionismo racial e o
surgimento de grupos supremacistas brancos. Ao mesmo tempo, o trabalho,
a resistência e a emancipação dos ex"-escravizados, conforme demonstrado
na obra, são pontos que igualmente merecem destaque para a compreensão
do período.

A obra \emph{Nascidos na escravidão} traz relatos sobre como era a
escravidão, do ponto de vista dos próprios escravizados, e essa
perspectiva rompe com uma historiografia tradicional, que dava
importância somente para documentos e fontes exclusivas dos senhores de
escravos, homens brancos e da elite daquele país.

\section{Sobre o gênero}

O gênero do relato, em linhas fundamentais, define"-se como a narrativa em que um sujeito, inscrito em um determinado tempo histórico, debruça"-se sobre fatos, descrições e interpretações desse momento histórico no qual vive. Para o historiador francês Paul Veyne, o relato histórico segue uma forma similar à forma tradicional de escrever história, seguindo um \textit{continuum} espaço"-temporal.

Apesar dessa relativa unidade, o relato, considerado como uma forma de fazer história,
é parcial e subjetivo, pois não consegue apreender a globalidade dos acontecimentos, apenas
aquilo que está ao alcance do narrador e, mesmo isso, não de uma forma pura, mas filtrado pela sua subjetividade e pelos objetivos de seu relato.
Para Veyne, estaríamos assim quase próximos do romance:

\begin{quote}
A história é uma narrativa de eventos: todo o resto resulta disso. Já que é, de fato, uma narrativa, ela não faz reviver esses eventos, assim como tampouco o faz o romance; o vivido, tal como ressai das mãos do historiador, não é o dos atores; é uma narração,
o que permite evitar alguns falsos problemas. Como o romance, a
história seleciona, simplifica, organiza, faz com que um século
caiba numa página.\footnote{\textsc{veyne}, Paul. \textit{Como se escreve a história}. Brasília: Editora Universidade de Brasília, 1999, p.\,18.}
\end{quote}

Seguindo nessa linha de pensamento, podemos observar, com o historiador francês Marc Bloch, que o relato é apenas um ``vestígio'' da história, um pequeno pedaço do factual que, pela pena de um narrador, pôde"-se cristalizar no tempo e ser transmitido a gerações posteriores, sendo apenas uma das infinitas possibilidades de apreensão e compreensão de determinados fenômenos:

\begin{quote}
Quer se trate das ossadas
emparedadas nas muralhas da Síria, de uma palavra cuja forma ou emprego revele um
costume, de um relato escrito pela testemunha de uma cena antiga [ou recente], o que
entendemos efetivamente por documentos senão um ``vestígio'' quer dizer, a marca,
perceptível aos sentidos, deixada por um fenômeno em si mesmo impossível de captar?\footnote{\textsc{bloch}, Marc. \textit{Apologia da história}. Rio de Janeiro: Zahar, 2002, p.\,73.}
\end{quote}

No caso destes \textit{depoimentos norte"-americanos}, evidencia"-se seu caráter
de relato pois os entrevistados estavam circunscritos a um determinado momento da escravidão norte"-americana --- como ressaltado acima, a maioria dos ex"-escravizados que ainda estavam vivos para contar as suas histórias tinham vivido a escravidão quando crianças, e em um momento no qual essa instituição passava a ser questionada e estava prestes a ser abolida com a Guerra Civil. Seus depoimentos, portanto, configuram relatos de um determinado momento da escravidão norte"-americana.
Além disso, é preciso levar em conta a situação na qual as entrevistas se deram.
Como ressalta o historiador Paul D.\,Escott, o fato de as conversas serem conduzidas, em sua maioria, por escritores brancos, em um momento de forte segregação racial nos Estados Unidos, influenciou muito o conteúdo dos depoimentos:

\begin{quote}
Henry Alsberg, o diretor nacional do Projeto Federal de Escritores,
insistiu com todos os entrevistadores que estes deviam tomar o máximo de
cuidado para não influenciar o ponto de vista do informante e enfatizou
que \emph{todas as histórias deveriam ser reproduzidas palavra por
palavra sempre que possível}. Infelizmente, as suas sugestões não tinham
como garantir que as narrativas seriam uma expressão imaculada, direta e
sem enfeites das perspectivas dos ex"-escravizados. Além de seus conselhos
somente terem sido recebidos após os entrevistadores já terem começado a
se encontrar com eles em diversos estados, e mais importante
ainda, as circunstâncias sociais da época exigiam que os entrevistados
usassem de notável cautela ao falarem com os entrevistadores.

A supremacia branca era uma realidade tirânica no Sul da década de 1930.
A subordinação rígida dos negros era a regra no Sul segregado, e isso
naturalmente moldou o jeito como os afro"-americanos interagiam com
pessoas brancas. Se ofendiam um branco, nada, nem mesmo sua idade avançada, poderia protegê"-los de consequências desagradáveis.
Como quase todos os entrevistadores do \versal{FWP} eram brancos, os ex"-escravizados
estavam cientes da necessidade de observar todas as regras da etiqueta
racial. O fato de os entrevistadores se apresentarem como agentes do
governo federal também afetou as conversas. Em geral, os entrevistados
eram pobres, lutando contra a fome e a pobreza, e tinham a esperança de
obter uma pensão ou alguma outra forma de auxílio do governo. Por
consequência, tanto os ex"-escravizados quanto a equipe do \versal{FWP} agiram de
maneiras que afetaram a natureza das narrativas.\footnote{Página \pageref{henry} da presente edição.}
\end{quote}

Outro gênero literário ao qual esses relatos se filiam são as ``narrativas de escravos''
(\textit{slave narratives}), autobiografias de ex"-escravizados focadas nas histórias de liberdade que detalhavam a reação de seus autores à escravidão e seus caminhos para a liberdade. No Estados Unidos, surgiram em princípios da década de 1820, quando escritores começaram a compor obras contra a emergente ficção regionalista do Sul, que representava a vida dos brancos e negros nas fazendas escravistas como um jardim das delícias.

Com a publicação de várias autobiografias de ex"-escravizados que fugiram do cativeiro --- como as de Harriet Jacobs, William Wells Brown e Frederick Douglass --- as narrativas de escravos consolidaram"-se como um gênero literário em meados do século \textsc{xx}, quando começaram a despontar mais narrativas, como essas coletadas pelo \textsc{fwp}, e passaram gradativamente a ser aceitas pelos historiadores e acadêmicos como peças fundamentais para entender a escravidão nas Américas.

Na definição do historiador William L.\,Andrews, professor da Universidade da Carolina do Norte, as ``narrativas de escravos'' podem ser entendidas como ``qualquer relato da vida, ou uma parte importante da vida, de um fugitivo ou ex"-escravo, escrito ou relatado oralmente pelo próprio escravo''.\footnote{Disponível em \emph{http://nationalhumanitiescenter.org/tserve/freedom/1609-1865/essays/slavenarrative.htm}.}


A primeira narrativa de escravo que se tem história é a 
\textit{Interesting Narrative of the Life of Olaudah Equiano, or Gustavus Vassa, the African}
(1789), em que Olaudah Equiano narra a sua trajetória de vida desde sua infância na África Ocidental, passando pelo tráfico transatlântico, sua situação de cativo, e encerrando"-se com sua liberdade e o sucesso financeiro como um cidadão britânico.
A história de Equiano, publicada na Inglaterra, chocou o público ao revelar os horrores da escravidão e balançou o senso"-comum da época que via na escravidão uma bem aceita e estabelecida instituição socioeconômica inglesa.

Já nos Estados Unidos da América, a primeira narrativa de escravo foi a \emph{Life of William Grimes, the Runaway Slave, Written by Himself}, de 1825.
Com seu relato, William Grimes revelou aos leitores do Norte o verdadeiro terror do cativeiro nos estados do Sul e as injustiças raciais da Nova Inglaterra.
Essas autobiografias que deram início ao gênero eram, na análise de Calvin Schermerhorn, pesquisador e professor da Universidade Estadual do Arizona,
``um híbrido de gêneros diversos, incluindo narrativas de cativeiro, literatura de protesto, confissão religiosa e relatos de viagem''.\footnote{\textsc{schermerhorn}, Calvin. ``Introdução''. In: \textsc{brown}, William Wells. \textit{Narrativa de William Wells Brown,
escravo fugitivo, escrita por ele mesmo}. São Paulo: Hedra, 2020, p.\,9.}

No entanto, como ressalta Schermerhorn, a maioria dessas biografias foram publicadas com auxílio editorial de brancos e para públicos brancos.  “Assim, desde os primeiros momentos da autobiografia negra na América, domina a pressuposição de que o narrador negro precisa de um leitor branco para completar o seu texto, para construir uma
hierarquia de significância abstrata referente ao simples
conjunto dos seus fatos, para oferecer uma presença onde
antes havia apenas um `Negro', uma ausência escura.”\footnote{\textsc{andrews}, William L. \textit{To Tell A Free Story: The First Century of Afro-American Autobiography, 1760–1865}, 1988, p.\,32--33 \textit{apud} \textsc{schermerhorn}, Calvin, op.\,cit., p.\,9--10.}

Desse conflito entre um gênero escrito por negros no meio de uma sociedade branca e racista, desponta outra característica das ``narrativas de escravo'', a modelação de sua história para ser bem"-aceita pelo público leitor branco:

\begin{quote}
O testemunho era a pedra fundamental da autobiografia de um ex-escravizado, e um dos principais desafios artísticos enfrentados pelos autores negros foi como atrair a
simpatia do leitor para que enxergassem as cenas de subjugação da maneira apresentada. Muitas vezes, isso significava aceitar boa parte da cultura anglo-americana como
normativa, sugerindo que a civilidade dos brancos seria a
razão por que era inaceitável a crueldade dos escravizadores. Os americanos civilizados não deveriam tolerar o barbarismo da escravidão. Para os escritores negros, isso
significou atenuar a importância das tradições, culturas, religiões e idiomas da África e dar preferência aos valores e tradições dos povos descendentes de europeus. Significou atenuar o radicalismo e a militância que caracterizou líderes descendentes de escravizados como Toussaint ou Jean-Jacques Dessalines no Haiti.\footnote{Ibidem, p.\,10.}
\end{quote}