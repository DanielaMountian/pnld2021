\chapter*{Introdução}
\addcontentsline{toc}{chapter}{Introdução, \emph{por Paul D. Escott}}
\hedramarkboth{Introdução}{}

\begin{flushright}
\versal{PAUL D. ESCOTT}\\\vspace{-3pt}
\versal{WAKE FOREST UNIVERSITY}
\end{flushright}




\section{As Narrativas de Escravos do Projeto Federal de Escritores e Suas
Características}



No início, a academia ignorou essas fontes. A Guerra Civil dera fim à
escravidão legal, mas não estabelecera direitos e oportunidades iguais
para os emancipados. Na década imediatamente após o conflito, os homens
negros conquistaram o direito ao voto, mas os governos estaduais que
simpatizavam com os seus interesses logo foram derrotados no Sul. Na
última década do século \versal{XIX}, governos reacionários em todos os estados
sulistas aprovaram leis que praticamente roubavam o voto dos
afro"-americanos. Outras leis impuseram um sistema de discriminação
humilhante em locais públicos e serviços governamentais, um sistema de
discriminação que conquistou o selo de aprovação da Suprema Corte dos
Estados Unidos em 1896.\footnote{A decisão da Suprema Corte é conhecida
  pelo nome \emph{Plessy v. Ferguson}.} Assim, no início do século \versal{XX}, a
segregação racial era uma força dominante na sociedade sulista e o
preconceito racial era forte e disseminado no resto do país.

Os mais influentes entre os primeiros estudiosos da escravidão, como U.
B. Phillips, em geral compartilhavam das ideias da supremacia branca.
Phillips escreveu que a escravidão fora uma escola para os africanos
incivilizados e rejeitou as narrativas de escravos, dizendo que não eram
confiáveis e não tinham valor. Foi só nas décadas de 1950 e 1960 que as
atitudes começaram a mudar significativamente, e então o Movimento dos
Direitos Civis despertou as consciências de muitos brancos à medida que
desmantelava as leis segregacionistas. Os historiadores agora ansiavam
para entender melhor o passado racista dos Estados Unidos. Muitos
voltaram a sua atenção para a escravidão e começaram a usar as
Narrativas de Escravos do \versal{FWP}. Em 1972, a Greenwood Publishing Company
publicou uma edição em 19 volumes que reproduzia as transcrições
datilografadas originais.\footnote{Federal Writers' Project, \emph{The
  American Slave: A Composite Autobiography}, George Rawick, General
  Editor (Westport, \versal{CT}: Greenwood Publishing Company, 1972). Greenwood
  também publicou o \emph{Supplement: Series \versal{I}} em 1977 e o
  \emph{Supplement: Series \versal{II}} em 1979.} Posteriormente, foram
publicados volumes adicionais, incluindo trabalhos do Projeto Federal de
Escritores que nunca haviam sido enviados para Washington. Nos últimos
anos, em resposta à popularidade da internet e às vantagens da
tecnologia digital, a Biblioteca do Congresso dos \versal{EUA} colocou as
Narrativas de Escravos do \versal{FWP} na rede. Graças à Biblioteca do Congresso,
hoje qualquer um tem acesso a um pouco da história de vida dos
escravizados.\footnote{Disponíveis no site da \emph{Library of Congress}.}



\section{As Narrativas de Escravos e Interpretações~da~Escravidão}

