\chapter{Vida e obra de João Athayde}

\section{Sobre o autor}

\noindent{}Filho de Belchior Martins de Lima e de dona Antônia
Lima de Athayde,
João Martins de Athayde nasceu em Cachoeira de Cebolas, município de
Ingá, no interior da Paraíba, em 23 de junho de 1877. A cinquenta
quilômetros de Campina Grande, Ingá está no meio do sertão da Paraíba,
e foi nesse ambiente que Athayde cresceu. Sem frequentar a escola
formal, alfabetizou"-se ainda criança com as chamadas cartilhas, nos
momentos de pausa do trabalho no campo.
O poeta andava com
as cartilhas no chapéu, pois seu maior sonho era
aprender a ler e a escrever. E tanto era assim que ele saía
perguntando as letras às pessoas e, como não tinha caderno
nem lápis, escrevia no chão, com o dedo.

Não se sabe se ele fugiu de casa ou se teve o
consentimento do pai, mas
já adolescente saiu de Ingá
fugindo da seca e foi para o Recife trabalhar no comércio e em
algumas fábricas. Em 1908 começou a escrever seus primeiros folhetos
que vendia nas principais feiras e mercados da cidade.
Com o dinheiro da venda dos folhetos e com o que
ganhava nos empregos, conseguiu comprar uma pequena
impressora manual, uma guilhotina para cortar o papel
dos folhetos, alugar uma casa, contratar vários empregados.
A procura de seus folhetos tornou"-se muito grande, mas ele
procurava atender à freguesia e aos seus 
agentes"-vendedores em diversas cidades do Nordeste.

Ganhou fama com
seus poemas e com essa pequena tipografia que, ainda no início do século
\textsc{xx}, especializou"-se na publicação de cordéis. Athayde escreveu centenas
de folhetos e publicou inúmeros outros, já que comprou os direitos
autorais de diversos poetas. Sendo considerado o príncipe do cordel no
Nordeste, Athayde deixou de escrever em 1949, quando sofreu um \textsc{avc}, e
morreu em 1959 na cidade de Limoeiro, região da Zona da Mata de
Pernambuco.

\section{Sobre a obra}

Ninguém sabe, com absoluta certeza, quantos folhetos
foram escritos e publicados por João Martins de Athayde.
Sua gráfica, trabalhando a todo vapor, quase que
semanalmente lançava um título novo ou mais uma
edição/impressão de um folheto que, na época, estivesse
fazendo sucesso. As encomendas recebidas de seus agentes"-vendedores espalhados por todo o Nordeste chegavam quase todos os dias, e ele procurava entregar, imprimindo
durante as madrugadas.

O estudioso Mário Souto Maior fez um levantamento de todos os cordéis publicados por Athayde:

\textit{Amor de perdição}, \textit{A vitória da revolução brasileira}, \textit{O triste
fim de um orgulhoso}, \textit{Os últimos dias da humanidade ou O
fim do mundo}, \textit{Uma viagem ao céu}, \textit{A vida de Nascimento
Grande}, \textit{Amor de pirata}, \textit{O amor de um estudante ou O poder
da inteligência}, \textit{A filha do boiadeiro}, \textit{O fim do mundo}, \textit{A
morte de Lampião}, \textit{A órfã abandonada}, \textit{A nobreza de um
ladrão}, \textit{História da imperatriz Porcina}, \textit{História de José do
Egito}, \textit{Peleja de João Athayde com Raimundo Pelado}, \textit{O
Jeca na praça}, \textit{Juvenal e o dragão}, \textit{A lamentável morte de
padre Cícero Romão Batista -- O patriarca do Juazeiro},
\textit{Lampião em Vila Bela}, \textit{As proezas de Lampião}, \textit{Sacco e
Vanzetti aos olhos do mundo}, \textit{Uma noite de amor}, \textit{A paixão
de Madalena}, \textit{A sorte de uma meretriz}, \textit{Proezas de João
Grilo}, \textit{O lobo do oceano}, \textit{História de Natanael e Cecília}, \textit{A
rainha que saiu do mar}, \textit{O retirante}, \textit{Romance do escravo
grego}, \textit{O Recife novo}, \textit{O prisioneiro do castelo da rocha negra},
\textit{Proezas de Lampião na cidade de Cajazeiras}, \textit{A garça
encantada}, \textit{Romance do príncipe que veio ao mundo sem ter
nascido}, \textit{Rachel e a fera encantada}, \textit{O segredo da princesa},
\textit{Discussão de um crioulo com um padre}, \textit{História de um
pescador}, \textit{O lobo do oceano}, \textit{A menina perdida}, \textit{O monstro
do Rio Negro}, \textit{O romance de um sentenciado}, \textit{Um amor
impossível}, \textit{A dama das camélias}, \textit{História de Paulo e Maria},
\textit{O dia de juízo}, \textit{Discussão de José Duda com João Athayde},
\textit{A grande surra que levou Cordeiro Manso de João Athayde
por desafiá-lo}, \textit{A chegada de Lampião e Maria Bonita a
Maceió e Corisco vingando o chefe}, \textit{Meia-noite no cabaré}, \textit{O
toureiro de Umbuzeiro ou O curandeiro misterioso}, \textit{O
namoro de um cego com uma melindrosa da atualidade}, \textit{A
princesa sem coração}, \textit{A grande batalha no reino da bicharia},
\textit{O homem do pulso de ferro}, \textit{A vida e os novos sermões do
padre Cícero}, \textit{Peleja de João Athayde com José Ferreira
Lima}, \textit{A pérola sagrada}, \textit{O primeiro debate de Patrício com
Inácio da Catingueira}, \textit{Mabel ou Lágrimas de mãe}, \textit{O poder 
oculto da mulher bonita}, \textit{João Batista Lusitano
desmascarado na sua mentirosa profecia do ano de 17},
\textit{História de Roberto do Diabo}, \textit{História de um homem que
teve uma questão com Santo Antônio}, \textit{O prêmio do sacrifício
ou Os sofrimentos de Lindoia}, \textit{Peleja de Ventania com Pedra
Azul}, \textit{Quatro poetas glosados: Ugulino, Romano, Nogueira
e o velho Mufumbão}, \textit{Peleja de Serrador e Carneiro}, \textit{O
homem que nasceu para não ter nada}, \textit{Martelo de José Duda
e Joaquim Francisco em Itabaiana}, \textit{Germano e Mufumbão},
\textit{Peleja de Laurindo Gato com Marcelino Cobra Verde},
\textit{História da princesa Elisa}, \textit{Peleja de Bernardo Nogueira
com Preto Limão}, \textit{História da moça que foi enterrada viva
ou a infeliz Sofia}, \textit{Peleja de Manoel Raymundo com Manoel
Campina}, \textit{História de Dimas -- O bom ladrão}, \textit{A guerra
dos animais}, \textit{Peleja de Antônio Machado com Manoel
Gavião}, \textit{Peleja de Patrício com Inácio da Catingueira}, \textit{O
prêmio da inocência}, \textit{A infelicidade de dois amantes}, \textit{História
do valente Vilela}, \textit{O efeito da passagem do eclipse total do
Sol e o alarme dos que não tinham visto o fenômeno},
\textit{Discussão de João Athayde com Leandro Gomes}, \textit{Alzira --
A morta viva}, \textit{O casamento do bode com a raposa}, \textit{História
de um rico avarento}, \textit{Fugida da princesa Beatriz com o conde
Pierre}, \textit{História da escrava Guiomar}, \textit{História de Joãozinho
e Mariquinha}, \textit{Décimas amorosas}, \textit{O bataclan moderno}, \textit{A
filha do bandoleiro}, \textit{O casamento do calango}, \textit{História de
Balduíno e o estudante que se vendeu ao diabo}, \textit{História do
menino da floresta}, \textit{As felicidades que oferece o casamento},
\textit{O balcão do destino ou a menina da ilha}, \textit{A filha das selvas},
\textit{O azar na casa do funileiro}, \textit{A desventura de um analfabeto
ou o homem que nunca aprendeu a ler}, \textit{Em homenagem às
mulheres}, \textit{O fantasma do castelo}, \textit{A condessinha roubada},
\textit{Discussão de João Athayde com João de Lima}, \textit{A fada e o
guerreiro}, \textit{Discussão de João Athayde com Mota Júnior},
\textit{Discussão de um operário com um doutor}, \textit{Doutor Caganeira},
\textit{A entrada de padre Cícero no céu vista por uma donzela de
13 anos}, \textit{Uma festa no sertão} e \textit{História do negrão André
Cascadura}.

Como escreve Souto Maior:

\begin{quote}
Como se vê, João Martins de Athayde não era um poeta
cuja temática fosse o sobrenatural, apesar de alguns de seus
folhetos enfocarem o céu, padre Cícero, o Diabo ou o
inferno. Não era, também, o poeta do circunstancial, de
fazer um jornalismo paralelo, como José Costa Leite, o
poeta-repórter. Era, sim, um poeta voltado para o amor,
para a aventura, para o grotesco, para o mundo da
imaginação.\footnote{\textsc{athayde}, João Martins. \textit{Cordel na escola}. São Paulo: hedra, 2000, p.\,20.}
\end{quote}

\subsection{Síntese dos poemas}

\paragraph{``Como Lampião entrou na cidade de
Juazeiro...''}

Na cidade de Juazeiro, Lampião entra com mais cinquenta cangaceiros. Diante da
presença ilustre, toda a cidade se preocupa. Como ele mantém"-se ali e as
autoridades nada fazem? Em entrevista a um repórter, Virgulino fala de
seu respeito a Padre Cícero, e que havia ajudado o tenente Chagas
contra um grupo revoltado, em Cipó de Pernambuco. Por conta da ajuda ao
exército e de sua devoção ao Padrinho, Lampião, para o desgosto da
polícia local, tem um salvo"-conduto na cidade, contanto que, em respeito
ao líder religioso, mantenha"-se desarmado. 

\paragraph{``A sorte de uma meretriz''}

Aulina, jovem de rara beleza e filha de um rico fazendeiro, para
desgosto do pai resolve se transformar em prostituta. Com muitos
amantes poderosos, ela é tratada com muitos mimos, ganha uma bela casa,
muitas roupas e joias. Por conta de seu destino cai doente, e é
abandonada por todos, inclusive seus amantes. Numa espécie de
penitência pelo seu comportamento, transforma"-se em miserável e morre
na rua. 

\paragraph{``A chegada de João Pessoa no céu''}

Narrativa que parte de um acontecimento real, o assassinato do
governador da Paraíba, João Pessoa, no Recife. Conta a chegada dessa
personagem no céu. Espécie de canto heroico, narra o julgamento, realizado por Deus, desse
homem, que auxiliado pela consciência, a
honestidade, a honra e o dever leva"-o para o reino dos céus. O poema
termina com a chegada dos assassinos do político, que recebem um
julgamento sumário e vão direto ao inferno. 

\paragraph{``A moça que foi enterrada viva''}

História de Sofia, filha de um fazendeiro muito poderoso e temido nos
sertões de Teresina, que não pode se casar, pois seu pai acredita que
não exista no mundo um homem suficientemente valente para ela. Certo
dia, um rapaz decide que vai casar"-se com ela, para tanto esconde"-se em
seu quarto à noite e, na manhã seguinte, na mesa do café da manhã,
demonstrando sua valentia, vai até o pai da moça e diz que vai se casar
com ela e que o fato já havia sido consumado. O fazendeiro
supostamente admirado com o rapaz diz que ele volte em quinze dias para o
casamento. Irado com o acontecido, o pai manda os irmãos da moça cavarem
um cova e a enterra viva. Por sorte, o noivo volta no dia seguinte e
percebendo o ocorrido, mata o pai, fere alguns dos irmãos e consegue
retirar ainda com vida a noiva da cova. Os outros irmãos acabam por
perseguir o casal, matando ambos e os retalhando. O poema termina com o
aviso moral de que aqueles que desafiam os pais tem um fim nefasto.

\paragraph{``As quatro classes corajosas''}

Poema de louvação às quatro profissões que o autor considera
indubitavelmente dignas. Os vaqueiros, que com destreza enfrentam o
sertão e os touros; os agricultores, que tudo suportam e alimentam o
mundo; os soldados, que trocam suas vidas para defender os outros; e os
pescadores, que com grande coragem enfrentam o mar e foram escolhidos
por Cristo, já que muitos dos apóstolos eram pescadores. 

\paragraph{``Discussão de José Duda com João
Athayde''}

Espécie de disputa entre cordelistas, entretanto o poema os apresenta
como amigos que há algum tempo não se veem. Tendo início como uma
espécie de exercício de ampliação de repertório. A disputa descreve
todo o abecedário de nomes de homens, partindo para os nomes das aves e
em seguida dos peixes. Tendo fim na afirmação de Athayde de que apesar
de saber que seu desafiante pode tratar de diversos assuntos, já está
tarde e o autor tem um compromisso a cumprir.

\paragraph{``Em homenagem às mulheres''}

Elegia à figura feminina, que parte de seus atributos físicos e de
comparações entre ela e a natureza. Separando as mulheres entre as que
ainda são virgens, e portanto ainda não desabrocharam, e as demais, o
autor, procura louvar os atributos físicos e sua capacidade de cuidar
do lar, fazendo uma ressalva de que naquele momento já havia um início
de equidade entre os homens e as mulheres. O autor termina por afirmar
que seu poema é apenas mais uma tentativa de descrever algo, que na
realidade é perfeito. 

\paragraph{``O retirante''}

Família que durante o período da seca é obrigada a migrar em busca de
trabalho. Apresentando as dificuldades dessa região do Nordeste, o
autor narra a chegada da família a uma fazenda para pedir trabalho, as
dificuldades na relação entre os retirantes e a polícia e a
dificuldade dessa situação, que se agrava cada vez mais. 

\section{Sobre o gênero}
%\section{Poesia de cordel: oralidade e escuta coletiva}

Para uma primeira definição de poesia enquanto gênero literário, poder"-se"-ia recorrer à definição do professor Domingos Paschoal Cegalla, para quem ``poesia é a linguagem subjetiva, carregada de emoção e sentimento, com ritmo melódico constante, bela e indefinível como o mundo interior do poeta visa a um efeito estético''.\footnote{\textsc{cegalla}, Domingos Paschoal. \textit{Novíssima Gramática da Língua Portuguesa}. São Paulo: Companhia Editora Nacional, 2008, p.\,640}

Aprofundando um pouco essa definição, o crítico Antonio Candido expande a definição de poesia ao diferenciá"-la do verso.
Para o crítico, a poesia enquanto ato criador do artista independe da forma métrica do verso, que passa a ser apenas um dos registros possíveis do poético:

\begin{quote}
A poesia não se confunde necessariamente com o verso, muito menos com o verso metrificado. Pode haver poesia em prosa e poesia em verso livre. [\ldots]
Pode ser feita em verso muita coisa que não é poesia.\footnote{\textsc{candido}, Antonio. \textit{O estudo analítico do poema}. São Paulo: Terceira leitura, 1993, p.\,13--14.}
\end{quote}

Delineada, de forma breve e geral, a forma poética, pode"-se pensar agora em seus três gêneros básicos: lírico, épico e dramático.
Para o crítico Anatol Rosenfeld, a lírica é o gênero mais subjetivo, no qual uma voz central exprime um estado de alma traduzido em orações poéticas.
Seria a expressão de emoções e experiências vividas, ``a plasmação imediata das vivências intensas de um Eu no encontro com o mundo, sem que se interponham eventos distendidos no tempo (como na Épica e na Dramática)''.\footnote{\textsc{rosenfeld}, Anatol. \textit{O teatro épico}. São Paulo: Perspectiva, 2006, p.\,22.}

Devido a essa característica central da lírica, a expressão de um estado emocional, Rosenfeld considera que o eu"-lírico, nesse gênero, não se delineia enquanto um personagem. Embora possa evocar personagens e narrar acontecimentos, a lírica entendida enquanto gênero puro afasta"-se sobremaneira da apreensão objetiva do mundo, que não existe independente da subjetividade intensa que o apreende e exprime. Assim, na lírica prevalece a fusão entre o sujeito e o objeto, que serve mais a realçar os estados profundos de alma do poeta.
Sobre os aspectos formais do gênero, Rosenfeld nota:

\begin{quote}
À intensidade expressiva, à concentração e ao caráter ``'imediato'' do poema lírico, associa"-se, como traço estilístico importante, o uso do ritmo e da musicalidade das palavras e dos versos. De tal modo se realça o valor da aura conotativa do verbo que este muitas vezes chega a ter uma função mais sonora que lógico"-denotativa. A isso se liga a preponderância da voz do presente que indica a ausência de distância, geralmente associada ao pretérito. Este caráter do imediato, que se manifesta na voz do presente, não é, porém, o de uma atualidade que se processa e distende através do tempo (como na Dramática) mas de um momento ``eterno''.\footnote{Ibidem, p.\,23.}
\end{quote}

No caso específico da poesia de cordel, dizem os especialistas, é uma poesia escrita para
ser lida, enquanto o repente ou o desafio é a poesia feita oralmente,
que mais tarde pode ser registrada por escrito. Essa divisão é muito
esquemática. Por exemplo, o cordel, mesmo sendo escrito e impresso para
ser lido, costumava ser lido em voz alta e desfrutado por outros
ouvintes além do leitor. A poesia popular, praticada principalmente no
Nordeste do Brasil, tem muita influência da linguagem oral, aproveita
muito da língua coloquial praticada nas ruas e na comunicação
cotidiana. 

Naturalmente, portanto, pode"-se considerar a poesia narrativa do cordel
uma forma de poesia mais compartilhada e desfrutada coletivamente, o
que lhe dá também uma grande ressonância social. Muitos dos temas do
cordel são originários das tradições populares e eruditas da Europa
medieval e moderna. Encontramos temas retirados das novelas de
cavalaria medievais e das narrativas bíblicas. Ao lado destes temas
mais literários, encontram"-se os temas locais, quase sempre narrados na
forma de crônicas de coisas realmente acontecidas, ou que apresentam
uma moral do sertão, como ``A moça que foi enterrada
viva'' e ``A sorte de uma
meretriz'', e as chamadas reportagens jornalísticas,
no caso ``Como Lampião entrou na cidade de
Juazeiro''. Também há as histórias fantásticas, que
se valem das tradições semirreligiosas, ligadas à experiência com o
mundo espiritual, perceptível em ``A chegada de João
Pessoa ao céu'', que também é um poema de exaltação
política do período varguista. 

Os grandes poemas de cordel são perfeitamente metrificados e rimados. A
métrica e a rima são recursos que favorecem a memorização e
tradicionalmente se costuma dizer que são resquícios de uma cultura
oral, na qual toda a tradição e sabedoria são sabidas de cor.

\subsection{O sertão geográfico e cultural}

O sertão tem mitos culturais próprios. Contemporaneamente, o sertão
evoca principalmente o sofrimento resignado daqueles que padecem a
falta de chuva e de boas safras na lavoura. Evoca a experiência
histórica de uma região empobrecida, embora tenha sido geradora de
riquezas, como o cacau e a cana"-de"-açúcar, ambos bens muito valiosos. 

O sertão formou também o seu imaginário por meio de grandes
personalidades e uma pujante expressão artística. Além do cordel, o
sertão viu nascer ritmos tão importantes quanto o forró e o baião.
Produziu artistas tão expressivos quanto Luiz Gonzaga, grande cantor da
vida do sertanejo em canções como ``Asa
branca''. Um escultor como Mestre Vitalino criou toda
uma tradição de representação da vida e dos hábitos sertanejos em
miniaturas de barro. A gravura popular, que sempre acompanha os
folhetos de cordel, também floresceu em diversos pontos e ficou mais
famosa em Juazeiro do Norte, no Ceará, e em Caruaru, no estado de
Pernambuco. 

Dentre os grande mitos do sertão, está certamente o do cangaço com seu
líder histórico, mas também mítico, Virgulino Ferreira, o Lampião. Até
hoje as opiniões se dividem: para alguns foi um grande homem, para
outros um bandido impiedoso. 

Uma figura muito presente na cultura nordestina é o Padre Cícero Romão,
considerado beato pela Igreja Católica. Consta que teria feito milagres
e dedicado sua vida aos pobres. 

\subsection{Variação linguística}

A linguística moderna usa o termo
``idioleto'' para marcar grupos
distintos no interior de uma língua. Um idioleto pode ser a fala
peculiar de uma região, de um grupo étnico ou de uma dada profissão. 

Uma das grandes forças da poesia popular do Nordeste se origina em sua
forma muito própria de falar, com um ritmo muito diferente dos falares
do sul, e também muito diferentes entre si, pois percebe"-se a diferença
entre os falares de um baiano, um cearense e um pernambucano, por
exemplo.

Além desse aspecto rítmico, quase sempre também há palavras peculiares a
certas regiões. 

%\pagebreak
%\section{Sugestões de atividades}
%\begin{enumerate}
%
%\item \textit{Atividade de leitura}. Esta atividade tem por objetivo sensibilizar os
%alunos para a escuta de poesia. O professor deve ler um conjunto de
%estrofes para exemplificar uma leitura que se construa com uma
%pronúncia clara, pausas e ênfases adequadas. Após isso, cada aluno deve
%ler uma estrofe, procurando marcar o ritmo e as rimas, bem como as
%pausas e ênfases expressivas. O trabalho de leitura pode auxiliar o
%professor na realização de um diagnóstico dos alunos, em relação à
%pontuação e ao ritmo do texto, além de possibilitar um desenvolvimento
%da percepção da voz e da fala como meios indispensáveis à boa
%convivência social.
%
%\item Nos poemas ``A sorte de uma
%meretriz'' e ``A moça que foi
%enterrada viva'', Athayde narra dois acontecimentos
%que terminam por apresentar uma moral. Essa moral relaciona-se a
%uma perspectiva religiosa e a um momento histórico das relações
%sociais no Brasil, associado ao machismo e à figura do coronel. A
%partir da leitura dos poemas o professor pode pedir para que os alunos
%identifiquem no texto as passagens em que o autor explicita seu ponto
%de vista sobre a condição de meretriz e de filha desobediente. Com os
%trechos selecionados, é possível conduzir uma discussão sobre o
%papel da mulher em uma sociedade que exalta a moral
%cristã como um valor rígido. 
%
%\item No poema ``A chegada de João Pessoa no
%céu'', Athayde apresenta o momento em que esse
%político paraibano chega ao céu depois de seu assassinato na cidade do
%Recife. A partir da leitura do poema o professor pode pedir aos alunos
%que caracterizem a figura do político. Em seguida, é possível trabalhar
%a importância desse acontecimento como justificativa para a chamada
%Revolução de 1930, que colocou Getúlio Vargas no governo. Por fim,
%pode-se refletir sobre como os valores defendidos por Vargas,
%principalmente a ideia do populismo, aparecem no momento do julgamento
%de João Pessoa, em especial, na ideia de um homem humilde que precisa
%cumprir seu dever, pois é aclamado pelo povo. Como pode ser visto no
%trecho abaixo:
%
%\begin{verse}
%
%
%Contra meus gestos humildes\\*
%Esse povo me aclamou\\*
%E as rédeas do governo\\*
%Nas minha mãos entregou;\\*
%Eu com o povo governei\\*
%Do povo não me afastei\\*
%Se errei o povo apoiou.
%
%\end{verse}
%
%\item Ainda em ``A chegada de João Pessoa ao
%céu'', o autor utiliza-se de uma estratégia de
%linguagem que transforma valores morais em personagens. Dessa forma, o
%ódio, a honra, o dever e a justiça são apresentados como personagens do
%julgamento de João Pessoa realizado por Deus. A partir da leitura do
%poema o professor pode pedir aos alunos que separem cada um desses
%valores em dois grupos, os favoráveis à vida celeste e os favoráveis à
%vida no inferno. Em seguida, o professor pode discutir como esse é um
%artifício que provém da tradição medieval, já que nas novelas de
%cavalaria os valores cavalheirescos estão diretamente ligados à
%devoção dos cavaleiros à religião católica. Como disparador dessa
%discussão o professor pode retirar uma passagem das histórias do Rei
%Arthur, em especial o momento em que Percival é caracterizado.
%
%\item Em ``Discussão de José Duda e João
%Athayde'', o autor apresenta uma disputa entre dois
%cordelistas, entretanto essa peleja é realizada através dos nomes de
%homens, pássaros e peixes. Sendo assim, as estrofes divididas em
%sextilhas utilizam-se da assonância, isto é, a semelhança de sons vogais,
%para construir as rimas. A partir da leitura do poema, o professor pode
%pedir aos alunos que elaborem uma lista de nomes com uma mesma letra,
%distintos daqueles apresentados no poema, e realizem um exercício de
%escrita que respeite o formato das sextilhas e que possua um esquema
%métrico de rimas. Para isso, é importante que o professor transcreva
%algumas das estrofes e apresente aos alunos, como as rimas são
%opostas, como no exemplo:
%
%\begin{verse}
%
%Z -- Com o H inda soletro\\*
%Heveu, Henrique e Hermano\\*
%Heriseu, Hermenerico\\*
%Helianeu, Hotulano,\\*
%Herundino, Heleodore,\\*
%Hermenegildo, Herculano
%
%\end{verse}
%
%\item No poema ``Em homenagem às
%mulheres'' o autor formula metáforas que
%apresentam as características femininas. De maneira geral, essas
%metáforas provém de imagens retiradas da natureza, como no Romantismo.
%A partir da leitura do poema, o professor pode pedir aos alunos que
%retirem do texto ao menos três dessas figuras de linguagem. Em seguida,
%a partir dessas metáforas, o professor pode refletir com os alunos como
%todas as características ressaltadas pelo texto são atributos físicos.
%Para concluir a discussão, o professor pode pedir aos
%alunos que comparem o papel da mulher hoje com o papel atribuído à
%mulher pelo autor do poema. 
%
%\item Em ``O retirante'', Athayde
%apresenta o drama de uma família de trabalhadores rurais que é obrigada
%a abandonar sua terra por causa da seca. Apesar de narrar a história de
%uma família, em especial de um homem, essa é uma temática ampla que
%engloba a própria condição social do nordestino, que devido à escassez
%provocada pelas condições climáticas e geográficas é sempre
%obrigado a migrar. A partir da leitura do poema, o professor pode pedir
%aos alunos que retirem extratos do texto que caracterizem essa condição
%climático-geográfica. Em seguida, o professor pode refletir sobre como
%as condições naturais dessa região contribuem para as condições
%econômico-sociais de seus habitantes. Entretanto, é importante
%mostrar como as desigualdades sociais, em especial as ligadas à
%terra, são um importante ponto de inflexão para a percepção de que essa
%não é uma questão restrita aos aspectos geográficos do sertão. Por fim,
%o professor pode trabalhar com os alunos a maneira como o autor
%justifica a condição do retirante ao fim do poema. Para incrementar a
%discussão, pode-se apresentar aos alunos o capítulo ``Fuga'', do romance \textit{%Vidas secas} (1938),
%de Graciliano Ramos, que possui diversos aspectos que podem contribuir e
%tensionar a perspectiva apresentada por Athayde.
%
%\item Ainda sobre o poema ``O
%retirante'', o professor pode pedir para que os
%alunos busquem no poema as situações que explicitam as dificuldades do
%retirante. Em seguida, como possibilidade de realização de uma leitura
%comparada, o professor pode trazer trechos do capítulo ``O soldado
%amarelo'', do romance \textit{Vidas secas} (1938), de Graciliano Ramos, para %discutir com
%os alunos como a figura da autoridade policial é controversa no sertão, pois,
%longe de auxiliar a população, ela se transforma, algumas vezes, em inimiga. 

%\end{enumerate}
%\pagebreak
%\section{Sugestões de leitura para o professor} 

\begin{bibliohedra}

\tit{DIEGUES JÚNIOR}, Daniel. \textit{Literatura popular em verso}. Estudos. Belo Horizonte: Itatiaia, 1986. 

\tit{MARCO}, Haurélio. \textit{Breve história da literatura de cordel}. São Paulo: Claridade, 2010.

\tit{TAVARES}, Braulio. \textit{Contando histórias em versos. Poesia e romanceiro popular no Brasil}. São Paulo: 34, 2005.

\tit{TAVARES}, Braulio. \textit{Os martelos de trupizupe}. Natal: Edições Engenho de Arte, 2004 

\end{bibliohedra}

%\paginabranca
%\paginabranca
%\paginabranca
%\paginabranca
%\paginabranca
%\paginabranca
%\paginabranca
%\paginabranca
