\chapter{Apresentação}

Nascido a 23 de junho de 1877, na então vila Cachoeira de Cebola, no município de Ingá, na Paraíba, João Martins de Athayde é considerado o príncipe dos poetas populares do Norte do Brasil. Um dos mais importantes cordelistas brasileiros, Athayde tornou"-se um verdadeiro ídolo popular. Não apenas da gente pobre e humilde, semi"-alfabetizada e mesmo analfabeta, do interior, mas da gente remediada e rica das zonas urbanas, capitais e cidades importantes, entre elas Salvador, Recife e Fortaleza.

A qualidade excepcional de sua poesia atesta o motivo de tal popularidade.
Para apresentá"-la ao jovem leitor, foram reunidos nessa coletânea oito poemas dos mais representativos de sua extensa obra. São poemas que perpassam os temas de predileção do poeta e são típicos da poesia de cordel e da vida no sertão: a moral religiosa, as lendas cristãs, as personalidades importantes para o seu povo, a disputa entre cordelistas, a condição do retirante na seca e o lugar social dos vaqueiros, pesqueiros e agricultores no sertão.

Ao longo desses poemas, vemos figuras históricas serem evocadas, como Lampião e seu bando de cangaceiros e o político paraibano João Pessoa.
Em ``A chegada de João Pessoa no céu'', Athayde apresenta de forma criativa o momento em que o político chega ao céu depois de seu assassinato na cidade do
Recife. No seu julgamento celestial, observa"-se como o poema também adquire um caráter político ao incorporar valores defendidos por Vargas, como a ideia de que um homem humilde precisa cumprir seu dever, pois é aclamado pelo povo. 
O poema também emprega figuras de linguagem que transformam valores morais em personagens, como o ódio, a honra, o dever e a justiça, que vão participar do julgamento de João Pessoa.
Ressalta"-se esse aspecto do poema pois é um
artifício que provém da tradição medieval --- nas novelas de
cavalaria os valores cavalheirescos estão diretamente ligados à
devoção dos cavaleiros à religião católica ---, evidenciando a ligação do poeta com as tradições populares do Nordeste que, como demonstra a pesquisa armorial, tem muita relação com a cultura popular medieval.

Em outros poemas, como ``A sorte de uma meretriz'' e ``A moça que foi
enterrada viva'', Athayde narra dois acontecimentos
que terminam por apresentar uma moral, relacionada a
uma perspectiva religiosa e a um momento histórico das relações
sociais no Brasil, associado ao machismo e à figura do coronel.
Outro poema da coletânea, ``Em homenagem às
mulheres'', também tem como protagonista uma mulher, que é cantada através de metáforas naturais pelas ``características femininas'' que o poeta nela observa.

Além da aproximação pela moral e pela relação com a mulher, também chegamos ao universo cultural do cordelista através de um poema como ``Discussão de José Duda e João
Athayde'', no qual o autor apresenta a clássica disputa entre dois
cordelistas, mas com uma inovação: a peleja é realizada através dos nomes de
homens, pássaros e peixes, o que leva o poeta, para manter a divisão das estrofes em
sextilhas, a utilizar da assonância, isto é, a semelhança de sons vogais,
para construir as rimas. 

Em ``As quatro classes corajosas'' o universo do autor também se faz presente com a louvação das que considera as profissões essenciais: vaqueiro, agricultor, soldado
e pescador. Mas é no último poema do livro, ``O retirante'', que vemos de forma mais intensa um dos grandes, e tristes, aspectos da cultura nordestina: o drama de uma família de trabalhadores rurais que é obrigada a abandonar sua terra por causa da seca.
Apesar de narrar a história de uma família, em especial de um homem, essa é uma temática ampla que engloba a própria condição social do nordestino, que devido à escassez
provocada pelas condições climáticas e geográficas é sempre
obrigado a migrar. 

Como no caso de ``O retirante'', os poemas aqui reunidos trazem temas fundamentais à tradição do cordel e à vida nordestina que, apesar da particularidade dos poemas, apontam para questões coletivas e gerais, que tangem as diferentes dimensões da vida e da cultura no Nordeste. Não obstante sua filiação a determinados motivos e imagens de seu meio, 
João Martins de Athayde soube imprimir a cada poema traços característicos e inventivos, singularizando a poesia desse que é um dos maiores poetas populares de todos os tempos.


\part{\textsc{Tipógrafo, poeta e editor}}

\chapter[Como Lampião entrou na cidade
de Juazeiro]{Como Lampião entrou na cidade
de Juazeiro acompanhado de
cinquenta cangaceiros e como
ofereceu os seus serviços à
legalidade contra os revoltosos}

\begin{verse}

O dia doze de março\\*
Foi alegre, alvissareiro,\\*
Porém para o sertanejo\\*
Tornou-se quase agoureiro,\\*
A polícia protestou\\*
Quando Lampião entrou\\*
Na cidade de Juazeiro

Cerca de cinquenta homens\\*
Cada qual mais bem armado\\*
Trajando roupa de cáqui\\*
Tudo bem municiado\\*
Desde o mais velho ao mais moço\\*
Tinha um lenço no pescoço\\*
Preso num laço amarrado

Compunha-se o armamento\\*
De fuzil, rifle e punhal\\*
Cartucheira na cintura\\*
Medonha e descomunal\\*
Conduzindo muitas balas\\*
Ninguém podia contá-las:\\*
Dizia assim o jornal

A maioria dos homens\\*
Que compõe-se o pelotão\\*
São filhos de Pernambuco\\*
Bem do centro do sertão\\*
Os de Pageú de Flores\\*
que também são defensores\\*
Do valente Lampião

Um deles chama-se Gaio\\*
Dizem que luta com três\\*
Pela fortidão parece\\*
Ser filho de holandês\\*
Diz que tem vinte e dois anos\\*
E é um dos pernambucanos\\*
Que nunca entrou no xadrez

Tem só dois homens casados\\*
No grupo de Lampião\\*
O mais é tudo solteiro\\*
Mostrando satisfação,\\*
Se entram em certo perigo\\*
Porém faz muitos amigos\\*
Por todo aquele sertão

De todos não tem um só\\*
Que se mostre arrependido\\*
Embora que da polícia\\*
Vivem sempre perseguidos\\*
Dizia assim um gaiato:\\*
Quem se confia no mato\\*
Vive sempre garantido

Chumbinho é um dos melhores\\*
Do grupo Ode Lampião\\*
Tem vinte anos de idade\\*
E com boa disposição,\\*
Só se vê ele contente\\*
Narrando constantemente\\*
Seus feitos pelo sertão

Causou admiração\\*
Ao povo do Juazeiro\\*
Quando Lampião entrou\\*
Mansinho como um cordeiro,\\*
Com toda sua regência\\*
Que lhe rende obediência\\*
Por ser leal companheiro

Em Juazeiro hospedou-se\\*
Em casa de um seu irmão\\*
Aglomerava-se o povo\\*
Todo em uma multidão,\\*
Dizendo: ``Não está direito\\*
Só vou daqui satisfeito\\*
Quando olhar pra Lampião''

De toda parte chegava\\*
Gente para o Juazeiro\\*
Alguns deles se vestiam\\*
Com as roupas d'um romeiro,\\*
Quem morava no deserto\\*
Vinha pra ver bem de perto\\*
O famoso cangaceiro

Um repórter da gazeta\\*
Com Lampião quis falar\\*
No meio da multidão\\*
Quase não pôde passar\\*
Machucando muita gente\\*
Pôde ele finalmente\\*
Com Lampião conversar

Ali se complementaram,\\*
E começou o jornalista\\*
Da vida de Lampião\\*
Saber por uma entrevista,\\*
Narrou tintim por tintim\\*
Do princípio até o fim\\*
Sem nada perder de vista

Começou logo a conversa\\*
De uma forma animada\\*
Lampião tinha a linguagem\\*
Muito desembaraçada,\\*
Mostrando sua importância\\*
Falando com arrogância\\*
Como quem não via nada

o repórter na conversa\\*
Prestava toda atenção\\*
Gravou na mente o retrato\\*
Bem fiel de Lampião,\\*
O seu perfil natural\\*
De um modo original\\*
Com a maior perfeição

Estatura mediana\\*
O corpo bem comedido\\*
O rosto bastante oval\\*
E queixo muito comprido\\*
Eis os traços principais\\*
Deste que entre os mortais\\*
Tornou-se tão conhecido

Ele traz o seu cabelo\\*
Americano cortado\\*
Traz a nuca descoberta\\*
Usa o pescoço raspado,\\*
Os dedos cheios de anéis\\*
Boa alpercata nos pés\\*
Pra lhe ajudar no serrado

Tinha a calça de bom pano\\*
Paletó de brim escuro\\*
No pescoço um lenço verde\\*
De xadrez e bem seguro\\*
Por um anel de brilhante\\*
Que se via faiscante\\*
Por ter um metal mais puro

Usava óculos também\\*
Pra encobrir um defeito\\*
Moléstia que Lampião\\*
Sofre no olho direito,\\*
Mesmo assim enxerga tudo\\*
Pois no sertão tem estudo\\*
Faz o que quer a seu jeito

o tempo que Lampião\\*
Com o repórter conversou\\*
Conversa que certamente\\*
Mais d'uma hora durou\\*
Conservou-se muito sério\\*
Mostrava com todo império\\*
A fama que conquistou

Não desprezou um momento\\*
Seu mosquetão de Vitória\\*
Aquela sinistra lenda\\*
Pra ele tem fama e glória,\\*
Se julga reconhecido\\*
Que seu dever tem cumprido\\*
Pra ter nome na história

Num tamborete sentado\\*
Lampião só respondia\\*
Às perguntas que o repórter\\*
Com assento lhe fazia,\\*
Sempre de arma na mão\\*
Prestando muita atenção\\*
Ao movimento que havia

Assim naquela atitude\\*
Rosto firme, olhar insano\\*
Quem o visse não dizia\\*
Ser um ente desumano,\\*
Prestava atenção a tudo\\*
Com um caráter sisudo\\*
Parecia um soberano

Suas armas pesam muito\\*
Porém Lampião não sente\\*
Mais de quatrocentas balas\\*
Carrega sobressalentes,\\*
Às vezes dói-lhe o espinhaço\\*
Porque o grande cangaço\\*
Empina ele pra frente

O repórter perguntou\\*
A Lampião sua idade\\*
Tenho vinte e sete anos\\*
Com toda serenidade,\\*
Sinto-me bastante forte\\*
Não tenho medo da morte\\*
Nem fujo da autoridade

Há dez anos me ajuntei\\*
Com o grupo do Pereira\\*
Inda não tive vontade\\*
De abandonar a carreira,\\*
Me dei bem com o negócio\\*
Ainda encontrando um sócio\\*
Não vou fazer esta asneira

Porém quando eu deixar\\*
Esta predileta arte\\*
Da melhor sociedade\\*
Eu tenho de fazer parte,\\*
Aí ninguém mais protesta\\*
Vivo numa vida honesta\\*
Sem usar do bacamarte

o repórter perguntou\\*
Se ele não se comovia\\*
Com os assaltos às fazendas\\*
Usando da tirania\\*
Na propriedade alheia\\*
Sempre de algibeira cheia\\*
Pelos roubos que fazia

Lampião lhe respondeu:\\*
``Não fiz mal a esta gente\\*
Se acaso peço dinheiro\\*
É muito amigavelmente''\\*
No meio da entrevista\\*
Lampião ergueu a vista\\*
De lado viu um tenente

Lampião disse ao repórter\\*
Se acaso eu for derrotado\\*
O meu irmão fica aí\\*
No meu lugar colocado,\\*
A casa está definida\\*
Quem vier tirar-me a vida\\*
Diga que está desgraçado

Ali chegou uma velha\\*
Com uma imagem na mão\\*
O repórter e mais alguém\\*
Prestaram toda atenção,\\*
Disse a velha paciente:\\*
``Eu trago aqui um presente\\*
Pro \textit{coroné} Lampião''.

Disse a velha: ``Aqui eu trago\\*
Remédio pra sua dor\\*
Guarde consigo esta imagem\\*
E tenha fé no Criador,\\*
Pelo poder do Messias\\*
Inda brigando dez dias\\*
Bala não fere o senhor''

Recebeu ele a imagem\\*
Da forma que lhe convinha\\*
Acreditando o milagre\\*
Que a velha disse que tinha,\\*
Pegou um dos seus anéis\\*
E mais um conto de réis\\*
Botou na mão da velhinha

Terminada a entrevista\\*
Falou assim Lampião\\*
Disse para o jornalista:\\*
``Me ofereça um cartão\\*
Seja"bom para comigo\\*
Escreva lá um artigo\\*
Pra ver se eu tenho perdão''

O povo do Juazeiro\\*
Todos queriam saber\\*
Ali naquela cidade\\*
Lampião que foi fazer\\*
De fato, a sua presença\\*
Produziu a mais imensa\\*
Dúvida que se pode ter

Dizia o jornal que ele\\*
Andava assim na cidade\\*
Na terra do padre Cícero\\*
Gozando da liberdade,\\*
É porque foi confirmado\\*
Que ele tinha prestado\\*
Serviço à legalidade

Em Cipó de Pernambuco\\*
Estava um combate travado\\*
Por contingentes legais\\*
Com um grupo revoltado,\\*
Se Lampião não chegasse\\*
Que aos legais ajudasse\\*
Tudo estava derrotado

De um batalhão patriota\\*
Da primeira companhia\\*
Do senhor tenente Chagas\\*
Por certo se acabaria,\\*
Se não fosse Lampião\\*
Que se meteu na questão\\*
Até o chefe morria

o combate foi renhido\\*
Foi uma luta de glória\\*
Uma espada da briosa\\*
É o facho da Vitória,\\*
Que Lampião apresenta\\*
Dizendo: ``Esta ferramenta\\*
Leva meu nome à história''

Bastante reconhecido\\*
Com este feito guerreiro\\*
O mesmo tenente Chagas,\\*
Como amigo verdadeiro\\*
Trouxe Lampião contente\\*
E entrou com ele à frente\\*
Nas portas do Juazeiro

Foi por sua conta e risco\\*
Que no Juazeiro entrou\\*
Na frente do padre Cícero\\*
Tenente Chagas provou,\\*
Deixando o povo ciente\\*
Que todo seu contigente\\*
Foi Lampião que salvou

Da polícia em Juazeiro\\*
Houve grande oposição\\*
Porque queriam prender\\*
O famoso Lampião,\\*
Não puderam conseguir\\*
Porque precisavam ouvir\\*
O padre Cícero Romão

Em menos de meia hora\\*
Juntou-se uma comissão\\*
Foram conferenciar\\*
Com o padre Cícero Romão,\\*
Temendo alguma censura\\*
Foram exígir a cultura\\*
Do povo de Lampião

Disse o padre: ``Nesse ponto\\*
Eu nada tenho a dizer\\*
Falsidade àquele homem\\*
Também não posso fazer\\*
Como é que eu vou maltratar\\*
Quem ajudou a livrar\\*
Nosso povo de morrer?''

Todos olham bem pra ele\\*
Com muito ódio e rancor\\*
``Eu sou chefe da igreja\\*
Dei provas de bom pastor,\\*
Não consinto violência\\*
Tenham santa paciência\\*
Não posso ser traidor''

o que eu posso arranjar\\*
Para não ser censurado\\*
É fazer por onde ele\\*
Só ande aqui desarmado,\\*
E tomo conta do resto,\\*
Faço dele um homem honesto\\*
Pacato e moralizado

\end{verse}

\chapter{A sorte de uma meretriz}

\begin{verse}

Não se engane com o mundo\\*
Que o mundo não tem que dar,\\*
Quem com ele se iludir\\*
Iludido há de ficar\\*
Pois temos visto exemplos,\\*
Que é feliz quem os tomar

Doze anos tinha Aulina\\*
Seu pai era fazendeiro,\\*
Casa que naquele tempo\\*
Havia tanto dinheiro\\*
Muitas joias de valor,\\*
Crédito no mundo inteiro

Aulina, eu creio, não tinha\\*
Outra igual na perfeição,\\*
Parece que a natureza\\*
Carregou mais nela a mão\\*
Pois nela via-se a força\\*
Do autor da criação

Os olhos dela fingiam\\*
Raios do sol da manhã,\\*
O rosto bem regular\\*
Corado corno a romã\\*
Parecia que as estrelas,\\*
Queriam chamá-la irmã

Os dedos alvos e finos\\*
Qual teclados de piano,\\*
Quem a visse só diria\\*
Qu~ não era corpo humano\\*
Parecia ser propósito,\\*
Do Divino Soberano

Também tinha tanto orgulho\\*
Que nem aos pais conhecia,\\*
Se julgava saliente\\*
.A todo mundo que via\\*
Julgando que todo mundo\\*
A ela se curvaria

Quando inteirou vinte anos\\*
Por si se prostituiu\\*
O pai quase enlouquece\\*
Tanto desgosto sentiu\\*
Porque em toda família\\*
Um caso assim nunca viu

Logo que caiu no mundo\\*
Por todos foi abraçada,\\*
Por as mais aItas pessoas\\*
Era sempre visitada\\*
Por fidalgos e militares,\\*
Por todos era adorada

Recebeu logo um presente\\*
De um palacete importante\\*
Com uma mobília sublime\\*
Dada pelo seu amante\\*
A obra de mais estima\\*
A quem se chama elegante

Para sala de visita\\*
Comprou um rico piano,\\*
Quatro consolos de mármore\\*
Um aparador de ébano\\*
Uma cômoda muito rica,\\*
Que só a de um soberano

Ricas cadeiras modernas\\*
Candeeiros importantes,\\*
Jarros de fino cristal\\*
Espelhos muito elegantes\\*
O retrato dela em um quadro\\*
Com quatro ou cinco brilhantes

Um grande damasco verde\\*
A sala toda cobria\\*
Toalha bordada a ouro\\*
Em qualquer quarto se via\\*
Era só de porcelana\\*
Toda a louça que existia

Nem é preciso falar\\*
No quarto onde ela dormia,\\*
Porque já se viu na sala\\*
A riqueza que existia\\*
Agora na cama dela,\\*
Faça idéia o que havia

Durante cinco ou seis anos\\*
A vida dela era assim\\*
A casa era um céu de estrelas\\*
Rodeada de marfim\\*
Vivia ela qual vive\\*
Um beija-flor no jardim

Adoeceu de repente\\*
Não cuidou logo em tratar-se\\*
Julgando que dos amantes\\*
Nenhum a desamparasse\\*
Devido à sua influência\\*
Qualquer médico curasse

Foi vice-verso o seu cálculo\\*
A si só chegaram dores,\\*
Foi perdendo a influência,\\*
Multiplicando os clamores\\*
Não foi mais em sua casa\\*
Nenhum dos adoradores

Pegou logo a empenhar\\*
As joias que possuía,\\*
Por menos do seu valor\\*
Diversas coisas vendia\\*
E a moléstia no seu auge\\*
Crescendo de dia a dia

No período de dois anos\\*
Gastou o que possuía,\\*
Pegou logo pelas joias\\*
De mais valor que existia\\*
Sofás, cadeiras e consolos,\\*
Vendeu tudo em um só dia

Os quadros, os aparadores\\*
Pianos, relógios, espelhos\\*
Vendeu-os para curar\\*
Duas fístulás nos joelhos\\*
Já desej ava encontrar\\*
Quem lhe desse alguns conselhos

Afinal vendeu a casa\\*
E a cama onde dormia\\*
Era o único objeto\\*
Que em seu poder existia\\*
Ainda um amante vendo\\*
Jamais a conheceria

``Meu Deus'', exclamava ela\\*
Vai infeliz meu futuro\\*
Nasci em berço dourado\\*
Para morrer ~o monturo\\*
Quanta diferença existe,\\*
Da seda para o chão duro

Quantos lordes aos meus pés\\*
Se esqueciam de seus cargos,\\*
Me adoravam como santa\\*
Me mostrando mil afagos\\*
Hoje não vejo nenhum,\\*
Nesses dias tão amargos

Quede os grandes militares\\*
Que não podiam passar,\\*
Três dias numa semana\\*
Sem me virem visitar\\*
E faziam de mim santa,\\*
De meu divã um altar

Nada disso existe mais\\*
Tudo já se dissipou,\\*
As promessas e os presentes\\*
O vento veio e levou\\*
Em paga de tudo isso\\*
Na miséria me deixou

Essas dores que hoje sofro\\*
É justo que sofra elas,\\*
Essas lágrimas que eu derramo\\*
Serão em pagas daquelas\\*
Que fiz gotej ar dos olhos\\*
Das casadas e das donzelas

Sinto dores com excesso\\*
Ouço a voz da consciência,\\*
Me dizer: ``Filha maldita\\*
Tua desobediência\\*
Clamará perante a Deus,\\*
E pedirá providência''

Ela em soluços exclamava:\\*
``Meu Deus, tende compaixão,\\*
Nega-me tudo na vida\\*
Mas me alcançai o perdão\\*
Santíssima Virgem, rogai,\\*
Pela minha salvação''

Que cobertores tão caros\\*
Já forraram meu colchão,\\*
Que cortinas de seda\\*
De grande admiração\\*
Hoje não tenho uma estopa\\*
Que forre aqui esse chão

Ricos vestidos de seda\\*
Lancei muitos no monturo,\\*
Saias ainda em estado\\*
Camisa de linho puro\\*
Não pensava na desgraça\\*
Que vinha para o futuro

Minha mesa nesse tempo\\*
Tinha de tudo que havia,\\*
Só mesa de um personagem\\*
De alta categoria\\*
Hoje o resto de uma sopa\\*
Quando agora me servia

Peço esmola a quem passa\\*
Esse nem me dá ouvido,\\*
Quem outrora me adorava\\*
Não ouve mais meu gemido\\*
Passa por mim torce à cara,\\*
Se finge desconhecido

Eu era como uma flor\\*
Ao despontar da manhã\\*
Representava outrora\\*
Aquela deusa louçã\\*
Meus amantes perguntavam,\\*
Se a lua era minha irmã

As majestades chegavam\\*
Antes da celebração,\\*
Humildes como um escravo\\*
Me faziam saudação\\*
Como se a render-me culto\\*
Seria uma obrigação

O Exército e o comércio\\*
A arte e agricultura,\\*
Todos me ofereciam\\*
Seu afeto de ternura\\*
Tudo vinha admirar\\*
Minha grande formosura

Mas eu vivia enganada\\*
Com essas tristes carícias,\\*
Eu bem podia saber\\*
Que o mundo não tem delícias\\*
É um gozo provisório,\\*
É um cofre de malícias

Donzelas eis o exemplo\\*
Para todos que estão vendo,\\*
Não me viram há poucos dias\\*
Como o sol que vem nascendo?\\*
Já estou aqui no chão,\\*
Os tapurus me comendo

Ah! meu pai se tu me visse\\*
N essa miséria prostrada,\\*
Embora que vossa face\\*
Foi por mim injuriada\\*
Talvez que ainda dissesse:\\*
``Deus te perdoe, desgraçada''

Ah! minha mãe carinhosa\\*
Se eu agora te abraçasse,\\*
Inda com essa agonia\\*
Talvez que me consolasse\\*
E antes de partir do mundo,\\*
Essa sede saciasse

Sinto o soluço da morte\\*
Já é hora de partir,\\*
Peço ao meu anjo da guarda\\*
Para comigo assistir\\*
Porque temo que o demônio,\\*
Não venha me perseguir''

Uma velha caridosa\\*
Trouxe água, ela bebeu,\\*
Matou a sede que tinha\\*
E graças a Jesus rendeu\\*
Erguendo os olhos ao céu,\\*
Nesse momento morreu

\end{verse}


\chapter{A chegada de João Pessoa no céu}

\begin{verse}

Parti pelo espaço em fora\\*
Cortando os ares ligeiro\\*
E levando presa aos ombros\\*
A viola de troveiro;\\*
Das nuvens rasguei o véu\\*
Dessa vez eu fui ao céu,\\*
Num vôo direto e certeiro

Não se admire o leitor\\*
De um trovador voar,\\*
Pois seu pensamento sempre\\*
Vive no espaço a vagar;\\*
Se alguém duvidar de mim\\*
Embarque num Zepelim\\*
E vá no céu indagar

Quando estava em grande altura\\*
O meu vôo interrompi\\*
Um palácio magnífico\\*
Então no espaço vi,\\*
Mil trombetas entoando\\*
Hinos de glória cantando\\*
Muitos arcanjos ouvi

Tinha grandes dimensões\\*
Como uma grande cidade,\\*
Num trono maravilhoso\\*
Estava sentada a Verdade;\\*
Em cima muito brilhante\\*
Vi bem claro e flamejante\\*
A palavra ``Eternidade''

Seduzido pelo canto\\*
E pela grande beleza\\*
De tudo que ali via\\*
Me aproximei com presteza\\*
E presenciei então\\*
A uma recepção\\*
Para mim grande surpresa

Chegava ao céu neste instante\\*
Uma alma heróica boa\\*
Uma legião de anjos\\*
Lhe acompanha e entoa\\*
Lindos cânticos anunciando\\*
Que ao céu ia chegando\\*
A alma de João Pessoa

São Pedro abriu-lhe a porta\\*
Veio fora o receber\\*
E pegando-o pelo braço\\*
Levou-o a comparecer\\*
Ao supremo-tribunal\\*
Aonde o bem e o mal\\*
Sua recompensa irão ter

Cheguei-me ao velho porteiro\\*
Com gestos mui respeitosos\\*
Pedi-lhe: ``Deixe-me entrar\\*
Ver esses cantos formosos''\\*
Ele a entrada cerrou\\*
Pra o sereno me mandou\\*
Que é o lugar dos curiosos

São Miguel que é meu padrinho\\*
Tendo queixa do porteiro\\*
Desde que o velho tomou-lhe\\*
O seu lugar de chaveiro\\*
Me arranjou um lugar\\*
De onde pude observar\\*
E ver a festa o dia inteiro

Vi então como um júri\\*
Mas num trono luminoso\\*
Sentado o juiz dos juízes\\*
Deus Pai, Todo Poderoso\\*
O julgamento presidia\\*
Ao lado a Virgem Maria\\*
Com o seu semblante formoso

Na tribuna da defesa\\*
Estava o Espírito Santo\\*
Cristo o filho de Deus vivo\\*
Noutro trono com um manto,\\*
O ódio, a inveja, o mal,\\*
Também tinham um local\\*
Estavam ali num recanto

Em torno da mesa estavam\\*
A justiça, honra, o dever,\\*
A razão,a honestidade,\\*
A consciência e o saber\\*
E mais além muito calma\\*
Estava sentada a alma\\*
Que ia ao júri responder

Era João Pessoa que então\\*
Ante o Todo Poderoso\\*
Curvou-se humildemente\\*
E lhe disse: ``Pai bondoso\\*
Bem sabeis o quanto fiz\\*
Julga-me reto juiz\\*
O mundo é falso e enganoso''

E fitando-lhe ternamente\\*
Disse-lhe o meigo pastor:\\*
``Sou o pai de misericórdia\\*
Sol da justiça e do amor\\*
Ouço lágrimas, gemidos,\\*
Sou o Deus dos oprimidos\\*
Fala, filho, sem temor''

João Pessoa levantou\\*
A sua fronte e então\\*
Por todo âmbito celeste\\*
Correu um grande clarão,\\*
Clarão divino e bendito\\*
Brilhando no infinito\\*
Como o sol pelo verão

``Pai! se na terra lutei\\*
Não verti sangue de irmão\\*
Lutei ao sol da justiça\\*
Com a vossa lei na mão\\*
Lutei pela integridade\\*
Pela honra e honestidade\\*
Justiça foi meu brasão

Lutei a favor do fraco\\*
Fui de encontro ao opressor\\*
Nunca empreguei a violência\\*
Contra o justo ou pecador;\\*
Minha lei foi a clemência\\*
Pra todos tive indulgência\\*
Ao meu próximo tive amor

Pai! meu peito só alberga\\*
Misericórdia e perdão\\*
Perdoei meus inimigos\\*
Pois nasci e morri cristão:\\*
``Chagas cruéis, cruéis dores\\*
Para mim são outras flores''\\*
Meu inimigo é meu irmão

Deram-me a toga, honrei-a,\\*
Não a manchei com vilezas;\\*
Nunca o suborno, a opressão\\*
Obrigaram-me a baixezas;\\*
Nunca ao forte me curvei\\*
Defendi a tua lei\\*
Jamais cedi a torpezas

Caluniaram-me meu pai\\*
Como fizeram a teu filho\\*
Da minha toga humilde\\*
Quiseram manchar o brilho;\\*
Mas do vosso ensinamento\\*
Não me afastei um momento\\*
Sempre andei por esse trilho

Nasci numa terra pequena\\*
Pela seca devastada\\*
Mas o povo em tudo via\\*
Vossa vontade e mais nada;\\*
E pelo sol requeimado\\*
Essa gente ajoelhada\\*
Venera a Virgem Sagrada

Contra meus gestos humildes\\*
Esse povo me aclamou\\*
E as rédeas de seu governo\\*
N as minhas mãos entregou;\\*
Eu com o povo governei\\*
Do povo não me afastei\\*
Se errei o povo apoiou

Fazer ``maior bem possível''\\*
Foi sempre minha intenção\\*
Guiei-o para o progresso\\*
Não preguei subversão\\*
Preguei-lhe lei e direito\\*
E estimulei-lhe no peito\\*
Honra, civismo e ação

Fiz esse povo vibrar\\*
De amor por seu torrão\\*
Mandei ensinar a seus filhos\\*
A santa religião\\*
Se errei tende piedade\\*
Errar é da humanidade\\*
E a lei do bom é o perdão

Os filhos que vós me destes\\*
Na vossa lei os criei\\*
Junto à miséria alheia\\*
Compadecido chorei\\*
Dei roupa aos esfarrapados\\*
Trabalho aos desempregados\\*
Muitas lágrimas enxuguei

Fiz que nos cárceres entrasse\\*
A vossa religião\\*
Levando aos infelizes\\*
Fé e regeneração\\*
Pai! eu fui justo e honrado\\*
E o dinheiro do Estado\\*
Não comi, não fui ladrão

Se lancei alguns impostos\\*
O fim desculpava o meio\\*
Foi pra salvar meu estado\\*
Dum fracasso horrível e feio;\\*
Nunca devi a ninguém\\*
Meu viver mancha não tem\\*
Jamais me apossei do alheio''

Levantou-se a inveja\\*
No papel do promotor\\*
Disse: ``Sempre espalhastes\\*
Revolução e pavor\\*
Mandaste prender, roubar,\\*
Teus inimigos saquear,\\*
Tua lei foi o terror''

Responde então a verdade:\\*
``Protesto em nome do Eterno\\*
A mentira aqui não entra\\*
Retira-te para o inferno;\\*
Mentes descaradamente\\*
Contra o justo, o inocente,\\*
Não tem poder o Averno''

Gritou o ódio iracundo\\*
(Querendo interromper):\\*
''Este homem é um malvado\\*
Foi quem fez a guerra arder\\*
E deste sangue derramado\\*
É ele o único culpado\\*
Por isso irá responder``

Disse a justiça: ``É falso\\*
Cala-te louca ambição\\*
A ele não cabe a culpa\\*
De sangue e revolução\\*
Os verdadeiros culpados\\*
Foram os grandes malvados\\*
Cujo ódio é um vulcão

Retira-te ódio, é demais\\*
Teu semblante no recinto''\\*
O mal então dando larga\\*
À baixeza, ódio instinto,\\*
Mentiu e caluniou\\*
A língua torpe ladrou\\*
Como um triste cão faminto

Depois do mal ter falado\\*
Levantou-se João Pessoa\\*
Disse aos três: ``Não conheço\\*
Tristes a errar à toa\\*
Procurando, os desgraçados\\*
Infelizes em quem o brado\\*
Da lei de Deus não ressoa

Quanto ao mal só pelo nome\\*
É que eu o conhecia\\*
E quando caí ferido\\*
Pela sua covardia,\\*
Quando aos berros gritou:\\*
`Conheces o mal? Eu o sou'\\*
Sorrindo então eu morria''

Do recinto foram expulsos\\*
A inveja, o ódio e o mal\\*
Falou então a justiça\\*
Entre outras coisas afinal\\*
E disse: ``Irmão, ficarás,\\*
E recompensa terás,\\*
No reino celestial''

A consciência falou:\\*
``Justo foste em tua vida\\*
O mundo é pequeno, é vil\\*
A humanidade é perdida\\*
Tua alma pura de cristal\\*
No reino celestial\\*
Será bendita, acolhida''

Falou a honestidade\\*
E disse: ``Justo varão\\*
Unicamente teu crime\\*
Foi não ter sido ladrão\\*
No reino de Deus amado\\*
E pelos anjos adorado\\*
Ficarás em união''

A honra disse também:\\*
``Paladino da nobreza\\*
Tua alma jamais cedeu\\*
À corrupção da riqueza\\*
Péncido por garra adunca\\*
Porém humilhado nunca\\*
Terás da glória a grandeza''.

O dever falou assim:\\*
``Mártir duma causa nobre\\*
É ditoso o firmamento\\*
Que os teus despoj os cobre\\*
À palavra não faltaste\\*
Teu sangue sacrificaste\\*
Em defesa do homem pobre

Por amigos ``solidários''\\*
Traído e abandonado\\*
Lutaste só até que um dia\\*
Caístes assassinado\\*
Jamais domou-te a descrença\\*
No céu agora a recompensa\\*
Por seres justo ou honrado''

A Razão falou também:\\*
``Teus inimigos não sabiam\\*
Que te prostrando na Glória\\*
Subir-te à gloria fariam\\*
Do Brasil és um padrão\\*
Morto não te vencerão\\*
Vivo não te venceriam

Depois que todos falaram\\*
Falou também o saber:\\*
``Esse homem sábio e prudente\\*
Jamais faltou ao dever\\*
Teve também seu calvário\\*
Cumpriu um cruel fadário\\*
Pelos homens quis morrer

Foi uma vítima na terra\\*
De ódio e perseguições\\*
Só governou com justiça\\*
Não teve contradições\\*
Fez um governo que é novo\\*
Foi do povo e para o povo\\*
Não pôs a pátria em leilões''

Chegou a vez do juiz\\*
Fez-se silêncio e então\\*
Disse o pai amantíssimo:\\*
``Filho de meu coração\\*
Vinde a mim escolhido\\*
O céu te foi prometido\\*
Na primeira geração''

E assim ficou no céu\\*
Aquela alma pura e boa\\*
A glória astral de seu nome\\*
Por todo espaço ressoa\\*
Na terra desconsolado\\*
Geme o povo contristado\\*
Relembrando João Pessoa

Estava terminada a festa\\*
Eu fui a outro lugar\\*
Aí tudo era terrível\\*
Só se ouvia praguejar;\\*
Era uma fornalha ardente,\\*
Via-se o ferro candente,\\*
Sob o malho ressoar

Havia muitos ferreiros,\\*
Ocupados em forjar,\\*
Umas cadeiras bem feitas,\\*
De encosto e espaldar;\\*
Porém era ``ferro em brasa,\\*
E disseram-me: "Nesta casa\\*
Vem gente boa morar''

Aquilo era o inferno,\\*
Não pude mais duvidar,\\*
Vi então um diabinho\\*
Que começou a me explicar\\*
Que aquelas grandes cadeiras,\\*
Era o prêmio das roubalheiras,\\*
Duma ``bancada'' sem par

Me disse mais o diabinho\\*
Cadeiras grandes tem duas,\\*
A ``bancada'' encomendou\\*
Aquelas cinco são suas;\\*
Tem uma de senador,\\*
E outra de desembargador,\\*
Dos Tapiocas e Gazuas

Vi no centro um cozinheiro,\\*
Fazendo um estranho café,\\*
Era chumbo derretido\\*
Perguntei: ``Para quem é?''\\*
Ele disse: ``É segredo''\\*
E apontou com o dedo\\*
O nome dum \textit{coroné}

Tinha ali um grande pátio\\*
De fogo todo cercado\\*
Disse-me o diabo: ``Aqui,\\*
Está sendo muito esperado\\*
Um rei truão e maluco,\\*
Que armado dum trabuco\\*
Fez da Princesa um estado''

Mostrou-me um prego comprido\\*
Onde há de ser pendurado\\*
Pelas orelhas unzinho,\\*
Que lá é advogado;\\*
E nunca encontrou uma causa\\*
Estamos fazendo uma pausa,\\*
Pois ele está pronunciado

Nisso entra Satanás\\*
De grande número acompanhado\\*
Ao ver tão grande séquito,\\*
Fiquei bastante espantado\\*
Era um enorme barulho\\*
Na mão trazia um embrulho,\\*
Que me entregou com cuidado

Com muita delicadeza,\\*
Ele disse: ``Por favor\\*
Quando for pra sua terra\\*
Dê isso lá ao doutor''\\*
E disse um nome conhecido;\\*
Eu fiquei estarrecido,\\*
Tremi até de pavor

Nisso Satanás mostrou-me,\\*
Todos os apartamentos,\\*
Disse: ``Muitos estão alugados\\*
A gente de conhecimento;\\*
São meus sócios e camaradas\\*
As camas estão preparadas\\*
E chegam a qualquer momento

Nisso ouvi um alarido,\\*
Chegava uma multidão\\*
Esfarrapados, sangrentos\\*
Manchas de sangue na mão,\\*
Na frente vinha um chacal;\\*
Com um riso bestial,\\*
Parecia um furacão

Traziam escritos na testa\\*
Ladrões, vis e assassinos;\\*
Era um conjunto terrível,\\*
De instintos negros ferinos;\\*
Do sangue humano manchados\\*
Como lobos esfaimados,\\*
Traziam o focinho canino

Um dizia: ``Eu menti'',\\*
O outro: ``Eu caluniei'',\\*
O outro: ``Os votos do povo,\\*
Eu descarado roubei,\\*
Fiz uma grande opressão\\*
Decretei intervenção,\\*
E ao povo trucidei''

Mais feroz que os outros todos,\\*
Levantou-se um esfarrapado,\\*
Gritou: ``Matei indefeso,\\*
Um homem inerme e sentado;\\*
No seu sangue me espojei,\\*
Sem vida em terra o prostrei;\\*
Com fúria de um condenado

Imediatamente eu vi\\*
O Satanás agarrá-lo,\\*
E numa cova de fogo,\\*
Nas profundezas jogá-lo;\\*
Berrava como mordido,\\*
Seu corpo estava roído,\\*
E vi o fogo devorá-lo

Eu estava aperreado\\*
Procurando uma saída\\*
O Satanás empurrou-me,\\*
Com uma raiva incontida\\*
No espaço arremessou-me\\*
A violinha quebrou-me\\*
Com a pressa da descida

Quando eu vinha pelos ares\\*
Uma pessoa me chama,\\*
Depois eu vi que morria\\*
Me segurei numa rama,\\*
Fazia um frio danado\\*
Eu vi que tinha sonhado\\*
E no sonho caí da cama

\end{verse}

\chapter{A moça que foi enterrada viva}

\begin{verse}

Nos sertões de Teresina\\*
Habitava um fazendeiro,\\*
Era materialista\\*
Além disso interesseiro\\*
Só amava a duas coisas\\*
Homem valente e dinheiro

Era quase analfabeto\\*
Ostentava o fanatismo\\*
Mostrava grande afeição\\*
Pelo imperialismo\\*
Ele era um potentado\\*
Nos tempos do carrancismo

Como era muito rico\\*
Confiava em sua sorte\\*
Era o temor dos sertões\\*
Naquela zona do Norte\\*
Que o que quisesse fazia,\\*
Ainda encarando a morte

Vivendo como casado\\*
Na mais perfeita harmonia\\*
Tinha quatro filhos homens\\*
Todos em sua companhia\\*
Tinha uma filha moça,\\*
Por nome dona Sofia

Esta moça era a caçula\\*
Vinte e um anos contava,\\*
Os irmãos eram mais velhos\\*
Mas nenhum se emancipava\\*
Só era dono de si\\*
No dia que se casava

O velho não se importava\\*
De fazer revolução,\\*
Para sustentar capricho\\*
Ou vingar sua paixão\\*
Seus filhos também seguiam\\*
Nessa mesma opinião

Quando ele conversava\\*
No meio de muita gente\\*
Dizia: ``Tenho uma filha\\*
É uma moça decente\\*
Porém só casa com ela\\*
Quem for um bicho valente''

Com poucos dias depois\\*
A notícia se espalhava,\\*
Qualquer um rapaz solteiro\\*
Que na estrada passava\\*
Já ia com tanto medo,\\*
Pra fazenda nem olhava

Sofia se lastimava\\*
Dizendo: ``Até onde vai,\\*
Este meu padecimento\\*
Sem se ver de onde sai\\*
Eu hei de ficar solteira,\\*
Pra fazer gosto a meu pai?''

Depois enxugou aS lágrimas\\*
Que banhavam o lindo rosto\\*
Dizia: ``Eu encontrando\\*
Um rapaz moço e disposto\\*
Eu farei com que meu pai\\*
Passe por esse desgosto''

Um rapaz sabendo disto\\*
Se condoeu da donzela\\*
Vendo que não encontrava\\*
Outra moça igual àquela\\*
Um dia determinou-se\\*
Dizendo: ``Vou roubar ela''

Escreveu logo um bilhete\\*
Dizendo: ``Dona Sofia,\\*
Eu ontem fui sabedor\\*
Do que a senhora sofria\\*
Fiquei muito indignado\\*
Pois lhe tenho simpatia

Conheço perfeitamente\\*
Que vou entrar num perigo\\*
Porque seu pai conhecendo\\*
Torna-se meu inimigo\\*
Basta saber que a senhora,\\*
Pretende casar comigo

Eu sou um rapaz solteiro\\*
Não tenho conta a quem dar\\*
Responda este bilhete\\*
Pra eu me desenganar\\*
Se me aceita como esposo,\\*
O jeito eu vou procurar''

Sofia mandou o sim\\*
Pela manhã muito cedo,\\*
Fazendo ver a seu noivo\\*
Que de nada tinha medo\\*
Queria falar com ele,\\*
No outro dia em segredo

O moço aí preveniu-se\\*
De um punhal e um facão,\\*
Pistola boa na cinta\\*
Cartucheira e munição\\*
Seguiu pra casa do velho,\\*
Porém com boa intenção

Encontrou uma criada\\*
Com um candeeiro na mão,\\*
Perguntou-lhe: ``Onde é o quarto\\*
Da filha de seu patrão?''\\*
Diz ela: ``Ao lado esquerdo\\*
Pela porta do oitão''

A noite era muito escura\\*
Por ali ninguém o viu,\\*
Ele tanto pelejou\\*
E tanto se retraiu\\*
Que entrou no quarto da moça\\*
E o velho nem pressentiu

Foi entardecendo a noite\\*
Acabaram de cear,\\*
Quando a moça entrou no quarto\\*
Para se agasalhar\\*
Foi avistando o rapaz,\\*
Ficou sem poder falar

O rapaz muito ligeiro\\*
Pegou ela pela mão,\\*
Porém com muito respeito\\*
Contou-lhe sua intenção\\*
Dizendo: ``Eu arranjo tudo,\\*
Sem precisar de questão''

Assim passaram a noite\\*
A moça muito assustada\\*
Quando amanheceu o dia\\*
Por sua mãe foi chamada\\*
Para cuidar dos trabalhos,\\*
Como era acostumada

O rapaz ficou no quarto\\*
Do povo se ocultou,\\*
Quando botaram o almoço\\*
Então a moça voltou\\*
De parelha com seu noivo,\\*
Ao velho se apresentou

O rapaz saiu do quarto\\*
Seu rosto não demudava,\\*
Fincou o punhal na mesa\\*
Dizendo se aproximava:\\*
``É este o homem valente!...\\*
Que o senhor procurava?''

Sou eu, seu futuro genro\\*
Que amo a esta donzela,\\*
Tudo isso que já fiz\\*
Não é criticando dela\\*
Embora me custe a vida,\\*
Só me casarei com ela!''

O velho conheceu logo\\*
Que não tinha jeito a dar\\*
Correu a vista nos filhos\\*
Como quem quer avisar\\*
Aí todos convidaram\\*
O moço para almoçar

Ele aceitou o convite\\*
Porque tinha precisão\\*
Disse o velho mansamente:\\*
``Entre nós não há questão\\*
Precisamos fazer logo,\\*
Toda esta arrumação

O senhor vá para casa\\*
Veja que falta arrumar,\\*
Arrumação para a noiva\\*
Eu também vou aprontar\\*
E o senhor no dia quinze,\\*
Venha para se casar''

Assim que o rapaz saiu\\*
O velho chamou Sofia,\\*
Dizendo: ``Filha maldita\\*
Quem te deu tanta ousadia?\\*
Me obrigaste a fazer\\*
O que nunca pretendia!''

Aí gritou para os filhos,\\*
Dizendo de cara dura:\\*
``Agarrem esta maldita\\*
Prendam ela bem segura\\*
E vão no quarto do meio\\*
Cavem uma sepultura''

Naquele mesmo momento\\*
Sofia foi amarrada,\\*
Para o quarto que estava\\*
A sepultura cavada\\*
Aonde a triste donzela\\*
Havia de ser sepultada

Reuniu-se em roda dela\\*
Toda aquela comitiva,\\*
O pai, a mãe, os irmãos\\*
Por infame tentativa\\*
Condenaram a pobre moça\\*
Para sepultarem-na viva

Naquela situação\\*
Que estava a pobre Sofia,\\*
Pedindo ao pai, em soluços,\\*
E o velho não atendia:\\*
``Meu pai, não me mate hoje\\*
Deixe eu viver mais um dia!''

Sofia se lastimava\\*
E o velho não dava ouvido,\\*
Depois disse para ela:\\*
``Nada vale o seu pedido\\*
A senhora está passando\\*
Da hora de ter morrido''

Sofia disse: ``Meu pai\\*
Tenha de mim compaixão\\*
Mande chamar o vigário\\*
Pra me ouvir em confissão\\*
Talvez que por este meio\\*
Eu possa alcançar perdão!''

``A senhora em parte alguma\\*
Podia ser perdoada,\\*
Não há sentença bastante\\*
Para filha excomungada\\*
Quem fez o que você fez\\*
Só paga sendo queimada''

O velho zangou-se e disse:\\*
``Não quero mais discutir\\*
Palavras de sua boca\\*
Não pretendo mais ouvir\\*
Siga; entre para a cova,\\*
Para eu mandar entupir''

Aí botaram Sofia\\*
Pra dentro da cova escura,\\*
O buraco foi cavado\\*
Com dez palmos de fundura\\*
Que sofrimento tirano\\*
Desta infeliz criatura

O velho como uma fera\\*
Mandou ela se deitar\\*
Ela na ânsia da morte\\*
Começou logo a gritar\\*
Pedia aos outros: ``Me acudam\\*
Que meu pai quer me matar!''

O velho era malvado\\*
Pior que o Satanás\\*
Pegou Sofia dizendo:\\*
``Veja bem como se faz!''...\\*
Botou-lhe terra por cima,\\*
Até que não gritou mais

Aí seguiram para a sala\\*
Ele, os filhos e a mulher\\*
Dizendo: ``Estou satisfeito\\*
Vou esperar o que houver\\*
Só fica mais perigoso,\\*
Se o noivo dela souber"

Logo preveniu-se tudo\\*
Contra o noivo de Sofia\\*
Nisto bateram à porta\\*
Mandaram ver quem batia\\*
Era o rapaz noivo dela,\\*
Porém de nada sabia

O velho disse pra ele:\\*
``O senhor de onde vem?\\*
Minha derrota está feita\\*
Aqui não me sai ninguém\\*
Matei sua noiva agora,\\*
E o senhor morre também''

Aí partiu para ele\\*
Como uma fera assanhada,\\*
O rapaz negou-lhe o corpo\\*
E deu-lhe uma punhalada\\*
O velho caiu gritando\\*
Não pode mais fazer nada

Reuniu-se contra ele\\*
Os quatro irmãos de Sofia,\\*
Atirando à queima-roupa\\*
Mas nem um tiro atingia\\*
E ele os poucos que dava,\\*
Lá um ou outro perdia

Com meia hora de luta\\*
Estava tudo sem ação,\\*
Os quatro irmãos de Sofia\\*
Dois morreram na questão\\*
Um correu espavorido\\*
E outro ficou no chão

O rapaz ficou sozinho\\*
Porém já muito ferido\\*
Quando foi pas~ando a porta\\*
Ouviu um grande gemido\\*
Diz ele: ``Talvez Sofia\\*
Inda não tenha morrido''

O rapaz muito ferido\\*
Conhecendo que morria,\\*
Seguiu pela casa adentro\\*
Procurando quem gemia\\*
Acertou logo no quarto,\\*
Onde enterraram Sofia

No mesmo canto encontrou\\*
A alavanca e a enxada\\*
Os ferros que tinham sido\\*
A dita cova cavada\\*
Com eles tirou Sofia,\\*
Quase morta asfixiada

O leitor preste atenção\\*
Sofia foi arrancada\\*
Não morreu por um motivo\\*
A cova não foi socada\\*
Só fazia quatro horas,\\*
Que tinha sido enterrada

O rapaz muito doente\\*
Ainda conduziu Sofia\\*
Pra casa de sua mãe\\*
Que nada disso sabia\\*
A velha quando viu ele\\*
Quase morre de agonia

Não fazia dez minutos\\*
Que o rapaz tinha chegado,\\*
N a casa de sua mãe\\*
Quando recebeu um recado\\*
Pelo irmão de Sofia\\*
Ia ser assassinado

Disse o rapaz a Sofia:\\*
``Me acabo aqui mas não corro\\*
Já estou muito ferido\\*
Desta conheço que morro\\*
E também não me sujeito\\*
Gritar pedindo socorro''

Aí ele pediu à mãe:\\*
``Veja as armas que aí tem\\*
O bacamarte, a espingarda\\*
E a pistola também\\*
E corra para bem longe\\*
Porque o povo já vem"

A velha morta de medo\\*
Trouxe as armas e entregou\\*
Traspassada de agonia\\*
Chorando o abençoou\\*
Temendo a morte fugiu\\*
Porém Sofia ficou

O rapaz entrincheirou-se\\*
Bem na porta da entrada\\*
Sofia estava por tudo\\*
Não se temia de nada\\*
Foi botar o seu piquete\\*
Atrás pela retaguarda

Sofia triste pensando\\*
Tão depressa se acabar\\*
Conhecendo que morria\\*
Talvez antes de casar\\*
Quando levantou a vista\\*
Foi vendo o grupo chegar

O rapaz que estava pronto\\*
Com o seu revólver na mão\\*
Amparou-se num portal\\*
Erifrentou o pelotão\\*
Cada tiro era um defunto

Que embolava no chão\\*
Sofia na retaguarda\\*
Inda emparelhou seis\\*
O bacamarte era bom\\*
Certa pontaria fez\\*
Quando puxou o gatilho,

Caiu tudo de uma vez\\*
Entrou um pela janela\\*
Sofia não pressentiu\\*
O rapaz estava lutando\\*
De forma nenhuma o viu\\*
Atirou nele nas costas\\*
Que o pobre rapaz caiu

Aí pegaram Sofia\\*
Que não podia escapar,\\*
Cortaram todo cabelo\\*
Mandaram os olhos furar\\*
Depois penduraram ela\\*
Dizendo: ``Vamos sangrar''

Sangraram devagarinho\\*
Pra inda mais judiar\\*
Antes da moça morrer\\*
Eles foram retalhar\\*
Em pedaços tão pequenos\\*
Que não puderam enterrar

Quem me contou esta história\\*
Foi um rapaz muito sério\\*
Foi testemunha de vista\\*
Daquele caso funéreo\\*
Os corpos foram levados\\*
Num cesto pro cemitério

O mundo está corrompido\\*
O erro vem de atrás\\*
Muitos acontecimentos\\*
De resultados fatais\\*
Só acontecem com as filhas\\*
Que vão de encontro aos pais

\end{verse}


\chapter[As quatro classes corajosas -- Vaqueiro, agricultor, soldado e pescador]{As quatro classes corajosas\\ Vaqueiro, agricultor,\\ soldado e pescador}

\begin{verse}

Tenho ouvido alguém dizer\\*
Sem ver que eu estou presente:\\*
``José Camelo não presta,\\*
Porque só fica contente\\*
Quando mete a língua dele\\*
Contra esse, ou contra aquele\\*
Inda sendo um seu parente''

Pois bem, se eu prestei estudos\\*
Para ser ruim demais\\*
Sou quem conheço os viventes\\*
Ruins, pois são meus iguais\\*
E o ruim que detesta\\*
Falar mal de quem não presta\\*
Inda mais ruim se faz

De gente ruim eu falo\\*
Mas de gente boa não;\\*
Portanto vou nestes versos\\*
Fazer uma exaltação\\*
Às quatro classes que eu vejo\\*
Que merecem sem gracejo\\*
Honras pela profissão

São quatro classes, já disse,\\*
Porém me é necessário\\*
Mencionar quais são elas\\*
Neste verso voluntário:\\*
Vaqueiros, agricultores,\\*
Soldados e pescadores;\\*
Destes não serei contrário

São quatro classes, porém\\*
Vou falar primeiramente\\*
Sobre a classe dos vaqueiros\\*
Fazendo ao mundo ciente\\*
O quanto são valorosos\\*
Ou por outra corajosos\\*
Honrando a sua patente

O vaqueiro é um herói\\*
Que não tem amor à vida\\*
Pois inda encontrando a morte\\*
N a frente de foice erguida\\*
Antes da morte matá-lo\\*
Ele lhe atira o cavalo\\*
E ela fica estendida

O vaqueiro não tem medo\\*
De pegar um mulo bravo\\*
E montar-se em cima dele\\*
Até fazê-lo um escravo\\*
E se o mulo der-lhe um coice\\*
Aí diga que danou-se\\*
Porque lhe paga o agravo

O vaqueiro não tem medo\\*
De se montar num cavalo\\*
E correr atrás d'um boi\\*
Num mato fechado ou ralo\\*
Pulando todo riacho\\*
E se for de serra abaixo\\*
Para ele é um regalo

O vaqueiro atrás de um boi\\*
Não respeita quixabeira\\*
Nem jurema, nem urtiga\\*
Nem macambirarasteira\\*
O pau bate o sangue sai\\*
Ele tomba mas não cai\\*
E continua a carreira

Quem nunca viu um vaqueiro\\*
Correndo dentro do mato\\*
Talvez julgue qu'eles tenham\\*
Pela vida algum precato\\*
Porém quem vê-los correndo\\*
Garanto ficar dizendo\\*
Que o que eu digo é muito exato

Já tenho visto vaqueiro\\*
Correndo em mato fechado\\*
Em lugar onde não pode\\*
Passar sequer um veado\\*
Porém ele na carreira\\*
Deixa o mato como esteira\\*
E já não fica engalhado

Faz medo ouvir-se de perto\\*
O estalo da madeira\\*
Produzido pel'um boi\\*
E um cavalo na carreira\\*
Parece uma tempestade\\*
Com sua temeridade\\*
Numa montanha altaneira

Além do vaqueiro ter\\*
Força, destreza e coragem\\*
São homens respeitadores\\*
E não gostam de vantagem\\*
Por isto gentes de bem\\*
Sempre para eles tem\\*
Um sorriso de homenagem

Numa ``corrida'' qualquer\\*
Onde esteja muita gente\\*
Os vaqueiros são os homens\\*
Pra quem se olha somente\\*
E as moças mais bonitas\\*
Muitas vezes compram fitas\\*
E lhes mandam de presente

Pois o homem que derriba\\*
Escanchado num cavalo\\*
Um touro de vinte arrobas\\*
Deve o mundo admirá-lo\\*
Por isto as moças sem pena\\*
Já para pagar-lhe a cena\\*
Lhe dão fitas, com regalo

Devido à mulher ser fraca\\*
Possui com ela o bom gosto\\*
De agradar e namorar\\*
A todo homem disposto\\*
Mas ao homem moleirão\\*
Ainda sendo um barão\\*
Ela lhe cospe no rosto

Por isto sempre os vaqueiros\\*
Devido à sua coragem\\*
São queridos das mulheres\\*
Por esta perfeita imagem\\*
Feita pelas mãos de Deus\\*
Por isto é que os filhos seus\\*
Lhe rendem tanta homenagem

Já falei sobre os vaqueiros,\\*
-- Classe muito valorosa --\\*
Agora posso falar\\*
Noutra classe corajosa\\*
Que são os agricultores\\*
-- Classe que merece flores\\*
Por ser muito proveitosa

A classe de agricultores\\*
É quem traz o mundo em pé;\\*
Pois é quem tira da terra\\*
O açúcar e o café\\*
O trigo, o milho, o feijão\\*
A farinha e o algodão\\*
E ninguém diz que não é

Tira também a batata\\*
O arroz, a rapadura,\\*
O inhame, a macaxeira,\\*
Coco e fumo com fartura;\\*
A laranja, o suanáz,\\*
A cebola e tudo mais\\*
Que pertence à agricultura

Portanto o agricultor\\*
Merece muito conceito\\*
Porque só ele é quem tem\\*
Disposição, força e jeito\\*
Já para tirar do chão\\*
A rica alimentação\\*
Que traz o mundo direito

Quem olhar para o serviço\\*
Que o pobre agricultor faz\\*
Achará que ele possui\\*
Força e coragem demais\\*
Pois vê que ele em seu trabalho\\*
Inda encontrando um engalho\\*
Já nunca dá para trás

O agricultor não teme\\*
Marimbondo nem formiga\\*
E não respeita o espinho\\*
Desde o cardo à urtiga\\*
Também a cobra não teme\\*
Qualquer lacrau ele espreme\\*
E lhe rebenta a barriga

O agricultor não teme\\*
A chuva, o sereno, o frio,\\*
O sol quente para ele\\*
É um pó-de-arroz macio\\*
O vento não lhe faz medo\\*
Pisa na lama bem cedo\\*
E não procura desvio.

O agricultor não teme\\*
Derrubar um mato grosso\\*
Onde existem baraúnas\\*
Já tão duras como um osso;\\*
Pois seu machado de aço\\*
Já não encontra embaraço\\*
Quando vai fazendo um roço

O agricultor não teme\\*
Descalço, andar no chão quente\\*
Pois inda o chão lhe queimando\\*
Não se mostra impaciente\\*
E quando às vezes se corta\\*
Ele calado suporta\\*
A dor que seu corpo sente

Portanto o agricultor\\*
É uma mola de aço\\*
Que movimenta este mundo\\*
Sem encontrar embaraço\\*
E quem disser o contrário\\*
Se não for um arbitrário\\*
É ruim, vil e devasso

Além da grande coragem\\*
Que tem o agricultor\\*
É dotado de ternura\\*
De paciência e amor\\*
Tem muita perseverança\\*
Fé e grande confiança\\*
No nosso Deus criador

Pois se a lagarta lhe come\\*
Sua lavoura primeira\\*
Ele não se desanima\\*
E sem menor choradeira\\*
Planta outra satisfeita\\*
Mostrando gosto perfeito\\*
Inda estando em quebradeira

E logo assim que lhe chega\\*
Milho e feijão com fartura\\*
Ele em vez de orgulhar-se\\*
Coitado, ainda procura\\*
Mandar para quem não tem\\*
Feijão, milho, arroz também\\*
Ou outra qualquer verdura

E quando bota nas feiras\\*
Sua farinha ou feijão\\*
O fiscal inda lhe diz\\*
Que ele não venda em porção\\*
Pois se vender ``atacado''\\*
Será logo castigado\\*
Como que seja um ladrão

Portanto o agricultor\\*
Se não fosse tão disposto\\*
Não trabalharia mais\\*
Tomado pelo desgosto\\*
De ver um fiscal mandar\\*
No que é seu, sem trabalhar\\*
Já além do grande imposto

Mas como o agricultor\\*
É dotado de coragem\\*
Continua trabalhando\\*
Coitado!\ldots{} como um selvagem,\\*
Exposto ao sol ardente\\*
À chuva e à terra quente\\*
Ao sereno e à friagem

Já falei sobre o prestígio\\*
Do agricultor; agora,\\*
Vou falar sobre o soldado\\*
Pois preciso nesta hora\\*
Dizer: que o soldado é\\*
Quem traz a justiça em pé\\*
Neste nosso mundo em fora

O soldado é um amigo\\*
Que não teme combater\\*
Defendendo a vida alheia\\*
Já sem pensar em morrer\\*
Pois entrando em luta forte\\*
Troca a vida pela morte\\*
Muitas vezes com prazer!

O soldado é quem defende\\*
A honra da casa alheia\\*
É quem leva o assassino\\*
Para os quartos da cadeia;\\*
É quem prende o cachaceiro\\*
Malcriado e desordeiro\\*
Quando ao sossego aperreia

O soldado é quem combate\\*
Os grupos de cangaceiros\\*
Que vivem martirizando\\*
Os incautos fazendeiros\\*
É quem prende o sedutor\\*
O gatuno e o roubador\\*
E chefes de bandoleiros

O soldado é quem garante\\*
O sossego d'uma feira\\*
Para não haver barulhos\\*
Onde haja bebedeira\\*
Pois onde não há soldado\\*
Todo ébrio anda armado\\*
E briga por uma asneira

O soldado, é ele quem\\*
Defende o fraco do forte\\*
Entrando dentro das lutas\\*
Ainda enxergando a morte\\*
Já nunca mostra fraqueza\\*
Ainda tendo a certeza\\*
Que a luta lhe rouba a sorte

Muita gente neste mundo\\*
Ainda diz com orgulho\\*
Que não gosta de soldado;\\*
Mas haja em casa um barulho\\*
E lhe apareça um soldado\\*
Que há de pedir-lhe ajoelhado\\*
Que ele acabe aquele embrulho

Pois um soldado sozinho\\*
Faz tremer dois valentões\\*
Faz chorar três cachaceiros\\*
E correr quatro ladrões\\*
Faz calar cinco insolentes\\*
Faz fugir seis renitentes\\*
Que estejam com questões

Só as vestes d'um soldado\\*
Por si merecem respeito\\*
Pois já está demonstrando\\*
Uma classe de conceito\\*
Uma classe que procura\\*
Com paciência e bravura\\*
Trazer o mundo direito

Inda um homem sendo fraco\\*
Mas se fazendo soldado\\*
Pela bandeira da pátria\\*
Não teme ser fuzilado\\*
Não é como o cangaceiro\\*
Que além de ser desordeiro\\*
Só briga estando emboscado

O soldado inda ferido\\*
Não perde a disposição\\*
Porque deseja cumprir\\*
Sua sagrada missão\\*
Talvez consigo lembrado\\*
Que houve um santo soldado\\*
Que foi são Sebastião

E com esse pensamento\\*
Às vezes muito ferido\\*
Inda procura lutar\\*
E se sai bem-sucedido\\*
Porque no fim da vitória\\*
Grava o nome na história\\*
Como herói destemido

Um galão ganha na luta\\*
Merece ser respeitado;\\*
Pois é como o diadema\\*
D'um santo martirizado\\*
É pois a relíquia santa\\*
Com que o soldado espanta\\*
A quem cometeu pecado

As flores que as moças jogam\\*
Sobre um soldado guerreiro,\\*
Quando volta d'um combate\\*
Valem mais do que dinheiro\\*
Pois aquelas flores são\\*
O mais perfeito galão\\*
Para o herói verdadeiro

Não há quem não tenha gosto\\*
De render uma homenagem\\*
A um soldado fiel\\*
Que se bate com coragem\\*
Em prol da ordem perfeita\\*
Quando a pátria está sujeita\\*
Aos pés da vilanagem

Se não houvesse soldado\\*
Não haveria respeito\\*
Pois todo bruto queria\\*
Trazer o mundo sujeito\\*
Aos seus caprichos malvados;\\*
Mas como existem soldados\\*
O mundo inda vai direito

Portanto os soldados são\\*
N ossos deuses defensores\\*
Pelo qual devemos todos\\*
Por cima jogar-lhes flores\\*
Pois sem eles, todo bruto\\*
Se tornava absoluto\\*
Nos causando dissabores

Já falei sobre os vaqueiros\\*
Agricultores também;\\*
Dos soldados já mostrei\\*
O valor que a classe tem\\*
Portanto vou dar louvores\\*
À classe dos pescadores\\*
Pois acho que me convém

Alguém diz que o pescador\\*
Não tem classe, então por isto\\*
Devo dizer neste versos\\*
Que alguns apóstolos de Cristo\\*
Foram homens pescadores\\*
E mais tarde pregadores\\*
Isto está mais do que visto

Se Cristo, sábio divino\\*
Procurava pescadores\\*
Para companhia dele\\*
E os fazia pregadores:\\*
É porque sabia então\\*
Que os pescadores são\\*
Os mais fortes lutadores

Todo mundo está ciente\\*
Da coragem sem medida\\*
Que o pescador representa\\*
Na sua pesada lida\\*
Bendigo: dentro dos mares\\*
Em perigosos lugares\\*
Onde ninguém fala em vida

Porém como o pescador\\*
Nasceu para não ter medo\\*
Acha que o mar furioso\\*
Não é mais que um brinquedo\\*
Pois se atira sobre as ondas\\*
Vendo visões hediondas\\*
Em quase todo rochedo

E se caso a ventania\\*
Roubar-lhe o mastro e a vela\\*
Ou lhe virar a jangada\\*
Ele fica em cima dela\\*
Depois põe-se a mergulhar\\*
E consegue revirar\\*
A mesma, e volta com ela

Quando chega é quase morto\\*
Na praia aonde pretende...\\*
Depois olha para o mar.\\*
Mas com tudo não se rende\\*
Pois na outra madrugada\\*
Naquela mesma jangada\\*
Volta a pescar onde entende

Se caso fisgar um peixe\\*
Que arraste sua jangada\\*
Ele não fica com medo\\*
Porque pra ele isto é nada;\\*
Então se a linha aguentar\\*
O peixe tem que cansar\\*
Porém a luta é pesada

Em cima da terra o homem\\*
Pode saltar e correr\\*
Porém em cima das águas\\*
Isto não pode fazer\\*
Já portanto o pescador\\*
É o maior lutador\\*
Que se pode conhecer

Como o pescador não há\\*
Classe que seja tão forte\\*
Esta certeza foi vista\\*
No Rio Grande do Norte\\*
Por dois homens pescadores\\*
Que dos mares seus horrores\\*
Não os fez pensar na morte

Pois esses dois pescadores\\*
Um do outro companheiro\\*
Subiram do Rio Grande\\*
Para o Rio de Janeiro\\*
Em jangadas de pescar\\*
Isto fez admirar\\*
O nosso Brasil inteiro

Entenderam de assistir\\*
A festa do centenário\\*
Feita em vinte e dois, no Rio:\\*
Mas não tendo o necessário\\*
Para irem num navio\\*
Em jangadas para o Rio\\*
Partiram, sem comentário

Quando chegaram no Rio\\*
Assombrou-se a grande massa\\*
Do povo que ali se achava\\*
E muitas gentes da praça\\*
Já para vê-los corriam\\*
Pois muitos inda entendiam\\*
Que a notícia era uma graça

\end{verse}

\chapter[Discussão de José Duda com João Athayde]{Discussão de José Duda com João Athayde --
Descrevendo todos os nomes próprios
masculinos, todas as aves ou pássaros
todos os peixes dos rios e do mar.}

\begin{verse}

Zé Duda estava cantando\\*
Uma noite em Limoeiro\\*
Quando bateu oito e meia\\*
Chegou o trem passageiro,\\*
Nesse trem vinha Athayde\\*
Mais outro seu companheiro

Athayde ia chegando\\*
Inda não tinha saltado;\\*
De onde estava Zé Duda,\\*
Chegou depressa um chamado;\\*
Sem falta fosse até lá\\*
Se não estivesse ocupado

Athayde nessa hora\\*
Em nada estava pensando\\*
Seguiu com o portador\\*
Pra ver quem estava chamando\\*
Encontrou na dita casa\\*
Zé Duda velho cantando

Aí o dono da casa\\*
Disse a ele: ``Pode entrar''\\*
Mostrando-lhe uma cadeira\\*
Mandou ele se sentar\\*
``Vou dizer qual o motivo\\*
Por que mandei lhe chamar''

Eu sabendo que o senhor\\*
É homem que sempre estuda\\*
Creio que seu repertório,\\*
Na poesia não muda\\*
Quero ouvir uma peleja\\*
Entre o senhor e Zé Duda

``Amigo, é muita verdade\\*
Eu gosto da poesia\\*
Porém isto é diferente\\*
Dos termos da cantoria\\*
É difícil, eu com Zé Duda\\*
Entrarmos numa porfia

Zé Duda rompeu dizendo:\\*
``O nosso amigo é quem quer\\*
Eu não enjeito a parada,\\*
Dê o caso no que der\\*
Previna seu repertório\\*
Pode vir como quiser''

A -- É este o primeiro assunto\\*
Pra discutir com você\\*
Descrevendo os nomes próprios\\*
Por meio de um ABC\\*
Sendo todos masculinos\\*
Desde o A até o Z

Z -- Aprígio, Alonso, Adriano\\*
André, Afonso, Ambrosino\\*
Alexo, Abel, Anastácio,\\*
Abílio, Adolfo, Agripino\\*
Ambroziolo, Anacleto\\*
Adamastor e Albino

A -- Antônio, Augusto, Agnelo\\*
Anselme, Anísio e Adão\\*
Ageu, Alcanjo, Aniceto\\*
Antero, Anito, Abraão\\*
Amon, Assis, Atanásio,\\*
Adalberto, Absalão

Z -- Alexandrino, Aristeu\\*
Aureliano e Alvim\\*
Arnobio, Alípio, Argemiro\\*
Amaranto e Aladim\\*
Apolinário, Algiberto\\*
Agapenor, Amorim

A -- Batista, Boaventura,\\*
Bianor, Berta e Balbino,\\*
Bartolomeu e Benício\\*
Basílio, Braz e Bertino\\*
Bento, Belo, Bonifácio\\*
Bonaparte e Brasilino

Z -- Brocassiano e Berilo\\*
Benedito e Baldoíno,\\*
Belchior, Bruno, Bernardo\\*
Benoni, Barão, Belmino\\*
Belisário e Banderico\\*
Bevenuto e Belarmino

A -- Calixto, Cleto e Caim\\*
Cacasseno e Coriolano\\*
Castor, Calino, Conrado\\*
Custódio, Cristo e Caetano\\*
Camilo, Cláudio, Canuto\\*
Constantino e Cipriano

Z -- Corinto e Constanciano\\*
Cantídio e Capitulino\\*
Cincinato e Claudimiro\\*
Carmelitano e Carvino\\*
Caro, Cupido e Colombo\\*
Cosmogênio e Carolino

A -- Doroteu, Ducas, Davino\\*
Duarte e Deocalião\\*
Domiciano e Deocrécio,\\*
Didiu, Donato e Durão\\*
Deocleciano e Delmiro\\*
Dagoberto e Damião

Z -- Com o Dinda soletro,\\*
Damásio e Diamantino,\\*
Divo, Dante e Deodoro\\*
Damiano e Durvalino\\*
Deodato e Domitilo\\*
Durval, Daniel, Delfino

A -- Eurico, Egeu, Elisbão\\*
Esculápio e Ernestino\\*
Elói, Edmundo, Elviro\\*
Estácio, Ernesto e Elmino,\\*
Evaristo, Eliotério,\\*
Epitácio, Evangelino

Z -- Felinto, Floro, Fiúza\\*
Fabriciano e Faustino,\\*
Felomeno e Ferrabraz,\\*
Faraó, Fábio e Fermino,\\*
Felonilo e Felizardo,\\*
Felisberto e Francilino

A -- Com F também soletro\\*
Felesmino e Floriano\\*
Felipe, Frazãoe Fausto,\\*
Francisco e Feliciano\\*
Felisberto e Frederico\\*
Furtunato e Fabiano

Z -- Gaspar, Genésio e Gonçalo\\*
Gastão, Gustavo e Galdino\\*
Galileu, Golo e Gonzaga\\*
Graciano e Genuíno\\*
Geror, Gilberto, Geraldo,\\*
Godofredo e Guilhermino

A -- Com o G também soletro\\*
Getúlio, Gentil, Gusmão\\*
Gregório, Gino e Gonçalves\\*
Giminiano e Galvão,\\*
Grigoriano e Gaudêncio,\\*
Garibalde e Gedeão

Z -- Honório, Hugo e Humberto\\*
Herodiano e Hermino,\\*
Hortêncio, Hircano e Horácio\\*
Heliberto e Heroíno,\\*
Heliciano, Hilário\\*
Honorato e Humbelino

A -- Com o H inda soletro\\*
Heveu, Henrique e Hermano\\*
Heriseu, Hermenerico\\*
Helianeu, Hortulano,\\*
Herundino, Heleodore,\\*
Hermenegildo, Herculano

Z - Izidro, Inácio, Inocêncio\\*
Irineu, Igo, Itervino\\*
Ivanoel, Izidoro\\*
Itmoneu, Ilissino\\*
Idílio, Ismael, Isaac\\*
Idomeneu, Idalino

A -- Josué, Justo e Juvêncio\\*
Justino e Joviniano\\*
Júlio, Joel, Januário\\*
José, Joaquim, Juliano\\*
Jerôncio, Judas, Jacinto\\*
Jazaviel, Jovino

Z -- Com o J inda soletro\\*
Jacob, Julião, Juvino\\*
Jesus, Jordão e Jovito\\*
Juvenal e Joventino\\*
Jeová, Jano e Juberto\\*
Juvenciano e Jozino

A -- Kapilas, Kéan, Kanaris\\*
Ketelaer e Keivino\\*
Kalidaza e Kalakua\\*
Kenuel e Kedovino\\*
Kentigerno e Koributh\\*
Kenibaldo e Kenerino

Z -- Leandro, Lídio, Lucrécio\\*
Lucas, Lindolfo, Livino\\*
Lúcio, Lionel, Libânio\\*
Lafaiete e Laurentino\\*
Lactor, Liberal, Lupércio\\*
Luminato e Landelino

A -- Com L ainda soletro\\*
Lupercino e Lauriano\\*
Leão, Lourival, Liberto,\\*
Lamartine e Luciano\\*
Ladislau, Lívio e Leôncio\\*
Leopoldino e Lusitano

Z -- Marco, Militão,Macedo,\\*
Materno, Manso e Miguel\\*
Moisés, Milano, Marinho\\*
Molisberto e Maciel\\*
Mendes, Macário, Mimoso\\*
Maomé e Mizael

A -- Com M também soletro\\*
Manoel, Modesto, Malvino\\*
Mário, Monetor, Murilo\\*
Mariano e Mareulino\\*
Marim, Mourão, Malaquias\\*
Mundoaldo e Minervino

Z -- Nazário, Nilo, Noêmio,\\*
Natal, Nicolau, Nabor\\*
Nereu, Nicácio, Nabuco.\\*
Nero, Narciso, Nestor\\*
Napoleão, Nazareno\\*
Nascimento e Nicanor

A -- Oscar, Olímpio, Ozário\\*
Odorico e Olindino\\*
Olinto, Odilon, Otelo\\*
Ogênio, Orfeu, Osvalino\\*
Otaviano, Olegário\\*
Odeberto e Ormandino

Z -- Pedro, Plutão e Pilatos,\\*
Ponciano e Prometeu\\*
Philomeno e Peregrino\\*
Pio, Pacato e Peleu\\*
Polenciano e Prescílio\\*
Pirro, Petrarca e Pompeu

A -- Com o P ainda soletro\\*
Pascoal, Penor e Paulino\\*
Prudente, Paulo, Procópio\\*
Ponsidônio e Pastorino\\*
Putifar, Pires, Patrício\\*
Pantalião, Pergentino

Z -- Com o Q também soletro\\*
Quaciano e Questorino\\*
Quiliano e Quatremero\\*
Questor, Queiroz e Quintino\\*
Quartim, Quevedo, Quixote\\*
Quintiliano e Quirino

A -- Raul, Rafael, Renato\\*
Roque, Rozendo, Rafino\\*
Rangel, Ramiro e Roberto\\*
Renê, Ricardo, Rolino\\*
Ramoaldo e Rigoberto\\*
Renovato e Rosalino

Z -- Com o R ainda soletro\\*
Reno, Raimundo, Roldão\\*
Romano, Rigo, Rogério\\*
Romeu, Rodolfo e Romão\\*
Reginaldo e Roderico,\\*
Renoberto e Rodião

A -- Samuel, Silva e Sotero,\\*
Santo, Sampaio e Simão\\*
Salazar, Silo, Sinésio\\*
Severino e Salomão\\*
Salviano e Saturnino,\\*
Sigismundo e Simião

Z -- Com o S também soletro\\*
Suzano, Sérgio e Santino\\*
Sansã0, Saldanha, Sêneca\\*
Severiano e Salvino\\*
Salomé, Sancho, Saturno\\*
Salustiano e Sabino

A -- Tasso, Tomé e Trindade\\*
Thomas e Tertuliano\\*
Tancredo, Teles, Tobias\\*
Toni, Tibárcio, Toscano\\*
Tamandaré, Teodorico\\*
Timoleão e Trajano

Z -- com o T inda soletro\\*
Tadeu, Timóteo, Targino\\*
Teotônio, Tigre, Torquato\\*
Taciano e Tributino\\*
Tandile, Teles, Teobaldo\\*
Teodomiro e Tranquilino

A -- Urbano, Urico e Umberto\\*
Unibaldo e Ursulino\\*
Urbiciano e Urquiza\\*
Uriel, Urso e Urbino\\*
Ulipiano, Ubarico\\*
Ubirajara, Ubaldino

Z -- Vital, Ventura, Vulcano\\*
Vespaziano e Virgino\\*
Victor, Vicente, Venâncio\\*
Vasco, Vidal, Vitorino\\*
Viracundo e Valdimiro\\*
Viriato e Valdivino

A -- com o V também soletro\\*
Virgílio e Vitaliano\\*
Victor, Vilarim, Valente\\*
Virgolino e Yeriano\\*
Vilar, Vibaldo, Venero\\*
Venceslau, Valeriano

Z -- Para o X não temos nome\\*
Direi aqueles que houver\\*
Xuto, Xenacho, Xafrido,\\*
Xenofanes e Xilander\\*
Xadrias e Xenócrito\\*
Xinodóio e Xavier

A -- com o Z também soletro\\*
Zito; Zebedeu, Zulino\\*
Zacarias e Zoroastro\\*
Zoé, Zanor e Zodino\\*
Zambuial, Zino e Zulmiro\\*
ZorobabeI, Zifirino

Z -- Athayde eu te conheço\\*
Como poeta cantor\\*
Por conhecer tua força\\*
Te peço como favor\\*
Vamos descrever as aves\\*
Do colibri ao condor

A -- Pavão, peru e galinha\\*
Pelicano e juriti\\*
Ema, crué e boleira\\*
Piacôco e parari\\*
Cegonha, pombo e curica\\*
Asa-branca e bem-te-vi

Z -- Perdiz, ferreiro e tucano\\*
Gaivota e gurinhatã\\*
Cisne, carão e sovela\\*
Sirigaita e araquã\\*
Corvo-marinho e bicudo\\*
Rabo branco e jaçanã

A -- Garça, guicé, viúva\\*
Faisão-dourado e condor\\*
Sereno, abutre e marreca\\*
João-de-barro e serrador,\\*
Codorniz, cuco ejandaia\\*
Tico-tico e beija-flor

Z -- Jacutinga e margarida\\*
Maracanã e socó\\*
Lira, cauã, solitário\\*
Rabijunco e noitibó\\*
Periquito e papagaio\\*
Patativa e curió

A -- Peito-celeste e canário\\*
Cegonha-branca e sofrê\\*
Anabato e jacupemba\\*
Galo-da-serra e gonguê\\*
Barriga-negra e barbudo\\*
Pinta-roxa e zabelê

Z -- Grajaú e carrapateiro,\\*
Águia, pompeu e jacu\\*
Fura barreira e pintada,\\*
Avestruz, ganso e nambu\\*
Alfaiate e feiticeiro\\*
Quebra-osso e jaburu

A -- Melro, mutum e barreiro\\*
Pardal, montês e guará\\*
Macó, chibé, figo-louro\\*
Jaru, kaci, guaraná\\*
Urubu-rei, palachina\\*
Maçarico e sabiá

Z -- Perdigão, pita e gasola\\*
Papa-capim, anagé\\*
Gaio, perrocho e frizada\\*
Água-peca e bambié\\*
Colibri, galo e peitica\\*
Ganso-bravo e caboré

A -- Cardeal, corvo e carriça\\*
Largateiro e lnanequim\\*
Pardal, rouxinol, fragata\\*
Seriema e cujubim\\*
Urubu, gralha e tesoura\\*
Caga-sebo e jacamim

Z -- Alma de mestre e gulino\\*
Rola, crió e serão\\*
Trepadeira e saracura\\*
Calafate e gavião\\*
Azulino e mensageiro\\*
Jipi-jipi e mergulhão

A -- Perota, polpa e tirano\\*
Luca, xaréu, acará\\*
Anum, pato e favorita\\*
Giva,cocó, biguá\\*
Barba-azul, mocho, araruna\\*
Boa-noite e guanamá

Z -- Cagacibito e burguesa\\*
Bacurau, grau e azulão\\*
Vru, caraúna e arara\\*
Concri, coruja e cancão\\*
Andorinha e lavandeira\\*
Papa-mosca e potrião

A -- As aves do paraíso\\*
De que nos faltou falar\\*
Outras de nome estrangeiro\\*
Que nã~ se pode rimar\\*
O abutre do Egito\\*
Katete e frango-da-mar

Z -- Para um trabalho insano\\*
Inda vou te convidar\\*
Você se diz competente\\*
Pode bem me acompanhar\\*
Vamos descrever os nomes\\*
Dos peixes que tem no mar

A -- Xaréu, baleia, cavala\\*
Carapeba e tubarão,\\*
Boto, sioba, navalha\\*
Bagre, viola e cação\\*
Mero, tainha, pescada\\*
Pata-rocha e bodião

Z -- Prego, salmão e rosada\\*
Piraíba, aruanã\\*
Morêa, solha, piaba\\*
Peixe-boi, curimatã\\*
Papa-terra e bicudinha\\*
Boca-mole e corimã

A -- Galé, bacalhau, bicuda\\*
Cigarra, polvo, delfim\\*
Balista, mola e biquara\\*
Dentão e camurupim\\*
Cabricunha e cuspe-cuspe\\*
Bailadeira e camurim

Z -- Barboto, salveI, piranha\\*
Biluca, baga e pacu\\*
Lavadinha e tintureira\\*
Jurupoca e camuru\\*
Pacupeba e pirarara\\*
Peixe-porco e baiacu

A -- Pirarucu e ferreiro\\*
Carovina e roncador\\*
Tromba, golfinho, donzela\\*
Bicançuda e ralhador\\*
Tagona, tuca e barbeiro\\*
Hipocampo e voador

Z -- Coió, misilão, cascudo\\*
Bico, dourado, mandi\\*
Carapau, lúcio e cutelo.\\*
Jula,cavaco e mugi\\*
Acaraúna e lanceta\\*
Peixe-serra e tamhaqui

A -- Frango-do-mar e torpedo\\*
Juliana e jacundá\\*
Alvacora e peixe-aranha\\*
Pampo-lixo e peroá\\*
Vintém, camboto e agulha\\*
Castanheta e jundiá

Z -- Gato, muçú e sardinha\\*
Gabião, freira e bacu\\*
Bonito, manta e arraia\\*
Louva-deus, peranambu\\*
Sirigado e carapeba\\*
Douradinha e timucu

A -- Garoupa e sapata-preta\\*
Esturjão e mobi\\*
Lampreia, frade e carepa\\*
Sapateiro e mavali\\*
Carapó, serra e coitada\\*
Patambeta e lambari

Z -- Arabaiana e boquinha\\*
Dragão e piramutá\\*
Briamente e salmonejo\\*
Zimbo, litão e cará\\*
Caico, lula e traíra\\*
Barriguinha e cambotá

A -- Já conheci teu talento\\*
Na teoria na prática\\*
Vamos entrar num tratado\\*
Da física matemática\\*
Segundo a gravitação\\*
Na ciência pneumática

Z -- Eu nunca fui titulado\\*
Como rara iriteligência\\*
Porém só digo uma coisa\\*
Quando tenho consciência\\*
Eu discuto com você\\*
Qualquer ramo de ciência

A -- Zé Duda é muito tarde\\*
Tenho um negócio a tratar\\*
Por este motivo justo\\*
Eu não posso demorar,\\*
Deixamos a discussão:\\*
Finda-se noutro lugar

\end{verse}

\chapter{Em homenagem às mulheres}

\begin{verse}

Vou descrever a mulher,\\*
Este arcanjo idolatrado,\\*
Cupido, o deus do amor,\\*
Vive a ela associado,\\*
Tem afeto e eloquência\\*
No céu da nossa existência,\\*
Tem ela um trono firmado

A natureza esmerou-se\\*
Em fazê-la assim formosa,\\*
Deu-lhe o cabelo tão lindo,\\*
A boca muito mimosa,\\*
A sua face corada,\\*
Ao romper da madrugada,\\*
Parece um botão de rosa

O sorriso da mulher,\\*
Não pode haver descrição,\\*
É justamente do riso,\\*
Que desabrocha a paixão,\\*
Os seus seios virginais,\\*
Para mim são dois punhais,\\*
Que nos fere o coração

Os olhos da mulher são\\*
Duas pedras brilhantes,\\*
São dois faróis pelo mar,\\*
A guiar os navegantes,\\*
Suas mãos são de cetim,\\*
Os seus dentes de marfim,\\*
Um colar de diamantes

O pranto de uma mulher\\*
Ninguém pode resistir,\\*
São como gotas de orvalho\\*
Serenamente a cair,\\*
Tudo fica comovido,\\*
Ali não há mais pedido,\\*
Que faça o homem sorrir

Deus pra fazer a mulher,\\*
Muito teve que lutar,\\*
Ela vive aqui no mundo,\\*
Para sofrer e amar,\\*
O homem mais valentão,\\*
Tem de pedir-lhe o perdão,\\*
Ante ela se curvar

Pois a mulher é um anjo,\\*
Que bem merece atenção,\\*
É ela quem nos consola\\*
Nas horas de ingratidão,\\*
Por nós o pranto derrama,\\*
E com carinho nos chama\\*
Filho do meu coração

O homem não sabe dar\\*
O merecido valor\\*
A esse ser sacrossanto,\\*
Que nos trata com fervor,\\*
E ela a mulher querida,\\*
Que expõe a própria vida,\\*
Em troco do nosso amor

Como uma flor no jardim,\\*
A mulher nasce no mundo,\\*
Tem beleza e tem primor,\\*
O seu perfume é fecundo,\\*
Desabrocha num momento,\\*
Também desfolha-se ao vento,\\*
Murchando assim num segundo

Sendo a mulher virgem ainda\\*
É corpo a flor em botão,\\*
Quando o dia vem nascendo,\\*
Ao soprar da viração,\\*
Ela ali no verde galho,\\*
Toda banhada de orvalho,\\*
Tem encanto e atração

Vem um dia o beija-flor,\\*
Dar-lhe um beijo com ternura,\\*
A flor pendida desmaia,\\*
Perdendo a sua candura,\\*
As pétalas caem no chão,\\*
Nascendo no coração,\\*
O sopro da desventura

O homem vive no mundo,\\*
Só na mulher a pensar,\\*
Quando chega aos vinte anos,\\*
O seu desejo é casar,\\*
Procura uma namorada,\\*
Meiga, formosa, educada,\\*
Que seu nome possa honrar

Ele tem muita razão,\\*
Em pensar dessa maneira,\\*
Procurando antes de tudo\\*
A querida companheira,\\*
Porque viver sem mulher,\\*
A vida perde o mister,\\*
É uma tristeza inteira

E mesmo o rapaz solteiro,\\*
Leva um viver desgraçado,\\*
Não tem um lar que descanse,\\*
Quando se acha enfadado,\\*
E seja lá como for,\\*
Jamais terá o valor,\\*
Que tem o homem casado

Se o pobre mora em castelo,\\*
É tudo desarranjado,\\*
A roupa suja num canto,\\*
O lixo ali ao seu lado,\\*
E quando quer passear,\\*
Ralha por não encontrar,\\*
O terno branco engomado

Não há dinheiro que chegue,\\*
Para fazer a despesa,\\*
O ordenado do mês,\\*
Vai embora com certeza,\\*
Além de tudo esse pobre,\\*
Esperdiçou o seu cobre,\\*
Vivendo assim na pobreza

Se ele fosse casado,\\*
Teria a mulher amada,\\*
Que lhe cuidasse de tudo,\\*
Não faltaria mais nada,\\*
Melhorava a sua vida,\\*
Tinha conforto e guarida,\\*
A sua roupa engomada

Antigamente a mulher,\\*
Somente em casa vivia,\\*
Trabalhándo na cozinha,\\*
Nas obrigações do dia,\\*
E passava a vida inteira,\\*
No seu lar prisioneira,\\*
Pra canto algum não saía

O homem não confiava,\\*
O seu trabalho à mulher,\\*
Julgando que ela fosse\\*
Um ser de pouco mister,\\*
Isso foi tempo passado,\\*
Porém hojé tem mostrado,\\*
Que faz tudo quanto quer

Agora vemos mulheres,\\*
Em suas obrigações,\\*
Trabalhando igual ao homem,\\*
Em várias colocações,\\*
Engrandecendo a nação,\\*
A pátria do coração,\\*
São as suas pretensões

Temos mulheres formadas\\*
Em comércio e medicina,\\*
Farmácia, Direito e tudo\\*
Quanto a ciência hoje ensina,\\*
Portanto a mulher de agora,\\*
Não é aquela de outrora,\\*
Seu valor já predomina

A mulher considerava-se\\*
Ao homem inferior,\\*
Mas no decorrer do tempo,\\*
Tem mostrado o seu valor\\*
Deu provas de inteligente,\\*
E que não nasceu somente,\\*
Para sofrer o amor

Até na luta ela tem\\*
Mostrado a sua façanha,\\*
O soldado vai à guerra,\\*
Sua mulher acompanha,\\*
Enfrenta todo perigo,\\*
Sem temer o inimigo,\\*
Nos horrores da campanha

Então no furor da guerra,\\*
Quando ribomba a metralha,\\*
Ela trata dos feridos,\\*
Pelos campos de batalha,\\*
Com muito zelo e cuidado,\\*
Cumprindo um dever sagrado\\*
Essa heroína trabalha

O soldado que levou\\*
Um ferimento no peito,\\*
Lutando em nome da pátria,\\*
Defendendo o seu direito,\\*
Tendo a mulher a seu lado,\\*
Sendo por ela tratado,\\*
Se morrer vai satisfeito

Agora caros leitores,\\*
Sobreo beijo vou falar,\\*
O beijo de uma mulher,\\*
Muito goza quem o levar,\\*
Sendo ele demorado,\\*
Da boca de um anjo amado,\\*
Que delícia faz causar!

Na terra o beijo contém,\\*
Um gosto excelso e fecundo,\\*
Embriaga o coração,\\*
Tem um perfume profundo,\\*
Quem um beijo assim levou,\\*
Também experimentou,\\*
O maior prazer do mundo

Se o beijo fosse comprado,\\*
Que fortuna não valia?\\*
Com certeza muito caro,\\*
Em toda parte seria,\\*
Pra quem quisesse comprar,\\*
Havia então de gastar,\\*
Muito dinheiro hoje em dia

Porém o beijo na vida,\\*
Qualquer um pode gozar,\\*
Dinheiro algum não precisa,\\*
Da algibeira ele tirar,\\*
Quem quiser experimente,\\*
Amando sinceramente,\\*
Só assim pode beijar

Do beijo nasce a paixão,\\*
Da paixão surge o amor,\\*
Se nada disso existisse,\\*
Tudo era pranto, era dor,\\*
O mundo era um deserto,\\*
De luto todo coberto,\\*
Nada teria valor

O homem não trabalhava,\\*
Nem ligava a sua vida;\\*
Porque não tinha na terra,\\*
A companheira querida,\\*
Não precisava lutar,\\*
Para viver sem amar,\\*
Sua luta era perdida

O homem que é casado,\\*
E tem família também,\\*
É quando pode saber,\\*
O valor que mulher tem,\\*
Vendo os grandes empecilhos,\\*
Que sofre pelos seus filhos,\\*
Para guiá-los ao bem

A nossa querida mãe,\\*
Nos seus braços nos criou,\\*
Grandes tormentos na vida,\\*
Por nossa causa passou,\\*
Quando ainda pequeninos,\\*
Seus santos seios divinos,\\*
Foi quem nos amamentou

Mesmo sendo muito pobre,\\*
A nossa mãe adorada,\\*
Nunca despreza o seu filho\\*
Morre com ele abraçada,\\*
Grande exemplo de carinho,\\*
Que tem para o seu filhinho,\\*
Sofrendo assim consolada

Mãe! palavra sacrossanta,\\*
Que devemos venerar,\\*
Bálsamo feito de luz,\\*
Para nos suavizar,\\*
Consolo dos desgraçados,\\*
Alívio dos desprezados,\\*
Hóstia santa no altar

A mulher quando é mãe,\\*
Seu amor é singular,\\*
Pelo filho estremecido,\\*
Luta sem nunca cansar,\\*
E passa uma noite inteira,\\*
Como uma santa enfermeira,\\*
Para o mesmo acalentar

Se o filho viver alegre,\\*
Ela está também contente,\\*
Porém se ele está triste,\\*
Ela sofre horrivelmente,\\*
Amor de mãe é suave,\\*
É como o gorjeio d'ave,\\*
Nas horas do sol nascente

Tudo passa sobre a terra,\\*
Com toda velocidade,\\*
Somente o amor de mãe,\\*
Reina toda a eternidade,\\*
Tudo definha e se esquece,\\*
Só ele nunca fenece,\\*
No seio da humanidade

Vem em segundo lugar,\\*
A nossa esposa adorada,\\*
Por ela é que nós deixamos\\*
Nossa primeira morada,\\*
Pois uma esposa exemplar\\*
É quem nos-pode guiar,\\*
Nesta vida amargurada

O homem abandona o lar,\\*
Para com ela viver,\\*
Pois a esposa consola,\\*
Nos momentos de sofrer,\\*
Esposa santa e querida,\\*
Anelo de nossa vida,\\*
Por ti devemos morrer

Desde o primeiro momento,\\*
Que Deus criou a Adão,\\*
Deu-lhe o santo paraíso,\\*
Pra nele viver então,\\*
Mas ele vivia triste,\\*
Dizendo ninguém resiste,\\*
Viver nessa solidão

Então Deus o vendo assim,\\*
Cheio de mágoa e tristeza,\\*
Ofereceu-lhe os tesouros\\*
Mais raros da natureza,\\*
E Adão agradecia,\\*
Dizendo que não queria,\\*
Aquela imensa riqueza

O Criador conhecendo,\\*
Que nada tinha mister,\\*
Disse então: ``Escolhas tudo!\\*
Que teu desejo quiser...''\\*
Disse Adão: ``Dessa maneira\\*
Preciso uma companheira'';\\*
E Deus formou a mulher

A morena brasileira,\\*
Possui todos os sinais\\*
Da formosura mais rara,\\*
Das mulheres mundiais,\\*
N as horas do arrebol,\\*
Parece os raios do sol,\\*
Quando beija os matagais

Quem vir uma brasileira,\\*
Com seu olhar fascinante,\\*
Fica logo apaixonado\\*
Pelo seu lindo semblante,\\*
Nunca mais pode esquecê-la\\*
Procura tornar a vê-la,\\*
Toda hora, todo instante

Qualquer um que se casar,\\*
Com tão bela criatura,\\*
Vai encontrar uma vida,\\*
Cheia de gozo e ventura,\\*
Porque toda brasileira,\\*
Tem o dom de feiticeira;\\*
Tem perfume e tem ternura

Quem viajar no sertão,\\*
Terá o gosto de ver\\*
A morena sertaneja,\\*
Sem a moda conhecer,\\*
Muito galante e bonita,\\*
Sem pó, sem carmim, sem fita,\\*
Na face o sangue a verter

Exposta ao vento e à chuva,\\*
Aos raios do sol tão quente,\\*
Trabalhando na fazenda,\\*
Fadiga alguma não sente,\\*
Corre atrás de qualquer rês\\*
E julga que nada fez,\\*
Fazendo abismar a gente

A morena brasileira,\\*
As faces cor de canela,\\*
Não pode existir no mundo,\\*
Tão formosa quanto ela,\\*
Quando é solteira ainda,\\*
É meiga, faceira e linda,\\*
Esta querida donzela

Filhos, amai vossa mãe,\\*
Com todo zelo e carinho,\\*
Porque ela é nesse mundo,\\*
Como o terno passarinho,\\*
No galho do pau florido,\\*
Entre os ramos escondido,\\*
Acolhe o filho no ninho

Esposo, amai vossa esposa,\\*
Esta fiel companheira,\\*
Que nos consola o sofrer\\*
Desta vida passageira,\\*
Pois assim tereis cumprido,\\*
Deveres de bom marido,\\*
Fazendo desta maneira

Mulher, tu és sobre a terra,\\*
O símbolo da perfeição,\\*
Eu tenho um altar construído\\*
Dentro do meu coração,\\*
Aonde irei ajoelhado,\\*
Anjo santo idolatrado,\\*
Fazer minha devoção

Leitores vou terminar\\*
A pequena descrição,\\*
Se não saiu do agrado,\\*
A todos peço perdão.\\*
Porque está pra nascer,\\*
Quem venha bem descrever,\\*
A mulher com perfeição

\end{verse}

\chapter{O retirante}

\begin{verse}

É o diabo de luto\\*
No ano que no sertão,\\*
Se finda o mês de janeiro\\*
E ninguém ouve trovão\\*
O sertanejo não tira,\\*
O olho do matulão

E diz à mulher:\\*
``Prepare o balaio,\\*
Amanhã eu saio\\*
Se o bom Deus quiser,\\*
Arrume o que houver\\*
Bote em um caixão\\*
Encoste o pilão\\*
Onde ele não caia\\*
Arremende a saia\\*
Bata o cabeção

Se meu padrim padre Cícero\\*
Quiser me favorecer,\\*
Eu garanto que amanhã\\*
Quando o sol aparecer\\*
Nós já sabemos da terra\\*
Onde ache o que comer

Vá logo ao chiqueiro\\*
Amarre a cabrinha,\\*
E mate a galinha\\*
Que está no terreiro\\*
Leve o candeeiro\\*
E duas panelas\\*
Arrume as tigelas\\*
E se tiver xerém\\*
Cozinhe o que tem\\*
Prepare as canelas''

E lá vai de estrada afora\\*
O velho com um matulão,\\*
Um chapéu velho de couro\\*
Uma calça de algodão\\*
Com uma enxada no ombro\\*
Dizendo adeus ao sertão

Já não tem mais força\\*
Vista muito menos,\\*
Dez filhos pequenos\\*
Quinze filhas moças\\*
Faltando-lhe as onças\\*
Além de não ver\\*
Ao ponto de ter\\*
Três filhos mamando\\*
Quatro se arrastando\\*
Cinco por nascer

Diz o velho: ``Minhas filhas\\*
Não era do meu desejo\\*
Eu ir degredar vocês\\*
Na terra dos carangueijos\\*
O Sul presta para tudo,\\*
Menos para sertanejo

Tem naqueles matos\\*
Um tal maruim,\\*
Filho de Caim,\\*
Neto de Pilatos\\*
E os carrapatos\\*
Mordem que faz pena\\*
Muriçoca em cena\\*
Com um canto grego\\*
Só música de negro\\*
Em tempo de novena''

Partem qual Eva e Adão\\*
Partiram do paraíso\\*
Não há um lábio entre tantos\\*
Que se veja nele um riso\\*
Se despedindo um dos outros,\\*
Até dia do juízo

E chega a ranchada\\*
Ao senhor de engenho\\*
Diz o velho: ``Eu tenho\\*
Esta filharada\\*
Família pesada\\*
E não tenho jeito\\*
Preciso e aceito\\*
Qualquer sacrifício\\*
Não tenho um ofício\\*
Vou cair no eito''

O senhor de engenho olha\\*
E vê gente em quantidade,\\*
Meninos de doze anos\\*
Até três meses de idade\\*
Inda o velho diz: ``Meus filhos\\*
Morreram mais da metade

Só em Juazeiro\\*
Tem doze enterrados,\\*
Fora os enjeitados\\*
uni inda solteiro\\*
Meu filho primeiro\\*
Também já morreu\\*
Desapareceu\\*
Outro pequenino\\*
E fora um menino,\\*
Que a onça comeu''

O senhor de engenho\\*
Vê mais de cem na estrada\\*
Umas moças, outras chegando\\*
É grande rapaziada\\*
A velha com a barriga\\*
Que chega vem empinada

O dono da terra\\*
Vê aquela tropa,\\*
Que só a Europa,\\*
Em tempo de guerra\\*
Ali não se encerra\\*
O grupo que tem;\\*
Atrás inda vem\\*
Fora o que ficou\\*
Os que lá deixou,\\*
Os que deu a alguém

Deu dezoito ao padre Cícero\\*
E quinze espalhou por lá;\\*
E uns dezesseis ou vinte\\*
Andam pelo Ceará;\\*
E na barriga da velha?\\*
Quem sabe quantos terá?

Ela de uma vez\\*
Que se confessou,\\*
Num dia abortou\\*
Bem uns cinco ou seis\\*
Devido um freguês\\*
Que teve uma briga\\*
Formando uma intriga\\*
Por um crime injusto\\*
Ela tevé um susto,\\*
Perdeu a barriga

Ela no mês de São João\\*
Teve Vicente e André\\*
Em julho teve Paulina\\*
Em agosto, Salomé\\*
Em setembro teve três\\*
Bernardo, Cosmo e Tomé

Em outubro, Ana\\*
E eu não me lembro\\*
Se foi em novembro\\*
Que nasceu Joana\\*
Rita e Damiana,\\*
Nasceram em janeiro\\*
E em fevereiro\\*
Nasceu um ceguinho\\*
Quando eu ia em caminho\\*
Para o Juazeiro

Exclama o senhor de engenho:\\*
``Que carritia danada!\ldots{}\\*
Nasceram tantos num ano?\\*
Sua história está errada''\\*
``Ou xente!'' respondeu o velho\\*
``Se admira? Isto é nada!''

Mulher do sertão\\*
Indo a Juazeiro\\*
Levando dinheiro\\*
Ouvindo o sermão\\*
Vendo a procissão\\*
Que faz meu padrinho\\*
No meio do caminho\\*
Ela tem de ver\\*
Menino nascer\\*
Que só bacurinho

E lá vai aquela prole\\*
Sujeitar-se ao cativeiro,\\*
Limpar cana o dia todo\\*
Por diminuto dinheiro\\*
Fazendo dez mil promessas\\*
Ao padre de Juazeiro

Dizia em oração\\*
Divino presbítero,\\*
Santo padre Cícero:\\*
``Tenha compaixão\\*
De vosso sertão\\*
Olhai para nós\\*
Que sofrer atroz\\*
Sem se ganhar nada\\*
De trouxa arrumada\\*
Confiamos em vós

Lançai vossos olhos santos\\*
Para as almas pecadoras\\*
Ouvi os grandes gemidos\\*
Das famílias sofredoras\\*
Vêde que o senhor do engenho,\\*
Não tome nossas lavouras''

Se quereis me ajudar\\*
Que chova em janeiro,\\*
Que em fevereiro\\*
Eu possa plantar\\*
E possa voltar\\*
Não morra em caminho\\*
Vou indo sozinho\\*
E rezo num dia\\*
Dez Ave-Maria\\*
Para meu padrinho

Oh! padre santo, nos tirai\\*
Desse país de mosquitos,\\*
As noites aqui são tão feias\\*
Os dias são esquisitos\\*
Ao passo que no sertão,\\*
Os campos são tão bonitos

Amanhece o dia\\*
Aqui nessa terra,\\*
Na mata e na serra\\*
Nem um grilo chia\\*
Não há alegria\\*
Ao romper da aurora\\*
Tudo vai embora\\*
Fica a solidão\\*
Foi aqui que o cão,\\*
Perdeu a espora

No sertão às cinco horas\\*
O carão canta no Rio,\\*
E no campo a seriema\\*
Grita o tetéu no baixio\\*
Passa voando aos pulos\\*
Nos ares o corrupio

Às vezes eu babo\\*
Da ira que tenho\\*
O senhor de engenho\\*
Tem um tal de cabo\\*
Esse é o diabo\\*
Pior que um dragão\\*
Eu faço tenção,\\*
De um dia pegá-lo\\*
Mandar encabá-lo\\*
Na foice do cão

Uma é ver outra é contar\\*
O Diabo corrio é,\\*
Como o cachorro do mal\\*
Desesperado da fé\\*
Ontem jurou de quebrar\\*
O cachimbo da muié''

Eu disse: ``Provoque\\*
Que eu agaranto\\*
Não haver um canto\\*
Que você se soque\\*
E se quiser toque\\*
No cachimbo dela\\*
Pra ver como ela\\*
De que jeito fica\\*
E se você não estica,\\*
Agora a canela''

A muié já não é boa\\*
No eito o sol esquentando,\\*
Um toco preto atrás dela\\*
Como quem está esporando\\*
Dizendo: ``Aqui está mal limpo''\\*
E de hora em hora falando

Além do sol quente\\*
Vem o cão de um negro\\*
Da cor de um morcego\\*
Perturbando a gente\\*
Nunca vi um ente\\*
Como o negro é\\*
Eu disse com fé:\\*
``Quer ver meu carimbo?\\*
Toque no cachimbo\\*
Da minha muié''

Ora, um pobre que trabalha\\*
No eito a semama inteira\\*
Depois que sai do serviço.\\*
Ir procurar macaxeira\\*
Pra cozinhar e comer,\\*
Com chá de erva-cidreira

Depois de cear\\*
Sentado no chão,\\*
Ao pé do fogão\\*
A se lastimar\\*
Onde vai falar\\*
Da grande pobreza\\*
E tendo a certeza.\\*
De findar na desgraça,\\*
Aquela fumaça\\*
É a sobremesa

O desgraçado do cabo\\*
Não deixa a gente fumar\\*
Porque disse que cachimbo\\*
Empata de trabalhar\\*
Minha muié acendendo\\*
Ele jura de quebrar

Naquele paul,\\*
É um mosquiteiro,\\*
Pior que um chiqueiro\\*
Nas casas do sul\\*
Quem já: vem azul\\*
Com fome e cansado\\*
Além de arranhado\\*
No mocambo entrou,\\*
Porém encontrou\\*
O fogo apagado

E o cabo agora\\*
Ali encostado\\*
Num pau escorado\\*
Gritando: ``Vambora!\\*
Avie e isso fora\\*
Não há outro jeito\\*
Levante sujeito!\ldots{}\\*
Que demora é essa?\\*
Almoço só presta,\\*
É mesmo no eito!''

Se ele for para o lado\\*
Onde tem um fogo feito\\*
``Onde vai?'' pergunta o cabo\\*
Um pouco mal satisfeito\\*
``Você se empalhando assim\\*
Está atrasando o eito''

É o resultado\\*
Do pobre que vem\\*
Sem nem um vintém\\*
E desarranchado\\*
Não acha um danado\\*
Que a porta lhe abra\\*
Que sorte macabra\\*
Com filhos demais\\*
A mulher atrás\\*
Puxando uma cabra

\end{verse}
