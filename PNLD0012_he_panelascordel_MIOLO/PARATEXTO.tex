\chapter{Vida e obra de Oliveira de Panelas}

\section{Sobre o autor}

Oliveira Francisco de Melo, ou Oliveira de Panelas, poeta, 
cantador e violeiro, nasceu em 24 de maio de 1946 no sítio 
Barroca, na cidade de Panelas, interior de Pernambuco. 
Começou a cantar versos ainda
na infância, com nove anos. Sem concluir a escola primária, Panelas
alfabetizou"-se lendo os folhetos de diversos cordelistas. Aos doze anos
passou a acompanhar o cantador Zé Rufino em suas andanças pelo interior
do Nordeste, trabalhando como ``estagiário'' desse outro
importante poeta. O estágio durou dois anos e com catorze anos Oliveira de Panelas
já percorria
as feiras de Pernambuco, Alagoas, Paraíba e Sergipe cantando seus
próprios versos. Em 1962 estabeleceu"-se em Garanhuns, no interior
de Pernambuco, onde fazia poemas e trabalhava como pedreiro para
complementar sua renda. Como muitos outros nordestinos, em 1972 veio a
São Paulo em busca de trabalho. Aqui cantou nas cantinas do Brás, Bom
Retiro e Bexiga, e foi um dos principais organizadores do I Congresso
de Repentistas Nordestinos de São Paulo. Frente a sua intensa atividade,
foi convidado por Otacílio Batista para retornar ao Nordeste como
cantador, chegando a João Pessoa em 1975 com o parceiro que o
acompanharia por mais de vinte anos. Panelas realizou trabalhos para o
cinema e a televisão, representou a poesia popular em salões de
literatura ao redor do mundo e cantou para diversos chefes políticos.
Sua poderosa voz e sua habilidade como violeiro ainda lhe renderam o
título de ``Pavarotti dos Sertões''.


\section{Sobre a obra}

Autodidata por excelência, Oliveira transitou entre diversas tradições distintas:
em matéria de cordel, técnica e arte, aprendeu o ofício com vários mestres sertanejos; na área da cultura geral, leu sistematicamente obras"-primas da literatura como também textos importantes de
filosofia, história e geografia. Prova disso é sua habilidade em subjugar os termos e nomes mais inesperados aos rigores do metro e da rima, capaz de
debater com intelectuais a respeito da arte e da técnica de
sua profissão.

Como analisa Maurice Van Woensel a respeito de sua poesia:

\begin{quote}
Sobre qualquer assunto que lhe indicam de improviso,
inventa na hora estrofe após estrofe, todas elas repletas de
alusões e citações, eruditas e populares, e inclui,
relacionados com o assunto, nomes, termos, tudo cabendo
nos minuciosos esquemas da versificação popular. Assim,
no caso de uma estrofe de ``galope à
beira"-mar'', há onze sílabas em cada verso, seguindo o intricado
esquema de rimas \textit{abbaaccddc}. Nesta modalidade, o mote
do último verso, que sempre termina com a palavra
``\textit{mar}'', exige logo que também o sexto e o sétimo versos de cada
estrofe tenham a rima em ``-\textit{ar}''.

(\ldots)

Oliveira não tem medo de improvisar em certos metros
pertencentes à tradição antiga porém usados por poucos
colegas. Nesta antologia reproduzimos versos nesses metros
menos comuns: o ``mourão perguntado'', a
``toada alagoana'', a ``parcela ligeira'', a ``toada gemedeira'', o
``oitavão rebatido'', o ``oito pés a quadrão'' e o ``dez pés a
quadrão''.\footnote{\textsc{Woensel}, Maurice Van. ``Introdução''. In: \textsc{panelas}, Oliveira das. \textit{Cordel na escola}. São Paulo: hedra, 2010, p.\,20.}
\end{quote}

\subsection{Síntese dos poemas}

\paragraph{``O poeta se apresenta''}

Caracteriza o poeta como o ``homem das
nuvens'', capaz de viajar aos céus e manter o planeta
em seu curso, possibilitando a aproximação do ato de escrever com o
sonho, ratificando a perspectiva de que a literatura é capaz de nos
fazer viajar.
 
\paragraph{``Saco de cego''}

Analogia entre o saco de um cego, que por causa de sua falta de visão
tem sua sacola cheia de alimentos diferentes misturados e o livro de
poesias, que versa sobre temas distintos.

\paragraph{``Dedicatória''}

Poema que dedica as composições do livro a todos aqueles que de alguma
forma sofreram e apresenta suas intenções de ensinar aos que fazem os
demais sofrerem a não mais fazê"-lo.

\paragraph{``Vida e morte de Frei Damião''}

Frei Damião foi uma importante figura religiosa no Nordeste brasileiro.
Por ocasião de sua morte o poema narra sua biografia, desde seu nascimento,
em Bozanno, na Itália, até sua chegada ao céu. Nesse caminho lembra os
preceitos religiosos que o frei defendeu. O poema é composto em
sextilhas.

\paragraph{``Frutos do amor'' (martelo com mote)}

Narrativa que retoma os feitos de Frei Damião, mostrando sua erudição e
generosidade com o povo do Nordeste. O poema em décimas apresenta ao
fim de cada estrofe seu mote de que frei Damião plantou a paz, colheu
os frutos do amor e agora partiu.

\paragraph{``Nas águas do mar'' (galope à beira"-mar)}

Apresentando os feitos de Frei Damião, o poema em décimas com rimas em
``ar'' apresenta esse religioso como
um grande missionário que fez da sua missão de vida um correlato da missão de
Jesus Cristo, já que também pregou para uma gente sofrida de uma terra
árida.

\paragraph{``O santo Frei Damião'' (décimas de sete pés com mote)}

Sob o mote de que Frei Damião é um santo e já nasceu canonizado, o autor
narra a história desse italiano que veio ao sertão pregar aos pobres
para que se esqueçam da dor, da tristeza e do pranto, dessa vez
utiliza"-se de décimas com setissílabos, fazendo uma poesia de rimas
mais rápidas.

\paragraph{``Nosso Pio Giannotti''}

Após a morte de Frei Damião o autor narra numa das clássicas métricas do
cordel, a septilha, as façanhas desse homem que na região em que viveu,
mesmo depois de morto, mantém"-se como uma espécie de porto seguro da fé.

\paragraph{``Ensinamentos''}

Narrando os ensinamentos de Frei Damião, o poema exalta principalmente
seu espírito pacífico e a qualidade de seus sermões, que mesmo com o
avanço da idade não perderam força.

\paragraph{``Encontro no céu de Frei Damião e Padre Cícero''}

Frei Damião encontra na porta do céu o Padre Cícero,
que, há tempos distante da terra, quer saber das novidades. 
Assustado com as diversas mudanças, Padre Cícero fica preocupado com o futuro da
humanidade.

\paragraph{``Salve, Salve Padre Cícero''}

Padre Cícero é a mais importante figura religiosa do Nordeste. Neste
poema em sextilhas, que foi musicado para o Cd ``Padre
Cícero'', Oliveira de Panelas narra a vida desse líder desde o momento
de seu nascimento até sua morte, ressaltando sua fé, mesmo quando 
foi impedido pela igreja de exercer suas funções sacerdotais, em consequência
de sua força política.

\paragraph{``Meu padrinho viajou'' (décimas de sete com mote)}

Sob o mote de ``meu padrinho viajou/ vai retornar
brevemente'', o poema narra a importância do legado de Padre
Cícero, que mesmo depois de sua morte está no imaginário do povo nordestino.

\paragraph{``Milagres'' (décimas com mote de dois pés)}

Com um mote sobre a vitória de Padre Cícero no processo de canonização,
o poema discorre sobre os milagres que o religioso teria realizado e
que ratificariam sua canonização.

\paragraph{``Meu Padim, Padre Ciço''}

Poema em sextilhas sobre os feitos do Padre Cícero, rememorando seus
ensinamentos e garantindo que ele se encontra no céu, com Deus e Jesus
Cristo, mas que logo estará de volta. Independentemente disso, o texto
garante que com os ensinamentos que fez tem sua crença multiplicada até
os dias de hoje.

\paragraph{``Virgem Mãe das Dores''}

Virgem Mãe das Dores acompanha os romeiros todos os anos até a cidade de
Juazeiro, espécie de guardiã dos cristãos, que contribui para a manutenção da
fé em Padre Cícero.

\paragraph{``Mourão perguntado''}

Em forma de versos que alternam perguntas e respostas, terminando sempre
com rimas em ``ão'', o poema
ressalta a fé em Padre Cícero mostrando que seu nome não foi esquecido
porque sempre conviveu com o povo.

\paragraph{``A volta do Padre Cícero'' (septilhas de setissílabos) }

Em função das mudanças do século \textsc{xx} e \textsc{xxi} que o poeta caracteriza como
horizontes turvos, representados pela falta de pudor e o fim dos
valores do espírito, o poema pede a volta de Padre Cícero para
restabelecer uma ordem que permitiria uma nova vida.

\paragraph{``Toda terra precisa do seu santo'' (martelo alagoano)}

Reflexão sobre o bom senso dos pedidos realizados a Deus, que, 
apesar de não faltar àqueles que precisam não pode oferecer em demasia.
Confirma que a fé faz do Nordeste uma região mais feliz.

\paragraph{``Do improviso que ficou''}

Refletindo sobre o ofício de produzir versos, o poema fala da
importância da erudição, da habilidade de misturar as temáticas e da
necessidade de deixar o poeta cantar os versos na hora que deseja.

\paragraph{``Uma cantoria em Paris''}

Poema que apresenta uma disputa entre Oliveira de Panelas e Lourinaldo Vitorino numa
apresentação no Salão do Livro, em Paris, versando sobre esse
acontecimento e as relações existentes entre os universos culturais da
França e do Brasil.

\paragraph{``A corrupção''}

Escrito em 1998, o poema versa sobre uma nova \textsc{cbf}, Corrupção dos
Brasileiros Famosos, apresentando nomes de políticos corruptos bem como
suas estratégias, terminando com a analogia de que no ano da Copa do
Mundo de Futebol, o penta está garantido na copa da corrupção, já que
os titulares são craques e sobram reservas em formação.

\paragraph{``Só porque sou um filho adulterino''}

O poema relaciona o nascimento do filho de um encontro adúltero com as
mazelas por que esse homem passa na vida, acusando a injustiça dos preconceitos
que seriam sofridos.
%questionando porque por conta
%disso, em sua vida tem seu direito sonegado.

\paragraph{``Tiradentes morreu sem ver a luz''}

Narrando os feitos de Tiradentes, o poema o apresenta como um herói que
buscou defender os fracos e os inocentes, mas que como Cristo morreu
sem ver a luz da liberdade.

\paragraph{``Nascimento de um gênio''}

Castro Alves é apresentado no poema, que mescla dados biográficos com
as temáticas dos poemas de Castro Alves. Oliveira de Panelas retrata o poeta como
defensor dos escravos fazendo menção ao livro \textit{Espumas flutuantes}.

\paragraph{``Problemas políticos''}

Em décimas que terminam com o mote ``A política tem
sido até agora/ O problema maior dessa nação'', o
texto apresenta a corrupção e a falta de representatividade como
importantes elementos que fazem da política uma maneira de manter as
desigualdades.

\paragraph{``Planeta recuperado''}

Na tentativa de recuperar o planeta em que vivemos, o texto versa sobre o
fim da venda dos valores morais, da discriminação social e sobre o
desejo de que todos sejam tratados como iguais.

\paragraph{``Gemidos da terra''}

Poema que discorre sobre o aumento da população no planeta e sua
exploração. Com um mote ao fim de cada estrofe, o texto
apresenta o sofrimento, através dos gemidos da terra, para sustentar
tantos pés ou tanta gente.

\paragraph{``O cantador''}

Ode aos poetas populares do Nordeste, demonstrando suas habilidades e a
maneira como utilizam o improviso para musicar a vida do
brasileiro, buscando nas belas fantasias pequenas pílulas de
felicidade.

\paragraph{``Ghandi''}

Poema de alusão à figura de Mahatma Ghandi, ressaltando seus feitos na
defesa do pacifismo como forma de luta contra a opressão colonial, bem
como a vida com menos recursos, possibilitando assim que a justiça seja
feita.

\paragraph{``As invenções''}

O poema apresenta os nomes e os principais inventos dos últimos séculos,
demonstrando a importância de cada um deles elogiando seus criadores 
e defende que se houvesse mais inventores todos os
problemas da humanidade já teriam sido resolvidos.

\paragraph{``Enigmas''}

Discorrendo sobre diversas temáticas polêmicas, o poema apresenta sempre
os dois lados da questão, como, por exemplo, o operário que faz greve,
porque passa fome, mas que mesmo com as greves não tem seus problemas
resolvidos.

\paragraph{``Guerra''}

Partindo de um acontecimento real, a Guerra das Malvinas, que opôs
argentinos e ingleses na década de 1980, o poema a insere na linha de
leitura da guerra fria, bem como defende o fim desses conflitos,
afirmando que das guerras o máximo que o poeta quer saber é o resultado
do telequete (espécie de luta livre combinada, que teve seu auge entre
os anos de 1980 e 1990)

\paragraph{``Festival de sereias'' (galope à beira"-mar)}

Narrativa sobre o encontro do autor com as sereias, mostrando esse
universo mágico e o modo como ele ficou completamente apaixonado por
uma delas, a ponto de trazê"-la para casa e transformá"-la em sua esposa.
Ressaltando ainda que não há mulher melhor que essa, já que é fonte de
amor, da cintura para cima, e fonte de alimento, da cintura para baixo.

\paragraph{``Oração da paz'' (toada alagoana)}

Espécie de elegia à paz, que retoma os muitos conflitos existentes no
mundo. Defendendo a igualdade entre os povos, o poema apela para que as
armas sejam trocadas por apertos de mão.

\paragraph{``Conformação'' (parcela ligeira)}

Evocando um lugar calmo e distante da correria das cidades e do mundo
atual, o autor apresenta o lugar em que ele gostaria de estar.
Sugerindo, portanto, que sua conformação estaria longe desse universo
caótico e tumultuado das cidades.

\paragraph{``Banquete não é jejum'' (oitavão rebatido)}

Poema de improviso que opõem diversas figuras a partir das expressões
``não tem'' e ``não é''. De leitura rápida, em
setissílabos, com rimas em ``ido'', o Poema reproduz 
a ideia de um desafio em forma de perguntas e respostas.

\paragraph{``Da causa vem o efeito'' (oito pés a quadrão)}

A partir da métrica do ``deixa"-prende'', em que o poeta
retoma a rima do verso anterior, o poema apresenta diversas temáticas,
em que cada estrofe, em formato de falas, retoma o verso anterior para a
realização da rima.

\paragraph{``Geme quem trai seu amor'' (toada gemedeira)}

Com o mote ``Ai, ai! Ui, ui!'' o
poema discorre sobre os diversos momentos em que um gemido pode
aparecer, apresentando, como esperado, os momentos de prazer, os gemidos
aparecem também no medo da sogra ou da prisão.

\paragraph{``O poeta se despede'' (dez pés a quadrão)}

Poema que encerra a coletânea e agradece aos leitores. 
Além de agradecer ele se desculpa por aqueles que porventura 
acham que os temas são simples ou que não gostam de poesia.

\section{Sobre o gênero} 

A poesia de cordel, dizem os especialistas, é uma poesia escrita para
ser lida, enquanto o repente ou o desafio é a poesia feita oralmente,
que mais tarde pode ser registrada por escrito. Essa divisão é muito
esquemática. Por exemplo, o cordel, mesmo sendo escrito e impresso para
ser lido, costumava ser lido em voz alta e desfrutado por outros
ouvintes além do leitor. A poesia popular, praticada principalmente no
Nordeste do Brasil, tem muita influência da linguagem oral, aproveita
muito da língua coloquial praticada nas ruas e na comunicação
cotidiana. 

Naturalmente, portanto, pode"-se considerar a poesia narrativa do cordel
uma forma de poesia mais compartilhada e desfrutada coletivamente, o
que lhe dá também uma grande ressonância social. Muitos dos temas do
cordel são originários das tradições populares e eruditas da Europa
medieval e moderna. Nesse livro de Oliveira de Panelas, a tradição
medieval possui dupla importância, se ela se dá em alguns momentos pela
temática, como nos poemas que abordam o juízo final, ela também está
presente na retomada de formas métricas da tradição
trovadoresca galego"-portuguesa, caso de diversos poemas, entre eles
``O santo frei Damião'' e ``Do improviso que ficou'', entre
outros. Outros temas são retirados de tradições orientais. 
Também encontramos na poesia popular em cordel temas
retirados das novelas de cavalaria medievais e das narrativas bíblicas.
Ao lado destes temas mais literários, encontram"-se os temas locais,
quase sempre narrados na forma de crônicas de coisas realmente
acontecidas, como em ``Meu padrinho
viajou'', e as reportagens jornalísticas partem de
acontecimento reais, e buscam interpretá"-los, caso de
``Guerra'', ``A corrupção''. Também há as histórias fantásticas,
que se valem das tradições semirreligiosas, ligadas à experiência com o
mundo espiritual. 

Os grandes poemas de cordel são perfeitamente metrificados e rimados. A
métrica e a rima são recursos que favorecem a memorização e
tradicionalmente se costuma dizer que são resquícios de uma cultura
oral, na qual toda a tradição e sabedoria são sabidas de cor. 


\subsection{O sertão geográfico e cultural}

O sertão tem mitos culturais próprios. Contemporaneamente, o sertão
evoca principalmente o sofrimento resignado daqueles que padecem a
falta de chuva e de boas safras na lavoura. Evoca a experiência
histórica de uma região empobrecida, embora tenha sido geradora de
riquezas, como o cacau e a cana"-de"-açúcar, ambos bens muito valiosos. 

O sertão formou também o seu imaginário por meio de grandes
personalidades e uma pujante expressão artística. Além do cordel, o
sertão viu nascer ritmos tão importantes quanto o forró e o baião.
Produziu artistas tão expressivos quanto Luiz Gonzaga, grande cantor da
vida do sertanejo em canções como ``Asa
branca''. Um escultor como Mestre Vitalino criou toda
uma tradição de representação da vida e dos hábitos sertanejos em
miniaturas de barro. A gravura popular, que sempre acompanha os
folhetos de cordel, também floresceu em diversos pontos e ficou mais
famosa em Juazeiro do Norte, no Ceará, e em Caruaru, no estado de
Pernambuco. 

Dentre os grande mitos do sertão, está certamente o do cangaço com seu
líder histórico, mas também mítico, Virgulino Ferreira, o Lampião. Até
hoje as opiniões se dividem: para alguns foi um grande homem, para
outros um bandido impiedoso. 

Uma figura muito presente na cultura nordestina é o Padre Cícero Romão,
considerado beato pela Igreja Católica. Consta que teria feito milagres
e dedicado sua vida aos pobres. 

\subsection{Variação linguística}

A linguística moderna usa o termo ``idioleto'' para marcar grupos
distintos no interior de uma língua. Um idioleto pode ser a fala
peculiar de uma região, de um grupo étnico ou de uma dada profissão. 

Uma das grandes forças da poesia popular do Nordeste se origina em sua
forma muito própria de falar, com um ritmo muito diferente dos falares
do sul, e também muito diferentes entre si, pois percebe"-se a diferença
entre os falares de um baiano, um cearense e um pernambucano, por
exemplo.

Além desse aspecto rítmico, quase sempre também há palavras peculiares a
certas regiões. 

%\section{Sugestões de atividades}
%\begin{enumerate}
%
%
%
%\item \textit{Atividade de leitura}. Esta atividade tem por objetivo sensibilizar os
%alunos para a escuta de poesia. O professor deve ler um conjunto de
%estrofes para exemplificar uma leitura que se construa com uma
%pronúncia clara, pausas e ênfases adequadas. Após isso, cada aluno deve
%ler uma estrofe, procurando marcar o ritmo e as rimas, bem como as
%pausas e ênfases expressivas. É possível enriquecer essa experiência
%com os vídeos do poeta cantando seus versos com diversos parceiros, que
%encontram"-se disponíveis no You Tube:
%https://www.youtube.com/watch?v=p2vii5KgXFw e
%https://www.youtube.com/watch?v=GdKG4rNs8UQ. O trabalho de leitura pode
%auxiliar o professor na realização de um diagnóstico dos alunos, em
%relação à pontuação e ao ritmo do texto, além de possibilitar um
%desenvolvimento da percepção da voz e da fala como meios indispensáveis
%à boa convivência social.
%
%\item No poema ``Encontro no céu de Frei Damião e Padre
%Cícero'' Panelas apresenta um tipo de história muito
%comum nos cordéis, são muitos os autores que narram os acontecimentos
%pós"-morte. A partir da leitura do poema o professor pode pedir aos
%alunos que pesquisem sobre os dois principais personagens. Com a
%pesquisa em mãos é possível refletir com os alunos sobre a importância
%dos líderes religiosos no interior do Nordeste. Como forma de
%complementar a discussão é importante ressaltar que apesar das
%personagens terem sido contemporâneas por alguns anos, viveram épocas
%muito diferentes. Nesse ponto, o poema apresenta diversos elementos
%capazes de contribuir com a discussão. Por fim, o professor pode
%narrar os acontecimentos de Canudos e finalizar a discussão com um
%debate sobre se a religião permite a emancipação social.
%
%\item Em ``Meu padrinho viajou'' e
%``Meu Padim, Padre Ciço'' o autor
%narra o momento da morte do Padre Cícero Romão e o desejo ainda
%existente de que ele retorne do céu para guiar o povo. Do ponto de
%vista da linguagem, o adjetivo possessivo
%``meu'' contribui para aumentar a
%importância desse líder e personalizá"-lo na vida de cada nordestino. A
%partir da leitura desses poemas o professor pode pedir aos alunos que
%identifiquem no texto os vocábulos que contribuem para a caracterização
%dessa figura como indispensável para a vida do sertanejo nordestino.
%Por fim, com os vocábulos identificados é possível refletir sobre o
%desejo de retorno desse personagem para guiar o povo, já que o poema
%refere"-se ao líder como um ente querido (a partir do adjetivo
%possessivo e do uso do apelido em forma de corruptela
%``Padim Ciço''). Essa reflexão pode
%ainda ser acompanhada de uma comparação com a figura do rei Dom
%Sebastião de Portugal, que no fim do século XVI foi ferido na batalha
%de Alcácer Quibir e desapareceu, mas cuja volta
%foi esperada durante séculos.
%
%\item No poema ``A corrupção'', Panelas
%parte de um acontecimento esportivo, a Copa do Mundo de 1998, para
%distorcer a sigla CBF -- Confederação Brasileira de
%Futebol --, transformando"-a em Corrupção dos Brasileiros Famosos. 
%A partir da leitura do
%poema, o professor pode pedir aos alunos que indiquem as imagens
%utilizadas pelo autor para representar a corrupção. Com as imagens em
%mãos, é possível refletir sobre os motivos que levam à corrupção,
%investindo principalmente na temática da impunidade bem como trantando do sistema
%paternalista da política, mostrando como muitos dos nomes citados
%possuem relação entre si, pois são apadrinhados por outros políticos.
%
%\item No poema ``Tiradentes morreu sem ver a
%luz'' essa personagem histórica é caracterizada como
%um grande herói, que virou mártir da luta pela república e que é
%diretamente relacionado a um Jesus Cristo moderno. A partir da leitura
%do poema, o professor pode pedir aos alunos que realizem uma pesquisa
%sobre essa personagem histórica. Com os resultados da pesquisa em mãos
%é possível discutir como a interpretação do autor pode e deve ser
%questionada, já que diversos historiadores procuram mostrar que
%Tiradentes só é visto como um herói após a proclamação da República,
%bem como sua proximidade com Jesus Cristo é resultado de uma construção
%posterior ao acontecimento. Para contribuir para essa reflexão o
%professor pode trazer o quadro ``Tiradentes
%esquartejado'', de Pedro Américo (1893), obra do
%momento em que a relação entre Cristo e essa personagem começa a ser
%construída na historiografia. 
%
%\item Em ``Nascimento de um gênio'' o
%autor narra parte da vida do poeta Castro Alves. O poema, construído em
%décimas de hendecassílabos, termina sempre com um mote
%``Castro Alves nasceu cantando amores/ Num castelo de
%espumas flutuantes'', que faz menção ao primeiro
%livro do poeta baiano, \textit{Espumas Flutuantes}, de 1870, mas que gera um falso
%contraste com a maneira como o poema o caracteriza, já que o autor
%investe numa longa descrição dos poemas de Alves que combatem a
%escravidão. Entretanto, a imagem do castelo de espumas flutuantes pode
%ser entendida como um lugar incerto, onde a escravidão não aconteça, ou que
%tem suas bases pouco firmes por ser de espuma, que pode ser interpretada como
%a fragilidade decorrente da escravidão. A partir da leitura do poema, o
%professor pode pedir aos alunos que elaborem hipóteses sobre a imagem
%de um castelo de espumas flutuantes. Com as hipóteses elaboradas é
%possível retomar a temática do texto, investindo principalmente nas
%imagens em que a defesa dos negros é explicitada. Por fim, o professor
%pode trazer um trecho do poema ``O navio
%negreiro'', de Castro Alves, como forma de mostrar
%que as imagens do belo e principalmente da natureza podem ser tratadas
%de forma densa, no interior do Romantismo.
%
%\item No poema ``Gemidos da terra''
%Panelas utiliza"-se de uma prosopopeia ou personificação, quando um
%objeto inanimado é capaz de expressar sentimentos. Nesse caso o planeta
%Terra é capaz de gemer, por causa das mazelas que ele sofreu, devido à
%exploração indiscriminada de seus recursos naturais, bem como do
%aumento da população no planeta. A partir da leitura do poema, o
%professor pode pedir aos alunos que apresentam a estratégia do autor
%para dizer que a terra está sofrendo (nesse ponto, caso os alunos
%não consigam apresentar o recurso da prosopopeia, é importante que o
%professor o apresente como uma figura de linguagem). Em seguida, é
%possível refletir sobre qual a interpretação do poema sobre esse
%acontecimento. Já que a princípio o autor sugere que a causa desses
%problemas está nos países do oriente e termina por também refletir
%sobre sua responsabilidade nesse movimento.  
%
%\item Em ``Guerra'', o autor parte da
%Guerra das Malvinas, de 1982, que opôs Argentina e Inglaterra, pelo
%controle de um pequeno arquipélago de ilhas do Atlântico Sul.
%Entretanto, a interpretação do poema caracteriza esse acontecimento
%como uma decorrência direta da guerra fria. A partir da leitura do
%poema, o professor pode pedir aos alunos que indiquem ao longo do
%poema quais acontecimento sugerem essa interpretação. Em seguida, é
%possível realizar um debate sobre por que um país comunista apoiaria uma
%ditadura na América Latina e dessa forma estimular a interpretação do
%poema. Como forma de incrementar a discussão, o professor pode trazer
%outros elementos sobre esse acontecimento, demonstrando como ele é
%apenas uma decorrência periférica da chamada guerra fria.
%
%\item Os poemas ``Vida e morte de Frei Damião'', ``Frutos do
%amor'', ``Nas águas do mar'' e ``O santo frei
%Damião'' trabalham a mesma temática, versando sobre
%os feitos de Frei Damião de Bozzano, um importante líder religioso do
%Nordeste. Entretanto, apesar de falarem sobre o mesmo tema, as rimas
%desses poemas são trabalhadas de maneira bastante diferentes, sendo o
%primeiro sextilhas clássicas da literatura de cordel, o segundo décimas
%em hendecassílabos com mote (espécie de repetição de um ou dois
%versos), o terceiro décimas em hendecassílabos com mote e rimas em
%``ar'' e o quarto décimas em
%setissílabos com mote. A partir da leitura dos poemas, o professor pode
%pedir aos alunos que procurem as diferenças existentes nos poemas, em
%seguida apresentar como a diferença da estrutura métrica modifica a
%leitura dos textos, investindo portanto, num exercício de leitura
%comparada para a compreensão da métrica, bem como para o
%desenvolvimento de uma leitura com ritmo, que seja capaz de apresentar
%as pausas e ênfases expressivas (caso, por exemplo, do mote), bem como
%a pontuação. 
%
%\item Os poemas ``O poeta se
%apresenta'' e ``O poeta se
%despede'' iniciam e concluem o livro. A partir da
%leitura desses textos, o professor pode pedir aos alunos que elaborem
%hipóteses sobre a necessidade da apresentação e da finalização de um
%livro de cordéis. Com as hipóteses elaboradas, o professor pode
%trabalhar a perspectiva do cordel como desdobramento da tradição oral,
%e o modo como a propaganda feita sobre sua cantoria é importante para
%que os ouvintes se aproximem, bem como o agradecimento que garante que
%esse cantador será ouvido mais vezes. 

\begin{bibliohedra}

\tit{DIEGUES JÚNIOR}, Daniel. \textit{Literatura popular em verso}. Estudos. Belo Horizonte: Itatiaia, 1986. 

\tit{MARCO}, Haurélio. \textit{Breve história da literatura de cordel}. São Paulo: Claridade, 2010.

\tit{TAVARES}, Braulio. \textit{Contando histórias em versos. Poesia e romanceiro popular no Brasil}. São Paulo: 34, 2005.

\tit{TAVARES}, Braulio. \textit{Os martelos de trupizupe}. Natal: Edições Engenho de Arte, 2004 

\end{bibliohedra}

