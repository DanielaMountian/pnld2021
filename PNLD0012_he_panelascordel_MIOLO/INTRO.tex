
\chapter[Introdução, por Maurice Van Woensel]{Introdução}

Dentre os poetas de cordel titulares desta coleção,
Oliveira de Panelas se destaca -- e, de certo modo, destoa
{}-- de duas maneiras. Primeiro, mesmo com 54 anos de
idade e 40 anos de profissão de cantador, ele pertence à
(relativamente) nova geração de artistas populares em
atividade, e faz parte da galeria de famosos
``cordelistas'' aos quais se dedica esta coleção. Destes, muitos já faleceram
e vários tiveram sua atuação no século XIX: eles contrastam
um tanto com Oliveira, que canta pelo século XXI adentro
com boa disposição e saúde.

Por outro lado, Oliveira de Panelas já não pertence mais
às gerações de poetas que atuavam principalmente nas
feiras e confeccionavam folhetos para cantar e vender. Ele
faz parte do que o professor Joseph Luyten chama de ``Novo
Cordel'': seu campo de atividade são as cidades, e só
publicou folhetos no início de sua carreira. Nesta época, o
rádio de pilha e as redes de TV começaram a tornar
obsoleta e inútil a tradicional função do cantador de feira:
trazer as novas e comentar desastres. e eventos, criando
assim uma das principais formas populares de
entretenimento e divulgação de informações.

Atualmente estão desaparecendo as tipografias manuais
que imprimem folhetos como antigamente, de maneira
artesanal e com uma xilogravura na capa: para o poeta
popular, não há como competir com a rapidez e os custos
reduzidos das empresas gráficas modernas. Os poetas que
ainda vivem e sobrevivem de sua arte procuraram público,
maneiras de divulgação e meios de comunicação
alternativos. Também Oliveira, que começou a carreira
ainda na época dos folhetos impressos, há tempos
diversificou, com êxito, seus meios de expressão. Desde
jovem tinha um programa de cantoria na rádio local de
Garanhuns. Tem muitos LP's e CD's
gravados. Tornou-se
uma figura conhecida nacionalmente devido à sua atuação
nos meios de comunicação modernos. Foi convidado
dezenas de vezes para cantar em programas de TV e já
cantou para as mais altas autoridades, no Brasil e no
exterior.

Pode-se pensar que Oliveira, pelo fato de ser um
cantador bem sucedido, teria renegado ou adulterado a arte
e a genialidade dos poetas de cordel. Ora, um livro que se
torna um \textit{best-seller} nem sempre é obra de arte espúria.
Tampouco um cantador popular que conquista um espaço
no circuito das Letras eruditas e na mídia é
necessariamente um traidor da tradição popular. Oliveira
não deixa de ser um artista genuíno, um herdeiro e um
continuador da longa tradição dos cantadores seus mestres.
Seu sucesso resulta exatamente do fato de que soube
conservar e até enriquecer o legado dos cantadores
populares e suas marcas: a aparente ingenuidade, a
jocosidade matuta, a criatividade verbal e a fidelidade às
rigorosas regras formais da cantoria. É por estes motivos
que apresentamos nesta série de poetas de cordel Oliveira
de Panelas, o artista e sua obra.

\section{Suas origens e formação}

Seu nome de batismo é Oliveira Francisco de Melo.
Nasceu em 24 de maio de 1946, filho de Antônio Francisco
de Melo e Maria Virtuoza dos Santos. Casou-se em 1968
com Ana Araújo de Melo com quem teve três filhos, Walter,
Wagner e Walber. Há vinte e cinco anos reside e\~{} João
Pessoa, em um sítio no bairro do Cristo Redentor; um lugar
sempre aberto aos amigos. O sobrenome
``Panelas'' tem
pouco a ver com o utensílio de cozinha de mesmo nome:
refere-se ao município (e ao rio) do interior pernambucano,
situado a 180 km de Recife, onde Oliveira nasceu, no sítio
Barroca. Seu pai era lavrador e pedreiro, não era cantador
nem poeta, mas, tendo uma bela voz, cantava para os
familiares e amigos. Os pais gostavam de ouvir os cantadores
na feira e organizavam em sua casa sessões de cantoria. O
jovem Oliveira também se apaixonou por essa arte, e os pais
incentivaram sua vocação quando, logo cedo, demonstrou
ter talento de poeta-cantador: aos nove anos de idade já
brincava de compor e cantar versos. Seu pai sempre lhe
apoiou nos primeiros anos de carreira. Oliveira nos disse a
seu respeito: ``Eu fui o cantador que ele não pôde
ser''.

Oliveira estudou as primeiras letras na escolinha local,
mas não chegou a concluir o curso primário. Em
compensação, já tinha uma curiosidade intelectual
insaciável: lia à luz de lamparina todos os livros que lhe
caíam nas mãos. Tanto que, anos depois, conseguiu passar
com facilidade nas provas do ensino supletivo.

Sua escola não tinha paredes: aos 12 anos começou
a ``estagiar'' com o poeta ambulante Zé
Rufino:
acompanhava-o, cantando nas feiras, e aprendia com ele a
arte. Aos 14 anos de idade, Oliveira iniciou sua carreira de
cantador, profissionalmente, em parceria com os cantadores
ambulantes João Vicente e Manoel Hermínio, e escolheu
o nome artístico de Oliveira de Panelas, em homenagem a
seu município de origem. Cantando como repentista,
percorria, na época, o interior de Pernambuco, Alagoas,
Sergipe e Paraíba.

A respeito destas andanças, Oliveira nos contou o
seguinte: ``Tínhamos três maneiras de atuar. Primeiro, a
excursão. Sem saber de antemão aonde ir, andávamos de
cidade a cidade, parando em qualquer casa ou terreiro,
oferecendo nossos serviços e aceitando a hospitalidade de
um ou outro fazendeiro ou notável. Outra maneira de viajar
era o trato: fulano de tal nos convidava para tal ou tal lugar,
mas sem compromisso: ali a gente cantava `de
bandeja': cada pessoa presente depositava, no fim da sessão, sua
moeda ou nota na bandeja, como colaboração espontânea.
Por fim, podíamos viajar \textit{de contrato}: neste caso já foi
estipulado de antemão o montante de nossos honorários e
indenização de despesas. Inútil dizer que por muitos anos
andamos de excursão e raramente de contrato!''\footnote{ Conforme 
entrevista concedida por Oliveira, em sua casa, ao
autor desta introdução, em 11 de julho de 2000.}

A partir de 1962 Oliveira fixou residência em
Garanhuns, no Pernambuco, onde passou oito anos.
Continuou cantando em parceria com os mestres da região.
Chegou a ser contratado por um programa de violeiros e
repentistas da Difusora de Garanhuns para fazer parceria
com Manoel Bezerra Diniz, o Voador da Paraíba. Mas era
difícil sustentar-se com o produto de sua arte, e, por longos
períodos, o poeta tinha que trabalhar com as mãos,
ajudando seu pai no ofício de pedreiro. Porém, nunca
desanimou: perseverava na sua vocação de poeta e aprendia
com paciência a técnica e arte necessárias ao ofício.

Em 1971, ele seguiu o caminho de milhões de
nordestinos, para o Sul: deixou Garanhuns e foi para São
Paulo, à procura de um futuro melhor. Radicou-se no bairro
do Brás, e por algum tempo teve que trabalhar como
pedreiro para sobreviver. Começou a cantar nos bares e
restaurantes do Brás, e já em 1972 os talentos poéticos do
pedreiro nordestino começaram a ser reconhecidos: fazendo
dupla com José Ferreira, ganhou seu primeiro prêmio no
Festival de Violeiros realizado no Cine Fontana, em Celso
Garcia. Assim começou o caminho do sucesso. Ainda em
São Paulo, tornou-se sócio-fundador e conselheiro da
Associação de Repentistas, Poetas e Folcloristas do Brasil.
Chegou a representar seu estado no I Congresso de
Repentistas Nordestinos de São Paulo.

A partir dali, Oliveira foi convidado para cantar em
restaurantes e bares, residências, teatros e casas de shows,
universidades e colégios, em numerosos eventos de caráter
público e privado. Em 1973 participou da trilha sonora de
dois filmes nacionais \textit{Os últimos dias de Lampião e Deus
deu a terra e o diabo cercou}. Em 1974 foi solicitado pela
Rede Globo de Televisão para uma participação especial
no programa dominical \textit{Fantástico}, tendo
como parceiro José Francisco de Souza. Em 1975, formando dupla com
João Quindingues, e com a participação de outros seis
repentistas, gravou três LPs para a \textit{Coletânea de
Repentistas}, na série ``Brasil Caboclo'',
apresentando 24 gêneros diferentes de repente.

No final de 1975, sua vida mudou. Foi em São Paulo
que se encontrou com um dos famosos três irmãos-poetas:
Otacílio Batista Patriota, irmão de Dimas e Lourival, da
linhagem e parentela dos patriarcas dos poetas populares
nordestinos, Ugolino e Nicandro Nunes da Costa. Esses,
misturados com a família dos Batista, deram origem à
estirpe de Antônio, Pedro, Chagas Batista e seus filhos:
Paulo Nunes Batista e Sebastião Nunes Batista. Foi Otacílio
Batista que convidou Oliveira a voltar para o Nordeste: os
dois começaram uma parceria que havia de durar mais de
vinte anos, e a dupla foi granjeando sucesso e popularidade
junto ao grande público como também na mídia. Otacílio
teve Oliveira como a um filho -- seus próprios filhos não
continuaram a tradição, assim como os filhos de Oliveira
não a continuam -- e transmitiu-lhe os segredos e as
técnicas da cantoria.

Enraizado na Paraíba, o poeta continuou atuando pelo
Brasil afora. Em 1976, tendo Juvenal de Oliveira como
parceiro, obteve o terceiro lugar no Congresso de Violeiros
de Campina Grande, onde foi considerado a revelação do
ano por conta de seu talento de improvisação. Em 1978
gravou, pela continental, um LP \textit{Só Deus pode mais e ainda},
pela CBS, e ainda O perguntador, produzido por Zé
Ramalho. Em 1980, a Ordem dos Músicos do Brasil
conferiu a Oliveira o diploma de habilitação profissional.
De 1981 a 1982 presidiu a Associação dos Poetas e
Repentistas do Brasil (Aperb). A entidade promove a cada
ano o Encontro Nacional de Poetas Cantadores do Brasil a
fim de abrir espaço para novos talentos. Participou ainda,
como membro da Comissão Julgadora, do Festival MPB-
Shell, no Rio de Janeiro.

Durante muitos anos, a dupla Oliveira-Otacílio era
titular do programa, ``O Nordeste canta'' na
Rádio Tabajara
de João Pessoa, campeão de audiência local. Oliveira
sempre tem na agenda convites para cantorias desde o Acre
até o Rio Grande do Sul. Já participou de mais de duzentos
congressos e concursos de violeiros, dos quais costuma sair
como vencedor ou premiado. Pelo microfone da Rádio
Globo, já cantou para o Papa, em sua passagem pela
Argentina, depois da Guerra das Malvinas. Cantou para o
então presidente de Portugal, Mário Soares, quando esteve
na casa de amigos em Brasília. Oliveira foi proclamado
vencedor do I Campeonato de poetas repentistas, realizado
no Memorial da América Latina, em São Paulo, em 1997.

Recentemente, no mês de maio de 2000, Oliveira foi o
vencedor de outro duelo artístico que mobilizou cantadores
em Foz de Iguaçu, numa promoção da Academia de Cultura
local. O Festival \textit{Norte versus Sul} reuniu em seu final dois
dos maiores expoentes do gênero no país. De um lado,
Oliveira de Panelas, defendendo a cultura nordestina. O
representante do sul era o conhecido José Estivalet, que
mostrou em verso e trovas as manifestações populares de
uma das regiões mais tradicionalistas do Brasil. Oliveira
venceu nas modalidades principais da poética repentista,
mostrando sua competência, tanto no conhecimento como
na eloquência. Em junho de 2000, comemorou junto aos
amigos seus 40 anos de carreira com uma série de shows e
debates. Neste contexto está previsto o lançamento de dois
livros: \textit{E Deus me fez cantador}, autobiográfico, e \textit{Mil
reflexões poéticas}, e ainda de dois CDs novos com os títulos
provisórios de \textit{Quarenta anos de cantoria} e \textit{Cântaro}.

Oliveira possui a arte de fazer amigos: jovial e prestimoso,
delicado e discreto, fica à vontade tanto em um palácio
quanto num remoto sítio do interior. Já foi parceiro de
microfone de Luís Gonzaga, o Rei do baião, e já cantou para
outro rei: Roberto Carlos. Deste encontro resultou uma
amizade e admiração recíproca. Por causa de sua melodiosa
voz de estentor já foi chamado de ``o Pavarotti dos
Sertões''. O verbete do \textit{Dicionário Literário da Paraíba} a respeito de
Oliveira dá um retrato fiel do artista e de seu estilo:

\begin{quote}

Canta gesticulando, rindo, empolgado, inquieto, atento
a tudo e a todos, procurando os motivos para sua poesia.
Com raciocínio rápido, em poucos segundos estrutura sua
estrofe, baseada no assunto em debate, buscando associar
o pensamento desenvolvido pelo seu contendor com a
circunstância, num esforço consciente de quem ostenta
competência e vivacidade.

Bom improvisador, desenvolve com naturalidade um
mote dado. Maneja com facilidade todas as formas poéticas,
embora dê preferência ao ``Galope à
beira-mar'' (estrofe de dez versos hendecassílabos), que considera o ``hino da
cantoria'', devido à musicalidade presente neste tipo de
composição poética.

Oliveira é também um músico. Seu conservatório foi o
sertão; seus mestres, os vaqueiros e aboiadores sertanejos. Boa
voz, afinado, executa com perfeição as várias toadas existentes
na poética dos cantadores, sendo também um inovador nesse
campo. Exercita constantemente sua sensibilidade musical,
assobiando pelas ruas as toadas das sextilhas, do martelo, do
galope, consciente de que a música é um elemento
imprescindível ao cantador e à cantoria, sobretudo quando é
executada harmoniosamente. Além do repente, canta também
poemas, canções, óperas etc. É um cantador que não se
preocupa em apenas fazer versos, mas que sabe unir o verso à
poesia e a poesia à música, cujo resultado ele transmite com
força, expressividade e, sobretudo, prazer.

Oliveira Francisco, mais conhecido como Oliveira de
Panelas, é atualmente considerado como um dos nomes
mais importantes da cantoria paraibana e nordestina, tanto
pelo talento e força poética, como pela liderança pessoal e
capacidade organizativa.\footnote{ Editora A União, João Pessoa, 1990, pp. 161-62.}

\end{quote}

\section{Sua arte poética}

Oliveira é um autodidata: em matéria de cordel, técnica
e arte, aprendeu o ofício com vários mestres sertanejos; na
área da cultura geral, ele leu sistematicamente obras-
primas da literatura como também textos importantes de
filosofia, história e geografia. Prova disso é que, à maneira
de um Augusto dos Anjos, subjuga os termos e nomes mais
inesperados aos rigores do metro e da rima e é capaz de
debater com intelectuais a respeito da arte e da técnica de
seu \textit{métier}.

Sobre qualquer assunto que lhe indicam de improviso,
inventa na hora estrofe após estrofe, todas elas repletas de
alusões e citações, eruditas e populares, e inclui,
relacionados com o assunto, nomes, termos, tudo cabendo
nos minuciosos esquemas da versificação popular. Assim,
no caso da estrofe que se segue, um ``galope à
beira-mar'',
há onze sílabas em cada verso, seguindo o intricado
esquema de rimas \textit{abbaaccddc}. Nesta modalidade, o mote
do último verso, que sempre termina com a palavra
``\textit{mar}'',
exige logo que também o sexto e o sétimo versos de cada
estrofe tenham a rima em ``-\textit{ar}''. Vejamos:

\begin{verse}

Eu canto em meus versos perfumes de fl\textit{ores}\\*
A curva de ondas, o verde da pl\textit{anta}\\*
O rio que corre, o pássaro que c\textit{anta}\\*
A noite que guarda segredos de am\textit{ores}.\\*
Tem o arco-íris com todas as c\textit{ores}\\*
Pintando aquarelas no corpo do \textit{ar}.\\*
Têm pássaros libertos levando o jant\textit{ar}\\*
São frutas maduras pros filhot\textit{inhos}\\*
Que dormem nas conchas macias dos n\textit{inhos}\\*
\textit{E eu canto galope na beira do mar}.

\end{verse}

Oliveira não tem medo de improvisar em certos metros
pertencentes à tradição antiga porém usados por poucos
colegas. Nesta antologia reproduzimos versos nesses metros
menos comuns: o ``mourão perguntado'', a
``toada alagoana'', a ``parcela ligeira'', a ``toada gemedeira'', o
``oitavão rebatido'', o ``oito pés a quadrão'' e o ``dez pés a
quadrão''.

\section{A versatilidade do poeta}

Oliveira sente-se à vontade perante qualquer tipo de
público. Já levou ao delírio turmas de universitários e de
sindicalistas, já arrancou os aplausos de freiras e de
aposentados. Evidentemente, prefere as sessões e shows
nos quais os presentes participam, cantam com ele,
respondem, gritam, comentam. Assustou-se um tanto
quando, de início, teve que enfrentar câmaras e holofotes
em estúdios de rádio ou TV, mas aos poucos foi se
acostumando àquela ausência de plateia. De qualquer
forma, sempre que possível, procura cantar em parceria
com um colega cantador. Porém, as sessões que prefere
mesmo, são, é claro, aquelas executadas ao vivo em sala
ou auditório lotado por uma rumorosa assistência. De fato,
o contato direto com o povo é indispensável a um poeta
popular.

Fomos testemunhas, um dia, da facilidade de
improvisação e da capacidade de memorização imediata
de Oliveira. Foi numa reunião de família, daquelas famílias
grandes do Nordeste: o patriarca fundador gerou, no início
deste século, doze filhos -- contando só os vivos! -- e cada
um desses deixou uma prole de pelo menos oito. Dos
quinhentos descendentes, trezentos se reuniram na fazenda
de um deles, e Oliveira foi convidado para cantar a história
da família. Bastavam poucos minutos para se inteirar dos
nomes e das façanhas de algumas dezenas de descendentes,
vivos e mortos, do grande clã. Oliveira acomodou o violão
e vinte minutos a fio foi debulhando sextilha após sextilha,
improvisando o histórico anedótico da família. Eis algumas
das estrofes produzidas na hora:

\begin{verse}

Dizem que o tio Manuel\\*
Ia a qualquer região\\*
O dinheiro no seu bolso\\*
Ele não guardava não\\*
Carregava toda a grana\\*
Dentro de seu sapatão

Dizem que nossa Florinda\\*
Gostava de oração\\*
Ia sempre à Gurinhém\\*
Para assistir à Missão\\*
Às palavras que saíam\\*
Do nosso Frei Damião

\end{verse}

Mas, à certa altura, arrebentou-se uma das cordas de
seu violão. Mal ressoou o estouro e Oliveira, continuando
a tocar com as cordas que restavam, ``intercalou, sem
pestanejar, no meio das anedotas da genealogia familial,
esta estrofe:

\begin{verse}

É que essa corda minha\\*
Quebrou-se nesse momento\\*
É apenas essa corda\\*
Que está em meu instrumento\\*
Porém, nunca vai quebrar-se\\*
A corda de meu pensamento.

\end{verse}

Os versos de Oliveira surpreendem também pelas notas
humorísticas, pela verve típica dos cantadores sertanejos.
Assim, no ``Encontro de Frei Damião com Padre Cícero
no céu'', Padre Cícero externa sua admiração perante o
relato de Frei Damião sobre as novidades do planeta,
dizendo: ``Eu morro e não acredito'':
expressão bem
popular, mas inesperada na boca de quem já morreu.

\section{O cordelista autodidata}

Possuidor de uma memória xerográfica, o poeta guarda
na cabeça uma infinidade de termos e nomes, anedotas e
histórias de todo tipo, provindos de seus contatos do dia-a-
dia, mas principalmente da leitura de centenas de livros
sobre filosofia, história, ciências. Além disso, Oliveira é um
dicionário de rimas ambulante: fora as rimas tradicionais
de nomes comuns, sempre aproveita as rimas de nomes
próprios, da atualidade ou da história, em qualquer idioma.

Assim, por exemplo, quando cantava em Paris para um
público seleto de intelectuais, deu uma amostra de seus
conhecimentos da História e Literatura da França. Foi
citando os gênios da literatura de cordel junto a outros
gênios:

\begin{verse}

Inácio da Catingueira\\*
Negro poeta tão bom\\*
Se lá tivemos Conselheiro\\*
Aqui tivemos Danton\\*
E ainda Afonso Daudet,\\*
Tartarin de Tarascon.

\end{verse}

Na mesma oportunidade, à guisa de contraste com as
obras de cunho sociológico de Gilberto Freyre, Oliveira
citou um romance pouco conhecido de Victor Hugo, \textit{Os trabalhadores do mar} (\textit{Les
Travailleurs de la mer}): muita gente, aliás, ignora o fato de
que Machado de Assis verteu esta obra para o português
sob o título \textit{Trabalhadores do mar}.

\begin{verse}

[\ldots]\\*
No grande Gilberto Freire\\*
Um rei espetacular.\\*
Aqui temos Victor Hugo\\*
E Trabalhadores do mar.\footnote{ Esses versos constam do relatório em
livro do XVIII Salon du Livre (março de 1998) em Paris,
incluindo a transcrição da ``Cantoria de Oliveira de
Panelas et Lourivaldo Vitorino'', pp. 216-21.
Neste livro, os versos vêm acompanhados de uma tradução
para o francês.}

\end{verse}

Às vezes Oliveira nos traz à memória a poesia de Augusto
dos Anjos, o poeta-mago que soube amalgamar fria
terminologia científica e sublime poesia. É o que acontece
nessa estrofe do martelo de mote invariável, intitulado
``Destino traçado''. (Vale destacar o fato de que o poeta 
respeita a regra dos acentos tônicos da modalidade métrica
em pauta: eles incidem sobre a terceira, a sexta e a décima
sílaba.)

\begin{verse}

Quando o óvulo fecundo acelerado,\\*
Gira o sangue com toda hemoglobina,\\*
Nisso a mãe necessita proteína\\*
Para dar a seu filho encarcerado\\*
Onde todo destino foi marcado\\*
Logo após o contato conjugal\\*
Na primeira carícia sensual;\\*
O espírito na carne se mistura,\\*
\textit{O destino de cada criatura}\\*
\textit{É traçado no ventre maternal}.

\end{verse}

\section{Resíduos da poética medieval}

O autor destas páginas, de formação humanística
europeia, ficou admirado ao descobrir o quanto a arte poética
de Oliveira e de muitos poetas populares nordestinos é
devedora dos trovadores e outros poetas medievais. De um
modo geral, os modelos específicos, os temas mais recorrentes
e o estilo da poesia medieval constituem, no âmbito da poesia
atual, uma linguagem extinta, apenas relembrada e
analisada pelos estudiosos: de fato, a mentalidade pragmática
das gerações modernas não simpatiza com a simplicidade, o
despojamento e a estrita observância das regras de métrica
da parte dos poetas medievais. Porém, é mister constatarmos
que, curiosamente, os poetas de cordel continuam cultivando
{}-- sem que se deem conta disso -- diferentes aspectos da
poesia medieval, tanto no que diz respeito à forma como ao
conteúdo.\footnote{ Ver Chico Viana \& Maurice Van Woensel, \textit{Poesia
medieval ontem e hoje}. João Pessoa: CCHLN Editora
Universitária, 1998.} Citamos a seguir algumas dessas marcas de
medievalidade, sempre presentes na poesia de Oliveira e
dos cordelistas em geral.

\asterisc

Lembramos, de início, que o metro preferido dos poetas
de cordel é a \textit{sextilha de setissílabos}, com rimas \textit{xaxaxa}, a
mesma rima no segundo, no quarto e no sexto verso. Ora,
esta fórmula é tão somente uma retomada do modelo dos
\textit{romances ibéricos} que floresceram na Península a partir
do século XV. Conforme Ramón Menéndez-Pidal, são
``poemas épico-líricos breves que se cantam ao som de um
instrumento quer em danças corais quer em reuniões para
simples recreio ou para trabalho em comum''.\footnote{ Ramón
Menéndez-Pidal, \textit{Flor nueva de romances viejos}. Madrid:
Espasa-Calpe, 1941, p. 7.} As mais
das vezes, os romances vêm em setissílabos, em estrofes de
tamanho irregular e não têm refrão. Geralmente os versos
se agrupam, semântica e sintaticamente, em pares, e esta
distribuição binária é mais destacada ainda pela rima --
nos romances antigos era assonante -- que só incide nos
segundos, quartos, sextos versos, e assim por
diante. É o que observamos nos versos iniciais de um romance
castelhano:

\begin{verse}

Passeava-se a Silvana\\*
Pelo corredor acima\\*
Viola de oiro levava,\\*
Oh, que bem a tangia!\\*
Essa ela bem a tangia\\*
Melhor romance fazia.\footnote{ Em J. David Pinto-Correia (org.).
\textit{Romanceiro tradicional português}. Lisboa: Editorial
Comunicação, 1984, p. 288.}

\end{verse}

Comparemos esses versos, quanto à métrica, com os
seguintes versos de cordel de autoria de Oliveira:

\begin{verse}

Morreu e foi para o céu\\*
O nosso Frei Damião\\*
Assim que no céu chegou\\*
Quem estava no portão?\\*
Era um santo nordestino\\*
O padre Cícero Romão.

\end{verse}

Ora, o modelo formal -- metro e rimas -- dos folhetos
é praticamente idêntico ao do romanceiro acima descrito:
versos setissílabos, geralmente formando pares até pela
força da rima em versos alternados (\textit{xaxaxa}), sem refrão:
a única diferença está no fato de que, nos folhetos, as estrofes
vêm em tamanho padronizado -- sextilhas, setilhas ou
décimas -- enquanto no romanceiro os versos não formam
estrofes mas vêm em blocos de tamanho aleatório. Por outro
lado, os poetas de cordel continuam outro elemento da
tradição antiga ao tocar o violão para acompanhar o canto
de seus folhetos. Este formato métrico e rimático, na sua
versão escrita, sob a forma de um folheto impresso
artesanalmente, foi difundido no Nordeste no final do século
XIX por Silvino Pirauá Lima e outros poetas.\footnote{ Átila Augusto F.
de Almeida \& José Alves Sobrinho. \textit{Dicionário bio-bibliográfico
de repentistas e poetas de bancada}. João Pessoa/Campina
Grande: Editora Universitária/Centro de Tecnologia, 1978,
p. 164.}

É provável que, em sua forma oral, já fosse este o formato
usado pelos poetas populares de gerações anteriores que não
deixaram traços documentados. De qualquer forma, temos
aqui um caso de maravilhosa simbiose do poeta popular do
interior nordestino com seus antecessores culturais tão
distantes no tempo e no espaço. Neste contexto, lembramos
que um artigo da revista Veja em 1998 chamava a atenção
para o fato de que muitos brasileiros, vivendo em
comunidades afastadas, continuam ainda usando dezenas e
dezenas de termos e expressões típicas da época de Camões,
tais como ``vosmercê'', ``entonces'', ``fremoso''. Aliás, vários 
estudiosos já chamaram a atenção para este fenômeno de
contato direto da poesia e da cultura popular nordestina com
as raízes ibéricas medievais.\footnote{ Sebastião Nunes Batista. \textit{Poética
popular do Nordeste}. Rio de Janeiro: Fundação Casa de Rui
Barbosa, 1982, pp. 1-9, 22, 42, 45, 59; Jerusa Pires Ferreira.
\textit{Armadilhas da memória}. Salvador: Fundação Casa de Jorge
Amado, 1991, pp. 71, 86; Luís da Câmara Cascudo. \textit{Vaqueiros
e cantadores}, pp. 178-190; Augusto de Campos. \textit{Verso, reverso e
controverso}. São Paulo: Perspectiva, 1988, pp. 257-62.}

\asterisc

Também as famosas \textit{pelejas} ou \textit{desafios}, discussões e
debates, constituem uma continuação e, às vezes, um
enriquecimento das tradicionais \textit{tenções} dos trovadores
ibéricos. As tenções são um modelo poético fixo que consiste
em ``uma discussão entre dois ou mais trovadores: um deles
sustentava uma tese, contrária à do adversário, sobre uma
determinada questão''.\footnote{ Pierre Bec. \textit{Anthologie des
troubadours}. Paris: UGE/14-18, p. 42.} As respostas são dadas em versos
e estrofes de metro e esquema de rimas iguais aos da
pergunta. Dezenas de ``cantigas d'escárnio
e de maldizer'' pertencem a esta categoria.

Ora, o desafio, um gênero difícil da poesia de cordel,
apresenta as mesmas marcas. Na tradição da cantoria
nordestina, a resposta do desafiado deve possuir estrofe,
metro, e rimas de mesmo tipo da estrofe inicial do desafiante,
mesmo se este muda de metro no meio do desafio.

Os poetas de cordel retomaram também várias
modalidades de ``\textit{topos}'' medievais, conforme a
definição de Ernst R. Curtius: ``um clichê, ou seja, um estereótipo
retórico. Um \textit{topos}, por exemplo, é o das \textit{adunata} ou
\textit{impossibilia}, isto é, coisas impossíveis de
acontecer''.\footnote{ \textit{Literatura europeia e Idade Média
latina}. Brasília: INL, 1979, pp. 99-102.}
Oliveira e outros colegas retomam aquela tradição antiga
das impossibilia, em particular quando praticam o que
chamam de ``auto-louvação'', na qual glosam um
mote tal como ``O que fiz até hoje ninguém faz/ O que é que me
falta fazer mais?'' Eles vão enumerando suas proezas
imaginárias, uma ainda mais espalhafatosa que a outra,
como podemos verificar nessas. estrofes de Oliveira:

\begin{verse}

Construí toda a arca de Noé\\*
Fui o mestre de Freud e o de Jung\\*
Fiz a China apoiar Mao Tse Tung\\*
Escrevi o Corão pra Maomé\\*
Ensinei futebol ao rei Pelé\\*
Campeão de três copas mundiais\\*
Escapei da prisão de Alcatraz\\*
Dei patente de rádio pra Marconi\\*
Detonei os canhões de Navarone\\*
\textit{O que é que me falta fazer mais?}\\*

\end{verse}

\asterisc

Outro modelo formal e temático que os cordelistas
herdaram dos trovadores é o \textit{pranto}, o lamento pela morte de
uma pessoa querida ou importante. Nestes prantos, redigidos
em termos muito hiperbólicos, seguia-se um esquema fixo:
de início, a consternação e tristeza do povo (muitas vezes
destacava-se que a natureza também dava sinais de luto),
enumeravam-se os-feitos, vitórias e conquistas do morto
pranteado, louvava-se eventualmente os uherdeiros ou
sucessores, e por fim pedia-se uma prece pela alma do falecido.
Conservaram-se muitos \textit{planhs}, prantos em língua provençal,
em galego-português e em castelhano. Típico é este pranto
em versos de romanceiro da autoria de Gil Vicente sobre a
morte de Dom Manuel de Portugal, ocorrida em 13 de
dezembro, festa de Santa Luzia:

\begin{verse}

Pranto fazem em Lixboa\\*
Dia de Sancta Luzia\\*
Poer el rey Dom Manuel\\*
Que se finou nesse dia.

Choram, duques, choram condes,\\*
Cada hu quem mais podia,\\*
As damas e as donzelas\\*
Muito tristes em porfia.\footnote{ Carolina Michaelis de Vasconcelos.
\textit{Notas vicentinas}. Lisboa: Revista Ocidente, s.d., p. 125.}

\end{verse}

Também nossos poetas de cordel consagraram dezenas 
de folhetos à morte de pessoas ilustres. Recentemente, em
1998, quando da morte de Frei Damião, foram impressos
e vendidos diferentes folhetos sobre ele, porém sem
xilogravura na capa.

A morte do frade foi para Oliveira a oportunidade, não
de escrever um folheto, mas sim de gravar um novo CD,
cantando, em parceria com Otacílio Batista, a vida e a morte
do grande missionário nordestino em sextilhas. Encontra-
se na antologia desta obra o texto integral desta
homenagem a Frei Damião. Reproduzimos aqui alguns
versos com as feições do pranto medieval:

\begin{verse}

I\\*
Frei Damião de Bozzano\\*
Nasceu e morreu feliz\\*
Andarilho do evangelho\\*
Assim o destino o quis,\\*
Ser seguidor da grandeza\\*
De São Francisco de Assis.

\end{verse}

\asterisc

Outro traço manifesto do parentesco entre os cordelistas
e a tradição medieval é o uso frequente que Oliveira faz do
\textit{mote glosado}. Temos aqui uma reminiscência do vilancete,
``poema lírico de velha origem popular 
galego-portuguesa''.\footnote{ Sebastião Nunes Batista, op. cit.,
p. 42.} Nele, um mote inicial era seguido de estrofes
nas quais o vilancete se desenvolvia, sendo parcial ou
inteiramente repetido. O próprio Camões praticava uma
variante do vilancete, conciliando a arte erudita e a popular.
Deixou várias ``redondilhas menores'' (de
pentassílabos) ou ``redondilhas maiores'' (de
heptassílabos), gêneros populares na época, encabeçadas por um mote seguido de
``voltas'' que o elaboram e repetem. Vejamos esta
redondilha maior:

\begin{verse}

\textit{Descalça vai pera a fonte}\\*
\textit{Lianor, pela verdura;}\\*
\textit{vai fermosa, e não segura}.

Leva na cabeça o pote,\\*
O testo nas mãos de prata,\\*
Cinta fina e escarlata,\\*
Sainho de chamalote;\\*
Traz a vasquinha de cote,\\*
Mais branca que a neve pura.\\*
\textit{Vai fermosa, e não segura}\ldots\footnote{ Nádia Battella Gotlib (org.) \textit{Luís Vaz de Camões} -- Seleção e textos, estudos biográfico, histórico e crítico e
exercícios. São Paulo: Abril Educação, 1980, pp. 25-26.}

\end{verse}

No caso da cantoria de Oliveira, alguém da assistência
ou o próprio poeta dá um \textit{mote} -- um verso que deve
servir como tema e refrão --, e o poeta começa a glosá-
lo. Improvisa verso após verso sobre o tema, e cada estrofe
deve ser arrematada com o verso-mote. O modelo métrico
preferido de Oliveira é de ``glosar motes em
décimas'':
ele improvisa estrofes de dez versos, sendo que as rimas
do mote -- geralmente de dois versos -- determinam as
rimas dos últimos dois versos: ``\textit{dc}'' da
fórmula rimática clássica da décima: \textit{abbaaccddc}. Assim, por exemplo, o
mote:

\begin{verse}

\textit{Trinta milhões de crianças\\*
Sem Noel e sem Natal}

Trombadinha analfabeto,\\*
O assalto é teu salário,\\*
Gíria, teu vocabulário,\\*
Palavrão, teu dialeto.\\*
O véu da noite é teu teto,\\*
Patota, teu ritu\textit{al},\\*
A droga, teu sonris\textit{al},\\*
Furtar, é tua poup\textit{ança},\\*
\textit{Trinta milhões de crianças}\\*
\textit{Sem Noel e sem Natal}.\footnote{ Oliveira de Panelas e Otacílio Batista.
\textit{Dois poetas do povo e da viola}. João Pessoa: Funesc, 1996,
pp. 73-4.}

\end{verse}

Muitas vezes, nas chamadas pelejas, os dois poetas-
contendores devem. alternadamente improvisar uma
estrofe deste tipo. O mote pode consistir em um único verso,
mas também em dois ou mais versos. Existem ainda outras
fórmulas, variantes da glosa: às vezes repetem-se no final
dos últimos versos uma ou mais palavras-mote.

\asterisc

Os cantadores praticam ainda outra forma métrica
herdada da tradição medieval: a chamada
``\textit{leixa-pren}'', um termo galego-português que quer dizer 
``deixa-prende''. O último verso, senão uma palavra, ou ainda a
última rima da primeira estrofe era retomada no primeiro
verso da segunda estrofe, a última rima da segunda estrofe
no primeiro verso da terceira,.e assim em adiante.\footnote{ Massaud
Moisés. \textit{Dicionário de termos literários}. São Paulo: Cultrix, 1985,
p. 364.} É o caso, por exemplo, na ``tenção'' (debate em
versos) em décimas (estrofes de dez versos) entre os trovadores galegos
Vasco Martins e Afonso Sanches:

\begin{verse}

I\\*
Vaasco Martins, poys vos trabalh\textit{ades}\ldots\\*
\ldots Morreu, por Deus, por quen \textit{trobades}?

II\\*
Afonso Sanches, vos me pregunt\textit{ades}\ldots\\*
\ldots E vós al de min saber non queir\textit{ades}.\footnote{ José Joaquim Nunes.
\textit{Crestomatia arcaica}. 7ª ed. Lisboa: Livraria Clássica, s.d.,
p. 189-90.}

\end{verse}

Na gíria dos cordelistas, a \textit{deixa} é um termo equivalente
ao ``\textit{leixa-pren}''.\footnote{ Ver Sebastião
Nunes Batista, op. cit., pp. 8-9.} Como na poesia trovadoresca, a
``deixa-prende'' deles consiste geralmente na rima
``\textit{deixada}'' no final da primeira estrofe, e que ``\textit{se
prende}'', que é repetida
no primeiro verso da segunda. Este procedimento é muito
usado nas pelejas. Encontramos um exemplo típico da
``deixa'' na cantoria que Oliveira ofereceu em
21 de março de 1998, no XVIII Salon du Livre, em parceria com
Lourinaldo Vitorino. Encontra-se na antologia deste livro
o texto integral da cantoria, da qual transcrevemos alguns
trechos contendo ``deixas''. (É bom lembrar
que, em março de 1998, Zidane, citado aqui, ainda não era considerado
uma ameaça para Ronaldinho\ldots)

\begin{verse}

V
Que ambiente tamanho\ldots\\*
...Tudo para nosso lu\textit{gar}.

VI
Simone de Beau\textit{voir}\ldots\\*
...Nossa Raquel de Queir\textit{oz}.

VII
A França mostra para n\textit{ós}\ldots\\*
...Temos que analis\textit{ar}.

VIII
E se quiser compar\textit{ar}\ldots\\*
E aqui temos Platin\textit{i}

IX
Zidane está por a\textit{í}
Brilha igual Ronaldinho\ldots

\end{verse}

\asterisc

O folclorista francês Claude Roy assinalou um clichê
temático que se encontra nas canções populares de vários
continentes, por séculos a fio: é o que ele chama de tema da
\textit{metamorfose}, em versos que dizem ``\textit{Queria ser
um}...''. Nele,
o poeta, dirigindo-se à bem-amada, queria se transformar em
um animal, uma flor ou um objeto para ficar perto dela. Desde
New Orleans até Changai, através dos milênios, o tema foi
registrado com variantes mil: ``Queria ser... uma lágrima
tua... teu espelho... uma flauta em tua boca... tua cintura...
andorinha... uma estrela''.\footnote{ Claude Roy. \textit{Trésor de
la littérature populaire}. Paris: Seghers, 1954, pp. 9-14.} Também
Oliveira e os poetas de
cordel recorrem frequentemente a este tema-chave da
literatura antiga e moderna. Foram registradas em uma
sessão informal de cantoria de Oliveira, essas sextilhas:

\begin{verse}

Ah! Se eu fosse teu travesseiro\\*
Revestido de cetim\\*
Perfumado por verbena\\*
Rosa, dália ou alecrim\\*
Pra meu bem passar a noite\\*
Roçando seu rosto em mim.

Ah! Se eu fosse um batom\\*
Róseo, creme ou outra cor\\*
Com o gosto de morango\\*
Ou quem sabe outro sabor\\*
Pra passar o dia inteiro\\*
Nos lábios de meu amor.

Ah! Se eu fosse um sutiã\\*
De forma bem desenhada\\*
Concha macia e cheirosa\\*
Para servir de morada\\*
Dos seios maravilhosos\\*
Do corpo de minha amada.

\end{verse}

\asterisc

Outra marca da poesia medieval são os conhecidos.
``ABC's'' de cordel,
produzidos às centenas no Nordeste:
ABC do amor, ABC do jogo do bicho, ABC da carestia etc.
Neles, cada estrofe começa com uma letra do alfabeto, na
sequência certa, o que facilita a memorização. Este tipo de
poesia encontra-se em uma dezena de Salmos, foi praticada
por Santo Agostinho e pelos clérigos medievais e outros
poetas, sempre por motivos didáticos, já que este truque
mnemotécnico facilitava a memorização. Os meios de
comunicação modernos fizeram com que esta moda caísse
em desuso, e não é de admirar que Oliveira não tenha
produzido nenhum ``ABC''.


\section{Esta antologia}

Geralmente, os cordelistas recorrem em seus folhetos
às formas de metro mais simples: a \textit{sextilha} de heptassílabos
com as rimas \textit{xaxaxa}, a \textit{setilha} de heptassílabos com rimas
\textit{xaxabb}, ou ainda a \textit{décima}, estrofe de dez versos, com rimas
\textit{abbaaccddc}.\footnote{ Seguimos aqui as definições e comentários a
respeito das formas da poesia de cordel de Sebastião Nunes
Batista -- ele mesmo folhetista -- em seu \textit{Poética popular
do Nordeste} (Rio de Janeiro: Fundação Casa de Rui Barbosa, 1982).} Nos
primeiros três textos, o poeta se apresenta
e dedica seus versos, usando a tradicional redondilha maior,
a sextilha de setissílabos. A seguir, reproduzimos a letra de
dois CD's recém-publicados sobre os dois santos
populares nordestinos, Frei Damião e Padre Cícero. A vida e a morte
de ambos é narrada bem no estilo do cordel, e não poderia
faltar o encontro no céu da dupla: o ``encontro (no
céu)'' de personagens conhecidos é um tema tradicional dos
cordelistas. É de notar que vários desses poemas não vêm
em sextilhas, mas em metros mais raros: em martelo com
ou sem mote; em ``\textit{galope à beira-mar}''
(décimas de hendecassílabos), sendo que o último verso deve conter a
última (ou mais) palavras do mote; em ``cantando galope
à beira do mar''; em ``\textit{mourão
perguntado}'' (oitavas de
setissílabos, dialogadas entre dois poetas) ou ainda em
``\textit{martelo alagoano}'' (décimas com rimas em
``-\textit{ano}'' no sexto, sétimo e décimo versos ).''\footnote{ Ibidem, p. 31.}

Selecionamos ainda três folhetos em sextilhas clássicas:
uma poética ruminação existencial; o
``repente'' que Oliveira cantou em parceria no XVIII \textit{Salon du Livre} em
1998, e o que o poeta fez questão de dedicar aos leitores
deste livro: um ``folheto de atualidade''
inédito, um repente
que improvisou sobre a corrupção no Brasil -- com
material ``quente'' e atualizado -- assim como
faziam os poetas das feiras do interior.

Retivemos ainda uma dúzia de ``\textit{martelos}'' com
mote, geralmente de poucas estrofes, sobre um leque de assuntos:
sérios ou humorísticos, devotos ou amorosos, sobre paz e
guerra, sobre reforma agrária e política, sobre o sertão e as
cidades. Em seguida vêm uma série de \textit{martelos sem mote}
e ainda três \textit{décimas de setilhas}, também, sobre vários
assuntos.

Selecionamos ainda um mostruário das várias facetas
da poética de Oliveiras: poemas de sua própria lavra em
metros variados: um \textit{galope à beira-mar} (dez versos de onze
sílabas, o sexto e o sétimo rimando com
``-\textit{ar}'' de ``mar''); uma \textit{toada alagoana} (estrofe de nove versos, seis setissílabos
com três trissílabos intercalados); uma \textit{parcela ligeira}
(décimas de pentassílabos, em ritmo acelerado); um
\textit{oitavão rebatido} (estrofes de setissílabos, com rima em
``ido'' no segundo, quarto e oitavo versos),
um \textit{oito pés a
quadrão} (versos de setissílabos, o quinto e sexto rimando
com ``-ão'' de
``quadrão'') e uma \textit{toada gemedeira} (uma
sextilha interrompida pela exclamação invariável: ``Ai, ai!
Ui, ui!'')

Como ``despedida do cantador'', o poeta
improvisou um \textit{dez pés a quadrão} (décimas de setissílabos, versos cantados
alternadamente por uma dupla, com rimas no sexto e
sétimo versos em ``-\textit{ão}'' de
``quadr\textit{ão}'').

\pagebreak

\section{Depoimento de Oliveira de Panelas}

\textit{O que é uma noite de cantoria?}\bigskip

Uma noite de cantoria é um dos mais misteriosos rituais
que podem existir: é suspense e encanto, é medo, é erro, é
acerto; é delírio, suor e desafio.

A chamada cantoria de pé-de-parede tem a duração de
quatro a cinco horas --, começa às 21:00 e vai até uma ou
duas da manhã. Antigamente cantava-se sempre até o
amanhecer. A vestimenta era a rigor, isto é: terno completo
e até colete. Era a chegada dos cantadores bem trajados
que impunha respeito pela arte que exerciam.

O ouvinte da cantoria, também chamado de apologista,
é o responsável maior pela desenvoltura poética dos
cantadores. O trabalho cantado dos repentistas violeiros é
com frequência realizado em duplas, pois existem
modalidades que precisam ser perguntadas e respondidas
alternadamente, além dos duelos em desafio etc. Um show,
uma apresentação, um disco, poderão ser feitos por um
único cantador; nunca uma noite de cantoria.

No palco, há uma distância de aproximadamente um
metro que separa os dois cantadores; qualquer diferença
será logo corrigida por um dos artistas. Nas pugnas de
cantoria de pé-de-parede, os cantadores devem, por
tradição, cantar sentados. A musa dos poetas se inibe
quando os ouvintes se acomodam de forma dispersa.

O silêncio, a atenção, o respeito e os aplausos são talismãs
que estimulam a competência quando são direcionados
equitativamente aos dois cantadores.

Na ética da cantoria, um dos parceiros deve, quando
tiver conhecimento, prosseguir o assunto que o colega
iniciou. Cada um começa por sua vez, segundo a norma.
Isso é chamado de \textit{Baionada}: a cantoria dividida em baiões.

``O desafio, o humor e a crítica política predominam na
preferência dos ouvintes.''\footnote{ Oliveira de Panelas.
\textit{Na cadência do martelo}. João Pessoa: s.e., s.d., pp. 6-7.}

\bigskip

\textit{Quais foram os melhores momentos?}\bigskip

Primeiro, em 1996, nossa viagem a Cuba -- Otacílio
Batista e eu --, a convite da ``Solidariedade ao Povo
Cubano''. Cantamos na presença de Fidel Castro no Teatro
José Heredia em Santiago de Cuba e a sessão foi transmitida
para a ilha toda. Fomos muito bem tratados e aplaudidos no
teatro, mas no hotel onde nos alojaram éramos como João
Ninguém.

Em 1997, por intermédio de Ariano Suassuna, então
Secretário de Educação do Estado de Pernambuco - mas
ele não foi conosco nunca sai do país --, Otacílio Batista e
eu fomos convidados para um congresso na cidade do Porto,
em Portugal. Era no mês de dezembro, e estava frio, frio
mesmo. Otacílio viajou com sua roupa fina que usava aqui,
achando que o frio de Portugal seria como o do Rio de
Janeiro. Mas chegando ali, a primeira coisa que fez foi ir
comprar um agasalho bem quente. Foram uns cinquenta
pernambucanos para participar deste congresso, que na
realidade tratava de gastronomia: e nós dois, paraibanos,
servimos ali para completar o cardápio de arte culinária
pernambucana.

A nossa experiência mais marcante foi nossa ida para
Paris em março qe 1998. Foi realizado ali o XVIII Salão
do Livro, que naquele ano dava um destaque especial à
literatura brasileira. A professora Idelette Muzart tinha
costurado nossa ida -- Leonardo Vitorino e eu -- para
ilustrar a noite dos autógrafos dos escritores brasileiros.
Estavam lá José Sarney, Paulo Coelho, Chico Buarque,
Jorge Amado, Lygia Fagundes Telles, João Ubaldo
Ribeiro, Saulo Ramos e muitos outros dos nossos famosos
autores. Aquela noite apresentamos duas cantorias: a
primeira, mais didática, dando uma amostra cantada de
uma série de tipos diferentes de poesia popular: martelo,
mourão, galope e outros, e eu explicando a coisa. Em
seguida demos um show, improvisando versos sobre o
evento. Tudo aquilo foi traduzido na hora, gravado e
publicado, no outro dia já, em forma de livro. Os autores
brasileiros puxaram uma conversa animada com a gente.
Estava ali também a primeira dama, Dona Ruth Cardoso:
por acaso ela me ouvira cantar, uma semana antes, na casa
de doutor José Paulo Cavalcanti: reconheceu-me e
conversou comigo''.\footnote{ Da entrevista com Oliveira,
citada acima.}

%Maurice Van Woensel
%
%
%\bigskip
%
%Universidade Federal da Paraíba

\addtocontents{toc}{\bigskip}
\section{Bibliografia}

\begin{bibliohedra}[]

\tit{}{\scshape\Large\MakeTextLowercase{Obras de Oliveira de Panelas}}

\bigskip
\tit{}{\bfseries\large Livros}


\tit{}\textit{O comandante do Planeta Médio}. João Pessoa: Tipografia Alvorada, 1978.

\tit{}\textit{Poesia liberdade: Temas inesquecíveis}. João Pessoa: Edições Jagurá, 1979.

\tit{}\textit{Poemas iluminados}. João Pessoa: Secretaria de Educação e Cultura, 1983.

\tit{}\textit{Gozação de cantador}. João Pessoa: Editora Gráfica Literária, 1983.

\tit{}\textit{Na cadência do martelo}. João Pessoa: edição particular, 1993.

\tit{}\textit{Poemas alternativos}. João Pessoa: edição particular, 1984.

\tit{}\textit{O Poeta gozador}. João Pessoa: edição particular, 1997.

\tit{}\textit{Nas pegadas de Champagnat}. (Cordel comemorativo da canonização de Marcelino Champagnat). João Pessoa: Colégio
Pio \textsc{x}, 1999.

\tit{}\textit{Dois poetas do povo e da viola} (Oliveira de Panelas e Otacílio Batista). João Pessoa: Funesc, 1996.

\bigskip
\tit{}{\bfseries\large Discos \textsc{lp}}


\tit{}\textit{Coletânea de repentistas}: Série Brasil Caboclo. Crazy, 1975.

\tit{}\textit{Só Deus improvisa mais}. Com Otacílio Batista. Continental, s.d.

\tit{}\textit{Nordeste e repentes}. Participação de Daudeth Bandeira. Discos Marcos Pereira, s.d.

\tit{}\textit{Regras de cantoria}. Com Otacílio Batista. \textsc{mec}, s.d.

\tit{}\textit{O perguntador}, \textsc{cbs}, s.d.

\tit{}\textit{O Nordeste anda, canta}. Com Geraldo Amâncio. Produções Cariri, s.d.


\bigskip
\tit{}{\bfseries\large Discos \textsc{cd}}


\tit{}\textit{Amor Cósmico}. João Pessoa: Produções Luz do Sol, 1995.

\tit{}\textit{\textsc{i} Campeonato de poetas-repentistas do Brasil}. (Com Ismael Pereira, Valdir Telles e Sebastião da Silva.) São
Paulo: Centro Popular de Cultura (\textsc{cpc}), 1997.

\tit{}\textit{Campeonato nordestino de poetas-repentistas}. Edição particular, 1998.

\tit{}\textit{O bardo cantador}. São Paulo: O Pequizeiro, 1998.

\tit{}\textit{Padre Cícero }(com Otacílio Batista). João Pessoa: Acácia Produções, 1998.

\tit{}\textit{Vida e morte de Frei Damião} (com Juvenal Oliveira). João Pessoa: Acácia Produções, 1998.

\tit{}\textit{40 anos de cantoria}. São Paulo: O Pequizeiro, 2001.


\bigskip
\tit{}{\scshape\Large\MakeTextLowercase{Bibliografia geral}}


\tit{Almeida}, Átila Augusto F. de \& José Alves Sobrinho. \textit{Dicionário bio-bibliográfico de repentistas e poetas de
bancada}. João Pessoa --- Campina Grande: Editora Universitária --- Centro de Tecnologia, 1978.

\tit{Batista}, Sebastião Nunes. \textit{Antologia da literatura de cordel}. Natal: Fundação José Augusto, 1977.

\titidem. \textit{Poética popular do Nordeste}.  Rio de Janeiro: Fundação Casa de Rui Barbosa, 1982.

\tit{Borges}, Neuma Fechine (Prefácio). In: Panelas, Oliveira de \& Otacílio Batista. \textit{Dois poetas do povo e da viola}.
João Pessoa: Funesc, 1996.

\tit{Cascudo}, Luís da Câmara. \textit{Vaqueiros e cantadores}. Belo Horizonte: Itatiaia, 1984.

\tit{Curtius}, Ernst R. \textit{Literatura europeia e Idade Média Latina}. São Paulo: Edusp, 1996.

\tit{}\textit{Dicionário literário da Paraíba}. João Pessoa: Editora A União, 1990.

\tit{Ferreira}, Jerusa Pires. \textit{Armadilhas da memória}. Salvador: Fundação Casa de Jorge Amado, 1991.

\tit{Gotlib}, Nádia Battella (org.). Luis Vaz de Camões. São Paulo: Abril Educação, 1980.

\tit{Menéndez-Pidal}, Ramón. \textit{Flor nueva de romances viejos}. Madrid: Espasa-Calpe, 1941.

\tit{Moisés}, Massaud. \textit{Dicionário de termos literários}. São Paulo: Cultrix, 1985.

\tit{Santos}, Idelette Muzart Fonseca dos. \textit{Em demanda da poética popular}. Campinas: Editora da Unicamp, 1999.

\titidem. \textit{La littérature de cordel au Brésil}. Paris: L'Harmattan, 1997.

\tit{Pinto-Correia}, J. David (org.). \textit{Romanceiro tradicional português}. Lisboa: Editorial Comunicação, 1984.

\tit{Polari}, Neide. \textit{Vinte anos de parceria}. In: Panelas, Oliveira de \& Otacílio Batista. \textit{Dois poetas do
povo e da viola}. João Pessoa: Funesc, 1996.

\tit{Roy}, Claude. \textit{Trésor da la littérature populaire}. Paris: Seghers, s.d.

\tit{Viana}, Chico \& Maurice Van Woensel. \textit{Poesia medieval ontem e hoje}. João Pessoa: \textsc{cchla} --- Editora Universitária,
1998.

\end{bibliohedra}
