
\cleardoublepage

\addtocontents{toc}{\bigskip}

\chapter{O poeta se apresenta}

\begin{verse}
Saí da inércia um dia,\\*
Que é estado latente;\\*
Em germe fui transformado;\\*
Tomei forma diferente:\\*
Desenvolvi-me no óvulo,\\*
Daí passei a ser gente

Escalo o voo duma águia;\\*
Passo acima do condor;\\*
Eu sou o \textsc{homem das nuvens}\\*
Dos Alpes, trago uma flor\\*
Para oferecer, na Terra,\\*
A quem mais me der amor

Velho planeta de guerra,\\*
De sua rota, não sai:\\*
Gira com velocidade,\\*
Saber, saber pra onde vai\ldots{}\\*
Talvez conduzindo um povo\\*
Pras moradas do meu Pai.
\end{verse}

\chapter{Saco de cego}

\begin{verse}
No saco de cego tem:\\*
Arroz, feijão e farinha\\*
Fubá de milho e sardinha,\\*
Tem pão, café, tem xerém\\*
Algum dinheiro também,\\*
Sal, bolacha, amendoim,\\*
Tem pé de porco e pudim,\\*
Tem tripa e carne de bode\\*
Só outro cego é quem pode\\*
Ter tanta salada assim

No meu livro tem doutrina,\\*
Moral e filosofia,\\*
Humorismo, poesia.\\*
Romantismo, disciplina,\\*
Oração e medicina,\\*
Filantropia e direito,\\*
Tem mentira e preconceito,\\*
Na misturada não nego:\\*
Meu livro e saco de cego\\*
É tudo do mesmo jeito.
\end{verse}


\chapter{Dedicatória}

\begin{verse}
Por tudo que aconteceu\\*
E está acontecendo agora

Pelos que pagam por seus crimes\\*
Entre os ferros das cadeias,

Pelas almas torturadas,\\*
de mil noites mal passadas,

Pelos fracos desgraçados,\\*
Pelos mal recompensados\\*
Sofredores e oprimidos,

Pelos que foram pra guerra\\*
E, depois, voltaram à terra,\\*
Com queixa dos hospitais,

Pelos que perderam a vida\\*
Em terra desconhecida\\*
Pra sustentar nossa paz,

Pelos que morrem de medo\\*
De perder logo cedo,\\*
A doce mulher querida,

Pelos que nada gozaram;\\*
Pelos que só encontraram\\*
Duras frustrações na vida,

Pelos que são humilhados;\\*
Pelos que sofrem calados,\\*
Temendo perder o pão,

Pelos que muito lutaram\\*
E ainda não acharam\\*
Um pedacinho de chão,

Esse livro é uma escola\\*
Pra quem prende na gaiola\\*
As aves que livres voam,

Pelos fracos que reclamam;\\*
Pelos filhos que não amam;\\*
Pelos pais que não perdoam.
\end{verse}


\chapter{Vida e morte de Frei Damião}

\begin{verse}
Frei Damião de Bozzano\\*
Nasceu e morreu feliz\\*
Andarilho do Evangelho\\*
Assim seu destino quis\\*
Ser seguidor da grandeza\\*
De São Francisco de Assis

Lenitivo dos humildes\\*
Que bebiam sua fé\\*
Seguidores incansáveis\\*
Faziam viagem a pé\\*
Pra terem perto um amigo\\*
De Jesus de Nazaré

Por um período de luto\\*
nossa crença se veste\\*
Morre um irmão da pobreza\\*
Vai-se um santo do Nordeste\\*
Pra fazer missões divinas\\*
Lá na milícia celeste

O seu carisma infinito\\*
Atravessou gerações\\*
Plantou sementes de fé\\*
Nos terrenos das missões\\*
Nasceram frutos da Paz\\*
Nas almas das multidões

Foi em mil e oitocentos\\*
E noventa e oito, o ano\\*
No dia 5 de agosto\\*
No torrão italiano\\*
Nasceu Pio Giannotti\\*
Frei Damião de Bozzano

Quase um século de existência\\*
De funções sacerdotais\\*
O mensageiro da fé\\*
O peregrino da Paz\\*
A missão de ganhar almas\\*
Aumentou cada vez mais

Defensor dos bons costumes\\*
Do pudor e da moral\\*
Regente das Leis de Deus\\*
Um protetor sem igual\\*
Conselheiro dos humildes\\*
Um irmão celestial

A sua fé poderosa\\*
Transparente como véu\\*
Na terra não quis medalha\\*
Taça, tesouro e troféu\\*
Seus sacrifícios terrenos\\*
Buscam as heranças do céu

Seus milagres foram muitos\\*
Nestas longas caminhadas\\*
Demônios foram expulsos\\*
Almas foram consoladas\\*
Nos corações dos fiéis\\*
São suas preces guardadas

Dizem seus ensinamentos\\*
Ligados à tradição\\*
O divórcio não é bom\\*
Pílula não é solução\\*
Que casamento de Igreja\\*
É a perfeita união

Assim foi Frei Damião\\*
Que teve a paz por irmã\\*
Aurora do Reino Santo\\*
Frutos da árvore cristã\\*
Lembrado por toda a vida\\*
Ontem hoje e amanhã.
\end{verse}


\chapter[Frutos do amor]{Frutos do amor\\\smallskip\textit{(martelo com mote)}}

\begin{verse}
Este homem de porte pequenino\\*
Foi exemplo de paz e liberdade\\*
Divulgando o amor e a caridade\\*
Para o homem sofrido nordestino\\*
Eis um dia por ordem do divino\\*
Foi marcado seu dia e sua hora\\*
O Nordeste entristece e gente chora\\*
E ninguém vai esquecê-lo nunca mais\\*
\textit{Nosso Frei Damião plantou a paz}\\*
\textit{Colheu frutos do amor e foi embora}

Nosso Frei Damião em seu reinado\\*
Tinha o fruto do amor por alimento\\*
Seu semblante mostrava sofrimento\\*
De ver tantos sofrendo ao seu lado\\*
Alertava pra todos que o pecado\\*
É um mal que afeta dentro e fora\\*
Quem não crê, quem não ama, quem não ora\\*
Não terá regalias siderais\\*
\textit{Nosso Frei Damião plantou a paz}\\*
\textit{Colheu frutos do amor e foi embora}

Nosso amado e irmão Frei Damião\\*
Do sertão fez sua pátria mor\\*
No Nordeste fez outra bem maior\\*
Para dar mais conselhos à multidão\\*
Seu remédio era feito de oração\\*
Sua luz era feita de aurora\\*
Onde Deus colocou os imortais\\*
\textit{Nosso Frei Damião plantou a Paz}\\*
\textit{Colheu frutos do amor e foi embora}

O conselho do Frade nos dizia\\*
Que o melhor casamento é da Igreja\\*
Que pílula tomada nunca seja\\*
Todo concubinato é sem valia\\*
Minissaia com dança é fantasia\\*
O divórcio é cruel pra quem se adora\\*
Confissão não tem dia e nem tem hora\\*
Namorar só na frente dos seus pais\\*
\textit{Nosso Frei Damião plantou a Paz}\\*
\textit{Colheu frutos do amor e foi embora.}

Aprendeu com Vernach e com Bilot\\*
Gazzarine Capello e com Degrand\\*
Muitos outros formando o mesmo clã\\*
Por Basílio Pompili se ordenou\\*
Frei Fernando depois lhe acompanhou\\*
Meio século de vida mundo afora\\*
Com as Leis do Divino colabora\\*
Pelas suas funções sacerdotais\\*
\textit{Nosso Frei Damião plantou a Paz}\\*
\textit{Colheu frutos do amor e foi embora}

Em direito canônico foi formado\\*
Em toda teoria e toda prática\\*
Pela teologia dogmática\\*
Nosso Frei Damião foi diplomado\\*
Foi em filosofia graduado\\*
Num saber que ao espírito revigora\\*
Conviveu com o Nordeste, fauna e flora,\\*
Com seu povo sofrido e com seus ais\\*
\textit{Nosso Frei Damião plantou a Paz}\\*
\textit{Colheu frutos do amor e foi embora.}
\end{verse}


\chapter[Nas águas do mar]{Nas águas do mar\\\smallskip\textit{(galope à beira mar)}}

\begin{verse}
Os grandes milagres do Frei Capuchinho\\*
O Mundo conhece, o Nordeste sabe\\*
São tantos já feitos que à terra não cabe\\*
Nas horas difíceis mostrou o caminho\\*
Àqueles que vivem sofrendo sozinhos\\*
Sem ter um amigo pra lhe acompanhar\\*
O Frei Capuchinho na hora de orar\\*
Aos seres sofridos lhes orientava\\*
Divinos conselhos confessando dava\\*
No céu e na terra, e \textit{nas águas do mar}

Pra Frei Damião são muitos os pecados\\*
Daqueles que vivem sem lei, sem moral\\*
Bebendo o efeito da seiva do mal\\*
Que são pervertidos e degenerados\\*
Roubando e matando os necessitados\\*
Que nunca tiveram pra quem apelar\\*
Então estes monstros que vivem a matar\\*
Jogando os valores morais para fora\\*
Com esses o diabo se atrai toda hora\\*
Querendo afogá-los \textit{nas águas do mar}

São mais de seis décadas constantes\\*
Que no Brasil mostra seus dons soberanos\\*
Morreu com um século faltando dois anos\\*
Em prol da virtude se fez um gigante\\*
De Cristo foi ele o representante\\*
N a grande tarefa de catequizar\\*
Por onde chegava fazia um altar\\*
Orava com o povo, cantava e saía\\*
Vivia com o povo pro canto que ia\\*
Até nas areias \textit{das águas do mar}

Os dez Mandamentos pra Frei Damião\\*
São do mesmo jeito que Moisés deixou\\*
Da forma que Cristo guardou e pregou\\*
Sem ter uma vírgula de alteração\\*
Porque lei divina não tem mutação\\*
Foi isto que ele vivia a pregar\\*
Os Frades na certa vão continuar\\*
A grande tarefa do missionário\\*
Com Deus a seu lado neste itinerário\\*
Falando com ele na terra \textit{e no mar}

Pregou numa terra de gente sofrida\\*
Politicamente tão discriminada\\*
No país Nordeste fez sua morada\\*
Mostrando a seu povo o caminho da vida\\*
Da simplicidade fez sua jazida,\\*
O ouro da fé ele quis ensinar\\*
E a caridade em primeiro lugar\\*
Assim eram todos seus ensinamentos\\*
Com base nas regras dos Dez Mandamentos\\*
Criados pra terra pro céu e \textit{pro mar}

Já vão dois mil anos que Cristo vivia\\*
Por lagos e praias, cidades e vilas\\*
Levando consigo plateias tranquilas\\*
Que só aos humildes Jesus pretendia\\*
Pescadores simples o Mestre queria\\*
Seu grande Evangelho na terra mostrar\\*
Frei Damião veio parar num lugar\\*
Tal qual Cristo fora pra lá enviado\\*
O sonho de Cristo por ele imitado\\*
Nas águas atlânticas \textit{da beira do mar}.
\end{verse}


\chapter[O Santo Frei Damião]{O Santo Frei Damião\\\smallskip\textit{(décimas de sete pés com mote)}}

\begin{verse}
Frei Damião conviveu\\*
Por vila, sítio e cidade\\*
Com sua simplicidade\\*
Milhões de conselhos deu\\*
Quase cem anos viveu\\*
Curando endemoniado\\*
Exorcizando o pecado\\*
E todo tipo de espanto\\*
\textit{Frei Damião é um Santo}\\*
\textit{Já nasceu canonizado}

Nosso Pio Giannotti\\*
O seu nome verdadeiro\\*
No Nordeste um caminheiro\\*
Um eterno sacerdote\\*
Recebeu de Deus o dote\\*
Pra divulgar seu reinado\\*
Porém por Deus foi chamado\\*
No céu está, eu garanto,\\*
\textit{Frei Damião é um Santo}\\*
\textit{Já nasceu canonizado}

Espírito conservador\\*
Alguém dirá, mas não é\\*
Em Jesus de Nazaré\\*
Teve sempre seu mentor\\*
Deu conselho ao pecador\\*
Que pedia angustiado\\*
Amparou o deserdado\\*
Na pureza do seu manto\\*
\textit{Frei Damião é um Santo}\\*
\textit{Já nasceu canonizado}

Humilde como ninguém\\*
Esse santo italiano\\*
Seu coração soberano\\*
À terra outro igual não tem\\*
Seu nome vai ao além\\*
Pela voz do povo amado\\*
Viveu longe do pecado\\*
Seu coração soberano\\*
\textit{Frei Damião é um santo}\\*
\textit{Já nasceu canonizado}

A luz de Frei Damião\\*
Tinha raios soberanos\\*
Com sessenta e cinco anos\\*
De constante pregação\\*
Ensinou na oração\\*
O que Deus tinha mandado\\*
No Evangelho Sagrado\\*
Encontrou seu melhor canto\\*
\textit{Frei Damião é um Santo}\\*
\textit{Já nasceu canonizado}

Conheci Frei Damião\\*
Pregando pelo Nordeste\\*
Na pureza que se veste\\*
Com o rosário na mão\\*
Rezando a santa oração\\*
De Jesus Imaculado\\*
Sempre amando sempre amado\\*
Longe do mundo profano\\*
\textit{Frei Damião é um Santo}\\*
\textit{Já nasceu canonizado}

Suas primeiras missões\\*
Tinha Padre Cícero vivo\\*
Lampião ainda ativo\\*
Reinando pelos sertões\\*
Frei Damião nos sermões\\*
Fazia: o povo abismado\\*
Esquecer o lado errado\\*
A dor, a tristeza, o pranto\\*
\textit{Frei Damião é um Santo}\\*
\textit{Já nasceu canonizado}.
\end{verse}


\chapter{Nosso Pio Giannotti}

\begin{verse}
Nosso Pio Giannotti\\*
Frei Damião trouxe o dote\\*
De excelso sacerdote\\*
De verdadeiro pastor\\*
Seu espírito pregador\\*
Buscando existências calmas\\*
Pro rebanho do Senhor

Este divino tripé\\*
Pra Frei Damião é\\*
Três coisas essenciais\\*
Pela confissão se faz\\*
O Santo depoimento\\*
Dando a Deus conhecimento\\*
De amor cada vez mais

Muito pouco ele dormia\\*
Quando amanhecia o dia\\*
Orava muito e saía\\*
Reunindo multidões\\*
Os fiéis se encontravam\\*
Atentamente escutavam\\*
Seus milagrosos sermões

Amigo irmão e profeta\\*
De Deus o grande estafeta\\*
Não esquecia a dieta\\*
Sobre pretexto nenhum\\*
Era fora do comum\\*
Mesmo já cansado e velho\\*
Ser da vida um Evangelho\\*
De oração e jejum

Frei Damião de Bozzano\\*
Um espírito sobre-humano\\*
Hoje está com o Soberano\\*
E com Jesus de Nazaré\\*
Frei Damião foi e é\\*
E será eternamente\\*
A bússola de minha gente\\*
E o farol de nossa fé.
\end{verse}


\chapter{Ensinamentos}

\begin{verse}
Frei Damião nosso santo\\*
Conviveu só com o bem\\*
Não feriu os poderosos\\*
Cruéis que a terra tem\\*
Ensinou sem fazer guerra\\*
Sua luz brilhou na terra\\*
Nos céus brilhará também

Este andarilho incansável\\*
Ganhou um lar diferente\\*
Foi residir nas estrelas\\*
Acredito piamente\\*
De lá enviando luzes\\*
Tirando o peso das cruzes\\*
Da existência da gente

Bendito seja este nome\\*
Que nesta terra viveu\\*
Que orou pelos pequenos\\*
Quem o ouviu aprendeu\\*
Partiu para a vida eterna\\*
Pra dimensão que governa\\*
O Cristo Rei Galileu

O tempo que aqui viveu\\*
Fez seus sermões pela Paz\\*
Pela lei dos bons costumes\\*
De amor pelos casais\\*
Pudor, justiça e verdade\\*
Que a nossa atualidade\\*
Está precisando demais

Seu espírito abnegado\\*
Nunca deixou-se render\\*
Pelas leis destruidoras\\*
Dos que matam pra vencer\\*
São facínoras disfarçados\\*
Sanguinários camuflados\\*
Que fazem o mal por prazer

Nunca incentivou seu povo\\*
A fazer combate armado\\*
Por contra os poderosos\\*
Ou o plano é bem traçado\\*
Numa forma inteligente\\*
Ou se faz inutilmente\\*
Muito sangue derramado

Sempre foi sua tarefa\\*
Por leis espirituais\\*
Buscando ensinar aos homens\\*
Verdades celestiais\\*
Exemplificando almas\\*
Com suas palavras calmas\\*
Frei Damião fez demais

Não trepidou na jornada\\*
Na canseira da idade\\*
Foi fiel a seus preceitos\\*
Puríssima sinceridade\\*
Os seus sermões começaram\\*
Do mesmo jeito findaram\\*
Na mesma fidelidade

Nunca fez nas suas prédicas\\*
Alusões extravagantes\\*
Porque de um povo sofrido\\*
Viu anseios cruciantes\\*
Disse pra os devotos seus\\*
Que estão no Reino de Deus\\*
As coisas mais importantes.
\end{verse}


\chapter{Encontro no Céu de Frei Damião e Padre Cícero}

\begin{verse}
Morreu e foi para o céu\\*
O nosso Frei Damião\\*
Assim que no céu chegou\\*
Quem estava no portão?\\*
Era um santo nordestino\\*
O padre Cícero Romão

Pergunta Cícero Romão:\\*
Damião você aqui?\\*
Pode entrar a casa é sua\\*
A entrada é por ali\\*
Você vai ter o espaço\\*
Do jeito que eu recebi

Frei Damião disse: aqui\\*
O clima é bem diferente,\\*
Ainda não conhecia\\*
Mas gostei do ambiente\\*
Tomara que não estejam\\*
Filmando o papo da gente

Frei Damião entre e sente\\*
Pode ficar à vontade\\*
Este céu é todo nosso\\*
Porém por sua bondade\\*
Me diga como deixou\\*
As coisas na humanidade

Trouxe muita novidade\\*
Aqui pra o Reino da Glória\\*
Talvez com cem anos conte\\*
O que guardei na memória\\*
Dê-me um canto pra sentar\\*
Que eu vou contar a história

Capriche bem na memória\\*
Pode contar detalhado\\*
Pois quando parti da terra\\*
Não tive mais resultado\\*
Ninguém me disse mais. nada\\*
Eu estou desinformado

Vou dizer bem compassado\\*
A terra como é que vai\\*
O homem perdeu a fé\\*
Filho não respeita pai\\*
Agora toda semana\\*
Um avião sobe e cai

Eu calculo como vai\\*
O povo do meu sertão\\*
Ó Damião conte mais\\*
Estou prestando atenção\\*
Diga se depois que eu vim\\*
Fizeram alguma invenção

Criaram televisão\\*
No seu tempo não havia\\*
Uma faca de dois gumes\\*
Eu mesmo não assistia\\*
Porque sempre atrapalhava\\*
As procissões que fazia

Credo em cruz Ave Maria\\*
Padre Cícero disse assim\\*
Depois disse: Damião:\\*
Você não ache ruim\\*
Me diga se tem ainda\\*
Romeiro falando em mim

O seu nome não tem fim\\*
Pôr lá é sempre lembrado\\*
Mas os romeiros de hoje\\*
Não são como os do passado\\*
Chamam seu nome em inglês\\*
Andam em carro importado

Está muito complicado\\*
Conte mais quero saber\\*
Me diga como é que vai\\*
O governo no poder\\*
Se está trabalhando bem\\*
Ou pondo o resto a perder

Faz medo até lhe dizer\\*
O que por lá acontece\\*
Tem tanto governo ruim\\*
Que eu não sei de onde aparece\\*
Do jeito que o mundo está\\*
Que você não conhece

Qual a invenção que cresce\\*
Lá na terra atualmente?\\*
A televisão já sei\\*
Diga outra diferente\\*
Se caso tiver lembrado\\*
Conte aqui pra mim somente

Tem robô metido a gente\\*
Que os ricos compram lá fora\\*
Basta apertar o botão\\*
Que o bicho atende na hora\\*
Quem tiver empregado perde\\*
Que o coisa manda ir embora

Valha-me Nossa Senhora\\*
Eu estou sem entender\\*
A máquina ganha do homem\\*
É isso que quer dizer\\*
E o homem perde emprego\\*
Essa aqui me fez tremer

Você nem queira saber\\*
Como vai o pecador\\*
Inventaram uma internet\\*
Um negócio sedutor\\*
Que até os santos estão sendo\\*
Feitos no computador

Mas Damião que horror\\*
E as coisas lá vão assim?\\*
Pelo que você me fala\\*
O mundo já está no fim\\*
Eu acho que lá tem gente\\*
Que nem se lembra de mim

Pode acreditar em mim\\*
Tem coisa mais complicada\\*
Homem casando com homem\\*
A Lei já foi aprovada\\*
Quem for conversar besteira\\*
Sai de barriga arranhada

Damião isso é piada\\*
Eu morro e não acredito\\*
Essa aí me deixou tonto\\*
Parado, pasmo e aflito\\*
Eu acredito em você\\*
Mas acho o troço esquisito

Você achou esquisito\\*
Ainda tem muito mais\\*
Estou só contando algumas\\*
Tem outras mais atuais\\*
Só não vou lhe contar tudo\\*
Senão você cai pra trás

Tenho pena dos mortais\\*
Das coisas que estão fazendo\\*
Pelo que você me disse\\*
Tem muita gente sofrendo\\*
Que lutam sem ter direito\\*
Que pagam sem estar devendo

Por lá ficaram dizendo\\*
Que você vai retornar\\*
Se quiser dou-lhe um conselho\\*
Melhor fique em seu lugar\\*
Quem está morando no céu\\*
Não tem pra que se mudar

Damião deixe eu pensar\\*
Que ainda não refleti\\*
Acho que vou atendê-lo\\*
Se de lá eu já saí\\*
Já me encontrei com você\\*
Voltar pra fazer o quê?\\*
Vou me aguentar por aqui.
\end{verse}


\chapter[Salve, Salve, Padre Cícero]{Salve, Salve, Padre Cícero\\\smallskip\textit{(do cd Padre Cícero)}}

\begin{verse}
Salve, salve Padre Cícero\\*
Meu Padim do Juazeiro\\*
Seus conselhos e seus sermões\\*
Me dão fé o tempo inteiro\\*
Redobrando a minha crença\\*
De coração de romeiro

Sou romeiro nordestino\\*
Das estradas do sertão\\*
Devoto da Mãe das Dores\\*
Do Padre Cícero Romão\\*
Ouço as mensagens da alma\\*
E trago a fé no coração

Foi em mil e oitocentos\\*
E quarenta e quatro o ano\\*
Em Crato no Ceará\\*
Com talento sobre-humano\\*
Que nasceu o Padre Cícero\\*
Por ordem do Soberano

No ano mil novecentos\\*
E trinta e quatro morreu\\*
Quase um século de existência\\*
O Santo Padre viveu\\*
Deixou milhões de conselhos\\*
Que o povo nunca esqueceu

Os sermões do Padre Cícero\\*
Todo devoto guardava\\*
O cangaceiro tremia\\*
O jagunço acreditava\\*
Até o ateu que ouvisse\\*
Cruzava as mãos e rezava

Seu lado misterioso\\*
Tinha grandeza inconteste\\*
Previa ano de inverno\\*
De fome de seca e peste\\*
Um homem predestinado\\*
Pra conviver com o Nordeste

Consolava os infelizes\\*
Nas confissões que fazia\\*
Nas orações que rezava\\*
Na força que possuía\\*
Triplicando a luz da fé\\*
Da multidão que lhe ouvia

As perseguições sofridas\\*
Suportou resignado,\\*
Foi suspenso pelo clero\\*
Depois foi reintegrado\\*
Deus não permite obstáculo\\*
A quem é predestinado

Quando houve a suspensão\\*
Das funções sacerdotais\\*
Não podia dizer missa\\*
Pelas ordens clericais\\*
Mas o número de devotos\\*
Aumentou cada vez mais

Foi o milagre da hóstia\\*
Que tanto espanto causou\\*
Que em sangue de Jesus\\*
Dizem que se transformou\\*
Como relíquia divina\\*
Que Deus do Céu enviou

Tinha o Padre Cícero auxílio\\*
De Jesus de Nazaré\\*
Romeiro pra lhe assistir\\*
Fazia viagem a pé\\*
Era distância vencida\\*
Pelo tamanho da fé

Foi padroeiro e padrinho\\*
Fiel amigo e irmão\\*
Foi Prefeito quinze anos\\*
Deu patente a Lampião\\*
Levando o Rei do Cangaço\\*
Ao posto de capitão

Era um condutor de massas\\*
Homem de espírito forte\\*
Mesmo assim por muitos anos\\*
Bem depois de sua morte\\*
É seu nome um mito vivo\\*
Em Juazeiro do Norte.
\end{verse}


\chapter[Meu Padrinho Viajou]{Meu Padrinho Viajou\\\smallskip\textit{(décimas de sete com mote)}}

\begin{verse}
Vou com Deus não vou sozinho\\*
Num caminhão de romeiro\\*
Eu vou para o Juazeiro\\*
Vou visitar meu Padrinho\\*
É muito longe o caminho\\*
Muita poeira e sol quente\\*
Mas o romeiro não sente\\*
Quando dá fé, já chegou\\*
\textit{Meu Padrinho viajou}\\*
\textit{Vai retornar brevemente}

Romeiro por devoção\\*
Devoto por consciência\\*
Meu Padrinho tem ciência\\*
Pra transformar um cristão\\*
Eu rezo aquela oração\\*
Que ele ensinou pra gente\\*
Deus na guia, Pai na frente\\*
Jesus me mandando eu vou\\*
\textit{Meu Padrinho viajou}\\*
\textit{Vai retornar brevemente}

Não morreu o Padre Santo\\*
Fez somente uma viagem\\*
Aqui ficou sua imagem\\*
Brilhando por todo canto\\*
Nós já esperamos tanto\\*
Pra vê-lo aqui novamente\\*
Tem romeiro impaciente\\*
Lembrando o que ele ensinou\\*
\textit{Meu Padrinho viajou}\\*
\textit{Vai retornar brevemente}

O Nordeste não esquece\\*
O Santo que foi embora\\*
Vai retornar qualquer hora\\*
Rezando uma nova prece\\*
Nossa ansiedade cresce\\*
Ele virá sorridente\\*
Curando a dor dessa gente\\*
Que até hoje não curou\\*
\textit{Meu Padrinho viajou}\\*
\textit{Vai retornar brevemente}

Quando lhe fiz um pedido\\*
A saúde estava em nada\\*
A graça foi alcançada\\*
E meu pedido atendido\\*
Eu fiquei agradecido\\*
Porque estava doente\\*
O senhor se fez presente\\*
Com um gesto me curou\\*
\textit{Meu Padrinho viajou}\\*
\textit{Vai retornar brevemente}

A matriz de Juazeiro\\*
Saudosa plange seu sino\\*
Alertando o Nordestino\\*
Lembrando a todo romeiro\\*
Que o Santo padroeiro\\*
Fez um passeio somente\\*
Mas estará novamente\\*
No canto que começou\\*
\textit{Meu Padrinho viajou}\\*
\textit{Vai retornar brevemente}.
\end{verse}


\chapter[Milagres]{Milagres\\\smallskip\textit{(décimas com mote de dois pés)}}

\begin{verse}
Se o povo diz santo, santo seja\\*
Ninguém tem muito tempo a esperar\\*
Pela grande vontade popular\\*
Padre Cícero é um santo em nossa igreja\\*
É o povo quem pede e quem deseja\\*
Que se atenda essa grande multidão\\*
Reunidos os fiéis em oração\\*
Esperando o momento glorioso\\*
\textit{Padre Cícero será vitorioso}\\*
\textit{No processo de canonização}

Trouxe a luz que o romeiro procurou\\*
Crato foi a cidade onde nasceu\\*
Juazeiro o lugar onde viveu\\*
Mil problemas na vida suportou\\*
Não tremeu, não gemeu, nem fraquejou\\*
Quando esteve em cruel perseguição\\*
Foi ao Papa, obteve permissão\\*
E prosseguiu seu trabalho precioso\\*
\textit{Padre Cícero será vitorioso}\\*
\textit{No processo de canonização}

Era um mês de setembro um belo dia\\*
Para ser' bem preciso, dia três,\\*
Um menino nascido com um mês\\*
Uma estranha linguagem principia\\*
Padre Cícero Romão compreendia\\*
Do guri, cem por cento, a expressão\\*
Ficou toda abismada a multidão\\*
Com aquele guri misterioso\\*
\textit{Padre Cícero será vitorioso}\\*
\textit{No processo de canonização}

Certa vez um sujeito foi fazer\\*
A limpeza de uma chaminé\\*
Nesse Padre, gritou, só tenho fé\\*
Se eu cair desta altura e não morrer.\\*
Nisso ele escorrega e ao descer\\*
Grita o nome do Padre e cai no chão\\*
Quebra braço, costela, pé e mão\\*
Escapou sem ficar defeituoso\\*
\textit{Padre Cícero será vitorioso}\\*
\textit{No processo de canonização}

Dos milagres que o povo tem contado\\*
Ninguém pode contá-lo num só dia\\*
O da moça que há tempo não dormia\\*
O chapéu na parede pendurado\\*
O ceguinho que foi recuperado\\*
Sem sofrer mais problemas de visão\\*
Padre Cícero só fez botar a mão\\*
E o ceguinho saiu todo orgulhoso\\*
\textit{Padre Cícero será vitorioso}\\*
\textit{No processo de canonização}.
\end{verse}


\chapter{Meu Padim, Pade Ciço}

\begin{verse}
Numa linda manhã, às cinco horas\\*
Embalado na voz da passarada\\*
Nasce o Padre Avatar do Nordestino\\*
Quando o sol conduzia a alvorada\\*
A vontade de Deus prenunciava\\*
A missão que lhe estava reservada

Homens brutos perversos do seu tempo\\*
Ao ouvir seus sermões estremeciam\\*
Sua voz era mansa como a brisa\\*
Seus conselhos pra todos que assistiam\\*
Criminosos que fossem a vida inteira\\*
Dos pecados mortais se arrependiam

Sua vida só foi de ensinamentos\\*
De sermão, de conselho e de jejum,\\*
Sua luz foi maior do que a nossa\\*
Por comum todo gênio é incomum\\*
Que a difícil missão de catequista\\*
Jamais pode ser dada a qualquer um

Meu Padim, Pade Ciço está com Deus\\*
Juazeiro, seu palco, iluminado\\*
O país nordestino, sua pátria,\\*
O romeiro é seu filho abençoado\\*
É um santo por tudo que foi feito\\*
E pelo povo já está canonizado

Meu Padim, Pade Ciço está com Deus\\*
Vai voltar qualquer hora, qualquer dia,\\*
Nós, romeiros da terra ensolarada,\\*
Precisamos demais de um anjo guia\\*
Nossa crença será multiplicada\\*
Pela sua imortal sabedoria

Em Jesus, em José e em Maria\\*
Padre Cícero firmou a sua fé,\\*
Antevia os fenômenos do seu tempo\\*
Nos sermões que fazia em sua Sé\\*
Toda graça que fez foi pela graça\\*
Do Messias Jesus de Nazaré

Grande nome que o povo não esquece\\*
Porque desde muito pequenino,\\*
Uma luz para o homem sertanejo\\*
Pela grande vontade do Divino\\*
Pade Ciço Romão do Juazeiro\\*
Anjo guia do povo nordestino

Meu Padim viajou mas voltará\\*
Com a Virgem das Dores, Padroeira,\\*
Nossa fé inda está do mesmo jeito\\*
Pela sua palavra verdadeira\\*
Seus amigos, seu povo e sua igreja\\*
Pacientes lhe esperam a vida inteira.
\end{verse}


\chapter{Virgem Mãe das Dores}

\begin{verse}
Virgem Mãe das Dores\\*
É a solução\\*
Pra todo cristão\\*
Que nela acredita\\*
É o lenitivo\\*
Pra nós pecadores\\*
Salve a Mãe das Dores\\*
De força infinita

Socorrei aqueles\\*
Que pedem conforto\\*
Que vão para o horto\\*
Que rezam na Sé\\*
Assim Padre Cícero\\*
Em vida ensinava\\*
A quem precisava\\*
De um pouco de fé

Viva Jesus Cristo\\*
Divino cordeiro\\*
São Pedro chaveiro\\*
E a Mãe das Dores,\\*
Que ajudam e libertam\\*
Quem vive cativo\\*
Que são lenitivo\\*
Para os pecadores

Quem ama e perdoa\\*
Quem reza contrito\\*
Ajuda o aflito\\*
E pratica jejum\\*
Quem faz caridade\\*
Ama o Pai Eterno\\*
Não vai pro inferno\\*
De jeito nenhum

Feliz quem proclama\\*
A graça alcançada\\*
Pela Virgem amada\\*
E Jesus Salvador\\*
A fé multiplica\\*
Agradece e canta\\*
Na bondade santa\\*
De Nosso Senhor

Romeiro sofrido\\*
De longas estradas\\*
Imensas jornadas\\*
De todo país\\*
Sai ano entra ano\\*
Visitando a Sé\\*
Vivendo da fé\\*
E cantando feliz

Nós somos devotos\\*
De meu Padim Ciço\\*
Nosso compromisso\\*
Era sempre assim\\*
Nossa Mãe das Dores\\*
É nossa Rainha\\*
Amiga e madrinha\\*
De um amor sem fim

Romeiros nós somos\\*
Por livre vontade\\*
Por afinidade\\*
E por devoção\\*
Os grandes milagres\\*
Que já recebemos\\*
Nós agradecemos\\*
Com o coração

Bendito e louvado\\*
Seja meu Padrinho\\*
Mostrando o caminho\\*
Nos conselhos seus\\*
Em nome da Santa\\*
Mãe imaculada\\*
Que mostra a estrada\\*
Da casa de Deus

Leprosos, morféticos,\\*
Já desenganados\\*
Que foram curados\\*
Libertos da dor\\*
Meu Padrinho via\\*
Pela medicina\\*
A força divina\\*
Do Pai Criador.
\end{verse}


\chapter{Mourão perguntado}

\begin{verse}
-- Caro poeta me diga\\*
Se ainda é bom romeiro\\*
-- Minha devoção antiga\\*
Me carrega ao Juazeiro\\*
-- O nome do Padroeiro\\*
Dos devotos do sertão\\*
-- O Padre Cícero Romão\\*
Eternamente lembrado\\*
\textit{Isto é mourão perguntado}\\*
\textit{Isto é responder mourão}

-- Diga pra nós por favor\\*
Um grande milagre seu\\*
-- A hóstia mudou a cor\\*
Em sangue se converteu\\*
-- Diga pra quem ele deu\\*
Patente de capitão\\*
-- Ao famoso Lampião\\*
Por muitos injustiçado\\*
\textit{Isto é mourão perguntado}\\*
\textit{Isto é responder mourão}

-- Padre Cícero, meu Padrinho\\*
Me diga com quem parece\\*
-- Com aquele capuchinho\\*
Que o povo não esquece\\*
-- Aquele também merece\\*
Sua canonização\\*
-- Seu nome é Frei Damião\\*
Será sim canonizado\\*
\textit{Isto é mourão perguntado}\\*
\textit{Isto é responder mourão}

-- Me diga qual o bendito\\*
Que você canta e encanta?\\*
-- Todo bendito é bonito\\*
Logo a gente aprende e canta\\*
-- Qual será o nome da santa\\*
Que o Padre fez devoção?\\*
-- A virgem da Conceição\\*
Mãe das Dores, ser sagrado\\*
\textit{Isto é mourão perguntado}\\*
\textit{Isto é responder mourão}

-- Os milagres por bondade\\*
Feitos pelo Padre Santo\\*
-- Para falar a verdade\\*
Não há quem saiba do tanto\\*
-- Foram muitos eu garanto\\*
Feitos em cima do chão\\*
-- Sei que foi uma porção\\*
Mas tudo não estou lembrado\\*
\textit{Isto é mourão perguntado}\\*
\textit{Isto é responder mourão}

-- Dos conselhos verdadeiros\\*
Cite alguns que recebeu\\*
-- Eu cito de alguns romeiros\\*
Que o Padre Cícero deu\\*
-- Uns dizem que o mundo seu\\*
Não é essa região\\*
-- Outros dizem que Abraão\\*
Também lhe fez convidado\\*
\textit{Isto é mourão perguntado}\\*
\textit{Isto é responder mourão}

-- Aqui em cima da terra\\*
Qual foi a sua doutrina?\\*
-- Que o amor supera a guerra\\*
A grande lei nos ensina\\*
-- Onde a crença nordestina\\*
Sentiu grande proteção?\\*
-- Num homem cuja missão\\*
Foi dar conselho ao errado\\*
\textit{Isto é mourão perguntado}\\*
\textit{Isto é responder mourão}

-- Por que é que o nordestino\\*
Seu nome nunca esqueceu?\\*
-- Porque desde pequenino\\*
Com seu povo conviveu\\*
-- Todo Nordeste gemeu\\*
Com a morte dele ou não?\\*
-- Foi tanto pranto no chão\\*
Que o chão ficou irrigado\\*
\textit{Isto é responder mourão}\\*
\textit{Isto é mourão perguntado}.
\end{verse}


\chapter[A volta do Padre Cícero]{A volta do Padre Cícero\\\smallskip\textit{(setilhas de setissílabos)}}

\begin{verse}
Sua figura tem trono\\*
No Cariri, no sertão\\*
Onde quer que Deus exista\\*
Toda e qualquer região\\*
Há um milagre contado\\*
E o nome lembrado\\*
Do Padre Cícero Romão

O seu nome é decantado\\*
Por poetas cantadores\\*
Violeiros, repentistas\\*
Por vários compositores\\*
Por artista de renome\\*
Todos enaltecem o nome\\*
Do filho da Mãe das Dores

Rei Gonzaga não cantou\\*
Só falando do vem-vem\\*
Serrote Agudo, Asa Branca\\*
Carolina e Xenhenhém\\*
O monarca sanfoneiro\\*
Do Padre do Juazeiro\\*
Foi um romeiro também

A sua estátua gigante\\*
Muito bem edificada\\*
Simbolizando no horto\\*
Sua longa caminhada\\*
Pelos mistérios da fé\\*
Nosso objetivo é\\*
Continuar a jornada

Tem vezes que nossa crença\\*
Enfraquece de repente\\*
Depois que a gente medita\\*
Firmada na nossa mente\\*
A fé renasce e dá fruto\\*
Num poder absoluto\\*
Que só Deus dá para a gente

Vem o terceiro milênio\\*
Iluminando a ciência\\*
Padrinho está de volta\\*
Qual um raio em refulgência\\*
Ao lado de Cristo Rei\\*
Mostrando a nova lei\\*
Da era da consciência

Os horizontes estão turvos\\*
O pudor chegou ao fim\\*
Os valores do espírito\\*
Nunca foram vistos assim\\*
O mundo mudou o tom\\*
Os maus estão achando bom\\*
E os bons estão achando ruim

Meu Padrinho, sua igreja\\*
Não pode ser esquecida\\*
Nem nossa fé apagada\\*
Nem nossa crença ferida\\*
Volte, estamos esperando\\*
O mundo está precisando\\*
Entrar numa nova vida.
\end{verse}


\chapter[Toda terra precisa do seu santo]{Toda terra precisa do seu santo\\\smallskip\textit{(martelo alagoano)}}

\begin{verse}
Todo santo precisa seu andor\\*
Toda igreja precisa do seu sino\\*
Todo homem já traz o seu destino\\*
Toda ópera precisa seu tenor\\*
Toda vida precisa de um amor\\*
Toda roupa só faz se tiver pano\\*
Todo justo nasceu pra ser humano\\*
Toda terra precisa do seu santo\\*
Vou chamar Padre Cícero por enquanto\\*
\textit{E lá vão dez de martelo alagoano}

Somos nós de um torrão sacrificado\\*
Onde tudo só vem por derradeiro\\*
As ofertas que vêm do .estrangeiro\\*
O produto já está ultrapassado\\*
Quando nosso produto é exportado\\*
É aquele terrível desengano\\*
Tem que haver um milagre soberano\\*
Ou então nossa raça fica louca\\*
Que a vontade política é muito pouca\\*
\textit{E lá vão dez de martelo alagoano}

Não é nada demais você comprar\\*
Um carrinho pra ir ao seu passeio\\*
Um lugar pra fazer um veraneio\\*
Uma casa modesta pra morar\\*
Trabalhar com direito a descansar\\*
Não faltar com o seu cotidiano'\\*
Receber suas férias todo ano\\*
Ter saúde, conforto e assistência\\*
Não há santo que tenha paciência\\*
\textit{Lá vão dez de martelo alagoano}

Quem tiver o seu santo padroeiro\\*
Faça preces pensando no trabalho\\*
Quem viver num emprego quebra-galho\\*
Não se meta a gastar o ano inteiro\\*
Quem tiver o ofício de pedreiro\\*
Não procure carrão americano\\*
Porque em tudo é preciso regra e plano\\*
Ninguém pode esperar só pelo santo\\*
Não existe milagre para tanto\\*
\textit{Lá vão dez de martelo alagoano}

É perdido o sonhar com grande altura\\*
Quando nossos limites são pequenos\\*
Essa história de vida mais ou menos\\*
Isto é só um jogo de cintura\\*
Porque quando a parada fica dura\\*
É comum fraquejar o ser humano\\*
Apelar para a luz do soberano\\*
Porque Deus quando manda é casa cheia\\*
Nunca vi Deus cobrar feijão de meia\\*
\textit{Lá vão dez de martelo alagoano}

O nordeste sem dúvida é um país\\*
Bem maior do que muitos europeus\\*
Aqui muitos apelam para Deus\\*
Pra cortar todo mal pela raiz\\*
Pela fé, o nordestino é mais feliz\\*
Várias vezes levado ao desengano\\*
Com promessa e com truque leviano\\*
Vive ao povo enganando o tempo inteiro\\*
Com discurso que Deus é brasileiro\\*
\textit{E lá vão dez de martelo alagoano}.
\end{verse}


\cleardoublepage

\part*{Sextilhas}

\chapter{Do improviso que ficou}

\begin{verse}
\textit{Que é que sei? E que sou,}\\*
\textit{Donde vim? Pra onde vou?}

Eu sou um substantivo,\\*
Tenho sede de aprender.\\*
Cantando aprendi amar;\\*
Amando aprendi sofrer\\*
Carrego dentro da alma\\*
A ventura de viver

Todo valer aparente,\\*
O tempo vem e consome.\\*
Quem sobre cascalho planta,\\*
É possível passar fome.\\*
Em casa de avarento,\\*
Quem pede esmola, não come

Todo homem inteligente\\*
Pode adquirir estudo;\\*
Interroga o “pai dos burros”,\\*
Faz consulta ao “mestre mudo”,\\*
Pergunta quando não sabe:\\*
Termina aprendendo tudo\ldots{}

O Universo é composto\\*
Por esferas soberanas,\\*
Entre átomos e moléculas,\\*
Entre células e membranas,\\*
Fluídos que se organizaram,\\*
Formando vidas humanas

Do abismo universal,\\*
O nada tomou a frente:\\*
Da água, nasce o germe,\\*
O lodo que se fez gente,\\*
Anjo que se fez arcanjo:\\*
Ponto final da semente

Vim do pó do universo,\\*
Das explosões estrelares,\\*
Dos princípios mais remotos,\\*
Desde a formação dos mares,\\*
Dos milênios vagarosos,\\*
Dos períodos seculares

Do veneno fulminante\\*
À força dos furacões;\\*
Do disparo das granadas\\*
À luz das constelações;\\*
Da queimadura dos raios\\*
Ao trompete dos trovões

Sente o cantador, na alma,\\*
Uma esperança divina;\\*
A luz que dissipa trevas,\\*
A mensagem da doutrina,\\*
A aura do corpo humano,\\*
O saber da medicina

A leveza da espuma,\\*
A dureza do brilhante,\\*
O pigmeu sem defesa,\\*
O portentoso gigante,\\*
A choupana desgraçada,\\*
O palácio cintilante

O aroma das campinas,\\*
O fascínio das paisagens,\\*
Zigue-zague das raízes,\\*
O requebrar das aragens,\\*
A superstição dos mitos,\\*
O fantasma das miragens

Cantando, sinto a dureza\\*
Do malefício da guerra.\\*
Vejo o reflexo dos céus\\*
Azular longínqua serra;\\*
De longe, dando impressão\\*
De unir o céu com a terra

Bilhões de coisas estranhas\\*
Precisamos conhecê-las,\\*
O abismo das distâncias,\\*
A quentura das estrelas,\\*
As formas do abstrato,\\*
Que nós não podemos vê-las

A solidão do deserto,\\*
O cintilar do mormaço,\\*
O rio serpenteante,\\*
Igual fumaça no espaço,\\*
Que vai tomando distância\\*
Sem alterar o compasso

Com alma o poeta canta\\*
A ciência que evolui;\\*
O sovina que não cede,\\*
O sábio que contribui.\\*
O macro que multiplica,\\*
O micro que diminui

Eu recrio nos meus versos,\\*
As galantes “Giocondas”,\\*
O ciciar dos insetos\\*
E o marujar das ondas;\\*
As distantes nebulosas,\\*
As profundezas das sondas

Cantando, eu sinto o encanto\\*
Da linda Mansão Aérea;\\*
Dos gênios que passam livres\\*
Lá na correnteza etérea:\\*
Hoje estão emancipados\\*
Das correntes da matéria

Dos fluidos ultrainvisíveis,\\*
Como bem sentia Cristo\\*
Com o “supermicroscópio”,\\*
Nenhum poderá ser visto,\\*
Mas a violência humana\\*
Poderá ver tudo isto

Cantando, eu vejo a ciência,\\*
Segredo da isotonia;\\*
O nêutron, próton, elétron,\\*
O íon, a isobaria;\\*
Membranas, células, moléculas,\\*
Física, biologia

Penetro no infinito,\\*
Nas camadas “marcianas”,\\*
Raio ultravioleta;\\*
Nos tecidos das membranas;\\*
Raio gama e raio \textsc{x};\\*
Nas ondas hertzianas

Como poeta, conheço\\*
A Reforma de Lutero,\\*
O Código de Hamurabi,\\*
A devassidão de Nero.\\*
Sei de todas essas coisas:\\*
Canto na hora que quero!

A combinação das coisas\\*
Tem seu ponto de partida\ldots{}\\*
A química dos elementos,\\*
Em forma desconhecida:\\*
Nêutron completando átomos,\\*
Fluidos produzindo vida

Estou precisando tanto\\*
De descansar meu juízo\\*
Num recanto adormecido,\\*
Num pequeno paraíso.\\*
Pra eu desfrutar um pouco\\*
A paz de que eu preciso

Preciso da paz dos campos,\\*
Um lugar longe daqui;\\*
Lugar que as flores conversam,\\*
A natureza sorri\\*
Com o mais bel'espetáculo\\*
O balé do colibri!

Quero libertar um pouco\\*
Minh'alma triste, cativa;\\*
Deixem-me passar as horas\\*
Em vida contemplativa\ldots{}\\*
Façam isso por bondade\\*
Para que feliz eu viva.
\end{verse}


\chapter[Uma cantoria em Paris]{Uma cantoria em Paris\\\smallskip\textit{(com Lourinaldo Vitorino no \textsc{xviii} Salão do Livro, em março de 1998)}}

\begin{verse}
Lourinaldo Vitorino --\\*
Que noite extraordinária\\*
Ao sopro da brisa mansa\\*
A viola é um piano\\*
Que no meu peito descansa\\*
Trazendo o sertão tostado\\*
Para as ruas frias da França

Oliveira de Panelas --\\*
Estou feliz como criança,\\*
Que volta à casa dos pais\\*
Eu tive aqui em Paris\\*
Meus momentos geniais\\*
E se eu não virar francês hoje\\*
Não vou virar nunca mais

LV -- Só vou pedir, rapaz,\\*
Que cante sem acanhez\\*
Jogue metáforas poéticas\\*
Mas cantando português\\*
Só não tente assassinar\\*
O idioma francês

OP -- Eu falo é meu português\\*
Eu falo e não me acanho\\*
Décimo oitavo Salão do Livro\\*
Na cultura eu tomo banho\\*
É a primeira vez que canto\\*
Num salão deste tamanho

LV -- Que ambiente tamanho\\*
Gostoso de se olhar\\*
A França que antigamente\\*
Viu o cordel no seu lar\\*
Cortando a península ibérica\\*
Indo pra o nosso lugar

OP -- Simone de Beauvoir\\*
A França lhe escuta a voz\\*
Sartre o grande filósofo\\*
Galeria de heróis\\*
Mas no Brasil conhecemos\\*
Nossa Raquel de Queiroz

LV -- A França mostrou pra nós,\\*
A poesia exemplar,\\*
Lamartine no passado\\*
Foi um poeta sem par,\\*
Mas o nosso Castro Alves\\*
Temos que valorizar

OP -- E se quiser comparar\\*
Lindas coisas que já vi,\\*
Entre o Brasil e entre a França\\*
Entre o Brasil e aqui,\\*
Temos o rei Pelé\\*
E aqui temos Platini

LV -- Zidane está por aí\\*
Brilha igual Ronaldinho\\*
Nem vou comparar os dois\\*
Cada um tem seu caminho.\\*
Mesmo dois pássaros tão grandes\\*
Não nascem no mesmo ninho

PO -- Gostei do meu amiguinho,\\*
Do poema que compôs,\\*
Na vértice de uma pirâmide\\*
Não há lugar para dois,\\*
Um tem que mandar primeiro\\*
E o outro manda depois

LV -- França e Brasil, vejo os dois\\*
Sobre os dois devo propor,\\*
Aqui tem a Torre Eiffel,\\*
Lá tem Cristo Redentor,\\*
É a engenharia humana\\*
É a expressão do Senhor

OP -- Cristo representa amor\\*
Eiffel tecnologia,\\*
E Sartre pra Jorge Amado\\*
A força se delicia,\\*
Um está com Gabriela,\\*
E o outro com filosofia

LV -- A França criou um dia\\*
Seu herói Napoleão\\*
O Brasil tem Virgulino\\*
Que é chamado Lampião,\\*
um matava por aqui,\\*
o outro matou no sertão

OP -- Com esta comparação\\*
Sei que a seu lado vou,\\*
Lá tivemos Marta Rocha,\\*
E aqui Brigitte Bardot,\\*
Marta Rocha ainda está livre\\*
Mas Brigitte se acabou

LV -- A França também criou\\*
Um poeta de primeira,\\*
Foi o genial Balzac\\*
Gênio pra toda a fronteira\\*
Mas o sertão nordestino\\*
Também criou Zé Limeira

OP -- Inácio da Catingueira\\*
Negro poeta tão bom,\\*
Se tivemos Conselheiro\\*
Aqui tivemos Danton,\\*
E também Afonso Daudet\\*
Tartarin de Tarascon

LV -- Falar de Flammarion\\*
Seria rima gostosa,\\*
Mas lembra advocacia\\*
Nosso grande Rui Barbosa,\\*
Gênio que a humanidade\\*
Com respeito ainda glosa

OP -- Ideia maravilhosa\\*
Que faz a gente pensar\\*
No grande Gilberto Freyre\\*
Um rei espetacular,\\*
Aqui Victor Hugo,\\*
E Trabalhadores do mar

LV -- Que magnífico lugar\\*
Citar sempre vale a pena,\\*
Paris que hoje é cortada\\*
Pelo grande rio Sena,\\*
Mas temos o Amazonas\\*
Que o brasileiro condena

OP -- Por aqui se vale a pena\\*
Por um motivo qualquer,\\*
Entre os gênios desta pátria\\*
Seja ele homem ou mulher,\\*
Por aqui temos Kardec,\\*
Por lá temos Chico Xavier.
\end{verse}


\chapter{A corrupção}

\begin{verse}
Os escândalos brasileiros\\*
Envergonham qualquer chão\\*
Desmoronam qualquer povo\\*
Denigrem qualquer nação\\*
Deixam de alma enlutada\\*
Qualquer constituição

Nas religiões, nas seitas\\*
Na política, na moral\\*
No senado, nos palácios\\*
Na medicina legal\\*
A corrupção comanda\\*
Até no judicial

A corrupção no Brasil\\*
Como um câncer prolifera\\*
Corrói a mente do povo\\*
Viça na mais alta esfera\\*
E temos que conviver\\*
Com a cruel besta-fera

O homem perdeu a fibra\\*
Moral e dignidade.\\*
Dos que não compactuam\\*
É pequena a quantidade\\*
Quando a mentira procura\\*
Desmoronar a verdade

O sexo também tem sido\\*
A corrupção patente\\*
Nem Sodoma, nem Gomorra\\*
Destruídos totalmente\\*
Praticaram os absurdos\\*
Que se faz atualmente

A política está passando\\*
Por um período infeliz\\*
Os coniventes não falam\\*
Quem está roubando não diz\\*
E assim a corrupção\\*
Vai destruindo o país

Enquanto a criança dorme\\*
Sem conforto na favela\\*
Sem roupa pra a aquecer\\*
Sem comida na panela\\*
Os corruptos levam tudo\\*
O que seria pra ela

O ladrão de sanduíche\\*
Esse vai pra cela fria\\*
Mas o ladrão magnata\\*
Que rouba na mordomia\\*
É bem raro que é chamado\\*
Pra ir à delegacia

Tem que se achar um jeito\\*
Para acabar esta laia\\*
Não podemos conviver\\*
No meio desta gandaia\\*
Que faz do povo sofrido\\*
A verdadeira cobaia

O país está chorando\\*
A bandeira está tristonha\\*
O povo está sufocado\\*
Por essa agressão medonha\\*
Tanto escândalo sobrando\\*
Faltando tanta vergonha

Até quando esses larápios\\*
Driblam nossa resistência\\*
Nos põem sapo guelra abaixo\\*
Torram nossa paciência\\*
São vândalos que patrocinam\\*
A máquina da violência

O sangue do operário\\*
É bebido toda hora\\*
Todo suor derramado\\*
O pranto que ele chora\\*
É transformado em moeda\\*
Para ser gasta lá fora

Sexo mercantilizado\\*
Faz a moral levar fim\\*
O narcotráfico progride\\*
Deixando tudo que é ruim\\*
Nenhum país vai à frente\\*
Com uma cambada assim

Os cambalachos perversos\\*
Programados em surdina\\*
Onde a calada da noite\\*
É quem dita e determina\\*
A negociata espúria\\*
Que trama nossa ruína

Esses dráculas sugadores\\*
Que há muito tempo vêm\\*
Fazendo banquete às custas\\*
Do sangue que o povo tem\\*
Sem responsabilidade\\*
Não prestam conta a ninguém

Com o dinheiro do povo\\*
São peritos na manobra\\*
O “filé” eles consomem\\*
Deixam para nós a sobra\\*
O antiofídico está pouco\\*
Pra suportar tanta cobra

A seleção do Brasil\\*
Tem seus craques mafiosos\\*
Uma nova \textsc{cbf}\\*
De atletas perigosos\\*
Com a sigla Corrupção\\*
De Brasileiros Famosos

Eles são habilidosos\\*
Para driblar tribunais\\*
Dar calote na Receita\\*
Comprar juízes venais\\*
Lesar o Banco Central\\*
É o que eles fazem mais

Fazem contratos gigantes\\*
Esses craques mascarados\\*
Licitação não existe\\*
Nos seus jogos complicados\\*
Que alistam seus coronéis\\*
E elegem seus deputados

Essa \textsc{cbf} possui\\*
Craque jogando lá fora\\*
Spa's, o cassino de Mônaco,\\*
Miami, a escrete adora,\\*
Caimã é a maior renda\\*
No seu campo até agora

Em Paris perdemos a copa\\*
Ninguém pôde com Zidane\\*
Desanimados torcemos\\*
Pr'os onze craques em pane\\*
Nossos \textsc{cbf} são “penta”\\*
E que Zagalo se dane

Esses \textsc{cbf} contam\\*
Com Collor na seleção\\*
Todos seus craques espertos\\*
Só fizeram gol de mão\\*
Poucos têm cartão vermelho\\*
Em forma de cassação

E o Nicolau Lalau\\*
Tem a prisão decretada\\*
Cento e sessenta milhões\\*
Levou de uma só lapada\\*
Sujou o \textsc{trt}\\*
E ainda não sofreu nada

Luiz Estevão começou\\*
Seu imenso capital\\*
Desviando seus milhões\\*
Com a construtora Incal\\*
O larápio mais votado\\*
Do Distrito Federal

O rato juiz Nestor\\*
Foi pego de mão na massa,\\*
Contra o \textsc{inss}\\*
Cometeu muita trapaça\\*
Juiz em cadeia pública\\*
No Brasil serve de graça

E Georgina de Freitas\\*
Ladra grã-fina danada\\*
Na justiça dos States\\*
Ali ela foi condenada\\*
Se fosse pelo Brasil\\*
Não seria molestada

Salvatore Cacciola\\*
O Banco Marka era seu\\*
Mais de um bilhão vírgula cinco\\*
Do Tesouro ele comeu\\*
Sabendo que ia preso\\*
Na hora se escafedeu

O famoso “Motoserra”\\*
O Hildebrando Pascoal\\*
Roubou, traficou, matou\\*
E foi cassado afinal\\*
Virou piada e vergonha\\*
Do Congresso Nacional

Dos anões do Orçamento\\*
João Alves ferrou-se um dia\\*
Abertamente roubava\\*
Cinicamente dizia\\*
Que Deus do céu lhe ajudava\\*
A acertar na Loteria

É dos Magalhães, os donos\\*
Do Banco Nacional,\\*
Um rombo de muitos “bi”\\*
Contra nosso capital.\\*
E o povo pagando o pato\\*
Do larápio oficial

Calmon de Sá, o banqueiro,\\*
Faliu bilionariamente\\*
\textsc{acm} veio e salvou\\*
Seu afilhado cliente\\*
Com o seu Banco Econômico.\\*
Quem paga a conta é a gente

E Maluf dos precatórios\\*
Do superfaturamento\\*
Desviou tanto dinheiro\\*
De carregar um jumento\\*
Continua candidato\\*
Isso é atrevimento

Celso Pitta e seus comparsas\\*
Puxa-sacos do prefeito:\\*
Só Jacome foi cassado,\\*
Preso por não ser direito\\*
E o restante da gandaia\\*
Livre está, deram um jeito

Eis a seleção dos onze\\*
Que roubando há tempo vêm\\*
Restam muitos camuflados\\*
Pode procurar que têm\\*
Quem tem tanto atleta desses\\*
Não perde para ninguém

O time acima citado\\*
É de primeiro escalão\\*
Milhares de bons reservas\\*
Em boa forma já estão,\\*
O “penta” ninguém nos tira\\*
Na copa da corrupção.
\end{verse}


\chapter[Só porque sou um filho adulterino]{Só porque sou um filho adulterino\\\smallskip\textit{(martelos com mote)}}

\begin{verse}
Eu preciso de minha identidade\\*
Para ver se o mundo me conhece\ldots{}\\*
Façam logo meu \textsc{inps}\\*
Que estou na maior necessidade.\\*
Também peço a legitimidade:\\*
Que, por Deus, sou também abençoado;\\*
Não me façam de um cão abandonado;\\*
Não me tratem da forma de um suíno.\\*
\textit{Só porque sou filho de adulterino}\\*
\textit{Meu direito está sendo sonegado}

Eu nasci duma noite mal dormida\ldots{}\\*
Na vontade da ejaculação;\\*
Vi dois corpos sentindo a sensação,\\*
Quando estava, em começo, minha vida!\\*
Demorei-me no ventre e fiz partida,\\*
No começo, por Deus, determinado.\\*
E por que é que sou repudiado,\\*
Se nasci da vontade do Divino?\\*
\textit{Só porque sou um filho adulterino.}\\*
\textit{Meu direito está sendo sonegado}

Olhe aqui, seu doutor, estou doente;\\*
Minha mãe, uma pobre sem dinheiro;\\*
Eu não sei, de meu pai, o paradeiro.\\*
Mas eu sou um coitado, um inocente!\\*
E pergunte pra Lei; veja se sente\\*
O que é ser um filho abandonado,\\*
Que é visto igualmente a um renegado,\\*
Porque pede um conforto de menino.\\*
\textit{Só porque sou um filho adulterino,}\\*
\textit{Meu direito está sendo sonegado}

É verdade: sou filho de amante!\\*
Minha mãe nunca quis o casamento,\\*
Mas, em Deus, há o grande mandamento:\\*
Respeitar e acatar o semelhante.\\*
Para mim o maior ignorante\\*
É quem tira o direito de um coitado:\\*
Que ninguém não pediu pra ser gerado!\\*
Façam lei, façam lei pra o pequenino.\\*
\textit{Só porque sou um filho adulterino,}\\*
\textit{Meu direito está sendo sonegado}

Eu nasci no ensejo da paixão,\\*
Se, assim foi, minha mãe, por que quiseste?\\*
Te pergunto, meu pai, por que fizeste\\*
O desejo de tua sensação?\\*
Por que, mãe, aceitaste a sedução?\\*
Tu devias, meu pai, ter evitado\ldots{}\\*
Mas é tarde demais: estou formado!\\*
Quero agora conforto em meu destino\ldots{}\\*
\textit{Só porque sou um filho adulterino,}\\*
\textit{Meu direito está sendo sonegado}.
\end{verse}


\chapter{Tiradentes morreu sem ver a luz}

\begin{verse}
Foste herói dos heróis, ó Tiradentes.\\*
Tu pagaste, com sangue, a tua glória:\\*
Teu espírito não viu a tua vitória,\\*
Na defesa dos fracos, inocentes.\\*
No sigilo dos teus “inconfidentes”\\*
Houve um que te fez a falsidade,\\*
No exemplo que deste à humanidade,\\*
Foste um Cristo morrendo noutra cruz\ldots{}\\*
\textit{Tiradentes morreu sem ver a luz}\\*
\textit{Da estrela febril da liberdade}

Preconceito de cor não existia\\*
Nas ideias do Mártir liberal:\\*
Uma luz que brilhasse por igual\\*
Era isso que ele pretendia.\\*
Não gostava de falsa hierarquia,\\*
Que oprime sem haver necessidade.\\*
Ele amava demais a igualdade,\\*
Na doutrina evangélica de Jesus.\\*
\textit{Tiradentes morreu sem ver a luz}\\*
\textit{Da estrela febril da liberdade}

Tu pediste aos céus que a desgraça\\*
Desse um voo de monstro e fosse embora,\\*
E a D\~{}us, suplicaste toda hora\\*
Que mudasse o destino de uma raça\ldots{}\\*
Mas, com isso, teu corpo foi à praça,\\*
Em pedaços de toda qualidade;\\*
Tua alma partiu pra divindade\\*
E a carne ficou pros urubus.\\*
\textit{Tiradentes morreu sem ver a luz}\\*
\textit{Da estrela febril da liberdade}.
\end{verse}


\chapter{Nascimento do gênio}

\begin{verse}
Castro Alves, não foste patriota\\*
Pois acima de tudo foste humano\\*
A canção que fizeste ao africano\\*
Conseguiste afinar nota por nota\\*
Teus poemas cobraram grande cota\\*
Aos terríveis senhores traficantes\\*
Que na compra dos pobres semelhantes\\*
Eram brutos, perversos mercadores\\*
\textit{Castro Alves nasceu cantando amores}\\*
\textit{Num castelo de espumas flutuantes}

Foste herói da defesa na contenda\\*
De Biafra, de Angola e de Bengala\\*
Derrubaste as paredes da senzala\\*
E protestaste os senhores da fazenda\\*
E não foste um mito ou uma lenda\\*
Pra tentar convencer ignorantes\\*
Tu vieste de cosmos mais distantes\\*
Onde tudo se vê com outras cores\\*
\textit{Castro Alves nasceu cantando amores}\\*
\textit{Num castelo de espumas flutuantes}

Tu de longe sentiste o sacrilégio\\*
Dos navios negreiros africanos\\*
Que por uso de meios desumanos\\*
Um mercado de mágica em sortilégio\\*
Mas a força do grande privilégio\\*
Transformou-a em gigante dos gigantes\\*
Tu sentiste na dor dos semelhantes\\*
O suplício de tuas próprias dores\\*
\textit{Castro Alves nasceu cantando amores}\\*
\textit{Num castelo de espumas flutuantes}

De castelos de luzes colossais\\*
Onde Deus, de propósito, fez pros bravos\\*
Deste canto o poeta dos escravos\\*
Condensou-se nas luzes de astrais\\*
Entre alvas esteiras de cristais\\*
Onde moram faíscas cintilantes\\*
Ou em ondas de brumas murmurantes\\*
Este gênio nasceu das próprias flores\\*
\textit{Castro Alves nasceu cantando amores}\\*
\textit{Num castelo de espumas flutuantes}

Na Bahia nasceu, amou, sofreu\\*
Não chegou se formar, morreu rapaz\\*
O veneno de um tiro lá no Brás\\*
Eis a causa que ele faleceu\\*
Antes disto, a Eugênia conheceu\\*
Sua estrela, a mais bela das amantes\\*
Que chegou transformar os seus instantes\\*
\textit{Castro Alves nasceu cantando amores}\\*
\textit{Num castelo de espumas flutuantes}.
\end{verse}


\chapter{Problemas políticos}

\begin{verse}
Na política dá bem para sentir\\*
Como o homem tem sede de poder,\\*
Os de cima, tem medo de descer\\*
Os de baixo, estão doidos pra subir,\\*
Os de cima, tem medo de cair,\\*
Os de baixo, angariam leite e pão,\\*
Os de cima, tem leite, mas não dão,\\*
No temor que a vaquinha vá embora,\\*
\textit{A política tem sido até agora}\\*
\textit{O problema maior desta nação}

O Nordeste, um senhor discriminado,\\*
Faz cuidado, dá ânsia, traz insônia,\\*
As queimadas cruéis da Amazônia,\\*
Pantanal, está sendo ameaçado\\*
Sete-quedas, foi queda, no passado;\\*
Itaipu, não tem sido a solução.\\*
Essa Angra dos Reis, é ilusão\\*
Se eu contar mais profundo o povo chora\\*
\textit{A política tem sido até agora}\\*
\textit{O problema maior desta nação}

Dá pra ver em Vergueiro e em Rocinha,\\*
E outras áreas de classe favelada,\\*
Não se sabe onde a verba é aplicada\\*
Quando rouba é talvez uma galinha\\*
O dilema não é o trombadinha\\*
O problema está mais no trombadão,\\*
Que não paga passagem de avião\\*
E que visita a Suíça qualquer hora,\\*
\textit{A política tem sido até agora}\\*
\textit{O problema maior desta Nação}.
\end{verse}


\chapter{Planeta recuperado}

\begin{verse}
Quando o homem souber que com papéis\\*
Não se compra caráter nem pudor,\\*
Que a seiva vital do grande amor\\*
Ninguém compra com cinco ou dez mil réis\\*
Quando o homem souber que as chaminés\\*
Deixam lucro com o ar contaminado,\\*
Que é melhor o fabrico de um arado\\*
Do que bala, metralha e cartucheira,\\*
\textit{Quando o homem fizer desta maneira,}\\*
\textit{O planeta será recuperado}

Quando o homem deixar de ser voraz,\\*
E começar botar paz em todo canto,\\*
Repartir sua dor na dor do pranto\\*
Dos que sofrem dilemas sociais,\\*
Pôr final nos conflitos raciais\\*
Onde tantos problemas nos têm dado\\*
Evitar novo sangue derramado\\*
Pra a Terra ter paz a vida inteira,\\*
\textit{Quando o homem fizer desta maneira}\\*
\textit{O planeta será recuperado}

Quando o homem notar que o semelhante\\*
Necessita de paz e de respeito\\*
Que ódio, o orgulho e o preconceito,\\*
São diplomas do vil ignorante,\\*
Quando o homem souber que neste instante\\*
Tem um ser pequenino desprezado,\\*
Sem remorso, sem culpa e sem pecado\\*
Como um trapo rolando na poeira,\\*
\textit{Quando o homem fizer desta maneira,}\\*
\textit{O planeta será recuperado}.
\end{verse}


\chapter{Gemidos da terra}

\begin{verse}
O país pequenino do Japão\\*
Não encontra mais campo pra seu povo,\\*
Planejando encontrar um mundo novo\\*
Invadir deve ser a solução,\\*
Quando alguém nesta terra perde o chão\\*
A tendência é achá-lo novamente,\\*
Ou buscá-lo em terreno diferente\\*
E é possível que morra procurando,\\*
\textit{Ouço a terra gemendo e soluçando}\\*
\textit{Com o peso dos pés de tanta gente}

Quando a China de outrora não sabia\\*
Que seria a maior população,\\*
Que um número passado de um bilhão\\*
Sua prole gigante passaria,\\*
Acontece que nasce todo dia\\*
Multidão no país do oriente,\\*
E ela vai exigir do continente\\*
Mais espaço, que o seu não está dando,\\*
\textit{Ouço a Terra gemendo e soluçando}\\*
\textit{Com o peso dos pés de tanta gente}

Terra mãe, no teu ventre frutifica\\*
Uma massa importante do teu colo,\\*
Somos filhos fecundos do teu solo\\*
Onde tudo ao nascer se multiplica,\\*
Uma raça que vai, outra que fica\\*
Sem parar o processo da semente,\\*
Olho a raça agitada em minha frente\\*
Também sinto meus pés lhe machucando\\*
\textit{Ouço a terra gemendo e soluçando}\\*
\textit{Com o peso dos pés de tanta gente}.
\end{verse}


\chapter{O cantador}

\begin{verse}
Repentista, poeta, cantador,\\*
Teu cantar livremente se levanta\\*
É teu grito holocausto da garganta\\*
Como quem quer matar a própria dor,\\*
Há um toque de sonho e de amor\\*
E um namoro de musa passageira,\\*
Teu cantar rasga o peito a vida inteira,\\*
'Na tangente da lira nordestina,\\*
Tua voz uma eterna clandestina\\*
Musicando a grandeza brasileira

Indomável titã do improviso,\\*
Quando cantas levita tua mente,\\*
Na cadência veloz do teu repente\\*
Solta fogo invisível, teu juízo,\\*
Quando buscas cantando um paraíso\\*
São fugazes demais as alegrias,\\*
Tua verve vestida de poesias '\\*
Faz de ti construtor das emoções,\\*
Pescador de fantásticas ilusões\\*
E caçador das mais belas fantasias.
\end{verse}


\chapter{Ghandi}

\begin{verse}
Lá na Índia distante nasceu Gandhi,\\*
Homem gênio, um apóstolo, um deus herói,\\*
Um amante das obras de Tolstoi\\*
Comprovou seu valor de alma grande,\\*
Do seu lar fez viagem para Dandi.\\*
Adotava a passiva resistência,\\*
Foi convicto, contrário à violência,\\*
Fez seu povo vencer sem fazer guerra\\*
Boicotou o governo da Inglaterra\\*
Pondo a Índia em real independência

Defensor de plebeus e camponeses\\*
Superou violência com a calma,\\*
Toda Índia chamou-o de grande alma\\*
Libertando seu mundo dos ingleses.\\*
Condenado à prisão por várias vezes\\*
Por ser sempre a favor do infeliz\\*
Foi à África e cortou pela raiz\\*
Todo mal que seu povo padecia,\\*
Fez justiça à sofrida maioria\\*
E tombou morto salvando seu país

O Decálogo de Gandhi é a verdade\\*
Nos ensina vencer sem violência\\*
São os homens iguais por excelência\\*
Nas crianças não há desigualdade\\*
É virtude manter a castidade\\*
Sê frugal, jejuando faz o bem\\*
Ajudar a seu próximo que não tem\\*
O supérfluo se faz desnecessário\\*
Do suor sê honesto em seu salário\\*
Não ter medo de nada e de ninguém.
\end{verse}


\chapter{As invenções}

\begin{verse}
Com Pasteur, se começa a lista extensa:\\*
Com Albert Einstein e com Dumont,\\*
César Lattes em busca do méson\\*
E Gutemberg, o maestro da imprensa,\\*
Com Da Vinci surgiu a Renascença\\*
Um dos grandes versáteis criadores,\\*
Raio laser, robôs, televisores\ldots{}\\*
Descobertas incríveis dos milênios\\*
Faço sim continência aos grandes gênios\\*
Desse no Planeta, os benfeitores

São dezenas, centenas e milhares,\\*
De façanhas das grandes invenções,\\*
Reatores, turbinas e aviões,\\*
Desfilando libertos pelos ares\\*
Moderníssimos navios singram os mares\\*
E o radar quantas vezes se acendeu,\\*
Telescópio brilhou com Galileu,\\*
Porém falta nascer o criador\\*
Pra mais bela invenção do grande amor\\*
Onde o homem jamais compreendeu

Thomas Edison, aposenta a lamparina,\\*
Foi Graham Bell inventor do telefone,\\*
Sendo o rádio criado por Marconi\\*
Sabin acha a relíquia na vacina,\\*
Fleming trouxe pra nós penicilina\\*
Enforcando a maldita enfermidade,\\*
Muitos homens com tal capacidade\\*
Se tivesse o planeta produzindo\\*
Facilmente seria resolvido\\*
Os problemas de toda humanidade.
\end{verse}


\cleardoublepage

\part*{Sextilhas de decassílabos}

\chapter{Enigmas}

\begin{verse}
Na história da crença ninguém sabe\\*
Quem está sendo certo ou sendo errado\\*
Ser herege ou ser crente é um problema\\*
Que deixou muito homem acabrunhado\\*
Se rezar, é carola, cheira-santo\\*
Não rezando, é herege excomungado

Esta história de briga é um tratado\\*
Onde o homem é quem mais se aperreia\\*
Se morrer, deixa a vida que era doce\\*
Se matar, sua vida fica feia\\*
Se morrer, vai morar no cemitério\\*
Se matar, vai pras grades da cadeia

Lavrador pequenino, perde a ceia\\*
Vendo a roça que a seca exterminou,\\*
Chuva e sol: dois problemas naturais\\*
Que o roceiro até hoje suportou\\*
Muita chuva, vai tudo na enxurrada\\*
E muito sol, perde a roça que plantou

E no jogo de bicho alguém já viu\\*
Que o sonho tem sido fantasia\\*
Não jogando, se der, ele pragueja\\*
Se jogar e não der, tem ingresia\\*
Não jogando, se der, vai dar encrenca\\*
E se não der, não faz feira neste dia

E o vestibular é um segredo\\*
De entrada de túnel pro deserto\\*
Pobre vestibulando sai chutando\\*
Sem saber o errado, nem o certo\\*
Sem bizu vai perder na concorrência\\*
E se pegar o bizu é descoberto

Nós poetas também passamos perto\\*
E esta bucha sem gosto a gente come\\*
Se disser que só canta por um xis\\*
Nem sustenta o que diz e suja seu nome\\*
Se pedir muito alto ninguém paga\\*
E não cantando é pior que passar fome

E o coitado operário passa fome\\*
Onde a greve ou ajuda ou tumultua\\*
Do falar pra o calar são dois extremos\\*
Que seu grande problema continua\\*
Não falando é boneco conformado\\*
E se falar é pior, que vai pra rua

Casamento mal feito é uma praga\\*
É purgante ruim de se beber\\*
A mulher trai o homem um pouco velho\\*
Aí sim, ele fica sem saber\\*
Se deixar, vai sofrer na solidão\\*
Se ficar, tem um sócio até morrer

Casamento de filha é um sofrer\\*
Que o pai de família o peso pega\\*
Vem um macho cretino e lhe namora\\*
Fica o pai sem saber se deixa ou nega\\*
Se deixar ela casa sem futuro\\*
E não deixando, é pior, que ele carrega

E a mulher seminua só sossega\\*
Quando o homem ficar daquele jeito\\*
Pouca roupa no corpo e rebolado\\*
Provocando o desejo do sujeito\\*
Se ninguém lhe olhar, é macho frio\\*
E se mexer, é tarado sem respeito

A mãe pobre engravida de um sujeito\\*
Que não é nem amante nem esposo\\*
Num orgasmo entre o medo e o prazer\\*
Decidir para ela é doloroso\\*
Se parir, cria um filho delinquente\\*
Se abortar, é um ato criminoso

E a mulher que brigar com o esposo\\*
No barulho ela vence o companheiro\\*
Quem lutar com mulher não faz negócio\\*
Nem que seja um navio de dinheiro\\*
Se apanhar, é vergonha pro sujeito\\*
E se bater, é covarde e garapeiro.
\end{verse}


\chapter{Guerra}

\begin{verse}
Eu vi El Salvador dilacerado\\*
Pela bomba, a granada e o fuzil\\*
Os Estados Unidos gargalhando\\*
Desta guerra da farda com o civil\\*
Se houver outro tranca em nossa pátria\\*
Correrá muito sangue no Brasil

Estou vendo problema atualmente\\*
Nações grandes insultando as pequeninas\\*
E os submarinos soviéticos\\*
Camuflados nas águas argentinas\\*
Defendendo o país do cone sul\\*
Reclamante das Ilhas das Malvinas

Alguém diz que são ricas suas minas\\*
De petróleo e de pedra preciosa,\\*
Sua orla de peixes saborosos\\*
É dotada de soma fabulosa\\*
E por riqueza do mar nós entraremos\\*
Numa guerra sinistra e criminosa

Inglaterra que está com Argentina\\*
Neste grande conflito militar\\*
Ou Estados Unidos “quer” encrenca\\*
Ou a Rússia está doida pra brigar\\*
Pra testar seus engenhos belicosos\\*
Do terrível armamento nuclear

As Malvinas nem correm, nem se mexem,\\*
Nem se afogam de vez no oceano\\*
Quando a Rússia defende a Argentina\\*
Inglaterra de Reagan adora o plano\\*
E nós comendo este truque armamentista\\*
Pelo russo e o norte-americano

Essas ilhas do sul do continente\\*
Foi o mar inquieto quem as fez,\\*
Bom seria que Deus pisasse em cima\\*
Dessas ilhas, afundando de uma vez\\*
Pra nenhum argentino brigar mais\\*
E degolar o orgulho do inglês

No Brasil onde moro eu saberei\\*
Quem é bom no fuzil e no bofete\\*
Se a vitória será de Galtieri\\*
Ou quem vence é a fêmea Margareth\\*
Eu só quero saber de minha rede\\*
Quem venceu ou perdeu no telequete

Quantos gastos sem lucro e sem futuro\\*
De ingleses e libras esterlinas,\\*
Abre o cofre do povo Galtieri\\*
E joga fora as moedas argentinas\\*
Pondo em prova de morte o sangue humano\\*
Por pedaços de ilhas pequeninas.
\end{verse}


\cleardoublepage

\part*{Poemas em metros variados}

\chapter[Festival de sereias]{Festival de sereias\\\smallskip\textit{(galope à beira-mar)}}

\begin{verse}
Eu fiz um passeio por águas alheias\\*
Juntando as sereias do mundo romântico\\*
Do glacial ártico, pacífico e atlântico\\*
Numa carruagem puxada a baleias\\*
Distantes de orla, das dunas de areias\\*
Abismos gigantes tive que passar\\*
Com uma poltrona de espuma no ar\\*
Sentei-me no colo da mãe maresia\\*
Beijava a sereia, cantava e sorria\\*
\textit{Cantando galope na beira domar}

Fiz um vesperal de voz de graúnas\\*
Nos lindos coqueiros, delgados e belos\\*
Entrei pelas águas, vi grandes castelos\\*
De pérolas douradas, imensas fortunas!\\*
Passando por cabo, por golfo e dunas\\*
A voz da sereia eu queria gravar\\*
Na festa dos mares vi se aproximar\\*
Netuno, trajado de água e espaço\\*
Olhou pra orquestra, marcou o compasso\\*
E a festa nascia \textit{na beira do mar}

Então pelos mares, vi mil caravelas\\*
No ritmo dos remos de outras barcaças\\*
Ao longe a cortina de finas vidraças\\*
Que transpareciam por entre as procelas\\*
No palco mistério das sereias belas\\*
Fiquei deslumbrado somente em olhar\\*
Já tinham guardado pra mim um lugar\\*
Que eu era um dos membros de sua seresta\\*
Senhor convidado, patrono da festa\\*
\textit{Cantando galope na beira do mar}

Mistérios incríveis eu observei\\*
Daquelas princesas do fundo das águas\\*
Não tinham nos olhos semblantes de mágoas\\*
Nem choro de escravo, nem riso de rei\\*
Falavam num código que a nossa lei\\*
Está bem distante de codificar\\*
Cantavam tão lindo da água parar\\*
Bequadros divinos, sublimes bemóis\\*
E o mar soluçava no ritmo da voz\\*
Das lindas sereias \textit{nas águas do mar}

A festa acabou-se, não tem mais apelo\\*
Eu trouxe a sereia mais bela pra mim\\*
Eu tenho um aquário de ouro-marfim\\*
A minha coberta vai ser seu cabelo\\*
Devido ela ser uma esposa-modelo\\*
Senti o seu corpo me hipnotizar\\*
Do meio pra cima, é pra gente amar\\*
Do meio pra baixo, tem peixe pra ceia\\*
Não vejo mulher igualmente a sereia\\*
Nem dentro, nem fora \textit{das águas do mar}.
\end{verse}


\chapter[Oração da paz]{Oração da paz\\\smallskip\textit{(toada alagoana)}}

\begin{verse}
Enquanto o homem da terra\\*
Forma guerra\\*
sete dias da semana\\*
Eu por fora me conservo\\*
E observo\\*
Tanta violência humana,\\*
Enquanto o mundo enlouquece\\*
Eu faço prece\\*
Na toada alagoana

Veja a tecnologia\\*
Todo dia\\*
Crescendo de mais a mais,\\*
Sei que o homem neste avanço\\*
Sem descanso\\*
Talvez queira encontrar paz\\*
Pergunta a si mesmo\\*
E fica a esmo\\*
A pergunta que ele faz

Neste solo que nascemos\\*
E vivemos\\*
Tem fartura em nosso chão\\*
Tem riqueza e abundância\\*
E substância\\*
Em cada germinação,\\*
O que falta é a ciência da consciência\\*
Pra se dividir o pão

Muitas invenções humanas\\*
E levianas\\*
Acarretam prejuízo\\*
É bom que neste milênio\\*
Nasça um gênio\\*
E faça o que for preciso\\*
E transformar tanto inferno\\*
Num eterno\\*
E sonhado paraíso

Que se acabe o tiroteio\\*
O bombardeio\\*
Ataques e explosões,\\*
Oxalá termine agora\\*
Nesta hora\\*
O comércio de canhões\\*
Quero ver a paz cantando\\*
E comandando\\*
O sossego das nações

Se este meu idioma\\*
Fosse a Roma\\*
À Arábia e a Belém\\*
Chegasse à Rússia e à China\\*
E à Palestina\\*
E dissesse em Jerusalém,\\*
Fora todas as metralhas\\*
E de batalhas\\*
Não vai mais morrer ninguém

Cairo diz a Israel\\*
Nosso papel\\*
É manter paz no Irã,\\*
Na Jordânia, no Iraque,\\*
Sem ataque\\*
Sem bombardeio amanhã\\*
Terra de São Saruê\\*
Cadê você?\\*
É o país da Canaã

Que Deus mande a esta esfera\\*
A nova era\\*
Da real fraternidade,\\*
Que os povos se entendam\\*
E compreendam\\*
Isentos de vaidade,\\*
E a paz desfile de branco\\*
Em passo franco\\*
Para toda a humanidade

Eu quero que cada irmão\\*
Estenda a mão\\*
A seu mano sofredor,\\*
Sem divisão de país\\*
E nem fuzis\\*
Nem preconceito de cor\\*
Todos unidos vibrando\\*
E solfejando\\*
O mesmo hino de amor.
\end{verse}


\chapter[Conformação]{Conformação\\\smallskip\textit{(parcela ligeira)}}

\begin{verse}
Um rio corrente\\*
De água potável,\\*
Um clima agradável\\*
Um bom ambiente,\\*
Um chazinho quente\\*
Após uma lua cheia,\\*
Uma lua cheia\\*
No céu a brilhar\\*
Eu faço é zombar\\*
De uma fortuna alheia

Um lugar distante\\*
De poluição\\*
Sem agitação\\*
Sem pressa um instante.\\*
Eu e minha amante\\*
Somente nós dois,\\*
Feijão com arroz\\*
Inhame e manteiga.\\*
Garanto que chega\\*
Herdeiro depois

Quero uma casinha\\*
Em plena floresta,\\*
Um tanto modesta\\*
Mas que seja minha,\\*
Com uma cozinha\\*
Sem fogão a gás,\\*
Senhor Pai dos pais,\\*
Pidão sei que sou\\*
Perdoa se estou\\*
Querendo demais

Quero uma panela\\*
De feijão fervendo\\*
E o fogo gemendo\\*
Por debaixo dela,\\*
Quero uma janela\\*
Aberta ao nascente,\\*
Um cão paciente\\*
Uma rede armada,\\*
Pra quem não tem nada\\*
É suficiente

Eu quero um viveiro\\*
De pássaro cantando,\\*
Rosas perfumando\\*
Todo o meu canteiro\\*
Pequeno dinheiro\\*
Que dê pra viver,\\*
Eu quero é saber\\*
Se estou sossegado,\\*
Que sou conformado\\*
Com o que aparecer

Não pago aluguel\\*
Onde me arrancho,\\*
Tou sempre em meu rancho\\*
Em lua de mel,\\*
Um amor fiel\\*
Vivendo pra mim,\\*
Senhor do Bonfim\\*
Eu te agradeço\\*
Porque não mereço\\*
Tanta coisa assim

Não quero jornais\\*
No meu pé-de-serra,\\*
Notícias de guerras\\*
Que o homem faz,\\*
Aqui tou em paz\\*
Com tudo que fiz,\\*
Que eu não tenha pressa,\\*
Que lhe interessa\\*
É que eu seja feliz

Adeus calçamento,\\*
Chaminés, apitos,\\*
Tumultos e gritos,\\*
Tanto movimento,\\*
Eu não aguento\\*
Para longe irei\\*
Dos meus interesses\\*
Qualquer dia desses\\*
Eu retornarei.
\end{verse}


\chapter[Banquete não é jejum]{Banquete não é jejum\\\smallskip\textit{(oitavão rebatido)}}

\begin{verse}
Banquete não é jejum\\*
Volúvel não tem partido\\*
No jogo trinta e um\\*
Com trinta não está batido\\*
Múmia não expõe o rosto\\*
Filho bom não dá desgosto\\*
Malandro não paga imposto\\*
\textit{No oitavão rebatido}

Mudo não faz propaganda\\*
Sábio não é convencido\\*
Fração não é uma banda\\*
Donzela não tem marido\\*
Herege não se confessa\\*
Ateu não paga promessa\\*
Tartaruga não tem pressa\\*
\textit{No oitavão rebatido}

Cachorro não teme gato\\*
Carrasco não quer pedido\\*
Gato não respeita rato\\*
Miado não é latido.\\*
Relógio não marca ano\\*
Gibão não se faz de pano\\*
Cotó não toca piano\\*
\textit{No oitavão rebatido}

Quem perde não desconta\\*
Prudente não é metido\\*
Velhaco não paga conta\\*
Suspiro não é gemido\\*
Desertor não quer fileira\\*
Rei não cochila em esteira\\*
Vagabundo não faz feira\\*
\textit{No oitavão rebatido}

Céu nublado não tem astro\\*
Branco não é colorido\\*
Fantasma não deixa rastro\\*
Essa daí eu duvido\\*
Cigano não tem patente\\*
Passarinho não tem dente\\*
Careca não usa pente\\*
\textit{No oitavão rebatido}.
\end{verse}


\chapter[Da causa vem o efeito]{Da causa vem o efeito\\\smallskip\textit{(oito pés a quadrão)}}

\begin{verse}
-- Da causa vem o efeito\\*
-- Do pudor vem o respeito\\*
-- Da razão vem o direito\\*
-- Vem da vontade a paixão\\*
-- Do trabalho vem o pão\\*
-- Do mestre vem o ensino\\*
-- Do homem vem o destino\\*
\textit{Nos oito pés do quadrão}

-- Casa de pobre é choupana\\*
-- Gente grã-fina é bacana\\*
-- Jantar de mico é banana\\*
-- Bicho valente é leão\\*
-- Ave bonita é pavão\\*
-- Casa de sapo é lagoa\\*
-- Chapéu de rei é coroa\\*
\textit{Nos oito pés a quadrão}

-- A nascente faz o rio\\*
-- O gelo alimenta o frio\\*
-- O verão faz o estio\\*
-- A chuva deita no chão\\*
-- A terra produz o grão\\*
-- Cresce o fruto a gente come\\*
-- Aí ninguém fala em fome\\*
\textit{Nos oito pés a quadrão}.
\end{verse}


\chapter[Geme quem trai seu amor]{Geme quem trai seu amor\\\smallskip\textit{(toada gemedeira)}}

\begin{verse}
Geme quem trai seu amor,\\*
Geme mais quem é traído,\\*
A mulher geme pensando\\*
Na traição do marido\\*
Que atire a primeira pedra\\*
Ai, ai! Ui, ui!\\*
Quem nunca deu um gemido

Hoje todo mundo geme\\*
O empregado, o patrão,\\*
Geme ao moço pra casar,\\*
Geme o ladrão na prisão,\\*
Eu também tenho gemido:\\*
Ai, ai! Ui, ui!\\*
No dia da prestação

Vovô gemeu com vovó\\*
Dois velhos que conheci.\\*
Papai gemeu com mamãe\\*
Com certo tempo eu nasci,\\*
E com gemidos dos dois\\*
Ai, ai! Ui, ui!\\*
Olh'eu escrevendo aqui

Um velho de minha terra\\*
Por nome João Aroeira,\\*
Morreu com oitenta anos\\*
E nunca quis companheira,\\*
Ou era medo de sogra\\*
Ai, ai! Ui, ui!\\*
Ou para não fazer feira.
\end{verse}


\chapter[O poeta se despede]{O poeta se despede\\\smallskip\textit{(dez pés a quadrão)}}

\begin{verse}
-- Em dez eu vou terminar\\*
-- Eu já estava com vontade\\*
-- É uma necessidade\\*
-- Que a gente tem de cantar\\*
-- Cantar, viver e sonhar\\*
-- É nossa obrigação\\*
-- Na força da profissão\\*
-- Que a natureza consente\\*
-- Você canta e ele sente\\*
-- \textit{Lá se vão dez a quadrão}

-- Vou me despedir agora\\*
-- Porque a hora é chegada\\*
-- A missão está terminada\\*
-- Nesta abençoada hora\\*
-- Eu necessito ir embora\\*
-- Começar outra missão\\*
-- Começo uma obrigação\\*
-- Outra obrigação tem fim\\*
-- Vida de artista é assim\\*
-- \textit{Lá se vão dez a quadrão}

Ao leitor muito obrigado\\*
Por ter lido meus poemas\\*
Minhas sextilhas, meus temas\\*
Gemedeira e agalopado\ldots{}\\*
Deixo um abraço apertado\\*
Pra quem gosta de poesia\\*
E o que não aprecia\\*
Tente aprender a gostar\\*
Nós vamos nos encontrar\\*
Adeus, até outro dia

Do leitor eu me despeço\\*
Me desculpe de tudo\\*
Do modesto conteúdo\\*
Se não foi um grande sucesso\\*
Muitas desculpas eu peço\\*
Pois é bom ser sempre assim.\\*
O trabalho está no fim\\*
Mas não vou ficar parado\\*
Quando estiver precisado\\*
Pode escrever para mim.
\end{verse}

\part{Na sala de aula}

\paginabranca

\section{Sobre o autor}

Oliveira Francisco de Melo, ou Oliveira de Panelas, poeta, 
cantador e violeiro, nasceu em 24 de maio de 1946 no sítio 
Barroca, na cidade de Panelas, interior de Pernambuco. 
Começou a cantar versos ainda
na infância, com nove anos. Sem concluir a escola primária, Panelas
alfabetizou-se lendo os folhetos de diversos cordelistas. Aos doze anos
passou a acompanhar o cantador Zé Rufino em suas andanças pelo interior
do Nordeste, trabalhando como ``estagiário'' desse outro
importante poeta. O estágio durou dois anos e com catorze anos Oliveira de Panelas
já percorria
as feiras de Pernambuco, Alagoas, Paraíba e Sergipe cantando seus
próprios versos. Em 1962 estabeleceu"-se em Garanhuns, no interior
de Pernambuco, onde fazia poemas e trabalhava como pedreiro para
complementar sua renda. Como muitos outros nordestinos, em 1972 veio a
São Paulo em busca de trabalho. Aqui cantou nas cantinas do Brás, Bom
Retiro e Bexiga, e foi um dos principais organizadores do I Congresso
de Repentistas Nordestinos de São Paulo. Frente a sua intensa atividade,
foi convidado por Otacílio Batista para retornar ao Nordeste como
cantador, chegando a João Pessoa em 1975 com o parceiro que o
acompanharia por mais de vinte anos. Panelas realizou trabalhos para o
cinema e a televisão, representou a poesia popular em salões de
literatura ao redor do mundo e cantou para diversos chefes políticos.
Sua poderosa voz e sua habilidade como violeiro ainda lhe renderam o
título de ``Pavarotti dos Sertões''.


\section{Síntese dos poemas}

\medskip

\paragraph{``O poeta se apresenta''}

Caracteriza o poeta como o ``homem das
nuvens'', capaz de viajar aos céus e manter o planeta
em seu curso, possibilitando a aproximação do ato de escrever com o
sonho, ratificando a perspectiva de que a literatura é capaz de nos
fazer viajar.
 
\paragraph{``Saco de cego''}

Analogia entre o saco de um cego, que por causa de sua falta de visão
tem sua sacola cheia de alimentos diferentes misturados e o livro de
poesias, que versa sobre temas distintos.

\paragraph{``Dedicatória''}

Poema que dedica as composições do livro a todos aqueles que de alguma
forma sofreram e apresenta suas intenções de ensinar aos que fazem os
demais sofrerem a não mais fazê-lo.

\paragraph{``Vida e morte de Frei Damião''}

Frei Damião foi uma importante figura religiosa no Nordeste brasileiro.
Por ocasião de sua morte o poema narra sua biografia, desde seu nascimento,
em Bozanno, na Itália, até sua chegada ao céu. Nesse caminho lembra os
preceitos religiosos que o frei defendeu. O poema é composto em
sextilhas.

\paragraph{``Frutos do amor'' (martelo com mote)}

Narrativa que retoma os feitos de Frei Damião, mostrando sua erudição e
generosidade com o povo do Nordeste. O poema em décimas apresenta ao
fim de cada estrofe seu mote de que frei Damião plantou a paz, colheu
os frutos do amor e agora partiu.

\paragraph{``Nas águas do mar'' (galope à beira-mar)}

Apresentando os feitos de Frei Damião, o poema em décimas com rimas em
``ar'' apresenta esse religioso como
um grande missionário que fez da sua missão de vida um correlato da missão de
Jesus Cristo, já que também pregou para uma gente sofrida de uma terra
árida.

\paragraph{``O santo Frei Damião'' (décimas de sete pés com mote)}

Sob o mote de que Frei Damião é um santo e já nasceu canonizado, o autor
narra a história desse italiano que veio ao sertão pregar aos pobres
para que se esqueçam da dor, da tristeza e do pranto, dessa vez
utiliza-se de décimas com setissílabos, fazendo uma poesia de rimas
mais rápidas.

\paragraph{``Nosso Pio Giannotti''}

Após a morte de Frei Damião o autor narra numa das clássicas métricas do
cordel, a septilha, as façanhas desse homem que na região em que viveu,
mesmo depois de morto, mantém-se como uma espécie de porto seguro da fé.

\paragraph{``Ensinamentos''}

Narrando os ensinamentos de Frei Damião, o poema exalta principalmente
seu espírito pacífico e a qualidade de seus sermões, que mesmo com o
avanço da idade não perderam força.

\paragraph{``Encontro no céu de Frei Damião e Padre Cícero''}

Frei Damião encontra na porta do céu o Padre Cícero,
que, há tempos distante da terra, quer saber das novidades. 
Assustado com as diversas mudanças, Padre Cícero fica preocupado com o futuro da
humanidade.

\paragraph{``Salve, Salve Padre Cícero''}

Padre Cícero é a mais importante figura religiosa do Nordeste. Neste
poema em sextilhas, que foi musicado para o Cd ``Padre
Cícero'', Oliveira de Panelas narra a vida desse líder desde o momento
de seu nascimento até sua morte, ressaltando sua fé, mesmo quando 
foi impedido pela igreja de exercer suas funções sacerdotais, em consequência
de sua força política.

\paragraph{``Meu padrinho viajou'' (décimas de sete com mote)}

Sob o mote de ``meu padrinho viajou/ vai retornar
brevemente'', o poema narra a importância do legado de Padre
Cícero, que mesmo depois de sua morte está no imaginário do povo nordestino.

\paragraph{``Milagres'' (décimas com mote de dois pés)}

Com um mote sobre a vitória de Padre Cícero no processo de canonização,
o poema discorre sobre os milagres que o religioso teria realizado e
que ratificariam sua canonização.

\paragraph{``Meu Padim, Padre Ciço''}

Poema em sextilhas sobre os feitos do Padre Cícero, rememorando seus
ensinamentos e garantindo que ele se encontra no céu, com Deus e Jesus
Cristo, mas que logo estará de volta. Independentemente disso, o texto
garante que com os ensinamentos que fez tem sua crença multiplicada até
os dias de hoje.

\paragraph{``Virgem Mãe das Dores''}

Virgem Mãe das Dores acompanha os romeiros todos os anos até a cidade de
Juazeiro, espécie de guardiã dos cristãos, que contribui para a manutenção da
fé em Padre Cícero.

\paragraph{``Mourão perguntado''}

Em forma de versos que alternam perguntas e respostas, terminando sempre
com rimas em ``ão'', o poema
ressalta a fé em Padre Cícero mostrando que seu nome não foi esquecido
porque sempre conviveu com o povo.

\paragraph{``A volta do Padre Cícero'' (septilhas de setissílabos) }

Em função das mudanças do século XX e XXI que o poeta caracteriza como
horizontes turvos, representados pela falta de pudor e o fim dos
valores do espírito, o poema pede a volta de Padre Cícero para
restabelecer uma ordem que permitiria uma nova vida.

\paragraph{``Toda terra precisa do seu santo'' (martelo alagoano)}

Reflexão sobre o bom senso dos pedidos realizados a Deus, que, 
apesar de não faltar àqueles que precisam não pode oferecer em demasia.
Confirma que a fé faz do Nordeste uma região mais feliz.

\paragraph{``Do improviso que ficou''}

Refletindo sobre o ofício de produzir versos, o poema fala da
importância da erudição, da habilidade de misturar as temáticas e da
necessidade de deixar o poeta cantar os versos na hora que deseja.

\paragraph{``Uma cantoria em Paris''}

Poema que apresenta uma disputa entre Oliveira de Panelas e Lourinaldo Vitorino numa
apresentação no Salão do Livro, em Paris, versando sobre esse
acontecimento e as relações existentes entre os universos culturais da
França e do Brasil.

\paragraph{``A corrupção''}

Escrito em 1998, o poema versa sobre uma nova CBF, Corrupção dos
Brasileiros Famosos, apresentando nomes de políticos corruptos bem como
suas estratégias, terminando com a analogia de que no ano da Copa do
Mundo de Futebol, o penta está garantido na copa da corrupção, já que
os titulares são craques e sobram reservas em formação.

\paragraph{``Só porque sou um filho adulterino''}

O poema relaciona o nascimento do filho de um encontro adúltero com as
mazelas por que esse homem passa na vida, acusando a injustiça dos preconceitos
que seriam sofridos.
%questionando porque por conta
%disso, em sua vida tem seu direito sonegado.

\paragraph{``Tiradentes morreu sem ver a luz''}

Narrando os feitos de Tiradentes, o poema o apresenta como um herói que
buscou defender os fracos e os inocentes, mas que como Cristo morreu
sem ver a luz da liberdade.

\paragraph{``Nascimento de um gênio''}

Castro Alves é apresentado no poema, que mescla dados biográficos com
as temáticas dos poemas de Castro Alves. Oliveira de Panelas retrata o poeta como
defensor dos escravos fazendo menção ao livro \textit{Espumas flutuantes}.

\paragraph{``Problemas políticos''}

Em décimas que terminam com o mote ``A política tem
sido até agora/ O problema maior dessa nação'', o
texto apresenta a corrupção e a falta de representatividade como
importantes elementos que fazem da política uma maneira de manter as
desigualdades.

\paragraph{``Planeta recuperado''}

Na tentativa de recuperar o planeta em que vivemos, o texto versa sobre o
fim da venda dos valores morais, da discriminação social e sobre o
desejo de que todos sejam tratados como iguais.

\paragraph{``Gemidos da terra''}

Poema que discorre sobre o aumento da população no planeta e sua
exploração. Com um mote ao fim de cada estrofe, o texto
apresenta o sofrimento, através dos gemidos da terra, para sustentar
tantos pés ou tanta gente.

\paragraph{``O cantador''}

Ode aos poetas populares do Nordeste, demonstrando suas habilidades e a
maneira como utilizam o improviso para musicar a vida do
brasileiro, buscando nas belas fantasias pequenas pílulas de
felicidade.

\paragraph{``Ghandi''}

Poema de alusão à figura de Mahatma Ghandi, ressaltando seus feitos na
defesa do pacifismo como forma de luta contra a opressão colonial, bem
como a vida com menos recursos, possibilitando assim que a justiça seja
feita.

\paragraph{``As invenções''}

O poema apresenta os nomes e os principais inventos dos últimos séculos,
demonstrando a importância de cada um deles elogiando seus criadores 
e defende que se houvesse mais inventores todos os
problemas da humanidade já teriam sido resolvidos.

\paragraph{``Enigmas''}

Discorrendo sobre diversas temáticas polêmicas, o poema apresenta sempre
os dois lados da questão, como, por exemplo, o operário que faz greve,
porque passa fome, mas que mesmo com as greves não tem seus problemas
resolvidos.

\paragraph{``Guerra''}

Partindo de um acontecimento real, a Guerra das Malvinas, que opôs
argentinos e ingleses na década de 1980, o poema a insere na linha de
leitura da guerra fria, bem como defende o fim desses conflitos,
afirmando que das guerras o máximo que o poeta quer saber é o resultado
do telequete (espécie de luta livre combinada, que teve seu auge entre
os anos de 1980 e 1990)

\paragraph{``Festival de sereias'' (galope à beira-mar)}

Narrativa sobre o encontro do autor com as sereias, mostrando esse
universo mágico e o modo como ele ficou completamente apaixonado por
uma delas, a ponto de trazê-la para casa e transformá-la em sua esposa.
Ressaltando ainda que não há mulher melhor que essa, já que é fonte de
amor, da cintura para cima, e fonte de alimento, da cintura para baixo.

\paragraph{``Oração da paz'' (toada alagoana)}

Espécie de elegia à paz, que retoma os muitos conflitos existentes no
mundo. Defendendo a igualdade entre os povos, o poema apela para que as
armas sejam trocadas por apertos de mão.

\paragraph{``Conformação'' (parcela ligeira)}

Evocando um lugar calmo e distante da correria das cidades e do mundo
atual, o autor apresenta o lugar em que ele gostaria de estar.
Sugerindo, portanto, que sua conformação estaria longe desse universo
caótico e tumultuado das cidades.

\paragraph{``Banquete não é jejum'' (oitavão rebatido)}

Poema de improviso que opõem diversas figuras a partir das expressões
``não tem'' e ``não é''. De leitura rápida, em
setissílabos, com rimas em ``ido'', o Poema reproduz 
a ideia de um desafio em forma de perguntas e respostas.

\paragraph{``Da causa vem o efeito'' (oito pés a quadrão)}

A partir da métrica do ``deixa-prende'', em que o poeta
retoma a rima do verso anterior, o poema apresenta diversas temáticas,
em que cada estrofe, em formato de falas, retoma o verso anterior para a
realização da rima.

\paragraph{``Geme quem trai seu amor'' (toada gemedeira)}

Com o mote ``Ai, ai! Ui, ui!'' o
poema discorre sobre os diversos momentos em que um gemido pode
aparecer, apresentando, como esperado, os momentos de prazer, os gemidos
aparecem também no medo da sogra ou da prisão.

\paragraph{``O poeta se despede'' (dez pés a quadrão)}

Poema que encerra a coletânea e agradece aos leitores. 
Além de agradecer ele se desculpa por aqueles que porventura 
acham que os temas são simples ou que não gostam de poesia.

\section{Poesia de cordel: oralidade e escuta coletiva} 

A poesia de cordel, dizem os especialistas, é uma poesia escrita para
ser lida, enquanto o repente ou o desafio é a poesia feita oralmente,
que mais tarde pode ser registrada por escrito. Essa divisão é muito
esquemática. Por exemplo, o cordel, mesmo sendo escrito e impresso para
ser lido, costumava ser lido em voz alta e desfrutado por outros
ouvintes além do leitor. A poesia popular, praticada principalmente no
Nordeste do Brasil, tem muita influência da linguagem oral, aproveita
muito da língua coloquial praticada nas ruas e na comunicação
cotidiana. 

Naturalmente, portanto, pode-se considerar a poesia narrativa do cordel
uma forma de poesia mais compartilhada e desfrutada coletivamente, o
que lhe dá também uma grande ressonância social. Muitos dos temas do
cordel são originários das tradições populares e eruditas da Europa
medieval e moderna. Nesse livro de Oliveira de Panelas, a tradição
medieval possui dupla importância, se ela se dá em alguns momentos pela
temática, como nos poemas que abordam o juízo final, ela também está
presente na retomada de formas métricas da tradição
trovadoresca galego-portuguesa, caso de diversos poemas, entre eles
``O santo frei Damião'' e ``Do improviso que ficou'', entre
outros. Outros temas são retirados de tradições orientais. 
Também encontramos na poesia popular em cordel temas
retirados das novelas de cavalaria medievais e das narrativas bíblicas.
Ao lado destes temas mais literários, encontram-se os temas locais,
quase sempre narrados na forma de crônicas de coisas realmente
acontecidas, como em ``Meu padrinho
viajou'', e as reportagens jornalísticas partem de
acontecimento reais, e buscam interpretá-los, caso de
``Guerra'', ``A corrupção''. Também há as histórias fantásticas,
que se valem das tradições semirreligiosas, ligadas à experiência com o
mundo espiritual. 

Os grandes poemas de cordel são perfeitamente metrificados e rimados. A
métrica e a rima são recursos que favorecem a memorização e
tradicionalmente se costuma dizer que são resquícios de uma cultura
oral, na qual toda a tradição e sabedoria são sabidas de cor. 


\section{O sertão geográfico e cultural}

O sertão tem mitos culturais próprios. Contemporaneamente, o sertão
evoca principalmente o sofrimento resignado daqueles que padecem a
falta de chuva e de boas safras na lavoura. Evoca a experiência
histórica de uma região empobrecida, embora tenha sido geradora de
riquezas, como o cacau e a cana-de-açúcar, ambos bens muito valiosos. 

O sertão formou também o seu imaginário por meio de grandes
personalidades e uma pujante expressão artística. Além do cordel, o
sertão viu nascer ritmos tão importantes quanto o forró e o baião.
Produziu artistas tão expressivos quanto Luiz Gonzaga, grande cantor da
vida do sertanejo em canções como ``Asa
branca''. Um escultor como Mestre Vitalino criou toda
uma tradição de representação da vida e dos hábitos sertanejos em
miniaturas de barro. A gravura popular, que sempre acompanha os
folhetos de cordel, também floresceu em diversos pontos e ficou mais
famosa em Juazeiro do Norte, no Ceará, e em Caruaru, no estado de
Pernambuco. 

Dentre os grande mitos do sertão, está certamente o do cangaço com seu
líder histórico, mas também mítico, Virgulino Ferreira, o Lampião. Até
hoje as opiniões se dividem: para alguns foi um grande homem, para
outros um bandido impiedoso. 

Uma figura muito presente na cultura nordestina é o Padre Cícero Romão,
considerado beato pela Igreja Católica. Consta que teria feito milagres
e dedicado sua vida aos pobres. 

\section{Variação linguística}

A linguística moderna usa o termo ``idioleto'' para marcar grupos
distintos no interior de uma língua. Um idioleto pode ser a fala
peculiar de uma região, de um grupo étnico ou de uma dada profissão. 

Uma das grandes forças da poesia popular do Nordeste se origina em sua
forma muito própria de falar, com um ritmo muito diferente dos falares
do sul, e também muito diferentes entre si, pois percebe-se a diferença
entre os falares de um baiano, um cearense e um pernambucano, por
exemplo.

Além desse aspecto rítmico, quase sempre também há palavras peculiares a
certas regiões. 

\section{Sugestões de atividades}
\begin{enumerate}



\item \textit{Atividade de leitura}. Esta atividade tem por objetivo sensibilizar os
alunos para a escuta de poesia. O professor deve ler um conjunto de
estrofes para exemplificar uma leitura que se construa com uma
pronúncia clara, pausas e ênfases adequadas. Após isso, cada aluno deve
ler uma estrofe, procurando marcar o ritmo e as rimas, bem como as
pausas e ênfases expressivas. É possível enriquecer essa experiência
com os vídeos do poeta cantando seus versos com diversos parceiros, que
encontram-se disponíveis no You Tube:
https://www.youtube.com/watch?v=p2vii5KgXFw e
https://www.youtube.com/watch?v=GdKG4rNs8UQ. O trabalho de leitura pode
auxiliar o professor na realização de um diagnóstico dos alunos, em
relação à pontuação e ao ritmo do texto, além de possibilitar um
desenvolvimento da percepção da voz e da fala como meios indispensáveis
à boa convivência social.

\item No poema ``Encontro no céu de Frei Damião e Padre
Cícero'' Panelas apresenta um tipo de história muito
comum nos cordéis, são muitos os autores que narram os acontecimentos
pós-morte. A partir da leitura do poema o professor pode pedir aos
alunos que pesquisem sobre os dois principais personagens. Com a
pesquisa em mãos é possível refletir com os alunos sobre a importância
dos líderes religiosos no interior do Nordeste. Como forma de
complementar a discussão é importante ressaltar que apesar das
personagens terem sido contemporâneas por alguns anos, viveram épocas
muito diferentes. Nesse ponto, o poema apresenta diversos elementos
capazes de contribuir com a discussão. Por fim, o professor pode
narrar os acontecimentos de Canudos e finalizar a discussão com um
debate sobre se a religião permite a emancipação social.

\item Em ``Meu padrinho viajou'' e
``Meu Padim, Padre Ciço'' o autor
narra o momento da morte do Padre Cícero Romão e o desejo ainda
existente de que ele retorne do céu para guiar o povo. Do ponto de
vista da linguagem, o adjetivo possessivo
``meu'' contribui para aumentar a
importância desse líder e personalizá-lo na vida de cada nordestino. A
partir da leitura desses poemas o professor pode pedir aos alunos que
identifiquem no texto os vocábulos que contribuem para a caracterização
dessa figura como indispensável para a vida do sertanejo nordestino.
Por fim, com os vocábulos identificados é possível refletir sobre o
desejo de retorno desse personagem para guiar o povo, já que o poema
refere-se ao líder como um ente querido (a partir do adjetivo
possessivo e do uso do apelido em forma de corruptela
``Padim Ciço''). Essa reflexão pode
ainda ser acompanhada de uma comparação com a figura do rei Dom
Sebastião de Portugal, que no fim do século XVI foi ferido na batalha
de Alcácer Quibir e desapareceu, mas cuja volta
foi esperada durante séculos.

\item No poema ``A corrupção'', Panelas
parte de um acontecimento esportivo, a Copa do Mundo de 1998, para
distorcer a sigla CBF -- Confederação Brasileira de
Futebol --, transformando"-a em Corrupção dos Brasileiros Famosos. 
A partir da leitura do
poema, o professor pode pedir aos alunos que indiquem as imagens
utilizadas pelo autor para representar a corrupção. Com as imagens em
mãos, é possível refletir sobre os motivos que levam à corrupção,
investindo principalmente na temática da impunidade bem como trantando do sistema
paternalista da política, mostrando como muitos dos nomes citados
possuem relação entre si, pois são apadrinhados por outros políticos.

\item No poema ``Tiradentes morreu sem ver a
luz'' essa personagem histórica é caracterizada como
um grande herói, que virou mártir da luta pela república e que é
diretamente relacionado a um Jesus Cristo moderno. A partir da leitura
do poema, o professor pode pedir aos alunos que realizem uma pesquisa
sobre essa personagem histórica. Com os resultados da pesquisa em mãos
é possível discutir como a interpretação do autor pode e deve ser
questionada, já que diversos historiadores procuram mostrar que
Tiradentes só é visto como um herói após a proclamação da República,
bem como sua proximidade com Jesus Cristo é resultado de uma construção
posterior ao acontecimento. Para contribuir para essa reflexão o
professor pode trazer o quadro ``Tiradentes
esquartejado'', de Pedro Américo (1893), obra do
momento em que a relação entre Cristo e essa personagem começa a ser
construída na historiografia. 

\item Em ``Nascimento de um gênio'' o
autor narra parte da vida do poeta Castro Alves. O poema, construído em
décimas de hendecassílabos, termina sempre com um mote
``Castro Alves nasceu cantando amores/ Num castelo de
espumas flutuantes'', que faz menção ao primeiro
livro do poeta baiano, \textit{Espumas Flutuantes}, de 1870, mas que gera um falso
contraste com a maneira como o poema o caracteriza, já que o autor
investe numa longa descrição dos poemas de Alves que combatem a
escravidão. Entretanto, a imagem do castelo de espumas flutuantes pode
ser entendida como um lugar incerto, onde a escravidão não aconteça, ou que
tem suas bases pouco firmes por ser de espuma, que pode ser interpretada como
a fragilidade decorrente da escravidão. A partir da leitura do poema, o
professor pode pedir aos alunos que elaborem hipóteses sobre a imagem
de um castelo de espumas flutuantes. Com as hipóteses elaboradas é
possível retomar a temática do texto, investindo principalmente nas
imagens em que a defesa dos negros é explicitada. Por fim, o professor
pode trazer um trecho do poema ``O navio
negreiro'', de Castro Alves, como forma de mostrar
que as imagens do belo e principalmente da natureza podem ser tratadas
de forma densa, no interior do Romantismo.

\item No poema ``Gemidos da terra''
Panelas utiliza-se de uma prosopopeia ou personificação, quando um
objeto inanimado é capaz de expressar sentimentos. Nesse caso o planeta
Terra é capaz de gemer, por causa das mazelas que ele sofreu, devido à
exploração indiscriminada de seus recursos naturais, bem como do
aumento da população no planeta. A partir da leitura do poema, o
professor pode pedir aos alunos que apresentam a estratégia do autor
para dizer que a terra está sofrendo (nesse ponto, caso os alunos
não consigam apresentar o recurso da prosopopeia, é importante que o
professor o apresente como uma figura de linguagem). Em seguida, é
possível refletir sobre qual a interpretação do poema sobre esse
acontecimento. Já que a princípio o autor sugere que a causa desses
problemas está nos países do oriente e termina por também refletir
sobre sua responsabilidade nesse movimento.  

\item Em ``Guerra'', o autor parte da
Guerra das Malvinas, de 1982, que opôs Argentina e Inglaterra, pelo
controle de um pequeno arquipélago de ilhas do Atlântico Sul.
Entretanto, a interpretação do poema caracteriza esse acontecimento
como uma decorrência direta da guerra fria. A partir da leitura do
poema, o professor pode pedir aos alunos que indiquem ao longo do
poema quais acontecimento sugerem essa interpretação. Em seguida, é
possível realizar um debate sobre por que um país comunista apoiaria uma
ditadura na América Latina e dessa forma estimular a interpretação do
poema. Como forma de incrementar a discussão, o professor pode trazer
outros elementos sobre esse acontecimento, demonstrando como ele é
apenas uma decorrência periférica da chamada guerra fria.

\item Os poemas ``Vida e morte de Frei Damião'', ``Frutos do
amor'', ``Nas águas do mar'' e ``O santo frei
Damião'' trabalham a mesma temática, versando sobre
os feitos de Frei Damião de Bozzano, um importante líder religioso do
Nordeste. Entretanto, apesar de falarem sobre o mesmo tema, as rimas
desses poemas são trabalhadas de maneira bastante diferentes, sendo o
primeiro sextilhas clássicas da literatura de cordel, o segundo décimas
em hendecassílabos com mote (espécie de repetição de um ou dois
versos), o terceiro décimas em hendecassílabos com mote e rimas em
``ar'' e o quarto décimas em
setissílabos com mote. A partir da leitura dos poemas, o professor pode
pedir aos alunos que procurem as diferenças existentes nos poemas, em
seguida apresentar como a diferença da estrutura métrica modifica a
leitura dos textos, investindo portanto, num exercício de leitura
comparada para a compreensão da métrica, bem como para o
desenvolvimento de uma leitura com ritmo, que seja capaz de apresentar
as pausas e ênfases expressivas (caso, por exemplo, do mote), bem como
a pontuação. 

\item Os poemas ``O poeta se
apresenta'' e ``O poeta se
despede'' iniciam e concluem o livro. A partir da
leitura desses textos, o professor pode pedir aos alunos que elaborem
hipóteses sobre a necessidade da apresentação e da finalização de um
livro de cordéis. Com as hipóteses elaboradas, o professor pode
trabalhar a perspectiva do cordel como desdobramento da tradição oral,
e o modo como a propaganda feita sobre sua cantoria é importante para
que os ouvintes se aproximem, bem como o agradecimento que garante que
esse cantador será ouvido mais vezes. 

\end{enumerate}

\section{Sugestões de leitura\\ para o professor} 

\begin{description}\labelsep0ex\parsep0ex
%\newcommand{\tit}[1]{\item[\textnormal{\textsc{\MakeTextLowercase{#1}}}]}
%\newcommand{\titidem}{\item[\line(1,0){25}]}

\tit{DIEGUES JÚNIOR}, Daniel. \textit{Literatura popular em verso}. Estudos. Belo Horizonte: Itatiaia, 1986. 

\tit{MARCO}, Haurélio. \textit{Breve história da literatura de cordel}. São Paulo: Claridade, 2010.

\tit{TAVARES}, Braulio. \textit{Contando histórias em versos. Poesia e romanceiro popular no Brasil}. São Paulo: 34, 2005.

\tit{TAVARES}, Braulio. \textit{Os martelos de trupizupe}. Natal: Edições Engenho de Arte, 2004 

\end{description}

\paginabranca
\paginabranca
\paginabranca
\paginabranca
