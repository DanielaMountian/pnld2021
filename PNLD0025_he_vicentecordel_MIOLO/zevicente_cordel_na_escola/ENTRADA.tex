
\vspace*{.15\textheight}

\section{Biblioteca de cordel}

A literatura popular em verso passou por diversas fases de incompreensão e
vicissitudes no passado. Ao contrário de outros países, como o México e a
Argentina, onde esse tipo de produção literária é normalmente aceita e incluída
nos estudos oficiais de literatura -- por isso poemas como ``La cucaracha'' são
cantados no mundo inteiro e o herói do cordel argentino, Martín Fierro, se
tornou símbolo da nacionalidade platina --, as vertentes brasileiras passaram por
um longo período de desconhecimento e desprezo, devido a problemas históricos
locais, como a introdução tardia da imprensa no Brasil (o último país das
Américas a dispor de uma imprensa), e a excessiva imitação de modelos
estrangeiros pela intelectualidade. 

Apesar da maciça bibliografia crítica e da vasta produção de folhetos (mais de
30 mil folhetos de 2 mil autores classificados), a literatura de cordel -- cujo
início remonta ao fim do século \textsc{xix} -- continua ainda em boa parte desconhecida
do grande público, principalmente por causa da distribuição efêmera dos
folhetos. E é por isso que a Editora Hedra se propôs a selecionar cinquenta
estudiosos do Brasil e do exterior que, por sua vez, escolheram cinquenta poetas
populares de destaque e prepararam um estudo introdutório para cada um, seguido
por uma antologia dos poemas mais representativos. 

Embora a imensa maioria dos autores seja de origem nordestina, não serão
esquecidos outros polos produtores de poesia popular, como a região
sul-riograndense e a antiga capitania de São Vicente, que hoje abrange o
interior de São Paulo, Norte do Paraná, Mato Grosso, Mato Grosso do Sul, parte
de Minas Gerais e Goiás. Em todos esses lugares há poetas populares que
continuam a divulgar os valores de seu povo. E isso sem nos esquecermos do Novo
Cordel, aquele feito pelos migrantes nordestinos que se radicaram nas grandes
cidades como Rio de Janeiro e São Paulo. Tudo isso resultará em um vasto
panorama que nos permitirá avaliar a grandeza da contribuição poética popular. 

Acreditamos, assim, colaborar para tornar mais bem conhecidos, no Brasil e afora,
alguns dos mais relevantes e autênticos representantes da cultura brasileira. 

\medskip

\begin{flushright}\begin{minipage}{.8\textwidth}

Dr.~Joseph M. Luyten (1941--2006)\smallskip

\small
Doutor pela \textsc{usp} em Ciências da Comunicação, Joseph Luyten foi um dos principais 
pesquisadores e estudiosos da literatura de cordel na segunda metade do século
\textsc{xx}. Lecionou em diversas universidades, dentre as quais a Universidade
de São Paulo, a Universidade de Tsukuba (Japão) e a Universidade de Poitiers
(França), onde participou da idealização do Centro Raymond Cantel de Literatura
Popular Brasileira. Autor de diversos livros e dezenas de artigos sobre literatura de cordel, 
reuniu uma coleção de mais de 15 mil folhetos e catalogou cerca de 5 mil 
itens bibliográficos sobre o assunto.

Joseph Luyten idealizou a Coleção Biblioteca de Cordel e a 
coordenou entre os anos de 2000 e 2006, período em que publicamos 22 volumes. 
Os editores consignam aqui sua gratidão.
\end{minipage}\end{flushright}

\cleardoublepage

