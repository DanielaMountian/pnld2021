\chapter[Introdução, por Vicente Salles]{Introdução}

A editora Guajarina foi o maior fenômeno editorial do Pará e
seguramente um dos maiores do Brasil, no campo da literatura de
cordel. Foi fundada por iniciativa do pernambucano Francisco
Rodrigues Lopes1.

Em torno da editora de Francisco Rodrigues Lopes – instalada em Belém
em 1914 – surgiu a primeira geração de cordelistas paraenses. Não é
possível recensear todos os nomes, nem todos os títulos de folhetos
publicados por essa casa. Mas é possível demonstrar que os poetas
paraenses assimilaram o modelo da literatura popular nordestina e que
alguns deles alcançaram indiscutível prestígio entre os consumidores
dessa literatura. Ernani Vieira, Romeu Mariz, Apolinário de Souza,
José Esteves e Lindolfo Marques de Mesquita estão nesse caso.

 É curioso constatar que os mais fecundos e inspirados poetas locais,
abundantemente editados pela Guajarina, ocultaram-se sob pseudônimos,
ao contrário do poeta nordestino, que — com raras exceções — assume
nominalmente a autoria de seus folhetos. Ernani Vieira foi Ernesto
Vera, às vezes precedido de um “dr.”, que lhe dava maior distinção.
Romeu Mariz, também “dr.”: dr. Mangerona-Assu. José Esteves foi
Arinos de Belém e Lindolfo Marques de Mesquita foi Zé Vicente,
cordelista e cronista, com farta colaboração na imprensa de Belém.
Apenas Apolinário de Souza manteve sua identidade no cordel.

Lindolfo Marques de Mesquita, ou Zé Vicente, foi o mais afortunado dos
cinco poetas da primeira geração de cordelistas. Fez carreira
administrativa e política. Prefeito municipal da Vigia (1933),
diretor do Deip (1943), diretor da Biblioteca e Arquivo Público
(1944), deputado estadual (1947/50), juiz do Tribunal de Contas, que
presidiu em dois mandatos (1957/58 e 1967). Alçado nessas elevadas
posições, repudiou a literatura de cordel. Mas, em tempos difíceis, o
folheto chegou a sustentá-lo. É o que declara na 35a sextilha do
folheto autobiográfico Agora sou revolucionário, que narra sua
transformação política em face dos acontecimentos de 1930. Então,
confessou:

\begin{verse}
Mas eu ainda continuo\\
À minha lira apegado,\\
Porque foi Deus quem ma deu\\
Para eu viver sossegado,\\
Pois desta lira dengosa\\
Já me tenho sustentado.
\end{verse}

Paraense, nascido em Belém em 11 de janeiro de 1898, era filho de
cearenses e foi casado com paraibana. Morreu na mesma cidade, em 17
de novembro de 1975.

 Durante longo tempo, quando jovem, fez jornalismo. Repórter da Folha
do Norte, aí criou a coluna com crônicas humorísticas “Na polícia e
nas ruas”, que assinava com o pseudônimo que o consagrou. Passou
depois para a redação de O Estado do Pará, onde continuou o mesmo
cronista-humorista.

Entre os dois empregos, deu-se a revolução de 1930. Embora vinculado a
um jornal que se colocava, muito polêmico, à frente de campanhas
políticas, Lindolfo Mesquita era funcionário público e, naquela
ocasião, perdeu o emprego e sofreu as conseqüências de sua
identificação com o regime deposto. Em seu folheto autobiográfico
narrou sua situação em 1930:


\begin{verse}
Eu já quase não sabia\\
Se ainda era brasileiro,\\
Pois até os meus patrícios\\
Me expulsaram do terreiro\\
E eu vivia, minha gente, \\
Na minha terra — estrangeiro.

— Você não presta, safado, \\
Você vai pra Arumanduba —\\
Disse-me um dia um sujeito,\\
Fazendo logo uma “truba”,\\
Como se eu tivesse sido\\
Algum moleque cotuba.

Eu vendo a coisa difícil\\
Fui ao Lloyd brasileiro\\
E comprei uma passagem\\
Para o Rio de Janeiro\\
Onde em novembro cheguei\\
Com muito pouco dinheiro.

Lá no Rio de Janeiro\\
Muita coisa me encantou,\\
Mas, também, comi o pão\\
Que Satanás enjeitou\\
Pois o dinheiro que tinha\\
Em pouco tempo acabou.
\end{verse}

Bom, o tempo também encarregou-se de ajustar as coisas e de acalmar os
ânimos:

\begin{verse}
Quando vi que já podia\\
Ter garantido o espinhaço, \\
Arrumei minha trouxinha\\
E pus debaixo do braço,\\
Vim chegando encabulado\\
E sem muito estardalhaço.
\end{verse}

E agora, no Pará, o poeta que servira ao governo deposto com tanto
ânimo a ponto de procurar o exílio, deveria servir aos novos donos do
poder. Ligou-se à facção local chefiada por Joaquim Cardoso de
Magalhães Barata, tenente, agora major, o todo-poderoso governador da
revolução. Opera-se, nele, grande transformação — agora é revoltoso: 

\begin{verse}
E quem quiser que se dane\\
Neste Pará perigoso\\
Mas autorizo também,\\
Satisfeito e bem dengoso\\
“Pode dizer, minha gente,\\
— Zé Vicente é revoltoso!
\end{verse}

Zé Vicente versejava bem, trabalhava com bom estilo, mas o folheto não
deixa de demonstrar vassalagem a uma nova ordem que acenava, para o
povo, transformações sociais profundas. O poeta rendia então
homenagem aos ricos e aos poderosos: 

\begin{verse}
Foi boa a Revolução,\\
Foi da “pontinha”, “daqui”\\
Porque veio reabilitar\\
A região do Jari\\
Que eu sei ser muito bonita\\
Por um escrito que li.
\end{verse}

Esse Jari, de singular história, já naquela época tinha no sertanejo
cearense José Júlio de Andrade o anfitrião hospitaleiro de eminentes
autoridades:

\begin{verse}
O próprio major Barata\\
Numa excursão feita ali\\
Mostrando não ter paixão\\
Elogiou o Jari\\
Que eu sei agora que é bom\\
Entretanto nunca vi.
\end{verse}

Humorista, Zé Vicente festejava assim sua adesão à revolução que o
expulsara do Pará. Satisfeito e bem dengoso, tornou-se, de fato,
revoltoso.

O folheto autobiográfico, editado pela Guajarina, data de fevereiro de
1932. Mostra portanto a rapidez da transformação. Até então Zé
Vicente tinha produzido pouco. O primeiro folheto, publicado pela
Guajarina, sem data, parece ser o duplo O azar, a cruz e o diabo
(história completa), composto de 38 sextilhas, seguido de O
Pixininga, com 22 sextilhas. O cronista-humorista já se revelava
também humorista-poeta. O primeiro poema é quase um divertimento; o
segundo narra a história de um novilho, o “Pixininga”, bicho soberbo
e temível. Em edição posterior, os dois poemas aparecem bastante
modificados.

Não localizei outros folhetos de Zé Vicente de sua fase anterior a
1930. Em seu “exílio” no Rio Janeiro, em 1931, escreveu A santa de
Coqueiros, datado de maio daquele ano, mas editado pela Guajarina em
1932, tão logo regressou. Narra um caso de misticismo acontecido em
Coqueiros, proximidade de Belo Horizonte. 

Guajarina editou ainda nesse ano os folhetos O rapto misterioso do
filho de Lindbergh, datado de 28 de maio de 1932 e A batalha naval de
Itacoatiara.

O poeta voltava assim com toda a força. Logo passaria a publicar
folhetos com assuntos políticos e sociais de grande repercussão,
histórias de bichos e de valentões: A guerra da Itália com a
Abissínia, outubro de 1935; Peleja de Chico Raimundo com Zé Mulato,
agosto de 1937; O golpe do seu Gegê ou o choro dos deputados,
novembro de 1937; Peleja de Armando Sales e Zé Américo, (possuo a 3a
edição, sem data); Combate e morte de Lampião, 10 de dezembro de 1938
(compulsei exemplar da 4ª edição) e outros. Da mesma época, todos
produzidos na década de 1930, são os notáveis folhetos do gênero
picaresco, A neta do Cancão de Fogo, 20 de janeiro de 1938 e
histórias de bichos, O macaco revoltoso, 5 de março de 1938 e A greve
dos bichos, um clássico, sem data. A partir de 1940 produziu nada
menos de sete folhetos sobre a II Guerra Mundial: A Alemanha contra a
Inglaterra, 1940; O afundamento do vapor alemão “Graff Spee”, 1940;
Alemanha comendo fogo, sem data; O Japão vai se estrepar!, 20 de
dezembro de 1941; A batalha da Alemanha contra a Rússia, 25 de julho
de 1941; O Brasil rompeu com eles, 20 de junho de 1943 e O fim da
guerra, sem data2.

A década de 1930 mostra o engajamento de Zé Vicente no ciclo das
revoluções, desdobramento dos acontecimentos da década anterior. Os
poetas nos dão enfoques muito precisos dos fatos que se desenrolaram
no mundo, no país e na região, alargando, por vezes, as perspectivas
dos historiadores oficiais. Diante dos acontecimentos, os poetas não
se mantiveram eqüidistantes. Esse elemento de “participação” é
importante até mesmo quando se esconde sob pseudônimos, ou
simplesmente quando entrega ao público seus folhetos anônimos,
refletindo, em qualquer caso, a forma de participação conseqüente
como catalisador — e decerto também formador — da opinião pública. A
editora Guajarina deu enorme contribuição, e o destaque do “ciclo das
revoluções” justificou a elaboração de um capítulo especial dedicado
à análise desses folhetos em meu ensaio Repente \& cordel. 


Zé Vicente deu a primeira mostra de sua participação no folheto
autobiográfico Agora sou revoltoso, em que narrou sua adesão à
revolução de 1930. Pouco depois entrava em cena no Pará a Aliança
Liberal, possibilitando — como em toda parte — o amplo debate de
idéias. O levante de São Paulo, em 1932, também ficou documentado em
vários folhetos publicados pela Guajarina. Mas o ciclo revolucionário
não teria muita oportunidade de se desenvolver depois de 1935, e
praticamente se exaure em 1937, sob o tacão do Estado Novo.
Precisamente nesse ano, Zé Vicente produz a obra-prima da ditadura
getulista, O golpe do seu Gegê ou O choro dos deputados, edição da
Guajarina, composta de 62 sextilhas muito divertidas, um verdadeiro
basta nas licenciosidades democráticas. Nos dias atuais, o folheto
suscita interpretações equívocas, pelo tom humorístico com que os
acontecimentos foram tratados. Na verdade, o poeta era partidário da
ditadura, e seu humor não era indistinto: divertia-se com a desgraça
alheia.

O folheto Peleja de Armando Sales e Zé Americo é outra deliciosa
sátira política da era estadonovista que escapou à citação de
Orígenes Lessa em seu ensaio Getúlio Vargas na literatura de cordel.
A peleja gira em torno da sucessão de Getúlio Vargas e, segundo o
poeta, aconteceu no próprio palácio do Catete: 

\begin{verse}
O presidente Getúlio\\
Fez um samba no Catete\\
E apareceu sorridente,\\
Mão no bolso e de colete,\\
Enquanto fora se ouvia\\
De vez em quando um foguete.
\end{verse}

Muita gente disputava as atenções do presidente e a divertida peleja
do paulista com o nordestino era o melhor da festa preparada por
Getúlio. Os contendores aprontam-se:

\begin{verse}
Zé Américo suspira\\
Enchendo o peito de alento,\\
Toma, depois... “bagaceira”3 \\
Para ajudar seu talento\\
E se bater com vontade \\
Sem sossegar um momento.

Armando Salles, então\\
Para dizer que tem fé,\\
Faz um cigarro de palha,\\
Toma depois um café,\\
Demonstrando para todos\\
Um paulista como é.
\end{verse}

A sátira, como se desenrola, é deliciosa. Zé Américo dobra o contendor
na peleja, na verdade uma espécie de “convenção” partidária, mas,
chegando Getúlio, desfaz todas as esperanças dos convencionais:

\begin{verse}
Nesse pé, toda a assistência\\
Fica mesmo tiririca\\
Porque chega seu Gegê\\
E para os homens explica:\\
“Vamos deixar como está\\
Para ver como é que fica”.
\end{verse}

O folheto O macaco revoltoso, com sugestiva capa de desenhista anônimo
e composto de 56 sextilhas, é uma verdadeira alegoria política,
criada nos tempos da ditadura getulista. Das histórias de bicho é,
sem dúvida, uma das mais divertidas, pela vivacidade dos versos e
pelo caráter satírico e humorístico. O macaco, no reino da bicharada,
aprontou uma grande confusão no Clube das Mariposas e o burro
aproveitou-se dessa confusão para

\begin{verse}
Revoltar a negrada\\
Fazendo um bruto discurso\\
Bastante considerado\\
Pelo jornalista urso.
\end{verse}

A borboleta faz correr o boato. As adesões são imediatas, vindo o
jumento fardado de capitão e o boi arrastando pesadíssimo canhão. O
papagaio dá um tiro e atinge o tatu-bola. O porco prega o comunismo e
o macaco manda fuzilá-lo por isso,

\begin{verse}
Dizendo o mesmo fazer\\
A quem quisesse imitá-lo.
\end{verse}

Daí, instala-se a junta provisória, e esse episódio nos lembra as
providências iniciais da revolução de 1930, com o macaco triunfante e
fardado de tenente. Alusão ao tenentismo? Talvez. O hino
revolucionário inclui um verso do “Hino da independência” de Evaristo
da Veiga... Colagem talvez proposital:

\begin{verse}
Já raiou a liberdade\\
Acabou-se a tirania\\
Do governo da cidade\\
E viva a democracia,\\
Abaixo a barbaridade!
\end{verse}

Podemos, sem muito esforço, ver outros corolários da revolução de
1930. E, de resto, a revolução dos bichos termina como certas
revoluções:

\begin{verse}
Mas no dia imediato\\
Da grande revolução \\
O Boi que era padeiro\\
Meteu o pé pela mão\\
E baixou uma tabela\\
Subindo o preço do pão.

[...]

O Tamanduá-bandeira\\
Foi pedir colocação\\
Mas a junta respondeu\\
Que fizesse petição\\
E ele nada conseguiu\\
Porque não tinha instrução.
\end{verse}

Além dessas habituais conseqüências revolucionárias, deram-se muitos
banquetes “em palácio”, ficando a revolta do macaco para exemplo do
bicho-homem:

\begin{verse}
Foi assim que o Macaco,\\
No mundo velho de guerra,\\
Fez a primeira revolta\\
Pondo um governo na terra,\\
Dando exemplo ao bicho homem\\
Que, mesmo assim, ainda erra.
\end{verse}

Na mesma linha do precedente coloca-se o clássico folheto A greve dos
bichos, também com sugestiva capa de desenhista anônimo4, composto de
62 sextilhas. É até hoje verdadeiro best-seller, tendo incontáveis
edições no Pará e nos estados do Nordeste. Seu lançamento deve-se à
editora Guajarina, por volta de 1939. Expandiu-se logo para o
Nordeste. Lido e reproduzido, está referenciado no catálogo da Casa
de Rui Barbosa sob o nº 616, p. 205.

Umberto Peregrino, já em 1942, chamava a atenção para esse folheto e
seu similar O macaco revoltoso. Embora a criação do gênero de
histórias de bichos se deva, incontestavelmente, a Leandro Gomes de
Barros, Zé Vicente, no Pará, deu-lhe notável dimensão.

Diz Umberto Peregrino:

\begin{hedraquote}
A propósito de A greve dos bichos vem um meeting em que se vê a massa
de animais em figura de gente, enquanto o macaco lhes discursa tendo
por tribuna o pescoço da girafa. Quadro irresistível! Tem-se mesmo a
impressão de um comício grevista, improvisado, em que o orador sobe
numa grade de jardim e pendura-se a um poste. No caso dos bichos
havia de ser, por força, o pescoço da girafa...

O macaco revoltoso não ilude ninguém. Está equilibrado num galho de
árvore disparando um revólver com cada mão, além de conduzir um fuzil
a tiracolo, uma espada na cinta, um punhal na ponta da cauda e uma
corneta na boca.

Sem que haja qualquer ligação consciente, vejo muito das malucas e
pérfidas imaginações do desenho animado, tanto nas gravuras como nas
histórias da editora Guajarina5. 
\end{hedraquote}

Umberto Peregrino transcreve grande parte do folheto. Dele temos cópia
de antiga edição da Guajarina, sem data, e de outras mais recentes,
tiradas em Belém por Raimundo Oliveira. 

A greve dos bichos não deixa de exprimir a luta de classes, tendo sido
convocada contra o jacaré, que

\begin{verse}
Era o grande imperador,\\
Sua corte era composta \\
Só de bichos de valor,\\		
Como a família Piranha\\
Onde tudo era doutor.
\end{verse}

E já observara Umberto Peregrino que o desenvolvimento dessa revolução
da bicharada “copia as coisas dos homens”. 

Colocado no contexto do Estado Novo, o folheto tem clara “missão
política”: alegoria à inutilidade das greves no mundo dos bichos, que
reflete, como na sociedade brasileira, interesses políticos espúrios,
degradação social, corrupção desenfreada. É uma crítica mordaz dos
costumes. E serve à ideologia estadonovista implantada com o golpe de
1937 com pruridos moralistas da velha concepção burguesa lusitana de
“restauração” dos “bons costumes” e da “dignidade nacional”.
Aspirações das oligarquias que se revezam no poder, em Portugal como
no Brasil, desde os tempos da revolução burguesa, independente do
regime político.

Poeta cordelista emérito, jornalista vocacionado, Zé Vicente
comporta-se como ativista político, engajado no “baratismo”,
expressão política local desvinculada da classe dominante mas que não
teve força para romper. Acabou por servir-lhes como principal
interlocutora dos interesses do capitalismo externo que a partir da
II Guerra Mundial submeteu o mundo à hegemonia capitalista. E que, no
Brasil, gerou o neocapitalismo tardio, que se autodefine neoliberal.
Herança da era Vargas...

Nesse sentido, o folheto de Zé Vicente, a despeito dele próprio,
continua sendo uma crítica mordaz e verdadeira a nossos costumes.
Atualíssima.

No gênero picaresco, Zé Vicente criou, com muito humor, um tipo
feminino em A neta do Cancão de Fogo (primeira edição impressa em 20
de janeiro de 1938 e a segunda, refundida, em 30 de abril de 1943),
descendente daquele famanaz velhaco e valente, formado em
“quengologia”. Era a Chica Cancão,

\begin{verse}
Que toda gente dizia\\
Que era mesmo um pancadão\\
Mas tinha o corpo fechado, \\
Não temia nem o cão.
\end{verse}

As proezas de Chica Cancão envolvem namoros, o casamento simulado com
um velho, que ela envenena para herdar-lhe a fortuna, a sedução de um
sacerdote e uma viagem ao Maranhão, como clandestina, sempre
seduzindo e enganando todo o mundo. O folheto não termina, ou não dá
solução às trapalhadas da neta do Cancão.

Humor à parte, o mundo, não apenas o Brasil, vivia nessa época
terríveis experiências. De uma guerra com efeitos revanchistas
começavam a brotar conseqüências inquietadoras. Em 1937 o Brasil
silencia. O poeta popular também tenta salvar a pele e volta-se para
os temas tradicionais. É mais fácil contribuir para a boa imagem de
Vargas e dos caudilhos locais do que revelar o que se passava nos
porões da ditadura. O lance universal, com os prenúncios de outra
hecatombe mundial, predispõe o poeta popular para a criação de um
novo ciclo, que documentou a II Guerra Mundial.

A editora Guajarina, no Pará, mostra-se mais uma vez atenta ao que
acontecia no mundo. O esperto editor Francisco Rodrigues Lopes, seu
proprietário, lançou 12 folhetos que narravam os acontecimentos,
assinados por Abdon Pinheiro Câmara, Zé Vicente, Arinos de Belém e
Apolinário de Souza.

O poeta paraibano Abdon Pinheiro, residente em Belém, assinou o
primeiro, O nascimento do anti-Cristo, inspirando-se nos profetas do
apocalipse, como o escritor Múcio Teixeira, que fazia sua pregação na
capital federal. Esse anti-Cristo, segundo Múcio Teixeira, nascera na
Itália. Seria o fascismo? Ninguém sabe. Mas os horrores do presente
são atribuídos ao comunismo. O anti-Cristo teria nascido na guerra da
Alemanha, que durou cinco anos e

\begin{verse}
Foi pelo seu nascimento\\
Que a mesma guerra espocou.
\end{verse}

O folheto data de 3 de março de 1939, pouco antes do início da II
Guerra Mundial. Ele se refere, portanto, à primeira, durante a qual
ocorreu a revolução soviética (1917), gerando reações em toda a parte
e a idéia bastante difundida de que o comunismo corromperia a
sociedade, destruindo a família. 

Os folhetos restantes são todos assinados por poetas paraenses: Zé
Vicente, Arinos de Belém e Apolinário de Souza. Os dois primeiros
tratam dos primórdios da guerra: A guerra da Itália com a Abissínia,
de Zé Vicente e A batalha do Sarre, de Arinos de Belém. Os três
seguintes foram assinados por Zé Vicente: O afundamento do vapor
alemão “Graff-Spee”, A Alemanha comendo fogo, A Alemanha contra a
Inglaterra. Arinos de Belém assinou o sétimo: A guerra da Alemanha e
da Polônia. E Zé Vicente mais quatro: A batalha da Alemanha contra a
Rússia, O Japão vai se estrepar!, O Brasil rompeu com eles e O fim da
guerra. O último, As escrituras e a guerra atual, de Apolinário de
Souza, traz de volta o sentimento místico do poeta popular. O
conjunto mostra a excelente contribuição de Zé Vicente, que assinou
nada menos de oito folhetos.

Nessa altura, podemos perceber que Lindolfo Mesquita, que repudiou os
folhetos, foi, como Zé Vicente, um dos poetas mais vigorosos,
criativos e originais da literatura popular paraense.

\begin{flushright}\begin{minipage}{.8\textwidth}
Vicente Salles
\\
Brasília, 3 de julho 1999
\end{minipage}\end{flushright}
