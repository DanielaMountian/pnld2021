\SVN $Id: TEXTO.tex 6637 2010-05-28 13:48:10Z jorge $

%%ol. falta folhetos e bibliografia

\chapter{A greve dos bichos}

\begin{verse}
Muito antes do dilúvio\\
Era o mundo diferente,\\
Os bichos todos falavam\\
Melhor do que muita gente\\
E passavam boa vida\\
Trabalhando honestamente.

O diretor dos correios\\
Era o doutor Jabuti,\\
O fiscal do litoral\\
Era o matreiro Siri\\
Que tinha como ajudante\\
O malandro do Quati.

O Rato foi nomeado\\
Para chefe aduaneiro,\\
Fazendo muita “muamba”\\
Ganhando muito dinheiro\\
Com Camundongo ordenança\\
Vestido de marinheiro.

O Cachorro era cantor\\
Gostava de serenata,\\
Andava muito cintado\\
De colete e de gravata\\
Passava a noite na rua\\
Mais o Besouro e a Barata.

A Cigarra, muito pobre,\\
Ainda não era “farrista”\\
Ganhava cinco mil réis\\
Para ser telefonista, \\
Mas foi cantar num teatro\\
E acabou como corista.

O Mosquito era enfermeiro\\
Tinha muita ocupação,\\
Andava sempre zunindo\\
Dando na tropa injeção,\\
Combatendo noite e dia\\
O micróbio da sezão.

O diretor do Tesouro\\
Era o doutor Gafanhoto,\\
Andava sempre apressado\\
Num bom cavalo de choto\\
Que uma vez quebrou a perna\\
Dentro dum cano de esgoto.

A Saúva se ocupava\\
Na podação dos jardins\\
E tinha como ajudantes\\
Quatrocentos Mucuins\\
Que já nesse velho tempo\\
Eram moleques ruins.
\pagebreak

O Macaco sempre foi\\
Muito bem expediente\\
Passava a vida feliz\\
Sempre baludo e contente,\\
Com sua sabedoria\\
Enganando toda gente.

O Burro, metido a sebo\\
Queria ser sabichão,\\
Até chegou mesmo a ser\\
Diretor da educação,\\
Onde baixou portaria\\
Metendo... os pés pela mão.

Do telégrafo sem fio\\
Era o chefe o Caranguejo,\\
Apesar de não saber\\
Daquele troço o manejo,\\
Dava melhor pra tocar\\
Berimbau ou realejo.

A Mucura era empregada\\
Numa fábrica de extrato,\\
O Peru era na terra\\
Consertador de sapato,\\
O Calango quitandeiro\\
Só não vendia barato.
\pagebreak

Dona Aranha era modista,\\
A Mosca sua empregada,\\
Quando errava no serviço\\
Levava muita pancada,\\
Mas no fim de pouco tempo\\
Já vivia acostumada.

A Guariba era uma negra\\
Destas mesmo bróbóbó,\\
Que não se dava a respeito,\\
Dançando no carimbó,\\
Num chamego vergonhoso\\
Com o sobrinho do Socó.

Por causa dela, uma vez,\\
Houve até pancadaria:\\
Quebraram a perna do Gato\\
Furaram os olhos da Jia\\
E o Mocó esmoreceu\\
Na presença da Cotia.

A Picota, coitadinha,\\
Teve um chilique na rua\\
Naquela barafunda\\
Apareceu a Perua,\\
Que ficou foi depenada\\
E completamente nua.
\pagebreak

Era o chefe de polícia\\
O comendador Jumento,\\
Que tomou as providências\\
Requeridas no momento,\\
Mostrando que para o cargo\\
Só lhe faltava talento.

Guariba foi deportada\\
Do centro da capital,\\
Depois da enorme sentença\\
De um processo federal,\\
Que condenava a vadia\\
Por ofensas à moral.

O jornal intitulado\\
“Gazeta dos animais”\\
Combateu esse processo\\
Chamando a todos venais\\
Porém, comprado o seu dono,\\
Fechou-se, não falou mais.

O sobrinho do Socó \\
Quando viu a coisa feia,\\
Foi falar com seu padrinho\\
Que tinha bom pé-de-meia,\\
E com peso de dinheiro\\
Pôs o juiz na cadeia.
\pagebreak

Nessa campanha medonha\\
Um Bode pai de chiqueiro\\
Foi “bancar” o moralista\\
Mas desertou do terreiro,\\
Por causa dumas histórias\\
Que revelou o Carneiro.

O Porco, então, prometeu\\
Fazer de todos a cama,\\
Dando lições de higiene\\
Querendo ter muita fama,\\
Mas todo bicho sabia\\
Que ele morava na lama.

Carrapato era fiscal\\
Preguiçoso e muito feio,\\
Onde havia uma tramóia\\
Estava sempre no meio,\\
Engordando doidamente\\
À custa do sangue alheio.

A Formiga era sovina\\
Mas amiga do trabalho\\
E tinha seu sindicato\\
Cada qual lá no seu galho,\\
Acumulando no inverno\\
Folhas de mato e retalho.
\pagebreak

Tartaruga, pescadora,\\
Era amiga da Baleia,\\
Tracajá guardava os ovos\\
Nos tabuleiros de areia,\\
Mas a Cobra só sabia\\
Falar mal da vida alheia.

O Tamanduá-bandeira\\
Era muito adulador,\\
Não saía do palácio\\
Mirando o governador,\\
Até que enfim conseguiu\\
Ser juiz corregedor.

E depois que se pegou\\
Naquela nova função,\\
Foi dizer que tudo aquilo\\
Era simples galardão\\
De seu talento elevado\\
Mas, favor, isso é que não!

Naquele tempo existia\\
Teatro da natureza\\
Borboleta era querida\\
Por sua grande beleza,\\
Era a melhor dançarina\\
Que se via na redondeza.
\pagebreak

Cururu era aplaudido\\
Como mágico perfeito,\\
Engolindo fogo em brasa\\
Como quem bate no peito,\\
Ganhando palmas “à beça”.

Fez uma festa o Veado\\
Em benefício da Arraia\\
Que já não tinha dinheiro\\
Nem pra comprar uma saia,\\
E o Cachorro foi cantar\\
Mas apanhou uma vaia.

Urubu já nesse tempo\\
Era um grande aviador,\\
Levando a correspondência\\
Aos bichos do interior,\\
Conduzindo pelos ares\\
Cartas, postais e valor.

A Coruja era ama seca\\
Dos filhos do Papagaio,\\
Que só viviam chorando\\
Dentro dum grande balaio,\\
Com medo de tempestade\\
De chuva grossa e de raio.
\pagebreak

Papagaio era estimado\\
E professor numa escola\\
Onde uma vez fez exame\\
A turma do Tatu-bola\\
Que foi toda reprovada\\
E levou pau na cachola.

Ia tudo muito bem\\
Ganhando alegre o seu pão,\\
Mas, uma vez o Quati\\
Se arvorando a sabichão,\\
Falou na necessidade\\
De fazer revolução.

Pedindo logo a palavra\\
Foi, de fato, extraordinário\\
Quando afirmou que o trabalho \\
Precisava de outro horário\\
E lembrou de se aumentar \\
Da bicharada o salário.

O Burro, então, bateu palma\\
Gritando, muito emproado:\\
“Muito bem, isto é verdade\\
Eu já vivo maltratado\\
De trabalhar para os outros\\
Como um pobre condenado”.
\pagebreak

O cavalo relinchando\\
Seu sofrimento descreve\\
E pede que o movimento\\
Seja mesmo para breve,\\
Que em todo o reino se faça\\
Estalar medonha greve.

Vendo a coisa pegar fogo\\
Cada qual melhor atiça\\
O Burro sempre na frente\\
Bufando vem para liça\\
Tudo que é bicho aderiu\\
Menos a dona Preguiça.

Havia imensa algazarra\\
Toda manhã, toda tarde,\\
O Quati não se calava\\
Promovendo grande alarde,\\
Enquanto o Boi só ficou\\
Pra não passar por covarde.

Acharam já os grevistas\\
Que nada estava direito;\\
Até Pipira arvorada\\
Batia o bico no peito,\\
Dizendo: “Pra me acalmar\\
Só mesmo com muito jeito”.
\pagebreak

O Macaco foi ao mato\\
Trouxe um rolo de cipó\\
E disse para o Quati:\\
“Isso é pra dar muito nó\\
No patife que fugir\\
E deixar a gente só!”

A Raposa convidada\\
Para a luta pela Aranha\\
Respondeu: Não acredito\\
Estou farta de patranha,\\
Tenho meu ponto de vista\\
Vou ver primeiro quem ganha.

Começando o movimento\\
A Formiga deu notícia,\\
O Tatu foi logo preso\\
Para o quartel de polícia,\\
Mas pensando na vitória\\
Até se riu com delícia.

O Peru, numa contenda,\\
Perdeu metade da crista;\\
Já tinha havido a traição,\\
Muitos estavam na lista;\\
O Galo foi deportado\\
Como sendo comunista.
\pagebreak

A questão não dava jeito,\\
Já passava uma semana;\\
O Porco entrou num roçado\\
Comeu tudo que era cana\\
E o Macaco foi pegado\\
Quando roubava banana.

O Quati viu-se perdido\\
Foi dando o fora apressado,\\
Enquanto o trouxa do Burro\\
Ali ficava enrascado,\\
Sem saber que jeito dava\\
Naquele caso encrencado.

Não havia mais comida\\
E nem tampouco dinheiro,\\
Mas a família Formiga\\
Tinha bem farto o celeiro\\
E quando foi procurada\\
Escondeu tudo primeiro.

O Jacaré, nesse tempo,\\
Era o grande imperador,\\
Sua corte era composta\\
Só de bichos de valor,\\
Como a família Piranha\\
Onde tudo era doutor.
\pagebreak

Tubarão, o comandante\\
De uma valente brigada,\\
Com corpos de infantaria\\
Do capitão Peixe-espada,\\
Mandava o “zinco” comer\\
Na costela da negrada.

Já quase desanimando\\
E arrependida da idéia,\\
Resolveu a bicharada\\
Se juntar numa assembléia\\
Que teve muita ovação\\
No grande dia da estréia.

Disse o Burro: “Minha gente\\
Só quero ver como é,\\
Ninguém mais hoje trabalha\\
Pra sustentar Jacaré\\
Ele agora o que merece\\
É certeiro pontapé”.

Pedido o auxílio da Onça\\
Esta se comprometeu\\
E, disfarçada, em palácio\\
Uma noite se meteu:\\
Quando chegou Jacaré\\
Passou-lhe o dente e comeu.
\pagebreak

Tomou conta do governo\\
Debaixo de aclamação\\
E baixou logo um decreto\\
Em que fazia questão\\
De só comer Jacaré\\
Que é de boa digestão.

O resto dos Jacarés\\
Vendo a vida por um fio,\\
Abandonou a cidade\\
Foi morar lá no rio,\\
Nunca mais na terra firme\\
A raça deles se viu.

Já para o fim, dona Onça\\
Foi ficando diferente,\\
Qualquer bicho que ela via\\
Passava logo no dente,\\
Ninguém teve mais direito\\
Tudo andava descontente.

Ao filho do Jacaré\\
Um grupo enorme aparece\\
E o governo da nação\\
De repente lhe oferece,\\
Mas este diz: “Cada povo\\
Com o governo que merece!”
\pagebreak

E subindo para a praia\\
Falou com muita razão:\\
“Vocês pensavam que a Onça\\
Ia salvar a nação,\\
Mas, querem ver o bonzinho\\
Bota-lhe a lança na mão”.

Os conselhos recebidos\\
Fez, então, que não ouviu,\\
E rematando a conversa\\
Os grandes olhos abriu:\\
“Vocês vão chorar na cama\\
Que ficou dentro do rio”.

“A mim ninguém pega assim\\
Como pegaram meu pai”\\
Disse o jovem Jacaré\\
Que no convite não cai,\\
E termina murmurando:\\
“Pra lá o diabo é quem vai”.

Voltaram todos os bichos\\
Se lamentando da sorte\\
A Coruja arrependida\\
Já preferia era a morte,\\
Ninguém tinha mais coragem,\\
Ninguém sentia-se forte.
\pagebreak

O Burro foi processado\\
Por mera perseguição,\\
Perdeu toda uma fortuna\\
Que ganhou com “cavação”\\
Ficou quase na miséria\\
E foi parar na prisão.

A Raposa era matreira\\
Mas se fingia de sonsa,\\
Vendo o rumo que tomava\\
Toda aquela geringonça,\\
Assinou um manifesto\\
Solidária com a Onça.

Cada vez a tirania\\
Manchava mais a nação:\\
A Onça só se empapando\\
Comendo farta ração,\\
Devorando os animais\\
Sem a menor compaixão.

O Bode compareceu\\
Num banquete oficial,\\
Mas quando quis regressar\\
Sofreu um golpe fatal,\\
Foi comido pela Onça\\
Sem choro, sem funeral.
\pagebreak

Todos os bichos fugiram\\
Ninguém mais contava broca,\\
Maribondo, amedrontado,\\
Já não saía da toca;\\
No reino arisco dos bichos\\
Tudo corria à matroca.

Quando acabou o governo\\
Desse tempo de sobrosso,\\
No palacete da Onça\\
Tinha um montão de caroço\\
E no tesouro do reino\\
Uma montanha de osso!
\end{verse}

\chapter{O Brasil rompeu com “eles”}

\begin{verse}
Chegou também para nós\\
O momento decisivo.\\
Nosso Brasil sobranceiro\\
Não nasceu para cativo.\\
Da liberdade, no peito\\
O sentimento tem vivo.

Vamos agora lutar\\
É contra a barbaridade.\\
O Brasil nessa missão\\
Age agora de verdade,\\
Pois vai bem alto gritar\\
Pelo bem da humanidade.

Não podia a nossa pátria\\
Se quedar covardemente.\\
Do contrário, os filhos seus\\
Rolariam na torrente.\\
O resultado seria\\
A peia, o tronco e a corrente.
\pagebreak

As nações totalitárias\\
Querem o direito esmagar.\\
Japão, Itália, Alemanha,\\
Querem o mundo escravizar,\\
Mas a nação brasileira\\
Tal não pode tolerar.

Em cada peito brasílio\\
Bate um livre coração.\\
Brasileiro não nasceu\\
Para arrastar-se no chão.\\
É preferível morrer\\
A ser servo de alemão.

Contra o regime nazista\\
Lutaremos com vigor.\\
Vamos mostrar firmemente\\
Também o nosso valor.\\
Assim faz o brasileiro\\
Que a seu país tem amor.

Nosso Brasil não deseja\\
Ver no mundo a escuridão.\\
Está soberbo, do lado\\
Onde se ergue a razão.\\
Vai em prol da liberdade\\
Trabalhar como um leão.
\pagebreak

O Novo Mundo não pode\\
Ficar de braço cruzado.\\
Este nosso continente\\
Hoje vive ameaçado.\\
Está exposto à cobiça\\
Do velho lobo esfaimado.

O nazismo alvoroçado\\
A sua vela hoje enfuna.\\
Ele tenta atravessar\\
Para cá na sua escuna.\\
É preciso pôr-se em guarda\\
Contra a tal Quinta Coluna.

Essa gente traiçoeira\\
Age na sombra da noite.\\
Mas infeliz do atrevido\\
Que nessa empresa se afoite:\\
Entra no arame farpado,\\
Se gritar muito, no açoite.

O Brasil não quer conversa\\
Com partidários do Eixo.\\
Foi por causa dessa gente\\
Que no mundo há remeleixo,\\
Que o nazismo impertinente\\
Só fala em murro no queixo.
\pagebreak

Rompemos as relações\\
Com japonês e alemão.\\
Italiano também\\
Vem de carrinho de mão;\\
Amizade só existe\\
Com país que for irmão.

Ninguém quer mais ter negócio\\
Com quem é totalitário.\\
É ser quinta-colunista\\
Quem pretender o contrário.\\
Quem infringir esta norma\\
É cabra muito ordinário.

O dever do brasileiro\\
É agora vigiar;\\
Ter cuidado de verdade\\
Com quem quiser se mostrar.\\
A decisão do Brasil\\
A gente tem que apoiar.

Nós estamos lado a lado\\
Das nações americanas.\\
Vamos mostrar quanto vale\\
Os que gostam de bananas\\
E vão agora enrascar\\
As tais potências tiranas.
\pagebreak

O Brasil furou na venta\\
E não quer mais lero-lero.\\
Japonês ou alemão\\
Pra nós não diz quero-quero;\\
Não vêm soprar deste lado\\
Com sua boca de mero.

O nazismo aqui não forma,\\
Aqui não tem cotação.\\
O Brasil é democrata, \\
De sangue, de coração,\\
Não vai nessa cantoria\\
Que canta o povo alemão.

Não queremos essa coisa,\\
Essa doutrina nazista.\\
Isso foi coisa inventada\\
Por gente malabarista\\
Pra tapear meia dúzia,\\
Igual a fogo de vista.

Japonês foi traiçoeiro\\
Contra a América do Norte,\\
Mas na sua falsidade\\
O Japão não teve sorte.\\
Agora, é vivo, ele vai,\\
Sentir o frio da morte.
\pagebreak

Japonês andou fingindo\\
Que era um anjo de candura,\\
Mas de repente mostrou\\
Quanto tem a cara dura,\\
Agredindo de emboscada\\
Pensando que era bravura.

Essa gentinha irritou\\
A gente do Novo Mundo.\\
Esse golpe, francamente,\\
Foi na verdade profundo,\\
Mas foi golpe que envergonha\\
Até mesmo um vagabundo.

O Brasil logo falou \\
Pela voz do presidente.\\
Essa ação do japonês\\
Foi ação muito inclemente.\\
Quem quiser diga o contrário,\\
Que fica logo é sem dente.

As nações americanas\\
Fizeram coligação.\\
Houve a grande conferência\\
Para maior união\\
E mostrar que nós aqui\\
Não temos medo do cão!
\pagebreak

Vamos agora ficar\\
Numa firme posição.\\
Vamos pôr a vista em cima\\
Do tal nazismo alemão;\\
Vamos pegar japonês\\
Que se mostrar espião.

Contra as três nações do Eixo\\
Nós vamos força fazer.\\
Não queremos transações\\
Com quem só quer nos morder,\\
Com quem vive eternamente\\
A discórdia a promover.

Quem não gostar da verdade\\
Que bem longe se reúna.\\
Quem achar ruim a coisa\\
Na certa é “quinta coluna”,\\
Merece arame farpado\\
Num campo lá da Pavuna.

Quem torcer por alemão\\
Só pode ser traidor,\\
Se torcer por japonês\\
Por nós nega o seu amor,\\
Se pela Itália torcer\\
Rebenta como um tumor.
\pagebreak

Nesta guerra pavorosa\\
A Alemanha se desgraça.\\
Na mão do russo valente\\
O tal Hitler se esbagaça,\\
Enquanto o povo inglês\\
Lhe criva bem a carcaça.

Vai a América do Norte\\
Também lhe dar um sopapo;\\
Nós vamos do nosso lado\\
Meter-lhe um tiro no papo,\\
Deixar “seu” Hitler cair\\
Como cai um jenipapo.

Mussolini vai ficar\\
É vendo o mundo da lua.\\
Japonês há de rodar\\
Direitinho uma perua,\\
Correndo de ventarola\\
Doido no mundo da lua.

Queremos ver brevemente\\
O tal nazismo por terra.\\
Alemanha hoje resiste\\
E pelo rádio inda berra,\\
Mas pode arrumar a trouxa,\\
Que na certa perde a guerra.
\pagebreak

Vive agora o Novo Mundo\\
Unido numa corrente.\\
Vamos mostrar nossa força\\
Deste lado do ocidente\\
E mandar muito reforço\\
Para o lado do oriente.

Japonês é traiçoeiro,\\
Mas agora come fogo;\\
Alemão é convencido\\
Mas na certa perde o jogo;\\
Italiano, coitado,\\
Vai chamar Chico a Diogo.

E no meio dessa luta\\
A França livre aparece;\\
Na figura de De Gaulle\\
cada vez mais se engrandece;\\
o francês livre na guerra\\
Nem um momento esmorece.

O polonês destemido\\
Também terá seu pedaço.\\
Quando chegar o momento\\
Nazista vira bagaço\\
A casa do tal de Hitler\\
Há de ficar num chumaço.
\pagebreak

Lá da África do Norte\\
Alemão já foi varrido;\\
Atrás dele italiano\\
Há muito que foi corrido;\\
Até lá na Abissínia\\
Já foi tudo resolvido.

Mussolini agora vê\\
Uma derrota de sobra.\\
Isso foi muito bem feito,\\
O final de sua obra,\\
Mas ele berra, esperneia,\\
Fica mesmo que uma cobra.

Quando chegar o momento\\
De se atacar o Japão,\\
Esse país vai sentir\\
O peso da aviação,\\
Vai chover bomba a valer,\\
Vai ser pior que um vulcão.

O hemisfério ocidental\\
Está dentro do barulho.\\
Isto é sinal que o nazismo\\
Agora vai de embrulho,\\
Vai esticar o cambito,\\
Ficar debaixo do entulho.
\pagebreak

O chinês vai dando conta\\
Direitinho do recado.\\
Chinês pega o japonês\\
E deixa bem machucado,\\
Quando o bruto mete a cara\\
Sai é todo arrebentado.

A China agora mostrou\\
Que tem é sangue na veia\\
Japonês quando se tem\\
Se esborracha na areia.\\
O china pega o moleque,\\
Sujiga e mete-lhe a peia.

O chinês agora mostra\\
Que não tem mais lero-lero.\\
Para o chinês japonês\\
Já não vale mais um zero,\\
Japonês faz a traição\\
Mas se finge de sincero.

Vamos abrir bem o olho\\
Contra qualquer espião.\\
Essa gente perigosa\\
Quando fere é à traição;\\
Podendo faz sabotagem\\
Prejudicando a nação.
\pagebreak

Brasileiro tem cuidado,\\
Sê um firme sentinela.\\
Defende o nosso Brasil\\
Contra a perfídia amarela.\\
Não te fies na conversa\\
Que te jogam por tabela.

Quando vires o manhoso\\
Tentando se aproximar\\
Fica logo prevenido,\\
Que ele pode te sondar\\
Para saber qualquer coisa\\
Que lhe possa aproveitar.

Não dá nunca confiança\\
Seja lá para quem for.\\
Só confia no aliado\\
Que como tu tem ardor,\\
Que luta por nossa causa\\
Porque lhe sabe o valor.

É preciso que tu saibas\\
Que tomamos posição;\\
Que descendemos também\\
Duma impávida nação,\\
Duma pátria sobranceira,\\
Valente como um leão.
\pagebreak

Brasileiro é de verdade\\
E não nasceu para o relho.\\
No Brasil nação alguma\\
Vem meter o seu bedelho,\\
Se tentar meter a cara,\\
A gente quebra o aparelho.

Quando damos a palavra\\
Não voltamos mais atrás.\\
Demos agora de fato,\\
Pois assim é que se faz.\\
Quando a formiga se assanha,\\
Bota-se em cima aguarrás.

A saúva nesse caso\\
É o tal soldado alemão.\\
Mas agora à nossa custa\\
Ele não come mais pão,\\
Não mandamos para os tais\\
Mais um tico de ração.

Deste lado do hemisfério\\
A porta ficou fechada.\\
Rompemos as relações,\\
Conosco não tem mais nada;\\
O Eixo agora vai ver\\
A coisa muito encrencada.
\pagebreak

Na conferência que houve\\
Lá no Rio de Janeiro,\\
O ministro Osvaldo Aranha\\
Foi quem falou por primeiro,\\
Mostrou que nosso país\\
No que diz é verdadeiro.

Entre todas as nações\\
O Brasil não vacilou.\\
Sua firmeza constante\\
Com orgulho demonstrou,\\
A América do Norte\\
Decidido acompanhou.

Uma nação nossa irmã\\
Não pode ficar sozinha.\\
Vamos lutar a favor\\
Dessa irmã nossa vizinha,\\
Que foi ferida nas costas\\
De maneira tão mesquinha.

Como fizeram com ela\\
Podia ser com o Brasil.\\
Por isso nós não devemos\\
Abandonar o fuzil,\\
Mostrando, em cima da bucha,\\
Disposição varonil.
\pagebreak

Ninguém aqui deste lado\\
Tem medo de japonês;\\
Ninguém teme atrevimento\\
De quem só luta de três;\\
Basta só um brasileiro\\
Pra dar conta de vocês.

Alemão por mais comprido\\
Também não causa impressão.\\
Acabou-se a valentia\\
Depois que veio o canhão;\\
Basta uma bala certeira\\
Pra pôr o bicho no chão.

Italiano também\\
Não faz ninguém se inquietar.\\
Basta uma boa investida\\
Para o rapaz disparar\\
E nunca mais neste mundo\\
Nem um trem o encontrar.

Guerra relâmpago agora\\
Nunca mais que mete medo.\\
Já foi desmoralizada,\\
Não constitui mais segredo.\\
A Rússia tirou-lhe a fama,\\
Na verdade muito cedo.
\pagebreak

Quem não gostar deste livro\\
Não nos merece atenção.\\
É torcedor de nazista\\
É simpático ao Japão.\\
E quem torce pelo Eixo\\
Não pode ser bom cristão.

Conter o hitlerismo\\
É combater a maldade;\\
É lutar pela defesa\\
Do bem, da luz, da verdade;\\
É saber sair a campo\\
Pelo amor da liberdade.

Morra, pois, a tirania,\\
Morra a negra escravidão! \\
Viva o mundo sempre livre,\\
Na mais perfeita união!\\
Morra o totalitarismo\\
Que só prega a servidão!
\end{verse}

\chapter[O azar, a cruz e o diabo\\ Divertida história do homem mais azarado do mundo]{O azar, a cruz e o diabo\break Divertida história do homem mais azarado do mundo}

\begin{verse}
Entenderam os maus fados\\
Que num dia de finados\\
Eu havia de nascer\\
Era um dia muito sério\\
De choro no cemitério\\
Quando este mundo vim ver.

Minha mãe morreu de parto\\
Entre as paredes do quarto\\
Em que me foi dar a luz\\
O meu pai, acabrunhado\\
Pelo golpe inesperado\\
Entregou a alma a Jesus.

Foi assim que dessa sorte\\
Meu avô um homem forte\\
Ficou cheio de aflição\\
E vendo tanta desgraça\\
Entregou logo a carcaça\\
À morte sem compaixão.
\pagebreak

Minha avó uma velhinha\\
Coitada, muito fraquinha\\
Uma semana chorou\\
Só não chorou mais aflita\\
Porque a barca  maldita\\
Com sua foice a matou.

Vendo sorte tão avessa\\
Meu irmão deu na cabeça\\
Cinco tiros de “Nagant”7 \\
A coisa foi tão sentida\\
Que morreu doida varrida\\
A minha única irmã.

Dos parentes só um tio\\
Vinte dias resistiu\\
Mas... mais tempo não viveu\\
A mulher teve igual sorte\\
Chorou tanto sua morte\\
Que no outro dia morreu.

A sogra dele, coitada\\
Ficou tão desesperada\\
Que de noite se enforcou\\
Uma vizinha com pena\\
Morreu ao ver essa cena\\
Porque do susto tombou.
\pagebreak

Ficou em casa o Formoso\\
Um cachorro, que saudoso\\
Muitos dias não durou\\
E uma cadela faceira\\
Que era a sua companheira\\
Comeu “bola” e se matou.

Um filho dessa cadela\\
Ao sentir a falta dela\\
Quatorze dias uivou\\
E magro, feio, pirento\\
Acabou seu sofrimento\\
Num rio em que se jogou.

Um cachorro, seu parente\\
Também ficou descontente\\
Não querendo viver mais\\
Com a coragem que tinha\\
Entregou-se à carrocinha\\
Da correção de animais.

Desse cão a namorada\\
Uma cachorra estimada\\
Não tomou mais nem café\\
A um rio foi destemida\\
E se jogou resolvida\\
Na boca dum jacaré.
\pagebreak

O jacaré num arranco\\
Foi subir para um barranco\\
Pra comê-la com desdém\\
Mas uma onça ligeira\\
Chegou por trás, traiçoeira\\
E deu-lhe cabo também.

E quando já satisfeita\\
Vai caminhando, à direita\\
Aparece um caçador\\
Esse homem vendo o perigo\\
Deu-lhe um tiro no umbigo\\
E a bicha morreu de dor.

Mas do tiro, o bacamarte\\
Deu-lhe um coice com tal arte\\
Que o jogou morto no chão\\
Caiu numa jurubeba\\
Em cima dum tatupeba\\
Que se enterrou pelo chão.

Minha gente, foi assim\\
No dia que ao mundo vim\\
Causando tanto terror\\
Criei-me sem um carinho\\
Andando sempre sozinho\\
Neste mundo sofredor.
\pagebreak

E quando eu tinha dez anos\\
Vivendo dos desenganos\\
Me apareceu Satanás\\
Olhou-me muito contente\\
E me disse sorridente:\\
– O que é que você faz?

Sem o menor medo dele\\
Eu disse tudo pra ele\\
Mas não contei meu azar\\
Ele então me convidou\\
E traiçoeiro implorou\\
Para ao seu reino eu chegar.

Contou-me então muita peta\\
O patife do Capeta\\
Pensando se disfarçar...\\
Mas do besta do Diabo\\
Eu já tinha visto o rabo\\
Um toquinho a balançar.

No bolso eu tinha uma cruz\\
Com a imagem de Jesus\\
Que ele não pôde avistar\\
Eu tive então um palpite\\
Aceitei logo o convite\\
De seu reino visitar.
\pagebreak

Quando foi à meia-noite\\
Da ventania no açoite\\
Fomos no inferno bater...\\
Mais de quinhentos diabinhos\\
Agitando os seus rabinhos\\
Vieram nos receber.

Pregada numa parede\\
Vi uma alma com sede\\
Pedindo água a chorar\\
A mãe do Diabo, de touca\\
Esfregava sal na boca\\
Da pobre alma a penar.

Metida numa cadeira\\
Vi durante a noite inteira\\
Uma pobre meretriz\\
Em meio à grande quentura\\
Fazia pena a criatura\\
Aquela alma infeliz.

E num quarto muito estreito\\
Eu vi um ex-sub-prefeito\\
Com fome pedindo pão\\
Um diabinho cor de sarro\\
Num peniquinho de barro\\
Lhe dava estercos do Cão.
\pagebreak

Um oficial de justiça\\
Arrolhado com cortiça\\
Do corpo em certo lugar...\\
Num castigo extravagante\\
Muitos litros de purgante\\
Era obrigado a tomar.

E um camarada que em vida\\
Tinha o vício da bebida\\
Eu vi também a penar\\
O pobre cheio de mágoa\\
Ali bebia era água\\
Para o tributo pagar.

E mais adiante eu deparo\\
As culpas pagando caro\\
Um advogado ladrão\\
Dos seus bens o despojavam\\
E depois o encarceravam\\
Em fedorenta prisão.

E numa escura saleta\\
Rodando numa roleta\\
Sem se poder abaixar\\
Um jogador via as fichas\\
Correndo qual lagartixas\\
Quando ele as ia pegar.
\pagebreak

Um velho saçariqueiro\\
Que gastou muito dinheiro\\
E foi na vida um imoral...\\
Sofria um frio danado\\
Nu, numa adega amarrado\\
Sem lenitivo pro mal.

Uma família orgulhosa\\
Que fora muito vaidosa\\
Recebia a provação\\
Sem mais valer o seu nome\\
No chão, comia com fome\\
Num cocho, ao lado dum cão. 

Um demoninho vermelho\\
Dava surra de relho \\
Numa velhota feroz\\
Essa velhota foi sogra\\
Pior do que uma cobra\\
Trazia o genro no cós.

Eram tais quadros terríveis\\
Eram cenas tão horríveis\\
Que me faziam tremer\\
Veio um diabinho zarolho\\
Satanás piscou-lhe um olho\\
Pro danado me prender.
\pagebreak

Pra que eu não desconfiasse\\
E a seu sequaz me entregasse\\
A seu lado me chamou\\
Deu-me um bruto saco de ouro\\
Pensando ser eu calouro\\
Mas comigo se enganou.

Quando eu, de posse do saco\\
Já enchia de tabaco\\
O meu cachimbo de pau\\
Ele gritou pro zarolho\\
Que me metesse num molho\\
E lhe fizesse um mingau.

Eu pulei igual um macaco\\
E tirei do meu casaco\\
A minha bendita cruz\\
A diabada, em disparada\\
Não teve tempo pra nada\\
Fugindo da santa luz.

E acabei com o sacrifício\\
Livrei do grande suplício\\
As almas, mais de um milhão\\
E só pelo desaforo\\
Peguei cem sacos de ouro\\
Da festa o melhor quinhão.
\pagebreak

E deixando o azar no inferno\\
Tomei nota no caderno\\
Dessa aventura sem par...\\
Desde essa noite bendita\\
Relata, então, minha escrita\\
Que Satanás tem azar.

Eu escapei por um triz\\
Mas hoje vivo feliz\\
Quando morrer, vou pro céu...\\
Quem sofrer de assombração\\
Quem tiver medo do Cão\\
E só pedir meu chapéu.
\end{verse}

\chapter{Peleja de Chico Raimundo e Zé Mulato}

\begin{verse}
Seu Quincas, homem festeiro, \\
Festejando São João,\\
Fez reunir em sua casa\\
Uma seleta sessão,\\
Enfeitou todo o terreiro\\
De bandeirinha e balão.

Em frente da residência\\
Fez uma bruta fogueira\\
E a gente da rendondeza\\
Convidou pra brincadeira,\\
Para tomar aluá\\
Com bolo de macaxeira.

Era uma data querida,\\
Era um fato extraordinário,\\
Compareceu Zé Paulino,\\
Mané Felipe e Macário, \\
Chico Dantas, Izidoro\\
E até “seu” próprio vigário.
\pagebreak

Muitos rapazes e moças\\
Compareceram à função\\
Para render homenagem\\
Ao glorioso São João,\\
Que batizou Jesus Cristo\\
Lá no rio do Jordão.

A fogueira era bonita,\\
Tinha dois metros de altura,\\
Tinha um metro de quadrado\\
Com madeira de espessura\\
E foi de tarde arrumada\\
Pra cedo fazer figura.

A meninada contente\\
Já queimava busca-pé,\\
Correndo pelo terreiro\\
Fazendo grande banzé,\\
Dando viva a São João\\
Que brincalhão também é.

Na cozinha a mulherada\\
Punha fogo na canjica,\\
Fazia broa de milho\\
Punha a pamonha na estica,\\
E pra guardar o aluá\\
Tirava o pote da bica.
\pagebreak

Seu Quincas migava fumo\\
Em meio àquele desleixo\\
Pra oferecer aos convivas\\
Que só trazem mesmo o queixo\\
Que a sua custa não fumam\\
Mas dado, vai mais de um feixo.

No terreiro em barulhada\\
A gente via a criação,\\
Galinhas, patos, perus,\\
Se misturando com o cão,\\
Um cachorro que acudia\\
Pelo nome de “Sultão”.

E no jirau cochilava\\
Um gato velho caduco,\\
Batendo a pata um cavalo\\
Perseguido de mutuca,\\
Tremia a espinha dorsal\\
Desde o rabo até a nuca.

Tinha pipoca de milho\\
Com rapadura socada, \\
Tinha jerimum com leite,\\
Pé-de-moleque e cocada,\\
Tapioca e mel de cana,\\
Não faltava mesmo nada.
\pagebreak

Matou-se um grande capado,\\
Um bode pai de chiqueiro, \\
Um peru, duas picotas,\\
Três galinhas, um carneiro,\\
Quatro patos, dois frangotes,\\
Mais um capão do terreiro.

Tudo aquilo inda era pouco\\
Pra gente que se aguardava,\\
Pois de todo o povoado\\
De instante a instante chegava\\
O pessoal que na festa\\
Se divertir esperava.

Todos os anos o Quincas\\
Fazia aquela função\\
Pois pelo santo querido\\
Tinha grande devoção\\
E o botava num altar\\
Onde fazia oração.

São João todo enfeitado\\
Com lindo manto de prega\\
Olhava pra santo Antônio\\
Como quem riqueza alega\\
E o pobre santo humilhado\\
Ralhava com o seu colega.
\pagebreak

Naquela noite encantada\\
Ia ali um cantador\\
De fama e muito temido\\
Por ter no verso vigor\\
E cantar quarenta dias\\
Com qualquer um trovador.

Diz que o cabra era valente\\
Fazia as cordas tinir,\\
No martelo, na embolada,\\
Da glosa, no repetir,\\
Cantava de dia e noite\\
Sem ter tempo de cuspir.

Já tinham mesmo avisado\\
Que eu temperasse a garganta\\
Pusesse o lombo de molho,\\
Rezasse para uma santa\\
Que se eu cantasse de tarde\\
Não chegava nem pra janta.

Me disseram que a viola\\
Eu botasse no seguro,\\
Que ele cantando no claro\\
Melhor cantava no escuro,\\
Da pedra fazia pau,\\
Do mole fazia duro.
\pagebreak

Que mais de cem cantadores\\
Lhe tinham pedido paz,\\
Cantavam só meia hora\\
Depois não queriam mais,\\
Voltavam de cara torta,\\
Com o calcanhar para trás.

Que com ele, cantador\\
Nunca pôde meter cunha,\\
Que ficava mais sem gosto\\
Que caroço de pupunha\\
E saía esmigalhado\\
Que nem piolho na unha.

Que ele era mesmo terrível,\\
Dos seus rivais dava cabo,\\
Que vencendo um cantador\\
Pegava então pelo rabo\\
E mandava direitinho\\
Para a casa do diabo.

Vencia o dono da casa\\
A mãe, o pai, mais a filha,\\
O sobrinho e o cunhado\\
E depois botava a cilha\\
E montava em todos eles\\
Pra ir cantar noutra ilha.
\pagebreak

Das cantorias que fez\\
Já tinha um grande caderno,\\
Cantando pelo verão\\
Ia entrando pelo inverno,\\
Dizia não ter receio\\
De cantar nem no inferno.

Toda gente só falava\\
Ao cantador Zé Mulato\\
Que dava surra de relho\\
E pisava com o sapato\\
Como quem mata saúva,\\
Barata, formiga ou rato.

Seu Quincas, fazendo pouco,\\
Me disse: prepara o couro,\\
Te encomenda logo a Deus,\\
Pois eu não quero ver choro,\\
Que ele te faz em coalhada\\
Bebe a nata, deixa o soro.

Eu respondia me rindo:\\
Seu Quincas, Chico Raimundo\\
É cantador afamado,\\
Inda não viu neste mundo\\
Garrafa sem ter gargalo,\\
Bolso de calça sem fundo.
\pagebreak

Zé Mulato, convencido,\\
Dizia pra toda gente: \\
— Vou pegar Chico Raimundo\\
Deixar-lhe a boca sem dente,\\
Botar-lhe o peito pra trás,\\
Virar-lhe o lombo pra frente.

Vou deixar-lhe todo o corpo\\
Pior que cana com broca,\\
Sangrar, tirar os miúdos\\
E depois fazer paçoca\\
Pra fazer isca de anzol\\
Quando não achar minhoca.

Vou mostrar com quantos paus\\
Zé Mulato faz jangada,\\
Vou deixar a cara dele\\
Pior que lata amassada,\\
E com palito de dente\\
Deixar-lhe a tripa virada.

Vou tirar-lhe o couro inteiro\\
Com serrote enferrujado, \\
Decepar-lhe os mocotós\\
Com canivete amolado,\\
Tirar-lhe o sangue das veias\\
Para estrumar meu roçado.
\pagebreak

Até que chegou a noite\\
Em que nos demos de cara,\\
Era uma noite de lua\\
Muito linda e muito clara\\
E dando início à festança\\
Uma ronqueira dispara.

O Zé Mulato vaidoso\\
Foi logo matando o “bicho”,\\
Depois disse presumido:\\
Desculpem que é meu capricho\\
Beber um trago de “cana”\\
Quando vou pisar no lixo.

Depois me disse: Atrevido!\\
Prepara logo o caixão,\\
Toma a bênção do teu pai\\
Faz a tua confissão,\\
Escreve teu testamento,\\
Que vivo não voltas, não.

Eu então lhe respondi: \\
— O caixão já preparei\\
Mas não fiz a confissão\\
E nem a bênção tomei,\\
Mas o caixão construído\\
Foi pra você que arranjei.
\pagebreak

Zé — Você pra mim é menino\\
Amarelão, come barro,\\
É taquari de cachimbo,\\
É bagana de cigarro,\\
É soluço, não gemido,\\
Nunca foi tosse, é pigarro.

Chico – E tu não és nem menino,\\
És um feto desumano,\\
És rede suja de cego,\\
És um boneco de pano,\\
És chave sem fechadura,\\
És água suja de cano.

Z – És uma faca sem cabo,\\
És espingarda sem cão,\\
És um balão sem esponja,\\
Lata velha de covão,\\
És selo já carimbado,\\
Colarinho sem botão.

C – És uma trempe sem pernas\\
És um velho ao pé dum novo,\\
És um pincenê sem vidro\\
És mictório do povo,\\
És uma galinha choca\\
Chocando casca de ovo.
\pagebreak

Z – Tu és um pano nojento\\
Limpando lugar imundo,\\
És um penico sem beira,\\
És uma mala sem fundo,\\
És pior do que bexiga,\\
Do que doença do mundo.

C – És uma telha de vidro\\
Numa casa em noite escura,\\
És duas bandas de saco\\
Sem ter boca e sem costura,\\
És mesmo que mentiroso\\
Que beija os dedos e jura. 

Z – Quando eu nasci foi cantando\\
E pedindo uma viola,\\
Aprendi conta e leitura\\
Mas nunca fui numa escola,\\
Pois eu não sou como tu\\
Um mentiroso e gabola.

C – Cantador melhor que tu\\
Eu já quebrei pelo meio\\
Depois com pena emendei\\
Botei sela e botei freio\\
Para em dia de domingo\\
Dar lá na feira um passeio.
\pagebreak

Z – A minha fama é medonha\\
Vencedor eu nunca achei\\
E duma vez um país\\
Com um eco enorme espantei,\\
Pensavam que era um vulcão\\
Mas foi um grito que eu dei.

C – Eu também sou perigoso\\
Sou mesmo de qualidade,\\
Pois quando ainda menino,\\
Com poucos anos de idade,\\
Fui soprar um fogareiro\\
E fiz uma tempestade.

Z – É minha sina ser forte,\\
Com gente fraca eu embirro\\
E vendo um cabra covarde\\
Mais o meu ânimo acirro,\\
Derrubei uma floresta\\
Ao dar na mata um espirro.

C – Não tenho medo de nada\\
De tudo faço remoque,\\
Pego baleia em paneiro\\
Mato onça de bodoque,\\
Acabo com um batalhão\\
Tendo na mão um estoque.
\pagebreak

Z – Mergulho com jacaré,\\
Dou cascudo em tubarão,\\
Eu faço cobra dançar,\\
Maneta jogar pião\\
E quando estou com calor\\
Eu pego o vento com a mão.

C – Eu dando um grito na praia\\
Faço o oceano parar,\\
Faço nuvem andar pra trás,\\
Faço a terra balançar,\\
Ensino boi a cuspir\\
E faço mudo falar.

Z – Eu pego boi de arapuca\\
Meto cavalo em gaiola\\
Faço tatu virar cobra\\
E sapo tocar viola,\\
Faço um rico esmoléu\\
E mendigo dar esmola.

C – Eu campeio tempestade\\
Montado num furacão,\\
Arrebento tromba d’água\\
Só com dois dedos da mão,\\
Pego o cabra mais valente\\
E dele faço pirão.
\pagebreak

Z – Eu derrubo uma montanha\\
Dando nela um pontapé,\\
Luto com vinte nações,\\
Furo ferro com quicé,\\
Faço cachorro miar\\
E gato tomar rapé.

C – Faço luz em lamparina\\
Sem querosene e pavio\\
E faço um rei ficar pobre\\
Chamando o gato meu tio,\\
Com quarenta graus de febre\\
Eu penso que tenho frio.

Z – Embora fique sabendo\\
Que teu avô não concorda,\\
Agüento a tua avó\\
Dou-lhe uma surra de corda,\\
Eu caindo dentro d’água\\
O mar dez dias transborda.

C – Te pego por uma orelha\\
Dou-te um murro no cangote\\
E rasgo pela barriga\\
Como quem rasga caçote\\
Dou-te de pau, porque tu\\
Desonra até um chicote.
\pagebreak

Z – Viro-te o queixo do avesso\\
Com o peso do meu murro,\\
Depois te puxo pra frente\\
Como quem arrasta um burro,\\
Pego uma peia bem grossa\\
E todo o tempo te surro.

C – Eu te esfolo de terçado\\
E penduro na parede,\\
Dou teu sangue pras formigas\\
Pra que não morram de sede,\\
Depois espicho teu couro\\
Pra fazer punho de rede.

Z – Eu te pelo n’água fria\\
Pra tirar todo o cascão,\\
Depois te quebro de pau\\
E te soco num pilão\\
E meto tudo num saco\\
Dou de presente pro cão.

C – Tu és mesmo que Caim\\
Depois que matou Abel,\\
Tens mais veneno na língua\\
Do que mesmo a cascavel,\\
Onde tu deixa a saliva\\
Ela logo vira fel.
\pagebreak

Z – Eu estou reconhecendo\\
Seres árvore de fronde,\\
Na toada que se fala\\
Nessa mesma tu responde,\\
És um cantador valente\\
Que de medo não se esconde.

C – Faço velho criar dentes, \\
Valentão vestir anágua,\\
Coruja cantar bonito\\
De encher o peito de mágoa,\\
Canto um mês sem descansar\\
Sem beber um pingo d’água.

Z – Um cantador como tu\\
Com certeza nasceu feito,\\
É neste mundo o primeiro\\
Que no cantar eu respeito\\
E a quem confesso também\\
Que me sinto satisfeito.

C – A minha fama é valente\\
Você agora a celebre, \\
Eu não sou empanzinado\\
Que anda morrendo de febre\\
Eu não sou pau de imbaúba\\
Que qualquer moleque quebre.
\pagebreak

— Quando você regressar\\
Não diga que fez furor,\\
Vá se curar desta surra\\
Que lhe deu um cantador\\
E pode por minha conta\\
Mandar chamar o doutor.

— Pra cantar lá na Inglaterra\\
Me chamaram outro dia\\
Eu mandei dizer ao rei\\
Que lá cantar não queria\\
Pois a Europa era titica\\
Só numa noite eu vencia.

Z – Agora o dono da casa\\
Por favor traga galinha,\\
Umas pipocas de milho\\
E café com bolachinha\\
E traga pra Zé Mulato\\
Purgante de cabacinha.
\end{verse}

\chapter{Combate e morte de “Lampião”}

\begin{verse}
Na famosa Vila Bela,\\
Estado de Pernambuco\\
Nasceu o chefe feroz\\
Do cangaço e do trabuco,\\
Que durante muitos anos\\
Nosso sertão pôs maluco.

José Ferreira da Silva\\
Era o nome de seu pai,\\
Morto também em combate\\
Há muito tempo já vai,\\
Mas o filho se levanta\\
Ainda bem ele não cai.

O seu nome de batismo\\
Foi Virgulino Ferreira,\\
Queria vingar o pai\\
Numa luta carniceira,\\
Tocar fogo no sertão,\\
Numa fúria verdadeira.
\pagebreak

Esse chefe de cangaço\\
Era o grande Lampião\\
Que comandava sozinho\\
Um terrível batalhão,\\
Esquartejando com raiva\\
Quem lhe caía na mão.

Os sertanejos viviam\\
Debaixo de pesadelo\\
O nome de Lampião\\
Arrepiava o cabelo,\\
Do banditismo sem nome\\
Era o perfeito modelo.

O próprio Antônio Silvino\\
Junto dele foi um santo,\\
Pois esse antigo jagunço \\
Nunca causou tal espanto,\\
Combateu com a polícia\\
Sem chegar jamais a tanto.

O terrível Lampião\\
Usava dente de ouro,\\
Tinha óculos da moda,\\
Bonito chapéu de couro;\\
Tinha um defeito num olho\\
Que lhe causava desdouro.
\pagebreak

Quando invadia um lugar\\
À frente da cabroeira,\\
Matava velho a punhal,\\
Castrava de “lambedeira”, \\
Na pobre localidade\\
Não deixava casa inteira.

Uma vez, numa cidade\\
Lampião apareceu,\\
Cinco rapazes pegou\\
E de punhal abateu;\\
Tirando o sangue de um deles\\
Um dos seus cabras bebeu.

Um rapaz que estava noivo \\
Num esteio ele amarrou,\\
A noiva fez ficar nua,\\
Com ferro em brasa a marcou\\
E o noivo, desesperado,\\
O criminoso castrou.

No tempo de Artur Bernardes\\
Ele foi a Juazeiro.\\
Padre Cícero o chamou,\\
Fez-lhe um sermão verdadeiro;\\
Lampião quase chorou,\\
Cabisbaixo no terreiro.
\pagebreak

O saudoso sacerdote\\
Condenou a sua vida,\\
Disse que aquilo era um crime,\\
Era uma grande ferida,\\
Que Lampião precisava\\
Ter a alma redimida.

O bandoleiro partiu\\
Conduzindo a sua gente,\\
Mas depois virou o cão\\
Matando até inocente,\\
Cometendo malvadez\\
A mais negra e repelente.

Prometeu, porém, deixar\\
O Ceará sossegado\\
Por causa do seu padrinho,\\
Padre Cícero adorado,\\
Do sertão do Juazeiro\\
O chefe mais respeitado.

Foi assim que o Ceará\\
De Lampião não sofreu,\\
Porque naquele sertão\\
Nunca mais ele mexeu,\\
Respeitando o seu padrinho\\
Que tais conselhos lhe deu.
\pagebreak

Rumou, então, o bandido\\
Para lugares distantes.\\
Teve combates violentos,\\
Teve terríveis instantes,\\
Matou soldado à vontade,\\
Das tais polícias volantes.

Uma vez, um pobre velho\\
denunciou Lampião.\\
O bandido, sabedor,\\
Prometeu botar-lhe a mão\\
E vingar-se do velhinho,\\
Fazendo dele pirão.

O infeliz ficou lesando,\\
Sem saber o que fizesse.\\
Tratou logo de arribar,\\
Mas Lampião não se esquece\\
E certo tempo depois\\
De repente lhe aparece.

Surrou o pobre ancião\\
E depois mandou castrá-lo,\\
Em seguida o colocou\\
Na cangalha de um cavalo,\\
À procura de um seu filho\\
Mandou um cabra tocá-lo.
\pagebreak

Quase morto, o desgraçado\\
Andou dez léguas e tanto.\\
Chegando à casa do filho\\
Causava dor o seu pranto,\\
Jamais se viu um cristão,\\
Oh! meu Deus, padecer tanto.

Bateu na porta do filho\\
Que no quarto se encontrava\\
E que seu pai nesse estado\\
Avistar não esperava,\\
Tão tranqüilo no seu lar\\
Nesse momento ele estava.

Quando veio abrir a porta\\
Viu o pai e o Lampião.\\
Sentiu na alma uma vertigem,\\
Um golpe no coração.\\
Aquele quadro terrível\\
Parecia assombração.

Qual é o filho, meu Deus,\\
Que vê seu pai nesse estado\\
E não fica de repente\\
Como que alucinado,\\
Como que se o coração\\
Tivesse sido esmagado?
\pagebreak

Para pintar esse quadro\\
Era preciso que a gente\\
Molhasse a pena no sangue\\
Desse velhinho inocente,\\
Cujo semblante magoado\\
Nunca mais nos sai da mente.

O cangaceiro feroz\\
Com raiva o velho cutuca.\\
Surge a nora do infeliz,\\
Fica assim como maluca;\\
Nesse instante Lampião\\
Dá-lhe um tiro bem na nuca.

A mulher rola por terra\\
Pondo sangue pela boca...\\
Foi melhor, porque senão \\
A pobre ficava louca.\\
Apenas deu um gemido\\
Sua garganta já rouca.

O rapaz, vendo essa cena,\\
Avança de peito aberto,\\
Mas o bandido, ligeiro,\\
Dá novo tiro, bem certo;\\
O moço cai fulminado,\\
De sangue todo coberto.
\pagebreak

Soluça aflito de dor\\
O desgraçado ancião,\\
Mas, de repente murmura:\\
“Mata, bandido, ladrão,\\
Que teu fim também será\\
Numa ponta de facão!”

Lampião, enfurecido\\
Puxa de novo o gatilho:\\
Matara a nora do velho,\\
Matara o pobre do filho,\\
Agora acaba a família,\\
Liquidando esse empecilho.

Um tiro parte violento\\
Num movimento instantâneo;\\
A bala atinge o velhinho\\
Esfacelando-lhe o crânio\\
E Lampião, debochando\\
Diz: “Adeus, meu conterrâneo!”

Essas façanhas horríveis\\
Era só o que fazia.\\
Lampião era uma fera\\
Que no sertão se escondia,\\
Que causava inquietação\\
No lugar que aparecia.
\pagebreak

Quando encontrava um sujeito\\
Que levava algum dinheiro\\
Roubava tudo sem pena\\
Esse grande bandoleiro\\
Que proclamou a desgraça\\
Do interior brasileiro.

Os governos, muito tempo,\\
Gastaram grande fortuna\\
Para ver se davam cabo\\
Desse bandido turuna,\\
Cujo nome foi citado\\
Até mesmo na tribuna.

Até que um dia um tenente\\
Da raça pernambucana\\
Prometeu do Lampião\\
Cortar a fama tirana,\\
Liquidá-lo com seu bando\\
Em menos de uma semana.

Era o tenente Bezerra,\\
Homem valente de fato,\\
Que na luta era feroz,\\
Na agilidade era um gato,\\
Tanto “brincava” no limpo\\
Como lutava no mato.
\pagebreak

Esse tenente Bezerra\\
Embrenhou-se no sertão.\\
Em pouco tempo mostrou\\
Que tinha convicção,\\
Que queria de verdade\\
Segurar o Lampião.

Deu cercos por toda parte,\\
Mas o bandido escapava,\\
Nas lutas esse tenente\\
Uns quatro ou cinco matava,\\
Mas o bando perigoso\\
Dias depois aumentava.

Lampião mandou dizer\\
Para o seu perseguidor:\\
“Seu Bezerra vá embora,\\
Tenho pena do senhor,\\
Se lhe pego com vontade\\
Você vai ver que horror”.

Mas o tenente Bezerra\\
Respondia a Lampião:\\
“Um de nós tem que morrer\\
Morder a terra no chão,\\
Me parece que você\\
É quem morre no facão”.
\pagebreak

Num cerco bem dirigido\\
Lampião foi descoberto,\\
Mas quando a tropa volante\\
Já dele andava bem perto,\\
O bandido se sumiu,\\
Mostrando ser bicho esperto.

Lampião tinha uma amante\\
Era a Maria Bonita.\\
Nos combates se metia\\
E também fazia “fita”,\\
Saltava nas capoeiras\\
Que parecia cabrita.

Essa mulher apanhava\\
Quando pedia perdão\\
Para algum prisioneiro\\
Do amásio Lampião,\\
Que podia tudo ter,\\
Só não tinha coração.

Tinha o tenente Bezerra\\
O seu grupo bem completo\\
Do mesmo fazendo parte\\
Bravo sargento Aniceto,\\
Bicho bom na pontaria,\\
Soldado mesmo correto.
\pagebreak

Até que enfim, no sertão\\
Do estado de Maceió\\
Se trava luta tremenda\\
De sair faísca e pó,\\
De enfiar faca afiada\\
Na barriga e no gogó.

Lá na fazenda de Angicos\\
Chega o tenente Bezerra.\\
Experimenta a garrucha,\\
Examina se está perra,\\
Como quem sabe as surpresas\\
Dos armamentos de guerra.

Bezerra manda dizer\\
Para o Terror do Sertão:\\
“Prepara o couro, bandido,\\
Dos crimes pede perdão;\\
Hoje é teu último dia,\\
Assassino Lampião”.

O bandoleiro, furioso,\\
Responde em cima da bucha:\\
“Bezerra velho prepara\\
O corpo para a garrucha:\\
O meu tiro é de verdade,\\
Não se perde nem a bucha”.
\pagebreak

O cerco então foi fechando\\
O grupo de Lampião.\\
Viu-se o sargento Aniceto\\
Pulando que nem um cão\\
E doido para acertar\\
A “pitomba” em Lampião.

Pipoca o fogo violento,\\
Só se vê grossa fumaça.\\
Bezerra grita valente:\\
“Eita, bicho, olha a desgraça!\\
Hoje salvo a minha vida\\
ou dou cabo da carcaça!”

Mas o sargento Aniceto\\
De repente se domina.\\
Vendo Maria Bonita\\
Grita com raiva canina:\\
“A mulher de Lampião\\
Tem cara de lamparina!”

Bezerra manda avançar\\
E fazer fogo à vontade.\\
Lampião do outro lado\\
Lhe resiste de verdade;\\
Era enorme no seu lado\\
A terrível mortandade.
\pagebreak

Grita Maria Bonita,\\
Amante de Lampião:\\
“Meu amor, sustenta o fogo\\
Que esse Bezerra é o cão,\\
Mas eu luto até de faca\\
Se acabar a munição!”

Nesse instante o tiroteio\\
Acerta nessa mulher.\\
Ela susteve-se bem firme\\
Com toda força inda quer,\\
Mas, adeus mesa sortida,\\
Foi-se a farinha e a colher!

Lampião vendo a defunta\\
Lhe disse: “Agora é que foi.\\
Adeus, Maria Bonita,\\
Pede a Deus que te abençoe,\\
Que eu por mim já vejo a canga\\
Preparada para o boi”.

Nisto, uma bala certeira\\
Atingiu o Lampião.\\
Ele corre ensangüentado,\\
Pondo a mão no coração,\\
Tropeça nos companheiros,\\
Já sem vida, pelo chão.
\pagebreak

Bezerra vendo o bandido\\
Fez de novo a pontaria.\\
A bala rompe violenta\\
Como Bezerra queria\\
Alcançando Lampião\\
Que estremece de agonia.

Estava findo o combate,\\
Essa luta pavorosa.\\
O bando de Lampião\\
Já não contava mais prosa,\\
O sertão estava livre\\
Dessa gente perigosa.

Depois da luta tremenda\\
Entrou em cena o facão.\\
Foi decepada a cabeça \\
Do famoso Lampião\\
Para Bezerra provar\\
Que matara o valentão.

Foram também decepadas\\
As cabeças principais.\\
Os corpos foram enterrados\\
Sem merecer funerais,\\
E não sei como não foram\\
Comidos por animais.
\pagebreak

Bezerra foi promovido\\
Mais o sargento Aniceto.\\
O chefe de Maceió\\
Tudo achou muito correto,\\
Angicos desde esse dia\\
Ficou de povo repleto.

Eis o relato perfeito\\
Da morte de Lampião.\\
Foi tirado direitinho\\
E sem nenhuma invenção,\\
Pois, meu leitor, eu não gosto\\
De contar carapetão.

Acabou-se, minha gente,\\
A fama do desordeiro;\\
Tranqüilizou-se a família\\
Do Nordeste brasileiro,\\
Contanto que não nos surja\\
Outro novo bandoleiro.
\end{verse}

\chapter{O golpe do seu Gegê ou O choro dos deputados}

\begin{verse}
O Brasil ultimamente\\
Andava muito agitado:\\
Era boato pra burro\\
Surgindo de lado a lado,\\
Deixando o pobre cristão\\
Já quase desanimado.

No dia 3 de janeiro\\
Ia haver uma eleição.\\
Os partidos se empenhavam\\
Em grande competição,\\
Cada qual querendo dar\\
Superior votação.

Um candidato nortista\\
Era bastante estimado,\\
Outro nascido em São Paulo\\
Também era desejado,\\
Mas cada qual a puxar\\
A sardinha pra seu lado.
\pagebreak

No Rio Grande do Sul\\
Flores da Cunha gritou.\\
Lá de Minas, Valadares\\
No mesmo som retrucou,\\
Pernambuco abriu a boca,\\
Mas Maceió se calou. 

Na Bahia Juraci\\
Prometeu tomar partido\\
Por Zé Americo, então,\\
Se demonstrou decidido,\\
Esperançoso de ver\\
O barulho resolvido.

Muita gente na carreira\\
Começou a se alistar,\\
Para na tal eleição\\
O seu votinho arriscar,\\
Que eleitor que não votasse\\
Na cadeia ia parar.

Quem não votasse, na certa, \\
Tinha que ser processado.\\
A lei mandava o eleitor\\
Dar o seu voto calado,\\
Não podia era ficar\\
Na sua cama deitado.
\pagebreak

Mas o Zé Povo, malandro,\\
Sempre o futuro prevê;\\
Na grande mão do destino\\
Sem lente mesmo ele lê,\\
E por isso propalava:\\
“Vai ficar é “seu” Gegê”.

Mas o tempo se passava\\
Nesse mesmo ramerrão.\\
A propaganda era feita\\
Para a futura eleição,\\
Enquanto que “seu” Gegê\\
Não ditava opinião.

Armando Sales andava\\
Na campanha sempre alerta.\\
Nos seus discursos dizia\\
Que venceria na certa,\\
Entrava pelo Catete\\
Cuja porta estava aberta.

Zéamerico, por fim,\\
Desconfiou do terreiro.\\
Não quis seu nome escrever\\
Do tribunal no letreito...\\
E também na propaganda\\
Não sacudiu seu dinheiro.
\pagebreak

Quando menos se esperava\\
Flores da Cunha correu.\\
Abandonando o governo\\
Para longe escafedeu,\\
Enquanto Armando de Sales\\
Desapontado tremeu.

Para o fim, o “seu” Gegê\\
Já estava tiririca,\\
Mas usando de prudência\\
O tema velho ele explica:\\
“Vamos deixar como está\\
Que é para ver como fica.”

O Zé Povo pela rua\\
Aguardava a hora H.\\
Uma canção perguntava:\\
“O tal homem quem será?\\
Será mesmo “seu” Manduca\\
Ou será o “seu” Vavá?”

No Pará, para a eleição\\
Havia ruma de flores.\\
Falaram de parte a parte\\
Uma porção de oradores;\\
Nesse dia votariam\\
Mais de cem mil eleitores.
\pagebreak

Nas asas dum avião\\
Desce um dia lá de cima\\
Um deputado mineiro\\
Chamado Negrão de Lima,\\
Que do Norte velozmente\\
Em dois tempos se aproxima.

Em cada estado que chega\\
Ele conversa em segredo,\\
A muito governador\\
Deixando morto de medo,\\
Prometendo em quem falasse\\
De rijo tocar o dedo.

O recado era avisando\\
Esta verdade fatal:\\
O Brasil não suportava\\
A campanha eleitoral,\\
Que viria nos deixar\\
Perante a crise bem mal.

Seu Negrão nesse passeio\\
Demorou-se muito pouco,\\
Deixou metade do povo\\
Sem saber, comendo coco,\\
Voltando, para esperar,\\
Lá de bem longe, o pipoco.
\pagebreak

Alguns jornais do Brasil\\
Começaram na “trancinha”\\
Dizendo pra “seu” Negrão:\\
“Quando tu ia eu já vinha;\\
Esse passeio, meu nego,\\
É porque tem boi na linha”.

Até que a 10 de novembro\\
Do ano de trinta e sete\\
O presidente Getúlio\\
Dissolveu quase a bofete\\
A tal Câmara e o Senado\\
Virando tudo em confete.

Deputado havia “à beça”\\
Do nosso pobre Brasil\\
Sete deles no Pará\\
Praticaram um ato vil,\\
Aquela negra traição\\
Do dia 4 de abril.

Presunçosos de verdade\\
Ainda contavam bravata,\\
Mas mereciam talvez,\\
Era apanhar de chibata,\\
Pela traição cometida\\
No governo do Barata
\pagebreak

Se existia deputado\\
Que presava seu mandato,\\
Também haviam daqueles\\
Que só jogando no mato,\\
Mais de três dedos abaixo\\
Do rabo sujo dum gato.

No Pará fazia pena\\
Ver-se a Câmara em sessão.\\
Só liberal se batia\\
No meio da discussão.\\
O resto todo ficava\\
Era de cara no chão.

O governo não podia\\
Livremente trabalhar,\\
Porque sete dissidentes\\
Andavam sempre a pular,\\
Prometendo a todo o tempo\\
O governo atrapalhar.

Uma vez, a “maresia”\\
Invadiu os arraiais.\\
Um bando de deputados\\
Não se conteve jamais;\\
Foi preciso a intervenção\\
Dos quatorze liberais.
\pagebreak

Se não fosse aquele auxílio\\
Haveria confusão;\\
No Pará se repetia\\
O caso do Maranhão\\
Em que Aquiles Lisboa\\
Apanhou um trambolhão.

Eu conheci deputado\\
Que nunca honrou a casaca,\\
Antes de ter esse posto\\
Não valia uma caraca,\\
No bolso velho furado\\
Não possuía pataca.

Mas depois que se elevou\\
Pisava a gente com os pés, \\
Afrontando todo mundo,\\
Os dedos cheios de anéis,\\
Ganhando sem trabalhar,\\
Mais de três contos de réis.

Deputado eu conheci\\
Que não sabia falar,\\
Só sabia era o dinheiro\\
Do Tesouro arrecadar\\
Para depois, na avenida,\\
Impertinente, gastar.
\pagebreak

Andavam cheios de luxo\\
Levando o rei na barriga\\
Pra meter medo ao governo\\
Sempre arranjando uma briga\\
Na gamela do Tesouro\\
Comendo que nem formiga.

Essa vida estava boa\\
Era só pra deputado,\\
Que vivia lordemente\\
Numa chupeta agarrado,\\
Sem coisa alguma fazer,\\
Ganhando bom ordenado.

Era uma “sopa” o mandato\\
Dessa gente comilona,\\
Que fazia, ameaçava,\\
Prometia dar tapona,\\
Rindo do povo à vontade,\\
Como quem toca sanfona.

Nunca se viu coisa assim,\\
Verdadeira safadeza,\\
No Brasil um deputado\\
Era quem tinha grandeza;\\
A fartura quando vinha\\
Era só pra sua mesa.
\pagebreak

Nos municípios, então,\\
Tinha os tais vereadores,\\
Um bando de periquitos\\
Gritalhões e comedores,\\
Politiqueiros de marca,\\
Verdadeiros “cavadores”.

Quando o prefeito queria\\
Alguma coisa fazer,\\
Esses tais vereadores\\
Pensavam logo em “morder”\\
Arranjando um embaraço\\
Para a coisa distorcer.

Prometiam dar pancada\\
E ao prefeito processar;\\
As contas das prefeituras\\
Nunca queriam aprovar,\\
Pelo gostinho somente\\
De tudo mais estragar.

Essa negrada vivia\\
Achando graça de tudo,\\
Enquanto o povo, coitado,\\
Olhava por um canudo,\\
Vendo em cada deputado\\
Um cabra muito orelhudo.
\pagebreak

Eles pensavam que a coisa\\
Nunca mais se acabaria;\\
Julgavam até que o mandato\\
Na certa prosseguiria...\\
Nos olhos deles a gente\\
Essa vontade já via.

O caso, então, do Pará\\
Só a coice de coronha\\
Disse um jornal de São Paulo\\
Nos mostrando a sua ronha:\\
“O Pará é paraíso\\
De sujeito sem vergonha”.

Major Barata caiu\\
Sorvendo a taça de fel,\\
Mas voltou de fronte limpa\\
Para a vida do quartel,\\
Sendo depois promovido\\
A tenente-coronel.

Aqueles que lhe passaram\\
Inesperada rasteira,\\
Ficaram rindo na rua\\
Daquela mesma maneira,\\
Pensando com seus botões\\
Que castigo era besteira.
\pagebreak

E passavam pela gente\\
Mostrando muito rompante,\\
Publicamente exibindo\\
O focinho de elefante,\\
Bancando o trunfo na vida,\\
Cada qual o mais chibante. 

Estavam mesmo pensando\\
Que ficavam sempre assim,\\
Gozando a vida à vontade\\
Depois dum ato ruim;\\
Que a função de deputado\\
Seguiria até o fim.

Porém no 10 de novembro\\
O “seu” Gegê deu o traço,\\
Pegou toda essa negrada\\
E fez virar em bagaço,\\
Deixando o tal de Congresso\\
Que nem calção de palhaço.

Depois do estado de sítio\\
Lá vem o estado de guerra;\\
Novo regime depois\\
Sobre o Brasil se descerra,\\
E deputado não vale\\
Aquilo que o gato enterra.
\pagebreak

“Seu” Gegê pegou a tropa\\
Dizendo: “Não tem café”.\\
Encangou toda a negrada\\
E sacudiu na maré\\
Depois de tê-la corrido\\
A peso de pontapé.

Deputado, se quiser,\\
Agora vai trabalhar.\\
Tem que pegar no pesado\\
Para o sustento cavar,\\
Não é mais como no tempo\\
Que vivia a vadiar.

Foi-se a “sopa” dos três contos\\
Para sentar na Assembléia\\
Principalmente pra muitos\\
Que nunca tiveram idéia\\
De ver na mesa de casa\\
Sobremesa de geléia.

Acabou-se a pretensão\\
Dos sete “cascas de manga”.\\
Eu bem que disse que havia\\
De um dia vê-los de tanga,\\
Caminhando para a rua\\
Tal e qual um boi de canga.
\pagebreak

Deputado no Brasil\\
A gente via era à ufa,\\
Mas, hoje em dia, essa gente\\
A peitorra já não tufa,\\
Perdeu de todo o prestígio,\\
Não vale mais uma bufa.

Essa gente presunçosa\\
Agora não vale nada,\\
Andava muito lampeira,\\
De cabeça levantada,\\
Mas “seu” Gegê pôs abaixo\\
Só duma simples penada.

Vão agora, seus malandros,\\
Trair de novo o Barata.\\
Vão comprar chapéus de Chile\\
E botar broche em gravata.\\
Vocês agora, no duro,\\
Vão mesmo plantar batata.

O povo achou foi bonito\\
Esse formidável tombo,\\
A paulada que vocês\\
Apanharam bem no lombo\\
Levando também na bolsa\\
Um “gostosíssimo” rombo.
\pagebreak

Acabou-se o falatório\\
Novo regime surgiu.\\
Eu quero ver deputado\\
Regressar de onde saiu,\\
Tomando benção a cachorro,\\
Chamando gato meu tio.

Quero ver essa tropilha\\
Pegando na picareta;\\
Cavando pedra na rua,\\
Limpando lixo em sarjeta,\\
Sentindo o peso da vida\\
É no cabo da marreta.

De ex-deputado preciso\\
Para dar colocação:\\
Vestir ceroula em macaco,\\
Pôr suspensório em cação,\\
Fazer colete pra burro,\\
Pegar mutuca de arpão.

Dou casa, mesa e comida,\\
De manhã café com pão;\\
Pra senador federal\\
Eu dou melhor refeição, \\
Penalizado de ver\\
Essa negrada na mão.
\pagebreak

Quero somente depois\\
O “seu” Gegê imitar:\\
Toda essa gente distinta\\
Numa cangalha juntar,\\
Sentar-lhe o pé nos fundilhos\\
E pra bem longe mandar!

Esse relato é fiel\\
Do que tudo aconteceu.\\
Minha pena sem mentir\\
Aqui tudo descreveu,\\
Nada de mais inventei\\
Do que, de fato, ocorreu.

Quem não gostar, me desculpe,\\
Embora não queira bem.\\
Eu falo assim é bufando\\
Que nem caldeira de trem,\\
Mas no fundo da minh’alma\\
Não tenho raiva em ninguém.
\end{verse}

\chapter{Peleja de Armando Sales e Zé Américo}

\begin{verse}
O presidente Getúlio\\
Fez um samba no Catete\\
E apareceu sorridente,\\
Mão no bolso e de colete,\\
Enquanto fora se ouvia\\
De vez em quando um foguete.

Nessa festa se falava\\
Numa tal de sucessão,\\
No futuro presidente\\
Da nossa grande nação,\\
Que virá com seu programa\\
Tomar parte na eleição.

No terreiro do Catete\\
“Seu” Gegê punha e dispunha.\\
Havia gente espremida\\
Que nem piolho na unha,\\
Além do bloco afamado\\
Do bamba Flores da Cunha.
\pagebreak

Veio o Macedo Soares,\\
Barriga cheia de banha;\\
Antônio Carlos, também,\\
Pisando sempre com manha;\\
Benedito Valadares,\\
Mangabeira e Osvaldo Aranha.

Representando com garbo\\
O Partido Liberal\\
Entrou Magalhães Barata,\\
No seu porte marcial,\\
Recebendo nesse instante\\
Aclamação colossal.

E um camarada pergunta:\\
“Oh! Valadares, vem cá,\\
Quem é esse militar\\
Como igual outro não há?”\\
E o Valadares responde:\\
“Esse é o trunfo do Pará!”

Um bando alegre tocava\\
O samba “Verde e amarelo”.\\
Muitos dançavam pulando\\
Que nem pintinho em farelo,\\
E, de batina, sambava\\
O padre Olímpio de Melo.
\pagebreak

Só quem não foi nesse samba,\\
Apesar de deslumbrante,\\
Com receio de ser preso\\
Por ter bancado o chibante,\\
Foi o nortista temido\\
O “seu” Lima Cavalcante.

O Juraci Magalhães\\
Também não foi no “brinquedo”.\\
Correu, por isso, o boato\\
De que ele estava com medo,\\
Mas Juraci declarou\\
Que pra dançar era cedo.

Mas o número melhor\\
Dessa festa no Catete\\
Ia ser um desafio\\
Em que tinisse o cacete,\\
Do qual quem fosse vencido\\
Virava mesmo sorvete.

Dois cantadores de fama\\
Iam medir resistência,\\
Armando Sales, paulista,\\
Cantador de consistência,\\
Zé Americo, nortista,\\
Campeão na persistência.
\pagebreak

Armando Sales chegou\\
Alvoroçando o Catete.\\
Cumprimentou toda gente\\
E sentou num tamborete,\\
Guardando embaixo do mesmo\\
Um formidável porrete.

Chegou, depois, Zé Americo,\\
Cabeçudinho e baixote.\\
Sentou num banco, defronte,\\
Pondo de lado o chicote,\\
Com seus óculos brilhando\\
Que nem um bruto holofote.

Em volta dos cantadores\\
Reuniu-se toda gente,\\
Pois daquele desafio\\
Sairia o presidente,\\
Venceria o que tivesse\\
O peito mais resistente. 

Armando Sales cuspiu,\\
Temperou sua viola,\\
Fez um risinho engraçado\\
Como quem diz: “Não me amola”,\\
Esgravatando as idéias\\
Lá no fundo da cachola.
\pagebreak

Seu Getúlio, descansado,\\
Por ser o dono da casa,\\
Para enganar os convivas\\
O relógio velho atrasa\\
E vai soltar busca-pé\\
Queimando o bicho na brasa.

Zé Américo suspira\\
Enchendo o peito de alento,\\
Toma, depois... “bagaceira”\\
Para ajudar seu talento\\
E se bater com vontade \\
Sem sossegar um momento.

Armando Sales, então\\
Para dizer que tem fé,\\
Faz um cigarro de palha,\\
Toma depois um café,\\
Demonstrando para todos\\
Um paulista como é.

Armando Sales:\\
Quando eu pego um desafio\\
Não quero mais acabar;\\
Seguro o cabra com raiva,\\
Obrigo o bicho a chorar,\\
Nem que ele seja nortista,\\
Que não sabe vadiar.
\pagebreak

Zé Americo: \\
Eu vendo muita bravata\\
Reconheço que é lambança.\\
Não tenho medo da luta,\\
Porque sou de confiança,\\
O meu sangue só se aquenta\\
Quando me meto na dança.

A. S.\\
Já peguei um cantador\\
Que tal coisa não supunha,\\
Virei-lhe a capa dos olhos\\
E depois meti-lhe a unha\\
Mandando os bagos redondos\\
Ao nosso Flores da Cunha.

Z. A.\\
Numa peleja sou baita,\\
Em qualquer muro me esfrego,\\
Em pau que tenha cupim\\
Não meto estopa nem prego,\\
E quando ingresso na luta\\
Dificilmente me entrego.

A. S.\\
Eu não sou de brincadeira,\\
Tão pouco de fantasia\\
E quando estou de veneta\\
Não temo nem minha tia,\\
Passo a mão no meu cacete,\\
Todo mundo se arrepia.
\pagebreak

Z. A.\\
Quando eu vejo um malcriado\\
Dou-lhe até de cocorote,\\
Seguro o cabra com jeito,\\
Meto a cabeça num pote,\\
Depois lhe dou nos fundilhos\\
Uma surra de chicote.

A. S.\\
Não sou menino amarelo\\
Que tem tosse de guariba\\
Nem matuto que se lava\\
Com sabão de copaíba,\\
Não sou bicho flagelado\\
Do sertão da Paraíba.

Z. A.\\
Você me vê pequenino,\\
Mas eu sou cabra travesso;\\
De reumatismo nas juntas\\
Felizmente não padeço,\\
Quando eu pego um paulistano\\
Eu viro logo do avesso.

A. S.\\
Quando eu começo a cantar\\
Do vermelho faço azul,\\
Faço pedra amolecer,\\
Brotar jardim num paul,\\
Se não chegar minha força\\
Vem o Rio Grande do Sul.
\pagebreak

Z. A.\\
Já lhe disse, “seu” Armando\\
Que na peleja sou forte,\\
Não temo cara zangada,\\
Não me amedronto da morte,\\
Eu dando um grito na serra\\
Comparece todo o Norte.

A. S.\\
Não tenho medo de ti\\
Quanto mais doutros valores,\\
Para a luta da eleição\\
Não preciso de favores,\\
Pois só da gente de casa\\
Tenho um milhão de eleitores.

Z. A.\\
Deixa de muita farofa\\
Recolhe teu berimbau.\\
Eu não sei em que te fias,\\
Meu nariz de barucau.\\
O teu padrinho de crisma\\
Era o “seu” Vicente Rao.

A. S.\\
Eu nunca tive padrinho,\\
É mentira de você.\\
Se tivesse escolheria\\
Era aqui o “seu” Gegê,\\
Mas a minha proteção\\
Felizmente ninguém vê.
\pagebreak

Z. A.\\
Eu nasci lá na Paraíba\\
Terra de cabra vaqueiro,\\
Depois cresci, vim pra cá,\\
E em 30 fiz um “banzeiro”,\\
Fui ministro de verdade,\\
Não me troquei por dinheiro.

A. S.\\
Eu nasci lá em São Paulo\\
Terra do bom algodão,\\
Do café que o Brasil\\
Manda até para o Japão; \\
Eu sou filho dum estado\\
Que não teme valentão.

Z. A.\\
Tu podes tudo fazer\\
Mas eu te lavro a sentença,\\
Podes ter muito eleitor\\
Como certa gente pensa,\\
Mas minha tropa pesada\\
Vai tirar a diferença.

A. S.\\
Um cantador como tu\\
Desmerece a minha fé,\\
Anda sempre se queixando\\
Com pulga e bicho-de-pé,\\
É melhor que dê o fora,\\
Vá se lavar na maré.
\pagebreak

Z. A.\\
Se tu começa a insultar\\
Meu Armandinho gabola,\\
Eu te quebro na cabeça\\
O fundo desta viola,\\
Tiro fogo, tiro estrela\\
Da tua bruta cachola.

A. S.\\
Eu sou comprido demais\\
Sou de elevada estatura.\\
Tu não me bates, menino,\\
Deixa dessa boca dura...\\
Para dar-me na cabeça\\
Falta-te muito na altura.

Z. A.\\
Isso agora de tamanho\\
Na minha missa não vai...\\
Quando eu era pequenino\\
Já me dizia meu pai:\\
“Quanto maior é o pau\\
Mais estronda quando cai”.

A. S.\\
Vamos ver nesta peleja\\
Quem abandona o terreiro.\\
Eu garanto de cantar\\
Nem que seja o ano inteiro,\\
Nem que seja lá por cima\\
Se despencando um pampeiro.
\pagebreak

Z. A.\\
Estás fingindo vigor\\
Mas já tens seca a garganta.\\
Agora que comecei\\
Que minha voz se levanta,\\
E um cantador como tu\\
Na minha mesa não janta.

A. S.\\
Eu não posso esmorecer\\
Nesta feroz hora H;\\
Se eu voltar arrependido\\
Banco na vida o gambá\\
E muita gente, zangada,\\
Vem, com certeza, me dá.

Z. A.\\
Tempero a minha viola\\
Com vontade do brinquedo;\\
Vou pra frente saltitando\\
Bem na pontinha do dedo,\\
Para mostrar que não vivo\\
Me amofinando de medo.

A. S.\\
Eu também sou bicho bamba\\
Sou duro no meu terreiro,\\
Não tenho medo de encrenca\\
Sou político matreiro,\\
Sou capaz de desarmar\\
O general Góis Monteiro.
\pagebreak

Z. A.\\
Quando eu vim pra convenção\\
Foi com gana de brigar;\\
Ou me fazem presidente\\
Ou o tempo vai fechar;\\
Passo a mão no cravinote,\\
Faço a negrada dançar.

A. S.\\
Estás com muita vontade\\
De ingressar na brincadeira;\\
Os olhos fitos tu tens\\
Na miragem feiticeira,\\
Mas vamos ver se o Gegê\\
Deixa mesmo a tal cadeira.

Z. A.\\
Estás fugindo da luta,\\
Fazendo insinuação,\\
Só porque já tens certeza\\
De perder na votação,\\
No resultado final\\
Dessa renhida eleição.

A. S.\\
Haja agora o que houver\\
O jeito mesmo é lutar,\\
Pois quando entrei na peleja\\
Prometi não fraquejar,\\
Eu não sou peru de roda\\
Que o papo vive a tufar.
\pagebreak

Z. A.\\
Eu fico tiriricando\\
Quando vejo um valentão;\\
Deixa de tanta folia\\
Que eu não quero falação;\\
Quando eu te vejo gritando\\
Digo: “Sossega, leão!”

A. S.\\
Tu só falas com rompante\\
Fazendo muita careta,\\
Dizes o verso ligeiro\\
E o que te dá na veneta;\\
Só tens mais força no dedo\\
Porque tocas de palheta.

Z. A.\\
Estou vendo que não podes\\
Sustentar o desafio,\\
Do teu jeito, camarada,\\
Eu bem que já desconfio,\\
Tens a garganta rouquenha,\\
Tens o semblante sombrio.

A. S.\\
Vai ficando muito tarde\\
E eu não posso demorar.\\
Minha gente, me desculpe,\\
Tenho que me levantar,\\
Nesta peleja medonha\\
Acabo por me entregar.
\pagebreak

Z. A.\\
Vais fugindo com receio\\
É duma grande derrota;\\
Começaste a valentona, \\
Contando tanta lorota,\\
Mas já vejo o povo em volta\\
Fazendo muita chacota.

Inda tenho resistência\\
Para cantar à vontade,\\
Este toque da viola\\
Sou cantor de qualidade,\\
Chamo vinte, chamo trinta,\\
Todo o povo da cidade.

Quando eu vim pra convenção\\
Foi preparado pra luta;\\
Você não fique pensando\\
Que a presidência disputa,\\
Porque, no fim, você leva\\
É com pau na cocuruta.

Nesse pé, toda a assistência\\
Fica mesmo tiririca\\
Porque chega seu Gegê\\
E para os homens explica:\\
“Vamos deixar como está\\
Para ver como é que fica”.
\end{verse}

