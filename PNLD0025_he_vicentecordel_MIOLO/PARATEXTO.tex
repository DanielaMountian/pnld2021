\chapter{Vida e obra de Zé Vicente}

\section{Sobre o autor}

\noindent{}Zé Vicente, pseudônimo de Lindolfo Marques de Mesquita, foi um dos
principais cordelistas brasileiros. Nascido ainda no século \textsc{xix}, em 11 de
janeiro de 1898, em Belém do Pará, Lindolfo foi jornalista, político,
editor e poeta, entretanto sempre separou sua produção poética de sua
vida pública. Até os anos 1920 trabalhou como repórter no jornal \textit{Folha
do Norte}, momento em que inventou seu pseudônimo para assinar algumas
crônicas humorísticas, que mais tarde levaria para \textit{O Estado do Pará}.
Concomitantemente à carreira de jornalista, Lindolfo era funcionário
público, mas perdeu seu emprego e foi demitido do jornal, com a chegada
de Vargas ao poder em 1930, já que era partidário da oposição.
Em um folheto autobiográfico narrou sua situação em 1930:

\begin{verse}
Eu já quase não sabia\\
Se ainda era brasileiro,\\
Pois até os meus patrícios\\
Me expulsaram do terreiro\\
E eu vivia, minha gente, \\
Na minha terra -- estrangeiro.

-- Você não presta, safado, \\
Você vai pra Arumanduba --\\
Disse-me um dia um sujeito,\\
Fazendo logo uma “truba”,\\
Como se eu tivesse sido\\
Algum moleque cotuba.

Eu vendo a coisa difícil\\
Fui ao Lloyd brasileiro\\
E comprei uma passagem\\
Para o Rio de Janeiro\\
Onde em novembro cheguei\\
Com muito pouco dinheiro.

Lá no Rio de Janeiro\\
Muita coisa me encantou,\\
Mas, também, comi o pão\\
Que Satanás enjeitou\\
Pois o dinheiro que tinha\\
Em pouco tempo acabou.
\end{verse}

Após um
período no Rio de Janeiro volta ao Pará, agora identificado com o
governo, onde exerce diversos cargos públicos chegando a ser prefeito
da cidade de Vigia, durante o Estado Novo, e diretor do Departamento
Estadual de Imprensa e Propaganda (\textsc{deip}), espécie de sucursal estadual
do Departamento de Imprensa e Propaganda (\textsc{dip}). 
As atividade poéticas de Zé Vicente acompanham, portanto, a
trajetória política de Lindolfo, sendo partidário do
``baratismo'' espécie de defesa dos pobres, transformando-as 
em longos poemas políticos sobre o período varguista.

Sua vocação política pode ser observada, por exemplo, logo no primeiro poema desse volume,
``A greve dos bicho'', em que Zé Vicente recorre à estrutura de uma fábula -- o tempo antes do dilúvio, em que os bichos ainda falavam -- para fazer uma analogia com as greves trabalhistas e a criação das leis do trabalho:

\begin{verse}
Ia tudo muito bem\\
Ganhando alegre o seu pão,\\
Mas, uma vez o Quati\\
Se arvorando a sabichão,\\
Falou na necessidade\\
De fazer revolução.

Pedindo logo a palavra\\
Foi, de fato, extraordinário\\
Quando afirmou que o trabalho \\
Precisava de outro horário\\
E lembrou de se aumentar \\
Da bicharada o salário.
\end{verse}

Estudioso da obra de zé Vicente, Vicente Salles, comenta acerca desse poema:

\begin{quote}
Colocado no contexto do Estado Novo, o folheto [``A greve dos bicho''] tem clara “missão
política”: alegoria à inutilidade das greves no mundo dos bichos, que
reflete, como na sociedade brasileira, interesses políticos espúrios,
degradação social, corrupção desenfreada. É uma crítica mordaz dos
costumes. E serve à ideologia estadonovista implantada com o golpe de
1937 com pruridos moralistas da velha concepção burguesa lusitana de
“restauração” dos “bons costumes” e da “dignidade nacional”.
Aspirações das oligarquias que se revezam no poder, em Portugal como
no Brasil, desde os tempos da revolução burguesa, independente do
regime político.\footnote{\textsc{salles}, Vicente. ``Introdução''. In: \textsc{vicente}, Zé. \textit{Cordel na escola}. São Paulo: Hedra, 2003, p.\,14.}
\end{quote}

\section{Sobre a obra}

Inserida no contexto sertanejo, a obra de Zé Vicente é influenciada pela tradição medieval, como a arte dos trovadores, que viria a influir na formação da poesia de cordel, marcada pela figura dos repentistas, violeiros e cantadores.
No entanto, o poeta diferenciou"-se de outros cordelistas por inserir muito das reportagens jornalísticas em seus poemas.

Como mencionado acima, Vicente trabalhou como jornalista em sua juventude, e esse traço de sua personalidade é visível em suas obras.
Como anota Vicente Salles: ``Poeta cordelista emérito, jornalista vocacionado, Zé Vicente
comporta"-se como ativista político, engajado no `baratismo',
expressão política local desvinculada da classe dominante mas que não
teve força para romper''.\footnote{Ibidem, p.\,15.}

Da época de jornalista, as crônicas humorísticas que assinava na \textit{Folha do Norte} também deixaram marcas em sua obra, eivada pelo humor e pela sátira.
É o caso dos folhetos ``O azar, a cruz e o diabo'', composto de 38 sextilhas, e ``O
Pixininga'', com 22 sextilhas. Nesse poemas, Vicente Salles observa uma transformação na obra do poeta, que de ``cronista-humorista já se revelava também humorista-poeta''.

A partir de 1932, a obra de Zé Vicente adquire mais traços engajados, abordando assuntos políticos e sociais de sua época. Como observa Vicente Salles:

\begin{quote}
A década de 1930 mostra o engajamento de Zé Vicente no ciclo das
revoluções, desdobramento dos acontecimentos da década anterior. Os
poetas nos dão enfoques muito precisos dos fatos que se desenrolaram
no mundo, no país e na região, alargando, por vezes, as perspectivas
dos historiadores oficiais. Diante dos acontecimentos, os poetas não
se mantiveram equidistantes. Esse elemento de “participação” é
importante até mesmo quando se esconde sob pseudônimos, ou
simplesmente quando entrega ao público seus folhetos anônimos,
refletindo, em qualquer caso, a forma de participação consequente
como catalisador -- e decerto também formador -- da opinião pública.\footnote{Ibidem, p.\,10.}
\end{quote}

A partir de 1940, nota Vicente Salles, Zé Vicente produziu nada
menos do que sete folhetos sobre a Segunda Guerra Mundial: ``A Alemanha contra a
Inglaterra'' (1940); ``O afundamento do vapor alemão \textit{Graff Spee}'' (1940);
``Alemanha comendo fogo'' (sem data); ``O Japão vai se estrepar!'' (20 de
dezembro de 1941); ``A batalha da Alemanha contra a Rússia'' (25 de julho
de 1941); ``O Brasil rompeu com eles'' (20 de junho de 1943); e ``O fim da
guerra'' (sem data).\footnote{Ibidem, p.\,9.}

\subsection{Síntese dos poemas}

\paragraph{``A greve dos bichos''}

Num mundo controlado por animais, cada um deles possui uma função na
máquina governamental, associando as características físicas desses
animais à sua função social. Em determinado momento, o Quati, segundo
o autor, metido a sabichão, propõe uma revolta. A greve é reprimida
pelo governo, mas em uma assembleia decide-se que o Jacaré,
chefe do governo, deve ser deposto. A Onça mata o Jacaré e assume o
governo, mas passa a reprimir os grevistas. 


\paragraph{``O Brasil rompeu com `eles'\,''}

O autor narra o momento que o estado brasileiro, em 1942, 
após a ratificação pelo governo da chamada Carta do Atlântico, 
rompe com as potências do eixo -- Alemanha, Itália e Japão --, e se une
aos aliados. Caracterizando cada país do eixo, e defendendo que são
povos vis, o poema avisa que a partir dali quem simpatizar com eles
passa a ser um inimigo.

\paragraph{``O azar, a cruz e o diabo. Divertida história do homem
mais azarado do mundo''}

História de um homem que nasce no dia de finados e que desde seu
nascimento só traz desgraça para o mundo. A mãe morre no parto, assim
como praticamente toda a família, em decorrência desse acontecimento.
Aos dez anos, azarado e sem muitas perspectivas, é convidado pelo diabo
a visitar o inferno, onde vê muitas das punições ali existentes. Quando
convidado a ficar, esse garoto tira do bolso uma cruz com a imagem de
Jesus Cristo, causando grande confusão no inferno.

\paragraph{``Peleja de Chico Raimundo e Zé Mulato''}

Típica disputa entre cantadores, modalidade de 
poesia oral muito praticada no Nordeste. No dia da
festa de São João, organizada por Seu Quincas, Chico Raimundo espera a
chegada de Zé Mulato, cantador de grande fama, que anda pelo sertão
destruindo a fama dos cantadores locais. A narrativa apresenta a
disputa entre ambos, e extrapola pela via do poema e da canção, aquilo
que poderia ser uma disputa em armas, já que se trata da defesa do
legado de Chico Raimundo, contra a chegada de um forasteiro.


\paragraph{``Combate e morte de `Lampião'\,''}

História do surgimento de Lampião, desde seu nascimento, no interior de
Pernambuco, até sua morte, na fazenda Angicos, no inteiro do Sergipe. O
autor narra as atrocidades de Lampião e a disputa pessoal que se cria
entre ele e o tenente Bezerra, seu captor e executor.

\paragraph{``O golpe do seu Gegê ou o choro dos deputados''}

Em 1937, o poema narra o momento em que Getúlio Vargas dissolve a câmara
dando o primeiro passo para a implantação do Estado Novo. Favorável a
Getúlio, o autor apresenta os deputados de vários estados e
principalmente os do Pará, seu estado de origem, como senhores
corruptos e pouco dedicados ao trabalho.

\paragraph{``Peleja de Armando Sales e Zé Américo''}

Retomando a forma de disputa entre cantadores, o autor apresenta uma
disputa entre os dois políticos que seriam candidatos à presidência no
ano de 1938, caso Getúlio Vargas não tivesse constituído o Estado Novo
e se mantido no poder. Sobre a disputa vale ressaltar que o autor opõe
a origem dos políticos, o primeiro membro da oligarquia paulista e o
segundo da oligarquia da Paraíba, assim como retoma a disputa entre
cantadores e homens no sertão. Um dado importante é que Sales seria o
candidato da oposição e Américo o da situação. 


\section{Sobre o gênero}

A poesia de cordel, dizem os especialistas, é uma poesia escrita para
ser lida, enquanto o repente ou o desafio é a poesia feita oralmente,
que mais tarde pode ser registrada por escrito. Essa divisão é muito
esquemática. Por exemplo, o cordel, mesmo sendo escrito e impresso para
ser lido, costumava ser lido em voz alta e desfrutado por outros
ouvintes além do leitor. A poesia popular, praticada principalmente no
Nordeste do Brasil, tem muita influência da linguagem oral, aproveita
muito da língua coloquial praticada nas ruas e na comunicação
cotidiana. 

Naturalmente, portanto, pode-se considerar a poesia narrativa do cordel
uma forma de poesia mais compartilhada e desfrutada coletivamente, o
que lhe dá também uma grande ressonância social. Muitos dos temas do
cordel são originários das tradições populares e eruditas da Europa
medieval e moderna. Encontramos temas retirados das novelas de
cavalaria medievais e das narrativas bíblicas. Como no caso de
``O azar, a cruz e o diabo. Divertida história do
homem mais azarado do mundo''. Ao lado destes temas
mais literários, encontram-se os temas locais, quase sempre narrados na
forma de crônicas de coisas realmente acontecidas, como em
``Peleja de Chico Raimundo e Zé
Mulato''. E as chamadas reportagens jornalísticas que,
no caso de Zé Vicente, misturam-se a poemas de exaltação política do
período varguista, caso de ``O Brasil rompeu com
`eles'\,'' e ``O golpe do seu Gegê ou O choro dos
deputados''. Também há as histórias fantásticas, que
se valem das tradições semirreligiosas, ligadas à experiência com o
mundo espiritual. 

Os grandes poemas de cordel são perfeitamente metrificados e rimados. A
métrica e a rima são recursos que favorecem a memorização e
tradicionalmente se costuma dizer que são resquícios de uma cultura
oral, na qual toda a tradição e sabedoria são sabidas de cor. 


\subsection{O sertão geográfico e cultural}

O sertão tem mitos culturais próprios. Contemporaneamente, o sertão
evoca principalmente o sofrimento resignado daqueles que padecem a
falta de chuva e de boas safras na lavoura. Evoca a experiência
histórica de uma região empobrecida, embora tenha sido geradora de
riquezas, como o cacau e a cana-de-açúcar, ambos bens muito valiosos. 

O sertão formou também o seu imaginário por meio de grandes
personalidades e uma pujante expressão artística. Além do cordel, o
sertão viu nascer ritmos tão importantes quanto o forró e o baião.
Produziu artistas tão expressivos quanto Luiz Gonzaga, grande cantor da
vida do sertanejo em canções como ``Asa
branca''. Um escultor como Mestre Vitalino criou toda
uma tradição de representação da vida e dos hábitos sertanejos em
miniaturas de barro. A gravura popular, que sempre acompanha os
folhetos de cordel, também floresceu em diversos pontos e ficou mais
famosa em Juazeiro do Norte, no Ceará, e em Caruaru, no estado de
Pernambuco. 

Dentre os grande mitos do sertão, está certamente o do cangaço com seu
líder histórico, mas também mítico, Virgulino Ferreira, o Lampião. Até
hoje as opiniões se dividem: para alguns foi um grande homem, para
outros um bandido impiedoso. Neste volume, o leitor vai encontrar 
um cordel dedicado ao mito do cangaço (``Combate e morte de `Lampião'\,''). 

Uma figura muito presente na cultura nordestina é o Padre Cícero Romão,
considerado beato pela Igreja Católica. Consta que teria feito milagres
e dedicado sua vida aos pobres. 

\subsection{Variação linguística}

A linguística moderna usa o termo
``idioleto'' para marcar grupos
distintos no interior de uma língua. Um idioleto pode ser a fala
peculiar de uma região, de um grupo étnico ou de uma dada profissão. 

Uma das grandes forças da poesia popular do Nordeste se origina em sua
forma muito própria de falar, com um ritmo muito diferente dos falares
do sul, e também muito diferentes entre si, pois percebe-se a diferença
entre os falares de um baiano, um cearense e um pernambucano, por
exemplo.

Além desse aspecto rítmico, quase sempre também há palavras peculiares a
certas regiões. 


\begin{bibliohedra}

\tit{DIEGUES JÚNIOR}, Daniel. \textit{Literatura popular em verso}. Estudos. Belo Horizonte: Itatiaia, 1986. 

\tit{MARCO}, Haurélio. \textit{Breve história da literatura de cordel}. São Paulo: Claridade, 2010.

\tit{TAVARES}, Braulio. \textit{Contando histórias em versos. Poesia e romanceiro popular 
no Brasil}. São Paulo: 34, 2005.

\titidem. \textit{Os martelos de trupizupe}. Natal: Edições Engenho de Arte, 2004


\end{bibliohedra}