\chapter{Prefácio\footnote{A
  organização dessa publicação se insere na minha pesquisa de
  pós"-doutorado, \emph{A literatura infantil russa e brasileira: uma
  análise comparada} (1919--1943), que desenvolvo no Departamento
  de Teoria Literária e Literatura Comparada da Universidade de São
  Paulo, com apoio da \textsc{fapesp} (processo 2017/24139-9).}} \label{part0}

\begingroup\small
\begin{flushright}
\hfill\textsc{daniela mountian}
\end{flushright}


\noindent{}A presente coletânea reúne quinze contos e doze autores representativos da
literatura russa juvenil, ainda pouco publicada e conhecida
no Brasil.

O livro percorre um período de quase 150 anos (o primeiro conto data de
1781 e o último de 1928) e não foi pensado para integrar todos os
autores consagrados que construíram essa história, mas organizado para
oferecer uma amostragem de estéticas e de questões expressivas que se
assinalaram ao longo desse século e meio.

Não se pode determinar quando uma literatura nasceu, mas se pode pensar
em quando ela começou a ser formalizada e se tornou objeto de reflexão.
Em sua expressão espontânea e oral, histórias russas para crianças
surgiram muitos séculos atrás com \emph{skázki} (contos maravilhosos
populares), \emph{bylíny} (narrativas épicas), provérbios,
canções, adivinhas, anedotas.

Antes de traduções e livros literários, o que se publicava para os
pequenos russos eram alfabetos, cartilhas, manuais de boas maneiras e
enciclopédias, e o caráter moralizante e religioso desses materiais é
inegável. O \emph{Abecedário} (1574), do tipógrafo Ivan Fiódorov
(c.1520--1583), nascido no Grande Principado de Moscou, é a mais antiga
edição infantil russa de que se tem conhecimento e trazia preces aos
educandos e recomendações de castigos aos educadores. Já o primeiro guia
de boas maneiras, \emph{Livro de correção,} foi produzido no século
seguinte, em 1683, pelo tipógrafo, professor de grego antigo e poeta
Karion Istómin (fins de 1640--1717), como um presente de aniversário
pelos onze anos do jovem Pedro: em forma de diálogo, este ``livro
ensinou ao futuro Pedro \textsc{i} a se portar em casa, na corte e na igreja'',
como assinalou Ben Hellman, pesquisador finlandês que fez uma competente
sistematização das letras russas infantojuvenis do séc. \textsc{xvi} ao
\textsc{xxi}.\footnote{\textsc{hellman}, Ben. \emph{Contos maravilhosos e histórias reais:
  a história da literatura russa infantil} (\emph{Skazka i byl: Istoria
  rússkoi diétskoi literatúry}). Moscou: \emph{Nóvoie
  literatúrnoie obozriénie,} 2016, p.\,9.}

A mando do próprio Pedro, que ocidentalizou a corte e obrigou todos os
cortesãos a aprender a ler e a escrever, foi produzido um guia de decoro
para jovens que se tornou referência por décadas: \emph{O espelho
honesto da juventude ou Indicações de como ter {[}boas{]} maneiras na
vida} (vários autores, 1717). Baseado em modelos europeus, o guia
continha uma cartilha e parábolas bíblicas e, na segunda parte,
orientações gerais ao jovem nobre: como se comportar privadamente,
diante dos pais e dos servos, e na vida pública. À mesa, por exemplo,
havia recomendações do tipo: ``Não se empanturre feito um porco!'',
``Não limpe os dentes com a faca (\ldots{})''. O manual, que também
trazia conselhos para meninas (inclusive de higiene), foi republicado
até 1767 e vinha em formato pequeno para que pudesse ser facilmente
carregado no bolso.

De forma mais sistemática, o começo da veiculação de livros infantis de
ficção pode ser datado da segunda metade do século \textsc{xviii}, época do
reinado de Catarina \textsc{ii} (1729--1796), embora, em meio a uma aristocracia
que se recusava a falar russo e a uma população não alfabetizada, a
maior parte das publicações literárias ainda fosse composta por títulos
estrangeiros. Em todo caso, nesse momento, como em outras áreas do
conhecimento, deram"-se passos importantes para a consolidação da
literatura russa infantojuvenil, ocorrida a partir da segunda década do
século \textsc{xix}.

Os primeiros contos russos para a infância saíram da pena de ninguém menos
que da própria Catarina, imperatriz da Rússia de 1762 a
1796. Nada mais justo e historicamente curioso, portanto, iniciarmos
nossa seleta com ela.

``Conto do tsarévitche Cloro'' (1781) e ``Conto dо
tsarévitche Fеvei'' (1783), escritos por ela, ``converteram"-se
nas primeiras obras literárias em prosa para crianças publicadas na
língua russa''. (\textsc{hellman}, p.\,15) Os textos, de caráter edificante, foram
feitos para seus netos, Alexandre, futuro imperador, e Constantino. Sem
viés nacionalista, as parábolas ressaltam valores universais: justiça,
honestidade, bondade.

Introduzindo a coletânea, o ``Conto do tsarévitche
Cloro'', um dos textos mais populares da imperatriz,\footnote{\textsc{kravtchenko},
  O. A. ``Conto do tsarévitche Cloro'' е o desenvolvimento do sistema
  alegórico e figurado na ode ``Felícia'' de G. R. Derjávin
  (``\emph{Skazka o tsariévitche Khlore'' i púti razvitia eio
  óbrazno"-allegorítcheskogo stroia v ode G. R. Derjávina ``Felitsa}'').
  \emph{Imalologuia i komparativistka}, 2015, nº1, pp.\,91--104.} se passa
no reino de um tsar bondoso e justo que tinha um filho, o tsarévitche
Cloro, dotado de beleza admirável. Um dia Cloro foi deixado aos cuidados
de sete babás, enquanto os pais saíam em defesa das fronteiras do reino.
A beleza e a inteligência do menino chegaram ao conhecimento de um cã
quirguiz, que por isso desejou conhecê"-lo e, tendo sido impedido pelas
babás de chegar até ele, atraiu o curioso Cloro, levando"-o para seu
acampamento. Para libertá"-lo, o cã exigiu, em troca, que o tsarévitche
encontrasse ``uma rosa sem espinhos, que não picasse''. No caminho cheio
de tentações à procura da flor, o menino recebeu ajuda de Juízo, neto do
cã. A parábola reflete o pensamento iluminista da tsarina, cuja
correspondência com Voltaire tornou"-se célebre. A imperatriz pregava o
aprimoramento pessoal por meio do conhecimento do mundo, que deveria ser
adquirido pela leitura desde a infância. Ela mesma organizou uma
cartilha (\emph{Cartilha elementar com estudos cívicos}, 1781) e se
aventurou por vários gêneros como escritora: comédia, dramas históricos,
operetas. Nada disso, no entanto, impediu que ela governasse a Rússia
com mãos de ferro.

Uma figura emblemática da época de Catarina foi о editor maçom Nikolai
Nоvikóv (1744--1818), que, entre outras empreitadas, publicou mais de
quarenta livros para crianças e foi criador da revista \emph{Leitura
infantil para o coração e a razão} (1785--1789), um suplemento gratuito
do jornal \emph{Notícias de Moscou} (1756--1917): ``a revista de Novikóv
não foi apenas a primeira do gênero da Rússia, como uma das primeiras da
Europa. Exatamente nesse momento, dos anos sessenta a noventa do séc.
\textsc{xviii}, apareceram na literatura europeia escritores que passaram a ser
chamados infantis''.\footnote{\textsc{golovin}, Valentin. Primeira experiência de
  edição de uma revista para jovens na Rússia (\emph{Piérvyi opit
  izdania jurnala dlia iúnochestva v Rossia}). \emph{Konstruíruia
  diétskoie: filologuia, istoria, antropologuia.} Moscou, São
  Petersburgo: \emph{Azimut,} 2011, p.\,11.} Para Novikóv, аs crianças
russas deveriam praticar sua língua natal (os nobres não raro dominavam
mais o francês do que o russo) e sua revista, com algumas propostas
progressistas, conquistou, depois, elogios do maior crítico
russo do século \textsc{xix}, Vissarion Belínski (1811--1848), que muito se
interessou pela questão da educação e da literatura infantil.

Ainda no fim do século \textsc{xviii} surgiu aquele que foi o maior fabulista
russo, Ivan Krylóv (1769--1844), que escreveu e adaptou muitas fábulas
de Esopo e de La Fontaine. No Brasil, histórias de Krylóv, escritas em
forma poética, foram adaptadas para o português por Tatiana Belinky
(1919--2013), que, por sinal, foi quem mais se dedicou à tradução de
textos russos infantis e juvenis aqui.

Foi no romantismo que a literatura infantil passou a despertar maior
atenção por parte dos escritores russos e apareceram os primeiros que se
dedicaram quase que exclusivamente às crianças (Anna Zontag, Vladímir
Lvóv, Piotr Furman, entre outros). A quantidade de publicações infantojuvenis aumentou
--- еm meados de 1830, o número de textos escritos em russo superou o de
traduções. (\textsc{hellman}, p.\,30)

Ganharam destaque histórias que exploram a fantasia --- que aparece em
contraste com a realidade ---, de maneira que o leitor saiba em qual dos
dois mundos está. Dois textos em prosa desse período se tornaram marcos,
a novela \emph{A galinha preta ou os habitantes do subterrâneo} (1829),
de Antóni Pogoriélski (1787--1836), tio do futuro escritor Aleksei
Tolstói; e ``A cidadezinha na tabaqueira'' (1834), parte
da nossa seleta, de Vladímir Odóievski (1802--1869), escritor e
musicólogo que cunhou a expressão, hoje repetida aos quatro ventos, ``o
sol da poesia russa'', referindo"-se ao mais admirado poeta do país,
Aleksándr Púchkin.

No conto de Odóievski, que faz referência a \textsc{e.\,t.\,a.} Hoffman, o menino
Micha inicia uma jornada pelo interior de uma caixinha de música, onde
ele compreende o mecanismo de funcionamento de sinos, martelos, ganchos
e cilindros, que ganham vida e, em última análise, se tornam uma
alegoria da sociedade, como notou depois Belínski, admirado. (\textsc{hellman}, p.\,34)

O romantismo também trouxe o interesse pelo folclore, uma das respostas
ao nacionalismo que aflorou na época. Surgiram estudos e coletâneas de
textos populares, que eram anotados e reunidos. Ganhou notoriedade a
coletânea \emph{Contos populares russos,} cuja primeira edição, com mais
de 600 textos, saiu em oito volumes (1855--1863), com organização de
Aleksándr Afanássiev (1826--1871). Em 1870 Afanássiev lançou uma edição
especial para crianças: \emph{Contos populares russos
infantis.}\footnote{\textsc{carolinski}, Flavia Cristina Moino. \emph{Aleksandr
  Nikoláevitch Afanássiev e o conto popular russo}. Dissertação
  (Mestrado em Letras) --- Universidade de São Paulo, 2008, p.\,61.}

Essas histórias coletadas ganharam refinamento literário nos poemas
folclóricos de Aleksándr Púchkin (1799--1837), Vassíli Jukóvski
(1783--1852) e Piótr Erschóv (1815--1869): ``A escolha da forma
poética podia parecer estranha, pois não havia nada parecido nem na
Rússia nem no estrangeiro. Os escritores esforçavam"-se por representar o
folclore em uma maneira mais elegante''. (\textsc{hellman}, p.\,35) A estrutura
narrativa dos contos populares também foi explorada pela prosa russa, a
começar por Nikolai Gógol (1809--1852).

A fase de grandes obras do realismo russo, consagrada mundialmente por
escritores como Ivan Turguêniev (1818--1883), Fiódor Dostoiévski
(1821--1881) e Lev Tolstói (1828--1910), foi, em geral, mais matizada
por questões sociais e humanitárias, o que também se refletiu nos contos
para a infância. E não poderia ser diferente: as reformas institucionais
dos anos 1860, realizadas no governo de Alexandre \textsc{ii}, foram sentidas por
todos os setores da sociedade.

É preciso destacar, nesse período, a atuação de importantes críticos
russos, como Nikolai Dobroliúbov (1836--1861) e Nikolai Tchernychévski
(1828--1889). Ambos produziram crítica de livros infantis nas pegadas do
já referido Vissarion Belínski, que, nos anos 1840, estabeleceu os
critérios artísticos da assim chamada Escola Natural, de viés social.

Belínski, que escreveu ``cerca de 200 trabalhos dedicados à literatura
infantil'',\footnote{\textsc{sigóv}, Vladímir (org.). \emph{Literatura infantil}
  (\emph{Diétskaia literatura}). Moscou: Iurait, 2019, p.\,64.} achava
que, nas obras para os pequenos, a realidade deveria ser descrita tal
como é, sem falseamentos, o que condizia com o que pensava sobre
literatura, em geral, nessa década. A demanda mais utilitária para ele
não significava, entretanto, desprezo à forma: era peremptório ao
afirmar que obras para jovens leitores tinham que ser elaboradas com
arte, considerando fundamental que eles aprimorassem o senso estético.
Para este opositor do absolutismo e do regime de servidão, os livros
infantis tinham a função de preparar as crianças para se tornarem
cidadãos conscientes, refinados esteticamente e munidos de ideais
humanitários, e, após os 12 anos, elas já deveriam ler textos literários
para adultos (o conceito de literatura juvenil é algo mais contemporâneo,
assim como o de adolescência; a idade adulta começava bem mais cedo).
Belínski criticava duramente as ``obras sentimentais de cunho
moralizante'' (\textsc{sigóv}, p.\,66) e recomendava, para as crianças, contos
populares russos, textos de Odóievski е as fábulas de Krylóv e, para os
mais crescidos, livros de história e de ciências naturais e obras de
Púchkin, Jukóvski, Lêrmontov e Gógol e, entre os estrangeiros,
Cervantes, Swift, Alexandre Dumas e, principalmente, Walter Scott e
Fenimore Cooper.

Aprofundando o lado político"-social de Belínski, alguns anos depois
Tchernychévski, autor da clássica obra panfletária \emph{Que fazer?}
(1863), assegurava que a literatura infantil deveria, além de ser
escrita de forma breve, clara e com ritmo ligeiro, ensinar a vida real
para as crianças, às quais também recomendava leituras adultas.
Considerava os jovens capazes de apreender a realidade sem floreados e
de participar ativamente dela. Por essa razão, livros juvenis de
história e de ciências naturais, com informações sobre natureza, plantas
e animais, eram recebidos por ele com entusiasmo, assim como por
Dobroliúbov.

Com a abolição da servidão em 19 de fevereiro (3 de março no calendário
gregoriano) de 1861 e milhões de camponeses recém"-libertos, as
\emph{escolas} \emph{populares}, com professores mais preparados,
receberam o impulso necessário para se desenvolverem: ``Entre a segunda
metade do século \textsc{xix} e o início do \textsc{xx}, surge no império russo a
necessidade de organizar um novo sistema educacional popular e de
conduzir uma educação comum''.\footnote{\textsc{vólik}, Elena. Escolas populares
  na segunda metade do século \textsc{xix} e início do \textsc{xx} na Rússia:
  particularidades na organização do processo de educação escolar
  (\emph{Naródnye chkóly vtorói polovíny \textsc{xix} --- natchala \textsc{xx} stoletia v
  Rossii: osóbnnosti organizátsii utchebno"-vospitátelnogo protsessa}).
  \emph{Viéstnik Tcheliábinskogo gosudárstvennogo pedagoguítcheskogo
  universitieta,} 2017, nº 5, p.\,35.} O Regulamento das Escolas Populares
Elementares foi estabelecido pelo Ministério da Educação Popular em
1864. Novos compêndios escolares foram criados e dois nomes tiveram
relevância nesse processo: Konstantin Uchínski (1823--1870) e Lev
Tolstói (1828--1910).

O pedagogo Uchínski, valorizando a cultura popular, a língua russa, a
religião e um aprendizado menos formal e mais pautado na experiência da
criança, alterou o curso da pedagogia infantil russa. Seus manuais
\emph{Mundo infantil} (1861) e \emph{Palavra nativa} (1864--69) foram
inúmeras vezes reeditados. Embora desse particular importância a
conteúdos ilustrativos da natureza, Uchínski foi o primeiro a incluir
textos de literatura (Derjávin, Púchkin, Jukóvski, Turguêniev, etc.)
como parte do aprendizado. (\textsc{sigóv}, p.\,118)

Muito engajado nas questões de seu tempo, Lev Tolstói se dedicou à
educação com especial afinco e praticamente a vida toda --- sobre o tema
escreveu ``nada menos que 629 trabalhos, muitos deles dedicados à
metodologia pedagógica e às recomendações aos professores, sem contar as
numerosas correspondências''.\footnote{\textsc{bernardini}, Aurora Fornoni. As
  cartilhas do Conde Lev Nikoláievitch Tolstói. \emph{Aulas de
  literatura russa}. São Paulo, Kalinka, 2018, p.\,159.} O conde
tinha grande interesse pela cultura popular e pesquisava ``quais livros
o povo lia e estimava''. (\textsc{sigóv}, p.\,134) Fundou, em 1859, uma escola
para camponeses em sua propriedade, Iásnaia Poliana, onde ``não
havia horário, nem punição, nem nota. Não havia dever de casa e exames,
e os estudantes tinham permissão para entrar na classe e sair dela a
qualquer momento''. (\textsc{hellman}, p.\,86) Tolstói acreditava em um processo
de aprendizado recíproco, em que o aluno aprende tanto quanto o
professor.

Ele criou em 1862 a revista mensal sobre pedagogia e educação
\emph{Iásnaia Poliana,} apreciada por Tchernychévski, além de ter
organizado a \emph{Cartilha} (1871--1872), que depois se desmembrou na
\emph{Nova cartilha} (1875) e em quatro \emph{Livros russos para
leitura} (1875--1885). А \emph{Cartilha} era ``um manual enorme
de 756 páginas dividido em 4 livros, cada um, por sua vez, subdividido
respectivamente em 4, 3, 3 e 3 partes'' (\textsc{bernardini}, p.\,160) e continha
o alfabeto cirílico, exercícios silábicos, provérbios, adivinhas, textos
de leitura, e notas aos professores. No preparo desse empreendimento
nada modesto, Tolstói estudou fábulas e contos populares alemães,
árabes, gregos, indianos e judeus para adaptá"-los às crianças
camponesas. Suas ideias democráticas e multiculturais foram tão
avançadas para a época que alguns pedagogos o censuraram ``pela
simplicidade e pelo caráter figurado da linguagem''. (\textsc{sigóv}, p.\,134)
Isso levou Tolstói a rever o material, criando a \emph{Nova cartilha,}
que, com um método progressivo de aprendizado, passou finalmente a ser
adotada pelas escolas populares, sendo reeditada 28 vezes até 1917,
``com uma tiragem geral que se aproximava de dois milhões de
exemplares''. (\textsc{hellman}, p.\,88) ``O trabalho sobre a língua é terrível. É
preciso que tudo seja belo, lacônico, simples e principalmente claro'',
disse Tolstói ao criar a \emph{Cartilha.} (\textsc{sigóv}, p.\,169)

Para suas cartilhas, o autor de \emph{Guerra e paz} escreveu ou adaptou
à sua maneira --- com sua visão moral e religiosa (que incluía preceitos
como praticar a não violência e perdoar ao inimigo) --- contos que
refletiam a realidade ou traziam conhecimentos sobre natureza (escritos
com uma linguagem concisa e procedimentos literários argutos) e fábulas
esopianas com lições de moral baseadas na vivência da criança.

O autor também escreveu textos especialmente para jovens que se tornaram
clássicos, como o aqui incluído ``О prisioneiro do Cáucaso'' (1872),
publicado na revista \emph{Aurora} e no quarto \emph{Livro russo para
leitura.} A história, tratando do embate clássico de russos e
tchetchenos e de colonizadores e colonizados, tornou"-se muito conhecida
na Rússia, ganhando duas adaptações para cinema --- não é para menos, o
texto é cinematográfico. Muito estimado pelo escritor e baseado em
sua própria experiência no serviço militar, o conto narra a história de
Ivan Jílin, um fidalgo russo que servia na Guerra do Cáucaso e foi
capturado por tártaros montanheses ao tentar regressar para casa, depois
de ter recebido uma carta da velha mãe doente pedindo para vê"-lo e para
conhecer a noiva que lhe arranjara --- a carta da mãe contém uma citação
bem"-humorada a ``Ivan Fiódorovitch Chponka e sua titia''
(1832), conto de Nikolai Gógol (1809--1852) sobre um oficial que volta para casa a fim de rever a velha tia, a qual tenta lhe
empurrar uma noiva, mas a ligação mais direta da história de Tolstói é
com um ciclo de poemas (1822) de Púchkin, também chamado ``O prisioneiro
do Cáucaso'', que é ainda o título de um poema longo (1828) de
Mikhail Lêrmontov (1814--1841). Essas relações não surpreendem: um dos
traços da literatura russa é esse jogo de amarelinha de citações e
homenagens contínuas (via paródia ou estilização), normalmente
culminando em Púchkin ou Gógol.

Com ritmo contagiante e temas universais (traição, amizade, esperança,
audácia, etc.), o narrador de Tolstói, que adota o ponto de vista
russo, mantém o colorido local ao descrever os tártaros, sem cair na
caricatura simplória, usando da técnica contrastiva e de uma linguagem
saborosa e ao mesmo tempo simples, sem excessos. O escritor considerava ``О
prisioneiro do Cáucaso'' um conjunto exemplar de procedimentos
literários que ele seguiria dali em diante para se comunicar com o
grande público.

Dos autores russos do séc. \textsc{xix} conhecidos do leitor brasileiro, foi
Tolstói quem mais se dedicou à literatura infantojuvenil, mas outros
nomes consagrados também se interessaram por ela. Ivan Turguêniev (1818--1883) gostava muito de crianças e chegou a planejar uma antologia de
contos infantis, que infelizmente não realizou, mas seu ``Mumu''
(1854), escolhido para esta coletânea, embora não tenha sido
escrito unicamente para jovens, é lido nas escolas russas e se tornou
uma das histórias mais conhecidas do país que trata da relação entre
pessoas e animais. Baseado num acontecimento real, o conto retrata o duro
destino da doce Mumu, a \emph{spaniel} de Guerássim, um brutamontes
surdo"-mudo, de caráter duro, mas inocente, que todos temiam e, de algum
modo, respeitavam. Com uma força descomunal, ele trabalhava como caseiro
em Moscou para uma fidalga velha e rabugenta que se tornou responsável
pelos dois infortúnios que marcaram a vida do criado.

Turguêniev, como Tchernychévski, lamentava a falta de bons livros russos
para os pequenos. Em Paris, ele se encantou com uma obra da escritora
Sofia Butkiévith (início de 1830--depois de 1880), \emph{Diário de
menina} (1862), em que objetos ganham vida e contam sua própria
história, um livro que informa ``sem pedantismo'', como Turguêniev
observa no pequeno prefácio que fez para ela, no qual acentua a
dificuldade de se escrever para crianças e a necessidade de
``aperfeiçoamento moral e social'' de quem escreve. Destaca ainda a
urgência de produção nacional: ``crianças russas precisam de livros
russos''.\footnote{\textsc{turguêniev}, Ivan. Prefácio para \emph{Diário de menina}
  de S. Butkiévitch (\emph{Predislovie k “Dnevniku diévotchki” S. Butkiévitch}). \emph{Pólnoie sobránie sotchiniénii i pissem v 30 tomákh.} Moscou,
  \emph{Naúka,} 1982.}

Na segunda metade do século \textsc{xix}, a escrita de denúncia, que busca
desvelar injustiças e desequilíbrios sociais, com desfechos não raro
infelizes, ganhou mais destaque, definindo uma vertente importante da
literatura juvenil russa. São dela representantes, entre outros,
Vladímir Korolenko (1853--1921), Pável Zassodímski (1843--1912), Dmítri
Mámin"-Sibiriák (1852--1912) e o poeta Nikolai Nekrássov (1821--1877),
que usou muito da cultura camponesa em seus poemas.

Nikolai Leskóv (1831--1895), que dava particular atenção a crianças desamparadas,
também tinha a vida camponesa muitas vezes como tema e utilizava
elementos narrativos do conto popular. Parte deste volume, ``Bobinho''
(\emph{Duratchók}), publicado pela primeira vez em 1891 na revista
infantil \emph{Brinquedinho}, faz parte de uma série de textos que
Leskóv escreveu aos pequenos. Pachka, personagem principal, é uma
espécie de \emph{iuródivyi,} que no dicionário pode ser definido como
``bobo'' (\emph{durák}), ``mendigo alienado'', ``vidente''. Figura
representativa da cultura russa, o \emph{iuródivyi,} misto de louco e
santo, é um devoto, um eremita que abre mão de riquezas terrenas e sai
vagando pelo mundo. А essa imagem se reúne uma personagem típica dos
contos populares maravilhosos, o Ivan"-\emph{durák}
(Ivanuchka"-\emph{duratchók}, diminutivo), que é normalmente o mais novo e mais fraco
de três irmãos que consegue passar por todas as provações e casar com a
filha do tsar. Leskóv se interessava pela religiosidade que nascia
espontaneamente do povo, mas não pela ``burocracia ortodoxa''; nesse
sentido, o conto ``Bobinho'' é emblemático: ``Escreveu uma série de
contos desse gênero, cujo personagem central é o justo, raramente um
asceta, em geral um homem simples e ativo, que se transforma em santo
com a maior naturalidade''.\footnote{\textsc{benjamin}, Walter. O narrador.
  Considerações sobre a obra de Nikolai Leskóv. \emph{Walter Benjamin:
  magia e técnica, arte e política}. São Paulo: Brasiliense, 3ª ed.,
  1985, p.\,200.}

Desvendar o máximo da vida com o mínimo de sentimentalismo foi a missão
que se colocou o escritor e dramaturgo Аnton Tchékhov (1860--1904), que
também era médico e mudou o curso do teatro e da prosa curta mundial com
procedimentos literários que se tornaram sinônimo de modernidade. As
tramas tchekhovianas se revelam quando nada parece ocorrer, o que
importa está em detalhes quase imperceptíveis ou no que foi deixado de
dizer. Em seus contos, muitas vezes historietas sobre existências
comuns, ele não busca resolver os conflitos, ``detesta colocar os pingos
no is'',\footnote{\textsc{vássina}, Elena. Anton Pavlovitch Tchekhov. Revista
  \emph{Cult.} Disponível em:
  \emph{https:/revistacult.uol.com.br/home/anton"-pavlovitch"-tchekhov/}.}
nas palavras de Elena Vássina, e deixa este trabalho para o leitor.

Tchékhov não fazia diferença entre textos para adultos e para crianças,
mas para elas selecionou, entre seus escritos, \emph{Kachtanka} e
\emph{Belolóby}. Outras obras do escritor entraram no repertório
infantojuvenil russo, como os dois contos escolhidos para nossa
coletânea (mais tocantes do que se espera do autor, mas, como muitas
obras suas, descrevem situações que não podem ser resolvidas
individualmente): ``Vanka'' (1886) e ``O fugitivo''
(1887), traduzidos para várias línguas e muito apreciados por Lev
Tolstói, que do segundo teria escrito: ``Como é bom ler isso! Volta e
meia, numa parte tocante ou engraçada, eu me emociono''.

Há quem defenda a ideia de que \emph{Vanka,} que retrata a noite de Natal
de um órfão miserável que apanha dos donos da casa onde vive, reflete a
própria infância do escritor, que veio de uma família humilde e sofrera
maus"-tratos do pai. O conto ``antinataliano'', com marcas de Andersen e
Dickens, foi publicado no dia 25 de dezembro de 1886 no suplemento
``Contos de Natal'' do \emph{Jornal de Petersburgo}. No mesmo jornal
saiu \emph{O fugitivo} (1887), história sobre um menino de sete
anos que adoecera e, ao ser internado, é rodeado de vidas se desfazendo.
Conforme testemunho do irmão do autor, foi baseada em fatos ocorridos no
hospital Tchikínski na época em que Tchékhov, ainda um jovem médico, lá
servia.

Lídia Avílova (1864--1943), que aparece aqui com o conto ``Primeira
mágoa'' (1906), é quase automaticamente relacionada a Anton Tchékhov, a
quem ela conheceu em 1889 e sobre quem escreveu uma biografia (\emph{A.
P. Tchékhov na minha vida}). A escritora costumava mandar seus
manuscritos para que Tchékhov, seu amigo, desse conselhos e pareceres e,
por vezes, marcas do estilo dele transparecem em suas obras. Em
``Primeira mágoa'', o menino Gricha, com a prisão do
cocheiro e amigo Ignát, se vê diante de seu primeiro dilema ético. O conto foi
elogiado por Lev Tolstói, que, à frente de vários projetos editorias, o
integrou a uma coletânea infantil.

Aleksándr Kuprin (1870--1938) conheceu Tchékhov na mesma época de
Avílova e o tinha como mestre. Numa carta de 1901,\footnote{\textls[-20]{\textsc{grómov},
  M.P., \emph{et al}. \emph{Correspondência de A. P. Tchékhov e A. I. Kuprin} (\emph{Peripiska A. P. Tchékhova i A. I. Kupriná}). Moscou:
  \emph{Khudójestvennaia literatura}, 1984.}} Kuprin fala de três peças
dele a que assistiu (\emph{A gaivota, Tio Vânia, Três irmãs}) e, no fim,
pede permissão para dedicar"-lhe uma coletânea de contos. Kuprin era um
bom prosador, muitas de suas narrativas descrevem com realismo a vida de
desvalidos e corajosos, elementos que não desaparecem dos contos que
escreveu para jovens e crianças, até hoje lidos e reeditados na Rússia,
como ``O poodle branco'' (1904) e ``O elefante'' (1907).

``O poodle branco'' narra as desventuras de uma trupe itinerante:
um velho tocador de realejo fora de moda, um jovem órfão acrobata, de
doze anos, e um poodle chamado Artô. Nas andanças pelas montanhas da
Crimeia, os saltimbancos vão de \emph{datcha} em \emph{datcha} fazendo
apresentações para conseguir trocados. Algumas cenas do conto revisitam
claramente ``Mumu'', que dá uma perspectiva mais dramática ao
suposto final feliz da história de Kuprin.

Com notas mais leves, ``O elefante'', estudado em escolas, traz um sopro
de vida com a história de Nadiejda (``esperança'' em russo), uma menina
de seis anos que adoece de tristeza e sonha conhecer um elefante. Em
desespero, seu pai vai até o circo e providencia que um deles visite a
filha. O tema do circo e as descrições vivas do elefante Tommy não
surpreendem --- Kuprin, que voava de balão e mergulhava com o
escafandro, era admirador de circos; ele mesmo foi ator, tinha amigos
lutadores, domadores e cantores e dizem que chegou a entrar numa jaula
de leões e por pouco não foi atacado.

O fim do século \textsc{xix} trouxe à arte russa o movimento simbolista e o
início do que se convencionou chamar ``Era de Prata'', que abrangeu um
período cronologicamente curto, mas que concentrou um número nada
desprezível de criações surpreendentes de áreas e tendências artísticas
diversas. Não à toa Nikolai Berdiáiev chamou essa época de ``Renascença
russa'', ``um momento que reuniu o simbolismo, o acmeísmo, futurismos e
um sem"-número de correntes e artistas de várias áreas que cobriram a
Rússia de manifestos, conceitos, obras e utopias arrojadas por cerca de
três décadas''.\footnote{\textsc{mountian}, Daniela. A pedra viva de Mandelstam.
  \emph{Revista Rosa,} série 2, nº1, 25 de maio de 2020. Disponível em:
  \emph{http:/revistarosa.com/1/a"-pedra"-viva"-de"-mandelstam}.}

Não poucos simbolistas, vanguardistas е modernistas em geral escreveram
textos em prosa e em versos direcionados para jovens e crianças. Algumas
coletâneas poéticas infantis reuniram poetas de envergadura: \emph{A
estrela da manhã} (1912), por exemplo, trazia, entre outros,
Aleksándr Blok, Ivan Búnin e Fiódor Sologub. (\textsc{hellman}, p.\,215) A revista
infantil \emph{Vereda} (\emph{Tropinka,} 1905--1912), sob direção de
Natália Manasseina e Poliksiena Soloviova, a Allegro, irmã do filósofo simbolista Vladímir Solovióv,
também reuniu muitos artistas importantes, principalmente simbolistas:
Aleksándr Blok, Aleksándr Kuprin, Aleksei Riémizov, Aleksei Tolstói,
Andrei Biéli, Konstantin Balmont, Kornei Tchukóvski, Maria Morávskaia,
Sacha Tchórny, Zinaída Guíppius, etc.

Devido ao interesse simbolista pelo mito e pelo folclore presente na
virada do século \textsc{xix} para o \textsc{xx}, houve nessa época uma busca pela
\emph{skazka} (conto maravilhoso popular). Fiódor Sologub (1863--1927),
expoente da velha geração de simbolistas de São Petersburgo, escreveu
contos maravilhosos para crianças pequenas, mas para a coletânea foi
selecionado o ``Conto da filha do fabricante de caixões'' (1915),
que não foi escrito apenas para o público juvenil, mas usa a estrutura
da \emph{skazka} e traz personagens jovens (a criança é tema central da
poética de Sologub). Na história, Elnítski se vê apaixonado pela
encantadora Zoia, filha do fabricante de caixões da cidade, que, para
construir as lápides dos futuros clientes, pede que ela fique do lado
deles e tira as medidas a olho. O elemento fantástico se une, como de
hábito no autor, ao insólito e ao sombrio, em uma narrativa construída
por mundos sobrepostos (uma história dentro da outra).

Também no universo do fantástico está o conto ``A pedrinha vermelha''
(1912), de Sacha Tchórny (1875--1937), mas aqui entram em cena o
cômico e o lúdico. Maria Morávskaia (1889--1947) e Sacha Tchórny
escreveram poemas infantis com um humor que contraria a lógica
cotidiana, anunciando, de algum modo, as criações dos anos 1920, como
observou Hellman (p.\,226).

``A pedrinha vermelha'' foi um dos primeiros trabalhos infantis de Sacha
Tchórny, que na época já era conhecido como poeta satirista. O conto fez
parte da antologia \emph{Livrinho azul} (1912), de que participou também
Maksim Górki (1868--1936), figura muito influente, como escritor e
editor, na esfera infantil no início da União Soviética.

No conto de Tchórny, o pequeno Jórjik, por meio de uma pedrinha recebida
de uma velha na floresta, ouvia a conversa dos animais; o narrador traz
o olhar e a lógica adorável da criança e os leva para o plano da
linguagem.

Longe de círculos artísticos e dos elogios dos críticos se encontrava a
favorita dаs russinhаs do início do século \textsc{xx}: Lídia Tchárskaia
(1875--1937). Tchárskaia produzia febrilmente e, por anos, tudo o que
publicava virava \emph{best"-seller}. Muitas de suas heroínas românticas,
sentimentais e positivas, de caráter marcante (não raro, órfãs), movidas
por valores de moral elevada (bondade, amizade, generosidade), eram
popularíssimas na Rússia (a uma delas Marina Tsvetáieva até dedicou um
poema).

Os dois contos escolhidos para a antologia trazem protagonistas
femininas. Em ``A prova'' (1907), temos as aflições de uma menina
às vésperas de um exame de geografia. Já ``A mãe'' (1912) retrata
a vida de uma viúva pobre e solitária que vivia de pequenos trabalhos de
costura com três filhos pequenos para criar.

Depois da Revolução de 1917, seus livros, não considerados adequados à
nova realidade, foram retirados das bibliotecas. Nos primeiros anos do
governo bolchevique, Tchárskaia, com um pseudônimo (N. Ivanova),
conseguiu, de alguma maneira, continuar escrevendo, mas em meados da
década de 1920 essa possibilidade também lhe foi negada: ``Até sua morte
em 1937 Tchárskaia viveu doente e na penúria, recebendo uma pensão
miserável e às vezes ajuda secreta de leitores devotados''. (\textsc{hellman}, p.\,180)

No incipiente mundo soviético, surge uma nova forma de escrever e de
ilustrar livros para crianças, desde o início condizente com a nova
ideologia proletária, mas delineada --- enquanto o realismo socialista
ainda não havia se tornado estilo hegemônico --- por artistas não raro
dotados de linguagem arrojada.

Se, na literatura russa destinada aos adultos, experiências poéticas e
gráficas da vanguarda do início do século \textsc{xx} circulavam, em geral, em
produções caseiras e em grupos restritos, na literatura infantil eram
arregimentados escritores e artistas gráficos --- vindos de diversos
círculos que então proliferavam na Rússia --- para a produção de livros
e revistas, impressos em tiragens consideráveis. Parece um contrassenso,
mas, enquanto os órgãos soviéticos censuraram as criações para adultos
desses artistas, aos seus livros infantis da década de 1920 eram
reservados recursos públicos vultosos (em meados da década seguinte, o
panorama mudou bastante). Muitos autores e pintores vanguardistas e
modernistas passaram a viver à custa de publicações infantojuvenis,
escrevendo com certa liberdade: ``(\ldots) tornou"-se a única chance, para
uma série inteira de escritores e poetas, de ganhar dinheiro com a
literatura. (\ldots) Livros infantis eram editados em tiragens
suficientemente grandes e rendiam bons vencimentos, assim como as
revistas infantis pagavam honorários bem razoáveis''.\footnote{\textsc{kóbrinski},
  Aleksándr. \emph{Daniil Kharms. Vida de pessoas notáveis --- série de
  biografias} (\emph{Jizn zametchátelnykh liudéi --- seriia
  biográfii}). Moscou: \emph{Molodaia gvárdiia,} 2009, p.\,136.}

Este foi o caso de Daniil Kharms (1905--1942), fundador de um coletivo
de vanguarda chamado \textsc{oberiu} (1928). Escreveu poemas, peças de teatro e
miniaturas hilariantes que, depois de haverem sido censurados durante
anos, caíram no gosto de artistas e jovens alternativos de São
Petersburgo. Se sua produção voltada para o público adulto só teve
reconhecimento depois de 1990 --- décadas depois de sua morte trágica e
precoce ---, seus textos infantis, além de terem sido seu ganha"-pão,
eram adorados pelas crianças russas (e continuam sendo).

Ele colaborava nas revistas infantis \emph{Ouriço} (\emph{Ioj}) e
\emph{Pinstasilgo} (\emph{Tchij}) com poemas e contos que, por vezes,
também viravam livros. Foi o caso de \emph{Sobre como Kolka Pánkin
viajou para o Brasil e sobre como Pietka Erchóv não acreditou em nada}
(1928), uma viagem fantástica de duas crianças de Leningrado para o
Brasil. Nos textos infantis de Daniil Kharms, dos anos 1920, não há
espaço para didatismos, mas para o ilógico, o irracional, o
\emph{nonsense}; e impressionam as técnicas refinadas de humor.
Diferentemente do conto de Odóievski, na história de Kharms não se pode
discernir a realidade da fantasia. Os pequenos leitores são levados a um
mundo lúdico, em que reinam a imaginação, a liberdade e a brincadeira,
hoje elementos valorizados no desenvolvimento do jovem.

Daniil Kharms entrou no mundo das edições infantis por intermédio do
poeta e editor Samuel Marchak, que, ao lado de Kornei
Tchukóvski, foi responsável por uma renovação na poesia
russa infantil. Marchak e Tchukóvski, tão conhecidos na Rússia como
Monteiro Lobato é no Brasil, escreviam numa linguagem mais próxima dos
pequenos --- com humor, novo ritmo poético e modernização de
procedimentos de contos populares tradicionais --- e foram seguidos por
uma nova geração de autores para crianças à qual também pertenciam
Kharms e seus amigos vanguardistas.

Além disso, a criação da União Soviética (1922) trouxe novos temas,
novas personagens e novos cenários às histórias para a infância e a
juventude, assim como a queda do regime, por isso a literatura russa
infantojuvenil dos séculos \textsc{xx} e \textsc{xxi} poderá ser assunto para um próximo
volume\ldots{}

Os autores e estilos destacados neste prefácio e neste livro formam um
breve contexto da literatura russa infantojuvenil até o início do século
\textsc{xx}, mas estão longe de esgotar o tema. De um lado, dão uma pequena
contribuição para que esta história comece a ser contada no Brasil. De
outro lado, são textos bons de serem lidos, por todas as idades,
traduzidos do russo por Irineu Franco Perpetuo, Moissei Mountian e
Tatiana Larkina e ilustrados por Fido Nesti.

No fim das contas, como diria Tatiana Belinky, ``o que é bom para a
criança é bom para o adulto''.

\endgroup
%\emph{Daniela Mountian}
