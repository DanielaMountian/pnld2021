\port

\begin{astanza}
  \edtext{Musas, que com cantos glorificam,}{\lemma{Musas\ldots{}glorificam,}{\nota{a repetição fônica
    em ``Musas'' (\textit{mo\=usai}) e ``glorifica'' (\textit{kleiousai}) confere 
    destaque às deusas e à sua função.}}} da Piéria&
  \edtext{%
    para cá, narrai de Zeus, vosso pai, celebrando-o:&
    ele faz homens mortais igualmente soados e ignorados,&
    conhecidos e desconhecidos devido ao grande Zeus.
  }{\lemma{para\ldots{}Zeus}{\nota{as formas gregas de ``Zeus'' são \textit{Di’} e \textit{Dios}, com o que
          se estabelece um trocadilho (ou figura etimológica) com a preposição \textit{dia} no
          verso 3, \textit{lit.}, ``através de quem os homens
          são igualmente soados e ignorados''.}}}&
  Pois fácil fortifica, fácil ao forte limita,                \num{5}&
  fácil ao conspícuo diminui e ao inconspícuo amplifica,&
  fácil endireita o torto e ao arrogante seca\&
  \end{astanza}

 \begin{astanza}
  Zeus troveja-acima, que a morada mais alta habita.&
  Atende, vendo e ouvindo, e com justiça endireita sentenças&
  tu; já eu, a \edtext{Perses}{\nota{o nome do irmão de Hesíodo evoca o verbo
  grego \textit{perthein} (``pilhar, destruir''), usado no segundo verso da \textit{Odisseia} para
  caracterizar o responsável pela conquista de Troia, Odisseu.}} o que é
  genuíno poria num discurso.                \num{10}&
  Ei, uma só família de \edtext{Disputas}{\nota{Hesíodo parece corrigir o que
  afirmara na \textit{Teogonia}, v.~225--32, em que a prole de Disputa engloba Labor,
  Esquecimento, Fome, Aflições, Batalhas, Combates, Matanças, Carnificinas,
  Brigas, Embustes, Contos, Contendas, Má-Norma, Ruína e Juramento, elementos
  que pertencem à espinha dorsal de \textit{Trabalhos e dias}.}} não havia, mas na
  terra&
  há duas: a uma apreciaria quem a aprendesse,&
  e a outra é censurável; e têm ânimo bem distinto.&
  Pois esta a guerra vil e a discórdia amplia,&
  terrível; mortal algum dela gosta, mas, constrangidos,                \num{15}\&
\end{astanza}



\begin{astanza}
  devido a planos de imortais, honram a pesada Disputa.&
  Aquela outra gerou antes a escura Noite,&
  e a pôs o Cronida trono-no-alto, que mora no éter,&
  nas raízes da terra, e é bem melhor para os homens;&
  ela até mesmo o inapto instiga ao trabalho.                \num{20}&
  \edtext{%
    Alguém, precisando trabalhar, olhando para outro,&
    rico, esse almeja arar e plantar&
    e arrumar a fazenda; vizinho invejaria o vizinho&
    que fartura almeja: boa essa Disputa para os mortais.
  }{\lemma{Alguém\ldots{}mortais}{\nota{Minha tradução não segue a
        interpretação mais aceita, possivelmente: ``Alguém, carente de
  trabalho, olhando para outro,/ rico, que almeja arar e plantar/ e arrumar a
  fazenda, emularia, vizinho, ao vizinho,/ ao que fartura almeja''.}}}&
  Ceramista ressente ceramista, e carpinteiro, carpinteiro,                \num{25}&
  mendigo inveja mendigo, e cantor, cantor.&
  Tu, ó Perses, isso deposita em teu ânimo,&
  e a Disputa sádica não te afaste o ânimo do trabalho,&
  espreitando brigas da ágora como ouvinte.&
  Pouco se preocupa com brigas e assembleias                 \num{30}\&
\end{astanza}


\begin{astanza}
  quem dentro não encerra suficiente sustento&
  sazonal que a terra produz, o grão de Deméter.&
  Tendo-se disso fartado, brigas e discórdia ampliarias&
  mirando bens alheios; e outra vez não te será possível&
  agir assim. Vamos, agora nos seja a briga decidida                \num{35}&
  com retos juízos, os que, de Zeus, são os melhores.&
  Pois já dividimos o patrimônio, e muita outra coisa&
  tentavas tomar e levar, enaltecendo bastante os reis&
  come-presente, eles que querem esse juízo pronunciar.&
  Tolos, não sabem quão maior é a metade que o todo                \num{40}&
  nem quão \edtext{grande valia há na malva e no asfódelo.}{\nota{sentido
  crítico, apontando para formas precárias de alimentação mística (a pobreza em
  oposição ao modo de vida dos reis), parte de uma dieta ``divina'',
  compondo uma vida de privação com benefícios após a morte (ao modo
  órfico-pitagórico)?}}&
  Pois deuses ocultaram e seguram o sustento dos homens.&
  De outro modo, fácil trabalharias até um só dia&
  de sorte a teres bastante, mesmo \edtext{inativo,}{\nota{tradução 
             de \textit{aergos}, formado a partir do 
             radical de \textit{trabalho} (\textit{ergon}) 
             e \textit{trabalhar} (\textit{ergazomai}).}}
             até por um ano;&
  ligeiro irias o leme sobre \edtext{a fumaça}{\nota{a lareira.}} depositar,                \num{45}\&
\end{astanza}


\begin{astanza}
  e findaria o trabalho dos bois e das mulas robustas.&
  Mas Zeus o escondeu, com raiva em seu juízo,&
  pois enganou-o Prometeu plano-torto.&
  Por isso para os homens armou agruras funestas,&
  e ocultou o fogo; a esse, então, o bom filho de Jápeto                \num{50}&
  para os homens roubou do astucioso Zeus&
  em cavo funcho-gigante sem Zeus prazer-no-raio notar.&
  Com raiva, disse-lhe Zeus junta-nuvens:&
  ``Filho de Jápeto, supremo mestre de armações,&
  regozijas após o fogo roubar e meu juízo iludir,                \num{55}&
  para ti mesmo e varões vindouros, grande desgraça.&
  Eu a eles, pelo fogo, darei um mal com que todos&
  se deleitarão no ânimo, seu mal abraçando''.&
  Assim falou, e gargalhou o pai de varões e deuses.&
  A Hefesto ordenou, ao mui glorioso, bem rápido                \num{60}\&
\end{astanza}


\begin{astanza}
  terra com água molhar, inserir, humanas, voz &
  e força e assemelhá-la à visão de deusa imortal,&
  bela aparência atraente de uma jovem; e a Atena,&
  ensinar as tarefas, tecer trama mui artificiosa;&
  e à dourada Afrodite, graça verter em torno da cabeça,                \num{65}&
  anseio aflitivo e preocupações devora-membro;&
  inserir um espírito canino e um modo finório&
  a Hermes impôs, o condutor Matador-da-Serpente.&
  Assim falou, e obedeceram a Zeus, o senhor Cronida.&
  De pronto da terra modelou o mui glorioso \edtext{Duas-Curvas,}{\nota{Hefesto.}}                \num{70}&
  pelos planos do Cronida, algo parecido à moça respeitada;&
  cinturou e adornou-a a deusa, Atena olhos-de-coruja;&
  em volta dela, as deusas Graças e a senhora Persuasão&
  corrente dourada puseram na pele; coroaram-na&
  as Estações bela-coma com flores primaveris;                \num{75}\&
\end{astanza}


\begin{astanza}
  e todo o adorno ajustou a seu corpo Palas Atena.&
  Em seu peito, o condutor Matador-da-Serpente&
  mentiras, histórias solertes e um modo finório&
  arranjou pelo plano de Zeus grave-ressoo; nela voz&
  pôs o arauto dos deuses, e nomeou essa mulher                \num{80}&
  Pandora, porque todos que têm casas olímpias&
  deram-lhe um dom, desgraça dos varões come-grão.&
  Mas após o íngreme ardil indestrutível completarem,&
  a Epimeteu o pai enviou o glorioso Matador-da-Serpente,&
  veloz mensageiro dos deuses, com o dom. E Epimeteu                \num{85}&
  não refletiu no que lhe disse Prometeu, nunca um dom&
  aceitar de Zeus Olímpio, mas reenviar&
  de volta, para não vir a ser um mal aos mortais.&
  E ele o recebeu e, quando já tinha o mal, percebeu.&
  Pois antes sobre a terra as tribos de homens viviam \num{90}\&
\end{astanza}


\begin{astanza}
  afastadas de males e longe de duro labor&
  e aflitivas doenças, as que dão morte aos varões.&
  Pois rápido, na miséria, envelhecem os mortais.&
  Mas a mulher tirou à mão a grande tampa do cântaro&
  e espalhou; para os homens, agruras funestas armou.                \num{95}&
  Lá mesmo só Esperança, na casa inquebrável,&
  ficou, dentro do cântaro sob as bordas, e não porta&
  afora voou: deixou antes tombar a tampa do cântaro&
  pelo plano do porta-égide, Zeus junta-nuvens.&
  Outras mil coisas funestas vagam entre os homens,                \num{100}&
  pois plena está a terra de males, pleno, o mar:&
  doenças para os homens, umas de dia, outras de noite,&
  espontâneas, vagam, males aos homens levando&
  em silêncio, pois tirou a voz o astuto Zeus.&
  Assim, é impossível da ideia de Zeus escapar.                \num{105}\&
\end{astanza}


\begin{astanza}
  Se queres, o outro relato para ti vou esboçar&
  bem e destramente, e tu, lança-o em teu juízo&
  como de origem igual são deuses e homens mortais.&
  De ouro a primeiríssima linhagem de homens mortais&
  foi feita pelos imortais que têm casas olímpias.                \num{110}&
  Existiram na época de Crono, quando reinava no céu:&
  como deuses viviam com ânimo sem aflição,&
  afastados de labor, longe de agonia. Nem a infeliz&
  velhice havia, e, sempre iguais nos pés e mãos,&
  apraziam-se em festejos, além de todos os males;                \num{115}&
  morriam como por sono subjugados. Toda benesse&
  possuíam: o fruto, que produzia o solo dá-trigo,&
  espontâneo, era farto e sem inveja; de bom grado,&
  tranquilos gozavam dos \edtext{grãos}{%
  			\nota{\textit{erga}, fruto do trabalho.}}
  					 com muita benesse,&
  \edtext{ricos em ovelhas, caros aos deuses ditosos.}{\lemma{ricos\ldots{}ditosos.}{\nota{a maioria
  dos editores considera este verso espúrio por estar ausente em todos os
  manuscritos e ser citado apenas por um autor antigo.}}}
  \num{120}\&
\end{astanza}


\begin{astanza}
  Mas depois que a essa linhagem a terra encobriu,&
  eles são numes, pelos planos do grande Zeus,&
  nobres, terrestres, guardiões dos homens mortais:&
  \edtext{eles guardam juízos e feitos terríveis,&
  envoltos em neblina, vagando por toda a terra,}{\lemma{eles\ldots{}terra,}{\nota{alguns editores rejeitam estes versos.}}}                \num{125}&
  dadores de riqueza; e essa honraria real receberam.&
  Uma segunda linhagem, muito pior, depois,&
  de prata, fizeram os que têm casas olímpias,&
  à de ouro semelhante nem no físico nem na ideia.&
  Mas por cem anos os meninos, junto às mães devotadas,                \num{130}&
  eram criados, divertindo-se, grandes tolos, em sua casa;&
  mas ao tornar jovens e alcançar o pico da juventude,&
  pouquíssimo tempo viviam, com aflições&
  pela imprudência: violência iníqua eram incapazes&
  de conter, recíproca, nem servir aos imortais                \num{135}\&
\end{astanza}


\begin{astanza}
  queriam ou sacrificar nos sacros altares dos ditosos,&
  norma dos homens pelos costumes. Esses então&
  Zeus Cronida removeu, furioso, pois honras&
  não deram aos deuses ditosos que o Olimpo ocupam.&
  Mas após a terra também essa linhagem encobrir,                \num{140}&
  esses, subterrâneos, são chamados mortais ditosos,&
  os segundos, e mesmo assim, também a eles segue honra.&
  E o pai Zeus a terceira, outra linhagem de homens mortais,&
  de bronze, fez, em nada semelhante à de prata,&
  a partir de freixos, terrível e ponderosa: a eles os feitos            \num{145}&
  de Ares importavam, tristes, e atos violentos. Pão&
  não comiam, com ânimo juízo-forte de adamanto,&
  inabordáveis: grande força e mãos intocáveis &
  dos ombros nasceram sobre os membros robustos.&
  Deles eram brônzeas as armas, brônzeas as casas, \num{150}\&
\end{astanza}


\begin{astanza}
  e com bronze obravam: negro ferro não existia.&
  E eles, subjugados pelas mãos uns dos outros,&
  rumaram à casa bolorenta do Hades gelado,&
  anônimos: embora assombrosos, a eles a morte&
  negra agarrou, e largaram a fúlgida luz do sol. \num{155}&
  Mas depois que a terra também essa linhagem encobriu,&
  de novo ainda outra, a quarta, sobre a terra nutre-muitos&
  Zeus Cronida produziu, mais justa e melhor,&
  a divina linhagem de varões heróis, esses chamados&
  semideuses, a estirpe anterior sobre a terra infinda.\num{160}&
  E a eles guerra danosa e prélio terrível,&
  a uns sob Tebas sete-portões, na terra cadmeia,&
  destruiu, ao combaterem pelos rebanhos de Édipo,&
  a outros, nas naus, sobre o grande abismo do mar,&
  levando a Troia por conta de Helena bela-coma.                \num{165}\&
\end{astanza}


\begin{astanza}
  Lá em verdade a alguns o termo, a morte encobriu,&
  e a outros, longe dos homens, ofertou sustento e casa&
  o pai, Zeus Cronida, e os alocou nos limites da terra.\numero{[168]}& 
  E eles habitam com ânimo sereno  \numero{[170]}&
  nas ilhas dos venturosos junto a Oceano\advanceline{1} funda-corrente,&
  heróis afortunados, aos quais delicioso fruto,&
  que três vezes ao ano floresce, traz a gleba dá-trigo.&
  Não mais, depois, eu devia viver entre os quintos&
  varões, mas ter antes morrido ou depois nascido.                \num{175}&
  De fato agora a linhagem é de ferro: nunca, de dia,&
  se livrarão da fadiga e da agonia, nem à noite,&
  extenuando-se: os deuses darão duros tormentos.&
  Todavia, para eles aos males juntar-se-ão benesses.&
  Zeus destruirá também essa linhagem de homens mortais                \num{180}&
  quando, ao nascer, cãs nas têmporas tiverem.\&
\end{astanza}


\begin{astanza}
  Nem o pai semelhante aos filhos, nem os filhos a ele,&
  nem anfitrião a hóspede, e companheiro a companheiro,&
  nem irmão será querido como o foi no passado.&
  Aos genitores, tão logo envelhecerem, desonrarão;                \num{185}&
  deles se queixarão, falando com palavras duras –&
  terríveis, ignorantes do olhar dos deuses –, nem eles&
  aos genitores idosos retribuiriam a criação.&
  Mão-justa: um aniquilará a cidade do outro.&
  Não haverá gratidão pelo honesto nem pelo justo \num{190}&
  nem pelo bom, e mais ao feitor de males, ao varão&
  violento, honrarão: justiça, nas mãos, e vergonha&
  não haverá, e o mau prejudicará o homem melhor,&
  enunciando discursos tortos sobre os quais jurará.&
  E \edtext{inveja}{\nota{``inveja'' traduz \textit{z\=elos}, palavra ambivalente, que
  também pode significar ``emulação'', que aqui tem um sentido claramente
  negativo.}} a todos os homens lastimáveis                \num{195}&
  cacofônica, seguirá, sádica, horripilante.\&
\end{astanza}


\begin{astanza}
  Então, de fato, da terra largas-rotas rumo ao Olimpo, &
  a bela pele tendo encoberto com brancas capas,&
  para junto da tribo de imortais irão, deixando os homens,&
  Respeito e Indignação. Isso deixarão, aflições funestas,                \num{200}&
  aos homens mortais; e defesa não haverá contra o mal.&
  Agora história aos reis contarei, eles conscientes.&
  \edtext{%
    Assim dirigiu-se falcão a rouxinol pescoço-variegado,&
    que, mui no alto, entre nuvens, levava, dominando.&
    Este, tristemente, transpassado por garras recurvas, \num{205}&
    chorava, e aquele, sobranceiro, o discurso lhe disse:&
    ``Insano, por que guinchas? Tem-te um mui melhor;&
    irás aonde eu te levar, embora sendo um cantor;
  }{\lemma{rouxinol\ldots{}cantor}{\nota{jogo de palavras entre \textit{a\=edon}
  (``rouxinol'') e \textit{aoidos} (``cantor'').}}}&
  de ti farei refeição, se quiser, ou te libertarei.&
  Insensato quem quer a mais fortes contrapor-se;                \num{210}&
  privado da vitória, além do vexame sofre aflições''.\&
\end{astanza}


\begin{astanza}
  Assim falou o falcão voa-ligeiro, ave asa-comprida.&
  Oh Perses, escuta Justiça e não fomentes Violência:&
  violência é nociva no mortal miserável, e nem o nobre&
  é capaz de fácil suportá-la, mas a ela sucumbe                \num{215}&
  ao topar desastres. Caminho distinto de percorrer&
  é mais forte, o rumo ao justo: Justiça sobrepuja Violência&
  ao se consumar; e após sofrer o tolo aprende.&
  De pronto corre Juramento ao lado de juízos tortos;&
  tumulto, ao ser arrastada, causa Justiça, que levam varões                \num{220}&
  come-presente e com tortos juízos escolhem sentenças:&
  ela fica, chorando a cidade e as moradas das gentes,&
  envolta em neblina, levando um mal aos homens,&
  eles que a expelem e não um direito atribuem.&
  Os que \edtext{juízos}{\nota{ou seja, um juízo ``direito''}.} dão a estrangeiros e nativos, \num{225}&
  direitos, e de modo algum se desviam da justiça,\&
\end{astanza}


\begin{astanza}
  para eles a cidade viceja, as gentes nela florescem.&
  Paz nutre-jovens vai pela terra, e nunca a eles&
  destina guerra aflitiva Zeus ampla-visão;&
  nunca a varões reto-juízo segue Fome                \num{230}&
  nem Desastre, e em festejos repartem o fruto da lida.&
  Produz-lhes a terra muito sustento, e nos morros carvalho&
  produz bolotas no alto e, no meio, abelhas;&
  ovelhas lanosas sentem o peso dos velos;&
  e as mulheres parem filhos semelhantes aos pais.                \num{235}&
  Vicejam com coisas boas direto: para os barcos&
  não vão, e fruto produz o solo dá-trigo.&
  A quem importa nociva violência e feitos terríveis,&
  a eles justiça destina o Cronida, Zeus ampla-visão.&
  Amiúde até urbe inteira perde com um mau varão,                \num{240}&
  um que ofensa comete e arma iniquidades.\&
\end{astanza}


\begin{astanza}
  Sobre eles, do céu o Cronida envia grande desgraça,&
  fome e peste, e as gentes perecem;&
  as mulheres não param e as fazendas fenecem&
  pelo plano de Zeus Olímpico. E outra vez                \num{245}&
  destrói seu amplo exército ou sua muralha&
  ou de suas naus o Cronida se vinga no mar.&
  Oh reis, também vós mesmos ponderai a fundo&
  esse juízo: estando perto entre os homens,&
  os imortais ponderam quem com tortos juízos                \num{250}&
  ralam-se uns aos outros ignorando o olhar dos deuses.&
  Três vezes dez mil existem, pela terra nutre-muitos,&
  imortais guardiões de Zeus sobre os homens mortais,&
  eles que guardam juízos e feitos terríveis,&
  envoltos em neblina, vagando por toda a terra.                 \num{255}&
  Esta é uma moça, Justiça, nascida de Zeus,\&
\end{astanza}


\begin{astanza}
  majestosa e respeitada por deuses que o Olimpo ocupam,&
  e quando alguém a lesa, aviltando-a tortuosamente,&
  de pronto junto ao pai Zeus, o Cronida, se senta&
  e proclama a ideia injusta dos homens para ele punir                \num{260}&
  o povo pela iniquidade dos reis, que, com ideias funestas,&
  desviam juízos, enunciando-os tortuosamente.&
  Guardando essas coisas, endireitai os discursos, reis&
  come-presente, e dos juízos tortos de todo esqueçais.&
  A si mesmo faz mal varão que faz mal a outro,                \num{265}&
  e o plano vil, para quem planeja, é o pior.&
  O olho de Zeus, que tudo vê e tudo apreende,&
  também isso, se quiser, observa e não ignora&
  o jaez desse juízo que a cidade dentro encerra.&
  Agora, eu mesmo, entre os homens, justo                \num{270}&
  não fosse nem meu filho, pois é ruim ser o varão\&
\end{astanza}


\begin{astanza}
  justo, se o mais injusto justiça maior receberá.&
  Mas isso não espero que consuma Zeus astuto.&
  Oh Perses, tu essas coisas em tua mente lança,&
  ouve Justiça e a força de todo esquece.                \num{275}&
  Essa norma para os homens o Cronida ordenou,&
  para peixes, feras e aladas aves &
  se entredevorarem, pois Justiça não está entre eles.&
  Aos homens deu Justiça, que de longe o melhor&
  é: se alguém quiser anunciar o que é justo                \num{280}&
  ao reconhecê-lo, fortuna dar-lhe-ia Zeus ampla-visão.&
  Quem, no testemunho, de bom grado falseia juramento&
  e mente, lesa Justiça e se fere sem cura,&
  e, mais débil, sua estirpe, no futuro, fica atrás;&
  e a estirpe do varão honesto, no futuro, será melhor.                \num{285}&
  A ti eu direi o que penso de bom, mui tolo Perses;\&
\end{astanza}


\begin{astanza}
  Miséria é possível, aos montes, agarrar&
  facilmente: é plano o caminho, e mora bem perto.&
  Mas diante de Excelência suor puseram os deuses&
  imortais: longa e íngreme é a via até ela,                \num{290}&
  e áspera no início; e quando se chega ao topo,&
  fácil depois ela é, embora sendo difícil.&
  Este o melhor de todos, quem por si tudo apreender&
  ao refletir no que será melhor, depois e no fim.&
  Distinto também o persuadido por quem fala bem.                \num{295}&
  Quem por si não apreender nem o que de outro ouve&
  lançar no ânimo, esse é um varão infrutífero.&
  Mas tu, sempre lembrando de nossa ordem,&
  trabalha, Perses de linhagem divina, para Fome a ti&
  odiar, e te querer bem Deméter bela-coroa,                \num{300}&
  respeitada, e encher teu celeiro de sustento:\&
\end{astanza}


\begin{astanza}
  Fome é de todo companheira do varão inativo;&
  contra ele se indignam deuses e varões, quem inativo&
  vive, semelhante, no caráter, a zangões \edtext{sem-ferrão,}{\nota{o que se
  traduziu por ``sem-ferrão'' é, na verdade,  um
  adjetivo grego de sentido desconhecido.}}&
  eles que esgotam a faina das abelhas, inativos,                \num{305}&
  comendo; que o trabalho bem organizado te seja caro&
  para de sustento sazonal se encherem os celeiros.&
  Pelo trabalho são os varões muita-ovelha e ricos,&
  e, ao trabalhar, muito mais caro aos imortais&
  \edtext{e mortais serás: demais abominam inativos.}{\lemma{e
  mortais\ldots{}inativos.}{\nota{verso ausente de muitos textos antigos do
  poema.}}}                \num{310}&
  Trabalho não é insulto algum, inação, um insulto.&
  Se trabalhares, logo te invejará o inativo&
  ao enricares; à riqueza excelência e glória acompanham.&
  Seja o que o destino te tornar, trabalhar é melhor,&              
 \edtext{se, longe de posses alheias, com ânimo insano}{\lemma{se\ldots{}insano}{\nota{verso de difícil interpretação; outra
      possibilidade seria ``Tal como eras pelo destino, trabalhar é melhor'',
      ou seja, ``como eras (e ainda és) pobre''.}}} \num{315}&
  voltado ao trabalho, tratares do sustento como te peço.\&
\end{astanza}


\begin{astanza}
  \edtext{Vergonha não é boa em cuidar de varão carente,&
    vergonha, ela que muito lesa e beneficia varões:&
    vergonha, sim, é da desfortuna; audácia, da fortuna.
  }{\lemma{Vergonha\ldots{}fortuna.}{\nota{\textit{vergonha} é \textit{aid\=os}, termo de conotação geralmente positiva, como, por ex., no verso 200 (``Respeito'').}}}&
  Bens não são para se tomar; dado por deus é melhor:                \num{320}&
  se um à força, no braço, grande fortuna adquire&
  ou pilha por meio da língua, o que amiúde&
  ocorre quando lucro engana o espírito&
  dos homens, e Pouca-Vergonha expulsa Vergonha,&
  fácil debilitam-no os deuses, degradam a fazenda                \num{325}&
  do varão, e pouco tempo fortuna o acompanha.&
  Age igual quem prejudica suplicante ou estranho,&
  quem sobe no leito do próprio irmão&
  às esconsas para sexo com a esposa, ação imprópria,&
  quem, insensato, ofende os filhos órfãos de outrem,                \num{330}&
  quem com o genitor idoso no vil umbral da velhice\&
\end{astanza}


\begin{astanza}
  briga, abordando com palavras duras:&
  quanto a ele, o próprio Zeus se irrita, e no fim,&
  pelas ações injustas, impõe dura compensação.&
  Mas tu, disso mantém longe, de todo, o ânimo insano. \num{335}&
  Faze aos deuses imortais, quanto puderes, sacrifícios&
  santos e limpos, e em cima queima coxas radiantes;&
  outra vez, propicia-os com libações e incensos&
  quando fores dormir e quando a sacra luz retornar&
  para que tenham coração e ânimo propício a ti                \num{340}&
  a fim de permutares a gleba de outros, não outro, a tua.&
  Chama o amigo ao banquete, e o inimigo, deixa estar:&
  sobretudo chama quem perto de ti habita;&
  se também evento outro na fazenda te atingir,&
  vizinhos vêm sem cinto, mas cinturam-se os parentes.                \num{345}&
  Vizinho ruim é desgraça, tanto quanto bom é grande ajuda:\&
\end{astanza}


\begin{astanza}
  partilha de honraria quem partilha de ótimo vizinho;&
  nem um boi se perderia se o vizinho não fosse ruim.&
  O que vem do vizinho seja bem medido, e devolve bem&
  com a mesma medida ou mais, se fores capaz,                \num{350}&
  para, se necessitado, também no futuro achares algo certo.&
  Não lucres de forma vil; lucros vis são como desastres.&
  Ao amigo quer bem, e visita quem te visita.&
  E dá a quem der e não dês a quem não der:&
  ao dador alguém dá, ao não dador ninguém dá.                \num{355}&
  Dádiva é boa, Usurpa é má, doadora de morte:&
  se um varão de bom grado der algo, mesmo grande,&
  compraz-se com a dádiva e deleita-se em seu ânimo;&
  se, ele mesmo, confiante na pouca-vergonha, pegar &
  ainda que pouco, isso congela seu \edtext{coração.}{\nota{não fica claro no
  texto grego se o coração é da vítima ou do autor do roubo.}}
  \num{360}&
  Pois se depositares, mesmo pouco, sobre pouco,\&
\end{astanza}


\begin{astanza}
  e amiúde isso fizeres, muito isso logo se tornaria:&
  quem adiciona ao que há, esse evitará a fome ardente.&
  O depositado na casa não aflige o varão;&
  é melhor estar na casa, pois o que sai corre risco.                 \num{365}&
  É bom tomar do que está aí, e desgraça para o ânimo&
  carecer do distante; peço que nisso reflitas.&
  Do cântaro iniciado ou em seu final te farta,&
  se, na metade, poupa; poupança na base é pífio.&
  \edtext{%
    Que paga a um caro varão, acordada, seja certa;                 \num{370}&
    após sorrir para o irmão, traze uma testemunha –&
    tanto confiar quanto desconfiar destruiu varões.
  }{\lemma{Que paga\ldots{}varões.}{\nota{versos de origem polêmica (o verso 372 tem também um
  problema métrico na sua primeira metade), ausentes de parte dos papiros; 
  talvez tenham sido acrescentados alguns séculos
  depois da composição do poema e rejeitados pelos seus primeiros editores
  antigos.}}}&
  Que mulher \edtext{balanç’-a-bunda}{\nota{adjetivo de difícil interpretação,
  talvez dissesse respeito ao modo de a mulher se vestir (destacando, aumentando
  as nádegas).}} não te engane a ideia,&
  tagarelando, solerte, esquadrinhando teu celeiro.&
  Quem confia em mulher, esse confia em larápios.  \num{375}&
  Que haja um filho único para a casa paterna\&
\end{astanza}


\begin{astanza}
  manter, pois assim riqueza crescerá nos salões.&
  Que morra velho, deixando para trás \edtext{outro filho.}{\nota{verso
  difícil; menção a um segundo filho ou a um neto?}}&
  Fácil para mais gente Zeus daria fortuna incontável;&
  mais empenho de mais gente, maior o excedente.                \num{380}&
  Se riqueza te almeja o ânimo em teu íntimo,&
  faze assim, e trabalha trabalho sobre trabalho.&
  Ao subirem as Plêiades, filhas de Atlas,&
  iniciai a colheita, e arada e semeadura ao se porem.&
  Essas, se sabe, por quarenta noites e dias                \num{385}&
  estão ocultas, e de novo, após o ano volver-se,&
  aparecem assim que o ferro é amolado.&
  Esta é a regra para as planícies, e os que do mar&
  perto habitam e os que vales profundos,&
  longe do mar faz-onda, fértil região,                \num{390}&
  habitam: semeia pelado, ara pelado,\&
\end{astanza}


\begin{astanza}
  colhe pelado se quiseres, na estação, todo trabalho&
  de Deméter realizar, para cada um deles&
  te crescer na estação, e que não, mais tarde carente,&
  mendigues em alheias casas e nada realizes –                \num{395}&
  como também agora vieste a mim. Mais eu não te darei,&
  algo adicional não medirei; trabalha, tolo Perses,&
  os trabalhos que, aos homens, deuses destinaram&
  para um dia, com filhos e mulher, aflito no ânimo,&
  não buscares sustento nos vizinhos, que irão ignorar-te.                \num{400}&
  Duas ou três vezes talvez arranjes; se ainda amolares,&
  o feito não realizarás, e dirás muita coisa inútil,&
  e infrutífero será o pasto de palavras. Mas te peço,&
  reflete como livrar-te das dívidas e evitar a fome.&
  Primeiríssima, a fazenda, e a mulher e o boi de arar –                \num{405}&
  adquirida, não desposada, uma que seguiria os bois.&
  Os instrumentos, na casa, deixa todos preparados,&
  para a outro não pedires, ele recusar, e tu careceres,&
  e a estação transcorrer, e fenecer tua lavoura.&
  Não adies para amanhã e o dia seguinte:                \num{410}&
  o varão lida-inútil não enche o celeiro&
  nem o que adia; empenho desenvolve o trabalho,&
  e o varão lida-adiada sempre luta com desastres.&
  Quando o ímpeto do sol lancinante cessa&
  a queimação sudorífera, manda chuva outonal                \num{415}&
  o potente Zeus, e muda a pele do mortal,&
  mui aliviada: isso é quando o astro Sírio&
  ligeiro sobre a cabeça de homens cria-p'ra-morte&
  move-se de dia e colhe maior porção da noite –&
  nisso é menos sujeita a vermes a por ferro cortada                \num{420}&
  madeira, e folhas tombam no solo, e cessa de brotar.&
  Então corta madeira, lembrando do trabalho na estação:&
  corta um graal de três pés, um pilão de três cúbitos&
  e um eixo de sete pés, pois assim o encaixe é bom.&
  Se fosse de oito pés, marreta dela também cortarias.                \num{425}&
  Corta roda de três palmos para carro de dez mãos.&
  Há muita madeira curva: quando achares, \edtext{leva o apo}{\nota{na
  sequência, mencionam-se as partes de um arado de garganta (ou arado curvo).
  \textit{Apo} é um termo genérico que seria mais propriamente traduzido por
  \textit{garganta}.}}&
  para casa, procurando na montanha e no campo –&
  de azinheiro: esse, p'ra arar com bois, é o mais estável,&
  quando \edtext{o servo}{\nota{o carpinteiro.}} de Atena, após fixá-lo no dental                \num{430}&
  com pregos, aproxima e prende ao timão.&
  Labuta e coloca dois tipos de arado na fazenda, &
  com apo natural e articulado, pois muito melhor assim;&
  se quebrasses um, o outro sobre os bois lançarias.&
  Timões de loureiro ou de olmo são os mais sem vermes,                \num{435}\&
\end{astanza}


\begin{astanza}
  o dental, de carvalho, o apo, de azinheiro. Bois nove-anos,&
  dois, adquire, machos, pois não é fraca sua força,&
  ainda na juventude: os melhores para trabalhar.&
  Os dois não quebrariam o arado, na trilha do sulco&
  brigando, e deixariam o trabalho, inútil, lá mesmo.                \num{440}&
  Que a eles seguisse um varão quadragenário&
  após almoçar pão \edtext{oito-partes, quadripartível,}{\nota{dois adjetivos
  obscuros neste verso; provável referência a um pão muito grande e ao tamanho
  – grande – da porção comida pelo lavrador.}}&
  que, empenhado na tarefa, dirigiria um sulco reto,&
  não mais focado nos amigos, mas na tarefa&
  tendo o ânimo; outro mais jovem que ele não é melhor                \num{445}&
  para sementes repartir e não esbanjar no plantio:&
  varão mais jovem em busca dos amigos excita-se.&
  Fica atento \edtext{quando ouvires o som do grou}{\nota{entre o final de outubro e o início de novembro.}}&
  que do alto das nuvens guincha anualmente,&
  o que traz o sinal da arada, a estação do inverno                \num{450}\&
\end{astanza}


\begin{astanza}
  chuvoso indica e morde o coração do homem sem boi;&
  nisso dentro alimenta os lunados bois,&
  pois é fácil dizer ``Dá dois bois e um carro''&
  e é fácil recusar: ``Há trabalho para meus bois''.&
  E pensa o homem rico no juízo em compor um carro;                \num{455}&
  tolo, não sabe: cem pedaços tem um carro;&
  empenha-te em antes torná-los teus próprios.&
  Tão logo a arada aos mortais aparecer,&
  nisso te lança, igualmente os servos e tu mesmo,&
  arando a seca e a úmida na estação da arada,                \num{460}&
  em cedo te animando, para se encherem as lavouras.&
  \edtext{Na primavera te move; a renovada no verão não te trairá;&
  pousio semeia quando ainda for fofa a lavoura.}{\lemma{Na primavera te move\ldots{}lavoura.}{\nota{segundo West (1978: 274), o que Hesíodo quer dizer é ``a
terra que semeias deveria ser terra de pousio que araste na primavera e, de
preferência, de novo no verão, e deveria ser arada na hora certa antes que
chuva demais tenha caído}; muita chuva deixa a terra pesada''.}}&
  Pousio afasta-dano \edtext{apazigua crianças.}{\nota{menção possível a crianças que não passam fome.}}&
  Reza ao terreno Zeus e à santa Deméter \num{465}\&
\end{astanza}


\begin{astanza}
  para o grão de Deméter, maduro, sentir o peso&
  tão logo inicies a lavra, quando, a ponta da rabiça&
  na mão, o dorso dos bois com vara atingires,&
  que, no jugo, puxam o timão com a correia. Pouco atrás,&
  outro servo, com enxada, faça as aves se esfalfarem,                \num{470}&
  a semente escondendo: organização é o melhor&
  para os homens mortais, desorganização, o pior.&
  Assim, na fartura, ao solo as espigas se inclinariam&
  se o próprio Olímpio um bom termo depois ofertasse,&
  e teias de aranha expulsarias das vasilhas: espero que tu                \num{475}&
  jubiles ao pegares dos víveres que dentro estiverem.&
  Bem sucedido chegarás à nublada primavera e outros&
  não fitarás; mas de ti outro homem precisará.&
  E se nos solstícios de inverno arares a terra divina,&
  colherás sentado, pouco encerrando na mão,                \num{480}\&
\end{astanza}


\begin{astanza}
  amarrando em paralelo, empoeirado, não muito alegre,&
  e num cesto levarás; poucos te contemplarão.&
  Sempre cambiante é a ideia de Zeus porta-égide,&
  e para os homens mortais é difícil apreendê-la.&
  Se arares tarde, isto te poderia ser um remédio:                \num{485}&
  quando um cuco cucula nas folhas do carvalho &
  pela prima vez e compraz mortais na terra infinda,&
  Zeus poderia chover \edtext{três dias sem parar,}{\nota{ou ``no terceiro dia''; trata-se de março.}}&
  nem cobrindo a marca do casco do boi nem faltando;&
  assim a arada tardia com a arada precoce competiria.                \num{490}&
  No ânimo a tudo atenta bem: que não te escape&
  a vinda da nublada primavera nem a chuva sazonal.&
  Passa ao largo do banco do ferreiro e do galpão protetor&
  na estação invernal, quando varões da lavoura o frio &
  afasta: então inabalável varão muito a casa fomentaria;                \num{495}\&
\end{astanza}


\begin{astanza}
  temo te capturem a Impotência do inverno vil&
  com Pobreza, e \edtext{com mão leve o encorpado pé apertes.}{\nota{sinais de
      desnutrição ou, mais provável, sugestão de masturbação; \textit{pé} pode se
  referir também no verso 524 ao membro masculino.}}&
  Amiúde varão inativo, que fica junto à vã esperança, &
  carecendo de recursos, de vilezas fala ao ânimo.&
  Esperança não é boa em cuidar de varão carente                \num{500}&
  sentado no galpão sem sustento protetor.&
  Aponta aos servos ainda no meio do verão:&
  ``Não será sempre verão; fazei-vos cabanas''.&
  O mês Lenaion, dias ruins, todos para couro de boi,&
  evita isso e também geadas, que, sobre a terra,                \num{505}&
  quando Bóreas sopra, são implacáveis,&
  ele que, pela Trácia nutre-cavalo, no amplo mar&
  sopra e o agita, e mugem terra e mato.&
  A muitos carvalhos alta-copa e abetos encorpados&
  faz nos vales dos montes tocar a terra nutre-muitos,                \num{510}\&
\end{astanza}


\begin{astanza}
  neles caindo, e toda a vasta floresta então grita.&
  Tremem os bichos e põem os rabos sob os genitais,&
  até os com lanugem sombreando a pele: mas por eles&
  sopra, frio, mesmo que tenham peitos hirsutos,&
  e atravessa couro de boi, que não o contém,                 \num{515}&
  \edtext{e sopra através da cabra longo-pelo; e ovelhas, não&
  é porque têm pelos fartos que por elas não sopra&
a força do vento Bóreas. Faz o ancião \edtext{recurvar-se}{\nota{sentido
    incerto; \textit{recurvar-se} (como uma roda) ou \textit{correr} (como uma roda ou sobre
rodas).}}}{\lemma{ovelhas\ldots{}Bóreas.}{\nota{este é um modo de traduzir a passagem referente às ovelhas: mesmo
elas, com seus pelos fartos, não sucumbem ao vento do norte. Outro modo seria
``e ovelhas não, / pois os pelos são fartos, por eles não sopra a força do
vento Bóreas''}.}}&
  e através da moça pele-macia não sopra,&
  ela que dentro da casa fica junto à cara mãe,                \num{520}&
  ainda não versada nos feitos de Afrodite muito-ouro:&
  após bem lavar a pele delicada e com óleo à farta&
  se ungir, dentro se protege, no recesso da casa&
  em dia invernal, quando o \edtext{sem-osso amolenta}{\nota{há dúvidas sobre o
      sentido do verbo traduzido por \textit{amolentar} e o referente do termo
      \textit{sem-osso}; para este último, \textit{polvo} foi a opção dos comentadores
      antigos (seguida por vários modernos) e \textit{pênis} aquela mais adequada ao
  contexto e bem demonstrada filologicamente.}} seu pé&
  na casa sem fogo, deplorável espaço,                 \num{525}\&
\end{astanza}


\begin{astanza}
  pois o sol não lhe mostra campo para avançar,&
  mas ao povo e à cidade dos homens negros&
  visita e a todos os helenos mais tarde brilha.&
  Então \edtext{os chifrudos e os sem-chifre}{\nota{a referência não é,
  necessariamente, somente a veados machos (com cornos) e fêmeas (sem
  cornos).}} toca-no-mato&
  rangem funestamente e pelos capões com ravinas                \num{530}&
  fogem, e no juízo a todos isto preocupa,&
  procurar abrigos e ter covis cerrados&
  ou furna pétrea; \edtext{então são iguais a mortal três-pés,&
  com as costas alquebradas, cuja cabeça olha o chão:}{\lemma{então
  são\ldots{}chão:}{\nota{outra adivinha; aqui, anciãos. Não podemos excluir uma
  alusão ao enigma da Esfinge decifrado por Édipo.}}}&
  semelhantes a ele vagam, evitando a neve branca.                \num{535}&
  Então veste uma defesa do corpo, como te peço,&
  manto macio e túnica até os pés.&
  Trama muito fio em miúda urdidura:&
  disso te reveste, para teus pelos ficarem imóveis&
  e não, retos, tremerem de pé pelo corpo.                \num{540}\&
\end{astanza}


\begin{astanza}
  Em volta dos pés, sandálias de boi morto à força,&
  ajustadas, prende, cobertas no interior com feltro;&
  de filhotes primogênitos, quando o frio sazonal vier,&
  cose as peles com tendão de boi para nas costas&
  lançares, proteção contra chuva; usa sobre a cabeça                \num{545}&
  boné de feltro bem feito, para não molhar as orelhas.&
  Pois a aurora é gelada \edtext{quando Bóreas cai,}{\nota{``quando perde sua intensidade'' é uma possibilidade de interpretação.}}&
  e, matutina, vinda do céu estrelado, sobre a terra&
  bruma tritícola se estende sobre lavouras de ditosos,&
  ela que é recolhida dos rios perenes,                \num{550}&
  ao alto, sobre a terra, erguida por lufada de vento,&
  e ora  chove ao anoitecer, ora venta,&
  o trácio Bóreas nuvens espessas aglomerando.&
  Antes dele, finda teu trabalho e volta para casa:&
  que não te encubra, vinda do céu, nuvem escura,                \num{555}\&
\end{astanza}


\begin{astanza}
  deixe a pele molhada e encharque as roupas.&
  Mas evita, pois o mês mais difícil é esse,&
  invernoso, difícil aos rebanhos, difícil aos homens.&
%Jorge: Revisar nota
  Nisso a metade dá aos bois, e ao homem a maior parte &
  de sua ração: \edtext{as longas noites são ajudantes.}{\nota{no grego, ao
  em vez de ``noite'' lê-se ``aquela que alivia o juízo'', uma expressão metafórica
  relativamente comum.}}\num{560}&
  Atento a isso \edtext{até se completar o ano,}{\nota{não confundir com nosso
  calendário civil; no imaginário grego de Hesíodo, o ano – ou seja, o ciclo da
  agricultura – começa quando os depósitos de comida estão cheios.}}&
  \edtext{contrabalança noites e dias}{\nota{ao longo do ano, à medida que as
  noites forem diminuindo, pode-se aumentar a ração diária.}} até que de novo&
  a terra, mãe de tudo, trouxer fruto variado.&
  Quando, depois do solstício, o \edtext{sexagésimo}{\nota{a segunda metade de fevereiro.}}&
  dia invernal Zeus completar, então o astro                \num{565}&
  Arcturo, após deixar a sacra corrente de Oceano,&
  brilhando todo, raia bem no início do lusco-fusco.&
      Após, a Pandionida vero-pranto, a \edtext{andorinha,}{\nota{outro epíteto
      transmitido é ``pranto antes da aurora''. Trata-se da Filomela do mito de
  Tereu e Procne (Rouxinol).}} se lança&
  à luz para os homens, a primavera de novo iniciando.&
  Antes dela poda as vinhas, pois assim é melhor.                \num{570}\&
\end{astanza}


\begin{astanza}
  Mas ao subir nas árvores, vindo da terra, o \edtext{leva-casa,}{\nota{lesma.}}&
  as Plêiades evitando, \edtext{então não mais caves junto às vinhas,}{\nota{em meados de maio.}}&
  não, amola a foice e aos escravos desperta.&
  Evita bancos sombreados e sono até a aurora&
  na estação da colheita quando o sol seca a pele:                \num{575}&
  nessa época te apressa e para casa o fruto leva,&
  de pé no nascente, para o sustento te bastar.&
  Aurora tem do trabalho a terça parte,&
  Aurora incita na jornada, incita no trabalho,&
  Aurora, ao surgir, faz pegarem a estrada muitos                \num{580}&
  homens e em muitos bois põe o jugo.&
  \edtext{Quando enflora o cardo-de-ouro, e a soante \edtext{cigarra,}{\nota{trata-se, provavelmente, do \textit{scolymus hispanicus.}}}&
  em árvore sentada, entorna soante canto}{\lemma{Quando enflora\ldots{}canto}{\nota{meados de julho.}}}&
  sem parar de sob as asas, na estação do fadigoso estio,&
  então as cabras, gordíssimas, o vinho, excelente,                \num{585}\&
\end{astanza}


\begin{astanza}
  as mulheres, lascivíssimas, os homens, \edtext{esgotadíssimos}{\nota{Hesíodo
  talvez aluda, por meio de uma figura etimológica com o adjetivo referente aos
  homens, a um esgotamento de sêmen.}}&
  estão, pois cabeça e joelhos Sírio seca,&
  ressequida a pele com o calor. Mas que nisso&
% Trácia?
  haja rochedo umbroso e vinho de \edtext{Biblos,}{\nota{Distrito da Trácia.}}&
  pão feito com leite, leite de cabras que secam                \num{590}&
  e carne de \edtext{vaca pasto-no-mato}{\nota{a referência é a gado selvagem.}} que nunca pariu&
  ou dos primeiros filhotes. Junto bebe faiscante vinho&
  sentado à sombra, saciado, no coração, de comida,&
  após a face voltar para Zéfiro sopra-do-alto;&
  de fonte perene, corrente e límpida,                 \num{595}&
  três partes de água tira, e lança quarta de vinho.&
  Ao sagrado grão de Deméter incita os escravos&
  a debulhar, assim que o \edtext{vigor de Orion surgir}{\nota{em torno de 20 de junho.}},&
  num espaço ventilado, uma eira bem aplanada.&
  Medido, traze direito em vasilhas. E após                \num{600}\&
\end{astanza}


\begin{astanza}
  todo o sustento, trancado, pores dentro de casa,&
  traze para dentro ração e limpadura: que haja                \numero{[606]}&
  suficiente para os bois e as mulas. E depois,                \numero{[607]}&
  escravos refresquem seus joelhos e soltem os bois.                \numero{[608]}&
  Empregado sem casa arranja e criada sem filho                \numero{[602]}&
  procura, eu te peço; problema, criada com bezerro.&
  Cuida do cão dente-afiado – não poupes na comida –&
  para nunca varão sono-diurno teus bens arrancar.                \numero{[605]}&
  \edtext{Quando Orion e Sírio chegarem ao meio}{\nota{meados de setembro.}}\numero{[609]}&
  do céu, e a Arcturo vir Aurora dedos-róseos,                \num{610}&
  Perses, colhe todo cacho de uva e leva para casa;&
  expõe-nos ao sol por dez dias e dez noites,&
  por cinco, na sombra, e verte no sexto em vasilhas &
  o dom de Dioniso muito-júbilo. E quando&
  as Plêiades, as Híades e a força de Orion                \num{615}\&
\end{astanza}


\begin{astanza}
  se puserem, nisso então te lembra da lavra&
  na estação: que \edtext{o ano}{\nota{o termo que traduzi por \textit{ano} – talvez
  mais especificamente \textit{ano rural} – é incerto; como se trata de algo dentro do
  solo, \textit{semente, cereal} é outra opção.}} à terra se adeque.&
  E se te toma o desejo pela encrespada navegação:&
  \edtext{quando}{\nota{novembro.}} as Plêiades da força ponderosa de Orion&
  fogem e caem no mar embaciado,                \num{620}&
  então grassam rajadas de todos os ventos;&
  então não mais o barco mantenhas no mar vinoso&
  mas, cônscio, trabalha a terra como te peço.&
  Puxa teu barco à praia e abriga-o com pedras &
  em volta para o ímpeto de ventos úmidos conterem,                \num{625}&
  e drena a água para a chuva de Zeus não o apodrecer.&
  Todo o material, trancado, põe em tua casa,&
  após dispor, bem ordenadas, as asas da nau cruza-mar;&
  e o leme bem-feito sobre a fumaça pendura.&
  Tu mesmo espera até vir o tempo da navegação;                \num{630}\&
\end{astanza}


\begin{astanza}
  então nau veloz puxa ao mar e nela carga&
  arranjada apronta para lucro à casa granjear,&
  como meu e teu pai, grande tolo Perses,&
  com barcos navegava, carente de vida nobre.&
  Ele um dia também veio aqui após muito mar cruzar,                \num{635}&
  tendo deixado a Cime eólia em negra nau,&
  não fugindo de abastança, riqueza e fortuna,&
  mas da vil pobreza, a que Zeus dá aos homens.&
  Fixou-se perto do Hélicon em miserável vilarejo,&
  Ascra, ruim no inverno, cruel no verão, boa jamais.                \num{640}&
  Tu, Perses, lembra-te dos trabalhos&
  na estação, de todos, sobremodo os da navegação.&
  %provavelmente algo como 'não, obrigado' (a 1ª parte).
  Louva nau pequena, e em grande põe a carga:&
  carga, maior, lucro – sobre lucro – maior&
  será, se os ventos suas vis rajadas contiverem.                \num{645}\&
\end{astanza}


\begin{astanza}
  Quando ao comércio voltares teu ânimo insano&
  e quiseres de dívidas fugir e da fome sem deleite,&
  mostrar-te-ei as medidas do mar ressoante,&
  de modo algum sábio em navegação ou barcos.&
  Pois nunca, com barco, naveguei pelo amplo mar,                \num{650}&
  exceto \edtext{de Aulis à Eubeia,}{\nota{o trecho em questão é de 65m., o que aponta para uma certa comicidade na passagem.}} onde um dia os aqueus,&
  esperando no inverno, grande tropa reuniram&
  para ir da sacra Hélade a Troia belas-mulheres.&
  Lá eu, atrás de prêmios pelo aguerrido Anfidamas,&
  até Cálquis cruzei – em profusão esses anunciados                 \num{655}&
  prêmios fixaram os filhos do enérgico –, onde afirmo eu&
  com um canto ter vencido e trazido trípode orelhuda.&
  Essa eu mesmo às Musas do Hélicon dediquei&
  onde no início puseram-me na via do canto soante.&
  Tanta é minha experiência em naus muito-prego;                \num{660}\&
\end{astanza}


\begin{astanza}
  mas mesmo assim direi a ideia de Zeus porta-égide,&
  pois as Musas ensinaram-me a cantar canto ilimitado.&
  Por cinquenta dias depois do \edtext{solstício,}{\nota{do final de junho até
  agosto. ``Por volta do quinquagésimo dia após o solstício'' também é uma
  tradução possível.}}&
  quando chega ao fim o verão, a estação da fadiga,&
  é hora, para os mortais, de navegar: nem ao barco                \num{665}&
  destroçarias, nem aos varões aniquilaria o mar,&
  se, de propósito, nem Poseidon agita-a-terra&
  ou Zeus, rei dos imortais, quiserem destruir,&
  pois neles igualmente está o termo de males e o de bens.&
  Quando as brisas são definidas, e o mar, seguro,                \num{670}&
  nisso, despreocupado, nau veloz confia aos ventos,&
  puxa-a ao mar e toda a carga nela põe.&
  Apressa-te o máximo em de novo à casa voltar&
  e não espera vinho novo, \edtext{chuva outonal,&
  o inverno chegando e as terríveis rajadas de Noto,}{\lemma{chuva outonal\ldots{}Noto}{\nota{parte final de setembro.}}}                \num{675}\&
\end{astanza}


\begin{astanza}
  que agita o oceano, companheiro da chuva de Zeus&
  abundante no outono, e deixa o mar terrível.&
  Outra navegação, \edtext{primaveril,}{\nota{final de abril.}} há para os homens:&
  quando chega a hora – o corvo, ao andar, tão grande&
  pegada faz quanto ao homem parece a folha                \num{680}&
  no alto da figueira –, nisso se pode no mar embarcar;&
  é primaveril essa navegação. Eu não a ela&
  louvaria pois não compraz meu espírito:&
  capciosa; difícil escapares de um mal. Também isso,&
  porém, homens fazem na ignorância do espírito:                \num{685}&
  bens são vida para mortais coitados.&
  Terrível é morrer entre as ondas; mas te peço,&
  no juízo considera tudo isso como exponho.&
  E nas cavas naus não ponhas todo o sustento,&
  mas deixa a maior parte, e a menor, carrega:                \num{690}\&
\end{astanza}


\begin{astanza}
  é terrível, entre as ondas do mar, atingir desgraças&
  e, terrível, se para cima do carro peso brutal ergueres,&
  romperes o eixo, e a carga se degradar.&
  Atenta às medidas; e \edtext{oportunidade}{\nota{\textit{kairos}, a mais
  antiga aparição do termo conhecida por nós. Por isso mesmo, seu sentido na
  passagem depende de interpretação. Acredito que ele se refira, ao mesmo
  tempo, ao momento exato em que tanto as atividades agrícolas quanto as de
  navegação devem ser realizadas quanto ao contexto imediato do conselho que
  segue.}} é o melhor em tudo.&
  Na tua hora faze conduzir esposa à casa,                \num{695}&
  nem muitos anos te faltando para os trinta,&
  nem com muitos a mais: essas te são bodas na hora.&
  A mulher deve ser moça há quatro, e no quinto, casar.&
  Desposa uma virgem  para ensinares hábitos de afeição,&
  e desposa em especial uma que more perto de ti                \num{700}&
  após olhar tudo ao redor: \edtext{não cases alegrando vizinhos.}{\nota{no caso da esposa ser infiel.}}&
  Nada melhor conquista o varão que uma esposa&
  boa; que uma ruim nada há mais frio, &
 \edtext{a embosca-no-jantar, que ao esposo, embora altivo,&
queima sem tição e o entrega à crua velhice.}{\lemma{a
embosca-no-jantar\ldots{}velhice.}{\nota{uma das inúmeras passagens bem-humoradas
(humor negro, claro) de Hesíodo; ela faz do marido uma refeição, mas crua. Para
alguns, alusão ao modo como Agamêmnon, no canto 11 da \textit{Odisseia}, conta a Odisseu
como foi morto por Clitemnestra.}}}                 \num{705}\&
\end{astanza}


\begin{astanza}
  Sê bem \edtext{atento ao olhar dos deuses}{\nota{esse olhar (\textit{opis}) diz respeito a uma punição retributiva.}} ditosos.&
  Não trates o camarada igual ao irmão.&
  Se o tratares, não lhe faças mal primeiro&
  nem mintas com graça da língua; se ele iniciar,&
  dizendo-te palavra detestável ou agindo,                \num{710}&
  lembra-te de puni-lo com o dobro. Mas se de volta&
  for levado à amizade e quiser dar compensação,&
  aceita: reles o varão que ora um, ora outro seu amigo &
  torna. Que de todo não infame o espírito tua aparência.&
  Que não te chamem muito-hóspede nem sem-hóspede,                \num{715}&
  nem companheiro de vis nem vituperador de bons.&
  Nunca ouses por destrutiva pobreza tira-vida &
  a um varão insultar, dom dos ditosos sempre vivos.&
  O melhor, entre os homens, é o tesouro da língua&
  parca, e o máximo de graça está na comedida:                \num{720}\&
\end{astanza}


\begin{astanza}
  se falares algo vil, logo tu mesmo mais ouvirás.&
  E não seja destemperado em banquete muito-hóspede:&
  na comunhão, a graça é máxima, a despesa, mínima.&
  Nunca, na aurora, libes a Zeus com vinho faiscante&
  nem aos outros imortais tendo mãos não lavadas,                \num{725}&
  pois não atendem e cospem as preces.&
  Não urines de pé voltado para o sol;&
  quando se pôr, lembrando-te, até ele raiar,&
  não o faça desnudado: as noites são dos ditosos.              \numero{[730]}&
  Nem na via nem fora dela, caminhando, urines.                \numero{[729]}&
  Agacha-se o varão divino, versado em bom senso,&
  ou se aproxima do muro de pátio bem cercado.&
  Nem as vergonhas, salpicado de esperma, em casa&
  exibas nas proximidades do fogo-lar, mas evita.&
  Nem ao retornar de um funeral mau-agouro                \num{735}\&
\end{astanza}


\begin{astanza}
  semeies prole, mas de um banquete de imortais.&
  Nem a água belo-fluxo dos rios permanentes&
  cruzes com os pés antes de rezar, mirando a bela corrente,&
  as mãos tendo lavado com água limpa e adorável.&
  Quem um rio atravessa sem lavar o mal e as mãos,                \num{740}&
  contra ele deuses se indignam e males dão no futuro.&
  \edtext{Nem da cinco-galhos,}{\nota{metáfora vegetal para uma interdição relativa ao corte das unhas.}} no farto banquete dos deuses,&
  cortes o seco do verdejante com fúlgido ferro.&
  Nem nunca ponhas \edtext{jarra sobre ânfora}{\nota{enquanto se bebe, a jarra é usada para tirar vinho da ânfora e servi-lo.}}&
  enquanto bebem: destino ruinoso para isso ocorre.                \num{745}&
  Nem, ao fazer a casa, a deixes rugosa em cima, &
  para um corvo grasnando, lá sentado, não crocitar.&
  Nem tires de caldeirão com pés, não consagrado,&
  para comer ou lavar-se, pois nisso recai pena.&
  Nem \edtext{assentes no que é imóvel,}{\nota{por exemplo, os túmulos.}} pois não é melhor,                \num{750}\&
\end{astanza}


\begin{astanza}
  um filho de doze dias, o que torna o varão desviril,&
  nem um de doze meses: ocorre igual evento.&
  Nem com água de mulher limpes o corpo&
  do varão: por um tempo, também nisso recairá triste&
  pena. Nem, ao encontrar sacrifícios chamejantes,                \num{755}&
  censura infernalmente: também com isso \edtext{se indigna o deus.}{\nota{por demais ou de menos da carne estar sendo sacrificada aos deuses.}}&
  Nem nas bocas dos rios que fluem ao mar&
  nem em fontes urines, mas de todo evita,&
  \edtext{nem neles soltes ar,}{\nota{``não defeques'' é outra interpretação possível.}} pois isso não é melhor.&
  Age assim; e evita a assombrosa reputação dos mortais, \num{760}&
  pois a reputação vil é leve para alguém erguer,&
  mui fácil, difícil para suportar e dura para pôr de lado.&
  Reputação nunca de todo perece, qualquer que muitas&
  gentes reputarem: também ela mesma é um deus.&
  Bem atento aos dias que vêm de Zeus, ordenadamente                 \num{765}\&
\end{astanza}


\begin{astanza}
  aponta-os aos escravos: o trigésimo do mês é o melhor&
  para observar os trabalhos e distribuir a ração,&
  quando as pessoas, distinguindo a verdade, o celebram.&
  Estes são os dias que vêm do astuto Zeus:&
  começando, o primeiro, o quarto, o sétimo, dia sacro –                \num{770}&
  nesse Leto gerou Apolo espada-de-ouro –,&
  o oitavo e o nono. Todavia, dois dias do mês&
  crescente se destacam para a faina de trabalhos mortais,&
  o dia onze e o dia doze. Ambos são ótimos&
  para tosquiar ovelhas e recolher o grão que alegra,                \num{775}&
  e o dia doze é muito melhor que o dia onze,&
  pois nele trama seus fios a aranha voa-alto&
  em pleno dia, quando a \edtext{habilidosa}{\nota{formiga.}} recolhe o monte.&
  Nele mulher deveria armar urdidura e pôr-se a obrar.&
  O mês crescendo, evita o décimo terceiro                \num{780}\&
\end{astanza}


  \begin{astanza}
    como início do plantio; o melhor para plantas encanteirar.&
    \edtext{O sexto}{\nota{os dias do mês são referidos de três formas,
    numericamente (do 1º ao 30º dia), pela lua crescente e decrescente ou por
uma divisão em três partes, a 1ª começando no dia 1º, a 2ª no dia 14 (por
exemplo, ``no nono do meio'' quer dizer ``o nono dia no segundo período de dez
dias'') e a 3ª no dia 21, quando, na Grécia histórica (em Hesíodo também?), os
dias eram referidos de forma decrescente:  ``o quarto dia do que míngua''
(798), neste caso, seria o 27º, não o 24º.}} do meio às plantas é bastante
desfavorável &
    e bom para nascerem varões; a meninas não é favorável,&
    nem, em primeiro lugar, para nascer nem para casar.&
    Nem o primeiro sexto dia para uma menina nascer                \num{785}&
    é adequado, mas para castrar cabritos e carneiros&
    e dia gentil para cercar em curral de rebanhos.&
    P'ra nascerem varões é ótimo: amaria provocações falar,&
    mentiras, histórias solertes e conversas escusas.&
    No oitavo do mês, javali e touro muito-mugido                \num{790}&
    castra, e no décimo segundo, mulas robustas.&
    No grande vigésimo, em pleno dia, homem sábio&
    nasça: na mente será muito arguto.&
    P'ra nascerem varões é ótimo o décimo; meninas, o quarto&
    no meio: nesse, ovelhas, lunadas vacas trôpegas,                 \num{795}\&
  \end{astanza}


  \begin{astanza}
    cão dente-pontudo e mulas robustas,&
    sobre esses põe a mão e amansa. Atenta, no ânimo,&
    a evitar o quarto dia do que míngua e do que cresce,&
    aflições come-ânimo: é um dia assaz completo.&
    No quarto do mês, faze conduzir a noiva para casa,                \num{800}&
    após discernir as aves, as melhores para esse feito.&
    Evita os dias cinco, pois duros e terríveis:&
    dizem que \edtext{no quinto}{\nota{o quinto dia de cada parte? O dia 5 de cada
    mês? A expressão grega deveria ser traduzida pelo singular ``o dia
cinco''?}} as Erínias zelaram&
    Jura ao nascer, que Disputa pariu, desgraça aos perjuros.&
    No sétimo do meio, o sagrado grão de Deméter                 \num{805}&
    muito bem observa e na eira bem aplanada&
    joeira-o, e corte o lenhador madeira para o tálamo&
    e muita lenha de barco que seja adequada a naus.&
    \edtext{No quarto,}{\nota{provavelmente deve-se subentender ``no quarto do meio''.}} começa a construir naus pequenas.&
    No nono do meio, o dia é melhor no entardecer;                \num{810}\&
  \end{astanza}


  \begin{astanza}
    o primeiro nono inofensível ao homem é de todo,&
    pois é ótimo para serem plantados e gerados&
    homem e mulher, e nunca é um dia de todo ruim.&
    Poucos sabem que o três vezes nono é \edtext{o melhor dia do mês}{\nota{provavelmente o dia 27.}}&
    \edtext{para iniciar um cântaro,}{\nota{começar o armazenamento em um cântaro vazio.}} pôr o jugo no pescoço                \num{815}&
    de bois, mulas e cavalos pés-ligeiros,&
    e uma nau muito-calço, veloz, ao mar vinoso&
    puxar; poucos o chamam pela forma verdadeira.&
    No quarto do meio, abre um cântaro: mais que todos, dia&
    sagrado. Poucos sabem ser o vigésimo primeiro o melhor                \num{820}&
    quando vem a aurora; no entardecer, é pior.&
    Esses dias são uma grande ajuda aos terrestres.&
    Os outros são incertos, sem destino, nada trazem.&
    Cada um louva dia distinto, e poucos sabem.&
    Ora é uma sogra, ora uma mãe um dia                \num{825}\&
  \end{astanza}


  \begin{astanza}
    desses. Venturoso e afortunado quem, tudo isso&
    conhecendo, trabalha, de nada culpado contra imortais,&
    aves discernindo e transgressões evitando.\&
  \end{astanza}


