
\chapter[Origem da literatura de cordel e a sua expressão de cultura nas letras de nosso país]{Origem da literatura de cordel e a sua expressão de cultura nas letras de nosso país}

\begin{verse}
Cordel quer dizer barbante\\*
Ou senão mesmo cordão,\\*
Mas cordel"-literatura\\*
É a real expressão\\*
Como fonte de cultura\\*
Ou melhor poesia pura\\*
Dos poetas do sertão.

Na França, também Espanha\\*
Era nas bancas vendida,\\*
Que fosse em prosa ou em verso\\*
Por ser a mais preferida,\\*
Com o seu preço popular\\*
Poderia se encontrar\\*
Nas esquinas da avenida.

Era em pequeno volume\\*
A edição publicada,\\*
Tamanho 15 por 12\\*
Pra melhor ser consultada,\\*
Isso no século \textsc{xviii}\\*
Depois de noventa e oito\\*
Foi aos poucos desprezada.

No Brasil é diferente\\*
O cordel"-literatura\\*
Tem que ser todo rimado\\*
Com sua própria estrutura\\*
Versificado em setilhas\\*
Ou senão em setilhas\\*
Com a métrica mais pura.

Nesse estilo o vate escreve\\*
Em forma de narração\\*
Fatos, romances, histórias\\*
De realismo, ficção;\\*
Não vale cordel, em prosa,\\*
E em décima na glosa\\*
Se verseja no sertão.

Pode o mote ser glosado\\*
Em sete sílabas também --\\*
Isso depende do ouvinte\\*
O mote rimado bem,\\*
Sem a métrica perfeita\\*
A glosa será mal feita\\*
Que não agrada a ninguém.

Os primeiros repentistas\\*
Residiam no Teixeira\\*
Cidade da Paraíba\\*
Como Inácio Catingueira\\*
Hogolino e outros mais\\*
Nossos primeiros jograis\\*
Na viola tinideira.

Fabiano das Queimadas\\*
Do tempo da escravidão\\*
E Romano da Mãe d'Água,\\*
O negro Preto Limão,\\*
Depois Antonio Marinho\\*
E o célebre Canhotinho\\*
Ídolos da multidão.

No começo a poesia\\*
Popular hoje cordel\\*
Era em quadras, realmente,\\*
Que usava o menestrel,\\*
Mas Silvino Pirauá\\*
Um novo sistema dá\\*
De maneira mais fiel.

Repetindo os últimos versos\\*
Da quadra forma setilha\\*
Cuja estrofe mais completa\\*
Na melodia mais brilha,\\*
Foi assim que começou\\*
E depois continuou\\*
aceitando a setilha.

No início os cantadores\\*
Cantavam com seu pandeiro\\*
Com triângulo, com rabeca\\*
No Nordeste brasileiro,\\*
As fazendas se alegravam\\*
E os ouvintes deliravam\\*
Nos salões ou no terreiro.

Os coronéis das fazendas\\*
Convidavam moradores\\*
Pra assistirem as pelejas\\*
Dos famosos cantadores\\*
Eram os grandes desafios\\*
Dos repentistas bravios\\*
Dos versos mais multicores.

Naquele tempo os poetas\\*
Não usavam profissão\\*
Embora fossem pagos\\*
No calor da discussão,\\*
O dinheiro que ganhavam\\*
Sua alforria compravam\\*
Saindo da escravidão.

Acontece que os vates\\*
Conhecidos de \textsc{bancada}\\*
Que não eram repentistas\\*
Escreviam bem rimadas\\*
As disputas que assistiam\\*
E o seu folheto vendiam\\*
Cuja obra era aceitada.

Essa poesia era\\*
Como folheto vendida\\*
Daí passavam escrever\\*
O cotidiano da vida,\\*
Os casos da região\\*
Ou história de valentão\\*
Que não era acontecida.

De tudo que acontecia\\*
No país ia escrevendo...\\*
Padre Cícero, Lampião,\\*
Ia o povo tudo lendo.\\*
Criou hábito no povo\\*
De ler um folheto novo\\*
Para a notícia ir sabendo.

O chamado trovador\\*
Ou poeta popular\\*
Era semi"-analfabeto\\*
Porém sabia rimar,\\*
Seus folhetos escrevia\\*
E os sertanejos os liam\\*
Por ser o seu linguajar.

Como o \textsc{mobral}, no Nordeste,\\*
Muito alfabetizou\\*
Nesses mesmos trovadores\\*
A gramática melhorou,\\*
Havia vates letrados\\*
No Nordeste consagrados\\*
Isso a história registrou.

Leandro Gomes de Barros\\*
Famoso Chagas Batista,\\*
José Camelo de Melo\\*
E um outro cordelista\\*
João Athayde --  editor\\*
Foi quem deram mais valor\\*
À classe de repentista.

O cordel é dividido\\*
Escrito, cantado, oral,\\*
Porém o cordel legítimo\\*
É aquele tipo jornal,\\*
Que trazia a notícia nova\\*
Em sextilhas, nunca em trova\\*
Que agrada o pessoal.

O cordel sendo cultura\\*
Hoje tem sua tradição,\\*
Chamado literatura\\*
Veículo de educação\\*
Retrata histórias passadas\\*
Que estão documentadas\\*
Para toda geração.

Nas famosas faculdades\\*
Da Itália e Grã"-Bretanha,\\*
Japão, Estados Unidos,\\*
França, Portugal, Espanha,\\*
Se formam hoje doutores\\*
Nos versos dos trovadores\\*
Como dá"-se na Alemanha.

Cadeira para cordel\\*
Hoje é uma realidade,\\*
Por exemplo hoje em São Paulo\\*
Em qualquer u'a faculdade\\*
Tem muita gente estudando\\*
Muitos jovens pesquisando\\*
Como especialidade.

Livro de Manoel d'Almeida,\\*
João José e Minelvino,\\*
E José da Costa Leite\\*
Que é bom vate nordestino\\*
São em Sorbonne estudados\\*
E no Arizona cotados\\*
dentro do melhor destino.

Dizer hoje quem é o maior\\*
Dos poetas do Brasil\\*
Escritores de cordel\\*
É uma opinião mais vil,\\*
São centenas de valores\\*
E falando em cantadores\\*
Hoje ultrapassam os mil.

Já foi tempo que diziam\\*
Que os folhetos do sertão\\*
Eram só de analfabetos,\\*
De poetas sem instrução,\\*
Há trovadores formados\\*
E outros conceituados\\*
Pela boa correção.

A nossa literatura\\*
De cordel pelos sertões\\*
Educa o povo e distrai\\*
Pelas suas narrações,\\*
Além disso documenta\\*
Um fato que apresenta\\*
Atraindo as multidões.

Assim, vocês estudantes,\\*
Devem ir enriquecendo\\*
Lendo a literatura\\*
De cordel que vai crescendo\\*
Pois lá nos seus escritos\\*
Serve para os eruditos\\*
Seus temas desenvolvendo.

Os governos do Nordeste\\*
Já estão colaborando\\*
Nos festivais do Nordeste\\*
Que estão se realizando,\\*
Na Paraiba, Bahia\\*
E em Alagoas a poesia\\*
Vem muito se propagando.

Piaui e Ceará\\*
Maranhão, todo o Nordeste\\*
Da mesma forma no Sul,\\*
No Leste até o Oeste,\\*
Cordel tornou"-se família\\*
Se projetando em Brasília,\\*
E na região do Leste.

No Território do Acre\\*
O nosso cordel fascina,\\*
Poetas do Amapá\\*
Sua platéia dominam.\\*
Porque tem o seu papel\\*
De propagar o cordel\\*
Lá da terra nordestina.

R -- omances dos trovadores\\*
O -- s temas são divulgados\\*
D -- entro das nossas tevês\\*
O -- s seus casos bem narrados,\\*
L -- ivros bons de folcloristas\\*
F -- alam sobre os cordelistas\\*
O -- s seus nomes consagrados.
\end{verse}

\chapter{Antônio Conselheiro}

\begin{verse}
Fim do século dezoito\\*
Na Bahia apareceu\\*
Um pregador cearense\\*
Que dizia: Quem sou eu?...\\*
-- Sou o Emissário Divino\\*
Salvador do nordestino\\*
Que ouve o conselho meu.

-- Trago a palavra de Deus\\*
Que é a Espada da Verdade.\\*
-- Sou o Caminho daquele\\*
Que deseja a Eternidade.\\*
Seja moço ou seja velho\\*
Ouvindo meu Evangelho\\*
Tem que fazer Caridade.

-- Só peço aos meus seguidores\\*
Que cuidem da Salvação\\*
Vendendo tudo que tem\\*
E entregando ao seu irmão,\\*
O Reino já se aproxima,\\*
Quem não olhar para cima\\*
Fica infincado no chão.

-- Caso e batizo de graça\\*
Não pago imposto também\\*
Porque a terra é de Deus\\*
Não pertencendo a ninguém...\\*
Construirei a Cidade\\*
Que se chama, na verdade:\\*
A ``Santa Jerusalém''

Era Antonio Conselheiro\\*
Um exótico pregador\\*
Que arrebanhava gente\\*
Quase em todo interior,\\*
A sua estranha Doutrina\\*
Se chamava ``\textsc{ordem divina}''\\*
Sendo ele o Salvador.

Guerreava contra Padres,\\*
Prefeitos e Coletores\\*
E à frente da multidão\\*
Doutrinava os pecadores,\\*
Pelo seu verbo inflamado\\*
Dizia ser Enviado\\*
De Jesus e a Mãe das Dores.

Em toda Aldeia que ele\\*
Passava ia construindo\\*
Capelas e mais capelas\\*
Com a multidão seguindo...\\*
Dessa forma era o retrato\\*
Do mais afoito Beato\\*
Que do Norte tinha vindo.

Vestia ele uma túnica\\*
Grosseira de azulão,\\*
De cabeça descoberta\\*
Apoiado num bastão,\\*
Barbas brancas e crescidos\\*
Seus cabelos, parecidos\\*
Semelhantes de Sansão.

Era Antonio Conselheiro\\*
Cearense foragido\\*
Que tinha sido Mascate,\\*
Coletor reconhecido,\\*
Devido um crime de morte\\*
Mudou ele até de porte\\*
Para não ser conhecido.

Conviveu com Padre Cícero\\*
Ouvindo suas Pregações,\\*
No Juazeiro do Norte,\\*
Aprendeu fazer sermões,\\*
Daí teve um tino novo\\*
De catequizar o povo\\*
Nas agrestes regiões.

Ao depois que Conselheiro,\\*
Catequizou muita gente\\*
Começou a agredir Padres,\\*
Governos e Presidente,\\*
No lugar que ele passava\\*
Ninguém imposto pagava\\*
Nem mesmo um tostão somente.

O estranho Missionário\\*
Na sua ``\textsc{santa missão}''\\*
Espalhava o fanatismo\\*
Prometendo Salvação.\\*
Pai de Família empregado\\*
Por ele catequizado\\*
Se juntava à multidão.

Pelo Cooperativismo\\*
Os fanáticos viviam,\\*
Trabalhavam sem salários,\\*
O que ganhavam comiam,\\*
Se conseguissem dinheiro\\*
Entregavam ao Conselheiro\\*
Assim nada possuíam.

Quem seguisse o Pregador\\*
À casa não mais voltava,\\*
Deixava mulher e filhos,\\*
De uma vez se separava...\\*
Era Conselheiro o Amigo\\*
E o mais era Inimigo\\*
Certo de que se salvava.

Era caso de Polícia\\*
O modo do Conselheiro,\\*
Pois já virava anarquia\\*
Contra o País brasileiro,\\*
Foi o Governo ciente\\*
Do Pregador insolente\\*
Contra um povo tão ordeiro.

O Doutor Luís Vianna\\*
Governador da Bahia\\*
Relatou ao Presidente\\*
Tudo quanto ele sabia,\\*
Fanáticos municiados\\*
Assassinavam soldados\\*
À qualquer hora do dia.

Começou em setenta e quatro\\*
O Conselheiro pregando\\*
Construindo suas Igrejas\\*
E ao povo doutrinando.\\*
Porém em noventa e quatro\\*
O sertão virou teatro\\*
Do conflito mais nefando.

Cumba virou um inferno\\*
Igualmente Cansanção.\\*
Canudos a Monte Santo\\*
Eram cidades do Cão,\\*
O Rio Vesa"-Barris\\*
Era o roteiro infeliz\\*
Da tragédia do sertão.

Soldados de Alagoas,\\*
De Sergipe e da Bahia\\*
Sofriam dentro do mato\\*
Com pesada Infantaria,\\*
Era a guerra dos lunáticos\\*
Onde Jagunços fanáticos\\*
Até o sangue bebiam.

De Chorrochó à Uauá\\*
Os Jagunços de emboscada\\*
Assassinavam soldados\\*
Até mesmo de pedrada,\\*
Nos arvoredos ocultos\\*
Sorriam, faziam insultos\\*
Em histérica gargalhada.

Tornou"-se o Inferno de Dante\\*
A guerra mais fratricida\\*
Do Nordeste brasileiro\\*
Outra não acontecida,\\*
Os jagunços na procela\\*
Comendo urtiga e favela\\*
Perdiam o gosto da vida.

Era o sofrer das caatingas,\\*
Dos mandacarus talhados\\*
Pelos golpes dos facões\\*
Entre jagunços, soldados,\\*
Os fanáticos matreiros\\*
Trepados nos oitizeiros\\*
Eram sagüins enraivados.

Os soldados pracianos\\*
Apenas com munição\\*
Não conheciam os segredos\\*
Dos mistérios do sertão,\\*
Por detrás da macambira\\*
Jagunços faziam mira\\*
Ouvindo o tombo no chão.

Foi na serra da Cambaia\\*
O quadro desolador\\*
De soldados e jagunços\\*
Que não houve vencedor,\\*
Em toda extensão da serra\\*
Foi a Batalha da guerra\\*
Que causou maior clamor.

Qualquer Tropa do Governo\\*
Contra os jagunços perdia,\\*
Porém em ``Taboleirinho''\\*
Com renovada energia\\*
A Tropa vence a Batalha\\*
E a jagunçada se espalha\\*
Pois do contrário rendia.

Brava e forte Artilharia\\*
Nova luta começou\\*
E em Bendegó de Baixo\\*
A Cobra nova fumou...\\*
Nessa Batalha Gigante\\*
Foi a Tropa triunfante\\*
Conselheiro recuou.

Finalmente em Monte Santo\\*
Houve nova mortandade\\*
Com baixas de parte a parte\\*
De maior atrocidade,\\*
Cada qual tinha o seu jogo,\\*
Foram cinco horas de fogo\\*
Sem saciarem a vontade.

Coronel Moreira César\\*
Heroicamente lutou\\*
Noutro Combate cerrado\\*
Porém alguém o alvejou...\\*
Ferido disse sorrindo:\\*
Tome conta Tamarindo\\*
Que minha hora chegou.

Dia cinco de outubro\\*
Do ano mil e oitocentos\\*
Noventa e Sete Canudos\\*
Entre os mares de tormentos\\*
O sangue ensopando a terra\\*
Terminou a triste guerra\\*
Que só causou sofrimentos.

Conselheiro estava morto\\*
Por sofrer tantos reveses\\*
Porém morreu como Líder\\*
Nos momentos mais cruéis,\\*
Foi ele um Santo Guerreiro\\*
Que teve o fim derradeiro\\*
Morrendo pelos fiéis.

Para concluir, leitores,\\*
Foi Antonio Conselheiro\\*
Um Bravo, um Herói, Fanático,\\*
Um cidadão brasileiro\\*
Que seria premiado\\*
Se ele lutasse ao lado\\*
De um ideal verdadeiro.

R -- uiu todo misticismo\\*
O -- nde a falsa pregação\\*
D -- issipou milhares de vidas\\*
O -- bscurecendo o sertão ...\\*
L -- ivre Deus"-Pai Verdadeiro,\\*
F -- indo Antonio Conselheiro\\*
O -- utro não apareça, não.
\end{verse}

\chapter{O boi que falou no Piauí}

\begin{verse}
Deu"-se o caso em Jaicós\\*
Que do Piauí é estado\\*
Onde um grande fazendeiro\\*
Era ali admirado\\*
Devido ao que possuía\\*
E a sua grande alegria\\*
Era ter bastante gado.

Mais de 20 mil cabeças\\*
De gado, com exatidão\\*
Tinha o coronel Praxedes\\*
Espalhadas no sertão\\*
Comprava Zebu de raça\\*
E levava para a praça\\*
Quando havia exposição.

Zebu de 30 milhões\\*
O coronel possuía\\*
``\textsc{guzerath}'' e ``\textsc{holandês}''\\*
Ele bastante vendia\\*
Sem usar de contrabando\\*
Mandava vir da Holanda\\*
Gado, quase todo o dia.

Um certo dia chegou\\*
Um cidadão estrangeiro\\*
Trazendo da Dinamarca\\*
Um boi bastante fagueiro\\*
Parece que por adredes\\*
Foi ao Coronel Praxedes\\*
Oferecer"-lhe primeiro.

Mil e cem quilos o boi\\*
Na realidade pesava\\*
Trazendo dois empregados\\*
Ambos o ``gringo'' pagava\\*
Para o tal boi ser tratado\\*
Com o máximo de cuidado\\*
Aonde o boi se instalava.

Era um boi de duas cores\\*
De uma pequena estatura\\*
Porém, devido a seu peso\\*
Era de uma enorme grossura\\*
Com seus olhares serenos\\*
Tinha dois chifres pequenos,\\*
De uma bonita figura.

Perguntou o coronel\\*
Ao dito dinamarquês:\\*
Quanto pede pelo boi\\*
Me responda de uma vez?\\*
Peço cinqüenta milhões\\*
Menos não aceito frações\\*
Se quer ganhar um freguês.

Disse o coronel: --  Dou trinta\\*
Pois, minha vontade aumenta\\*
Levá"-lo na exposição\\*
Já que o amigo me tenta...\\*
Tornou o dinamarquês:\\*
--  Para findar de uma vez...\\*
Me passe logo \textsc{quarenta}!

--  Está feito --  respondeu\\*
O coronel animado.\\*
Passou o cheque e mandou\\*
Levar o boi ao cercado.\\*
Numa véspera de São João\\*
Houve uma exposição\\*
Em Teresina, de gado.

Quando o boi do fazendeiro\\*
Chegou nessa exposição\\*
Recebeu quatro medalhas\\*
Cada qual de um milhão\\*
Entre os congêneres mil\\*
Foi ele em todo o Brasil\\*
Verdadeiro campeão.

Voltando da exposição\\*
O tal boi descomunal\\*
Ao chegar em Jaicós\\*
Penetrou no matagal\\*
Não houve nenhum vaqueiro\\*
Que mostrasse ao fazendeiro\\*
Nem o rastro do animal.

Porém o preto Zé Telles\\*
Um certo dia avistou\\*
O tal boi numa Santa Cruz\\*
Bem perto dele encostou\\*
E ouvindo o boi falando\\*
Foi o diálogo escutando\\*
Tudo ao Coronel contou.

Disse o boi: --  Não adianta\\*
Ter"-se bonita figura\\*
Para dar medalha ao dono\\*
Que tem a mente obscura\\*
Pois ele com seus cruzeiros\\*
Vai comprar meus companheiros\\*
Prá levá"-los à tortura!

--  É raro o dia que ele\\*
Não leva pro Matadouro\\*
Meus irmãos e minhas irmãs,\\*
Come a carne e vende o couro,\\*
Outros numa vida tirana\\*
Vivem carregando cana\\*
Só para encher"-lhe de ouro. --

Surge um cavalo que estava\\*
Perto do boi escutando\\*
E diz --  Olá companheiro\\*
O que é que estás reclamando? --\\*
O boi de testa caída\\*
Disse: --  Lamento a minha vida\\*
Não estou mais suportando --

--  Bebem o leite de mamãe\\*
Pegam o meu pai assassinam...\\*
Quando a gente é muito gordo\\*
Que ao mundo inteiro fascina\\*
Nos levam para a Exposição\\*
Porém quem ganha é o patrão\\*
Se enchendo de ``granolina''.

--  Aproveitas o teu tempo\\*
Enquanto estás cobiçado\\*
(Disse o cavalo tristonho),\\*
Já fui bem gordo e cevado\\*
Hoje vivo padecendo\\*
Meus ossos aparecendo\\*
Pelo patrão desprezado.

--  Quando era forte e sadio\\*
Comia gorda ração:\\*
Milho, capim, rapadura,\\*
A mandado do patrão:\\*
Hoje cheio de bicheira\\*
Não há ninguém que me queira...\\*
Homem não tem coração! --

--  Retruca o boi: --  É por isso\\*
Que eu não quero mais saber\\*
Dos agrados do patrão\\*
Resolvi me defender.\\*
Vou virar"-me endiabrado\\*
Porém, aquele malvado\\*
Nunca mais há de me ver.

Nisto chegou um Carneiro\\*
Que pastava ali ao lado\\*
E disse: --  Meus bons amigos\\*
Não vejo nada acertado\\*
Que Jesus Cristo os perdoe\\*
Principalmente o irmão boi\\*
Que vive mais revoltado.

O boi olhou pro Carneiro\\*
E tristonho respondeu.\\*
--  É porque seu sofrimento\\*
Não se compara com o meu\\*
Vejo o meu dono matando\\*
Aos meus irmãos maltratando\\*
Com o mau instinto seu. --

Disse o carneiro: --  Eu também\\*
Morrerei assassinado\\*
Porém é a lei do destino\\*
Para isso fui gerado\\*
Quando chegar o meu dia\\*
Mesmo sofrendo agonia\\*
Terei que morrer calado. --

Disse o cavalo: --  Carneiro\\*
Ouça a minha opinião:\\*
Eu acho que o irmão boi\\*
É que tem toda razão\\*
Eu tenho isso de cor:\\*
``Não existe dor maior\\*
Do que a dor da ingratidão!''

Falou o boi: --  É exato\\*
Meu prezado companheiro\\*
O homem além de ingrato :\\*
Tem instinto carniceiro\\*
Cria a gente pra matar\\*
Botando pra trabalhar\\*
Para ele o tempo inteiro. --

Disse o Carneiro: --  Me ouçam\\*
Por favor, por um momento...\\*
Vos peço que fiquei certos\\*
Que o nosso sofrimento\\*
É permitido por Deus\\*
Por isso os sofreres meus\\*
Eu morro, mas não lamento.

Somente sofrem aqueles\\*
Pela lei determinada\\*
Pois o sofrimento traz\\*
Nossa alma lapidada\\*
Nisto explica a Evolução\\*
E um dia o nosso patrão\\*
Trilhará em outra estrada.

--  Que filosofia é esta?\\*
(Disse o boi com emoção)\\*
Quase não estou compreendendo\\*
Esta sua explicação...\\*
Pro meu patrão sou tesouro\\*
E amanhã me tira o couro,\\*
Qual a lei da gratidão?

Disse o Carneiro: --  Vocês\\*
Trilham num roteiro errado\\*
Se o homem é criminoso\\*
Ele por si é julgado\\*
Quem é vítima, depois\\*
Há de ver o próprio algoz\\*
Por seu crime condenado! --

Disse o cavalo: --  Gostei\\*
Desta sua explicação\\*
Vou terminar os meus dias\\*
Sem odiar meu patrão\\*
Se hoje vivo maltratado\\*
Porém mió serei julgado\\*
Por crime de ingratidão!

Chegando aos pés do Cruzeiro\\*
O boi para o céu falou:\\*
--  Desejo morrer agora\\*
Como Deus determinou\\*
Mesmo sem dizer um ai,\\*
Muito obrigado meu Pai! --\\*
E pra Fazenda voltou.

Contou o vaqueiro ao patrão\\*
Todo o diálogo que ouviu\\*
Mas ninguém acreditou\\*
Até Praxedes sorriu .\\*
Eu não sei se isto é lenda\\*
Mas dizem que da Fazenda\\*
O vaqueiro se sumiu.

Se o tal boi do Piauí\\*
Na verdade não falou\\*
Que isto sirva de fábula\\*
Como o trovador narrou.\\*
Deus perdoe quem é marchante\\*
E a \textsc{rodolfo cavalcante}\\*
O poeta que versou.
\end{verse}

\chapter[O encontro de Rodolfo Cavalcante\\ com Lampião Virgulino]{O encontro de Rodolfo Cavalcante com Lampião Virgulino}

\begin{verse}
Com 15 anos de idade\\*
Viajando com um irmão\\*
Pelo sertão de Alagoas\\*
Encontrei com Lampião\\*
E mais 30 cangaceiros\\*
Que pareciam vaqueiros\\*
Com os seus rifles na mão.

Eu saí de Pão de Açúcar\\*
Com Ari acompanhado,\\*
Viajava o mano a pé\\*
Enquanto eu ia montado,\\*
Quase me ardendo de febre\\*
Eu avistei um casebre\\*
Por cangaceiros cercado.

Ari puxou o jumento\\*
Para fora da estrada\\*
E ficamos escutando\\*
Gemidos e gargalhada,\\*
Quando todos deram fora\\*
Passando quase uma hora\\*
Foi que tomamos chegada.

Encontramos u'a senhora\\*
Com o seu dedo anular\\*
Decepado pelo meio\\*
Com o sangue inda a jorrar...\\*
Chorava como criança,\\*
Tinham levado a aliança\\*
E o chefe do seu lar.

Disse ela que o marido\\*
Que sofria de sezão\\*
Foi obrigado, coitado,\\*
Viajar com Lampião\\*
Para ensinar o roteiro\\*
Da casa de um Fazendeiro\\*
Lá daquela região.

Ao depois de confortá"-ta\\*
Eu com o mano saímos,\\*
Em direção à Santana\\*
De Ipanema nós seguimos,\\*
Viajamos com receio\\*
Quando um novo quadro feio\\*
Lampião bem perto vimos.

Debaixo de uma ingazeira\\*
Lampião se descansava\\*
Junto com seus cangaceiros,\\*
Porém um que se encontrava\\*
De vigia observando\\*
Foi depressa nos levando\\*
Aonde o ``Capitão'' estava.

Estava o homem da vítima\\*
Que deixei atrás chorando\\*
Sentado, muito tristonho,\\*
Talvez no seu lar pensando,\\*
E Lampião bem gentil\\*
Pegado no seu fuzil\\*
Com u'a mulher conversando.

O ``cabra'' que nos levou\\*
Parecia ser um mudo,\\*
Nos olhava atravessado\\*
Como quem fazia estudo\\*
E à frente de Lampião\\*
Foi dizendo: --  Capitão\\*
Agora resolva tudo!...

Quando ouvi essa conversa\\*
Todo meu corpo tremeu,\\*
Meu irmão mudou de cor\\*
Coitado, empalideceu,\\*
Rezei a ``\textsc{salve rainha}''\\*
Sabendo que mais farinha\\*
Nunca mais comia eu...

Lampião todo fechado\\*
O meu nome perguntou\\*
E o nome do meu irmão\\*
Do meu mano se informou,\\*
De onde vinha para onde ia\\*
E se eu de fato sabia\\*
Onde a ``\textsc{volante}'' ficou.

Respondi"-lhe: --  ``Capitão\\*
Nós somos pobres artistas,\\*
Viemos de Pão de Açúcar...\\*
Fixando bem suas vistas\\*
Para nós, sério, falou"-nos,\\*
Com bons modos perguntou"-nos:\\*
--  Vocês são Malabaristas?...

--  ``Não senhor, lhe respondi,\\*
Com Bonecos trabalhamos,\\*
Quebramos pedras nos peitos\\*
E a nossa cara melamos'',\\*
Lampião pensou... pensou...\\*
Dois minutos demorou\\*
Ao depois que nós falamos.

Disse ele: --  ``Pra aonde vão\\*
Fazendo esta jornada?...''\\*
--  Nós vamos para \textsc{santana}\\*
Se é que esta é a estrada...\\*
Disse: ``Vai no rumo certo,\\*
Daqui lá não é tão perto...\\*
Três horas de caminhada!...''

``Escutem bem --  vocês: dois\\*
Quando em Santana chegar\\*
Não vão dizer que me viu,\\*
Cuidado o que vão falar...\\*
Eu só aviso uma vez,\\*
E se disserem, vocês\\*
Não poderão escapar.''

Eu falei: --  Muito obrigado\\*
Capitão, nada diremos. --\\*
Lampião disse: --  Descansem...\\*
Nisso o animal soltamos,\\*
Meia hora demoramos\\*
E ao depois nós rumamos\\*
Ao destino que escolhemos.

Em Santana do Ipanema\\*
Chegamos no mesmo dia,\\*
Com o Coronel Lucena\\*
Eu pedi a garantia\\*
Para poder trabalhar,\\*
Sem nada lhe revelar,\\*
Pois se dissesse morria ...

Felizmente que a nós\\*
Nada ninguém perguntou.\\*
Trabalhamos quatro dias\\*
Que todo o povo gostou,\\*
Seguimos para Cascão --\\*
Pequena povoação --\\*
Que um cidadão nos chamou.

À noite quando o Salão\\*
Estava superlotado\\*
Já no meio do espetáculo\\*
Um cidadão, apressado,\\*
Com força à porta bateu,\\*
Pelo recado que deu\\*
Deixou o povo assombrado.

Disse ele, que Curisco\\*
Vinha o espetáculo acabar,\\*
Nesta hora todo o povo\\*
Começou a se agitar,\\*
Era mulher desmaiando,\\*
Rapazes, moças gritando,\\*
Toda criança a chorar.

Com dez minutos depois\\*
Naquela povoação\\*
Não ficou uma criatura...\\*
Até eu e meu irmão\\*
Dentro do mato dormimos\\*
E de manhã nós seguimos\\*
Para outra direção.

Eu não soube se o Curisco\\*
O lugarejo atacou,\\*
O certo é que em Águas Belas\\*
Um cidadão me informou,\\*
Era verdade a notícia\\*
Que Lampião à Polícia\\*
Um tiroteio travou.

Muita gente afirma, que\\*
Lampião era ruim,\\*
Eu não posso condená"-lo\\*
Porque nada fez a mim,\\*
Ao contrário: a mim e ao mano\\*
Deu a prova de humano\\*
Como cavalheiro enfim.

Lampião pelo destino\\*
Na senda triste vivia\\*
Devido às circunstâncias\\*
De tanta selvageria\\*
Pela injustiça da era,\\*
Se tornou uma Besta"-Fera\\*
Pelas mortes que fazia.

Muitas coisas que se narram\\*
Em nome de Lampião\\*
São maldades inventadas\\*
Como lendas do sertão,\\*
Muitos dramas de fazendas\\*
Mais da metade são lendas\\*
Da própria imaginação.

Foi Virgulino Ferreira\\*
Pobre homem injustiçado\\*
E por isto vingativo\\*
Se tornou um celerado,\\*
Se a justiça fosse reta\\*
Nem jornalista ou poeta,\\*
O teria decantado.

Lampião era um bom filho\\*
Nunca se pode negar,\\*
Foi também bom companheiro\\*
Como pode se provar\\*
No epílogo da desdita\\*
Junto a Maria Bonita\\*
Os seus dias foi findar.

O homem por mais cruel\\*
Tem seu lado positivo\\*
Por ser Centelha Divina,\\*
E enquanto ele for vivo\\*
Há esperança divina\\*
Se su'alma cristalina\\*
Ter um rumo objetivo.

Embora sendo criança\\*
Com meus 15 anos de idade\\*
Pude ver em Lampião\\*
Vítima da sociedade,\\*
Talvez ele em outro meio\\*
(Posso dizer sem receio)\\*
Era útil à humanidade!

Não vi a hiena feroz\\*
Nem o leão devorador,\\*
Ao contrário: vi um ser\\*
Realmente sofredor\\*
Que só andava assustado\\*
E para não ser caçado\\*
Matava seja quem for...

Vamos mudar a imagem\\*
Do lendário Lampião,\\*
Morreu o analfabetismo\\*
O flagelo do sertão,\\*
Que foi o grande fator\\*
Do drama desolador\\*
Do temível Capitão.''

Essa história é verdadeira\\*
Fica como um comprovante\\*
Que Lampião Virgulino\\*
Teve o seu lado brilhante,\\*
Se não foi um benfeitor\\*
Não matou o trovador\\*
\textsc{rodolfo c. cavalcante}.
\end{verse}

\chapter[Cuíca de Santo Amaro -- O poeta popular que conheci]{Cuíca de Santo Amaro\break O poeta popular que conheci}

\begin{verse}
Na ``Baixa dos Sapateiros''\\*
Entrada do Taboão,\\*
Eu e Cuíca fizemos\\*
Um pacto na profissão,\\*
Quando um morresse primeiro\\*
Versaria o derradeiro\\*
A História do seu irmão.

Mil Novecentos e Quarenta\\*
Conheci a vez primeira\\*
Cuíca de Santo Amaro\\*
Vendendo Livros na feira,\\*
``O Homem que inventou\\*
o trabalho'' ele versou\\*
De engraçada maneira.

Quarenta e Cinco voltando\\*
A morar em Salvador\\*
Cuíca já tinha fama\\*
Na vida de Trovador,\\*
Quando ele ainda novo\\*
Era o Poeta do Povo\\*
Popular versejador.

Era baiano da gema\\*
Pois em Salvador nasceu,\\*
Se chamava José Gomes,\\*
Porém o apelido seu\\*
Ficou sendo muito claro:\\*
``Cuíca de Santo Amaro''\\*
Nome que ele mesmo o deu.

Dizem que ele muito moço\\*
Tinha como distração\\*
Fazer belas serenatas\\*
Ao som de um violão\\*
Quando em Santo Amaro ia,\\*
Dest'artes se distraía\\*
Atraindo multidão.

Mil novecentos e nove\\*
Na Castanheda nascia,\\*
No Centro da Capital,\\*
A nossa amada Bahia,\\*
Abraçou a profissão\\*
Perante a população\\*
No ramo da Poesia.

Tinha o Signo de ``Peixe''\\*
Porque em março nasceu,\\*
Dia dezessete, portanto,\\*
Como trovador viveu,\\*
Teve fama, teve glória\\*
E na sua trajetória\\*
Na vida também sofreu.

Todo caso que se dava\\*
Na Capital da Bahia\\*
Ou mesmo no interior\\*
Cuíca logo sabia,\\*
Não versava ele boato\\*
E sim o concreto fato\\*
Que o jornal confirmaria.

Nunca levantou um falso\\*
Para fazer sensação,\\*
Só versava o que dava\\*
Da Capital ao sertão,\\*
Era um Jornal do Estado\\*
Que o povo tinha cuidado\\*
De ler sua narração.

Muita gente da política\\*
De Cuíca tinha medo,\\*
Pois o seu estro era fogo\\*
Que derretia rochedo,\\*
Quem mal ato praticasse\\*
Por certo que esperasse\\*
Que terminava o segredo.

Normalistas que vivessem\\*
Certas ``Casas'' freqüentando\\*
Tivessem muito cuidado\\*
Pois Cuíca até sonhando\\*
Se soubesse da safadeza\\*
Um folheto, com certeza,\\*
Ia logo publicando.

Muita gente receosa\\*
Temia do rimador,\\*
Entretanto, tinha ele\\*
Muita gente ao seu favor,\\*
Até os homens do Cais\\*
Já não compravam jornais\\*
Para ler o Trovador.

Cinco horas da manhã\\*
Com seus folhetos na mão\\*
Cuíca estava na ``Leste''\\*
Gritando na Estação,\\*
E dentro dos trens mercando\\*
Ia os Títulos anunciando\\*
Chamando o povo atenção.

``A Negra que se casou\\*
Com um americano cortês\\*
Que ensinou ela falar\\*
O Castelhano e o Inglês\\*
Teve um filho diferente,\\*
Não é bicho e nem é gente\\*
Que nasceu todo pedrês''.

``A história do Veado\\*
Que matou o Caçador''\\*
Ou senão: ``Mulher de Brotas'',\\*
``Falta Pão em Salvador'',\\*
``Misérias da Circular''\\*
E ``Um Padre Popular''\\*
Que está morrendo de Amor''.

Toda Obra de Cuíca\\*
Chamava ó Povo a atenção,\\*
Já tinha fregueses certos\\*
Do ``Lacerda'' ao ``Taboão'',\\*
Do ``Terreiro'' à ``Barroquinha'',\\*
De Salvador à Serrinha\\*
E em todo o nosso sertão.

Uns oitocentos folhetos\\*
Cuíca em vida escreveu,\\*
Como Poeta"-Repórter\\*
Outro igual não apareceu,\\*
A Bahia tem saudade\\*
Do Poeta da Cidade\\*
Que pouco tempo viveu.

Se ele pecava na métrica\\*
Seus versos tinham sabor\\*
Da própria alma do Povo,\\*
Por isso que o trovador\\*
Atraía a multidão\\*
Que até numa Eleição\\*
Quase foi Vereador.

Defendia ele os direitos\\*
Do povo quando explorado,\\*
Fazia críticas severas\\*
O que visse de errado,\\*
Combatia os ``Tubarões''\\*
Por todas explorações\\*
Que houvesse no Estado.

Quem não andasse direito\\*
Cuíca sentia o faro\\*
E o seu folheto saía\\*
Sem pedir segredo, é claro,\\*
E como ele não temia\\*
Na capa ele escrevia:\\*
``Cuíca de Santo Amaro''.

Os desenhos dos folhetos\\*
Sinézio Alves fazia\\*
E também os seus Cartazes\\*
Mostrando o fato do dia,\\*
Cuíca ficou famoso\\*
Como vate valoroso\\*
Que não bancou covardia.

Passou ele seus vexames\\*
Só por versar a verdade,\\*
Foi Cuíca em Salvador\\*
O Poeta da Cidade,\\*
Morreu pobre esquecido,\\*
De ninguém foi protegido,\\*
Hoje o povo tem saudade.

Era um Gregório de Matos\\*
Que em Salvador vivia\\*
Versando todos os fatos\\*
Do Estado da Bahia,\\*
Vivendo da inteligência\\*
Não temia a conseqüência\\*
Quando o folheto saía.

À tardezinha Cuíca\\*
Com seus folhetos na mão\\*
E com um embrulho de lado\\*
Pegava sua lotação\\*
Levava pros ``enchadinhos''\\*
Como chamava aos filhos,\\*
A sua alimentação.

Usava Fraque e Cartola\\*
Como hoje usa o ``Chacrinha''\\*
Camisa branca, engomada\\*
E na Cartola u'a peninha,\\*
Com o povo gracejava\\*
E todo mundo o estimava\\*
Pelo jeito que ele tinha.

Respeitava as famílias\\*
No lugar onde passava.\\*
Somente nos seus folhetos\\*
Cuíca pilheriava,\\*
Era assim que ele vivia\\*
Pelas ruas da Bahia\\*
No labor que tanto amava.

Era demais esforçado\\*
Na luta do ganha"-pão,\\*
Se alguém dele criticasse\\*
Nunca ligava atenção,\\*
Era um repórter inteirado\\*
Conhecido em todo o Estado\\*
Com sua pena na mão.

De manhã no Elevador\\*
Lacerda ele mercava\\*
Seus folhetos, com cartazes,\\*
E à tarde trabalhava\\*
Na ``Baixa dos Sapateiros''\\*
Levando os tabuleiros\\*
Todo freguês que passava.

Faleceu ele em janeiro\\*
De sessenta e quatro o ano,\\*
Tive a notícia em Itabuna\\*
Num matutino baiano,\\*
Perdeu a nossa Bahia\\*
Melhor repórter do dia,\\*
De um coração humano.

Era um chefe de família\\*
Concretamente exemplar,\\*
Cumpridor dos seus deveres,\\*
Zelando pelo seu lar,\\*
Morreu e nada deixou\\*
Porém o nome ficou\\*
Para o povo se lembrar.

A viúva de Cuíca\\*
Precisa ser amparada,\\*
Em memória do poeta\\*
De uma profissão honrada\\*
Isso já há muito devia\\*
Ter se feito na Bahia\\*
E a hora está chegada.

Cuíca, caro Colega,\\*
Nosso trato foi cumprido,\\*
Desculpe ter demorado\\*
Mas você não é esquecido,\\*
Deus o tenha em sua glória\\*
E receba esta história.\\*
Conforme meu prometido.
\end{verse}

\chapter[A mulher que foi surrada pelo Diabo]{A mulher que foi surrada\\ pelo Diabo}

\begin{verse}
Eu vou narrar uma história\\*
Que Hermes Gomes contou\\*
Cuja moral do enredo\\*
A muita gente abalou;\\*
Fica o caso como exemplo\\*
Que o lar é o maior templo\\*
Que Deus no mundo deixou.

Disse Hermes que vivia\\*
Um casal conceituado\\*
Numa cidade que o nome\\*
Agora não estou lembrado,\\*
O esposo um lutador,\\*
Cidadão trabalhador\\*
Por todo povo estimado.

Tinha o casal três filhinhos\\*
Dois meninos e u'a menina,\\*
Nada faltava no lar\\*
Pois ele como carpina\\*
Era ativo interesseiro,\\*
Ganhava muito dinheiro\\*
Na sua própria oficina.

Nada faltava à mulher\\*
Tendo tudo que queria,\\*
Mas por sua vaidade\\*
Tudo desfez"-se num dia,\\*
Resolveu ela deixar\\*
O seu mais ditoso lar\\*
Para entregar"-se à orgia.

Dizia ela consigo:\\*
--  No meu lar não tenho nada,\\*
O meu marido é um santo\\*
Vivo bem como casada,\\*
Entretanto a minha vida\\*
Me deixa a alma oprimida\\*
Numa casa enclausurada.

Irei percorrer o mundo\\*
Para minha vida gozar\\*
Conhecer novas paragens,\\*
Beber, fumar e dançar,\\*
Ter a minha liberdade\\*
Para que a mocidade\\*
Possa melhor desfrutar.

Já chega de lavar roupa,\\*
Varrer casa e cozinhar\\*
Quero sair pelo mundo\\*
Somente para gozar...\\*
Sou moça, tenho saúde,\\*
Não perco minha juventude\\*
Escravizada num lar.

E assim ela pensando,\\*
Sua bagagem arrumou.\\*
Fez um bilhete ao marido\\*
E numa manhã viajou,\\*
Deixando os filhos dormindo,\\*
Sem o remorso sentindo\\*
Do esposo que deixou.

Quando o marido acordou"-se\\*
E logo o bilhete leu\\*
Chorou igual uma criança\\*
Pela esposa que perdeu,\\*
Sentindo n'alma os espinhos\\*
Foi cuidar dos seus filhinhos\\*
Os anjos que Deus lhe deu.

A mulher aventureira\\*
Quando da casa saiu\\*
Na primeira encruzilhada\\*
Um arrepio sentiu...\\*
Quando ela olhou para trás\\*
Foi avistando um rapaz\\*
Que lhe saudou e sorriu.

--  Para onde vai Sá Dona\\*
Com a mala carregando?...\\*
Disse ela alegremente:\\*
--  Da forma que estou pensando\\*
Irei ao mundo gozar,\\*
Pois abandonei o lar\\*
E agora vou viajando.

Esse moço era o Diabo\\*
Que respondeu para ela:\\*
--  Sá Dona tenha juizo,\\*
Cuidado, muita cautela,\\*
Que o mundo é duvidoso,\\*
Falso, vil e mentiroso,\\*
Cheio de toda procela.

--  Tome um conselho Sá Dona!\\*
Regresse para o seu lar,\\*
Tomar conta dos seus filhos\\*
Para amanhã não chorar...\\*
Reflita neste momento\\*
Antes do arrependimento\\*
Que um dia há de chegar.

Que nada! --  disse a mulher\\*
Eu já estou resolvida.\\*
Eu quero gozar no mundo\\*
Todos prazeres da vida...\\*
Cinco anos de casada\\*
Eu já estou enjoada\\*
De só viver oprimida!

--  Não faça isso Sá Dona...\\*
Seus filhos estão chorando\\*
E seu esposo inocente\\*
Está triste se acabando;\\*
Abrande seu coração\\*
Não use de ingratidão\\*
Para não findar penando.

Quem é você que me vem\\*
Dar conselhos que não quero\\*
Se dane com seus conselhos\\*
Que isso tudo é lero"-lero.\\*
Desejo minha liberdade\\*
Pra gozar minha mocidade\\*
E a ninguém eu considero.

O Satanás novamente\\*
Parecendo um bom cristão\\*
Mostrou"-a o quadro do mundo\\*
Em toda sua extensão,\\*
Porém a mulher teimosa\\*
Traiçoeira e vaidosa\\*
Não mudou de opinião.

Por fim disse o Satanás\\*
Da forma que aqui estou\\*
Um dia lhe esperarei.\\*
Diga que não me escutou,\\*
Quando a senhora sofrer\\*
Eu só peço não dizer:\\*
O Diabo me enganou!...

Dizendo isto o Satã\\*
Foi abrindo uma cancela\\*
E ao despedir"-se lhe disse\\*
Que esperaria ela,\\*
A mulher se admirou\\*
Quando defronte avistou\\*
Bonita casa amarela.

Deixando o Diabo ela\\*
Prosseguiu sua aventura\\*
Numa primeira cidade\\*
Gostou de uma criatura,\\*
Essa vivendo com ela\\*
Logo desgostou"-se dela\\*
Deixando"-a na amargura.

Empregou"-se numa casa\\*
De um ricaço Doutor\\*
Um dia um certo ladrão\\*
Roubou jóias de valor,\\*
Sendo ela suspeitada\\*
Foi logo trancafiada\\*
Sofrendo terrível dor.

Se vendo ela perdida\\*
No seu viver infeliz\\*
Com dois anos, justamente,\\*
Transformou"-se em meretriz,\\*
Depois de muito sofrer\\*
Veio ela compreender\\*
quem tem um lar é feliz.

Manchou"-se sua epiderme\\*
Desbotou sua feição\\*
Os seus lábios cor de rosa\\*
Ficaram como algodão,\\*
Os cabelos desgrenhados\\*
Deixaram ser ondulados\\*
Perdendo toda expressão.

Apareceram eczemas,\\*
O corpo todo chagando...\\*
Numa terrível megera\\*
Começou ela esmolando:\\*
Dormindo pelas calçadas,\\*
Nas chuvosas madrugadas\\*
Ia o corpo definhando.

Lembrou"-se ela do moço\\*
Da linda casa amarela\\*
Que tanto lhe aconselhou,\\*
Bem pertinho da cancela...\\*
Via seus filhos chorando\\*
E o esposo se lastimando\\*
Sentindo saudades dela.

Depois de muito sofrer\\*
Resolveu ela voltar\\*
Para implorar o esposo\\*
Tomar conta do seu lar\\*
E assim perambulando\\*
Saiu no mundo esmolando\\*
No mais tristonho penar.

Quando completou três anos\\*
Que ela no mundo sofreu\\*
Um dia viu a cancela\\*
Que o moço apareceu...\\*
Logo foi o avistando\\*
E tristemente chorando\\*
Contou"-lhe o que sucedeu.

Disse ela: --  Ai seu moço\\*
O mundo é mui diferente...\\*
O Diabo me enganou\\*
Pois nada estava ciente!\\*
Pensei gozar minha vida\\*
Hoje estou arrependida\\*
Do que fiz antigamente.

O Satanás nessa hora\\*
Tinha um chicote na mão,\\*
E logo se transformou\\*
Com rabo, chifre, esporão,\\*
E disse: --  Mulher danada\\*
Você que foi a culpada.\\*
Da sua situação!

--  Não se lembra desgraçada\\*
Quando aqui lhe avisei?...\\*
Como diz que o Diabo\\*
Lhe enganou?... eu lhe esperei\\*
Para dar"-lhe boa sova\\*
Que ficará como prova\\*
Que eu nunca lhe enganei!

Nisto o Diabo surrou"-a\\*
Que ela ficou estirada,\\*
Quando ela levantou"-se\\*
Viu que estava retalhada;\\*
Após o pobre marido\\*
Viu o seu corpo estendido\\*
Semimorta na calçada.

Ela abençoando os filhos\\*
Deu um suspiro e gemeu\\*
E nos braços do esposo\\*
Uma lágrima desceu,\\*
Foi a hora dolorosa\\*
Quando a mulher vaidosa\\*
Entre soluços morreu.
\end{verse}

\chapter{Tudo na terra tem fim}

\begin{verse}
Grande foi na Babilônia\\*
O rei Nabucodonosor\\*
Salomão --  Pai da Ciência\\*
Moisés --  Legislador,\\*
Davi, Isac, Abraão,\\*
Hoje debaixo do chão\\*
Prestam contas ao Senhor

Alexandre --  Vulgo ``O Grande''\\*
Também desapareceu,\\*
Napoleão Bonaparte\\*
Em Santa Helena morreu,\\*
Kaiser, Hitler, da Alemanha,\\*
Cada qual com sua façanha\\*
De dominar pereceu!

Herodes, Pôncio Pilatos,\\*
Com Anás e com Caifás.\\*
Crucificaram Jesus --\\*
Esses não existem mais,\\*
Júlio César até Tibério\\*
Também Nero, o Deletério,\\*
Morreram tempos atrás!

Quem mais fala em Mussolini\\*
Que a Itália dominou?\\*
Stálin, Marx, Lênin,\\*
Cada um se evaporou...\\*
Daniel, Abimeleque,\\*
Vítor Hugo, Allan Kardec,\\*
Nenhum na Terra ficou!

Aonde se encontram os Dumas\\*
Bons Escritores franceses?\\*
Voltaire, mesmo Balzac\\*
E os grandes portugueses\\*
Camilo, Guerra Junqueiro?\\*
Ficaram só no tinteiro,\\*
Morreram, por suas vezes!

Que é dos gregos da História\\*
Como Sócrates, Platão,\\*
Aristóteles e Demóstenes,\\*
E Aristóteles?... então\\*
Agora pergunto eu:\\*
--  Onde está Ptolomeu\\*
Senão debaixo do chão?

Que é da Rainha Cleópatra\\*
Soberana do Egito?\\*
Messalina --  a Dissoluta,\\*
Agripina em seu conflito\\*
Com Cláudio e o seu filho Nero,\\*
Se esvaíram eu assevero,\\*
Hoje estão no infinito!

Que é de Roosevelt, Kennedy,\\*
Degaulle, Churchill --  varões\\*
Que se tornaram famosos --\\*
Defensores de Nações?...\\*
Hoje já estão sepultados\\*
Embora glorificados\\*
Para todas gerações!

Onde está Getúlio Vargas\\*
Com seu riso triunfal,\\*
Cujo nome foi um ídolo\\*
De fama internacional?\\*
Por isso que digo assim:\\*
Tudo na Terra tem fim\\*
Neste plano material!

Que é dos velhos Faraós\\*
Que somaram mais de dez,\\*
Construindo suas Múmias\\*
Trazendo o povo aos seus pés?\\*
As Pirâmides ergui das\\*
Foram elas construídas\\*
Por mais de doze Ramsés!

Que é de Lao"- Tsé, Confúcio,\\*
Mactube, Zoroastro,\\*
Tiradentes --  nossa glória,\\*
Da Independência --  o Astro?\\*
Tudo desapareceu,\\*
Matusalém pereceu\\*
Que não deixou nenhum rastro!

Onde está Augusto Comte\\*
Com o seu Positivismo?\\*
Lutero e o próprio Calvino  --\\*
Luzes do Protestantismo?\\*
Krishna, Buda, Maomé,\\*
Só Jesus de Nazaré\\*
Que não ficou no abismo!

Que é da grande valentia\\*
Do famoso Lampião,\\*
Corisco, Maria Bonita --\\*
Bandoleiros do sertão?\\*
Tudo ficou no passado\\*
Cada qual exterminado\\*
Para servir de Lição!

Que é do nefando Nazismo,\\*
E o Fascismo do estrangeiro?\\*
Até mesmo o Comunismo\\*
O terror do mundo inteiro?...\\*
Uns estão dilacerados\\*
E outros repudiados\\*
Pelo Povo Brasileiro!

Aonde esta o Anarquismo\\*
Que há tempo se findou?\\*
E o idioma Aramaico?...\\*
Tudo, tudo terminou.\\*
E desta maneira assim\\*
Tudo na Terra tem fim\\*
Para o Além retornou!...

Que de Freud --  o Criador\\*
Do mais alto sexualismo\\*
Que na sua psicanálise\\*
Quis mostrar o sensualismo\\*
Até na idade infantil?\\*
Cujas teorias mil\\*
Não passaram de lirismo!

Que é do famoso Rostaing\\*
Com a sua filosofia\\*
Chamada ``Corpo Fluídico''?...\\*
A mais tola teoria...\\*
O notável Advogado\\*
Blasfemou em seu Tratado\\*
Contra o Filho de Maria!

O que fizeram os nossos\\*
Astronautas, lá na Lua?\\*
Gastaram tantos bilhões\\*
E a Terra continua\\*
Pela Lua Iluminada.\\*
Cuja missão encerrada\\*
No mundo inteiro flutua...

Onde estão Truman, Abraão\\*
Lincoln --  homens de valores?\\*
Todos morrem neste mundo:\\*
Democratas, Ditadores,\\*
Marechais e Cientistas,\\*
Nobres, plebeus e Artistas\\*
E os mais Sábios Pensadores.

Que é do grande Carlos Magno\\*
Oliveiras e Roldão\\*
Com outros Pares de França,\\*
Da mesma forma Sansão?\\*
Morre o fraco e morre o forte\\*
Ninguém escapa da morte\\*
Com seu ``Cutelo'' na mão!

Não adianta o orgulho,\\*
O egoísmo, a vaidade,\\*
O dinheiro, a opulência,\\*
O abuso de autoridade.\\*
Morre o bom, morre o ruim,\\*
Tudo na Terra tem fim\\*
É a Lei da Divindade!

Morre a árvore mais frondosa,\\*
Morre o rio, morre o outeiro,\\*
Morre a mulher que é bonita,\\*
Morre o homem do dinheiro,\\*
Morre quem faz tirania,\\*
Só não morre a Poesia\\*
Dada por Deus verdadeiro!

Tudo, tudo se transforma\\*
Já disse Lavoisier.\\*
Porém a morte é um sonho\\*
Que é difícil se entender...\\*
Sonho esse prolongado,\\*
Que oculta nosso passado\\*
Para depois reviver!

Tudo no Terra tem fim\\*
Como já deixei provado,\\*
Mas o espírito continua\\*
O qual será transformado\\*
Em outro ser diferente,\\*
Gerado de uma semente\\*
Pelo fruto do passado!

Há muitas filosofias\\*
De cada para escolher...\\*
Uma diz que a gente morre\\*
Depois volta renascer,\\*
Outra afirma que na hora\\*
Morre o corpo e se evapora\\*
E a alma ninguém vê.

Umas dizem que há inferno\\*
Já outros dizem que não,\\*
Uma existe o purgatório\\*
Para a purificação,\\*
Cada tem sua teoria\\*
Com própria filosofia\\*
Da mais pura ficção!

Por que morreu Joana D'Arc\\*
Queimada numa fogueira\\*
E ainda sendo taxada\\*
Como a pior feiticeira?...\\*
Se ela ao passado voltasse\\*
Via de Judas --  a face\\*
Sua imagem verdadeira!

O Circo de Niterói\\*
Que alguém fogo tocou\\*
Matando tantas crianças,\\*
Muita gente não pensou\\*
No grande incêndio de Roma\\*
Que foi a pior Sodoma\\*
Que o tempo não perdoou!...

A história se repete --\\*
Não é tola teoria.\\*
Tudo no mundo se acaba\\*
Mas a natureza cria\\*
Novo seres transformando,\\*
Na pureza condenando\\*
Pela Divina Energia!

Não existe retrocesso\\*
Pela Lei da Evolução,\\*
Ninguém escapa da morte,\\*
Mas a morte é ficção;\\*
Morte, sim, é a partida\\*
Para a verdadeira Vida\\*
Não há outra explicação!

``Amai"-vos Uns aos Outros''\\*
Toda grandeza se encerra.\\*
Nem o próprio Rockefeller\\*
Levou dinheiro da Terra...\\*
Para que ódio e vingança\\*
Apagando a esperança\\*
Do mundo fazendo guerra?

Tudo na Terra tem Fim\\*
É a pura realidade,\\*
Mas depois do fim começa\\*
A vida"-continuidade,\\*
Se um volta o outro fica,\\*
É assim que se explica\\*
A Lei de Deus, em verdade!
\end{verse}

\chapter[Paulista virou tatu viajando pelo metrô]{Paulista virou tatu\\ viajando pelo metrô}

\begin{verse}
A reportagem presente\\*
Desse \textsc{metrô} encantado\\*
Quem me deu foi um baiano --\\*
Hoje um paulista inteirado;\\*
Além de ser Jornalista\\*
É um grande Folclorista,\\*
Seu Nome: --  \textsc{franklin machado}

São Paulo modernizou"-se\\*
Numa mais alta expressão,\\*
O povo anda nas ruas\\*
Como se anda em avião,\\*
Sem querer ser puxa"-saco\\*
É a cidade sem buraco,\\*
Do \textsc{metrô}, do Minhocão!

Antigamente os pedestres\\*
Puseram a boca a gritar\\*
Dizendo que aos subúrbios\\*
Não podia viajar,\\*
Eram demais os distúrbios\\*
Os pobres que nos subúrbios\\*
Iam ao centro trabalhar.

Por causa disto o Governo\\*
Logo o \textsc{metrô} construiu\\*
Onze quilômetros percurso\\*
Que agora concluiu.\\*
E assim desta maneira\\*
Ninguém mais leva poeira\\*
Depois que o \textsc{metrô} abriu.

Antes era o Minhocão\\*
Que o pedestre viajava\\*
Vários quilômetros a pé\\*
Para ver se ele chegava\\*
A hora de seu serviço,\\*
O Minhocão nem por isso\\*
O resultado alcançava.

Paulista virou tatu\\*
Pelo \textsc{metrô} viajando\\*
Uma obra arquitetônica\\*
Da Engenharia falando,\\*
Desta forma o Engenheiro\\*
Realmente brasileiro\\*
Ao mundo está invejando.

Miguel Ferraz deu início\\*
Fazendo o primeiro trecho,\\*
Veio Miguel Colasuono\\*
E disse: O \textsc{metrô} não deixo,\\*
O Olavo Setúbal\\*
Na conclusão afinal\\*
Diz: Da obra não me queixo!

Começou com Prestes Maia\\*
Esse \textsc{metrô} tão falado,\\*
Ao depois foi Paulo Lauro\\*
Que mostrou todo o traçado,\\*
Rua abaixo, rua acima,\\*
Prefeito Faria Lima\\*
Deixou tudo planejado.

Dizem que São Paulo é\\*
O Motor desta Nação\\*
Com os 22 Vagões\\*
Que formam a Federação,\\*
S. Paulo não há regresso\\*
É a Capital Progresso,\\*
Do Brasil --  o Coração!

--  Creio que na minha gestão\\*
Diz Olavo alegremente --\\*
Esse \textsc{metrô} de concreto\\*
Cada dia vai pra frente.\\*
Quebro pedra, quebro broca,\\*
Viro Tatu ou Minhoca\\*
Mas dou paz a minha gente!

Os Governos de São Paulo\\*
Desde o tempo de Ademar,\\*
Sodré e Lauro Natel\\*
Com o Governo atual,\\*
Todos à obra ajudando\\*
Hoje o \textsc{metrô} transitando\\*
Desafoga o pessoal.

Merece nossos aplausos\\*
Plínio Assan --  o Presidente\\*
Do \textsc{metrô} que demonstrou\\*
Ser perito competente,\\*
Provando que o brasileiro\\*
É o melhor Engenheiro\\*
Deste mundo, atualmente.

Já disse Euclides da Cunha\\*
Que o ``sertanejo é um forte'' ...\\*
Os Candangos de Brasília,\\*
Os seringueiros do Norte,\\*
E os homens do Nordeste\\*
Chamados ``cabras da peste''\\*
Trocam a vida pela morte.

São Paulo --  Grande Cidade\\*
Pelo seu deslumbramento,\\*
Com a obra do \textsc{metrô}\\*
Findou"-se o engarrafamento,\\*
Seus habitantes transitam\\*
E jamais se precipitam\\*
A morrer qualquer momento!

São Paulo --  a primeira linha\\*
Do \textsc{metrô} já começou,\\*
De Santana a Jabaquara\\*
Há tempo que inaugurou\\*
Brevemente o suburbano\\*
Do começo ao fim do ano\\*
Dirá: --  Com este é que eu vou!

São Paulo tem Minhocão\\*
Para o povo transitar,\\*
O suburbano já pode\\*
Ir ao centro trabalhar,\\*
Se não quer o Minhocão\\*
Anda debaixo do chão\\*
Para o seu pão ir ganhar.

Esta obra gigantesca\\*
Da moderna Engenharia\\*
Breve terá Porto Alegre,\\*
Belo Horizonte e Bahia,\\*
Pois na verdade, leitor,\\*
Nossa velha Salvador\\*
Cresce o trânsito dia a dia.

Os nossos compositores\\*
Já entoaram canções\\*
Musicadas ao \textsc{metrô}\\*
Que agradaram multidões,\\*
Eu ao Paulista fiel\\*
Rimo o folheto em cordel\\*
Para levá"-lo aos sertões!

Salve São Paulo querido\\*
Que mais cresce dia a dia,\\*
Salve Venâncio --  o Poeta\\*
Benfeitor da Cantoria,\\*
Que ajuda aos Trovadores\\*
E aos meus irmãos Cantadores\\*
Com a sua fidalguia!

Salve o Paulista que honra\\*
O Trabalho que é o fator\\*
Do Progresso de um País\\*
Que é independente, leitor,\\*
São Paulo com seu orvalho\\*
Brilha a rosa do Trabalho\\*
Como Jardim do Amor!

O Paulista que é Paulista\\*
Não usa de utopia\\*
Pois é contrário a Maconha\\*
Que só destrói a energia\\*
Do brasileiro que é nobre,\\*
Do opulento ao mais pobre\\*
Nos ópios não se vicia!

Todo paulista de brio\\*
Tem orgulho de sua terra,\\*
Assim como o Nordestino\\*
Do asfalto ao pé da serra\\*
Com chuva solou orvalho\\*
Mete os peitos no trabalho\\*
Como morre até na guerra!

Se o povo de São Paulo\\*
Virou Tatu viajando\\*
No \textsc{metrô}, debaixo do chão,\\*
Porque não vive esperando\\*
Que o progresso ausente dele\\*
E é por isto que ele\\*
Virou tatu trabalhando.

Quando o \textsc{metrô} terminar\\*
Os seus concretos roteiros\\*
Cobrindo toda a Cidade\\*
Milhares de brasileiros\\*
Podem residir distante\\*
O \textsc{metrô} leva bastante\\*
Um bilhão de passageiros!

O \textsc{metrô} leva operários,\\*
Funcionários, Doutores,\\*
Verdureiros, ambulantes,\\*
Engraxates, Professores,\\*
Leva também Magistrados,\\*
Jornalistas e Soldados,\\*
Lavadeiras, Cantadores!

Leva rico e leva pobre,\\*
Leva judeus e franceses,\\*
Baianos e Cearenses,\\*
Espanhóis e Portugueses,\\*
Mineiros, pernambucanos,\\*
Capixabas, sergipanos,\\*
Alemães e Japoneses.

Juscelino Kubitschek\\*
O \textsc{metrô} já visitou\\*
E o Presidente Médici\\*
Nele um dia viajou,\\*
Ernesto Geisel --  presente\\*
Nosso Grande Presidente\\*
Agora em setembro andou.

É uma obra de arte\\*
Pelos seus encantos mil,\\*
Cujo trabalho arrojado\\*
Feito de modo viril\\*
Demonstra a grande energia\\*
Da Suprema Engenharia\\*
Do nosso amado Brasil!

Hoje o Rio de Janeiro\\*
Seu \textsc{metrô} está construindo,\\*
Breve construirá Recife\\*
Porque isso está sentindo,\\*
E Curitiba, em verdade,\\*
Já necessita a cidade,\\*
Breve o \textsc{metrô} será vindo.

Salvador --  velha Cidade\\*
Hoje já modernizada\\*
O trânsito de dia a dia\\*
Grita a pobreza coitada,\\*
É tanto engarrafamento\\*
Que o povo no sofrimento\\*
Leva hora malfadada.

Salve minha Pátria querida\\*
Meu Brasil verde e amarelo,\\*
Que é o Coração do Mundo!\\*
Por tudo que tem de belo,\\*
Tudo lhe encanta e fascina\\*
É a Santa Palestina\\*
Do mais puro e santo anelo!

R -- ealmente que o \textsc{metrô}\\*
O -- seu papel é importante,\\*
D -- progresso de uma cidade,\\*
O -- transporte que é constante,\\*
L -- eitores, muito obrigado,\\*
F -- olheto por mim versado:\\*
O -- \textsc{rodolfo cavalcante}.
\end{verse}

\chapter[O desencanto da moça que bateu na mãe\\ e virou cachorra]{O desencanto da moça que bateu na mãe e virou cachorra}

\begin{verse}
Já muitos leitores leram\\*
A história comovente\\*
Da Moça que Virou Cachorra\\*
Por se tornar insolente;\\*
Agora a linda donzela\\*
Do seu corpo de cadela\\*
Voltou a forma de gente.

Dona Matilde viveu\\*
O seu tempo de amargura\\*
Em ver sua própria filha\\*
Na mais horrenda figura.\\*
Deixando Helena a masmorra\\*
Deixou de ser u'a cachorra\\*
Para ser uma criatura.

Dona Matilde pediu\\*
Ao Divino Bom Jesus\\*
Da Lapa, que se a filha\\*
Alcançasse a Santa Luz\\*
De ser gente novamente\\*
Ia ser uma penitente\\*
Do Santo Varão da Cruz.

Prometia ao Bom Jesus\\*
Que faria uma Romaria,\\*
Saindo do Canindé\\*
Atravessando a Bahia\\*
Com sua filhinha ao lado\\*
Pra ver o Morro Sagrado\\*
Que ainda não conhecia.

E assim aconteceu\\*
Sua promessa pagou\\*
Porque Helena -- a cadela\\*
Há pouco desencantou.\\*
No dia 6 de agosto\\*
Todo terrível desgosto\\*
Para ela terminou.

Helena viu um velhinho\\*
Segurado num bastão,\\*
Quando partiu para ele\\*
O caridoso ancião\\*
Lhe disse, piedosamente,\\*
-- Helena, voltas a ser gente\\*
Findou a tua missão!...

Helena se aproximou...\\*
Quis falar, porém latia,\\*
Em vez de chorar uivava\\*
E assim nesta agonia\\*
O seu corpo de donzela\\*
Deu"-se um milagre, que ela\\*
Em si mesma não sentia.

Helena foi conhecendo\\*
O Padre Cícero Romão,\\*
E ao narrar sua história\\*
Foi pedindo o seu perdão.\\*
Disse o Santo Reverendo:\\*
-- Eu estou compreendendo\\*
Tua terrível aflição!...

Helena, foi Bom Jesus\\*
da Lapa, que me mandou\\*
Atender da tua mãe\\*
O que ela suplicou.\\*
Agora tu vais a pé\\*
Ao teu Santo Canindé,\\*
Pois Jesus te perdoou!

-- Tu tens bastante sofrido\\*
Em cadela transformada,\\*
Mas teu arrependimento\\*
É a prova detalhada\\*
Que precisas reviver\\*
Para o mundo conhecer\\*
Que um filho sem mãe é nada!

-- Em que lugar me encontro\\*
Meu Divino Protetor?...\\*
Disse o velhinho: -- É Inhambupe\\*
Bem perto de Salvador,\\*
Amanhã, muito cedinho\\*
Seguirás o teu caminho\\*
Porque Jesus é Amor!

Nisso o resto do velhinho\\*
Logo desapareceu ...\\*
E a imagem de Jesus\\*
Para ela apareceu.\\*
Tornou"-se o quadro mais lindo\\*
Quando Jesus foi subindo\\*
Entre as nuvens para o Céu!

Helena viu que estava\\*
Em donzela transformada.\\*
Debaixo de um oitizeiro\\*
Passou a noite acordada,\\*
Rogando à Virgem Maria\\*
Que amanhecesse o dia\\*
Para seguir sua jornada.

Rezou várias orações\\*
Também a Salve Rainha\\*
E ofereceu na intenção\\*
Da sua velha mãezinha.\\*
Pediu à Nossa Senhora\\*
De ser uma filha agora\\*
Na bondade que não tinha!

Fez uma cama de folha\\*
E de pedras travesseiro\\*
Meditando em sua vida\\*
Debaixo do oitizeiro...\\*
Na manhã do outro dia\\*
Deixou a velha Bahia\\*
Com destino ao Juazeiro.

Com oito dias depois\\*
Ela alcançou Petrolina,\\*
Com 20 chegou ao Crato\\*
Se banhando na neblina,\\*
Fez ela vários pernoites\\*
Se descansando nas noites\\*
Entre vales e campina.

Chegando no Juazeiro\\*
Do Padre Cícero Romão\\*
Visitou Nossa Senhora\\*
Das Dores, com devoção,\\*
Vendo seu passado morto\\*
Foi agradecer no horto\\*
Como era de obrigação.

Saindo do Juazeiro\\*
Do Norte fez a jornada\\*
Com destino ao Canindé\\*
Pra rever sua morada.\\*
Sua alegria era tanta\\*
Que parecia uma santa\\*
Caminhando na estrada.

Eram seis horas da tarde\\*
O sol ia descambando\\*
Nas serras do Ceará\\*
Com o seu matiz doirando,\\*
Nisto ela mais feliz\\*
Foi avistando a Matriz\\*
Com suas torres brilhando

Chegou na sua morada\\*
Pela mãe dela chamou\\*
Dona Matilde rezando\\*
Depressa se levantou.\\*
Foi a mais tocante cena\\*
Entre Matilde e Helena\\*
Quando a mocinha chegou.

-- Bênção minha mãe querida!\\*
Disse chorando a donzela\\*
Deus te abençoe minha filha!\\*
Foi respondendo a mãe dela.\\*
Ambas ali se abraçaram\\*
E todas duas choraram\\*
Como se vê na novela.

Tanto a velha como a filha\\*
Segundo a imprensa diz\\*
Saíram com a multidão\\*
Para a Igreja da Matriz\\*
Foi um ofício rezado\\*
Pelo povo e acompanhado\\*
Pelo Padre João Luiz.

Com um mês e quinze dias\\*
Depois que Helena chegou\\*
Ao lado da mãe querida\\*
Para a Lapa viajou,\\*
Ajudando na viagem\\*
Deu o prefeito a passagem\\*
Que todo mundo gostou.

Lá em Bom Jesus da Lapa\\*
Subiram até o \textsc{cruzeiro},\\*
Visitaram toda gruta\\*
Guiadas por um romeiro\\*
E assim com muita fé\\*
Regressaram ao Canindé\\*
No princípio de janeiro.

Fica o caso exemplo\\*
Para ser melhor gravado\\*
Do filho que não respeita\\*
O nome de mãe sagrado.\\*
Segundo diz a escritura\\*
Mãe é a Santa criatura\\*
Gerou o Verbo Encarnado.

Ditoso sempre é o filho\\*
Que obedece seus pais,\\*
Goza na vida ventura\\*
E não se perde jamais,\\*
Todo aquele que é bom filho\\*
Apresenta o santo brilho\\*
Do amor em seus sinais.

Helena foi castigada\\*
Pelo seu mau coração\\*
Ou talvez por não saber\\*
Que uma filha sem bênção\\*
É como um jardim sem flor\\*
Porque Mãe é o Santo Amor\\*
Da Divina Criação!

``Honra teu pai, tua mãe''\\*
Isto a Escritura nos diz,\\*
Por isto que o bom filho\\*
Na vida sempre é feliz,\\*
Porém quem desobedece\\*
Uma mãe sempre padece,\\*
Não podendo ser feliz!

Helena Matias Borges\\*
Vive hoje no seu lar\\*
Sustentando a pobre velha\\*
Que a tanto fez chorar,\\*
Trabalhado noite e dia\\*
Suplica a Virgem Maria\\*
Para nunca a lhe faltar.

Jesus -- Cordeiro do Mundo,\\*
Salvador do mundo inteiro,\\*
Dai a vista quem é cego\\*
Para enxergar o roteiro.\\*
Que todo filho educado\\*
Seja por Deus amparado\\*
Tendo saúde e dinheiro!

Toda mocinha que leu\\*
Este folheto versado\\*
Se mire no grande exemplo\\*
Do drama que foi passado,\\*
Seja sempre obediente,\\*
Que o porvir sempre é latente\\*
Diferente do passado.

R -- esta agora agradecer\\*
O -- s meus prezados leitores\\*
D -- ando a melhor preferência\\*
O -- s livros dos trovadores,\\*
D -- e Helena a ex"-cadela\\*
F -- ica aqui o exemplo dela\\*
O -- mais termino, senhores!
\end{verse}

\chapter[Gregório de Matos Guerra -- O Pai\\ dos Poetas Brasileiros]{Gregório de Matos Guerra\\ O Pai dos Poetas Brasileiros}

\begin{verse}
Bahia -- Mãe dos Poetas\\*
Que toda Poesia encerra,\\*
Terra que deu Castro Alves --\\*
O maior Gênio da terra.\\*
Também deu um Grande Vulto,\\*
Dos Poetas -- o mais Culto:\\*
Gregório de Matos Guerra.

No ano Mil e Seiscentos\\*
E Vinte e Três, na Bahia,\\*
Nasceu Gregório de Matos\\*
Guerra -- seu nome dizia\\*
Que era Vate corajoso,\\*
Inflamado, mavioso,\\*
Um Mestre na Poesia!

Pedro Gonçalves de Matos\\*
Era o seu progenitor\\*
E Dona Maria Guerra\\*
Era a mãe do Trovador;\\*
Filho de \textsc{matos} e \textsc{guerra}\\*
Só podia aqui na Terra\\*
Ser um leão no furor!

Desde a infância o Poeta\\*
Tinha amor pela leitura.\\*
Foi crescendo e aprimorou"-se\\*
Numa melhor estrutura;\\*
No idioma de Camões\\*
Fulguraram seus clarões\\*
Na alta literatura.

No ano de Trinta e Seis\\*
Foi a Coimbra estudar\\*
Onde graduou"-se em Leis\\*
De uma maneira sem par,\\*
Passando a sua pessoa\\*
A advogar em Lisboa\\*
Sem à Pátria regressar.

No ano Setenta e nove\\*
Resolveu vir à Bahia\\*
Onde começou viver\\*
Somente em Advocacia,\\*
Já nesse tempo Gregório\\*
De Matos era notório\\*
No setor da Poesia.

No ano de Oitenta e Quatro\\*
O velho bardo casou"-se,\\*
Ao ver Maria de Povos\\*
Por destino apaixonou"-se,\\*
Com pouco tempo casado\\*
Chegou a ser degredado\\*
Porque a ninguém dobrou"-se.

Era um Poeta do povo\\*
Por sua maneira nova\\*
De atacar ao adversário\\*
Mesmo abrindo a sua cova,\\*
Não dispensava a porfia\\*
Satirizando em poesia\\*
Tornou"-se Mestre na Trova!

Nunca Gregório de Matos\\*
Teve medo de ninguém,\\*
A palavra \textsc{covardia}\\*
Dele se ausentava além,\\*
Era mordaz numa crítica\\*
Por causa disso a Política\\*
Não lhe olhava tão bem.

Em decassílabo Gregório\\*
De Matos foi Pioneiro,\\*
Cuja métrica de verso\\*
Improvisa o Violeiro\\*
O ``Martelo Agalopado'',\\*
Sendo esse gênero criado\\*
Pelo Vate brasileiro!

Formado ele em Direito\\*
Foi um grande Advogado,\\*
Da Universidade Coimbra\\*
Tinha todo predicado,\\*
No brilho da inteligência\\*
Tinha a suma competência\\*
No Direito comprovado.

Não nos registra a História\\*
Que o Grande Trovador\\*
Fugisse de uma polêmica\\*
Até com Governador,\\*
Por esta mesma razão\\*
Sofreu a degradação\\*
Por ser forte opositor.

Por todo seu satirismo\\*
Deixava de ser fraterno\\*
Quando sentia que o erro\\*
Jamais deve ser eterno,\\*
Por sua verve inflamada\\*
Foi sua pessoa chamada\\*
Como ``Boca do Inferno''.

Jamais Gregório de Matos\\*
Inclinou"-se ao ateísmo,\\*
Nem tão pouco alimentou\\*
A idéia do anarquismo,\\*
Sua lira era um vulcão\\*
Porém o seu coração\\*
Era à qualquer extremismo.

Até dos próprios amigos\\*
O poeta criticava\\*
Quando havia precisão\\*
E em trova o erro mostrava,\\*
Por esta forma a política\\*
Não suportando sua crítica\\*
Cada vez mais o odiava.

Como Homem do Direito\\*
Defendia a Liberdade,\\*
Porém a rima da prosa\\*
Penetra com gravidade,\\*
Desta maneira o degredo\\*
Por causa de tanto enredo\\*
Sofreu a penalidade.

A Academia de Letras\\*
Do Brasil como Homenagem\\*
Aos Escritores antigos\\*
Patronou a sua imagem.\\*
A \textsc{cadeira dezesseis}\\*
Quem senta nela uma vez\\*
Goza de toda vantagem.

``Sacra'' -- Primeiro Volume\\*
Que A. B. L. publicou,\\*
Um livro em que o Poeta\\*
Sua alma extravasou,\\*
Chegando quase ao empirismo\\*
Deixou o seu \textsc{gregorismo}\\*
Que toda América aceitou.

Muita gente já conhece\\*
Os versos de ``Graciosa'',\\*
Da ``Lyrica'' já publicados,\\*
``Satírica'', ``Licenciosa''...\\*
Nem \textsc{conselheiro xx}\\*
Talvez fosse mais feliz\\*
Em matéria primorosa

Uma piada picante\\*
Para quem sabe escrever\\*
Se torna \textsc{arte}, por isso\\*
O \textsc{censor} faz que não vê,\\*
Mais quem não sabe versar\\*
Deixa de ser popular\\*
Para a censura caber.

Da mesma forma o Poeta --\\*
-- O Trovador Repentista\\*
Que ao compor os seus versos\\*
Mesmo sem ser moralista\\*
Pode narrar uma piada\\*
Que o leitor acha engraçada\\*
Pelo versar do Artista.

Assim Gregório de Matos\\*
Não era um Vate Imoral,\\*
Ao contrário: um Trovador\\*
De inspiração genial,\\*
Se não gozava de estima\\*
Foi grande Mestre na rima --\\*
Um Camões de Portugal.

No Ano Mil e Seiscentos\\*
Noventa e Cinco, em verdade,\\*
Com seus setenta e dois anos\\*
Vindo da adversidade\\*
Em Recife faleceu,\\*
Porém lutou e venceu\\*
Ganhando a Imortalidade!

Vinha o Vate de Angola\\*
Bastantemente ferido,\\*
Dos martírios do degredo\\*
Já bastante envelhecido,\\*
Não chegou ver a Bahia\\*
Terra que tanto queria --\\*
Seu torrão estremecido.

Todos os poetas do mundo\\*
São semelhantes às flores\\*
Que possuem melhor perfume\\*
Tendo pétalas multicores,\\*
Trazem consigo os espinhos\\*
Que em troca dos carinhos\\*
São coroados de dores

Gregório não usou calúnia\\*
Nem abraçou a traição,\\*
Foi um Vate corajoso\\*
Como o rugir do leão,\\*
Era um ser que no Universo\\*
Não podia usar o verso\\*
Da sua imaginação.

Calar a voz de um Poeta\\*
É parar o próprio vento\\*
O destino que ele segue\\*
Porque o seu pensamento\\*
Não pode ser asfixiado\\*
Dentro do peito guardado\\*
Ferindo seu sentimento!

Hoje Gregório de Matos\\*
Guerra é nome venerado,\\*
Mas no seu tempo por todos\\*
Era por certo odiado,\\*
Hoje o ``Boca do Inferno'',\\*
Na Glória do Pai Eterno!\\*
É um ser abençoado

Gregório de Matos Guerra\\*
Pela versificação,\\*
Pelo estilo que criou\\*
Na mais alta inspiração,\\*
É o \textsc{pai dos trovadores}\\*
E dos Poetas, leitores,\\*
Desta nova geração!

Foi Castro Alves um Gênio\\*
Por seu estilo elevado,\\*
Um moço de Vinte Séculos\\*
De espírito adiantado,\\*
Foi Mestre no Condorismo,\\*
Gregório no Satirismo\\*
Jamais será imitado!

Salve Gregório de Matos\\*
Grande Vate brasileiro\\*
Que tornou"-se o Grande Vulto\\*
Deste País Brasileiro,\\*
Pelo seu estilo ardente\\*
Ficará eternamente\\*
Como um perfeito Luzeiro!

R -- ealmente Gregório\\*
O -- satírico Trovador\\*
D -- eixou o nome na História\\*
O -- nde cintila o fulgor,\\*
L -- uz Excelsa da Bahia,\\*
F -- ulgurado em Poesia\\*
O -- Bardo de Salvador.

\end{verse}

\part{Na sala de aula}

\pagebreak
\section{Sobre o autor}

Rodolfo Coelho Cavalcante nasceu em Rio Largo (AL) em 1919, embora em
seu registro de nascimento conste 1917. 

Aos treze anos de idade, deixou a casa dois pais e foi percorrer todo o
interior de Alagoas, Ceará, Sergipe, Piauí e Maranhão, trabalhando como
palhaço de circo, camelô e propagandista. Em 1945 fixou-se em Salvador
(BA), onde passou a escrever seus versos e atuar no jornalismo. Foi
membro fundador da Associação de Imprensa Periódica da Bahia e filiado
à Associação Baiana de Imprensa. Como trovador entusiasta que era,
fundou A voz do trovador, O trovador e Brasil poético, órgãos do
movimento trovadoresco.

Rodolfo Cavalcante pode ser considerado o maior líder da história da
literatura de cordel, tendo idealizado muitos movimentos visando à
união dos cantadores. Em julho de 1955, junto com outros expoentes da
poesia popular, realizou o I Congresso Nacional de Trovadores e
Violeiros, ocasião em que foi fundada a Associação Nacional de
Trovadores e Violeiros, hoje Grêmio Brasileiro de Trovadores, com sede
em Salvador. Morreu em 1986, atropelado em frente à sua casa no bairro
da Liberdade, em Salvador.

O conjunto de sua obra inclui inúmeros folhetos, entre eles: A chegada
de Lampião no céu, diversos ABCs, como ABC dos namorados, do amor, do
beijo, da dança, Cuíca de Santo Amaro – O poeta popular que conheci e O
desencanto da moça que bateu na mãe e virou cachorra.

\section{Síntese dos poemas}

\medskip

\paragraph{“Origem da literatura de cordel e a sua expressão de cultura nas letras
de nosso país”}

O poema explica o que é o cordel, comentando a sua origem oral, mantida
pelos repentistas desde o tempo da escravidão, e sobre os poetas de
bancada, que escrevem seus versos e os publicam em folhetos. Discorre
também sobre a métrica, as rimas dos versos, a importância do cordel
para a divulgação da cultura popular e das notícias, os temas mais
frequentes e as figuras míticas nordestinas mais celebradas, como Padre
Cícero e Lampião.

\paragraph{“Antônio Conselheiro – O santo guerreiro de Canudos”}

Em seus versos, Rodolfo narra eventos importantes na vida de Antônio
Conselheiro, como a Guerra de Canudos, em que liderou seu grupo contra
as forças militares. Descreve sua aparência extravagante e seu carisma
de “santo e profeta”, ora o apresentando como herói, ora como um
místico e fanático contra o qual era preciso lutar: “Um Bravo, um
Herói, Fanático,/ Um cidadão brasileiro/ Que seria premiado/ Se ele
lutasse ao lado/ De um ideal verdadeiro”.

\paragraph{“O boi que falou no Piauí”}

O poema relata a compra de um boi da Dinamarca pelo Coronel Praxedes, um
rico fazendeiro do Piauí. Inicialmente, a história parece baseada em
uma notícia real, descrevendo o fazendeiro, sua criação de gado de
raça, os leilões que costumava frequentar e a visita que um certo
estrangeiro lhe fez, oferecendo-lhe o tal boi. Mas o que pode ter sido
fato vai adquirindo tons de folclore e lenda quando um dos empregados
da fazenda relata que ouviu o boi se queixando com os outros animais da
forma como seu dono os tratava...

\paragraph{“O encontro de Rodolfo Cavalcante com Lampião Virgulino”}

Rodolfo e seu irmão Aristófeles, quando ainda eram muito jovens,
realmente chegaram a cruzar com Lampião e seu bando. O famoso bandido
os capturou mas, ao verificar a sua insignificância, deixou-os partir
sem lhes fazer mal algum, como estes versos atestam: “Muita gente
afirma que/ Lampião era ruim,/ Eu não posso condená-lo/ Porque nada fez
a mim,/ Ao contrário: a mim e ao mano/ Deu a prova de humano/ Como
cavalheiro enfim”.

Ao longo do poema Rodolfo constrói a imagem de um Lampião temível e
cruel, mas, ao mesmo tempo, um grande líder, bom filho e companheiro,
mais vítima do Brasil daquela época do que propriamente vilão. E no
final dedica estes versos ao “Capitão”: “Vamos mudar a imagem/ Do
lendário Lampião,/ Morreu o analfabetismo/ O flagelo do sertão,/ Que
foi o grande fator/ Do drama desolador/ Do temível Capitão”.

\paragraph{“Cuíca de Santo Amaro — O poeta popular que conheci”}

Como diz Eno T. Wanke na introdução, Cuíca de Santo Amaro foi o mais
famoso colega de trabalho de Rodolfo, “com quem mantinha boas relações
e trocava seus folhetos”. Rodolfo era o oposto do amigo: enquanto Cuíca
era uma figura extravagante e extrovertida, sempre de cartola, fraque e
óculos escuros, Rodolfo vendia seus cordéis de terno e gravata. Em seus
folhetos, Cuíca geralmente narrava os assuntos em tom escandaloso e
sensacionalista, ao passo que Rodolfo apelava para o drama e os
sentimentos do leitor.

Apesar das diferenças, tornaram-se muito amigos, chegando a realizar um
pacto: quem sobrevivesse ao outro lhe dedicaria alguns versos. E assim
surgiu o poema, que já no início esclarece:


\begin{verse}

Na “Baixa dos Sapateiros”\\*
Entrada do Taboão,\\*
Eu e Cuíca fizemos\\*
Um pacto na profissão,\\*
Quando um morresse primeiro\\*
Versaria o derradeiro\\*
A História do seu irmão.

\end{verse}

\paragraph{“A mulher que foi surrada pelo Diabo”}

O poema conta as desventuras de uma mulher casada, mãe de três filhos,
que troca a família e o lar, no qual se sente oprimida, pela ilusão de
liberdade e juventude eterna. Embora fantasiosa, a história ilustra de
forma interessante a insatisfação da “vida moderna” e os ideais
feministas, que nas últimas décadas inspiraram tantas mulheres a
desistir da vida doméstica e evitar ter filhos. O poema tem um tom
profundamente moral, sugerindo que não se pode fugir das
responsabilidades, que no caso da mulher – principalmente a nordestina
– é cuidar da família.

\paragraph{“Tudo na terra tem fim”}

O poema fala da transitoriedade da vida e da inevitabilidade da morte.
Não importa se somos ricos ou pobres, cultos ou ignorantes, cientistas
ou lavradores, famosos como Napoleão ou anônimos como a maioria –
estamos aqui apenas de passagem e, queiramos ou não, com o tempo
acabaremos sendo esquecidos.

\paragraph{“Paulista virou tatu viajando pelo metrô”}

Este poema, inspirado numa reportagem do famoso jornalista e folclorista
baiano Franklin Machado, é uma verdadeira apologia do metrô e da cidade
grande. Com sua visão ingênua e otimista, Rodolfo discorre sobre o
“progresso” de metrópoles como São Paulo e Rio de Janeiro, exaltando os
administradores públicos e suas obras de engenharia.

\paragraph{“O desencanto da moça que bateu na mãe e virou cachorra”}

Segundo o próprio autor, este foi seu folheto de maior sucesso, tendo
vendido milhares de exemplares em 36 edições. Escrito no fim da década
de 1940, narra a história de Helena, uma linda jovem que, por ter sido
insolente e haver agredido a própria mãe, foi transformada em uma
cachorra. 

Dona Matilde, sua mãe, inconformada com a triste situação da filha,
rogou ao Bom Jesus da Lapa que a socorresse. E o milagre aconteceu:
depois de um encontro com um misterioso velhinho, Helena recupera sua
forma humana e parte numa longa viagem de regresso a Canindé, sua terra
natal. 

\paragraph{“Gregório de Matos Guerra — O Pai dos Poetas Brasileiros”}

O poema é uma homenagem ao poeta colonial baiano Gregório de Matos
Guerra (1636-c. 1696), o primeiro grande poeta brasileiro. Gregório foi
também advogado, tendo se formado em Direito em Coimbra, Portugal.
Embora sua obra também inclua poesia lírica, religiosa e erótica,
recebeu o apelido de “Boca do Inferno” devido a seus irreverentes
poemas satíricos.

\section{Poesia de cordel: oralidade e escuta coletiva} 

A poesia de cordel, dizem os especialistas, é uma poesia escrita para ser lida,
enquanto o repente ou o desafio é a poesia feita oralmente, que mais tarde pode
ser registrada por escrito. Essa divisão é muito esquemática. Por exemplo, o
cordel, mesmo sendo escrito e impresso para ser lido, costumava ser lido em
volz alta e desfrutado por outros ouvintes além do leitor. A poesia popular,
praticada principalmente no Nordeste do Brasil, tem muita influência da
linguagem oral, aproveita muito da língua coloquial praticada nas ruas e na
comunicação cotidiana. 

Naturalmente, portanto, pode-se considerar a poesia narrativa do cordel uma
forma de poesia mais compartilhada e desfrutada coletivamente, o que dá também
uma grande ressonância social. Muitos dos temas do cordel são originários das
tradições populares e eruditas da Europa medieval e moderna. Outros temas são
retirados de tradições orientais, como neste livro de Patativa “A história de
Aladim e a lâmpada maravilhosa”. O personagem Aladim pertence ao Livro das mil
e uma noites, um dos famosos conjuntos de histórias de todos os tempos. Também
encontramos temas retirados das novelas de cavalaria medievais e das narrativas
bíblicas. Ao lado destes temas mais literários, encontram-se os temas locais,
quase sempre narrados na forma de crônicas de coisas realmente acontecidas,
como em “Padre Henrique e o dragão da maldade”. Também há as histórias
fantásticas, que se valem das tradições semirreligiosas, ligadas à experiência
com o mundo espiritual. 

Os grandes poemas de cordel são perfeitamente metrificados e rimados. A métrica
e a rima são recursos que favorecem a memorização e tradicionalmente se costuma
dizer que são resquícios de uma cultura oral, na qual toda a tradição e
sabedoria são sabidas de cor.  


\section{O sertão geográfico e cultural}

O sertão tem mitos culturais próprios. Contemporaneamente, o sertão evoca
principalmente o sofrimento resignado daqueles que padecem a falta de chuva e
de boas safras na lavoura. Evoca a experiência histórica de uma região
empobrecida, embora tenha sido geradora de riquezas, como o cacau e cana de
açúcar, ambos bens muito valiosos. 

O sertão formou também o seu imaginário por meio de grandes personalidades e
uma pujante expressão artística. Além do cordel, o sertão viu nascer ritmos tão
importantes quanto o forró e o baião. Produziu artistas tão expressivos quanto
Luiz Gonzaga, grande cantor da vida do sertanejo em canções como “Asa branca”.
Um escultor como Mestre Vitalino criou toda uma tradição de representação da
vida e dos hábitos sertanejos em miniaturas de barro. A gravura popular, que
sempre acompanha os folhetos de cordel, também floresceu em diversos pontos e
ficou mais famosa em Juazeiro do Norte, no Ceará, e em Caruaru, no estado de
Pernambuco. 

Dentre os grande mitos do sertão, está certamente o do cangaço com seu líder
histórico, mas também mítico, Virgulino Ferreira, o Lampião. Até hoje as
opiniões se dividem: para alguns foi uma grande homem, para outros um bandido
impiedoso. 

Uma figura muito presente na cultura nordestina é o Padre Cícero Romão,
considerado beato pela Igreja Católica. Consta que teria feito milagres e
dedicado sua vida aos pobres. 

\section{Variação linguística}

A línguística moderna usa o termo “idioleto” para marcar grupos distintos no
interior de uma língua. Um idioleto pode ser a fala peculiar de uma região, de
um grupo étnico ou de uma dada profissão. 

Uma das grandes forças da poesia popular do Nordeste se origina em sua forma
muito própria de falar, com um ritmo muito diferente dos falares do sul, e
também muito diferentes entre si, pois percebe-se a diferença entre os falares
de um baiano, um cearense e um pernambucano, por exemplo.

Além desse aspecto rítmico, quase sempre também há palavras peculiares a certas
regiões. 

\section{Sugestões de atividades}
\begin{enumerate}

\item \textit{Atividade de leitura}. Esta atividade tem por objetivo sensibilizar
os alunos para a escuta de poesia. O professor deve ler um conjunto de
estrofes para exemplificar uma leitura que se construa com uma
pronúncia clara, pausas e ênfases adequadas. Após isso, cada aluno deve
ler uma estrofe, procurando marcar o ritmo e as rimas, bem como as
pausas e ênfases expressivas.

Para enriquecer a experiência com outros recursos, o professor pode
mostrar o poeta Patativa do Assaré lendo alguns de seus poemas no link
a seguir, do site Youtube: www.youtube.com/watch?v=RTEfYnMNNpc. 

Uma atividade como essa pode auxiliar no desenvolvimento da percepção da
voz e da fala como meio indispensáveis à boa convivência social.

\item “Antônio Conselheiro – O santo guerreiro de Canudos”
Nota: Esta atividade poderá ser desenvolvida antes da leitura do poema.

A situação do Nordeste brasileiro era muito difícil no final do século
XIX. A região, que já havia abastecido a economia brasileira em outros
tempos, passava por maus bocados com a seca, fome, miséria e violência.
O abandono político fez com que inúmeros conflitos surgissem na região,
dentre eles, a Guerra de Canudos, no sertão da Bahia.

Os motivos aparentes da guerra eram o fanatismo religioso e o
messianismo, mas suas razões profundas eram o latifúndio, o
coronelismo, a escravidão, o isolamento cultural e geográfico em que
vivia seu povo, no ambiente árido e hostil do sertão. Enquanto os
políticos disputavam por seus interesses, algumas regiões passavam por
uma profunda crise de miséria e pobreza. 


a. Orientar os alunos na pesquisa de informações sobre esse conflito:
quando começou e por quê; quantas baixas (mortes) houve de cada lado do
conflito (soldados X “conselheiristas”) etc.

b. Ao contrário do que muitos imaginam, o cearense Antônio Vicente
Mendes Maciel – o Antônio Conselheiro – não era analfabeto nem
ignorante: teve aulas de português, latim e francês na infância e
trabalhou como comerciante, professor e advogado dos pobres. Ele se
acreditava enviado por Deus para acabar com as diferenças sociais e,
sempre que podia, reconstruía igrejas e cemitérios, coordenava mutirões
para a construção de casas, as colheitas e a distribuição de alimentos
entre os mais pobres. Suas ações acabaram por torná-lo popular entre os
nordestinos, que sofriam com a miséria e a fome e viam nele a esperança
de liberdade.

Pedir aos alunos que busquem informações sobre Conselheiro em pelo menos
duas fontes distintas: uma o retratando apenas como um líder fanático e
outra com um ponto de vista diferente e mais favorável a ele.

c. O filme Guerra de Canudos (links abaixo), dirigido por Sérgio
Resende, retrata o episódio sob a ótica de uma família em que há
opiniões conflitantes sobre Conselheiro. Seria interessante reunir a
classe para uma sessão, antes da atividade final (item d). 

Links para as partes 1 e 2 do filme:

http://www.youtube.com/watch?v=KCH9Om\_X6Lc

http://www.youtube.com/watch?v=Q3ElPSxxKzU

d. Para concluir a atividade, propor um debate acerca de Conselheiro
e sua revolta contra as forças do governo. A classe poderá ser dividida
em dois grupos: um com argumentos a favor de Conselheiro e outro
defendendo a posição do governo. No final do debate, propor que ambos
os grupos reflitam sobre as consequências do conflito (e que
responsabilidade atribuem a cada um dos lados) e sugiram como ele
poderia ter sido evitado.

Links com informações adicionais sobre Canudos e Antônio Conselheiro:

http://redes.moderna.com.br/2011/10/05/fim-da-guerra-de-canudos/

http://www.algosobre.com.br/biografias/euclides-da-cunha.html

http://www.oolhodahistoria.ufba.br/03moura.html


\item Orientar os alunos para uma leitura atenta do poema, situando a história
no tempo e no espaço. Deverão identificar passagens que revelem a
postura do poeta em relação ao conflito de Canudos e seu líder.
Questões para disparar o debate:

a. Segundo o poema, em que época Antônio Conselheiro passou a pregar?
E quando teve início o conflito? Estas informações estão corretas?


b. O poema cita alguns dos locais dos principais conflitos e
massacres. Quais são eles? Tente localizá-los em um mapa atual. Quais
ainda existem e quais já desapareceram?

c. Que visão o poeta tem de Conselheiro e seus seguidores? Esta visão
é favorável aos revoltosos ou não? Justifique sua resposta.

d. Em sua opinião, a visão do poeta é perfeitamente objetiva, ou
seja, corresponde inteiramente à realidade dos fatos?


\item “O encontro de Rodolfo Cavalcante com Lampião Virgulino”

a. Após a leitura do poema, propor aos alunos uma conversa informal,
em que possam opinar sobre o poema e a impressão do autor a respeito de
Lampião: O poema o apresenta de forma negativa ou positiva? Que
qualidades (positivas e negativas) Rodolfo atribui a ele? (As respostas
devem ser justificadas com exemplos retirados do próprio poema.)

b. Pedir aos alunos que releiam o poema e reflitam sobre o
significado e a importância para Rodolfo e seu irmão – adolescentes na
época – de seu encontro com o famoso e temível Lampião. Como eles,
alunos e também adolescentes, teriam reagido numa situação dessas? Os
alunos poderão selecionar os trechos mais emocionantes do poema e
colocar-se, então, no lugar dos dois irmãos, explicando como teriam
reagido. Para tornar a atividade bem criativa e divertida, os alunos
poderão substituir alguns versos de Rodolfo por outros criados por eles
mesmos, e que expressem sua reação diante de Lampião e seu bando.

c. Poucas pessoas fora do cangaço tiveram a chance de conhecer
Lampião pessoalmente. Uma delas foi o fotógrafo libanês Benjamin
Abrahão, que não apenas o viu em carne e osso como também o fotografou
e filmou com Maria Bonita e outros membros de seu bando. 

Os links abaixo são para dois vídeos: 

O primeiro (14 min.) é a única parte que restou do documentário de
Abrahão. Todo em preto e branco e sem som, ele mostra o próprio
fotógrafo sendo recebido pelos cangaceiros, bem como Lampião e os
demais em diversas situações. O texto na introdução fornece mais
informações importantes.

O segundo é o filme Baile perfumado (1h32min), de 1996, que narra a
história desse incrível encontro.

Link para o vídeo 1:
http://www.youtube.com/watch?v =O33Flqcp5B4\&list=PL2orKizuWz-9oO05TRetvCS8 WgHyzyh1i

Link para o vídeo 2: http://www.youtube.com/watch?v =NnmWmTl217k

\pagebreak

\item “A mulher que foi surrada pelo Diabo”

a. Seria interessante propor uma pesquisa sobre a história do
movimento feminista no Brasil e sua influência (ou não) na sociedade,
tanto nos grandes centros quanto nas pequenas cidades. Os alunos,
organizados em grupos, podem investigar (em notícias ou crônicas da
época e de hoje, vídeo-documentários, entrevistas, etc.) quais eram/são
as reivindicações feministas, que impacto tiveram/têm nas famílias, que
regiões ou centros urbanos do país foram/são mais afetados etc. 

b. Depois de concluída a pesquisa, poderão apresentar os resultados
obtidos. É importante que percebam que a realidade socioeconômica e
cultural nos grandes centros difere muito daquela observada em cidades
pequenas ou povoados, onde a vida é mais pacata e a tradição e o
conservadorismo ainda são muito fortes. Para ilustrar seus argumentos,
podem selecionar algumas histórias verídicas semelhantes à que inspirou
o poema de Rodolfo. 

c. Outra atividade interessante seria um debate sobre a influência da
televisão sobre os costumes locais, especialmente através de novelas e
programas de entretenimento. É importante que eles percebam como este
veículo afeta a opinião pública e modifica seus valores, não só através
dos programas como também das propagandas e do merchandising inserido
nas novelas e séries.

\item “Cuíca de Santo Amaro — O poeta popular que conheci”

O documentário O cordel esquecido num país sem memória (link abaixo)
constitui uma boa oportunidade para os alunos conhecerem o processo
artesanal de produção dos folhetos de cordel: as tipografias, que hoje
estão se tornando obsoletas; as antigas xilogravuras, que não mais
estampam e enobrecem os livretos; a figura do poeta declamador, que
emocionava o povo nas feiras e praças públicas; e o repentista, poeta
que cria versos de improviso.

Além disso, o vídeo resgata a memória dos mais importantes cordelistas
da Bahia – Cuíca de Santo Amaro e Rodolfo Cavalcante. Isaías, filho de
Rodolfo, concede uma entrevista em que fala do pai e sua amizade com
Cuíca, bem como do “trato” que fizeram de prestar uma homenagem ao
primeiro que viesse a falecer. Muito interessante.

Link para o vídeo: http://www.youtube.com/watch?v= 6dKQTRQH3EU

\end{enumerate}

\section{Sugestões de leitura\\ para o professor} 

\begin{description}\labelsep0ex\parsep0ex
%\newcommand{\tit}[1]{\item[\textnormal{\textsc{\MakeTextLowercase{#1}}}]}
%\newcommand{\titidem}{\item[\line(1,0){25}]}

\tit{DIEGUES JÚNIOR}, Daniel. \textit{Literatura popular em verso}. Estudos. Belo Horizonte: Itatiaia, 1986. 

\tit{MARCO}, Haurélio. \textit{Breve história da literatura de cordel}. São Paulo: Claridade, 2010.

\tit{TAVARES}, Braulio. \textit{Contando histórias em versos. Poesia e romanceiro popular no Brasil}. São Paulo: 34, 2005.

\tit{TAVARES}, Braulio. \textit{Os martelos de trupizupe}. Natal: Edições Engenho de Arte, 2004 

\end{description}