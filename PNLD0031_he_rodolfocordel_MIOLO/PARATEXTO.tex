\chapter[Vida e obra de Rodolfo Coelho Cavalcante]{Vida e obra de Rodolfo\\ Coelho Cavalcante}

\section{Sobre o autor}

\noindent{}Rodolfo Coelho Cavalcante nasceu em Rio Largo (\textsc{al}) em 1919, embora em
seu registro de nascimento conste 1917. 

Aos treze anos de idade, deixou a casa dois pais e foi percorrer todo o
interior de Alagoas, Ceará, Sergipe, Piauí e Maranhão, trabalhando como
palhaço de circo, camelô e propagandista. Em 1945 fixou"-se em Salvador
(BA), onde passou a escrever seus versos e atuar no jornalismo. Foi
membro fundador da Associação de Imprensa Periódica da Bahia e filiado
à Associação Baiana de Imprensa. Como trovador entusiasta que era,
fundou A voz do trovador, O trovador e Brasil poético, órgãos do
movimento trovadoresco.

Rodolfo Cavalcante pode ser considerado o maior líder da história da
literatura de cordel, tendo idealizado muitos movimentos visando à
união dos cantadores. Em julho de 1955, junto com outros expoentes da
poesia popular, realizou o \textsc{i} Congresso Nacional de Trovadores e
Violeiros, ocasião em que foi fundada a Associação Nacional de
Trovadores e Violeiros, hoje Grêmio Brasileiro de Trovadores, com sede
em Salvador. Morreu em 1986, atropelado em frente à sua casa no bairro
da Liberdade, em Salvador.

O conjunto de sua obra inclui inúmeros folhetos, entre eles: A chegada
de Lampião no céu, diversos \textsc{abc}s, como \textsc{abc} dos namorados, do amor, do
beijo, da dança, Cuíca de Santo Amaro – O poeta popular que conheci e O
desencanto da moça que bateu na mãe e virou cachorra.

\section{Sobre a obra}

Um dos primeiros êxitos em cordel de Rodolfo Coelho Cavalcante foi \textit{A volta de Getúlio}, publicado dois dias depois da queda do ditador, ocorrida em 29 de
outubro. Começava assim: 

\begin{verse}
Pode o porco ser granfino, \\
Pode o pato não nadar, \\
Pode o leão ser mofino, \\
Pode o gato não miar, \\
A galinha criar dente, \\
Gente virar serpente, \\
Mas Getúlio vai voltar!

Pode um padre ser batista, \\
Protestante não cantar, \\
Católico não ir à missa, \\
Freira deixar de rezar, \\
O ateu ter salvação, \\
Cobra tocar violão, \\
Se Getúlio não voltar!
\end{verse}

Os primeiros mil exemplares esgotaram"-se em dois dias.
E Getúlio continuou sendo um dos assuntos
favoritos de Rodolfo nos próximos anos, tanto na deposição, em 1945,
como na campanha eleitoral e reeleição à presidência da República em
1950, e no suicídio ocorrido em 24 de agosto de 1954. 

Como relembra Eno T. Wanke, estudioso de sua obra, seus folhetos, a maioria com oito páginas mais a capa, eram impressos
em tipografias. O resto do trabalho, a composição dos versos, a
obtenção da ilustração da capa (geralmente uma xilogravura de seu amigo
Sinésio Cabral ou um clichê obtido em jornal), a dobragem, o corte da
capa, o grampeamento -- era feito em casa, ``nas horas
vagas da noite'', quer pelo próprio Rodolfo, quer
``pelos meninos'' aos quais ele dava pequena
gratificação. 

\begin{quote}
Seu ritmo de produção era intenso: nos anos iniciais, entre quatro e
oito títulos por semana, dele ou de outros autores, comentando os mais
diversos acontecimentos. Trabalhava, geralmente, na praça Cayru, junto
ao Mercado, um dos mais movimentados pontos de Salvador. Aos poucos,
foi ampliando seus ``pontos de venda'' com os
agentes, aos quais fornecia ``em grosso'' .
Chegou a ter trinta agentes, espalhados desde o interior da Bahia até
Alagoas e Sergipe.\footnote{\textsc{wanke}, Eno T. ``Introdução''. In: \textsc{Cavalcante}, Rodolfo Coelho. \textit{Cordel na escola}. São Paulo: Hedra, 2011, p.\,17.}
\end{quote}

\subsection{Síntese dos poemas}

\paragraph{“Origem da literatura de cordel e a sua expressão de cultura nas letras
de nosso país”}

O poema explica o que é o cordel, comentando a sua origem oral, mantida
pelos repentistas desde o tempo da escravidão, e sobre os poetas de
bancada, que escrevem seus versos e os publicam em folhetos. Discorre
também sobre a métrica, as rimas dos versos, a importância do cordel
para a divulgação da cultura popular e das notícias, os temas mais
frequentes e as figuras míticas nordestinas mais celebradas, como Padre
Cícero e Lampião.

\paragraph{“Antônio Conselheiro – O santo guerreiro de Canudos”}

Em seus versos, Rodolfo narra eventos importantes na vida de Antônio
Conselheiro, como a Guerra de Canudos, em que liderou seu grupo contra
as forças militares. Descreve sua aparência extravagante e seu carisma
de “santo e profeta”, ora o apresentando como herói, ora como um
místico e fanático contra o qual era preciso lutar: “Um Bravo, um
Herói, Fanático,/ Um cidadão brasileiro/ Que seria premiado/ Se ele
lutasse ao lado/ De um ideal verdadeiro”.

\paragraph{“O boi que falou no Piauí”}

O poema relata a compra de um boi da Dinamarca pelo Coronel Praxedes, um
rico fazendeiro do Piauí. Inicialmente, a história parece baseada em
uma notícia real, descrevendo o fazendeiro, sua criação de gado de
raça, os leilões que costumava frequentar e a visita que um certo
estrangeiro lhe fez, oferecendo"-lhe o tal boi. Mas o que pode ter sido
fato vai adquirindo tons de folclore e lenda quando um dos empregados
da fazenda relata que ouviu o boi se queixando com os outros animais da
forma como seu dono os tratava...

\paragraph{“O encontro de Rodolfo Cavalcante com Lampião Virgulino”}

Rodolfo e seu irmão Aristófeles, quando ainda eram muito jovens,
realmente chegaram a cruzar com Lampião e seu bando. O famoso bandido
os capturou mas, ao verificar a sua insignificância, deixou"-os partir
sem lhes fazer mal algum, como estes versos atestam: “Muita gente
afirma que/ Lampião era ruim,/ Eu não posso condená"-lo/ Porque nada fez
a mim,/ Ao contrário: a mim e ao mano/ Deu a prova de humano/ Como
cavalheiro enfim”.

Ao longo do poema Rodolfo constrói a imagem de um Lampião temível e
cruel, mas, ao mesmo tempo, um grande líder, bom filho e companheiro,
mais vítima do Brasil daquela época do que propriamente vilão. E no
final dedica estes versos ao “Capitão”: “Vamos mudar a imagem/ Do
lendário Lampião,/ Morreu o analfabetismo/ O flagelo do sertão,/ Que
foi o grande fator/ Do drama desolador/ Do temível Capitão”.

\paragraph{“Cuíca de Santo Amaro — O poeta popular que conheci”}

Como diz Eno T. Wanke na introdução, Cuíca de Santo Amaro foi o mais
famoso colega de trabalho de Rodolfo, “com quem mantinha boas relações
e trocava seus folhetos”. Rodolfo era o oposto do amigo: enquanto Cuíca
era uma figura extravagante e extrovertida, sempre de cartola, fraque e
óculos escuros, Rodolfo vendia seus cordéis de terno e gravata. Em seus
folhetos, Cuíca geralmente narrava os assuntos em tom escandaloso e
sensacionalista, ao passo que Rodolfo apelava para o drama e os
sentimentos do leitor.

Apesar das diferenças, tornaram"-se muito amigos, chegando a realizar um
pacto: quem sobrevivesse ao outro lhe dedicaria alguns versos. E assim
surgiu o poema, que já no início esclarece:


\begin{verse}

Na “Baixa dos Sapateiros”\\*
Entrada do Taboão,\\*
Eu e Cuíca fizemos\\*
Um pacto na profissão,\\*
Quando um morresse primeiro\\*
Versaria o derradeiro\\*
A História do seu irmão.

\end{verse}

\paragraph{“A mulher que foi surrada pelo Diabo”}

O poema conta as desventuras de uma mulher casada, mãe de três filhos,
que troca a família e o lar, no qual se sente oprimida, pela ilusão de
liberdade e juventude eterna. Embora fantasiosa, a história ilustra de
forma interessante a insatisfação da “vida moderna” e os ideais
feministas, que nas últimas décadas inspiraram tantas mulheres a
desistir da vida doméstica e evitar ter filhos. O poema tem um tom
profundamente moral, sugerindo que não se pode fugir das
responsabilidades, que no caso da mulher – principalmente a nordestina
– é cuidar da família.

\paragraph{“Tudo na terra tem fim”}

O poema fala da transitoriedade da vida e da inevitabilidade da morte.
Não importa se somos ricos ou pobres, cultos ou ignorantes, cientistas
ou lavradores, famosos como Napoleão ou anônimos como a maioria –
estamos aqui apenas de passagem e, queiramos ou não, com o tempo
acabaremos sendo esquecidos.

\paragraph{“Paulista virou tatu viajando pelo metrô”}

Este poema, inspirado numa reportagem do famoso jornalista e folclorista
baiano Franklin Machado, é uma verdadeira apologia do metrô e da cidade
grande. Com sua visão ingênua e otimista, Rodolfo discorre sobre o
“progresso” de metrópoles como São Paulo e Rio de Janeiro, exaltando os
administradores públicos e suas obras de engenharia.

\paragraph{“O desencanto da moça que bateu na mãe e virou cachorra”}

Segundo o próprio autor, este foi seu folheto de maior sucesso, tendo
vendido milhares de exemplares em 36 edições. Escrito no fim da década
de 1940, narra a história de Helena, uma linda jovem que, por ter sido
insolente e haver agredido a própria mãe, foi transformada em uma
cachorra. 

Dona Matilde, sua mãe, inconformada com a triste situação da filha,
rogou ao Bom Jesus da Lapa que a socorresse. E o milagre aconteceu:
depois de um encontro com um misterioso velhinho, Helena recupera sua
forma humana e parte numa longa viagem de regresso a Canindé, sua terra
natal. 

\paragraph{“Gregório de Matos Guerra — O Pai dos Poetas Brasileiros”}

O poema é uma homenagem ao poeta colonial baiano Gregório de Matos
Guerra (1636-c. 1696), o primeiro grande poeta brasileiro. Gregório foi
também advogado, tendo se formado em Direito em Coimbra, Portugal.
Embora sua obra também inclua poesia lírica, religiosa e erótica,
recebeu o apelido de “Boca do Inferno” devido a seus irreverentes
poemas satíricos.

\section{Sobre o gênero} 

Para uma primeira definição de poesia enquanto gênero literário, poder"-se"-ia recorrer à definição do professor Domingos Paschoal Cegalla, para quem ``poesia é a linguagem subjetiva, carregada de emoção e sentimento, com ritmo melódico constante, bela e indefinível como o mundo interior do poeta visa a um efeito estético''.\footnote{\textsc{cegalla}, Domingos Paschoal. \textit{Novíssima Gramática da Língua Portuguesa}. São Paulo: Companhia Editora Nacional, 2008, p.\,640}

Aprofundando um pouco essa definição, o crítico Antonio Candido expande a definição de poesia ao diferenciá"-la do verso.
Para o crítico, a poesia enquanto ato criador do artista independe da forma métrica do verso, que passa a ser apenas um dos registros possíveis do poético:

\begin{quote}
A poesia não se confunde necessariamente com o verso, muito menos com o verso metrificado. Pode haver poesia em prosa e poesia em verso livre. [\ldots]
Pode ser feita em verso muita coisa que não é poesia.\footnote{\textsc{candido}, Antonio. \textit{O estudo analítico do poema}. São Paulo: Terceira leitura, 1993, p.\,13--14.}
\end{quote}

Delineada, de forma breve e geral, a forma poética, pode"-se pensar agora em seus três gêneros básicos: lírico, épico e dramático.
Para o crítico Anatol Rosenfeld, a lírica é o gênero mais subjetivo, no qual uma voz central exprime um estado de alma traduzido em orações poéticas.
Seria a expressão de emoções e experiências vividas, ``a plasmação imediata das vivências intensas de um Eu no encontro com o mundo, sem que se interponham eventos distendidos no tempo (como na Épica e na Dramática)''.\footnote{\textsc{rosenfeld}, Anatol. \textit{O teatro épico}. São Paulo: Perspectiva, 2006, p.\,22.}

Devido a essa característica central da lírica, a expressão de um estado emocional, Rosenfeld considera que o eu"-lírico, nesse gênero, não se delineia enquanto um personagem. Embora possa evocar personagens e narrar acontecimentos, a lírica entendida enquanto gênero puro afasta"-se sobremaneira da apreensão objetiva do mundo, que não existe independente da subjetividade intensa que o apreende e exprime. Assim, na lírica prevalece a fusão entre o sujeito e o objeto, que serve mais a realçar os estados profundos de alma do poeta.
Sobre os aspectos formais do gênero, Rosenfeld nota:

\begin{quote}
À intensidade expressiva, à concentração e ao caráter ``'imediato'' do poema lírico, associa"-se, como traço estilístico importante, o uso do ritmo e da musicalidade das palavras e dos versos. De tal modo se realça o valor da aura conotativa do verbo que este muitas vezes chega a ter uma função mais sonora que lógico"-denotativa. A isso se liga a preponderância da voz do presente que indica a ausência de distância, geralmente associada ao pretérito. Este caráter do imediato, que se manifesta na voz do presente, não é, porém, o de uma atualidade que se processa e distende através do tempo (como na Dramática) mas de um momento ``eterno''.\footnote{Ibidem, p.\,23.}
\end{quote}

No caso específico da poesia de cordel, dizem os especialistas, é uma poesia escrita para
ser lida, enquanto o repente ou o desafio é a poesia feita oralmente, que mais tarde pode
ser registrada por escrito. Essa divisão é muito esquemática. Por exemplo, o
cordel, mesmo sendo escrito e impresso para ser lido, costumava ser lido em
voz alta e desfrutado por outros ouvintes além do leitor. A poesia popular,
praticada principalmente no Nordeste do Brasil, tem muita influência da
linguagem oral, aproveita muito da língua coloquial praticada nas ruas e na
comunicação cotidiana. 

Naturalmente, portanto, pode"-se considerar a poesia narrativa do cordel uma
forma de poesia mais compartilhada e desfrutada coletivamente, o que dá também
uma grande ressonância social. Muitos dos temas do cordel são originários das
tradições populares e eruditas da Europa medieval e moderna. Outros temas são
retirados de tradições orientais, como no famoso poema de Patativa “A história de
Aladim e a lâmpada maravilhosa”. O personagem Aladim pertence ao Livro das mil
e uma noites, um dos famosos conjuntos de histórias de todos os tempos. Também
encontramos temas retirados das novelas de cavalaria medievais e das narrativas
bíblicas. Ao lado destes temas mais literários, encontram"-se os temas locais,
quase sempre narrados na forma de crônicas de coisas realmente acontecidas,
como em “Antônio Conselheiro” ou ``O encontro de Rodolfo Cavalcante com Lampião Virgulino''. Também há as histórias fantásticas, que se valem das tradições semirreligiosas, ligadas à experiência com o mundo espiritual. É o caso, por exemplo, da presença da figura do Diabo em um poema como ``A mulher que foi surrada pelo Diabo''.

Os grandes poemas de cordel são perfeitamente metrificados e rimados. A métrica
e a rima são recursos que favorecem a memorização e tradicionalmente se costuma
dizer que são resquícios de uma cultura oral, na qual toda a tradição e
sabedoria são sabidas de cor.  


\subsection{O sertão geográfico e cultural}

O sertão tem mitos culturais próprios. Contemporaneamente, o sertão evoca
principalmente o sofrimento resignado daqueles que padecem a falta de chuva e
de boas safras na lavoura. Evoca a experiência histórica de uma região
empobrecida, embora tenha sido geradora de riquezas, como o cacau e cana de
açúcar, ambos bens muito valiosos. 

O sertão formou também o seu imaginário por meio de grandes personalidades e
uma pujante expressão artística. Além do cordel, o sertão viu nascer ritmos tão
importantes quanto o forró e o baião. Produziu artistas tão expressivos quanto
Luiz Gonzaga, grande cantor da vida do sertanejo em canções como “Asa branca”.
Um escultor como Mestre Vitalino criou toda uma tradição de representação da
vida e dos hábitos sertanejos em miniaturas de barro. A gravura popular, que
sempre acompanha os folhetos de cordel, também floresceu em diversos pontos e
ficou mais famosa em Juazeiro do Norte, no Ceará, e em Caruaru, no estado de
Pernambuco. 

Dentre os grande mitos do sertão, está certamente o do cangaço com seu líder
histórico, mas também mítico, Virgulino Ferreira, o Lampião. Até hoje as
opiniões se dividem: para alguns foi uma grande homem, para outros um bandido
impiedoso. 

Uma figura muito presente na cultura nordestina é o Padre Cícero Romão,
considerado beato pela Igreja Católica. Consta que teria feito milagres e
dedicado sua vida aos pobres. 

\subsection{Variação linguística}

A linguística moderna usa o termo “idioleto” para marcar grupos distintos no
interior de uma língua. Um idioleto pode ser a fala peculiar de uma região, de
um grupo étnico ou de uma dada profissão. 

Uma das grandes forças da poesia popular do Nordeste se origina em sua forma
muito própria de falar, com um ritmo muito diferente dos falares do sul, e
também muito diferentes entre si, pois percebe"-se a diferença entre os falares
de um baiano, um cearense e um pernambucano, por exemplo.

Além desse aspecto rítmico, quase sempre também há palavras peculiares a certas
regiões. 

%\section{Sugestões de atividades}
%\begin{enumerate}
%
%\item \textit{Atividade de leitura}. Esta atividade tem por objetivo sensibilizar
%os alunos para a escuta de poesia. O professor deve ler um conjunto de
%estrofes para exemplificar uma leitura que se construa com uma
%pronúncia clara, pausas e ênfases adequadas. Após isso, cada aluno deve
%ler uma estrofe, procurando marcar o ritmo e as rimas, bem como as
%pausas e ênfases expressivas.
%
%Para enriquecer a experiência com outros recursos, o professor pode
%mostrar o poeta Patativa do Assaré lendo alguns de seus poemas no link
%a seguir, do site Youtube: www.youtube.com/watch?v=RTEfYnMNNpc. 
%
%Uma atividade como essa pode auxiliar no desenvolvimento da percepção da
%voz e da fala como meio indispensáveis à boa convivência social.
%
%\item “Antônio Conselheiro – O santo guerreiro de Canudos”
%Nota: Esta atividade poderá ser desenvolvida antes da leitura do poema.
%
%A situação do Nordeste brasileiro era muito difícil no final do século
%XIX. A região, que já havia abastecido a economia brasileira em outros
%tempos, passava por maus bocados com a seca, fome, miséria e violência.
%O abandono político fez com que inúmeros conflitos surgissem na região,
%dentre eles, a Guerra de Canudos, no sertão da Bahia.
%
%Os motivos aparentes da guerra eram o fanatismo religioso e o
%messianismo, mas suas razões profundas eram o latifúndio, o
%coronelismo, a escravidão, o isolamento cultural e geográfico em que
%vivia seu povo, no ambiente árido e hostil do sertão. Enquanto os
%políticos disputavam por seus interesses, algumas regiões passavam por
%uma profunda crise de miséria e pobreza. 
%
%
%a. Orientar os alunos na pesquisa de informações sobre esse conflito:
%quando começou e por quê; quantas baixas (mortes) houve de cada lado do
%conflito (soldados X “conselheiristas”) etc.
%
%b. Ao contrário do que muitos imaginam, o cearense Antônio Vicente
%Mendes Maciel – o Antônio Conselheiro – não era analfabeto nem
%ignorante: teve aulas de português, latim e francês na infância e
%trabalhou como comerciante, professor e advogado dos pobres. Ele se
%acreditava enviado por Deus para acabar com as diferenças sociais e,
%sempre que podia, reconstruía igrejas e cemitérios, coordenava mutirões
%para a construção de casas, as colheitas e a distribuição de alimentos
%entre os mais pobres. Suas ações acabaram por torná-lo popular entre os
%nordestinos, que sofriam com a miséria e a fome e viam nele a esperança
%de liberdade.
%
%Pedir aos alunos que busquem informações sobre Conselheiro em pelo menos
%duas fontes distintas: uma o retratando apenas como um líder fanático e
%outra com um ponto de vista diferente e mais favorável a ele.
%
%c. O filme Guerra de Canudos (links abaixo), dirigido por Sérgio
%Resende, retrata o episódio sob a ótica de uma família em que há
%opiniões conflitantes sobre Conselheiro. Seria interessante reunir a
%classe para uma sessão, antes da atividade final (item d). 
%
%Links para as partes 1 e 2 do filme:
%
%http://www.youtube.com/watch?v=KCH9Om\_X6Lc
%
%http://www.youtube.com/watch?v=Q3ElPSxxKzU
%
%d. Para concluir a atividade, propor um debate acerca de Conselheiro
%e sua revolta contra as forças do governo. A classe poderá ser dividida
%em dois grupos: um com argumentos a favor de Conselheiro e outro
%defendendo a posição do governo. No final do debate, propor que ambos
%os grupos reflitam sobre as consequências do conflito (e que
%responsabilidade atribuem a cada um dos lados) e sugiram como ele
%poderia ter sido evitado.
%
%Links com informações adicionais sobre Canudos e Antônio Conselheiro:
%
%http://redes.moderna.com.br/2011/10/05/fim-da-guerra-de-canudos/
%
%http://www.algosobre.com.br/biografias/euclides-da-cunha.html
%
%http://www.oolhodahistoria.ufba.br/03moura.html
%
%
%\item Orientar os alunos para uma leitura atenta do poema, situando a história
%no tempo e no espaço. Deverão identificar passagens que revelem a
%postura do poeta em relação ao conflito de Canudos e seu líder.
%Questões para disparar o debate:
%
%a. Segundo o poema, em que época Antônio Conselheiro passou a pregar?
%E quando teve início o conflito? Estas informações estão corretas?
%
%
%b. O poema cita alguns dos locais dos principais conflitos e
%massacres. Quais são eles? Tente localizá-los em um mapa atual. Quais
%ainda existem e quais já desapareceram?
%
%c. Que visão o poeta tem de Conselheiro e seus seguidores? Esta visão
%é favorável aos revoltosos ou não? Justifique sua resposta.
%
%d. Em sua opinião, a visão do poeta é perfeitamente objetiva, ou
%seja, corresponde inteiramente à realidade dos fatos?
%
%
%\item “O encontro de Rodolfo Cavalcante com Lampião Virgulino”
%
%a. Após a leitura do poema, propor aos alunos uma conversa informal,
%em que possam opinar sobre o poema e a impressão do autor a respeito de
%Lampião: O poema o apresenta de forma negativa ou positiva? Que
%qualidades (positivas e negativas) Rodolfo atribui a ele? (As respostas
%devem ser justificadas com exemplos retirados do próprio poema.)
%
%b. Pedir aos alunos que releiam o poema e reflitam sobre o
%significado e a importância para Rodolfo e seu irmão – adolescentes na
%época – de seu encontro com o famoso e temível Lampião. Como eles,
%alunos e também adolescentes, teriam reagido numa situação dessas? Os
%alunos poderão selecionar os trechos mais emocionantes do poema e
%colocar-se, então, no lugar dos dois irmãos, explicando como teriam
%reagido. Para tornar a atividade bem criativa e divertida, os alunos
%poderão substituir alguns versos de Rodolfo por outros criados por eles
%mesmos, e que expressem sua reação diante de Lampião e seu bando.
%
%c. Poucas pessoas fora do cangaço tiveram a chance de conhecer
%Lampião pessoalmente. Uma delas foi o fotógrafo libanês Benjamin
%Abrahão, que não apenas o viu em carne e osso como também o fotografou
%e filmou com Maria Bonita e outros membros de seu bando. 
%
%Os links abaixo são para dois vídeos: 
%
%O primeiro (14 min.) é a única parte que restou do documentário de
%Abrahão. Todo em preto e branco e sem som, ele mostra o próprio
%fotógrafo sendo recebido pelos cangaceiros, bem como Lampião e os
%demais em diversas situações. O texto na introdução fornece mais
%informações importantes.
%
%O segundo é o filme Baile perfumado (1h32min), de 1996, que narra a
%história desse incrível encontro.
%
%Link para o vídeo 1:
%http://www.youtube.com/watch?v =O33Flqcp5B4\&list=PL2orKizuWz-9oO05TRetvCS8 %WgHyzyh1i
%
%Link para o vídeo 2: http://www.youtube.com/watch?v =NnmWmTl217k
%
%\pagebreak
%
%\item “A mulher que foi surrada pelo Diabo”
%
%a. Seria interessante propor uma pesquisa sobre a história do
%movimento feminista no Brasil e sua influência (ou não) na sociedade,
%tanto nos grandes centros quanto nas pequenas cidades. Os alunos,
%organizados em grupos, podem investigar (em notícias ou crônicas da
%época e de hoje, vídeo-documentários, entrevistas, etc.) quais eram/são
%as reivindicações feministas, que impacto tiveram/têm nas famílias, que
%regiões ou centros urbanos do país foram/são mais afetados etc. 
%
%b. Depois de concluída a pesquisa, poderão apresentar os resultados
%obtidos. É importante que percebam que a realidade socioeconômica e
%cultural nos grandes centros difere muito daquela observada em cidades
%pequenas ou povoados, onde a vida é mais pacata e a tradição e o
%conservadorismo ainda são muito fortes. Para ilustrar seus argumentos,
%podem selecionar algumas histórias verídicas semelhantes à que inspirou
%o poema de Rodolfo. 
%
%c. Outra atividade interessante seria um debate sobre a influência da
%televisão sobre os costumes locais, especialmente através de novelas e
%programas de entretenimento. É importante que eles percebam como este
%veículo afeta a opinião pública e modifica seus valores, não só através
%dos programas como também das propagandas e do merchandising inserido
%nas novelas e séries.
%
%\item “Cuíca de Santo Amaro — O poeta popular que conheci”
%
%O documentário O cordel esquecido num país sem memória (link abaixo)
%constitui uma boa oportunidade para os alunos conhecerem o processo
%artesanal de produção dos folhetos de cordel: as tipografias, que hoje
%estão se tornando obsoletas; as antigas xilogravuras, que não mais
%estampam e enobrecem os livretos; a figura do poeta declamador, que
%emocionava o povo nas feiras e praças públicas; e o repentista, poeta
%que cria versos de improviso.
%
%Além disso, o vídeo resgata a memória dos mais importantes cordelistas
%da Bahia – Cuíca de Santo Amaro e Rodolfo Cavalcante. Isaías, filho de
%Rodolfo, concede uma entrevista em que fala do pai e sua amizade com
%Cuíca, bem como do “trato” que fizeram de prestar uma homenagem ao
%primeiro que viesse a falecer. Muito interessante.
%
%Link para o vídeo: http://www.youtube.com/watch?v= 6dKQTRQH3EU
%
%\end{enumerate}
%
%\section{Sugestões de leitura\\ para o professor} 

\begin{bibliohedra}

\tit{DIEGUES JÚNIOR}, Daniel. \textit{Literatura popular em verso}. Estudos. Belo Horizonte: Itatiaia, 1986. 

\tit{MARCO}, Haurélio. \textit{Breve história da literatura de cordel}. São Paulo: Claridade, 2010.

\tit{TAVARES}, Braulio. \textit{Contando histórias em versos. Poesia e romanceiro popular no Brasil}. São Paulo: 34, 2005.

\tit{TAVARES}, Braulio. \textit{Os martelos de trupizupe}. Natal: Edições Engenho de Arte, 2004 

\end{bibliohedra}