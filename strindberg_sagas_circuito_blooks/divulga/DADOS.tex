\titulo{Sagas}
\autor{Strindberg}
\organizador{Tradução do sueco}{Carlos Rabelo}
\isbn{978-85-7715-080-9}
\preco{22}
\pag{134}
\release{\textbf{Sagas}, de August Strindberg, traz ao público brasileiro um
conjunto de contos composto por breves peças morais e narrativas oníricas,
passando por temas históricos, humorísticos e heróicos. \textit{Sagas} foi
escrito em 1903 para a filha de Strindberg, Anne-Marie, à época com um ano de
idade, mas sem que o autor tenha deixado de lado a preocupação com o aspecto
literário, o que torna a obra leitura fascinante para todas as idades. 

Como destaca Ivo Barroso na introdução, trata-se da "oportunidade de captar uma das
facetas do grande criador de obras-primas: a de contador de histórias infantis.
\textit{Sagas} não contém apenas os feitos épicos dos grandes heróis, nem as
longas narrativas do folclore escandinavo. Há na obra lugar para o herói
mínimo, o herói obscuro, o herói negativo e até para o anti-herói. São contos
strindberguianos, de estilo personalíssimo, que antecipam a literatura
fantástica dos nossos dias. E, é claro, são ainda deliciosas histórias da
carochinha, contos infantis, relatos para crianças, escritos quase na linguagem
delas."    

\noindent Escritor em tempo integral, \textbf{August Strindberg} (1849--1912) deixou como
legado uma vasta produção de grande valor literário. A edição sueca de suas
obras completas compreende 75 volumes, sem contar os milhares de cartas que
escreveu. Dentre outras inovações, Strindberg imprimiu à literatura de sua
época um estilo coloquial, direto e realista. No Brasil, onde a produção de
Strindberg ainda é pouco conhecida, algumas de suas peças teatrais são
encenadas regularmente, a exemplo de \textit{Senhorita Júlia} e \textit{O Pai}.

\noindent A tradução, a cargo de \textbf{Carlos Rabelo}, foi feita diretamente do idioma
original, o sueco. Rabelo também traduziu \textit{Camaradagem}, peça teatral de
Strindberg. O responsável pela Introdução é o poeta, tradutor e ensaísta
\textbf{Ivo Barroso}, autor da tradução de \textit{Inferno}, outra obra de
Strindberg.

}

