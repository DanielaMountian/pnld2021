\chapter[Introdução, por Ivo Barroso]{introdução}
\hedramarkboth{Introdução}{Ivo Barroso} 


\noindent\textsc{O dia 22} de janeiro de 1912 pode ser considerado aquele em que um santo
fez milagre em sua própria terra: uma procissão de estudantes e
trabalhadores, empunhando tochas, marcha para o nº 85 da
Drottninggatan, em Estocolmo, onde reside August Strindberg. É o dia de
seu aniversário (que será o último) e a multidão vem aplaudi"-lo em
reconhecimento de seu gênio. Durante grande parte de sua vida
(1849-1912), ele fora vítima da incompreensão de seus compatriotas,
do escárnio da sociedade calvinista e conservadora de seu país, do
desprezo dos críticos e de seus colegas escritores, banido dos teatros
e dos cargos públicos. Além disso, a Academia Sueca, outorgante do
prêmio Nobel de literatura, durante décadas, lhe havia subtraído
sistematicamente a consagradora láurea, embora a contribuição de
Strindberg no campo das letras fosse mundialmente reconhecida. A
procissão daquele 22 de janeiro, além de louvar os méritos do
homenageado, tinha a finalidade de outorgar"-lhe o ``Anti"-prêmio
Nobel'', de reconhecimento literário generalizado e não apenas
acadêmico, que consistia numa expressiva soma em dinheiro obtida por
subscrição nacional.

August Strindberg era sem dúvida um homem de gênio: fotógrafo avançado
para o seu tempo na arte dos efeitos luminosos e das extravagâncias de
composição e enfoque, teria sido certamente um grande diretor
cinematográfico à maneira de seu compatriota Ingmar Bergman. Artista
plástico amador, conviveu com Paul Gauguin, e foi o primeiro a
reconhecer o talento do pintor francês. Escritor, jornalista, cronista,
contista, ensaísta, dramaturgo e poeta, escandalizou seus
contemporâneos com a crueza dos temas de seus romances, entre os quais
\textit{A defesa de um louco} e \textit{O filho de uma criada}, ambos
de fundo biográfico, em que narra as suas desditas matrimoniais e as
agruras que sofreu em consequência da condição social inferior de sua
mãe, que tinha sido a governanta de seu pai. Em 1884, ele e seu editor
foram processados por blasfêmia em razão de passagens irreverentes
sobre a instituição matrimonial contidas em seu livro de contos \textit{Giftas}
(Casados). O insucesso de seu casamento com a atriz e baronesa Siri von
Essen levou"-o a um extremo estado de misoginia, refletido em suas
obras, capaz de chocar a sociedade estrita e conservadora de seu tempo.
Siri von Essen, de origem finlandesa, era casada com o decadente barão
Carl Gustaf Wrangel, mas tinha ambições de atriz. Strindberg
conheceu"-a no teatro onde tentava encenar uma de suas inúmeras peças.
Ficou amigo do barão e tomou"-lhe a esposa, com a atenuante de que o
marido queria mesmo livrar"-se dela para assumir seu romance com a
sobrinha. Siri obtém o divórcio e casa"-se com Strindberg em 1877, mas
o \textit{affaire} resultou em escândalo e acusações de traição,
tornando insuportável a vida do casal na provinciana Estocolmo da
época. Embora Strindberg tenha obtido sucesso com seu drama
\textit{Mäster Olof} em 1881, com o que inaugura o realismo e o drama
social no teatro, o casal acabou, num exílio voluntário, viajando por
vários países da Europa para afinal fixar"-se na França. Em 1888, já
estão na Dinamarca, onde Strindberg escreve seu mais conhecido drama,
\textit{Senhorita Júlia}, protagonizado pela esposa, Siri, no ano
seguinte, no Teatro Experimental Escandinavo, fundado por ele. A peça
retrata a luta de classes sociais, com um olhar incisivo sobre a
ascensão do proletariado e a prevalência dos instintos sobre a razão,
conceitos que iriam influenciar o movimento expressionista alemão e
fazer a cabeça de futuros dramaturgos como Antonin Artaud, Luigi
Pirandello, Eugene O´Neil e Arthur Adamov. Até hoje representada em
todo o mundo, já teve várias encenações, inclusive recentes, no Brasil.

As relações do casal nunca tinham sido das melhores embora gerassem três
filhos; agravadas pelas dificuldades econômicas, acabam por se
deteriorar nos últimos anos e conduzir inevitavelmente ao divórcio em
1891. August vai residir em Berlim, onde participa de um círculo
artístico de vanguarda na companhia do pintor norueguês Edvard Munch e
do poeta polaco Stanislaw Przybyszewski. Entra em crise, deixa de
escrever para o teatro por uns tempos para se dedicar às ciências
naturais, à pintura e à fotografia. Envia, de quando em quando,
estranhos artigos pseudo"-científicos para revistas francesas
dedicadas ao ocultismo, com o que obtém algum dinheiro para o seu
sustento, coberto igualmente pela ajuda financeira de alguns
compatriotas. Em 1893, conhece a jornalista austríaca Frida Uhl, com
quem se casa e vai morar com a família desta, na Áustria. Frida dá à
luz a menina Kerstin, que exercerá vital influência sobre a vida do pai. 

O casamento durou pouco. Strindberg despede"-se da esposa e da filhinha
em 1894 e vai viver em Paris, onde sofre uma crise depressiva que o
leva à beira da loucura. Morando em pensões miseráveis, assusta a todos
com suas experiências alquímicas no quarto, tentando converter metais
em ouro, enchendo o ambiente de vapores pestilentos e colocando o
albergue sob ameaça de incêndio. Não tarda entrar em surto e ser
recolhido ao Hospital de La Salpetrière, uma espécie de manicômio de
Paris. São seus ``anos infernais'', que ele descreve em seu livro escrito
em francês, \textit{Inferno},\footnote{ Ver August Strindberg, \textit{Inferno},
trad.~Ivo Barroso (Hedra: São Paulo, 2010). [N.~da~E.]} no qual narra
sua redenção graças ao pensamento na filhinha Kerstin, que lhe surge
na mente como um anjo da guarda. Volta a escrever e compõe a primeira
parte de sua trilogia \textit{A caminho de Damasco}, que viria a ser o
ponto de partida do teatro expressionista do século \textsc{xx}.

De regresso à Suécia, em 1899, ao completar 50 anos, Strindberg escreve
a seu amigo Nils Andersson, pesquisador de música folclórica e notário:
``Agora que retornei a minha terra e suas paisagens, conquistei uma
ligação com meu passado e minha juventude; encontro"-me cada vez mais,
crio novas raízes e aos poucos o caule começa a enverdecer''. A luminosa
paisagem do arquipélago de Estocolmo, depois de uma ausência de sete
anos no exílio, traz de volta seu sentimento pela terra natal,
infunde"-lhe uma visão otimista e colorida que vai se refletir também
em sua obra. No verão de 1900, escreve na pequena ilha de Furusund,
próxima de Estocolmo, a leve e folclórica comédia de verão
\textit{Midsommar} (Solstício de verão), a que ele no entanto dá o
subtítulo de ``Uma comédia séria em seis quadros'', já que usa os
personagens e a paisagem campestre para expor vários tópicos da reforma
social, tais como bem"-estar, democratização e educação. Volta seu
interesse para os temas históricos e produz vários dramas sobre reis e
heróis suecos. Consegue encenar sua peça \textit{A caminho de Damasco},
trilogia (iniciada em 1898, mas só concluída em 1904) que retrata de
maneira simbólica os lances de seu casamento com Siri von Essen.
Strindberg escolhe, para viver o papel da protagonista, a jovem atriz
sueco"-norueguesa Harriet Bosse, por quem -- era fatal! -- se apaixona
profundamente e se casa em 1901.

August Strindberg foi escritor em tempo integral. A edição nacional
(sueca) de suas obras completas compreende 75 volumes e nelas não se
incluem os milhares de cartas que escreveu, quase todas de grande teor
literário. Infelizmente, no Brasil sua produção é pouquíssimo
conhecida: a não ser o drama \textit{Senhorita Júlia} e a autobiografia
\textit{Inferno}, raros são os leitores que tiveram \textit{A dança da
morte} e \textit{Gente de Hemsö} em mãos. Daí a excelente
contribuição que a Hedra traz ao nosso público dando a lume mais esta
faceta do grande escandinavo: a de contador de histórias infantis. Mas,
em se tratando de um escritor da estirpe de Strindberg, é bom assinalar
que essas ``histórias para crianças'' têm uma dimensão e uma profundidade
que transcendem o conceito de literatura infantil. São sagas, e
fábulas, e lendas, e sátiras, e parábolas, e contos, e divagação, e
sonho, e poesia. Um modo de se captar, com um pequeno livro, grande
parte da essência multíplice desse grande criador de obras"-primas.
