\chapter{Strindberg para crianças?}

\section{Sobre o autor}

Johan August Strindberg (Estocolmo, 1849--\textit{id.}, 1912) 
foi escritor, dramaturgo, pintor e fotógrafo sueco. Após concluir
seus estudos, dedica"-se à carreira de professor, ao mesmo tempo em que
estuda medicina. Mais tarde tenta lançar"-se como ator, mas em 1870 vai
estudar na universidade de Uppsala, onde começa a escrever. Dois anos
mais tarde interrompe os estudos por razões financeiras; passa a
trabalhar no jornal \textit{Dagens Nyheter} e, a seguir, na
Kungliga Bibliotek -- a Biblioteca Nacional da Suécia.
Em 1879, a publicação do livro \textit{Röda Rummet} (A sala vermelha) e
a encenação da peça \textit{Mäster Olof} trazem"-lhe o reconhecimento
merecido. Em 1882, o aparecimento de \textit{Det Nya Riket} (O novo
reino) -- obra de cunho realista, repleta de críticas às instituições
sociais vigentes na época -- rende"-lhe tantas críticas que o autor vê"-se
obrigado a deixar seu país natal. Strindberg muda"-se com a primeira
mulher, Siri von Essen, e os filhos para Paris e então para a Suíça. No
exterior, escreve uma parcela significativa de sua obra, ao mesmo tempo
em que luta contra graves problemas psicológicos. Em 1897, após
divorciar"-se de Frida Uhl, sua segunda esposa, a condição mental de
Strindberg -- já delicada na época -- deteriora"-se ainda mais. Em um
período de crise profunda, atormentado pela paranoia e por surtos
psicóticos, escreve o romance \textit{Inferno}. Após tornar à Suécia em
1897, casa"-se pela terceira vez, em 1901, com a atriz Harriet Bosse,
que lhe dá a filha Anne"-Marie. Nesse período, as leituras e crenças
pessoais de Strindberg influenciam seu estilo, que passa do realismo ao
expressionismo. Rechaçado pela Academia Sueca, que até 
hoje concede o Prêmio Nobel, Strindberg foi agraciado com uma 
distinção sem precedentes: o Prêmio Anti"-Nobel, uma arrecadação 
pública de dinheiro promovida por seus conterrâneos. 
No fim da vida, o autor instala"-se na Blå Tornet -- a “Torre azul” 
onde hoje funciona o museu em sua memória.
Strindberg morreu no dia 14 de maio de 1912, deixando como legado uma
vasta produção de grande valor literário -- entre elas, as peças
\textit{Senhorita Júlia}, \textit{A dança da morte}, \textit{O pai},
\textit{A caminho de Damasco} e \textit{A sonata espectral}, além dos
romances \textit{Inferno}, \textit{O~filho da criada}, \textit{Defesa
de um louco} e \textit{Gente de Hemsö}.

\section{Sobre a obra}

Em 1902, após ver seu terceiro casamento fracassar, Strindberg escreve estas \textit{Sagas} para a recém"-nascida Anne"-Marie, filha do escritor com a atriz Harriet Bosse. Embora separado da esposa, Strindberg mantinha com ela e a menina relações
cordiais, motivo de composição dessas fábulas, pelas quais o escritor reforçava os laços com Bosse e a pequena Anne"-Marie.

Publicadas um ano depois, em 1903, \textit{Sagas} reúne contos que vão de breves peças morais a narrativas oníricas, passando por temas históricos, humorísticos e heróicos com igual desenvoltura, trazendo ora um tom fabulesco, ora um tom decididamente strindberguiano. Grande parte dessas histórias 
foram inspiradas por acontecimentos na vida do autor. Ainda que as tenha escrito para a filha Anne'-Marie, Strindberg não deixou de preocupar"-se com seu aspecto literário, o que faz dessa obra uma leitura fascinante para todas as idades.

Ao longo das histórias, o narrador aborda a história da Suécia, fábulas regionais, acontecimentos contemporâneos, entes mitológicos e lições de moral destinadas à filha. Em várias narrativas pode"-se observar a influência que animais e plantas exercem na ordem humana, traço típico das fábulas. Em ``Nos dias de verão'' uma pomba canta as alegrias do reino de Deus, pinheiros e rouxinóis vão à ajuda de uma desamparada mãe com sua filha. Já em ``A grande peneira'', as principais personagens são enguias, percas, bacalhaus e outros animais marinhos. 

O viés moral, com aconselhamentos éticos para a vida, é nítido em três narrativas em especial: ``Os segredos do celeiro de fumo'' ``Jubal sem identidade'', ambos sobre artistas desgraçados pela sua arrogância, quando deixam o talento subir à cabeça; e ``O herói e o bobo'', em que o espelho deformado do herói, na figura do bobo, serve à retificação de caráter do herói, enquanto o bobo cai em desgraça por incorporar a \emph{hybris} do herói.

Ressaltam"-se igualmente as narrativas em que fica evidente o domínio de Strindberg em manipular o real e o sonho, criando cenas que navegam por diversos espaços e tempos. Em ``Nos dias de verão'', mãe e filha embarcam sozinhas em uma embarcação que as leva para outra terra, semelhante ao paraíso. Os efeitos do sol em um quarto super-aquecido fazem do protagonista de ``O dorminhoco'' alucinar um cenário feérico em sua casa. O sol, incidindo sobre as lembranças de sua esposa morta, criam outro espaço ilusório para o dorminhoco músico que protagoniza a história.

Em ``Agruras do timoneiro'', essa confusão entre o espaço real e onírico é alçada ao máximo, trazendo o narrador a confusão de sua própria filha ao ouvir a história. Como escreve o tradutor e escritor Ivo Barroso:

\begin{quote}
Em ``As agruras
do timoneiro'', como que escrito para usar termos náuticos da arte de
velejar (que todas as crianças suecas aprendem na escola), Strindberg
toma Anne"-Marie como personagem da história que se complica com tal
número de elementos oníricos e surreais, a ponto de a menina perguntar
no fim: ``Mas como foi que o timoneiro conseguiu passar do barco para o
passeio? E depois, ele voltou ou foi tudo um sonho?'' A saída é fazer
com que a mãe (também personagem da narrativa) assim responda: ``Isso eu
conto outra hora, menina curiosa''. Neste mesmo conto há referência a
conchas marinhas, que são descritas minuciosamente e chamadas por seus
nomes científicos ou populares, certamente uma lembrança da Birger
Jarlspassagen, de Estocolmo, com suas lojas de conchas e apetrechos
náuticos no tempo de Strindberg.\footnote{\textsc{barroso}, Ivo. ``Introdução''. In: \textsc{Strindberg}, August. \textit{Sagas}. São Paulo: Hedra, 2008, p.\,14.}
\end{quote}

Já em ``A história do São Gotardo'' percebemos um Strindberg mais realista, próximo a elementos contemporâneos, como a perfuração de um túnel que uniu a Suíça à Itália em 1880. Com marcas de uma narrativa jornalística, a narrativa acompanha a trajetória de Andrea, italiano que participa das escavações para chegar à sua amada Gertrud na Suíça, a quem prometera atravessar montanhas pelo amor. A narrativa, que acompanha os trabalhos de Andrea perfurando a montanha com sua picareta, demonstram que o autor certamente visitou o local das obras e se inteirou do processo de escavação do túnel. Vale lembrar que na época, em 1883, Strindberg estava morando na Suíça.

Um dos pontos altos do livro talvez seja ``A grande peneira'', descrita com maestria por Ivo Barroso:

\begin{quote}
Claro que o humorismo não podia ficar de fora desses contos encantadores
por sua variedade estilística e conceitual. Em ``A grande
peneira'', os personagens principais são uma enguia e seu filhote que
assistem das profundezas a queda de um piano que estava sendo
transportado por um barco a vapor, de propriedade de um inspetor de
minas. Ao ser arrastado pela ponte, o piano cai ao mar e todos os
peixes em redor vêm sondar o inusitado objeto que emite sons, cada qual
dando seu palpite sobre o que seria aquela forma estranha: um armário
com espelho! uma armadilha! um tear! Quem acerta a charada é a perca
que diz ser uma peneira. Uma peneira de mineração. E aqui há um belo
jogo de palavras do original, infelizmente irreproduzível numa
tradução. Essa peneira -- um grande retângulo de madeira trançado
de arame -- é, em sueco, \textit{grusharpa}, ou seja, literalmente,
``harpa de cascalho'', e, tendo assim a aparência de instrumento musical,
podia perfeitamente soar sob as águas. Um cardume de peixes"-espada
passa por dentro da caixa tangendo as cordas com as caudas,
fazendo"-as soar ``de um jeito diferente, pois agora estavam em outra
afinação''. Que bela imagem, digna dos melhores contadores de
histórias.\footnote{Ibidem.}
\end{quote}


\section{Sobre o gênero}

\begin{quote}
Para estreitar esse convívio, Strindberg elabora uma série de contos
destinados não só a divertir a pequenina Anne"-Marie, mas igualmente
para introduzi"-la nos ensinamentos primários da ética, da história,
da política e do convívio social. O título abrange esse complexo de
intenções. Sagas não são apenas os feitos épicos dos grandes heróis,
nem as longas narrativas do folclore escandinavo. Há aqui lugar para o
herói mínimo, o herói obscuro, o herói negativo e até para o
anti"-herói. Além disso, são igualmente fábulas por terem seu conteúdo
moral, exemplificativo, no personagem que serve de modelo para a
formação do caráter. E são contos, contos strindberguianos, de estilo
personalíssimo, que antecipam a literatura fantástica dos nossos dias,
na qual o elemento onírico assume papel de destaque, permitindo a
brusca mudança de cenários e situações, sem que haja no final a
necessidade de explicar as razões do ocorrido. E, é claro, são ainda
deliciosas histórias da carochinha, contos infantis, relatos para
crianças, escritos quase na linguagem delas, vivamente inspirados nas
estripulias de Andersen e nas travessuras de Selma Lagerlöf.\footnote{Ibid., p.\,13.}
\end{quote}

A partir da conceituação de Ivo Barroso, acima exposta, percebe"-se a dificuldade de inserir as \textit{Sagas} dentro de um determinado e fechado gênero literário. Elas permitem aproximações com o conto, a narrativa folclórica, a escrita jornalística e a fábula infantil. 

Pensando"-se nesse último gênero, na perspectiva de que, afinal, são histórias escritas para sua pequena filha, pode ser interessante relembrar a estrutura que o intelectual russo Vladimir Propp aplica às narrativas infantis. Para Propp, a história é estruturada no dano ou na carência inicial do herói, que vai passar por diversas fases e situações para recuperar seu estado inicial, voltando ao contexto de normalidade do cotidiano.

Em ``Nos dias de verão'', por exemplo, para suprir a ausência do pai e do marido, uma mãe sai com sua filha em direção à cidade, para realizar os desejos da pequena. Para lá chegar, no entanto, as duas terão que passar por um caminho de provação: são seis cercas e seis porteiras até chegar à cidade, e cada uma guarda um perigo e uma provação para a sensibilidade das duas protagonistas. Seja um tropel de cavalos, um ataque de carneiros, a cegueira provocada pela neblina no meio de um pântano, ou o repentino surgimento de um touro desgovernado, sempre aparece algum elemento a impedir o passo de mãe e filha e ameaçar"-lhes as vidas. Sempre, igualmente, uma força mágica aparece ao seu favor: uma pomba que canta o caminho correto, um pinheiro que se verga para tirá"-las de uma situação perigosa, ou um rouxinól que espanta o tropel de bestas.

Em outra história, ``Quando a
andorinha pousou no espinheiro'',  um condenado à prisão perpétua habita
o presídio em uma ilhota de pedra, junto ao ancoradouro, e nela passa a
existência a quebrar o mineral, a ponto quase de se transformar em um homem de pedra. Sua existência ali, como dos demais presos, é miserável: as condições no presídio são precárias, e passam o dia na montanha pedregosa, sem nenhuma vegetação, debaixo do sol calcinante. Governadores ascedem e caem sucessivamente, mas a condição dos condenados não se altera. Certa feita, são convocados para criar um novo acesso a barcos no porto, e assim tem de dragar toda a imundície do lago:

\begin{quote}
Começaram a tirar todas a sujeira que havia no fundo do lago. Apareceram
gatos afogados e sapatos velhos, gordura podre da fábrica de vela de
estearina, filamentos coloridos da tinturaria Mão Azul, cascas de
árvore do curtume e todas as misérias humanas, que as lavadeiras por
cem anos enxaguaram nas margens. O fedor de enxofre e amoníaco era tão
forte que apenas os prisioneiros suportavam"-no.\footnote{Nesta edição, página \pageref{lago}.}
\end{quote}

O governo os obriga a jogar esse lixo na própria montanha pedregosa em que ficava o presídio, e assim passam a viver entre o lixo: ``O ar logo
ficou empesteado; Os presos andavam no meio da imundície, sujando as
roupas, as mãos e o rosto''.\footnote{Ibidem.} Após uma longa vida de provação, totalmente sem esperança, o prisionerio volta para essa montanha, de onde tinha sido afastado para cumprir outras tarefas, e percebe estupefado que todo aquele lixo criara lama, terra e, daquelas primeiras ervas"-daninhas que nasceram ali,  surgira uma exuberante mata. Ao final, percebe que chegou ao paraíso pela redenção de seu crime. Novamente, percebe"-se a estrutura de Propp em funcionamento: o herói --- no caso um anti"-herói, visto tratar"-se de um criminoso --- que passa por tremendas injustiças e, quase sem esperança, recebe a chave mágica para o paraíso.

Como ressalta Ivo Barroso, no entanto, mesmo através do tom infantil o autor não deixa de tecer críticas ao seu tempo:

\begin{quote}
Embora trabalhando com temas infantis, Strindberg não perde a chance de 
abordar, aqui também, os conflitos sociais, de criticar o sistema
político sueco, fazendo desfilar em ``Os Elmos Dourados de
\r Alleberg'', diante do gigante Svensk (que representa a Suécia), todos os
grandes governantes e eventos notáveis do passado. A cada nome ou fato
citado, o contista faz um comentário abonador ou desairoso, numa aula
de história sintética que certamente passou de liso pela cabecinha de
Anne"-Marie.
\end{quote}