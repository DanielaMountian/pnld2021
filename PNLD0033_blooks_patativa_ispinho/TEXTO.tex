\chapter{Patativa do Assaré -- Depoimento}

Eu nasci no dia 5 de março de 1909, no lugar denominado Serra de
Santana, que fica no interior do Estado do Ceará, pertencendo ainda a
região do Cariri. A Serra de Santana esta distante da cidade de Assaré
18km. O meu pai, um pobre agricultor, Pedro Gonçalves da Silva, e mi-
nha mãe, Maria Pereira da Silva. Deste casal nasceram cinco filhos:
José, Antônio, Joaquim, Pedro e Maria. Eu sou o segundo filho, o
Antônio Gonçalves Silva. Quando meu pai morreu, eu fiquei apenas com 9
anos de idade. Meu pai morreu muito moço. E eu, ao lado dos meus irmãos
e da minha mãe, tivemos que enfrentar a vida de pobre agricultor, no
diminuto terreno que meu pai deixou como herança. Na idade de 12 anos
eu freqüentei uma escola lá mesmo no campo, onde vivia e onde ainda
estou vivendo. Nesta escola o professor era muito atrasado, embora muito
bom, muito cuidadoso, mas o coitado não conhecia nem sequer a pontuação.
Eu aprendi apenas a ler, sem ponto de português, sem vírgula, sem ponto,
sem nada, mas como sempre a minha maior distração foi a poesia e a
leitura, quando eu tinha tempo, chegava da roça, ao meio-dia ou à noite,
a minha distração era ler, ler e ouvir outro ler para mim, o meu irmão
mais velho, José. Ele lia sempre os folhetos de cordel e foi daí de
onde surgiu a minha inspiração para fazer poesia. Eu comecei a fazer
verso com 12 anos de idade. E continuei sempre na vida de agricultor e
ali entre meus irmãos e ao lado da minha mãe. Com 16 anos, eu comprei
uma viola e comecei a cantar de improviso. Naquele tempo, 16 anos, eu já improvisava, mesmo glosando, sem ser ao pé
da viola. Comprei a viola e que comecei a cantar também, não fazendo
profissão. Eu cantava assim por esporte, atendendo convite especial,
renovação de santo, casamento que não ia haver dança, também aniversá-
rios de pessoas amigas. O certo que eu só cantava ao som da viola
atendendo convite especial.

Com 20 anos de idade, um primo legítimo da minha mãe, um negociante que
morava no Pará, veio visitar a família que aos 15 anos havia saído do
Assaré, então foi à casa da minha mãe e me ouviu cantar ao som da viola.
Ficou encantado e maravilhado com os meus improvisos e pediu
carinhosamente à minha mãe para que deixasse eu ir com ele ao Pará, que
custearia todas as despesas e ela não tivesse cuidado que eu voltaria
quando quisesse. Então, a minha mãe, muito chorosa, pela amizade e
atenção que tinha ao primo, consentiu que eu fosse. Eu viajei ao Pará,
eu tinha 20 anos naquele tempo. Viajei com tio, chegando lá ele me
apresentou ao escritor Cearense José Carvalho de Brito, autor do livro
``O Matuto Cearense e o Caboclo do Pará'', em cujo volume eu tenho um
capítulo. José Carvalho me recebeu com a maior atenção e me pediu uns
versos para publicar no ``Correio do Ceará''. E ele era redator do
``Correio do Ceará''. Ele colaborava no ``Correio do Ceará''. Então, no
final dos versos, ele faz a apreciação dele, fazendo uma referência
sobre meus versos e disse que a espontaneidade da minha poesia tinha
semelhança, se assemelhava ao canto sonoro da patativa do Nordeste, a
nossa patativa aqui do Ceará. E então o jornal circulou, daquele tempo
para cá, eu já com 20 anos, foi que começaram a me chamar Patativa.
Posso dizer que foi José Carvalho Brito que pôs esse apelido que o povo
hoje conhece, esta alcunha. Patativa do Assaré. Depois começou a surgir
outro Patativa por aí afora também fazendo versos, cantando ao som da
viola, e o povo, para distinguir, quando se falava de uma poesia que
povo gostava, perguntava logo: é o do Patativa do Assaré? Só quero se
for do Patativa do Assaré, sendo do Patativa do Assaré eu quero. Então
começaram a me tratar Patativa do Assaré. E com muito direito, porque
Assaré é a minha terra, a minha cidade. Sim, como eu ia dizendo, lá em
Belém do Pará eu desci para Macapá, onde morava outro primo legítimo
da minha mãe. Lá eu passei um dia, mas achei a vida trancada, uma vida
insípida, uma vida sem distração, eu só podia sair de dentro de uma casa
se levado por outra pessoa, porque lá a gente sai de dentro de casa já é
na canoa, desce da porta não é escada, sai de dentro da canoa e então
vai para outra casa, que é tudo alagado. Então eu não agüentei e
passei apenas dois meses lá. Voltei a Belém do Pará, para casa do outro
tio, daí fui às colônias do Pará, cantei com os cantadores das colônias,
Francisco Chagas, Antônio Merêncio, Rufino Galvão, mas a saudade danada
não me deixou demorar mais no Pará. Passei apenas 5 meses e tantos dias
e voltei ao Ceará.

De volta ao Ceará, José Carvalho de Brito, que era muito amigo de Dra.
Henriqueta Galeno, filha do afamado poeta Juvenal Galeno, me deu uma
carta de recomendação para a Dra. Henriqueta Galeno. Eu, chegando
aqui, entreguei a carta, ela leu e me recebeu no salão de Juvenal
Galeno, como ela sempre recebeu um poeta de classe, um poeta de cultura,
um poeta erudito. Ali fiz alguns improvisos, cantei ao som da viola,
porque eu trazia minha viola. Então eu voltei novamente ao Assaré, 5
meses e tantos dias. Cheguei lá, recomecei a minha vida de roceiro,
sempre trabalhando e sem nunca mais viajar, porém sempre fazendo
versos. E já tinha uma farta bagagem de produções, quando o latinista
José Arrais de Alencar, vindo do Rio de Janeiro, onde ele morava,
visitar a D. Silvinha, a sua mãe, ouviu o programa na rádio Araripe,
onde eu estava recitando verso.
Perguntou de quem era, quem era aquela pessoa que recitava versos tão
dignos de atenção e próprios de divulgação. Aí disseram a ele que era um
caboclo, um roceiro, um agricultor. Então ele mandou me chamar. Eu fui
à presença dele, ele ficou muito satisfeito, recitei muita poesia para
ele, pois a minha bagagem, que dava um volume, eu tinha toda na mente,
toda guardada na memória, e ele perguntou: porque você não publica
essa poesia, esta coisa tão admirável que você tem, tantos versos
próprios de divulgação! Eu respondi: doutor, porque eu não posso, eu sou
pobre, sou roceiro, nem sequer nunca pensei em publicar alguma coisa.
Ele disse: pois você vai publicar o seu livro. Você, eu publico seu
livro e você pagará o impresso com a venda do próprio livro. Então eu
respondi: doutor, e se o livro não tiver sorte, como é que acontece?
Então ele disse: você é um vencido, não tem coragem. E com certeza a sua
honestidade é grande. Ficou com medo de ficar devendo algum di-
nheiro? Não, não acontecerá isso. E se assim acontecer, você não ficará
devendo um vintém a seu ninguém. Você não tá pedindo para ninguém
publicar seu livro. E na presença estava o Dr. Moacir Mota, que era
gerente do Banco do Brasil na cidade do Crato, filho do saudoso Leonardo
Mota, poeta, e se ofereceu para datilografar as minhas produções, sem me
cobrar um vintém. E assim fez. A cópia foi datilografada na cidade de
Crato pelo Dr. Moacir Mota, foi remetida para o Rio de Janeiro, lá o
Dr. José Arrais de Alencar, esse latinista, homem de profundo
conhecimento, publicou meu livro na editora Borçoi e remeteu para o
Banco do Brasil, foi guardado no Banco do Brasil, de onde eu tirava os
volumes e vendia aí pelo Assaré, no meio do meu conhecimento. Fizemos
lançamento também no Crato, o certo é que eu paguei com facilidade o
impresso desse livro. No ano de 66, o mesmo livro foi editado, a 2a
edição, com o aumento de um livrozinho que eu tinha, com o título
\emph{Cantos de Patativa}. \emph{Cantos de Patativa} era um livrozinho que
eu tinha, um livro inédito, com o qual eu ampliei o ``Inspiração
Nordestina'', na sua 2a edição, na mesma editora Borçoi, no Rio de
Janeiro onde estive 4 meses. Lá no Rio de Janeiro, quando saiu a
impressão do meu livro, eu tive um dos prazeres maiores da minha vida. É
que lá, eu sem conhecimento para a venda do meu livro, e o Borçoi, o
dono da editora, publicou fazendo o mesmo negócio, para eu pagar com a
venda do próprio livro, eu dando apenas uma entrada de meu. Aí então eu
sabendo que no Ceará era onde eu poderia vender com facilidade, fui à
presença do Dr. Borçoi e falei para ele: digo, doutor, eu venho aqui
tratar de negócios com o senhor. É que o meu livro aqui eu não posso
vender com facilidade, não tenho conhecimento, sou muito tímido, sou
muito pessimista e eu vou voltar ao Ceará, que lá eu vendo e então
enviarei o dinheiro. Venha para que a gente assine aqui um documento,
uma promissória de tudo e ele respondeu: poeta, tem quatro meses que
você está aqui no Rio, eu já estou lhe conhecendo, já conheço assim, fi-
quei conhecendo a sua índole, a sua honestidade, sua capacidade. Olha,
volta lá para o teu Ceará, com os teus livros, eu apenas te dou este
cartão do banco para onde você vai remeter o dinheiro e pode me pagar
parceladamente e eu estou confiando e sei que recebo o dinheiro. Ora, eu
voltei muito satisfeito dele me confiar e, se eu tinha desejo de pagar
com brevidade, ainda mais me cresceu esse desejo de fazer isso com a
maior facilidade.

No ano de 70, o Prof. J. de Figueiredo Filho publicou um livro, esse
livre eu não posso dizer que ele é meu, porque o comentarista do livro
é o J. de Figueiredo Filho. A poesia é toda minha, mas o livro foi
apresentado por ele, que é: ``O Patativa do Assaré''. Ele mesmo se
explica e diz: o livro é meu? Não, o livro não é meu. O livro é do poeta
Patativa. Eu sou apenas o comentarista do livro, sou apenas
o apresentador. Então o ``Patativa do Assaré'' já foi esgotado, o
``Inspiração Nordestina'' foi também esgotado, eu só tenho publicado os
meus livros por iniciativa dos homens de cultura, como agora mesmo o
``Cante Lá que Eu Canto Cá''. O ``Cante Lá que Eu Canto Cá'' foi
iniciativa do homem de letras, o prof. Plácido Cidade Nuvens.

Ele veio a mim e disse: olhe, vamos publicar o seu livro, o livro, um
novo livro. Eu disse: você pode? Pode, porque a fundação Pe. Ibiapina
está aqui para trabalhar e apresentar aquilo que de melhor tem na região
e eu não vejo outra coisa melhor do que a sua capacidade de fazer
versos, essa sua cultura popular, esse seu pensamento de penetrar em
todos os assuntos sociais e cantar a vida do povo. E nós vamos publicar
o seu livro. Eu mesmo serei o portador para me entender com a Editora
Vozes, faço negócio e então vamos publicar o seu livro. Eu faço isso
não é interesse de você ganhar dinheiro porque o poeta, aqui no Brasil,
ele não ganha dinheiro, mas ele é a riqueza da divulgação. É um
documentário que eu quero deixar aqui na fundação Pe. Ibiapina e esse
documentário ficará não só aqui como em outros lugares. E assim foi, que
a minha vida tem sido assim. Tido isso sem eu deixar meu trabalho na
roça. Eu nunca de mim próprio procurei voluntariamente publicar um
livro. São os apreciadores, os interessados pela cultura popular que me
procuram, pois até mesmo da Inglaterra veio o Dr. Collin à minha casa,
passou aí 3 dias, conversou muito comigo, é um escritor que já escreveu,
já tem livros publicados, como ele tem um livro de título Gente da
Gente, que é sobre os índios da Guiana Inglesa. Recebi uma carta desse
escritor, Dr. Collin, lá de Londres, pedindo licença para traduzir o meu
livro ``Cante Lá que Eu Canto Cá'' em língua inglesa. Eu disse a ele que
sim. E ele disse: olha, Patativa, o apresentador e o tradutor que sou
eu, não quero ganhar um centavo nesse trabalho. Será todo seu.
E você querendo poderá oferecer à Fundação Pe. Ibiapina. Pois bem, isto
aqui é uma história, é uma parte da história da minha vida. Tudo isso eu
tenho feito sem deixar o meu trabalho de roça lá na Serra de Santana,
lugar onde eu nasci, tenho vivido e hei de viver o resto da minha
vida, porque nunca me habituei à vida da cidade, sempre o meu mundo
foi a minha poesia e a minha família e aonde eu quero viver o resto da
minha vida. E ali mesmo eu hei de morrer, se Deus quiser, um dia feliz.

Voltando do Pará e demorando uns dias aqui em Fortaleza, fui parar no
Assaré. Lá recomecei minha vida de agricultor, nos meus 21 anos de
idade. Quando cheguei aos 25 anos de idade eu casei com uma serrana, uma
rapadeira de mandioca, uma cabocla que eu já conhecia desde menina,
que é a Belarmina Gonçalves Cidrão, conhecida por Belinha. Aí comecei,
recomecei, continuei a minha vida de casado, me senti muito feliz e essa
felicidade ainda hoje continua. Sou pai de 7 filhos, 4 homens e 3
mulheres: Afonso, Pedro, Geraldo, João Batista, Lúcia, Inês e Miriam.
É esta a minha família, é esse meu mundo que me sinto feliz e vivo entre
eles e é por isso que eu quero estar semprte na Serra de Santana, pois é
onde está toda essa minha família, todos continuando na vida do velho
Patativa, tratando da agricultura, naquela vida pobre do camponês, a
minha espossa muito paciente, muito trabalhadora, muito carinhosa e
graças a Deus já estou com 70 anos, mas minha felicidade sempre
continua, porque a felicidade para mim não é possuir dinheiro, não é ser
um fazendeiro, não é esse estado financeiro muito fraco. A felicidade
consiste em a pessoa viver dentro da harmonia com todos e principalmente
com seus familiares. E é por isso que me sinto muito feliz.

Em 1973, sendo convidado aqui para o sesquicentenário de Fortaleza, no
mês de agosto, tive a infelicidade de ser acidentado. Ia atravessando a
Av. Duques de Caxias, fui colhido por um carro e quando recobrei os sentidos eu já estava em cima
da cama de operação, no hospital, e então foi uma infelicidade para mim.
Bota gesso, tira gesso, e ali passei 11 meses e não recuperei. Então
resolvi ir ao Rio de Janeiro, pois eu tenho parentes e amigos. Dr. Mário
Dias Alencar, que é meu parente e é filho de Assaré, mandou me buscar
para o Rio de Janeiro. Ele não é ortopedista, ele é operador de outras
coisas, viu? Mas me pôs lá no Hospital
S. Francisco de Assis, onde um professor de ortopedia operou minha
perna, pelejou, ainda houve duas operações, mas lá já cheguei retardado
e fui obrigado a pôr um aparelho ortopédico, com o auxílio do mesmo é
que eu vivo me locomovendo e ando, vou por onde quero. Queriam amputar
minha perna, mas eu me danei, não deixei. Não deixei, não, eu não queria
minha perna cortada, não. E o médico teimou comigo e disse: você não vai
agüentar que é um aparelho ortopédico, dói muito e talvez até ainda tire
ele para mandar amputar a perna. Então fiz a seguinte pergunta:
doutor, há perigo de infeccionar? A perna vai infeccionar com esse
aparelho ortopédico? Ele disse: não, não infecciona, não. Dói é muito
pra que você possa se acostumar. Eu digo: ah, doutor, pois eu já sei
que me acostumo. Eu sou é cabeça do mato, acostumado a levar pancada de
pau quando estou brocado, coice de animais, quanta coisa tem. Eu já tô
acostumado com o embate da vida. Aí então botou o aparelho ortopédico,
doeu por mim, parente, amigo, o diabo a sete, mas me acostumei e hoje
estou andando para onde quero, embora com dificuldade, mas não me dói.
Em compensação eu não relembro aquele verso, mas pelo menos meu acidente
foi no dia 13 de agosto e eu não tenho superstição. E o povo sempre
comentava: mas Patativa, além de ser no mês de agosto, ainda mais no dia
13. Você foi muito feliz, porque este mês, não sei o quê\ldots{} Então fiz
este soneto:

\begin{verse}
Foi a 13 de agosto que um transporte\\
Me colheu quebrou a minha perna\\
E ainda hoje padeço o duro corte\\
Que me aflige, me atrasa e me consterna.

Diz alguém que esta data é quem governa\\
Os desastres, nos dando triste sorte\\
Apesar da ciência tão moderna\\
Nossa estrela se apaga e não tem norte.

Mesmo sofrendo a minha sorte crua,\\
Não direi nunca que esta culpa é tua\\
13 de agosto de 73.

Porém, tratado com desdém será\\
E a classe ingênua não perdoará\\
Porque te chama de agourento mês.
\end{verse}

Eu sou um caboclo roceiro que, como poeta, canto sempre a vida do
povo. O meu problema é cantar a vida do povo, o sofrimento do meu
Nordeste, principalmente daqueles que não têm terra, porque o ano
presente, esse ano que está se findando, não foi uma seca, podemos dizer
que não foi a seca. Lá pelo interior, mesmo no município de Assaré, lá
no Assaré, tem duas frentes de serviço, com muita gente. Mas naquela
frente de serviço nós podemos observar que é só dos desgraçados que
não possuem terra. Os camponeses que possuem terra não sofrem estas
conseqüências e não precisam recorrer ao trabalho de emergência, como
os agregados e esses outros desgraçados trabalham na terra dos patrões.
E é isso que eu mais sinto: é ver um homem que tanto trabalha, pai de
família e não possui um palmo de terra. É por isso que é preciso que
haja um meio da reforma agrária chegar, uma reforma agrária que chegue para o povo que não tem terra. Por isso eu digo neste meu soneto
``Reforma Agrária'':

\begin{verse}
Pobre agregado, força de gigante,\\
escuta, amigo, o que te digo agora.\\
Depois da treva vem a linda aurora\\
e a tua estrela surgirá brilhante.

Pensando em ti eu vivo a todo instante\\
minh'alma triste, desolada chora\\
quando eu te vejo pelo mundo afora\\
vagando incerto qual judeu errante.

Para saíres da fatal fadiga\\
do invisível jugo que cruel te obriga\\
a padecer a situação precária,

lutai altivo, corajoso e esperto\\
pois só verás o teu país liberto\\
se conseguires a reforma agrária.
\end{verse}

E esta luta pela reforma agrária e pelo sindicato dos camponeses, mas o
verdadeiro sindicato conduzido pelos próprios camponeses, procurando,
reivindicando os seus direitos, é preciso que continue até chegar o
tempo do camponês sofrer menos do que vem sofrendo. Precisa fazer como
eu digo nos meus versos ``Lição do Pinto'', pois o pinto sai do ovo
porque trabalha. Ele belisca a casca do ovo, rompe e sai. É assim que o
povo também deve fazer, unido sempre, trabalhando.

\hfill{}Depoimento concedido a Rosemberg

\hfill{}Cariry, no Crato, em 1979.

\chapter{Eu e meu campina}

\begin{verse}
Assaré, terra querida,\\
Nestes versos que componho\\
Te digo que em minha vida\\
Tu és o meu grande sonho,\\
Desde o vale até o monte\\
És a milagrosa fonte\\
De minhas inspirações,\\
Meu Torrão de sol ardente\\
Banhado pela corrente\\
Do rio dos Bastiões.

Foi em mil e novecentos\\
E nove que eu vim ao mundo,\\
Meus pais naqueles momentos\\
Tiveram prazer profundo,\\
Foi na Serra de Santana\\
Em uma pobre choupana\\
Humilde e modesto lar,\\
Foi onde eu nasci\\
Em cinco de março vi\\
Os raios da luz solar.

Eu nasci ouvindo cantos\\
Das aves de minha terra\\
E vendo os lindos encantos\\
Que a mata bonita encerra,\\
Foi ali que eu fui crescendo,\\
Fui lendo e fui aprendendo\\
No livro da Natureza\\
Onde Deus é mais visível,\\
O coração mais sensível\\
E a vida tem mais pureza.

Sem poder fazer escolhas\\
De livro artificial,\\
Estudei nas lindas folhas\\
Do meu livro natural\\
E assim longe da cidade\\
Lendo nesta faculdade\\
Que tem todos os sinais,\\
Com estes estudos meus\\
Aprendi amar a Deus\\
Na vida dos animais.

Quando canto o sabiá\\
Sem nunca ter tido estudo,\\
Eu vejo que Deus está\\
Por dentro daquilo tudo,\\
Aquele pássaro amado\\
No seu gorjeio sagrado\\
Nunca uma nota falhou,\\
Na sua canção amena\\
Só diz o que Deus ordena,\\
Só canta o que Deus mandou.

Cresci entre os campos belos\\
De minha adorada Serra,\\
Compondo versos singelos\\
Brotados da própria terra,\\
Inspirado nos primores\\
Dos campos com suas flores\\
De variados formatos\\
Que pra mim são obras-primas,\\
Sem nunca invejar as rimas\\
Dos poetas literatos

Vivendo naquele meio\\
Sentindo prazer infindo\\
De doces venturas cheio\\
Naquele quadro silvestre\\
A voz do Divino Mestre\\
Falando dentro de mim:\\
--- Não lamentes a pobreza\\
Pois tu tens grande riqueza,\\
Felicidade é assim.

E eu passava sorridente\\
A tarde, a noite e a manhã\\
Sem nunca me entrar na mente\\
A falsa riqueza vã,\\
Mas por capricho da sorte\\
Vi que a estrela do meu norte\\
Deixou de me proteger,\\
Saí do meu paraíso\\
Porque na vida é preciso\\
Gozar e também sofrer.

Com setenta anos de idade\\
O destino me fez guerra,\\
Fui residir na cidade\\
Deixando a querida Serra,\\
Minha Serra pequenina,\\
Mas um galo de campina\\
De trazer não me esqueci\\
Porque neste passarinho\\
Estou vendo um pedacinho\\
Lá do sítio onde eu nasci.

Me envolve nesta cidade\\
Certa sombra de tristeza\\
Sentindo a roxa saudade\\
Das vozes da Natureza,\\
Longe daquele ambiente\\
Tão puro e tão inocente\\
Que me prende e que me encanta,\\
Tenho apenas esta lira,\\
Um coração que suspira\\
E um passarinho que canta.

Canta campina, o teu canto\\
Faz diminuir meu tédio\\
Para aplacar o meu pranto\\
A tua voz é o remédio,\\
Neste nosso esconderijo\\
És o único regozijo\\
Para os tristes dias meus,\\
Tu és meu anjo divino\\
E este teu canto é um hino\\
Louvando o poder de Deus.

Por dentro da mesma linha\\
Nossa vida continua\\
A tua sorte é a minha\\
E a minha sorte é a tua,\\
Se vivendo na cidade\\
Tu cantas uma saudade,\\
Saudade o teu dono tem,\\
Meu querido companheiro\\
Se tu és prisioneiro\\
Eu vivo preso também.

Se tu tens a tua história\\
Que o mau destino te deu,\\
Perdi também uma glória,\\
O mesmo padeço eu,\\
Meu querido passarinho\\
Vamos no mesmo caminho\\
Seguimos a mesma meta,\\
Padecem a mesma sina\\
O poeta do campina\\
E o campina do poeta.

Era boa a tua vida\\
Porque vivias liberto\\
E para tua dormida\\
Tu tinha o ponto certo,\\
Mas não lamentes o fado\\
Vivemos hoje preso ao lado\\
Deste teu pobre senhor,\\
Quem sabe se no porvir\\
Tu não irias cair\\
Nas armas do caçador?

Eu te conduzi do mato\\
Com desvelo e com carinho\\
Porque neste mundo ingrato\\
Ninguém quer viver sozinho,\\
Se a mesma sorte tivemos\\
Juntinhos nós viveremos\\
Por ordem do Criador,\\
Neste sombrio recanto\\
Tu, consolando meu pranto\\
E eu cantando a tua dor.
\end{verse}

\chapter{Ispinho e fulô}

\begin{verse}
É nascê, vivê e morrê\\
Nossa herança naturá\\
Todos tem que obedecê\\
Sem tê a quem se quexá,\\
Foi o autô da Natureza\\
Com o seu pudê e grandeza\\
Quem traçou nosso caminho,\\
Cada quá na sua estrada\\
Tem nesta vida penada\\
Pôca fulô e muito ispinho.

Até a propa criança\\
Tão nova e tão atraente\\
Conduzindo a mesma herança\\
Sai do seu berço inocente,\\
Se passa aquele anjo lindo\\
Hora e mais hora se rindo\\
E algumas horas chorando,\\
É que aquela criatura\\
Já tem na inocença pura\\
Ispinho lhe cutucando.

Fora da infança querida\\
No seu uso de razão\\
Vê muntas fulô caída\\
Machucada pelo chão,\\
Pois vê neste mundo ingrato\\
Injustiça, assassinato\\
E uns aos outros presseguindo\\
E assim nós vamo penando\\
Vendo os ispinho omentando\\
E as fulô diminuindo.

Nosso tempo de rapaz\\
Quando a gente ama e qué bem\\
Tudo que é bom ele traz,\\
Tudo que é bom ele tem,\\
Nossa vida é um tesôro,\\
De bêjo, abraço e namoro\\
De fantasia e de canto\\
De ilusão e de carinho\\
Não se vê nem um ispinho,\\
É fulô por todo o canto.

Depois vem o casamento\\
Trazendo a lua de mé,\\
O maió contentamento\\
Que goza o home e a muié,\\
Mas depois que a lua passa\\
Já vão ficando sem graça\\
Pois é preciso infrentá\\
A obrigação que eles tem\\
Porque Deus não faz ninguém\\
Pra vivê sem trabaiá.

Mais tarde chega a criança\\
Que o casal tanto queria,\\
Rosinha como a esperança\\
Enche a casa de alegria,\\
No dia da sua vinda\\
Todos diz: Ô coisa linda!\\
Pra repará todos vem\\
A criancinha mimosa\\
Tão linda iguamente a rosa,\\
Mas traz ispinho também.

Quando um casá se separa\\
Rebenta duas ferida,\\
Ferida que nunca sara,\\
Pois a dô é repartida,\\
Cumprindo a sorte misquinha,\\
Nem mesmo uma fulôzinha\\
Aos desgraçados acompanha,\\
Cada quá no seu caminho\\
Topa toceira de ispinho\\
Que o chique-chique não ganha.

A vida tem um tempêro\\
De alegria e de rigô\\
Derne o mais pobre trapêro\\
Ao mais ricaço dotô\\
Na roda desta ciranda\\
O mundo intêro disanda,\\
Não ficou pra um sozinho,\\
O sofrimento é comum\\
A estrada de cada um\\
Sempre tem fulô e ispinho.

Sem chorá ninguém tulera\\
De uma sêca a tirania,\\
O rapapé da misera\\
Ispaia as pobre famia,\\
O ispinho da precisão\\
Fura em cada coração,\\
Seca as águas no regato,\\
A mata fica dispida,\\
Não se vê fulô no mato.

Para o véio que ficou\\
Sem corage e sem assunto\\
Só resta as triste fulô\\
Com que se enfeita difunto,\\
Vem a doença e lhe inframa\\
E ele recebe na cama\\
Na sua eterna partida\\
Sem tá sabendo de nada\\
A derradêra furada\\
Do ispinho da nossa vida.
\end{verse}

\chapter{Antonio Conselheiro}

\begin{verse}
Cada um na vida tem\\
O direito de julgar\\
Como tenho o meu também\\
Com razão quero falar\\
Nestes meus versos singelos\\
Mas de sentimentos belos\\
Sobre um grande brasileiro\\
Cearense meu conterrâneo,\\
Líder sensato e espontâneo,\\
Nosso Antonio Conselheiro.

Este cearense nasceu\\
Lá em Quixeramobim,\\
Se eu sei como ele viveu\\
Sei como foi o seu fim,\\
Quando em Canudos chegou\\
Com amor organizou\\
Um ambiente comum\\
Sem enredos nem engodos,\\
ali era um por todos\\
e eram todos por um.

Não pode ser justiceiro\\
E nem verdadeiro é\\
O que diz que o Conselheiro\\
Enganava a boa fé,\\
O Conselheiro queria\\
Acabar com a anarquia\\
Do grande contra o pequeno,\\
Pregava no seu sermão\\
Aquela mesma missão\\
Que pregava o Nazareno.

Seguindo um caminho novo\\
Mostrando a lei da verdade\\
Incutia entre o seu povo\\
Amor e fraternidade,\\
Em favor do bem comum\\
Ajudava cada um,\\
Foi trabalhador e ordeiro\\
Derramando o seu suor,\\
Foi ele o líder maior\\
Do nordeste brasileiro.

Sem haver contrariedades\\
Explicava muito bem\\
Aquelas mesmas verdades\\
Que o Santo Evangelho tem,\\
Pregava em sua missão\\
Contra a feia exploração\\
E assim, evangelizando,\\
Com o progresso estupendo\\
Canudos ia crescendo\\
E a notícia se espalhando.

O pobrezinho agregado\\
E o explorado parceiro\\
Cada qual ia apressado\\
Recorrer ao Conselheiro\\
E o líder recebia\\
Muita gente todo dia,\\
Assim fazendo seus planos\\
Na luta não fracassava\\
Porque sabia que estava\\
Com os direitos humanos.

Mediante a sua instrução\\
Naquela scociedade\\
Reinava paz e união\\
Dentro do grau de igualdade,\\
Com a palavra de Deus\\
Ele conduzia os seus,\\
Era um movimento humano\\
De feição socialista,\\
Pois não era monarquista\\
Nem era republicano.

Desta forma na Bahia\\
Crescia a comunidade\\
E ao mesmo tempo crescia\\
Uma bonita cidade,\\
Já Antonio Conselheiro\\
Sonhava com o luzeiro\\
Da aurora de nova vida,\\
Era qual outro Moisés\\
Conduzindo seus fiéis\\
Para a terra prometida.

E assim bem acompanhado\\
Os planos a resolver\\
Foi mais tarde censurado\\
Pelos donos do poder,\\
O tacharam de fanático\\
E um caso triste e dramático\\
Se deu naquele local,\\
O poder se revoltou\\
E Canudos terminou\\
Numa guerra social.

Da catástrofe sem par\\
O Brasil já está ciente,\\
Não é preciso eu contar\\
Pormenorizadamente\\
Tudo quanto aconteceu,\\
O que Canudos sofreu\\
Nós guardamos na memória\\
Aquela grande chacina,\\
A grande carnificina\\
Que entristeceu a nossa história.

Quem andar pela Bahia\\
Chegando ao dito local\\
Onde aconteceu um dia\\
O drama triste e fatal,\\
Parece ouvir os gemidos\\
Entre os roucos estampidos\\
E em benefício dos seus\\
No momento derradeiro\\
O nosso herói brasileiro\\
Pedindo a justiça a Deus.
\end{verse}

\chapter{O galo egoísta e o frango infeliz}

\begin{verse}
Alguém diz que nos vêm dores fatais\\
É porque com certeza nós pecamos,\\
E porque é que também observamos\\
A nega sorte contra os animais?

Eu vejo um animal que é bem feliz\\
E vejo outro que em crises permanece,\\
É um segredo que só Deus conhece\\
Porque ele é o Criador e é o Juiz.

Ninguém sabe os segredos da natura,\\
De opiniões há grande variedade,\\
Bom leitor, ouça agora por bondade\\
Esta história do frango sem ventura.

O terreiro de um rico um frango tinha\\
Tão mofino e medroso, que tolice!\\
Inda estava donzelo embora visse\\
No terreiro galinha e mais galinha.

De gozar de uma franga o seu calor\\
Muitas vezes pensava ele em segredo,\\
Porém muito assombrado, tinha medo\\
Dos cruéis esporões do seu senhor.

Mariscando tristonho sobre o chão\\
Vivia contra as leis da Natureza,\\
Andando a passo lento e perna tesa\\
E não era outra coisa, era paixão.

Resolveu coma sua dor incrível\\
Exigir do seu chefe uma galinha,\\
Outro meio de vida ele não tinha\\
E viver como estava era impossível.

Com o fim de fazer esta conquista\\
Muito alegre o seu sonho alimentava.\\
A seiva galinácea borbulhava\\
Mostrando o sangue na vermelha crista.

E chegando presente ao velho galo\\
Foi dizendo com grande reverência:\\
Grande e nobre senhor de alta potência\\
Com sobrada razão é que vos falo.

Eu preciso senhor neste momento\\
Muito humilde dizer a vossa alteza\\
Neste mundo da mágoa e de tristeza\\
Quanto é duro e cruel meu sofrimento.

Minha sorte é tão pouca, é tão mesquinha,\\
Que eu já sei encrespar as minhas asas\\
As glândulas parecem duas brasas,\\
Porém nunca beijei uma galinha.

Se uma franga das vossas cucurica\\
De desejos carnais fico tremendo.\\
Só eu sei meu senhor, só eu entendo\\
Aqui dentro de mim como é que fica.

Eu vos peço com toda cortesia\\
Para mim o calor de uma franguinha\\
Ou de qualquer espécie de galinha\\
Ao menos uma vez por dia.

Será este o remédio com certeza,\\
Para o mal que constante me devora,\\
Eu não posso viver assim por fora\\
Dos direitos da leis da Natureza.

Pela glória da vossa grande crista,\\
Pelo vosso poder e majestade,\\
Eu confio sair da crueldade,\\
Desta minha sentença nunca vista.

Conhecendo estar cheio de razão\\
O franguinho falava paciente,\\
Cabisbaixo, tristonho, descontente\\
E não era outra coisa era paixão.

Quando o galo escutou se arrebitou\\
E raivoso se encheu de fúria tanta\\
Fez um tal grugulejo na garganta\\
Que a cristinha do frango amarelou.

E nervoso, raivoso e presunçoso,\\
Foi mostrar seu prestígio e seu conceito,\\
Até os pintos lhe ouviam com respeito\\
Parecia um sermão religioso.

Forte canto primeiro ele soltou\\
Como prova de grande e de valente\\
E depois, para o pobre penitente\\
Da maneira seguinte começou:

Seu patife, atrevido e desordeiro,\\
Minhas penas são sempre respeitadas,\\
Sou o grande cantor das madrugadas,\\
A minha fama está no mundo inteiro.

Minha vida é a mais bela epopéia,\\
Sou querido de toda humanidade,\\
Avisei de São Pedro a falsidade\\
Contra Cristo na antiga Galiléia.

O poeta me louva e me quer bem,\\
Dos terreiros do mundo eu sou o rei,\\
Eu bati minhas asas e cantei\\
Quando Cristo nasceu lá em Belém.

Meu valor sublimado eu não regulo,\\
Sem limite é a minha posição,\\
Da mais linda e suave inspiração\\
Fui a fonte da lira de Catulo.

Frango estúpido, veja quem sou eu,\\
Vai cumprir paciente o seu tormento\\
E não queira invejar com o seu lamento\\
Esta sorte que Deus me concedeu.

Seja bom, seja honesto, seja casto,\\
Não queira desonrar as minha penas,\\
De galinha só tenho três dezenas\\
E isto mesmo só dá para o meu gasto.

Em vez daquilo que você requer,\\
Terá outro remédio, outra meizinha\\
Um quicé, uma agulha e uma linda\\
Com um dedo comprido de mulher.

Sofrerá de uma faca a crueldade,\\
Só assim pagará em um momento\\
Este seu monstruoso atrevimento\\
Contra minha suprema autoridade.

Com a dura e cruel repreensão\\
O franguinho voltou desenganado\\
Cabisbaixo, tristonho, desolado,\\
E não era outra coisa era paixão.

Veja agora leitor o resultado\\
Da predição do chefe do terreiro,\\
Botaram o coitado no chiqueiro\\
E no dia seguinte foi capado.

Veja só que existência tão mesquinha\\
Deste frango que a sorte o desprezou\\
Durante sua vida não gozou\\
Da presença feliz de uma galinha.

Pensando nestas dores tão fatais\\
Eu pergunto ao decifrador da sorte,\\
Será que há também depois da morte\\
Um Paraíso para os animais?
\end{verse}

\chapter{Nordestino, sim, nordestinado, não}

\begin{verse}
Nunca diga nordestino\\
Que Deus lhe deu um destino\\
Causador do padecer,\\
Nunca diga que é o pecado\\
Que lhe deixa fracassado\\
Sem condição de viver.

Não guarde no pensamento\\
Que estamos no sofrimento\\
É pagando o que devemos,\\
A Providência Divina\\
Não nos deu a triste sina\\
De sofrer o que sofremos.

Deus o autor da criação\\
Nos dotou com a razão\\
Bem livres de preconceitos,\\
Mas os ingratos da terra\\
Com opressão e com guerra\\
Negam os nosso direitos.

Não é Deus que nos castiga,\\
Nem é a seca que obriga\\
Sofrermos dura sentença,\\
Não somos nordestinados,\\
Nós somos injustiçados\\
Tratados com indiferença.

Sofremos em nossa vida\\
Uma batalha renhida\\
Do irmão contra o irmão,\\
Nós somos injustiçados,\\
Nordestinos explorados,\\
Mas nordestinados, não.

Há muito gente que chora\\
Vagando de estrada afora\\
Sem terra, sem lar, sem pão,\\
Crianças esfarrapadas,\\
Famintas escaveiradas\\
Morrendo de inanição.

Sofre o neto, o filho e o pai,\\
Para onde o pobre vai\\
Sempre encontra o mesmo mal,\\
Esta miséria campeia\\
Desde a cidade à aldeia,\\
Do sertão à capital.

Aqueles pobres mendigos\\
Vão à procura de abrigos\\
Cheios de necessidade,\\
Nesta miséria tamanha\\
Se acabam na terra estranha\\
Sofrendo fome e saudade.

Mas não é o Pai Celeste\\
Que faz sair do Nordeste\\
Legiões de retirantes,\\
Os grandes martírios seus\\
Não é permissão de Deus,\\
É culpa dos governantes.

Já sabemos muito bem\\
De onde nasce e de onde vem\\
A raiz do grande mal,\\
Vem da situação crítica\\
Desigualdade política\\
Econômica e social.

Somente a fraternidade\\
Nos traz a felicidade,\\
Precisamos dar as mãos,\\
Para que vaidade e orgulho\\
Guerra, questão e barulho\\
Dos irmãos contra os irmãos.

Jesus Cristo, o Salvador,\\
Pregou a paz e o amor\\
Na santa doutrina sua,\\
O direto banqueiro\\
É o direito do tropeiro\\
Que apanha os trapos na rua.

Uma vez que o conformismo\\
Faz crescer o egoísmo\\
E a injustiça aumentar,\\
Em favor do bem comum\\
É dever de cada um\\
Pelos direitos lutar.

Por isto, vamos lutar,\\
Nós vamos reivindicar\\
O direito e a liberdade\\
Procurando em cada irmão\\
Justiça, paz e união,\\
Amor e fraternidade.

Somente o amor é capaz\\
E dentro de um país faz\\
Um só povo bem unido,\\
Um povo que gozará\\
Porque assim, já não há\\
Opressor nem oprimido.
\end{verse}

\chapter{O boi zebu e as formiga}

\begin{verse}
Um boi zebu certa vez\\
Moiadinho de suó,\\
Querem sabê o que ele fez?\\
Temendo o calô do só\\
Entendeu de demorá\\
E uns minutos cuchilá\\
Na sombra de um juazêro\\
Que havia dentro da mata\\
E firmou as quatro pata\\
Em riba de um formiguêro.

Já se sabe que a formiga\\
Cumpre a sua obrigação,\\
Uma com outra não briga\\
Veve em perfeita união\\
Paciente trabaiando\\
Suas fôia carregando\\
Um grande inzempro revela\\
Naquele seu vai e vem\\
E não mexe com ninguém\\
Sem ninguém mexê com ela.

Por isto com a chegada\\
Daquele grande animá\\
Todas ficaro zangada,\\
Começaro a se açanhá\\
E fôro se reunindo\\
Nas pernas do boi subindo,\\
Constantimente a subi,\\
Mas tão devagá andava\\
Que no começo não dava\\
Pra ele nada senti.

Mas porém como a formiga\\
Em todo canto se soca,\\
Dos casco até na barriga\\
Começou a frivioca\\
E no corpo se espaiando\\
O zebu foi se zangando\\
E os casco no chão batia\\
Mas porém não miorava,\\
Quanto mais coice ele dava\\
Mais formiga aparecia.

Com esta formigaria\\
Tudo picando sem dó,\\
O lombo do boi ardia\\
Mais do que na luz do só\\
E ele zangado as patada,\\
Mais a força encorporada\\
O valentão não agüentava,\\
O zebu não tava bem,\\
Quando ele matava cem,\\
Chegava mais de quinhenta.

Com a feição de guerrêra\\
Uma formiga animada\\
Gritou para as companhêra:\\
--- Vamo minhas camarada\\
Acabá com o capricho\\
Deste ignorante bicho\\
Com nossa força comum\\
Defendendo o formiguêro\\
Nós somo muntos miêro\\
E este zebu é só um.

Tanta formiga chegô\\
Que a terra ali ficou cheia\\
Formiga de toda cô\\
Preta, amarela e vremêa\\
No boi zebu se espaiando\\
Cutucando e pinicando\\
Aqui e ali tinha um móio\\
E ele com grande fadiga\\
Pruqué já tinha formiga\\
Até por dentro dos óio.

Com o lombo todo ardendo\\
Daquele grande aperreio\\
O zebu saiu correndo\\
Fungando e berrando feio\\
E as formiguinha inocente\\
Mostraro pra toda gente\\
Esta lição de morá\\
Contra a farta de respeito\\
Cada um tem seu direito\\
Até nas lei da naturá.

As formiga a defendê\\
Sua casa, o formiguêro,\\
Botando o boi pra corrê\\
Da sombra do juazêro,\\
Mostraro nesta lição\\
Quanto pode a união;\\
Neste meu poema novo\\
O boi zebu qué dizê\\
Que é os mandão do podê,\\
E estas formiga é o povo.
\end{verse}

\chapter{A triste partida}

\begin{verse}
Setembro passou, com oitubro e novembro,\\
Já tamo em dezembro,\\
Meu Deus que é de nós?\\
Assim fala o pobre do seco Nordeste,\\
Com medo da peste,\\
Da fome feroz.

A treze do mês ele fez esperiença,\\
Perdeu sua crença\\
Nas pedra de sá.\\
Mas noutra esperiença com gosto se agarra\\
Pensando na barra\\
Do alegre Natá.

Rompeu-se o Natá, porém barra não veio,\\
O só bem vermeio,\\
Nasceu munto além.\\
Na copa da mata buzina a cigarra,\\
Ninguém vê a barra,\\
Pois barra não tem.

Sem chuva na terra descamba janêro,\\
Despois feverêro,\\
E o mermo verão.\\
Entonce o rocêro, pensando consigo,\\
Diz: isso é castigo!\\
Não chove mais não!

Apela pra maço, que é o mês preferido\\
Do Santo querido,\\
Senhô São José.\\
Mas nada de chuva! Tá tudo sem jeito,\\
Lhe foge do peito\\
O resto da fé.

Agora pensando segui outra tria,\\
Chamando a famia\\
Começa a dizê:\\
Eu vendo meu burro, meu jegue e cavalo,\\
Nós vamo a Sã Palo\\
Vivê ou morrê.

Nós vamo a Sã Palo, que a coisa tá feia,\\
Por terras alêia\\
Nós vamo vagá.\\
Se o nosso destino não fô tão misquinho,\\
Pro mermo cantinho\\
Nós torna a vortá.

E vende o seu burro, o jumento e o cavalo,\\
Inté mermo o galo\\
Vendero também,\\
Pois logo aparece feliz fazendêro\\
Por pôco dinhêro\\
Lhe compra o que tem.

Em riba do carro se junta a famia;\\
Chegou o triste dia,\\
Já vai viajá.\\
A seca terrive, que tudo devora\\
Lhe bota pra fora\\
Da terra natá.

O carro já corre no topo da serra.\\
Oiando pra terra,\\
Seu berço, seu lá,\\
Aquele nortista partido de pena,\\
De longe inda acena:\\
Adeus, Ceará!

No dia seguinte, já tudo enfadado,\\
E o carro embalado,\\
Veloz a corrê,\\
Tão triste, coitado, falando sodoso,\\
Um fio choroso\\
Excrama, a dizê:

--- De pena e sodade, papai, sei que morro!\\
Meu pobre cachorro,\\
Quem dá de comê?\\
Já ôto pregunta: --- Mãezinha e meu gato?\\
Come fome, sem trato,\\
Mimi vai morrê!

E a linda pequena tremendo de medo:\\
--- Mamãe meus brinquedo!\\
Meu pé de fulô!\\
Meu pé de rosêra, coitado, ele seca!\\
E a minha boneca\\
Também lá ficou.

E assim vão dêxando, com choro gemido\\
Do berço querido\\
O céu lindo e azul.\\
Os pai pesaroso, nos fio pensando,\\
E o carro rodando\\
Na estrada do Sul.

Chegaro em Sã Palo --- sem cobre, quebrado\\
O pobre, acanhado,\\
Procura um patrão.\\
Só vê cara estranha, da mais feia gente,\\
Tudo é diferente\\
Do caro torrão.

Trabaia dois ano, três ano e mais ano,\\
E sempre no prano\\
De um dia inda vim.\\
Mas nunca ele pode, só veve é devendo\\
E assim vai sofrendo\\
Tromento sem fim.

Se arguma nutiça das banda do Norte\\
Tem ele por sorte\\
O gôsto de uvi,\\
Lhe bate no peito sodade de móio\\
E as água dos óio\\
Começa a caí.

Do mundo afastado, sofrendo desprezo\\
Ali veve preso,\\
Devendo ao patrão.\\
O tempo rolando, vai dia, e vem dia,\\
E aquela famia\\
Não vorta mais não!

Distante da terra tão seca mais boa,\\
Exposto à garoa,\\
À lama e ao paul.\\
Faz pena o nortista, tão forte, tão bravo,\\
Vivê como escravo\\
Nas terra do Sul.
\end{verse}

\chapter{Carta do padre Antonio Vieira ao Patativa do Assaré}

\begin{verse}
Toda cheia e toda farta\\
De verdade e de razão,\\
Receba aí esta carta,\\
Seu poeta do sertão,\\
Um livro você pediu\\
Eu lhe dei e você sumiu\\
Cheio de contentamento,\\
Ainda esperando estou\\
E até hoje não chegou\\
O seu agradecimento.

É um poeta da fama\\
Patativa do Assaré,\\
O povo todo o aclama\\
Mas dele eu perdi a fé,\\
Você tem inteligência\\
Porém não tem consciência,\\
Não cumpre com o dever,\\
Só quer crescer e subir\\
Você só sabe pedir\\
Mas não sabe agradecer.

Dei meu livro tão querido,\\
Foi boa a minha proposta,\\
Mas me deixou esquecido\\
Não mandou sua resposta,\\
Você pode ser famoso\\
Mas também é orgulhoso,\\
Sou obrigado a dizer\\
Pois não gosto de mentir,\\
Você só sabe pedir\\
Mas não sabe agradecer.

Gosta muito de cultura,\\
Está de livro na mão\\
Saboreando a leitura\\
Do Jumento Nosso Irmão,\\
Porém, enquanto vai lendo\\
Do autor vai se esquecendo,\\
Do mesmo não quer saber\\
Já disse e vou repetir,\\
Você só sabe pedir\\
Mas não sabe agradecer.

Por carta e por telefone\\
Tem recebido homenagem\\
E falam que da Sorbone\\
Já recebeu reportagem,\\
A sua capacidade\\
Sem usar fraternidade\\
De nada pode valer,\\
Vai lhe atrasar no porvir,\\
Você só sabe pedir\\
Mas não sabe agradecer.

É muito forte na rima\\
Mas fraco de gratidão\\
E só quer estar de cima\\
Sem olhar para o irmão,\\
Quando um presente recebe\\
O seu dever não percebe,\\
Vaidoso se destaca\\
E não diz nem obrigado,\\
É um bezerro enjeitado\\
Mama e dá coice na vaca.

Agora já lhe conheço,\\
Patativa do Assaré,\\
Do seu papel não me esqueço,\\
Eu já sei você quem é,\\
É bom para receber\\
Porém para agradecer\\
Tem a natureza fraca,\\
Está certo e comprovado\\
É um bezerro enjeitado,\\
Mama e dá coice na vaca.

Nesta sua poesia\\
A verdade a gente vê\\
E é por isto que eu vivia\\
Enganado com você,\\
Com um sentimento nobre\\
Defendendo o povo pobre\\
Aos políticos ataca,\\
Porém não vive lembrado\\
Que é um bezerro enjeitado\\
Mama e dá coice na vaca.

Muito revoltado estou,\\
O meu livro recebeu,\\
Sei que do mesmo gostou,\\
Não me deu prova de amigo,\\
O que você fez comigo\\
Foi feio, é quase um ardil,\\
É com razão que me queixo,\\
Por isso agora eu lhe deixo\\
Entre a pedra e o fuzil.

Depois desta carta lida\\
Você vai se arrepender,\\
Fica de cara lambida\\
Sem saber o que fazer,\\
Ao terminar este assunto\\
Curioso eu lhe pergunto:\\
Quando ler essa missiva\\
Onde é que você se esconde\\
Ou o que é que me responde\\
Seu poeta Patativa?
\end{verse}

\chapter{Resposta do Patativa ao padre Antonio Vieira}

\begin{verse}
Meu prezado sacerdote,\\
Distinto Padre Vieira,\\
Por piedade não me bote\\
Entre a bola e a chuteira\\
Dizendo que eu sou ingrato,\\
Com você não fui exato,\\
Não cumpri com meu dever,\\
Para quem quiser ouvir\\
Fala que eu só sei pedir\\
Mas não sei agradecer.

Só porque me fez presente\\
De o Jumento Nosso Irmão\\
E eu por estar inocente\\
Não lhe mandei gratidão,\\
Diz que eu só sei receber\\
Mas não sei agradecer,\\
De tal maneira me ataca\\
Que até me faz comparado\\
Com o bezerro enjeitado,\\
Mama e dá coice na vaca.

Diz que eu sou forte na rima\\
Mas fraco de gratidão\\
E spo quero estar de cima\\
Sem olhar para o irmão,\\
Padre, eu não fico calado,\\
Vou defender o meu lado,\\
Isso é quase um insolência,\\
Me debochou e me lascou\\
Porém eu agora vou\\
Provar a minha inocência.

Não me jogue entre os abrolhos\\
Seja reto e justiceiro,\\
Tire a trave dos seus olhos\\
Para ver se eu tenho argueiro,\\
Você recebeu batina\\
E prega a santa doutrina,\\
É um pastor exemplar\\
Que a nossa igreja precisa,\\
Casa, confessa e batiza,\\
Mas não sabe perdoar.

É muito conceituado\\
Tem grande capacidade\\
Pois já foi até caçado\\
Que é prova de honestidade,\\
Segue um belo itinerário\\
Conduzido o breviário,\\
Sabe a ovelha apascentar,\\
É um grande confessor\\
Amigo do pecador,\\
Mas não sabe perdoar.

Sem eu ter culpa nenhuma\\
Você vem tirar meu couro\\
Me sacudindo esta ruma\\
De pilhéria e desaforo;\\
Quando o livro eu recebi\\
Se logo não escrevi\\
Mandando agradecimento,\\
O culpado não sou eu,\\
A culpa vem do lopeu,\\
Este azar é do jumento.

Esta culpa não é minha\\
E sim do pobre jumento\\
Que tem a sorte mesquinha\\
E é carga de sofrimento,\\
Você escreveu sobre o jegue,\\
Me dê razão, não me negue\\
E nem fique aborrecido,\\
Este animal de transporte\\
É desgraçado e sem sorte\\
Em todo e qualquer sentido.

Todo aquele que se atreve\\
Como você se atreveu\\
E um famoso livro escreve\\
Falando sobre o lopeu,\\
Quando um bonito exemplar\\
Resolve presentear\\
Com atenção e prazer,\\
O que recebe o volume\\
Mesmo tendo bom costume\\
Se esquece de agradecer.

E por isto, meu amigo,\\
Veja que eu tenho razão,\\
A culpa vem do castigo\\
Do Jumento nosso Irmão,\\
Você escreveu com arte\\
Porém ficou esta parte\\
Com a qual me justifico,\\
Eu me defendo e me vingo,\\
Não venha com choromingo,\\
Esta culpa é do jerico.

E para mostrar ao Padre\\
Que eu tenho bom coração,\\
Respeitando a santa Madre\\
Eu vou lhe dar o perdão\\
Pois me atacou inocente,\\
Me atacou inconsciente,\\
Um grande ataque me deu,\\
Fez a maior anarquia\\
Porque inda não sabia\\
Se a culpa vem de lopeu.

O jumento cochilando\\
Na sombra de um juazeiro\\
Ele está filosofando\\
Sobre o grande cativeiro\\
Do qual foi sempre sujeito,\\
O jegue é do mesmo jeito\\
Do matuto agricultor\\
Que trabalha até morrer\\
Pro mundo inteiro comer\\
Mas ninguém lhe dá valor.
\end{verse}

\chapter{Coração doente}

\begin{verse}
Quando o coração não tem\\
Nenhum sinal de doença\\
O corpo se sente bem\\
Tem vigô e tem resistença,\\
Se o coração tem prazê\\
Alegra o resto do sê\\
Pois é ele o condutô,\\
Veve sempre sastisfeito\\
Quem pissui dentro do peito\\
Um coração sonhadô.

É ele um orgo incelente,\\
É por onde o sangue reve\\
Se o corpo sente, ele sente,\\
Sem coração ninguém veve,\\
Sempre a pursá sem demora,\\
Se a gente chora, ele chora\\
O que chega em nossa mente\\
Logo o coração precebe\\
Pois é ele quem recebe\\
O que vai do consciente.

Quando se encronta o sujeito\\
Por uma afrição passando\\
Tá também dentro do peito\\
O coração chucaiando,\\
Se o sujeito fica triste.\\
Na tristeza ele pressiste,\\
Se o sujeito tá risonho,\\
Logo ele muda de jeito\\
Batendo dentro do peito\\
Cheio de esperança e sonho.

Conheci um coração\\
Iguamente o da criança\\
Todo cheio de inluzão,\\
De paz, de amô e de esperança,\\
Tinha a pancada suave\\
Como o relojo agradave\\
Que não atrasa o pontêro,\\
Sempre a parpitá seguro\\
Prometendo um bom futuro\\
Ao resto do corpo intêro.

Inquanto alegre se rindo\\
Dentro do peito batia,\\
Cada membro ia sentindo\\
Aquela mesma alegria,\\
Tudo bem continuava\\
Por ele nunca passava\\
Uma sombra de tristeza,\\
Tudo era paz e bonança\\
Recebendo a substança\\
Da divida Natureza.

Este coração sadio\\
Começou a adoecê,\\
Mas os dotô, os seus fio,\\
Não quisero defendê\\
E os microbe das doença\\
Entraro com insistença\\
Numa invistida danada\\
E o coitado assim doente,\\
Foi discompassadamente\\
Diminuindo as pancada.

Era precioso e caro\\
Este grande coração,\\
Mas porém lhe abandonaro\\
Os dotô cirugião,\\
Microbe de toda sorte\\
Foi nele fazendo corte\\
E o pobre se consumindo\\
Já sem força, quase inzangue,\\
Os verme chupando o sangue\\
E os membro diminuindo.\\

Com o medonho fracaço\\
Tudo ficou deferente;\\
Na referença que eu faço\\
Este coração doente\\
Que eu mencionei aqui,\\
É tu, querido Brasi,\\
Pois teus fio te abandona,\\
Pra ti já não há mais jeito,\\
Agoniza no teu peito\\
Teu coração a Amazona.
\end{verse}

\chapter{O desgosto do Medêro}

\begin{verse}
Ô Joana este mundo tem\\
Sujeito com tanta faia\\
Que quanto mais qué sê bom\\
Mais no erro se escangaia,\\
Istuda mas não prospera\\
E pra sê burro de vera\\
Só farta levá cangaia.

Ô Joana, tu já deu fé,\\
Tu já prestou atenção,\\
Que tanta gente que tinha\\
Com nós boa relação\\
Anda agora deferente\\
Sem querê sabê da gente\\
Pru causa da inleição?

Óia Joana, o Benedito\\
Que era camarada meu\\
Anda agora todo duro\\
Sem querê falá com eu\\
Na maió intipatia\\
Pruquê vota em Malaquia\\
E eu vou votá no Romeu.

Se ele vota em Malaquia\\
E eu no Romeu vou votá\\
Cada quá tem seu partido\\
Isto é munto naturá,\\
Disarmonia não traz\\
E este motivo não faz\\
Nossa relação cortá.

O Zé Loló que me vendo\\
Brincava e dizia trova\\
Anda todo infarruscado\\
Com certa manêra nova\\
Sem morá e ingnorante,\\
Com a cara do istudante\\
Que não passou pela prova.

Ô meus Deus, nunca pensei\\
De vê o que agora tô vendo,\\
Joana, basta que eu lhe diga\\
Que até mesmo o Zé Rozendo\\
Anda falando grossêro\\
Não fala mais no dinhêro\\
Que ele ficou me devendo.

Pra que tanta deferença,\\
Pra que tanta cara estranha?\\
O mundo intêro conhece\\
Que quando chega a campanha\\
Tudo alegre pega fogo,\\
Inleição é como jogo\\
Quem tem mais ponto é quem ganha.

Ô meu Deus como é que eu vivo\\
Sem tê comunicação?\\
Ô Joana, só dá vontade\\
De sumi num sucavão\\
Pra ninguém me aborrecê\\
E somente aparecê\\
Quando passá as inleição.

---  Medêro, não seja tolo\\
Pruquê você se aperreia?\\
Tudo isto é gente inconstante\\
Que sempre fez ação feia,\\
É gente que continua\\
Na mesma fase da lua,\\
Crescente, minguante e cheia.

--- Medêro, não entristeça\\
Você não vai ficá só\\
O que fez o Benidito,\\
Zé Rozendo e Zé Loló\\
Eu sei que foi munto ruim\\
Porém se os home é assim\\
As muié são mais pió.

---  Medêro, tanta muié\\
que dizia a todo istante?\\
Como é que tu vai Joaninha?\\
Toda fofa e elegante,\\
Pruquê voto no Romeu\\
Agora passa pru eu\\
Com a tromba de elefante.

Eu onte vi a Francisca\\
A Ginuveva e a Sofia\\
Dizendo até palavrão\\
Com Filismina e Maria,\\
No maió ispaiafato\\ 
Pro causa dos candidato\\
O Romeu e o Malaquia.

Tu não vê a Zefa Peba,\\
Que é até colegiá?\\
Nunca mais andou aqui\\
E agora vou lhe contá\\
O que ela já fez comigo\\
Que até merece castigo\\
Mas eu vou lhe perdoar.

A Zefa Peba chegou\\
Reparou e não vendo eu\\
Subiu na nossa carçada\\
Se isticou, gunzou, se ergueu\\
Com os óio de cabra morta\\
E tirou da nossa porta\\
O retrato de Romeu.

Eu tava escondida vendo\\
E achei aquilo bem chato\\
Será que ela tá pensando\\
Que rasgando este retrato\\
O Romeu fica pequeno\\
E tem um voto de meno\\
Para o nosso candidato?

Eu vi tudo que ela fez\\
Porém não quis arengá,\\
Mas no momento que vi\\
A Peba se retirá,\\
Provando que sou muié\\
Agarrei outro papé\\
Preguei no mesmo lugá.

Por isso você Medêro\\
Não se importe com pagode\\
Se lembre deste ditado\\
E com nada se incomode,\\
Tudo é farta de respeito,\\
``Quem é bom já nasce feito\\
Quem qué se fazê não pode''.
\end{verse}

\chapter{A fonte milagrêra}

\begin{verse}
O finado meu avô\\
Era munto rezadô\\
Em milagre acreditava\\
E na minha meniniça\\
Cheio de amô e cariça\\
Munto histora me contava

Muntas vez hora e mais hora\\
Passava contando histora,\\
Contava pruquê sabia\\
E tudo que ia dizendo\\
Era mesmo que eu tá vendo,\\
Pois meu avô não mentia.

Um dia tando sentado\\
Com eu também do seu lado\\
Sobre um banco de aruêra,\\
Contou bastante sodoso\\
O passado vantajoso\\
De uma fonte milagrêra.

Disse com munto carinho:\\
Você tá vendo, netinho,\\
Aquele monte acolá\\
Com aquela altura imensa\\
Que quem repara até pensa\\
Que ele no céu qué tocá?

Aquele monte pelado\\
Com o chão todo escarvado\\
Já foi munto encantadô,\\
Tinha beleza sem fim,\\
Já foi um lindo jardim\\
Do mais bonito verdô.

Ele tem a sua histora,\\
Seu passado de gulora\\
De alegria e de ventura\\
E hoje tá naquele estado,\\
Bem diz um véio ditado\\
Que o qué bom, bem pôco dura.

É preciso que eu lhe conte,\\
Bem no pé daquele monte\\
Tem uma grande pedrêra\\
E entre aquelas pedra havia\\
As água pura e sadia\\
De uma fonte milagrêra.

Quem tinha fé e se banhava\\
Naquelas água curava\\
Qualqué doença reimosa,\\
Munta gente todo dia\\
Tomava banho e bebia\\
Das água maraviosa.

Como quem faz romaria\\
Munta gente todo dia\\
No pé do monte chegava,\\
E até mesmo o mal do peito\\
Que os dotô não dava jeito\\
Aquelas água curava.

De gente que ali chegava\\
As estrada friviava\\
Como caminho de fêra\\
E todos que tinha crença\\
Curava suas doença\\
Na água da milagrêra.

Mas nosso mundo sem fim\\
Tem munto sujeito ruim\\
De baxo procedimento\\
Sem arma e sem coração\\
Que qué obtê perdão\\
Sem tê arrependimento.

Certo dia um delegado\\
Se largou do seus coitado\\
E um banho nela tomou,\\
Com este banho danado\\
Ele não ficou curado\\
E a linda fonte secou.

Depois que secou a fonte\\
Ficou triste aquele monte\\
Como o doente que chora,\\
Veja, querido netinho,\\
Até mesmo os passarinho\\
Voaro e foro se embora.

Para a pessoa descrente\\
Milagre não dá pra frente,\\
Quem sabe! Quem advinha!\\
Quem é que pode jurgá,\\
Quantos pecado mortá\\
Este delegado tinha?
\end{verse}

\chapter{Vicência e Sofia ou o castigo de mamãe}

\begin{verse}
Vou dá uma prova franca\\
Falando pra seu dotô.\\
Gente preta e gente branca\\
Tudo é de Nosso Sinhô,\\
Mas tem branco inconsciente\\
Que querendo sê decente\\
Diz que o preto faz e nega,\\
Que o preto tem toda fáia,\\
Não vê os rabo de páia\\
Que muntos branco carrega.

Pra sabê que o preto tem\\
Capacidade e valia\\
Não vou mexê com ninguém\\
Provo é na minha famia;\\
Eu sou branco quage lôro,\\
Mas no premêro namoro,\\
Com a santa proteção\\
Da Divina Providença,\\
Eu casei com a Vicença\\
Preta da cô de carvão.

Ela não tinha beleza,\\
Não vou menti, nem negá,\\
Mas tinha delicadeza\\
E sabia trabaiá.\\
Venta chata, beiço grosso\\
E muito curto o pescoço,\\
Disto tudo eu dava fé,\\
A feiúra eu não escondo,\\
Os óio grande e redondo\\
Que nem os do caboré.

Mas Deus com sua ciença\\
Em tudo faz as mistura,\\
A bondade de Vicença\\
Tirava a sua feiúra\\
E o amô não é brinquedo,\\
Amô é grande segredo\\
Que nem o saibo revela.\\
Quando a Vicença falava\\
Parece que Deus mandava\\
Que eu me casasse com ela.

Houve um baruio do Diacho,\\
Papai e mamãe não queria\\
Foro arriba e foro abaxo\\
Mode vê se eu desistia,\\
Um falava, outro falava,\\
Porém do jeito que eu tava\\
Eu não podia deixá,\\
Eu tava que nem ureca\\
Que depois que prega e seca,\\
Não tem quem possa arrancá.

Mamãe dizia: Romeu,\\
Veja a grande deferença,\\
Vejo a cô que Deus lhe deu\\
E o pretume da Vivença,\\
Tenha vergonha, se ajeite,\\
Aquela pipa de azeite,\\
Não serve de companhia,\\
Isto é papé do Capeta,\\
Você com aquela preta\\
Desgraça nossa famia.

Isso muito me aborrece\\
Que futuro você acha\\
Nesta preta que parece\\
Um tubo sujo de graxa?\\
Lhe dou um conseio agora;\\
Deixe tudo e vá se embora\\
Ganhá dinhêro no sul,\\
Venda o meu burro e o cavalo\\
Vá se embora pra São Palo,\\
Acabe com esse angu.

Mande a sua opinião,\\
Se não você fica à tôa,\\
Eu não lhe boto benção\\
E o sei pai lhe amardiçôa.\\
Este infeliz casamento\\
Só vai lhe dá sofrimento\\
Isto, eu digo e em Deus confio,\\
Você vai se arrependê\\
Depois, mais tarde vai tê\\
Vergonha até de seus fio.

Fio com mãe não discute,\\
Mas porém com esta briga,\\
Eu disse: mamãe, escute,\\
É preciso que eu lhe diga,\\
Não fale da fia alêia,\\
A Vicença é preta e feia,\\
Não vou lhe dizê que não\\
Disto tudo eu já dei\\
fé, Mas eu não quero muié\\
Pra botá na exposição.

Mamãe, eu quero muié\\
É promode me ajudá\\
Fazê comida e café\\
E a minha vida zelá\\
E aquela é uma pessoa\\
Que pra mim tá muito boa,\\
O que é que a senhora pensa?\\
Lhe digo sem brincadêra\\
Mamãe é trabaiadêra\\
Mas não vai com a Vicença.

Dotô, mamãe desta vez\\
De raiva ficou cinzenta,\\
Fungou igual uma rez\\
Quando cai água nas venta,\\
Com raiva saiu de perto\\
E eu achei que eu tava certo\\
Defendendo meu amô,\\
Pois tenho na minha mente\\
Que o negro também é gente,\\
Pertence a Nosso Sinhô.

Eu disse: eu vou é botá\\
Meu casamento pra riba,\\
Tenho idade de casá\\
Não vejo quem me improíba,\\
Saí como quem não foge\\
Fui na casa de seu Joge\\
Cheguei lá, pedi licença\\
E tratei do meu noivado;\\
Ficou tudo admirado\\
Do meu amô por Vicença.

E eu disse: mamãe e papai\\
O casamento não qué,\\
Mas porém a coisa vai\\
Mesmo havendo rapapé.\\
Seu Joge eu quero é depressa\\
Já dei a minha promessa\\
E eu prometendo não nego,\\
Mesmo, eu conheço o direito,\\
Casamento deste jeito\\
Se faz é trás e zás, nó cego.

Seu Joge com muito gosto\\
Fez as obrigação dele\\
Pois era forte e disposto,\\
Que eu nunca vi como aquele,\\
Depois que fez os preparo\\
Convidou seu Januaro\\
Um bom tocadô que eu acho\\
Que é com seu dom soberano,\\
O maió pernambucano\\
Pra tocá nos oito baxo.

Com a pressa que nós tinha,\\
Seu Joge tomou a frente\\
Como quem caça mezinha\\
Quando tá com dô de dente,\\
E depressa, sem demora,\\
Veio o dia e veio a hora\\
Do mais feliz casamento\\
E perto do só se pô\\
Seu Januaro chegou\\
Montando no seu jumento.

Eita festona animada\\
Maó não podia sê\\
O tamanho da latada\\
Não é bom nem se dizê,\\
Sogra, sogro e seus parente\\
Brincava tudo contente,\\
Cada quá o mais feliz,\\
Porém, ninguém puxou fogo,\\
Nem houve banca de jogo\\
Porque seu Joge não quis.

Era noite de luá\\
E a lua o mundo briando\\
Dentro das lei naturá,\\
Lá pelo espaço, vagando,\\
Pura como a consciença\\
Da minha noiva Vicença,\\
O meu amparo e meu bem,\\
Parece até que se ria\\
E pras estrela dizia:\\
Romeu tá de parabém.

Seu Januaro sem medo\\
Tomou um pequeno gole\\
E foi molegando os dedo\\
No tecrado do seu fole,\\
E véio, o moço e a criança\\
Caíro dentro da dança\\
Com uma alegria imensa\\
E eu com a noiva dançando,\\
Já ia me acostumando\\
Com o suó de Vicença.

Seu dotô, eu sei que arguém\\
Não me acredita e me xinga,\\
Mas do suó do meu bem\\
Eu nunca senti catinga,\\
Esta vaidade tola\\
Da branca cronta a criola,\\
A maió bestêra é,\\
Com tudo a gente se arruma,\\
Quarqué home se acostuma\\
Com o chêro das muié.

Seu moço, não ache ruim,\\
Pois eu vou continuá\\
Uma histora boa assim\\
Só se conta devagá.\\
Já disse com paciença\\
Que eu casei com a Vicença,\\
É este o premêro trecho,\\
O mais mió deste mundo,\\
Agora eu conto o segundo\\
Pra o sinhô vê o desfecho.

Nem com a força do vento\\
A luz de Deus não se apaga\\
E quando chega o momento,\\
Aquele que deve, paga.\\
Munto ignorante foi\\
Mamãe, que Deus lhe perdoi,\\
E papai o seu marido,\\
Nenhum falava com eu,\\
Pra eles dois, o Romeu\\
Tinha desaparecido.

Mas nosso Deus Verdadeiro\\
Com a providença sua,\\
Escreve certo e linhêro\\
Até num arco de pua,\\
Lá um dia a casa cai,\\
Com a mamãe e com papai\\
Um desastre aconteceu,\\
Escute bem o que digo\\
E veja como o castigo\\
Na casa deles bateu.

O meu irmão, o José\\
Que ainda tava sortêro,\\
Lesado, besta e paié\\
Que nem peru no pulêro,\\
Se largou do seus coidado\\
E por mamãe atiçado,\\
Intendeu de se casá\\
E casou com a Sofia,\\
A mais bonita que havia\\
Praquelas banda de lá.

A Sofia era alinhada\\
Branca do cabelo lôro,\\
Diciprinada e formada\\
Nas escola de namoro,\\
O que tinha de fromosa,\\
Tinha também de manhosa\\
Dos trabaio da cozinha\\
Ela não sabia nada\\
E pra sê bem adulada\\
Tomou mamãe por madrinha.

Foi a maió novidade\\
O casoro de José,\\
Pra lhe dizê a verdade\\
Sortaro até buscapé,\\
Foguete, traque e chuvinho,\\
Com o prazê que eles tinha\\
Foi comida pra sobrá,\\
Houve armoço, janta e ceia,\\
Mataro até minha uveia\\
Que eu tinha dêxado lá.

Foi grande o contentamento\\
Como iguá eu nunca vi\\
E depois do casamento,\\
Era Sofia prali\\
E Sofia pracolá,\\
A mamãe que pra cantá\\
Nunca teve intiligença,\\
Sorfejava toda hora\\
Só porque tinha uma nora\\
Deferente da Vicença.

Mas pra fazê trapaiada\\
Sofia era cobra mansa,\\
Inventou umas andada\\
Por aquelas vizinhança\\
E o meu irmão sem receio\\
Não ligava estes passeio\\
Confiando na muié,\\
Mas porém a descarada\\
Tava naquelas andada\\
Botando chifre em José.

A coisa inda tava assim\\
Na base da confusão,\\
Arguns dizia que sim,\\
Outros dizia que não,\\
Mas foi pegada em fraglante\\
Lá dentro duma vazante\\
Nuns escondidos que tinha,\\
E quer sabê quem pegou?\\
Não foi eu, nem seu dotô,\\
Foi mamãe sua madrinha.

A mamãe toda tremendo\\
Naquele triste segundo,\\
Como se tivesse vendo\\
Uma coisa do outro mundo,\\
Vortou pra casa chorando\\
Lamentando e cramunhando\\
O caso que aconteceu\\
E a Sofia foi embora,\\
Largou-se de mundo afora\\
Nunca mais apareceu.

Por causa daquele imbruio\\
Minha mamãe acabou\\
Com a suberba e o orguio\\
Que sempre lhe acompanhou\\
Mandou pedi com urgença\\
Que eu fosse mais a Vicença\\
Mode me botá benção\\
Pois ela e o seu marido,\\
De tudo que tinha havido\\
Queria pedi perdão.

Com o que fez a Sofia\\
Mamãe virou gente boa\\
E dizia, minha fia\\
Vicença, tu me perdoa,\\
Como pobre penitente\\
Que dentro da sua mente\\
Um fardo de curpa leva,\\
Mamãe na frente da nora\\
Parecia a branca orora\\
Pedindo perdão a treva.

Se acabou a desavença\\
Se acabou a grande briga,\\
Pra ela, hoje a Vicença\\
É nora, fia e amiga,\\
Hoje o seu prazê compreto\\
É pintiá seus três neto\\
Do cabêlo arrupiado,\\
Cabelo mesmo de bucha\\
Mas mamãe puxa e ripuxa\\
Até que fica estirado.

E é por isso que onde eu chego,\\
No lugá onde eu tivé,\\
Ninguém fala mal de nêgo\\
Que seja home ou muié;\\
O preto tendo respeito\\
Goza de justo dereito\\
De sê cidadão de bem,\\
A Vicença é toda minha\\
E eu não dou minha pretinha\\
Por branca de seu ninguém.

Se de quarqué parte eu venho,\\
Entro na minha morada\\
E aquilo que eu quero tenho,\\
Tudo na hora marcada\\
Da sala até a cozinha\\
E a Vicença é toda minha\\
E eu também sou dela só,\\
Eu sou home, ela é muié\\
E o que eu quero ela qué,\\
Pra que coisa mais mió.

Seu dotô, muito obrigado\\
Da sua grande atenção\\
Escutando este passado\\
Que serve até de lição.\\
Neste mundo de vaidade,\\
Critero, honra e bondade\\
Não tem nada com a cô,\\
Eu morro falando franco\\
Tanto o preto como o branco\\
Pertence a Nosso Senhô.
\end{verse}

\chapter{O meu livro}

\begin{verse}
Meu nome é Chico Braúna\\
eu sou pobre de nascença,\\
diserdado de fortuna\\
mas rico de consciença.\\
Nas letra num tive istudo\\
sou mafabeto de tudo\\
de pai, de mãe, de parente.\\
Mas tenho grande prazê\\
Pruquê aprendi lê\\
duma forma deferente.

\textsc{abc} nem beabá\\
no meu livro não se encerra.\\
O meu livro é naturá\\
é o má, o céu e a terra,\\
cum a sua imensidade.\\
Livro cheio de verdade,\\
de beleza e de primô,\\
tudo incadernado, iscrito\\
pelo pudê infinito\\
do nosso pai Criadô

O meu livro é todo cheio\\
de muita coisa incelente,\\
em suas foia é que leio\\
o pudê do Onipotente.\\
Nesta leitura suave\\
eu vejo coisa agradave\\
que muita gente não vê\\
por isso sou conformado\\
sem eu nunca tê pegado\\
numa carta de \textsc{abc}.

Num é preciso a pessoa\\
cunhecê o beabá\\
pra sê honesta e sê boa\\
e em Jesus acreditá\\
Deus e seu milagre ixato\\
eu vejo mesmo nos mato\\
justiça, verdade e amô\\
de minha mente não sai\\
deste jeito era meu pai\\
e o finado meu avô.

De que adianta a ciença\\
do professô istudioso\\
se ele não crê na existença\\
de um grande Deus Puderoso?\\
Eu sem tê letra nem arte\\
vejo Deus em toda parte.\\
O seu pudê radiante\\
tá bem visive e presente\\
na mais piquena simente\\
e no maió elefante.

Deus é a força infinita\\
é o espírito sagrado\\
que tá vivendo e parpita\\
em tudo que foi criado.\\
Não há quem possa contá\\
é assunto que não dá\\
pra se dizê no papé\\
não inxiste professô\\
nem sábio, nem iscritô\\
pra sabê Deus cuma é.

Apenas se tem certeza\\
que ele é a santa verdade\\
e é a subrime grandeza\\
em bondade e divindade.\\
Porém se ele é infinito\\
é soberano e bendito\\
de tudo superiô\\
que até os bicho lhe adora\\
pruquê muitos tão pru fora\\
das orde do Criadô?

Deus quando o mundo criou\\
ordenou a paz comum\\
e com amô insinou\\
o devê de cada um,\\
Os home pra trabaiá\\
Um ao outro respeitá\\
e a boa istrada segui\ldots{}\\
e os bicho irracioná\\
prumode se alimentá\\
produzi e reproduzi.

Ainda hoje os animá\\
as orde santa obedece\\
sem uma virga faltá\\
se alimenta, omenta e cresce\\
eles que nada magina\\
que nada raciocina\\
não pensa nem tem razão\\
continua sem disorde\\
sempre obedecendo as orde\\
do sinhô da criação.

Segue o seu caminho ixato\\
até a própria furmiga\\
trazendo foia dos mato\\
dentro da terra se abriga\\
sem nada contrariá,\\
cumprindo as lei naturá\\
ao divino mestre atende.\\
Sabe até fazê iscôia\\
pois ela só corta a fôia\\
das fôia que não lhe ofende.

Se o João de Barro, o Pedreiro,\\
sabendo que não se atrasa\\
faz de dezembro a janêro\\
a sua bunita casa\\
com a porta pro poente\\
pois nunca faz pro nascente\\
é orde do Sumo bem.\\
Nunca aquele passarinho\\
faz a porta do seu ninho\\
do lado que a chuva vem.

Tudo segue as orde santa\\
sem havê ninhuma fáia\\
inquanto a cigarra canta\\
as formiguinha trabáia\\
bria o lindo vagalume\\
faz a aranha o seu tissume\\
e o passo beija-fulô\\
voa pra frente e pra trás\\
e o certo é que todos faz\\
aquilo que Deus mandou.

Será que o home, esse ingrato\\
dotado de intiligença\\
vendo os bichinho do mato\\
cum tamanha obidiença\\
não se sente incabulado,\\
acanhado, invergonhado,\\
por não sigui as lição\\
da istrada da sua vida\\
esta graça concedida\\
pelo autô da criação?

A Divina Providença\\
com o seu imenso pudê\\
deu ao home intiligença\\
foi pra ele se regê.\\
Não precisa o Soberano\\
chegá a dizê: Fulano\\
seu caminho é por ali\\
Deus lhe deu o dom divino\\
o dom do raciocino\\
pra ele se conduzi.

Ninguém vem contrariá\\
a mim, o Chico Braúna\\
não precisa Deus mandá\\
que a humanidade se una\\
pois todos tem cunsciença\\
tem o dom da intiligença\\
por dereito e gratidão\\
todos tem de obedecê\\
cada um tem o devê\\
de defendê seu irmão.

Se todos observasse\\
a lei da divina doutrina\\
e um ao outro ajudasse\\
como manda a lei divina\\
num fizesse papé feio\\
defendesse o que é aleio\\
cum amô e cum respeito\\
não precisava formado\\
cum ané de advogado\\
e nem de juiz de dereito.

Mas a farça humanidade\\
continua disunida\\
cheia de prevessidade\\
dos irmãos tirando a vida.\\
Fulano xinga bertrano\\
bertrano bate em sicrano\\
e de suja consciença\\
vão impregando o crime\\
o que tem de mais subrime\\
que é a sua intiligença.

Com a inveja e o goirmo\\
cum a suberba e a vaidade\\
vão se socando no abirmo\\
se afastando da verdade\\
muitos não preza o seu dom\\
fazendo aquilo que é bom\\
cumo manda o Criadô\\
Pru fora das lei divina\\
Pru segue a santa doutrina\\
que Jesus Cristo insinou.

Eu sempre pensei assim.\\
Deus cum a sua consciença\\
Não premite o que é ruim\\
é justo por incelença.\\
Se inxiste luta e mais luta\\
é fruto da má conduta\\
a divina Majestade\\
nunca quis briga na terra\\
o assassinato e a guerra\\
é obra da humanidade.

Quem será que não cunhece\\
quando sunda e pensa um pouco\\
que o nosso mundo parece\\
um asi, cheio de louco?\\
por causa dessa loucura\\
só lá na vida futura\\
as arma no Paraíso\\
tão sujeita a julgamento\\
não se sarva dez por cento\\
vai sê grande o prijuízo.

Este mundo está perdido\\
e o povo perdeu a fé\\
é muié cronta\\
marido marido cronta muié\\
tá tudo materiá\\
ninguém pode potrestá\\
esta certeza que digo\\
do campo inté a cidade\\
os amigo de verdade\\
cum certeza tão cumigo.

Eu sou Chico Braúna\\
não digo palavra om vão\\
fala o dotô na tribuna\\
e eu falo no meu sertão.\\
O que achá ruim me perdôi\\
mas o mundo sempre foi\\
duro de se cuncertá.\\
Dispois criaro o divorço\\
e cum a lei desse troço\\
acabou de disgraçá.
\end{verse}

\chapter{A derrota de Zé Côco}

\begin{verse}
Assim que o sinhô me viu\\
E começou a chamá\\
Chamando e dando pisiu,\\
Conheci que é um fiscá\\
E qué vê meus documento,\\
É com prazê que apresento\\
Minha documentação,\\
Eu nunca fui contra a lei,\\
Até hoje não botei\\
O pé adiante da mão.

Veja aqui este papé\\
Que os iscrivão pede tanto,\\
Meu nome todo é José\\
Filiciano do Santo,\\
Mas como o povo inxirido\\
Sempre botou apelido\\
Da gente fazendo pôco,\\
No meu sertão todo povo\\
Preto, branco, véio e novo\\
Me conhece por Zé Côco.

Eu venho do meu Nordeste,\\
Comprei lá uma passage,\\
Foi sacrifiço da peste,\\
Pra fazê esta viagem,\\
Eita São Paulo distante,\\
Na viagem extravagante\\
Eu vim na carroceria\\
Deitado, sentado e impé,\\
Num caminhão chervolé\\
Cheio de mercadoria.

Foi o maió sacrifiço\\
Mas afiná tou aqui\\
Vendo um grande rebuliço\\
Como iguá eu nunca vi,\\
Faz uns dias que eu cheguei\\
E inda não me acostumei,\\
O São Paulo arvorossado,\\
Eu vejo constantimento\\
Um frumiguêro de gente\\
Andando pra todo lado.

Seo moço, eu sou nordestino,\\
Falo que só papagaio,\\
Eu no tempo de menino\\
Bibi água de chucaio,\\
Muntas vez eu considero\\
Que mode dizê o que eu quero\\
Uma língua só é pôca,\\
As vez eu tenho pensado\\
Que Deus devia tê dado\\
Duas língua em minha boca.

Mas com meu palavriado,\\
Com este meu falatoro,\\
Eu vou dexando de lado\\
As coisa que eu ignoro,\\
Só digo a pura verdade\\
E lhe peço por bondade\\
Quêra me escutá um pôco,\\
Sei que o sinhô não istranha\\
De uvi as grande façanha\\
De seu criado Zé Côco.

Pra falá me dê licença,\\
Lhe juro em nome da lei\\
Tenho limpa a consciença,\\
Nunca matei nem rôbei,\\
Tenho horrô ao pistolêro,\\
Matá pra granhá dinhêro\\
Não adianta nem omenta,\\
Isto não ficou pra mim,\\
Porém surra em cabra ruim\\
Dei mais de cento e cincoenta.

Mas com as minhas proeza\\
Sempre fui home de bem\\
E lhe digo com certeza\\
Eu nunca insurtei ninguém,\\
Meu negoço era inxemprá\\
Cabra que procura azá\\
E faz do diabo a pintura,\\
Sem carate e imbuancêro\\
Que apaga até candiêro\\
Pra sala ficá iscura.

Eu sempre fui convidado\\
Pras festa do meu sertão\\
E quando o povo animado\\
Pra mais mió diversão\\
Uma festa organizava,\\
Quando a nutiça vagava\\
Que era bom o tocadô\\
E munta muié havia\\
Os mais medroso dizia:\\
Só vou se Zé Côco fô.

Pruquê sabendo que eu indo\\
A festa corria em paz,\\
Tudo alegre divirtindo\\
Home, casado e rapaz,\\
Pois com a minha presença\\
Não havia desavença,\\
Ninguém temia perigo\\
Bá bá bá ou rapapé,\\
Por isto toda muié\\
Queria dançá comigo.

E se caso acontecesse\\
De chegá um inxirido,\\
Por ali aparecesse\\
Um vagabundo atrivido\\
Querendo que as pobre moça\\
Dançasse com ele à força\\
Com certos chove e não móia,\\
Logo o jucá vadiava\\
Que o cabra uns dia passava\\
Com o braço na tipóia.

No São João do João Conrado,\\
No mais alegre brinquedo,\\
Chegou um Zé Bronzeado\\
Com raiva e fazendo medo,\\
Tava revortoso e brabo,\\
Eu fui lá e meti-lhe o diabo\\
Dei-lhe um surra tão boa\\
Que o malandro ficou zonzo,\\
Naquele São João o Bronzo\\
Virou lama de lagoa.

De outra vez fui convidado\\
Pro mode acabá o abuso\\
De um cara desaforado\\
Chamado João Parafuso,\\
Eu fui lá e lhe disse assim:\\
Se retire cabra ruim,\\
Tenha vergonha, não teime\\
E ele não quis ir embora,\\
Mas porém na mesma hora\\
Cantou Maria valei-me.

Como a onça carnicêra\\
Pra mim o cabra partiu\\
E eu joguei-lhe uma rastêra\\
Chega a poêra cubriu,\\
Ali o jucá vadiou\\
Que o Parafuso ficou\\
Com o corpo incalombado\\
Foi se embora cachingando\\
Foi cachingando e chorando\\
Pra dexá de sê danado.

Assim seu moço eu vivia\\
No meu papé de machão,\\
Sendo o fiscá e o vigia\\
Das festa do meu sertão,\\
Com amô e com respeito\\
Eu defendia o dereito\\
E a razão daquela gente,\\
Mas nunca fiz imbuança\\
E nem andei com lambança\\
Dizendo que era valente.

Porém bem sabe o sinhô\\
Que esta vida é um caminho,\\
Se aqui tem uma fulô\\
Acolá tem um ispinho,\\
Sempre dizia o meu pai\\
Lá um dia a casa cai\\
Pra gente se machucá,\\
Escute o que eu tou dizendo,\\
O sinhô não tá sabendo\\
Onde é que eu quero chegá.

Festa boa de incumenda,\\
O coroné Zé Davi\\
Fazia em sua fazenda\\
Pro povo se divirti,\\
Era um ricaço famoso\\
Do coração generoso,\\
Nas festa que ele fazia\\
A custa do seu dinhêro\\
O praciano e o rocêro\\
Comia e se divirtia.

Iscute o que eu tou falando,\\
Veja a sorte como é,\\
Eu tava fiscalizando\\
A festa do coroné\\
Mode não havê chafurdo\\
Confusão nem absurdo,\\
Quando apareceu por lá\\
Um magote de estudante\\
Cada quá mais elegante\\
Que veio da capitá.

Entre a turma de estudante\\
Vinha um de preta cô\\
Munto simpate e importante,\\
Um criolo de valô\\
Gostava de brincadêra,\\
Seu cabelo de tocêra,\\
Corpo dergado e bonito,\\
Aquele inducado home\\
Tinha a cô e o mermo nome\\
Do santo São Benedito.

Divido arguém lhe contá\\
Que eu era mesmo o maió,\\
Na rastêra e no jucá\\
No sertão eu tava só,\\
Ele ficou sastifeito\\
E com o maió respeito\\
Chegou bem perto de mim\\
Com munta delicadeza,\\
Me tratando com fineza\\
E foi me dizendo assim:

Seu Zé Côco, meu amigo,\\
Um pedido eu vou fazê,\\
Pra você lutá comigo\\
Sem nenhum se aborrecê,\\
Munto interessado eu tou,\\
Não é briga de rancô\\
É uma vadiação,\\
Nós dois vamo pelejá\\
Você com o seu jucá\\
E eu só com os pé e as mão.

Eu lhe respondi: dotô,\\
Isto assim não pode sê,\\
Um bom conseio eu lhe dou,\\
Cace um pau pra vamicê,\\
Veja que é grande perigo\\
O sinhô lutá comigo\\
Com as mão desocupada,\\
É um medonho fracasso\\
Sendo assim eu lhe desgrasso\\
Da primêra bilotada.

E ele munto prazentêro\\
Disse se rindo, seu Côco\\
Pra fazê o meu tempêro\\
O seu azeite é bem pôco.\\
Achando que ele insistia\\
Queria porque queria\\
Esta luta extravagante,\\
Eu disse pra meu jucá:\\
Nós vamo disinterá\\
Esta turma de estudante.

De lá pra cá ele veio\\
E eu parti daqui pra lá,\\
Sem ninguém entrá no meio\\
Sem ninguém se incomodá,\\
Quando o jucá eu descia\\
Ele uns caracó fazia\\
Rodando que nem pião,\\
Eu botava pra lascá\\
Mas porém o meu jucá\\
Só acertava no chão.

Nós dois já tava suado\\
Naquela vadiação\\
E eu bastante incabulado\\
Jogando o jucá em vão\\
Quando ele gritou Zé Côco,\\
Sua corage é de um lôco,\\
Mas porém tenha cuidado\\
Repare como é que eu luto,\\
Antes de dá três minuto\\
Você fica desarmado.

Era um fuxico esquisito\\
Aquele nosso duelo\\
E o valente Benedito\\
Querendo vê meu fragelo,\\
Ligêro iguamente um gato\\
Com o bico do sapato\\
Deu um chute em minha mão\\
Que o jucá saiu vuando,\\
Foi vuando e rebolando\\
E adiante caiu no chão.

E ele gritou animado\\
Agora amigo Zé Côco\\
Você ficou desarmado\\
Nós vamo lutá de soco\\
Veja bem como é que eu faço\\
Me agarrou e troceu meu braço\\
E eu com o braço trucido\\
Com a dô maió do mundo\\
Senti naquele segundo\\
o meu corpo esmorecido.

Fiquei naquela prisão\\
Sem tê corage pra nada\\
E ele com a outra mão\\
Que tava desocupada,\\
Como quem faz um brinquedo,\\
Com as junta dos seus dedo\\
Pra servi de mangação,\\
Batia em minha cabeça\\
Dizendo Zé Côco esqueça\\
Que você é valentão.

Bem me falava o meu pai\\
Como eu disse a seu fiscá\\
Lá um dia a casa cai\\
Pra gente se machucá\\
Medonho desgosto sinto\\
Mas falo sero, não minto\\
E pra dizê não me acanho,\\
Foi uma pisa tão feia\\
Que eu berrei como a uveia\\
Separada do rebanho.

E ele sortando meu braço\\
Vendo que tava no fim,\\
Disse me dando um abraço\\
Não tenha raiva de mim,\\
Nesta nossa brincadêra\\
Moiei a sua foguêra\\
E apaguei a sua brasa,\\
Vá apanhá seu jucá\\
E vá pra sala dançá\\
Ou então vorte pra casa.

Meu jucá eu apanhei\\
Me jurgando um pobre troço,\\
Mas com raiva não fiquei\\
Pruquê negoço é negoço,\\
Vortei pra casa pensando\\
Pensando e quage jurando\\
Que aquela grande istrução,\\
Aquele grande ixerciço\\
Era reza de feitiço\\
Ou a pintura do cão.

Porém era ilusão minha,\\
Vei um sinhô me dizê\\
Que aquele estudante tinha\\
Aprendido caratê\\
Que eu inda não conhecia\\
E nem sabia se havia\\
Esta escola de brigá\\
De um ixerciço bonito\\
Com o quá o Benedito\\
Se defendeu do jucá.

Com isto que se passou\\
Eu fiquei incabulado,\\
Mas não perdi meu valô\\
Eu era sempre chamado\\
Mode dá banho de pau\\
Em lombo de cabra mau\\
Para as festa defendê,\\
Mas quando alguém me chamava\\
Eu com medo preguntava\\
Ele sabe caratê?
\end{verse}

\chapter{Meu avô tinha razão e a justiça tá errada}

\begin{verse}
Do campo até a cidade\\
Atrapaiando a verdade\\
Sempre inxiste uma caipora,\\
Vou falá pro mundo intêro\\
Como eu era de premêro\\
E como eu tô sendo agora.

De alegria todo cheio\\
Uvindo os belo conseio\\
Do finado meu avô,\\
E sastisfeito vivia,\\
Neste tempo eu não mentia\\
Nem mode fazê favô.

Meu avô, munto correto,\\
Dizia: Querido neto\\
Escute bem o que digo,\\
Use da sinceridade\\
Pruquê quem diz a verdade\\
Nunca merece castigo.

Veve bem acompanhado\\
E também é respeitado\\
Quem sempre a verdade diz,\\
A mentira é treiçuêra\\
E a verdade é companhêra\\
Que faz a gente feliz.

Com a minha intiligença\\
Estas lição de sabença\\
Que o meu avô me ensinava,\\
Eu sastisfeito aprendia\\
E tudo quanto eu dizia\\
O povo me acreditava.

Mas por pintura do diabo\\
O coroné Mané Brabo\\
Começou uma questão\\
E tomou no pé da Serra\\
Trinta tarefa de terra\\
Do Francisco Damião.

Damião tinha dereito,\\
Mas porém não houve jeito\\
Perdeu para o fazendêro,\\
A gente logo descobre\\
Que o Damião era pobre\\
E o Brabo tinha dinhêro.

Um dia numa bodega\\
Onde o pessoá chombrega\\
Cada quá sua bicada,\\
Começaro a conversá\\
Em quem gosta de inricá\\
Por meio de trapaiada.

Eu nada tinha bebido,\\
Mas como tinha aprendido\\
As lição do meu avô,\\
Disse com munta razão\\
A terra do Damião\\
Seu Mané Brabo tomou.

Dois fio do coroné\\
Gritaro logo: o que é\\
Que você falou aí?\\
E eu que prezava a verdade\\
Com munta sinceridade\\
Minha histora repeti.

Mas bem não abri a boca,\\
Os dois com uma fura lôca\\
Me derrubaro no chão\\
E com a força do braço\\
Batia em meu espinhaço\\
Como quem bate fejão.

Os dois safado fizero\\
Comigo o que bem quisero\\
E ali ninguém se importou,\\
Era os peste me surrando\\
E eu chorando e me lembrando\\
Das lição do meu avô.

Fiquei de corpo banido,\\
Mas vendo que era perdido\\
Não dei nem parte a puliça,\\
Tratei de me retirá\\ 
Fui morá noutro lugá\\
E nunca mais dei nutiça.

Invergonhado daquilo\\
Procurei vivê tranquilo\\
Em uma terra afastada\\
Dizendo com os meu butão:\\
Meu avô tinha razão\\
E a justiça tá errada.

Fui morá na Lagoinha\\
Uma cidade onde tinha\\
Uma moça bem bonita\\
Era um anjo tão prefeito\\
Que se eu fô dizê dereito\\
Bem pôca gente acredita.

Parecia tá presente\\
Um anjo em forma de gente\\
Filisberta era o seu nome,\\
Quem conhece considera\\
Que aquela garota era\\
O pára-raio dos home.

Eu vendo a linda donzela\\
Disse bem pertinho dela\\
Filisberta, tu é jóia\\
E a moça ficou me oiando\\
Oiando e também inchando\\
Que nem a cobra gibóia.

E me gritou: --- Atrevido,\\
Seu sem vergonha, inxirido\\
Drobe a língua macriado\\
E uns palavrão me dizendo\\
Saiu depressa correndo\\
Pra dá parte ao delegado.

Fiquei pensando e dizendo:\\
Nada a ela tô devendo\\
Não tenho medo nem corro,\\
Mas sabe o que aconteceu?\\
A puliça me prendeu\\
E apanhei que nem cachorro.

Por orde da Filisberta\\
Me fizero triste oferta\\
Lá num quarto da prisão\\
Que eu fiquei cheio de imbombo,\\
Era chicote no lombo\\
E parmatora nas mão.

Quando me dero surtura\\
Saí da prisão escura\\
Com uma raiva danada,\\
Dizendo com os meu butão:\\
Meu avô tinha razão\\
E a justiça tá errada.

Pensando na Filisberta\\
Pensei numa coisa certa\\
E fiz a comparação\\
Do prédio de um potentado,\\
Por fora fantasiado\\
E por dentro a exploração.

Só pruquê disse a verdade\\
Me sacudiro na grade\\
E apanhei de fazê dó,\\
Com esta sorte misquinha\\
Eu saí da Lagoinha\\
Fui batê no Siridó.

Porém sempre onde eu chegava\\
Se com razão eu falava\\
Defendendo o injustiçado,\\
Depressa o côro caía\\
Com aquilo eu já vivia\\
Incabulado e afobado.

Divido tanto castigo\\
Um dia eu disse comigo:\\
Eu já apanhei com sobra,\\
Não vou mais dizê verdade,\\
Deste mundo de mardade\\
A mentira é quem manobra.

Fiquei munto desgostoso\\
Desgostoso e revortoso\\
Com o que me aconteceu\\
E hoje eu sô um vagabundo\\
E não inxiste no mundo\\
Quem minta mais do que eu.

Se a mentira é apoiada\\
E a verdade é desprezada,\\
Mudei o meu pensamento,\\
Tudo meu é sem assunto,\\
Vendo um cavalo eu pregunto\\
De quem é esse jumento.

Dos peixe que anda no mato\\
Os mió é peba e gato;\\
As lição do meu avô\\
Não tô mais obedecendo\\
Por onde eu ando é dizendo\\
Que a infração se acabou.

Derne o campo até a praça\\
Gasolina tá de graça\\
E já conheci também\\
Que o pessoá operaro\\
Tá tudo milionaro\\
Gozando e comendo bem.

Nunca houve um assartante\\
Na terra dos bandêrante\\
Nem no Rio de Janêro\\
E também já conheci\\
Que nosso grande Brasi\\
Nada deve ao estrangêro.

Respeitando um grande amigo\\
Só uma verdade eu digo,\\
Esta verdade sagrada\\
Que tá no meu coração,\\
Meu avô tinha razão\\
E a justiça tá errada.
\end{verse}

\chapter{Dois anjo}

\begin{verse}
Compade Mané Lorenço,\\
Às vez eu fico a pensá\\
Que pra pensá como eu penso\\
Não é preciso istudá,\\
Minha mãe preta não lia,\\
Mas porém ela dizia\\
Que as coisa mais boa e pura\\
Nasce da simpricidade;\\
Amô, justiça e verdade\\
Não tem nada com leitura.

O mundo de tudo tem,\\
Disto eu já sou sabedô,\\
Tem muntos home de bem\\
Que nenhuma vez pegou\\
Numa carta de \textsc{abc},\\
Mas mesmo sem sabê lê\\
Vale mais que ôro em pó\\
E tem sujeito formado\\
Com ané de advogado\\
Que não vale um cibazó.

É coisa bastante certa\\
Que pra cada criatura\\
Tem duas estrada aberta\\
Uma quilara, outra escura,\\
Minha mãe preta dizia\\
Com munta filosofia\\
Que todos fio de Adão,\\
Pode sê de quarqué raça\\
Carrega por onde passa\\
Dois anjo no coração.

Dois anjo que a gente leva\\
Um é repreto de luz\\
E o outro é da cô da treva\\
E só o que é ruim podruz,\\
O anjo bom aconseia\\
Pra não fazê coisa feia,\\
Não menti nem censurá\\
E o anjo mau dá conseio\\
Pra fazê tudo que é feio,\\
Disonrá, matá e robá.

Com a sua graça boa\\
Nosso Pai Celestiá\\
Deu força a cada pessoa\\
Para os anjo guverná,\\
O sujeito arruacêro\\
Trapacêro e disordêro\\
Cheio de intriga e moitim,\\
Sem acompanhá Jesus\\
Abandona o Anjo de Luz\\
Pra andá com o anjo ruim.

Quando o sujeito só pende\\
Pro lado da estrada escura\\
Istuda na escola e aprende\\
A linguage da leitura,\\
Mas quanto mais estudá\\
Mais aprende a trapaçá,\\
Sobe nos mais alto grau\\
Dos bancos das facurdade\\
Mas nunca dexa a ruindade\\
Que aprende com o anjo mau.

Para a pessoa gozá\\
Harmonia, paz e amô\\
Deve no peito zelá\\
O seu anjo potretô,\\
Este isprito luminoso\\
Tão bom e tão milagroso\\
Dado pela providença\\
Quá fulô que o chêro ixala\\
E dentro da gente fala\\
Pela voz da conciença.

Quem deseja sê feliz\\
Sempre na mimora tem\\
Este ditado que diz:\\
Quem pranta o bem cói o bem.\\
A pessoa que só pensa\\
Em questão e desavença\\
Arengando aqui e ali,\\
Por si prope se atropela,\\
A semente de favela\\
Nunca botou bugari.

Onde a inveja faz morada\\
Tudo que é ruim acontece\\
O anjo bom entristece,\\
Quem qué sê ruim nunca muda\\
Nos mais mió livro istuda\\
Sempre com boa linguage\\
Subindo de grau em grau\\
Alimentando o anjo mau\\
Com tudo que é safadage.

É mió dá com um pau\\
Neste isprito sem respeito\\
Este prevesso anjo mau\\
Que cada um tem no peito\\
É bom acabá com isto,\\
Vamo querê Jesus Cristo\\
Que por nós morreu na cruz,\\
Vamo sê fié e exato,\\
O anjo bom é o retrato\\
Da doutrina de Jesus.
\end{verse}

\chapter{Um sonho desfeito}

\begin{verse}
Causou tristeza e lamento\\
Em nosso grande Brasil\\
O desaparecimento\\
De um presidente civil,\\
O mesmo se achando eleito\\
Teve o seu sonho desfeito,\\
Triste coisa aconteceu,\\
Em um momento precário\\
Dizia o noticiário:\\
Tancredo Neves morreu.

Aquela notícia triste\\
Foi grande desolação\\
Que continua e persiste\\
Em nossa grande nação,\\
Dando rigorosa prova\\
Logo espalharam a nova\\
\textsc{tv}, revista e jornal\\
Anunciando a surpresa\\
E ao mesmo tempo a tristeza\\
No rosto de cada qual.

Brasil, Brasil, o teu filho\\
Na Eterna Mansão está,\\
Porém no bronze o seu brilho\\
O tempo não gastará,\\
Com os louros da vitória\\
Ele partiu para a glória\\
Ingressar na Santa Grei\\
Da Divina Majestade,\\
Sentindo a dor da saudade\\
Suspira São João Del Rei.

Tal qual o forte guerreiro\\
Que pela pátria lutou,\\
O nosso herói brasileiro\\
Vitorioso tombou\\
Com as honras de civismo\\
De amor e patriotismo\\
Seu merecido troféu\\
Duas vitórias encerra\\
A primeira, aqui na terra\\
E a Segunda lá no céu.

O povo todo em delírio\\
Por sua saúde orou\\
E ele com o seu martírio\\
Sobre o leito apresentou\\
De Jesus a semelhança,\\
Jesus ferido por lança\\
E ele pelos bisturis,\\
No sofrimento profundo,\\
Um para salvar o mundo\\
E outro salvar o país.

O valor da medicina\\
Grande vantagem nos traz,\\
Mas sem a ordem divina\\
O doutor não é capaz,\\
Cirurgiões, cientistas,\\
Médicos especialistas\\
Do nosso e de outro país,\\
Esperançosos lutavam,\\
Mas o que eles (ilegível)\\
A natureza não quis.

Com as dores cruciantes\\
Da moléstia que o atacou,\\
Pra terra dos bandeirantes\\
Um avião o levou,\\
Depois de corte e mais corte\\
A junta médica sem sorte\\
Todo trabalho perdeu,\\
Foi bem triste o resultado\\
Por doutores rodeado\\
Tancredo Neves morreu.

A sua promessa franca\\
José Sarney assumiu\\
E ele com a faixa branca\\
Do mundo se despediu,\\
Bem perto de governar\\
E com amor trabalhar\\
Em favor do bem comum,\\
Deixou na pátria adorada\\
Uma saudade gravada\\
No peito de cada um.
\end{verse}

\chapter{Assaré de 1957}

\begin{verse}
Assaré meu! Assaré meu!\\
Terra do meu coração!\\
Sempre digo que tu é\\
A terra mió do chão.\\
Me orguio quando me lembro\\
Que tu também é um membro\\
Do valente Ceará.\\
Pra mim, que te adoro tanto,\\
Te jurgo o mió recanto\\
Da terra de Juvená.

Foi aqui, foi nessa Serra\\
De Santana, onde nasci,\\
Que da água da tua terra\\
A premera vez bebi.\\
Nesta Serra, eu pequenino,\\
No meu vivê de menino,\\
Tão inocente, tão puro,\\
Dei as premera passada,\\
Triando as tua estrada\\
No rumo do meu futuro.

Eu sou um dos teu caboco\\
Que toda vida te quis,\\
E eu não invejo nem pôco\\
O resto do meu país.\\
Eu aqui tou sussegado,\\
No teu seio incalocado,\\
De tudo eu gozo contente:\\
Do crima, saúde franca,\\
Da noite, uma lua branca,\\
Do dia, um só resprendente.

Tanto te quero e dou parma,\\
Que às vez à lembrança vem\\
Que tu tem corpo e tem arma\\
Como toda gente tem.\\
Quando saio da paioça\\
Mode trabaiá na roça,\\
Prantando mio e fejão,\\
Eu inté penso que peco\\
Em batê meu enxadeco\\
Em riba deste teu chão.

Não quero que chegue a hora\\
Deu de tu me separá,\\
Pra saí de mundo afora\\
Cumo cigano, sem lá.\\
Se aqui foi meu nascimento\\
Te juro, digo e sustento\\
Que é de vivê sempre aqui\\
Do sertão inté na serra,\\
No punhadinho de terra\\
Do nosso grande Brasi.

Quero a minha vida intêra\\
Aqui te vendo e te amando,\\
E não é de brincadêra\\
O que eu te juro cantando,\\
Quando na viola toco:\\
Que não te dou nem te troco\\
Por terra de seu ninguém.\\
Quero é que Deus me dê vida\\
Uma vida bem cumprida\\
Pra gozá o que tu tem.

O vento assopra manêro\\
Durante os teus mês de estio,\\
Trazendo um certo tempêro\\
Que não faz calô nem frio.\\
E no tempo da invernada\\
Canta alegre a passarada\\
Cada quá sua canção,\\
E a garça de branca pena\\
Encruza as água serena\\
Do rio dos Bastião.

Tuas nuve arriunida\\
Traz, quando as águas derrama,\\
Sossego, alegria e vida\\
Nos coração de quem ama.\\
Como é belo a gente oiá\\
As águas cô de cristá,\\
Quando corre no riacho!\\
Parece dentro da mata\\
Uma cabrona de prata\\
Descendo de serra abaixo.

Tudo escuta e tudo vê\\
Quem por as brenha passeia.\\
Canta a linda zabelê,\\
Zoa o enxame de abêia,\\
E o cipó-laça-vaquêro\\
Se atrepa e faz um barsêro\\
Por riba do cramunzé,\\
Cumo cobra que se enrosca,\\
E o vento fazendo cosca\\
Na copa do catolé.

Da baxa inté na chapada\\
Tudo é paz, tudo é beleza,\\
Gozando a musga animada\\
Da festa da natureza.\\
Vai fazendo pirueta\\
Um cordão de brabulêta\\
Arrodeando as lagoa.\\
Tudo forga de contente,\\
Por toda parte se sente\\
Um chêro de coisa boa.

O sabiá e o sofreu,\\
O canaro e outros tanto,\\
Com as voz que Deus lhe deu,\\
Cada um sorta o seu canto.\\
E o beija-fulô penera\\
Nas fulô da primavera,\\
Chêrando o doce prefume,\\
E afiná tem fromosura\\
Inté sua noite escura\\
Pintada de vagalume.

Toda essas coisa que eu digo,\\
E outras que eu não sei dizê,\\
Tu tem, Assaré amigo:\\
Quem nunca viu venha vê.\\
Se é cantando estas beleza\\
Que a minha arma aqui tá presa,\\
E o coração preso aqui,\\
Sou o mais feliz dos cabôco,\\
E não invejo nem pôco\\
O resto do meu Brasi.

Mas, meu Assaré amado,\\
Sinto munto a tua sorte!\\
Tu és dos mais deserdado\\
Daqui das banda do Norte.\\
Tu nada goza da histora,\\
Não tem fama, nem gulora,\\
Nunca arguém te protegeu.\\
Tu só tem essas riqueza,\\
As coisa da Natureza,\\
Que Nosso Senhô te deu.

Eu também, meu Assaré,\\
Sou pobre, quage de esmola;\\
Sou assim como tu é,\\
Só pissuo esta viola.\\
Nasci e me criei nos mato,\\
Sem paletó, sem sapato,\\
E sem recebê lição.\\
Pobre, sem nada, na rapa,\\
Não conheço nem as capa\\
Dos livro de estudação.

Não posso te protegê,\\
Nada tenho pra te dá,\\
Mas porém quando eu morrê\\
Com razão vou te dêxá\\
Uma pequena lembrança,\\
Uma pequenina herança,\\
Em prova do meu amô:\\
O nome de um passarinho,\\
Uma viola de pinho\\
E os versos de um cantadô.

Não te dêxo abandonado.\\
Durante enquanto eu vivê,\\
Eu é de tá do teu lado\\
Te ajudando a padecê.\\
Nunca, nunca te desprezo,\\
E todas vez quando rezo\\
Te encomendo horas intêra\\
À santa que mais se adora\\
A Virge Nossa Senhora,\\
Tua santa padroêra.

Quando em tua igreja vou\\
Fazê minhas oração,\\
Com amô e com frevô\\
Mode te dá proteção\\
A Nossa Senhora peço.\\
Pruquê a luz do progresso\\
Tu não tem na tua vida,\\
É grande o teu sofrimento,\\
Tu não pissui carçamento,\\
Nem colejo, nem vinida.

Tu também não tem cinema,\\
Também não tem hospitá,\\
Veve preso nas argema\\
Sem ninguém te libertá.\\
É grande a dô que padece,\\
Quem te visita conhece\\
A tua situação.\\
Veve sofrendo um desprezo\\
Com o criminoso preso\\
Piado de pá pra mão.

Tu é o mermo Assaré\\
Do véio tempo passado,\\
Sem esperança, sem fé,\\
De rôpa xuja e barbado,\\
Teu sembrante nunca muda,\\
Só pruquê ninguém te ajuda.\\
Que seja inverno ou estio,\\
Sempre é triste, dando ai!\\
Tu é como um véio pai\\
Abandonado do fio.

Conheço cidade rica\\
Que quem repara tá vendo\\
Que quanto mais véia fica,\\
Mais nova tá parecendo.\\
Mas tu, Assaré querido,\\
Véio, cacundo, incuído,\\
Com o óio fito pro chão,\\
Não tem conforto nem nome!\\
Só é lembrado dos home\\
Quando é tempo de inleição.

Mas tu, meu torrão que chora,\\
Vai suportando o desgosto\\
Inté que uma nova orora\\
Venha briá no teu rosto.\\
Pode sê minha cidade,\\
Que um dia a felicidade\\
Te alevante do paul,\\
E a Divina Providença,\\
Com o seu pudê e cremença\\
Mande um oxilo pra tu.

Não fique no desengano,\\
Neste desgosto profundo\\
Que a desfilada dos ano\\
Traz tudo no véio mundo.\\
Vai aguentando este escuro,\\
Que um dia no teu futuro\\
Pode sê que chegue a vez\\
De teus fio ajueiado\\
Se arrependa dos pecado,\\
Do grande má que te fez.

E agora, Assaré amigo,\\
Peço descurpa e perdão\\
Destas coisas que te digo\\
Com pesar no coração.\\
Se acha que tou te ofendendo\\
Dispense o que tou dizendo,\\
Me perdoi por caridade,\\
De falá pubricamente,\\
Descubrindo a toda gente\\
A tua necessidade.

Satisfazendo um desejo\\
Que sempre me acompanhava,\\
Como cantô sertanejo\\
Eu fui vê se te cantava,\\
Meu torrão querido e nobre.\\
Mas te vendo assim tão pobre,\\
Tão pobre, tão sem recuço,\\
Quebrou-se a minha viola,\\
E os verso em minha cachola\\
Se acabou tudo em saluço.
\end{verse}

\chapter{Pergunta de moradô}

\begin{flushright}
\emph{Autor: Geraldo Gonçalves de Alencar}
\end{flushright}


\begin{verse}
Meu patrão, não tenho nada,\\
O sinhô de tudo tem,\\
Porém a razão de cada\\
É coisa que me convém.\\
Meu patrão tem boa vida,\\
Tem gado, loja surtida,\\
Farinha, mio e fejão,\\
Já eu não pissuio nada\\
Vivo de mão calejada\\
Na roça de meu patrão.

Meu patrão, seja sincero,\\
Seja franco, honesto, exato,\\
Preguntando assim não quero\\
Metê a mão em seu prato;\\
O que desejo sabê\\
Pode o sinhô me dizê\\
Sem medo e sem afrição,\\
Pode se firme e sincero\\
Lhe juro como não quero\\
Usá de tapiação.

Fale sero sem tapiá\\
Certo cumo a exatidão,\\
Qué que meu patrão fazia\\
Se eu passasse a sê patrão\\
E meu patrão de repente\\
Tomasse a minha patente\\
De cativo moradô,\\
Morando numa paioça\\
Trabaiando em minha roça\\
Sendo meu trabaiadô?

E enquanto no meu roçado\\
Tratasse do meu legume\\
Me visse todo equipado\\
Todo pronto de prefume\\
Entrá pra dentro dum carro\\
Fumando belo cigarro\\
Sem oiá seu sacrifiço\\
E o sinhô se acabrunhando\\
Trabaiando, trabaiando,\\
Acabando meu serviço?

Qué que meu patrão fazia\\
Se fosse meu moradô\\
Trabaiando todo dia\\
Bem por fora do valô?\\
Me vendo num palacete\\
Sabureando banquete\\
Daqueles que o sinhô come\\
E o sinhô no meu roçado\\
Trabaiando no alugado\\
Doente e passando fome?

Qué que meu patão fazia\\
Nessas condição assim\\
Trabaiando todo dia\\
Num sacrifiço sem fim,\\
Sem obtê resurtado\\
Daquele grande roçado\\
Onde muito trabaiô,\\
O sinhô não se desgoste\\
Se fô possive arresposte,\\
O que fazia o sinhô?
\end{verse}

\chapter{Resposta de patrão}

\begin{verse}
O que você perguntou,\\
Pobre infeliz agregado,\\
Com a resposta que eu dou\\
Ficará mais humilhado.\\
Se você fosse o patrão\\
E eu na sua sujeição,\\
Seria um estado horrendo\\
O meu grande padecer\\
E teria que fazer\\
O que você está fazendo.

Porém, eu tenho cuidado,\\
Meus planos sempre são certos\\
E o povo tem um ditado\\
Que o mundo é dos espertos;\\
Eu fui um menino pobre\\
Mas sempre arranjava cobre\\
No meu papel de estradeiro.\\
Esta tal honestidade\\
É contra a felicidade\\
De quem quer juntar dinheiro.

No papel de mexerico\\
Tirei primeiro lugar,\\
Fui o leva-e-traz do rico\\
Que vive a politicar,\\
Quando fiado eu comprava,\\
Depois a conta eu negava\\
E nunca me saí mal,\\
E pra fazer mão de gato\\
Em favor de candidato,\\
Já fui cabo eleitoral.

Roubar no peso e medida\\
Sem o freguês perceber,\\
Foi coisa que em minha vida\\
Nunca deixei de fazer;\\
Com a minha inteligência\\
Repleta de experiência\\
Eu sempre me saí bem,\\
Assim eu fui pelejando,\\
Me virando, me virando\\
E hoje sou rico também.

Tenho fazenda de gado,\\
Tenho grande agricultura\\
E é à custa do agregado\\
Que eu faço grande fartura,\\
Toda vida eu me preparo\\
Para sempre vender caro\\
E sempre comprar barato\\
E o voto dos eleitores,\\
Que são os meus moradores\\
Eu vendo ao meu candidato.

Hoje, sou homem do meio,\\
Tenho o nome no jornal,\\
Tenho carro de passeio\\
E freqüento a capital;\\
Se um homem ao outro explora,\\
Sei que ninguém ignora,\\
É fraqueza da matéria\\
E você, pobre agregado,\\
Tem que me escutar calado\\
E se acabar na miséria.

Me pergunta o que eu faria\\
Se eu fosse seu morador\\
Trabalhando todo dia\\
Bem por fora do valor!\\
E pergunta com o gesto\\
De quem é correto e honesto,\\
Porém, você está sabendo\\
Que em minha terra morando,\\
Passa a vida me pagando\\
E vai morrer me devendo.

Com a minha habilidade\\
Eu me defendo e me vingo,\\
Expondo minha verdade\\
Acabo seu choramingo,\\
Quando você perguntava\\
Achou que me encabulava\\
Com o seu grande clamor,\\
Mas tomou errado o bonde,\\
É assim que patrão responde\\
Pergunta de morador.
\end{verse}

\chapter{Pé quebrado}

\begin{flushright}
\emph{(Paródia de Fulô de Puxinanã de Zé da Luz)}
\end{flushright}

\begin{verse}
Neste país invejado,\\
De tanto já ter sofrido\\
O nosso índio é conhecido\\
Pelo sinal.

Foi Pedro Álvares Cabral\\
O causador desta guerra\\
Quando descobriu a terra\\
Da Santa Cruz.

Como cruéis canguçus\\
Contra os índios se bateram\\
Sofrer o que ele sofreram\\
Livre-nos Deus!

Para que o índio e os seus\\
Tenham terra e domicílio,\\
Pedimos o vosso auxílio\\
Nosso Senhor.

Os índios com o seu valos\\
Eram donos deste chão\\
Merecem a gratidão\\
Dos nossos.

Se não respeitam seus troços\\
E as reservas onde estão\\
Estão provando que são\\
Inimigos.

Por causa destes castigos,\\
O medonho padecer\\
Recorremos ao poder\\
Em nome do padre.

Daí Senhora Santa Madre,\\
Ao indígena mantimento,\\
Está faltando alimento\\
Do filho.

Para que cesse o empecilho,\\
Perseguição e trapaça\\
Pedimos a santa graça\\
Do Espírito Santo.

Os que maltrataram tanto\\
Os nossos índios queridos\\
Mais tarde serão punidos\\
Amém.
\end{verse}

\chapter{A terra é nossa}

\begin{verse}
Deus fez a grande natura\\
Com tudo quanto ela tem,\\
Mas não passou escritura\\
Da terra para ninguém.

Se a terra foi Deus quem fez\\
Se é obra da Criação\\
Deve cada camponês\\
Ter uma faixa de chão.

Esta terra é desmedida\\
E com certeza é comum,\\
Precisa ser dividida\\
Um tanto pra cada um.

A Argentina e a Inglaterra\\
Formaram duros engodos\\
Por uma faixa de terra\\
Que Deus deixou para todos.

Faz pena ver sobre a terra\\
O sangue humano correr\\
O grande provoca guerra\\
Para o pequeno morrer.

Vive o mundo sempre em guerra\\
Ambicioso e sanhudo,\\
Tudo brigando por terra\\
E a terra comendo tudo.
\end{verse}

\chapter{Egoísmo}

\begin{verse}
Sem ver as grande cegueiras\\
Da sua própria pessoa\\
Vive o homem sempre às carreiras\\
Atrás de uma coisa boa,\\
Quando a coisa boa alcança,\\
Ele ainda não descansa,\\
Sente um desejo maior,\\
Esquece aquela ventura,\\
E corre logo à procura\\
De outra coisa bem melhor.

Se a segunda ele alcançar,\\
Aumenta mais a cegueira,\\
Fica sem se conformar\\
Correndo atrás da terceira,\\
Vem a quarta, a quinta, a sexta\\
E ele sendo o mesmo besta\\
Correndo atrás de ventura,\\
Assim esta vida passa\\
Assim o desgraçado fracassa\\
No fundo da sepultura.
\end{verse}

\chapter{Saudade}

\begin{verse}
Saudade dentro do peito\\
É qual fogo de monturo,\\
Por fora tudo perfeito,\\
Por dentro fazendo furo.

Há dor que mata a pessoa\\
Sem dó e sem piedade,\\
Porém não há dor que doa\\
Como a dor de uma saudade.

Saudade é um aperreio\\
Pra quem na vida gozou,\\
É um grande saco cheio\\
Daquilo que já passou.

Saudade é canto magoado\\
No coração de quem sente\\
É como a voz do passado\\
Ecoando no presente.

A saudade é jardineira\\
Que planta em peito qualquer,\\
Quando ela planta cegueira\\
No coração da mulher,\\
Fica tal qual a frieira\\
Quanto mais coça mais quer.
\end{verse}

\chapter{A garça e o urubu}

\begin{verse}
Certa garça presunçosa\\
Pousando à margem de um rio,\\
Tecia muito vaidosa\\
A si um grande elogio.

E com um riso escarninho\\
Criticava com desdém\\
De um urubu coitadinho\\
Que se achava ali também.

Dizia a garça: Urubu,\\
Preto, nojento e sombrio,\\
O que andas fazendo tu\\
Na margem deste meu rio?

Ante o raio zombeteiro\\
Sentiu no peito uma dor\\
E voando bem ligeiro\\
Foi se queixar ao condor.

Com esforços incansáveis\\
Foi depressa ao infinito\\
E contou ao rei das aves\\
O que a garça tinha dito.

O condor co a voz pausada\\
Respondeu por sua vez:\\
Urubu meu camarada\\
Cada qual como Deus fez.

Eu lhe afirmo e falo franco\\
Para deus o sumo bem\\
O valor que tem o branco\\
Tem o crioulo também.

Com o que a garça falou\\
Não queira triste ficar\\
Pois só Deus que nos\\
criou É quem nos pode julgar.
\end{verse}

\chapter{Três beijos}

\begin{verse}
Jesus o Verbo Encarnado\\
Com o seu amor profundo\\
Quando andou por este mundo\\
Por três vezes foi beijado,\\
Primeiro por Madalena\\
A pecadora morena,\\
Quando no seu coração\\
Entrou um raio de luz\\
E ela procurou Jesus\\
Para lhe pedir perdão.

Depois o segundo beijo,\\
O beijo da falsidade,\\
Quando aproveitando o ensejo\\
Judas cheio de maldade\\
Cometeu naquele dia\\
A maior hipocrisia\\
Que pode haver contra Deus,\\
De Cristo se aproximou\\
E em sua face beijou\\
Lhe entregando aos fariseus.

Depois quando perseguido\\
Nosso Cristo Redentor\\
Já cruelmente ferido\\
Lhe trataram com rigor\\
Pregando sobre o madeiro,\\
Lhe veio o beijo terceiro\\
Quando a sua mãe querida\\
Com um olhar puro e terno\\
Lhe aplicou o beijo materno\\
O beijo da despedida.
\end{verse}

\chapter{Rosa e rosinha}

\begin{verse}
Uma rosinha mostrando\\
Sua beleza e perfume,\\
Me olhava triste chorando\\
Com inveja e com ciúme.

Eu disse à pobre coitada\\
Não tenha raiva de mim,\\
Eu não sou disto culpada\\
Foi Deus quem me fez assim.

Você é rosa e eu sou Rosa,\\
Por Rosa fui batizada\\
E se eu nasci mais formosa,\\
Eu não sou disso culpada.

Se alegre com o que é seu,\\
Ter inveja não convém,\\
Você não é como eu\\
Mas é formosa também.

A rosinha no seu galho\\
Me ouviu e se conformou,\\
As suas lágrimas de orvalho\\
A luz do sol enxugou.
\end{verse}

\chapter{Meu passarinho}

\begin{flushright}
\emph{Ao meu neto Expedito}
\end{flushright}

\begin{verse}
Eu andando no mato achei um ninho\\
Bem fofinho e macio, que beleza!\\
Quando olhei para dentro, que surpresa!\\
Tinha um lindo e mimoso passarinho.

Era lindo e mimoso de encantar,\\
Não podia voar porque as penas\\
Inda estavam pequenas, bem pequenas,\\
Não podiam seu corpo transportar.

Eu correndo ia lá toda manhã\\
Para ver o bichinho encantador\\
O retrato fiel do puro amor\\
No seu ninho feliz feito de lã.

Mas um dia fui lá com todo orvalho,\\
Era cedo e fazia muito frio,\\
Vi o ninho sem nada, bem vazio\\
E o maroto juntinho sobre um galho.

Do seu ninho o maroto esta fora\\
E quando um jeito de pega-lo fiz,\\
Saiu ele a voar como quem diz:\\
Meu adeus, Expedito, eu vou me embora.
\end{verse}

\chapter{Óios redondo}

\begin{verse}
Nesta vida aperriada\\
Pra me livrá das furada\\
Destes teus óios redondo,\\
Caboca onde é que eu me soco\\
Caboca onde eu me coloco?\\
Caboca onde é que eu me escondo?

Pra me esquece dos teus óio\\
Eu canto, eu grito, eu abóio,\\
Faço tudo que é preciso,\\
Mas por onde eu vou passando\\
Sinto teus óio briando\\
Por dentro do meu juízo.

Meu padecê, minha cruz,\\
É tuas bolas de luz\\
Que me dêxa incandiado,\\
Estas duas jóias prima\\
Com a força de dois ímã\\
Me puxando pra teu lado.

Vendo os teus óio prefeito\\
Sinto entrando no meu peito\\
Dois ferrão de marimbondo\\
Caboca, não seja ingrata,\\
Tu me martrata e me mata\\
Com estes óio redondo.

Me tire desta sentença,\\
Tu só parece que pensa\\
Que eu não tenho coração,\\
Tu me amofina e me aleja\\
De ruêdera, de inveja,\\
De ciúme e de paixão.

Sabe quá é a meizinha\\
Pra essa doença minha?\\
Pregunta que eu te respondo,\\
Era se tu me quisesse\\
E de coração me desse\\
Estes teus óio redondo.
\end{verse}

\chapter{Crime imperdoável}

\begin{verse}
Com sua filha de bondade infinda,\\
Maria Rita, encantadora e bela,\\
Morava a viúva dona Carolinda,\\
Junto ao engenho do senhor Favela.

Paciente e boa e cheia de carinho,\\
Passava os dias sem pensar na dor,\\
Reinava ali, naquele tosco ninho,\\
Um grande exemplo do mais puro amor.

A linda jovem, flor de simpatia,\\
De olhos brilhantes e cabelo louro,\\
Além de arrimo e doce companhia\\
Era da mãe o virginal tesouro.

Tinha uma voz harmoniosa e grata\\
Maria Rita, a filha da viúva,\\
Igual à voz do sabiá da mata,\\
Quando ele canta na primavera chuva.

Maurício, um filho do senhor do engenho,\\
Um estudante, bacharel futuro,\\
Apaixonou-se, com o maior empenho\\
De saciar o coração impuro.

E com promessas de um porvir brilhante,\\
Fazendo juras de casar com ela,\\
Tanto insistiu o traidor pedante\\
Que conquistou a infeliz donzela.

Tornou-se em pranto da menina o riso,\\
Anuviou-se o doce amor materno,\\
Aquele rancho que era um paraíso,\\
Foi transformado em verdadeiro inferno.

Depois, expulsas pelo mundo afora,\\
Sorvendo a taça de amargoso fel,\\
Soluça mãe e a triste filha chora,\\
Horrorizadas do chacal cruel.

Vive hoje o monstro a prosseguir no estudo,\\
Enquanto o manto da miséria as cobre,\\
Porque só o rico tem direito a tudo,\\
Não há justiça para quem é pobre.
\end{verse}

\chapter{Curioso e miudinho}

\begin{verse}
C. Quem é você, que alegre se apresenta\\
Com a altura de dois metros e oitenta?

M. Onde eu ando me chamam Miudinho,\\
Tudo vejo e decifro em meu caminho.

C. Miudinho, e com tanta dimensão,\\
No volume do corpo e na noção?

M. Se o mundo sempre foi contradição,\\
O que assim me tratar possui razão.

C. Miudinho, com o seu saber profundo,\\
Conhece alguma coisa do outro mundo?

M. Não há mesmo quem possa saber nada\\
Do que vive por trás de uma murada.

C. Miudinho me diga o que é política?

M. É um dilema de onde nasce a crítica.

C. E este argumento para onde se lança?

M. Para os dois pratos de uma só balança.

C. E na campanha quem vitória alcança?

M. Quem mais mentira sobre o prato lança.

C. Miudinho, obrigado por ser franco,\\
Nas eleições eu vou votar em branco.
\end{verse}

\chapter{Linguage dos óio}

\begin{verse}
Quem repara o corpo humano\\
E com coidado nalisa,\\
Vê que o Autô Soberano\\
Lhe deu tudo o que precisa,\\
Os orgo que a gente tem\\
Tudo serve munto bem,\\
Mas ninguém pode negá\\
Que o Autô da Criação\\
Fez com maió prefeição\\
Os orgo visioná.

Os óio além de chorá,\\
É quem vê a nossa estrada\\
Mode o corpo se livrá\\
De queda e de barruada\\
E além de chorá e de vê\\
Prumode nos defendê,\\
Tem mais um grande mister\\
De admirave vantage,\\
Na sua muda linguage\\
Diz quando qué ou não qué.

Os óio consigo tem\\
Incomparave segredo,\\
Tem o oiá querendo bem\\
E o oiá sentindo medo,\\
A pessoa apaxonada\\
Não precisa dizê nada,\\
Não precisa utilizá\\
A língua que tem na bôca,\\
O oiá de uma caboca\\
Diz quando qué namorá.

Munta comunicação\\
Os óio veve fazendo,\\
Por inxempro, oiá pidão\\
Dá siná que tá querendo\\
Tudo apresenta na vista,\\
Comparo com o truquista\\
Trabaiando bem ativo\\
Dexando o povo enganado,\\
Os óio pissui dois lado,\\
Positivo e negativo.

Mesmo sem nada falá,\\
Mesmo assim calado e mudo,\\
Os orgo visioná\\
Sabe dá siná de tudo,\\
Quando fica namorado\\
Pela moça desprezado\\
Não precisa conversá,\\
Logo ele tá entendendo\\
Os óio dela dizendo,\\
Viva lá que eu vivo cá.

Os óio conversa munto\\
Nele um grande livro inxiste\\
Todo repreto de assunto,\\
Por inxempro o oiá trste\\
Com certeza tá contando\\
Que seu dono tá passando\\
Um sofrimento sem fim,\\
E o oiá desconfiado\\
Diz que o seu dono é curpado\\
Fez arguma coisa ruim.

Os óio duma pessoa\\
Pode bem sê comparado\\
Com as água da lagoa\\
Quando o vento tá parado,\\
Mas porém no mesmo istante\\
Pode ficá revortante\\
Querendo desafiá,\\
Infuricido e valente;\\
Nestes dois malandro a gente\\
Nunca pode confiá.

Oiá puro, manso e terno,\\
Potretô e cheio de brio\\
É o doce oiá materno\\
Pedindo para o seu fio\\
Saúde e felicidade,\\
Este oiá de piedade\\
De perdão e de ternura\\
Diz que preza, que ama e estima\\
É os óio que se aproxima\\
Dos óio da Virge Pura.

Nem mesmo os grande oculista,\\
Os dotô que munto estuda,\\
Os mais maió cientista,\\
Conhece a linguage muda\\
Dos orgo visioná\\
E os mais ruim de decifrá\\
De todos que eu tô falando,\\
É quando o oiá é zanôio,\\
Ninguém sabe cada ôio\\
Pra onde tá reparando.
\end{verse}

\chapter{Três moça}

\begin{flushright}
\emph{(Paródia de Fulô de Puxinanã de Zé da Luz)}
\end{flushright}

\begin{verse}
Três moça, três atração,\\
Três anjo andando na terra,\\
Eu vi lá no pé da serra\\
Numa noite de São João.

A premera era a Benvinda\\
E eu juro pro Jesus Cristo\\
Como eu nunca tinha visto\\
Uma coisinha tão linda.

Benvinda, o premero anjo\\
Tinha a voz harmoniosa\\
Como as corda sonorosa\\
Do bandulim dos arcanjo.

A segunda, a Filisberta,\\
Era um mundo de beleza,\\
Não sei como a Natureza\\
Acertou pra fazê ela.

Os óio era dois primô\\
Com tanta quilaridade\\
Como quem sente a sodade\\
De um bem que nunca vortou.

A tercera, a Conceição,\\
Era a mais nova das três\\
Parecia santa Inês\\
Quando sai na procissão.

Nunca houve sobre a terra\\
E não pôde havê ainda\\
Quem diga qual a mais linda\\
Das moça do pé da Serra.

Se arguém mandasse eu jurgá\\
E a mais bonita iscuiê\\
Eu ficava sem sabê,\\
Pois todas três era iguá.

Quando oiei pras três menina\\
Oiei tornei a óiá,\\
Eu fiquei a maginá\\
Nas coisa santa e divina.

E o que ninguém desejou\\
Desejei naquela hora\\
Sê o grande Rei da Gulora\\
O Divino Criadô.

Mode agarrá as três donzelas,\\
Invorvê num santo véu\\
E levá viva pro céu\\
Pra ninguém mexê com elas.
\end{verse}

\chapter{Injustiça}

\begin{verse}
O nosso selvático vivia contente\\
Quando estranha gente\\
Na taba chegou\\
E o índio liberto foi subordinado,\\
Foi escravizado,\\
Sem terra ficou.

Se é grande injustiça tomar o que é alheio,\\
Se é um ato feio,\\
Se é crime de horror,\\
Na culpa medonha os brancos caíram\\
Porque transgrediram\\
A lei do Senhor.

Faz pena sabermos que muitas aldeias\\
Outrora bem cheias\\
Já tiveram fim,\\
É triste sabermos que os índios coitados\\
Sem serem culpados\\
Sofreram tanto assim.

Em nome daqueles que vivem sem terra\\
E não querem guerra\\
Procuram a paz,\\
A igreja reclama amor e piedade\\
E a fraternidade\\
Que o gozo traz.

Queremos a paz que tanto ensinava\\
Qua tanto pregava\\
Jesus nosso Rei,\\
Direitos humanos, o grau de igualdade,\\
E a voz da verdade\\
Em nome da lei.

Os índios precisam de um ponto seguro\\
Um belo futuro\\
Para os filhos seus\\
Eles não merecem tamanhos castigos\\
São nossos amigos,\\
São filhos de Deus.
\end{verse}

\chapter{Assaré e Mossoró}

\begin{verse}
Curiosidade e vontade\\
De ver mossoró eu tinha\\
Hoje vi e sei que a cidade\\
É diferente da minha,\\
Porém, isto é natural\\
Este mundo desigual\\
Tem o rico e o pobre Jó\\
No meu estilo singelo\\
Faço aqui um paralelo\\
Entre Assaré e Mossoró.

Eu vejo que é Mossoró\\
Diferente do Assaré\\
Um com a tônica no o,\\
O outro a tônica no e\\
Assaré modesto e pobre,\\
Mossoró pomposo e nobre\\
Onde a cultura se expande,\\
Um longe do outro está\\
Assaré, no Ceará,\\
Mossoró, no Rio Grande.

Assaré não é lembrado\\
Por entre as folhas da história,\\
Mossoró, no seu Estado\\
É grande padrão de glória.\\
Não me convém que eu afaste\\
A prova deste contraste,\\
Com evolução completa\\
Mossoró desenvolvido\\
E Assaré só conhecido\\
Através do seu poeta.

Nos meus versos justifico,\\
Cada qual tem clima ameno,\\
Mas Mossoró grande e rico\\
E Assaré pobre pequeno,\\
Para que ele se destaque\\
Vou rogar ao Deus de Isaque,\\
De Abrahão e de Jacó,\\
Talvez assim ele um dia\\
Tenha a mesma fantasia\\
Que agora tem Mossoró.

Nesta singela linguagem,\\
Neste estilo popular\\
Ofereço esta mensagem\\
Ao bom povo Potiguar\\
Onde fui bem recebido\\
E ao meu Assaré querido\\
Minha terra e meu xodó,\\
Voltarei bem satisfeito\\
Levando dentro do peito\\
Saudade de Mossoró.
\end{verse}

\chapter{Ele e ela}

\begin{verse}
Ela dizia, fora da razão,\\
Qual fera brava, toda enfurecida:\\
--- Eu me casei porque fui iludida,\\
Por isso fiz a minha perdição.

Maldita a hora em que te dei a mão,\\
Sem dar valor à minha própria vida,\\
Deixei meu pai e minha mãe querida\\
Para abraçar este infiel tição.

Negro nojento, sem asseio, imundo\\
Não terei mais satisfação no mundo,\\
Em conseqüência deste compromisso

Com quem vivia de baralho e pinga.\\
Foi bruxaria, catimbó, mandinga!\\
Você, marmanjo, me botou feitiço!

E ele, bem calmo, com um gesto amigo,\\
Depois de um trago de aguardente fria:\\
Eu já escutei o seu bê-a-bá, Maria\\
Escute agora o que também lhe digo.

Não sou a causa deste seu castigo.\\
Se, com certeza, muito lhe queria,\\
Atrás de mim você também vivia;\\
Seu belo sonho era casar comigo.\\
A sua história não está bem certa.\\
Nossa amizade sempre foi liberta,\\
Um só afeto nunca me negou.

Como ao contrário do que você fala,\\
Ainda guardo lá na minha mala\\
Todas as cartas que você mandou.
\end{verse}

\chapter{Língua ferina}

\begin{verse}
Se alguém te chama de cabrita feia,\\
Tudo isto é inveja, é ambição e ciúme,\\
Gente ferina de malícia cheia,\\
Negra navalha de afiado gume.

Não te amofines quando alguém te tacha\\
De sassarica e de coruja choca,\\
Pois testemunha do que diz não acha\\
Quem te insulte, te fere e te provoca.

Tu és tão simples como flor silvestre\\
E pouco importa a tua forma rude,\\
Para o nosso Jesus Divino Mestre\\
A riqueza maior é a virtude.

Sempre é cheia de espinho a vida nossa\\
E o mal precisa de um perdão clemente,\\
Neste mundo cruel não há quem possa\\
Com língua mordaz de certa gente.
\end{verse}

\chapter{Barriga branca}

\begin{verse}
Quando vive o marido atravancado\\
De cabresto, cambão, canga e tamanca,\\
Aos caprichos da esposa escravizado,\\
Recebe o nome de barriga branca.

Nunca pode fazer o que ele quer\\
O pobre diabo, o tal barriga branca,\\
Sempre cumprindo as ordens da mulher,\\
Ele é o dono da casa e ela da tranca.

Ele escuta calado e sempre mudo\\
Sua esposa da língua de tarisca,\\
Ela é quem manda e quem comanda tudo,\\
Ele só corta por onde ela risca.

Em qualquer festa do melhor brinquedo\\
Se ela nota que o pobre está contente,\\
Logo lhe ordena com um gesto azedo:\\
Vamos voltar, está doendo um dente.

Na sua ordem rigorosa e dura\\
Ninguém pode tirar suas razões,\\
Aos amigos do esposo ela censura\\
E procura cortar as relações.

Tu és, barriga branca, um desgraçado,\\
Por onde passas todos te dão vaia\\
Teu destino é viver subordinado\\
Sob o jugo humilhante duma saia.

Tu és um carro que não sai da pista,\\
Rodas constante velozmente e bom,\\
Tua esposa é o único motorista\\
Pé no teu freio e mão no teu guidom.

É lamentável teu viver profundo\\
Nunca serás autoridade franca,\\
Tens um inferno neste nosso mundo,\\
É muito triste ser barriga branca!
\end{verse}

\section{O nadador}

\begin{flushright}
\emph{Ao meu sobrinho e colega}\\
\emph{Geraldo Gonçalves Alencar}
\end{flushright}

\begin{verse}
Desde novo, gostou de ver as águas\\
Do oceano, tão verdes e tão belas\\
E ele pensava que vagando nelas\\
Poderia aplacar as suas mágoas.

Conduzido por este pensamento,\\
Aprendeu a nadar em tempo breve,\\
Como se fosse canoa leve\\
Singrando as ondas no soprar do vento.

Seus amigos, lhe vendo sobre o mar,\\
Tranqüilamente, sem temer as brumas,\\
Transpondo as vagas, sacudindo espumas,\\
Sentiram ânsia de também nadar.

Sem temerem das ondas o furor,\\
Cada qual, a sorrir, dizia: Eu entro!\\
E se jogaram de oceano adentro\\
Com a mesma intenção do Nadador.

E assim tangidos por vontade louca,\\
Alguns lhes até fazendo apostas,\\
Uns nadavam de frente, outros de costas\\
Vendo as águas lhe entrando pela boca.

O mar, raivoso, nunca fez carinho,\\
Os teimosos e ousados aprendizes,\\
Foram todos, coitados! Infelizes\\
Deixando o bravo Nadador sozinho.

Foram todos das águas se afastando,\\
Receosos da forte maresia\\
E daqueles, no mar da poesia\\
Só Geraldo Alencar ficou nadando.
\end{verse}

\begin{flushright}
\emph{(Hospital São Francisco, Rio de Janeiro -- 1975)}
\end{flushright}

\chapter{Filho de gato é gatinho}

\begin{verse}
Era o esposo assaltante perigoso,\\
O mais famoso dentre os marginais,\\
Porém, se ele era assim astucioso\\
Sua esposa roubava muito mais

A ladra certo dia se sentindo\\
Com sintonia e sinal de gravidez\\
Disse ao marido satisfeito e rindo:\\
--- Eu vou ser mãe pela primeira vez!

Ouça, querido, eu tive um pensamento\\
Precisamos viver com precaução\\
Para nunca saber nosso rebento\\
Desta nossa maldita profissão

Nós vamos educar nosso filhinho\\
Dando a ele as melhores instruções\\
Para o mesmo seguir o bom caminho\\
Sem conhecer que somos dois ladrões

Respondeu o marido: --- está direito,\\
Meu amor, você disse uma verdade\\
De hoje em diante eu procurarei um jeito\\
De roubar com maior sagacidade

Aspirando o melhor sonho de rosa\\
Ambos riam fazendo os planos seus\\
E mais tarde a ladrona esperançosa\\
Teve um parto feliz, graças a Deus

Ai, como é linda, que joinha bela\\
Diziam os ladrões, cheios de amor\\
Cada qual desejando para ela\\
Um futuro risonho e promissor

Mas logo viram com igual surpresa\\
Que uma das mãos da mesma era fechada,\\
Disse mãe, soluçando de tristeza:\\
--- Minha pobre menina é aleijada

A mãe, aflita, teve uma lembrança\\
De olhar a mão da filha bem no centro\\
Quando abriu a mãozinha da criança\\
A aliança da parteira estava dentro
\end{verse}

\chapter{Lição do pinto}

\hfill\emph{Versos recitados pelo autor em um}

\hfill\emph{comício em favor da anistia}


\begin{verse}
Vamos meu irmão,\\
A grande lição\\
Vamos aprender,\\
É belo o instinto\\
Do pequeno pinto\\
Antes de nascer.

O pinto dentro do ovo\\
Está ensinando ao povo\\
Que é preciso trabalhar,\\
Bate o bico, bate o bico\\
Bate o bico tico-tico\\
Pra poder se libertar.

Vamos minha gente,\\
Vamos pra frente\\
Arrastando a cruz\\
Atrás da verdade,\\
Da fraternidade\\
Que Pregou Jesus.

O pinto prisioneiro\\
Pra sair do cativeiro\\
Vive bastante a lutar,\\
Bate o bico, bate o bico,\\
Bate o bico tico-tico\\
Pra poder se libertar.

Se direito temos,\\
Todos nós queremos\\
Liberdade e paz,\\
No direito humano\\
Não existe engano,\\
Todos são iguais.

O pinto dentro do ovo\\
Aspirando um mundo novo\\
Não deixa de beliscar\\
Bate o bico tico-tico\\
Bate o bico, bate o bico,\\
Pra poder se libertar.
\end{verse}

\chapter{Seca dágua}

\hfill\emph{Musicada e cantada pelos artistas do}

\hfill\emph{Nordeste Já -- 1985}

\begin{verse}
É triste para o Nordeste\\
O que a Natureza fez\\
Mandou 5 anos de seca,\\
Uma chuva em cada mês\\
E agora em 85\\
Mandou tudo de uma vez.

\textsc{refrão}

A sorte do nordestino\\
É mesmo de fazer dó,\\
Seca sem chuva é ruim\\
Mas seca dágua é pió.

Quando chove brandamente\\
Depressa nasce o capim,\\
Dá milho, arroz e feijão,\\
Mandioca e amendoim,\\
Mas como em 85\\
Até o sapo achou ruim.


\textsc{refrão}

Maranhão e Piauí\\
Estão sofrendo por lá\\
Mas o maior sofrimento\\
É nestas bandas de cá,\\
Pernambuco, Rio Grande,\\
Paraíba e Ceará.

\textsc{refrão}

O Jaguaribe inundou\\
A cidade do Iguatu\\
E Sobral foi alagado\\
Pelo rio Aracajú,\\
O mesmo estrago fizeram\\
Salgado e Banabuiú.

\textsc{refrão}

Ceará martirizado\\
Eu tenho pena de ti,\\
Limoeiro, Itaiçaba,\\
Quixeré e Aracati,\\
Faz pena ouvir o lamento\\
Dos flagelados dali.

\textsc{refrão}

Meus senhores governantes\\
Da nossa grande Nação\\
O flagelado das enchentes\\
É de cortar o coração,\\
Muitas famílias vivendo\\
Sem lar, sem roupa e sem pão.

\textsc{refrão}
\end{verse}

\chapter{Tereza Potó}

\begin{verse}
Quem já nasceu azarado\\
Veve de mal a pió\\
O seu mamoêro é macho,\\
Não cai preá no quixó,\\
Mesmo ele andando no prano\\
Cai dentro do brocotó\\
E ninguém tira a mandinga\\
Ninguém tira o catimbó,\\
Nem reza de feiticêro\\
Lá das matas de Copo,\\
De tudo que eu tou dizendo\\
Dou a prova, veja só.

Num dia de sexta-fêra\\
Eu vi Tereza Potó\\
Com seu vestido de chita\\
Infeitado de filó,\\
Os beiço bem tinturado\\
E o rosto branco de pó,\\
Doidinha por namorado\\
Chorando de fazê dó\\
Já se achava impaciente\\
Afobada e zuruó,\\
Às cinco e meia da tarde\\
Se encontrou com Siridó,\\
Siridó era tão feio,\\
Tão feio como ele só\\
O cabelo era assanhado\\
Parecia um sanharó\\
Um braço, todo injambrado\\
Todo cheio de nopró\\
Os óio da cô de brasa\\
E um lubim no gogó,\\
Mas Tereza se agradou\\
E começou o chodó\\
Preguntou: você me qué?\\
E ele respondeu ó ó,\\
Eu quero você todinha\\
Da cabeça ao mocotó\\
E dali saíro os dois\\
Para dançá num forró\\
Que havia naquela noite\\
Na casa de Zé Tingó\\
Na latada de palmêra\\
Da terra subia o pó\\
Tereza Potó já tava\\
Moiadinha de suó\\
Mas assim mesmo dizia:\\
Não tem coisa mais mió\\
Depois compraro passage\\
No carro de Zé Jacó\\
E saíro do Iguatu\\
Para casá no Icó,\\
Porém lá naqueles meio\\
Perto do açude do Oró\\
O carro faltou o freio\\
E desceu num cafundó,\\
Por riba de pau e pedra\\
No maió torototó,\\
Morreu logo o motorista\\
Chamado José Jacó,\\
Morreu Tereza Potó\\
E morreu o Siridó,\\
O namoro deles dois\\
Se acabou tudo no ó.
\end{verse}

\chapter{A enfermeira do pobre}

\begin{verse}
Não consiste na riqueza\\
Na posição ou grandeza\\
A maior felicidade,\\
É de Deus que ela nos vem,\\
É muito feliz quem tem\\
O dom da fraternidade.

Quem tem prazer em servir\\
A sua fé no porvir\\
Cada vez mais evolui,\\
Faço referência agora\\
Sobre uma simples senhora\\
Que o meu Assaré possui.

Nós todos a conhecemos\\
Testemunhamos e vemos\\
O seu sentimento nobre,\\
Toda casa de família\\
Conhece Maria Ermília\\
A enfermeira do pobre.

Com muita abnegação\\
Dentro de sua missão\\
Maria Ermília oferece\\
A caridade, o amor\\
E a beleza interior\\
Ao doente que padece.

No caminho da virtude\\
Nunca mudou de atitude\\
Quer de noite quer de dia,\\
Qualquer pobre domicílio\\
Conta com o grande auxílio\\
Da enfermaria Maria.

Vai enfermeira de Deus\\
Com estes cuidados teus\\
Praticando a penitência,\\
Que dentro deste teu grêmio\\
Tu receberá o prêmio\\
Da Divina Providência.

Não há quem saiba na terra\\
A caridade que encerra\\
Teu coração de mulher\\
Pelo valor que tu tens\\
Recebas os parabéns\\
De quem te preza e te quer.
\end{verse}

\chapter{Quadras}

\begin{verse}
Toda Natureza cheia\\
Com os possuídos seus\\
É um grãozinho de areia\\
Na palma da mão de Deus.

Morena, você me deixe\\
Viver quieto e sossegado,\\
Por causa da isca o peixe\\
Vai preso e sacrificado.

Você que passa grã-fina\\
Lavando o ninho dos ninhos\\
Tenha cuidado menina\\
Com estes dois passarinhos.

Correntes de grossas águas,\\
Que tudo levas à toa,\\
Por que não levas as mágoa\\
Da mágoa que me magoa?

Naquele belo ambiente\\
Ninguém pode estar seguro,\\
Muitos morrem no presente\\
Lá na praia do Futuro.

A mulher também tem isca\\
Da forma que o anzol tem,\\
Um belisca outro belisca\\
Até que o mais tolo vem.

A rosa do meu ciúme\\
Me negou os seus carinhos,\\
Levou beleza e perfume\\
Só me deixou os espinhos.

Quando o amor não é fiel\\
É o instinto quem domina,\\
Depois da lua-de-mel\\
Vem o fel da quinaquina.

Como é que amor verdadeiro\\
Pode haver entre nós dois?\\
Se tudo teu é primeiro\\
E o que é meu sempre depois?

Cada qual na sua lida\\
Trabalha constantemente,\\
Anda a gente atrás da vida\\
E anda a morte atrás da gente.

Pra gente saber que é besta\\
Precisa estudar besteira,\\
terça, quarta, quinta, sexta,\\
Sábado e segunda-feira.

A língua é tal qual a faca,\\
É como outra arma qualquer,\\
Pois a mesma só ataca\\
A quem seu dono quiser.

Ao amor nasci propenso,\\
Só nele tenho pensado\\
E tanto pensei que penso\\
Que sele fui dispensado.

Desde o dia em que partiste\\
Deste sertão para rua\\
Meu coração ficou triste\\
Tal qual a noite sem lua.

Com três meninas, meu fado\\
É feliz, graças a Deus,\\
Tu, que vives ao meu lado\\
E as duas dos olhos teus.

Desde o dia de tristeza\\
Quando te deixei na cova,\\
Foi sepultada a beleza\\
Dos versos da minha trova.

Que alguém morre por alguém,\\
Sempre ouvi alguém dizer\\
Porém quando a morte vem\\
Não há quem queira morrer.

Quando eu te vejo, Maria,\\
Com tua pele de lixa,\\
Me lembro quando eu vivia\\
Atirando em lagartixa.

Mesmo a mãe dando gemido\\
Desfalecida de fome\\
O seu filhinho querido\\
Será o primeiro que come.

Sempre satisfeito estou\\
Com os sofrimentos meus,\\
Cada passada que dou\\
Fico mais perto de Deus.

Em um jardim eu entrei\\
Vi rosas de várias cores\\
Porém lá não encontrei\\
A rosa de meus amores.

A lua foi a ilusão\\
Do poeta sonhador,\\
Porém hoje está na mão\\
Do astronauta voador.
\end{verse}

\chapter{Ao padre Miracapilo}

\begin{verse}
Porque vivias trabalhando ao lado\\
Dos operários que padecem tanto,\\
Porque enxugavas do mendigo o pranto,\\
Pelos ricaços foste censurado.

Por seres justo, foste injustiçado,\\
Porém teu gesto, fraternal e santo,\\
Qual nota linda de um saudoso canto\\
Eternamente ficará gravado.

Grande saudade no país deixaste\\
Entre os humildes onde doutrinaste\\
Apresentando um sentimento nobre,

Mesmo vivendo do Brasil distante,\\
Deus estará contigo a todo instante\\
Miracapilo, protetor dos pobres.
\end{verse}

\chapter{No cemitério}

\begin{verse}
Querendo um dia ver algum mistério\\
Das coisas tristes que este mundo tem,\\
Impressionado fui ao cemitério,\\
Onde podemos mais pensar no Além.

No Campo Santo, no sombrio império,\\
De olhar piedoso a recordar alguém\\
Entre soluços em um tom funério\\
Muitos choraram e eu chorei também.

Mas nesta vida sempre há um sujeito\\
Que da miséria tira o seu proveito.\\
Vi no semblante do senhor coveiro,

A indiferença ao doloroso assunto,\\
Interessado a receber defunto\\
Fazia cova pra ganhar dinheiro.
\end{verse}

\chapter{Prezado amigo}

\begin{verse}
Prezado amigo, uma expressão comum\\
Que muitos dizem, como eu também digo,\\
Porém no mundo, verdadeiro amigo\\
É feliz quem possui ao menos um.

Se alguém convida procurando algum:\\
Prezado amigo, venha ao meu abrigo,\\
Venha amanhã para almoçar comigo,\\
Sem atender não ficará nenhum.

Porém, dizendo: meu amigo eu morro\\
Venha depressa, venha em meu socorro,\\
São negativas todas as respostas,

Aquele mesmo que gostou do prato\\
Mostrando agora o seu papel de ingrato\\
Indiferente vai lhe dando as costas.
\end{verse}

\chapter{Chico Forte}

\begin{verse}
O Chico Forte era um sujeito duro\\
Cruel, malvado que matava gente,\\
Este assaltante, monstro delinqüente\\
Deixava o claro para andar no escuro.

Roubava casas e pulava muro,\\
Vil, assassino, traidor, valente,\\
Nonguém ousava lhe tomar a frente,\\
Era um assombro o seu viver impuro.

Mas, disse um dia o cemitério à morte:\\
Você procure o cabra Chico Forte\\
Se este eu abraço, não desprezo aquele,\\

Se eu quero os filhos, não desprezo os pais,\\
Na minha mesa todos são iguais\\
Traga o Francisco que eu preciso dele.
\end{verse}

\chapter{Gratidão}

\begin{verse}
Não podemos negar a proteção\\
Ao coitado infeliz, pobre mendigo,\\
Quando às vezes recorre ao nosso abrigo\\
Procurando comer do nosso pão.

Precisamos ao mesmo dar a mão\\
Relembrando este dito muito antigo\\
Todo aquele que sabe ser amigo\\
Sempre encontra um papel de gratidão.

Tu, mangueira, delgada e tão bonita,\\
Recebendo em teu tronco parasita\\
Com certeza do mesmo não gostaste,

Porém hoje, pagando a substância,\\
Com as flores repletas de fragrância\\
Ele vai perfumando a tua haste.
\end{verse}

\chapter{O padre Henrique e o dragão da maldade}

\begin{verse}
Sou um poeta dos matos\\
Vivo afastado dos meios\\
Minha rude lira canta\\
Casos bonitos e feios,\\
Eu canto meus sentimentos\\
E os sentimentos alheios.

Sou caboclo nordestino,\\
Tenho mão calosa e grossa,\\
A minha vida tem sido\\
Da choupana para a roça,\\
Sou amigo da família\\
Da mais humilde palhoça.

Canto da mata frondosa\\
A sua imensa beleza,\\
Onde vemos os sinais\\
De pincel da natureza,\\
E quando é preciso eu canto\\
A mágoa, a dor e a tristeza.

Canto a noite de São João\\
Com toda sua alegria,\\
Sua latada de folha\\
Repleta de fantasia\\
E canto o pobre que chora\\
Pelo pão de cada dia.

Canto o crepúsculo da tarde\\
E o clarão da linda aurora,\\
Canto aquilo que me alegra\\
E aquilo que me apavora\\
E canto os injustiçados\\
Que vagam de mundo afora.

E por falar de injustiaça\\
Traidora da boa sorte\\
Eu conto ao leitor um fato\\
De uma bárbara morte\\
Que se deu em Pernambuco\\
Famoso Leão do Norte.

Primeiro peço a Jesus\\
Uma santa inspiração\\
Para escrever estes versos\\
Sem me afastar da razão\\
Contando uma triste cena\\
Qua faz cortar coração.

Falar contra as injustiças\\
Foi sempre um dever sagrado\\
Este exemplo precioso\\
Cristo deixou registrado,\\
Por ser reto e justiceiro\\
Foi no madeiro cravado.

Por defender os humildes\\
Sofreu as mais cruéis dores\\
E ainda hoje nós vemos\\
Muitos dos seus seguidores\\
Morrerem barbaramente\\
Pelas mão dos malfeitores.

Vou contar neste folheto\\
Com amor e piedade,\\
Cujo título encerra\\
A mais penosa verdade:\\
O Padre Antônio Henrique\\
E o Dragão da Maldade.

O Padre Antônio Henrique\\
Muito jovem e inteligente\\
A 27 de maio\\
Foi morto barbaramente\\
Do ano 69\\
Da nossa era presente.

Padre Henrique tinha apenas\\
29 anos de idade,\\
Dedicou sua vida aos jovens\\
Pregando a santa verdade\\
Admirava a quem visse\\
A sua fraternidade.

Tinha três anos de padre:\\
Depois que ele se ordenou\\
Pregava a mesma missão\\
Que Jesus cristo pregou\\
E foi por esse motivo\\
Que o Dragão lhe assassinou.

Surgiu contra Padre Henrique\\
Um fúria desmedida\\
Ameaçando a igreja\\
Porque estava decidida\\
Conscientizando os jovens\\
Sobre os problemas da vida.

Naquele tempo o Recife,\\
Grande e bonita cidade,\\
Se achava contaminada\\
Pelo Dragão da Maldade,\\
A rancorosa mentira\\
Lutando contra a verdade.

Nesse clima de tristeza\\
Os dias iam passando,\\
Porém nosso Padre Henrique\\
Sempre a verdade explicando\\
E ameaças contra a igreja\\
Chegavam de vez em quando.

Por causa de seu trabalho\\
Que só o que é bom almeja\\
O espírito da maldade,\\
Que tudo estraga e fareja,\\
Fez tristes acusações\\
Contra D. Helder e a igreja.

Com o fim de atemorizar\\
O apóstolo do bem\\
Chegavam cartas anônimas\\
Com insulto e com desdém,\\
Porém quem confia em Deus\\
Jamais temeu a ninguém.

Anônimos telefonemas\\
Com assuntos de terror\\
Chegavam, constantemente,\\
Cheios de ódio e rancor\\
Contra Padre Henrique, o amigo\\
Da paz, da fé e do amor.

Os ditos telefonemas\\
Faziam declaração\\
De matar 30 pessoas\\
Sem ter nem compaixão\\
Que tivessem com D.\,Helder\\
Amizade ou ligação.

Veja bem leitor amigo\\
Quanto é triste esta verdade:\\
O que defende os humildes\\
Mostrando a luz da verdade\\
Vai depressa perseguido\\
Pelo Dagrão da Maldade.

Mas o ministro de Deus\\
Possui o santo dever\\
De estar di lado dos fracos\\
Sua causa a defender,\\
Não é só salvar a alma\\
Também precisa comer.

Os poderosos não devem\\
Oprimir de mais a mais,\\
A justiça é para todos,\\
Vamos lutar pela paz.\\
Ante os direitos humanos\\
Todos nós somos iguais.

A igreja de Jesus\\
Nos oferece orações\\
Mas também precisa dar\\
Aos humildes instruções,\\
Para que possam fazer\\
Suas reivindicações.

Veja meu caro leitor\\
A verdade o quanto é:\\
O Padre Henrique ensinava\\
Cheio de esperança e fé,\\
Aquelas mesmas verdades\\
De Jesus de Narazé.

E foi por esse motivo\\
Que surgiu a reação,\\
Foi o instinto infernal\\
Com a fúria do Dragão,\\
Qua matou o nosso guia\\
De maior estimação.

A 27 de maio,\\
O santo mês de Maria,\\
No ano 69\\
A Natureza gemia\\
Por ver o corpo de um padre\\
Morto sobre a terra fria.

Naquele dia de luto\\
Tudo se achava mudado,\\
Parece até que o Recife\\
Se encontrava envergonhado\\
Por ver que um triste segredo\\
Estava a ser revelado.

Rádio, \textsc{tv} e jornais,\\
Nada ali noticiaram\\
Porque as autoridades\\
Estas verdades calaram\\
E o Padre Antônio Henrique\\
Morto no mato encontraram.

Estava o corpo do Padre\\
De faca e bala furado,\\
Também mostrava ter sido\\
Pelo pescoço amarrado\\
Provando que antes da morte\\
Foi bastante judiado.

No mato estava seu corpo\\
Em situação precária,\\
Na região do lugar\\
Cidade Universitária,\\
Foi morto barbaramente\\
Pela fera singuinária.

Por aquele mesmo tempo\\
Muitos atos agravantes\\
Surgiram lá no Recife\\
Contra os jovens estudantes\\
Que devem ser no futuro\\
Da pátria representantes.


Invadiram o Diretório\\
Estudantil, um recinto\\
Universidade Católica\\
De Pernambuco e, não minto,\\
Foi atingido por bala\\
O estudante Cândido Pinto.

Foi seqüestrado e foi preso\\
E estudante Cajá\\
O encerramento no cárcere\\
Passou um ano por lá\\
Meu Deus! A democracia\\
Deste país onde está?

Cajá, o dito estudante,\\
Pessoa boa e benquista,\\
Pra viver com os pequeninos\\
Deixou de ser carreirista\\
E por isto o mesmo foi\\
Tachado de comunista.

Será que ser comunista\\
É dar ao fraco instrução,\\
Defendendo os seus direitos\\
Dentro da justa razão,\\
Tirando a pobreza ingênua\\
Das trevas da opressão.

Será que ser comunista\\
É mostrar certeiros planos\\
Para que o povo não viva\\
Envolvido nos enganos\\
E possam se defender\\
Do jugo dos desumanos.

Será que ser comunista\\
É saber sentir as dores\\
Da classe dos operários,\\
Também dos agricultores\\
Procurando amenizar\\
Horrores e mais horrores.

Tudo isto, leitor, é truque\\
De gente sem coração\\
Que com o fim de trazer\\
Os pobres na sujeição,\\
Da palavra comunismo\\
Inventa um bicho-papão.

Porém a igreja dos pobres,\\
Fiel se comprometeu,\\
Cada um tem o direito\\
De defender o que é seu\\
Para quem segue Jesus\\
Nunca falta um Cirineu.

Mostrando a mesma verdade\\
De Jesus da Palestina\\
O movimento se estende\\
Contra a opressão que domina\\
Sobre os nossos irmão pobres\\
De toda América Latina.

Quando Jesus Cristo andou\\
Pregando sua missão,\\
Falou sobre a igualdade,\\
Fraternidade e união,\\
Não pode haver injustiça\\
Na sua religião.

Por este motivo a igreja\\
Nova posição tomou\\
Dentro da América Latina\\
A coisa agora mudou,\\
O bom cristão sempre faz\\
Aquilo que Deus mandou.

É justo por excelência\\
O autor da criação,\\
Devemos amar a Deus\\
Por direito e gratidão,\\
Cada um tem o dever\\
De defender seu irmão.

Por isto, os nossos pastores,\\
Trilham penosas estradas\\
Observando de Cristo\\
Suas palavras sagradas,\\
Trabalhando em benefício\\
Das classes desamparadas.

Declarando dessa forma\\
A santa luz da verdade\\
Para que haja entre todos\\
Amor e fraternidade\\
E boa organização\\
Dentro da sociedade.

Pois vemos o estudante\\
Pelo poder perseguido,\\
Operário, agricultor,\\
O nosso índio querido\\
E o negro? Pobre coitado!\\
É o mais desprotegido.

Vendo a medonha opressão\\
Me vem à mente o que disse\\
O grande bardo baiano\\
\quad(Castro Alves)\\
O poeta dos escravos\\
Apelando ao soberano.

Senhor Deus dos desgraçados\\
Dizei-me vós, Senhor Deus,\\
Se é mentira, se é verdade\\
Tenho horror perante os céus\ldots{}

Meu caro leitor desculpe\\
Esta falta que cometo\\
Me desviando do assunto\\
Da história que lhe remeto,\\
O caso do Padre Henrique,\\
Motivo deste folheto.

Se me desviei do ritmo,\\
Não queira se aborrecer,\\
É porque as outras coisas\\
Eu queria lhe dizer,\\
Pois tudo que ficou dito\\
Você precisa saber.

Mas, agora lhe prometo\\
Com bastante exatidão\\
Terminar para o amigo\\
Esta triste narração\\
Contando tudo direito\\
Sem sair da oração.

Eu disse ao caro leitor\\
Que foi no mato encontrado\\
Nosso Padre Antônio Henrique\\
De faca e bala furado,\\
Agora conto direito\\
Como ele foi sepultado.

Na igreja do Espinheiro\\
Foi o povo aglomerado\\
E ao cemitério da Várzea\\
Foi pelos fiéis levado\\
O corpo do Padre Henrique\\
Que morreu martirizado.

Enquanto o cortejo fúnebre\\
Ia levando o caixão\\
Este estribrilho se ouvia\\
Pela voz da multidão\\
``Prova de amor maior não há\\
Que doar a vida pelo irmão.''

Treze quilometros de pé\\
Levaram o corpo seu\\
Lamentando lacrimosos\\
O caso que aconteceu,\\
A morte de um jovem padre\\
Que pelos jovens morreu.

Ia naquele caixão\\
Que grande exemplo deixou\\
Em defesa aos oprimidos\\
A sua vida entregou\\
E foi receber no Céu\\
O que na terra ganhou.

O corpo ia acompanhado\\
Em forma de procissão,\\
Com as vozes dos fiéis\\
Ecoando na amplidão:\\
``Prova de amor maior não há\\
Que doar a vida pelo irmão.''

A vida do Padre Henrique\\
Vamos guardar na memória,\\
Ele morreu pelo povo,\\
É bonita a sua história\\
E foi receber no Céu\\
Sua coroa de glória.

Pensando no triste caso\\
Entristeço e me comovo,\\
O que muitos já disseram,\\
Eu disse e digo de novo,\\
O Padre Henrique é um mártir\\
Que morreu pelo seu povo.

Prezado amigo leitor\\
Esta dor é minha e sua\\
De ver morrer Padre Henrique\\
De morte tirana e crua,\\
Porém a igreja dos pobres\\
Sua luta continua.

Quem da igreja do Espinheiro\\
Santa casa de oração\\
As cemitério da Várzea,\\
Palmilhar aquele chão\\
A 27 de maio,\\
Sentirá recordação.

Do corpo de um Padre jovem\\
Conduzido em um caixão,\\
Me parece ouvir uns versos\\
Com sonora entoação:\\
``Prova de amor maior não há\\
Que doar a vida pelo irmão.''
\end{verse}

\chapter{Desilusão}

\begin{verse}
Como a folha no vento pelo espaço\\
Eu sinto o coração aqui no peito,\\
De ilusão e de sonho já desfeito,\\
A bater e a pulsar com embaraço.

Se é de dia, vou indo passo a passo\\
Se é de noite, me estendo sobre o leito,\\
Para o mal incurável não há jeito,\\
É sem cura que eu vejo o meu fracasso.

De Parnaso não vejo o belo monte,\\
Minha estrela brilhante no horizonte\\
Me negou o seu raio de esperança,

Tudo triste em meu ser se manifesta,\\
Nessa vida cansada só me resta\\
As saudades do tempo de criança.
\end{verse}

\chapter{Reforma agrária}

\begin{verse}
Pobre agregado, força de gigante,\\
Escuta amigo o que te digo agora,\\
Depois da treva vem a linda aurora\\
E a tua estrela surgirá brilhante.

Pensando em ti eu vivo a todo instante,\\
Minha alma triste desolada chora\\
Quando te vejo mundo afora\\
Vagando incerto qual judeu errante.

Para saíres da fatal fadiga,\\
Do horrível jugo que cruel te obriga\\
A padecer situação precária

Lutai altivo, corajoso e esperto\\
Pois só verás o teu país liberto\\
Se conseguires a reforma agrária.
\end{verse}

\chapter{Mote}

\begin{verse}
Coronel, tenha cuidado,\\
Que o comunismo aí vem.
\end{verse}

\section*{Glosas}

\begin{verse}
A nossa crise fatal,\\
Cada dia mais aumenta\\
O pobre já não agüenta\\
Esta opressão atual,\\
O peso deste costal\\
É carga pra mais de um trem,\\
Isso assim não nos convém,\\
O povo está revoltado,\\
Coronel, tenha cuidado,\\
Que o comunismo aí vem.

Fale com os seus patrões\\
Para que, por suas vezes,\\
Protejam os camponeses\\
Que vivem pelos sertões\\
Sem terra, sem instruções\\
E sem auxílio de alguém;\\
Eles são filhos também\\
Deste Brasil adorado:\\
Coronel, tenha cuidado,\\
Que o comunismo aí vem.

É coisa bem necessária\\
Minorar o sofrimento,\\
Ao pobre dando instrumento\\
Concernente à vida agrária;\\
A situação precária\\
Dos camponêses vai além,\\
Há matuto que não tem\\
Com que tratar do roçado:\\
Coronel, tenha cuidado,\\
Que o comunismo aí vem.

Dentro da classe cativa\\
Chora o rude camponês,\\
Sob jugo do burguês,\\
Que fala com voz altiva,\\
Querendo que o povo viva\\
Sem proteção de ninguém,\\
Por séculos sem fim, amém,\\
Para sempre abandonado:\\
Coronel, tenha cuidado,\\
Que o comunismo aí vem.

O pobre não pode mais\\
Se expor a tantas fadigas,\\
Para aumentar as barrigas\\
Dos chupões nacionais.\\
Quem vai por estes canais\\
Morre e não ganha um vintém,\\
Perde noventa por cem,\\
Pois vive sempre explorado:\\
Coronel, tenha cuidado,\\
Que o comunismo aí vem.

O sertanejo sem luz\\
De letra e civilidade,\\
Cheio de necessidade,\\
Vergado ao peso da cruz,\\
Tão pobre como Jesus,\\
Quando nasceu em Belém,\\
Não tem mais fé no porém\\
Do povo civilizado:\\
Coronel, tenha cuidado,\\
Que o comunismo aí vem.
\end{verse}

\chapter{Saudação ao Juazeiro do Norte}

\begin{verse}
Mesmo sem eu ter estudo,\\
Sem ter da escola o bafejo,\\
Juazeiro, eu te saúdo\\
Com o meu verso sertanejo.\\
Cidade de grande sorte,\\
De Juazeiro do Norte,\\
Tens a denominação,\\
Mas teu nome verdadeiro\\
Será sempre Juazeiro\\
Do Padre Cícero Romão.

O Padre Cícero Romão,\\
Que, por vocação celeste,\\
Foi, com direito e razão,\\
O apóstolo do Nordeste.\\
Foi ele o teu protetor,\\
Trabalhou com grande amor,\\
Lutando sempre de pé\\
Quando vigário daqui,\\
Ele semeou em ti\\
A sementeira da fé.

E com milagre estupendo\\
A sementeira nasceu,\\
Foi crescendo, foi crescendo,\\
Muito ao longe se estendeu.\\
Com a virtude regada,\\
Foi mais tarde transformada\\
Em árvore frondosa e rica.\\
E com a luz medianeira\\
Inda hoje a sementeira\\
Cresce, flora e frutifica.

Juazeiro, Juazeiro,\\
Jamais a adversidade\\
Extinguirá o luzeiro\\
Da tua comunidade.\\
Morreu o teu protetor,\\
Porém a crença e o amor\\
Vive em cada coração.\\
E é com razão que me expresso:\\
Tu deves o teu progresso\\
Ao Padre Cícero Romão.

Aquele ministro amado,\\
Que tanto favor nos fez,\\
Conselheiro consagrado\\
E o doutor do camponês.\\
Contradizer não podemos\\
E jamais descobriremos\\
O prodígio que ele tinha:\\
Segundo a popular crença,\\
Curava qualquer doença,\\
Com malva branca e jarrinha.

Juazeiro, Juazeiro,\\
Tua vida e tua história\\
Para o teu povo romeiro\\
Merece um padrão de glória.\\
De alegria tu palpitas,\\
Ao receber as vistas\\
De longe, de muito além.\\
Grande glória tu tiveste!\\
Do nosso caro Nordeste\\
Tu és a Jerusalém.

Sempre me lembro e relembro,\\
Não hei de me deslembrar:\\
O dia 2 de novembro,\\
Tua festa espetacular,\\
Pois vêm de muitos Estados\\
Os carros superlotados\\
Conduzindo os passageiros\\
E jamais será feliz\\
Aquele que contradiz\\
A devoção dos romeiros.

No lugar onde se achar\\
Um fervoroso romeiro,\\
Ai daquele que falar,\\
Contra ou mal do Juazeiro.\\
Pois entre os devotos crentes,\\
Velhos, moços e inocentes,\\
A piedade é comum,\\
Porque o santo reverendo\\
Se encontra ainda vivendo\\
No peito de cada um.

Tu, Juazeiro, és abrigo\\
Do amor e da piedade.\\
Eu te louvo e te bendigo\\
Por tua felicidade,\\
Me sinto bem quando vejo\\
Que tu és do sertanejo\\
A cidade predileta.\\
Por tudo quanto tu tens\\
Recebe estes parabéns\\
Do coração de um poeta.
\end{verse}

\chapter{Um cearense desterrado}

\begin{verse}
Fiz uma coisa no mundo\\
Que hoje arrependido me acho\\
Vivi com um vagabundo\\
Sempre andando arriba e abaixo\\
Remexi o Sul e o Norte\\
Andei a pé e de transporte\\
De todo jeito eu andei\\
Sem tirá do coração\\
O pedacinho de chão\\
Onde eu nasci e me criei.

Sou fio do Ceará\\
Minha terra é bem distante\\
E aquele que nasce lá\\
É como o judeu errante,\\
Parece que uma formiga\\
Friviando nele obriga\\
A dexá tudo que é seu\\
Pra vagá na terra estranha\\
Tecendo que nem aranha\\
Caçando o que não perdeu.

Quando eu tinha dezoito ano\\
Me larguei de mundo afora\\
Assim à moda cigana\\
Que onde chega não demora,\\
E hoje sem vê minha gente,\\
Véio, cansado e doente,\\
Me doendo as carne e os osso\\
Tou prisionêro daqui\\
Como tatu no giqui,\\
Quero vortá mas não posso.

É muito triste o meu pranto\\
E dura a minha sentença,\\
Não há dô pra doê tanto\\
Como dói a dô da osença\\
O meu maió desengano\\
É pensá nos meus seis ano\\
Quando eu vivia a brincá\\
Cantando pelo terrêro:\\
``Meu limão meu limoêro\\
Meu pé de jacarandá''.

Com os fio do meu tio\\
Era boa a brincadêra\\
De pião, de currupio,\\
De bodoque e baladêra,\\
Pensando na minha infança\\
Sinto o espinho da lembrança\\
Furando meu coração,\\
Pensando naquela idade\\
O meu armoço é sodade\\
E a janta recordação.

Hoje vejo e tou ciente\\
Que a gente só goza a vida\\
Vivendo como os da gente\\
Na mesma terra querida,\\
Bem que o meu avô dizia\\
Com munta filosofia\\
Este dito populá\\
Que a gente não contrareia:\\
``O boi pelas terra aleia\\
Até as vaca lhe dá.''

Eu vejo que munto peco,\\
Dêxei meu torrão querido\\
Pra vivê nos inteleco\\
Deste Brasi desmedido,\\
Quage na extrema da estranja\\
Onde o pobre não arranja\\
Um jeito para vortá,\\
Minha sentença é de réu,\\
Sei que daqui vou pro céu\\
Sem vê mais meu Ceará.

Meus querido conterrano\\
Iscute o que eu tou dizendo\\
Pra não sofrê o desengano\\
Do jeito que eu tou sofrendo,\\
Tope fome, peste e guerra\\
Mas não dêxe a sua terra,\\
Tenha corage, resista,\\
Não quêra mudá o destino,\\
O Sul é para o sulino\\
E o Norte é para o nortista.

Nordestino, meu amigo,\\
Tope fome, guerra e peste\\
Mas não dêxe o seu abrigo\\
Não saia do seu Nordeste,\\
Lute pelos seus dereito\\
Até incrontá um jeito\\
De ninguém lhe escravizá\\
Estas terra por aí\\
Pertence ao mesmo Brasi\\
Descoberto por Cabrá.

Eu vejo que munto erra\\
Quem não óia com amô\\
Para sua prope terra\\
Do seu pai e do seu avô,\\
Nordestino, nordestino,\\
Não quêra mudá o destino,\\
Não despreze o seu Estado\\
Por aquilo que é aleio,\\
Guarde na mente os conseio\\
Deste pobre desterrado.
\end{verse}

\chapter{Carta à doutora Henriqueta Galeno em 1929}

\begin{verse}
Incelentíssima dotoura,\\
Peço perdão à senhora\\
Desta carta lhe enviá;\\
Mas leia os verso rastêro\\
De um cabôco violêro\\
Do sertão do Ceará.

Sou o cantadô Patativa,\\
Que trôxe aquela missiva,\\
Aquele papé escrito\\
E cantou no seu salão,\\
Com a recomendação\\
De Zé Carvaio de Brito.

Tou lembrando e não me engano,\\
Faz hoje vinte e dois ano\\
Que do Pará eu vortei,\\
E aí, nesse salão,\\
Com grande satisfação\\
Na viola provisei.

Mode prová quem sou eu,\\
Já dixe o que aconteceu\\
Com a verdade compreta.\\
Agora vou lhe contá\\
Quá é o fim principá\\
Desta carta de poeta.

É pedi um objeto,\\
O tesouro mais dileto\\
Que a gente pode estimá,\\
Jóia de valô prefeito,\\
Que vale mais que os infeito\\
Da corôa imperiá.

Esse objeto incelente\\
Que veve na minha mente,\\
E um só momento não sai,\\
É o volume precioso,\\
Do poeta primoroso,\\
Juvená, o seu papai.

É aquele volume de ôro,\\
Aquele rico tesôro\\
Do maió dos trovadô,\\
É o livro de um bom poeta,\\
Que cantou as onda inquieta\\
E a vida do pescadô.

É o livro do poeta honrado\\
Que tem o nome gravado\\
Na história do Ceará,\\
E foi quem cantou primêro\\
Neste pais brasilêro\\
As cantiga populá.

Já percurei fortemente,\\
E quando fiquei ciente\\
Que não podia obtê\\
Vim me valê da dotôra,\\
Porque somente a senhora\\
Pode me sastisfazê.
\end{verse}

\chapter{Um candidato político na casa de um caçador}

\begin{verse}
Seu dotô vem de viage\\
Deve tá munto enfadado,\\
O dia já tá findando,\\
Seu cavalo tá cansado\\
Vou botá ele na roça\\
E esta casa aqui é nossa\\
Pra quem quisé se arranchá,\\
Conceição minha muié\\
Já vem trazendo o café,\\
Depois nós vamo jantá.

Mas primêro eu lhe pergunto,\\
O sinhô come tatu?\\
Lapichó, viado, peba,\\
A juriti, o jacu,\\
Asa branca e zabelê?\\
Se come, pode dizê,\\
Não vá se acanhá, dotô,\\
Tudo isto eu tenho guardado,\\
O sinhô tá hospedado\\
Na casa dum caçadô.

O sinhô disse que come?\\
Então já tenho certeza\\
Que o moço vai passar bem,\\
A janta já tá na mesa,\\
Se assente nesta cadêra\\
Que fica na cabicêra\\
E não quêra se acanhá,\\
Que eu não me acanho também\\
Com esses home que vem\\
Da banda da capitá.

Se o dotô fala bonito,\\
Pruquê na escola aprendeu,\\
Mas sou gente como ele\\
E ele é gente como eu\\
A beleza de linguagem,\\
Tudo é bestêra, é bobage,\\
Eu só na verdade creio,\\
Fica uma coisa isquisita\\
Tanta palavra bonita\\
Com mentira pelo meio.

Tô vendo que o sinhô gosta,\\
Já inrolou mais dum prato\\
Parece que vamicê\\
Também já morou no mato;\\
Se achou bom pode comê\\
Que pra mim é um prazê\\
E esta nossa moradia\\
De caça é bem previnida\\
E pro lado de comida\\
Ninguém faz inconomia.

Agora que nós jantemo,\\
Inchemo nossas barriga\\
Vou fazê uma pregunta,\\
Quero que o dotô me diga:\\
Que é que aqui anda fazendo,\\
Por este sertão sofrendo,\\
Parando aqui e acolá,\\
Sem sabê caminho certo\\
Será porque já tá perto\\
Da campanha eleitorá?

Garanto como acertei,\\
Sei que tô falando exato,\\
Este seu jeito parece\\
Com jeito de candidato,\\
Se vem com este sentido\\
O seu trabaio é perdido,\\
Eu não gosto de inleição\\
E pra mió lhe dizê\\
Aqui ninguém sabe lê,\\
Nem eu e nem Conceição.

E se eu tivesse leitura,\\
Vou lhe dizê e fica dito\\
Votava mas era em branco\\
Pra garanti o meu tito,\\
Eu presto atenção e noto\\
Que aqui os que dero voto\\
Todo trabaio perdeu,\\
Aqui para o nosso lado\\
O povo que tem votado\\
Nunca favô recebeu.

O meu avô e o meu pai\\
Votava em toda inleição,\\
Morrero de trabaiá\\
Sem recebê proteção;\\
Isto sempre foi assim,\\
Sempre de ruim a mais ruim\\
E onte uvi arguém contá\\
Que continua a caipora\\
E que o Brasi só miora\\
Se este rejume mudá.

Eu não entendo de nada,\\
Meu mundo é bem deferente\\
E nunca quis e nem quero\\
Relação com certa gente,\\
Vou levando a vida minha\\
Prantando minha rocinha\\
E dando minhas caçada,\\
É mió sê caçadô\\
do que sê um inleitô\\
pra votá sem ganhá nada.

Agora eu mudo de rumo\\
Vou de outro assunto tratá,\\
Se eu não entendo política,\\
Não vou meu tempo gastá,\\
Vou é falá de caçada\\
Que o dotô não sabe nada\\
Das coisa do meu sertão,\\
Vou falá de ribuliço\\
De marmota e sacrifiço\\
Nas noite de assombração.

Ói seu moço pode crê\\
Como eu que vivo a caçá\\
Vejo coisas nessas mata\\
Do cabelo arrupiá,\\
Já uvi dentro da brenha\\
Uma voz gritando, venha!\\
E outra respondê, já vou!\\
Tudo aquilo é a caipora\\
Que dentro das mata mora\\
Presseguindo os caçadô.

Que é que vai vê um vaquêro\\
Noite de escuro nos mato,\\
Isfolando memelêro\\
No maió espaiafato?\\
Correndo disisperado,\\
Como quem vai animado\\
Para uma rês derrubá?\\
Lhe juro em nome de Cristo\\
As coisa que eu tenho visto\\
É do sibito piá.

Tem delas que até parece\\
Com a pintura de Diogo,\\
Se vê um pau ramaiudo\\
Se ardendo e pegando fôgo,\\
A coisa não é brinquedo,\\
É mesmo de fazê medo\\
Esta grande assombração,\\
O vento forte assoprando\\
E a lavareda roncando\\
Jogando brasa no chão.

Com isto o caçadô vorta\\
Desenganado pra casa\\
E no outro dia vai lá\\
Pra vê as cinza e as brasa,\\
Mas brasa e cinza não tem,\\
Tudo lá tá tudo bem\\
Com a forma naturá,\\
Ele repara de perto\\
E vê que tá tudo certo,\\
Sem tê de fogo um siná.

Tudo aquilo é a caipora\\
Com a sua arrumação,\\
Não há quem conte nos mato\\
Os tipo de assombração,\\
Na noite dessas pirraça\\
Ninguém arranja uma caça\\
Fica tudo deferente,\\
Quando essas coisa acontece\\
Até o cachorro entristece,\\
Não sai de perto da gente.

O que é isso seu dotô!\\
No seu jeito eu tô notando\\
Que de tudo que eu lhe disse\\
O sinhô tá duvidando,\\
Se qué sabê se é exato\\
Vamo comigo pro mato,\\
Pra Chapada do Espigão,\\
Hoje é bom que é sexta-fêra\\
Vao sabê se é brincadêra\\
Histora de assombração.

O sinhô não sabe nada\\
Das coisa do meu sertão,\\
Só conhece futibó,\\
Cinema e televisão,\\
Se vossimicê nos mato\\
Incrontasse ispaiafato\\
Como eu já tenho encontrado,\\
La do mato o sinhô\\
vinha Com a carça moiadinha\\
Da braguia ao imbanhado.

O sinhô que é um dotô\\
E sabe lê e escrevê,\\
Tarvez passasse três dias\\
Sem conhece o \textsc{abc}\\
E quando em casa chegasse\\
Que a sua muié oiasse\\
Lhe dizia achando ruim:\\
Meu véio, o que diabo é isto?\\
Eu nunca tinha lhe visto\\
De carça moiada assim.

E agora que já contei\\
Minha verdade sagrada\\
E é tarde, vamo drumi,\\
Sua rede tá arrumada,\\
Aqui zuada não tem,\\
O dotô vai drumi bem\\
Um sono reparadô\\
Com o seu amô sonhando,\\
Não tem carro businando\\
Nem zuada de motô.

Bom dia seu candidato,\\
Só pôde acordá agora?\\
Com certeza drumiu bem\\
Já passou das nove hora,\\
Depois que o rosto lavá\\
O sinhô vai merendá\\
Tomá café com beju\\
Fazendo misturada\\
Traçando com carne assada\\
De titela de jacu.

Já passou das nove hora,\\
Coma bem munto seu moço\\
A merenda tarde assim\\
Não é merenda é armoço,\\
Tem munta carne nos prato\\
E dessas caça dos mato\\
Eu vi que o sinhô gostou,\\
Come munto e come bem,\\
Parece que o sinhô vem\\
De raça de caçadô.

Eu já fiz o seu pedido\\
Seu cavalo tá celado,\\
Mas querendo demorá\\
Tem às orde um seu criado,\\
Se vamicê demorasse,\\
Tarvez a gente caçasse\\
Na Chapada do Espigão\\
Lá o sinhô se assombrava\\
E nunca mais duvidava\\
De histora de assombração.

Mas como já qué parti,\\
Vá fazê sua campanha,\\
Que o inleitô é quem perde\\
E o candidato é quem ganha,\\
Vá em paz seu candidato\\
Que eu fico aqui pelo mato\\
Fazendo minhas caçada,\\
É mió sê caçadô\\
Do que sê um inleitô\\
Pra votá sem ganhá nada.
\end{verse}

\chapter{Raimundo Jacó}	

\hfill\emph{(Versos recitados pelo autor no dia do vaqueiro)}

\begin{verse}
Aqui o que a dizer tenho\\
Eu sei que o povo admite,\\
É que de ano em ano eu venho\\
Atendendo este convite,\\
Nos três dias festejados\\
Gente de vários Estados\\
No parque se manifesta\\
E eu sinto triste impressão\\
Por saber qual a razão\\
E o motivo desta festa.

Como poeta roceiro\\
Me encontro aqui novamente\\
Para falar de um vaqueiro\\
Que morreu barbaramente\\
Este vaqueiro afamado\\
É lembrado e relembrado\\
Como a luz que não se apaga,\\
Jamais o tempo destrói\\
A fama do grande herói\\
Primo de Luiz Gonzaga.

Com expressão compassiva,\\
Bom vaqueiro aboiador,\\
Aqui fala o Patativa\\
Teu grande admirador,\\
Neste dia alvissareiro\\
Nacional do vaqueiro,\\
Escutando a nossa voz\\
Tu desces lá do infinito\\
Para agradecer contrito\\
Vagando aqui entre nós.

Vem do campo e da cidade\\
Sem enfado nem preguiça\\
O povo com piedade\\
Assistir a tua missa\\
E eu também que não me esqueço\\
Comovido te ofereço\\
Na minha simplicidade\\
De caridade repleta,\\
Homenagem de um poeta\\
Da justiça e da verdade.

De chapéu, gibão e perneira\\
Na tua terra bonita,\\
Desde o Sítio dos Moreira\\
Às quebradas de Serrita,\\
Eras forte e destemido\\
Sem nunca temer tecido\\
De unha de gato e cipó,\\
O teu valor permanece\\
E hoje o Brasil já conhece\\
Quem foi Raimundo Jacó.

Foste num dia inditoso\\
Vítima de negra traição\\
Por um colega invejoso\\
Sem dó e sem compaixão,\\
Deixasse na terra a história\\
E partiste para a glória\\
Do nosso Pai Soberano\\
Na Santa Mansão Celeste,\\
Caboclo do meu Nordeste\\
Vaqueiro pernambucano.

Tu, vaqueiro nordestino,\\
Que foste um dia abatido\\
Pela mão do mau destino\\
Jamais serás esquecido,\\
Tua fama continua,\\
Sinto na noites de lua\\
A sensação de escutar\\
O teu aboio distante\\
E o lengo tengo distante\\
Do chocalho a badalar.

Longe da vida terrestre\\
Ao lado do santo trono,\\
Aos pés do Divino Mestre\\
Enquanto dormes teu sono,\\
Tu és lembrado Raimundo,\\
Com sentimento profundo\\
Com amor e devoção\\
Neste poema que eu rimo\\
E pela voz do teu primo\\
Luiz o Rei do Baião.
\end{verse}

\chapter{Juazêro e Petrolina}

\begin{flushright}
\emph{(Poesia recitada pelo autor por ocasião do segundo festival dos
violeiros em Petrolina-\textsc{pe}.)}
\end{flushright}

\begin{verse}
Vou vortá bem satisfeito,\\
A viage não perdi\\
E vou falá com respeito\\
Sobre uma coisa que eu vi,\\
Eu nunca gostei de enredo\\
Mas vou contá um segredo\\
E sei que o povo combina,\\
Existe aqui um pobrema\\
Sobre este amoroso tema\\
Juazêro e Petrolina.

É coisa bastante certa\\
Que com relação ao amô\\
Só faz grande discoberta\\
Quem é bom pesquisadô;\\
Vocês quera discurpá,\\
Mas o bardo populá\\
Patativa do Assaré\\
Vai já falá pra vocês\\
De uma coisa que tarvez\\
Ninguém tenha dado fé.

Aqui na bêra do rio\\
Tem uma dô que consome,\\
Vejo a verdade e confio\\
Vi que o Juazêro é home\\
E Petrolina é muié,\\
Vi que o Juazêro qué\\
Com Petrolina se casá,\\
Porém corre um grande risco,\\
As água do São Francisco\\
Não deixa os dois se abraçá.

Tem um riso feiticêro\\
A linda pernambucana\\
E gosta do Juazêro\\
Moço da terra baiana,\\
Vejo tudo e tô ciente\\
Que o que ele sente ela sente\\
Mas esperança não tem\\
De sastifazê o desejo\\
Apenas envia beijo\\
Pela brisa que vai e vem.

Juazêro vai passando\\
Com a arma apaxonado\\
Do outro lado reparando\\
Para a sua namorada\\
E Petrolina conhece\\
E a mesma paixão padece\\
O namorado não esconde,\\
Lá do outro lado do rio\\
Juazêro dá o picio\\
E Petrolina responde.

Que seca ou inverno\\
Nesta terra nordestina,\\
Tem sempre um amô eterno\\
Juazêro e Petrolina,\\
Sei que é firme este namoro,\\
Porém existe um agôro,\\
Um azá, e um aperreio,\\
Ele do lado de lá\\
Ela do lado de cá\\
E o rio a roncá no meio.

Juazêro e Petrolina\\
De amô veve ardendo em brasa\\
Pois é munto triste a sina\\
De quem namora e não casa,\\
Quando o rio dá enchente\\
Mais os namorado sente,\\
Um de lá outro de cá\\
Cada quá faz sua quêxa\\
Cronta o rio que não dêxa\\
O sonho realizá.

Juazêro este baiano\\
Moço forte e destemido\\
De realizá seu prano\\
Já veve desinludido\\
Com a arma apaxonada\\
Óio para a namorada\\
Mas é grande o sofrimento\\
Pois não pode sê isposo\\
Divido o rio orguiso\\
Impatá seu casamento.

Inquanto descê nas água\\
Bascuio, barcêro e cisco,\\
Causando paxão e mágua\\
Este rio São Francisco\\
Capricho da Natureza,\\
Com a sua correnteza\\
Neste baruio maluco\\
Toda noite e todo dia,\\
Não será sogra a Bahia\\
E nem sogro o Pernambuco.

Se oiando de face a face\\
Petrolina o seu querido\\
Não pode fazê o inlace\\
Pruquê o rio intrometido\\
Do seu leito nunca sai,\\
É um suspiro que vai\\
E outro suspiro que vem\\
E nesta sentença crua\\
O namoro continua\\
Por um século sem fim amém.
\end{verse}

\chapter{Ao reis do baião}

\begin{verse}
Caboco Luiz Gonzaga!\\
Tu és do céu de Nabuco,\\
A estrela que não se apaga\\
Gulora de Pernambuco,\\
Tu é o dono da conquista\\
O mais fino e grande artista\\
Que canta baião pra nós,\\
De alegria pressiona\\
Quem ouve a tua sanfona\\
Ligada na tua voz.

Caboco de geno forte\\
Eu nunca vi como tu\\
Leva semente do Norte\\
Prumode prantá no Sul.\\
Tu sempre foi preferido\\
Em toda parte querido,\\
Mas porém tu é mais caro\\
Na terra pernambucana,\\
Prazê de dona Santana\\
E orguio de Januaro.

Por capricho do destino\\
Outrora tu foi sordado,\\
Mas Deus, nosso Pai Divino,\\
Te vendo um dia humiado\\
Disse: --- ``Luiz, dêxa a farda!\\
Esta vida de espingarda\\
Para tu é um horrô;\\
Pega a sanfona, Luiz!\\
Vai de país em país\\
Eu serei teu potretô.''

E tu, pegando a sanfona,\\
Como bom e obediente,\\
Foi espaiando na zona\\
Da terra dos penitente,\\
Com tua rica cachola,\\
Este baião que não cai,\\
Este baião que já vai\\
Da terra inté lá no céu.

Neste requebro gaiato\\
Do teu grito de vaquêro,\\
Eu vejo o fié retrato\\
Do nordestino brasilêro\\
Oiço da vaca o gemido,\\
Do chucaio oiço o tinido\\
E a gaita do véio tôro,\\
E vejo a festa comum\\
Do sertão dos Inhamum\\
Terra de chapéu de côro.

Tua sanfona sodosa,\\
Com quem tu veve abraçado,\\
É a santa milagrosa\\
Ressuscitando o passado;\\
Inté mermo a criatura\\
Sisuda, de cara dura,\\
E de crué coração,\\
Fica branda como a cera\\
Uivando a voz prazentêra\\
Do grande rei do baião.

Quando tu dêxou de sê\\
Da filêra de sordado,\\
Não querendo mais sabê\\
Da luta do pau furado,\\
Que com teu geno profundo\\
Foi espaiando no mundo\\
A tua voz de tenô\\
De milagre incomparave\\
A vida ficou suave\\
E o Nordeste miorou.

De prazê ninguém sussega\\
Tudo sarta de animado,\\
Na hora que tu molega,\\
Os teus dedo no tecrado;\\
Nosso caboco daqui\\
Tudo forga, tudo ri,\\
Ninguém se lembra de praga,\\
Nem de fome, nem de peste,\\
Quando escuta no Nordeste\\
A voz de Luiz Gonzaga.

Caboco do geno forte,\\
Contigo ninguém se engana,\\
Tu é do Sul e é do Norte,\\
Do palaço e da chupana,\\
Tu veve provando a raça,\\
Derne o campo inté a praça,\\
Na vida de sanfonêro\\
É grande rei soberano\\
Moreno pernambucano\\
Que sabe sê brasilêro.
\end{verse}

\chapter[O bicho mais feroz sátira imperdoável]{O bicho mais feroz\\ sátira imperdoável}

\begin{verse}
O dia amanheceu, era verão,\\
A tomar seu café lá no fogão\\
O seu Tonho dizia para Solidade,\\
O sonho muitas vezes é realidade,\\
E esta noite eu sonhei quando dormia\\
Que um cachorro na roça me mordia,\\
Por aqui hoje o dia vou passar\\
E não vou para a roça trabalhar.

--- Respondeu Solidade, isto é besteira\\
O sonho não é coisa verdadeira,\\
E se o bicho morder você se atrasa\\
Tanto faz lá na roça como na casa,\\
O melhor é você não pensar nisso\\
E ir pra roça cuidar do seu serviço.

O seu Tonho saiu dando cavacos\\
Com o seu cavador cavar buracos,\\
Felizmente o coitado não foi só\\
Pois o filho o seguiu no mocotó.

No primeiro buraco que cavou\\
Para ele o perigo não chegou,\\
Porém quando passou para o segundo,\\
Viu estrelas brilhando no outro mundo,\\
Uma feia raposa sem respeito\\
Agarrou-lhe na mão de certo jeito\\
E rosnando raivosa arrepiada\\
Com os dentes lhe dava safanada,\\
O coitado sozinho a pelejar\\
E a raposa filada sem soltar.

Sem poder defender-se do perigo\\
O seu Tonho a gemer disse consigo\\
Bicho doido dos diabos tu me pagas\\
E gritou pelo filho: venha, Chagas,\\
Venha logo depressa em meu socorro\\
Que estou preso nos dentes de um cachorro,\\
Quando Chagas ouviu disse de lá:\\
Vou fazer um cigarro e chego lá\\
E fazendo o cigarro de repente\\
Foi provar que é um filho obediente.

Chagas vendo a raposa foi dizendo:\\
Pra seu Tonho poder ficar sabendo,\\
Ou papai, me desculpe, por bondade,\\
O senhor tem sessenta anos de idade\\
E por fora daqui já tem andado\\
Pois já foi passeas em outro estado\\
No roçado um só dia nunca falha\\
E se em casa tiver também trabalha\\
Torce corda, faz peia de capricho\\
Faz cabresto e faz mais alguma coisa\\
E ainda não conhece uma raposa?\\
Eu lhe digo e o senhô sei que combina,\\
Isso aí é raposa até na China.

O seu Tonho já quase esmorecido\\
Respondeu para o filho, aborrecido,\\
--- Ou raposa ou cachorro ou qualquer raça,\\
Por favor mate logo essa desgraça,\\
Foi que o Chagas cortando um grosso pau\\
Acabou com aquele bicho mau.

Não podendo o ferido ter demora\\
Para a casa voltou na mesma hora\\
E achando que a vida estava em risco\\
Foi chegando com ar de São Francisco.

E dizendo pra sua boa esposa:\\
Veja aqui o que fez uma raposa,\\
Solidade ficou bastante aflita\\
Porém como em Jesus muito acredita,\\
Respondeu: --- Você inda foi feliz\\
Se a raposa mordeu, porque Deus quis\\
E se fosse na casa também vinha,\\
Lhe mordia e levava uma galinha.

Tudo aquilo seu Tonho ouviu calado\\
Disfarçando que estava conformado\\
A Tosinha, a Canginha, a Margarida,\\
Reparando o xaboque da mordida\\
Aplicaram remédio bem ligeiro\\
E foram dá risada no terreiro.

E seu Tonho no seu comportamento\\
Passou dias fazendo tratamento,\\
Desfrutando café, cigarro e bóia\\
Paciente de braço na tipóia\\
E o Chagas por causa do acidente\\
Passou dias folgado, bem contente,\\
Pois quando ia pra roça era sozinha,\\
Muitas vezes voltava do caminho.

Felizmente, o seu Tonho está curado,\\
Porém nunca deixou de ter cuidado\\
E ele até com razão fez uma jura\\
Não tirar sua faca da cintura,\\
Outro dia com o Souza conversando\\
Em diversos assuntos e falando\\
Sobre os bichos ferozes do país\\
Ele disse mostrando a cicatriz:\\
--- Pode crer meu prezado amigo Souza\\
Não há um tão feroz como a raposa.
\end{verse}

\chapter{Eu e o padre Nonato}

\begin{verse}
Meu caro Padre Nonato\\
Eu nestes versos relato\\
Uma prova verdadeira\\
Da nossa grande coragem\\
Fazendo aquela viagem\\
A Lavras da Mangabeira.

Quando o convite me fez\\
Eu senti por minha vez\\
Que o tom de suas palavras\\
Me deu prazer estupendo\\
E fui com o Reverendo\\
Até São José de Lavras.

Seu jipe velho coitado!\\
Todo desparafusado\\
Deu voltas de bicicleta,\\
Corria muito veloz\\
Sempre conduzindo nós\\
Um vigário e um poeta.

O poeta diminutivo\\
E o vigário aumentativo\\
Pegado na direção,\\
Quem reparava dizia\\
Que o transporte conduzia\\
Um gigante e um anão.

Você muito corajoso\\
E eu um tanto nervoso\\
Que às vezes não percebia\\
Devido a marcha apressada\\
Sua palestra animada\\
Cheia de filosofia.

Eu muitas vezes pensava\\
Que aquele jipe rodava\\
Com asas de passarinho\\
E nós dois, Padre Nonato\\
Às vezes dentro do mato\\
E outras vezes no caminho.

Cheios de esperança e fé\\
Chegamos em São José\\
Graças a Deus, felizmente,\\
Onde fomos recebidos\\
Recebidos e acolhidos\\
Por sua querida gente.

Com algumas companhias\\
O Doutor Francisco Dias\\
Bendizia o belo mês\\
Em que por Deus protegido\\
Sua esposa tinha sido\\
Mãe pela primeira vez.

Que bela reunião\\
E que doce animação,\\
Patativa do Assaré,\\
O Reverendo Nonato\\
Gente da praça e do mato\\
E haja festa em São José.

Na casa de Francisquinha\\
Desde a varanda à cozinha\\
Uma entrava e outro saía,\\
Foi tudo bem misturado,\\
A festa do batizado\\
Carne, arroz e poesia.

Não podíamos ficar,\\
Precisávamos voltar\\
Da viagem com urgência,\\
Deixando tanta alegria,\\
Tanta paz, tanta harmonia,\\
Lá naquela residência.

Para nossa despedida\\
Houve gente reunida,\\
Abraço, beijo e carinho,\\
Nós voltamos com saudade\\
E para felicidade\\
Anoiteceu no caminho.

A lua no céu brilhava\\
Parecendo até que estava\\
Muito mais encantadora,\\
Estendendo pura e franca\\
A sua toalha branca\\
Sobre a terra sofredora.

Já era tarde da noite,\\
O vento um suave açoite\\
Soprava constantemente\\
Bafejando com amor\\
Para esfriar o motor\\
Do jipe velho doente.

Ouvimos que alguém falava\\
E baixinho conversava\\
Como quem conta um segredo,\\
Era um grupo de cigano\\
Com as barracas de pano\\
Sob um frondoso arvoredo.

E você, meu Reverendo,\\
Foi alegre me dizendo\\
Depois que o carro parou:\\
Vamos ouvir um artista\\
Cantor e violonista\\
Conhecido por Chator.

Logo uma cigana veio\\
Naquele grande aperreio\\
Com o jeito que ela tem\\
De nos chamar de ganjão,\\
Leu depressa a sua mão\\
E leu a minha também.

Falou que em nosso porvir\\
Ia a gente possuir\\
Bom dinheiro e muito gado,\\
Com isso recebeu ela\\
Uma pequena parcela\\
Do nosso cobre minguado.

O Chator, nossa atração,\\
Cantava em seu violão\\
Com muita sonoridade,\\
Com bom ritmo e com estilo,\\
Aumentando com aquilo\\
A nossa felicidade.

Foi grande a nossa surpresa\\
Ante a beleza e a riqueza\\
Na mente de um ser humano,\\
Cá nos julgamentos meus\\
Eu via as graças de Deus\\
Na voz daquele cigano.

Foi alegria completa\\
Para um padre e um poeta\\
Amigos da poesia,\\
Enquanto o Chator cantava\\
A lua no céu brilhava\\
E a Natureza sorria.

Naquela simplicidade\\
Estava a realidade,\\
Só conhecem o porquê\\
Do prazer que ele nos deu\\
Um poeta como eu\\
E um padre como você.

Muito além da prata e do ouro\\
Estava ali um tesouro\\
Num quadro de singeleza,\\
O grande milionário\\
Egoísta e usurário\\
Não conhece o que é riqueza.

Ao terminar este assunto,\\
Padre Nonato, eu pergunto:\\
Cadê o cigano Chator\\
Com o seu dom soberano\\
Aquel humilde cigano\\
Se morreu já se salvou.

Responda meu confessor\\
Aonde está o condutor\\
Daquela simples equipe\\
O Chator onde andará?\\
E também onde estará\\
A sucata do seu jipe?
\end{verse}

\chapter{Castigo do mucuim}

\begin{verse}
\textbf{1}

Bom dia compade Chico,\\
Como você tem vivido?\\
Eu às vez pensando fico\\
Que você anda escondido,\\
Pra festa você não vai,\\
Da sua casa não sai\\
Só qué trabaiá e drumi,\\
Meu compade Chico, a gente\\
Pro mode vivê contente\\
É preciso divirti.

\textbf{2}

Isto eu lhe digo e garanto,\\
Não viva desta manêra\\
Escondido lá num canto\\
Como bode com bichêra,\\
Compade a gente trabaia,\\
Trabaia que se escangaia\\
Na roça de só a só,\\
Mas pra disinfastiá\\
É preciso freqüentá\\
De quando em vez um forró.

\textbf{3}

A maió felicidade\\
Tá na comunicação\\
Sincera sem farcidade\\
Na mais prefeita união,\\
Brincá sem arranjá briga\\
Deus ajuda, não castiga,\\
Divirti não é pecado;\\
Compade você se arrasa\\
Só do roçado pra casa\\
E da casa pro roçado.

\textbf{4}

Ah! Meu compade Mané,\\
Com esta minha estatura\\
Você sabe como é,\\
Todo mundo me censura,\\
Eu sei que neste sertão\\
Nunca farta diversão\\
Mas delas não me aproximo\\
Pruquê bastante me acanho\\
De andá com o meu tamanho\\
De um metro e vinte centimo.

\textbf{5}

Pode crê como é verdade\\
O meu disgosto é compreto\\
 Esta farça humanidade\\
Não deixa ninguém tá queto\\
E nem da gente tem dó,\\
Fui outro dia um forró\\
Na casa de Zé Davi\\
E vi quando fui chegando\\
Arguém dizendo e mangando:\\
Ói o anão do Brasi!

\textbf{6}

Não posso tá prazentêro\\
Em uma reunião\\
Com sangue de brasilêro\\
E o tamanho dum anão,\\
Se com arguém vou falá\\
É preciso escangotá,\\
Fica uma posição feia\\
E munto me desanima\\
Tá com a cara pra cima\\
Que nem caçadô de abeia.

\textbf{7}

--- Meu compade Chico, amigo,\\
Esta sua pequenez\\
Pra você não é castigo\\
Cada quá como Deus fez,\\
Ninguém manga de você,\\
Eu lhe digo e pode crê,\\
Lhe juro de consciença,\\
Não viva nisso pensando,\\
Se arguém mangá tá mangando\\
Da Divina Providença.

Eu de você me aproximo\\
Com amô e com respeito,\\
Dois metro e vinte centimo,\\
Mas o juízo é prefeito,\\
Sua forma não é nula,\\
Você sarta, você pula,\\
E é bastante inteligente,\\
Por um castigo não tome,\\
Compade você é home,\\
Compade você é gente.

Cada quá tem seu destino\\
É um naturá segredo\\
E nas orde do Divino\\
Ninguém vai metê o dedo,\\
Compade, magine e pense,\\
Que isto tudo a Deus pertence,\\
Foi Deus quem lhe fez assim;\\
Pra você se conformá\\
Vou um inxempro contá\\
Do atrivido mucuim.

Nas areia duma estrada\\
Um mucuim vagabundo\\
Vendendo azeite às canada\\
Falou de Deus e do mundo;\\
Dizia ele raivoso:\\
Vivo munto desgostoso\\
Não posso tê paciença\\
Sou pobre diminutivo,\\
Mode vivê como eu vivo\\
Não vale a pena a inxistença.

Vejo animá neste mundo\\
Tão grande que se escangaia\\
E eu pequenininho imundo\\
Pedacinho de migaia,\\
Eu calado não tulero,\\
Não sei porque foi que dero\\
Meu nome de mucuim,\\
Esta mágua me consome,\\
Seis letras tem o meu nome,\\
E eu tão pequeno assim.

Fico zangado e condeno\\
Quando vejo a todo istante\\
Bicho do nome pequeno\\
Com um tamanho gigante,\\
E eu sacudido no pó\\
Com esta sorte cotó,\\
O tal enredo ou segredo\\
Não sei mesmo porque foi,\\
Só três letras tem o boi\\
E é grande que causa medo.

Fico raivoso e afobado\\
Quando reparo o camelo\\
Com um tamanho alarmado\\
Sobrando ainda um novelo\\
E ainda mais o elefante,\\
De um tamanho extravagante\\
Que cresceu, cresceu, cresceu,\\
É o maió de todas raça\\
E só duas letras passa\\
Do nome que arguém me deu.

Parece mesmo um capricho,\\
Eu me envergonho e me acanho,\\
Uns pedaço destes bicho\\
Pra botá no meu tamanho\\
Fazia eu ficá maió,\\
Pois vejo que sou menó\\
Do que um grãozinho de areia\\
Só sabe arguém que eu inxisto\\
Quando revortado insisto\\
Mexendo na péia alêia.

Com este grande castigo\\
Eu não me conformo não,\\
Foi munto ingrato comigo\\
O Sinhô da Criação,\\
Com esta dura sentença\\
Não posso tê paciença\\
E nem vivo satisfeito\\
Vivo cheio de rancô\\
E vejo que o criadô\\
Não fez o mundo dereito.

--- Nas areia do caminho\\
Todo cheio de rancô\\
Tava aquele falerinho\\
Defamando o criadô\\
E só não se estribuchava\\
Pruquê o tamanho não dava\\
Pra ele se estribuchá,\\
Mas porém quem faz assim\\
Como fez o mucuim\\
Vê seu castigo chegá.

A Divina Majestade\\
Pra castigá aquele mau\\
Mandou uma tempestade\\
Virando cepa de pau,\\
Disparou um vento forte\\
Roncando do sul ao norte\\
Arrastando o pó da estrada\\
Como quem diz: não caçoi;\\
E aquele mucuim foi\\
Batê na taba lascada.

Veja aí, compade Chico,\\
O castigo, o grande horrô,\\
Cronta o pequenino tico\\
Debochando o Criadô,\\
As graça de Deus é tanta\\
Que até nas ave que canta\\
A voz dele nós uvimo,\\
Deste inxempro não se esqueça\\
Adore a Deus e agradeça\\
Seu metro e vinte centimo.
\end{verse}

\chapter{Zé Limeira em carne e osso}

\begin{flushright}
\emph{(Ao poeta e jornalista doutor Orlando Tejo)}
\end{flushright}

\begin{verse}
Nesta vida passageira\\
Há coisa que muito pasma,\\
Disse alguém que Zé Limeira\\
É um poeta fantasma,\\
É um cantador fictício,\\
Por isto, com sacrifício\\
Querendo ser sabedor,\\
Viajei com paciência\\
Para saber da existência\\
Do famoso cantador.

Saí do meu Ceará\\
Para tirar esse engano,\\
Parando aqui, acolá,\\
No mapa paraibano,\\
Para colher a verdade\\
Do campo até a cidade\\
Perguntei ao velho e ao moço\\
E não dei um passo a esmo,\\
José Limeira foi mesmo\\
Um poeta em carne e osso.

O seu improviso tinha\\
Versos espalhafatosos\\
Deixando fora da linha\\
Os cantadores famosos,\\
Nas rimas da sua lavra\\
Ele criava palavra\\
Que dominava a assistência,\\
E os camponeses que ouviam\\
Batiam palma e diziam:\\
Está cantando ciência!

O que ele tinha na mente\\
Não pode saber ninguém,\\
Seu mundo era diferente\\
E a poesia também,\\
Foi o célebre Limeira\\
Lá na Serra do Teixeira\\
Um grande e frondoso arbusto\\
No embalo da brisa mansa\\
E hoje na glória descansa\\
A sua alma de justo.

Depois que eu tive a certeza,\\
Com um sentimento nobre\\
E a mais alegre surpresa,\\
Com a minha lira pobre\\
De verbo e de fantasia,\\
Sem usar de hipocrisia,\\
Fiz com atenção e amor\\
E completa consciência\\
A seguinte referência\\
Ao seu apresentador.

Quando a doutrina espantânea\\
Jesus andou a pregar,\\
Lá nas terras da Betânea\\
Fez Lásaro ressuscitar,\\
Porém, Jesus é divino,\\
Admiro é o nordestino\\
Nesta nação brasileira,\\
Orlando Tejo, escritor,\\
Sendo um grande pecador\\
Ressuscitar Zé Limeira.

O poeta Orlando Tejo\\
Com sua capacidade\\
De sertão até ao brejo,\\
Da praia até a cidade,\\
Depois de algum vai não vai\\
Vem não vem e sai e não sai,\\
Confusão, quase chafurdo,\\
Sua pena alvissareira\\
Mostrou quem foi Zé Limeira\\
O poeta do absurdo.

Só o Orlando este famoso\\
Poeta gênio liberto,\\
Que hoje me sinto ditoso\\
Por conhecê-lo de perto,\\
Teve o cuidado e a coragem\\
De guardar esta bagagem\\
Que mais tarde publicou\\
Mostrando o vocabulário\\
Deste cantador lendário\\
Que a Paraíba criou.

Zé Limeira, este portento\\
De disparate sem par,\\
No mundo do esquecimento\\
Não poderia ficar,\\
Com improvisos vibrantes\\
E assuntos extravagantes\\
Todos lhe queriam bem,\\
Cantando ao som da viola\\
Ele criou uma escola\\
Sem ajuda de ninguém.

Orlando, o pesquisador\\
De qualidade primeira\\
Registrou com muito amor\\
A escola de Zé Limeira,\\
Tudo ele pôde gravar\\
Pois não queria deixar\\
Este tesouro perdido\\
E hoje eu me sinto contente\\
Lendo um livro diferente\\
De todos que eu tenho lido.

E por isto, sempre quando\\
Vou o livro folhear\\
Me parece ouvir Orlando\\
Com Limeira a conversar;\\
Colega, eu muito agradeço\\
O presente não me esqueço\\
De guardá-lo com cuidado,\\
Você acerta e não erra,\\
Quem não ama a própria terra\\
É desnaturalizado.
\end{verse}

\chapter{O beato Zé Lourenço}

\begin{verse}
Sempre digo, julgo e penso\\
Que o beato Zé Lourenço\\
Foi um líder brasileiro\\
Que fez os mesmos estudos\\
Do grande herói de Canudos,\\
Nosso Antônio Conselheiro.

Tiveram o mesmo sonho\\
De um horizonte risonho\\
Dentro da mesma intenção,\\
Criando um sistema novo\\
Para defender o povo\\
Da maldita escravidão.

Em Caldeirão trabalhava\\
E boa assistência dava\\
A todos os operários,\\
Com sua boa gente\\
Lutava pacificamente\\
Contra os latifundiários.

Naquele tempo passado\\
Canudos foi derrotado\\
Sem dó e sem compaixão,\\
Com a mesma atrocidade\\
E maior facilidade\\
Destruíram Caldeirão.

Por ordem dos militares\\
Avião cruzou os ares\\
Com raiva, ódio e com guerra,\\
Na grande carnificina\\
Contra a justiça divina\\
O sangue molhou a terra.

Porém, por vários caminhos,\\
Pisando sobre os espinhos,\\
Com um sacrifício imenso,\\
Seguindo o mesmo roteiro\\
Sempre haverá Conselheiro\\
E Beato Zé Lourenço.
\end{verse}

\begin{flushright}
\emph{(Composta para o filme ``O Caldeirão da Santa Cruz do Deserto'',
de Rosemberg Cariry)}
\end{flushright}

\chapter{A verdade e a mentira}

\begin{verse}
Foi a verdade e a mentira\\
Nascida no mesmo dia,\\
A verdade, no chão duro\\
Porque nada possuía\\
E a mentira por ser rica\\
Nasceu na cama macia\\
E por causa disto mesmo\\
Criou logo antipatia,\\
Não gostava da verdade,\\
Temendo a sua energia,\\
Pois onde a mentira fosse\\
A verdade também ia\\
O que a mentira apoiava\\
A verdade não queria\\
Oque a mentira formava\\
A veldade desfazia\\
O segredo da mentira\\
A verdade descobria,\\
E a mentira esmorecendo\\
Vendo que não resistia\\
Chamou depressa o dinhêro\\
Para sua companhia,\\
Levou o dinhêro com ele\\
A inveja, a hipocrisia,\\
A ambição, a calúnia,\\
O orgulho, o crime e a ironia,\\
A soberba e a vaidade\\
Que são da mesma famia\\
E fizero um tal fofó\\
Um ingôdo, uma ingrizia\\
Que a verdade pelejava\\
Pra desmanchá e não podia\\
E a mentira aposentou-se\\
Com esta grande quadria.

Depois, casou-se o dinhêro\\
Com sua prima anarquia\\
E com quatro ou cinco mês\\
Dela nasceu uma fia,\\
Caçaro logo os padrinho\\
Mas no mundo não havia\\
Satanaz com a mãe dele\\
Lhe apresentaro na pia\\
E com todo atrevimento\\
Com toda demagogia\\
Caçaro um nome bonito\\
Na sua infernal cartia\\
E dissero: essa menina\\
Se chama democracia,\\
Tudo se danou de quente\\
E a verdade ficou fria
\end{verse}

\chapter{A realidade da vida}

\begin{verse}
Na minha infança adorada\\
Meu avô sempre contava\\
Muntas histora engraçada\\
E de todas eu gostava,\\
Mas uma delas havia\\
Com maió filosofia\\
E eu como poeta sou\\
E só rimando converso,\\
Vou aqui conta em verso\\
O que ele em prosa contou.

Rico, orguioso, profano,\\
Rifrita no bem comum\\
Veja os direitos humano\\
As razão de cada um,\\
Da nossa vida terrena\\
Desta vida tão pequena\\
A beleza não destrua,\\
O dereito do banquêro\\
É o direito do trapêro\\
Que apanha os trapo na rua.

Pra que vaidade e orguio?\\
Pra que tanto confusão\\
Guerra, questão e baruio\\
Dos irmão contra os irmão?\\
Pra que tanto preconceito\\
Vivê assim deste jeito\\
Esta inxistença é perdida\\
Vou um inxempro contá\\
E nestes versos mostrá\\
A realidade da vida.

Quando Deus nosso sinhô\\
Foi fazê seus animá\\
Fez o burro e lhe falou:\\
--- Tua sentença eu vou dá,\\
Tu tem que sê escravizado\\
Levando os costá pesado\\
Conforme o teu dono quêra\\
E sujeito a toda hora\\
Aos fino dente da espora\\
Mais a brida e a cortadêra.

Tu tem que a vida passá\\
Com esta dura sentença\\
E por isto eu vou te dá\\
Uma pequena inxistença,\\
Já que em tuas carne tora\\
Brida, cortadêra, espora,\\
E é digno de piedade\\
E crué teu padicê\\
Para tanto não sofrê\\
Te dou trinta ano de idade.

O burro ergueu as ureia\\
E ficou a lamentá\\
Meu Deus ô sentença feia\\
Esta que o Sinhô me dá,\\
Levando os costá pesado\\
E de espora cutucado\\
Trinta ano quem é que agüenta?\\
E mais outras coisa lôca;\\
A brida na minha boca\\
E a cortadêra na venta.

Vivê trinta ano de idade\\
Deste jeito é um castigo,\\
É grande a prevessidade\\
Que o meu dono faz comigo\\
E além desse escangaio\\
Me bota mais um chucaio\\
Que é pra quando eu me sortá\\
De longe ele uvi o tom;\\
Dez ano pra mim tá bom,\\
Tenha dó de meu pená!

A Divina Majestade\\
Fez o que o burro queria\\
Dando os dez ano de idade\\
Da forma que ele pedia\\
Mode segui seu destino\\
E o nosso artista divino\\
A quem pode se chamá\\
De artista santo e prefeito\\
Continuou sastisfeito\\
Fazendo mais animá.

Fez o cachorro e ordenou\\
Tu vai trabaiá bastante,\\
Do dono e superiô\\
Será guarda vigilante,\\
Tem que a ele acompanhá\\
Fazendo o que ele mandá\\
Nas arriscada aventura,\\
Até fazendo caçada\\
Dentro da mata fechada\\
Nas treva da noite escura.

Tu tem que sê sentinela\\
Da morada do teu dono\\
Para nunca ficá ele\\
No perigo e no abandono,\\
Tem que sê amigo exato\\
Na casa e também no mato\\
Mesmo com dificurdade\\
Subindo e descendo morro;\\
Teu nome é sempre cachorro\\
E vinte ano é tua idade.

Quando o cachorro escutou\\
Aquela declaração\\
Disse bem triste: Senhô\\
Tenha de mim compaixão!\\
Eu desgraço meu focinho\\
Entre pedra, toco e espinho\\
Pelo mato a farejá\\
Ficando sujeito até\\
A presa de cascavé\\
E unha de tamanduá.

Vinte ano neste serviço\\
Sei que não posso agüentá\\
É grande o meu sacrifiço\\
Não posso nem descansá\\
Sendo da casa o vigia\\
Trabaiando noite e dia\\
Neste grande labacé,\\
Tenha de mim piedade,\\
Dos vinte eu quero a metade\\
E os dez dê a quem quisé.

O cachorro se alegrou\\
E ficou munto feliz\\
Pruquê o Senhô concordou\\
Da manêra que ele quis,\\
Ficou bastante contente\\
E o Deus Pai Onipotente\\
Fez o macaco em seguida\\
E depois da expricação\\
Qual a sua obrigação\\
Lhe deu trinta ano de vida.

E lhe disse: --- O teu trabaio\\
É sempre fazê careta\\
Pulando de gaio em gaio\\
Com as maió pirueta,\\
Tu tem que sê buliçoso\\
Fazendo malicioso\\
Careta pra todo lado\\
Pulando, sempre pulando\\
Muntas vez até ficando\\
Pela cauda pendurado.

O macaco uviu afrito\\
E ficou cheio de espanto\\
Deu três pulo e deu três grito\\
Se coçou por todo canto\\
E disse: --- Ô que sorte preta,\\
Pulando e a fazê carêta\\
Trinta ano, assim eu me acabo,\\
Senhô, será que eu caio\\
Lá da pontinha do gaio\\
Pendurado pelo rabo.

É bem triste a minha sina,\\
Trinta ano de cambaiota\\
Com esta cintura fina,\\
A minha força se esgota,\\
Ô Divina Majestade\\
Me discurpe esta verdade,\\
Mas vejo que é um capricho\\
A idade que Deus me deu\\
Tire dez anos dos meus\\
Pra idade doutro bicho.

Deus concordou e ele disse:\\
Já saí do aperreio\\
Fez diversas macaquice\\
Deu dez pinotes e meio\\
Agradecendo ao Senhô\\
E o Divino Criadô\\
Com o seu sabê profundo\\
Lhe dando o esboço e o nome\\
Num momento fez o home\\
E ao mesmo entregou o mundo.

E lhe disse: Esta riqueza\\
É para tu governá\\
Toda esta imensa grandeza\\
O espaço, a terra e o má,\\
Vou te dá inteligença\\
Mode tratá da ciença,\\
Mas com a tua noção\\
Use do grau de iguardade\\
Não faça prevessidade\\
Não pressiga teu irmão.

Nunca deixe te inludi\\
Com ôro, prata e briante,\\
O que não quisé pra ti\\
Não dê ao teu simiante,\\
Vivendo nesta atitude\\
Será o dono da virtude\\
Que é um dom da providença,\\
Para bem feliz vivê\\
E tudo isto resorvê\\
Trinta ano é tua inxistença.

O home inchou de vaidade\\
E com egoirmo lôco\\
Gritou logo: Majestade\\
Trinta ano pra mim é pôco,\\
Vinte ano, o burro injeitou\\
Me dá para mim Senhô\\
Mode eu pudê sê feliz,\\
Dez o cachorro não quis\\
Me dá que eu faço sessenta\\
E ainda mais me destaco\\
E quero os dez do macaco\\
Mode eu compretá setenta.

O nosso Pai Soberano\\
Atendeu o pedido seu;\\
Vive o home até trinta ano\\
A idade que Deus lhe deu\\
De trinta até os cinqüenta\\
A sua tarefa aumenta\\
Veve cheio de cancêra\\
De famia carregado\\
Levando os costá pesado\\
E é burro nem que não quêra.

De cinqüenta até sessenta\\
Já não pode mandá brasa,\\
Aqui e aculá se assenta\\
Botando sentido a casa,\\
Pruquê já força não tem,\\
Veve neste vai e vem\\
Do cargo que ele assumiu\\
Se encronta liberto e forro\\
Tá na vida do cachorro\\
Que ele mesmo a Deus pediu.

De sessenta até setenta\\
Já com a cara enrusgada,\\
Constantemente freqüenta\\
Os prédio da fiarada\\
Fazendo graça e carinho\\
Para turma de netinho,\\
Beija neto e abraça neto\\
Sentado mesmo no chão\\
E naquela arrumação\\
É um macaco compreto.

Rico, orguioso, profano,\\
Rifrita no bem comum\\
Veja os dereitos humano\\
A razão de cada um\\
Em vez de fraternidade\\
Praque tanta vaidade\\
Orguioso inchendo o saco?\\
Este inxempro tá dizendo\\
Que os home termina sendo\\
Burro, cachorro e macaco.
\end{verse}

\chapter{O alco e a gasolina}

\begin{verse}
Neste mundo de pecado\\
Ninguém qué vivê sozinho\\
Quem viaja acompanhado\\
Incurta mais o caminho\\
Tudo que no mundo existe\\
Se achando sozinho é triste,\\
O alco vivia só\\
Sem ninguém lhe querê bem\\
E a gasolina também\\
Vivia no caritó.

Alco tanto sofreu\\
Sua dura e triste sina\\
Até que um dia ofreceu\\
Seu amô à gasolina\\
Perguntou se ela queria\\
Ele em sua companhia,\\
Pois andava aperriado\\
Era grande o padecê\\
Não podia mais vivê\\
Sem companhêra ao seu lado.

Disse ela: dou-lhe a resposta\\
Mas fazendo uma proposta\\
Sei que de mim você gosta\\
E eu não lhe acho tão feio\\
Porém eu sou moça fina,\\
Sou a prenda gasolina\\
Bem recatada, granfina\\
E gosto muito de asseio.

Se você não é nojento\\
É grande o contentamento\\
E tarvez meu sofrimento\\
Da solidão eu arranque,\\
Nós não vamo nem casá\\
Do jeito que o mundo tá\\
Nós dois vamo é se juntá\\
E morá dentro do tanque.

Se quisé me acompanhá,\\
No tanque vamo morá\\
E os apusento zelá\\
Com carinho e com amô,\\
Porém lhe dou um conseio\\
Não vá fazê papé feio\\
Quero limpeza e asseio\\
Dentro do carboradô.

Se o meu amô armeja\\
E andá comigo deseja,\\
É necessário que seja\\
Limpo, zeladô e esperto,\\
Precisa se controlá,\\
Veja que eu sou minerá\\
E você é vegetá,\\
Será que isto vai dá certo?

Disse o alco: meu benzinho\\
Eu não quero é tá sozinho\\
Pra gozá do teu carinho\\
Todo sacrifiço faço,\\
Na nossa nova aliança\\
Disponha de confiança\\
Com a minha substança\\
Eu subo até no espaço.

Quero é sê feliz agora\\
Morá onde você mora\\
Andá pelo mundo afora\\
E a minha vida gozá,\\
Entre nós não há desorde\\
Basta que você concorde\\
Nós se junta com as orde\\
Da senhora Petrobá.

Tudo o alco prometia\\
Queria porque queria\\
Na Petrobá neste dia\\
Houve uma festa danada,\\
A Petrobá ordenou\\
Um ao outro se entregou\\
E o querosene chorou\\
Vendo a parenta amigada.

Porém depois de algum dia\\
Começou grande narquia,\\
O que o alco prometia\\
Sem sentimento negou,\\
Fez uma ação traiçoêra\\
Com a sua companhêra\\
Fazendo a maió sujêra\\
Dentro do carboradô.

Fez o alco uma ruína,\\
Prometeu à gasolina\\
Que seguia a diciprina\\
Mas não quis lhe obedecê,\\
Como o cabra embriagado\\
Descuidado e deslêxado,\\
Dêxava tudo melado,\\
Agúia, bóia e giclê.

A gasolina falava\\
E a ele aconseiava,\\
Mas o alco não ligava,\\
Inchia o saco a zombá\\
Lhe respondendo, eu não ligo,\\
Se achá que vivê comigo\\
Tá sendo grande castigo\\
Se quêxe da Petrobá.

E assim ele permanece\\
No carro a tudo aborrece,\\
Se a gasolina padece\\
O chofé também se atrasa\\
Hoje o alco veve assim\\
Do jeito do cabra ruim\\
Que bebe no butiquim\\
E vai vomitá na casa.
\end{verse}

\chapter{Aos irmãos Aniceto}

\begin{verse}
Vocês, irmão Aniceto,\\
Com a banda cabaçá\\
É um conjunto compreto\\
Que faz tudo se alegrá,\\
Com este turututu\\
Já andaro pelo Sul\\
E foro apoiado lá,\\
Também já foro apraudido\\
E munto bem recebido\\
No Distrito Federá.

De prazê tudo parpita\\
Quando vocês toca e dança\\
Eu sinto que ressuscita\\
O meu tempo de criança,\\
Nesta idade prazentêra\\
Na festa da Padroêra\\
Havia no meu lugá\\
Um parrapapá sodoso,\\
Atraente e milagroso\\
De uma banda cabaçá.

Quando escuto sastisfeito\\
Os Aniceto tocando\\
Sinto dentro do meu peito\\
O coração balançando,\\
Balançando de sodade\\
Daquela felicidade\\
Que eu vi desaparecer,\\
Para quem sabe jurgá\\
É gostoso rescordá\\
Aquilo que dá prazê.

Destas coisa populá\\
Que a gente preza e qué bem\\
Uma das mais principá\\
De todas que o Crato tem,\\
Com grande capacidade\\
Que merece de verdade\\
Proteção, amô e afeto\\
É a bela inzecução\\
Dos três artistas irmão\\
De sobrenome Aniceto.

O prazê não é pequeno\\
Quando tá mandando brasa\\
Esta turma de moreno\\
Que pertence à mesma casa,\\
O trancilim é dançá\\
Gingando daqui pra lá,\\
Outro vem de lá praqui;\\
É um trancilim compreto;\\
Viva os irmão Aniceto\\
Gulora do Cariri.
\end{verse}

\chapter{Brosogó, Militão e o Diabo}

\begin{verse}
O melhor da nossa vida\\
É paz, amor e união,\\
E em cada semelhante\\
A gente vê um irmão\\
E apresentar para todos\\
O papel de gratidão.

Quem fez um grande favor,\\
Mesmo desinteressado,\\
Por onde quer que ele ande\\
Leva um tesouro guardado\\
E um dia sem esperar\\
Será bem recompensado.

A gratidão é virtude\\
Do mais sublime valor,\\
Neste singelo folheto\\
Eu vou mostrar ao leitor\\
Que até o diabo agradece\\
A quem lhe faz um favor.

Em um dos nossos Estados\\
Do Nordeste brasileiro\\
Nasceu Chico Brosogó,\\
Era ele um miçangueiro\\
Que é o mesmo camelô\\
Lá no Rio de Janeiro.

O Brosogó era ingênuo,\\
Não tinha filosofia\\
Mas tinha de honestidade\\
A maior sabedoria\\
Sempre vendendo ambulante\\
A sua mercadoria.

Em uma destas viagens,\\
Numa certa região\\
Foi vender mercadoria\\
Na famosa habitação,\\
De um fazendeiro malvado\\
Por nome de Militão.

O ricaço Militão\\
Vivia a questionar,\\
Toda sorte de trapaça\\
Era capaz de inventar,\\
Vendo assim desta maneira\\
Sua riqueza aumentar.

Brosogó naquele prédio\\
Não apurou um tostão.\\
E como na mesma casa\\
Não lhe ofereceram pão\\
Comprou meia dúzia de ovos\\
Para sua refeição

Quando a meia dúzia de ovos\\
O Brosogó foi pagar\\
Faltou dinheiro miúdo\\
Para a paga efetuar\\
E ele entregou uma nota\\
Para o Militão trocar.

O rico disse: eu não troco,\\
Vá com a mercadoria,\\
Qualquer tempo você vem\\
Me pagar esta quantia,\\
Mas peço que seja exato\\
E aqui me apareça um dia.

Brosogó agradeceu\\
E achou o papel importante,\\
Sem saber que o Militão\\
Estava naquele instante\\
Semeando uma semente\\
Para colher mais adiante.

Voltou muito satisfeito\\
Na sua vida pensando,\\
Sempre arranjando fregueses\\
No lugar que ia passando,\\
Vendo sua boa sorte\\
Melhorar de quando em quando.

Brosogó no seu comércio\\
Tinha bons conhecimentos,\\
Possuía com os lucros\\
Daqueles seus movimentos,\\
Além de casa e terrenos\\
Meia dúzia de jumentos.

De ano em ano ele fazia\\
Naquele seu patrimônio\\
Festejo religioso\\
No dia de Santo Antônio\\
Por ser o aniversário\\
Do seu feliz matrimônio.

No festejo oferecia\\
Vela para São João,\\
Santo Ambrósio, Santo Antônio,\\
São Cosmo e São Damião,\\
Para ele qualquer santo\\
Dava a mesma proteção.

Vela para Santa Inês\\
E para Santa Luzia,\\
São Jorge e São Benedito,\\
São José e Santa Maria,\\
Até que chegava a última\\
Das velas que possuía.

Um certo dia voltando\\
Aquele bom sertanejo\\
Da viagem lucrativa,\\
Com muito amor e desejo\\
Trouxe uma carga de velas\\
Para queimar no festejo.

A casa naquela noite\\
Estava um belíssimo encanto,\\
Se viam velas acesas\\
Brilhando por todo canto\\
Porém sobraram três velas\\
Por faltar nome de santo.

Era linda a luminária\\
O quadro resplandecente\\
E o caboclo Brosogó\\
Procurava impaciente\\
Mas nem um nome de santo\\
Chega na sua mente.

Disse consigo: o diabo\\
Merece vela também,\\
Se ele nunca me tentou\\
Para ofender a ninguém\\
Com certeza me respeita,\\
Está me fazendo o bem.

Se eu fui um menino bom,\\
Fui também um bom rapaz\\
E hoje sou pai de família\\
Gozando da mesma paz,\\
Vou queimar estas três velas\\
Em tenção do satanaz.

Tudo aquilo Brosogó\\
Fez com naturalidade,\\
Como o justo que apresenta\\
Amor e fraternidade\\
E as virtudes preciosas\\
De um coração sem maldade.

Certo dia ele fazendo\\
Severa reflexão,\\
Um exame rigoroso\\
Sobre a sua obrigação,\\
Lhe veio na mente os ovos\\
Que devia ao Militão.

Viajou muito apressado\\
No seu jumento baixeiro\\
Sempre atravessando rio\\
E transpondo taboleiro,\\
Chegou no segundo dia\\
Na casa do trapaceiro.

Foi chegando e desmontando\\
E logo que deu bom dia\\
Falou para o coronel\\
Com bastante cortesia:\\
Venho aqui pagar os ovos\\
Que fiquei devendo um dia.

O Militão muito sério\\
Falou para o Brosogó:\\
Para pagar esta dívida\\
Você vai ficar no pó,\\
Mesmo que tenha recurso\\
Fica pobre como Jó.

Me preste bem atenção\\
E ouça bem as razões minhas:\\
Aqueles ovos no chôco\\
Iam tirar seis pintinhas,\\
Mais tarde as mesmas seriam\\
Meia dúzia de galinhas.

As seis galinha botando\\
Veja a soma o quanto dá\\
São quatrocentos e oitenta,\\
Ninguém me reprovará,\\
Galinha aqui faz postura\\
De oitenta ovos pra lá.

Preste atenção Brosogó,\\
Sei que você não censura,\\
Veja que grande vantagem,\\
Veja que grande fartura\\
E veja o meu resultado\\
Só na primeira postura.

Dos quatrocentos e oitenta\\
Podia a gente tirar\\
Dos mesmos cento e cinqüenta\\
Para no chôco aplicar,\\
Pois basta só vinte e cinco\\
Que é pra o ovo não gorar.

Os trezentos e cinqüenta\\
Que era a sobra eu vendia\\
Depressa sem ter demora,\\
Por uma boa quantia,\\
Aqui, procurando ovos\\
Temos grande freguesia.

Dos cento e cinqüenta ovos,\\
Sairiam com despacho\\
Cento e cinqüenta pintinhas\\
Pois tenho certeza e acho\\
Que aqui no nosso terreiro\\
Não se cria pinto macho.

Também não há prejuízo\\
Posso falar pra você\\
Que maracajá e raposa\\
Aqui a gente não vê\\
Também não há cobra preta,\\
Gavião nem saruê.

Aqui de certas moléstias\\
A galinha nunca morre\\
Porque logo a medicina\\
Com urgência se recorre\\
Se o gôgo se manifesta\\
A empregada socorre.

Veja bem, seu Brosogó\\
O quanto eu posso ganhar\\
Em um ano e sete meses\\
Que passou sem me pagar,\\
A conta é de tal maneira\\
Que eu mesmo não sei somar.

Vou chamar um matemático\\
Pra fazer o orçamento,\\
Embora você não faça\\
De uma vez o pagamento,\\
Mesmo com mercadoria,\\
Terreno, casa e jumento.

Porém tenha paciência,\\
Não precisa se queixar,\\
Você acaba o que tem,\\
Mas vem comigo morar\\
E aqui, parceladamente,\\
Acaba de me pagar.

E se achar que estou falando\\
Contra sua natureza,\\
Procure um advogado,\\
Pra fazer sua defesa,\\
Que o meu já tenho e conto\\
A vitória com certeza.

Meu advogado é\\
Um doutor de posição\\
Pertence a minha política\\
E nunca perdeu questão\\
E é candidato a prefeito\\
Para a futura eleição.

O coronel Militão\\
Com orgulho e petulância\\
Deixou o pobre Brosogó\\
Na mais dura circunstância,\\
Aproveitando do mesmo\\
Sua grande ignorância.

Quinze dias foi o prazo\\
Para o Brosogó voltar\\
Presente ao advogado\\
Um documento assinar\\
E tudo que possuía\\
Ao Militão entregar.

O pobre voltou bem triste\\
Pensando, a dizer consigo:\\
Eu durante a minha vida\\
Sempre fui um grande amigo,\\
Qual será o meu pecado\\
Para tão grande castigo!

Quando ia pensando assim\\
Avistou um cavaleiro\\
Bem montado e bem trajado\\
Na sombra de um juazeiro,\\
O qual com modos fraternos\\
Pergunto ao miçangueiro!

Que grande tristeza é esta!\\
Que você tem Brosogó?\\
O seu semblante apresenta\\
Aflição, pesar e dó,\\
Eu estou ao seu dispor,\\
Você não sofrerá só.

Brosogó lhe contou tudo\\
E disse por sua vez\\
Que o coronel Militão\\
O trato com ele fez\\
Para as dez horas do dia\\
Na data quinze do mês.

E disse o desconhecido:\\
Não tenha má impressão,\\
No dia quinze eu irei\\
Resolver esta questão\\
Lhe defender da trapaça\\
Do ricaço Militão.

Brosogó foi para casa\\
Alegre sem timidez,\\
O que o homem lhe pediu\\
Ele satisfeito fez\\
E foi cumprir o seu trato\\
No dia quinze do mês.

Quando chegou encontrou\\
Todo povo aglomerado\\
Ele entrando deu bom dia\\
E falou bem animado\\
Dizendo que também tinha\\
Arranjado um advogado.

Marcou o relógio dez horas\\
E sem o doutor chegar\\
Brosogó entristeceu\\
Silencioso a pensar
E o povo do Militão\\
Do coitado a criticar.

Os puxa-sacos do rico\\
Com ares de mangação\\
Diziam ao miçangueiro\\
Vai-se arrasar na questão\\
Brosogó vai pagar caro\\
Os ovos do Militão.

Estavam pilheriando\\
Quando se ouviu um tropel,\\
Era um senhor elegante\\
Montado no seu corcel\\
Exibindo em um dos dedos\\
O anel de bacharel.

Chegando disse aos ouvintes:\\
Fui no trato interrompido\\
Para cozinhar feijão\\
Porque muito tem chovido\\
E o meu pai em seu roçado\\
Só planta feijão cozido.

Antes que o desconhecido\\
Com razão se desculpasse,\\
Gritou o outro advogado:\\
Não desonre a nossa classe\\
Com essa grande mentira!\\
Feijão cozido não nasce.

Respondeu o cavaleiro:\\
Esta mentira eu compus\\
Para fazer a defesa\\
É ela um foco de luz\\
Porque o ovo cozinhado\\
Sabemos que não produz.

Assim que o desconhecido\\
Fez esta declaração,\\
Houve um grande silêncio na sala,\\
Foi grande a decepção\\
Para o povo da política\\
Do coronel Militão.

Onde a verdade aparece\\
A mentora é destruída,\\ %mentira?
Foi assim desta maneira\\
Que a questão foi resolvida\\
E o candidato político\\
Ficou de crista caída.

Mentira contra mentira\\
Na reunião se deu\\
E foi por este motivo\\
Que a verdade apareceu,\\
Somente o preço dos ovos\\
O Militão recebeu.

Brosogó agradecendo\\
O favor que recebia,\\
Respondeu o cavaleiro,\\
Era eu quem lhe devia\\
O valor daquelas velas\\
Que me ofereceu um dia.

Eu sou o diabo a quem todos\\
Chamam de monstro ruim,\\
E só você neste mundo\\
Teve a bondade sem fim\\
De um dia queimar três velas\\
Oferecidas a mim.

Quando disse estas palavras\\
No mesmo instante saiu\\
Adiante deu um pipoco\\
E pelo espaço sumiu\\
Porém pipoco baixinho\\
Que o Brosogó não ouviu.

Caro leitor nesta estrofe\\
Não queira zombar de mim,\\
Ninguém ouviu o estouro\\
Mas juro que foi assim\\
Pois toda história do diabo\\
Tem um pipoco no fim.

Sertanejo, este folheto\\
Eu quero lhe oferecer,\\
Leia o mesmo com cuidado\\
E saiba compreender,\\
Encerra muita mentira\\
Mas tem muito o que aprender.

Bom leitor tenha cuidado,\\
Vivem ainda entre nós\\
Milhares de Militões\\
Com instinto feroz\\
Com trapaças e mentiras\\
Perseguindo os Brosogós.
\end{verse}

\chapter{Rogando praga}

\begin{verse}
Dizia o velho Agostinho\\
Que este mundo é cheio de arte\\
E se encontra em toda parte\\
Pedaço de mau caminho\\
Um pessoal meu vizinho,\\
Sem amor e sem moral,\\
Atrás de fazer o mal,\\
Para feijão cozinhar,\\
Começaram a roubar\\
As varas do meu quintal.

Toda noite e todo dia\\
Iam as varas roubando\\
E eu já não suportando\\
Aquela grande anarquia\\
Pois quem era eu não sabia\\
Pra poder denunciar,\\
Com aquele grande azar\\
Vivia de saco cheio,\\
Até que inventei um meio\\
Pra do roubo me livrar.

Eu dei a cada freguês\\
Com humildade, o perdão,\\
E lancei a maldição\\
Em quem roubasse outra vez\\
E com muita atividez\\
Na minha pena peguei,\\
Umas estrofes e rimei\\
Sobre as linhas de uns papéis\\
Rogando pragas cruéis\\
E lá na cerca botei.

Deus permita que o safado,\\
Sem vergonha, ignorante,\\
Que roubar de agora em diante\\
Madeira no meu cercado,\\
Se veja um dia atacado\\
Com um cancro no toitiço,\\
Toda espécie de feitiço\\
Em cima do mesmo caia\\
E em cada dedo lhe saia\\
Um olho de panariço.

O santo Deus de Moisés\\
Lhe mande bexiga rôxa,\\
Saia carbúnculo na côxa,\\
Cravo na sola dos pés,\\
Entre os incômodos cruéis\\
Da doença hidrofobia\\
Iterícia e anemia,\\
Tuberculose e diarréia\\
E a lepra da morféia\\
Seja a sua companhia.

Deus lhe dê o reumatismo\\
Com a sinusite crônica,\\
A sezão, o impaludismo\\
E os ataques da bubônica,\\
Além de quatro picadas\\
De quatro cobras danadas\\
Cada qual a mais cruel\\
E de veneno fatal\\
A urutu, a coral,\\
Jararaca e cascavel.

Eu já perdoei bastante\\
Os que puderam roubar,\\
Para ninguém censurar\\
Que sou muito extravagante\\
Mas de agora por diante,\\
Ninguém será perdoado,\\
Deus queira que um cão danado\\
Um dia morda na cara\\
De quem roubar uma vara\\
Na cerca do meu cercado.

E o que não ouvir o rogo\\
Que faço neste momento\\
Tomara que tenha aumento\\
Como correia no fogo,\\
Dinheiro em mesa de jogo\\
E cana no tabuleiro\\
E no dia derradeiro\\
A vela pra sua mão,\\
Seja um pequeno tição\\
De vara de marmeleiro.
\end{verse}

\chapter{Mãe de verdade}

\begin{verse}
Boa noite, amigo Jacó,\\
Eu não lhe disse que vinha?\\
--- Boa noite! Veio só?\\
Proquê não trôxe Zefinha?
--- Zefinha não veio não\\
Ficou mamentando o João,\\
Gordo que tá um cartuxo,\\
O menino tá dum jeito\\
Que quando agarra no peito\\
Só larga quando enche o bucho.

Migué, tudo isto é o amô,\\
O que ela faz com o João\\
Não tá fazendo favô,\\
É a sua obrigação,\\
Mãe qué não dá de mamá,\\
Não que bem, não sabe amá\\
Nem merece confiança,\\
Faz o papé de ladrona,\\
Proquê dos peito ela é dona\\
Mas o leite é da criança.

Depois do fio nascê\\
O leite que os peito têm\\
Pertence todo ao bebê\\
É dele e de mais ninguém,\\
Toda mãe que não mamenta\\
Prá mim nada representa\\
Pois comete um grande erro\\
É disamorosa e fraca,\\
Eu comparo com vaca\\
Quando ela enjeita o bizerro.

Você sabe que Zabé\\
Já é mãe de quatro fio,\\
Tudo mamou e tudo é\\
Gordo, ribusto e sadio,\\
Este que ali vai passando\\
Correndo alegre e brincando,\\
Ainda não quis dexá,\\
Tem três ano este Luiz\\
E ainda pedindo diz:\\
--- Mamãe, eu quero mamá.

E Zabé ali demora\\
E adulando e lhe bejando\\
Bota os seus peito prá fora\\
E o Luiz fica mamentando,\\
Mama num, noutro depois,\\
O certo é que em todos dois\\
Mama o tanto que ele qué,\\
O tanto que tem vontade,\\
Esta é que é mãe de verdade,\\
Esta sabe sê muié.

Mas há tantas por aí\\
Ingrata e sem piedade\\
Só aprendeu produzi,\\
Mas não é mãe de verdade,\\
Mãe de verdade é aquela\\
Que bêja, mamenta e zela\\
O seu fiinho bonito\\
E depois do nascimento\\
O mais mió alimento\\
Como o dotô já tem dito.

Por doença ou por defeito\\
Há mãe que, pobre coitada,\\
Não cria leite nos peito,\\
Esta já tá perdoada\\
Esta não pode, é doente\\
Cria sei fio inocente\\
Com este leite que vem\\
Impacotado nas lata\\
Uma coisa feia e chata\\
Que eu não sei que gosto tem.

--- Jacó tudo isto é ixato,\\
Você disse uma verdade,\\
As nossa muié dos mato\\
Não é como as da cidade,\\
Às nossa muié dos mato\\
Às vez veve no matrato\\
Magra iguá uma rabeca,\\
No fejão e no muncunzá\\
Mas dêxa o fio mamá\\
Até quando o leite seca.

E as muié lá da cidade,\\
Não digo com todas não,\\
Mas porém mais da metade\\
Não faz esta obrigação,\\
Não faz do fio o disejo,\\
Come carne, arroz e queijo,\\
Tem corpo gordo e sadio\\
E às vez com cada peitão\\
Que parece dois mamão,\\
Mas nega leite ao seu fio.

Isto é cronta a Natureza,\\
Cronta a lei do Criadô,\\
Eu tenho prena certeza\\
Que o Menino Deus mamou\\
Na Santa Mãe potretora\\
E agora estas pecadora\\
Não qué o inxempro tomá\\
E tem delas que até caça\\
Remédio pelas farmaça\\
Pro mode o leite secá.

Sê mãe é um grande brio,\\
Sê mãe é coisa subrime,\\
Mas pra negá leite ao fio\\
Sê mãe é um grande crime,\\
Às vez o bebê na cama\\
Chorando o peito recrama\\
E ela não liga o coitado,\\
A mãe que faz deste jeito\\
Devia nascê sem peito\\
Ô com os peito alejado.

--- Colega vamo dexá\\
Nós já conversemo munto\\
Pode arguém inguinorá\\
Não gostá do nosso assunto\\
E depois andá contando\\
Que quem tava assim falando\\
Era o Jacó e o Migué,\\
Vamo dexá, meu colega,\\
A tesoura já tá cega\\
De tanto cortá muié.

--- Jacó, se língua é tesoura,\\
Cortemo sem dizê pêta,\\
Cortemo péia de lôra,\\
De morena, branca e preta,\\
Nós ataquemo as muié\\
Seja de lugá quarqué,\\
Do Brasi até a Russa,\\
Seja feia ou seja bela,\\
Mas só ataquemo aquela\\
Que lhe assenta a carapuça.
\end{verse}

\chapter{Eu e a pitombêra}

\begin{verse}
Aqui dentro do meu peito\\
O meu coração idoso\\
Tá sendo do mesmo jeito\\
Do relojo priguiçoso\\
Que no compasso demora\\
Marcando fora das hora\\
E que tanto se aperreia\\
Andando fora da tria\\
Que quando dá meio-dia\\
Ele tá nas dez e meia.

Mas porém meu coração\\
Naquela vida passada\\
Já teve a mesma feição\\
Da pitombêra copada\\
Que havia no meu terrêro\\
Onde os passo prazentêro\\
Cantava era mesmo que escutá\\
Cantava com tanto amô\\
Que era mesmo que escutá\\
Um coro celestiá\\
Do anjo do Criadô.

Parece que os passarinho\\
Para aquele verde abrigo\\
Convidava seus vizinho,\\
Seus parente e seus amigo\\
Crescendo os nurmo das ave\\
Cada quá mais agradave\\
Sobre a copa hospitalêra,\\
Quanto mais dia passava\\
Mais passarinho chegava\\
Na copa da pitombêra.

As ave cantava mansa\\
Os mais sonoroso hino\\
As vez me vinha à lembrança\\
Que Deus nosso Pai Divino\\
Mandou um dia que um santo\\
Remexesse no encanto\\
Que havia no mundo intêro\\
E depois que inxaminasse\\
O mais bonito que achasse\\
Botasse no meu terrêro.

A pintombêra frondosa\\
Se ria toda contente\\
Uvindo as voz sonorosa\\
Daqueles musgo inocente,\\
Mas quem repara conhece\\
Que as ave também padece,\\
Triste coisa aconteceu,\\
Tudo aquilo se acabou,\\
A pitombêra secou,\\
A pitombêra morreu.

A pitombêra morreu\\
Depois que tanto gozou,\\
A natureza lhe deu\\
E a natureza tomou\\
E as suas fôia caindo\\
As ave fôro fugindo\\
Atrás doutro paradêro,\\
Se acabou a poesia,\\
Aquela grande alegria\\
Que havia no meu terrêro.

Da frondosa pitombêra\\
Ficou o isqueleto horrendo,\\
Coisa assombrosa e agorêra\\
Até mesmo parecendo\\
Com um bocado de braço\\
Apontando para o espaço,\\
Ou um bocado de frecha\\
E o pica-pau sem piedade\\
Martelando sobre a grade\\
Tirando broca nas brecha.

Pitombêra, pitombêra,\\
Teu fracasso continua,\\
Nesta vida passagêra\\
Minha sorte é como a tua,\\
Pitombêra distruída,\\
Eu também na minha vida\\
Tive amô e tive carinho,\\
Eu já fui do mesmo jeito,\\
Aqui dentro do meu peito\\
Cantava meus passarinho.

As aves, sastisfação,\\
Prazê, aventura, alegria,\\
Dentro do meu coração\\
Grogeava noite e dia,\\
A felicidade, o sonho\\
E o riso todo risonho\\
Junto com outros irmão\\
Formava um côro incelente\\
Cantando constantimente\\
No conjuto da inlusão.

Um quadro cheio de lume\\
Briava na minha sorte\\
Como os lindo vagalume\\
Nas noite escura do norte,\\
Uvindo os mais belo som\\
Tudo pra mim era bom\\
O meu gozo era sem fim,\\
Toda sorte de harmonia\\
Que neste mundo existia\\
Cantava dentro de mim.

Porém, como a feia bruxa\\
Quando qué fazê mardade\\
Foi chegando as fôia mucha\\
Do jardim da mocidade,\\
Depois as fôia caíro\\
E os passarinho fugiro,\\
Onde tudo era beleza\\
Se arranchou sem piedade\\
O passarinho sodade\\
Cantando a minha tristeza.

Vendo desaparecer\\
Aquele belo istribio\\
Começou meu padecer,\\
Meu mundo ficou vazio,\\
Sem dó e sem compaxão\\
O conjunto da inluzão\\
Do meu peito se afastou,\\
Hoje só resta a lembrança,\\
Até o passo esperança\\
Bateu as asa e voou.
\end{verse}

\chapter{Inleição direta 1984}

\begin{verse}
Bom camponês e operaro\\
A vida tá de amargá\\
O nosso estado precaro\\
Não há quem possa aguentá\\
Neste espaço dos vinte ano\\
Que a gente entrou pelo cano\\
A confusão é compreta\\
Mode a coisa miorá\\
Nós vamo bradá e gritá\\
Pelas inleição direta.

Camponês, meu bom irmão\\
E operaro da cidade,\\
Vamo uni as nossas mão\\
E gritá por liberdade\\
Levando na mesma pista\\
Os estudante, os artista\\
E meus colega poeta\\
Vamo todos reunido\\
Fazê o maió alarido\\
Pelas inleição direta.

Vamo cada companhêro\\
Com nosso potresto forte\\
Por este país intêro,\\
Leste, oeste, sul e norte\\
Com as inleição direta\\
Nós vamo por outra meta\\
De uma forma deferente,\\
Esta marcha tá puxada\\
E esta canga tá pesada\\
Não há cangote que aguente.

Senhora dona de casa\\
Lavadêra e cozinhêra,\\
É preciso mandá brasa,\\
Ingrossá nossa filêra,\\
Vamo abalá toda massa\\
Derne o campo até a praça,\\
Agora ninguém se aqueta,\\
Vamo lutá fortimente\\
E elegê um presidente\\
Com as inleição direta.

Se o povo veve sujeito\\
Sem tê a quem se quexá,\\
É preciso havê um jeito\\
Pois deste jeito não dá,\\
Cadê a democracia\\
Que o pudê tanto irradia\\
Nas terra nacioná?\\
Tudo isto é demagogia\\
Quem já viu democracia\\
Sem direito de votá?

Democracia é justiça\\
Em favor do bem comum\\
Sem trapaça e sem maliça\\
Defendendo cada um,\\
O povo tá sem sossego\\
Com a fome e o desemprego\\
Arrochando o brabicacho,\\
Será que o Brasi de cima\\
Não tá vendo o triste crima\\
Que tem no Brasi de baixo?

Seu dotô, seu deputado,\\
Seu ministro e senadô\\
Repare que o nosso estado\\
É mesmo um drama de horrô;\\
Se vossimicê baixasse\\
E uns dez ano aqui passasse\\
Ferido da mesma seta,\\
Fazia assim com nós\\
Gritando na mesma voz\\
Queremo inleição direta!

Isto qu eu digo é exato\\
É uma verdade certa\\
Pois só o que carça o sapato\\
Sabe onde é que o mesmo aperta,\\
Nosso país invejado\\
Tá todo desmantelado\\
O que observa descobre\\
E com certeza tá vendo\\
A crasse pobre morrendo\\
E a média ficando pobre.

Nestes verso que rimei\\
Disse apenas a verdade\\
Eu aqui não afrontei\\
A nenhuma outoridade\\
Quem fala assim deste jeito\\
Defendendo seus dereito\\
Todos já sabe quem é,\\
É um poeta do povo\\
Véio do coração novo\\
Patativa do Assaré.
\end{verse}

\chapter{O agregado e o operário}

\begin{verse}
Sou matuto do Nordeste\\
Criado dentro da mata,\\
Caboclo cabra da peste,\\
Poeta cabeça chata,\\
Por ser poeta roceiro\\
Eu sempre fui companheiro\\
Da dor, dia mágoa e do pranto\\
Por isto, por minha vez\\
Vou falar para vocês\\
O que é que eu sou e o que canto.

Sou poeta agricultor\\
Do interior do Ceará\\
A desdita, o pranto e a dor\\
Canto aqui e canto acolá\\
Sou amigo do operário\\
Que ganha um pobre salário\\
E do mendigo indigente\\
E canto com emoção\\
O meu querido sertão\\
E a vida de sua gente.

Procurando resolver\\
Um espinhoso problema\\
Eu procuro defender\\
No meu modesto poema\\
Que a santa verdade encerra,\\
Os camponeses sem terra\\
Que o céu deste Brasil cobre\\
E as famílias da cidade\\
Que sofrem necessidade\\
Morando no bairro pobre.

Vão no mesmo itinerário\\
Sofrendo a mesma opressão\\
Nas cidades o operário\\
E o camponês no sertão,\\
Embora um do outro ausente\\
O que um sente o outro sente\\
Se queimam na mesma brasa\\
E vivem na mesma guerra\\
Os agregados sem terra\\
E os operários sem casa.

Operário da cidade,\\
Se você sofre bastante\\
A mesma necessidade\\
Sofre o seu irmão distante\\
Levando vida grosseira\\
Sem direito de carteira\\
Seu fracasso continua,\\
É grande martírio aquele,\\
A sua sorte é a dele\\
E a sorte dele é a sua.

Disto eu já vivo ciente,\\
Se na cidade o operário\\
Trabalha constantemente\\
Por um pequeno salário,\\
Lá nos campos o agregado\\
Se encontra subordinado\\
Sob o jugo do patrão\\
Padecendo vida amarga,\\
Tal qual o burro de carga\\
Debaixo da sujeição.

Camponeses meus irmãos\\
E operários da cidade,\\
É preciso dar as mãos\\
Cheios de fraternidade,\\
Em favor de cada um\\
Formar um corpo comum\\
Praciano e camponês\\
Pois só com esta aliança\\
A estrela da bonança\\
Brilhará para vocês.

Uns com os outros se entendendo\\
Esclarecendo as razões\\
E todos juntos fazendo\\
Suas reivindicações\\
Por uma democracia\\
De direito e garantia\\
Lutando de mais a mais,\\
São estes os belos planos\\
Pois nos direitos humanos\\
Nós todos somos iguais.
\end{verse}

\chapter{Acuado}

\begin{flushright}
\emph{(Canto de um povo acuado}\\
\emph{ao poeta Patativa do Assaré)}
\end{flushright}

\begin{verse}
melhor seria silenciar\\
diante de um canto de medo\\
quando as palavras saem filtradas\\
do pulmão de um povo morto

em cada gesto dos pássaros\\
chamaremos camarada\\
o povo lírico nordestino,\\
onde a foice doida\\
decapitou a mãe e o filho\\
num desespero de retrato 3x4\\
daquela gente rude de alpargatas currulepe\\
e ``bornó'' cheio de sonhos\ldots{}

acuado no canto da sala\\
bem na esquina do mundo,\\
esmaga as tristezas ouvindo versos\\
do Patativa do Assaré\\
que canta no rádio a pilha\\
o retorno do príncipe misterioso\\
que veio à terra dar o grito\\
de Independência ou Morte\\
do Terceiro Mundo

Patativa\ldots{}\\
como tínhamos vergonha do Ceará\\
daquelas estórias de seca\\
diante do eixo da terra\\
sempre azul\\
tínhamos vergonha dos grampos\\
das cercas de arame farpado\\
onde de um lado --- era só jurema\\
e do outro lado --- era só jurema

figuras oprimidas\\
acuadas\\
no canto da sala,\\
bem na esquina do mundo,\\
esmagam tristezas ouvindo versos\\
do Patativa que canta\\
o canto deste povão\\
de uma terra onde Deus\\
plantou, semeou e colheu\\
a Esperançar.
\end{verse}


\hfill\emph{Cândido B. C. Neto}

\chapter{Ao poeta B.\,C.\,Neto}

\begin{verse}
\textbf{I}

Sei que não tenho sabença\\
Sou um pobre nafabeto\\
Mas vou fazer referença\\
Sobre você, B. C. Neto\\
Proque, seus verso moderno\\
No meu coração fraterno\\
Senti de leve tocá,\\
Pois você canta a cidade\\
Mas também diz a verdade\\
Das nossas coisas de cá.

\textbf{II}

Imbora seja polida\\
Sua bonita linguagem\\
A mesma retrata a vida\\
Da nossa gente servage\\
No meu verso eu tou notando\\
Que você tá me ajudando.\\
Obrigado meu amigo,\\
Pelo trabalho distinto\\
Sentindo aquilo que eu sinto\\
Dizendo aquilo que eu digo.

\textbf{III}

Você bastante aprendeu\\
É jornalista iscritô\\
Mas parece que mexeu\\
Na coisa do interiô,\\
Na poesia que escreve\\
Sabe dizer como veve\\
O caboco do roçado\\
No sofrimento medonho\\
Sempre alimentando um sonho\\
Que nunca vai realizado.

\textbf{IV}

O seu livro eu não me ingano,\\
tem dois temas ispiciá\\
Um tema do meio urbano\\
Ôtro do meio rurá;\\
Mesmo sem letra e sem arte\\
É desta segunda parte\\
Que esta referença faço;\\
Nem que me chame invejoso\\
Do seu livro precioso\\
Tenho também um pedaço.

\textbf{V}

Tem a parte pequenina\\
Do seu livro que me toca,\\
O amargo da quinaquina\\
E o gosto da tapioca\\
Tem o canto magoado\\
Do camponês no roçado\\
Gotejando de suó,\\
Tem o sentimento nobre\\
Do choro de uma mãe pobre\\
Com oito fio em redó.

\textbf{VI}

Canta, amigo, a nossa gente\\
Do nosso meio rurá,\\
Meio munto diferente\\
Do meio da capitá.\\
Imensa alegria tive\\
Lendo os seus verso sensive\\
Cantando a vida sem vida\\
Deste povo abandonado\\
Que veve e morre apoiado\\
Numa esperança perdida.
\end{verse}

