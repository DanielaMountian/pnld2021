\part{\textsc{paratexto}}

\chapter{Uma viagem pela poesia grega antiga}
\hedramarkboth{introdução}{giuliana ragusa}

\begin{flushright}
\textsc{giuliana ragusa}
\end{flushright}

%\section*{Safo revisitada: viagem pela\break poesia grega
%antiga\footnoteInSection{ Amparam estas páginas os seguintes estudos:
%Ragusa (2005, pp.~23--53, 55--78; 2010, pp.~23--53, 55--97).}}
\section{Sobre a autora\footnoteInSection{ Amparam estas páginas os seguintes estudos: Ragusa (2005, pp.~23--53, 55--78; 2010, pp.~23--53, 55--97).}}

\epigraph{Estas mulheres de divinas línguas o Hélicon nutriu --- e o  %\\
rochedo macedônio de Piéria --- com hinos: %\\
Praxila, Mero, Anite eloquente, feminino Homero, %\\
Safo, adorno das lésbias de belos cachos, %\\
Erina, Telesila mui gloriosa e tu, Corina, %\\
o impetuoso escudo de Atena cantando, %\\
Nóssis de feminina língua, e Mirtes, doce de ouvir ---  %\\
todas fazedoras de eternos escritos. %\\
Nove Musas do grande Urano, e nove mesmas %\\
Gaia pariu, para a imperecível alegria dos mortais.}
{\textit{Antologia palatina} (\textit{\textsc{ap}} livro \textsc{ix}, epigrama 26, de Antípatro de Tessalônica,
séculos \textsc{i} a.C.--\textsc{i} d.C.)\footnotemark}
\footnotetext{ Tradução de Ragusa (2005, p. 57) aqui com pequenas alterações. Todas as
traduções neste texto, salvo quando indicado, são minhas.}


\noindent{}Ao contrário do que faz supor o mito --- e quando se trata da poeta a que se
dedica este livro, mito e realidade se confundem sem cessar, mal se
distinguindo entre si ---, Safo não é o único nome feminino da poesia da Grécia
	antiga, mas de sua primeira fase histórica, a arcaica\footnote{ O adjetivo é
	usado no sentido de “antiga, remota”; a era arcaica é, por assim dizer, a mais
	antiga da Grécia antiga, e divide"-se em duas etapas: a arcaica, até \textit{c.}
	550 a.C., e a tardo"-arcaica, \textit{c.} 550--450 a.C.; ver Shapiro (2007, pp.
	1--3) a respeito.} (\textit{c.} 800--480 a.C.). Nascida em 630~a.C., de família
aristocrática, na costeira Êresos, oeste da ilha de Lesbos, ela viveu na
proeminente Mitilene, costa oriental, contemporaneamente ao poeta e guerreiro
Alceu. Ambos são os primeiros poetas lésbios dos quais sobreviveram, para cada
um, corpos de obra substanciosos; suas práticas, porém, se beneficiaram,
ressalta Angus M. Bowie, de uma forte e bem reputada tradição
poética lésbio"-eólica, em que se inserem nomes como os dos célebres citaredos
Terpandro (séculos \textsc{viii}--\textsc{vii} a.C.) e Árion (séculos
\textsc{vii}--\textsc{vi} a.C.), que levaram a
outras geografias do mundo grego, e a dois polos culturais da era arcaica ---
	Esparta e Corinto ---, suas práticas métrico"-musicais.\footnote{ Bowie (1984,
	pp.~7--10).} Mais não podemos dizer,
pois do primeiro há só dois fragmentos de autoria duvidosa, e do segundo, nada
resta. De todo modo, a relevância dessas figuras e o peso que conferiram a uma
tradição lésbio"-eólica de canção bem conhecida e firmada se
fazem sentir na imagem que os antigos projetaram de Terpandro, tido como
inovador da música grega num século \textsc{vii} a.C. de ricas experimentações, e
inventor da lira de sete cordas, algo que a arqueologia prova insustentável,
uma vez que o instrumento era já conhecido no mundo minoico"-micênico, que
antecede o que chamamos “Grécia histórica”. E Árion é dado como o
poeta do ditirambo, canção de forte aspecto narrativo.


%\section*{Em busca de Safo: poeta de Lesbos}

Um papiro encontrado na antiga cidade egípcia de Oxirrinco, próxima a\EP[]
Alexandria, isto nos conta da vida de Safo (\textit{Papiro de Oxirrinco} 1800,
fr.~1, século \textsc{ii} ou início do \textsc{iii} d.C.): seu pai seria Escamandro ou
Escamandrônimo; seus irmãos, Cáraxo, o mais velho, Erígio e Lárico, o mais
novo; sua filha, Cleis, que levaria o nome da mãe de Safo ``foi acusada
por alguns de ser irregular e amante de mulheres”; era feia, mirrada e de
compleição escura. Já o léxico bizantino \textit{Suda}, compilado no século \textsc{x},
no verbete à poeta repete o segundo nome para o pai de Safo, acrescentando à
lista outros sete; reitera o nome Cleis como sendo de sua mãe e de sua filha, e
os de seus três irmãos; diz ainda que Safo teria sido casada; que teria mantido
amizades impuras com jovens meninas, como Átis, e, por isso, adquirido má
reputação; e teria tido pupilas, como Gongila. Alguns desses nomes se registram
na obra de Safo, ou nas fontes que as preservaram. Mas, como se vê, os
testemunhos antigos se retomam uns aos outros, discordando aqui e ali, tornando
mais intricada a rede de inconsistências, e embasando"-se, claramente, na
leitura dos fragmentos de Safo, em circularidade viciosa. O pequeno painel
biográfico composto pelo papiro e pelo léxico é, pois, indigno de confiança,
para dizer o mínimo, mesmo que associado a outros testemunhos.\footnote{ Na
edição bilíngue de Campbell (1994; 1ª ed.: 1982), são arrolados 61 testemunhos
sobre a poeta e sua obra. O volume é, \mbox{comparado} aos testemunhos de outros
poetas, considerável; de seu contemporâneo Alceu, há 27 testemunhos na mesma edição. 
Quando me referir a testemunhos sobre Safo, uso
sempre a compilação de Campbell.}

O mistério, então, persiste e as perguntas que estimula --- e que com frequência
ganham dimensão desproporcional à própria inviabilidade de solução
corroborada em sólidas evidências --- dificilmente podem ser respondidas: quem
foi Safo e que figura portava? Como se fez a poeta? Como construiu e fez
circular suas canções? Como viveu na Lesbos arcaica? Sustenta"-se a imagem da
Safo \textit{lésbica}? Vejamos um pouco do muito que se diz, e do
pouco que podemos dizer.

Embora seja a mais velha das poetas mulheres listadas no epigrama que abre esta
introdução, atuante numa Grécia da oralidade, em quem a escrita ainda estava
por se consolidar como veículo principal de produção artística, Safo é aquela
cujo \textit{corpus} é o mais extenso (cerca de 200 fragmentos, um dos quais é,
na verdade, uma canção completa, o “Hino a Afrodite”) e cujos testemunhos são
os mais numerosos, referindo"-a desde sua época até a romana. Será a poeta
privilegiada pela sorte, ou sua reconhecida superioridade garantiu sua
preservação mais generosa do que a de outros poetas arcaicos e das poetas
mulheres? Impossível simplificar assim a explicação
para a sobrevivência de Safo ou de qualquer outro poeta grego antigo: nem a
mera sorte, nem a mera fama podem ser responsabilizadas exclusivamente por
nossa boa ou má fortuna quanto ao estado e volume do \textit{corpus}
remanescente de cada um deles.

Se não é Safo a única poeta grega antiga, é a única do período
arcaico. Esse dado, todavia, não deve ser superestimado, pois desconhecemos as
razões que guiaram a sobrevivência da obra dos poetas antigos.
Em princípio, nada impede que Safo tenha sido, de fato, a primeira
poeta grega; mais equilibrada se revela, no entanto, a ideia de que tenha
herdado, como os demais poetas gregos, uma tradição\footnote{ Ver West (1973, pp.~179--92), que mostra ser a
tradição métrica lésbio"-eólica mais antiga do que outras, como a dórica e a
ático"-jônica. Para Skinner (1996, p. 183), haveria, dado o caso de Safo, “uma
longa linha de predecessoras mulheres” poetas, mas nada há que permita sustentar em evidências essa ideia, nem as ideias de uma tradição separada de mulheres na poesia grega e de audiências igualmente separadas e femininas. Ver a propósito Bowman (2004, pp. 1--27).} que, em seu fazer --- haja vista
sua grande reputação firmemente atestada desde sua época ---, alcançou um grau
elevado de maturidade que, no caso das poetas mulheres, pode ter alargado os
caminhos depois dela percorridos.\footnote{ Ver Cantarella (1991, pp.~73--6), De
Martino (1991, pp.~17--75), Bowman (2004, pp. 1--27) e o estudo de Klinck (2008).} A reputação de Safo, atestam os
testemunhos, não encontrou em nenhuma poeta mulher um nome que a
superasse; Estrabão (\textsc{i} a.C.--\textsc{i} d.C.) ressalta isso: 

\begin{quote}
em todo o tempo recordado não sei de mulher
alguma aparecida que seja páreo ou que se aproxime um pouco dela, de Safo,
quanto à graça de sua poesia.\footnote{ \textit{Geografia} (13. 2. 3).}
\end{quote}


%Da busca de Safo, a mulher, e da formação da poeta, voltamos de mãos
%praticamente vazias --- praticamente, porque podem carregar ideias plausíveis,
%mas estas serão inverificáveis. Da busca da Safo \textit{lésbica}, à qual
%aludem os testemunhos tardios do papiro e do \textit{Suda}, o que traremos nas
%mãos no retorno será um resultado semelhante, ou até mais esvaziado.

A condição de mulher poeta cujas canções têm na temática erótico"-amorosa e no
universo feminino suas linhas de força --- canções estas lidas em chave
biografizante já pelos antigos que nelas buscavam a substância da figura
histórica para eles, como para nós, esvaecida --- faz emergir a questão da
sexualidade feminina e de seu exercício a partir da figura de Safo,
amiúde reelaborada em tramas antigas e modernas tecidas no correr dos
séculos, desprovidas quase de historicidade atestada, mas
inseridas na rede que se pretende explicativa da poeta e de seus textos. Judith
P.~Hallett\footnote{ Hallett (1996, pp.~125--7). Ao tratar das
imagens da poeta, ela fornece as indicações das fontes antigas.} comenta as
imagens que se formaram, sobretudo, do século \textsc{iv} a.C. em diante; e assim resume
Glenn W.~Most as díspares e desencontradas --- não raro,
extremadas --- imagens de Safo: 

\begin{quote}
As várias fontes que fluíram juntas para criá"-la creditaram"-na com um marido,
uma filha, muitos irmãos, numerosas amigas e companheiras (com as quais, ao
menos segundo alguns relatos, ela teve relações sexuais), numerosos amantes, um
homem que rejeitou as investidas de Safo, e um salto suicida de um penhasco. Em
princípio, decerto, não há razão para que uma vida social tão variada e rica
não tenha sido possível --- embora se pudesse cogitar como, entre um e outro
compromisso, Safo teria encontrado tempo para compor sua poesia [\ldots{}] Mas tanta
complexidade apresenta um desafio a qualquer um que tente imaginar um retrato
coerente da vida de Safo, pois requer que elementos potencialmente divergentes
sejam trazidos a uma relação plausível uns com os outros. Mais
fundamentalmente, a recepção de Safo pode ser interpretada como uma série de
tentativas de chegar a termo com a complexidade dessa gama de informações.\footnote{ Most (1996, p. 14).}
\end{quote}

Ora, a leitura dos modernos dessa intricada trama biográfica não foi, no mais
das vezes, menos infeliz que a dos antigos. Most comenta, por exemplo,
a dos românticos:

\begin{quote}
\textit{Condensando} numa única pessoa as muitas contradições com as quais a
tradição tinha suprido Safo, inventaram uma figura intensamente paradoxal
[\ldots{}] A Safo romântica é a primeira que é, essencialmente, uma poeta --- mas uma
poeta romântica, insatisfeita com a realidade banal e lutando para alcançar a
perfeição espiritual incompatível com a vida e somente alcançável às custas da
morte.\footnote{ Most (1996, p. 20).}
\end{quote}

Em verdade, uma revisão das leituras modernas de Safo mostra que cada época,
cada contexto histórico"-social e cultural criou para si a imagem desejada da
poeta; tal liberdade explica"-se pela carência de conhecimento consistente sobre
Safo e sua vida na arcaica Mitilene --- liberdade, diga"-se ainda, criativa, que
em sedutoras e intrigantes projeções pseudobiográficas, acabaram por
ganhar mais atenção do que merecem, do que sua arte
muitas vezes usada para alavancar --- eis novamente a circularidade viciosa que
já outrora mencionei --- tais projeções. 

\section{Sobre a obra}

Tendo isso em mente, e passeando por versos de poetas que ou precederam Safo ou
a sucederam, melhor entenderemos a similaridade tremenda entre as imagens, a
linguagem e o tom de seus fragmentos eróticos e de versos marcados pelo
erotismo em Hesíodo (ativo em \textit{c.} 700 a.C.), Arquíloco (\textit{c.} 680--640
a.C.), Íbico (ativo em \textit{c.} 550 a.C.) e Eurípides (\textit{c.} 482--406 a.C.). Eis
uma pequena amostragem de como, no registro erótico de gêneros, poetas e
tradições distintos, é símil a linguagem para falar da paixão erótica, do
desejo nas traduções possíveis para o grego \textit{éros} que, longe de nomear
o amor romântico, designa a força controlada por Afrodite --- mesmo na forma do
deus Eros, sempre a ela subordinado --- que, segundo seus desígnios, toma de
assalto sua vítima e se apodera de seu corpo e de sua mente, como “uma invasão,
uma doença, uma insanidade, um animal selvagem, um desastre natural”, que vem a
“provocar o colapso, consumir, queimar, devorar, exaurir, entontecer, picar,
aguilhoar, [\ldots{}]”, resume Anne Carson:\footnote{ Carson (1998, p. 148).}

\begin{quote} \parindent=0em
%\pagebreak
{\centering
\textsc{Hesíodo, \textit{Teogonia} (poesia didática)\footnote{ Tradução:
Torrano (2003).}}
\par}

\smallskip

	[\ldots{}] Eros: o mais belo entre os Deuses imortais,\\
	solta-membros, dos Deuses todos e dos homens todos\\
	ele doma no peito o espírito e a prudente vontade.\\
	\mbox{}\hfill (vv. 120--2)

\smallskip

	\hspace*{2em}[\ldots{}] Eurínome de amável beleza virgem de Oceano\\
	\hspace*{4.5em}terceira esposa gerou-lhe Graças de belas faces:\\
	\hspace*{4.5em}Esplendente, Agradábil e Festa amorosa\\
	\hspace*{4.5em}de seus olhos brilhantes esparge-se o amor \textit{[éros]}\\
	\hspace*{4.5em}solta-membros, belo brilha sob os cílios o olhar.\\
	\mbox{}\hfill (vv. 907--11)

\bigskip

{\centering
\textsc{Arquíloco, Fragmentos 193 e 196 W (poesia jâmbica)}
\par}

\smallskip

	\hspace*{4.5em}mísero jazo em desejo,\\
	\hspace*{2em}sem ar, de atrozes dores por querer dos deuses\\
	\hspace*{4.5em}transido até os ossos.\footnote{ Tradução: Souza (1984, p. 92). Edição: West (1998).}

\smallskip

	Mas o desejo solta-membros, ó companheiro, subjuga-me
	\ldots{}\footnote{ Tradução: Corrêa (2010, p.~504). Edição: West (1998)}

\bigskip

{\centering
\textsc{Íbico, Fragmento 287 Dav. (poesia mélica)\footnote{ Tradução: Ragusa (2010, p.
650; ver estudo às pp.~480--507). Edição: Davies (1991).}}
\par}

\smallskip

Eros, de novo, de sob escuras\\
pálpebras, com olhos me fitando derretidamente,\\ %[F020?][F020?][F020?][F020?][F020?] 
com encantos de toda sorte, às inextricáveis\\
redes de Cípris \textup{[Afrodite]} me atira.\\ %[F020?][F020?][F020?][F020?] 
Sim, tremo quando ele ataca, \num{5}\\
tal qual atrelado cavalo vencedor, perto da velhice,\\
contrariado vai para a corrida com carros velozes.

\bigskip

{\centering
\textsc{Eurípides, \textit{Hipólito} (tragédia)}\footnote{ Tradução: Oliveira (2010), em
edição bilíngue que segue o texto grego de Barrett (1992).}
\par}

\smallskip

Eros, Eros que nos olhos \num{525}\\
destilas desejo, incutindo doce\\ 
prazer n’alma dos que atacas,\\
que jamais me apareças com dano\\
nem venhas desmedido.\\
Pois nem dardo de fogo e nem dos astros é forte \num{530}\\
como o de Afrodite, que atira das mãos\\
Eros, filho de Zeus.\\
Em vão, em vão às margens do Alfeu \num{535}\\
e na morada pítica de Febo\\
bovinos sacrifícios prodigaliza a terra heládica,\\
se não veneramos Eros,\\
o soberano dos homens,\\
claviculário da deleitosa alcova de Afrodite \num{540}\\
exterminador que atira os mortais\\
em todos os desastres quando vem.
\end{quote}

Nos fragmentos traduzidos de Safo, o leitor decerto perceberá os ecos intensos
dessas linguagem e imagens sobre a paixão erótica, da qual lançam mão
reiteradamente os poetas gregos, como mostra a seleção citada. Esse movimento é
característico da prática genérica de composição dessa poesia de tradição oral, que lida sobretudo com motivos consolidados e embasados em
percepções mantidas pela repetição, pelo seu retomar por vezes incrementado em
escolhas estilísticas que parecem singulares a um ou outro poeta --- parecem, é
prudente dizer, em vista de nosso magro \textit{corpus} de textos, a partir do
qual afirmar a inovação é um risco.

Confrontados com os versos que reproduzi de Hesíodo, de Arquíloco, de Eurípides,
poetas homens de gêneros poéticos distintos, os fragmentos de Safo que envolvem
a temática da ação da paixão sobre sua vítima e a concepção sobre \textit{éros}
não se distinguem deles a ponto de tornar sustentável o argumento questionável
de uma “literatura feminina”; ao contrário, ecoam versos
dos dois primeiros poetas, e ressoam nos do terceiro, pois trabalham com
suas imagens e, se lhes acrescentam outras, aparentemente singulares à poeta
lésbia, fazem"-no calcados num tratamento da temática referida que não se pode
dizer masculino, nem feminino, mas tradicional. Daí porque é tão complicado
inferir, a partir da leitura das canções em que o “eu” --- nem sempre
identificável para além de uma voz em 1ª pessoa do singular --- acontece de ser
feminino e de se relacionar a figuras femininas eroticamente, que Safo seja
\textit{lésbica}, que represente a “literatura homossexual” --- outro rótulo
demasiado modernizante e anacrônico, em se tratando dos poetas antigos
(homens ou mulheres).

O recorte de um universo que é sobretudo feminino em sua poesia, inclusive
a de caráter intensamente erótico, poderia ser pensado de outro modo. No mundo
de Safo, como na Grécia antiga em geral, estavam cindidos o universo masculino
e o feminino, e Jan N. Bremmer observa que mesmo as relações
heterossexuais no casamento eram distanciadas. Conclui o helenista:
“Dificilmente foi por acaso que o homossexualismo moderno”, que em geral impõe
a rejeição ao heterossexualismo, “se desenvolveu na mesma época em que a
relação heterossexual no casamento adquiria um caráter muito mais íntimo”.\footnote{ Bremmer (1995, p. 24).} Na
Grécia, prossegue Bremmer, evidências literárias e iconográficas --- vasos
atenienses da segunda metade do século \textsc{vi} a.C., “que circulavam nos banquetes
aristocráticos”\footnote{ Bremmer (1995, pp.~20--21).} --- “mostram que as relações homossexuais normalmente
ocorriam só entre adultos e rapazes”,\footnote{ Bremmer (1995, p. 12).} sendo “um aspecto estabelecido do
caminho de um rapaz rumo à idade adulta”.\footnote{ Bremmer (1995, p. 12). Ao menos, na visão
externa dessas relações; da interna, pouco se pode dizer, lembra Bremmer (p.
20).} E frisa Bremmer que tais relações não implicavam a rejeição “do contato
heterossexual”, necessário à procriação e à preservação das linhagens e das
comunidades, mas eram vivenciadas no mundo dos homens, nitidamente distinto do
mundo das mulheres.\footnote{ Bremmer (1995, p. 26) ainda observa que era,
porém, a “relação pederasta que transformava o rapaz em um verdadeiro homem” ---
pois se voltava ao benefício do ensinamento intelectual e social do jovem, e
não apenas ao benefício do prazer sexual ao adulto --- e abria as portas para o
universo da “elite social”.}

Ao falar de \textit{éros} numa linguagem tradicional e a partir de uma concepção
reafirmada na poesia de temática erótica, Safo segue as práticas poéticas de
seu tempo e de lugar histórico, viabilizando assim a circulação de sua poesia
que por todos poderia ser entendida e fruída. 

%É decerto nesse sentido que
%devemos tomar uma conhecida frase --- na verdade, um verso hexamétrico --- do poeta
%latino Horácio (século \textsc{i} a.C.), numa \textit{Epístola} (1. 19. 28):
%``Regula a Musa pelo pé de Arquíloco a máscula Safo”,\footnote{ Agradeço
%ao colega latinista Prof.~Dr.~Marcos Martinho dos Santos (\versal{USP}) a %discussão sobre esse verso e
%sua tradução.} ou seja, a poeta lésbia regulou seus ritmos pelos de Arquíloco,
%numa compreensão possível de um verso controverso. A rigor, porém,
%estudos métricos mostram que a tradição lésbio"-eólica é mais antiga que a
%jônica, de que se vale Arquíloco, e que a dórica.\footnote{ Ver o estudo de West
%(1973, pp.~179--92).} Mas talvez possamos, apoiados nos versos dos dois poetas
%gregos arcaicos, pensar a ideia do ritmo em termos mais largos: a linguagem e
%as imagens que Safo emprega no tratamento da paixão são as mesmas que
%encontramos em Arquíloco e em outros poetas homens, antes ou depois dela. E
%quanto à conhecida expressão ``máscula Safo”, a primeira das duas
%explicações oferecidas por Porfírio (século \textsc{iii} d.C.), comentador de %Horácio,
%de que a poeta foi excelente na poesia em que homens mais amiúde se
%destacam, é, decerto, aquela que pode ser entendida coerentemente e com
%respaldo vasto e sólido das práticas poéticas antigas testemunhadas na
%composição dos textos sobreviventes. A segunda, de que Safo foi difamada como
%dissoluta tríbade --- termo grego de que se vale o comentador, que significa
%mulher homossexual ou \textit{lésbica} --- é de pouca valia, inclusive para a
%compreensão do verso epistolar horaciano, conduzindo"-nos para longe de sua
%poesia e do fazer poético de que resultam suas canções, e para perto das
%ficções de Safo e de polêmicas e ansiedades de nosso tempo. 


\section{Sobre o gênero}

%\section*{A mélica de Safo}

%Melhor é, portanto, ir para perto da obra da poeta lésbia, como espero que esta
%antologia evidencie aos que pensam conhecer Safo, mas talvez tenham se perdido
%nas tramas lendárias em que foi enredada, antes mesmo de chegar às ruínas da
%matéria viva que resta da poeta, de um sopro quente e fragrante que os séculos
%não abafaram.

No começo desta introdução, observei o fato de que o gênero poético praticado
por Safo é o que os antigos chamavam simplesmente “canção”
(\textit{mélos},\footnote{ Termo mais usado nos períodos arcaico e clássico da
Grécia. O sentido primeiro de \textit{mélos} é “membro” do corpo, daí “membro
musical, frase” e “canção” (palavra, melodia e ritmo), na compreensão
explicitada na \textit{República} (398c), de Platão (séculos \textsc{v}--\textsc{iv} a.C.).
Budelmann (2009, p. 2) lembra que \textit{mélos} “é usado por vários líricos
arcaicos com referência a suas composições”.} \textit{âisma, oidé}) --- daí a
designação \textit{mélica} --- e que se destinava à \textit{performance} ao som
da lira --- daí \textit{lírica} em acepção específica, termo mais tardio, em
circulação da era helenística (323--31 a.C.) em diante. A propósito
desses nomes, Rudolf Pfeiffer afirma: “Em tempos
modernos, toda a poesia não épica e não dramática é usualmente chamada lírica.
Mas os antigos teóricos e editores faziam a distinção entre, de um lado, poemas
elegíacos e jâmbicos, e, de outro, mélicos”.\footnote{ Pfeiffer (1998, pp.~182--3).} Mélica ou lírica, prossegue,
designava o “verso que era cantado para a música e, muito frequentemente, a
dança, e era composto de elementos de ritmos e tamanhos variados”:
\textit{mélos} era, na literatura grega arcaica, o poema lírico; o poeta,
\textit{melopoiós} --- o “fazedor de canções”, literalmente --- ou mélico; o
gênero, mélica ou poesia mélica. Ainda segundo Pfeiffer, tais nomes
“permaneceram usuais em mais tardias pesquisas sobre a teoria poética e a
classificação da poesia”, mas “líricos” era o termo que designava os autores
“em referências a edições de textos e em listas de ‘fazedores’\,”; e do século \textsc{i}
a.C. em diante, lírica sobrepõe"-se a mélica para designar a canção
cantada ao som da lira, o instrumento mais importante de seu acompanhamento, e
os latinos acabam por adotar o primeiro, a despeito do uso ocasional do
segundo.

Como se vê, o modo de \textit{performance} nomeia o gênero;
mas o que mais o caracteriza? Na \textit{performance} ainda, a canção podia ser
entoada em solo e com acompanhamento da lira, ou em coro, com o acréscimo de
outros instrumentos e da dança, a configurarem um espetáculo. Na métrica, a
composição da mélica dá"-se em estrofes mais breves e menos
complexas na modalidade monódica do que na coral. No conteúdo, a canção
monódica apresenta grande variedade de temas --- principalmente os vinculados à
vida cotidiana na \textit{pólis}, a eventos de um passado recente, à
experiência humana, sempre em relação direta com a voz poética ---, formas e
linguagem, enquanto à coral e seus subgêneros são comuns a celebração, a
narrativa mítica e o canto de autorreferência à \textit{performance} em
execução pelo coro. Não são mais estas palavras do que uma síntese para dar uma ideia ao leitor de um cenário mais complexo do que aparenta ser à primeira vista. 

Considerando que as composições da mélica grega arcaica “estavam destinadas,
desde o início, à execução pública ou privada, e constituíam por definição uma
poesia \textit{de} e \textit{para} a voz”, Gustavo Guerrero
afirma decorrer daí que a mélica

\begin{quote}
apareça dominada por um maior traço --- o caráter circunstancial do discurso --- de
literatura oral, que reflete a relação direta do texto com um local e um
momento precisos, um espaço e um tempo ritualizados [\ldots{}]; daí os índices
textuais de um discurso situacional, que se expressam através do uso de certas
figuras pronominais e de marcas do presente, signos que traduzem a interação
geral entre o sujeito da enunciação e seus destinatários.\footnote{ Guerrero (1998, pp.~20--1).}
\end{quote}

Inserida e movida “culturalmente no seio de um sistema de comunicação oral”, a
mélica se concretizava, “portanto, numa prática artística performática”,
conclui Guerrero.\footnote{ Guerrero (1998, p. 21).} Carecemos, porém, de informações que permitam uma
reconstrução clara e precisa de sua \textit{performance}, entre as quais, as
relativas à música, que logo se perdeu. No que se refere à mélica monódica,
lembra Giovan B. D’Alessio que se destinava à apresentação “em
contextos mais próximos aos da comunicação espontânea, face a face”; logo, nela
“espera"-se grau maior de imediatismo”.\footnote{ D’Alessio (2004, p. 270).} 

Recompondo minimamente o cenário, haveria para a canção solo várias audiências e
ocasiões de \textit{performance} possíveis, com destaque para o simpósio,
central também para gêneros como a elegia e o jambo; como anota Massimo Vetta,
no mundo grego de até meados do século \textsc{v} a.C., em que não
estava previsto “um público de leitores”, o simpósio 

\begin{quote}
é o lugar de conservação
e evolução da cultura ‘literária’ relativa a todos os temas que resultam
alternativos ao interesse ecumênico do \textit{epos} e à ambientação
exclusivamente pública do canto religioso oficial e da lírica agonística
[temas estes muito trabalhados na elegia, no jambo e na mélica monódica].\footnote{ Vetta (1995, p. \textsc{xiii}).}
\end{quote}

Entenda"-se por simpósio o que Pauline Schimitt"-Pantel define, em
sentido restrito e etimológico, como o “momento, após a refeição, em que todos
passam a beber”, e, em sentido mais amplo e corrente, “uma prática, de beber
junto, e uma instituição” que “é a expressão do modo de vida aristocrático” dos
homens na \textit{pólis}.\footnote{ Schimitt"-Pantel (1990, p. 15).} O simpósio
é, portanto, coletivo do ponto de vista do
evento, mas restrito do ponto de vista da classe e do gênero;\footnote{ Só
indivíduos do sexo masculino podiam dele tomar parte, reclinando"-se nos sofás ---
prática oriental --- ou servindo aos comensais comida e bebida. A presença
feminina limitava"-se às tocadoras de \textit{aulós}, instrumento de sopro, às
dançarinas --- normalmente cortesãs ---, e às servidoras de bebida. Não por acaso a
sala da casa do aristocrata em que se dava o simpósio era chamada
\textit{andrṓn} (“sala dos homens”).} e era pautado por um
“código rígido e próprio de honra”, ressalta Oswyn Murray, visando
garantir a moderação, caminho para a harmonia essencial à atmosfera
simposiástica.\footnote{ Murray (1990, p. 7).}

No andamento do simpósio, o beber era privilegiado e, por isso, tornou"-se
altamente ritualizado. A \textit{(re)performance} amadora ou profissional da
poesia, pelo poeta ou não, tinha lugar em meio a essa fase, e o simpósio acaba
por exercer um papel fundamental na sua preservação e difusão. Relaxados,
sorvendo o vinho, os gregos ouviam e cantavam e/ou recitavam elegias, trechos
dos poemas homéricos e, claro, exemplares da mélica, competindo uns com os
outros, e demonstrando a habilidade e desenvoltura esperadas de sua formação
aristocrática. Entre os textos escolhidos pelos simposiastas, poderia haver
versos em que o “eu” poético fosse de uma mulher, admitindo"-se, no contexto, a
representação de um papel feminino.\footnote{ Ver Bowie (1986, pp.~16--7).}
Nesse sentido, mais uma vez, não haveria obstáculo às canções de Safo em que a
1ª pessoa do singular é feminina. E não por acaso se diz ser o simpósio uma
ocasião bastante adequada ao caráter mais informal e privado da mélica
monódica; mas Most ressalta:

\begin{quote}
\mbox[\ldots{}] a aparente privacidade da canção monódica não é do individual espontâneo,
introspectivo, mas, antes, do pequeno grupo fora do qual o sujeito grego
arcaico mal pode ser concebido. [\ldots{}] Por sua própria natureza, portanto,
centra"-se nas relações pessoais entre um poeta individual e outro membro de seu
grupo de amigos, ou entre ele e o grupo como um todo, ou ainda entre ele e
indivíduos de fora desse grupo. [\ldots{}] Logo, em geral, a poesia monódica tem
dois modos principais: erótico para com os de dentro do mesmo grupo, de
invectiva, contra os de fora.\footnote{ Most (1982, p. 90).}
\end{quote}

O simpósio é mais voltado ao “mundo do privado”, frisa Schimitt"-Pantel, menos
formal e oficial do que o festival próprio à \textit{performance}
da canção coral; nem por isso, todavia, deixa de ser público e
ritualizado.\footnote{ Schimitt"-Pantel (1990, p. 25).} Na
“tentativa de generalização”, Most diz sobre a \textit{performance} e função da
mélica monódica: essa poesia 

\begin{quote}
era apresentada em ocasiões informais, para
pequenos grupos ligados por laços de amizade e interesse comum, e cumpria a
função social de unir esses grupos em todos coesos e separá"-los ou colocá"-los
em oposição a outros grupos numa mesma cidade.
\end{quote}

A mélica nas modalidades solo e coral está representada no \textit{corpus} sobrevivente
de Safo, que conta ainda com ao menos um fragmento de narrativa de
caráter epicizante (não sabemos se coral ou solo na \textit{performance}).

Feita para a \textit{performance} no festival cívico"-religioso, patrocinado
pelos governos e aristocracias das cidades gregas, no qual se
desenvolviam muitas atividades --- como o \textit{agṓn} (“competição”) poético
para cada gênero poético, para a música e para o corpo (os jogos) \mbox{---,} a canção coral tinha no poeta o compositor das palavras e
da música, e o diretor do coro de cidadãos, que, liderado por um de seus
membros, conduzia a apresentação, cantando e dançando ao som dos instrumentos.
Nessa atmosfera festiva e de celebração pública e coletiva, o canto tinha por
“tônica dominante”, frisa Herington, “o \textit{prazer}, humano e
divino”, pois eram homens e deuses homenageados a um só tempo.\footnote{ Herington (1985, p. 6).} Mas, para além
do festival, também podiam servir de ocasião à
\textit{performance} da mélica coral os grandes funerais e as grandes bodas,
menos públicos e mais privados. São notáveis nessas três ocasiões a ideia da
solenidade, o caráter religioso e a comemoração em chave de elogio; e esses
três elementos permeiam a construção das canções corais, refletindo"-se
em sua linguagem altamente elaborada em registro elevado, em seu forte
componente mítico, no canto autodramático do coro, e nas máximas de tom moral
e ético que pontuam seus versos, a (re)validar e reiterar valores e
ensinamentos compartilhados pela comunidade, pela audiência, pelos
\textit{performers}, pelo poeta cuja voz ganha dimensão mais
pública do que privada, na medida em que sua poesia, diz Most,
“tem papel vital na autoconsciência pública da cidade”.\footnote{ Most (1982, p. 94).}

