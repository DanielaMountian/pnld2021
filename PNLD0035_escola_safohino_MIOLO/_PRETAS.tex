\textbf{Safo de Lesbos} nasceu, segundo a tradição, de uma família 
aristocrática em Êresos, na costa ocidental da ilha de Lesbos
(mar Egeu), em torno de 630~a.C. A~poeta grega passou a maior parte de
sua vida numa cidade da costa oriental, a próspera e proeminente Mitilene, onde
teria morrido em cerca de 580~a.C. Seu nome figura desde seu tempo entre os
expoentes da poesia grega e de um de seus gêneros mais importantes, a
\textit{mélica} ou \textit{lírica}, e é o único nome feminino no conjunto de
poetas da Grécia arcaica (\textit{c.} 800–480~a.C.). Muitos outros dados sobre
sua vida podem ser colhidos nos testemunhos antigos; vistos de perto, porém,
eles se mostram demasiado frágeis, contraditórios, anedóticos, configurando-se
antes como peças de uma biografia ficcionalizante, sempre em (re)construção,
baseada no que nos restou da obra sáfica.

\textbf{Hino a Afrodite e outros poemas} reúne os textos traduzidos e anotados 
remanescentes da mélica sáfica, ou seja, de suas canções para \textit{performance} 
ao som da lira, em solo ou em coro. Mais precisamente, dessa poesia de
tradição oral, foram selecionados a única canção completa e os fragmentos mais legíveis  
de canções do \textit{corpus} de Safo, que sobreviveram ao tempo. 
Anotações de leitura buscam lançar luz sobre elementos
relevantes da estrutura, conteúdo ou transmissão dos fragmentos organizados
tematicamente. Precede a tradução anotada uma introdução sobre Safo, sua poesia
e o contexto em que se produziu e circulou, o gênero mélico, a fortuna crítica
sobre a poeta, a transmissão de sua obra, e as outras poetas mulheres de que
se tem notícia. 

\textbf{Giuliana Ragusa} é professora livre"-docente de Língua e Literatura Grega na Faculdade de Filosofia, Letras e Ciências Humanas (Departamento de Letras Clássicas e Vernáculas) da Universidade de São Paulo, onde ingressou como docente em 2004, tendo ali se graduado Bacharel em Letras (1999) e obtido os títulos de Mestre (2003) e Doutora (2008) em Letras Clássicas, e Livre"-Docente em Literatura Grega (2019). Entre 2012--2013, fez pós"-doutorado nos \versal{EUA} (University of Wisconsin, Madison, Bolsa Fapesp). Dentre seus livros publicados destacam-se: 
\textit{Fragmentos de uma deusa: a representação de Afrodite
na lírica de Safo} (Editora da Unicamp, 2005), contemplado com o 2º lugar do 
Prêmio Jabuti de 2006, na categoria Teoria/Crítica Literária,
\textit{Lira, mito e erotismo: Afrodite na poesia mélica grega arcaica} 
(Editora da Unicamp, 2010), e \emph{Lira grega: antologia de poesia arcaica} (Hedra, 2013). 
Tem publicado artigos em periódicos especializados na
área de Estudos Clássicos, e desenvolve projetos de pesquisa sobre a mélica grega arcaica. Atualmente, integra o Programa de Pós-Graduação em Letras Clássicas (\versal{FFLCH-USP}).





