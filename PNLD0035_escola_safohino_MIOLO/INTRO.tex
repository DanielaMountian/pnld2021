\chapterspecial{Introdução}{}{Giuliana Ragusa}

\noindent{}As composições de Safo são canções, mais
precisamente, fragmentos de algo mais próximo do que chamamos “canção” do que
daquilo que chamamos “poema”, retirada do primeiro termo a ideia moderna da
prevalência da música sobre o texto, ou de uma relação de paridade entre ambos,
diz a advertência de Guerrero,\footnote{ Guerrero (1998, p. 18).} uma vez que, na era arcaica, completa Eric A. Havelock, 

\begin{quote}
a melodia permaneceu serva
das palavras, e seus ritmos foram moldados para obedecer à pronúncia
quantitativa da fala.\footnote{ Havelock (1996, p. 132).}
\end{quote}

 A poesia grega antiga, cabe recordar, não se
escandia por acento, mas por duração breve ou longa de pronúncia da sílaba, o
que já lhe confere uma sonoridade bem ritmada. Em Safo, até onde permite
afirmar seu \textit{corpus} preservado,
há canções solo e corais, majoritariamente de estrofes breves, de variados temas,
linguagens, tons e modos, \textit{performance} em modalidade solo e coral.
Permanecem, contudo, problemáticas a
audiência e a ocasião específicas da apresentação original das canções
de Safo, presumivelmente pela própria poeta, na Mitilene arcaica, com participação de seu grupo de \textit{parthénoi} --- meninas virgens, condição transicional na rota ao casamento que as insere no mundo do sexo e da sociedade ---, em festivais cívico"-cultuais públicos e em cerimônias de casamento, ocasiões valorizadas no estudo de Franco Ferrari.\footnote{Ferrari (2010) se baseia nas novas reflexões a que levaram os avanços nos estudos de \textit{performance} e a descoberta do novo fragmento da poeta (``Canção sobre a velhice''), em 2004, que nos fez reavaliar a coralidade em Safo e a natureza coral de sua associação de \textit{parthénoi}, e traz um olhar fresco ao aproximá"-la de Álcman, o poeta dos partênios --- canções para coros de virgens --- que atuava em sua época. Ver ainda Ragusa (2019a, pp. 85--111; 2019b, pp. 211--39).}

Se olharmos para a canção poética, seus temas mais recorrentes e a inserção da
1ª pessoa do singular, e para a canção popular --- algumas do universo grego nos
foram transmitidas ---, perceberemos as origens pré"-literárias da mélica. Afinal,
o canto é uma forma de expressão verbal própria do homem, e a canção popular é
“atributo quase universal de sociedades tradicionais”, frisa Bowie.\footnote{
Bowie (1984, p. 3).} E, pensando na mélica grega, Lesky comenta suas formas
pré"-literárias, como os cantos de culto aos deuses, de lamento ou de celebração
nos “momentos culminantes da vida e da morte”, o “canto que acompanha o
trabalho” nos teares, na colheita das uvas.\footnote{ Lesky (1995, pp.~133--4).} Esses cantos ligam"-se ao ritmo da
vida cotidiana e às festividades das comunidades envolvidas na sua
\textit{performance}. Note"-se, enfim, que boa parte da música entoada pelas
personagens épicas da \textit{Ilíada} e da \textit{Odisseia} ilustra as formas
lembradas por Lesky, e, por conseguinte, relaciona"-se principalmente à mélica
arcaica e a seus subgêneros.

Em síntese, a mélica é um gênero de evidentes raízes
encravadas em tradições populares, visíveis em subgêneros
como o epitalâmio ou canção de casamento, que muito pode nos contar, tomada
junto a outras fontes escritas e iconográficas, sobre a cerimônia
da boda. Segundo Cecil M. Bowra, a despeito das variações
impostas por tradições locais, os epitalâmios apontam características gerais:
o banquete inicial na casa do pai da noiva; os sacrifícios aos
deuses do casamento na festa; a noiva, escondida por um véu, sentada junto às
outras virgens, aguardando a apresentação ao noivo, feita por parentes e amigos
dela; a procissão que acompanhava os noivos a seguirem em carruagem rumo à nova
morada; a condução dos noivos ao aposento nupcial, para consumar a boda,
condução esta feita pelos amigos de ambos, com danças, cantos, brincadeiras e
tochas, visando amenizar a tensão dos noivos --- em geral, estranhos um ao outro
--- e propiciar sua união. Aliás, a temática dos epitalâmios varia entre a
solene, quando centrada na procissão da carruagem dos noivos, jocosa, quando na
condução deles ao leito, ou elogiosa, quando na aparência física dos noivos de
modo a estimular a atração entre eles, passo importante à sua necessária união
sexual.\footnote{ Bowra (1961, pp.~214--8).}

Como se percebe, a execução dos epitalâmios tende a ser sobretudo coral, dado
que reforça seus elos com a tradição popular; e no caso dos fragmentos sáficos
dessas canções, anota Lesky, 

\begin{quote}
\mbox[\ldots{}] vemos como a poesia popular
tradicional é captada em toda a sua frescura e espontaneidade, por uma grande
poetisa que, no âmbito de sua arte, a modela em composições que alcançam uma
forma perfeita, sem perderem o encanto daquilo que surgiu do povo.\footnote{ Lesky (1995, p. 168).}
\end{quote}

Considerados os epitalâmios sáficos, bem como o já referido fragmento do que
poderíamos chamar mélica narrativa (“As bodas de Heitor e Andrômaca”), neles
encontramos uma poesia menos “pessoal”, por assim dizer, logo, bem menos
propícia à tão frequente leitura romântico"-biografista do \textit{corpus} de
Safo. O oposto disso ocorre com a maior parte da mélica da poeta. Nela, a temática
erótico"-amorosa --- uma de suas linhas de força ---, a autonomeação da
\textit{persona} como “Safo”, a força da 1ª pessoa do singular, o caráter
aparentemente intimista, as numerosas figuras
femininas contempladas em chave erótica ---, tudo isso estimula o estudo das
canções de Safo em viés flagrantemente romântico"-biografista. Mas tais elementos, incluindo o homoerotismo, mais coerentemente devem ser pensados, como se tem argumentado após a descoberta da ``Canção sobre a velhice'', no âmbito da coralidade e das associações corais femininas, atestadas com evidências consistentes pelo mundo arcaico afora.\footnote{Ver estudos indicados à nota 64 e ainda Lardinois (1994, pp. 57--84; 1996, pp. 150--72). Todos tratam da relação das meninas e de sua líder, Safo, a partir da ideia da coralidade e das associações corais em que se dava a \textit{paideía} feminina, atestadas no mundo arcaico. Tal formação, sentido do termo grego, era conduzida por uma mulher adulta, e tinha no casamento uma de suas orientações, bem como na formação ético"-moral e nas atividades femininas, no cantar e no dançar, nos ritos e em outros âmbitos.} A propósito
disso, afirma Odysseus Tsagarakis, em lúcida advertência à
problemática leitura da 1ª pessoa do singular da canção
monódica:

\begin{quote}
É bastante compreensível, dada a atitude do poeta lírico grego diante da vida,
que alguma experiência humana real subjaza à situação \textit{poética} (ela não
precisa ser específica). A relação da experiência poética com uma experiência
real é de importância secundária no que se refere à apreciação de um poema como
peça de arte. Mas quando se trata de julgar o caráter de um poeta e de
reconstruir sua biografia a partir do poema, a questão se torna séria. Mesmo se
assumirmos, em prol da argumentação, que Safo relembra uma situação pessoal,
não podemos provar que o que está descrito e o modo como está descrito refletem
a experiência real. Uma experiência real [\ldots{}] foi transformada em peça de
poesia que --- este agora é o problema --- tem um propósito.\footnote{ Tsagarakis (1977, p. 71).}
\end{quote}

Adiante, ele completa: “O caráter de Safo e de sua vida privada não pode ser
julgado a partir de sua poesia”.\footnote{ Tsagarakis (1977, pp.~81--2).} Se a identificação é problemática,
a separação radical também o é; na verdade, há que reconhecer certa medida de
proximidade e outra de distância entre o “eu” do poeta e o de seus versos. Qual
medida? Eis o enigma próprio da natureza poética dos textos; arriscar
responder à esfinge, em se tratando dos poetas antigos, cujas figuras nossos
olhos bem mal alcançam, é deixar devorar"-se a frágil matéria que deles nos foi
transmitida pela especulação e pela simplificação.

Vale reiterar que a relação na poesia entre a 1ª pessoa do
singular de um poema e o sujeito empírico do poeta é, desde o século \textsc{xix}, um
dos grandes nós no estudo da elegia, do jambo e da mélica arcaicos,\footnote{
Ver Slings (1990, pp.~1--30). Corrêa (2009, pp.~31--93) discute, com indicação de
bibliografia pertinente, os trabalhos de Fränkel (1975; 1ª edição
1951) e Snell (2001; 1ª edição 1951), da “escola Fränkel"-Snell”, de
caráter romântico"-hegeliano; também Ragusa (2005, pp.~23--44), com o foco em
Safo.} mas que o biografismo é já corrente entre os antigos, que alimentavam a
imagem dos poetas tão admirados quanto desconhecidos a partir de seus poemas,
agindo esses estudiosos como “poetas da ficção biográfica”, sintetiza Diskin
Clay.\footnote{ Clay (1998, p. 10). Não por acaso “a separação entre poeta e
\textit{persona} chegou tarde e com grande dificuldade”, prossegue ele (p. 16),
tendo sido o latino Catulo (século \textsc{i} a.C.) o primeiro poeta a clamar por tal
separação, num protesto poético em chave de virulenta vituperação de tradição
jâmbica, o que, portanto, compromete a argumentação. Cito o Poema 16 (tradução
Oliva, 1996): ``Meu pau no cu, na boca, eu vou meter"-vos,/ Aurélio bicha
e Fúrio chupador,/ Que por meus versos breves, delicados,/ Me julgastes não
ter nenhum pudor./ A um poeta pio convém ser casto/ Ele mesmo, aos seus
versos não há lei./ Estes só têm sabor e graça quando/ São delicados, sem
nenhum pudor,/ E quando incitam o que excite não/ Digo aos meninos, mas esses
peludos/ Que jogo de cintura já não têm/ E vós, que muitos beijos (aos
milhares!)/ Já lestes, me julgais não ser viril?/ Meu pau no cu, na boca, eu
vou meter"-vos''. As questões que se impõem, insolúveis, são: quem fala
aqui? Catulo ou seu “eu” poético? Quão identificados estão um e outro? E quanto
pesa na linguagem o gênero com o qual dialoga o poema, o jambo, de vituperação,
de ataque aos inimigos, de sarcasmo e de sátira?} Como bem resume Simon R.~Slings, 
todavia, as modernas teorias sobre a poesia “lírica”
pressupõem a leitura; mas a “lírica” grega arcaica e clássica destinava"-se à
\textit{performance} perante uma audiência, em dada situação e de certo modo
que se relacionavam ao gênero --- por conseguinte, à linguagem, à matéria e ao
metro --- do poema apresentado de viva voz. Assim, o poeta grego arcaico,
“precisamente porque é o causador de uma experiência estética, é, em certa
medida, despersonalizado”:

\begin{quote}
Se olharmos para o problema desse ângulo, tornar"-se"-á claro, de imediato, que
a oposição Eu ficcional x Eu biográfico é, na verdade, uma simplificação
irresponsável. O Eu é o Eu do \textit{performer}, que se move através de um
\textit{continuum} no qual o Eu biográfico e o Eu ficcional são os dois
extremos: na maior parte do tempo, ele não é nenhum deles.\footnote{ Slings (1990, pp.~11--12).}
\end{quote}


\section*{Poesia oral e o problemático nome «lírica»}

Safo e Alceu, mencionado no início dessa introdução, são dois dos nomes notáveis de um gênero poético, a
\textit{mélica} ou a lírica propriamente dita --- a canção destinada à
\textit{performance} em solo ou coral,
com o acompanhamento da lira (e de outros instrumentos e da dança, na
modalidade em coro). Se falo em \textit{performance} é porque, recorde"-se
desde já, sobretudo no período arcaico e depois no clássico (\textit{c.}
480--323 a.C.), pelo menos até \textit{c}. 400 a.C., a poesia grega é eminentemente de tradição oral e inserida no
que John Herington chama de “cultura da canção”,\footnote{ Herington (1985, p. 3).} na
qual, recitada ou cantada na \textit{performance}, disseminava
“ideias morais, políticas e sociais”. A oralidade, portanto, marca
a composição e a circulação dessa poesia em \textit{performances} e
\textit{reperformances} profissionais e/ou amadoras a determinada
audiência, de certo modo, em dada ocasião, colocada assim em ligação estreita
com a vida cotidiana da comunidade em que se fazia e pela qual passava, ligação
esta que lhe confere um caráter em essência pragmático. A mélica grega,
como bem ressalta Bruno Gentili, “não foi intimista, no senso
moderno”,\footnote{ Gentili (1990b, p. 9).} uma vez que só existia integrada na
vida da comunidade em meio à qual
circulava oralmente. Não por acaso, a voz poética, apresentada numa
situação de diálogo entre o \textit{performer} e sua audiência, está sempre em
diálogo; em vez do “eu” a falar consigo mesmo ou a ninguém, o “eu”/“nós” sempre
se dirige ao outro, ao “tu”/“vós” com que estabelece a interlocução. Se por
vezes esta não nos é de todo discernível, isso se deve aos problemas
materiais de preservação dos textos. Ora, o diálogo, dimensão viva da
comunicação verbal humana, é um elemento crucial da oralidade, incorporado com
grande força aos gêneros poéticos da Grécia antiga desde a épica homérica e
seus poemas monumentais, a \textit{Ilíada} e a \textit{Odisseia}, em que há uma
divisão quase equivalente entre narrativa de ação e diálogo. 

A oralidade se evidencia na composição da mélica, que se vale regularmente de
estruturas e procedimentos estilísticos de caráter mnemônico, que,
de maneira mais flagrante na era arcaica, refletem a tradição poética oral,
mesmo que já possamos pensar, naquele momento, no uso da escrita --- o alfabeto
grego, adaptação do fenício, se disseminava desde fins do século \textsc{ix} a.C. ---
pelos poetas nos processos e técnicas de construção de seus
versos.\footnote{ Ver a respeito Gentili (1990a, pp.~14--23) e Svenbro (1993, pp.
27--30).} Pensando o caso de Safo, Jesper Svenbro acredita
que ela teve seus textos escritos à sua época e com sua interferência direta,
ela mesma os escrevendo --- algo que pode ser excessivo e que não podemos
comprovar; de todo modo, afirma ele:

\begin{quote}
Um grego que vivesse por volta de 600 a.C., se refletisse sobre o problema de
registrar o poema sob a forma escrita, provavelmente consideraria a questão em
termos de uma \textit{transcrição} de algo que já tinha uma existência
socialmente reconhecida e que tenha sido tecnicamente controlado num estado
oral ou memorizado. Considerar a transcrição como uma operação que tornava o
poema duradouro e famoso não seria necessário; a tradição oral era bastante
capaz de fazer isso, sem o auxílio da escrita.\footnote{ Svenbro (1993, pp.~145--59).}
\end{quote}

Em outras palavras, ainda que aceitemos a possibilidade de que Safo e outros
poetas arcaicos, principalmente, tenham feito uso da escrita, o estudo atento
aos elementos estruturais e estilísticos de suas obras dá a perceber que a oralidade as gera e sustenta.

Decorre do modo de \textit{performance}, justamente, o nome tardio,
\textit{lírica}, que prevalecerá na referência moderna ao variado corpo de
textos --- de vários autores, metros, dialetos, tradições culturais, temas,
\textit{performances}, tons --- que nem são poesia hexamétrica --- como a
épica ---, nem dramática --- tragédia e comédia. Esse uso moderno do nome é decerto
prático, mas acaba por vestir, com um mesmo manto,
gêneros poéticos autônomos e distintos, entre os quais destacam"-se, na era
arcaica, a elegia, o jambo e a mélica. \textit{Mélica}, essa palavra não
dicionarizada em nosso vernáculo, é o termo que os antigos identificavam à
\textit{lírica}, rigorosamente, o gênero da canção para a lira. 

Fica, então, exposta a primeira armadilha de “lírica”, que por isso grafo
entre aspas: suas acepções antiga e moderna não são correspondentes uma
à outra. A segunda reside no fato de que uma mesma designação é empregada para
um gênero ou um conjunto de gêneros de poesia antiga --- fruto de uma cultura em
que tudo se realizava no trânsito entre boca e ouvido --- e para um gênero de
poesia moderna --- fruto de uma cultura da escrita. Isso cria uma falsa impressão
de familiaridade no que tange à poesia antiga, que, se não revertida, acarreta
para sua leitura um olhar modernizante potencialmente equivocado, sobretudo se
guiado pelas expectativas de uma noção comum --- não menos errônea para certa
poesia moderna --- sobre a “lírica” e o “lírico”. Tal noção agrega as ideias da
brevidade, da subjetividade --- amparada na abundante e intensa constância da voz
poética em 1ª pessoa do singular ---, da explosão dos sentimentos. Pautar"-se por
ela, todavia, é esquecer o filtro da dimensão estética, que faz com que
experiências e sentimentos, ainda que pessoais, conhecidos, vividos pelo poeta,
passem pelo processo da elaboração artística. Nele, a linguagem é trabalhada
estilisticamente, formalmente, transformando a experiência ou a emoção --- pouco
importa, a rigor, se vividas ou não pelo poeta --- em experiência
\textit{representada} no presente da composição. Há, portanto, uma filtragem
que impõe um distanciamento que precisa ser observado, mesmo que a voz poética
assuma o nome do próprio poeta, pois, ainda assim, estamos diante de sua
\textit{persona}, artisticamente elaborada em linguagem diversa --- não importa
quão próxima ela se pretenda --- da cotidiana.

Se é preciso atentar para tudo isso já na poesia moderna, mais ainda o é na antiga, em que, dada sua composição genérica, devem
estar articuladas as escolhas que faz o poeta da matéria, do metro, da ocasião
e do modo de \textit{performance}, bem como da linguagem, do caráter, do tom, e
assim por diante. Noutras palavras, a composição dessa poesia --- que, à diferença
da moderna, tem notável caráter pragmático --- centra"-se no gênero, cujas regras,
na época arcaica, são tradicionalmente preservadas e praticadas, sem que
estejam escritas, o que nos obriga a pensar os gêneros de poesia arcaica e
clássica menos como identidades de leis severamente fixas e mais como
“tendências firmadas o suficiente para permitir que afinidades e influências
sejam discerníveis”, observa Chris Carey, gerando
“expectativas na audiência”, mas sem prejuízo da flexibilidade que dá margem “à
frustração e à redefinição de tais expectativas” pelo poeta e pela sua
audiência.\footnote{ Carey (2009, p. 22).}

Claro está, a esta altura, que há uma defasagem considerável, para não dizer
enorme, entre nossas prática e cultura literárias e as antigas, às quais não
podem ser associados, sem grande prejuízo para a compreensão e leitura dos
textos --- na letra fria, artificial e estática de uma poesia feita para a
\textit{performance} em viva voz --- e de seus universos, conceitos relativos à
ideia moderna de “literatura”, tais como originalidade e criatividade. A poesia
elegíaca, jâmbica e mélica da Grécia arcaica, sobretudo, e clássica, resume
Anne P.~Burnett: 

\begin{quote}
[\ldots{}] é mais engenhosa e menos apaixonada,
mais convencional e menos individual do que desejariam os que advogam essa noção [da explosão da individualidade].\footnote{ Burnett (1983, p. 2).} 
\end{quote}

Essa poesia antiga, oral e de
ocasião, “fincada no sistema social de uma \textit{pólis} grega arcaica”,
recorda Wolfgang Rösler, é essencialmente discurso.\footnote{ Ver
Rösler (1985, p. 139), Johnson (1982, p. 72) e Clay \mbox{(1998, p. 11)}.}

Em conclusão a essas palavras, cabe esclarecer que não se trata, aqui, de negar
qualquer medida de identificação entre o poeta e seu “eu” poético, mas de
afirmar que, ao lidarmos com os antigos, o somatório do modo de composição, da
falta de conhecimento sobre a biografia dos poetas e seus contextos
histórico"-sociais, e da precariedade mais ou menos intensa dos textos
sobreviventes, torna essa identificação ainda mais complexa do que é.
E o terceiro ingrediente dessa soma não é de importância menor, como
bem mostra este aviso aos que buscam a obra de Safo que, reunida com rigor,

\begin{quote}
contém apenas um poema completo, aproximadamente dez fragmentos substanciais,
uma centena de citações breves de autores antigos e cerca de 50 peças de textos
em papiro, que emergiram das areias do deserto egípcio.
\end{quote}

Daí ser “mais exato
falar em fragmentos de Safo”,\footnote{ Lardinois (1995, p. 29).} e não em
poemas. Essa síntese citada de André Lardinois é acurada, e de modo algum restrita ao
caso da poeta de Lesbos.
Antes, o cenário da preservação das obras de outros poetas da “lírica” antiga ---
a poesia jâmbica, a elegíaca e a mélica --- é similar ou pior do que
o da obra de Safo. A propósito disso, cito a fala marcante de Walter R. Johnson:

\begin{quote}
Nós todos, quando lemos a lírica grega, ficamos desapontados em certo sentido:
não porque a poesia não impressione --- antes, é supremamente bela ---, mas
porque existe para nós apenas em cacos e farrapos. [\ldots{}] E quando a comparamos
aos outros remanescentes da literatura e da cultura gregas, essas ruínas são de
machucar o coração. Nenhuma experiência de leitura, talvez, é mais deprimente e
mais frustrante do que a de abrir um volume dos fragmentos de Safo e
reconhecer, ainda uma vez --- pois sempre se espera que desta vez seja diferente
---, que essa poesia está perdida para nós.

Esse é um fato que escolhemos não encarar --- não de frente e constantemente.
Logo, divisamos uma ficção --- [\ldots{}] que chamamos poesia lírica grega ---,
mesmo que saibamos que ela está em meros fragmentos. Na verdade,
\textit{porque} sabemos que está em meros fragmentos, agimos, falamos
e escrevemos como se o impensável não tivesse acontecido, como se bispos pios,
monges descuidados e ratos famintos não tivessem consignado Safo e seus colegas
líricos ao esquecimento irremediável. [\ldots{}] Naturalmente, qualquer helenista a
quem você perguntar admitirá o fato da fragmentação. Pode até ter prazer em
descrever o estado verdadeiro dos textos e a incerteza fascinante de
restaurações e conjecturas. Mas se você persistir em seu escrutínio e conversar
sobre a poesia, sobre os poemas não existentes, à medida que a conversa
esquentar, o esqueleto ganhará carne e cor, e a ruína se esvaecerá. Isso não é
prevaricação ou enganação, isso é a natureza humana: nós queremos aqueles
poemas e, nos momentos em que nos desarmamos, nós os imaginamos de volta à
existência. [\ldots{}]

É o leitor, então, que deve se lembrar, quando eu me esquecer, que a lírica
grega [\ldots{}] nos é essencialmente inacessível.\footnote{ Johnson (1982, pp.~25--6).}
\end{quote}

Nesta introdução e na tradução que se lhe seguirá ---
para não esquecer essa realidade e para que o leitor dela tenha consciência,
acrescento à síntese de Lardinois esta outra: aos que procuram a poesia de
Safo, restarão seus fragmentos vivos, desafiadores, pulsantes há séculos, a
despeito de sua fragilidade material; aos que procuram Safo, a mulher, ou Safo,
a \textit{lésbica}, restará pouco mais do que ficções e anedotas. Penso que,
entre a substância não obstante precária dos fragmentos e a névoa tão sedutora
quanto impalpável da biografia, mais profícuo será privilegiar a primeira
opção. Mas falemos um pouco da névoa sáfica.


\section*{Outras poetas}

Considerado o desenho geográfico composto pelos nomes de poetas mulheres
lembrados no epigrama da \textit{Antologia palatina} --- compilação de quinze livros de
epigramas datados dos séculos \textsc{vii} a.C. ao \textsc{v} d.C. ---, que abre esta introdução, Albin Lesky atenta para um detalhe
notável: há poetas mulheres de Lesbos, da Beócia e do Peloponeso, mas não da
Ática. “Assim se manifesta uma posição diferente, mais livre, da mulher”
nessas outras regiões, do que aquela “que conhecemos no mundo de Atenas”.\footnote{ Lesky (1995, p. 210).} A
conclusão do helenista, em princípio possível, leva a outra: nas sociedades
que geraram poetas mulheres, deve ter havido acesso a uma forma de educação
feminina e abertura para a participação maior das mulheres na vida da
comunidade, incluindo as de origem aristocrática, como Safo; daí a provável
aceitação do fazer poético exercido por uma mulher.

A circulação da obra de Safo, e de outras poetas mulheres mais tarde, parece
apontar nessa direção, bem como o silêncio que ressoa da Ática, de Atenas, onde
prevalecia, ao menos na era clássica, o confinamento das mulheres ao \textit{oîkos} (“casa”), espaço
feminino por excelência na Grécia; salvo em ocasiões específicas, como um rito
religioso"-cultual, as mulheres atenienses não deveriam ser vistas, nem suas
vozes ouvidas.\footnote{ Para mais sobre a mulher ateniense, ver os estudos de
Mossé (1991, pp.~49--61) e Murray (1993, p. 41). Reitero que é eloquente --- se
não for obra do acaso --- a ausência de poetas provindas da Ática em nosso
\textit{corpus}, considerada a exclusão social feminina em Atenas. E ressalta
Mossé (pp.~152--3) que só uma esfera de atuação civil das mulheres lhes era
permitida em Atenas: a religiosa, da qual participavam ativamente, em funerais,
casamentos, rituais e festivais públicos e privados.} Na casa, talvez elas
tivessem acesso à educação e mesmo à escrita, mas, à diferença do que teria
ocorrido noutras partes do diversificado mundo grego, não haveria em Atenas
condições, segundo o que se sabe, para que suas manifestações artísticas
pudessem circular.

Nada disso pode ser afirmado com segurança, mas é plausível pensar que o acesso
à educação aristocrática e a maior liberdade de ação e inserção social
contribuíram e muito para que poetas mulheres tivessem condições de existir,
poetas como Safo,\footnote{ Diz Bennett (1994, p. 346):
“Para tornar"-se uma poeta, Safo teve de ser treinada, em expressão
e composição, e nós naturalmente suporíamos que tal treino era aquele de outras
meninas aristocráticas de Mitilene”.} cujas composições, como as dos demais
poetas arcaicos e clássicos, tinham que ser, necessariamente, apresentadas em
determinados modo e ocasião de \textit{performance} --- e dadas a celebridade e a variedade da mélica sáfica, não pode ter sido esta uma só, e nem limitada
a grupos secretos de mulheres. 

Lamentavelmente, não é fácil avaliar a condição feminina na Grécia
antiga, menos ainda na era arcaica, e menos ainda na lésbia Mitilene, cuja
especificidade, em qualquer de suas dimensões, nos escapa quase que
de todo. As evidências escritas ou iconográficas são em toda parte muito
escassas, mas não podemos deixar de mencionar que nos vasos atenienses que
retratam figuras femininas associadas à escrita e/ou à leitura, estas são Safo
--- porque é a poeta de grande fama --- e as Musas --- porque são deusas.
Essas imagens, em última análise, nada provam quanto a Safo --- e outras mulheres
aristocratas --- no que concerne às habilidades da escrita e da leitura --- as
quais são separadas na Antiguidade, o domínio de uma não implicando
automaticamente o da outra, lembra Susan G. Cole, limitando"-se a
leitura, sempre em voz alta, a poucos textos.\footnote{ Cole (1992, p. 220).} Na representação iconográfica,
prossegue ela, os livros --- na verdade, rolos de papiros --- amparam a
recitação, e não a “leitura solitária” e silenciosa; e nas imagens de mulheres,
é típico o desenho de uma que lê para uma moça ou para um grupo feminino.
Quando a leitora é Safo, diz Cole, a imagem quer antes celebrar a poeta, do que
retratar “uma cena familiar ou tipicamente doméstica”, até porque, ressalta,
mesmo em vasos onde figuram cenas domésticas, além dos que trazem a poeta ou as
deusas da poesia, jamais vemos mulheres escrevendo.\footnote{ Cole (1992, p. 224).}


Somadas as nove poetas mulheres nomeadas no epigrama --- nove são as Musas, e nove é o número comum às listas de cânones de poetas, como a do epigrama, que proliferam como reflexo dos trabalhos na Biblioteca de Alexandria ---,
elas compõem um modesto conjunto de obra e de testemunhos biográficos
não de todo confiáveis, dado o caráter altamente ficcionalizante da
biografia antiga, que, sendo antes um gênero de discurso do que um relato
factual, está comprometido com o verossímil.

\subsection*{1. Grécia clássica:\break Mirtes, Praxila, Telesila (e
Corina?)\footnoteInSection{ Para todas essas poetas, suas obras e testemunhos sobre
elas, ver a edição bilíngue de Campbell (1992).}}

\smallskip

Mirtes (Beócia, fins do século \textsc{vi} a.C.), da qual nada resta, teria sido mestra
de Píndaro (séculos \textsc{vi}--\textsc{v} a.C.) e de Corina --- neste caso, datada do século \textsc{v}
a.C., não do \textsc{iii} a.C., como parece também ser possível. Todos os três são
mélicos predominantemente corais, e estão relacionados no Fr.~664(a)
P\footnote{ Abreviação para a edição Page (1962).}  de Corina, cuja compreensão
é bastante nebulosa, como a própria imagem de Mirtes:


\begin{quote}
\mbox[\ldots{}] eu censuro também a de clara-voz,\\
Mirtes, porque, sendo mulher,\\
entrou em rivalidade com Píndaro [\ldots{}]
\end{quote}

Praxila, ativa em meados do século \textsc{v} a.C., assim surge num passo do gramático
Ateneu (séculos \textsc{ii}--\textsc{iii} d.C.), no \textit{Banquete dos eruditos} (\textsc{xv}. 694a):
``Praxila de Sícion era também admirada pela composição de cantos
convivais” para acompanhar o beber do vinho, nos simpósios. De sua obra,
porém, temos apenas o fragmento de um “Hino a Adônis” (747 P), outro
de um “Ditirambo: Aquiles” (748 P), três de canções convivais. No fragmento
hínico, o mítico jovem amante de Afrodite, Adônis, arrola as mais belas coisas
que deixou no mundo dos vivos ao morrer, no auge de sua juventude e virilidade:

\begin{quote}
\mbox[\ldots{}] a mais bela coisa que eu deixo é a luz do sol,\\
a segunda, as estrelas brilhantes e da lua sua face,\\
e também pepinos maduros e maçãs e peras [\ldots{}]
\end{quote}

O retórico Zenóbio (século \textsc{ii} d.C., \textit{Provérbios} 4. 21), fonte única
desses versos, cita"-os para explicar o antigo provérbio ``mais ingênuo
que o Adônis de Praxila”, pois “é fraco da cabeça quem igualmente
lista com o sol e com a lua os pepinos e o resto”.

Do fragmento de ditirambo --- gênero mélico de difícil classificação, com forte
componente narrativo ---, há um verso citado por Heféstion (século \textsc{ii} d.C.,
\textit{Sobre os metros}, \textsc{ii}. 3): ``[\ldots{}] mas teu coração no peito nunca
eles persuadiram [\ldots{}]”. Nele, Praxila lembra a intolerância de Aquiles e sua
inabalável rejeição à ajuda aos gregos na luta contra os troianos, em razão da
grave ofensa à honra que lhe havia feito o chefe da expedição, o Atrida
Agamêmnon, ao arrebatar"-lhe Briseida, seu \textit{géras} (“prêmio”) --- parte do
botim de guerra que é a medida da honra de quem o recebe ---, conforme o canto \textsc{i}
da \textit{Ilíada}.

Telesila (Argos, meados do século \textsc{v} a.C.), é associada a um episódio marcial
supostamente biográfico: segundo Plutarco (séculos \textsc{i}--\textsc{ii} d.C., \textit{As
virtudes das mulheres}, 4. 245c"-f) e Pausânias (século \textsc{ii} d.C.,
\textit{Descrição da Grécia}, 2. 20. 8--10), ela teria liderado um conflito
militar contra Cleomenes, de Esparta, o que jogaria sua datação para \textit{c.}
494 a.C. Pausânias descreve, no santuário de Afrodite em Argos, diante da
estátua sentada da deusa, a imagem de Telesila, ``a compositora de
canções”, numa estela:\footnote{ Placa funerária comum nos túmulos gregos, com
inscrições e desenhos sobre os mortos.} ``seus livros caídos aos seus
pés, ela olha para um elmo, segurando"-o com a mão e prestes a pô"-lo sobre sua
cabeça”. Para o viajante, a poesia deu à poeta ainda ``maior honra” do
que outras coisas. Interessa notar, sobre a imagem de guerreira e poeta de
Telesila, que a temática marcial deve ter sido relevante em sua produção, a
confiarmos em Máximo de Tiro (século \textsc{ii} d.C., \textit{Oração} 37. 5), segundo
quem ``os espartanos eram exaltados pelos versos de Tirteu” --- 
poeta elegíaco (meados do século \textsc{vii} a.C.) ---, ``os argivos, pelas
canções de Telesila, e os lésbios, pelas odes de Alceu”. 

A referida imagem de Telesila, dada na fonte tardia, pode ter se nutrido dos
versos da poeta, tanto quanto o episódio biográfico, cuja historicidade não
podemos comprovar. Infelizmente, não se percebe a temática guerreira na obra
remanescente da poeta: cinco fragmentos, o maior, com dois versos (Fr.~717 P),
ao que se percebe, sobre o mito da paixão do deus"-rio Alfeu pela irmã de Apolo,
a virgem caçadora Ártemis.

Por fim, se aceitarmos uma de suas possíveis datações, temos Corina (Beócia) na
era clássica --- mas ela pode ter vivido na helenística, disse
anteriormente.\footnote{ Se a primeira opção proceder, há que supor que “seu
trabalho se perdeu logo depois de ser escrito, e só em fins do século \textsc{iii} a.C.
foi redescoberto”, comenta Campbell (1998, p. 409), pois a análise da
ortografia dos papiros (séculos \textsc{i} e \textsc{ii} d.C.) de Corina “prova que o nosso texto
foi escrito em \textit{c.} 200 a.C.” (p. 408). A questão cronológica permanece
em disputa, mas fato é que o nome de Corina apenas se registra, para nós, em
fontes de 50 a.C. em diante; Lesky (1995, p. 209) ressalta que os “grandes
gramáticos alexandrinos não se ocuparam” dela, que somente na Antiguidade
tardia parece ter desfrutado de certa notoriedade.} De sua obra, temos
fragmentos em que se nota a presença bem marcada de mitos beócios. O mais longo
(654 P) deles, cuja fonte é o bastante precário \textit{Papiro de Berlim}
(século \textsc{ii} d.C.), traz a competição poética entre duas montanhas beócias,
Hélicon (abrigo das Musas) e Citero (no limite com a vizinha região da Ática),
e algo sobre as filhas de Asopo, deus"-rio beócio. Outro fragmento (655 P)
canta, no trecho legível:

\begin{quote}
\mbox[\ldots{}] a musa Terpsícore me (chama?) [\ldots{}]\\
belos contos cantar\\
às de Tanagra, de brancas vestes;\\
e grandemente a cidade se alegrou com minha\\
clara sedutora voz.
\end{quote}

Há, ainda, fragmentos de alguns versos, uma única linha ou palavra, os quais
formam a maioria dos pequenos e corrompidos textos do \textit{corpus} de
Corina, como o 666 P, conservado na citação de Apolônio Díscolo (século \textsc{ii}
d.C., \textit{Pronomes}, 95bc):

\begin{quote}
\mbox[\ldots{}] por tua causa Hermes contra Ares\\
boxeia [\ldots{}]
\end{quote}

Destaco, por fim, o Fr.~664(b), também citado por Apolônio; nele, a voz poética
declara o tema de sua mélica:

\begin{quote}
\mbox[\ldots{}] eu mesma canto dos heróis as excelências\\
e das heroínas [\ldots{}]
\end{quote}

\smallskip

\subsection*{2. Grécia helenística:\break Mero, Erina, Anite, Nóssis}

\smallskip

Esse grupo nos remete à primeira parte da era helenística, cujo centro não é
mais Atenas, mas Alexandria, no Egito ptolomaico, onde o grego foi tornado
língua oficial da administração, do comércio e da educação, e onde proliferaram
escribas, sobretudo na cidade, sede da Biblioteca que era, na
verdade, uma sala entre outras do \textit{Mouseîon} (“a casa das Musas”,
“Museu”) erguido pelo faraó Ptolomeu \textsc{i}, o Sóter, que reinou entre 305--285 a.C.,
e que fora general de Alexandre, o pupilo de Aristóteles. Seu objetivo era um
só: edição e cópia das grandes obras dos antigos, em organização de forte
inspiração aristotélica; e Ptolomeu \textsc{ii}, o Filadelfo, no poder entre 285--246
a.C., ampliou a Biblioteca, de modo a permitir a intensificação desses
trabalhos depois interrompidos por uma catástrofe em 47 a.C., mas retomados
eventualmente e ativos até meados do século \textsc{v} d.C. 

Nesse mundo, “a poesia do passado”, declara Gentili, “passou a
ser lida como literatura pura e simples”, embora fosse ainda recitada --- reflexo
ainda vivo da cultura oral em que se produziu.\footnote{ Gentili (1990a, p. 37).} E em termos da produção poética,
destaca"-se entre os gêneros praticados no período o epigrama, “poema curto em
dísticos elegíacos” --- metro próprio da poesia elegíaca --- e de conteúdos
variados, lembra Jane M. Snyder, entre os quais, “lamentos,
dedicatórias, casos amorosos, animais de estimação e assim por diante”.\footnote{ Snyder (1989, p. 66).}
Originalmente, o epigrama “limitava"-se a servir de epitáfio”, anota a
helenista; e “a palavra grega \textit{epigramma} significa ‘inscrição’\,”. Os
epigramas das poetas de que passo a me ocupar encontram"-se na \textit{Antologia
palatina}.

Mero, de Bizâncio, é a mais desconhecida; dela só há dois epigramas na
\textit{\textsc{ap}} (livro \textsc{vi}, 119 e 189): uma dedicatória às uvas viníferas,
outra às ninfas das águas. Mas em Ateneu (\textsc{xi}. 491b) encontram"-se ainda dez
versos hexamétricos --- metro da épica grega e da poesia didático"-sapiencial,
sobretudo ---, que tratam da constelação das Plêiades. Esse trânsito por entre os
gêneros não é novidade nem na produção poética dos gregos, nem no período, mas
se atesta desde a era arcaica.

Outra poeta, Erina --- não sabemos ao certo sua origem ---, também praticou o
epigrama e a poesia épica. Magro, porém, é seu \textit{corpus}: três epigramas
provavelmente de sua lavra, um fragmento com dois versos hexamétricos, pedaços
d’“O fuso do tear”, longo poema (cerca de 300 versos) em hexâmetros e dialetos
lésbio"-eólico e dórico, preservado num papiro em que se leem, quando muito,
algumas palavras. Um epigrama anônimo (\textit{\textsc{ap}} livro \textsc{ix}, 190)
sobre Erina a dá por lésbia, conta sua morte ainda virgem aos 19 anos, e
afirma que os versos d’“O fuso” são ``iguais” aos de Homero. Por fim,
declara que ``tanto quanto Safo supera Erina em canções, Erina supera
Safo em hexâmetros” (vv. 7--8), metro da epopeia, que teria sido contemplado na
obra sáfica.

Sobre os epigramas, atribuídos a Erina, Snyder observa
serem todos concernentes às mulheres: o 352 do livro \textsc{vi}
retrata uma mulher de nome Agatárquis; o 710 e o 712 do \textsc{vii}, a morte da jovem
Báucis, personagem que reencontramos n’“O fuso do tear”, poema cujo sentido de
todo nos escapa.\footnote{ Snyder (1989, p. 90).}

Já a Anite estão dados cerca de vinte e quatro
epigramas, informa Snyder.\footnote{ Snyder (1989, p. 67).} Segundo o \textit{Onomástico} (5. 48) de
Pólux (século \textsc{ii} d.C.), ela seria de Tegeia, na Arcádia, dados o dialeto dos
versos, as referências a elementos naturais e a imagem do deus Pã, própria da
mitologia local. Cito dois epigramas da poeta, na tradução de José Paulo Paes:

\begin{quote}\parindent=0em 
Para o seu gafanhoto, rouxinol dos campos, e a sua\\
cigarra das árvores, fez Miro um duplo túmulo\\
e o regou com lágrimas de menina: pois o cruel Hades\\
levou-lhe embora os dois bichinhos de estimação.\\
\mbox{}\hfill (\textsc{\textit{ap} vii}, 190)

\smallskip

Vivo, este homem era Manes, um escravo; morto,\\
vale agora o mesmo que o grande Dario.\\
\mbox{}\hfill (\textsc{\textit{ap} vii}, 538)\footnote{ Paes (1995, pp.~34--5).}
\end{quote}

Vê"-se acima a diversidade de conteúdos, própria do gênero; e são os temas
pastorais especialmente relevantes nessa poeta da Arcádia --- região configurada
como “o ideal da paisagem bucólica do pastor”, resume Snyder. Eis o epigrama
313 (\textsc{\textit{ap}}, \textsc{ix}):

\begin{quote}
Senta-te, de todo sob as belas folhas vicejantes do loureiro\\
e tira doce porção d’água de beber da graciosa nascente,\\
para que descansem teus membros cansados da labuta\\
do verão, tocados pelo sopro de Zéfiro.
\end{quote}

Merecem nota, ainda, os curiosos epitáfios para animais --- dos quais o epigrama
190 do livro \textsc{vii}, já citado, é exemplo ---, o que pode ser visto como uma
brincadeira de Anite com a expectativa de quem ouve um epigrama e espera
“grande solenidade”, observa Snyder.\footnote{ Snyder (1989, p. 70).} 

Nossa última poeta, Nóssis, nasceu em Lócris, “colônia grega no sul da Itália”,
ressalta Snyder,\footnote{ Snyder (1989, p. 77).} fundada no século \textsc{vii} a.C. Seu \textit{corpus} se
compõe de doze epigramas, quase todos centrados no
universo feminino e nas deusas Hera e Afrodite. E, como Safo, ela em três deles
se autonomeia, observa a helenista, “criando a vívida \textit{persona} de uma
mulher que celebra as delícias de Eros e que se proclama, ela própria,
seguidora da tradição poética” da poeta de Lesbos. Cito dois epigramas,
novamente em tradução de Paes:

\begin{quote}\parindent=0em
Nada mais doce que o amor; tudo quanto haja de ditoso\\
lhe fica atrás e eu cuspo da boca até o mel.\\
Eis o que diz Nóssis; aquela a quem a Cípria\footnote{ Outro nome de
Afrodite, a deusa do amor erótico, da beleza, da sedução.} não beijou,\\
essa não sabe sequer que flores são as rosas.\\
\mbox{}\hfill (\textsc{\textit{ap} v}, 170)

\smallskip

Se fores, estrangeiro, à Mitilene de formosas danças,\\
a qual fez Safo, a flor das Graças, consumir-se,\\ 
diz que a terra locriana produziu, dileta das Musas,\\
alguém que lhe é igual, de nome Nóssis. Vai!\\
\mbox{}\hfill (\textsc{\textit{ap} vii}, 718)\footnote{ Paes (1995, pp.~36--7).}
\end{quote}

Ambos sugerem que o tema amoroso deve ter preenchido alguns de seus textos. No
primeiro epigrama, os dois versos finais parecem referir"-se exatamente à
paixão, pois alinhavam Afrodite, o beijo e as rosas, flores prediletas da
deusa. No segundo, ao igualar"-se a Safo, cujo tema principal, até onde o
\textit{corpus} de sua poesia mélica e a sua reputação na Antiguidade permitem
afirmar, gira em torno de \textit{éros} (paixão, amor, desejo erótico), Nóssis
declara, indiretamente, que sua poesia se afina à mesma temática: Mitilene
gerou uma poeta de \textit{éros}; Lócris, outra que lhe é ``igual”.

Muitos nomes, escassa substância:\footnote{ Alguns nomes mais nebulosos ainda são
os de Megalóstrata, mencionada por um poeta mélico ativo na Esparta de fins do
século \textsc{vii} a.C., Álcman, no Fr.~59(b) (edição Davies, 1991;
tradução Ragusa, 2010, p. 653): ``[\ldots{}] isto mostrou, das doces
Musas/ o dom, uma das virgens venturosa ---/ ela, a loira Megalóstrata [\ldots{}]'' Há
também o de Cleobulina, filha de Cleóbulo de Lindos, o colecionador de enigmas,
e Carixena, a quem o \textit{Léxico} de Fócio (patriarca de Constantinopla,
século \textsc{ix}) se refere para explicar a expressão ``do tempo de Carixena”:
``Carixena foi uma antiquada tocadora de flauta e compositora de música,
mas alguns a dizem também poeta lírica”. Um provérbio no \textit{Léxico} de
Hesíquio (século \textsc{v} d.C.) buscava já explicar essa expressão depois lembrada em
Fócio: ``Carixena foi famosa por sua estupidez, porque não sabia que era
antiquada. Alguns dizem que ela fazia canções eróticas. Há um provérbio também,
‘o tipo de coisa que é do tempo de Carixena’\,”. Nada resta dessas duas poetas
provavelmente do século \textsc{v} a.C.} é o que se pode espremer do acúmulo dos
séculos que encobrem as obras e as poetas do epigrama de abertura desta
introdução. Distinta e bem mais feliz fortuna teve a obra de Safo --- não sua
figura que, decerto pelo fascínio exercido por sua poesia, ao menos em parte,
tem sido preenchida com múltiplas ficções desde a Antiguidade, as quais a
tornam ainda mais impalpável. 

\section*{A Safo lésbica}


A imagem da Safo \textit{lésbica} construiu"-se já com os textos antigos mencionados, e ganhou fôlego
na esteira de vogas na crítica literária --- \textit{gay studies}, \textit{women
studies} --- que valorizam aspectos do entorno dos textos que passam a ser
estudados de modo secundário, e não no primeiro plano, para sustentar a análise
de tais aspectos. Nessa linha, certa leitura de Safo e de certa fatia de
testemunhos sobre a poeta amparam a afirmação de que trata"-se da primeira poeta
\textit{lésbica} do Ocidente, não porque seja filha da ilha de Lesbos, mas
porque, segundo um olhar modernizante e romântico --- que
desconsidera ou minimiza o fato de que a defasagem entre nós e os clássicos
reside também no modo como percebemos a sexualidade, e que o uso da poesia como
história é, no mínimo, imprudente ---, Safo dividia seu leito com mulheres e por
elas era tomada de paixão. Digo certa leitura, porque há outros modos, mais
seguros e coerentes com o que de fato sabemos do mundo antigo grego, de
compreender a prevalência de meninas --- e não mulheres ---  na poesia erótica de Safo; e digo
certa fatia de testemunhos antigos, porque naqueles que enfocam sua
sexualidade, Safo é retratada majoritariamente como promíscua e engajada em relações hetero, e não homoeróticas. Vejamos. 

A partir de Aristófanes (séculos \textsc{v}--\textsc{iv} a.C.) e da comédia clássica, pelo menos, a
referência às mulheres de Lesbos e o uso de verbos como \textit{lesbiázein} e
\textit{lesbízein} (“agir como uma mulher de Lesbos”) conotavam luxúria e
lascívia; em particular, diz Hallett, a prática da
felação, que as lésbias teriam inventado\footnote{ Hallett (1996, p. 129).} --- algo sem qualquer respaldo
histórico. A esse respeito, Gentili observa: 

\begin{quote}
Já na segunda metade do século \textsc{v} a.C. --- e seu uso é certamente bem mais antigo
--- as palavras \textit{Lésbia} ou \textit{moça de Lesbos} tinham a típica
conotação de \textit{fellatrix}, e não de \textit{Lésbica} no sentido moderno
do termo. \textit{Lesbís} e \textit{lesbiázein} eram essencialmente peças de
terminologia sociológica com um significado específico e inequivocamente
erótico.\footnote{ Gentili (1990a, p. 95).}
\end{quote}

Não há como precisar a razão dessa ligação das mulheres de Lesbos a práticas
sexuais específicas, mas talvez isso se deva à fama da beleza incomparável e da
sensualidade das mulheres da ilha, já atestada na \textit{Ilíada} (\textsc{ix}, 128--30),
e/ou à intensidade erótica da poesia de Safo. Seja como for, a habilidade
sexual --- com o sexo oposto --- atribuída às lésbias e a imagem da poeta parecem
emanar, confundidas, no uso do adjetivo “Lésbia” mesmo tarde, em Catulo
(século \textsc{i} a.C.) --- repare"-se, sempre em contextos heteroeróticos, frisa Hallett.\footnote{ Hallet 
(1996, (pp.~129--30). Ver ainda os estudos de Brasete (2003, pp. 17--26; 2009, pp. 289--303) e Ragusa (2019b, pp. 211--39).} 

A despeito dessas ressalvas, \textit{Safo de Lesbos} é designação que
recorrentemente projeta em nosso imaginário certa Safo, a poeta
\textit{lésbica}. E é essa projeção que vem sendo aqui
problematizada, uma vez que não só carece de respaldo em evidências antigas,
confiáveis e consistentes, mas que também resulta de uma percepção da
sexualidade que em muito se distingue da dos antigos, seja em largos termos ---
os concernentes à sexualidade feminina e à masculina ---, seja em termos
restritos, os relativos ao caso de Safo. 

Repare"-se, a propósito, que o adjetivo \textit{lésbica} não existia na
Antiguidade, tampouco o \textit{lesbianismo} em sua concepção moderna --- o amor
entre mulheres, que exclui o amor heterossexual, dado que a moderna
categorização do homossexualismo se faz a partir da oposição a este, lembra Mossé.\footnote{ Mossé (1991, p. 156).} Não há,
ademais, qualquer base com lastro mínimo na afirmação de que teria sido esta
uma opção sexual da poeta que, enquanto sujeito histórico, é pouco mais, se
tanto, do que uma neblina para nossos olhos, como já para os dos antigos. O
termo \textit{lésbica}, enfim, frisa Sue Blundell, “é uma
invenção moderna” que passa a nomear “uma mulher homossexual no final do século
\textsc{xix}, como resultado da publicidade criada por uma controvérsia acadêmica em
torno da sexualidade da própria Safo”\footnote{ Blundell (1995, p. 83).} --- polêmica esta que permanece tão acesa
nos debates quanto estéril nos seus resultados.
Lardinois precisa as datas em que, em
língua inglesa, o adjetivo se verifica, a partir de 1890, e o substantivo
\textit{lesbianismo} (homossexualismo feminino), primeiro em 1870, escrito com
inicial maiúscula, em clara alusão à ilha de Lesbos; e ele indaga:

\begin{quote}
Será justificada a relação entre a ilha de Lesbos e o homossexualismo das
mulheres? Existiriam razões para crer que Safo de Lesbos fosse uma ‘lésbica’?

É essa a Grande Questão Sáfica [\ldots{}] [que] já era debatida na Antiguidade, mas
os estudiosos ainda não foram capazes de chegar a um consenso. Provavelmente,
jamais o consigam, não apenas porque as evidências sejam muito escassas, mas
porque existe algo intrinsecamente errôneo na forma de colocar a questão.\footnote{ Lardinois (1995, pp.~27--8).}
\end{quote}

Concluindo seu estudo, afirma Lardinois:

\begin{quote}
Podemos concluir que, no caso de Safo, estamos, no máximo, diante de
relacionamentos breves entre uma mulher adulta e uma jovem prestes a se casar.
Chamar de ‘lésbicas’ essas relações é um anacronismo. É impossível avaliar se a
palavra se aplica à própria Safo ou à sua vida íntima. Na verdade, essa é uma
questão sem sentido. Mesmo se, pelos padrões modernos, Safo devesse ser
considerada lésbica, sua experiência deve ter sido muito diferente, vivendo,
como viveu, em uma era diferente com diferentes noções e tipos de sexualidade.\footnote{ Lardinois (1995, p. 50).}
\end{quote}

O grau elevado de complexidade da chamada “Questão Sáfica” é um alerta
fundamental contra afirmações simplistas como as que anunciam em Safo a
primeira poeta (engajadamente) \textit{lésbica} da literatura ocidental,
servindo"-se de modo igualmente simplista e redutor de sua poesia para provar a alegada
verdade do rótulo. E se não basta tudo o que aqui se disse, cito as palavras
enfáticas de Holt Parker: 

\begin{quote}
O texto de Safo está em fragmentos [\ldots{}] A linguagem é difícil, a sociedade,
obscura. Voltamo"-nos a manuais e comentários em busca de auxílio. Isso
significa, porém, que chegamos a Safo já cegos pelas assunções largamente não
examinadas de gerações prévias de estudiosos; e no caso de Safo, o acúmulo de
assunções é profundamente milenar e inclui comédias gregas, romances italianos
e pornografia francesa. O caso é pior com Safo do que com qualquer outro autor,
incluindo Homero. Pois aqui não lidamos apenas com a literatura arcaica, mas
com a sexualidade; os comentários são pesadamente carregados de emoção e de
nossos preconceitos. Mais importante, estamos lidando com homossexualidade (ou
melhor, o que construímos como homossexualidade) e sexualidade feminina.\footnote{ Parker (1996, p. 149).}
\end{quote}

As considerações de Parker e Lardinois são de válidas, embora raras, lucidez e
prudência, tanto mais se considerarmos que, segundo Blundell, em
textos antigos, a única “clara alusão a um comportamento homossexual de uma
mulher de Lesbos ocorre num diálogo entre duas prostitutas escrito pelo
satirista Luciano, no século \textsc{ii} d.C.”, o \textit{Diálogo das cortesãs}
(5).\footnote{ Blundell (1995, p. 82).} A
alusão é clara, mas sua historicidade está comprometida pela natureza
literária, mais do que isso, satírica, da fonte em que se encerra --- gêneros como a
comédia e a sátira, nunca é demais frisar, valem"-se com frequência da liberdade
para tratar do sexo para fazer rir e, no caso do segundo, que visa ao riso dos
cúmplices do satirista e à destruição do alvo de seu texto, da linguagem do
vitupério. 

Deve estar evidente, agora, por que evito termos como
\textit{lésbica}, por que deve ser relativizada a imagem da nomenclatura “Safo
de Lesbos”, e por que prefiro falar em homoerotismo, em vez de homossexualismo
que, entre as mulheres da Grécia antiga é muito difícil de ser estudado, dada a
escassez de fontes que induz, muitas vezes, às analogias decerto problemáticas
com o abundante material do atestado homossexualismo masculino --- que,
diga"-se logo, também não funciona entre os antigos como funciona no imaginário
que com olhos modernizantes o contempla. Isso tudo está bastante
bem explorado, documentado e analisado no estudo fundamental de Kenneth J.
Dover, originalmente publicado em 1978.\footnote{ Dover (1994).}

Quanto à referida preferência, reitero: embora haja em Safo forte erotismo
dirigido a outras mulheres --- mas não exclusivamente a elas ---, suas canções não
são registros biográficos ou documentos de sua sexualidade, mas discursos
poéticos em condição fragmentária, que compromete o alcance de nossas leituras,
filiados a determinado gênero poético, o mélico, e inseridos em certa tradição
poética de linguagem erótica. Estamos fadados a fracassar e a nos perder em
especulações, se tentarmos explicar Safo e sua poesia a partir das
características da vida cotidiana das mulheres em Lesbos, ou pela sua
biografia, mesmo porque teciam"-na os antigos extraindo"-a das canções de Safo, e
sem qualquer constrangimento preenchiam as lacunas biográficas com narrativas
que criavam segundo a verossimilhança, que é categoria discursiva. Mas entramos em terreno mais seguro
quando atentamos para o fato de que a poesia grega antiga é de composição
genérica, predominantemente: metro, matéria, linguagem, tema, tudo se articula
em termos das práticas dos gêneros e das expectativas que criam nas audiências.



\section*{A transmissão da mélica de Safo}

%\epigraph{Ó Píndaro, boca sacra das Musas, e loquaz Sirena --- %\\
%Baquílides! ---, e graças eólias de Safo, %\\
%e escrita de Anacreonte, e quem da fonte homérica %\\
%extraiu sua própria obra --- Estesícoro! ---, %\\
%e doce página de Simônides, e quem de Peitó e dos %\\
%meninos colheu a doce flor --- Íbico! ---, %\\
%e espada de Alceu, que o sangue de tiranos muitas vezes %\\
%derramou, protegendo as leis da pátria, %\\
%e rouxinóis de suaves cantos de Álcman --- sede graciosos, vós %\\
%que fincastes o início e o fim de toda a lírica. %\\
%\mbox{}\hfill (\textsc{\textit{ap}} \textsc{ix}, 184)}
%
%\smallskip
%
%\epigraph{Gritou alto de Tebas Píndaro; soprou deleites %\\
%com voz doce"-mel a musa de Simônides; %\\
%brilha Estesícoro e também Íbico; era doce Álcman; %\\
%deleitáveis sons dos lábios entoou Baquílides; %\\
%e Peitó falou junto a Anacreonte; e coisas variegadas canta %\\
%Alceu, cisne lésbio na Eólida; %\\
%e dentre os homens Safo não é a nona, mas entre as amáveis %\\
%Musas a décima Musa registrada. %\\
%\mbox{}\hfill (\textsc{\textit{ap}} \textsc{ix}, 571)}%


Não podemos precisar as razões que favoreceram ou não a
preservação das obras dos poetas gregos. Mas somou"-se aos fatores favoráveis,
em princípio, junto à reputação do poeta e outros elementos, sua edição na
Biblioteca de Alexandria. Aliás, ao tratar do termo \textit{lírica}, disse"-o
tardio, porque seu uso nos remete justamente a esse trabalho de cópia e estudo
que, no caso dos poetas mélicos, teve em Aristófanes de Bizâncio
(\textit{c.} 258--180 a.C.) seu principal executor. Segundo Pfeiffer, desse
erudito pode ser a autoria do cânone dos “nove líricos”,
dado nos dois epigramas declamatórios anônimos que citei
acima.\footnote{ Pfeiffer (1998, p. 205). Traduções: Ragusa (2010, pp.~27--8). Peitó (v. 5 de cada
epigrama) é a deusa Persuasão.}

A edição dos mélicos listados, além de tardia, seguiu critérios variados e
arbitrários, para nós nem sempre discerníveis: a compilação de Safo pautou"-se
pelo critério métrico, e foi dividida no eloquente número de nove
livros\footnote{ O nono livro seria de epitalâmios, segundo uma hipótese cuja
aceitação não é consensual; ver Lesky (1995, pp.~168--9). Sobre os critérios
adotados para a edição de Safo, ver ainda Nicosia (1976, pp.~31--2).}  --- rolos de papiros; já seu contemporâneo Alceu pode ter
tido dez, mas o número é incerto e os critérios de
organização, ignorados; as obras de outros, como as de Píndaro e Baquílides,
poetas dos séculos \textsc{vi}--\textsc{v} a.C., organizaram"-se pelo gênero mélico, o
que é muito problemático, pois a definição é usualmente discutível,
em virtude da ausência de sistematização deles antes dos trabalhos na
Biblioteca.\footnote{ Ver os estudos que discutem as definições de gêneros entre os
antigos: Harvey (1955, pp.~157--75), Rossi (1971, pp.~69--94), Calame (1974, pp.
113--28).}

Antes da Biblioteca, a circulação da poesia grega antiga, incluindo a jâmbica,
elegíaca e mélica, foi viabilizada, como vimos, por \textit{performances} e
\textit{reperformances} --- nos mesmos moldes ou não, profissionais ou amadoras ---, pela
simples repetição propiciada pela memória,\footnote{ Ver Herington (1985, pp.
45--8) a respeito.} por inscrições comemorativas em monumentos, por possíveis
edições\footnote{ Harvey (1955, p. 159) afirma que “não há razão para pensar
que as edições alexandrinas foram as primeiras a existir”; e na Atenas clássica
circulavam edições disponíveis dos grandes poetas.} --- termo que deve ser
entendido como cópias de um registro original, em quantidade e difusão muito
restritas e, certamente, custosa.\footnote{ Ver Havelock (1996, p. 26). Tais
cópias eram feitas sobretudo em papiro, material do qual o Egito, sua fonte,
detinha o monopólio, e que, a partir do século \textsc{vi} a.C., adentra o mundo heleno.
O “livro” é, na verdade, um \textit{bíblos} ou \textit{biblíon}, isto é, rolo
de papiro, sendo tardio o formato do \textit{codex}, do século \textsc{ii} d.C. Sobre
os copistas, muitos devem ter sido escravos, e não necessariamente saberiam ler
o que copiavam.} Tudo isso contribui para a sobrevivência dos textos até os
alexandrinos e para o trabalho destes em suas próprias edições e classificações
numa época em que mudaram demais “as condições fundamentais de produção
poética, assim como a relação entre o poeta e sua audiência”, anota Clay.\footnote{ Clay (1998, p. 28).} 

Tendo a obra de Safo se inserido nesse cenário geral --- que não pode ser mais
específico, por falta de conhecimento nosso --- de circulação e
preservação, como nos foi transmitida? Como chegou até nós? Por dois caminhos
trilhados por toda a literatura produzida na Grécia antiga: por fontes de
transmissão direta --- papiros, manuscritos, inscrições em monumentos, e assim
por diante --- e por fontes de transmissão indireta --- citações. Vejamos. 

Desde a década final do século \textsc{xix} a meados do século \textsc{xx}, sobretudo,
intensos trabalhos de escavações conduzidos no Egito trouxeram à luz uma
incrível massa de papiros literários e não literários, provindos,
majoritariamente, da cidade de Oxirrinco que, para Salvatore Nicosia, “tinha
estreito contato com Alexandria”; diz ele ainda que, “em geral, os
textos lá descobertos reportam à atividade filológica e crítica dos grandes
gramáticos alexandrinos”.\footnote{ Nicosia (1976, p. 32).}

Com os acréscimos, que hoje ocorrem em ritmo bem mais lento --- em 2005, foi
publicada uma nova elegia de Arquíloco, o fragmento do Télefo, contido no
\textit{Papiro de Oxirrinco} 4708 (século \textsc{ii} d.C.),\footnote{ Ver tradução de
Corrêa (2009, pp.~337--9).} e em 2004, uma nova canção de Safo, o fragmento da
``Canção sobre a velhice'', preservado no \textit{Papiro de Colônia} 21351 (século \textsc{ii} d.C.) ---,
muitas obras passaram a ser de fato conhecidas, outras ganharam mais
substância, como a da poeta lésbia. Por outro lado, o volume recuperado demanda
o reconhecimento das pesadas perdas sofridas, com as quais devemos conviver. A
condição do \textit{corpus} da poesia jâmbica, elegíaca e mélica, em termos
quantitativos, melhorou --- o estado material dos papiros, porém, trouxe textos
em geral precários; Frederic G. Kenyon frisa:
“É no período lírico, talvez, que as nossas perdas foram maiores; e aqui os
papiros não fizeram muito por nós”.\footnote{ Kenyon (1919, p. 9).} Animado com os acontecimentos então
recentes, a despeito das frustrações, Kenyon afirmava, ao final de seu
artigo: 

\begin{quote}
Verdadeiramente, para todos aqueles que amam a literatura e reconhecem
na literatura grega a mais alta expressão do pensamento humano, os desertos do
Egito floresceram como uma rosa.\footnote{ Kenyon (1919, p. 13).}
\end{quote}


Quase cinquenta anos depois, William H.~Willis oferecia ao
leitor um censo dos papiros literários encontrados no Egito, com cerca de 3000
exemplares publicados.\footnote{ Willis (1968, pp.~205--41).} Eis sua avaliação: 

\begin{quote}
Devemos, é claro, ter em mente as severas limitações de nossa evidência. Quase
todos os nossos papiros vêm de uma única província do mundo greco"-romano; e o
Egito, de muitas maneiras --- na geografia, na tradição e no isolamento político
--- foi uma província atípica. Tampouco podem os nossos textos preservados
derivar uniformemente de todo o Egito. Uma vez que a sobrevivência dos papiros
depende da completa proteção da umidade, as chuvas de Alexandria e a nascente
do Delta, as inundações anuais do Nilo, a irrigação, e o crescimento gradual do
lençol freático ao longo dos séculos --- para não mencionar os inimigos naturais
--- devem, necessariamente, ter"-nos roubado a vasta maioria dos textos antigos.\footnote{ Willis (1968, pp.~205--6).}
\end{quote}

Além dos fatores relativos ao clima, há que se considerar a sorte, as limitações
relacionadas às próprias escavações e o interesse das equipes quanto ao que
gostariam de ver renascer das areias egípcias. Tudo somado, temos uma medida
dos estragos sofridos pelos papiros: a maior parte desapareceu, e os que
sobreviveram estão corrompidos, mutilados, demasiado escurecidos. Mesmo assim,
tê"-los descoberto foi grande fortuna; e grande foi a sorte dos jâmbicos,
elegíacos e mélicos, que contaram com os esforços de Edgar Lobel, helenista
inglês que trabalhou intensamente com os papiros de Oxirrinco, recorda Willis,
em cujo censo os de Safo concentram"-se nos períodos
romano (31 a.C.--476 d.C.) e bizantino (476--1453).\footnote{ Willis (1968, pp.~211--3).}

Quanto às fontes de transmissão indireta, paráfrases e citações em escritos
antigos variados, Nicosia observa que devem ter dependido
sobretudo da memória falível e seletiva de quem cita, da versão do texto por
ele conhecida e/ou disponível em cópia escrita, e das suas necessidades para o
uso dos textos citados, as quais influíram no tamanho destes, em geral
reduzido. Tais textos sofreram ainda, lembra ele, alterações decorrentes da
aticização dos dialetos nos quais os poemas foram compostos --- no caso de Safo,
o lésbio"-eólico, e não o ático que, em parte pelo impulso de uma Atenas
culturalmente muito poderosa na era clássica, prevaleceu sobre os demais
dialetos gregos. Não obstante os problemas, a maioria dos poetas
arcaicos, notadamente, têm nesse tipo de transmissão uma grande aliada.\footnote{ Nicosia (1976, pp.~23--5).}

Diante desse quadro, os textos que contam com mais de uma fonte comumente
apresentam variações, diferenças, que precisam ser resolvidas por escolhas do
editor no trabalho com as obras, ressalta Nicosia.\footnote{ Nicosia (1976, pp.~28).} Insere"-se na lista
de dificuldades do trabalho com a poesia jâmbica, elegíaca, mélica
ainda isto: o problema do estabelecimento dos textos fragmentários, salvo raras
exceções. 

%Vamos, pois, às canções de Safo.


\begin{bibliohedra}
\tit{battistini}, Y. (introd., trad., notas). \textit{Poétesses grecques:
Sapphô, Corinne, Anytè \ldots{}}. Paris: Imprimerie Nationale Éditions, 1998.

\tit{bennett}, C. “Concerning ‘Sappho schoolmistress’\,”. \textit{TAPhA} 124,
1994, pp.~345--7.

\tit{blundell}, S. \textit{Women in ancient Greece.} London: British Museum
Press, 1995.

\tit{bowie}, A. M. \textit{The poetic dialect of Sappho and Alcaeus.} Salem:
Ayer, 1984. 

\tit{bowie}, E. L. “Early Greek elegy, symposium and public festival”.
\textit{JHS} 106, 1986, pp.~13--35.

\tit{bowra}, C. M. \textit{Greek lyric poetry.} 2ª
ed. Oxford: Clarendon Press, 1961.

\tit{bremmer}, J. N. “Pederastia grega e homossexualismo moderno”. In:
\line(1,0){25}. (org.). \textit{De Safo a Sade: momentos na história da
sexualidade.} Campinas: Papirus, 1995, pp.~11--26.

\tit{budelmann}, F. “Introducing Greek lyric”. In:
\line(1,0){25}. (ed.). \textit{The Cambridge
Companion to Greek lyric.} Cambridge: Cambridge University Press, 2009, pp.
1--18.

\tit{burnett}, A. P. \textit{Three archaic poets: Archilochus, Alcaeus,
Sappho.} Cambridge: Harvard University Press, 1983. 

\tit{calame}, C. “Refléxions sur les genres littéraires en Grèce archaïque”.
\textit{QUCC} 17, 1974, pp.~113--28.

\tit{campbell}, D. A. (ed. e trad.). \textit{Greek lyric \textsc{i}.} Cambridge:
Harvard University Press, 1994. [1ª ed.: 1982].

\titidem. (ed. e trad.). \textit{Greek lyric \textsc{iv}.} Cambridge: Harvard
University Press, 1992.

\titidem. (coment.). \textit{Greek lyric poetry.} London: Bristol
Classical Press, 1998. [1ª ed.: 1967]. 

\tit{cantarella}, E. \textit{Pandora’s daughters. The rôle and status of
women in Greek and Roman antiquity.} Trad. M. B. Fant. New York: Penguin, 1991.

\tit{carey}, C. “Genre, occasion and performance”. In:
\textsc{budelmann}, F. (ed.). \textit{The Cambridge Companion to Greek
lyric.} Cambridge: Cambridge University Press, 2009, pp.~21--38.

\tit{carson}, A. \textit{Eros, the bittersweet: an essay.} Chicago:
Dalkey Archive Press, 1998.

\tit{clay}, D. “The theory of the literary \textit{persona} in Antiquity”.
\textit{MD} 40, 1998, pp.~9--40.

\tit{cole}, S. G. “Could Greek women read and write?”. In: \textsc{foley}, H.
P. (ed.). \textit{Reflections of women in antiquity.} Philadelphia: Gordon and
Breach, 1992, pp.~219--45.

\textsc{corrêa}, P. da C. \textit{Armas e varões: a guerra na
lírica de Arquíloco.} 2ª ed. revista e ampliada. São Paulo: Ed. da Unesp, 2009.

\titidem. \textit{Um bestiário arcaico: fábulas e imagens
de animais na poesia de Arquíloco}. Campinas: Editora da Unicamp, 2010. (Apoio:
Fapesp)

\tit{d’alessio}, G. B. “Past, future and present past: temporal
\textit{deixis} in Greek archaic lyric”. \textit{Arethusa} 37, 2004, pp.
267--94.

\tit{davies}, M. (ed.). \textit{Poetarum melicorum Graecorum fragmenta \textsc{i}.}
Oxford: Clarendon Press, 1991.

\tit{de martino}, F. “Appunti sulla scrittura al femminile nel mondo antico''.
In: \line(1,0){25}. (ed.). \textit{Rose} \textit{de Pieria.} Bari: Levante Editori,
1991, pp.~17--75.

\tit{dover}, K. J.  \textit{A homossexualidade na Grécia antiga.}
Trad. L. S. Krausz. São Paulo: Nova Alexandria, 1994. [1ª ed. orig.: 1978].

\tit{easterling}, P. E.; \textsc{knox}, B.W. (ed.). \textit{The Cambridge History of
classical literature --- \textsc{i}: Greek literature.} Cambridge: Cambridge University
Press, 1990.

\tit{ferrari}, F. \textit{Sappho’s gift: the poet and her community.} Trad.
B. Acosta"-Hughes e L. Prauscello. Ann Arbor: Michigan University Press, 2010.

\tit{foley}, H. P. (ed.). \textit{Reflections of women in antiquity.}
Philadelphia: Gordon and Breach, 1992.

\tit{fränkel}, H. \textit{Early Greek poetry and philosophy.} Trad. M. Hadas
e J. Willis. Oxford: Basil Blackwell, 1975. [1ª ed. orig.: 1951]. 

\tit{gentili}, B. \textit{Poetry and its public in ancient Greece.} Trad. A.
T. Cole. Baltimore: The Johns Hopkins University Press, 1990a.
[1ª ed. orig.: 1985]

\titidem. “Lo ‘io’ nella poesia lirica greca”. \textit{AION (filol)} 12,
1990b, pp.~9--24.

\tit{guerrero}, G. \textit{Teorías de la lírica.} México: Fondo de
Cultura Económica, 1998.

\tit{johnson}, W. R. \textit{The idea of lyric. Lyric modes in ancient and
modern poetry.} Berkeley: University of California Press, 1982.

\tit{hallett}, J. P. “Sappho and her social context”. In: \textsc{greene}, E.
(ed.). \textit{Reading Sappho: contemporary approaches.} Berkeley: University
of California Press, 1996, pp.~125--42.

\tit{harvey}, A. E. “The classification of Greek lyric poetry”. \textit{CQ}
5, 1955, pp.~157--75.

\tit{havelock}, E. \textit{A revolução da escrita na Grécia e suas
consequências culturais.} Trad. O. J. Serra. São Paulo, Rio de Janeiro: Editora
da Unesp, Paz e Terra, 1996.

\tit{henderson}, W. J. ``Received responses: ancient testimony on Greek lyric imagery.'' 
\textit{AClass} 41, 1998, pp.~5--27. 

\tit{herington}, J. \textit{Poetry into drama.} Berkeley: University of
California Press, 1985.

\tit{kenyon}, F. G. “Greek papyri and classical literature”.
\textit{JHS} 39, 1919, pp.~1--15.

\tit{lardinois}, A. “Safo lésbica e Safo de Lesbos”. In: \textsc{bremmer}, J.
(org.). \textit{De Safo a Sade: momentos na história da sexualidade.} Campinas:
Papirus, 1995, pp.~27--50.

\tit{lesky}, A. \textit{História da literatura grega.} Trad. M. Losa. Lisboa:
Fundação Calouste Gulbenkian, 1995. [1ª ed. orig.: 1957].

\tit{lourenço}, F. (trad.). \textit{Poesia grega de Álcman a Teócrito.}
Lisboa: Livros Cotovia, 2006.

\tit{mossé}, C. \textit{La femme dans la Grèce antique.} Paris: Éditions
Complexe, 1991.

\tit{most}, G. W. “Greek lyric poets”. In: \textsc{luce}, T. J. (ed.). \textit{Ancient writers -- \textsc{i}: Greece and Rome}. New York: Charles Scribner's Sons, 1982, pp. 75--98. 

\titidem. ``Reflecting Sappho.'' In: \textsc{greene}, E.~(ed.) 
\textit{Re"-reading Sappho: reception and transmission}. Berkeley: University of California Press,
1996, pp.~11--35.

\tit{murray}, O. “Sympotic history”. In: \line(1,0){25}. (ed.).
\textit{Sympotica. A symposium on the symposion.} Oxford: Clarendon Press,
1990, pp.~3--13.

\titidem. \textit{Early Greece.} 2ª ed.
Cambridge: Harvard University Press, 1993.

\tit{nicosia}, S. \textit{Tradizione testuale diretta e indiretta dei poeti
di Lesbo.} Roma: Ateneo, 1976. 

\tit{oliva neto}, J. A. (trad., introd. e notas). \textit{Catulo. O Livro de
Catulo.} São Paulo: Edusp, 1996.

\tit{oliveira}, F. R. (introd., trad. e notas). \textit{Hipólito. Eurípides.}
São Paulo: Odysseus, 2010.

\tit{paes}, J. P. (trad., notas, posfácio). \textit{Poemas da Antologia grega
ou palatina, séculos \textsc{vii} a.C. a \textsc{v} d.C.} São Paulo: Companhia das Letras, 1995.

\tit{page}, D. L. (ed.). \textit{Poetae melici Graeci.} Oxford: Clarendon
Press, 1962.

\titidem. \textit{Sappho and Alcaeus.} Oxford: Clarendon
Press, 2001. [1ª ed.: 1955].

\tit{parker}, H. “Sappho schoolmistress”. In: \textsc{greene}, E. (ed.).
\textit{Re"-reading Sappho: reception and transmission.} Berkeley: University of
California Press, 1996, pp.~146--83.

\tit{pfeiffer}, R. \textit{A history of classical scholarship --- \textsc{i}.} Oxford:
Clarendon Press, 1998. [1ª ed.: 1968]

\tit{ragusa}, G. \textit{Fragmentos de uma deusa: a representação de Afrodite
na lírica de Safo}. Campinas: Editora da Unicamp, 2005. (Apoio: Fapesp).

\titidem. \textit{Lira, mito e erotismo: Afrodite na poesia mélica grega
arcaica}. Campinas: Editora da Unicamp, 2010. (Apoio: Fapesp).

\tit{robb}, K. \textit{Literacy and paideia in ancient Greece.} New York,
Oxford: Oxford University Press, 1994.

\tit{rösler}, W. “Persona reale o persona poetica?”. \textit{QUCC} 19, 1985,
pp.~131--44.

\tit{rossi}, L. E. “I generi letterari e le loro leggi scritte e non scritte
nelle letterature classiche”. \textit{BICS} 18, 1971, pp.~69--94.

\tit{schimitt"-pantel}, P. “Sacrificial meal and symposion: two models of
civic institutions in the archaic city?” In: \textsc{murray}, O. (ed.).
\textit{Sympotica. A symposium on the symposion.} Oxford: Clarendon Press,
1990, pp.~14--33.

\tit{shapiro}, H. A. “Introduction”. In: \line(1,0){25}.
(ed.). \textit{The Cambridge Companion to archaic Greece.} Cambridge: Cambridge
University Press, 2007, pp.~1--9.

\tit{skinner}, M. B. “Woman and language in archaic Greece, or, Why is Sappho
a woman?”. In: \textsc{greene}, E. (ed.). \textit{Reading Sappho: contemporary
approaches.} Berkeley: University of California Press, 1996, pp.~175--92.

\tit{slings}, S. R. “The \textit{I} in personal archaic lyric: an
introduction”. In: \line(1,0){25}. (ed.). \textit{The poet’s I in archaic Greek
lyric.} Amsterdam: VU University Press, 1990, pp.~1--30.

\tit{snell}, B (ed.). \textit{A cultura grega e as origens do pensamento
europeu.} Trad. P. de Carvalho. São Paulo: Perspectiva, 2001. [1ª ed. orig.:
1955].

\tit{snyder}, J. M. \textit{The woman and the lyre: women writers in
classical Greece and Rome}. Carbondale: Southern Illinois University Press,
1989.

\tit{stehle}, E. “Romantic sensuality, poetic sense”. In: \textsc{greene}, E.
(ed.). \textit{Reading Sappho: contemporary approaches.} Berkeley: University
of California Press, 1996, pp.~143--9.

\tit{svenbro}, J. \textit{Phrasiklea. An anthropology of reading in ancient
Greece.} Trad. J. Lloyd. Ithaca: Cornell University Press, 1993.

\tit{torrano}, J. (estudo e trad.). \textit{Hesíodo. Teogonia --- A origem dos
deuses.} 5ª ed. São Paulo: Iluminuras, 2003.

\tit{tsagarakis}, O. \textit{Self"-expression in early Greek lyric, elegiac
and iambic poetry.} Wiesbaden: Franz Steiner, 1977.

\tit{vetta}, M. “Poesia simposiale nella Grecia arcaica e classica”. In:
\line(1,0){25}. (ed.). \textit{Poesia e simposio nella Grecia arcaica.}
Bari: Laterza, 1995, pp.~xi--lx.

\tit{west}, M. L. “Corinna”. \textit{CQ} 20, 1970, pp.~277--87.

\titidem.“Greek poetry 2000--700 \textsc{b.c.”.} \textit{CQ} 23, 1973,
pp.~179--92.

\titidem. (ed.). \textit{Iambi et elegi Graeci.} Oxford: Oxford University
Press, 1998. vols. 1--2. [1ª ed.: 1971].

\tit{willis}, W. H. “A census of the literary papyri from Egypt”.
\textit{GRBS} 9, 1968, pp.~205--41.



\section*{adendo bibliográfico} %à 2ª edição


\tit{BOWMAN}, L. “The ‘women’s tradition’ in Greek poetry”. \textit{Phoenix} 58, 2004, pp. 1--27.

\tit{BRASETE}, M. F. “O amor na poesia de Safo”. In: \textsc{ferreira}, A. M. (ed.). \textit{Percursos de Eros -- representação do erotismo}. Aveiro: Universidade de Aveiro, 2003, pp. 17--26.

\titidem. “Homoerotismo feminino na lírica grega arcaica: a poesia de Safo”. In: \textsc{fialho}, M. do Céu et alii (eds.). \textit{A sexualidade no mundo antigo}. Lisboa, Coimbra: Centro de História da Universidade de Lisboa/Centro de Estudos Clássicos e Humanísticos U. Coimbra, 2009, pp. 289--303. 

\tit{BRUSSE}, J. S. “Epigram”. In: \textsc{clauss}, J. J.; \textsc{cuypers}, M. (eds.). \textit{A companion to Hellenistic literature}. Malden: Wiley"-Blackwell, 2010, pp. 117--35. 

\tit{BUZZI}, S. et alii (eds.). \textit{Nuove acquisizioni di Saffo e della lirica greca}. Alessandria: Edizioni dell'Orso, 2008.

\tit{CAZZATO}, V.; \textsc{lardinois}, A. (eds.) \textit{The look of lyric: Greek song and the visual. Studies in archaic and classical Greek song, vol. 1}. Leiden: Brill, 2016.

\tit{GREENE}, E.; \textsc{skinner}, M. B. (eds.). \textit{The new Sappho on old age. Textual and philosophical issues}. Washington, \textsc{d.c.}: Center for Hellenic Studies, 2009.

\tit{KLINCK}, A. \textit{Woman’s song in ancient Greece}. Montreal: McGill-Queen’s University Press, 2008. 

\tit{LARDINOIS}, A. “Subject and circumstance in Sappho’s poetry”. \textit{TAPhA} 124, 1994, pp. 57--84.

\titidem. “Who sang Sappho’s songs?”. In: \textsc{greene}, E. (ed.). \textit{Reading Sappho}. Berkeley: University of California Press, 1996, pp. 150--72.

\titidem. ``Lesbian Sappho reviseted''. In: \textsc{dijkstra}, J. et alii (eds.). \textit{Myths, Martyrs, and Modernity. Studies in the History of Religions in Honour of Jan N. Bremmer}. Leiden: Brill, 2010, pp. 13--30.

\tit{RAGUSA}, G. (org., trad.). \textit{Lira grega: antologia de poesia arcaica}. São Paulo: Hedra, 2013.

\titidem. “Memória, a terra prometida dos poetas: o tema na mélica grega arcaica”. \textit{Forma Breve} 15, 2018, pp. 143--52. 

\titidem. “A coralidade e o mundo das \textit{parthénoi} na poesia mélica de Safo”. \textit{Revista Aletria} 29.4, 2019a, pp. 85--111. 

\titidem. “Safo de Lesbos: de liras e neblinas”. In: \textsc{rede}, M. (org.). \textit{Vidas Antigas. Ensaios Biográficos da Antiguidade}. São Paulo: Editora Intermeios, 2019b, pp. 211--39.

\tit{REEDER}, E. D. (ed.). \textit{Pandora’s box. Women in classical Greece}. Baltimore, Princeton: The Walters Art Gallery/University Press, 1995.

\end{bibliohedra}