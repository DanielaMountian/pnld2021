\chapter{Nota à segunda edição}

Quase uma década decorreu da publicação deste volume de traduções,
acompanhadas de pequenos comentários, dos fragmentos da poesia mélica
arcaica de Safo, precedidos por longa introdução abarcando a poeta, seu
universo cultural, o gênero poético que praticou, as demais poetas
mulheres e a transmissão do \emph{corpus} de sua obra até nós.

A antologia teve por critério oferecer ao leitor todo fragmento
minimamente legível. Isso significa a exclusão de fragmentos em que mal
vemos uma palavra inteira preservada, de outros em que há apenas
palavras esparsas que, poucas e soltas, mal permitem a reconstituição de
qualquer leitura, e ainda de outros que contêm uma única palavra ---
estes incluídos somente quando podem integrar um conjunto que viabilize
alguma compreensão. Logo, a palavra ``antologia'' é usada em termos
relativos: não indica uma escolha de textos mais famosos ou algo desse
tipo, mas uma reunião dos textos que podemos ler e entender pelo menos
de modo a vislumbrar algo de palpável.

O critério se mantém, por certo, mas alguns fragmentos, reconsiderados,
foram incluídos desta vez, e distribuídos nas seções temáticas já
existentes ou abertas para acomodá"-los e a alguns outros que foram
reorganizados.

Todo o material passou por revisão, e mesmo as traduções, em alguns
casos, trazem pequenas modificações nesta edição. Para alguns, a
tradução é tarefa que, uma vez feita, não mais se altera. Para mim, a
tradução é tarefa contínua, porque articulada à compreensão, à análise,
e mesmo à sensibilidade ao texto original que, quero crer, sofre os
efeitos positivos do amadurecimento da tradutora. Claro, restrinjo as
alterações ao mínimo necessário --- ou irresistível na oportunidade que
me é dada. Mas elas se fazem presentes, aqui e ali, no conjunto dos
fragmentos apresentados.

A revisão da introdução detalhada que preparei quando da 1ª edição não
se limitou à redação --- que, seguramente, sempre pode melhorar ---, mas
procurou rever certos problemas e elaborações que hoje me pareceram
insatisfatórias. De novo, só interferi naquilo que de fato merecia ser
reformulado, sobretudo o que é relativo à nova e mais consistente
apreciação da coralidade na mélica de Safo e na natureza do grupo por
ela liderado de \emph{parthénoi} --- o termo que nomeia tecnicamente o
estágio transicional das moças não casadas, virgens, mas já na puberdade
e, portanto, prontas para a boda (\emph{gámos}). Tal apreciação é fruto
do impacto nos estudos da poeta causado pelas pesquisas sobre a canção
coral e a \emph{performance}, e, sobretudo, da descoberta em 2004 de um
novo fragmento, a ``Canção sobre a velhice'' --- a única não numerada,
porque, recente, não está incluída nas edições de autoridade ---, o
último traduzido neste volume.

No mundo acadêmico, a produção de conhecimento não para, e não há que
ansiar por verdades inamovíveis, absolutas --- menos ainda no universo
das Humanidades, que lidam com a cultura e seus objetos, como a poesia,
e no campo das Letras Clássicas que, vez por outra, é obrigado a rever
teorias, posturas e o mais, seja porque algo novo vem à luz, como o referido fragmento sáfico, seja porque para algo se elabora uma compreensão mais
sólida.

Esta 2ª edição deu"-me a chance de mostrar esse movimento natural e
esperado das pesquisas em torno de Safo e da mélica, que estimulou
estudos recentes sobre a poeta, incluindo alguns meus, listados no
último acréscimo que fiz a este volume --- um adendo à bibliografia
inicialmente lançada.

Que o leitor possa, com a mediação deste trabalho de tradução organizada
e abordagem contextualizada, admirar uma das maiores vozes da poesia que
atravessa os séculos, e que já os antigos poetas celebravam com versos
como estes que deixo, encerrando estas breves linhas.

\begin{verse}
Ó Safo, o mais doce travesseiro das paixões \qb{}aos jovens que amam,\\
a ti, junto às Musas, a Piéria adorna, ou o\\
Hélicon coberto de hera --- a ti que sopras tal \qb{}qual\\
elas, a ti, Musa na Ereso eólia.\\
Ou Hímen Himeneu, portando sua tocha \qb{}brilhante,\\
contigo fica sobre o tálamo nupcial;\\
ou junto a Afrodite enlutada, lamentando o \qb{}jovem rebento de\\
Ciniras, contemplas o bosque sacro dos \qb{}venturosos.\\
Em toda parte, ó soberana, te saúdo como aos \qb{}deuses, pois tuas canções\\
ainda hoje consideramos filhas dos imortais.\\[5pt]
(Dioscúrides, século \versal{III} a.C., epigrama 407 do livro \versal{VII} da
\emph{Antologia palatina})
\end{verse}
