

\thispagestyle{empty}
\mbox{}\vfill
{\noindent\itshape “A Chapter on Dreams” foi publicado pela primeira vez em 1892, 
e é uma reflexão posterior sobre o processo criativo de Stevenson através 
do sonho, que não ocorreu apenas em \emph{Jekyll e Hyde}, mas, como ele deixa 
claro, em muitas outras histórias suas.  O caráter extremamente vívido desses 
sonhos e a sua duração parecem ser propícios a sua transposição para a literatura.  
De certo modo, o sonho dispensa o escritor de inventar: no momento em que desperta, 
a história já está pronta em sua mente, e cabe"-lhe apenas o trabalho intelectual e físico de escrevê"-la.   
Este artigo é ainda hoje um dos textos fundamentais para se compreender a 
relação entre sonho e literatura.}

\chapter[Um capítulo sobre o sonho\\ \textit{R.L.~Stevenson}]{Um capítulo sobre o sonho\subtitulo{R.L.~Stevenson}}
\hedramarkboth{um capítulo sobre o sonho}{r.l.~stevenson}

\textsc{O passado} possui uma única textura, seja ela fingida ou experimentada,
seja vivida em três dimensões ou apenas presenciada naquele pequeno
teatro mental que mantemos brilhantemente iluminado durante toda a
noite, depois que as luzes se apagam, e a escuridão e o sono reinam,
sem serem perturbados, sobre o restante do corpo.  Não há uma distinção
clara entre as nossas variadas experiências; uma é mais vívida, a outra
opaca, uma é agradável, outra é dolorosa à lembrança; mas qual delas
constitui o que chamamos verdade, e qual delas é um sonho, é algo de
que não temos como prova sequer um fio de cabelo.  O passado é algo que
se equilibra de modo precário; basta uma mudança ínfima no campo da
metafísica para que nos vejamos destituídos dele.  São poucas as
famílias cujo conhecimento de si mesmas remonta a mais de quatro
gerações, mas ainda assim muitas são capazes de alegar direitos sobre
algum obscuro título de nobreza, algum castelo, alguma propriedade; uma
reivindicação que não poderiam provar diante de um tribunal, mas que
lhes gratifica a fantasia e lhes serve de passatempo durante os
momentos de ócio.  A certeza de um homem sobre seu próprio passado é
ainda menos sólida do que essa.  Um documento pode vir a ser descoberto
(como costuma acontecer nos romances) numa gaveta secreta de uma velha
escrivaninha de ébano, e devolver à vossa família honrarias antigas, e
restituir à minha a propriedade de uma certa ilhota nas Índias
Ocidentais (não longe de St.~Kitt’s, como uma cara tradição familiar
sussurrava aos meus ouvidos) que um dia já foi nossa e agora está em
outras mãos, ilhota que aliás (dada a situação atual do mercado do
açúcar) já não tem grande valor para quem quer que seja.  Não afirmo
que reviravoltas dessa natureza se deem com frequência; mas nenhum
homem pode afirmar que sejam impossíveis;  nosso passado, por outro
lado, está perdido para sempre; nossos dias e nossas ações de tempos
atrás, a pessoa que fomos, e o próprio mundo em que aqueles fatos
ocorreram, tudo se diluiu até tornar"-se igual ao tênue resíduo de um
sonho que tivemos na noite anterior, algumas imagens descontínuas, e um
eco distante nas câmaras do cérebro.  Nada somos capazes de resgatar,
nem um momento, nem uma sensação, nem um olhar; tudo se perdeu, sem
possibilidade de recuperação.

E no entanto imaginemos que fôssemos privados disso que nos resta; que
esse delgado fio de memória que se estende atrás de nós se rompesse ao
roçar na borda do nosso bolso; como nos veríamos reduzidos à mais nua          
das nulidades!  Porque só nos guiamos, e só nos conhecemos, graças a
essas reproduções fantasmagóricas do nosso passado.

Em vista disso, alguns de nós alegam ter vivido vidas mais longas e mais
ricas do que as vidas dos seus semelhantes; afirmam que, quando
adormecem, continuam em atividade; e entre os tesouros da memória que
todos os homens revisitam para seu próprio entretenimento, eles não
relegam a segundo plano a colheita dos seus sonhos.  Existe um
indivíduo desse tipo que observo agora, e cujo caso talvez mereça ser
relatado, de tão pouco usual que é.  Desde criança ele costumava ter
sonhos intensos e desconfortáveis.  Quando lhe sobrevinha uma pequena
febre à noite, e o quarto parecia dilatar"-se e encolher"-se, e suas
roupas, penduradas num prego, ora se agigantavam até as dimensões de
uma igreja, ora se reduziam ao horror de uma infinita distância e
infinita pequenez, aquela pobre alma tinha consciência do que iria se
seguir, e lutava contra a aproximação daquele sono que era o começo das
suas desventuras.

Mas essa sua luta era em vão; cedo ou tarde a feiticeira noturna o
agarrava pela garganta e o arrastava, sufocado e aos gritos, para o
sono.  Seus sonhos eram às vezes bastante comuns, outras vezes
estranhos, às vezes quase sem forma; ele se sentia amedrontado, por
exemplo, por nada mais que uma certa tonalidade de marrom, que não lhe
produzia o mínimo desconforto quando estava desperto, mas que lhe dava
medo e repulsa durante o sonho; outras vezes, os sonhos desenhavam"-se
nos mínimos detalhes e circunstâncias, como certa vez quando ele sonhou
que iria ser forçado a engolir o mundo e toda sua população, o que o
fez despertar gritando ao terror dessa ideia.  Os dois maiores
tormentos de sua vida tão limitada --- o tormento prático e cotidiano dos
seus deveres escolares, e aquele outro, menos palpável e mais profundo,
da ideia do inferno e do Juízo Final --- foram mais de uma vez fundidos
num único pesadelo aterrorizante.  Ele se via comparecendo diante do
Grande Trono Branco, e recebia, pobre diabo, a ordem de recitar algumas
frases das quais dependia seu destino; sua língua se paralisava, sua
memória ficava vazia, o inferno se escancarava aos seus pés; e ele
despertava, agarrando"-se ao varal da cortina do seu leito, com os
joelhos colados ao queixo.

Eram péssimas experiências, de um modo geral; e naquela fase da vida
esse sonhador a que me refiro abriria mão, de muito bom grado, da sua
capacidade de sonhar.  Mas aos poucos, à medida que foi crescendo, os
gritos e as contorções físicas se atenuaram, ao que parece para sempre;
suas visões continuavam a ser terríveis, em sua maioria, mas ele já as
suportava com mais firmeza; e ele passou a despertar pela manhã sem
nenhum sintoma mais extremo do que um coração palpitante, o couro
cabeludo e o corpo banhados em suor frio, e o terror mudo que lhe
sobrevinha à meia"-noite.  Seus sonhos, também, conforme acontece com as
memórias mais bem providas de detalhes, tornaram"-se mais precisos, e
tinham uma certa aparência de continuidade com sua vida.  Como o mundo
exterior começava a despertar de modo mais intenso a sua atenção, os
seus cenários passaram a aparecer com destaque nos seus pensamentos
noturnos, tal como ocorria durante as horas de vigília, de modo que ele
passou a fazer viagens longas e sem incidentes notáveis durante o
sonho, e a visitar cidades estranhas e belas paisagens.  E, o que é
mais significativo, a estranha predileção que ele tinha pela época
Georgiana e por histórias ambientadas naquele período da história
inglesa passou a dar o tom dos seus próprios sonhos, de modo que ele
costumava disfarçar"-se com um chapéu de três bicos e envolver"-se em
conspirações jacobitas entre a hora de deitar"-se e a do café da manhã. 
Por volta dessa mesma época, ele começou a ler durante o sonho ---
contos, em sua maior parte; e a maioria deles ao estilo de G.P.R.
James,\footnote{ George Payne Rainsford James (1799--1860) foi um historiador 
e ficcionista inglês.} 
mas contos muito mais vívidos e emocionantes do que os dos
livros impressos, a tal ponto que desde então ele passou a ficar
insatisfeito com a mera literatura.

E então, quando ele era ainda estudante, ocorreu"-lhe um sonho"-aventura
que ele não faz nenhuma questão de repetir; ele principiou, por assim
dizer, a ter sonhos em sequência, e com isto a levar uma vida dupla ---
uma de dia, outra de noite --- uma que ele tinha todos os motivos para
crer que fosse a verdadeira, e outra que não tinha meios de provar que
era falsa.  Faltou"-me dizer que ele então estudava, ou fingia estudar,
na Universidade de Edinburgh, e foi assim que (supõe"-se) vim a
conhecê"-lo.  Bem, na sua vida onírica ele costumava passar um longo dia
no anfiteatro cirúrgico, com o coração na boca, os dentes chacoalhando,
vendo malformações monstruosas e presenciando a abominável destreza dos
cirurgiões.  Ao chegar a noite, pesada, enevoada, chuvosa, ele se
encaminhava para a South Bridge, virava a esquina em High Street e
cruzava a porta de um alto edifício, em cujos andares superiores estava
presumivelmente alojado.  E a noite inteira, com a roupa molhada de
chuva, ele subia as escadas, andar por andar, numa série infinita, e de
dois em dois andares uma lâmpada com refletor iluminava o ambiente.  A
noite inteira ele cruzava com pessoas que desciam sozinhas --- mulheres
que mendigavam na rua; trabalhadores corpulentos, cansados, sujos de
lama; homens magros como espantalhos; criaturas que pareciam pálidos
arremedos de mulheres --- mas todas elas sonolentas e fatigadas como ele
próprio, e todas a sós, e todas roçando as roupas nas suas ao se
cruzarem.  No final, através da janela virada para o norte ele começava
a ver o céu clarear por sobre o Firth, e ele renunciava a continuar		
subindo.  Ao se virar para descer a escada, num piscar de olhos ele se
via de volta à rua, com a roupa ainda molhada, no meio de uma alvorada
chuvosa e sinistra, rumo a mais um dia de monstruosidades e operações. 
O tempo passava mais rápido nessa vida onírica, em que sete horas mais
ou menos (ou assim lhe parecia) equivaliam a uma; e, além disso,
transcorria de modo mais intenso, de modo que a sensação lúgubre
produzida por essas experiências imaginárias nublava todo o seu dia, e
ele ainda não tinha sido capaz de libertar"-se dessa sombra quando
chegava a hora de repetir tudo que se passara.  Não sei dizer durante
quanto tempo ele foi capaz de suportar essas provações, mas foi o
bastante para deixar uma mancha negra em sua memória, e para fazê"-lo ir
bater, trêmulo de receio pela própria sanidade mental, à porta de um
certo médico; de onde uma simples poção foi capaz de restituir"-lhe a
condição normal de qualquer homem.

Desde então, este pobre cavalheiro não tem sido perturbado por
ocorrências desse tipo; na verdade, suas noites passaram a ser por um
certo tempo iguais às de qualquer outro indivíduo, às vezes nulas, às
vezes consteladas de sonhos, os quais ora eram encantadores, ora
decepcionantes, mas, exceto quando ocasionalmente vívidos, nada tinham
de extraordinário.  Registrarei apenas uma dessas ocasiões, antes de
tratar do que faz do meu sonhador um tema interessante.

Ele tinha a impressão de estar no andar superior de uma casa de fazenda
situada numa colina.  O recinto demonstrava algumas tentativas de
decoração: um tapete no chão, um piano, acho, encostado à parede; mas
apesar desses pequenos refinamentos ele não tinha dúvida de que se
encontrava na região das charnecas, entre os montanheses, em meio a
grandes extensões de urzes.  Ele olhava pela janela lá para baixo, para
o terreno deserto, que parecia estar abandonado há muito tempo.

Um enorme e angustiante torpor parecia pesar sobre o mundo.  Não havia
sinal de qualquer dos habitantes da fazenda ou mesmo dos animais, salvo
por um velho cão \textit{retriever}, de pelo castanho e encaracolado,
deitado de encontro à parede, e que parecia cochilar.  Algo nesse cão
provocou uma certa inquietação no sonhador; um sentimento sem nome,
porque o animal parecia bastante comum, e sem dúvida, era tão velho e
cansado e sujo e alquebrado que era mais suscetível de inspirar
piedade; e no entanto cresceu no homem que sonhava a convicção de que
aquilo não era propriamente um cão, mas algum ser infernal.  Muitas
moscas típicas do verão esvoaçavam zumbindo pelo ambiente, e a certa
altura o cão estendeu a pata, colheu uma mosca na palma aberta, levou"-a
à boca num gesto igual ao de um macaco e, erguendo de súbito o olhar na
direção do homem que sonhava, piscou para ele.  O sonho
prosseguiu, não importa agora como; foi um sonho interessante como os
sonhos costumam ser, mas nada do que ocorreu depois pôde se comparar
àquele diabólico cachorro castanho.  E o ponto mais significativo jaz
parcialmente nesse mesmo fato: que, tendo se deparado com um incidente
tão extraordinário, o meu imperfeito sonhador se revelasse incapaz de
conduzir o sonho até um desfecho adequado, e permitisse que ele
acabasse degenerando em ruídos indescritíveis e horrores imprecisos. 
Hoje, seria bem diferente; ele agora conhece melhor o seu ofício!

Pois, para chegarmos ao cerne da questão: esse honesto indivíduo
cultivava de há muito o hábito de adormecer contando histórias para si
mesmo, como seu pai o fizera antes dele; mas essas histórias eram
invenções sem compromisso, contadas pelo simples prazer de contar, sem
nenhuma preocupação com um público pouco perceptivo ou com críticos
ultraexigentes; histórias em que uma linha de ação podia ser deixada
de lado, ou uma aventura substituída por outra, ao sabor da mera
vontade.  De modo que as pequenas criaturas que dirigem o teatrinho
íntimo da mente não tinham recebido ainda um treinamento muito
rigoroso; atuavam ali como crianças que tivessem entrado numa casa e a
encontrado deserta, não como atores experientes representando uma peça
de verdade perante um oceano de rostos.  Mas houve um momento em que
esse sonhador começou a usar esse divertimento de contar histórias
(como se diz) em benefício próprio, ou seja, começou a escrever e a
vender os seus próprios contos.  Agora, tanto ele quanto as
criaturinhas que executavam parte do seu trabalho viram"-se projetados
em diferentes circunstâncias.  Agora, as histórias precisavam ser
desbastadas e polidas até serem capazes de ficar de pé sozinhas;
precisavam ter começo e fim, além de poderem se encaixar (de algum
modo) às realidades da vida; o prazer, para resumir, transformara"-se em
negócio; e isto não apenas para o sonhador, mas para as criaturinhas
que habitavam seu teatro.  Elas compreenderam essa mudança tão bem
quanto ele próprio.  Quando ele se deitava e fazia seus preparativos
para adormecer, já não pensava em divertimento, mas em produzir
histórias publicáveis que lhe dessem algum lucro; e assim que começava
a cochilar no leito suas criaturinhas entravam em atividade imbuídas
dos mesmos propósitos mercantis.  Todas as outras formas de sonho o
abandonaram, com exceção de duas: ele ainda lê de vez em quando os
livros mais encantadores, e ainda visita às vezes os lugares mais
deslumbrantes; e é importante registrar que ele retorna a esses
lugares, e a um deles em particular, com intervalos de meses ou de
anos, descobrindo ali novas veredas nos campos, visitando novos
vizinhos, contemplando aquele vale hospitaleiro à luz do meio"-dia, ou
da aurora, ou do pôr do sol.  Mas todos os outros tipos de visão estão
perdidos para ele: a mistura comum e desordenada dos acontecimentos do
dia, os pesadelos repletos de cabeças sanguinolentas e ossos partidos,
que se diz serem produto de torradas com queijo; tudo isso, e suas
variantes, desapareceu.  E, quase sempre, seja desperto ou adormecido,
ele está ocupado --- ele e suas criaturinhas --- em criar
conscientemente histórias visando um mercado.  Esse sonhador (como
muitas outras pessoas) já se deparou com grandes vicissitudes da sorte.
 Quando o banco começa a enviar"-lhe cobranças e o açougueiro a
emboscá"-lo junto à entrada de serviço, ele começa a forçar a imaginação
em busca de uma história, porque essa é a sua forma mais imediata de
ganhar o pão; e no mesmo instante as criaturinhas se mobilizam para
ajudá"-lo, e trabalham a noite inteira, e toda a noite vêm mostrar"-lhe
fragmentos de cenas em seu teatro iluminado.  Ele não mais teme a
ameaça do terror; o coração aos pulos e o suor frio na pele são coisas
do passado; agora, as emoções que ele experimenta nesses dramas
noturnos são o aplauso, aplauso cada vez maior, um interesse crescente
e um entusiasmo crescente com sua própria habilidade (porque ele se
atribui todo o crédito), até que ele acorda com um salto, gritando, “É
isso!  Achei!”, interrompendo a peça, como Cláudio, em pleno
transcurso.  Muitas outras vezes, ele acorda desapontado: dormiu um
sono demasiado profundo (é assim que interpreto esses fatos), e a
sonolência se apoderou de suas criaturinhas, que interpretaram seus
papéis aos tropeções e balbucios; e a peça, examinada agora pela mente
desperta, não passa de um tecido de absurdos.  E no entanto quantas
foram as vezes em que esses incansáveis Brownies\footnote{ Os “Brownies”
são criaturas do folclore britânico, que
recebem uma variedade de nomes na Escócia, Irlanda, Gales etc. São pequenas
criaturas com pele marrom e roupas marrons, que eventualmente ajudam nas
tarefas caseiras, afeiçoam"-se aos membros da família, e recebem pequenos
lanches em troca dos seus serviços, embora esse “pagamento” não possa ser feito
diretamente (a comida deve apenas ser deixada ao seu alcance).} lhe prestaram seus
honestos serviços, e lhe entregaram, enquanto ele se divertia
confortavelmente em seu camarote, histórias melhores do que ele sozinho
teria sido capaz de compor. 

Eis aqui uma delas, exatamente como lhe foi mostrada.  Ao que parece ele
era então o filho de um homem rico e cruel, proprietário de vastas
terras e de um temperamento brutal.  O sonhador (ou seja, o filho)
vivera no exterior durante anos, com o propósito de manter"-se afastado
do pai; e quando acabou retornando à Inglaterra, foi apenas para
encontrá"-lo casado, pela segunda vez, com uma esposa jovem, que parecia
sofrer terrivelmente e odiar a condição a que estava sujeita.  Por
causa desse casamento (o sonhador compreendeu, sem muita clareza) era
necessário que pai e filho se encontrassem para conversar, mas, como
ambos eram orgulhosos e havia rancor de parte a parte, nenhum dos dois
tomava a iniciativa de visitar o outro.

Acabaram marcando um encontro, numa área desolada e arenosa à beira"-mar;
ali tiveram uma altercação, e o filho, ferido por algum insulto
intolerável, golpeou mortalmente o pai.  Ninguém suspeitou de nada; o
corpo foi encontrado e sepultado, e o sonhador viu"-se herdeiro de
grandes propriedades, e instalou"-se sob o mesmo teto em que vivia a
viúva de seu pai, à qual nada coubera como herança.  Os dois passaram a
viver ali sozinhos, como tantas vezes ocorre após uma tragédia:
sentavam juntos à mesa, compartilhavam as tardes intermináveis, e aos
poucos foram se tornando bons amigos; até que de súbito ele começou a
achar que ela estava invadindo uma zona perigosa, que de algum modo
suspeitava de sua culpa, que o vigiava e lhe fazia perguntas
indiscretas.  Ele começou a evitar sua companhia como um homem que
recua diante de um precipício subitamente avistado; e no entanto era
tão forte a atração que sentia que uma vez e mais outra voltava a ceder
à antiga intimidade, até recuar mais uma vez ao se deparar com uma
pergunta sugestiva ou uma expressão inexplicável no olhar da mulher.  E
assim viviam nesse impasse, uma vida cheia de frases interrompidas, de
olhares desafiadores e emoções reprimidas; até que, certo dia, ele viu
a mulher saindo sorrateiramente de casa, o rosto coberto por um véu, e
a seguiu até a estação, e depois a seguiu no mesmo trem até o litoral e
a região arenosa onde o crime tinha sido praticado.  Ali ela começou a
remexer nos arbustos, enquanto ele a observava, deitado no chão; e de
repente ela se ergueu segurando algo --- não lembro agora o que era, mas
tratava"-se de uma prova incriminadora contra o filho --- e no momento em
que deu alguns passos para o lado a fim de examinar melhor o objeto seu
pé escorregou, e ela agarrou"-se, numa posição perigosa, às plantas da
beira do barranco.  Não veio à mente dele outra coisa senão pular de
onde estava e correr ao seu socorro; e ali os dois se defrontaram, face
a face, ela com aquela prova mortal na mão, e a mera presença dele,
ali, servindo como confirmação da prova.  Era claro que ela ia dizer
alguma coisa, mas isso era mais do que ele podia suportar; suportaria
com bravura a própria desgraça, mas não era capaz de ouvir o que ela
pudesse dizer"-lhe; e ele acabou por interrompê"-la, falando de algum
assunto trivial.  De braços dados, os dois voltaram até o trem, falando
sobre uma coisa ou outra, fizeram juntos a viagem de volta para casa,
sentaram"-se para jantar, e passaram o resto da noite na sala, como
costumavam fazer no passado.  Mas o suspense e o medo faziam bater mais
forte o coração do sonhador.  “Ela não me denunciou ainda,” pensava
ele, “mas quando irá fazê"-lo?  Será amanhã?”  Mas não foi no dia
seguinte, nem no outro, nem no que veio depois; e a vida dos dois foi
se acomodando aos poucos ao formato antigo, com a única diferença de
que ela agora parecia mais gentil do que antes, o que tornou ainda mais
insuportável para ele o fardo do suspense e do espanto, a tal ponto que
ele começou a consumir"-se como um homem que tem uma doença grave.  Um
dia, ele ultrapassou todos os limites da decência, aproveitou uma
ocasião em que ela estava fora e revistou por completo o seu quarto,
até encontrar por fim, escondida entre as suas joias, a prova maldita. 
Ele parou ali, segurando nas mãos o objeto, que podia significar sua
vida ou sua morte, perplexo com o comportamento inexplicável da
mulher, que procurara por aquela prova, encontrara"-a, e agora a
conservava consigo sem usá"-la; nesse instante a porta se abriu e
lá estava ela.  E assim, pela segunda vez, os dois se defrontaram,
olhos nos olhos, com a prova do crime interposta entre eles; e pela
segunda vez ela o encarou com uma expressão indefinível; e mais uma vez
ele evitou falar no assunto e se retirou.  Mas antes de deixar o
quarto, que ele tinha revirado por completo, ele colocou de volta a
prova do crime no lugar onde a tinha encontrado, e a esse gesto o rosto
dela se iluminou.

A próxima coisa que ele ouviu foi quando ela se dirigiu à criada, com
uma engenhosa mentira a respeito da desarrumação do aposento.  Carne e
sangue não eram capazes de aguentar por mais tempo uma tamanha tensão,
e acho que foi na manhã seguinte (embora a cronologia do teatro da
mente seja sempre um tanto imprecisa)  que ele decidiu quebrar o
próprio silêncio.  Os dois tinham acabado de tomar juntos o café da
manhã, numa sala ampla, forrada de tacos, com muitas janelas e pouca
mobília; durante todo o transcorrer da refeição ela o havia torturado
com alusões veladas; e assim que os criados se retiraram e os dois
viram"-se sozinhos, ele se pôs de pé com um salto.  Ela também se
ergueu, com o rosto pálido;  pálida o escutou desabafar suas queixas.
Por que ela o torturava assim?  Ela sabia tudo, sabia que ele
não era seu inimigo; por que não o denunciava de uma vez por todas?  O
que significava aquele comportamento?  Por que insistia em torturá"-lo? 
E quando ele finalmente se deu por satisfeito, ela caiu de joelhos, e
com as mãos estendidas disse: “Será que você não entende?  Eu o amo!”.

E nesse instante, com um sobressalto de espanto e de deslumbramento
mercantil, o sonhador despertou.  Esse deslumbramento mercantil não
durou por muito tempo; porque logo ficou claro que nessa história tão
cheia de espírito havia elementos que não se adequariam ao mercado; e
esta é a razão pela qual ela vai aqui resumida de forma tão breve.  Mas
sua curiosidade não deixou de crescer, como certamente ocorrerá com a
do leitor, se ele considerar o assunto com mais vagar.  Pois agora
ele sabe por que me refiro àquelas criaturinhas como grandes
criadores e atores.  Elas souberam manter seu segredo até o fim.  Posso
assegurar em nome do homem que sonhou esta história (e tenho excelentes
motivos para crer na sua sinceridade) que em momento algum ele
suspeitou das motivações da mulher, desse sentimento em torno do qual
girou todo o enredo, até o instante em que ela fez essa declaração tão
dramática.  Porque a história não pertencia a ele, e sim às
criaturinhas!  E observem: não apenas este segredo foi mantido, mas
toda a história foi contada com uma habilidade fora do comum.  A
direção do casal de atores foi (para usar o jargão desses casos)
psicologicamente correta, e a emoção foi manipulada gradativamente até
o clímax surpreendente.  Agora estou acordado; e conheço bem este
ofício; e ainda assim confesso que não poderia fazer melhor.  Estou
acordado, vivo deste trabalho, e sei que não poderia superar (talvez
nem mesmo igualar) esses hábeis artifícios (dignos de velhos artesãos
teatrais como Dennery ou Sardou) através dos quais a mesma situação nos
é apresentada por duas vezes e os dois atores, por duas vezes, são
levados a encarar aquela prova, na primeira vez na mão dela, na segunda
vez na dele, e isto na ordem correta, ou seja, vindo primeiro a cena de
menor impacto dramático.  Quanto mais penso nisto, mais me sinto
impelido a propor ao mundo esta pergunta: quem são as criaturinhas? 
Que têm uma relação estreita com o sonhador, isto está fora de dúvida;
elas participam de suas preocupações financeiras e têm sempre um olho
em sua caderneta bancária; elas compartilham evidentemente de sua
educação; aprenderam, tanto quanto ele, a construir o esquema de uma
história bem arquitetada, e a dosar progressivamente a emoção; penso
somente que elas demonstram ter mais talento; e uma coisa está fora de
dúvida: elas podem contar"-lhe uma história, pedaço por pedaço, como um
seriado, e mantê"-lo o tempo inteiro em completa ignorância sobre o seu       
final.  Quem são elas, então?  E quem é esse sonhador?

Bem, com relação ao sonhador, posso dar uma resposta precisa, pois ele
não é outro senão eu mesmo, como eu poderia ter"-lhes revelado desde o
princípio, não fosse pelo hábito dos críticos de se queixarem do meu
consistente egoísmo; mas sou forçado a confessá"-lo agora, para poder
prosseguir com minha narrativa.  Quanto às criaturinhas, nada posso
dizer além de que eles são os meus Brownies, e Deus os abençoe! --- e
são eles que fazem metade do meu trabalho enquanto durmo, e,
provavelmente, também fazem todo o restante, quando estou desperto e
julgo que estou inventando coisas sozinho.  A parte que é feita
enquanto estou adormecido é produto dos Brownies, quanto a isto não há
dúvida; mas a que é produzida enquanto estou desperto e ativo não é
necessariamente minha, pois tudo indica que a mão dos Brownies também
não está ausente dela.  Reside aí uma dúvida que me assalta a
consciência.  Porque o meu Eu --- refiro"-me aqui ao meu ego consciente, o
habitante da glândula pineal (a menos que ele tenha mudado de
residência desde Descartes), o homem dotado de consciência e de uma
conta bancária, o homem que usa chapéu e botas, que tem o privilégio de
votar e de não eleger seu candidato nas eleições gerais --- esse Eu,
arrisco"-me a supor que ele não é de modo algum um contador de
histórias, mas um indivíduo tão pragmático quanto um vendedor de
queijos qualquer ou até mesmo um queijo; e um realista, mergulhado até
as orelhas no mundo à sua volta; de sorte que, no fim das contas, o
conjunto dos meus livros publicados deve ser o produto do trabalho
solitário dos meus Brownies, de algum demônio familiar, algum
colaborador invisível, que eu mantenho trancafiado em algum quarto dos
fundos, enquanto recebo todos os louvores e ele apenas um pequeno
pedaço (que não posso deixar de conceder"-lhe) do bolo.  Sou um
excelente conselheiro, um pouco como aquele criado de Molière; sei
suprimir, sei recortar, sei revestir o conjunto com as melhores
palavras e frases que sou capaz de encontrar e de produzir; sou também
o que segura a pena; e sou o que se assenta à mesa, que é de todas as
partes a pior; e quando tudo está pronto sou eu que preparo o
manuscrito e providencio o seu registro; de modo que, no cômputo geral,
tenho certos direitos a reivindicar nos lucros do empreendimento,
embora não tantos quanto os que de fato exerço.

Posso dar"-lhes um ou dois exemplos da parte deste trabalho que é
desenvolvida durante o sono e a que se dá durante o estado de vigília,
e deixar a critério do leitor a divisão, entre mim e meus
colaboradores, dos méritos que possam existir.  Para isto, recorrerei
em primeiro lugar a um livro que um certo número de pessoas me fez a
gentileza de ler, \textit{O estranho caso do Dr.~Jekyll e Mr.~Hyde}. 
Há muito tempo eu tentava escrever uma história sobre aquele tema,
tentava encontrar um corpo, um veículo, para aquela poderosa sensação
da duplicidade do ser humano, que às vezes se apossa do espírito de
toda criatura pensante.  Eu já escrevera uma, ``O companheiro de
viagem'', que me foi devolvido pelo editor sob o pretexto de que se
tratava de uma obra genial, porém indecente, e que queimei algum tempo
atrás por considerar que não era uma obra genial, e que \textit{Jekyll} o havia
suplantado.

Sobreveio"-me então uma daquelas oscilações financeiras às quais (com
elegante modéstia) já me referi na terceira pessoa.  Durante dois dias
maltratei meu cérebro tentando extrair dele um enredo de qualquer
natureza; e na segunda noite sonhei com a cena da janela, e depois com
outra cena, dividida em duas, em que Hyde, perseguido por um crime
qualquer, bebe a poção e sofre a transformação diante de testemunhas. 
Todo o restante escrevi desperto, e consciente, embora eu creia 
perceber nele o traço característico dos meus Brownies.  O
significado da história pertence portanto a mim, e já existia há muito
tempo em meu jardim de Adônis, e tinha tentado em vão encarnar"-se neste
ou naquele corpo; na verdade, sou o responsável pela maior parte da
moralidade do conto, e meus Brownies não têm sequer os rudimentos do que
consideramos uma consciência.  Também é meu o cenário, e são minhas as
personagens.  Tudo que me foi dado foi um conjunto de três cenas, e a
ideia central de uma mudança voluntária que passa a ser involuntária. 
Serei acusado de falta de generosidade se, depois de ter elogiado de
forma tão liberal os meus colaboradores invisíveis, eu os atirar agora,
de pés e mãos atados, na arena dos críticos?  Porque o detalhe
relativo aos pós e às poções, que muitos censuraram, não é de modo
algum meu, afirmo"-o com alívio, mas todo fornecido pelos Brownies.  E
há um outro conto sobre o qual, caso o leitor não tenha ainda lançado
os olhos sobre ele, posso dizer algumas palavras: aquela história não
muito defensável intitulada “Olalla”.

Ali, o pátio, a mãe, o nicho da mãe, Olalla, o quarto de Olalla, os
encontros na escada, a janela quebrada, a desagradável cena da mordida,
tudo isto me foi dado no conjunto e nos detalhes, enquanto escrevia; e
contribuí apenas com o cenário exterior (porque em meu sonho eu nunca
ia além do pátio), o retrato, as personagens de Felipe e do padre, a
moral, valha ela o que valer, e as derradeiras páginas, ai de mim --- com
o valor que tenham.  E posso dizer também que neste caso a própria
moral me foi fornecida; porque ela surgiu imediatamente da comparação
entre a mãe e a filha, e do terrível traço de atavismo existente na
primeira.  Às vezes, um sentido de parábola está inegavelmente presente
em um sonho; às vezes não posso deixar de supor que meus Brownies
andaram imitando Bunyan, e, no entanto, jamais com aquilo que
poderíamos chamar de uma moral livresca, nunca de uma maneira
eticamente estreita, mas, ao contrário, fornecendo pistas sobre as
limitações da vida, e sobre o sentido que imaginamos perceber ao
contemplar os arabescos do tempo e do espaço.

Na maior parte dos casos, como se verá, os meus Brownies são bastante
fantásticos, gostam de suas histórias cheias de fogo e de paixão,
pitorescas, movimentadas e cheias de incidentes vívidos; e não têm
preconceito contra o sobrenatural.  Dias atrás, contudo, fizeram"-me uma
surpresa, distraindo"-me com uma história de amor, uma pequena comédia
de abril que eu deveria certamente transmitir ao autor de \textit{A
Chance Acquaintance},\footnote{ Referência a William Dean 
Howells (1837--1920), escritor norte"-americano.} porque ele poderia escrevê"-la como ela deveria
ser escrita, e estou certo (embora bem possa tentar) de que não sou
capaz.  Mas quem poderia imaginar que um dos meus Brownies seria capaz
de inventar uma história para Mr.~Howells?

\clearpage
\ifodd\thepage ~ \clearpage\else\relax\fi
\thispagestyle{empty}
\mbox{}\vfill
{\noindent\itshape Frederick William Henry Myers (1843--1901) foi poeta, crítico           \label{cartastev}
literário e pesquisador dos fenômenos da mente, tendo sido presidente e
um dos fundadores da Society of Psychical Research.  Dele, disse
William James ter sido um homem formado principalmente nas áreas da
literatura e da história, e que veio a se interessar pela poesia e pela
religião; e que não foi um filósofo no sentido técnico e específico do
termo, mas que se tornou um crítico meticuloso das evidências, um hábil
formulador de hipóteses, um neurologista erudito e um leitor onívoro de
assuntos biológicos e cosmológicos.  Após o lançamento de}
O estranho caso do Dr.~Jekyll e Mr.~Hyde, \textit{ele e Stevenson passaram a travar
uma calorosa correspondência.  Quando Myers leu ``Um capítulo
sobre o sonho'', escreveu novamente a Stevenson uma carta, em 17
de abril de 1887, elogiando aquele texto.  A carta a seguir, de
Stevenson, foi conservada nos anais da} \textsc{s.p.r.}

\chapter[Esse outro Eu, meu companheiro\ldots{}\\ \textit{R.L.~Stevenson}]{“Esse outro Eu,\break meu companheiro\ldots{}”\subtitulo{R.L.~Stevenson}}
\hedramarkboth{esse outro eu, meu companheiro\ldots{}}{r.l.~stevenson}

\vspace*{.8em}

Fazenda de Vailima, Upoho, Ilhas Samoas

14 de julho de 1892 


\bigskip

Caro Dr.~Myers,

Gostaria de comunicar"-lhe algumas das minhas experiências, que me
parecem (por mais ignorante que eu seja) de um certo interesse
psicológico.

Tive sempre uma péssima saúde na minha infância, durante a qual sofri de
terrores noturnos espantosos.  Depois, estes desapareceram, e até os
trinta anos deixei de experimentar aqueles fortíssimos ataques de febre
em que sentia meu espírito me fugir.  Quando aquelas experiências
recomeçaram, elas se impuseram a mim com um absoluto frescor, o que
talvez explique, a menos que eu seja um indivíduo muito peculiar, a
exatidão com que fui capaz de registrá"-las. 

\textit{Experiência A.}  Durante uma longa enfermidade, em Nice, passei
uma noite inteira acordado, em meio a terríveis sofrimentos.  Desde o
início da noite \textit{uma parte do meu espírito} ficou literalmente
obcecada por uma noção tão grotesca e disforme que não consigo
descrevê"-la de outro modo senão como “uma forma de palavras”.  Eu
estava convencido de que minha dor estava relacionada a um toroide, ou
um rolo de cordas.\footnote{ Toroide é a forma geométrica
que se assemelha a uma rosquinha ou um pneu, um círculo
espesso fechado sobre si mesmo e aberto no meio.}  Em que consistia
ela? Do que se tratava, precisamente?  Eu não procurava saber: pensava
apenas que se as duas extremidades desse toroide se juntassem, minha
dor cessaria.  Durante todo esse tempo, com \textit{uma outra parte do
meu espírito}, algo que eu me arriscaria a definir como \textit{eu
mesmo}, eu estava plenamente consciente do absurdo desta ideia, sabia
que ela era indício de uma sanidade mental em perigo, e travava com
\textit{meu outro eu} uma luta furiosa.  Meu \textit{eu} não tinha
outra preocupação senão não permitir que minha mulher, que me velava,
tomasse conhecimento desse conflito, e não deixar escapar nenhum alusão
a essa alucinação ridícula.  \textit{O outro}, por sua vez, estava
convencido de que era preciso dizer"-lhe tudo, para que ela me ajudasse
a unir aquelas duas extremidades.  Foi somente antes do amanhecer,
creio eu, que a febre (ou o \textit{outro companheiro}) venceu, e eu
chamei minha esposa ao meu lado, agarrei"-a com força pelos pulsos e,
fitando"-a com um ar furioso, exclamei: “Por que você não junta essas
extremidades, para que minha dor possa passar?!”.

\textit{Experiência B.}  Um dia, em Sidney (era uma segunda"-feira,
creio), fui tomado subitamente por uma febre muito forte.  Ao longo da
tarde, pus"-me a repetir mecanicamente um som, que poderia ser
representado pelas letras “mhn”, e me surpreendi (detendo"-me de
imediato) tentando explicar a minha mãe, que estava no aposento
vizinho: “É o sinal de que estou começando a delirar, e que preciso
resistir desde logo”.  Depois adormeci e acordei várias vezes seguidas,
e passei o resto da noite a repetir uma palavra sem nenhum sentido, da
qual não consegui me lembrar na manhã seguinte.

No dia anterior, eu havia lido uma biografia de Swift, e durante o resto
da noite uma parte do meu espírito (\textit{“o outro companheiro”}) me
assegurou que não era eu quem repetia a palavra daquele modo, mas
Swift, durante sua derradeira enfermidade, sobre a qual eu lera naquele
livro.  A tentação de comentar com outras pessoas este absurdo foi
vista por \textit{mim mesmo} com aborrecimento, e desta vez foi esta
vontade que prevaleceu, e a enfermeira que estava de guarda ao meu
leito, naquela noite, nada entendia sobre Swift ou sobre essa
misteriosa palavra, e não entenderia nada que não fosse racional e
totalmente a propósito.  

Isto é o que eu posso atribuir claramente a uma ou à outra das minhas
duas consciências; resta ainda toda uma outra parte dos meus
pensamentos, cuja atribuição me parece bem mais delicada.  Uma parte do
meu espírito louvava sem cessar o gozo transracional daquela palavra,
destacando cada sílaba dela, mostrando"-me que nenhuma delas era
significativa por si mesma, e, portanto, que o seu conjunto exprimia
com perfeição  a imensa aflição de alguém que ardia em febre e tentava
em vão chamar a atenção da enfermeira.  Era provavelmente a mesma parte
 (que imagino ser \textit{“o outro amigo”}) que me fazia comparar essa
palavra às palavras “nonsense” de Lewis Carroll, como quem compara as
invenções de um cérebro doentio às de um homem de espírito são.  Mas
era certamente \textit{eu} (e eu num perfeito estado de lucidez)  que
tentava ao longo daquela noite decorar a tal palavra, repetindo para
mim mesmo que ela poderia me ser útil mais tarde, caso num dos meus
livros eu precisasse criar um personagem louco.  Digo que quem assim
pensava era eu, porque o outro amigo acreditava (ou fingia acreditar) 
que estava lendo um trecho de um livro onde sempre seria possível
reencontrá"-la, caso fosse necessário.

\textit{Experiência C.}  Na noite, seguinte, o \textit{“outro
companheiro”} tinha pronta uma explicação para todo o meu sofrimento,
da qual tudo que posso dizer é que tinha alguma coisa a ver com a
Marinha, o que era um absurdo total, algo sem pé nem cabeça, e que não
podia ser expresso através de palavras.  \textit{Eu mesmo} sabia disto,
mas não pude me conter, e minha enfermeira teve assim o prazer de me
ouvir discursar a respeito da Marinha.  Entretanto, meu outro
companheiro (ou eu mesmo?) ficou duplamente entediado: primeiro, porque
não tinha conseguido tornar sua mensagem compreensível, e depois porque
a enfermeira acabou não lhe dando a menor atenção.  \textit{O outro
companheiro} desejaria ter se explicado de uma maneira mais completa,
mas \textit{eu mesmo},\textit{ }muito chocado por ter sido colocado
numa posição tão equívoca, recusei"-me a ouvir a explicação.  

Nos casos A e C, a ilusão não tinha uma forma definida, mas, mesmo
consciente disso, eu sucumbi à tentação de tentar comunicá"-la a alguém.
 No caso B, a ideia era coerente e fui capaz de me controlar.  Em
outras palavras: minhas “duas consciências” foram menos afetadas no
caso B do que nos casos A e C.  Talvez não seja sempre assim: pode ser
que a autoridade racional do espírito se veja suspensa, mesmo quando a
ilusão é coerente: não é precisamente isto a alienação mental?

No caso A, eu estava perfeitamente consciente do fato de estar com a
mente desorientada, e de que minhas palavras não faziam nenhum sentido:
por isso eu estava tão ansioso para esconder meu estado, e por isso,
quando cedi à tentação de falar, meu rosto ficou convulso pela raiva, e
cerrei os dedos de modo tão cruel sobre o pulso de minha esposa.  Uma
atitude tão pouco natural e tão distanciada do meu caráter, resultando
de uma ideia que eu mesmo considerava insana e que durante tantas horas
tentei dissimular --- não é justamente o que acontece com os doentes
mentais?

Chamei a uma dessas pessoas “eu” ou “eu mesmo”, e à outra “o outro
companheiro”.  Era “eu mesmo” quem falava e se agitava; “o outro
companheiro” parecia não ter nenhum controle sobre meu corpo e minha
língua, e podia agir somente através de “mim mesmo”, sobre o qual
exercia, portanto, uma enorme pressão, que acabou derrotada em um caso,
e vitoriosa nos outros dois.  Sou tentado a crer que conheço “o outro
companheiro”, e que ele é o sonhador descrito por mim no meu “Capítulo
sobre o sonho”, ao qual o senhor se refere.

Aqui está, por fim, um sonho, oriundo desse mesmo período, mas desta vez
um sonho puro, uma ilusão, eu diria, que desapareceu com a volta do		\EP[-1]
sentido da visão, e não uma dessas ilusões que se prolongam durante o
tempo da vigília, quando eu era capaz de falar e de tomar meus
remédios.  Ele ocorreu um dia depois da experiência B, e antes da
experiência C.

\textit{Experiência D.}  Uma tempestade de vento se ergueu no começo da
tarde, levantando enormes nuvens de poeira. Meu quarto parecia estar
situado sobre uma colina escarpada, e sob o efeito do vento os galhos
das árvores se inclinavam todos na mesma direção --- de tal modo que o
mundo desfilava diante de mim como por uma calha de moinho.  Eu nadava
em meio àquela confusão, por entre o tumulto e todo aquele
movimento, mas sem experimentar nenhuma aflição, o que me deixava ainda
mais espantado, porque em circunstâncias normais um vento forte tem
sempre um efeito doloroso sobre os meus nervos.  Eu tinha acabado de
adormecer.  Pouco antes estava lendo \textit{A vida de Dryden}, de
Scott, e tinha me espantado ao descobrir que Dryden traduzira alguns
hinos latinos.  Como era possível que eu jamais os tivesse encontrado
em nenhuma de suas obras?  Quando adormeci, me veio com o sonho a razão
pela qual o som do vento e a visão das nuvens de poeira não tinham me
assustado. Na verdade não existiam nem o vento nem a poeira, mas apenas
Dryden que cantava a sua tradução dos hinos \textit{numa só direção} e
todos aqueles que o tinham ofendido ou atacado durante a Revolução os
cantavam \textit{na direção oposta}.  Este detalhe das duas direções		\EP[-1]
era sem dúvida de uma insanidade total.  Na verdade, Dryden estava o
tempo inteiro esvoaçando no ar mas sem poder ultrapassar a minha
janela, e seus detratores faziam o mesmo, na direção oposta, mas eles
também sem poder ultrapassar a janela.  Mas o mais curioso é que as
“duas direções” se aplicavam também às palavras, e mesmo à música --- de
uma maneira que me seria difícil explicar.

Era um sonho --- e no entanto ele reproduzia exatamente o método de “meu
outro companheiro” durante a vigília.  Não seria possível, a partir
deste ponto, encontrar uma explicação para esses estados do espírito e
do corpo, que tantas vezes se veem apanhados nas redes de uma loucura
feroz, completa e inexprimível? 

\bigskip

Seu sincero

Robert Louis Stevenson

\clearpage
\ifodd\thepage ~ \clearpage\else\relax\fi
\thispagestyle{empty}
\mbox{}\vfill
{\noindent\itshape Lloyd Osbourne (1868--1947) foi o filho
adotivo de R.L.~Stevenson.  Tinha doze anos quando sua mãe, Fanny
Osbourne (nascida Vandegrift) se casou com o escritor.  Os dois vieram
a colaborar em várias obras. O depoimento a seguir foi extraído de
“Stevenson at thirty"-five”, texto escrito para a edição de 1924 de
\emph{O estranho caso do Dr.~Jekyll e Mr.~Hyde}, a chamada “Edição Tusitala”. A tradução foi feita a
partir da versão francesa incluída em \emph{Essais sur l’art de la fiction},
editada por Michel LeBris (Payot, 1992).}

\chapter[Quando ocorreu o pesadelo de Mr.~Hyde\ldots{}\\ \textit{Lloyd Osbourne}]{Quando ocorreu o\break pesadelo de Mr.~Hyde\ldots{}\subtitulo{Lloyd Osbourne}}
\hedramarkboth{quando ocorreu o pesadelo de Mr.~hyde\ldots{}}{lloyd osbourne}

A saúde dele, é claro, sofria altos e baixos.  Havia períodos em que se
sentia melhor, e era capaz de passar alguns dias em Londres.  Uma vez,
chegou a visitar Paris.  Outra vez, deteve"-se em Dorchester para
visitar Thomas Hardy, e dali seguiu até Exeter, onde foi derrubado por
uma crise que durou três semanas e quase lhe foi fatal.  Mas na maior
parte do tempo ele vivia como um prisioneiro em sua própria casa, e não
avistava praticamente nada de Skerryvore a não ser seu pequeno jardim. 
Não se pode fingir que ele não era um inválido, um doente grave!  Tinha
hemorragias terríveis, e longos períodos durante os quais tinha que
ficar imóvel na cama, porque o menor movimento podia provocar		
derramamento de sangue.  Nesses momentos conseguia falar apenas
através de sussurros, enquanto que nós, sentados à sua volta,
procurávamos distraí"-lo --- e provavelmente aquele quarto teria se
tornado sua câmara funerária se ele não tivesse demonstrado sempre uma
força de vontade excepcional.

Como conseguia ele, debilitado a esse ponto, escrever tantos livros, é
um dos mistérios da literatura; livros tão fortes, tão cheios de vida
que não se consegue imaginá"-los brotando do quarto de um doente; livros
tão bem construídos, sem uma única quebra de intensidade, que ninguém
poderia pensar que sua redação era tantas vezes interrompida, enquanto
o seu autor jazia às portas da morte.  Os anos passados em Skerryvore
foram extremamente produtivos.  \textit{O estranho caso do Dr.~Jekyll e
Mr.~Hyde} foi escrito ali, assim como \textit{Raptado!}, “Markheim” e
muitos outros entre os seus melhores contos, sem esquecer \textit{A
vida de Fleeming Jenkin}. 

Certo dia, ele desceu para o café da manhã com um ar preocupado, comeu
sem dar muita atenção ao que ocorria em volta --- uma atitude quase
inacreditável, em se tratando dele --- e ao se levantar explicou que
estava escrevendo, com ótimos resultados, uma história que tinha lhe
ocorrido em sonho, e cuja criação ele não queria interromper ou
atrapalhar de modo algum, nem que a casa pegasse fogo.

Durante três dias uma redoma de silêncio envolveu Skerryvore.
Caminhávamos todos na ponta dos pés, e quando eu enfiava a cabeça pela
sua porta podia vê"-lo sentado na cama, acumulando página sobre página,
sem aparentemente fazer nenhuma pausa para descansar.  Ao cabo de três
dias, o misterioso trabalho foi concluído, e ele o leu para nós em voz alta,
para mim e para minha mãe, a primeira versão de \textit{Jekyll e Hyde}.

Escutei fascinado.  Stevenson tinha uma voz capaz de causar inveja em
muitos grandes atores, e lia com uma intensidade que me causava
arrepios na medula.  Quando chegou ao fim, ficou na expectativa pela
nossa aprovação, com um ar de exultação íntima, como que arrebatado por
um êxtase, e enquanto esperávamos as exclamações de entusiasmo de minha
mãe, fiquei aterrado ao perceber como ela estava hesitante.  Seus
elogios pareciam meramente formais, as palavras lhe saíam com
dificuldade, e de repente ela começou a fazer"-lhe críticas.  Disse que
ele havia falhado no essencial, pois tinha perdido a dimensão
alegórica; criara apenas uma história, um magnífica peça de literatura
sensacionalista, quando poderia ter escrito uma obra"-prima.

Stevenson ficou furioso.  Tremia dos pés à cabeça, segurando nas mãos o
manuscrito, e tudo em sua atitude mostrava o quanto estava mortificado.
 Sua voz, amarga, desafiante, encobria a da minha mãe como uma onda
corrosiva.  Eu nunca o vira assim transtornado, nem tão ofendido.  A
cena me era tão penosa que preferi deixar o aposento, incapaz de dar
um sentido àquilo tudo.  E foi com a sensação de uma tragédia irreparável
que fiquei ouvindo suas vozes, no aposento ao lado; palavras
indistintas, apenas, mas carregadas de uma tal emoção que me deixavam
de coração partido.

Quando voltei, minha mãe estava sozinha.  Estava sentada, pálida e
abatida, junto à chaminé, olhando fixamente as chamas.  Nenhum de nós
falou.  Se eu o tivesse feito, certamente a teria coberto de censuras,
porque a meu ver ela tinha sido cruelmente injusta.  Depois, ouvimos
Louis descendo a escada, e pelo modo como entrou no aposento chegamos a
crer, com um sobressalto no peito, que ele iria retomar a discussão. 
Mas tudo que ele disse foi: “Você tinha razão!  Eu de fato perdi essa
dimensão alegórica, que no final das contas é o ponto central, a
essência de toda a história!”.  E ao falar assim, como se quisesse
brincar com o abatimento de minha mãe e dos seus preparativos para
enfrentá"-lo de novo, ele jogou o manuscrito, tranquilamente, no meio
das chamas!  Imaginem meus sentimentos e os da minha mãe enquanto as
páginas se inflamavam diante dos nossos olhos, se encarquilhavam,
enegreciam, e se desmanchavam por entre as labaredas!

Minha primeira impressão foi de que ele tinha agido assim por um impulso
de despeito.  Mas não foi nada disso; ele tinha se convencido, de
fato, e aquele auto de fé estava lhe servindo como uma punição.  Quando
minha mãe e eu lhe dissemos que era loucura destruir daquela forma um
manuscrito, ele protestou com veemência, dizendo que “nada daquilo era
bom”, e que se tentasse preservar qualquer parte dele acabaria perdendo
a direção novamente, e a única maneira de escapar era evitar qualquer
tentação.

Seguiram"-se para ele três outros dias de febre criadora, e para nós três
dias de caminhadas na ponta dos pés, de refeições em que ele não dizia
uma palavra, de noites monótonas devido a sua ausência, de olhadas
sub"-reptícias através de sua porta, tudo isso enquanto ele, sentado na
cama, escrevia, escrevia, as folhas se amontoando em desordem sobre a
colcha. O resultado foi o \textit{Jekyll e Hyde} que o mundo
inteiro conhece, aquele que, traduzido em todas as línguas europeias, e
em várias do Oriente, deu ao mundo uma nova expressão.

Sua redação foi uma extraordinária façanha, sob qualquer ângulo que se
considere.  Sessenta e quatro mil palavras em seis dias, mais de dez
mil palavras por dia!  Para os que não conhecem bem os aspectos deste
ofício, eu diria que mil palavras por dia já é uma quantidade razoável,
para qualquer escritor de obras de ficção.  Anthony Trollope se
contentava com este ritmo, bem como Jack London; é --- como era naquela
época --- uma espécie de padrão para a criação literária cotidiana. 
Stevenson o multiplicou por dez, e além disso o recopiou por inteiro
durante mais dois dias, colocando"-o no correio no terceiro!

Foi uma bela façanha, mais espantosa ainda pelo fato de que depois
disso, ao invés de demonstrar qualquer manifestação de cansaço, ele
parecia ao contrário repousado, rejuvenescido; andava de lá para cá com
aspecto jubiloso, e exultante como se tivesse acabado de receber uma
herança; há meses que não exibia um aspecto tão saudável. 


\clearpage
\ifodd\thepage ~ \clearpage\else\relax\fi
\thispagestyle{empty}
\mbox{}\vfill
{\noindent\itshape Fanny Van De Grift (1840--1914) era norte"-americana, e casou aos
dezessete anos com o militar Samuel Osbourne, com quem teve três
filhos, e de quem se separou em 1875.  Conheceu Stevenson em
Paris, onde ficaram amigos, antes que ela voltasse para os Estados
Unidos.  Apaixonado por ela (que era dez anos mais velha), Stevenson
economizou dinheiro durante três anos para viajar aos \emph{\textsc{eua}} e
encontrá"-la.  Os dois se casaram e ficaram juntos até a morte do
escritor em 1894. O texto que se segue foi, como o de seu filho,
escrito para a “Edição Tusitala” de 1924, e traduzido da mesma
coletânea organizada por Michel LeBris.}

\chapter[Recordações de Mr.~Hyde\\ \textit{Fanny Van de Grift"-Stevenson}]{Recordações de Mr.~Hyde\subtitulo{Fanny Van de Grift"-Stevenson}}
\hedramarkboth{recordações de Mr.~hyde}{fanny van de grift"-stevenson}

Quando meu marido e eu deixamos Hyères e fomos para a Inglaterra, foi
com a intenção explícita de retornar no inverno seguinte.  Mas a saúde
de meu sogro piorou rapidamente, e nos vimos diante da evidência de que
uma nova ausência do seu filho seria um sério golpe para ele. 
Decidimos então, sem hesitação, pelo menos da minha parte, permanecer
em Bournemouth quanto tempo fosse necessário.  Para agradecer minha
decisão e sem dúvida com a esperança de que nossa presença ali se
tornasse permanente, meu sogro me presenteou uma casa pequena e
encantadora que batizamos de Skerryvore.  Não era muito espaçosa, mas
este aspecto era compensado pela presença de um gramado, de muitas
flores e de uma pequena horta; além disso, tinha ao lado um vale de
mata virgem por onde corria um pequeno rio.  Perto do estábulo, que
nunca usávamos, e próximo a casa, ficava um pombal coberto de hera.

Desde a nossa chegada a Bournemouth Mr.~W.E.~Henley foi nos fazer
companhia, com o projeto de escrever peças de teatro em colaboração com
meu marido; e ele voltou a nos visitar quando nos instalamos em
Skerryvore.  \textit{Deacon Brodie} tinha sido encenada em Londres sem
obter mais do que um \textit{succès d’estime}, mas Mr. Henley esperava
aproveitar esta experiência para produzir desta vez alguma coisa
suscetível de agradar o grande público.

No quarto que meu marido ocupou durante a infância, em Edinburgh, havia
uma estante de livros e uma cômoda fabricadas pelo famoso Deacon Brodie
--- um respeitável artesão durante o dia, e assaltante durante a noite. 
Cummy (Allison Cunningham), a quem meu marido havia dedicado seu livro
\textit{A Child’s Garden of Verses}, compôs, com sua viva imaginação de
escocesa, numerosas canções, sobre esses móveis tão prosaicos, para
distrair uma criança de quem ela cuidava.  Alguns anos mais tarde meu
marido ficou muito impressionado pela leitura de um artigo sobre o
subconsciente, aparecido numa revista científica francesa.  Este
artigo, combinado com suas lembranças de Deacon Brodie, foi a origem da
ideia que ele veio a desenvolver depois, primeiro numa peça, depois no
conto “Markheim”, e que por fim culminou, depois de uma intensa febre
que se sucedeu a uma hemorragia pulmonar, no pesadelo que resultou em
\textit{Jekyll e Hyde}.

Meu marido não tinha muito gosto pela composição dramática, se bem que
\textit{Prince Otto} foi concebido inicialmente como uma peça de
teatro; mas Mr.~Henley possuía uma capacidade extraordinária de
insuflar nos outros seu próprio entusiasmo.  Eu mesma me vi
involuntariamente arrastada nesse turbilhão.  Lembro"-me que chegaram a
me prometer um bracelete de rubis, com os lucros que esperavam para a
primeira apresentação da peça, por uma sugestão que fiz para
\textit{Admiral Guinea}.  As peças eram imaginadas e escritas desde o
início no estilo apaixonado e cheio de exageros de Mr.~Henley, cuja
influência predominava, menos no que dizia respeito à forma literária
propriamente dita.  Um primeiro esboço do roteiro, ainda muito pequeno
e superficial, era traçado, para ser desenvolvido a seguir, aprofundado
em uma série de parágrafos escritos por cada um deles.  Tinha sido
estabelecido entre os dois que tudo que um deles criticasse nos textos
do outro seria suprimido sem apelações, um processo que não posso
deixar de considerar prejudicial ao trabalho em conjunto.  Em matéria
de teatro, meu marido procurava produzir um \textit{tour de force}
literário, apoiando"-se nas convenções clássicas.  Quanto a Mr.~Henley,
ele sonhava, acima de tudo, em causar espanto ao público.  É possível que
cada um deles, a sós, tivesse alcançado sucesso, mas em conjunto
tudo parecia uma tarefa impossível. “Isso vai fazê"-los pular até o		      
teto!”, gritava Mr.~Henley, esmurrando a mesa com tal força que fazia
pular o tinteiro. “Não, não, Henley”, dizia meu marido com ar
preguiçoso, “você é muito brutal. É preciso atenuar isso mais um
pouco”.  Mas em virtude do seu acordo prévio, a cena era rejeitada, e
na nova versão que faziam alguma coisa dela se perdia.

Durante a estadia de Mr.~Henley em Skerryvore, muitos amigos adquiriram
o hábito de vir à noite passar algumas horas em nossa companhia. 
Entre eles, vieram Mr.~Henry James, Mr.~John Sargent, Mrs.~De Mattos,
Mr.~Sully, Mr.~Walter Lemon, Miss Taylor e Miss Ferrier, e Robert Allan
Stevenson.  Estas noites que passamos em discussões sábias,
apaixonadas, brilhantes, estão entre as experiências mais agradáveis da
vida de meu marido.  Quando elas foram vetadas pelo médico, sob o
pretexto de serem extenuantes para os seus nervos, ele voltou, com uma
passividade que chegava a ser patética, ao “reino da colcha” (seu leito
de dormir), onde matava o tempo tocando flajolé; e quando até isso lhe
foi proibido passou a se dedicar aos problemas de xadrez que eram
publicados nos jornais.  Talvez o xadrez não fosse uma panaceia para
curar seu espírito e seus nervos desgastados pelas jornadas de trabalho
na companhia estafante de Mr.~Henley.  Felizmente meu marido tinha o
dom de poder dormir conforme sua vontade.  Era capaz de dizer:
“Acordem"-me daqui a meia hora”, pousar a cabeça no travesseiro e
mergulhar de imediato num sono reparador.  Ora, pela primeira vez em
sua existência, seu sono passou a ser agitado, intermitente.  Os
Brownies despertavam nele durante toda a noite, atormentavam"-no com
problemas de xadrez deixados sem resolver, com cenas de acontecimentos
pessoais há muito esquecidos e que retornavam para tirar"-lhe o sossego.
 Foi justamente durante uma pausa forçada na sua colaboração com Mr.
Henley que lhe veio a inspiração para \textit{O estranho caso do 
Dr.~Jekyll e Mr.~Hyde}.  Os gritos de horror que ele soltou durante o sono
me obrigaram a despertá"-lo --- para sua enorme indignação.  “Eu estava
justamente sonhando com uma ótima história de terror”, disse ele, em
tom de censura; e me fez um breve resumo da narrativa de Jekyll e Hyde,
até a cena da transformação, que era a que estava presenciando em sonho
no momento em que o despertei.

Quando o dia amanheceu, ele se atirou febrilmente ao trabalho.  Em três
dias o primeiro esboço, com trinta mil palavras, foi terminado, depois
foi destruído e refeito sob outro ponto de vista --- o da alegoria, que
estava latente na narrativa mas de forma incompleta, provavelmente
devido à pressa excessiva e ao fato do pesadelo ser ainda demasiado
recente.  Ao cabo de mais três dias o livro estava pronto para ser
impresso, precisando apenas de algumas correções menores.  A quantidade
de trabalho que ele executou nesse \textit{tour de force} foi enorme. 
Que um homem inválido, como o meu marido, fosse capaz de executar um
tal esforço, pondo no papel sessenta mil palavras em seis dias, parece
quase inimaginável.  Ele sofria de hemorragias contínuas e falava com
dificuldade; sua conversa se dava muitas vezes pelo emprego de uma
ardósia e um lápis.  Não podia receber duas pessoas ao mesmo tempo em
seu quarto, e qualquer um que obtinha a autorização do médico para
vê"-lo podia demorar"-se no máximo quinze minutos --- e cabia a mim a
tarefa ingrata de ficar de guarda à porta, no patamar, de relógio em
punho, para avisar o visitante de que seu tempo tinha se esgotado.

O sucesso de \textit{Jekyll e Hyde} foi imediato e fenomenal, tanto na
Inglaterra quanto nos Estados Unidos, onde surgiram várias edições
piratas.  A história era mencionada por padres durante os seus sermões,
e foi adaptada para o palco pelo menos três vezes; a única versão
satisfatória foi a de Mr.~T.R.~Sullivan, que enviou o texto de sua
adaptação ao meu marido, para que ele fizesse as correções necessárias
e lhe desse sugestões.  É curioso perceber até que ponto o público pode
identificar um autor com personagens dos seus livros.  A aparência
de meu marido foi descrita assim como uma mistura grotesca do Dr.~Jekyll 
e Mr.~Hyde.  Um crítico chegou a escrever: “Ele se assemelha a
um afogado que tenha sido retirado da água no derradeiro instante, com
seus longos cabelos ainda úmidos e colados ao rosto”.  Mesmo os
pintores que o retrataram tentaram sugerir algo de bizarro e de
espectral na sua fisionomia.  Nenhum deles, por outro lado, teve a
ideia de imaginá"-lo como o Príncipe Otto, com o qual ele se parece
muito mais.

Ele recebeu muitas cartas estranhas, principalmente da parte de
espíritas e teósofos, os quais imaginavam ter recebido certas
“diretivas” do além para guiá"-lo nas suas descrições da vida dupla. 
Uma condessa alemã lhe perguntou se a narrativa era de fato resultante
de um sonho, assegurando"-lhe que se de fato fosse este o caso ele se
achava numa situação das mais perigosas, porque as forças “da magia
branca e da magia negra” estavam disputando sua alma.  A condessa lhe
implorava que aceitasse os ensinamentos da Teosofia, porque caso contrário as
forças da magia negra venceriam a disputa e as consequências, garantia
ela, “seriam terríveis”.  


\clearpage
\ifodd\thepage ~ \clearpage\else\relax\fi
\thispagestyle{empty}
\mbox{}\vfill
{\noindent\itshape Frederic Myers (1843--1901) foi um professor de Cambridge, poeta,
e psicólogo amador.  Diferentemente de Maudsley, que tinha uma visão
mais materialista e fisiológica da mente humana, ele tinha uma
abordagem mais experimental, e fez várias tentativas de provar a
existência da telepatia, em seus trabalhos pela Society of Psychical
Research. (Atribui"-se a ele, aliás, a criação do termo “telepatia”.) A
carta que Stevenson lhe escreveu, reproduzida à página \pageref{cartastev} deste		
livro, mostra o respeito mútuo, e as curiosidades que
compartilhavam.}

\chapter[A personalidade multiplex\\ \textit{Frederic Myers}]{A personalidade multiplex\subtitulo{Frederic Myers}}
\hedramarkboth{a personalidade multiplex}{frederic myers}

Meu propósito neste artigo é propor alguns tópicos para reflexão,
tópicos que precisarão ser desenvolvidos com maiores detalhes em outros
estudos. Meu tema é o caráter mutável e “multiplex” daquilo a que
chamamos a Personalidade do ser humano, e as vantagens práticas que
podemos auferir ao perceber essa mutabilidade até então ignorada e
trabalharmos levando"-a em consideração.  Começarei citando alguns
exemplos de transferência histérica, e de desintegração mórbida;
mostrarei em seguida que nem todos esses reajustamentos espontâneos do
nosso ser são patológicos ou regressivos; na verdade, mudanças que nos
são tão familiares quanto os estados do sono e da vigília nos dão a
pista para outras alterações que podem ser empregadas em nosso
proveito.

Começarei, então, com um ou dois exemplos do nível que pode ser
alcançado por essas dissociações de nossas memórias, faculdades, e
sensibilidades, sem que isso resulte num caos insano ou num
obscurecimento demente da memória. Por enquanto, esses casos existem em
pequeno número. Foi apenas nos anos mais recentes --- e
principalmente na França --- que os sábios registraram com o devido
cuidado essas lições psíquicas, mais profundas do que nossa própria
arte pode nos proporcionar, que nos foram dadas pelo estudo de
anomalias naturais e de casos aberrantes. 

Entre os mais extraordinários documentos vivos que a natureza oferece ao
nosso estudo estão os singulares personagens conhecidos como Louis V. e
Félida X.  O nome de Félida, pelo menos, deve ser familiar a um certo
número dos meus leitores; mas o caso de Louis V. é menos conhecido, e
embora alguns registros a seu respeito já tenham sido publicados em
inglês, é necessário recordar algumas de suas particularidades para
depois apresentar nossas especulações a respeito.

A vida de Louis V. começou (em 1863) como uma criança abandonada por uma
mãe turbulenta.  Aos dez anos de idade foi mandado para um
reformatório, e ali revelou"-se, e foi assim sempre que aquela
organização lhe deu a oportunidade, um menino quieto, bem comportado e
obediente.  Então, aos catorze anos, sofreu um grande susto ao ser
atacado por uma víbora --- um susto que o deixou desequilibrado e iniciou
uma série de oscilações psíquicas das quais ele continua sendo vítima
desde então.  A princípio os sintomas eram apenas físicos: epilepsia e
paralisia histérica das pernas; e no asilo de Bonneval, para onde ele
foi enviado em seguida, trabalhou normalmente numa oficina de alfaiate
durante cerca de dois meses.  Depois, teve um súbito ataque
histérico"-epiléptico --- cinquenta horas de convulsões e êxtase --- e
quando despertou não estava mais paralítico, não demonstrava o menor
conhecimento de alfaiataria, e não exibia bom comportamento.  Sua
memória tinha regredido, por assim dizer, até o momento do ataque da
víbora, e ele não recordava coisa alguma do que lhe acontecera desde
então.  Seu caráter se tornou violento, cobiçoso e encrenqueiro; e
seus gostos haviam mudado de forma radical.  Por exemplo, embora antes
do ataque ele fosse completamente abstêmio, agora ele não apenas bebia
vinho como roubava o vinho de outros pacientes.  Fugiu de Bonneval, e
os anos que se seguiram foram turbulentos; sua vida pode ser rastreada
através de suas entradas em hospitais e manicômios, até que retornou
ao asilo de Rochefort, sob a identidade de um fuzileiro naval preso por
furto mas demonstrando instabilidade mental. Em Rochefort e em La
Rochelle, por um grande golpe de sorte, foi parar nas mãos de três
médicos, os professores Bourm e Burot, e o Dr.~Mabille, que estavam
capacitados e dispostos a retomar e expandir as observações já feitas
pelo Dr.~Camuset em Bonneval e pelo Dr.~Jules Voisin em Bicêtre, quando
ambos tiveram a oportunidade de examinar esse precioso espécimen de
“estranho caso”, em momentos anteriores de sua contraditória
existência.

Louis V. não se encontra mais em Rochefort, e o Dr.~Burot me informa que
ele melhorou bastante de saúde e que suas peculiaridades em grande
medida já desapareceram; devo, contudo, a bem da clareza, usar o tempo
presente ao fazer esta breve descrição de sua condição na época em que
essas experiências foram realizadas.

O estado físico e mental em que ele se encontra é muito preocupante. Há
paralisia e insensibilidade do seu lado direito e também (como ocorre
muitas vezes em caso de hemiplegia direita) sua fala é dificultosa e
pouco compreensível.  Mesmo assim, ele discute sem parar com qualquer
pessoa que lhe dê ouvidos, ofendendo os médicos, ou então fazendo
pregações de radicalismo político ou de ateísmo antirreligioso, e se
comporta mais como um macaco que procura escandalizar do que como
alguém que emprega argumentos claros e razoáveis. Faz piadas de mau
gosto, e se alguma pessoa lhe agrada ele imediatamente tenta fazer"-lhe
carícias.  É capaz de lembrar acontecimentos recentes, ocorridos
durante sua residência em Rochefort, mas antes dessa data consegue
recordar apenas dois fragmentos do seu passado: o vicioso período que
passou em Bonneval e parte do tempo que passou em Bicêtre.

Com exceção dessa memória estranhamente fragmentada não há nada de
extraordinário em sua condição, e na maior parte dos asilos ninguém
teria se dado o trabalho de explorá"-la mais a fundo.  Por sorte, os
médicos de Rochefort eram familiarizados com a eficácia do contato de
metais para transferir a hemiplegia histérica de um lado do corpo para
o outro. Foram experimentados diferentes metais em contato com o corpo
de Louis V.: chumbo, prata e zinco não produziram efeito.  O cobre
produziu um leve retorno da sensibilidade no braço paralisado.  Mas o
aço, aplicado ao braço direito, transferiu toda a insensibilidade para
o lado esquerdo do corpo.

Por mais inexplicável que pareça este fenômeno, ele é suficientemente
comum (de acordo com os relatos dos médicos franceses), em casos de
histeria, e não desperta surpresa alguma.  O que deixou perplexos os
médicos foi a mudança de caráter que acompanhou essa inversão da
sensibilidade.  Quando Louis V. emergiu da crise provocada pela
transferência, cerca de um minuto de expressão ansiosa e respiração
arquejante, era o que se pode chamar de um novo homem.  A agitação
insolente, a impulsividade selvagem, tinham desaparecido por completo.
O paciente é agora gentil, respeitoso e modesto.  Pode falar com
clareza, mas o faz apenas quando alguém lhe dirige a palavra.  Se lhe
pedem sua opinião sobre religião ou política, ele diz que prefere
deixar esses assuntos para mentes mais capazes do que a sua. A
impressão que se tem é de que tanto moralmente quanto intelectualmente
a cura do paciente foi completa. 

Mas agora eu lhe pergunto o que ele acha do asilo de Rochefort, e o que
nos diz do tempo em que serviu ao regimento da Marinha. Ele responde,
admirado, que não conhece Rochefort, e que nunca foi fuzileiro naval
durante toda sua vida.  “Onde você está, então, e que dia é hoje?”
“Estou em Bicêtre, hoje é 2 de janeiro de 1884, e espero encontrar"-me
hoje com Monsieur Voisin, como me encontrei ontem”.

Descobriu"-se, então, que ele agora é capaz de recordar dois curtos
períodos de sua vida (períodos diferentes daqueles que ele recordava
quando era seu lado direito que se encontrava paralisado), e durante os
quais, na medida em que nos foi possível confirmar, seu caráter era da
mesma natureza respeitosa, e seu lado paralítico era o esquerdo.

Essas duas condições foram então denominadas de “primeira” e “segunda”,
em uma série de seis ou mais pelas quais ele pode ser induzido a
passar.  Para não alongar em demasia este relato, descreverei em
seguida apenas o seu “quinto” estado.

Se Louis é colocado num banho elétrico, ou se um ímã é colocado por cima
de sua cabeça, à primeira vista somos levados a pensar que ocorreu uma
cura completa.  Toda a sua paralisia e todas as suas perdas de
sensibilidade desaparecem.  Seus movimentos são ágeis e ativos; sua
expressão, gentil e tímida.  Mas se lhe perguntam o local onde se
encontra descobre"-se que ele regrediu até ser um menino de catorze
anos, que está em Saint Urbain, seu primeiro reformatório, e que sua
memória abarca os anos da infância e se detém no dia exato em que foi
assustado pela víbora. Se é pressionado a lembrar este incidente,
ocorre uma violenta crise epileptiforme, que encerra bruscamente esta
fase de sua personalidade.

O leitor pode perguntar, então: existirá alguma lei discernível que
controle essas estranhas transformações?  Existe algum motivo concreto
para que Louis V. seja num momento um simples lunático ou selvagem, em
outro se transforme num adulto bem comportado, e num terceiro recupere
suas plenas condições físicas, mas mentalmente volte a ser uma criança?
 Reproduzo a seguir, com brevidade, e omitindo numerosas ressalvas e
pormenores técnicos, a opinião dos médicos que o têm examinado.

Um choque súbito, numa pessoa de constituição instável, provocou nesse
rapaz uma cisão profunda entre os hemisférios direito e esquerdo do seu
cérebro, cisão de uma magnitude que não havia sido registrada antes. 
Estamos acostumados, é claro, a ver uma pessoa com o lado direito do
corpo paralisado e insensível em decorrência de uma lesão no hemisfério
esquerdo, que o governa; e vice"-versa.  E estamos acostumados a ver em
casos de histeria --- caso em que não se verifica nenhuma lesão física
perceptível em qualquer dos hemisférios --- as perturbações de
sensibilidade e de mobilidade mudarem com grande rapidez, mudarem, por
assim dizer, a um simples toque, de um lado do corpo para o outro.  Mas
usualmente não somos capazes de perceber nenhuma mudança correspondente
no modo de funcionamento daquilo a que consideramos os “centros
superiores”, os centros que determinam as nossas manifestações de
inteligência, caráter, memória, e dos quais depende em grande medida o
nosso senso de identidade.  No entanto, em alguns casos de afasia e de
outras formas de assemia (nossa perda de poder sobre os signos, a
palavra falada ou escrita e outras manifestações análogas) têm ocorrido
fenômenos que em certa medida nos prepararam para perceber que a perda
do uso do hemisfério esquerdo --- que sem dúvida é o mais desenvolvido,
em alguns aspectos --- pode conduzir a uma regressão nas características
superiores da vida humana.  E o singular fenômeno da “escrita
automática” parece muitas vezes depender de uma ação obscura sobre o
hemisfério cerebral menos utilizado.  Os pesquisadores que seguem estas
linhas de observação podem estar prontos para achar possível que no
caso de Louis V. a predominância alternada do hemisfério direito e do
esquerdo afeta a memória e o caráter na mesma medida em que o faz às
suas inervações motoras e sensoriais.  

Ao ser inibido o seu cérebro esquerdo (e o lado  direito do corpo) ele
se torna, por assim dizer, não apenas canhoto, mas também “sinistro”;
manifesta"-se através de combinações nervosas que alcançaram um grau
inferior de evolução.  E ele pode evocar em sua memória apenas aqueles
períodos em que sua personalidade assumia aquela mesma atitude,
cristalizada a certa altura de sua vida. 

Ao ser inibido o seu cérebro direito, suas qualidades superiores de
caráter são mantidas intactas, tanto quanto o dom da fala. Existe
autocontrole, existe o pudor; existe um senso de obrigação ---
qualidades que o ser humano desenvolveu ao se erguer acima do nível dos
selvagens.  No entanto, ele é apenas metade de si próprio.  Além da
ocorrência da hemiplegia, que é uma consequência direta, sua memória
vê"-se também truncada, e ele é capaz de evocar apenas aqueles
fragmentos do seu passado que se acham ligados a esse estado anormal, e
impossibilitado de lembrar não apenas os períodos dessa ascendência
“sinistra” em sua mente, mas também o período normal de sua infância,
antes que a sua natureza fosse dividida em duas.  E se agora, por algum
triunfo da técnica, fôssemos capazes de restaurar o perfeito equilíbrio
entre os seus dois hemisférios, se pudéssemos colocá"-lo num estado em
que não restasse nenhum traço físico dessa cisão que se tornou para ele
uma segunda natureza, o que poderíamos esperar encontrar como
contrapartida dessa integridade recomposta?  O que de fato encontramos
no paciente é uma mudança que, pelas possibilidades físicas que nos faz
vislumbrar, é a mais interessante de todas.  Ele renasce, por assim
dizer; torna"-se de novo uma criança, conduzido de volta, em sua
memória, caráter, conhecimentos, poderes, aos dias anteriores ao
problema que o vitimou, e ao predomínio do seu “eu” menos evoluído.

Comecei com a descrição de um caso extremo, um caso que para muitos dos
meus leitores talvez venha a parecer inacreditável, de tão bizarro. 
Mas, embora extremo, ele não é de modo algum um caso isolado; ele se
relaciona, em diferentes aspectos, com numerosos outros casos bem
conhecidos.  A retomada da vida num momento situado no passado, por
exemplo, é apenas a forma exagerada de um fenômeno muitas vezes
observado em casos de lesão cerebral.  O treinador de cavalos, atingido
por um coice violento, completa sua ordem para afrouxar as rédeas no
momento em que sua trepanação é concluída com sucesso no hospital.  A
senhora idosa que sofre um ataque durante um jogo de baralho, e cuja
consciência é restaurada após um longo período de insensibilidade,
surpreende sua família em prantos com a pergunta: “Qual é o trunfo?”. 
Nestes casos, porém, é apenas uma fatia da vida que é subtraída; a
personalidade retorna como se despertasse de um sono, e é a mesma de
antes. No caso de Louis V. não é isso o que ocorre; as lembranças dos
sucessivos estados não ficam perdidas, mas justapostas, de certo modo,
em diferentes compartimentos; ninguém é capaz de dizer quais são as
épocas que brotam como interpolações, nem qual é o canal central onde
está fluindo a correnteza do seu ser.

Essas divisões do Eu tão graves quanto a de Louis V. podem ser
encontradas com maior facilidade nos asilos de lunáticos. É ali que
encontramos a duplicação\ \ da individualidade em suas formas mais
grotescas.  Encontramos o homem que vive perdendo a si mesmo e
procurando por si mesmo embaixo da cama.  Há o homem que acha que
existem dois dele, e que na hora da refeição apresenta seu prato pela
segunda vez, dizendo: “Já comi bastante, mas o outro sujeito não comeu
ainda”.  Há também o homem que diz ser ao mesmo tempo ele mesmo e seu
irmão, e quando lhe perguntam como é possível ser duas pessoas ao mesmo
tempo, ele diz: “Ora, através de mães diferentes”.

Há casos também em que a personalidade oscila de um foco para outro, e
esses impulsos contraditórios, que em nós provocam apenas uma mudança
no estado de espírito, costumam se objetivar adquirindo uma “persona”
própria.  Uma penitente histérica acredita numa semana ser a “Irmã
Marta das Cinco Chagas” e na semana seguinte torna"-se uma imaginária
“Madame Pulmaire” cujos hábitos não seriam nem um pouco adequados para
um convento.  Outro paciente dá a impressão de ser um indivíduo
razoavelmente sadio, mas de tempos em tempos deixa a barba crescer e se
transforma num arrogante tenente da artilharia. Quando o acesso passa,
ele raspa a barba e se torna novamente um homem lúcido mas melancólico,
que se dedica a estudar os antigos doutores da Igreja. É claro que
essas mudanças de caráter podem ser rápidas e variadas, na medida em
que a vida do paciente o permita.  Num caso bastante conhecido uma
mulher muito pobre costumava mudar de história, de personalidade, e
mesmo de sexo, a cada dia.  Num dia apresentava"-se como a noiva do
imperador, no outro como um estadista prisioneiro:

\begin{quote}
Juvenis quondam, nunc femina, Caeneus,\\
Rursus et in veterem fato revoluta figuram.\footnote{
Versos de Virgílio na \textit{Eneida}, \textsc{vi}, 449--450: “E aqui também
está Caeneus, outrora um homem jovem, depois uma mulher / e agora
devolvido pelo destino ao seu sexo de origem”.}
\end{quote}

E ainda mais instrutivos, embora mais tristes, são os casos em que a
desintegração da personalidade não atingiu um ponto de insanidade, mas
desembocou numa impotência perplexa, no horror de uma vida que parece
um sonho.  De um modo geral, casos assim se dividem em duas categorias
principais: aqueles onde a perda de controle se dá principalmente sobre
os centros motores, de modo que o paciente é capaz de sentir, mas não
de agir; e os casos em que a perda de controle atinge principalmente os
centros sensoriais, e o paciente age mas não é capaz de sentir.

A impossibilidade de agir como gostaríamos de agir é um problema que
muitos de nós compartilhamos.  Provavelmente todos temos momentos em
que somos capazes de simpatizar com aquele provocante paciente de
Esquirol\footnote{ Jean Etienne Dominique Esquirol
(1772--1840), médico e reformador dos asilos da França.}
que, depois de um ataque de monomania, recuperou todos os dons para a
convivência social que faziam dele a alegria dos seus amigos, mas não
podia mais ser induzido a dedicar cinco minutos de atenção mesmo aos
assuntos mais urgentes.  “Seus conselhos”, disse ele cordialmente a
Esquirol, “são imensamente úteis. Eu não quereria nenhuma outra coisa
senão ser capaz de segui"-los, desde que o senhor pudesse me conceder a
vontade de fazer o que quero”.  Às vezes a vida inteira é desperdiçada
na execução dos atos mais irrelevantes --- como ocorria com um paciente
de Monsieur Billod, que passava quase uma hora tentando executar o
floreio da pena sob sua assinatura, num documento para o advogado; ou
tentou em vão durante três horas, com o chapéu posto e as luvas
calçadas, deixar os seus aposentos para ir assistir um espetáculo que
tinha imenso desejo de ver.  Casos assim precisam de um tratamento
verdadeiramente heroico, e este cavalheiro teve a sorte de acabar sendo
envolvido e curado pela Revolução de 1848.  

Ainda mais lamentáveis são aqueles casos em que são principalmente os
centros sensoriais que se situam, por assim dizer, exteriormente à
personalidade; onde o pensamento e a vontade permanecem intactos, mas o
mundo ao redor não mais excita os sentimentos nem atinge a alma
solitária.  “Em todos os meus atos falta uma coisa --- o senso de esforço
que deveria acompanhá"-los, a sensação de prazer que eles deveriam
me proporcionar.” “Todas as coisas”, disse outro paciente, “parecem
incomensuravelmente distantes de mim; estão recobertas por um ar
pesado.”  “As pessoas parecem se mover à minha volta”, disse outro,
“como sombras ambulantes”.  Gradualmente, esta sensação de esvaziamento
fantasmagórico se estende até a própria pessoa do paciente. “Cada um
dos meus sentidos, cada parte de mim, está separada de mim mesmo.” “Eu
existo, mas fora da vida real.” É como se Tirésias, o único capaz de
manter"-se verdadeiramente vivo no mundo insubstancial do Hades, fosse
aos poucos convertendo"-se ele próprio numa sombra. 

Todos estes casos são exemplos da mudança regressiva de personalidade, a
dissolução da estrutura do nosso ser num amontoado de elementos sem
ordem. Vemos a situação do ser humano como uma cidade sitiada, como um
grande império desmoronando a partir do seu núcleo. E é claro que uma
perturbação espontânea e não dirigida numa máquina tão complexa é mais
suscetível de modificá"-la para pior do que para melhor. No entanto,
chegamos agora à questão que para mim precisa ser colocada de forma
mais urgente neste ensaio.  Quero dizer que mesmo estas perturbações
espontâneas e não dirigidas provocam em certos casos uma mudança  que é
um nítido melhoramento.  Com exceção das experiências diretas, são elas
a prova de que, de fato, somos capazes de nos reestruturar num padrão
superior ao que exibíamos previamente, que somos capazes de nos fundir
e de nos cristalizar novamente com maior clareza do que antes; ou,
digamos de forma mais modesta, que a inconstante duna de areia do nosso
ser pode se recompor de repente numa estrutura mais firme e mais bem
equilibrada.

Entre os casos desta natureza registrados até agora, nenhum é mais 
notável do que o daquela conhecida paciente do Dr.~Azam, Félida X.

Muitos dos meus leitores poderão lembrar que nessa mulher a vida
sonambúlica havia ocupado o lugar da vida normal; esse seu “segundo
estado”, que a princípio aparecia somente em acessos curtos,
semelhantes a um sonho, gradualmente substituiu o “primeiro estado”,
que atualmente retorna apenas durante algumas horas e a longos
intervalos.  Mas o ponto que quero salientar é este: que o segundo
estado de Félida X. é em conjunto superior ao primeiro --- fisicamente
superior, desde que as dores nervosas que a atormentavam desde a
infância desapareceram, e moralmente superior, na medida em que sua
atitude morosa e egoísta foi trocada por um estado de atividade e boa
disposição que lhe permite cuidar de seus filhos e de sua loja com
muito mais eficiência do que quando se encontrava em seu “estado
estúpido”, que é como ela chama agora aquela única outra personalidade
de que tinha conhecimento. Neste caso, portanto, que já vai para trinta
anos de evolução, o reajustamento espontâneo das atividades nervosas 
--- o segundo estado, do qual ela não guarda memória alguma quando se
encontra no primeiro --- resultou numa melhora mais profunda do que
poderia ter sido previsto por qualquer tratamento moral ou médico que
seja do nosso conhecimento.  Este caso nos mostra com que frequência a
palavra “normal” significa apenas “aquilo que existe”. Porque o estado
normal de Félida era na verdade o seu estado mórbido; e a nova
condição, que a princípio parecia não passar de uma mera anormalidade
histérica, a conduziu para uma vida de saúde física e mental,
tornando"-a uma mulher equivalente à média das mulheres de sua classe.  

Um ou dois breves exemplos poderão dar uma ideia dos benefícios morais e
físicos que a hipnose está trazendo para o alcance da medicina prática.
Primeiro, narrarei um dos casos --- por enquanto raros --- em que uma
pessoa insana foi hipnotizada com resultados benéficos e permanentes.

No verão de 1884 havia no asilo da Salpêtrière uma mulher jovem do tipo
mais deplorável.  Jeanne Sch. era uma lunática criminosa, de hábitos
imundos, atitudes violentas, e com uma vida inteira de devassidão e
furtos.  Monsieur Auguste Voisin, um dos médicos da equipe, começou a
hipnotizá"-la em 31 de maio daquele ano, num período em que ela só podia
ser mantida quieta pelo emprego da camisa de força e do “boné de
irrigação”, ou seja, uma permanente ducha de água fria na cabeça. Ela
não queria, ou na verdade não podia, encarar de frente o operador, mas
praguejava e cuspia na sua direção. Monsieur Voisin manteve o rosto
próximo ao rosto dela, e seguiu os seus olhos para onde quer que eles
se movessem. Em cerca de dez minutos conseguiu produzir nela um sono
cheio de estertores, e com cinco minutos mais ela entrou num estado de
sonambulismo e começou a falar de forma incoerente.  Este processo foi
repetido ao longo de vários dias, e aos poucos ela se tornou sã quando
estava em transe, embora voltasse a ser uma lunática furiosa ao
despertar.  Gradualmente, tornou"-se capaz de obedecer no estado de
vigília ordens recebidas durante o transe hipnótico --- a princípio
ordens triviais (varrer o quarto, e assim por diante) e depois ordens
que envolviam uma acentuada mudança de conduta.  Não, mais do que isso:
em seu estado hipnótico ela se arrependeu voluntariamente de sua vida
anterior, fez uma confissão que envolvia uma quantidade de atos
malévolos que a polícia desconhecia por completo (embora estivesse em
concordância com fatos já do conhecimento público) e finalmente, por
iniciativa própria, tomou uma série de resoluções concernentes ao seu
futuro.  Dois anos se passaram, e Monsieur Voisin me escreveu, em 31 de
julho de 1886, que agora ela trabalha como enfermeira num hospital em
Paris e que sua conduta é irrepreensível.  Neste caso, e em outros
casos recentes de Monsieur Voisin, há espaço para controvérsia,
naturalmente, quanto à natureza precisa e à prognose, com exceção do
hipnotismo, da loucura que foi curada.  Mas o meu ponto de vista está
plenamente justificado pelo fato de que esta pobre mulher, cuja
história desde os treze anos tinha sido de ininterrupta loucura e de
vício, hoje é capaz de exercer uma profissão como a de enfermeira num
hospital, que necessita de firmeza e autocontrole, e que a
personalidade reformada manifestou"-se primeiramente através da hipnose,
em parte obedecendo a sugestões, e em parte como o resultado natural da
atenuação de suas paixões mórbidas.

Mas aqui preciso encerrar o meu ensaio.  Quero concluir com uma única
reflexão que de certo modo pode ir de encontro aos receios daqueles a
quem não agradam quaisquer interferências em nossa personalidade, e
receiam que esse tipo de análise invasora possa vir a roubá"-los de seu
verdadeiro Eu.  Todas as criaturas vivas, ao que se diz, lutam para
conseguir o máximo de prazer.  Em que momentos, então, e em que
circunstâncias, consideramos que os seres humanos têm conseguido
experimentar essa alegria tão intensa? Nossos pensamentos não se voltam
instintivamente, em busca da resposta, para aquelas cenas e aqueles
momentos em que todas as nossas preocupações pessoais e todos os
cuidados com nossos interesses individuais se perdem numa sensação de
união espiritual, seja com aquela alma a quem amamos mais
profundamente, ou com uma poderosa nação, ou com “o mundo inteiro e as
criaturas de Deus”?  Pensamos em Dante com Beatriz, pensamos em Nelson
em Trafalgar, ou em São Francisco naquela colina da Úmbria. E aqui,
certamente, como no grito de Sir Galahad (“Se eu perder a mim mesmo
encontrarei a mim mesmo!”) temos uma indicação de que muitas coisas,
uma grande quantidade de coisas, que nos acostumamos a considerar uma
parte integral de nós mesmos pode ser deixada para trás, e não obstante
ficaremos com uma consciência do nosso próprio ser muito mais vívida e
mais pura do que tivéramos antes.  Esta rede de hábitos e de gostos, de
paixões e de medos, não é, talvez, a manifestação mais perfeita do que
de fato somos.  É a capa protetora em que os nossos rudes antepassados
se envolveram para se proteger das tempestades cósmicas; mas estamos
aos poucos aprendendo a modificá"-la e refazê"-la de acordo com as
necessidades de um clima mais ameno, e se por acaso ela escorregar dos
nossos ombros num dia de sol então poderemos ter o vislumbre de algo
mais antigo e mais glorioso que existe dentro de nós mesmos. 




\clearpage
\ifodd\thepage ~ \clearpage\else\relax\fi
\thispagestyle{empty}
\mbox{}\vfill
{\noindent\itshape Este texto foi extraído da seção \emph{\textsc{vi}} da parte 
\emph{\textsc{iii}} do livro \emph{Body and Will, Being an Essay concerning Will 
in its Metaphysical, Physiological and Psychological Aspects}, na 
sua edição de 1883, a de data mais próxima ao surgimento de \emph{O estranho caso 
do Dr.~Jekyll e Mr.~Hyde}. Seu autor, Henry Maudsley (1835--1918),
foi um médico e psiquiatra considerado uma das maiores autoridades de sua época
sobre doenças mentais.}

\chapter[As desintegrações do ego\\ \textit{Henry Maudsley}]{As desintegrações do ego\subtitulo{Henry Maudsley}}
\hedramarkboth{as desintegrações do ego}{henry maudsley}

Nessa organização intrincadamente sofisticada e complexa que é a base
física da nossa mente estão representados, de forma direta ou indireta,
todos os interesses do nosso corpo inteiro, toda a nossa energia
orgânica: não existe nada em nossa realidade exterior que não esteja
representado, por assim dizer, na nossa realidade interior.  Não apenas
um órgão, mas todos os órgãos; não apenas uma estrutura, mas todas as
estruturas; não apenas um movimento, mas todos os movimentos; não
apenas uma sensação, mas todas as sensações, todas as vibrações da
energia, sejam de que natureza forem, de todas as partes do corpo, das
mais próximas às mais remotas, as mais vis e as mais nobres,
conscientes e infraconscientes --- toda essa corrente flui para um centro
unificador e traz as suas contribuições, sensíveis ou insensíveis,
para o desempenho das funções de nossa consciência.  A mente é o órgão
central das sínteses, das simpatias e das sinergias do corpo, e a
vontade é, nas circunstâncias mais favoráveis, a suprema expressão
dessa unidade.  Desse modo, é na vontade que está contido o caráter:
não apenas o caráter da mente, como é comumente entendido, mas o
caráter de cada órgão do nosso corpo, cujas funções concordantes
contribuem para a expressão completa da nossa individualidade.

Sendo assim, é evidente que a desordem na união dos centros cerebrais
superiores pode se tornar mais ou menos uma dissolução do eu
consciente, do Ego, de acordo com a profundidade do dano produzido na
unidade fisiológica.  Se qualquer órgão do corpo se tornar defeituoso,
isso significa uma ruptura na suprema unidade da consciência, por se
tornar um déficit de energia, e uma perturbação na medida em que ele
exerce influência sobre outros órgãos; é como quando, numa junta de
cavalos, um deles não desempenha seu trabalho de maneira uniforme com
os demais.  A sensação constante de identidade pessoal, que os
metafísicos indicam de forma tão enfática como uma intuição fundamental
da consciência, discernindo nela o traço incontestável e a prova de um
\textit{ego} espiritual com o qual eles não conseguem estabelecer
contato de nenhuma outra maneira, pode, pela nossa experiência, ser às
vezes incerta e cambiante, em outros casos discordante e dividida, e,
em casos extremos, ver"-se extinta.  Mas esta é uma expectativa muito
propensa ao fracasso quando se considera o “eu”, o “ego” --- aquele
\textit{ens unum et semper cognitum in omnibus notitiis}\footnote{ “O ser
que é só um e sempre compreendido como um só.”}
--- o qual, segundo se alega, é percebido de forma mais ou menos clara
por nós em cada manifestação de nossa inteligência.  Olhemos com
franqueza para os fatos, então, e vejamos que conclusões eles nos
permitem tirar.  Será que existe a mínima consciência do seu “ego” em
um idiota, um indivíduo destituído de sensatez, do uso da fala, um
indivíduo que uiva, se espoja, se suja, não consegue se defender e vive
à mercê de tudo?  Um indivíduo cujos centros cerebrais defeituosos são
incapazes de responder mesmo às impressões vagas e imperfeitas que seus
sentidos embotados lhe transmitem, e incapaz de fazer associações entre
as poucas, confusas e vagas impressões que recebe?  Não há dúvida de
que do seu corpo, enquanto ele permanece coeso em virtude dos cuidados
alheios, pode"-se dizer que possui um “ego”, ou um “si mesmo”, mas do
ponto de vista humano o que é isto?  Não é um “ego” mental, uma vez que
o mecanismo orgânico central em que as energias inferiores do corpo
poderiam obter uma representação mais elevada, e onde poderia ter lugar
a organização mental --- onde poderia se dar essa já mencionada síntese,
simpatia e sinergia --- ou está completamente ausente ou mal formado de
modo irremediável.  Este miserável espécimen de degeneração não sabe e
não tem condições de saber que é um “eu”, ou mesmo de sentir que existe
um “eu” humano degradando"-se nele.  Se a prova certa e segura da
existência de uma alma independente do organismo, e a esperança, nela
baseada, de que ela pode obter uma ressurreição para a vida eterna,
jazem na existência, permanente e distinta, da consciência de uma
identidade por entre todas as mudanças e todos os acidentes por que
passa nossa estrutura mortal, então é sem dúvida uma pena que esta
prova nos falte justamente naqueles casos em que sua certeza nos seria
mais necessária, mais reconfortante e consoladora, e seu sucesso mais
triunfante.

A verdade é que as múltiplas variedades do desarranjo mental produzem
exemplos de diminuição gradativa no brilho da consciência do eu, até a
sua total extinção, bem como todos os tipos de desarranjo e confusão
sobre ela, desde a mais leve até a mais grave interferência.  A
dificuldade em cada caso particular é sempre saber exatamente qual é o
defeito ou a confusão, uma vez que não podemos penetrar na mente de
outra pessoa, perceber o estado de sua consciência, e desse modo
medi"-la e apreciar seu estado ou sua qualidade.

Acontece com frequência, nos casos de desordem mental, principalmente em
suas primeiras fases, que o indivíduo se queixe de ter mudado de uma
maneira completa e dolorosa, que não é mais ele mesmo, mas que se sente
indescritivelmente estranho; e que as coisas ao seu redor, embora
mantenham o seu aspecto costumeiro, de algum modo lhe parecem muito
diferentes.  “Estou tão mudado que sinto como se não fosse eu mesmo, e
sim outra pessoa; embora eu saiba que é uma ilusão, é uma ilusão de que
não consigo me livrar; todas as coisas me parecem estranhas e eu não
consigo apreendê"-las por completo, mesmo sendo familiares; elas
me parecem distantes e mais parecem figuras de um sonho do que coisas
reais, e na verdade é como se eu estivesse dentro de um sonho e minha
vontade estivesse paralisada.  É impossível descrever a sensação de
irrealidade que eu tenho para com todas as coisas; eu garanto a mim
mesmo o tempo todo que eu sou eu, mas não consigo fazer com que minhas
impressões se instalem corretamente em mim e se encaixem numa relação
de familiaridade com o meu verdadeiro eu; entre o meu eu atual e o meu
eu passado parecem estar interpostas uma eternidade de tempo e um
infinito de espaço; o sofrimento que sinto é indescritível\ldots{}” --- esta é
a linguagem com que essas pessoas procuram exprimir a profunda mudança
ocorrida nelas, que sentem da mais dolorosa das maneiras mas não
conseguem descrever adequadamente.

Um exemplo interessante e notável de uma mudança de identidade pessoal
nos é fornecido por uma forma de desordem mental que, por oscilar com
regularidade entre duas fases opostas e alternadas, foi chamada pelos
estudiosos franceses “insanidade circular”, mas seria melhor
denominá"-la “insanidade alternada”.  Uma crise de muita excitação
mental, com grande exaltação de pensamento, sentimentos e conduta é
seguida por uma fase oposta e sombria com depressão, desalento e
apatia, cada um desses estados com a duração de semanas ou meses, e com
a sucessão usual deles retornando de tempos em tempos, depois de
intervalos mais ou menos longos de sanidade.  O contraste entre esses
dois estados é extraordinário, e fácil de imaginar.  No primeiro, o
indivíduo está entusiasmado, exultante, autoconfiante, orgulhoso,
transbordante de energia; fala abertamente sobre assuntos íntimos que
jamais teria mencionado no seu estado normal, e trata com familiaridade
pessoas tanto acima quanto abaixo de sua condição social, pessoas às
quais, quando são, ele jamais se dirigiria; do mesmo modo, escreve
longas cartas, cheias de detalhes e de opiniões, negociações e
projetos, para pessoas com quem tem apenas o mais superficial dos
relacionamentos; gasta dinheiro desordenadamente, embora não seja essa
a sua disposição natural; faz projetos para aventuras ousadas e mesmo
fantasiosas; demonstra estar sempre disposto e contente em falar em
público, mesmo que nunca o tenha feito antes; mostra"-se descuidado
quanto às convenções sociais e chega mesmo a não observar as
reticências e precauções de ordem moral; escuta os conselhos prudentes
que recebe, mas não lhes dá ouvidos, parecendo sempre imbuído de uma
extraordinária sensação de bem"-estar, de poder intelectual, de uma
inteligência e uma vontade totalmente sem amarras.  O que se vê aí não
é uma disruptura do ego, mas uma extraordinária exaltação dele, de fato
uma extrema alienação de natureza moral mais do que intelectual.  Esta
condição em muito se assemelha àquela que precede a mania aguda, onde
se observa uma grande exaltação mental sem que haja verdadeira
incoerência, alienação do caráter sem alienação da inteligência, mas
não é, tal como ela, seguida por uma turbulência degenerativa; pois
quando passa a excitação sobrevém a segunda fase, de extremo abatimento
mental e prostração moral. 

Como esse indivíduo está agora diferente do que era!  Está tão inseguro
quanto era antes autossuficiente; tão retraído quanto era antes
invasivo; tão tímido e silencioso quanto era antes barulhento e loquaz;
tão impotente em pensar e agir quanto era antes disposto e cheio de
energia para o que tentasse fazer; tão inteiramente oprimido por uma
sensação avassaladora de incapacidade física e mental quanto antes
estava possuído por uma sensação exultante dos próprios poderes.  Para
todos os fins e todos os propósitos ele é outra pessoa, outro ego, pelo
menos em tudo que diz respeito à consciência --- subjetivamente, embora
não objetivamente --- uma vez que em todos os seus relacionamentos ele
sente, pensa e age de maneira muito diferente.  Não menos acentuada do
que a transformação mental é a verdadeira transfiguração corporal que a
acompanha em alguns casos: porque durante a fase de exaltação há uma
intensificação geral das funções corporais que fazem o indivíduo
parecer, e sentir"-se, anos mais jovem.  A pele é mais fresca e macia,
suas rugas são suavizadas, os olhos tornam"-se brilhantes, vivazes e
cheios de animação, o cabelo menos grisalho do que talvez tenha sido, o
pulso mais vigoroso, a digestão mais pronta, a atividade dez vezes
aumentada, e uma mulher já entrada em anos pode tornar"-se fértil
novamente.  Durante a prostração que se segue, o contraste é tão
pronunciado que este indivíduo não seria considerado como a mesma
pessoa por alguém que o conhecesse apenas superficialmente; pois cada
um dos sinais de juventude e vigor antes descritos deu lugar agora a
sinais correspondentes de idade avançada e de fraqueza física.  No
primeiro estado, é como se ele tivesse experimentado o elixir da vida;
no outro, como se tivesse provado a apatia da morte. 

Um fato interessante, e que não pode deixar de chamar nossa atenção, é
que durante o estado de exaltação dessa desordem alternada a pessoa refaz
com uma repetição quase automática as coisas que já fez, e tem os mesmos
pensamentos e sentimentos que teve nos seus estados de exaltação
anteriores; e durante o estado de prostração ele pensa, sente e faz
tudo de acordo com o que fez em seus estados de prostração anteriores.
Em cada um dos estados, contudo, ele não tem uma lembrança clara e
precisa dos eventos do outro; provavelmente não chega a esquecê"-los por
completo, mas guarda deles apenas aquele tipo de recordação vaga,
enevoada e incompleta que geralmente se tem dos acontecimentos de um
sonho, ou que um homem tem, quando sóbrio, das sensações que teve e dos
atos que praticou quando embriagado.  E de fato, como poderia ele
recordá"-los com clareza, se é evidente que lhe seria forçoso, para
tanto, reproduzir exatamente em si mesmo aquele estado, encontrando"-se
no estado oposto? É impossível, portanto, que ele possa ter uma ideia
clara das experiências de um quando está no outro, embora ele possa ter
conhecimento do fato simples de que aquelas coisas aconteceram com ele,
e, sentindo"-se um tanto envergonhado por essas lembranças, e
preocupando"-se com as demais coisas de que não consegue se recordar,
não tenha como evocá"-las na memória e falar sobre elas.

Portanto, a despeito de tudo que a teoria psicológica possa afirmar em
contrário com base em seus próprios oráculos internos, está provado de
maneira incontestável pela observação de casos concretos que existem 
estados de desordem da consciência que, sendo bastante distintos dos
estados de consciência normais, não podem ser tratados como eles, e
cujos eventos só podem ser lembrados de maneira nebulosa, pressentidos
difusamente mais do que recordados, ou completamente esquecidos.  A
lição que eles nos dão é uma lição que vem sendo afirmada cada vez mais
no terreno da psicologia, ou seja, que a consciência do eu, a
unidade do \textit{ego}, é uma consequência, e não uma causa; é a
expressão do funcionamento completo e harmonioso de um agregado de
centros mentais diferenciados, e não uma misteriosa entidade metafísica
que jaz por trás dessas funções, inspirando"-as e guiando"-as; é uma
síntese ou unidade subjetiva baseada nas sínteses ou unidades objetivas
do organismo. Nessa condição, ela pode ser obscurecida, desarranjada,
dividida, aparentemente transformada.  Para cada falha na unidade dos
centros unidos existe uma falha nela; subtraia"-se cada um dos centros
mentais dessa íntima cooperação fisiológica, e o eu é na mesma medida
enfraquecido ou mutilado; obstrua"-se ou desarranje"-se a função
condutora dos feixes de ligação entre os vários centros, de forma que
eles fiquem dissociados ou desunidos, e o eu perde num grau
correspondente seu senso de união e de continuidade; estimule"-se um ou
dois centros ou grupos de centros até uma hipertrofia mórbida de modo a
que eles absorvam uma parte maior da alimentação mental e se tornem
numa função exclusiva ou predominante, e a personalidade parece ter
sido transformada; faça"-se a excisão de uma camada inteira dos centros
superiores --- aquela parte mais elevada que se encarrega do raciocínio
abstrato e dos sentimentos morais --- e o ser humano será reduzido à
mesma condição de qualquer do animais superiores; suprimam"-se de uma só
vez todos os centros, e ele se tornará uma criatura apenas senciente;
removam"-se os centros correspondentes aos sentidos, e ele será reduzido
a uma existência meramente vegetativa, em que, como uma hortaliça, ele
terá um Eu objetivo mas não um subjetivo. Estas são as conclusões a que
somos conduzidos quando, sem obscurecer os fatos, observamos
sinceramente a natureza e a interpretamos com fidelidade,
experimentando os fatos para compor o nosso entendimento, ao invés de
apelar para nossa própria imaginação para nos servir de oráculo. 





