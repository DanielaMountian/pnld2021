\chapter[Nota sobre a organização \bigskip]{Nota sobre a organização}

O livro \emph{Zeppelin} consiste em uma antologia de crônicas de
Mário de Andrade, publicadas em diversos jornais e revistas, como
\emph{Diário Nacional} e \emph{O Estado de S. Paulo}. A maioria dos
textos selecionados foram extraídos do livro \emph{Os filhos da
Candinha} (1943); os demais, de obras organizadas por especialistas no
autor, como \emph{Táxi e crônicas no Diário Nacional} (1976) e
\emph{Será o Benedito!} (1992). Nas crônicas, Mário apresenta alguns dos
principais elementos que caracterizam sua literatura, como a relação
entre as linguagens escrita e falada, a cultura popular, o conflito de
classes, os contrastes sociais da modernização e a busca por uma
identidade nacional. Em alguns textos, atua também como crítico,
colocando em cena nomes importantes da literatura, das artes plásticas e
da música.

Mário de Andrade é conhecido pela sua atuação em diversos gêneros e
áreas do saber, e sua atuação como cronista foi uma das mais
exercitadas. Assim, esse livro vem para enriquecer a apreensão
desse grande escritor brasileiro,
apresentando também uma das facetas instigantes de Mário de Andrade,
tão importante para a valorização da nossa cultura e para
que possamos, desta maneira, nos (re)conhecer cada vez mais.

Para o estabelecimento do texto, cotejamos todas estas edições, além de
a mais recente edição de \emph{Os filhos da Candinha}, publicada em 2013
pela editora Nova Fronteira, com texto fixado por João Francisco
Franklin Gonçalves e revisado por Aline Nogueira Marques. A grafia foi
atualizada conforme o Novo Acordo Ortográfico, mas foram mantidas as
idiossincrasias da escrita de Mário de Andrade, a fim de manter o ritmo
da frase e o sabor de sua ``aventura intelectual'' com a palavra escrita
na sua desvairada relação com o cotidiano arlequinal, mas, por vezes,
pintado de neblinas finas.

%\part{Algumas crônicas}

\chapter{Zeppelin}

E eis que senão quando os brasileiros sentiram sobre suas cabeças um
formidável tumulto de ideias e gritaram: --- Zeppelin! Zeppelin!

Era o Zeppelin que vinha chegando.

Por enquanto essa máquina voadora ainda é muitas coisas pra brasileiro.
É um susto pra alguns. Pra muitos será um monstro de feitiçaria, o Ogum
das nossas macumbas mas ainda dotado do corpo de cobra do Ogum Badagris
dos vodus haitianos, mandando na tempestade. Pra algumas será apenas um
balão a fogo, antecipando a descida de S. João pra namorar no randevu
das cacimbas. E é preciso não esquecer que patrioticamente, seguindo o
versinho tradicional, Zeppelin será neto do santista Bartolomeu Lourenço
de Gusmão e filho do mineiro Santos Dumont. Enfim pra uns poucos será
apenas um dirigível.

Mas será principalmente pra todos, o que já falei no princípio: um
tumulto de ideias. Os jornalistas escreverão piadas. E já estou vendo
daqui todo o Nordeste cantador, botando o Zeppelin em toada de romance,
em dança e nos reisados de Natal. Zeppelin é um tumulto de ideias.

Onde iremos nós?\ldots{} Agora mesmo os médicos americanos descobriram um
jeito de acabar com o cancro e de repente as bocas que se abriam no
movimento dos cinemas principiaram deitando sons como a vida. Etc. etc.
E o Zeppelin veio provar pra São Tomé o sofisma gracioso de que uma casa
dum andar pode ser mais alta que o Martinelli.

Ora, conquanto não se possa chamar esse andar propriamente de andar
térreo, não tem dúvida que se pode ordenhar as maiores ilações dessa
casa que voa. Se uma voa, duas podem voar e não muito pra que chegue o
dia em que o próprio Martinelli e demais arranha"-céus da terra se
mostrarão de novo mais altos que o papiri Zeppelin, convertido em
arranha"-céus dos ares. E então será um novo Juízo Final pra esta
humanidade que aliás vive nele desde o princípio dos séculos.

O progresso há de ser enorme e o homem ficará mesmo vastíssimo. Bota um
objetinho no bolso e záz! vira pintassilgo. Onde que vai? Vai na escola
aérea, uma formidável universidade ambulante que resolverá duma vez o
problema universitário do Brasil. (É verdade que o Brasil desse tempo
não será mais Brasil\ldots{}). Os enterros não buscarão mais sete palmos de
terra deteriorante, em vez, ascenderão lentamente ar em cima, até o
limite do ar, onde coveiros parecidos com escafandros jogarão os
cadáveres no éter. E passarão nos ares palácios presidenciais dando
inveja, monumentos celebrando heróis mundiais, catedrais dizendo
adeusinho, teatros de ópera xingando a arte musical e livros luminosos
em alemão. Livros com três mil e oitocentas páginas cada um.

Não há meios da imaginação parar. E então as próprias cidades é que se
multiplicarão pelos ares. Não serão apenas arranhacéus"-cidades nem as
ruas a vários andares que profetiza o sr. Le Corbusier. As cidades é que
terão vários andares, erguidas sobre grandes placas de algum metal que
não sabemos, destituídas de ladeiras e provavelmente de viadutos. Digo
``provavelmente'' porque se não há propriamente limites pra fantasia,
tem momentos em que ela se quer modesta como todas as coisas poderosas e
conscientes do perigo do mundo. Provavelmente não haverá viadutos
mais\ldots{}

Mas isso ainda não é nada. Então havemos de nos rir do deus Tezcatlipoca
que, embora com a facilidade dos deuses, sempre se deu ao trabalho de
construir uma escada inteirinha dos céus à Terra pra nos vir ensinar o
divino mistério da música. Sem escada e sem nada, já que a Índia não
quer mesmo se sujeitar ao domínio de ninguém, nem por intermédio de Ford
e outras fordescas diabruras a Amazônia virará possessão do ingrêis do
Novo (!) Mundo, os Estados Unidos e a Inglaterra terão como colônias, as
estrelas. A frase é bonita, mas a infâmia continua a mesma, está claro.
Nossa vingança é que nesse tempo a Inglaterra não será Inglaterra mais,
e os Estados Unidos idem. E a pátria mundo\ldots{} provavelmente, não terá
mais reis, nem imperadores, nem presidentes, nem mussolinis, nem mesmo
governos comunistas como os de agora. O chefe da pátria mundo será um
botão ponhamos, elétrico. A gente aperta o botão, pronto: sai uma ordem
que foi escrita pela sabedoria de todos os séculos, compendiada
naturalmente por um grupo de professores alemães. E sinceros.

E não há meios mais da imaginação parar! Imaginem o que será a
psicologia desse tempo! Que Freud nada! Havemos de rir de Freud e de
Zeppelin. Se Bartolomeu e Santos Dumont ficam, é porque foram os
primeiros nas suas direções. Só que nesse tempo não terão mais a
miserável miséria de dar patriotismo pra ninguém, na vasculhada pátria
mundo. Bom, melhor é deixar uma linha em branco onde cada um dos
leitores escreverá o tumulto de ideias que no momento se chama ``seu''
Zeppelin.

\noindent(\dotfill{})

Mas o que me assusta pavorosamente é que mudadas as leis, pátrias e
felicidades, nem por isso a vida humana deixará de ser o que é agora e
já foi no começo dos séculos: inflexivelmente quotidiana.

\chapter{Educai vossos pais}

Nós ainda somos apenas educados pelos nossos pais\ldots{} Se vê a criança
detestando quanto os pais detestam\ldots{} Depois começam desequilíbrio e
hipocrisia. É o tempo do ``no meu tempo''\ldots{} O rapaz, a morena é um
bloco maciço de modas novas. Os pais detestam essas modas e querem
torcer a gente para o caminho que eles fizeram, na bem"-intencionada
vaidade de que são exemplos dignos de seguir. A gente, não é que não
queira, nem pode! Se vive em briga, mentira, dá vontade de morrer.

Creio que, para a felicidade voltar, tudo depende do moço. O melhor é a
gente se fazer passar por maluco. Faz umas extravagâncias bem daquelas,
descarrila exageradamente umas três vezes, depois organiza uma temporada
dramática aí de uns quinze espetáculos. Fazendo isso com arte e amor até
é gostoso. ``Nosso filho é um perdido'', se dizem os pais. Sofrem a
temporada toda, você com muito carinho abana o sofrimento mas sustenta a
mão. E eles a final sossegam, reeducados. E você conquistou a liberdade
de existir.

Há por exemplo o caso da cadelinha Lúcia que me impressiona bem. A
cadelinha Lúcia era uma espécie de Greta Garbo, mais maravilhosa que
linda. Você podia ficar tempo contemplando bem de perto os olhos
inconcebíveis dela, a cadelinha Lúcia não dava um avanço pra abocanhar
nosso nariz. Então chegava a primavera. Você cansava, não dando mais
atenção e a cadelinha Lúcia virava um sabá de flores de retórica,
latidos, estalidos, luzes, festa veneziana, a guerra de 14 e o cisne de
Saint Saëns"-Paulova.

Me esqueci de contar que ela era branca. Um branco tamisado de
esperanças de cor, duma riqueza reflexiva tão profunda que, não sei se
por causa dela se chamar Lúcia, a gente sentia naquele esborrifo andante
os valores da única maravilha desse mundo que tem direito a se chamar de
Lúcia, a pérola.

Lúcia era brabinha como já contei e alimentava grandes ideais. Ninguém
entrava no jardim sem sabá. Isso ela vinha que vinha possuída de toda a
retórica do furor e mais os dentes. Mordia. Estragava a roupa, era uma
dificuldade.

Porém a gente percebia que a cadelinha Lúcia não era feliz. Não lhe
satisfazia arremeter eficaz contra os humanos, e pouco a pouco, na
contemplação latida das grades do seu jardim lhe brotara um ódio
poderoso contra os vultos gigantes da rua. Um dia afinal, pilhando o
portão aberto, saiu como uma sorte grande, era agora! Olhou arrogante, e
enfim vinha lá longe, num heroísmo de polvadeira, o grandioso bonde.
Lúcia esperou, acendrando ódio na impaciência, e quando o bonde já
estava a uns vinte metros, ei"-la que sai em campo, enfunada, panda,
côncava de pérolas febris. Atira"-se e compreende enfim. O bonde só fez
juque! e quebrou a mãozinha direita da cadelinha Lúcia.

Vocês imaginam o que foi aquela morte de filho em casa. Correrias,
choro, médico, telefonemas, noite em claro\ldots{} A cadelinha Lúcia
salvava"-se, mas ficava manquinha pra sempre. Quando veio do hospital,
convalescente e com o enorme laço de ita no pescoço, milagre! o laço
parava no lugar. Continuava o maravilhoso bichinho, mas a alma era
outra.

Dantes preferira a glória ao amor. Agora queria apenas a beleza e o
amor. Mansa, pusera de parte os dentes e os ideais, e todos a adoravam.
E pouco depois dos tratamentos do hospital, teve os primeiros filhos.
Pedidos, presentes, mas um ficou na casa.

Chincho foi educado nessa mansidão. Não era possível a gente imaginar
uma doçura mais suave que a do cachorrinho Chincho. Ora, bons tempos
depois, eu indo naquela casa, a cadelinha Lúcia estava no gramado
entredormida. Eis que ergue a cabecinha, se esborrifa toda e geme um
ladrido desafiante, porém muito desamparado. O que eu vejo! Sai detrás
da casa o cachorrinho Chincho, e vem num sabá furioso sobre mim, quase
recuei, sai, passarinho! Que sair nada! Olhou pra mãe, lá na sua grama,
hesitante:

--- Ajuda, minha velha!

E ela veio reeducada, furiosa pra cima de mim. Me chatearam, quase me
morderam. Depois, quando a criada me salvou, ficaram brincando na grama,
parceiros, muito em família. Educai vossos pais! Não dou três meses, e o
cachorrinho Chincho fará a cadelinha Lúcia odiar os bondes outra vez.
Pode ser que ambos percam a vida nisso, mas não é a vida que tem
importância. O importante é viver.

\chapter{O diabo}

--- Mas que bobagem, Belazarte --- fazer a gente entrar a estas horas
numa casa desconhecida!

--- Te garanto que era o Diabo! Com uma figura daquelas, aquele cheiro,
não podia deixar de ser o Diabo.

--- Tinha cavanhaque?

--- Tinha, é lógico! Se toda a gente descreve o Diabo da mesma maneira!
Está claro que não hei de ser eu o primeiro a ver o Diabo, juro que era
ele!

--- Mas aqui não está mesmo, vam'bora. Engraçado\ldots{} parece que a casa
está vazia\ldots{}

--- Vamos ver lá em cima. Está aí uma prova que era o Diabo! se vê que a
casa é habitada e no entanto não tem ninguém.

--- Mas se era mesmo o Diabo decerto já desapareceu no ar.

--- Isso que eu não entendo! Quando vi ele e ele pôs reparo em mim, fez
uma cara de assustado, deitou correndo, entrou por esta casa sem abrir a
porta.

--- Pensei até que você estava maluco quando gritou por mim e
desembestou pela rua fora\ldots{}

--- Bem, vamos ficar quietos que aqui em cima ele deve estar na certa.
Remexemos tudo. Foi então que de raiva Belazarte inda deu um empurrão
desanimado na cesta de roupa suja do banheiro. A cesta nem mexeu,
pesada. Belazarte levantou a tampa e

--- Credo!

Gelei. Mas imaginava que ia ver o Diabo em pessoa, em vez, dentro da
cesta, muito tímida, estava uma moça.

--- Não me traiam, que ela falou soluçando, com um gesto lindo de pavor
querendo se esconder nas mãos abertas. Era casada, se percebia pela
aliança. Belazarte falou autoritário:

--- Saia daí! O que você está fazendo nessa cesta! A moça se ergueu
abatida.

--- Sou eu mesmo\ldots{} Mas, por favor, não me traiam!

--- Eu, quem!

Ela baixou a cabeça com modéstia:

--- Sim, sou o Diabo\ldots{}

E nos olhou. Tinha certa nobreza firme no olhar. Moça meia comum, nem
bonita nem feia, delicadamente morena. Um ar burguês, chegando quando
muito à \emph{hupmobile}.

--- A senhora me desculpe, mas eu imaginei que era o Diabo; se soubesse
que era uma diaba não tinha pregado tamanho susto na senhora.

Ela sorriu com alguma tristeza:

--- Sou o Diabo mesmo\ldots{} Como diabo não tenho direito a sexo\ldots{} Mas Ele
me permite tomar a figura que quiser, além da minha própria.

--- Então aquela figura em que a senhora estava na frente da igreja de
Santa Teresinha.

--- Aquela é a minha caderneta de identidade.

--- Não falei!

--- Só quando é assim quase de madrugada e já ninguém mais está na rua,
é que vou me lastimar na frente das santas novas.

--- Mas por que que a senhora\ldots{} isto é\ldots{} o Diabo toma forma tão pura
de mulher!

--- Porque só me agradam as coisas puras. Já fui operário, faroleiro,
defunto\ldots{} Mas prefiro ser moça séria.

--- Já entendo\ldots{} É deveras diabólico\ldots{}

A moça nos olhou, vazia, sem compreender.

--- Mas por quê?

--- Porque assim a senhora torna desgraçada e manda pro inferno uma
família inteira duma vez.

--- Como o senhor se engana\ldots{} Pois então não façam bulha.

E por artes do Diabo principiamos enxergando através das paredes. Lá
estava a moça dormindo com honestidade junto dum moço muito moreno e
chato. No outro quarto três piazotes lindos, tudo machinho, musculosos,
derramando saúde. Até as criadas lá embaixo, o fox"-terrier, tudo tão
calmo, tão parecido! Mas a felicidade foi desaparecendo e o Diabo"-moça
estava ali outra vez.

--- Foi pra evitar escândalo que quando os senhores entraram fiz minha
família desaparecer sonhando. Meu marido esfaqueava os senhores\ldots{}

--- Estavam tão calmos\ldots{} pareciam felizes\ldots{}

--- Pareciam, não! Minha família é imensamente feliz (uma dor amarga
vincou o rosto macio da moça). É o meu destino\ldots{} Não posso fazer senão
felizes\ldots{}

--- Mas por que a senhora está chorando então?

--- Por isso mesmo, pois o senhor não entende! Meu marido, todos, todos
são tão felizes em mim\ldots{} e eu adoro tanto eles!\ldots{}

Feito fumaça pesada ela se contorcia num acabrunhamento indizível. De
repente reagiu. A inquietação lhe deformou tanto a cara que ficou duma
feiúra diabólica. Agarrou em Belazarte, implorando:

--- Não! pelo que é mais sagrado nesse mundo pro senhor, não revele o
meu segredo! Tenha dó dos meus filhinhos!

--- É! mas a final das contas eles são diabinhos! A senhora assim de
moça em moça quantos diabinhos anda botando no mundo!

--- Que horror! meus filhos não são diabos não! lhe juro! eu como Diabo
não posso ter filho! Meus filhos são filhos de mulher de verdade, são
gente! Não desgrace os coitadinhos!

Nem podia mais falar, engasgada nas lágrimas. Belazarte indeciso me
consultou com os olhos. A final era mesmo uma malvadeza trazer
infelicidade, assim sem mais sem menos, pra uma família inteira. A moça
creio que percebeu que a gente estava titubeando, fez uma arte do Diabo.
Principiamos enxergando de novo a curuminzada, o fox, tão calminhos\ldots{}
Só o moço estava mexendo agitado na cama, sem o peso da esposa no peito.
Se acordasse era capaz de nos matar\ldots{} A visão nos convenceu. Seria uma
cachorrada desgraçar aquela família tão simpática. Depois o bruto
escândalo que rebentava na cidade, nós dois metidos com a Polícia,
entrevistados, bancando heróis contra uma coitada de moça. Resolvi por
nós dois:

--- A senhora sossegue, nós vamos embora calados.

--- Os senhores não me traem mesmo!

--- Não.

--- Juram\ldots{} juram por Ele!

--- Juro.

--- Mas o outro moço não jurou\ldots{}

Belazarte mexeu impaciente.

--- Que é isso, Belazarte, seja cavalheiro! Jure!

--- \ldots{}juro\ldots{}

A moça escondeu depressa os olhos numa das mãos, com a outra se apoiando
em mim pra não cair. Era suave. Pelos ofegos, a boca mordida, os
movimentos dos ombros, me pareceu que ela estava com uma vontade danada
de rir. Quando se venceu, falou:

--- Acompanho os senhores.

E sempre evitando mostrar a cara, foi na frente, abriu a porta, olhou a
rua. Não tinha ninguém na madrugada. Estendeu a mão e teve que olhar pra
nós. Isso, caiu numa gargalhada que não parava mais. Torcia de riso, e
nós dois ali feito bestas. Conseguiu se vencer e virou muito simpática
outra vez.

--- Me desculpem, mas não pude mesmo! E vejam bem que os senhores
juraram, hein! Muito! muito obrigada!

Fechou depressa a porta. Estávamos nulos diante do desaponto. E também
daquela placa:

``\textsc{doutor} Leovigildo Adrasto Acioly de Cavalcanti Florença, formado em
Medicina pela Faculdade da Bahia, Diretor Geral do Serviço de Estradas
de Rodagem do Est. de São Paulo. Membro da Academia de Lettras do Siará
Mirim e de vários Institutos Historicos, tanto nacionaes como
extrangeiros.''

\chapter{O culto das estátuas}

Fenômeno bem curioso de psicologia social é a deformação por que passa
frequentemente nas cidades o culto dos mortos mais ou menos ilustres. O
culto verdadeiro, sendo subsidiário por demais, raro existe de homens
pra mortos. A gente cultua facilmente Deus, deuses e assombrações,
porque pra com essas forças conspícuas do além, o culto é mais
propriamente uma barganha de favores, um dá"-cá"-e"-toma"-lá em que sempre
nos sobra esperança de mais ganhar do que dar. Outro culto propício é o
dos vivos poderosos ou célebres. Os poderosos poderão nos dar um naco da
sua força. E viver ao pé dos célebres é o processo mais seguro de sair
nas fotografias.

Já o culto dos mortos é pouco rendoso e os homens o foram substituindo
pelo culto das estátuas. No fundo não deixa de ter bom resultado este
culto: nós substituímos a memória do morto, difícil de sustentar, por um
minuto vivo de beleza. Em verdade a função permanente da estátua não é
conservar a memória de ninguém não, é divertir o olhar da gente. O fato
é que bem pouco as estátuas divertem\ldots{} Não só porque são raríssimas as
estátuas bonitas, como porque saber se divertir com o feio é já um grau
muito elevado de sabedoria, pra ser de muitos.

Até aqui não foi doloroso falar, porém agora principia sendo\ldots{} Além de
ser muitíssimo relativa a memória do morto na estátua, será mesmo que
muito cadáver ilustre merece a eternização da escultura? Toda estátua
pública tem de representar um culto público, a rua é de todos. Sei bem
que uma unanimidade é coisa democraticamente impossível, porém certos
homens, mais pela ideia que representam que por si mesmos, podem merecer
um culto geral. E se a maioria dos praceanos talvez ignore esse homem,
carece não esquecer que a estátua deve ter uma função educativa.

Neste ponto é que a porca torce o rabo. Só enxergo um jeito do
monumento, ser educativo: é pela grandiosidade obstruente e
incomodatícia. O monumento, pra chamar a atenção de verdade, não pode
fazer parte da rua. O monumento tem que atrapalhar. Uma dona em toalete
de baile é muito mais monumental na rua Quinze, mesmo sendo
catatauzinha, que a estátua de Feijó e a própria escadaria de Carlos
Gomes. A gente passa e indaga logo: Quem será! Isso os comerciantes
perceberam muito bem, principalmente depois que chegaram os Estados
Unidos e a eletricidade. É incontestável que o anúncio erguido à
``memória'' de tal cigarro ou sabonete, no Anhangabaú, é monumento que
jamais Colombo não teve.

Tudo isso são coisas que se provam por si pra que eu insista sobre. Em
São Paulo, com exceção do monumento do Ipiranga e o do Conde Matarazzo
que são os únicos monumentais e educativos, todas as outras estátuas não
passam de mesquinharias. Por que existem?\ldots{} Se não nos cansamos de
espetar estátuas nas ruas é porque o nosso egoísmo substituiu o culto
dos mortos pelo culto das estátuas.

A egolatria não consiste apenas na adoração do eu: tudo o que não seja
humanidade como amor social não passa de manifestações interessadas de
egolatria. Existe egolatria de família, de classe. E a mais monstruosa
de todas: a nacional. Mas se esta é monstruosa, a egolatria de facção é
a mais mesquinha. A que desmandos estatuários leva o grupo de amigos!\ldots{}

Em torno dum homem de certo valor, os admiradores vão se transformando,
pela frequência, em ``grupo dos amigos de''. Uma quarta"-feira morre o
homem. O grupo dos amigos fica despojado, num mal"-estar medonho.
Sentimentalizados pela proximidade do vazio, carecem dar um derivativo
ao cincerro do sofrimento pessoal, e se torna imprescindível sufocar o
abatimento por meio duma vitória qualquer. Não é o morto que tem de
vencer, esse já está onde vocês quiserem, quem tem de vencer é o grupo
dos amigos. E se note que muitas vezes esses amigos (do morto) nem se
dão entre si! O ``grupo'' se justifica pela admiração sentimentalizada do
morto, e esses indiferentes entre si, se percebem irmãos. Não deixa de
ser comovente. Só não acho comovente é o derivativo: ---Vamos fazer uma
estátua!

E a estátua se faz. Quais são os que apenas conhecem mais intimamente
Carlos Gomes em suas obras, dentre os que povoaram com porcelanas
ocasionalmente de bronze a esplanada do Municipal? O que significará
Verdi pra uma cidade em que a própria colônia italiana preferirá mil
vezes Os Palhaços ao Falstaff?\ldots{} Sim, mas quando um do ``grupo dos
amigos de'' escuta falar na estátua ou mais raramente no morto, jamais
não se esquecerá de sentir (e às vezes proclamar) que foi ele, Ele, quem
ajudou a erguer\ldots{} a memória do morto? Não, a estátua. O ególatra incha
todo na satisfa pessoal duma vitória. O culto continua inexistente. O
morto mais que morto.

Os transeuntes passarão pela estátua, a primeira vez olhando. ``É uma
estátua'', dirão. Os de maior atividade espiritual irão mais longe e
prolongarão o pensamento até um ``É feia'' ou ``É bonita''. Poucos irão ler
o nome, não do morto, mas da estátua. Raríssimos saberão quem é, mas a
estes será desnecessário o culto da estátua para que cultivem o morto. E
quando muito a estátua daí em diante servirá uma vez por outra como
ponto de referência ou marcação de randevu. E para sempre só os turistas
a olharão, não pra saber do morto, mas pra se distrair com a estátua. E
todos os turistas verificarão indiferentemente, só pra meter o pau na
terra visitada, ``É feia''.

E de"-fato será sempre ``feia'' porque apenas estátua. Um bronzinho magro,
uns granitos idiotas. A tal de vitória do ``grupo dos amigos'' não passou
também duma auto"-sugestão. A festa se resumiu a uma subscrição de má
vontade e no presentinho coletivo das alunas pelos anos da professora:
um bibelô.

A rua é de todos, e nela Pereira Barreto, Ramos de Azevedo, Feijó e a
infinita maioria dos mortos se nulifica. Na rua que é quotidiana, de
trabalho e vida viva, eles não adiantam nada. Não passarão jamais de
estátuas. Feias.

\chapter{Conversa à beira do cais}

Outro dia, dia útil, almocei com um amigo num dos restaurantes do
mercado no Rio. Talvez tenha me excedido na moqueca de peixe, excelente.
Sei que me senti bastante completo, com desejo de ficar só, como nos
momentos raros de perfeição. Por isso, meu amigo tendo seus afazeres e
eu nada, me despedi gostosamente dele, e fiquei banzando, pensamento
entrecerrado, pelo cais da Praça Quinze. Olhava o mar que se descortina
dali, curto, sem a menor oceanidade.

Foi quando se aproximou um homem, isto é, não se aproximou não, quando
ele falou comigo, pus reparo num homem debruçado no parapeito do cais.
Apontando uma barca rápida que entrava, comentou que decerto estava
cheia de tainhas. Não sei por que sentimento de complacência fui
perguntando ao homem qualquer coisa sobre a pesca das tainhas, se não
era feita de noite, se a barca então saíra na véspera e vinha atrasada,
ou se, marupiara, chegava primeiro, nem lembro. O homem secundava manso,
com forte acento português, mirando o mar. Era operário, de fato
português, com apenas dois anos de Brasil e nenhum conhecimento das
tainhas.

De repente parou a frase, meio que se virou me olhando pela primeira vez
e perguntou:

--- \emph{Êtes"-vous français}?

--- \emph{V'oui}, que eu falei, com uma espontaneidade tão absurda que
nem pude me divertir com a pergunta. E esta era perfeitamente maluca:
nada tenho de francês nem no corpo nem na fala. Donde vinha agora aquele
portuga me pensar francês! Porém nada me divertia, estava era
espaventado com a espontaneidade da minha mentira. E logo imaginei com
primaridade acomodatícia: a resposta fora um puro relexo de vaidade,
pois não tinha dúvida que me timbrava como um elogio pelo ser inteiro,
aquela invenção do operário.

A conversa continuou em francês e, está claro, desde então dirigida
cuidadosamente por mim. Quis logo saber se o homem estivera na França
(sim), quanto tempo (dois anos), em que cidade (Lião, Marselha).
Sobretudo os lugares em que o operário estivera me preocuparam muito no
começo, pra não coincidir com eles nalguma resposta. Então me situei
desafogadamente ao Norte, em Paris onde nasci, no Havre e em mais duas
ou três invenções que meu juízo reputou bem pobres geograficamente.
Depois fui consertando com paciência a invencionice: meu pai é que era
bem francês, viera ao Brasil, servira como gerente numa casa de sedas em
São Paulo (hoje me envergonho da associação de imagens larvar: Lião
---sedas), e se casara a final nesta São Paulo com uma senhora
paranaense que, pra viver em São Paulo, fiz ser ilha de militar. Foram
em viagem de núpcias para a França e eu nasci em Paris, voilà. Minha mãe
fez questão de me registrar no consulado, mas educado num liceu de
Versalhes (gostei de Versalhes que aumentou a geogra ia), tendo passado
a mocidade em França, me sinto muito mais francês que brasileiro. De
resto, estava no Brasil apenas de passagem, colhendo uma herança curta (
fiz questão do ``curta'' pro homem não me pedir dinheiro) e voltava
dentro de uns três meses. Então metemos o pau no Brasil.

Mas fiquei logo com vontade de me vingar do companheiro e meti o pau em
Portugal, por causa do fachismo. Mas o português me ajudou. Então meti o
pau na França, meti o pau na Europa e, é incrível! não uso patriotices,
mas não sei o que me deu: me deu uma vontade enorme de elogiar o Brasil,
fiz. Ele aceitou, com indiferença, meus comentários sobre a doçura,
apesar de tudo, desta nossa vida brasileira.

De vez em quando eu argotizava com aplicação pra me naturalizar bem
francês. Ele, por si, contou uma história labiríntica que me pareceu
muito mal contada, em que entrava uma mulher bastante rica apaixonada
por ele, mais a ilha dela e com a mesma paixão. Tudo de uma imoralidade
exemplar, certos detalhes!\ldots{} De repente, tive a noção absolutamente,
posso dizer, concreta de que o homem estava mentindo. Engoli a mentira
toda bem quietinho. E concebi concomitantemente o pensamento de que
talvez ele já me perguntasse se eu era francês por simples mentira,
apenas pra poder contar sua estadia verdadeira em Marselha e Lião. Mas
fui besta, botando importância em mim; e ele tivera que esperar aquela
conversalhada, pra achar ouvidos que escutassem o seu mundo imaginário
de sexualidades escatológicas. Estava tão distraído nestas dúvidas, que
a conversa entreparou, desiludida. O operário aproveitou a estiada e se
despediu tocando levemente no chapéu. Isto é, boné.

\chapter{Morto e deposto}

Jesus Cristo morreu mais uma vez. Os processos humanos de adoração,
repetindo a tragédia medonha, puseram de novo a imagem Dele num caixão,
as tochas conduziram vultos lerdos, o sepultamento se fez. Fizeram com
alarde que percebêssemos a morte de Jesus e muitas almas ficaram cheias
de cuidados.

De alguns anos para cá, principalmente depois que a guerra grande se
acabou, a morte de Jesus se torna cada vez mais insofismável. A
humanidade contemporânea, como coletivo, se afasta cada vez mais da
imagem de Jesus. A morte Dele é um enterro anônimo que atravanca as
ruas, tem um rito impossível. As prefeituras e as polícias deviam de
proibir isso, como proíbem outros hábitos que não são da época mais.

É que Jesus não está morto apenas, está morto e deposto. Bem que Ele
falara, no seu conhecimento extemporâneo, que o reino Dele não era deste
mundo\ldots{} Mas possuía uma grandeza tão imensa que, além de salvação
individual dos homens, Jesus se tornou uma razão"-de"-ser social e deu
origem a uma civilização.

Os homens também não tiveram a culpa disso, porque as civilizações
transcendem às vontades humanas, mas essa foi a causa dos cuidados de
agora. Se fez a Civilização Cristã que, apesar de todas as grandezas
dela, é um insulto à grandeza de Jesus.

Mas isso não é o pior. O pior é que ela, que nem todas as civilizações,
tinha que se acabar e se acabou. A humanidade de hoje, apesar de todas
as ligações que ainda a prendem à Civilização Cristã, tem outra maneira
concreta de ser, outra moral prática, outros sentimentos, ideais e
paixões imediatas. E a gente assiste a essa fragilidade humana ridícula
que faz com que os destroços da Civilização Cristã despenquem sobre o
corpo morto Daquele que, por sua grandeza, teve a fatalidade de a criar.
Assim Jesus não está só morto não. Está deposto. Ele perdeu toda a
magnitude social e nem um prisioneiro no Vaticano é mais.

Ora graças a Deus!

Até que enfim vai se acabar também toda a cegueira que desvirtuava quase
que completamente a vida terrestre de Deus. Jesus morreu pra nos salvar
da terra, não pra nos salvar num pedaço de terra. Liberta a figura Dele
de todas as condições sociais e apetites civilizadores que A mascaravam,
Ela se livra da vaidade nossa. Agora Jesus está bem liberto mesmo, e
talvez o ato mais prodigioso da simbologia católica seja essa troca de
Roma por dinheiro. A covardia de Mussolini inventou o gesto mais genial
desse agudíssimo ``moderno''. Quanto ao gesto do papa, depondo
contratualmente Jesus de pí ias realezas de homem, se choca a
\emph{sensiblerie} de muitos católicos frouxos, é de deveras uma
inspiração divina. Jesus preso por despique! preso por um muro!\ldots{} Não
havia maior rebaixamento da divindade. Não havia concepção mais
anticonceitual da divindade. E não havia símbolo mais inútil.

Morto e deposto, Jesus se libertou enfim. Agora é um deus unicamente
divino. Não é mais culpa de nada e nem desculpa. Está isento da
derradeira evidência. Os que acreditam Nele, seja por fé, seja por
conhecimentos filosóficos ou religião, estão com as orações libertas.
Ficou até inútil discutir se a oração tem de ser adoração primeiro e
pedido depois. Ato de medo, ato de coragem: a oração não possui mais
nenhuma validade terrestre. O reino de Jesus não é deste mundo. As curas
se fazem com ou sem Ele, os ganhos de dinheiro, de futebol, de amor, com
ou sem Ele. A gente não pode mais estatisticar as forças de Deus. Jesus
está morto e deposto. A união com Ele agora é como o brilho inútil das
estrelas.

Porque a oração cada vez mais adianta menos. Ou melhor: não adianta
nada. Todas as prerrogativas individuais não têm mais nenhuma validade
para a civilização que está nascendo. A própria felicidade humana é uma
exacerbação espúria do individualismo. Cacoete pessoal que não interessa
às preocupações sociais, que não adianta nem atrasa o conceito nascente
de civilização. Pro facho; pro hitlerismo; pros sovietes, a própria
grandeza moral do indivíduo é um fantasma desprovido dos seus sustos.
Inútil ao mecanismo social, onde tudo e todos estarão controlados. Quem
quiser que a tenha, é indiferente. E por isso aqueles que agora se unem
a Jesus fulgem do brilho inútil das estrelas.

\chapter{A pesca do dourado}

Quando chegamos na barranca do Moji, andadas oito léguas de cabriolante
forde, era madrugada franca. O rio fumava no inverninho delicioso.
Vento, nada. E a névoa do rio meio que arroxeava, guardando na brancura
as cores do sol futuro.

Estivemos por ali, esquentando no foguinho caipira que é o cobertor da
nossa gente. Estivemos por ali, esfregando as mãos, tomando café,
preparando as varas. Eu, como não tinha esperança mesmo de pescar nenhum
dourado, fui pescar iscas no ceveiro. Isso, era atirar o anzolzinho
desprezível n'água, vinha cada lambari enganado, cada tambiú e mesmo uma
piabinha comovente. Nove bastavam, me falaram.

E a rodada principiou. João Gabriel dizia que era preciso pescar já,
porque depois, com o dia, a água esfriava, entenda"-se! Do lado do
oriente o horizonte se cartão"-postalizava clássico; e os vultos das
``ingaieiras'', dos jatobazeiros e do timboril do rumo já se vestiam de um
verde apreciável.

Eu me esforçava por pescar direito. Olhava a altura da vara do outro
pescador, copiava com aplicação os gestos dele. Às vezes me dava uma
raiva individualista e, só por independência ou morte, batia com a isca
onde bem queria, longe dos lugares de água tumultuosa, preferidos pelos
dourados. Foi numa dessas ocasiões que atrapalhei o io de aço do anzol
na vara, e o lambari da isca, juque! me bateu no nariz. A natureza
inteira murmurou ``Bem feito!'' e me deu uma vontade de morrer. João
Gabriel, que ia de proa, olhou pra mim, não riu, não censurou, nada.
Continuou proando a canoa. Essa inexistência de manifestação exterior
destes que me rodeiam, a deferência desprezante, a nenhuma esperança
pelo moço da cidade, palavra de honra, é detestável. Castiga a gente. Oh
vós, homens que viveis no sertão, por que me tratais assim! Quero ser
como vós, vos amo e vos respeito!

Estava eu no urbano entretenimento deste pensamentear, quando a canoa
tremeu com violência. Olhei pra trás e o companheiro pescador dançava
num esforço lindo, às voltas com a vara curva. Por trás dele a aurora,
me lembro muito bem; e tive a sensação de ver um deus. Mas o dourado,
não sei o que fez, a vara descurvou. O peixe se livrara e o deus virou
meu companheiro outra vez. Fiquei com uns vinte contos de satisfação.

Mas Nêmesis não me deixou feliz. Veio um desejo tão impetuoso de um
dourado, pelo menos beliscar meu anzol, coisa dolorosa! torcia com
paixão, pedia um dourado, me lembrei de fazer uma promessa a Nossa
Senhora do Carmo, minha madrinha, me lembrei das feitiçarias de catimbó,
e principiei por dentro cantarolando a prece da Sereia do Mar. Eis que
percebi a minha linha de aço navegando por si mesma rio abaixo.

--- Puxe!

Isso, dei um arranco de três forças, fiquei gelado, meu coração ploque!
ploque!

--- Não bambeie a linha!

A linha, não era eu que bambeava não! bambeou, ergui a vara, o dourado
pulou um metro acima d'água, que Virgílio nem Camões nada! um peixe
imenso!

--- Não puxe a vara!

Não puxe a vara, não bambeie a vara, sei lá! o dourado é que dava cada
puxão, cada bambeio que queria.

--- Canse ele!

Mas como é que se cansa dourado! isso é que nenhum dos meus livros me
contara! A segunda vez que o bicho pulou fora, eu já não podia mais de
comoção. Palavra"-de"-honra; estava com medo! Tinha vontade de chorar, os
companheiros não falavam mais nada, tinham me abandonado! ôh que ser
mais desinfeliz!

Mas pesquei! Teve alguém finalmente que me ajudou a tirar o dourado da
água, cuidou dele, guardou"-o no viveiro da canoa. Eu, muito simples, pra
um lado, jogado fora, pela significação do bicho que era mesmo
importante.

Não fazia mal não\ldots{} eu mesmo estava me dando uns ares de coisa muito
fácil. Mas quem pescou o bicho fui eu! A respeito de dourado, estou
ganhando de um a zero contra Manuel Bandeira, Guilherme de Almeida,
Augusto Meyer, Alberto de Oliveira, Castro Alves e outros poetas maiores
que eu.

\chapter{Brasil"-Argentina}

Na véspera, o meu amigo uruguaio confessou que viera torcer pelos
argentinos. Arroubadamente, com excesso de boa"-educação, fui afirmando
logo que isso não fazia mal, que diabo! etc. Ficou desagradável foi
quando ele se imaginou no direito de explicar porque torcia pelos
argentinos:

--- Você compreende, amigo, nós, uruguaios, temos muito mais afinidade
com os argentinos, apesar de já termos feito parte do Brasil. Até por
isso mesmo!\ldots{} Por mais que se explique historicamente o que levou um
tempo o Uruguai a participar do Brasil, nós não sentimos (repare que
emprego o verbo ``sentir''), não sentimos a coisa como se tivéssemos
participado do Brasil, e sim como tendo pertencido a ele. A modos de
colônia\ldots{} E isso, por mais esforços que a gente faça, irrita bem.
Quanto a afinidades com os argentinos, há muitas\ldots{} muitas\ldots{}

Aqui meu amigo uruguaio parou de supetão. Percebi que não queria me
machucar. Mas nesse terreno de boa"-educação ninguém ganha de brasileiro,
não insisti. Não ousei dar uma liçãozinha de humanidade no meu hóspede,
falando na minha simpatia igual por argentinos, turcos e australianos, e
outras invencionices maliciosas. Me preocupei apenas em disfarçar a
ansiedade que me enforcava por causa do jogo.

No campo me acalmei com segurança. Estávamos em pleno domínio do
``nacioná'', com algumas bandeiras argentinas por delicadeza. Mas na
verdade, por causa daquele jogo, estávamos todos odiando os argentinos e
a Argentina ali. E dizem que futebol estreita relações, estreita nada!
Mas aqueles milhares de brasileiros, que piadas cariocas! brilhavam na
certeza da vitória. Desconfio que, em casa ou ilhados nos bondes, também
tinham sentido a mesma inquietação que eu disfarçava, mas a unanimidade
é um estupefaciente como qualquer outro. De forma que nem bem cada
brasileiro se arranjava em seu lugar, olhava em torno, tudo era
nacional! e a certeza vinha: Vamos ganhar na maciota.

E foi nessa atmosfera de vitória que principiou o famoso jogo
Brasil"-Argentina, de que certamente não tiraremos nenhuma moral. Os
nacionais escolheram o lado pior do campo, com uma ventania dos diabos
contra, varrendo tudo, calor, bola e argentino contra o nosso gol.
Principiou o jogo.

Os argentinos pegaram com os pés na bola e\ldots{} Mas positivamente não
estou aqui pra descrever jogo de futebol. Só quero é comentar.

Ora, o que é que se via desde aquele início? O que se viu, se me
permitirem a imagem, foi assim como uma raspadeira mecânica,
perfeitamente azeitada, avançando para o lado de onze beija flores.
Fiquei horrorizado. Procurei disfarçar, vendo se me lembrava a que
família da História Natural pertencem os beija flores, não consegui! Nem
sequer conseguia me lembrar de alguma citação latina que me consolasse
filosoficamente! Enquanto isso, a raspadeira elétrica ia assustando
quanto beija flor topava no caminho e juque! fazia mais um gol. Era
doloroso, rapazes.

Mas era também admirável. Quem já terá visto uma força surda, feia mas
provinda duma vontade organizada, que não hesita mais, e diante de um
trabalho começado não há transtorno político, financeiro, o diabo! que
faça parar!\ldots{} Eram assim os argentinos, naquela tarde filosófica. Não
que eles se alardeassem professores de ordem, de energia ou de
coisíssima nenhuma. Se alguém desejar saber exatamente o que eu senti,
eu senti a Grécia, a Grécia arcaica, no tempo em que se fazia a futura
grande Grécia. Dezenas de tribos diferentes se organizando, se
entrosando, recebendo mil e uma influências estranhas, mas aceitando dos
outros apenas o que era realmente assimilável e imediatamente
conformando o elemento importado em ibra nacional. Quem quiser me
compreender, compreenda, mas no fim do quarto gol eu tinha me
naturalizado argentino, e estava francamente torcendo pra que\ldots{} nós
fizéssemos pelo menos uns trinta gols. Mas logo bem brasileiramente
desanimei, lembrando que seria inútil uma lavada exemplar. Não serviria
de exemplo nem de lição a ninguém. Ao menos meu amigo uruguaio foi
generoso comigo, não teve o menor gesto de piedade. Comentava
navalhantemente:

--- Era natural que vocês perdessem\ldots{} Os brasileiros ``almejaram''
vencer, mas os argentinos ``quiseram'' vencer, e uma coisa é almejar,
outra é querer. Vocês\ldots{} é um eterno iludir"-se sem fazer o menor gesto
para ao menos se aproximar da ilusão. Sim, os argentinos escalaram o
quadro e este se preparou para o jogo; mas o que a gente percebe é que,
na verdade, há trinta anos que os argentinos vêm se preparando para o
jogo de hoje. A força verdadeira de um povo é converter cada uma das
suas iniciativas ou tendências em norma quotidiana de viver. Vocês?\ldots{}
nem isso\ldots{}

Os argentinos, desculpe lhe dizer com franqueza, mas os argentinos são
tradicionais.

Eu é que já estava longe, me refugiado na arte. Que coisa lindíssima,
que bailado mirífico um jogo de futebol! Asiaticamente, cheguei até a
desejar que os beija flores sempre continuassem assim como estavam
naquele campo, desorganizados mas brilhantíssimos, para que pudessem
eternamente se repetir, pra gozo dos meus olhos, aqueles hugoanos
contrastes. Era Minerva dando palmada num Dionísio adolescente e já
completamente embriagado.

Mas que razões admiráveis Dionísio inventava pra justificar sua
bebedice, ninguém pode imaginar! Que saltos, que corridas elásticas!
Havia umas rasteiras sutis, uns jeitos sambísticos de enganar, tantas
esperanças davam aqueles volteios rapidíssimos, uma coisa radiosa,
pânica, cheia das mais sublimes promessas! E até o fim, não parou um
segundo de prometer\ldots{} Minerva porém ia chegando com jeito, com uma
segurança infalível, baça, vulgar, sem oratória nem lirismo, e juque!
fazia gol.

\chapter{Abril}

Agora é de novo abril e voltam os dias perfeitos da cidade paulistana.
Na quarta"-feira passada ainda era março, mas de repente, com firmeza, a
tarde amaciou o calorão do dia, veio tão maravilhosamente exata de
bonita e boa, que eu percebi dentro de mim abril chegando, o grande mês
da natureza da cidade.

Não me importam comparações, me esqueço de outras terras. Abril será
também maravilhoso em Caldas\ldots{} Mas São Paulo é uma cidade ruim,
compadre. Os viajantes que desde o primeiro século andaram por
Piratininga, todos exaltam o clima paulistano, a sua salubridade. Me dão
a ideia de que passaram aqui todos por abril. Porque São Paulo é uma
cidade ruim, bem traiçoeira. Aqui moram as laringites, os resfriados e a
pneumonia. Eis que o calor grosso se enlameia de chuva e, nascendo dos
bueiros, bate de supetão uma friagem de morte, que mata mesmo muitas
vezes, é a morte encontrada nos desvãos do trabalho do dia.

Porém São Paulo possui abril. Um mês, pouco mais, de tardes sublimes, de
manhãs arrebitadamente frias, cheias de vontade de trabalhar audacioso.
E noites de meia"-estação, macias, cordatas que nem flanela. É preciso
que os paulistanos soltem do ser fechado os sugadores de prazer, o
olhar, a boca, o passo, o amoralismo, a preguiça, pra receber com
plenitude a ventura do nosso abril.

Eu não me esqueço não que a vida anda medonha sobre a terra, nem que aos
paulistas o tempo é de solidão e abatimento. Mas por que as desventuras
humanas e mesmo as dores pessoais hão"-de se contrapor à perfeição do
ser? Por que se há"-de reduzir a felicidade, que é especialmente uma
concordância do indivíduo consigo mesmo e o seu destino, a uma
contingência externa? A própria dor é uma felicidade, quando aceita
entre os bens que a vida fornece para o equilíbrio do ser e a sua
perfeição livre.

Este desejo agora de que os paulistanos saibam gozar abril não é
provocado em mim por nenhum diletantismo, nenhuma displicência desumana
que ignore a nossa humanidade. Temos que continuar devorando os
telegramas da China, reagindo por todas as formas contra a colonização
do Brasil, impondo maior justiça entre os homens de má vontade. Tudo
isso não esqueço, faz parte do nosso destino, serão elementos sempre da
nossa perfeição humana.

Porém abril chegou de lá detrás da serra, veio verde, luminoso, tão bom
como um caju do Norte. Nada impedirá que, depois de uma distribuição de
veneno e algum gesto de não"-colaboração ou sabotagem, o paulistano desça
na sua alameda e vá por aí.

Que celestial o céu está! É uma mistura de rosa, verde e azul, em que um
sol estilhaçado deixou milhares de partículas de ouro. As coisas de
baixo se escurentam, casas, árvores, com aquela gravidade agradável que
os bois têm. As feiúras urbanas se amansam nessa escureza modesta e as
próprias bulhas são como aparições evocadas. A gente se dispersa numa
pansensualidade também virtuosa, em que o ser se percebe tão idêntico,
tão refarto de pausas, complacência e gostos, que nada, compadre, ôh
nada supera nesse mundo a gostosura do nosso abril!

Vamos fugir de norteamericanos, italianos e nortistas, que são gentes
cheias de vozes e de gesticulação. Vamos cultivar com paz e muita
consciência nossas rosas, ruas, largos e as estradas vizinhas. Calmos,
vagarentos, silenciosos, um bocado trombudos mesmo, nessa espécie
tradicional de alegria, que não brilha, nem é feita pra gozo dos outros.
Vamos exercer o nosso paulistismo famoso, em sua expressão maior, abril:
as coisas estão desaparecidas, mansinhas, e o céu claro, claro, lá.

\chapter{Anjos do Senhor}

Os jornais deram um telegrama de Paris que me deixou meio aéreo\ldots{}

Diz"-que um brasileiro, Muniz, realizou num dos aeródromos de lá
experiências de um novo tipo de aeroplano estafeta. E que, apesar do mau
tempo, chuva, nuvens, uma ventania danada, as experiências tinham sido
``coroadas do mais completo êxito''. Ora qual será o futuro desse
invento novo de brasileiro: virará Zeppelin, como o aerostato de
Bartolomeu Lourenço, ou aeroplano como a libélula de Santos Dumont? Ou
dará em nada como o infeliz sonho de Augusto Severo\ldots{}

Está claro que, de mim, desejo o mais completo desenvolvimento ao já
completo êxito telegráfico de Muniz, porém o que me deixou um bocado
aéreo não foi isso não: foi esta mais ou menos curiosa especialidade de
brasileiro pela aviação. O que será que os brasileiros têm com os
ares!\ldots{} É extraordinário. O nosso papel na América tem sido viver no
ar. Desde a nossa pré"-história que os brasileiros, aliás então nem
brasis chamados, vivemos no ar. Porque, sem contar que, segundo a
tradição ameríndia, qualquer desgosto que brasileiro tenha, pronto, vai
pro céu e vira estrelinha: nós possuímos duas lendas que, segundo os
processos universitários de exegese, são deveras precursoras da aviação.
Uma delas é a da aranha que faz fio no chão e espera que passe o vento.
Vem um terral, ergue o fio no céu, e lá se vai, gostando bem, a aranha
pelos ares. A outra, mais bonita, é a que Afonso Arinos batizou, não sei
bem por que, com o nome de Tapera da Lua. A moça, percebendo"-se
descoberta pelo mano que no negrume da noite marcara a amante misteriosa
com as tintas do mato, planta uma semente do cipó matamatá que tem mesmo
forma de escada. O cipó cresce num átimo, e a traída sobe por ele,
transformada em lua, e fica pra sempre banzando no céu, se mirando na
água parada das ipueiras, pra ver se ainda não se acabaram as manchas da
tinta.

Qualquer pessoa medianamente nutrida na facilidade das explicações está
vendo logo que estas lendas são precursoras do mais pesado que o ar. O
que me enquizila em ambas é o ligamento que prende as aviadoras à terra,
num caso o fio da teia, noutro o cipó. Parece que a interpretação mais
aceitável é que em ambos os casos se trata de balão cativo. Isso prova
pelo menos que os antigos habitantes deste mato sem saída eram mais
sensatos que nós. Andavam no ar, que dúvida! porém sempre e
sensatamente, em perfeita comunhão com a terra. É verdade que depois
Santos Dumont, também brasileiro sensato, subiu aos ares com a intenção
de saber onde levava o nariz e resolver a dirigibilidade dos balões. E
de fato: mesmo quando mais pesado que o ar, levou o dito nariz onde
muito bem quis. Mas nós outros\ldots{} não sei não. Me parece que preferimos
o anedótico destino de Gusmão, que subiu sem saber onde que iria parar.
É exatamente o caso da valorização do café que beneficiou a América
Central, e de todos nós, oh, todos, brasileiros natos, gente pesada que
vive no ar e não sabe mesmo nada onde que vai parar.

É triste. Não queremos escutar os experientes cantadores nordestinos
que, muito antes de mim, impressionados com a especialidade aerostática
do brasileiro, fizeram um coco fabuloso. O solista entusiasmado exclama
assim:

---Eu vi um aeroplano Avuano!

O coro conselheiro avisa logo:

--- Divagá co'a mesa!\ldots{}

Mas o solista romântico insiste no entusiasmo, mentindo como eu quando
pesco dourado:

--- E eu fui no Jaú, Aribu!

Mas o coro sempre conselheiral:

--- Divagá co'a mesa!\ldots{}

Qual o quê: jamais que iremos devagar com a nossa mesa!\ldots{} Não temos
interesse pelo nosso destino; o que nos entusiasma é a nossa
predestinação. Dê a ciência aviatória no que der, caia o Brasil em que
mares encapelados cair, o que nos dirige é a predestinação aviatória que
faz com que nos imaginemos uns águias. Quando somos apenas uns
borboleteantes anjos do Senhor.

\chapter{Romances de aventura}

Depois do romance psicológico moderno, a gente não pode mais negar que
todas as existências de homem são romances legítimos. Já não tem
significação nenhuma isso da gente exclamar: ``A minha vida é um
romance!''\ldots{} Todas o são. E se noventa e cinco por cento dos seres
psicológicos deste mundo pensam que não têm muito que contar, não é
porque não tenham não. É por simples fenômeno de timidez.

Agora: já é bem mais raro a vida humana se parecer com os romances
chamados ``de aventura''. Acho incontestável que o homem no geral se
conduz pela fadiga. Já exaltaram demais a curiosidade humana\ldots{} Tem
muitos animais que são curiosíssimos, e uma das coisas mais graciosas
deste mundo é a curiosidade da mosca. Não quero desenvolver esta minha
última afirmativa pra não me tornar o que chamariam de ``anticientífico'',
porém, moscas, em vão sois cacetes às vezes, sois mais curiosas que
cacetes!

Voltando ao assunto: a história do homem tem sempre sido mal escrita,
vive inútil e sem eficiência normativa, porque a vaidade nos faz
escrever a história das nossas grandezas e não a manifestação evolutiva
da nossa vulgaridade. São nossas ideias, nossas descobertas, nossos
gênios, nossas guerras, nossa economia que a gente enumera, sai mosca!
imaginando que isso é a história do homem. E por isso acreditamos por
demais em nossa curiosidade, quando realmente ela é esporádica e só de
alguns.

O homem no geral se orienta muito mais pela fadiga que pela curiosidade.
E se tão rara é a vida humana que se equipare aos romances de aventuras,
isso vem principalmente da falta de força em seguir para diante. A
fadiga cessa o homem pelo meio, ele fica tipógrafo, sapateiro, médico,
fazendeiro, e numa quarta"-feira morre assim. O homem não tem curiosidade
nenhuma de viver, e as mais das vezes as aventuras chegam sem que ele as
procure. Eu estava imaginando em Jimmy da Paraíba, e foi mesmo por causa
dele que andei curioseando estas considerações, desculpem.

Jimmy era um negrinho como outro qualquer da Paraíba, se chamando
Benedito, Pedro, um nome assim. Doze ou treze anos. Feio como o Cão,
porém tipo da gente ver. Nariz não havia, ou era a cara toda, com as
ventas maravilhosamente horizontais, dum nordestinismo exemplar.

Quando o conheci, este ano, o menino já estava pedantizado, homem feito,
só respondendo ao nome de Jimmy. A aventura passara e o pernóstico até
não gostava de falar nela! Ficara do romance apenas a vaidade de se
chamar Jimmy,a mania de cantar a Madelon e desprezar as cantigas
brasileiras, diante dos seus doutores.

Jimmy era rapazinho esperto. Um moço paraibano, rico, meio estourado, se
engraçou pelo menino e o levou à Europa. Tinha um escritório de qualquer
coisa em Paris, Jimmy chasseur, recebendo gente, levando recado, numa
farpela encarnada que o coroava rei da simpatia, um sucessão. Aprendeu o
francês, decorou a Madelon, viajou a Itália, vivo que nem
galinho"-de"-campina.

Uma feita o patrão de Jimmy se viu nuns apuros de dinheiro. Um não sei
bem se rajá indiano estava tão entusiasmado com o negrinho que propôs
comprá"-lo. E Jimmy, sem saber comprado, salvava o patrão dos apuros, e
embarcava, escravo, em Londres, rumo do império das Índias.

Que libertação\ldots{} ser escravo em pleno século vinte!\ldots{} Afirmo que não
tem nenhuma imoralidade neste desejo meu. Tem mas é fadiga. Desprover"-se
de vontades, ser mandado, nirvanização\ldots{}

Bom, mas Jimmy é que não quis saber deste descanso, escreveu. A mãe dele
andou chorando desesperada pelas ruas da Paraíba. Havia um brasileiro
escravo de estimação, em Bombaim. A cidade de Bombaim fui eu que escolhi
pra contar o caso. Mas os jornais principiaram falando. O Ministério do
Exterior se mexeu. E Jimmy foi repatriado, livre, nessa ilusão de
liberdade com que nós vivemos dizendo ``hoje vou ao cinema'', ``sou
inglês'', ``me passe o pão''. Em vez: trinta anjos diabólicos peneiram
invisíveis sobre nós, mandando ir no cinema, pedir pão e ser inglês. Nós
obedecemos quase sempre\ldots{}

\chapter{Na sombra do erro}

Outro dia eu errei, e querendo falar em Caldas Barbosa, num artigo sobre
o Aleijadinho, falei em Sousa Caldas.

Erro desses produz sempre na gente uma impressão tão desagradável que
torna"-se inesquecível. A final foi bom! e por causa da impressão
péssima, acho que tomar Sousa Caldas por Caldas Barbosa não me acontece
mais. Agora ficam na espera duma impressão dessas pra desaparecer, só
mais duas confusões permanentes minhas, jamais saber entre Silva
Alvarenga e Alvarenga Peixoto qual é o Silva, e entre farinha de
mandioca e farinha de milho, qual das duas é a de milho.

Quão inexploráveis são as restrições do espírito!\ldots{} Desde minha mais
tenra infância que minha mãe me ensinava a distinguir estas farinhas,
mas até hoje um recalque inabalável as conserva sem batismo em mim. Sei
que diferem. Distingo"-as pelo olhar, gosto só duma\ldots{} Porém se me são
oferecidas pra valorizar a carne"-de"-sol ou o feijão"-de"-coco, me acanho
apontando, murmurando ``Essa!'' pianíssimo, ansioso por saber e sem poder
falar a língua das farinhas. Me consolo é recordando meu pai, homem de
vontade, mas que morreu sem conseguir jamais saber qual é o lado mais
doce da laranja. E como não passava sem ela, almoço e janta, setecentas
e trinta vezes por ano tínhamos que lhe ensinar esse imprescindível
b"-a"-bá cítrico.

Aliás tenho mesmo uma memória muito fraca, razão pela qual preciso duma
biblioteca muito grande. Minha memória repousa nas folhas impressas,
porém não me lastimo. Imaginação desarreiada galopa mais livre, e, já
viram café florado? assim são minhas surpresas. Além disso a precaução
me obrigou a esta sabedoria de jamais não discutir em bate"-bocas que o
vento leva. Quando falam uma enormidade ao pé de mim, digo ``Sei'' com
bem"-aventurança. Eu amo a minha paciência. É mais lenta que um buço, e o
fato dela não aparecer nos meus escritos não a desmente não. É que eu
sou pelo menos ``em dois'', que nem falam os italianinhos dos manos que a
noite lhes deu. Sou em dois: esse um de que quereis saber, oh exigências
do mundo, que sou eu apenas como um animal de raça que me dei de
presente para os meus passeios no Flamengo, e o outro, o cheio da
paciência, o que não tem nenhuma razão"-de"-ser terrestre, o que faz a
minha felicidade incomparável e sou eu.

Agora esta matemática de dois está me lembrando um dos incidentes mais
aborrecidos que me sucederam e a que nem posso chamar falta de
memória\ldots{} Foi uma troca de personalidades e nomes, coisa maluca duma
vez. Estava com um amigo e conversa vai conversa vem, ele dizia:

--- Beethoven, pouco depois de escrever a Nona Sinfonia\ldots{} Meio que
sorri e cortei a frase:

--- Que bobagem, Luís! A Nona Sinfonia é de Mozart.

Ele me olhou muito sarapantado e afirmou que a Nona Sinfonia era de
Beethoven. Nasceu uma rápida discussão penosa, eu levado logo ao máximo
da exasperação pela coragem do leiguinho me contradizer a mim,
profissional do assunto.

Caí em mim mas foi pra ter ódio de mim. Naquele tempo eu inda não era
sábio, isto é, não tinha paciência. Deu"-se esta confusão temível: o meu
amigo pronunciava ``Beethoven'', eu ouvia ``Beethoven'' bem certo, mas meu
espírito traduzia ``Mozart''. Então a verdade me obrigava a ensinar que
quem fizera a Nona Sinfonia fora Beethoven, eu imaginava ``Beethoven'',
mas pronunciava ``Mozart'' e escutava ``Mozart''!

Hoje ainda, quando penso nesse fato, as mais modernas explicações da
fadiga mental não me satisfazem. E lhes asseguro que o meu sofrimento
por vários dias foi medonho. Imaginei que estava ficando louco e
esperei. Mas se não me engano esta bem"-aventurança não chegou.

\chapter{Cai, cai, balão!}

Imagino que deviam fazer uma aplicação da lei de Mendel pra explicar
certas manifestações do nosso espírito misturado, há coisas incríveis\ldots{}
A gente vai indo, vai indo, bastardizando o espírito nas tradições de
todas as culturas do mundo, mesclando as tendências duma idade com as
das outras épocas do homem, eis que de supetão risca um gesto puro no
ar, fui eu? O homem, nas alturas sábias dos quarenta anos, vai e pratica
um ato de menino de grupo. Guardo, pra me embelezar a vida, uma peça de
cerâmica feita por um caipira das vizinhanças de Taquaritinga. A
decoração pintada copia descaradamente as cores de Marajó, e a forma
reproduz com semelhança de pasmar uma figurinha grega arcaica. Bem sei
que se fala nas ideias elementares que tanto podem nascer na cabeça dum
botocudo como dum maori ou de um fachista, eu sei. Nem foi minha
intenção, que bobagem! afirmar que o caipira de Taquaritinga provinha
esteticamente de Marajó e da Grécia. Mas por outro lado, a realização
espontânea duma faculdade infantil num homenzarrão meditabundo que já
enterrou a infância num cemitério repudiado mostra que o indivíduo, por
maior técnica do ser que possua, guarda pra sua riqueza a inexperiência
do aprendiz. Por mais organização que tenha, o indivíduo segue mandado
por correntes marinhas incontroláveis. A rota pode ser muitíssimo bem
norteada, se vai de Belém ao cabo Horn, que nem Carlos Gomes vai da
Noite no castelo ao Escravo. Mas nada impede e nada indica, porém, o que
a gente vai topar e vai mandar a gente, nesses incontroláveis oceanos.
Pode ser tubarão. Pode ser a princesa de Trípoli.

É por isso que agora eu já não tenho mais vergonha do que me sucedeu
outro dia e vou contar. Mas que cadeias misteriosas me puxaram dos
desígnios tão pretos do homem"-feito e me colocaram de novo como aprendiz
desses desígnios, plena infância? Tanto mais que nunca na minha vida
infantil fui pegador de balão!\ldots{} O melhor é contar logo.

Era noite avançada, quase vinte"-e"-quatro horas. São João, a festa já
estava acaba"-não"-acaba nos barulhos raros do ar. Eu vinha\ldots{} suponhamos
que tivesse errado o caminho, eu vinha de destinos do homem"-feito, forte
e designado em mim. Vinha em procura da rua dos bondes que me levasse
pra casa outra vez. Era longe, num bairro que dorme cedo. Foi exatamente
quando virei a esquina: enxerguei no céu perto a chama viva do balão.
Caía com fúria, por ali mesmo onde eu estava, em três minutos se
transformava no cisco do chão. Corri, não tinha ninguém, corri. Fui até
na esquina em frente que virei, o balão vinha cair mesmo na rua, oi lá!
como ele vem certinho, sem moça na janela pra me ver, pego o balão!
Estou falando em moça porque decerto lá nos fundos de mim, se tivesse
alguma possibilidade de moça na janela, não vê que eu corria pra pegar
balão! Lá nos fundos de mim talvez estas noções persistissem, mas o fato
é que não pensei em moça, pensei em nada, mas pego o balão.

E tanto não pensei em ninguém que agora vai suceder o espantoso. Não fui
eu só que virei esquina. Também na esquina lá da outra ponta do
quarteirão, viraram uma molecada duns cinco ou seis, já dos maiores,
pelos doze, quinze anos, com paus nas mãos. Chegamos quase juntos no
espaço em que se percebia que o balão vinha cair. Nos olhamos. Houve um
primeiro receio na molecada, e em mim a sombra da infelicidade, ia
perder o balão! Quis reagir com a autoridade de gente bem vestida, falei
respeitável:

--- Deixem que eu pego.

De"-certo a primeira noção deles foi deixar, mas eram meia dúzia, com
paus nas mãos. Reagiram manso, umas frasinhas resmungadas, depois mais
nítidas, e então já eram inimigos.

--- Já disse que quem pega sou eu!

Eu odiava, me desculpem, mas odiava. Me arrebentava, brigando com a
molecada se fosse preciso, mas quem pegava o balão havia de ser eu.
Minha vontade até ficara meia distraída, eu queria já brigar, bater,
machucar muito, se algum ficasse aleijado que bom!

--- Não atirem pedra! fiquei desesperado, iam estragar o meu balão!
Inventei quase gritando: Sou secreta! eu levo vocês pra\ldots{} Me ajeitei
melhor. O balão só roçou no fio do telefone, veio direitinho para as
minhas mãos apaixonadas que tremiam lá no alto do ar, feito flor que
come mosca. E peguei o balão. Ainda foi um caro custo apagar a mecha,
não tinha prática, sempre olhando de banda, com rancor de morte, os
moleques ali me odiando. Agora estava em mim de novo, o balão meu.
Chegaram em disparada as vergonhas, as censuras, e um passado em que
nunca fui moleque de rua, nunca jamais peguei balão. Mas os homens
depreciam tanto a humanidade que trocam qualquer honra por dinheiro. Foi
o que fiz, cruel. Sorri para os moleques, entreguei o balão a eles,
também o quê que eu ia fazer com balão!

--- Não sou secreta não, estava enganando. Pra quê que vocês não vão pra
casa dormir, é tão tarde! Olhem\ldots{} repartam isso entre vocês\ldots{} vão
tomar um café\ldots{}

A gente fala ``café'' por comodidade, mas está pensando cerveja, pinga, os
desejados prêmios da alegria. Tanto que dei cinco mil"-réis à molecada. O
que fiquei pensando, já escrevi no princípio.

O que trouxera um pouco mais de brilho àquela vidinha quando muito de
candonga fora a fundação do Partido Democrático. Não é possível perceber
as razões que levaram certos pracistas daquela terra mansa a virarem
democráticos, tudo ia tão bem! Mas agora as conversas do largo da Matriz
e do clubinho das moças dançarem no domingo chegavam a parecer
discussões; e como não era possível a cidadinha ir melhor do que ia, a
questão de perrepista ou democrático atingira as raias do idealismo.
Discutiam, não os atos locais do compadre prefeito, mas a eloquência de
Bergamini, os camarões da Light paulistana em que dona Arlinda, a mulher
do chefe democrático, levara um tombo, coisas assim. Idealismo puro,
como se vê. E o Comunismo. Ah, discutia"-se ardentemente o Comunismo,
esse perigo imediato, que àqueles pracistas se demonstrava absolutamente
horroroso, porque ninguém sabia bem o que era.

Repartidos os homens, os democráticos menos numerosos mas sempre de cima
pela mais cômoda posição de oposição, nem por isso as amizades, os
compadrismos e parentagens se embaçaram. Continuou tudo na mesma. Só que
agora os jornais da capital eram mais atentamente decorados, os homens
eram mais altivamente gentis, e havia certa emulação na toalete das
senhoras. A missa das oito valia a pena ver.

Quando arrebentou a revolução, os democráticos exultaram. Xingaram o
presidente de uma porção de culpas. Os perrepistas apreensivos
secundavam que não era tanto assim. Entre boatos e comunicados oficiais,
tudo mentira, não se sabia a quantas o Brasil andava e pela primeira
vez, fora os casos de doença grave, a angústia sufocava o peito
mansamente respirador da cidadinha. Oito dias, doze dias, não se
aguentava mais! Os chefes perrepistas se reuniram, confabularam bem
pedros"-segundos, depois saíram da Câmara e foram procurar os
democráticos que se reuniam na porta do Comércio e Indústria.

--- Boa tarde.

--- Boa tarde.

--- Boa tarde.

Alguém arriscou um ``Como vai'' mas foi logo censurado pelos olhos
correligionários.

--- Olhem, vamos fazer uma coisa: não vale a pena a gente derramar
sangue, nem estar agora diz que brigando por causa de revolução. O
melhor é fazer assim: se a revolução ganhar, nós entregamos tudo pra
vocês, tomem conta da Câmara, da coletoria, do jornal, tá tudo em dia,
só falta fechar o balanço do mês. Não se deve nada e tem vinte e dois
contos da arrecadação em caixa. Mas se o governo ganhar, continua tudo
na mesma, está feito?

--- Está feito.

E comentaram mais sossegadamente os comunicados e sucessos do dia. Os
sucessos do dia estavam sintetizados num sucesso guaçu. Os perrepistas
tinham sido obrigados a organizar um batalhão que fosse defender o
governo. Entre desocupados dos sítios e das vendas, por causa das
promessas de dinheiro e principalmente por causa da janta excelente que
veio da casa da prefeita ajudada pelas amigas de partido, o certo é que
uns setenta rapazes se exercitavam ao mando de um sargentinho, lá no
largo da Cadeia, mais longe pra não fazer muito barulho. Mas cortava o
coração de todos, o Juca! o Amadeuzinho! até o Treque"-treque golquipa
que nunca deixara a cidadinha perder\ldots{} Cortava o coração. Ora a partida
do batalhão fora determinada pra esse dia. O certo é que foi chegando
perrepista, foi chegando perrepista na estação, até o prefeito, que não
falava nem por nada, preparara um discurso. Mas soldado mesmo! Dos
setenta apareceram vinte. Vinte tristes, assustados. O sargento xingou
todos de negros, de covardes, e aquilo até ficara doendo no coração dos
chefes perrepistas. Desaforo! xingar nossa gente de negros! são da roça,
não entendem!\ldots{}

Era hora dos jornais, o trem já apitara na curva, mas as verdades correm
mais depressa que os jornais. Nem o trem ainda pousara na estação, seu
Marcondes parou o forde na porta do Comércio e Indústria e contou
brilhando. Os vinte mártires tinham desertado na estação seguinte, e o
sargento fora obrigado a voltar sozinhíssimo com armas e bagagens. Isso
foi uma gargalhada geral de satisfação. Peste! negro era ele! bem"-feito!
E foram todos pra casa jantar, ler jornais. Depois seria o cavaco, feito
de largos silêncios, na sublime tardinha da nossa terra.

O resto já se imagina. Viveram mais uns dias de não saber nada, o
Palácio da Liberdade não parava de ser bombardeado em Belo Horizonte,
Cruzeiro não acabava mais de cair, etc. A final chegou a notícia
nocaute, essa verdadeiríssima: ele fora pro forte de Copacabana. Os
democráticos já estavam com vontade de tomar conta de tudo, mas os
perrepistas aconselharam mais calma, vamos esperar confirmação. Veio a
confirmação. Então os perrepistas entregaram a Câmara, a cadeia, o
jornal, tudo. E foram conversar na porta do Comércio e Indústria, à
espera dos jornais. Só que agora estavam de cima, já bem menos
numerosos, porém, mais vivazes e argumentadores, por gozarem das
regalias da posição de oposição.

\chapter{Sociologia do botão}

A sociologia está milagrosamente alargando os seus campos de
investigação. Hoje pesquisa"-se sobre qualquer elemento da vida, com
resultados inéditos da mais grave importância. Estamos todos, para maior
felicidade, unanimemente convencidos que uma análise dos nomes das casas
que vendem colchões pode fornecer a razão do excesso de divórcios; e se
uns destroem a verdade poenta dos alfarrábios ciscando anúncios de
jornais, outros constroem doutrinas inteiras sobre a urbanização da
humanidade, estudando a rapidez do voo dos mosquitos. Ora foi meditando
sobre isso com os meus botões, que estes me comunicaram a teoria ilustre
que venho vos expor.

Porque, senhores, estou agora sendo vítima dos meus botões. Arrancado,
sem nenhuma alegria, do meu lar paulistano, eu vivo agora a vida aberta
de um arranha"-céu carioca. Pois nem bem se passaram três meses deste
processo de viver, me vi forçado a encarar pela primeira vez na
existência, o problema do botão. E principiou se valorizando em mim
aquele sublime silêncio com que minha Mãe repassava semanalmente as
minhas roupas vindas da lavadeira, reforçando botões bambos e pregando
novos no lugar dos que tinham me abandonado. Agora não. Minha Mãe ficou
lá no seu lar de província, eu bracejo na descarinhosa luta da cidade
grande, com trinta e seis botões bambeados. E pouco a pouco,
insensivelmente, já vou me acostumando com esta nova insegurança e com a
ameaça imodesta de uma repentina nudez.

Alguns retrucarão que o meu caso é particular, pois sou solteiro. Meu
caso é o da maioria, pois nem são as esposas modernas mais hábeis que
nós, barbados, no ofício de pregar botões (sei de muitas que se recusam
altivamente a fazê"-lo), como nem são pouco numerosos os totalmente
solteiros. De resto, aproveito este ensejo tão íntimo para vos
apresentar minha criada Maria, fluminense benedita e dedicada que se
sujeita a ser mulher de botão pra mim. Solicitude não lhe falta: lhe
falta é ter vindo ao mundo naqueles tempos de dantes, em que minha Mãe
aprendeu a pregar botões tão garantidores como um fio de barba de meu
avô.

O homem, de uns tempos pra cá, não usa mais dos alastramentos abusivos
da metáfora, quando pensa, fala ou briga com os botões. O que foi
metáfora um dia, hoje é realidade amarga. O botão contemporâneo apenas
contemporiza. E por isso vê"-se o homem condicionado a um botão
aproximativo que está provocando vastas mudanças sociais. A mim me
parece mesmo que está se criando toda uma vida e mentalidade
desabotoada, a que bem se poderia chamar de Civilização do Botão.
Vejamos:

\textsc{primo}:

O homem, ao abotoar ou desabotoar um botão, já não o faz pensando em
álgebra, indissolubilidade do matrimônio ou próximas eleições, pensa
botão. Ora, como sabeis e ficou assentado pela psicologia, o botão é
imagem sexual. Esta imagem persegue o homem dia inteirinho, pois ele
está preocupado com o déficit de casas abotoadas, botões no fracasso e a
possibilidade de encontrar pelo caminho alguma mulher que seja
verdadeiramente de botão. Como negar portanto que o exaspero sexual da
atualidade se origina do botão! Avanço mais: o freudismo é uma
consequência direta do botão. É uma doutrina de ensejo visivelmente
pragmatista, só imaginada porque estamos em plena civilização do botão.

E já concordareis que este é um assunto da maior gravidade. A
civilização do botão, muito mais que o cimento armado, é que levou"-nos à
vida endogâmica do arranha"-céu. Até que ponto o apartamento prova uma
atitude, não apenas técnica, mas desabotoada da sociedade?\ldots{} Em
verdade, vos digo: o apartamento é o filho adotivo da civilização do
botão. Filho postiço, como diria Machado de Assis.

\textsc{secundo}:

O botão mal pregado, que salta irreverentemente quando o milionário
estufa a peitaria gritando ``Viva a Humanidade!'', enfim, o botão
inconfiável, o botão"-acaso, acaba acostumando a gente a viver na
insegurança e no relativo. A Relatividade é outra doutrina conformista
só admissível dentro da civilização do botão. Pois que os botões já não
são mais integérrimos botões, nem as casas femininamente casas, nem o
que está abotoado o está senão relativamente, brota e se fixa em nós a
complacência espiritual com o aproximativo. A falta de\ldots{} de nitidez
indumentária tem como consequência natural a gente se acostumar com a
falta de nitidez intelectual, moral, social. Se aceitamos as roupas mais
ou menos lotericamente abotoadas, somos como consequência levados a
aceitar doutrinas e ideologias também inseguramente arrematadas, sem a
garantia de verdade desta que vos exponho. enfim, se as arianizações
tempestivas, os fachismos ilusórios e ofuscante grandeza nacional dos
totalitarismos, arregimentam facilmente as multidões rubicundas, é
porque estas já foram arregimentadas pela psicologia de um botão
improvável. As ideologias como os botões só têm valor ocasional. O
importante é que o botão tergiverse.

\textsc{tércio}:

O meu amigo uruguaio me faz notar, porém, que mesmo no meio de tais e
tamanhos desabotoamentos sociais, sempre houve uma reação digna da parte
do homem. Infelizmente esta mesma reação veio manchada pelo cinismo
aventureiro da civilização do botão. Já percebestes, por certo, que
estou me referindo ao fecho éclair ou zip, que não sei como se escreve.
Mas que contraditória coisa o fecho éclair! Duas carreiras de dentinhos
eis que se unem e desunem a um risco de mão, zip! Como se fecharam ou
abriram? mistério. E a consciência de cinismo oportunista se acentua.
Será certo que o substitutivo zip representa um derradeiro esforço do
ser social, na preservação de sua integridade! Mas de nulo ou
contraproducente valor normativo, pela rapidez sem esforço que exige, e
sempre desleal e conformistamente aproximativo, o fecho zip representa
bem uma sociedade de panos quentes, misterioso e mítico, sem aquele
severo, realista e irretorquível sentido abotoante do botão de minha
Mãe. Só as mães, não mãe"-de"-apartamento, só as \textsc{mães} sabem pregar botões!
E tenho dito.

\chapter{O dom da voz}

Estava hoje à procura de um assunto quando tive a felicidade de
encontrar um homem que admiro muito. Professor dos mais dedicados, só
uma vez duvidei desse justo: foi quando ele me comunicou que abrira um
curso de oratória. Não cheguei propriamente a me espantar, tive a
certeza imediata de que o meu amigo estava irreparavelmente louco.
Depois, muito aos poucos, pude me convencer de que não se tratava de
loucura não, mas de um possível ideal, parece incrível.

É sabido que os gregos se entregavam muito à oratória e os selvagens
também. À noitinha, acabados os descansos do dia, os brasis se ajuntavam
em torno do fogo e falavam, falavam, falavam. Os gregos também falavam,
falavam, falavam. E os brasileiros também. E agora, com essa guerra,
também os enormes guerreiros estão falando que é um incontestável
despropósito. Se ao menos as guerras se limitassem a bate"-bocas de
chefes valentíssimos, ah! como eu havia de abençoar o dom da voz! Em
vez, segundo a lição europeia das guerras, está mais que provado que
discursos não dão pra ganhar batalha nem fazem valer o direito da gente.
De forma que foi necessário organizar uma entrosagem muito conivente de
armas palpáveis e armas impalpáveis. Primeiro os grandes chefes deitam
muito discurso e conseguem convencer do uso da guerra os que já estavam
convencidos disso. Imediatamente em seguida chega o instante menos
imaginoso do exercício das armas palpáveis, canhões, escondimento
apressado das crianças que participarão da guerra próxima, cidades
bombardeadas. Mas eis que nasce o medo, hoje intitulado guerra de
nervos, porque o homem, como todos os seus irmãozinhos do mato, dos ares
e das águas, é fundamentalmente medroso. E é então que volta
salvadoramente esse dom da voz que, de acordo com os prospectos dos
cursos de oratória, ``permite convencer os outros e nos aumenta a
confiança em nós mesmos''. E eis cada qual convencido e confiantíssimo.
``Morres de fraco? Morre de atrevido!'' O medo desaparece.

Ora valha"-me Deus! Está claro que, pelo simples fato de escrever estas
linhas amargas, não me convenço de ter arrancado do poroso saco das
ideias um argumento a mais contra a oratória. Já porém essa
justificativa de que saber falar em público nos aumenta a confiança em
nós, coisa provada, é que muito me melancoliza. Os conselhos da
desconfiança e timidez me parecem bem mais fecundos e intelectuais. Ah,
essa marca da minha viagem amazônica\ldots{}

Íamos um pequeno grupo de paulistas, dirigidos por D. Olívia Guedes
Penteado, que por lá, os jornais e os pobres intitularam de ``rainha do
café''. Mas não só do café ela era rainha, o que nos proporcionou várias
recepções oficiais e numerosos discursos. Eu, que era o homem do grupo,
tivera até esse dia a timidez intelectual de jamais falar em público,
jamais improvisar. Já algumas vezes lera em público, manifestação
honrada e pertencente ao domínio da inteligência, e que nada tem a ver
com falar. Eis que de repente, logo num primeiro almoço íntimo, em
Belém, quando chegou a hora prima pós"-meridiana, que era a da sobremesa,
um orador se levantou e veio pra cima de mim com um discurso de
saudação. Digo que o discurso veio pra cima de mim não porque fosse a
mim dirigido, era, com toda a justiça, endereçado à rainha de nós todos.
Mas logo percebi que a mim, homem do grupo e tido às vezes por poeta,
caberia responder. Ainda a timidez me obrigou a hesitar em meu bom"-senso
açaimado, mas D. Olívia me fez um graciosíssimo pedido com o olhar. E
penetrei na onda convulsa da pororoca.

O papelão que fiz não se descreve, embora, nos momentos de lucidez, eu
conserve um inconfessável orgulho do meu fracasso. A única coisa de que
me lembro é que, súbito, depois de uns quatro ou cinco minutos de
palavras que eu falava, me nasceu uma ideia! Ideia não muito rara eu
sei, mas enfim sempre era uma ideia. Significava mais ou menos que, no
meio das coisas tão bonitas e novas que víamos, jamais inda nos
lembráramos do Sul, porque os homens eram os mesmos, parafraseando o
grande poeta: ``Tendo um só coração, tendo um só rosto.'' Ainda
encompridei a ideia, acrescentando qualquer coisa sobre o sentimento
perfeito que tínhamos da ``inexistência dos limites estaduais'' (não sou
centralista, sou municipalista, mas não fazia mal me trair em palavras).
E acabei o discurso. E quando me namoraram os ouvidos uns aplausos
delicados, palavra de honra que o meu único desejo era levantar outra
vez e fazer mais discurso! coisa fácil as palavras!\ldots{}

Não faltou ocasião. Em quase todas as cidadinhas do rio imenso, tive que
parafrasear o ``grande poeta'' e reconhecer mais numerosas vezes a
``inexistência dos limites estaduais''. E quando chegamos a Iquitos,
capital do departamiento de floreto, no Peru, a coisa ficou
prodigiosamente fácil. Tornei a parafrasear o ``grande poeta'' (não havia
meios de me lembrar do nome dele!) e troquei os ``limites estaduais'' por
``nacionais'', com ardente sentimento de americanismo.

Quem me trouxe à razão foi o ``grande poeta''. Não houve meios, durante a
viagem toda, de lembrar o nome dele. O nome sagrado de Machado de Assis,
que nunca fez discursos de improviso, se recalcara no subconsciente,
como terrível censura à minha confiança em mim. Só quando estava já no
mar oceano, de volta, sem probabilidades de botar discurso mais, o nome
do poeta me tornou à lembrança. Caí em mim. Nesse mesmo dia, recebi um
rádio de Luís da Câmara Cascudo, amigo íntimo, que ainda nos preparava
uma recepção oficial, em Natal. Dizia: ``Quer almoço presidente discurso
ou sem? Abraços.'' Respondi: ``Sem. Abraços.''

\chapter{Memória e assombração}

Outro dia maltratei bastante o valor da linguagem como instrumento
expressivo da vida sensível. Agora conto um caso que exprime bem a força
dominadora das palavras sobre a sensibilidade. Quem reflita um bocado
sobre uma palavra há de perceber que mistério poderoso se entocaia nas
sílabas dela. Tive um amigo que às vezes, até na rua, parava, nem podia
respirar mais, imaginando, suponhamos, na palavra ``batata''. ``Ba'' que
ele, ``ta'' repetia, ``ta'' assombrado. Gostosissimamente assombrado. De
fato, a palavra pensada assim não quer dizer nada, não dá imagem. Mas
vive por si, as sílabas são entidades grandiosas, impregnadas do
mistério do mundo. A sensação é formidável. Porém o caso que eu quero
contar não é esse não, e se passou com a minha timidez.

Entre as pessoas que mais estimo está Prudente de Morais, neto, o
escritor que tanto fez com a Estética, pra dar uma ordem mais serena ao
movimento das nossas letras modernas. Há muitos Prudentes nessa família
e nós tratávamos o nosso por Prudentinho.

Uma feita ele veio a São Paulo e fui visitá"-lo. Cheguei no portão duma
casa nobre, alta como a tarde desse dia. Uma senhora linda tornava
tradicional um jardim plantado entre duas moças. Meu braço aludiu à
campainha com delicadeza e uma das moças perguntou o que eu queria.
``Falar com o Prudentinho'' secundei. A moça me contou que o Prudentinho
estava no Rio.

--- A senhora me desculpe, mas hoje mesmo ele telefonou pra mim. Ela
sorriu:

--- Ah, então é o Prudentão.

Fiquei numa angústia que só vendo, senti corpos de gigantes no ar.
Jamais um aumentativo não me fez perceber com tamanha exatidão a
malvadez humana. Decerto a moça teve dó porque esclareceu:

--- Naturalmente é o Prudentão, filho do dr. Prudente de Morais\ldots{}

--- Deve ser, minha senhora!\ldots{} arranquei da minha incompetência.

Então a moça foi boa pra mim e respondeu que o Prudentão não estava.
Fugi com tanta afobação da casa do gigante, uma casa mui alta, fugi com
toda a afobação.

Estava muito impressionado e passei uma noite injusta. Não é que
sentisse medo nem sentira ---positivamente eu já não posso mais ter medo
de gigante.

Porém tivera a sensação do gigante, e ele produzia em mim efeitos de
estupefaciente. Eu enxergava um despotismo de Prudentes sobre um estrado
comprido, procurava, procurava e não achava o meu. Quando cheguei lá no
fim do estrado, enxerguei novo estrado cheio de novos Prudentes\ldots{} Eram
decerto encontradiços de rua, alguns rostos pude identificar por estarem
nas memórias desse dia. Dum fulano parado na esquina me lembrava bem.

Veio um momento em que não pude sofrer mais e reagi. Murmurei com
autoridade: Prudentico! Essas confianças que se toma com os companheiros
são bem consoladoras\ldots{} Nos inundam dessa intimidade que a presença de
nós mesmos. Dormi.

É sempre assim. As memórias que a gente guarda da vida vão se
enfraquecendo mais e mais. Pra dar a elas ilusoriamente a força da
realidade, nós as transpomos para o mundo das assombrações por meio do
exagero. Exageros malévolos, benéficos. E um dos elementos mais
profícuos de criar esse exagero é a palavra. Poesias, descrições, ritos
orais\ldots{}

É um engano isso de afirmarem que a gente pode reviver, tornar a sentir
as sensações e os sentimentos passados. As memórias são fragílimas,
degradantes e sintéticas, pra que possam nos dar a realidade que passou
tão complexa e intraduzível. Na verdade o que a gente faz é povoar a
memória de assombrações exageradas. Estes sonhos de acordado,
poderosamente revestidos de palavras, se projetam da memória para os
sentidos, e dos sentidos para o exterior, mentindo cada vez mais. São as
assombrações. Estas assombrações, por completo diferentes de tudo quanto
passou, a gente chama de ``passado''\ldots{}

\chapter{Meu engraxate}

É por causa do meu engraxate que ando agora em plena desolação. Meu
engraxate me deixou.

Passei duas vezes pela porta onde ele trabalhava e nada. Então me
inquietei, não sei que doenças mortíferas, que mudança pra outras portas
se pensaram em mim, resolvi perguntar ao menino que trabalhava na outra
cadeira. O menino é um retalho de hungarês, cara de infeliz, não dá
simpatia nenhuma. E tímido o que torna instintivamente a gente muito
combinado com o universo no propósito de desgraçar esses desgraçados de
nascença. ``Está vendendo bilhete de loteria'', respondeu antipático, me
deixando numa perplexidade penosíssima: pronto! estava sem engraxate! Os
olhos do menino chispeavam ávidos, porque sou dos que ficam fregueses e
dão gorjeta. Levei seguramente um minuto pra definir que tinha de
continuar engraxando sapatos toda a vida minha e ali estava um menino
que, a gente ensinando, podia ficar engraxate bom. É incrível como essas
coisas são dolorosas. Sentei na cadeira, com uma desconfiança infeliz,
entregue apenas à ``fatalidade inexorável do destino''.

Pode parecer que estou brincando, estou brincando não. Há os que fazem
engraxar os sapatos no lugar onde estão, quando pensam nisso. Há os como
eu, que chegam a tomar um bonde comprido, vão até a rua Fulana, só pra
que os seus sapatos sejam engraxados pelo ``seu'' engraxate. Há indivíduos
cujo ser como que é completo por si mesmo, seres que se satisfazem de si
mesmos. Engraxam sapato hoje num, amanhã noutro engraxate; compram
chapéu numa chapelaria e três meses depois já compram noutra; conversam
com a máxima comodidade com os empregados duma e doutra casa e com todos
os engraxates desse mundo. Indivíduos assim me dão uma impressão
ostensiva de independência feliz, porém não os invejo.

De primeiro, faz talvez vinte anos, meu engraxate foi trabalhar com o
meu freguês barbeiro. Era cômodo, ficava tudo perto da minha casa de
então. Meu barbeiro, serzinho de uma amabilidade tão loquaz que acabou
me convencendo da perfeição da gilete, logo me falou que aquele
engraxate falava o alemão. Perguntei por passatempo e o italiano fizera
a guerra, preso logo pelos austríacos. Era baixote, atarracado, bigode
de arame e uma calvície fraternal. Se estabeleceu uma corrente de forte
interdependência entre nós dois, isso o homenzinho trabalhou que foi uma
maravilha e meus sapatos vieram de Golconda. Nunca mais nos largamos.
Entre nós só se trocaram palavras tão essenciais que nem o nome dele
sei, Giovanni? Carlo? não sei. Um dia ele me contou baixinho, rápido,
que mudava de porta. Foi o que me deu a primeira noção nítida de que o
meu barbeiro era mesmo duma amabilidade insustentável. Mudei com o meu
engraxate e, pra não ferir o barbeiro, que a final das contas era um
homem querendo ser bom, me atirei nos braços da gilete a que até agora
sou fiel.

Veio o dia em que a engraxadela aumentou de preço. Só soube muito mais
tarde, por acaso, meu engraxate não me contou nada, preferindo ficar sem
gorjeta, não é lindo! Nos fins de ano, jamais pediu festas, eu dava
porque queria. Hoje, tanto as festas como as pequenas gorjetas me
produzem um sentimento de mesquinhez, não sei por que dificuldades meu
engraxate terá passado, quanto lutou consigo e com a mulher. A final não
aguentou mais esta crise, vamos ver se vender bilhete rende mais!

O menino, até me deu raiva de tanto que demorou. (Meu engraxate também
demorava demais quando era eu, mas não dava raiva.) O menino, pra falar
verdade, engraxou tão bem como o meu engraxate e meus sapatos
continuaram vindo de Golconda. Não sei\ldots{} não voltei mais lá. Faz semana
que não engraxo meus sapatos. Sei que isso não pode durar muito e o mais
decente é ficar mesmo freguês do menino, porém minha única e verdadeira
resolução decidida é que vou comprar bilhetes de loteria. Não tenho
intenção nenhuma de tirar a sorte grande mas\ldots{} mas que mal"-estar!\ldots{}

\chapter{Biblioteconomia}

O contato com os livros e manuscritos dessas idades que irreverentemente
costumamos chamar de ``passado'', será que nos deixa o ser mais antigo?\ldots{}
Parece. Positivamente não é a mesma coisa a gente ler Matias Aires numa
edição primeira ou numa reimpressão contemporânea. A transposição
moderna conterá sempre a mesma substância, e mesmo nas raríssimas
edições honestas, a substância estará enriquecida de comentários,
correções, esclarecimentos. Mas o importante é que não são apenas os
dados da verdade que um livro pode nos fornecer. Quem julgar assim sabe
ler pelo meio.

O livro não é apenas uma dádiva à compreensão, é, deve ser
principalmente um fenômeno de cultura. Quem lê indiferentemente um
escrito numa edição do tempo ou noutra moderna, numa edição mal impressa
ou noutra tipograficamente perfeita, num bom como num mau papel, esse é
um egoísta, cortado em meio em sua humanidade. Lê porque sabe ler, e
apenas. O livro lido apenas para se saber o teor do escrito é sempre
singularmente subversivo da humanidade que trazemos em nós. O fenômeno
mais característico desse individualismo errado, a gente encontra nos
estudantes que, na infinita maioria, são pervertidos pelos seus livros
de estudo. Não que todos os livros escolares sejam ruins, os rapazes é
que ainda não aprenderam a ler. Leem pra saber a verdade que está nos
livros, e apenas. O resultado são essas almas imperialistas, tão
frequentes nos ginásios, vivendo em decretos desamorosos, incapazes de
distinguir, comendo, dormindo, respirando afirmações. O estudante
pernóstico, corrigindo os erros do pai!

Nas civilizações contemporâneas mais energicamente respeitosas do homem,
as universidades, os livreiros se esforçam por apresentar o livro, não
apenas como um repositório de verdades, mas como um fenômeno duma
totalidade muito mais fecunda que isso. Pela boniteza da impressão, pela
generosidade do papel, pelo conselho encantador das gravuras, os bons
livros modernos não querem nos obrigar apenas a saber a vida, mas a
gostar dela porém.

Ora já de muito, bem que venho matutando em que talvez a verdade menos
deva ser um objeto de conhecimento que de contemplação\ldots{} Não será essa
diferença fundamental que separa o encanto maravilhoso de Platão da
secura sem beijo de Aristóteles, no entanto bem mais verdadeiro?\ldots{} Não
será esse engano das nossas civilizações que as torna tão rasteiras,
monetárias, dogmáticas, em oposição às grandes civilizações da Ásia, bem
mais gostosas e sutis?\ldots{}

E cheguei com certo esforço adonde pressentia que desejava chegar: o
livro antigo, o manuscrito original, pela sua venerabilidade, pelo
esforço de acomodação à leitura, pela exigência permanente de controle
do que diz, não nos deixa nunca apenas na psicologia individualista de
quem aprende, mas no êxtase amplíssimo, difuso, contagioso da
contemplação. Ele nos reverte à nossa antiguidade.

Deixem que eu diga, mas nas civilizações novatas que nem as desta
América, os seres são profundamente imorais, no sentido em que a moral é
uma exigência derivada aos poucos do ser tanto indivíduo como social.
Não nos custa a nós, americanos, aceitar religiões, filosofias, e mesmo
importar civilizações aparentemente completas. O nosso dicionário vai de
A pra Z, direitinhamente. Tem F tem L e tem R: Fé, Lei, Rei. O que não
nos é possível importar é a precedência orgânica dessa Fé, dessa Lei e
desse Rei, nascidos de outras experiências. Nós existimos pouco,
demasiado pouco. Nós existimos em desordem. É que nos falta antiguidade,
nos falta tradição inconsciente, nos falta essa experiência por assim
dizer fisiológica da nossa moralidade que, só por si, torna a palavra
``passado'' duma incompetência larvar.

Isso nem o ótimo livro moderno conseguirá nos fornecer. O livro antigo é
moral, com a sutil prevalência de não ser uma moral ensinada (que é
sempre pelo menos duvidosa) mas uma moral vivida. É um banho
inconsciente de antiguidade. E se na mão do bibliófilo o livro antigo é
duma volúpia incomparável, estou que devemos arrancá"-lo dessas mãos
pecaminosas e botá"-lo nas mãos rápidas dos moços. Convém tornar os moços
mais lentos, e iniciar no Brasil o combate às velocidades do espírito.
Que abundância de meninos"-prodígios transfere a vida agora da beca
difícil dos clérigos pro quepe chamariz dos generais\ldots{} Vivo meio
sufocando.

Eu desconfio que ninguém achará razão nestas palavras, quando o que me
intitula é a Biblioteconomia. Mas pra mim foram os pensamentos
sossegados que pensei e quis dizer. Para mim, que envelheço rápido, o
pensamento como a vista já vão preciosamente perdendo aquele dom de
precisão categórica, que define as ideias como as coisas nos seus
limites curtos. De fato a biblioteconomia é, dentre as artes aplicadas,
uma das mais afirmativas. Diante desse mundo misteriosíssimo que é o
livro, a biblioteconomia parece desamar a contemplação, pois categoriza
e ficha. É engano quase de analfabeto imaginar tal desamor; e não foi
senão por um velho hábito biblioteconômico que, faz pouco, me fichei na
categoria dos envelhecidos, o que posso jurar ser pelo menos uma
precipitação.

Isso é a grandeza admirável da biblioteconomia! Ela torna perfeitamente
acháveis os livros como os seres, e alimpa a escolha dos estudiosos de
toda suja confusão. Este o seu mérito grave e primeiro. Fichando o
livro, isto é, escolhendo em seu mistério confuso uma verdade, pouco
importa qual, que o define, a biblioteconomia torna a verdade
utilizável, quero dizer: não o objeto definitivo do conhecimento, pois
que houve arbitrariedade, mas um valor humano, fecundo e caridoso de
contemplação. E pelo próprio hábito de fichar, de examinar o livro em
todos os seus aspetos e desdobrá"-lo em todas as suas ofertas, a
biblioteconomia rallenta os seres e acode aos perigos do tempo, tornando
para nós completo o livro, derrubando os quepes e escovando as becas.

\chapter{Voto secreto}

Não sei, todos ficaram entusiasmados porque as eleições em São Paulo
correram na ``maior ordem''\ldots{} Talvez haja o que distinguir. É
incontestável que todos e a própria tendência para a disposição boa das
coisas, que é tão da gente paulista, fizeram com que as eleições
terminassem dentro do remanso dos nossos monótonos entusiasmos. Mas
porém tenho a impressão bastante melancólica de que, se houve a ``maior
ordem'' prática, nem por isso deixou de ser devastadora a desordem
mental.

Voto secreto. É possível que um dia a gente venha a usar, com direito de
propriedade, desse presente subitâneo que nos deram. O presente ficou
bonito lá fora, pra podermos falar que também no Brasil se emprega o
tal. Mas aqui dentro, por enquanto, ele fez foi despertar aquele
farrancho temível de mitos. Uma completa mitomania vai grassando por aí
de vária forma. Mitomania não somente no sentido em que as paixões
partidárias deformaram, falsificaram, esconderam. Até neste sentido é
que a nossa mitomania atual é mais perdoável. Eu me convenci de que a
imensa maioria estão absolutamente convencidos de que estão com a razão.
Se perrepistas como peceístas deformaram, esconderam, falsificaram
verdades, não fizeram nada disso intencionalmente. Pelo menos, em geral.
Acreditavam no que diziam ou faziam, ardentemente movidos pelo desejo de
salvar (!) o Estado. Infelizmente a salvação se resumiu em criar mitos.
Culpa do voto secreto.

Não duvido que a ideologia democrática tenha tido o seu valor, mas hoje,
diante das exigências do tempo, nem se sabe mais o que é. É um mito,
duma largueza aquosa, tão adaptável ao \textsc{pc} como ao \textsc{prp}. O resultado disso
é que o voto secreto não conseguiu que adiantássemos um passo sobre
1930. Não se discutiu ideologias, ninguém se dedicou por sistemas, porém
por indivíduos. Novos mitos também, estes indivíduos\ldots{} Não discuto o
valor de ninguém aqui, mas os chefes de partidos cujas ideias já não
conseguem mais se agrupar em sistemas definidos se viram guindados a
deuses do Bem e do Mal. O chefe que pra uns era o mito do Bem, pra
outros era mito do Mal. Daí não haver ideia possível. Dedicações de
fanáticos, ódios erruptivos de apóstatas. Digo de apóstatas, porque
neste caso incolor de \textsc{prp} e \textsc{pc}, constituiu verdadeira apostasia deixar
um partido por outro, quando ambos tinham a mesma religião!

Mas o voto secreto provocou nova mitomania mais curiosamente particular.
Observei isso em centenas de pessoas com quem conversei. Foram talvez
milhares os eleitores que se elevaram a mitos de si mesmos. Um dos
processos desse individualismo empafioso consistiu (principalmente entre
pessoas de certa idade e cultura de maior pretensão) consistiu em votar
fora das chapas partidárias. O eleitor se julgando honestíssimo e dono
de sua consciência (talvez outro mito\ldots{}) batucava na
máquina"-de"-escrever uma chapa mesclada, com os mitozinhos da sua
simpatia, uns tantos do \textsc{pc} e outros tantos do \textsc{prp}. As mais das vezes,
nem se dava ao trabalho de escrutar se não existiria por acaso no Estado
alguém, fora dos partidos, com mais possibilidades de formar um
pai"-da"-pátria. As chapas estavam ali, escolhendo pela gente!\ldots{}
Cozinhavam dentro das chapas. Pouco se lhes dava atrapalhar, pouco se
lhes dava a contradição do sistema, pois votavam em indivíduos de
partido que seguirão seus partidos: importante era ele, eleitor,
provável proprietário do seu voto. Voto em quem Eu quero. Imaginava
estar usando do seu voto, quando estava apenas abusando do seu Eu.

Mas entre os moços e na burguesia mais pobre, observei outro abuso,
muito curioso também. O voto secreto causou uma legítima bebedeira de
liberdade que constatei em muitos. De posse duma arma que poderiam usar
pra castigo de chefes maus, estes eleitores se esqueceram completamente
de julgar por si mesmos se os chefes eram de fato maus. Votaram contra.
Votaram contra, não porque estivessem conscientemente contra, mas só
para provar que podiam votar contra, t'aí!

E nesses brinquedos de primeira vez se desperdiçou a infinita maioria do
eleitorado. Cada qual votou num mito: uns votaram nos seus chefes
ideologicamente inexistentes, outros votaram em si mesmos, também
ideologicamente inexistentes. Jogados pra um canto, em minoria esmagada,
outros eleitores ainda havia, a meu ver os únicos dignos de maior
atenção. Os esquerdistas vermelhos e os fachistas. Cegos de ódio e
antagonismo. Se diria presos a mitos também\ldots{} Não eram mitos não: estão
presos, não tem dúvida, mas conscientemente presos a ideologias
perfeitamente delimitadas e fixas. Uns cantando antecipadamente a sua
vitória, verdes, mas dum verde que não me dá muita esperança não.
Outros, por uma aberração hedionda e, meu Deus! natural dos
princípios\ldots{} liberais, não tendo direito de viver à luz do sol,
perseguidos como leprosos. Ou como os primeiros cristãos\ldots{} E não se
pense que pretendo ficar assim, de camarote, assuntando o desfilar das
tragédias humanas. Todos os sintomas do meu ser me lembram que sou filho
de tipógrafo, alma de tipógrafo. Só que nesta desordem mental que o voto
secreto açulou, vermelhos como verdes têm de mim o mesmo respeito. Foram
os únicos que tiveram consciência do mundo e confiança nas ideias.

\chapter{A sra. Stevens}

--- Mme. Stevens.

--- Sim, senhora, faz favor de sentar.

--- Fala francês?

--- \ldots{}ajudo sim a desnacionalização de Montaigne.

--- Muito bem. (Ela nem sorriu por delicadeza.) O sr. pode dispor de
alguns momentos?

--- Quantos a sra. quiser. (Era feia.)

--- O meu nome é inglês, mas sou búlgara de família e nasci na
Austrália. Isto é: não nasci propriamente na Austrália, mas em águas
australianas, quando meu pai, que era engenheiro, foi pra lá.

--- Mas\ldots{}

--- Eu sei. É que gosto de esclarecer logo toda a minha identidade, o
sr. pode examinar os meus papéis. (Fez menção de tirar uma papelada da
bolsa"-arranha"-céu.)

--- Oh, minha senhora, já estou convencido!

--- Estão perfeitamente em ordem.

--- Tenho a certeza, minha senhora!

--- Eu sei. Estudei num colégio protestante australiano. Com a mocidade
me tornei bastante bela e como era muito instruída, me casei com um
inglês sábio que se dedicara à Metafísica.

--- Sim senhora\ldots{}

--- Meu pai era regularmente rico e fomos viajar, meu marido e eu. Como
era de esperar, a Índia nos atraía por causa dos seus grandes filósofos
e poetas. Fomos lá e depois de muitas peregrinações nos domiciliamos nas
proximidades dum templo novo, dedicado às doutrinas de Zoroastro. Meu
marido se tornara uma espécie de padre, ou melhor, de monge do templo e
ficara um grande filósofo meta físico. Pouco a pouco o seu pensamento se
elevava, se elevava, até que desmaterializou"-se por completo e foi vagar
na plenitude contemplativa de si mesmo, fiquei só. Isto não me pesava
porque desde muito meu marido e eu vivíamos, embora sob o mesmo teto, no
isolamento total de nós mesmos. Liberto o espírito da matéria, só ficara
ali o corpo de meu marido, e este não me interessava, mole, inerte,
destituído daquelas volições que o espírito imprime à matéria
ponderável. Foi então que adivinhei a alma dos chamados irracionais e
vegetais, pois que se eles não possuíssem o que de qualquer forma é
sempre uma manifestação de vontade, estariam libertos da luta pela
espécie, dos fenômenos de adaptação ao meio, correlação de crescimento e
outras mais leis do Transformismo.

--- Sim senhora!

--- Como o sr. vê, ainda não sou velha e bastante agradável.

--- Minh\ldots{}

--- Eu sei. Com paciência fui dirigindo o corpo do meu marido para um
morro que havia atrás do templo de Zoroastro, donde os seus olhos, para
sempre inexpressivos agora, podiam ter, como consagração do grande
espírito que neles habitara, a contemplação da verdade. E o deixei lá.
Voltei para o bangalô e fiquei refletindo. Quando foi de tardinha
escutei um canto de lauta que se aproximava. (Aqui a sra. Stevens começa
a chorar.) Era um pastor nativo que fora levar zebus ao templo. Dei"-lhe
hospitalidade, e como a noite viesse muito ardente e silenciosa, pequei
com esse pastor! (Aqui os olhos da sra. Stevens tomam ar de alarma.)

--- Mas, sra. Stevens, o assunto que a traz aqui a obriga a essas
confissões!\ldots{}

--- Não é confissão, é penitência! Fugi daquela casa, horrorizada por
não ter sabido conservar a integridade meta física de meu esposo, e
concebi o castigo de\ldots{}

--- Mas\ldots{}

--- Cale"-se! Concebi meu castigo! Fui na Austrália receber os restos da
minha herança devastada e agora estou fazendo a volta ao mundo, em busca
de meta físicos a quem possa servir. Cheguei faz dois meses ao Brasil,
já estive na capital da República, porém nada me satisfez. (Aqui a sra.
Stevens principia soluçando convulsa.) Ontem, quando vi o sr. saindo do
cinema, percebi o desgosto que lhe causavam essas manifestações
específicas da materialidade, e vim convidá"-lo a ir pra Índia comigo. Lá
teremos o nosso bangalô ao pé do templo de Zoroastro, servi"-lo"-ei como
escrava, serei tua! oh! grande espírito que te desencarnas pouco a pouco
das convulsões materiais! Zoroastro! Zoroastro! lá, Tombutu, Washington
Luís, café com leite!\ldots{}

Está claro que não foram absolutamente estas as palavras que a sra.
Stevens choveu no auge da sua admiração por mim (desculpem). Não foram
essas e foram muito mais numerosas. Mas com o susto, eu colhia no ar
apenas sons, assonâncias, que deram em resultado este verso maravilhoso:
``lá, Tombutu, Washington Luís, café com leite''. Sobretudo faço questão
do café com leite, porque quando a sra. Stevens deu um silvo agudo e
principiou desmaiando, acalmei ela como pude, lhe assegurei a
impossibilidade da minha desmaterialização total e, como a coisa
ameaçasse piorar, me lembrei de oferecer café com leite. Ela aceitou.
Bebeu e sossegou. Então me pediu dez mil réis pra o templo de Zoroastro,
coisa a que acedi mais que depressa.

Aliás, pelo que soube depois, muitas pessoas conheceram a sra. Stevens
em São Paulo.

\chapter{Mesquinhez}

Tenho que distinguir porém. Se é fato que existe em certas classes de
sublimadores de vida (poetas, mendigos etc.) sincera incompetência pra
viver, não é menos certo que muito indivíduo se aproveita disso para não
tomar atitude ante os fenômenos sociais. Dado que o artista, o cientista
é um ser à parte, pois então vamos nos aproveitar disso. Sistematizam
esse ``estado de off"-side'' que é inerente à psicologia deles, e na
verdade já não estão apenas à parte mais, criam mas é uma salvaguarda de
indiferentismo e até sem"-vergonhismo que lhes permite aceitar tudo em
proveito pessoal.

Quando Julien Benda estabeleceu no livro bulhento dele a condição do
clerc, ele não esqueceu de especificar bem que a contemplatividade do
intelectual às direitas não impedia este de se manifestar a respeito dos
movimentos políticos e tomar parte neles. Mas disto os cultos
brasilianos não querem saber. Se, entre escritores, ainda existem alguns
que, talvez por mais acostumados a pensar, tomam partido, é absurdo como
se estabeleceu tacitamente que pintores, músicos, arquitetos, fisiólogos
e tisiólogos e teólogos são da neutralidade.

Neutralidade? É, eles chamam de neutralidade o que é muito boa falta de
caráter. É a neutralidade que consiste v. g. no governo de Carlos de
Campos, em tudo quanto era concerto, qualquer pecinha desse compositor
lamentável, aparecer no programa. E me vinham: --- ``Você compreende, essa
música é banalíssima, porém nós que pertencemos à classe dos músicos
devemos honrar um, sim, um músico que está na presidência do Estado.'' E
como a bondade pessoal de Carlos de Campos era mesmo um fato, aquilo
rendia bem ao colega. Dele.

Este ano um pintor ainda me expunha suas teorias sobre a honradez
profissional. Era assim: --- ``Você compreende (usam e abusam do 'você
compreende' arranjador gratuito de cumplicidade) você compreende, tudo
isto que eu faço, não é minha arte não! Mas é disso que o povo gosta!
Estas orquídeas, isso é passadismo do miúdo, mas Fulano, que você sabe a
importância dele na Prefeitura, queria por força que eu pintasse
orquídeas. Eu pintei e ele comprou! Estou envergonhado de ter um quadro
assim na exposição, mas, você compreende, é por causa do nome do
comprador. Mais tarde, quando eu não tiver mais cuidados pecuniários,
então hei de fazer a arte que sinto em mim!''

E com isto, se receberem a encomenda de uma sinfonia pra Mussolini e um
retrato de Napoleão a cavalo, fazem. Porque diz"-que o assunto não tem
importância na obra"-de"-arte!\ldots{} Me parece incontestável que estamos
atravessando um momento muito importante, que pode despertar no povo
brasileiro a consciência social ---coisa que ele não tem. Ora não só
músicos, fisiólogos e fisiólogos, mas até entre os literatos, vou
percebendo uma pouca vontade vagarenta em tomar atitude. Parece que
estão muito preocupados em cantar a mãe"-preta, o seu rincãozinho, a sua
religiãozinha, pra tomarem consciência verdadeira do momento que a
nacionalidade atravessa, e vai bastante mais além desses lugares"-comuns
temáticos do novo ``modernismo'' de agora. No poema de Martin Fierro vem
aquela estrofe honesta, que gosto muito:

\begin{verse}
Yo he conocido cantores\\
Que era un gusto escuchar\\
Mas no quieren opinar\\
Y se divierten cantando;\\
Pero yo canto opinando\\
Que es mi modo de cantar.
\end{verse}

Eu acho que também temos que cantar opinando agora. Há muito mais nobre
virilidade em se ser conscientemente besta que grande poeta da arte
pura.

\chapter{Tacacá com tucupi}

Quem me chamou uma atenção mais pensamentosa para a cozinha brasileira
foi, uns quinze anos atrás, o poeta Blaise Cendrars. Desde que teve
conhecimento dos pratos nossos, ele passou a sustentar a tese de que o
Brasil tinha cultura própria (ou melhor: teria, se quisesse\ldots{}), pois
que apresentava uma culinária completa e específica. Sem impertinência
doutrinária, era apenas como viajante de todas as terras que Blaise
Cendrars falava assim. A tese lhe vinha da experiência, e o poeta
garantia que jamais topara povo possuindo cozinha nacional que não
possuísse cultura própria também. Pouco lhe importava que a maioria dos
nossos pratos derivasse de outros vindos da África, da Ásia ou da
península ibérica, todos os povos são complicadas misturas arianas. O
importante é que, fundindo princípios constitucionais de pratos
asiáticos e elementos decorativos de condimentação africana, modificando
pratos ibéricos, chegamos a uma cozinha original e inconfundível. E
completa.

Alguns comedores bons discordam de que a nossa cozinha seja completa.
Acham"-na pesada e incapaz de criar jantares dignos, leves e cerimoniais.
Culinária própria de almoço, exclusivamente. Não há dúvida que a maioria
dos nossos pratos principais é pesada, mesmo grosseira. Pratos como a
panelada de carneiro nordestina, o vatapá baiano, o tutu com torresmo
são de violência estabanada. O efó preparado à baiana é tão brutalmente
delirante que nem somos nós que o comemos, ele é que nos devora. A
primeira vez que ingeri uma colherada de efó, a sensação exata que tive
foi essa de estar sendo comido por dentro. Pratos que implicam a sesta
na rede e o entre"-sono\ldots{} Alguns mesmo nos deixam num tal estado de
burrice (de sublime burrice, está claro) que não é possível, depois
deles, comentar sequer Joaquim Manuel de Macedo.

Mas isto é meia verdade, e dentro da nossa culinária variadíssima temos
o que comer a qualquer hora do dia e da noite. O sururu alagoano bem
como o dulcíssimo pitu nordestino são espécies delicadíssimas de manjar.
Em todo caso, de modo grosseiro, pode"-se dizer que há uma ascensão
geográfica quanto ao refinamento e delicadeza da culinária nacional. À
medida que avançamos para o Norte, mais os pratos se tornam delicados.

Se principiamos no Sul, o churrasco gaúcho nem pode"-se dizer que seja
prato de mesa; é antes comida de campo que tira parte do seu encanto em
ser provada de pé, entre os perfumes do vento e do fogo perto. E faz
grande exceção em toda a nossa culinária característica, por ser um
prato simples, que não se inspira apenas no seu elemento básico para
combinações mais complexas, antes procura revelar a carne em toda a sua
mensagem. Dir"-se"-ia, neste sentido, um prato inglês. Porque,
filosoficamente falando, desculpem, diremos que a culinária pode se
orientar por duas apenas das três grandes ideias normativas que regem
nossa humanidade: pelo Bem e pelo Belo. Está claro que, sendo
necessariamente verdadeira e não interessando imediatamente ao\ldots{}
pensamento puro, a culinária põe a Verdade de banda. As cozinhas
francesa e inglesa podem comparecer como protótipos das duas orientações
normativas da culinária. A inglesa se orienta pela ideia do Bem: mais
simples, mais franca, buscando apenas variar pelos molhos a monotonia
das suas bases. Até o seu uísque de após janta, mais digestivo e
funerário, é um valor fácil como a maioria dos heróis shakespearianos,
se o compararmos ao sabor montaigne de uma fine. A cozinha francesa se
orienta francamente pela ideia do Belo. As bases alimentares quase
desaparecem, sutilizadas às vezes em combinações de um inesperado
miraculoso. Isso é invenção desnecessária, é arte às vezes do mais
gratuito hedonismo.

Em geral a nossa culinária se dirige também pelas normas do Belo. Vindo
do Sul para esta zona caipira, os nossos pratos já são ricas
multiplicações. Em alguns deles chega a ser difícil descobrir qual a
base alimentar inspiradora. A feijoada, por exemplo, em que o feijão
deixou de ser o fundamento, pra se tornar o dissolvente das carnes
fortes. E quase o mesmo diríamos do nosso cuscuz paulista, que pondo de
parte a farinha, se determina pela combinação principal, ``cuscuz de
galinha'', ``cuscuz de camarão''.

Com a Bahia a violência dos pratos se acentua em mesa bem mais variada.
Estamos no auge da influência negra: e uma brutalidade de zabumba,
agressivamente misteriosa, cheia de carícias estupefacientes, arrasa os
paladares, que caem no santo, completamente divinizados.

Da Bahia pro Norte, os grandes pratos vão se tornando cada vez mais
delicados. É certo que continuam ainda pratos ásperos, vem a panelada,
vem o trágico tacacá com tucupi. Mas o Nordeste concorre com os seus
pitus e sururus; e então uma sioba cremosa deslizando sobre o feijão de
coco em calda, servida em porcelana translucidamente branca, isso é
prato para o mais granfino jantar.

Mas, a meu ver, onde a culinária brasileira atinge suas maiores
possibilidades de refinamento é na Amazônia. Todos já perceberam que pus
de lado certas caças, encontráveis mais ou menos por todo o país, que
podem nos dar pratos da maior delicadeza. O macuco baixa do poleiro com
seu sabor tão silencioso; e vem a ingênua paca no seu gosto irônico de
estarmos prejudicando virgens; e principalmente o tatu"-galinha, uma das
nossas mais perfeitas carnes como sutileza do tecido. Mas são carnes que
ainda não se culturaram e não sabemos tratar. A rusticidade jesuítica
dos nossos costumes rurais ignora esse requinte pecaminoso de descansar
suficientemente uma caça, de modo que a asperidade do mato fique apenas
como um\ldots{} background do paladar.

Não. É na Amazônia que melhormente podemos jantar. É lá que se encontra
o nosso mais fino pescado de água"-doce, ninguém pode imaginar o que seja
uma pescadinha do Solimões! Ninguém pode imaginar o que é um ``casquinho
de caranguejo'' distraidamente pulverizado com farinha"-d'água. A
tartaruga, principalmente a tracajá mais risonha, dá vários pratos
suaves, e o pato de Marajó vagamente condimentado com o tucupi
picante\ldots{} Devo acabar aqui, pois estou ficando com vontade de comparar
tais sabores com Morgan, Bergson e o engenhoso fidalgo Valéry. E certas
frutas, principalmente o bacuri perfume puro, tratadas sem açúcar,
viriam finalizar tais jantares, como versos de Rilke. E assim é que,
nestes tempos aviatórios, a minha experiência já vos pode dar este
conselho: Almoça"-se pelo Brasil, janta"-se no Amazonas.

\chapter{Fábulas}

No quase fundo do pastinho desta chácara, junto do aceiro da cerca, tem
uma arvoreta importante, com seus quatro metros de altura e folhagem
boa. No sol das treze horas quentes passa um velho arrimado a um bordão.
Pára, olha em torno, vê no chão um broto novo, ainda humilde, de futura
arvoreta, e o contempla embevecido. Quanta boniteza promissora nesta
folhinha rósea, ele pensa. Já descansado, o velho vai"-se embora. Diz o
broto: --- ``Está vendo, dona arvoreta? A senhora, não discuto que é o
vegetal mais corpudo deste pastinho, mas que valeu tamanha corpulência!
Velho parou, foi pra ver a boniteza rósea de mim.'' --- ``Sai, cisco!'' que
a arvoreta secundou; ``velho te viu mas foi por causa da minha sombra, em
que ele parou pra gozar.''

Ora andava o netinho do velho brincando no pasto, catando gafanhoto,
nisto enxergou longe, lá na beirada do aceiro, um tom vermelho. Correu
pra ele, era o broto, na intenção capitalista de o arrancar. Mas
chegando perto, faltou ar pro foleguinho curto do piá, ele parou
erguendo a carinha pra respirar e se embeveceu contemplando a arvoreta
que lhe pareceu imensa no pastinho ralo. Esqueceu o broto que, de perto,
já nem era encarnado mais, porém dum róseo sem força. Depois cansou
também de contemplar a arvorita que nem dava jeito de trepar, deu um
pontapé no tronco dela e foi embora. --- ``Ah, ah'', riu a arvoreta, ``está
vendo, seu broto? Você, não discuto seja mais colorido que eu, porém
columim parou foi pra espantar com a minha corpulência.'' --- ``Sai,
ferida!'' que o broto respondeu; ``diga: por que foi que o columim te
enxergou, diga! por quê! Ahm, não está querendo dizer!\ldots{} pois foi minha
cor, ferida! foi minha cor lindíssima que o chamou. Sem mim, jamais que
ele parava pra te ver, ferida!''

Ora sucedeu chegar a fome num formigueiro enorme que tinha pra lá da
cerca, e as saúvas operárias saíram campeando o que lhes enchesse os
celeiros. Toparam com uma quenquém inimiga que, só de malvadeza, pras
saúvas ficarem sofrendo mais fome, contou a existência do broto
encarnado. As saúvas foram lá e exclamaram: --- ``Isso não dá pra cova dum
nosso dente! antes vamos fazer provisão nesta enorme árvore.'' Deram em
cima da arvoreta que, numa noite e num dia, ficou pelada e ia morrer. O
broto, sementezinha da arvoreta mesmo, noite e dia que chorava e que
gemia, soluçando: --- ``Minha mãe! minha mãe!''

Carecendo de fogo em casa, no outro dia, o velho saiu pra lenhar. Passou
pela arvoreta que era só pau agora e ficou furibundo: --- ``Pois não é que
essas danadas de saúvas me acabaram com a única sombra que eu tinha no
pasto!'' E de raiva, deu uma machadada no chão. Acertou justo no broto
que se desenterrou bipartido e ia morrer. O velho foi buscar formicida e
matou todas as saúvas que, aliás, já estavam morre"-não"-morre porque a
folha da arvoreta era veneno. E o velho pegou de novo no machado e foi à
procura dum pau pra lenhar. Enquanto isso, a arvoreta moribunda, com
vozinha muito fraca, olhava o broto arrancado no chão: --- ``Meu filho!
meu filho!''

--- Onde que vai, vovô? exclamou o netinho topando o importante machado
no ombro do velho.

--- Vou lenhar.

O columim logo lembrou a árvore enorme que tanto o espantara na véspera:
--- ``Pois então pra que você não derruba aquele pau grande que está na
beirada do aceiro, lá''? --- ``Ora, que cabeça a minha!'' pensou o velho;
``pois senão dá sombra mais e está perdida, melhor é derrubar a arvinha
mesmo.'' Porém muito já que tinha se movimentado no ardente sol. Nem bem
derrubou o tronco, veio um mal"-estar barulhento por dentro, nem soube o
que teve, fez ``ai, meu neto!'', deu um baque pra trás e morreu.

No outro dia, enquanto andavam fazendo o enterro chorado do velho, o
netinho estava entretidíssimo com o tronco derrubado da arvoreta. Assim
retorcido como era, fazia um semicírculo que nem de ponte chinesa, sobre
o chão. Isso o menino fez, só que não imaginando na China. Era uma ponte
formidável sobre um imenso rio. O columim atravessava a ponte, chegava
do outro lado e era o porto. Então embarcava num galho da arvoreta,
caído por debaixo da ponte, remava com outro galhinho e estava tão
satisfeito que pegando a folhinha já roxa do broto, solta ali, enfeitou
com ela o chapéu. Uma única saúva salva, que estava agarrada na folhinha
do broto, mordeu a orelhinha do piá, que deu um imenso berro e foi
embora chorando pra casa. Pra consolar o filho, a mãe deu uma sova no
broto que ocasionara a mordida da saúva.

O menino viveu mais cincoenta"-e"-sete"-anos, casou"-se, fez política,
deixou vários descendentes. Uma quarta"-feira morreu.

\chapter{Rei Momo}

Também quis celebrar o rei Momo e logo me vesti de azul e de encarnado.
Então me olhei no espelho e esmoreci. Não é que eu imagine indignidade
desumana a gente estar se preocupando de alegria num momento como este,
em que positivamente o mundo vai de mal a pior. A final das contas, eu
já cheguei também àquele alto de montanha, muito avançado no caminho da
experiência, pra estar mais ou menos desconfiado de que sempre o mundo
foi de mal a pior. E apesar disso ele ainda está aí, bastante ativo e um
pouco perturbado. Talvez não faça mal que, de permeio a missões de \textsc{sos}
financeiro e leis de segurança para os governinhos bastante perturbados,
a gente se enlambuze, por uma quartaHfeira apenas, com o zarcão da
alegria. O que me fez esmorecer foram as cores que logo preferi pra me
enfeitar.

São as cores tradicionais da alegria brasileira\ldots{} Com elas se vestiram
os zambis de mentira, tirados da escravaria, a que os padres, num gesto
meio aborrecido, como viu Koster, entregavam cetro e coroa à porta das
igrejas. Com o azul e o encarnado se distinguiam as facções guerreiras
nas danças dos Congos e Gingas, bem como as dos nossos Congados e
Moçambiques caipiras. Nas Cavalhadas, de Norte a Sul, eram ainda o azul
e o encarnado que aformoseavam a gesticulação enfeitada de cristãos e
mouros. Cristãos de azul, cor de Deus, mouros de vermelho, cor do Sujo.
E ainda até agora nalgum raro lugar, a rapaziada briga por um pedacinho
de ita das mulatas do Cordão Encarnado ou do Cordão Azul, nos
Pastoris\ldots{}

Ora eu soube que chegou a esta cidade de São Paulo, quem? o rei Momo em
pessoa. Mas eu não conheço o rei Momo, nunca tive argent pra ir na
Europa. Qualquer dia havemos de ter por aí o Bruder Alex, o Sultão
T"-Tulba e os outros bodes expiatórios, também enlambuzados de alegria,
que permitimos reinem por toda uma quarta"-feira, pra que, destruindo"-os
depois, levem consigo o nosso mal humano. É inútil: não levam não e
ignoro as cores do rei Momo europeu.

Por que não se tentar trazer de novo a São Paulo o Sultão do Meio"-Sol e
da Meia"-Lua, das Cheganças, o Arrelequim do Bumba"-meu"-Boi, o Matroá
fabulosíssimo dos Caiapós, ou melhor, o rei Congo e a rainha Ginga?\ldots{}
Não estou censurando comissões de alegria, não é censura, é saudade. É
este anseio meu, rabugento, de unir presente e passado, anseio de quem
vê dia por dia o homem sempre o mesmo, incapaz de beneficiar de suas
próprias experiências. Pois já não estão querendo criar o Conselho
Nacional do Algodão! Em cada gesto humano a gente percebe sempre, não a
experiência, mas a macaqueação de trezentos séculos. ``Repetição, tudo
repetição'', pra corrigir a atrasadíssima sabedoria salomônica. E se
repetimos diariamente os erros milenários, se eles renascem com
facilidade de erva e fecundidade suína, por que não tentar o
renascimento de costumes que só desapareceram pela desordem dos chefes?

Dantes, as festas dadas pelos chefes pra que o povo se\ldots{} se esqueça
tocavam base popular. Não era a Marinha que subia no tablado nem o rei
Momo. Era o Armirante Mascaranha e o Surtão de Trugue"-e"-Metrogue. Nascia
lá no transatlântico Portugal o, quem? o príncipe da Beira Baixa. Então
estava convencionado que o brasileiro ficou alegríssimo e queria se
divertir. Mandavam avisar de lá que os coloniais de cá tinham que mandar
pra lá um presente de suponhamos um milhão de cruzados de cá pro
bercinho do herdeiro macho. E isso era motivo de alegria fenomenal. E o
suponhamos Ilmo. e Exmo. Sr. D. Luís Gonzaga de Sousa Botelho e Mourão,
Governador e Capitam General da Capitania de São Paulo, decretava
grandes alegrias públicas, de que algum áulico poeta escreveria, em
letra iluminada, a Relação. Ah, como esta vestimenta de azul e encarnado
me amarga!\ldots{} Estou imaginando num rei Congo diante do qual o próprio
Governador se abaixasse, pra lhe pegar o cetro caído, como sucedeu no
Tejuco\ldots{} Ou nalgum armirante de mentira, que nem aquele João Pacheco
baiano, que fazia parar divisões do exército inteiras, pra trocar
continências de estilo com generais de verdade\ldots{} Um pouco de orientação
em poucos anos faria renascer tudo isso\ldots{} E talvez isso trouxesse pelo
menos uma justificativa mais humana aos decretos oficiais de alegria.

\chapter{Idílio novo}

Oh! quem são esses entes fugazes, duendes fagulhando na luxuosa
cidade!\ldots{} Eles brotam dos bueiros, das portas, das torres, rostos
lunares, e a dentadura abrindo risos duma intimidade ignorada no
ambiente gélido. Chatos, troncudos eles barulham, pipilam, numa fala
mais evolucionada que a nossa, fulgurante de vogais sensíveis e de sons
nasais quentes, que são carícias perfeitas. Quem são esses sacizinhos
felizes, confiantes nos trajes improvisados, portadores da alegria nova,
estrelando na ambiência da agressiva cidade!\ldots{}

Naquele recanto de bairro a casa não era rica mas tinha seu parecer. Aí
moravam uma senhora e seus filhos. Era paulista e já idosa, com bastante
raça e tradição. Cultivava com pausa, cheia de manes que a estilizavam
inconscientemente, o jardinzinho de entrada e o silêncio de todo o ser.
Suas mãos serenas davam rosas, manacás, consolos e, abril chegando,
floresciam numa esplêndida trepadeira de alamandas, que fora compor seu
buquê violento num balcão. Nos abris e maios do bairro, os automóveis
passando, até paravam pra contemplar.

Outro dia estava a senhora lidando com as sedas sírio"-paulistas da ilha,
quando a criada veio falar que tinha na porta um sordado. A senhora
percorreu logo a criada com olhos de inquietação. Com as últimas
revoluções a senhora tivera portão vigiado, armas escondidas, filho
preso. Ergueu"-se, recompôs o estilo do rosto, foi ver. No portão estava
um tenentinho moreno, cheio da elegância, mais a mulher dele, menos
flexível, achando certa dificuldade, se via, em se vestir com distinção.
Mas ambos figuras duma simpatia imediata, confiantes como água de beber.

--- Bom dia! sorriram.

--- Bom dia.

--- Madame é a dona da casa!

--- Sou.

--- Nós passamos sempre por aqui! Achamos muito linda essa trepadeira
que a senhora tem no balcão!\ldots{}

--- Como se chama essa flor!\ldots{}

--- Alamanda.

--- Alamanda!

--- É.

--- Nós moramos ali em cima naquela casa grande da esquina, e Hosana
sempre me chama a atenção para a sua trepadeira! aquela ali de frente
não é tão bonita assim!

--- É tão bonita. É a mesma.

--- Não parece não! não acha, Catita!

--- Até parece outra, olha só a cor! Nós queríamos pedir à senhora que
nos desse um galho pra plantarmos em nosso jardim!

--- Eu dou o galho\ldots{} Mas não sei se esta planta pega de galho. Comprei
ela já crescida.

--- A senhora quer que ajude! Hosana, vá com madame!

--- Muito obrigado, não carece.

Então a senhora já completamente sossegada subiu ao balcão. Com a
natural cordialidade pouco visível exteriormente nela, escolheu três ou
quatro galhos bem robustos, foi cortando. Ela de lá, eles de baixo,
estabeleceu"-se logo uma conversação agradável, toda criada pelo
tenentinho e sua mulher, que ainda durou no portão, com muitos
agradecimentos do casal, já uma certa familiaridade e oferecimentos de
casa e dos préstimos.

A senhora não soube corresponder porque nunca aprendera isso tão
depressa. Meio que a assustava aquela intimidade com vizinhos, pedir
coisas, mandar presentinhos, jeitos que jamais não tivera na vida nem
lhe ordenavam os seus manes. Não pôde oferecer nada, aquela colaboração
com vizinhos lhe desarranjava todo o silêncio. Mas soube sorrir com um
``possível carinho'' no adeus. E era incontestável que lhe ficava no peito
uma espécie de felicidade. Não era verdade, ela sabia, mas sempre tinha
no mundo alguém que achara as flores dela mais bonitas que as do jardim
rico defronte. Estava próxima de querer bem o tenentinho e sua mulher.

\chapter{Calor}

Calor\ldots{} O Rio de Janeiro está na sua maior festa física de terra onde
quem mandou o homem vir morar? O contraste é violentíssimo: percebe"-se
claro que tudo quanto não é ser humano ou animal de cultura está
gozando, se expandindo, se multiplicando, enquanto o homem sofre
pavorosamente. Um pensamento só me preocupa o espírito vagarento: tudo
quanto é ser humano sofre insofismavelmente, sofrem os pobres como os
ricos, não há distinção de casta, nem de raça, nem de idade,
martirizados pelo calorão. Mas tudo o que é desumano se deslumbra e
revive num escandaloso esplendor. Pois é incontestável que também a
falange das mulheres floresce traidoramente, adere franco ao delírio da
vitalidade mineral e vegetal, tanto mais esplêndidas que o macho se
mostra chucro e charro. Isto me inquieta pouco aliás, porque eu pago
imposto, mas hei de continuar solteiro. Em todo caso, ajuntando
recordações esparsas pelos anos, sinto mesmo que deve haver qualquer
coisa de mineral nas mulheres.

Desta janela, os meus olhos vão roçando a folhagem vertiginosamente
densa da Glória e da praça Paris, buscar no primeiro horizonte os
arranha"-céus do Castelo. A superfície da folhagem é feia, de um verde
econômico, desenganadamente amarelado. Mas em baixo, dentro dessa crosta
ensolarada, o verde se adensa, negro, donde escorre uma sombra candente,
toda medalhada de raios de sol. Passam vultos, passam bondes, ônibus,
mas tudo é pouco nítido, com a mesma incerteza linear dos arranha"-céus
no longe, ou, mais longe ainda, no último horizonte, a Serra dos Órgãos.
Porque a excessiva luminosidade ambiente dilui homens e coisas numa
interpenetração, num mestiçamento que não respeita nem o mais puro
ariano. Os corpos, os volumes, as consciências se dissolvem numa
promiscuidade integral, desonesta. E o suor, numa lufa"-lufa de lenços
ingênuos, cola, funde todas as parcelas desintegradas dos seres numa
única verdade causticante: \textsc{calor}!

Estou me recordando dos outros grandes momentos de calor que já vivi\ldots{}
Três deles se gravaram pra sempre em minha vida, momentos sarapantados
de infelicidade, desses que depois de vividos a gente sente certo
orgulho em recordar. O mais conscienciosamente sofrido dos três foi numa
errada de meio"-dia, alto sertão da Paraíba, junto à Borborema. Íamos
de auto e fazia já seguramente duas horas que não encontrávamos ninguém,
na estrada incerta que tomáramos. O mundo era pedra só, do seixo ao
rochedo erguido feito um menir, tudo pulverizado de cinza, sob a
galharia sem folha das juremas sacrais. Sob elas, o deus"-menino do
Nordeste, Mestre Carlos, o ``que aprendeu sem se ensinar'', adormecera pra
sempre e se desencarnara, indo com mais amplitude fazer bem aos homens
lá nos altos reinos. A hora aproximava do meio"-dia, quando topamos a
final com uma casa, algum ``morador'' de fazenda, com certeza. Chamamos
por gente, e no fim de certo tempo apareceu, palavra de honra que
tivemos a noção perfeita de que o homem era Jesus. Um sertanejo
belíssimo, completamente igual ao Jesus de Guido Reni, ou das verônicas
que se vende por aí. Ficamos estarrecidos. Mas Jesus foi péssimo pra
nós, a estrada que deveríamos tomar não era aquela não, mas a outra que
fazia encruzilhada com a nossa, umas três horas de caminho atrás: era o
pino do dia. Desde alta madrugada viajávamos assim, vindos do Açu, sem
comer, recusando a água barrenta dos pousos, pois contávamos em breve
almoçar e tirar um bom naco de conversa em Catolé do Rocha, espreitando
os domínios do Suassuna. E agora só iríamos alcançar a cidadinha pela
boca"-da"-noite, se Deus quisesse. Ah, ninguém não ouse imaginar o calor
que principiou fazendo de repente! um calor de raiva, um calor de
desespero e de uma sede pavorosa que a raiva inda esturricava mais. Esse
foi o maior calor que nunca senti em vida, o calor dos danados, em que
falei palavras"-feias, pensei crimes e me desonrei lupulentamente.

De outra espécie, dolorido mas magnificentemente vicioso, foi o calor
que aguentei no centro de Marajó, lago Arari. Entre as venturas da ilha,
o verde inglês dos pastos, visita a búfalos e os sublimes pousos de
aves, coisa de indescritível fantasmagoria, a nossa ingenuidade de
turistas culminara de bom"-humor com a vista do vilejo lacustre que
bóia na boca do lago. Nos transportamos para os tempos neolíticos,
descobrimos a cerâmica, polimos a pedra e várias outras conversas de
fácil erudição. Depois decidimos dar um passeio no lago e tomarmos assim
um gostinho das inatingíveis jazidas do Pacoval, que ficavam do outro
lado e estariam submersas naquela época de cheia. Porém naqueles mundos
amazônicos não tem água que não guarde traição: nem bem avançávamos uns
quinhentos metros no lago, que a lancha estremeceu, mordendo fundo no
areão invisível, parou. Depois de uns três esforços para nos safarmos do
encalhe, o mestre percebeu que a coisa era grave e o melhor era mandar o
único bote em busca de socorro. Imaginamos logo o que seria de tempo,
descer num bote de remo todo o rio mole, arranjar socorro e o socorro
chegar até junto de nós\ldots{} O calor já vinha afastando com severidade as
brisas matinais do lago e o céu era sem nuvens. Nem foi tanto questão de
calor, foi mais questão de luz. Aos poucos uma luz imensa, penetrante,
foi engolindo tudo. Já mal se enxergava o vilarejo lacustre, as margens
tinham desaparecido. O amarelado solar foi clareando impassível, foi se
tornando cada vez mais branco, incomparavelmente branco, e o vilejo
desapareceu também, imerso no algodão que escaldava. O azul do céu
diluiu"-se na alvura de fogo que as águas espelharam sem piedade,
brancas, assombrosamente brancas. A primeira consciência de sofrimento
que tive foi de estupor, não tem dúvida, espaventado com aquela trágica
massa de brancos luminosos em que tudo se engolfou tumultuariamente, num
estardalhaço espalhafatoso de cataclisma. Não havia mais olhar que
ousasse apenas entreabrir, mas as próprias pálpebras fechadas eram
incompetentes para nos livrar da fatalidade da luz. O branco penetrava
pelos poros, pelos ouvidos, pela boca, nada agressivo agora, nada
impetuoso, mas certo, irrevogável, irrecorrível, alcançando os ossos,
alcançando o cérebro que de repente como que parava, convulsamente
branco também.

Hoje, às vezes, tenho desejo de sentir de novo a sensação medonha que
sofri, tenho como que uma saudade daquele branco em fogo. Mas isso deve
ser vício, pureza é que não é. Se escolheram o branco para simbolizar a
pureza, deve ser mesmo porque a pureza é impossível de sustentar. Mas
agora estou lembrando aqueles tapuios do vilejo lacustre, que lá viviam
e ainda vivem, na convivência do assombro. Pois então mudemos a
conclusão e convenhamos que até com a pureza há gentes que conseguem se
acostumar.

Enfim a terceira lembrança de calor que guardo nos transporta a Iquitos,
no Peru. Mas nesta, o calor não se colore de raiva nem de luz, nem de
coisíssima nenhuma, é um calor só calor, e talvez por isso mesmo
degradante e de pouco interesse experimental. Nós chegáramos à cidade
(assim mais ou menos do tamanho de Moji das Cruzes) com a indumentária
de célebres, recebidos com aparato e o nobre presidente, de ponto em
branco, no cais lutuante, para nos saudar. Fazia um calor de estafa, e
depois de todo um cerimonial longo, e por aí uma centena de
apresentações e consequentes apertos"-de"-mão, ``muito prazer'', o
presidente se retirou enquanto o secretário dele me advertia em segredo
que dentro duma hora seríamos esperados em palácio, para retribuição
oficial da visita oficial. Teríamos que vestir pelo menos um linho mais
escanhoado e o suor nascia como fonte, diluindo qualquer esperança de
discrição. Me lembrei de tomar um banho frio, daquele frio relativo e
sempre sujo, das águas barrentas do Amazonas. Mas quando principiou a
cerimônia de enxugar o corpo é que se deu o acontecimento cruel:
verifiquei apavorado que não havia nenhuma possibilidade de me enxugar.
Nem bem enxugava de um lado, que o outro chovia em suores inesgotáveis,
que calor! Foi então que sentei na cama da cabina e tive, palavra de
honra, tive, aos trinta e muitos anos daquela existência seca, uma
sensação degradante: vontade de chorar. Me nasceu uma vontade manhosa de
chorar, de chamar por Mamãe, me esconder no seio dela e me queixar, me
queixar muito, contar que não aguentava mais, que aquele calor estava
insuportável, desgraçado, maldito! Enquanto ela docemente enxugaria as
minhas lágrimas, murmurando: ``Tenha paciência, meu filho, o calor é
assim mesmo''\ldots{} Se não chorei foi de vergonha dos espelhos. Porém jamais
me percebi mais diminuído em mim, mais afastado das bonitas forças da
dignidade.

O calor desmoraliza, desacredita o ser, lhe tira aquela integridade
harmoniosa que permitiu aos suaves climas europeus suas bárbaras noções
cristãs, sua moral sem sutileza, e suas forças brutamontes de criação.
Que se tenha conseguido implantar, neste calor brasileiro, laivos bem
visíveis da civilização europeia, me parece admirável de força e
tenacidade. E talvez tolice enorme\ldots{} Melhormente nos formaríamos talvez
como chins ou indianos, de místico e vagarento pensar.

\chapter{Tempo de dantes}

Este é um caso brasileiro da terra potiguar.

No município de Penha suponhamos que Antônio de Oliveira Bretas era
senhor de engenho, homem já de seus trinta e cinco anos, casado com dona
Clotildes, já sabe: cabeça"-chata atarracado, falando alto. Dona
Clotildes chamava ele ``seu Antônio'' e ele respondia ``a senhora''. A mana
dela também morava no engenho que não era grande não, produção curta mas
com uma aguardente famosa no bairro.

Na véspera de Ano Bom dançavam um pastoril muito preparado na vila da
Boa Vista, ficada a umas três léguas do engenho, e dona Clotildes quis
ver. Chamou a negrinha:

--- Vá dizer a seu Antônio que eu quero que ele me leve na Boa Vista,
ver o pastoril.

A negrinha foi.

--- Fale pra dona Clotildes que não quero ir na Boa Vista hoje.

A negrinha foi e voltou falando que dona Clotildes mandava dizer que
queria mesmo ir ver o pastoril. O senhor de engenho embrabeceu:

--- Pois se ela quiser ir que vá sozinha! Levo ninguém não! Dona
Clotildes teve raiva.

--- Clotildes!\ldots{} ôh Clotildes!\ldots{}

Que Clotildes nada. O vestido caseiro estava sacudido na cama. Os
sapatos caseiros por aí. Dona Clotildes tinha partido com a mana. Três
de janeiro um vizinho portou no engenho, chamou Antônio de Oliveira
Bretas e deu o recado. Diz que dona Clotildes mandava pedir ao marido ir
buscá"-la, passado Reis.

--- Foi sozinha! pois que venha sozinha! Vou buscar ninguém não!

E não foi mesmo. Dona Clotildes decerto achou desaforo aquilo e ficou
esperando na vila. Um mês passou. Mas, e agora? O senhor de engenho
careceu de ir na vila por amor duns negócios, ir lá?\ldots{} Parecia por
causa da mulher\ldots{} Mandou um amigo. Dona Clotildes soube, se moeu de
raiva: agora é que não voltava sem seu Antônio ir buscá"-la!

Dois meses passaram, três\ldots{} Passou um ano, passaram dois, meus amigos!
No engenho, seu Antônio vivia sozinho, não mostrando tristeza.

Mas mandava limpar o quarto de casados sem que mudassem nada do lugar. O
sapato direito sacudido no meio do quarto. O vestido caseiro dormindo de
atravessado na cama aqueles anos inativos. E nove anos passaram.

Numa noite de lua dona Clotildes voltou. Antônio de Oliveira Bretas
fumava na sala de entrada, conversando com o amigo que viera comprar
aguardente. Este chegou na porta da casa, se calou de repente, aprumou a
vista:

--- Compadre!

--- Eu?

--- Homem, parece que é dona Clotildes que vem lá na estrada!\ldots{}

--- Hum.

Era dona Clotildes com a mana. Apeou do cavalo e chegou na porta.

--- Dá licença, seu Antônio?

--- A senhora não carece de pedir licença nesta casa.

Não houve uma explicação, uma recriminação, nada. Dona Clotildes entrou,
foi até o quarto. O vestido caseiro dela, aquele, meu Deus! faziam nove
anos, estava até sacudido com raiva, de atravessado na cama. Os sapatos,
mesma coisa, no chão, sem alinhamento. Quarto na mesma. Ar, na mesma.
Nove anos passados. Dona Clotildes se trocou e, como estavam na hora da
ceia, mandou a agora moça"-feita da negrinha botar a mesa. Cearam. Vieram
as palavras quotidianas, quer isto? quer aquilo? quero, não quero não,
dormiram, se levantaram, etc.

\chapter{Esquina}

É chegado o momento de vos descrever minha esquina.

Eu moro exatamente na embocadura dum desses igarapés cariocas feitos de
existências em geral apressadas ---ruazinhas, vielas que, nascidas no
enxurro do morro próximo, desembocam na famosa rua do Catete. Estranha
altura este quarto andar em que vivo\ldots{} Não é suficientemente alta para
que a vida da esquina se afaste de mim, embelezada como os passados; mas
não chega a ser bastante baixa pra que eu viva dessa mesma vida da rua e
ela me marque com seu pó. Mas apesar dos quartos"-andares e outras
comodidades modernas que a cercam nos becos e praias próximas, a rua do
Catete é ainda caracteristicamente uma rua a dois andares. O andar
térreo, onde mascateia um comércio miúdo sem muitas ambições, e, tenham
as casas três ou quatro andares, um só andar superior, onde se enlata no
ar antigo, muitas vezes respirado, uma gentinha de aluguel.

Contemplando essa gente do segundo andar, me ponho imaginando a classe a
que pertence. É um lento exército de infiéis, que fazem todos os
esforços pra não pertencer à classe operária. Mas é fácil verificar que
não chegam a ser essa pequena burguesia que vive agarrada ao seu bem"-bom
e indiferente a tudo mais. Não. É uma casta de inclassificáveis, cuja
forma essencial de vida é a instabilidade. Enorme parte dela é pessoal
do biscate, que a audácia faz pegar qualquer serviço, qualquer. Ou são
empregados baratos que insistem em bancar alturas, e só começam vivendo
quando de noite, no sábado, se trans figuram na roupa cinza e no sapato
de praia, e vão por aí, feito gatos, buscando amor. Ou são
costureirinhas, bordadeiras, chapeleiras que não trabalham na oficina,
isso não! trabalham ``particular'', menos vivendo do seu recato ou
tradição renitente que da espera de algum príncipe que as eleve a
frequentadoras de bar. Há também as famílias: pai cansado, cujo
exclusivo sinal de vida é o cansaço, mãe desarranjada que dá pensão pra
estudantes de fora e as crianças, muitas crianças, de dois até treze
anos. Porque é uma coisa terrivelmente angustiosa esta do andar superior
da rua do Catete: a quase completa ausência de adolescentes. Com a rara
exceção de algum estudantinho pensionista, não se vê uma só garota, um
só rapaz de quinze até vinte anos. Não sei se morrem, se fogem ---em
qualquer dos dois casos buscando vida melhor.

Instáveis no trabalho, instáveis na classe, estes seres são
principalmente instáveis na moradia. É mesquinho, mas ninguém mora mais
de três meses na mesma casa. As famílias, os sozinhos chegam e da mesma
forma partem, quase mensalmente. Mas sem ruído, com humildade
sorrateira, mudanças tão reles que não chegam sequer a colorir a
existência da esquina. E o andar superior da rua do Catete se enfeita de
barbantes em cuja ponta acenam papelões, fazendo o sinal do ``Aluga"-se''.

Minto. No meio de toda essa instabilidade, há um caso altivo que tem me
preocupado até demais. Quase em frente da esquina, há uma casa de
janelas fechadas. Desde que cheguei aqui, faz um ano e oito meses, essa
casa viveu sempre assim. De primeiro imaginei que ninguém morasse ali, e
o andar estivesse condenado pela Higiene, que ideia minha! Se a Higiene
quisesse agir, creio condenaria toda a rua do Catete. A final, uma
feita, era pela manhã, percebi que uma nesga tímida se abria numa das
portas de sacada da tal casa. A nesga foi se abrindo com muita lentidão,
e a final se aventurou pela abertura uma cabecinha de criança. Criou
coragem, entusiasmada com o dia, entrou todinha na sacada, chamou outra
da mesma idade e graça, e ambas se debruçaram sobre a rua, olhando tudo,
mostrando tudo. E de repente, esquecidas, principiaram soltando felizes
risadas. Pela abertura, se percebia que a sala estava inteiramente
despida, nenhum móvel. Então apareceu uma senhora que não olhou pra
nada, nem inquieta parecia. Apenas deu uns petelecos nas crianças e
fechou tudo outra vez. De vez em longe a cena se repete inalterável. As
crianças conseguem abrir a porta e se debruçam, brincando de ver a
esquina. Não dura muito, surge a senhora que não olha mundo, dá uns
petelecos nas crianças e fecha tudo outra vez.

E há o caso do rapaz que se olhava nu, altas horas, num jogo de
espelhos\ldots{} E há o caso da gorda, o do paralítico a quem morreu a mulher
que o tratava, o das duas irmãs, mas tenho que descer para o andar
térreo. Na rua, quem vive são os operários. Este operariado do Catete,
que mora por aqui mesmo, no fundo das casas, no oco dos quarteirões, nos
vários cortiços que arriscam desembocar na própria rua. Muitos vivem de
pé"-no"-chão, mesmo aqui, bem junto da sublime praça Paris. Não é gente
triste, embora todos sejam de físico tristonho. O nível de vida é
baixíssimo, só as mocinhas se disfarçam mais. Os outros, mesmo os
jovens, mesmo os lusíadas resistentes, mostram sempre qualquer ombro
tombado ou peito fundo, marca de imperfeição. Deles a vida não é
instável, pelo contrário. São sempre os mesmos e já os conheço a todos.
Esta gente, passados os vinte"-e"-dois anos e o ``ajuntamento'' legal ou
não, não se movimenta mais: são os homens que vêm até a esquina. De
noite, após a janta, ou nos domingos de camisa limpa, eles têm que
descansar e se divertir um bocado. Então vêm na esquina, se encostam nas
árvores ou se ajuntam na porta dos botequins, conversandinho. Os bondes
passam cheios do futebol que nos faz esquecer de nós mesmos. Mas estes
homens nem de futebol precisam. Só conseguem é vir até a esquina,
reumáticos de miséria.

Mas o bom"-humor brinca assim mesmo nas bocas, até em horas de trabalho,
e a esquina é um espetáculo em que há qualquer coisa de desumano, de
macabro até. Como é que este pessoal consegue conservar um bom"-humor que
pipoca em malícias e graças! Esta gente parece ter a leviandade
escandalosa do mar de praia que está próximo e se atreve a jogar
banhistas quase nus até nesta esquina tão perfeitamente urbana. Mar
também achanado, sem crista, de baixo nível de vida, este mar de
porto\ldots{} Nem ao seu parapeito podemos chegar em passeio, porque são tão
numerosos os casais indiscretos quanto numerosíssimos os exércitos de
baratas, baratinhas, baratões, num assanhamento de carnaval. E é
monstruoso, é por completo inexplicável este amor entre baratas, coberto
destas baratas que qualquer calorzinho põe doidas, avançam pelo bairro,
cruzam lépidas a esquina, invadem o arranha"-céu.

Gasto mais de metade do meu ordenado em venenos contra as baratas. Vivo
sem elas, mas só eu sei o que isto me custa de energia moral. Altas
horas, quando venho da noite, há sempre uma, duas baratas ávidas, me
esperando. Se abro a porta incauto, perdido nos pensamentos insolúveis
desta nossa condição, isso elas dão uma corridinha telegráfica, entram e
tratam logo de esconder, inatingíveis. Eu sei que, feito de novo o
escuro no apartamento, elas irão morrer se banqueteando com os venenos
que me custam a metade do ordenado. Mas me vem uma saudade melancólica
dos meus ordenados inteiros, dos livros que não comprei, dos venenos com
que não me banqueteei. Pra dar banquete às baratas. Às vezes eu me
pergunto: por que não mudo desta esquina?\ldots{} Mas sempre o meu pensamento
indeciso se baralha, e não distingo bem se é esquina de rua, esquina de
mundo. E por tudo, numa como noutra esquina, eu sinto baratas, baratas,
exércitos de baratas comendo metade dos orçamentos humanos e só
permitindo até o meio, o exercício da nossa humanidade. Não é tanto
questão de mudança. Havemos de acabar com as baratas, primeiro.

\chapter[Congresso Nacional de Língua Cantada]{Congresso Nacional\\ de Língua Cantada}
\hedramarkboth{Congresso Nacional\ldots}{}

O que mais importa verificar a respeito do Congresso de Língua Nacional
Cantada, realizado ultimamente em São Paulo, pelo Departamento de
Cultura, é o seu espírito de iniciativa. Em várias manifestações o
Congresso foi primeiro no Brasil, como início de trabalhos. Foi o
primeiro congresso musical num país em que a música já alcançou
esplêndida qualidade e tem numerosíssimos cultores. Desconfio mesmo que
foi o primeiro da América do Sul.

Também foi este Congresso o primeiro na América do Sul a cuidar do
estabelecimento de uma língua"-padrão para as artes de dizer. Pouco
importa seja ela adotada ou não. O que interessa em principal é
verificar o início de uma ordem, dantes inexistente, de preocupações
estéticas.

Por outro lado, desde os tempos da Imperial Academia de Música e Ópera
Nacional, certos jornais denunciavam a necessidade urgente de se
estabelecer a pronúncia cantada da língua nacional. A preocupação
existia, pois, mas só agora, quase um século mais tarde, é que se fez o
primeiro esforço organizado para conseguir tão grande elemento de
progresso musical.

São estas primazias de máxima importância artística, que dão um impulso
incontestável à música, às artes de dizer e ao canto em língua nacional.

Depois dos tempos imperiais em que a citada Academia levou em nossa
língua óperas estrangeiras, e a Traviata se chamava Transviada, nunca
mais grandes obras foram traduzidas e executadas aqui, que eu me lembre.
Hoje, com exceção da língua portuguesa, não há língua europeia
universalizada em que não se cante Bach traduzido. O Congresso atirou"-se
justamente a uma das obras capitais do grande polifonista, e
apresentou"-lhe o Actus Tragicus, pela primeira vez em língua nacional.

Essa primeira tradução, eu sei, já está provocando imitações de enorme
interesse, como seja, a da Ode à Alegria, de Schiller para execução da
Nona Sinfonia de Beethoven, em língua nacional.

Dos trabalhos provocados pelo Congresso ou apresentados a ele, vários
foram primeiros. Em primeiro lugar estão os registos, em 16 discos, das
variantes regionais de pronúncia nacional, feitos pela Discoteca
Pública, de colaboração com Manoel Bandeira e o prof. Antônio de Sá
Pereira. Creio que esta pesquisa fonética é a primeira da América do
Sul, tendo coincidido em data com outra idêntica, realizada na Alemanha.

Ainda primeiros no gênero, pelo menos no Brasil, são os mapas
geográficos de variações linguísticas, apresentados pela Sociedade de
Etnografia e Folclore. São ainda estes mapas, as primeiras manifestações
de cartografia folclórica que se realizam entre nós.

Os estudos sobre pronúncia nacional cantada e o problema do nasal
brasileiro, apresentados pela Discoteca Pública tomando por base a
discografia brasileira existente, são também os primeiros que se
realizaram na América do Sul.

O trabalho de fonética experimental aplicada a vozes brasileiras,
apresentado pelo prof. Roquette Pinto, também é o primeiro no gênero,
entre nós.

Estas são as principais manifestações do espírito de iniciativa que
caracterizou enormemente este Congresso. Outras houve, estudos fonéticos
de pronúncia regionais de vários Estados, estudos de musicologia sobre
assuntos ainda não percebidos entre nós.

Não cabe nesta reportagem salientar a validade intrínseca desses
trabalhos e tese. Isso me levaria longe e para dentro da crítica de que,
graças a Deus, me afastei. Os Anais, já em vias de publicação, darão a
todos o direito, excessivamente humano talvez, de ajuizar desse valor.
Quis apenas, aqui, salientar a importância social que teve o Congresso
rasgando um campo à iniciativa que, sob certos pontos de vista, música,
artes de dizer, fonética, musicologia, folclore, pôs o Brasil, pela
primeira vez, dentro de certas realidades das ciências e das artes, de
que ele até agora estivera ausente.

\chapter{Pintura nova}

O veterano pintor Campofiorito, dispondo por algum tempo da sala de
exposições da Associação dos Artistas Brasileiros, teve a boa ideia de
convidar para apresentarem seus trabalhos, de parceria com os dele,
alguns discípulos do pintor Portinari. Saliente"-se de passagem a alta
generosidade de um artista, suficientemente liberto de rivalidades,
capaz de proporcionar a discípulos de outrem a ocasião de se mostrarem
em público. Resultou de tudo isto uma exposição muito curiosa em que, a
par da firmeza veterana de Campofiorito e sua esposa, apareceram três
valores novos: o paulista Ruben Cassa, o carioca Aldary Toledo, e o
italiano Enrico Bianco. Não quero profetizar nada, tanto mais que dois
destes pintores, Ruben Cassa e Aldary Toledo já têm profissão definida e
provavelmente continuarão pintando apenas nas horas vagas, sem aquela
mais honesta fatalidade do profissionalismo, que obriga a mais profundas
confidências; mas tenho a sensação muito nítida de que qualquer destes
três artistas poderá alcançar honrosa posição em nossa pintura.

O mais inquieto deles é Ruben Cassa. É também dos três o que demonstra,
por agora, não menos valor, mas maior indisciplina de personalidade.
Apresenta"-se por isso mais experimentador que os outros dois e mais
sujeito a influências. Dos quatro óleos que mostra, dois revelam
imediatamente a admiração do moço pelo seu mestre, com processos de
composição e elementos de lirismo característicos de Portinari. Nas
naturezas mortas, porém, já o artista se apresenta mais pessoal e mais
independente, embora uma delas, com bastante felicidade, se aproveite de
certas colorações e processos de Vlaminck. O quarto trabalho, último em
data e reproduzido nesta página, é já um trabalho de excelente
qualidade. A segurança do desenho, a riqueza intensa do colorido se
equilibram numa profunda harmonia e numa intimidade raras. Trata"-se
positivamente de um belo trabalho.

Aldary Toledo será por ventura dos três rapazes o melhor desenhista.
Apresenta"-se, fazendo valer a sua qualidade, com uma série de desenhos
em preto e branco, em sanguino ou coloridos a pastel, e três óleos.
Ótimos desenhos, de muita segurança de traço e dosagem do claro"-escuro.
Dão porém a impressão de muito academismo, por enquanto, belos
incontestavelmente de fatura, mas pouco livres e impessoais. Mas há
neste jovem artista uma natureza generosa muito sensível, que o faz
entregar"-se aos seus temas e modelos, conseguindo deles, certas vezes,
muito caráter. Salientam"-se a meu ver, na sua colaboração, a forte
cabeça de Mestiço, brutal, vibrante de uma insensibilidade quase
grotesca, ou pelo menos quase perversa e uma excelente paisagem com
operários. Nesta, a uniformidade rítmica das figuras do primeiro plano,
a claridade crua do conjunto, a gravidade quase trágica da composição
formam um todo excelente, muito expressivo e muito seriamente belo.

Enrico Bianco é, dos três rapazes, o que me parece mais firme de sua
técnica e orientação estética. Muito cuidadoso na fatura, chegando mesmo
a carícias de um miniaturista, todos os seus trabalhos apresentam uma
feição segura de acabamento. As suas obras nos dão, todas, uma impressão
muito cômoda de facilidade, mas não creio que a facilidade seja um
valor, nem mereça elogio. Pelo contrário, as mais das vezes é um escuso
perigo, levando o artista a se entregar a ela e esquecer"-se de si e das
graves exigências da arte. Com efeito, Enrico Bianco fica por vezes a
dois milímetros apenas do decorativo. As linhas do seu desenho são tão
afetivas, as suas cores isoladamente tão gostosas e suas combinações tão
epidermicamente felizes, que todas as obras deste moço encantam logo à
primeira vista. Enrico Bianco, se quiser, poderá a qualquer momento
conhecer um largo sucesso público com os seus deliciosos retratos. Mas
uma visão mais refletida e exigente saberá em pouco tempo distinguir o
que é facilidade instintiva e o que é valor. Sinto a obrigação de
confessar mesmo um certo temor pelo destino deste artista que em nenhum
momento de sua vida deverá esquecer que o bonito é o maior inimigo do
belo. Em todo o caso, Enrico Bianco apresenta, entre suas obras, um
retrato de homem de uma matéria muito saborosa e construído com um
carinho e uma honestidade verdadeiramente filiais. É que o jovem artista
estava retratando seu pai, e talvez a incumbência que se dera de
reproduzir na tela uma figura muito venerada, não tenha sido estranha à
severidade e à profundeza desse ótimo quadro.

\chapter{Obras novas de Cândido Portinari}
\hedramarkboth{Obras novas\ldots}{}

Desde a conquista de um dos prêmios Carnegie, a obra do nosso pintor
Cândido Portinari não cessou mais de preocupar a atenção de revistas e
críticos norte"-americanos. Ainda recentemente o sr. Rockwell Kent, sem
dúvida alguma uma das mais acatadas autoridades americanas na crítica de
artes plásticas, dava ao pintor brasileiro as provas mais
desinteressadas e úteis da sua admiração. Em fevereiro passado, a
revista Life mandava buscar obras de Portinari para reproduzi"-las em
policromia. Os quadros enviados, conforme carta recentíssima, causaram
tamanho interesse, que de Nova York solicitaram os preços de todos eles.

Foi portanto escolha muito acertada a que fez o Comissariado brasileiro,
encarregado de organizar o pavilhão do Brasil na Feira Internacional de
Nova York, encomendando ao artista os três painéis que irão decorar o
pavilhão. Foi assim que, interrompendo a feitura dos afrescos que está
fazendo no edifício novo do Ministério da Educação, Cândido Portinari
executou os três vastos painéis, inspirados na vida popular brasileira,
que aqui se reproduzem.

Apaixonado como está atualmente pela pintura mural, o primeiro cuidado
do pintor foi criar, por processos especiais de preparação da tela e
escolha de tintas, um material que, sem imitar o afresco, apresentasse
características idênticas de força, peso e monumentalidade. Conseguiu"-o
plenamente. As reproduções só muito vagamente podem dar a impressão da
qualidade material destas obras novas do pintor. A matéria se apresenta
forte e áspera, epidermicamente arenosa, mas com uma profundeza dura,
compacta, de uma intensidade dramática, um verdadeiro achado.

As pinturas se apresentam com aquela mesma grande liberdade de criação
com que, já nos afrescos do Ministério da Educação, o pintor está
criando as suas paredes, tomando quase que exclusivamente como ponto de
referência os desenhos e cartões que para elas fizera. Esse é o processo
que Cândido Portinari sistematizou atualmente para a criação das suas
obras de grande tamanho. Estuda"-lhes primeiro, com a maior paciência, a
composição e os detalhes. Para os afrescos do Ministério, todas as
figuras foram separadamente estudadas com verdadeira minúcia anatômica.
Depois desses estudos longos, e que formam às vezes coleções esplêndidas
de desenhos perfeitamente clássicos, Portinari como que faz tábua rasa
de tudo quanto desenhou e compôs. Liberta"-se de tudo e pinta livremente,
sem se preocupar de perspectivas, sombras e realidades anatômicas. Às
vezes, um detalhe de inspiração primeira, rápido, impressionista, ele o
reproduz minuciosamente, conservando"-lhe toda a rápida espontaneidade.
Ora obedece, ora desobedece à perspectiva ou ao sistema das sombras.
Pinta livremente enfim. Obras como estes painéis da Feira de Nova York
são verdadeiros compêndios de composição pictórica, em que concorrem
desde a figura anatomicamente clássica até a sombra voluntariamente
errada. Com isto, embora empregando as três dimensões pela acentuação
dos volumes, Portinari consegue uma unidade bidimensional
impressionantemente exata, a superfície vibra sem o menor
enfraquecimento. Por outro lado, como é fácil de observar pelas
fotografias, usando sempre a perspectiva pela gradação de tamanho das
figuras, o pintor recusa a fuga dos chãos, e compõe os planos de base em
que se apoiam as figuras, estritamente dentro de duas dimensões, por
meio de massas mais claras centralizadoras da composição.

O que é extraordinário é, dentro dos seus princípios e pintando sempre
com vagar, o pintor conservar intacta a impressão de impetuosidade
criadora e de entusiasmo lírico que as suas obras atuais nos dão. Nada é
frio nestas obras novas, tudo arde vigorosamente, dentro da composição
mais segura, conservando um ar de improviso, de movimento, de vida, que
nos faz lembrar Rubens e o Greco. O painel da jangada, principalmente, é
de um lirismo, de uma beleza de colorido e de formas, que lhes dão uma
qualidade rara.

E a tudo isto deveremos ajuntar os dons de caracterização destas obras.
São quadros que só poderiam ser concebidos por alguém profundamente
brasileiro. Não apenas os costumes, tudo é nosso, o ar, o cheiro, o
clima destes painéis. Aquela tradição que Almeida Junior quis abrir, só
agora parece retomada por este pintor, que em vez de perder tempo em
buscar a cor do nosso céu, está verdadeiramente fazendo obra de
sentimento nacional.

\chapter{A exposição Machado de Assis}

Realmente o centenário de Machado de Assis deu lugar a uma das mais
significativas manifestações de coesão nacional de que se tem notícia.
Ninguém poderia supor, no início do ano, que a figura do genial criador
do Brás Cubas comovesse tanto a alma nacional, mas a intelectualidade
toda do país cerrou fileiras em tomo do morto. E o mais admirável, o que
demonstra realmente que a inteligência brasileira está em profundo
adiantamento cultural e social, é que as manifestações quase todas, com
exceção evidentemente das acadêmicas e de algumas oficiais, nada têm de
apologéticas e patrioticamente servis. É um culto esclarecido, em que o
carinho, a atenção ambiciosa em torno de qualquer gesto ou obra de
Machado de Assis, não exclui a clarividência de julgamento, as censuras
a certas atitudes sociais ou filosóficas do grande humorista, e
restrições à obra dele.

A própria Exposição Machado de Assis, organizada oficialmente sob a
direção do jovem Instituto Nacional do Livro, com a colaboração da
Biblioteca Nacional, que lhe emprestou local e documentos, é uma
exposição de caráter crítico, o que pela primeira vez se faz no Brasil.
Aliás, sob vários pontos de vista ela é uma novidade entre nós.
Planejada conjuntamente por Augusto Meyer, um dos mais argutos críticos
de Machado de Assis, e pelo arquiteto Oscar Niemeyer, ela é de uma nobre
expressão. A beleza das cores dos painéis e sua sóbria decoração, a
escolha dos documentos mais característicos e raros, a sugestividade das
suas divisões e das fotografias apresentadas, a comodidade do olhar e da
compreensão imediata, bem como a transformação quase miraculosa de um
recinto imprestável, tornam esta exposição uma réussite completa.

Convertido o saguão da Biblioteca mais ou menos num recinto circular,
este se divide em gomos, cada qual correspondente a uma seção do
conjunto. Estas seções se referem à ``Infância'' de Machado de Assis, à
sua ``Formação'', ``Vida íntima'', ``Maturidade'', ``Crepúsculo'',
``Consagração'' e ``Obra''. Na parte da infância, vemos o registro de
batismo, o croqui da Chácara do Livramento com o desenho da capela em
que o escritor foi batizado, além de várias fotografias desse morro do
Livramento, que o próprio clássico evocou, muito mais tarde, na sua
frase: ``caçar ninhos de pássaros ou perseguir lagartixas nos morros do
Livramento e da Conceição, ou simplesmente arruar à toa''. Este painel é
reproduzido aqui por uma das fotografias. Ainda mais feliz e evocativo é
o painel seguinte, ``Formação'', também reproduzido numa das fotografias
deste suplemento. É o período em que o mestre se ensaia nas letras. Além
da documentação original, contida na mesa vidrada, vemos na parede as
reproduções ampliadas, dos primeiros amigos e mentores do escritor,
Paula Brito, Manoel Antônio de Almeida, José de Alencar, Francisco
Otaviano, Bocaiúva, Faustino Xavier de Novaes, a quem ele dedicaria um
dos seus poemas mais admiráveis, além dos jornais em que colaborou no
tempo, em especial a Marmota Fluminense. Quanto à ``Vida íntima'' quase
nada. O gênio que detestava os derramados, nos deixou muito pouco do que
foi no lar. Ele mesmo afirmava que ``a descrição da vida não vale a
sensação da vida''. Um belo retrato de Carolina, no tempo em que se
casou com o poeta em 1869, outro do Machado de Assis dessa mesma época,
e numa redoma, o seu tinteiro, a caneta, o ``pince"-nez''. Chegamos então
à maturidade. Uma vista do Ministério da Viação de que ele foi
funcionário até morrer, uma admirável evocação da Academia que ele
fundou e de que se tornou o protótipo (nas coisas boas\ldots{}). O fac"-símile
da primeira página das Memórias Póstumas de Brás Cubas, e frontispícios
de jornais em que colaborou então, A Gazeta de Notícias, A Estação.

No ``Crepúsculo'', vemos Machado de Assis cultuado em vida e feliz. Um
delicioso retrato de Carolina já grisalha, outro do clássico, o ramo do
carvalho de Tasso, e a horrível máscara mortuária que nos enche de
pavor. Qualquer coisa de um mico inchado que se risse. Um derradeiro
traço de humorismo. Mas este foi a morte quem fez\ldots{} Na ``Consagração'',
coisas mais frágeis, a se excetuar os livros até agora escritos sobre
Machado de Assis. Algumas das traduções de obras dele para o francês e o
italiano, e a pavorosa estátua que, além de outros defeitos de feia,
entope a deliciosa fachada do Petit Trianon. E enfim a parte mais
audaciosa da exposição, em que os organizadores dela tomam uma atitude
exclusivamente crítica, e se reproduz aqui. Aí vêm de qualquer forma
evocados os gênios que mais influenciaram a formação técnica e
espiritual do nosso gênio literário, Camões, Frei Luiz de Souza,
Garrett, Pascal, o Eclesiastes, Daniel Sterne, além das obras do nosso
mestre principal. E talvez em contradição com alguns desses nomes
evocados, vem ainda a frase de Machado de Assis, dizendo que ``de todas
as coisas humanas, a única que tem o seu fim em si mesma é a arte''.
Essa arte pela qual ele tudo sacrificou, até o equilíbrio da sua
humanidade\ldots{}

\chapter{Eros Volúsia}

O caso da bailarina Eros Volúsia é de enorme interesse. Filha de Gilka
Machado, Eros Volúsia traz nos músculos e no sangue aquele mesmo grito,
aquele mesmo ímpeto de um Eros singularmente dionisíaco, que deu à
poetisa os mais ardentes versos femininos de nossa língua. Mas Eros
Volúsia não baila sobre ritmos verbais, como a sua ilustre mãe, e
poderemos dizer que felizmente, pois é lei quase sem exceções, na
hereditariedade artística, que filho de peixe não sabe nadar. Eros
Volúsia preferiu os ritmos plásticos, e é uma bailarina verdadeira, que
tem a sua dança naquela mesma necessidade mística do gesto imitativo
pela qual, no dizer de alguns etnógrafos, o baile foi a primeira
expressão estética dos homens sobre a terra. Para Eros Volúsia, dança e
religião se confundem; e deve ser por isso, talvez, que em todas as suas
criações, pelo menos as que já vi, ela vai num graduando de exaltação
admirável e atinge sempre verdadeiros estados de possessão mística,
verdadeiros paroxismos frenéticos em que, sem perder o equilíbrio e o
controle artístico, ela alcança uma expressão viva de realidade.

Respeitosa de sua arte, Eros Volúsia aprendeu o baile clássico, e faz
questão de mostrar que o conhece, como na visão ``Ânsia de Azul'' que
nos deu no seu último espetáculo do Teatro Ginástico. Acho que a artista
fez bem estudando os recursos do baile clássico e que o deve praticar
sempre em seu treinamento pessoal. Não tenho a impressão porém que deva
insistir nele nas suas apresentações públicas, porque é inútil a gente
fantasiar"-se daquilo que não é, a imitação transparece, e através dos
esforços a gente se dispersa na monotonia dos pastiches.

Porque Eros Volúsia é essencialmente uma bailarina brasileira, uma
expressão nacional de bailado, e este é o seu grande mérito, que ninguém
lhe poderá mais tirar. Foi ela a primeira a tentar sistematicamente a
utilização artística da nossa mímica coreográfica popular, e a transpor
sambas, maxixes, maracatus, danças místicas de candomblé, e até mesmo
ameríndias para o plano da coreografia erudita. E é incontestável que o
faz com muita inteligência. Algumas das suas criações são deliciosas de
espírito, ricas de ritmos plásticos inéditos no bailado teatral, e de um
sabor brasileiro marcado.

Suas danças ameríndias foram aliás as que menos me impressionaram, mas
força é confessar que o problema aqui era dificílimo. Além da artista só
ter podido colher pessoalmente pequena documentação, e esta ainda ser
muito fraca através das descrições dos viajantes e da cinematografia, a
dança ameríndia, não tendo jamais passado do plano interessado para o da
coreografia pura, não fornece dose suficiente de caráter e de elementos
transponíveis para o domínio da virtuosidade. É nos batuques, na
interpretação dos passos e mímicas do antigo maxixe ou do atual frevo,
nas movimentações místicas das macumbas cariocas ou dos maracatus
pernambucanos que a artista se toma positivamente de maior interesse e
verdadeiro espírito criador. Ora a vemos evocando o antigo Lundu, numa
graça mestiça que lembra delicadamente os desenhos de Rugendas e Debret.
Ora no terreiro de Umbanda, aos chamados do pai de santo o espírito de
Xangô, deus do raio e do fogo, desce sobre ela, e Eros samba com
verdadeiro esplendor muscular, em pinchos e meneios de um frenesi
endemoninhado. Ora ela se converte na Dama do Passo, dos maracatus,
carregando a Calunga, a boneca inspiratriz, e o rito lento se move com
volúpias de êxtase, interrompidos por quedas bruscas de braços, numa
primaridade tão negra, tão impressionantemente grave, que se desdobram
em beleza aos nossos olhos, expressões que nos atiram para as formas
larvares, iniciais da vida. Eros Volúsia está agora no dever de criar
escola, para que a chama, o destino que ela traz consigo transborde de
sua própria vida, formando uma tradição que nos será fecunda.

\chapter{Será o Benedito!}
\hedramarkboth{Será o Benedito\ldots}{}

A primeira vez que me encontrei com Benedito, foi no dia mesmo da minha
chegada na Fazenda Larga, que tirava o nome das suas enormes pastagens.
O negrinho era quase só pernas, nos seus treze anos de carreiras livres
pelo campo, e enquanto eu conversava com os campeiros, ficara ali, de
lado, imóvel, me olhando com admiração. Achando graça nele, de repente o
encarei fixamente, voltando"-me para o lado em que ele se guardava do
excesso de minha presença. Isso, Benedito estremeceu, ainda quis me
olhar, mas não pôde aguentar a comoção. Mistura de malícia e de
entusiasmo no olhar, ainda levou a mão à boca, na esperança talvez de
esconder as palavras que lhe escapavam sem querer:

--- O hôme da cidade, chi!\ldots{}

Deu uma risada quase histérica, estalada insopitavelmente dos seus
sonhos insatisfeitos, desatou a correr pelo caminho, macaco"-aranha, num
mexe"-mexe aflito de pernas, seis, oito pernas, nem sei quantas, até
desaparecer por detrás das mangueiras grossas do pomar.

Nos primeiros dias Benedito fugiu de mim. Só lá pelas horas da tarde,
quando eu me deixava ficar na varanda da casa"-grande, gozando essa
tristeza sem motivo das nossas tardes paulistas, o negrinho trepava na
cerca do mangueirão que defrontava o terraço, uns trinta passos além, e
ficava, só pernas, me olhando sempre, decorando os meus gestos, às vezes
sorrindo para mim. Uma feita, em que eu me esforçava por prender a rédea
do meu cavalo numa das argolas do mangueirão com o laço tradicional, o
negrinho saiu não sei de onde, me olhou nas minhas ignorâncias de
praceano, e não se conteve:

--- Mas será o Benedito! Não é assim, moço!

Pegou na rédea e deu o laço com uma presteza serelepe. Depois me olhou
irônico e superior. Pedi para ele me ensinar o laço, fabriquei um
desajeitamento muito grande, e assim principiou uma camaradagem que
durou meu mês de férias.

Pouco aprendi com o Benedito, embora ele fosse muito sabido das coisas
rurais. O que guardei mais dele foi essa curiosa exclamação, ``Será o
Benedito!'', com que ele arrematava todas as suas surpresas diante do
que eu lhe contava da cidade. Porque o negrinho não me deixava aprender
com ele, ele é que aprendia comigo todas as coisas da cidade, a cidade
que era a única obsessão da sua vida. Tamanho entusiasmo, tamanho ardor
ele punha em devorar meus contos, que às vezes eu me surpreendia
exagerando um bocado, para não dizer que mentindo. Então eu me
envergonhava de mim, voltava às mais perfeitas realidades, e metia a
boca na cidade, mostrava o quanto ela era ruim e devorava os homens.
``Qual, Benedito, a cidade não presta, não. E depois tem a tuberculose
que\ldots{}''

--- O que é isso?\ldots{}

--- É uma doença, Benedito, uma doença horrível, que vai comendo o peito
da gente por dentro, a gente não pode mais respirar e morre em três
tempos.

--- Será o Benedito\ldots{}

E ele recuava um pouco, talvez imaginando que eu fosse a própria
tuberculose que o ia matar. Mas logo se esquecia da tuberculose, só
alguns minutos de mutismo e melancolia, e voltava a perguntar coisas
sobre os arranha"-céus, os chauffeurs (queria ser chauffeur\ldots{}), os
cantores de rádio (queria ser cantor de rádio\ldots{}), e o presidente da
República (não sei se queria ser presidente da República). Em troca
disso, Benedito me mostrava os dentes do seu riso extasiado, uns dentes
escandalosos, grandes e perfeitos, onde as violentas nuvens de setembro
se refletiam, numa brancura sem par.

Nas vésperas de minha partida, Benedito veio numa corrida e me pôs nas
mãos um chumaço de papéis velhos. Eram cartões postais usados, recortes
de jornais, tudo fotografias de São Paulo e do Rio, que ele colecionava.
Pela sujeira e amassado em que estavam, era fácil perceber que aquelas
imagens eram a única Bíblia, a exclusiva cartilha do negrinho. Então ele
me pediu que o levasse comigo para a enorme cidade. Lembrei"-lhe os pais,
não se amolou; lembrei"-lhe as brincadeiras livres da roça, não se
amolou; lembrei"-lhe a tuberculose, ficou muito sério. Ele que reparasse,
era forte mas magrinho e a tuberculose se metia principalmente com os
meninos magrinhos. Ele precisava ficar no campo, que assim a tuberculose
não o mataria. Benedito pensou, pensou. Murmurou muito baixinho:

--- Morrer não quero, não sinhô\ldots{} Eu fico.

E seus olhos enevoados numa profunda melancolia se estenderam pelo plano
aberto dos pastos, foram dizer um adeus à cidade invisível, lá longe,
com seus ``chauffeurs'', seus cantores de rádio, e o presidente da
República. Desistiu da cidade e eu parti. Uns quinze dias depois, na
obrigatória carta de resposta à minha obrigatória carta de
agradecimentos, o dono da fazenda me contava que Benedito tinha morrido
de um coice de burro bravo que o pegara pela nuca. Não pude me conter:
``Mas será o Benedito!''\ldots{} E é o remorso comovido que me faz celebrá"-lo
aqui.

\chapter{Fantasias de um poeta}

Leitor, ouve este conselho: se jamais fizeste fotomontagens, nunca te
metas neste processo novo de criação lírica. Ou de brincadeiras, se
quiseres. É tão empolgante, que em pouco tempo vira vício mais pegajoso
que outro qualquer, perderás tempo e dinheiro, brigarás com a esposa,
discutirás com os filhos, etc. Pior que futebol ou religião. É a coisa
mais apaixonante do século.

A fotomontagem parece brincadeira, a princípio. Consiste apenas na gente
se munir de um bom número de revistas e livros com fotografias, recortar
figuras, e reorganizá"-las numa composição nova, que a gente fotografa ou
manda fotografar. A princípio as criações nascem bisonhas, mecânicas e
mal inventadas. Mas aos poucos o espírito começa a trabalhar com maior
facilidade, a imaginação criadora apanha com rapidez, na coleção das
fotografias recortadas, os documentos capazes de se coordenar num todo
fantástico e sugestivo. Os problemas técnicos da luminosidade são
facilmente resolvidos, e, com imensa felicidade, percebemos que, em vez
de uma pura brincadeira de Passatempo, estamos diante de uma verdadeira
arte, de um meio novo de expressão!

Porque esta é a realidade mais inesperada e fecunda: a fotomontagem é um
processo de expressão lírica. As nossas tendências mais recônditas,
nossos instintos e desejos recalcados, nossos ideais, nossa cultura,
tudo se revela nas fotomontagens. E é mesmo natural que seja assim.
Dentro de uma centena de imagens recortadas, que estejam à nossa
disposição, dois temperamentos diversos fatalmente escolherão as imagens
que lhes são mais gratas, descobrirão combinações diferentes, movidos
pelas suas verdades e instintos. E assim, os principais criadores de
fotomontagens se distinguem facilmente; as suas personalidades divergem
e se tornam tão características como as de um poeta ou de um pintor.
Entre um criador de monstros inimagináveis como Salvador Dali e um
sereno e construtivo Jean Hugo, a distância psicológica é muito grande.

Esta página apresenta algumas fotomontagens do poeta Jorge de Lima.
Talvez não seja grande elogio afirmar que o poeta da ``Negra Fulô'' é o
maior criador de fotomontagens que temos no Brasil. Porque estes ainda
são tão poucos que não é grande mérito ser o maior deles. Mas a meu ver,
Jorge de Lima, que há muito tempo se dedica à fotomontagem, já chegou a
tal habilidade técnica e possibilidades expressivas, que pode sofrer
perfeitamente comparação com outros artistas célebres, que as revistas
estrangeiras nos mostram.

As composições fotográficas que apresento nesta página me parecem
admiráveis. O temperamento místico e profundamente compassivo do poeta
está perfeitamente expresso na mais simples destas fotomontagens, a
religiosa. Realmente nada mais sugestivo e impressionante que na aridez
trágica desses morros pedrentos, a aparição assombrada, o grito
prodigiosamente sofredor do Crucificado. Não se sabe se Ele vai surgindo
em seu martírio ou se vai desaparecendo da terra, como se desaparecesse
da memória dos homens\ldots{} Mas não quero estar fazendo sonhos e
imaginações por mim. Deixo ao leitor que contemple essa criação tão
dramática, a liberdade de imaginar por si mesmo.

A fotomontagem da mulher dormindo é uma das invenções mais encantadoras
de Jorge de Lima. Porém, dentre as composições aqui reveladas, a que me
parece mais perfeita é a da piscina. Nesta, não apenas a força inventiva
do poeta é notável, pela concatenação e o inesperado dos elementos, e o
lindo (desculpem\ldots{}) e moderníssimo monstro do primeiro plano, mas o
conjunto assume um valor artístico muito bem conseguido como
distribuição da luz. Porque a fotomontagem não deve ser apenas uma
variedade de poesia sobre"-realista, que, por princípio mesmo, não se
sujeita a nenhum controle estético; é uma arte da luz, como a
fotografia, o cinema e a pirotécnica. E realmente nesta fotomontagem a
tonalidade geral tão macia da composição, os ritmos criados pelos jatos
luminosos são tão serenos e equilibrados, que é uma verdadeira delícia
contemplar essa fotografia.

A fotomontagem é uma espécie de introdução à arte moderna. Ainda há
muita gente que não sabe olhar um quadro de Picasso ou um desenho
aguarelado de Flávio de Carvalho. Mas toda pessoa que se mete a fazer
fotomontagens, em pouco tempo fica perfeitamente habilitada a entender
certas doutrinas artísticas da atualidade e a distinguir o que há de
valor técnico em um quadro cubista e o que há de sugestividade
psicológica e sonhadora no Sobre"-realismo. Neste sentido, é bem possível
que a moda atual da fotomontagem, que está se espalhando com quase tanta
rapidez como a das palavras cruzadas, ainda venha a ser uma força de
importância para a cultura artística da atualidade.

\chapter{O homem que se achou}

Pelo número de dezembro da \emph{Revista do Brasil}, na sua crônica
sobre cinema, a romancista Raquel de Queiroz, comentando a miséria do
cinema brasileiro, pedia que se desse ao sr. Jorge de Castro ``uma
máquina e liberdade'', afiançando que este artista faria certamente
belas coisas. A autora das Três Marias estava, com muita razão,
entusiasmada com a exposição que em novembro último esse moço realizou
no salão do Palace Hotel, aqui no Rio. Pois o sr. Jorge de Castro é o
homem que se achou, desta crônica. Andou algum tempo, de Seca em Meca,
pelas artes plásticas, ora se dedicando ao desenho, ora avançando
pintura adentro, até que um dia encontrou o meio de expressão em que
podia revelar a sua sensibilidade artística. Era ainda um processo a
duas cores, porém ao mesmo tempo, ainda mais realístico e ainda mais
fantasmagórico que o desenho: era a fotografia. A sua exposição recente
foi de fato interessantíssima. Porém, preliminarmente, neste mundo torvo
das artes em que são tão numerosos os destinos errados e os seres que
jamais se encontram a si mesmos, vale que nos alegremos agora por mais
um artista que se achou.

Se eu disse atrás que a fotografia era uma arte ainda mais realística e
mais fantasmagórica que o desenho, foi aliás muito de propósito para
ressaltar as duas qualidades mais características das fotografias do sr.
Jorge de Castro. Na sua exposição figuravam uma série de retratos de
intelectuais brasileiros, paisagens e fotografias ``de gênero'', para me
utilizar da terminologia da pintura. Não estranhei não aparecerem nela
nem naturezas"-mortas, nem puros jogos de luz, nem ainda fotos de criação
super"-realista. A natureza"-morta fotográfica, sem ser propriamente um
erro, tende a roubar da pintura os seus princípios de composição e
equilíbrio, que são naturalmente impostos pelo retângulo da tela, e que
se reforçam nas exigências da multiplicidade dispersiva das cores. São
frequentíssimas as naturezas"-mortas fotográficas que mais se assemelham
a fotografias de quadros.

Ora, a fotografia é antes de mais nada um fato de luz; e apanha, a bem
dizer, campos ilimitados. Se é certo que também pelo processo
fotográfico podemos inventar livremente, provocando manifestações de luz
de nossa arbitrária invenção, creio que ninguém negará ser destino
essencial da fotografia, ser a sua fecundidade, ser a sua mensagem
infatigável, registrar a realidade enquanto luz. Deus me livre negar a
beleza de certas composições fotográficas abstratas ou super"-realistas
de alguns artistas nossos contemporâneos, nem muito menos negar o
direito de criar manifestações luminosas desses gêneros; apenas
reconheço que estas manifestações fatigam facilmente, desprovidas da
sugestividade da cor, ou confundidas com manifestações pictóricas\ldots{}
fotografadas.

Por outro lado, aquilo em que a fotografia artística se eleva sobre a
puramente documental, reside não na máquina ou na luz, como imaginam
confusionistamente os manipuladores de truques fotográficos ou os
fotografadores de eternos crepúsculos românticos, mas na criação humana
do artista. Enfim: há que ter esse dom especial de apanhar ``a poesia do
real'', como disse muito bem o desenhista Santa Rosa, justamente a
propósito das fotografias do sr. Jorge de Castro: ``Procurando na
aparência dos objetos e dos seres o seu momento de transfiguração
poética, o artista vai registrando, ora um ramo que o vento verga, ora a
superfície rugada de um velho muro, ou a dura face de um homem''. Nisso
está a característica do sr. Jorge de Castro. Fundamentalmente realista,
amando as visões da vida, ele as interpreta, porém, captando o momento e
o ângulo rico, ou compondo o ambiente em que a realidade capitula diante
da luz e se converte numa expressão sugestiva e bela. Os quatro
admiráveis retratos que ornam esta página são exemplos característicos
da arte fotográfica do sr. Jorge de Castro.

\chapter{Moringas de barro}

Às vezes o progresso vai engulindo as tradições, sem as substituir por
qualquer melhoria sensível, é uma pena\ldots{} Veja"-se o que está sucedendo
com as nossas deliciosas moringas de barro populares. Quem quer
contemple os quatro documentos apresentados nesta página verificará a
verdade desta observação.

Uma das manifestações do barro cozido, no Brasil, são as moringas com
forma de mulher. É uma tradição das mais graciosas, que daria lugar para
algumas sedentas considerações de psicanálise, não me preocupassem agora
aspectos de outra ordem, desta arte secular.

Ela se estende por quase todo o Brasil. Além dos dois exemplares aqui
expostos, o da mulher de chapéu oriundo do Paraná e o da sem"-chapéu
(este era representado pela tampa) manipulado por mãos caipiras do
município de Araraquara, ainda conheço outros documentos nordestinos,
pertencentes à coleção do pintor e arquiteto de jardins, Roberto Burle
Marx.

A manufatura popular destas moringas antropomorfas ainda permanece entre
nós, mas tende a desaparecer com a invasão das feiras rurais pela
pequena indústria, ainda popular ou, pelo menos, popularesca e de
caráter utilitaristamente urbano, das moringas de forma simples,
posteriormente enfeitadas com pinturas. Desta espécie são os outros dois
documentos apresentados. Um deles ainda é um excelente exemplar,
bastante erudito, dos tempos da monarquia. Mas o outro é um exemplar
feiosíssimo do ``progresso'' industrial moderno. O seu valor é apenas
iconográfico, por pertencer ele àquela aluvião de objetos populares
enfeitados com a extinta bandeira estadual, surgidos durante e logo após
a revolução de 32.

A diferença de princípio conceptivo entre estas moringas enfeitadas e as
esculpidas é facilmente perceptível. Nestas últimas a invenção artística
implica a própria forma útil do objeto, com ela se identificando. É um
princípio esteticamente bem mais perfeito em que arte e utilidade se
confundem na concepção mesma do objeto. Já nas duas outras
``quartinhas'', a arte é uma superfetação, senão de todo defeituosa e
desaconselhável, pelo menos abusivamente retardatária, movida pelo
exclusivo instinto de atrair. E creio que nessa manifestação ainda e
sempre de luta pela vida e vitória do mais forte que a arte humana
também é, há necessidades e interesses mais profundos e graves que a
duvidosa vitória do mais enfeitado. E então no caso da moringa de 32, o
enfeite é tão esteticamente falsário que nem sequer se orienta pelas
formas do objeto.

Os que amam, como eu, as criações do povo, pelo que elas guardam de
intensa humanidade e achados de beleza, são todos mais ou menos
desafetos do progresso. Realmente, o progresso por muitos lados é uma
coisa antipática e ilusória que se mete em tudo e tudo muda, sem muitas
vezes dirigir o homem para o aperfeiçoamento de si mesmo ou da vida. Não
é possível a gente ser contra o progresso, não seria razoável semelhante
generalização. Mas antipatizar com ele, olhá"-lo com desconfiança ou,
pelo menos, lhe guardar rancor por tudo quanto ele deturpa nas formas da
vida, é quase um instinto nos que amam verdadeiramente o povo, e o veem
roubado, pelo progresso, dos seus direitos de artista.

\chapter{Uma capela de Portinari}

Na residência de sua família em Brodowski, Cândido Portinari acaba de
pintar uma capela. O ambiente é pequenino e discreto, apenas uma sala de
proporções modestas, mas nela o grande artista criou uma das suas mais
importantes obras murais.

Ainda não se poderá dizer completamente terminada a capela, pois o
pintor ainda pretende lhe decorar a parede da porta de entrada que
defronta o altar. Talvez mesmo ainda lhe introduza algumas modificações
arquitetônicas pelo que me disse, no sentido de alongá"-la e conseguir
melhor iluminação natural. Como está, além dos elementos meramente
decorativos, como os deliciosos vasos de flores enfeitando as colunas
onde se erguem as duas interessantes imagens em gesso e cimento,
esculpidas por um dos irmãos do artista, Loi Portinari, a decoração
atual consta de doze figuras de santos em tamanho natural. Tamanho
natural dos homens, entenda"-se, que não sei que grandeza física possam
ter os santos celestiais.

O processo de pintura usado é à têmpera e é difícil explicar por
palavras todo o partido e variedade de efeitos que o artista obteve. Há
coloridos de uma intensidade prodigiosa, especialmente certos azuis,
cor, aliás, em que Portinari sempre foi mestre. Mas o Ultramar da
pintura a cola, combinado às vezes com o azul"-pavão, lhe deu toda uma
escala nova de azuis. O manto da Senhora, no grupo da Sagrada Família, é
de uma beleza esplêndida de colorido, de um ardor inesperado, assim como
o vestido de Maria na Visita a Santa Isabel, já agora aproveitando a
frieza da cor, e dentro da qual, nas gradações para o branco das
ondulações do pano, o artista conseguiu uma delicadeza de modelado de
grande habilidade técnica. Aliás, também os rostos das duas figuras da
Visita, depois das modificações que Portinari lhes fez, se não serão dos
mais místicos, ficaram magníficos pela sutileza e a graça do tratamento.
Ainda como colorido ficam inesquecíveis os pardos profundos obtidos com
vigor para os buréis do Santo Antônio e o São Francisco de Assis. Neste,
aquela voz brasileira tão irreprimível no artista deixou escapar o pio
vermelho de um cardeal, pousado na mão do santo.

Mas talvez o verdadeiro milagre de colorido está no branco com que
Portinari vestiu completamente o seu São Pedro --- um branco de uma
audácia lancinante, de pompa magnífica, de uma invenção verdadeiramente
divinatória pela grandiosidade papal que outorga ao santo. Poucas vezes
o nosso grande artista terá ultrapassado a força extraordinária com que
realizou este São Pedro. Um São Pedro brutal, enérgico, bem pescador no
corpo, barba desleixada, rosto, mãos e pés ásperos, mas de olhar severo
e duro, trabalhado de rugas e preocupações. Chega a ser bastante difícil
ao noticiarista não fazer um pouco de\ldots{} literatura sobre criações tão
expressivas como este São Pedro e o Cristo. Se a beleza da pintura é
admirável, não menos admirável é a expressividade que o artista imprimiu
a essas figuras. Como firmeza de desenho, o São Pedro alcança aquela
nobre melodia de um Nuno Gonçalves, de um Van Eyck, enquanto o Cristo,
tratado mais evasivamente no traço, pela riqueza dos entretons mais
delicados entre o marfim do rosto e o louro dos cabelos, é de uma
efusividade mística impressionante.

E é preciso não esquecer a Santa Luzia. Será o preito inconsciente do
artista aos seus olhos de pintor\ldots{} É a glorificação da beleza física e
há que nos transportarmos à Itália renascentista para encontrar santas
assim tão belas. Os roxos, os amarelos, os encarnados e azuis da
roupagem, o rosto colorido de saúde, os grandes olhos negros são uma
festa luminosa em que o artista exprimiu toda a sua felicidade de
colorista. E tudo sem desperdício, com aquela lógica cromática, aquela
energia impositiva dos afresquistas italianos.

Não estou fazendo exatamente uma crítica, não falei do São João e do
Santo Antônio encantadores, nem detalhei a realização técnica da Sagrada
Família e do São Francisco, que me pareceram um pouco desenhados em
oposição ao tratamento mais de pintura, mais livre de traços de limite
das outras figuras. Mas quis dar imediatamente notícia desta nova
criação do nosso maior pintor. É certo que se estivéssemos num país de
cultura estética mais desenvolvida, Brodowski se tomaria imediatamente
lugar de turismo e romaria de artistas.

\chapter{\textsc{sphan}}

A 6ª Região, do Serviço do Patrimônio Histórico e Artístico Nacional,
que é a de São Paulo, apresentou alguns dos seus trabalhos, na exposição
nacional aberta agora no Parque Antártica. É um alívio entrar"-se no
pavilhão arejado, onde tudo se expõe com bom gosto e clareza.

Pela exposição é possível seguir os trabalhos principais executados pela
6ª Região. Alguns deles são especialmente importantes, como as
restaurações de Embu e São Miguel, os trabalhos preparatórios para
restaurar o sítio de Santo Antônio, bem como o arrolamento geral de
arquitetura militar, civil e eclesiástica, e ainda de escultura e
pintara religiosas.

Embu e São Miguel são obras já terminadas. Não é possível, numa simples
notícia, descrever o enorme e delicadíssimo trabalho que implicam
empreitadas como estas. O restaurador, além das pesquisas e da forte
capacidade técnica, é obrigado a auscultar cada pedra, cada muro, cada
reverso de caliça caída, para descobrir o segredo de um passado mais
antigo, que a impiedade de reformas apenas utilitárias deformou. De
fato, em São Miguel como em Embu, houve transformações muito grandes, só
devidas a essa perquirição infatigável. Embu, graças a um documento
descoberto num fundo de gaveta caipira, pôde recompor a sua torre.
Perdeu aquele ar desconexo de chapéu de palha, que abobalhava a sua
fachada, readquirindo uma graça, uma humildade macia, que o branco da
cal e o azul intenso de janelas e portas ainda fazem mais encantadora.
Quanto a São Miguel, nas obras de fortalecimento da fachada lateral
direita, descobriram"-se provas decisivas de que essa fachada não fora
primitivamente um muro, mas um vasto alpendrado que uma grade de madeira
defendia do exterior. Modificou"-se muito o aspecto da deliciosa
igrejinha, tão mais inesperadamente deliciosa agora, que outras
pesquisas na pintura das madeiras provaram que antes elas eram pintadas
com um verde de aceto arseniato de cobre, divulgado no séc. \textsc{xvii} pelo
mundo. Creio se tratar de um único exemplo de igreja brasileira,
decorada com essa cor. E hoje São Miguel, com a surpresa de suas cores e
a raridade das suas soluções arquitetônicas, só espera a praça que a
livre da polvadeira absurda, para receber a última camada de cal e o
ambiente que a valorize.

Este ano a 6ª Região vai iniciar as restaurações do sítio de Santo
Antônio, em São Roque, o forte Santiago na Bertioga e a proteção do
altar de Vuturuna, que ameaça desaparecer, comido pelo cupim. Do
primeiro, a exposição oferece uma maquette do estado atual, feita para
estudos preliminares sobre a recomposição total primitiva da propriedade
de Fernão Paes de Barros. Aliás, dele e de sua beata mulher, d. Maria
Mendonça, causadora da capela que é hoje um dos tesouros mais preciosos
deste nosso Estado tão pobre de belezas tradicionais.

Não tão pobre assim\ldots{} O maior interesse da exposição apresentada pelo
\textsc{sphan} regional é justamente demonstrar que São Paulo ainda guarda muitas
e preciosas expressões artísticas do nosso passado. É certo que não
podemos nos comparar com a importância americana de Minas e a grandeza
da Bahia e Pernambuco, mas nem por isso seremos desprezíveis. Nos
diversos ``stands'' dedicados à escultura, pintura e arquitetura, uma
coleção de admiráveis fotografias arrola documentos de grande valor, que
já foram tombados. Assim as imagens de barro do baiano frei Agostinho,
as pinturas do santista frei Jesuíno em Itu, as anônimas pinturas
profanas de São Sebastião, e esse graciosíssimo forte da Bertioga, por
certo o mais diplomático e pacífico de todos os fortes. Tem"-se a
impressão de que os arquitetos do Santiago só se preocuparam em fazê"-lo
bem proporcionado e gracioso. Nem na Bahia, onde os fortes coloniais são
numerosos, se encontra exemplo tão gentil do quanto eram minuetes essas
guerras de dantes. Junto aos guerreiros de agora, Santiago é o mais
mozartiano dos blindados deste mundo.

A exposição da 6ª Região do \textsc{sphan} é um verdadeiro refúgio que, sem
alardear grosseiramente os méritos do seu trabalho, nos reverte às
nossas mais saudáveis tradições.

\chapter{Qual é o louco}

Recebi esta carta uns pares de dias. Como o autor dela não pediu segredo
posso edificar os meus leitores com ela.

Diz assim:

``Muito distinto sr.

``Sou brasileiro, nasci em Mato Grosso, nos limites do Paraguai e por
questões de herança (minha família materna é argentina), moro há perto
de dez anos em Buenos Aires. Sou, aliás, mais argentino que brasileiro,
ou antes: não sou nada.

``Por uma certa curiosidade fatigada do passado, ainda conservo,
todavia, algumas ligações com o Brasil. Assino alguns periódicos
brasileiros. Entre estes o \emph{Diário Nacional}, onde o sr. escreve.
Pude por isso notar que o seu espírito se preocupa com assombrações e
casos misteriosos. Ora eu, embora não acredite em assombrações, vivo
estes últimos tempos mui preocupado.

``Tenho um tio que resultou louco devido ao seguinte fato. Era muito
rico e solteiro. Não tendo nenhuma coisa a fazer, tornou"-se colecionador
bastante estimável, e entre alguns magníficos `faux' da coleção, se
destacavam também objetos legítimos e de indiscutível autoridade. Na
vasta residência em que meu tio morava, a escuridão constante se enchia
cada vez mais de objetos, cuja única função era serem visíveis. Pois meu
tio se especializara também em colecionar relógios"-de"-parede e alguns
possuía mui curiosos. No `hall' central da casa, peça magnífica e
discreta em que meu tio passava o maior número de suas horas, ele tivera
o bom gosto familiar de não expor nada de curioso. Algumas poltronas,
uma mesa com flores, um pequeno armário com livros eróticos, única
literatura de meu tio. Isto é, ele também tinha alguma literatura
misteriosa, livros esotéricos, manuais de espiritismo, obras que
considerava especialmente eróticas --- `o desvio' --- como ele dizia.
Nesse `hall' havia apenas três grandes relógios, as peças mais preciosas
da coleção e que com seu tique"-taque secular faziam um barulho
formidável no vazio do ambiente.

``Uma noite, precisamente às 23 horas e 46 minutos, quando meu tio lia
nesse `hall', sucedeu algo estranhíssimo. Os três relógios tinham
acabado de musicalizar a casa toda com as batidas do último quarto de
hora, quando meu tio escutou um ruído seco de metal que se quebra. Logo
depois outro. Parou a leitura. Fazia agora um silêncio tão novo no
`hall' que meu tio a princípio ficou perfeitamente desnorteado.
Custou"-lhe muito verificar o que sucedera. Afinal pôde perceber que os
três relógios tinham parado quase simultaneamente.

``Até aqui vai o que meu tio conta e que na sua tristíssima obsessão da
loucura atual, repete dez e mais vezes por dia, sendo notável que depois
de falar tem sempre um acesso furioso que o torna um louco muito
perigoso. Atira"-se a quem estiver presente e enfia"-lhe os dedos nos
olhos, dizendo que quer ver lá por dentro se `tem corda para dois dias'.

``Os freudianos daqui consideram meu tio um caso simples de
`refoulement' sexual num solteirão tímido, mas para mim o que preocupa
não é o diagnóstico, senão que a razão por que os relógios pararam. Me
entreguei ao estudo paciente do sucedido e pude verificar o seguinte: um
mês antes dessa noite maldita, meu tio recolhera, por piedade, uma velha
que fora serva da família, antigamente. Para que ela tivesse alguma
ocupação, meu tio dera a ela o serviço de dar corda aos 53 relógios da
casa. Todavia, essa era uma ocupação difícil, porque como as cordas
variavam em durabilidade, o trabalho de as renovar tinha dias e horas
certos.

``Assim os três relógios do `hall', um possuía corda para 15 dias, outro
para um e outro para dois dias. A velha que, fato importante, era uma
bruxa sincera e passava seu tempo em bruxarias, para facilitar, dava
corda todos os dias de manhã aos três relógios. É fácil, pois, verificar
que o relógio que tinha corda para 15 dias, com tensão demasiada pela
renovação diária, não resistiu, o metal delicado da corda se quebrou.

``Também se pode aceitar que o relógio de corda diária partiu a corda já
gasta, por acaso no mesmo instante que o outro de 15 dias.

``Mas, e o terceiro? O terceiro, sr.! O relógio com corda para dois dias
e que parou sem nenhuma razão de ser, sem mesmo partir a corda!

``O seu metal estava intacto e não era nem novo nem velho. Perfeitamente
normal. O relógio parou, no mesmo instante dos outros, ainda com a corda
pelo meio! E quem nos diz que parou no mesmo instante!

``A velha bruxa resultou louca também e vive rindo pacífica, dizendo
somente estas palavras: `Meus queridos vingativos'!\ldots{}

``O sr. pode bem compreender o estado de exaltação em que me acho e que
seguramente me levará ao manicômio também. Peço"-lhe encarecidamente que
me envie seu alvitre sobre o terceiro relógio. Minha vida passa
amargurada e na mais profunda e inquieta pesquisa. A única solução que
encontro é matar"-me e se o senhor não me acudir imediatamente, sinto que
me matarei.

``Aflitivamente, seu admirador,

X.X.''

\chapter{Café queimado}

Paul Morand foi a Santos exclusivamente porque queria ver a famosa
queima do café paulista. Foi e viu o que era: coisa mesquinha,
minúscula, sem grandiosidade nenhuma. Fazem montes pecurruchos do
trabalho paulista e tacam fogo. E o café vai queimando, queimando,
enfumaçado, se consumindo. Mesmo de noite o espetáculo não tem nada que
ver: dá a impressão duma quantidade, nem ela impressionante, de
fogueiras de São João, depois de festa acabada. Braseiros, nada mais!
Está claro que isso contado pelo escritor francês virou em itatiaias
horrendos numa inferneira de labaredas formidandas que nem Dante!\ldots{}
Ficou lindíssimo. E esplêndido para se contar, assim, numa conferência
literária.

Agora estão queimando café aqui mesmo, nas barbas da cidade. Da minha
casa se enxerga. Não a queima propriamente; se enxerga é o resultado do
ar, noite e dia, coisa martirizante, noite e dia, mesquinha, noite e
dia, sem grandiosidade nenhuma. Se fosse norte"-americano já tinha
arranjado um jeito elétrico de queimar todos esses milhões de sacas de
trabalho paulista duma vez só. Então ficava grandioso. O cenário havia
de se encher de jornalistas, haveria fotógrafos, e todos os jornais
cinematográficos haviam de falar e mostrar a heroica resolução
norte"-americana. Mas a queima assim noite e dia acaba mas é por deixar o
ânimo da gente esgotado duma vez, que suplício!

A gente chega de"-tarde em casa, e lá pra banda do poente enxerga a
fumaceira. É o trabalho paulista queimando. Não pensem não que estou
rebatendo sempre esse ``paulista'' por qualquer paulistanismo estaduano
e estreito. O fato seria igualmente medonho se fosse de inglês ou de
pernambucano. ``Paulista'' no caso, está por ``humano'': o que acaba com
a gente é isso: presenciar noite e dia toda essa prodigiosa massa de
trabalho humano, de esperanças de salvação, de ambições, de fadiga,
engenho, atividade, tudo sacrificado pela estupidez\ldots{} humana. Que
também aqui é estupidez paulista, convenhamos. Pelo menos em máxima
parte.

De"-tarde é a hora em que os perfumes do arvoredo e dos jardins saem na
rua para nos descansar. Porém agora no meu bairro não tem mais perfume
de flor. A tarde só cheira a café queimado, um cheiro meio de podrume
também, muito desagradável, não aquele odor sublime das torrefações.
Deste é que se devia fazer uma essência pra paulista usar. Paulista\ldots{}
Sim, aquele paulista de dantes pelo menos, que criou todo o nosso
progresso pra que depois nós, filhos desses criadores, nos orgulhássemos
daquilo que os nossos pais fizeram. E que nós não estamos sabendo fazer
mais.

Pra divertir um bocado toda esta minha amargura irritada com a queima do
café aqui perto de casa, quero citar um fato verdadeiro, passado outro
dia comigo. É apenas uma anedota, mas me parece típica da situação de
bem"-estar que os nossos criaram e não soubemos manter. Eu estava
passeando por aí de automóvel e num momento tivemos que atravessar um
campo vasto, completamente salpicado de cupim. E uma senhora paulista
que estava conosco, muito acostumada com a sua tradição e o seu bem bom,
perguntou:

--- O que é isso que está espalhado no pasto?

--- É cupim, falei.

--- Parece feno, ela respondeu com toda a simplicidade.

Não parecia não. Infelizmente aquilo só parecia cupim mesmo. O que
importa dolorosamente, no caso, é a conversão duma imagem de castigo, em
promessa vigorosa de bem"-estar, pra não dizer da riqueza. Virar cupim em
feno, só mesmo paulista duma estirpe antiga, que por todas as
aparências, não quer dar fruta mais. Gente da têmpera dessa mulher, ou
queimava todo o café duma vez, ou emperrava, não queimava, e acabava
dando certo. Porque a mais terrível das experiências históricas é que no
fim, nesse fim sem data que é o tempo que passa, tudo acaba dando certo.

Não é bom a gente pensar assim. E sobretudo é inútil\ldots{} Não vale de nada
se saber que no fim tudo dá certo, quando a gente chega em casa, altas
horas e enxerga por detrás da casa o clarão assombrado do incêndio.
Estão queimando café\ldots{} A sensação fica logo tão sinistra como naqueles
dias inenarráveis da revolução de 24, em que do longe a gente enxergava
os clarões dos incêndios das fábricas lá pela Mooca, pelo Belenzinho.
Acabem com isso logo! Não se aguenta um martírio pequenino de noite e
dia! acabem com isso! A gente vai ficando também com vontade de acabar
tudo, arranjar brigas, matar! acabem com isso!

\chapter{O mar}

--- Vamos pra Santos?

--- Mas fazer o quê!

--- \ldots{} ver o mar.

--- \ldots{}então vamos.

E assim constantemente, ninguém sabendo, feito os amantes clandestinos,
vou para Santos ver o mar.

A primeira vez que vi o mar, não me esqueço. Parece que isso ficou
dentro de mim como uma necessidade permanente do meu ser. Teria uns dez
anos, se tanto, estava no Ginásio N. S. do Carmo. Quando chegou o tempo
de junho, uns parentes próximos alugaram uma casa muito grande na praia
do José Menino, e na mesa falou"-se vagamente num convite. E na precisão
que todos tínhamos de ir ver o mar. Senti um frio ardente em mim, não
sei, o mar das pinturas, verde"-limão, com a praia boa da gente disparar.
Porém tudo ficara num talvez tão mole, que o mais certo era mesmo ir pra
fazenda como sempre. E a minha incapacidade de ser desistiu do mar.
Havia a esperança, mais fácil de sentir, que eram as férias pra d'aí a
duas semanas, fiquei todo nessa esperança.

Ora uma tarde, inda faltavam uns quatro dias pras férias, apareceu um
padre na classe, cochichou com o professor, e este disse alto que tinham
mandado me chamar de casa e fosse. Agarrei nos livros, papéis, lápis,
meio atordoado, sem saber o que era, crivado de olhares invejosos, na
orgulhosa petulância de que tivesse acontecido uma desgraça danada em
casa, morte de alguém, não sei. Sei que saí muito assustado, mas
competentemente feliz. Na porta do colégio estava um primo, homem de
idade, rindo pra mim com essa barulhenta complacência dos que uma vez na
vida se resolvem praticar um ato de caridade. Que fôssemos depressa pra
casa, que se eu não queria ir ver o mar, mamãe tinha deixado, o trem
partia numa hora. Fiquei simplesmente gelado de comoção. Não é questão
do sangue português, isso é bobagem, e parece mesmo que não tenho sangue
português: foi de chofre, o mar, uma grandeza longínqua, enorme, fiquei
doentio, não ia, me levavam\ldots{}

Em casa arrumavam a mala, havia pressa, mas francamente eu estava numa
ventura mais dolorosa que feliz. Nenhuma ideia do mar tomara até então
importância, nem na minha vida de família nem nas histórias que eu vivia
em mim. Era mesmo tamanha a insuficiência de dados, de noções, de
concepções do mar em mim, que naquela prodigiosa inércia em que eu
ficara, meio abobado, a única ``realidade que me sustentava a angústia
assombrada, eram as palavras: o mar\ldots{}

``O mar\ldots{} o mar\ldots{}'' estava ecoando meu pensamento, sem mais nada, mas
em bimbalhes formidáveis porém. Entregue inteiramente à prática do meu
primo que me levava, tomei o trem sem ver, viajei sem ver. É uma
tendência minha, muito pouco provável pelo que tenho sido publicamente,
mas é mesmo uma tendência minha, isso de sentir sempre um respeito quase
religioso por qualquer espécie de grandeza. Na minha vida particular
acabei por tomar uma decisão enérgica: me afastar sistematicamente de
todos os indivíduos altamente colocados, porque a presença deles me
insulta muito, tenho porte de escravo, uma coisa qualquer me obriga a
uma atitude de preestabelecido respeito, que é quase subserviência. Até
quando os graúdos falam alguma coisa razoável, com que é inteligente
concordar, meu apoiado me fere como se fosse um consentimento de
capacho. Creio que achei a explicação do assombro angustioso em que
estava, naquela viagem pra ir ver o mar. Era o mar na realidade a
primeira grandeza conceptivamente entendida por minha meninice, com que
eu ia me encontrar na vida. Positivamente viajava sem ver. Não me fica
dessa viagem senão a contrariedade topada em Santos. Rebentara uma greve
qualquer, ou coisa assim, os bondinhos de burro não estavam circulando.
Já era tarde avançada, e meu primo, um sovina de marca maior, nos fez
batermos a pé pro José Menino.

Só noite fechada chegamos junto da praia, que cheiro! Não me
desfatigava, entontecia muito, mais ruim que bom naquele primeiro
encontro. A praia parecia acabar perto das casas, comida pela noite
nublada. E o mar era aquele ruído incessante, grandiosamente ameaçador,
que rolava sem pressa na escureza. Me mandaram dormir mas dormi mal.
Acordava a todo momento, e levantava a cabecinha do travesseiro, pra
haurir no silêncio da casa dormida todos os sintomas do mar. Havia
sempre o ruído, é certo, mas os parentes conhecidos, o conforto da casa,
me decepcionavam vagamente, como seu eu tivesse imaginado que de noite é
que todo mundo devia estar em guarda, aos gritos erriçados, em frente do
mar. E só no retardado dia seguinte, enfiado numa roupa"-de"-banho
emprestada, me avistei com o mar. Tenho essa primeira presença dele até
hoje nos meus olhos, um mar pardo, cor das nuvens muito baixas, com
frio. Mas toda a criançada estava alegríssima com as brincadeiras do
banho, e eu era o novato, mimado por todos por isso, na superioridade
real que tinham sobre mim, não eram hóspedes e já tinham tomado muitos
banhos de mar. Toda a comoção do encontro se diluiu por isso numa
criançada, em que tomei um conhecimento excessivo das ondas, pra que
continuasse respeitando o mar.

Eu sei que essas recordações sem interesse, fazem pouco caso deste meu
cantinho de jornal que, como toda a folha, é destinado aos leitores. Mas
sinto que algum dia tinha mesmo que fazer parte aos outros destas
confissões. Gosto muito do mar; e é junto dele, nalguma praia lá do
Nordeste, que pretendo morar.

