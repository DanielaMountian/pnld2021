\chapterspecial{Vida e obra}{}{Rodrigo Jorge Ribeiro Neves}


\epigraph{Na rua Aurora eu nasci\\
Na aurora de minha vida\\
E numa aurora cresci.\\[5pt]
No largo do Paiçandu\\
Sonhei, foi luta renhida,\\
Fiquei pobre e me vi nu.\\[5pt]
Nesta Rua Lopes Chaves\\
Envelheço, e envergonhado.\\
Nem sei quem foi Lopes Chaves.\\[5pt]
Mamãe! me dá essa lua,\\
Ser esquecido e ignorado\\
Como esses nomes de rua.}{}

\section{Sobre o autor}

\noindent{}Publicados em
\emph{Lira paulistana}, em 1945, os versos acima traçam uma espécie de
síntese lírica bio"-cartográfica da vida de Mário de Andrade, embora não
seja tarefa simples descrever em poucas linhas um indivíduo tão múltiplo
e diverso como ele, ``trezentos, trezentos e cinquenta'', como atestara
em outro de seus poemas mais famosos.

Mário Raul Morais de Andrade teve a sua aurora no dia 9 de outubro de
1893, na cidade de São Paulo, no número 320 da rua que carrega
simbolicamente a projeção e permanência de sua produção intelectual e
artística. Desde a infância, demonstrou talento para a música,
destacando"-se como exímio pianista, levando"-o a ser matriculado no
Conservatório Dramático e Musical de São Paulo ao completar 18 anos.
Como autodidata, se dedicou também ao estudo da literatura e outras
artes, mas foi por meio da poesia que se lançou como escritor, atividade
a qual se dedicou intensamente até o fim da vida.

Seu primeiro livro foi \emph{Há uma gota de sangue em cada poema}
(1917), publicado sob o pseudônimo de Mário Sobral. Nos textos de
estreia, ainda estão presentes as influências das tendências estéticas
da virada do século \textsc{xix} ao \textsc{xx}, como o simbolismo e o parnasianismo,
embora estejam esboçadas as questões que permeariam toda a sua obra
posterior. Poucos anos depois, Mário de Andrade se engajou no
modernismo, movimento que, em sua fase inicial, se oporia radicalmente a
essas estéticas anteriores, influenciado pelas vanguardas artísticas
europeias.

Em 1922, Mário publicou o livro de poemas \emph{Pauliceia desvairada},
sua obra"-manifesto, no mesmo ano em que trabalhava na organização de um
dos eventos mais importantes da vida intelectual e cultural brasileira
no século \textsc{xx}, a Semana de Arte Moderna, ocorrida entre os dias 11 e 18
de fevereiro de 1922, no Theatro Municipal de São Paulo, e que contou
ainda com a participação de artistas como Oswald de Andrade, Anita
Malfatti e Heitor Villa"-Lobos. Com o ensaio \emph{A escrava que não é
Isaura} (1925), Mário lançou mais um manifesto, mas, desta vez, por meio
de uma reflexão mais séria sobre a poesia moderna no Brasil.

No mesmo decênio, Mário de Andrade começou a se embrenhar pelos caminhos
da prosa de ficção. Como contista, publicou o livro \emph{Primeiro
andar} (1926), mas foram seus romances \emph{Amar, verbo intransitivo}
(1927) e \emph{Macunaíma, o herói sem nenhum caráter} (1928) que o
destacaram como prosador. Também se dedicou à crítica de artes
plásticas, de música e de literatura, além de ter se dedicado a
importantes estudos sobre a cultura popular nacional.

Como missivista, Mário de Andrade foi um dos mais prolíficos
intelectuais de seu tempo. A correspondência do escritor paulista é
volumosa e diversa tanto em sua dimensão numérica quanto no que diz
respeito aos temas e interlocutores envolvidos no diálogo epistolar, de
escritores a pintores, de folcloristas a políticos. As cartas de Mário
não são apenas os bastidores da intimidade dos seus correspondentes, mas
também espaço de memória da formação e transformação da vida cultural,
intelectual e política do Brasil no século \textsc{xx}.

Em 22 de fevereiro de 1945, um infarto no miocárdio abreviou sua vida.
No entanto, a importância de sua obra, já reconhecida por seus
contemporâneos, se ampliou ainda mais com o passar dos anos, sendo hoje
objeto de estudos de variadas perspectivas críticas e suscitando um
interesse cada vez maior dos leitores.

\section{Sobre a obra}

Esta edição reúne crônicas publicadas nos livros \emph{Os filhos da
Candinha} (1943), \emph{Táxi e crônicas do Diário Nacional} (1976) e
\emph{Será o Benedito!} (1992). Destes, como se pode notar, apenas o
primeiro foi publicado em vida pelo autor. Os demais reúnem a farta
produção jornalística de Mário de Andrade mantida em seu arquivo, no
Instituto de Estudos Brasileiros da Universidade de São Paulo. Como
confessa na ``Advertência'' do livro de crônicas que publicou em vida,
era a sua ``aventura intelectual'', em que o cronista ``brincava de
escrever''. O múltiplo Mário de Andrade tem, na crônica, um espaço
descomprometido para a escrita, mas também de confissão, fuga,
encenação, memória e de afetos que buscou salvar do esvanecimento do
tempo.

A edição de \emph{Os filhos da Candinha} foi pensada por ocasião do
projeto das \emph{Obras completas}, pela Livraria Martins Editora. O
livro reúne crônicas publicadas nos jornais paulistanos \emph{Diário
Nacional}, \emph{Diário de S. Paulo} e \emph{O Estado de S. Paulo}, e
nos magazines cariocas \emph{Movimento Brasileiro} e \emph{Revista
Acadêmica}, além da soteropolitana \emph{Letras}. Das 43 crônicas do
livro, selecionamos para esta edição 27. Mário colaborou com diversas
publicações mandando seus textos curtos ao sabor das inquietações
cotidianas. Embora o caráter lúdico de quem queria ``brincar de
escrever'', ele não se furta a expor as tensões de uma sociedade
ricamente diversa social e culturalmente, mas com problemas arraigados
lá no fundo, que se refletem nos conflitos de classe e nas relações de
poder.

As outras duas edições de onde foram recolhidos textos que trazemos aqui
são póstumas, a partir do trabalho exaustivo de pesquisadores e
estudiosos da obra de Mário de Andrade em seu arquivo de documentos
pessoais. O livro \emph{Táxi e crônicas no Diário Nacional}, organizado
por uma das maiores especialistas no escritor paulista, Telê Ancona
Lopez, e publicado em 1976, junta todas as crônicas publicadas em sua
coluna ``Táxi'' no periódico paulistano, entre 1927 e 1933. Algumas
foram selecionadas pelo escritor para figurar em \emph{Os filhos da
Candinha}.

Quanto ao livro \emph{Será o Benedito!}, de 1992, além da crônica que o
intitula, também foram resgatados para esta edição os textos em que
Mário de Andrade atua também como crítico, falando sobre pintura,
literatura, artes plásticas e patrimônio artístico e cultural. São
textos publicados no ``Suplemento em Rotogravura'', de \emph{O Estado de
S. Paulo}, entre setembro de 1937 e novembro de 1941.

As crônicas de Mário de Andrade merecem ser lidas porque retratam, de
forma simples e divertida, a diversidade da sociedade brasileira a
partir de situações cotidianas, seja por meio da ficção ou pelo registro
documental de um acontecimento. São textos ainda bastante atuais, pois
expõem as inquietações do indivíduo comum diante das transformações
observadas no dia a dia, mas que também valoriza as tradições culturais
da sociedade em que vive e sua diversidade.

Como cronista, Mário dialoga com grandes escritores de seu tempo que
também se dedicaram ao gênero, como Manuel Bandeira, Carlos Drummond de
Andrade e Rubem Braga. Entre os cronistas contemporâneos, podemos
observar relações entre os textos de Mário e os de João Ubaldo Ribeiro,
Aldir Blanc e Luiz Antonio Simas. Este, aliás, merece destaque, não
apenas pela simplicidade e qualidade de texto, mas pelo diálogo com os
estudos de cultura popular e pela valorização das miudezas do cotidiano,
temas tão caros ao universo de Mário de Andrade, que ainda exerce
profunda influência sobre muitos autores brasileiros contemporâneos.

Para o estabelecimento do texto, cotejamos todas estas edições, além de
a mais recente edição de \emph{Os filhos da Candinha}, publicada em 2013
pela editora Nova Fronteira, com texto fixado por João Francisco
Franklin Gonçalves e revisado por Aline Nogueira Marques. A grafia foi
atualizada conforme o Novo Acordo Ortográfico, mas foram mantidas as
idiossincrasias da escrita de Mário de Andrade, a fim de manter o ritmo
da frase e o sabor de sua ``aventura intelectual'' com a palavra escrita
na sua desvairada relação com o cotidiano arlequinal, mas, por vezes,
pintado de neblinas finas.

\section{Sobre o gênero}

Na definição do crítico literário Antonio Candido, ``a crônica é filha do jornal e da era da máquina, onde tudo acaba tão depressa. Ela não foi feita originalmente para o livro, mas para essa publicação efêmera que se compra num dia e no seguinte é usada para embrulhar um par de sapatos ou forrar o chão de cozinha''.\footnote{\textsc{cândido}, Antonio. ``A vida ao rés do chão''. In: \textit{Para gostar de ler: crônicas}, volume 5. São Paulo: Ática.}

Como se observa, para Antonio Candido esse gênero literário se caracteriza por sua aparição no jornal, revelando uma de suas características que mais a afasta do conto, com o que é constantemente confundida: a crônica aborda os acontecimentos do dia a dia, fatos banais, cotidianos, pequenos eventos e conversas observadas pelo cronista e plasmada, muitas vezes de forma irônica e sarcástica, nos jornais e periódicos de circulação regular.

Apesar de seu estreito vínculo com os acontecimentos contemporâneos, lastreado pela atualidade e imediaticidade dos jornais, a crônica não se limita a registrar a realidade tal qual uma notícia de jornal, pois, no que tem de brevidade na forma, ganha muitas vezes na distenção em profundidade do que aborda. Em uma crônica como ``O mineirinho'', por exemplo, de Clarice Lispector, o texto não se prende ao acontecimento imediato. Na crônica, a escritora narra uma notícia de jornal da época, o assassinato do bandido Mineirinho, morto com treze tiros pela polícia.
A violência e a brutalidade da execução, no entanto, demovem a escritora do mero relato estilizado. Ela alça sua crônica a indagações morais, éticas e existenciais que, ao cabo, não dizem respeito apenas ao Mineirinho ou ao ato de roubar e matar, mas a todos enquanto humanidade.

Sob a ótica do cronista, qualquer fato banal ganha relevância e passa a ter uma existência autônoma em relação ao seu desdobramento fatídico.
Nas palavras de Massaud Moisés, a crônica

\begin{quote}
move-se entre ser no e para o jornal,
uma vez que se destina a ser lida na folha diária ou na revista. Difere, porém,
da maneira substancialmente jornalística naquilo em que, apesar de fazer do
seu cotidiano o seu húmus permanente, não visa à mera informação: o seu
objetivo, confesso ou não, reside em transcender o dia a dia pela universalização das suas virtualidades latentes, objetivo esse via de regra minimizado
pelo jornalista de ofício. O cronista pretende-se não o repórter, mas o poeta ou
o ficcionista do cotidiano.\footnote{\textsc{moisés}, Massaud. \textit{A criação literária II}. São Paulo: Cultrix, 2006, p.\,104.}
\end{quote}

\section{Sobre nossa equipe}

Rodrigo Ribeiro Neves é crítico literário e pesquisador, com doutorado em Estudos de Literatura e mestrado em Letras pela Universidade Federal Fluminense (\textsc{uff}). Foi pesquisador visitante na Princeton University, nos \textsc{eua}, e bolsista da Fundação Casa de Rui Barbosa. Atuou como docente de literatura brasileira na Universidade Federal Fluminense (\textsc{uff}) e na Universidade Federal do Rio de Janeiro (\textsc{ufrj}). Desenvolveu pesquisa de pós"-doutorado no Instituto de Estudos Brasileiros da Universidade de São Paulo (\textsc{ieb"-usp}) e na Universidad de Alcalá, na Espanha.