
\textsc{Poemas da Negra}

\textsc{Marco de Viração}

\begin{quote}
\textsc{Improviso do mal da América}

\textsc{Momento}

\textsc{Pela noite de barulhos espaçados...}
\end{quote}

\textsc{Poemas da Amiga}

\begin{quote}
I -- ``A tarde se deitava nos meus olhos''

II -- ``Se acaso a gente se beijasse uma vez só...''

III -- ``Agora é abril, ôh minha doce amiga,''

IX -- ``Vossos olhos são um mate costumeiro.''
\end{quote}

A COSTELA DO GRÃ CÃO

\textsc{Reconhecimento de Nêmesis}

\textsc{Toada}

\textsc{Grã Cão do Outubro}

\begin{quote}
\textsc{II -- Os gatos }

\textsc{IV -- Poema tridente }

\textsc{V -- Dor }
\end{quote}

\textsc{Quarenta anos}

\textsc{Momento}

\textsc{Brasão}

\textsc{Canção}

LIVRO AZUL

\textsc{Rito do irmão pequeno}

\textsc{Girassol da madrugada}

\textsc{O Grifo da Morte}

\begin{quote}
I -- ``Milhões de rosas''

\textsc{IV -- ``}Quando o rio Madeira''

\textsc{V -- ``}Silêncio monótono,''
\end{quote}

O CARRO DA MISÉRIA

I -- ``O que que vêm fazer pelos meus olhos tantos barcos''

II -- ``Meu baralho dois ouros''

V -- ``Plaff! chegou o Carro da Miséria''

XI -- ``Enquanto isso os sabichões discutem''

XIV -- ``Vou-me embora vou-me embora''

Lira Paulistana

``Minha viola bonita,''

``Garoa do meu São Paulo,''

``Ruas do meu São Paulo,''

``O bonde abre a viagem,''

``O céu claro tão largo, cheio de calma na tarde,''

``Tua imagem se apaga em certos bairros,''

``A catedral de São Paulo''

``Agora eu quero cantar''

``Na rua Aurora eu nasci''

``Quando eu morrer quero ficar,''

\textsc{A meditação sobre o Tietê}

CAFÉ

\textsc{Café: tragédia coral em três atos}

\begin{quote}
\textsc{Porto parado} (primeiro ato -- primeira cena)

\textsc{I -- Coral do queixume}

\textsc{Câmara-Balé} (segundo ato -- primeira cena)

\textsc{III -- A endeixa da Mãe }

\textsc{O Êxodo} (segunda cena)

\textsc{III -- Coral da vida }

\textsc{O Dia Novo} (terceiro ato -- cena única)

\textsc{I -- 1}\textsuperscript{\emph{o}} \textsc{Parlato do rádio}

\textsc{VII -- Grande coral de luta}

\textsc{VIII -- O Rádio da Vitória}

\textsc{IX -- Hino da fonte da vida}
\end{quote}

POEMAS ESPARSOS E PÓSTUMOS

\textsc{Anteriores ao modernismo}

\textsc{O poema}

\textsc{Soneto}

\textsc{O }

\textsc{Epitalâmio}

\textsc{A }

\textsc{Minha epopeia}

\textsc{Eterna }

\textsc{Poemas em jornais e revistas}

\textsc{Obsessão}

\textsc{Franzina}

\textsc{Losangos arlequinais ( condensados)}

\textsc{Seção : urgente}

\textsc{A morte que ri!...}

\textsc{Poemas na correspondência}

\textsc{Noturno n° 4}

\textsc{Reza de de} \textsc{(5° )}

\textsc{em manuscritos}

\textsc{Viola quebrada}

\textsc{Canção }

\textsc{Cântico}

\textsc{Nova de dixie}

\textsc{Esboço VI}

\textsc{Traduções feitas por Mário de Andrade}

\textsc{Premier nocturne} -- Mário de Andrade

\textsc{Paysage n° 4} -- Mário de Andrade
