\chapter[Introdução, por Marcus Baccega]{introdução}
\hedramarkboth{introdução}{marcus baccega}


% 2.6.17. Cada obra inscrita deverá incluir, 
% no final do volume (10 a 15 páginas), 
% paratexto com informações que contextualizem 
% o autor, a obra e o(s) gênero(s) literário(s).



\section{Sobre as origens históricas do gênero da Távalo Redonda}

A primeira referência à Távola Redonda ocorre em uma hagiografia (narrativa
moralizante e exemplar acerca da vida de um santo) bretã, redigida em latim, a
\textit{Legenda Sanctii Goeznovii}. Tal ocorrência é antecedida por extensa
produção textual efetuada no lastro da cultura celta, especialmente nas regiões
das Ilhas Britânicas e na Armórica (território da Gália celta hoje
correspondente à Bretanha francesa). Sendo consensual a transmissão do espólio
arturiano por meio da oralidade, pensa"-se na possibilidade de uma influência de
narrativas celtas galesas sobre os ciclos de compilação, versificação e
prosificação da Matéria da Bretanha nos séculos \versal{XII} e \versal{XIII}. Neste sentido, uma
referência residual seriam os contos compilados sob o nome de
\textit{Mabinogion}\footnote{ Estes contos celtas, cujo título original galês é
	\textit{Y Mabinogi} constituem"-se de quatro ramos de narrativas, cujos
	manuscritos completos remanescentes são o \textit{White Book of Rhydderch}
	(\textit{Llyfr Gwyn Rhydderch}, c.1350) e o \textit{Red Book of Hergest}
	(\textit{Llyfr Coch Hergest}, c.1400). Um possível local de compilação destes
	contos orais seria a abadia galesa de Llanbadarn. Muitas vezes atribuídos ao
	monge local Rhygyfarch, tais escritos podem ter sido produzidos na segunda
	metade do século \versal{XI}.} 
(contos para a infância). Há, entretanto, problemas de
datação referentes a tal compilação, sendo que a associação dos
\textit{romans} de compiladores bretões como Chrétien de Troyes e Robert de Boron
aos \textit{Mabinogion} (\textit{Y Mabinogi}) enfrentaria problemas também de
definição do vetor de influência. Como as narrativas celtas apenas se deixaram
conhecer tardiamente, podem ser antes tributárias dos \textit{corpora}
arturianos que suas ancestrais.

Os \textit{Mabinogion}, provenientes da tradição medieval dos celtas de Gales
(\textit{mittelkymrische Erzählungen}), legaram à posteridade quatro textos,
“de conteúdo muito arcaico e próximo ao mito e, para o erudito vienense Helmut
Birkhan, de indubitável lugar na literatura mundial”.\footnote{ (...) \textit{Zu
Recht tragen nur vier Texte sehr archaischen, mythosnahen Inhalts und von
unzweifelbar weltliterarischem Rang den Titel: “Vier Zweige des Mabinogi”
(Pedeir Keinc y Mabinogi)}. Cf. \versal{birkhan}, Helmut. \textit{Keltische Erzählungen
vom Kaiser Arthur}. Wien: Lit Verlag, 2004, p.~33. } Esse autor assinala que
se poderia tratar, neste caso, de manuais de instrução para aprendizes de
bardos, portadores de aventuras heroicas, a serem memorizadas, que encontrariam
paralelo nas \textit{Enfances }francesas ou nos \textit{Macgnímartha} ou “Atos
dos meninos” dos celtas da Hibérnia (atual Irlanda). Esses escritos são atribuídos a uma
personagem constante de seu próprio enredo, o bardo (\textit{cyfarwydd}) celta
Dafydd ap Gwilyn (provavelmente no século \versal{VI} d.C.), que Birkhan compara ao
trovador alemão Walther von der Vogelweide. Tais fontes não compõem, todavia,
uma unidade, apesar de manterem um traço comum, justamente a presença de
elementos depois apropriados pela Matéria da Bretanha. 

 São relevantes, para os estudos arturianos, os contos \textit{Kulhwch e Owein}
e \textit{O sonho de Rhonabwy}, com notórios paralelos na cultura escrita
medieval de expressões francesa e alemã. De acordo com Helmut Birkhan, o
primeiro conto, por denotar afastamento em relação à morfologia clássica das
novelas de cavalaria centro"-medievais, é decisivo para equacionar a “questão
dos Mabinogion”. Esse escrito apresenta um indício para a especulação sobre o
possível itinerário de apropriação pelo qual teriam transitado os
\textit{Mabinogion}, se não influenciado pela recepção, nas Ilhas Britânicas,
das narrativas arturianas continentais. Apesar de presentes aos contos celtas, Artur e seus
cavaleiros são referidos por caracteres diversos daqueles assinalados nos
\textit{romans} de Chrétien de Troyes.

\textit{Kulhwch e Owein }traz ainda outra especificidade ante as novelas
centro"-medievais, qual seja, um tema arcaico, conforme o qual as aquisições de
Owein não se devem a suas habilidades heroicas, mas à ação coletiva da corte, o
que permite a aparição do rei Artur como um \textit{primus inter pares} quanto
aos cavaleiros dessa corte, destacando"-se por seus atos de heroísmo. Os
\textit{romans} da tradição continental apresentam um Artur que se aparta das
batalhas e se vê ameaçado quando ocupa a posição de protagonista. Com efeito,
os escritos romanescos arturianos revestem"-se de um caráter de rito de
iniciação à cavalaria, uma vez que a aventura é “distribuída” por personagens que
agem solitárias, aumentando seu valor por meio de seus atos. 

Além dos \textit{Mabinogion}, outros escritos de antiga tradição celta insular
apresentam referências ao rei Artur, como o \textit{Livro Negro de Carmathen}
(\textit{Das Schwarze Buch von Carmathen}), que data de cerca do ano 1000
(portanto pré"-normando), em que o monarca se faz acompanhar de Key, figurando
ambos como campeões de Hexen, ocasião em que teriam conhecido um gato
gigantesco maravilhoso. O mesmo livro relata uma pugna, nos montes que
circundam Edimburgo, entre os dois heróis e homens cinocéfalos. Da mesma forma,
em outro conto galês, \textit{O saque do inframundo} (\textit{Preideu
Annwvyn}), narra"-se a imersão do rei Artur no Além céltico (a Ilha dos Mortos,
Avalon), de onde teria trazido um caldeirão mágico e sua espada maravilhosa
\textit{Caledvwlch,} depois denominada \textit{Excallibur}, que havia estado
sob a tutela de nove virgens no supramundo (\textit{Oberwelt}).\footnote{ O nome
\textit{Excallibur} aparece em uma novela inglesa de fins do século \versal{XIII},
denominada \textit{Arthour and Merlin}. Patrick Ford propõe que
\textit{Caledvwlch }deriva do galês \textit{caled} (“duro, forte”) e
\textit{vwlch }ou \textit{bwlch} (“ponta”). Já outro autor, Heinrich Zimmer,
preconiza que a fonte da referência para a espada de Artur seria
\textit{Caladbolg}, a espada da personagem"-título do poema holandês
\textit{Fergus}. Em algumas versões, como \textit{Morte Arthure}, o rei possui
duas espadas, \textit{Clarent} (\textit{Guerra}) e \textit{Claris}
(\textit{Paz}). O nome Excallibur, por sua vez, advém de outra versão das
narrativas arturianas, conforme a qual a espada originária do rei,
\textit{Calliburnus}, teria sido partida em duas em uma batalha contra um
cavaleiro anônimo que guardava uma fonte. Aconselhado por Merlin, Artur lança a
espada partida no lago onde habita a Dama do Lago, que lhe restitui uma nova
espada, forjada a partir dos fragmentos da anterior, portanto, uma espada
\textit{ex Calliburnu}, daí se originando \textit{Excallibur}.\textit{ }Cf.
\versal{littleton}, Scott. \versal{malcor}, Linda. \textit{From Scythia to Camelot}. A Radical
Reassessment of the Legends of King Arthur, The Knights of the Round Table and
The Holy Grail. New York: Routledge, 1994, p.~190. } Tal narrativa foi
atribuída ao bardo galês do século \versal{VI} Taliesin, declamador na corte do rei
Urien (ou Urbgen), do reino celta escocês de Rheged. Em \textit{Bran, Filha de
Ll\^yr}, também se fala de uma expedição militar à Hibérnia (atual Irlanda),
comandada por Artur, com o fito de apossar"-se de um caldeirão mágico. O
caldeirão de \textit{Annwvyn}, ao lado de inúmeros outros que grassam nas
culturas celtas, e a cornucópia celta da fartura são prováveis antecessores do
Santo Graal. 

Outro exemplo da recorrência do tema arturiano entre os celtas provém de
\textit{A Estória da Condessa da Fonte}, em que se apresenta o herói Owein (ou
Ewein), filho do já mencionado Uryen de Rheged, localidade na Escócia austral,
esse último um dos três “reis benditos” (\textit{gwynderyn}) das Ilhas
Britânicas. O aludido conto também se atribui ao bardo da corte de Uryen,
Taliesin, sendo que, na \textit{Historia Britonnum}, Nennius relata que o velho
rei enfrentou o rei anglo Teodorico (que teria reinado entre
572 e 579). Esse embate celta contra os anglos foi retratado no
\textit{Lamento} (\textit{Klagenlied}) de Taliesin quando Teodorico é
denominado “flamejante” (\textit{Flamddwyn}) e referido como
\textit{Theodoricus de Bernicia}, o Dietrich von Bern.\footnote{ A mesma
descrição pode ser encontrada na canção de outro bardo celta coevo,
identificado como Llywarch Hen. Cf. \versal{BIRKHAN}, Helmut, \textit{op. cit.}, p.~21.
} Essa personagem foi referida, ainda, na \textit{Canção dos Nibelungos}
(\textit{Nibelungenlied}), como o rei dos amelungos. 

Em outra fonte, \textit{O sonho de Ronabwy} (\textit{Breudwyt Ronabwy}),
desenha"-se uma rivalidade entre Owein e Artur. Ambos são descritos disputando
um jogo de tabuleiro, enquanto seus guerreiros assassinavam"-se reciprocamente
em uma contenda. Cada movimento efetuado no tabuleiro implicava um golpe no
campo de batalha. Interessa salientar que, em \textit{Peredur, Filho de
Evrawc}, o “Castelo das Maravilhas”, objeto da demanda do herói, observa"-se
descrito como castelo do tabuleiro mágico de xadrez. 

Ainda em terras britânicas, os temas arturianos conheceram ampla difusão em
latim, em obras da Primeira Idade Média (séculos \versal{IV} a \versal{VIII}) como \textit{De
excidio et conquestu Britaniae}, do prelado galês Gildas (c. 504--570),\footnote{
Helmut Birkhan apresenta uma narrativa galesa de cerca de 1188, o
\textit{Itinerarium Kambriae}, atribuído a Giraldus Cambrensis, em que Artur
teria assassinado o irmão do próprio Gildas. O narrador semi"-anônimo ainda se
refere, em Caerlon, a primeira corte do rei Artur, à presença de um mago,
Myrddin, uma possível prefiguração do Mago Merlin. Giraldus estatui um vínculo
entre os videntes celtas de Gales e a vidente Cassandra, de Tróia, reverberando
a tendência messiânica de tais populações celtas, bem como seu desejo de
estabelecer uma mitologia das origens que os vinculasse aos troianos. Cf.
\versal{BIRKHAN}, Helmut. \textit{op. cit.}, p.p.~19, 24 e 25. } que descreve a invasão
de hordas anglo"-saxãs à Britânia romana e as tentativas de resistência da
população romano"-bretã, sob a liderança de Artorius. Após \textit{Historia
Britonnum} (c. 800 d.C.), de Nennius, há a \textit{Gesta regum anglorum }(1125), em
que William of Malmesbury apresenta Artur e seu sobrinho, Galwain, como
personagens históricas referidas à narrativa das origens da monarquia
britânica, confirmando suas virtudes guerreiras e denegando as expectativas
messiânicas acerca do retorno do rei da Ilha de Avalon. Ademais, em
\textit{Historia regum Britaniae }(1100--1155), que o erudito arturiano Volker
Mertens considera o “momento fundador” da tradição arturiana, Geoffrey of
Monmouth (1100--1155) alude, ao lado das virtudes bélicas do herói, a sua generonisade,
datando sua ascensão ao trono de Logres aos quinze anos de idade, predicando"-lhe o
mesmo estatuto de figura histórica atribuído a Carlos Magno. Este compilador
clamava ter escrito com base em \textit{auctoritates} como Nennius, o Venerável
Beda ou Gildas, além de um livro escrito “em língua britânica”, que estaria
traduzindo, entregue pelo arquediácono Walter Map. Tal prelado presidiu a sé de Oxford
e seria cortesão do futuro rei Henrique \textsc{ii} da Inglaterra.

% A contribuição fundamental de Geoffrey of Monmouth para a gesta mítica de Artur
% seria sua caracterização --- inaugural --- como conquistador celto"-romano
% contemporâneo do imperador romano do Oriente Leão \versal{I} (457 a 474 d.C.).
% Algumas fontes adicionais são a \textit{Historia Anglorum} (c. 1129), de Henry
% of Hudingdon, que situa o reinado de Artur entre 527 e 530, e o
% \textit{Chronicon Montis Sancti Michaelis in Periculo Maris}, que associa o rei
% à data de 421.

Também se encontram alusões à corte de Artur na \textit{Historia Ecclesiastica
Gentis Anglorum }(731), do Venerável Beda.\footnote{ Cf. \versal{MEGALE}, Heitor.
\textit{op. cit.}, pp.~30--1. } Não por acaso, Helmut Birkhan assinala que,
na Alta Idade Média (séculos \versal{IX} a \versal{XI}), os habitantes de Gales estavam
plenamente convencidos da existência “histórica” do rei Artur, a quem se
atribui um túmulo no elenco de sepulcros \textit{Beddeu}, constante do
\textit{Livro Negro de Camarthen}. 

Na Europa continental, o advento dos mitos arturianos, oralmente cultivados
desde o século \versal{VI} nas Ilhas, deve"-se, de acordo com a maioria dos estudiosos,
ao contato com os principados celtas britânicos a partir da Batalha de
Hastings. Os primeiros \textit{romans} em verso com estes motivos originaram"-se
na Armórica (depois chamada Bretanha), sob a pluma do clérigo intermediário
Chrétien de Troyes, na segunda metade do século \versal{XII}. Sua primeira versão de
\textit{Perceval ou Le conte dou Graal} foi redigida sob os auspícios do conde
Felipe, de Flandres, e evidências indicam que a primeira versão continuadora,
anônima, guarda relações com a Burgúndia ou a Champanha em princípios do século
\versal{XIII}, atingindo a Picardia e a região de Paris apenas décadas mais tarde. 

A segunda versão, atribuída a Wauchier de Denain, provavelmente foi composta
para Joana, a neta do conde Felipe de Flandres, entre 1212 e 1244, para quem o
mesmo já havia dedicado alguns escritos, bem como algumas hagiografias a seu
tio, o conde de Namur. A terceira versão seria redigida por Manessier, tão
incógnito como Wauchier de Denain, possivelmente para a mesma destinatária,
como narrativa de legitimação de sua pretensão ao trono de Flandres,
questionada por um nobre que se pretendia seu pai. A corte senhorial dos condes
flamengos desenvolveu a tal grau o mecenato que o historiador inglês Richard Barber incorre na
afirmação, algo temerária, de que o \textit{roman} redigido por Chrétien de
Troyes era reputado “propriedade da família governante”, em virtude das
associações dinásticas.\footnote{ \versal{BARBER}, Richard. \textit{Op. Cit.}, p.p.~28 e 29. } A propósito, um manuscrito da terceira versão, sem o
prólogo e os versos finais, teria sido compilado para João \versal{II} de Avesne, que
clamava o trono de Flandres no século \versal{XIII}, com vinculações linhageiras
referidas ao imperador romano"-germânico.

Por outro lado, o autor da quarta versão permite"-se conhecer melhor, em função
de outros escritos que o notabilizaram, como o \textit{Roman de la Violette},
dedicado à condessa de Poitou, entre 1227 e 1229. Trata"-se de Gerbert de
Montreuil, que Barber supõe frenquentador da corte régia em Paris e, de forma
muito indiciária, um ator social da cultura intermediária, que “parece ter tido
um pé em ambos os mundos, clerical e aquele dos menestréis e dos jograis”, esses
últimos em estreito contato com a cultura popular. 

A par das obras completas \textit{Le Chevalier de la Charrette }(1179), que
apresenta a infâmia dos amores clandestinos do cavaleiro Lancelot e da rainha
Guinevere, e \textit{Eric et Enide}, uma ode às proezas guerreiras da cavalaria
e uma advertência para que os bravos cavaleiros não contraiam matrimônio, sob
pena de acovardar"-se, Chrétien de Troyes falece antes de concluir
\textit{Perceval ou Le conte dou Graal}. Além das quatro continuações
mencionadas, houve ainda dois prólogos ao \textit{roman} em verso, ambos de
autoria anônima. O primeiro deles, menos extenso, identificado por Richard
Barber como \textit{O Prólogo de Bliocadran}, enfatiza a genealogia de Perceval
(Bliocadran aqui figura como seu pai), a própria linhagem sagrada de
protetores do Santo Graal. Há uma notória insistência em aspectos negativos --
assim considerados sob a perspectiva de uma normativa eclesial -- da cavalaria,
como a propensão à guerra e suas formas ressignificadas, o torneio e a justa.
Trata"-se de um alerta moralizante acerca dos perigos da guerra e de como a
demanda por glórias nos feitos de bravura redunda em traição aos ideais da
cavalaria cristã.

Já no segundo escrito, o \textit{Prólogo da Elucidação}, narram"-se aventuras
preliminares ao conto do Santo Graal, em que ocorre o roubo de copas mágicas e
o estupro de virgens habitantes de uma floresta, nas imediações de poços. A
violência se dá por parte do rei Amangons e seus guerreiros, assumindo Artur e
seus cavaleiros o dever de vingá"-las. No mais, há uma clara repetição do enredo
apresentado por Chrétien de Troyes, com a distinção de que aqui o Santo Graal,
cuja aparência ecoa a primeira e anônima continuação a \textit{Perceval ou Le
conte dou Graal}, encontra"-se bastante dessacralizado, sendo apenas mais uma
entre tantas aventuras cavaleirescas. O que esse Prólogo apresenta de inédito é
uma descrição dos sete ramos da estória da corte do rico rei Pescador, que
aludem aos sete sacramentos da ortodoxia católica, definidos desde o século \versal{XII}
e ratificados no Quarto Concílio de Latrão (1215).

A denominada \textit{Vulgata da Matéria da Bretanha} representa a primeira
prosificação pela qual passou o espólio anterior em versos, ao redor de 1220.
Abrange a sequência narrativa das novelas \textit{Estoire de Merlin},
\textit{Estoire dou Graal, Lancelot du Lac }(\textit{roman} redigido em três
livros, que ocupa mais da metade desse primeiro ciclo), \textit{La Queste del
Saint Graal} e \textit{La Mort le roi Artu. }Com efeito, detectou"-se que
\textit{Lancelot du Lac}, \textit{La Queste del Saint Graal} e \textit{La Mort
le roi Artu} foram redigidos antes de \textit{Estoire dou Graal} \textit{e
Estoire de Merlin}, cabendo a primazia cronológica ao primeiro. 

Como expõe Heitor Megale, a constituição plena do \textit{Ciclo da Vulgata}
exigia a redação das \textit{Suites} ao \textit{roman} sobre o Mago Merlin, com
as necessárias acomodações para tornar coerentes as novelas. Esse primeiro
ciclo de prosificação denominou"-se também \textit{Ciclo do}
\textit{Lancelot"-Graal}, o que desvela a fusão das massas narrativas
pertinentes ao Cavaleiro Lancelot, mais antiga, e ao Santo Graal, posterior. A
propósito, a narrativa relativa a Lancelot não figura no \textit{Ciclo da
Post"-Vulgata}. O \textit{Ciclo do} \textit{Lancelot"-Graal} conheceu incontáveis
cópias que geraram uma abundante tradição manuscrita no Ocidente europeu
medieval, o que atesta, à evidência, uma difusão ímpar, sem qualquer paralelo
conhecido, da \textit{Matéria da Bretanha }no universo medieval. No
\textit{Ciclo da Post"-Vulgata}, a \textit{Estoire dou Graal} passa também a ser
referida como \textit{O Livro de José de Arimateia}. Alguns autores referem"-se
a \textit{Lancelot du Lac}, \textit{Queste del Saint Graal} e  \textit{La mort
le Roi Artu}, em conjunto, como \textit{Lancelot en prose}, apesar de outros
empregarem tal expressão apenas para designar o \textit{Lancelot du Lac}.
Observe"-se que as expressões \textit{Ciclo da Vulgata} e \textit{Ciclo da
Post"-Vulgata} devem"-se à terminologia proposta pela estudiosa Fanny Bogdanow,
em seu ensaio \textit{The Romance of the Grail }(1966). 

O \textit{Ciclo da Vulgata}, conservado em seis manuscritos dos cem compilados
entre os séculos \versal{XIII} e \versal{XV}, foi identificado a um só autor ou compilador,
apesar da improbabilidade de se deverem todas as novelas a uma pena solitária.
Esse escriba seria o galês Gautier Map ou Walter Map, porém já há tempos é
denominado Pseudo"-Map, pois já era falecido tal compilador quando da primeira
prosificação. O primeiro \textit{roman} a integrar esse ciclo inicial de
prosificação da Matéria da Bretanha, \textit{Lancelot du Lac}, atribuído ao
suposto clérigo galês Gautier Map, foi compilado em francês. O escriba era
arquediácono da sé de Oxford e cortesão do rei Henrique \versal{II}, tendo falecido em
cerca de 1209. Os antropólogos estadunidenses Scott Littleton e Linda Malcor assinalam que o compilador exibia
bons conhecimentos da geografia da região de Poitou, parcas noções sobre aquela
relativa ao sudeste da Bretanha e praticamente nenhuma acerca de Gales. Para os
mencionados autores, existiria um consenso entre os especialistas no
\textit{Ciclo da Vulgata}: o \textit{roman} teria, efetivamente, sido escrito nas
cercanias de Poitou, em cerca de 1200--1210., combinando elementos de
\textit{Le Chevalier de la Charrette}, de Chrétien de Troyes, e de
\textit{Lanzelet}, do poeta Ulrich von Zatzikhoven, escrito entre 1194 e
1205.\footnote{ Cf. \versal{LITTLETON}, Scott; \versal{MALCOR}, Linda, \textit{op. cit.}, pp.
82--84. } O compilador, ou os compiladores, ocultou"-se sob seu nome para
atrair, em procedimento muito comum para a Idade Média, seu prestígio e a
aceitação futura de seus manuscritos. O que se pôde averiguar, posteriormente,
foi a possível autoria da \textit{Estoire dou Graal} e da \textit{Estoire de
Merlin}, atribuídas a Robert de Boron.

Já no segundo ciclo de prosificação, o do \textit{Pseudo"-Boron}, houve uma
expressiva redução da matéria narrativa, com a eliminação daquela relativa a
Lancelot, ao passo que \textit{A Demanda do Santo Graal} e \textit{A Morte do
rei Artur} foram acoplados em um único volume, reduzindo"-se a matéria do
derradeiro. O \textit{Livro de José de Arimateia} encerra praticamente o mesmo
conteúdo da versão primeira da \textit{Estoire dou Graal.} Essa segunda
prosificação, inicialmente atribuída a Robert de Boron, fez"-se conhecer como
\textit{Ciclo do }\textit{Pseudo"-Boron} ou da \textit{Post"-Vulgata}. Richard
Barber, entretanto, pondera que a matéria narrativa concernente ao Graal pode
ter sido uma interpolação advinda de outro compilador. A \textit{Estoire dou Graal} apresenta uma
introdução coerente ao principal \textit{roman}, \textit{A Demanda do Santo
Graal}. O mencionado historiador inglês propõe ainda outro argumento, afirmando
que o vínculo entre as novelas não elidiu completamente suas contradições. A
\textit{Estoire de Merlin} compunha o ciclo de prosificação de Rober de Boron
(\textit{Ciclo da Vulgata}), a que subjaz uma coerência narrativa entre as
novelas acerca de Artur e do Graal. Confirma"-se que o segundo ciclo de
prosificação ducentista deve"-se mesmo a um Pseudo"-Boron, como evidenciou Heitor
Megale. Mais uma menção introdutória ao Graal ocorre em
\textit{Lancelot du Lac}, provavelmente fruto de outra interpolação tardia.
Nesse \textit{roman}, institui"-se, pela primeira vez, a aventura cavaleiresca
para descobrir"-se quem é o virtuoso cavaleiro digno do Graal\footnote{ Cf.
\versal{BARBER}, Richard.\textit{ The Holy Grail}. Imagination and Belief. Cambridge:
Harvard University Press, 2004,  pp.~71--2. }. 

No entanto, o Graal é citado, de forma inaugural, no \textit{Roman d'Alexandre}
(c. 1170--1180), epopeia dedicada a construir um inventário mítico sobre a vida de 
Alexandre Magno.

A explanação da gesta do Ciclo Arturiano, bem como de \textit{Tristan}, deveu"-se
a autores do renome de Ferdinand Lot, Albert Pauphilet, Jean Frappier e
Alexandre Micha, porém competiu a Fanny Bogdanow a proposição de uma exegese
unitária do \textit{Ciclo da Post"-Vulgata,} como possível conversão da
\textit{Vulgata} em enredos mais breves e homogêneos. Consagra"-se uma trilogia
de \textit{romans} que não se encontram representados em manuscritos completos, mas em
inúmeros fragmentos de duas naturezas, vestígios de códices despedaçados, ou
seções narrativas inseridas em manuscritos de outros ciclos, propensos à
transmissão oral.

\section{Versões do texto}

No concernente ao texto de \textit{A Demanda do Santo Graal}, permaneceram
apenas três versões, vale citar, a bretã, a portuguesa e a alemã. Com relação a
uma possível versão inglesa, não se reteve qualquer exemplar original ou
posterior, porém sua existência poderia ser supostamente comprovada, para alguns
autores, de forma indireta, pela influência sobre a obra de Sir Thomas Malory,
\textit{Le Morte d’Arthur} (\textit{sic}), concluída em 1470 e impressa em 1485,
para o rei Henrique \versal{VII} da Inglaterra. Tal a hipótese aventada pela
estudiosa Elizabeth Bryan.\footnote{ Cf. \versal{BRYAN}, Elizabeth. ``Introduction''. In
\versal{MALORY}, Thomas. \textit{Le Morte d'Arthur}. New York: The Modern Library, 1999,
p. \textsc{ix}. } Como leciona Richard Barber, o século \versal{XV} testemunhou
uma demanda por textos mais breves e coerentes sobre o rei Artur e o Santo
Graal, a que Malory responde com um desejo de conhecimento verídico de sua história.
Neste espectro, \textit{Le Morte d’Arthur} também pode ter sido
escrito com base em um poema inglês medieval, \textit{Morte Arthure}. No texto
quatrocentista, o Santo Graal, claramente uma relíquia, aparece no contexto da
desolação do reino de Logres, tema logo abandonado, no entanto. Em que pese o
fato de que Malory procura introduzir uma visão laica a respeito do Santo Graal,
sobretudo aludindo à Crucificação e ao \textit{Evangelho Apócrifo de Nicodemos}
(fins do século \versal{IV}), aqui o Santo Graal se apresenta indissoluvelmente
vinculado ao Mistério Eucarístico e à doutrina ortodoxa da Transubstanciação.
Há, ainda, uma adaptção de \textit{A Demanda do Santo Graal}  para o irlandês,
denominada \textit{Logaireacht an tSoidigh Naomhtha}, bem como para o galês,
\textit{Y Seint Greal.}

\section{A Demanda do Santo Graal alemã (Gral"-Queste)}

Quanto ao universo cultural alemão, observe"-se que  \textit{A Demanda do Santo
Graal} \textit{(Die Suche nach dem Gral) }e \textit{A Morte do rei Artur (Der
Tod des Königs Artus)} compõem a terceira parte do Códice 147 da Biblioteca
Palatina Germânica de Heidelberg (\textit{Codex Palatinus Germanicus} 147). A
primeira parte do manuscrito contém uma adaptação do \textit{Lancelot du Lac} do
Primeiro Ciclo de Prosificação. Esses textos baseiam"-se, essencialmente, no
\textit{Ciclo da Vulgata} ou \textit{Ciclo do Lancelot"-Graal}, e não no
\textit{Ciclo da Post"-Vulgata}, em que pese situar"-se o início de sua compilação
na segunda metade do século \versal{XIII}.\footnote{ Cf. \versal{ANDERSEN}, Elizabeth A.
“The Reception of Prose: The \textit{Prosa"-Lancelot}”. In \versal{VVAA}. \textit{The
Arthur of the Germans}. The Arthurian Legend in Medieval German and Dutch
Literature. Cardiff: University of Wales Press, 2000, p.~156. } O manuscrito
alemão integral, também designado como \textit{Prosa"-Lancelot} pelos estudiosos,
apresenta lacunas. O erudito que analisou e adaptou, para o alemão
contemporâneo, a versão de Heidelbeg de\textit{ A Demanda do Santo Graal},
Hans"-Hugo Steinhoff, elucida que, no caso alemão, esse processo de prosificação
foi multissecular. A compilação de \textit{A Demanda do Santo Graal} data da
segunda metade do século \versal{XIII}, mas a integralidade do códice 147 já aludido não
se configurou antes de 1455, tendo existido interpolações do século
\versal{XVI}. O mencionado estudioso adverte, entretanto, que o texto alemão
apresenta variações e especificidades que o afastam do \textit{corpus} bretão
que embasou os trabalhos de compilação.\footnote{ \versal{STEINHOFF}, Hans"-Hugo. “Der
deutsche Text”.  In \textit{Die Suche nach dem Gral. }Heidelberg: Deutscher
Klassiker, 2004.  } Os estudiosos da versão alemã de \textit{A Demanda
do Santo Graal}, por vezes referida como \textit{Gral"-Queste}, supõem que a
mesma não teria sido adaptada para o alemão, diretamente, com base no texto
homônimo francês, mas a partir de um hipotético escrito que se teria compilado
na região do Reno em meados do século \versal{XII}, em alemão ou
holandês.\footnote{ Cf. \versal{JACKSON}, W.H.;  \versal{RANAWAKE}, Silvia, \textit{op. cit.}, pp.
3--4. Cf. \versal{SPECKENBACH}, Klaus.  “Prosa"-Lancelot”. In
\textit{Interpretationen}. Mittelhochdeutsche Romane und Heldenepen. Stuttgart:
Reclam, 2007, p.~328. }

Com efeito, um manuscrito holandês preservado, que narra as aventuras do
cavaleiro Lancelot, apresenta forte similaridade com a parte primeva do
\textit{Prosa"-Lancelot} alemão, datada de cerca de 1250, correspondente ao
\textit{Lancelot du Lac} do Ciclo da Vulgata\footnote{ Cf. \versal{ANDERSEN}, Elizabeth
A., \textit{op. cit.}, p.~156. }.  Elizabeth Andersen detectou uma lacuna de
cerca de 1/10 da narrativa entre os dois primeiros livros do
\textit{Prosa"-Lancelot}, se comparados aos escritos franceses. Ainda assim, o
livro \versal{I}, relativo a Lancelot, é três vezes mais longo que as narrativas
combinadas de \textit{A Demanda do Santo Graal }e \textit{A Morte do rei Artur}.
A mesma pesquisadora supõe a eventual existência de uma versão não cíclica de
\textit{Lancelot}, que depois teria sido melhor desenvolvido no \textit{Lancelot
du Lac} do \textit{Ciclo do Pseudo"-Map}. Sua hipótese se fundamentou na
percepção de que há muitas homologias entre o primeiro livro do
\textit{Prosa"-Lancelot} e o texto francês, como o episódio da humilhação de
Lancelot ao adentrar a charrete, para resgatar Guinevere ou a assunção de
Galahad aos Céus, no final da \textit{Gral"-Queste}.\footnote{ \textit{Idem
ibidem}. } 

O \textit{corpus} alemão integral parece ter sido dedicado aos condes de Hesse,
Frederico \versal{I}, o Vitorioso (1425--1476), e sua irmã Mathilde de Rottenburg, cuja
corte se localizava em Heidelberg. Como os exemplares francês e português, 
\textit{A Demanda do Santo Graal} alemã revela forte influência do pensamento
cisterciense, sobretudo referido ao monastério de Gottesthal, no ducado de
Lemburgo.\footnote{ Cf. \versal{ANDERSEN}, Elizabeth A., \textit{Op. Cit.}, p.~157.
Entretanto, outro estudioso do documento alemão, Klaus Speckenbach, afirma a
inexistência do espírito cruzadista próprio à Ordem de Cister quanto a Galahad.
Cf.  \versal{SPECKENBACH}, Klaus, \textit{op. cit.}, p.~340. } A propósito, convém
observar que os \textit{romans} arturianos alemães foram cultivados,
principalmente, pelas cortes senhoriais, raramente pelas principescas,
interessadas aquelas na difusão da imagem do cavaleiro laico enquanto campeão da
justiça. Em tais ambientes aristocráticos alemães, entrevia"-se forte presença de
mulheres letradas, que conviviam com cavaleiros por vezes iletrados, tendo
impulsionado a propagação de escritos em vernáculo. Os estudiosos artunianos
Jackson e Ranawake salientam que os motivos arturianos conheceram especial
difusão nas regiões meridionais do Sacro Império Romano"-Germânico.\footnote{ Cf.
\versal{JACKSON}, W. H.; \versal{RANAWAKE}, Silvia,  \textit{op. cit.}, p.~6. } 

%J:  parei aqui
O \textit{Codex Palatinus Germanicus} 147 gerou dez manuscritos copiados, além
de uma edição impressa de 1576, tendo representado, ainda, um ponto de inflexão
nos escritos em prosa da tradição alemã. Em primeiro lugar, como destaca Volker
Mertens, o \textit{roman} introduz, nos círculos letrados alemães, a exemplo do
francês e do flamengo, o hábito da leitura em pequenos grupos ou mesmo
individual, paralela à declamação que ainda ocorre, nas cortes, do conteúdo
desses escritos.\footnote{ Cf. \versal{MERTENS}, Volker, \textit{op. cit.}, p.~146. }
Antes do \textit{Prosa"-Lancelot}, não havia, na tradição escrita alemã, a prosa
novelesca, e sim a bíblica e a jurídica. A familiaridade dos compiladores
anônimos com os códices jurídicos e sermões clericais evidencia"-se na sintaxe
textual flexível deste \textit{Roman}, bem como na visível impregnação da
espiritualidade laica das ordens mendicantes do século \versal{XIII}, estando
presente uma perspectiva de História da Salvação.\footnote{ \textit{Idem}, p.
150. }

Recentemente, a estudiosa alemã Katja Rothstein tem"-se dedicado a perscrutar a
história da compilação e difusão da \textit{Gral"-Queste} no Sacro Império
Romano"-Germânico, explorando, sobretudo, o Ms. Allem. 8017--8020(\versal{H--A}) da
\textit{Bibliothèque de L'Arsenal}, em Paris.\footnote{ O manuscrito fez parte,
durante o século \versal{XVIII}, da biblioteca de um bibliófilo alsaciano, o
Barão Joseph von Heiss, tendo sido adquirido pela \textit{Bibliothèque de
L'Arsenal} ainda em fins do mesmo século. Cf. \versal{ROTHSTEIN}, Katja. ``Eine
Entstehungsgeschichte der Lancelot"-Handschrift. Ms. Allem.  8017--8020 (a)''. In
\versal{RIDDER}, Klaus; \versal{HUBER}, Christoph (org.), \textit{Lancelot --- Der mittelhochdeutsche
Roman im europäischen Kontext}. Tübigen: Max Niemeyer, 2007, p.~282. } No
Fólio 925 \textit{retro}, consta a sua data de apresentação, o já citado dia 12
de setembro de 1576. Em sua investigação, esteve em cotato com o \versal{CPG} 147 de
Heidelberg (intitulado por ela \textit{*P"-Redaktion}).

A erudita alemã concluiu que as 935 páginas do manuscrito parisiense parecem ter
sido originadas de um só escriba. Com efeito, há uma autoidentificação do
compilador como Christophorus Crispinus, que teria atuado a serviço de um
patrício alemão de Strassburg, Wolffhelm Bock. O brasão de armas fabricado para
esse mercador opulento, em seu afã de enobrecimento, um bode em salto para a
esquerda, figura no códice Ms. Allem. 8017--8020. Há evidências da proeminência
de sua família na Alsácia desde o século \versal{XIII}, variando as grafias de
seu nome, entre Bock, Böckle e Böcklin.

Para este segundo códice, Rothstein acredita que o compilador pode ter"-se
servido não apenas do original alemão, mas também de uma versão francesa da
\textit{Demanda do Santo Graal} do Ciclo do Lancelot"-Graal. Para pesquisadores
como reinhold Kluge, o \versal{H--A} é inteiramente dependente do \versal{CPG} 147 de Heidelberg.
De qualquer forma, é notório que o manuscrito quinhentista é muito mais similar
à versão francesa que o manuscrito de fins do século \textsc{xiii}, objeto de
análise do presente estudo.

Ademais, há um outro códice, anterior e não idêntico ao Ms. Allem. 8017--8020,
dito Incunabulum 1488, mas não pode ter sido, na análise de Rothstein, a única
fonte para a compilação do manuscrito do século \versal{XVI}, já que o códice do
Arsenal coincide em muitos pontos com o texto de Heidelberg, que, por sua vez,
desvia do Incunabulum 1488. Não se trata aqui de acaso, mas de evidência de que,
a partir do \versal{CPG} 147 da \textit{Bibliotheca Palatina Germaniae}, surgiu e se
consolidou uma tradição alemã peculiar em torno da narrativa da \textit{Demanda
do Santo Graal}.\footnote{ \textit{Idem}, p.~283. } Já outra estudiosa, Monika
Unzeitig"-Herzog acredita que as incongruências entre o \versal{H--A} do Arsenal e o
códice 147 de Heidelberg denotam um propósito de afastar"-se da tradição, por
meio do recurso às narrativas francesas.\footnote{ \textit{Idem ibidem}. }

Quanto a tal debate no meio acadêmico alemão acerca do \textit{Prosa"-Lancelot},
Katja Rothstein afirma que a \textit{*P"-Redaktion} filia"-se, mais claramente, à
versão francesa já cristianizada de \textit{La Queste del Saint Graal}, ao passo
que o Ms. Allem. 8017--8020 seguiria uma versão mais antiga. As convergências
maiores entre esse último e o CPG 147 se dão quanto à narrativa sobre o Mago
Merlin. A conclusão da estudiosa é de que houve uma preferência deliberada pela
tradição arturiana alemã, o que sugere a percepção de que a mesma conheceu um
processo minimamente bem sucedido de afirmação. Mas o que se procurou elaborar,
nas cortes senhoriais do Sacro Império Romano"-Germânico, foi uma fusão entre o
códice de Heidelberg e a versão francesa do \textit{Ciclo da Vulgata}.\footnote{
\textit{Idem}, p.~284 }

No itinerário de transmissão do Ms. Allem. 8017--8020, Rothstein descobriu o
parentesco entre topolinhagens da nobreza de espada e o patriciado urbano
ascendente.
Uma parente de Wolffelm Bock, Sophia Bock, casou"-se com um filho ilegítimo do
Conde Frederico, o Vitorioso (e da cortesã Clara Tott), de nome Ludwig von
Löwenstein und Scharfeneck, após enviuvar de seu primeiro marido, o também conde
Konrad \versal{III} von Tübingen. Desta forma, Christophus Crispinus teria
conseguido acesso ao manuscrito produzido para a Corte de Heidelberg,
provavelmente uma cópia posterior a 1475.

Na perspectiva de Katja Rothstein, os textos alemães da \textit{Gral"-Queste}
representam um intertexto em que dialogam as tradições francesa e a propriamente
alemã, forjando mitemas arturianos modificados, enriquecidos e divergentes das
demais versões europeias do mito.\footnote{ \textit{Idem} p.~286}

Sobre os três enredos distintos que compõem o \textit{Prosa"-Lancelot} alemão,
Rothstein noticia a existência do códice \versal{H--S}, de Schaffhausen, na atual Suíça
alemã (correlato alemão ao \textit{Agravain} francês) e do manuscrito \versal{H--K}, de
Colônia (correlato alemão ao \textit{Le Chevalier de la Charrette}, de Chrétien
de Troyes). O texto do cantão de Schaffhausen é similar ao \versal{CPG} 147 de
Heidelberg, tendo sido elaborado no mosteiro beneditino de Ochsenhausen, c.
1530. Rothstein assinala, no entanto, que não se tratou de mera reprodução, e
sim de dois textos autônomos que se teriam baseado em uma fonte comum da
tradição alemã. Supõe"-se que tenha sido produzido para um nobre de Schaffhausen,
Heinrich vom Stain zu Hürben (o nome se encontra no próprio códice), 
%Pedro: Hürben ou Hürden???
com parentesco por afinidade com a linhagem de Frundsberg, propritária de uma
das maiores bibliotecas aristocráticas da Idade Média, em que foram localizados
quatro \textit{Romans} arturianos.

Interessa ainda observar que um exemplar do \textit{Prosa"-Lancelot} (\versal{CGM} 573)
abriga um brasão de armas simbolizando a aliança entre as linhagens de
Frundsberg e vom Staim zu Hürden. Este texto passou à propriedade do Duque
Albrecht \versal{V}, conservado, na cidade de Munique, em sua \textit{Münchener
Hofbibliothek}. Já o Códice de Colônia traz uma narrativa, a
\textit{Karrenepisode} (Relato sobre a charrete), ausente do \textit{Lancelot
von dem Lache} alemão, de c. 1250. Por esta razão, parece que os copistas
alemães baixo"-medievais utilizaram esse manuscrito \versal{H--K} como padrão de prova
para ``corrigir'' ou ``complementar'', já em princípios do século \versal{XVI},
o \textit{Prosa"-Lancelot} de Heidelberg, com fulcro apenas na tradição alemã,
sem recorrer aos códices franceses.

O Códice de Colônia parece descortinar, por conseguinte, um esforço de
preenchimento das lacunas dos textos alemães, o que se deveu, em grande parte,
ao impulso dos meios cortesões letrados de Heidelberg e Rottenburg, cujos
senhores exerceram a função de príncipes eleitores do Imperador Romano
Germânico, enquanto Condes Palatinos do Reno.\footnote{ Cf. \versal{BORST}, Otto.
\textit{Geschichte Baden"-Württembergs}.  Ein Lesebuch. Stuttgart: Theiss, 2005,
pp.~126--155. } No afã de forjar uma versão alemã completa, não lacunar, do
mito arturiano, tal códice baseou"-se também na adaptação efetuada por Ulrich
Füetrer a partir do próprio \textit{Prosa"-Lancelot} de fins do século
\versal{XIII}. Katja Rothstein não admite a hipótese de que a fonte de Füetrer
possa ter sido o Códice de Heidelberg. Junto a ela, outro pesquisador alemão,
Rudolf Voss, acredita que Füetrer precisou, necessariamente, basear"-se em uma
fonte menos lacunar, em relação ao texto francês do \textit{Ciclo do
Pseudo"-Map}, que o \versal{CPG}147.\footnote{ Cf. \versal{ROTHSTEIN}, Katja, \textit{op. cit.},
p.~288. } O próprio compilador quatrocentista o fez a serviço da corte de
Munique, também interessada em se apropriar de uma tradição arturiana
propriamente alemã.

Acredita"-se que outros aristocratas laicos alemães tenham tido acesso a cópias
do \textit{Prosa"-Lancelot}, destacando"-se o Conde de Manderscheid"-Blankenheim,
bem como seu cunhado, o Conde de Nassau"-Hessen e o Conde de Waldeck. O primeiro
também possuía um exemplar francês do \textit{Lancelot du Lac}, de 1520, que lhe
veio da parte do Conde Palatino reichart von Simmern, aparentado por casamento
aos condes de Heidelberg, tendo registrado algumas glosas no livro. Noticia"-se
ainda um códice do século \versal{XVI} (\versal{CPG}92), também da \textit{Bibliotheca
Palatina Germaniae} de Heidelberg, que continha as versões alemãs de \textit{A
Demanda do Santo Graal} e \textit{A Morte do rei Artur}, além do
\textit{Donaueschinger Manuskript} 147.
	
Quanto ao último, de acordo com a correspondência entre Lassberg e Jakob Grimm,
a terceira parte (\textit{A Morte do rei Artur}) estaria evadida. Também o
aristocrata Johann Werner von Zimmern, o Velho, que viveu seus últimos anos na
corte bávara de Albrecht \versal{IV}, teria lá travado contato com as
compilações de Ulrich Füetrer. Saliente"-se, finalmente, uma conclusão da
pesquisa inaugural de Katja Rothstein: embora tanto o \versal{CPG}147 de Heidelberg
quanto o Ms. Allem. 8017--8020 sejam as bases fundamentais para a difusão do
\textit{Prosa"-Lancelot} pelo Sacro Império Romano"-Germânico, o último parece
encarnar o ponto culminante e terminal da história de transmissão do
mesmo.\footnote{ \textit{Idem}, pp.~289 a 291. }


\begin{bibliohedra}

\tit{BARBER}, Richard. \textit{ The Holy Grail}. Imagination and Belief. Cambridge:
Harvard University Press, 2004.

\tit{BIRKHAN}, Helmut. \textit{Keltische Erzählungen vom Kaiser Arthur}. Wien: Lit,
2004.

\tit{FRANCO JR}, Hilário. “Meu, teu, nosso. Reflexões sobre o conceito de cultura
intermediária”. In \textit{A Eva Barbada. }Ensaios de Mitologia Medieval. São
Paulo: Edusp, 1996.

\tit{JACKSON}, W. H.; \versal{RANAWAKE}, Silvia A. (org.) \textit{The Arthur of the Germans}. The
Arthurian Legend in Medieval German and Dutch Literature. Avon: University of
Wales Press, 2000. 

\tit{LITTLETON}, Scott.; \versal{MALCOR}, Linda. \textit{From Scythia to Camelot}. A Radical
Reassessment of the Legends of King Arthur, The Knights of the Round Table and
The Holy Grail. New York: Routledge, 1994.

\tit{MEGALE}, Heitor. \textit{A Demanda do Santo Graal}. Das origens ao Códice
Português. Cotia: Ateliê, 2000.

\tit{MERTENS}, Volker. \textit{Der deutsche Artusroman}. Stuttgart: Reclam,
2007Editorial, 2000.

\tit{SCHMITT}, Jean"-Claude. \textit{Le corps, les rites, les rêves, le temps}: essais
d’anthropologie médiévale. Paris: Gallimard, 2007.

\tit{RÉGNIER"-BOHLER}, Danielle. “O amor cortesão”. In \versal{LE GOFF}, Jacques; \versal{SCHMITT},
Jean"-Claude. \textit{Dicionário temático do ocidente medieval}. São Paulo:
Edusc, 2002.
\end{bibliohedra}


