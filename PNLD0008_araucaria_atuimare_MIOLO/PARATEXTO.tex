\chapter{A nova ficção científica brasileira}

\section{Sobre o autor}


\noindent{}Não são poucos os médicos (atuantes ou não na prática da medicina) que escrevem literatura --- célebres nomes como Anton Tchekhov, Guimarães Rosa, Moacyr Scliar e Arthur Conan Doyle figuram nessa lista. Fabio Atui é também um desses clínicos"-literatos --- médico gastroenterologista e proctologista, formou"-se em 1995 pela Faculdade de Medicina da \textsc{usp}. É cirurgião"-geral do Hospital das Clínicas de São Paulo, onde opera diversas enfermidades que assolam o trato digestivo. Também é responsável pelo ambulatório de doenças sexualmente transmissíveis e proctologia do mesmo hospital, onde atende semanalmente. Ainda no Hospital das Clínicas, Fabio é membro ativo do Navis (Núcleo de Assistência à Vítima de Violência Sexual), amparando e assistindo pessoas sexualmente agredidas.

Desde 2004, Fabio atua também como cirurgião voluntário e coordenador de cirurgia dos Expedicionários da Saúde, um grupo de médicos voluntários que levam atendimento especializado a populações indígenas em regiões isoladas. 

À toda essa vivência em contato direto com pessoas, soma"-se a crença de Fabio de que a disseminação de informações de qualidade é a receita certa para que as pessoas vivam melhor. O conhecimento humano adquirido em tantos anos de dedicação à área da saúde agrega"-se ao fazer artístico: Fabio assina, junto com dois outros autores, o roteiro do documentário \emph{Expedicionários}, longa"-metragem narrativo do trabalho da equipe de médicos voluntários para com indígenas que habitam regiões remotas do país, onde a infraestrutura e o trabalho clínico especializado não chegam. O filme foi dirigido por Octávio Cury e lançado em 2012. 

Mas a vivência com o campo das artes acompanha Atui desde a infância: Fabio é sobrinho de Fauzi Arap, um dos principais nomes da dramaturgia brasileira, de quem toma emprestado o título \emph{Mare nostrum} e a inspiração e ideia inicial do romance.

\section{Sobre a obra}

No livro encontramos a história de Theo, jovem, estudante de medicina,
afeito às tecnologias e que recentemente logra sucesso junto aos
investimentos de renda variável.

Theo também é capaz de se valer do \emph{mare nostrum}, uma espécie de
poder psíquico que permite com que se conecte com outras pessoas, mesmo
a distância.

Entretanto, esse poder parece gerar o efeito contrário na vida da
personagem principal, uma vez que ela se encontra cada vez mais distante
de seus amigos e colegas, parecendo perder o interesse nas situações
cotidianas. E, não obstante, resta claro que Theo vale"-se desse dom de
forma frívola, usando"-o para ganhar vantagens e habilidades indiferentes
ao seu crescimento pessoal. Em uma dessas situações, à beira de sua
formatura, seu desempenho em uma avaliação é colocado em xeque. Seus
professores, pressupondo uma ``cola'', ou acesso à informação
privilegiada, indicam que não se pode tratar a medicina de maneira tão
fútil, afinal, em primeiro lugar, deve"-se ter em mente que se lida com
outro ser humano, devendo haver para com este, toda atenção e dedicação
possível. Assim, diante da suspeita de fraude e de um utilitarismo
barato por parte de Theo, seus professores decidem puni"-lo, enviando à
selva amazônica, esperando, com isso, incutir em seu pupilo o verdadeiro
significado de ser médico.

\subsection{Superficial versus o essencial}

Na leitura do livro, vê"-se o desenvolver da personagem, que demonstra
ser um rapaz pouco empático, autocentrado, para alguém atento ao outro e
desejoso de fazer"-se útil a um maior número de pessoas, e, também, em
muitas frentes.

Ao longo da história também se percebe como o protagonista vai deixando
o apreço a tecnologia em segundo plano. O apreço ao aparto tecnológico é
gritante já no início da obra, onde o autor descreve a situação que
Theo, estudante de medicina, toma conta do pai adoentado por meio de um
sistema de domótica, controlado por aplicativo.

Isso também ocorre com seu \emph{mare nostrum}, inicialmente utilizado
pela personagem de forma mesquinha, para fugir das frustrações
cotidianas. Depois, Theo o oferece a pessoas em desalento, como quando o
faz em auxílio do índio Carú.

O próprio tom da escrita evolui ao longo da obra, deixando um início
truncado e arredio para uma fluência mais sublime no caminhar da
história, sendo um importante elemento condutor da história, revelando
ao leitor que a libertação do frívolo, do superficial, conduzem a
caminhos mais brandos.

\subsection{De muitos, um, e de um, muitos}

O tema de um médico de um grande centro encaminhado a uma província
distante, ou a um lugar inóspito é recorrente na literatura. Como
exemplo, podemos retomar as cartas que Tchekhov escreveu em sua viagem à
Sacalina, ou a experiência que Carlo Levi viveu nas vilas de Grassano e
Aliano. E tais vivências são narradas sempre como um desabrochar de
humanidade, uma reflexão sobre os pontos importantes da vida e sobre a
beleza contida nos detalhes.

A obra \textit{Mare Nostrum: Paranã Tipi} não fica para trás nesse sentido,
havendo também nela a riqueza de se desenvolver em solo nacional,
desvelando situações e características muito próprias à nossa realidade.

Apesar de ficcional, a história traz narrativas reais amalgamadas no fio
condutor do enredo. Alguns nomes na história são pessoas reais e, mesmo
as personagens que aparecem são, por vezes, um mosaico de diversas
outras. O próprio protagonista resulta de experiências do autor e de
seus colegas.

O próprio autor afirma que a comunidade que compõe boa parte do enredo é
ficcional, sendo na verdade uma colcha de retalhos de muitas outras,
essas sim, existentes.

Assim, \textit{Mare Nostrum: Paranã Tipi} representa a união em poucas
personagens das experiências de diversos seres humanos. Mas também, ao
se pensar em termos de humanidade, denota o quão próximo podemos ser um
dos outros e como, apesar de diferentes, somos similares em nossos
âmagos.

\subsection{Jogo de palavras}

Há na obra uma escolha perspicaz de palavras. Logo ao ler o título, já
somos postos em confronto com dois universos: do latim, \textit{Mare Nostrum}, e
do Nheengatu, \textit{Paranã Tipi}. De um lado, a menção ao mundo romano, do
\emph{imperium}, da \emph{civitas}, do idioma do Lácio, a partir do qual
o português surgiria. Do outro, a língua geral indígena, a língua da
terra, a língua que remete a uma vida consoante à natureza. Ambos os
termos denotam água: em latim temos o ``nosso mar'', e, em nheengatu, temos
``rio profundo''.

Ambas as referências se entremeiam pela obra, afinal, o chamado \textit{mare
nostrum} é o ambiente onde Theo consegue se conectar com outras pessoas.
Porém, quando também descobre que esse mesmo lugar já fora chamado de
\textit{paranã tipi}, é o momento onde o protagonista passa a entrar em contato
mais profundo consigo mesmo.

A oposição entre a \textit{civitas} e a selva também é pesada pelo próprio
protagonista, que passa a contestar a necessidade de todo o aparato
tecnológico que se encontra envolto, passando a ver com mais apreço
sentimentos e ações.

A ideia de profundidade das águas é muito explorada dentro da
psicologia. Faz alusão a um mergulho profundo dentro de si, da busca
pelos naufrágios e tesouros há muito escondidos. Havia também, entre os
gregos, o mito de Nereu, deidade marinha que tinha a resposta para todas
as perguntas. Porém habitava as águas profundas, sendo necessária uma
árdua busca para encontra"-lo. Quando isso se dava, contudo, era
necessário capturar Nereu que, para fugir, assumia as formas mais
assustadoras. Era preciso ter resiliência e domá"-lo, para então obter as
respostas buscadas.

Tal qual a captura de Nereu, o herói da história tem de enfrentar
monstros, mas não de um ser marinho, mas coisas assustadoras de sua
própria profundidade. E nesse conflito sai, como suas dúvidas e seus
pacientes, sanado, e também pleno, pois foi exposto à um universo mais
sublime.

Aliás, o próprio choque de universos que o título parece indicar é, de
fato, apenas aparente, pois a similitude não se encontra apenas na
semântica das palavras serem ligadas à água, mas também a característica
mais elementar de que todas as culturas são humanas. E quanto mais
humanos somos mais nos realizamos.

A leitura desse livro é um convite à pausa, e a consequente reflexão. De
onde estão nossos limites. Isto é, até que ponto ``sou eu'', e onde
começa o externo? E de que modo o indivíduo não se distancia de sua
essência, preso a frugalidades do cotidiano? Tergiversando, inclusive, a
ditames da ética e da urbanidade?

O protagonista é exposto a essa reflexão de forma abrupta. E contra sua
vontade. É apenas após um caminhar áspero que vai, aos poucos,
percebendo o que realmente importa, ao mesmo tempo que vai desenvolvendo
um trato mais humano com seus pacientes. Esses não mais representam
números como seus ativos na bolsa, mas verdadeiros universos que
precisam ser cuidadosamente regados para a manutenção de sua boa saúde.

De início, o autor faz parecer que o \textit{Mare Nostrum} é, de certo modo, uma
alusão à tecnologia. Entretanto, depois vemos que não, que na verdade é
um acesso ao próprio potencial humano que Theo detinha, mas que até
então usava de forma equivocada. Apenas com o choque de realidade que
vivenciou é que foi capaz de compreender isso. Com isso, o escritor nos
abre os olhos, convidando"-nos a sermos escafandristas de nós mesmos,
para, de dentro, vermos como realmente estamos tocando nossas vidas.

O próprio autor afirma que:

\begin{quote}
Não quero dizer com isso que precisamos ir para o meio da floresta para
encontrar nosso verdadeiro eu. Cada um tem o seu caminho próprio, mas
afirmo sem medo de errar que sair de seu cotidiano, abrir mão do
conforto e ter uma perspectiva externa de si mesmo faz milagres para o
crescimento pessoal e para estabelecer as prioridades na vida. Foi o que
aconteceu com o Theo, comigo e com muitos outros Expedicionários. Eu
aconselho.
\end{quote}

Assim, resta o convite à leitura da obra, e do mergulho que ela instiga,
de modo que, eventualmente, possa se tirar uma experiência
engrandecedora desse exercício.


\section{Sobre o gênero}

Colocando, ao lado da discussão sobre o inconsciente e o poder do sonho, questões relativas aos povos indígenas, como seus hábitos e costumes em confronto com o pensamento de um jovem médico da cidade, \emph{Mare nostrum: Paranã Tipi} é um texto atual, num gênero de ficção que se aproxima da ficção científica, esbarrando nos subgêneros da ficção metacientífica e psicocientífica. A ficção científica é um gênero pouco explorado na escola, mas de muita identificação com o universo do adolescente, principalmente por meio de produções audiovisuais estrangeiras. Pouco se menciona de ficção com caráter científico produzida e ambientada no Brasil.

O livro parte de uma potência comum a toda a humanidade: a capacidade de sonhar. A trama ficcional não se desenvolve em torno de complexos aparatos tecnológicos de vocabulário intrincado ou especialista, facilitando a compreensão do adolescente e direcionando"-o para reflexões acerca do próprio conhecimento, dos mistérios em torno da mente humana, daquilo que é pressuposto como verdade científica e da importância do acesso à informação. 

Enquanto gênero literário, a ficção científica é um tipo específico de ficção, que constrói o enredo a partir da ciência --- real ou imaginada --- e de seu impacto em determinada sociedade. Surgiu a partir do século \versal{XIX}, principalmente em razão do desenvolvimento científico e tecnológico nas áreas da Física, Química, Geologia, Astronomia e Informática, que introduziram descobertas, invenções e aparatos tecnológicos que modificaram a vivência da sociedade humana.

Alguns dos principais representantes desse gênero são os escritores
Isaac Asimov (autor do famoso \textit{Eu, robô}); Arthur C. Clarke (que escreveu o livro adaptado às telas por Stanley Kubrick, \textit{2001: uma odisseia no espaço}); H. G. Wells (dos clássicos \textit{A guerra dos mundos}, \textit{O homem invisível} e \textit{A máquina do tempo}); Philip K. Dick (que também teve seu livro \textit{Androides sonham com ovelhas elétricas?} adaptado ao cinema por Ridley Scott sob o título \textit{Blade Runner}); William Gibson (\textit{Neuromancer}); Ursula K. Le Guin (autora de ficções científicas políticas como \textit{Os despossuídos}); e Douglas Adams (da série \textit{O mochileiro das galáxias}).

Apesar de ser mais relacionado ao século \textsc{xx} e às revoluções científicas que essa época presenciou, a gênese da ficção científica pode ser encontrada ainda no século \textsc{xix}, em autores como Mary Shelley e Julio Verne.
Na primeira, normalmente associada ao romantismo gótico, já se percebe o elemento científico distópico, tão caro à ficção científica do século \textsc{xx}, com a construção de seu Frankenstein.
Apesar de ser uma resposta ao Iluminismo francês do século \versal{XVIII}, o gótico conjugava a modernidade da medicina, do tranporte e da ciência ao ambiente medieval.
Dessa transposição das inovações técnicas à literatura foi beber a ficção científica, assim como de Verne e suas invenções \textit{avant la lettre} de submarinos e foguetes espaciais.

Muito popular em obras estrangeiras, especialmente em filmes e séries para a televisão, o gênero ainda tem poucos autores de referência no Brasil. \emph{Mare nostrum} poderia ser categorizado como ficção psicocientífica, por ficcionalizar algumas teorias psicanalíticas do inconsciente, do material onírico, daquilo que é o sonhado, ou ainda como ficção metacientífica, por orbitar a esfera da indagação da ciência acerca da própria ciência, lançando mão de pressupostos e teorias científicas para a construção do enredo.

Através dessa amálgama de teorias psicanalíticas e da ciência, 
\emph{Mare nostrum} empreende, concomitante à viagem espacial, uma jornada e um mergulho em si mesmo.
É curioso pensar que Theo detinha um mecanismo de se conectar com as
pessoas do mundo, mas mesmo assim era alguém distante, incapaz de
priorizar o outro no lugar de si próprio.

Forçado a realizar a grande jornada rumo à Amazônia, é confrontado com
uma realidade muito distante da sua. Inconformismo, raiva, menosprezo,
são todos sentimentos que lhe envolvem a alma em um primeiro momento.
Entretanto, numa breve epifania, registrada de forma tão natural, o
protagonista observa a capacidade que possui, se tiver boa vontade, de
tornar a vida de outra pessoa menos miserável, tomando, para isso,
apenas pequenas atitudes.

E nessa jornada ao desconhecido, acaba por se conhecer melhor, detendo
de si uma visão mais clara, construída a partir de seu reflexo na retina
do outro. 

Pode assim relacionar a história da personagem principal com outro gênero narrativo, um dos mais arcaicos da raça humana: a jornada do herói, em que o herói se constrói saindo de seu ambiente de conforto,
encontrando as forças que deve enfrentar para se realizar e, então,
tornar"-se digno de retornar, refeito, fortalecido e enobrecido.

Theo, efetivamente, passa por todas as etapas da jornada do herói, tal como teorizada por Joseph Campbell. O protagonista é surpreendido, na normalidade do cotidiano --- seus estudos em medicina, a vida costumeira na cidade ---, por um desafio: sua viagem compulsória à floresta amazônica. Lá encontra os mais diversos desafios, tanto pessoais --- com o choque cultural, a incompreensão de outra forma de ver e viver o mundo, o conflito com o \textit{mare nostrum} ---, quanto exteriores à sua psique, como os planos do Sr.\,D de conquistar e subjugar o \textit{mare nostrum}.

Em sua situação conflitiva, o herói é auxiliado por um poder extraordinário, que subverte a ordem recorrente das coisas, ainda na tipificação de Campbell. Trata"-se, na obra de Atui, do contato de Theo com os mitos amazônicos, com o poder que os rituais indígenas têm para auxiliá"-lo a compreender o \textit{mare nostrum} e a si mesmo. Por fim, quando consegue superar o obstáculo inicial --- compreender a razão de sua viagem, retificar seus hábitos e costumes e deter os planos do Senhor D --- ele retorna à sua vida cotidiana, mas não como antes, e sim maior, acrescido de sabedoria, autoconhecimento e poder.

%Deslumbrado com o poder de navegar o Mare nostrum -- o oceano que interliga os nossos inconscientes --, Theo é enviado à Amazônia, onde estranhamento e familiaridade se interligam em uma jornada rumo ao próprio passado. Em contato com culturas e saberes diferentes, o aprender se impõe como urgência diante da ameaça que parece cada vez mais próxima: poderá esse mar ser cerceado e controlado? \textit{Mare nostrum: Paranã Tipi} é o segundo volume da aventura de Theo rumo ao conhecimento do Mare nostrum, que é também a busca de si mesmo no alvorecer da vida adulta.

