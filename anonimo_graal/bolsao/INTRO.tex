\SVN $Id: INTRO.tex 7286 2010-08-24 13:20:34Z bruno $
\chapter[Introdução, por Marcus Baccega]{introdução}
\hedramarkboth{introdução}{marcus baccega}

\textsc{O historiador} francês Jérôme Baschet aborda, ao estudar a civilização
feudal, um tema relevante para compreender a trajetória histórica do Ocidente.
Formulando-o como uma pergunta -- se há algo como um “singular destino do
Ocidente -- Baschet percebe o papel central da Idade Média na formação de nossa
identidade e cultura no Ocidente contemporâneo. Tal destino significa, em
última análise, ressignificar permanentemente nossa herança medieval. Dentre os
produtos culturais da Idade Média ainda atuantes no mundo contemporâneo, a
matéria cavaleiresca apresenta especial destaque, um verdadeiro culto
nostálgico a nossas raízes medievais. E os motivos da cavalaria não se
dissociam, no imaginário ocidental contemporâneo, da fascinante figura do Rei
Artur, sempre acompanhado pelos 150 cavaleiros da Távola Redonda. A mesma
temática origina, na atualidade, ressignificações na literatura e no cinema, de
que se pode mencionar, por exemplo, o \textit{cavaleiro} \textit{jedi} na série
\textit{Guerra nas Estrelas} (\textit{Star Wars}), dirigida pela cineasta
estadunidense George Lucas. Seria ingênuo não aferir como a construção das
personagens da \textit{Ordem dos Jedi} remete à noção de um cavaleiro-monge, ou
o monge-guerreiro, a exemplo dos Templários, cuja regra monástica foi elaborada
pelo próprio São Bernardo de Claraval (1130). 

\section{O ethos cavaleiresco}

O ideal do cavaleiro medieval foi, em grande medida, desenhado pela
ortodoxia católica, principalmente pela doutrina da \textit{Militia Christi} (o
“exército de Deus”) de São Bernardo de Claraval (abade da Ordem de Cister), que
preconizava um monge-guerreiro, portador das virtudes teologais da fé,
esperança e caridade. Esse cavaleiro cristão, tendo a missão bíblica de
amparar, preferencialmente, órfãos, viúvas e pobres, deveria ser capaz de
sobrepor a glória de Deus e a destruição dos infiéis a suas ambições pessoais,
sobretudo aquelas inspiradas pela cupidez. Tal monge-guerreiro seria, por
extensão, a força armada da Igreja, sacramento que, conforme a Teologia
católica, torna Deus presente na história dos homens, devendo combater pela
justiça, paz e supremacia deste Corpo Místico de Cristo. 

Não se pode dissociar a construção deste \textit{ethos} (conjunto de valores,
crenças, tradições, hábitos e usos que definem a identidade de certo grupo
social) cavaleiresco cristão da Reforma Pontifical, dita por vezes “Reforma
Gregoriana”, assim referida em virtude do Papa Gregório \versal{VII} (1073--1085), um de
seus principais impulsionadores. Essa tentativa de reforma, não apenas do corpo
clerical, mas de toda a sociedade cristã, iniciou-se antes ainda o pontificado
de Gregório \versal{VII} e conheceria seu término por ocasião do \versal{IV} Concílio de Latrão
de 1215. Tratava-se, no plano das representações ideológicas, de um retorno
urgente e imediato à pureza originária das primeiras comunidades cristãs, como
descritas nos \textit{Atos dos Apóstolos}. Este ideal consagrou a denominada
\textit{vita vere apostolica} (“vida verdadeiramente apostólica”), como seria
referida pela Ordem Franciscana no século \versal{XIII}, que remetia a um intuito de
imitar a Cristo ao longo da existência terrena (\textit{imitatio Christi}). Na
verdade, a Reforma Pontifical significou, do ponto de vista das relações de
poder, um projeto de construção de uma monarquia papal universal, uma teocracia
pontifícia, que elevasse o Pontífice à condição de \textit{dominus mundi}
(senhor do mundo), sobrepondo-se aos poderes laicos, notadamente ao Sacro
Império Romano-Germânico. 

Todavia, no seio do fenômeno que o historiador Jacques Le Goff designou como
“reação folclórica” da aristocracia laica, outro ideal cavaleiresco foi
cunhado. Na Idade Média Central (séculos \versal{XI} a \versal{XIII}), em especial no século
\versal{XIII}, ocorreu uma ascensão social da cavalaria à condição de pequena nobreza.
Anteriormente mercenários ou camponeses incumbidos de proteger os castelos
senhoriais, ou elementos marginais que realizavam assaltos em estradas e
florestas, os cavaleiros não apenas ascenderam ao \textit{ordo} (como eram
designadas as camadas sociais no Medievo) nobre, como tal estrato social passou
a identificar-se -- e ser identificada pela aristocracia clerical -- com a função
bélica. O cavaleiro passa a encarnar a própria auto-representação traçada pela
nobreza feudal. Neste lastro, projetam-se valores próprios desta camada
nobiliárquica nobre sobre o tipo social do cavaleiro, valores que se contrapõem
à investida normatizadora da Reforma Pontifical. 

A “reação folclórica” correspondeu à constituição de verdadeiras mitologias de
origem -- que exercem a função auto-legitimadora de narrativas identitárias --
para as casas nobiliárquicas, apelando-se, por vezes, a entes fantásticos do
imaginário pagão pré-cristão, advindos da cultura oral híbrida céltica,
germânica e greco-romana, caso da fada Melusina, ancestral mítica da casa
borgonhesa de Lusignan. No seio desta efabulação, estabeleceu-se um outro
\textit{ethos} cavaleiresco, que apresentava o cavaleiro como um guerreiro
audaz, capaz de superar aventuras, provar seu valor e suas proezas em justas e
torneios promovidos nas cortes senhoriais e principescas, além de um praticante
do \textit{fin’amor} ou amor cortês. Em breves linhas, pode-se definir esse
amor cortês como um jogo estilizado e sublimado de disciplina do desejo erótico
e reafirmação dos vínculos feudovassálicos entre os nobres laicos.  O primeiro
registro da expressão \textit{amor cortesão} deveu-se a Gaston Paris, em 1883,
a propósito de um artigo acerca do \textit{Le Chevalier de la charrette}, de
Chrétien de Troyes, \textit{roman} que narra a relação de amor arquetípica, ou
“mais que perfeita”, entre o cavaleiro Lancelote e Guinevere, a esposa do Rei
Artur. O vínculo amoroso conduz o herói à prática de proezas e a jurar
ilimitada obediência às ordens de sua dama.

Também denominado \textit{vraie amour} (“amor verdadeiro”), cultivado pelos
trovadores cortesãos, o amor cortês conheceria um registro doutrinário
constitutivo de uma verdadeira normativa, no \textit{Tractatus de amore
}(“tratado sobre o amor”), de 1184, de André, o Capelão, composto sob o
mecenato da condessa Maria da Champanha. Na poesia trovadoresca, o amor cortês
figura virtualmente como uma relação adúltera, já que a dama é habitualmente
casada, destinatária de um cortejo amoroso, de uma súplica sentimental
veiculada por poemas (cantigas líricas de amor). Passa-se a conceber o amor
cortês consoante o nível social, já que vedado a clérigos e plebeus, para então
associá-lo a uma perfeição ideal, definida de acordo com “julgamentos de amor”
proferidos por grandes damas. Excluindo o \textit{fin’amor} do contexto
matrimonial, após enunciar as regras amorosas, Capelão finalmente desiste de
trilhar, ele mesmo, a senda desse amor cortês, na medida em que desafia e
contesta a moral cristã. Denota-se, pois, um eixo de tensão extrema entre a
elaboração literária e os valores do esteio social, que faculta a gênese mesma
do amor e sua recepção. 

Convém destacar como o fundamento feudovassálico do amor cortês está explícito
no vocativo, frequente nos poemas trovadorescos, “minha Senhora” (\textit{mi
dona}, em occitânico, \textit{minha Senhor}, em português), posto que o
eu-lírico se encontre ao dispor da dama como um vassalo ante seu Senhor,
devendo-lhe, inclusive, prestar homenagem. Mesmo não se especificando o ritual
da homenagem feudal nas cantigas trovadorescas, trata-se da questão da
\textit{saisine}, da posse pelo ósculo, como a se tomar posse de um feudo. Se
bem cortejada a dama, o poeta poderá, talvez, aspirar a um \textit{guerredon},
uma recompensa, que pode consistir em um olhar, um beijo, uma sempre incerta
declaração de amor, talvez mesmo uma união carnal, denominada “algo a mais”.

A posição amorosa deve sempre estar vinculada ao valor pessoal daquele que
aspira a cortejar uma dama, devendo mostrar-se leal e cortês, dedicar toda a
atenção ao elogio da amada e, nas regiões ao norte da atual França, inserir-se
na problemática bem romanesca do aperfeiçoamento pessoal por meio da ostentação
das virtudes bélicas nos torneios e combates. Não por acaso, o termo
\textit{prouesse}, em \textit{langue d’oïl }(variante do francês medieval
falada nas regiões setentrionais), remete às virtudes guerreiras,
enquanto \textit{proeza}, em \textit{langue d’oc }(variante do francês medieval
falada no sul), abrange as qualidades de um “fino amante”.  

O amor cortês instaura uma verdadeira religião do amor, cujo culto orbita a dama
nobre. Neste espectro, a alegoria do deus Amor, que permeia os escritos
novelescos cortesãos, revela a sujeição do eu-lírico ao sentimento que,
doravante, é sua única razão para viver. Com efeito, a intensidade da vida
interior é a todo tempo sugerida pela lírica, bem como pelas passagens
romanescas em que o herói se encontra cativo de uma imagem fascinante, em
estado de \textit{dorveille} (torpor). Nesta liturgia amorosa, para aceder à
alegria final, o enamorado precisa enfrentar a provação da castidade
(\textit{assag}, em \textit{langue d’oc}), o que demanda um extraordinário
domínio sobre o desejo, mesmo em uma situação em que o cavaleiro-trovador
amante jaz nu ao lado de sua dama. 

Nestes termos, o \textit{fin’ amor} consiste em uma erótica do controle do
desejo, que conduz o enamorado a suspirar (sendo \textit{fenhador} ou
suspirador), e adorar, muitas vezes de longe. Para aspirar à aceitação por
parte da dama, deve tornar-se suplicante (\textit{precador}), para assim
exprimir seu desejo com maior clareza, sem, no entanto, insistir. Se aceito
(sendo, doravante, \textit{entendedor}, em \textit{langue d’oc}, ou
\textit{merceians}, em \textit{langue d’oïl}), poderá ser admitido ao
\textit{assag} e, ao depois, talvez se torne um amante carnal (\textit{drut}).
A harmonia pretendida (\textit{joy}), que transcende o puro prazer físico
(\textit{gauc}), é elevação repleta de alegria, que transmuta o ser pela força
do desejo. Nas cortes principescas e castelãs setentrionais (região da
\textit{langue d’oïl}), essas longas etapas de perfecção do amor cortês são
menos usuais, sendo menos explícitas as promessas mais sensuais. A dama parece
sempre ausente e seus favores são dificilmente obtidos. Enfim, nas palavras do
erudito Jean Frappier, o amor figura como “extremo refinamento da cortesia”, no
espelho da mesma como virtude de sociabilidade. 

Ao passo que a cortesia setentrional (\textit{courtoisie}) aparece como conjunto
das virtudes da sociabilidade, a que pertence a arte de amar, a
\textit{cortezia} dos trovadores meridionais encontra-se mais intimamente
ligada à arte do \textit{fin’amor}. Vincula-se, com efeito, à \textit{mezura},
autocontrole e domínio do desejo, suscitando que a dama deve recompensar quem a
serviu fielmente. A \textit{mezura} designa, enfim, a atitude de espírito do
amante cortês, sua paciência e humildade. Outro predicado inescapável do amante
cortês é a “juventude” (\textit{joven}), que não se refere apenas às virtudes
de uma camada etária, mas à generosidade, à aptidão para a dádiva, à
liberalidade, e obviamente ao serviço de cortejo das damas, assim transcendendo
a mera galanteria. Ao atingir a consumação do vínculo amoroso, os amantes
precisam resguardar-se de uma personagem constante, que encarna o perigo de que
a relação seja revelada, o bajulador (\textit{losengier}, em \textit{langue
d’oïl}). Nas palavras da historiadora Danielle Régnier-Bohler, “o culto secreto
a que se devota o discípulo do \textit{fin’amor} é passível de espreita e o
nome da dama jamais deve ser revelado”.\footnote{ Cf. RÉGNIER-BOHLER, Danielle.
“O amor cortesão”. In: LE GOFF, Jacques. SCHMITT, Jean-Claude.
\textit{Dicionário Temático do Ocidente Medieval}, São Paulo: Edusc, 2002. , p.~51.} 
Os escritos arturianos são permeados pela dualidade do
\textit{ethos} cavaleiresco, mas nas versões ducentistas de \textit{A Demanda
do Santo Graal}, o amor cortês está plenamente convertido em ascese mística, de
acordo com a normativa clerical. Isto descortina uma progressiva apropriação
das novelas de cavalaria pelos clérigos, sobretudo os já mencionados
cistercienses. 

\section{A Matéria da Bretanha e o gênero retórico do \textls{roman}}

No que concerne aos gêneros escritos medievais, há duas formas de composição que
apresentam o idílio do amor cortesão: a canção de amor (\textit{canso}),
imitada pelos poetas do norte, e o \textit{roman}, desenvolvido principalmente
em \textit{langue d’oc}. As tonalidades do \textit{fin’amor} são bastante
matizadas entre os diversos trovadores. É notório como, entre os
\textit{trouvères} (os poetas líricos da região de \textit{langue d’oc}),
conserva-se distância entre a dama e o enamorado. Ante o caráter inacessível da
amada, qualquer palavra pode parecer excessivamente ousada, um ultraje, uma
desmesura. Na verdade, o tormento de amor é simultaneamente prazer e morte. 

Antes de se detalhar a questão do \textit{roman}, suporte por excelência da
Matéria da Bretanha (narrativas concernentes ao Rei Artur, aos cavaleiros da
Távola Redonda e ao Santo Graal), quando de sua compilação na Idade Média
Central, faz-se necessário aclarar que tais escritos não podem ser considerados
\textit{literatura}, propriamente dita. No lastro de estudiosos como Paul
Zumthor e Michel Zink, não é possível pensar em uma \textit{literatura}, como a
definimos atualmente, o que remete tal conceito a uma convenção de
ficcionalidade. Isto significa, em termos breves, que o conteúdo narrado em uma
obra literária é verossímil, mas não verídico, ou seja, o público receptor da
obra sabe de que se trata de um enredo fictício. Para os medievais, escritos
como os arturianos eram considerados portadores de acontecimentos verídicos,
“históricos”. Isto equivale a afirmar que esses escritos integravam uma
convenção retórica de veracidade, que se denota, por exemplo, no fato de que
palavras como \textit{roman} e \textit{estoire} eram intercambiáveis no período
medieval. Conceitualmente, o mais correto é pensar em gêneros retóricos
medievais, na medida em que se trata de discursos moralizantes, disciplinares,
vocacionados para a normatização das condutas de todos os estratos sociais por
parte da Igreja. 

Assim, essa espécie narrativa não deve ser confundida com a noção atualde
romance como narrativa afeita à convenção de ficcionalidade da literatura. Com
efeito, o vocábulo \textit{roman} adveio da expressão \textit{mettre en roman},
vale afirmar, traduzir determinado \textit{corpus} do latim para os idiomas
vernáculos que florescem a partir do século \versal{VIII}, conhecem seus primeiros
registros em documentos normativos de meados do século \versal{IX} (com destaque para o
\textit{Juramento de Estrasburgo} de 842 d.C.) e afirmam-se na Idade Média
Central, denotando o fenômeno da \textit{diglossia}. Daí a existência de
expressões como \textit{romanice loqui}, do latim clássico, ou os derivados
\textit{fabulare} e \textit{parabolare romanice,} que confirmam a etimologia do
\textit{roman}.\footnote{ Cf. MEGALE, Heitor.\textit{ A Demanda do Santo Graal}.
Das origens ao Códice Português. Cotia: Ateliê Editorial, 2000., p. 36.} Se
for desejável ensaiar uma tradução para o português, o vocábulo mais correto
seria \textit{romanço}, vale afirmar, o falar popular derivado do latim tardio,
que se praticava na Península Ibérica até a Alta Idade Média. 

A diglossia significa a que a produção cultural erudita se expressa em latim,
não nos nascentes idiomas vernáculos, ao passo que a havia uma cultura letrada
não dominada pela Igreja e se expressa nos idiomas vernáculos europeus. O latim
correspondia à norma culta herdada da Antiguidade Clássica e monopólio dos
setores clericais mais versados na erudição dos autores clássicos e na teologia
dos Padres da Igreja (Patrística), bem como na Escolástica coetânea à Idade
Média Central. Torna-se notório que o latim, adequado aos textos sagrados
(\textit{Vulgata} de São Jerônimo), litúrgicos e aos grandes tratados de
teologia, constitui a língua da memória em um contexto social em que memória e
verdade tornam-se sinônimos, em virtude da manipulação ideológica do idioma
latino pelos \textit{oratores}, em um contexto assinalado pela hegemonia da
oralidade sobre a escrita. Desta forma, o \textit{ordo clericorum}
(eclesiásticos) consagra sua posição de elite intelectual de \textit{litterati
}(letrados), manuseadores unitários da escrita latina e do saber formal e
erudito, e do sagrado. 

Por outro lado, autores como o linguista russo Mikhail Bakhtin identifica a
existência de um variado esteio de cultura popular na Idade Média Central,
relativo aos camponeses, vilões, cavaleiros analfabetos e outros homens
desprovidos de formação intelectual formal, os \textit{ilitterati}. Sua
manifestação simbólica poderia ser identificada na ampla gama de gestos,
hábitos, celebrações, tradições, contos e sagas transmitidas pelo viés da
oralidade. Sua singularidade religiosa poderia ser apreendida nas formas
concretas de adaptação dos ritos e cânones católicos aos usos e costumes
cotidianos de cada população, formulando-se expressões concretas e peculiares
de representação do sagrado e interação com o sobrenatural. Em alguma medida,
essa cultura popular passaria a conhecer registro, expressão e transmissão
escritos a partir do surgimento dos idiomas vernáculos locais, evoluídos de
formas dialetais híbridas de elementos latinos, celtas e germânicos, a partir
do século \versal{VIII}. 

Os significativos estudos de Mikhail Bakhtin e demais linguistas do Círculo de
Tartu sobre a história cultural medieval superaram a noção de que o espólio
cultural deste período histórico estaria reduzido à produção cultural erudita
dos \textit{litterati}, desconsiderando a pluralidade de manifestações
simbólicas das camadas populares. A interação entre a cultura de alto
repertório e a popular ocorre, na concepção de Bakhtin, por meio de uma
circularidade de seus produtos culturais, que se interpenetram, ressignificam,
invertem e reconfiguram a todo instante. A tese da circularidade cultural
também se encontra em outros autores relevantes para o debate historiográfico
medievalístico contemporâneo, como Carlo Ginzburg, Aaron Gurevitch e Georges
Duby\footnote{ FRANCO JR, Hilário. “Meu, teu, nosso. Reflexões sobre o conceito
de cultura intermediária”. In: \textit{A Eva Barbada. }Ensaios de Mitologia
Medieval, São Paulo: Edusp, 1996, p.p. 34 e 35.  }.

Questionando também o rigor da clivagem entre cultura erudita e cultura popular,
Jean-Claude Schmitt\footnote{ SCHMITT, Jean-Claude. \textit{Le corps, les
rites, les rêves, le temps}: essais d’anthropologie médiévale, Paris: Editions
Gallimard, 2007, pp. 130--147.} afirma que ambos extremos foram a todo tempo
intermediados por ampla esfera de interface entre seus produtos culturais. De
fato, as sociedades européias ocidentais do século \versal{XIII} contaram com atores
culturalmente híbridos e versáteis, responsáveis pelo trânsito entre os dois
pólos da cultura medieval. Pode-se exemplificar este permanente diálogo entre
os registros da cultura européia medieval mediante a ação evangelizadora do
clero católico. Os próprios sacerdotes de menor grau hierárquico,
intermediários entre os grandes pensadores cristãos e os estratos populares,
sempre adaptaram os cânones da dogmática ortodoxa produzida nos mosteiros e
abadias às peculiaridades culturais das regiões em que atuavam, sobretudo em
vista do esforço catequético. Denotaram elevado grau de tolerância e mesmo
claro propósito sincrético ante as manifestações laicas de religiosidade
cristã, profundamente impregnadas de crenças, ritos e signos do patrimônio
ancestral celta, germânico e greco-latino pagão. 

É necessário ponderar, neste momento, que não é o fato de tais agentes
transitarem pelos dois outros pólos culturais que torna intermediário seu
estrato de cultura. Como leciona Hilário Franco Júnior, trata-se exatamente do
oposto. Havendo uma esfera híbrida e polissêmica, em que se verificam fenômenos
de hibridação, retro-alimentação, ressignificação e reconversão de elementos da
cultura de alto repertório e da cultura popular, é que se torna possível a
existência de personagens de cultura intermediária. Os contrastes de ritmo e
intensidade com que \textit{litterati} e \textit{illitterari} se apropriam do
espólio híbrido dependem não apenas do conflito entre valores e interesses em
tela, mas também da detenção de instrumentos culturais diferenciados em cada
estrato social. Ponto de convergência entre as demais esferas de registro
cultural, a cultura intermediária permite a migração de determinados elementos
comuns, alargando as identidades de cada qual dos \textit{ordines} e
constituindo o próprio fenômeno da intermediação cultural, hoje muito estudado
por historiadores da cultura e antropólogos. Essa migração se processa, em
primeiro lugar, com uma recepção e ressignificação dos espólios na teia da
cultura intermediária, que fornece a matéria-prima, já híbrida e inédita, que
retorna para os estratos originários transformada em algo inédito\footnote{
FRANCO JR, Hilário. “Meu, Teu, Nosso: Reflexões sobre o Conceito de Cultura
Intermediária”. In: \textit{A Eva Barbada}. Ensaios de Mitologia Medieval. São
Paulo: Edusp, 1996., p.p. 34--36. }.

O \textit{roman }floresce precisamente no seio desta cultura intermediária,
fundamentalmente impulsionado pelo Ciclo Arturiano, cultivado por atores
sociais intermediários como jograis, menestréis e trovadores, que declamavam
nas cortes senhoriais e principescas. Este gênero retórico desenvolveu-se a
partir das canções de gesta da Alta Idade Média (séculos \versal{IX} a \versal{XI}) e das
crônicas historiográficas. As canções de gesta revelavam uma verdade do
enunciador, proclamada e reverberada de forma circular e sempre idêntica ao
próprio canto, referindo-se a feitos heróicos e grandes batalhas façanhas
militares e vitórias de grandes heróis. Seu exemplar mais célebre é a
\textit{Chanson de Roland} (“Canção de Rolando”), datada de cerca de  1080.
Como assinala o estudioso Heitor Megale, o cantor procura integrar o caos do
vivido em uma ordem, e encerrar as dúvidas em uma moldura de justiça. Ademais,
o diálogo -- evidentemente virtual -- do cantor com seu público, instituiria uma
“duração permanente”, ou “uma intemporalidade”, ao conteúdo narrado. Este
fenômeno decomporia, por conseguinte, a narrativa em células relativamente
autônomas que se sucedem, por vezes ordenadas em conjuntos justapostos, cuja
progressão, absolutamente não linear, seria percebida por paralelismo e
repetições, que, entretanto, não desfazem as contradições entre os episódios
singulares.

Se as canções de gesta apresentam uma circularidade do canto, que se perfaz por
uma captação da memória coletiva, Megale está plenamente correto em inferir que
se instaura uma realidade de expressão coletiva de que muitos homens
participam, mediados pelo cantor. Assim, não apenas se reflete uma camada de
historicidade sentida como real, porém se verifica uma compensação simbólica
pela ruptura entre o vivido e o imaginado em uma sociedade regida pela
oralidade. Enfim, a história seria ratificada, mas  simultaneamente convertida
em ficção, por intermédio da rigidez do discurso e da proeminência de um estilo
formal e seus recortes\footnote{ Cf. MEGALE, Heitor. \textit{Op. Cit}., p.p. 32
e 33. }.

 A transição da canção de gesta para os gêneros da historiografia e do
\textit{roman} encerraria uma ruptura dos laços da narrativa com a memória
coletiva, contemporânea a um desejo emergente por um conhecimento não
ficcional, porém marcado pela História, assumindo o antigo ouvinte uma condição
similar àquela de um “aluno consentido”. A  Antiguidade Clássica legou à Idade
Média uma concepção moral do discurso sobre a História, implicando uma moldura
do mesmo como exposição estilizada e persuasiva destinada à instrução e
conversão dos hábitos sociais. Eclipsada durante a primeira época feudal, tal
tradição seria retomada em tempos de desenvolvimento político do Sacro Império
Romano-Germânico, em fins do século \versal{XI} (com destaque para a região da
Lotaríngia). Mas seu pleno renascimento, com efeito, ocorreu no século \versal{XII},
sendo que a ela aderiram os anglo-normandos, quando de sua expansão pela
Grã-Bretanha e Itália, sendo tal tendência consignada pelo inegável impacto das
Cruzadas. 

A produção das formas arcaicas de \textit{roman}, ainda em verso, ocorreu nos
domínios continentais da dinastia dos Plantagenetas, onde também se havia
desenvolvido a historiografia. Um novo estrato senhorial, nessas áreas de maior
estruturação política, percebe a nocividade da guerra e a correspondente ética
veiculada pelas canções de gesta\footnote{ Cf. JACKSON, W. H. RANAWAKE, Silvia
A. “Introduction”. In \textit{The Arthur of the Germans}. The Arthurian Legend
in Medieval German and Dutch Literature. Avon: University of Wales Press,
2000., p. 06. }. Ao mesmo tempo em que clérigos e cavaleiros instruídos aspiram
a liberar-se da palavra poética, a partir do fenômeno das escolas medievais,
cria-se o prazer pelo códice, pelo texto em prosa, em torno do qual se
desenvolve, inclusive, uma forma de comércio. As próprias autoridades políticas
passam a comunicar-se por textos escritos com seus vassalos. Nos territórios
britânicos, a monarquia normanda instaurada por Guilherme, o Conquistador,
desde seu triunfo na Batalha de Hastings (1066), incentivou a compilação e
reprodução de motivos arturianos para atrair a simpatia das populações celtas e
desconstituir a anterior hegemonia saxã, proveniente das penetrações de anglos
e saxões na transição entre os séculos \versal{V} e \versal{VI} da Era Cristã e da formação de
sua heptarquia. 

Todavia, a difusão dos motivos arturianos parece ter despertado, entre as
populações celtas britânicas, o intuito de resgate de um glória pretérita então
imaginada, o que implicou a necessidade de a monarquia normanda enfrentar o
messianismo em torno da figura de Artur. O mesmo se vinculava,
fundamentalmente, à Abadia de Glastonbury, em Gales, que reunia escritos de
origem céltica e muitos monges irlandeses. Reputava-se fundada por José de
Arimatéia quando de seu êxodo da Palestina, tendo-se lá desenvolvido uma
devoção cultual ao fundador mítico que, em última instância, pretendia
legitimar a própria antiguidade da Igreja Católica inglesa. Essa última
poderia, desta forma, clamar paridade eclesiológica com Roma enquanto
verdadeira fundação apostólica, uma vez que José de Arimatéia teria sido,
consoante o \textit{Evangelho de São Mateus} (27:57), um discípulo secreto de
Cristo, mesmo permanecendo membro do Sinédrio. Na \textit{Gesta regum
anglorum}, William of Malmsbury afirma que a fundação da Abadia de Glastonbury
se deu por determinação do Papa Eleutério, que teria enviado uma missão ao
mítico Rei Lucius, no século \versal{II} d.C., registrada em escritos historiográficos
anglo-saxões. Elaborou-se, posteriormente, a versão de que dois missionários,
Phagan e Deruvian, teriam encontrado, no local da abadia, uma igreja
pretensamente fundada por discípulos de São Felipe e São Tiago, em 63 d.C.

A associação entre a abadia e José de Arimatéia parece ter sido também uma
efabulação do \textit{roman} arturiano \textit{Perlesvaus}, em que Glastonbury
é textualmente mencionada. De acordo com Richard Barber, o apogeu do projeto
plantageneta de identificar Glastonbury a José de Arimatéia e à própria ilha
mítica de Avalon deu-se no século \versal{XV}, sob a influência do abade Richard Bere,
que impulsionou o culto ao primeiro guardião do Santo Graal, vindo a alterar o
selo de armas da abadia, para incluir a figura de José de Arimatéia. Já era
corrente, todavia, a versão oral de que Arimatéia estaria sepultado na igreja
antiga de Glastonbury, incendiada em 1184, e de que o decurião teria aportado a
esse mosteiro duas ampolas, com o sangue e o suor de Cristo. Tal narrativa foi
atribuída a um escritor semi-anônimo de nome Melkin, para quem a descoberta do
sepulcro de Arimatéia teria revelado um cadáver incorrupto. Ainda em 1345, o
rei inglês Eduardo \versal{III} encorajou a busca por este túmulo, missão que Henrique \versal{V}
ordenaria aos próprios monges em 1419, o que revela um projeto político de
autonomização da Igreja inglesa com relação a Roma. Tal temática, discutida no
Concílio de Constança (1414--1418), foi posteriormente esquecida, sendo apenas
resgatada na Inglaterra anglicana, quando se atribuiu a origem de uma relíquia
identificada com o próprio Graal à Abadia de Glastonbury.

A partir do século \versal{XII}, a fabricação da coincidência do túmulo de Artur com o
território de Glastonbury propiciou a difusão das narrativas arturianas por
diversos estratos sociais, o que se denota pela alusão ao regresso messiânico
de Artur e à própria Távola Redonda no \textit{Roman de Brut} (1155), do
normando Robert Wace. Tal \textit{roman }correspondeu à adaptação poética da
\textit{Historia regum Britaniae}, de Geoffrey de Monmouth, por solicitação de
Henrique \versal{II} (1154--1189), que desejava uma epopéia versificada para consagrar
uma narrativa de laude à Dinastia Plantageneta, apta a suplantar a celebridade
da \textit{Canção de Rolando}, de que os Capetíngios se valiam para exaltar
suas glórias e, assim, legitimar seu poder político. 

Com efeito, o rei normando Henrique \versal{II} teria ordenado, em 1189, que se
procurassem os vestígios mortais de Artur no terreno da abadia, identificada à
ilha de Avalon (já que se situava em meio a um pântano), o que teria redundado
na descoberta de ossos gigantescos de um homem e uma mulher, juntos a uma cruz
de chumbo, em que se encontrava a inscrição latina \textit{Hic jacet Artorius,
Rex Britonum} (“aqui jaz Artur, rei dos bretãos”). Henrique II interpretou a
descoberta como evidência inconfutável de que Artur estaria morto, como
qualquer outro homem, o que deveria condenar as esperanças messiânicas dos
celtas sobre o futuro Reino de Artur, o que implicava, evidentemente, sua
projeção, ressignificada, sobre a casa normanda dos Plantagenetas.

Assim como nas primeiras expressões do gênero historiográfico, nos primeiros
\textit{romans} cavaleirescos escritos em versos octassílabos, de rimas
paralelas e sem estrofes definidas, exclui-se o canto, bem como a fragmentação
dos versos ordenados em estrofes. Isto implica uma forma de inflexão do texto
sobre si mesmo, ou, no entender de Heitor Megale, uma “concentração sobre a
intenção formalizante” que determina tais escritos. Observa-se, portanto, um
incipiente desenvolvimento da prosa, que se tornará plena no século \versal{XIII}, tanto
nas novelas de cavalaria como na historiografia. Como resulta evidente, esses
textos são multifacetados, recebendo acréscimos e continuações. Ademais,
mostram-se independentes do ritmo poético, que conferia unidade à canção de
gesta a partir do ato único e real da voz, que era ato de sua produção. Volker
Mertens percebe que, a exemplo das crônicas, nas canções de gesta o acontecido
legitima e valida o narrado, desenhando-se uma sincronia de núcleos de ação
narrativa, por meio de remissões recíprocas entre os episódios\footnote{ Cf.
MERTENS, Volker. \textit{Der deutsche Artusroman}. Stuttgart: Reclam, 2007., p.
151. }. Já o novo paradigma de texto revela uma finalidade, não mais
coincidente com a expressão da voz mesma da comunidade que o ouve, mas um
propósito de descrever o mundo para essa comunidade, para aprofundar um elo
fictivo, artificial mesmo, entre o passado da memória e o futuro que ainda se
deve desenhar\footnote{ Cf. MEGALE, Heitor. \textit{Op. Cit}., p.p. 35 e 36.}. 

Como salienta Volker Mertens, a prosa, linguagem da Bíblia, é, por excelência, a
forma discursiva do verídico. Distinguindo-se, em certa medida, do verso como
“linguagem das \textit{res fictae}”, da ficção, a prosa estaria apta a produzir
a impressão de veracidade, como narrativa das \textit{res factae}, os episódios
verídicos, o que aproxima os gêneros retóricos do \textit{roman} e da crônica
historiográfica\footnote{ Cf. MERTENS, Volker.\textit{ Op. Cit.}, p. 145. }. 

\section{A gesta da Matéria da Bretanha}

A primeira referência à Távola Redonda ocorre em uma hagiografia (narrativa
moralizante e exemplar acerca da vida de um santo) bretã, redigida em latim, a
\textit{Legenda Sanctii Goeznovii}. Tal ocorrência é antecedida por extensa
produção textual efetuada no lastro da cultura celta, especialmente nas regiões
das Ilhas Britânicas e na Armórica (território da Gália celta hoje
correspondente à Bretanha francesa). Sendo consensual a transmissão do espólio
arturiano por meio da oralidade, pensa-se na possibilidade de uma influência de
narrativas celtas galesas sobre os ciclos de compilação, versificação e
prosificação da Matéria da Bretanha nos séculos \versal{XII} e \versal{XIII}. Neste sentido, uma
referência residual seriam os contos compilados sob o nome de
\textit{Mabinogion}\footnote{ Estes contos celtas, cujo título original galês é
\textit{Y Mabinogi} constituem-se de quatro ramos de narrativas, cujos
manuscritos completos remanescentes são o \textit{White Book of Rhydderch}
(\textit{Llyfr Gwyn Rhydderch}, c.1350) e o \textit{Red Book of Hergest}
(\textit{Llyfr Coch Hergest}, c.1400). Um possível local de compilação destes
contos orais seria a abadia galesa de Llanbadarn. Muitas vezes atribuídos ao
monge local Rhygyfarch, tais escritos podem ter sido produzidos na segunda
metade do século \versal{XI}.} (contos para a infância). Há, entretanto, problemas de
datação referentes a tal compilação, o que sendo que a associação dos
\textit{romans} de autores bretões como Chrétien de Troyes e Robert de Boron
aos \textit{Mabinogion} (\textit{Y Mabinogi}) enfrentaria problemas também de
definição do vetor de influência. Como as narrativas celtas apenas se fizeram
conhecer tardiamente, podem ser antes tributárias dos \textit{corpora}
arturianos que suas ancestrais.

Os \textit{Mabinogion}, provenientes da tradição medieval dos celtas de Gales
(\textit{mittelkymrische Erzählungen}), legaram à posteridade quatro textos,
“de conteúdo muito arcaico e próximo ao mito e, para o erudito vienense Helmut
Birkhan, de indubitável lugar na literatura mundial”\footnote{ (...) \textit{Zu
Recht tragen nur vier Texte sehr archaischen, mythosnahen Inhalts und von
unzweifelbar weltliterarischem Rang den Titel: “Vier Zweige des Mabinogi”
(Pedeir Keinc y Mabinogi)}. Cf. BIRKHAN, Helmut. \textit{Keltische Erzählungen
vom Kaiser Arthur}. Wien: Lit Verlag, 2004., p. 33. }. Esse autor assinala que
se poderia tratar, neste caso, de manuais de instrução para aprendizes de
bardos, portadores de aventuras heróicas, a serem memorizadas, que encontrariam
paralelo nas \textit{Enfances }francesas ou nos \textit{Macgnímartha} ou “Atos
dos Meninos” dos celtas da Irlanda. Esses escritos são atribuídos a uma
personagem constante de seu próprio enredo, o bardo (\textit{cyfarwydd}) celta
Dafydd ap Gwilyn (provavelmente no século \versal{VI} d.C.), que Birkhan compara ao
trovador alemão Walther von der Vogelweide. Tais fontes não compõem, todavia,
uma unidade, apesar de manterem um traço comum, justamente a presença de
elementos depois apropriados pela Matéria da Bretanha. 

 São relevantes, para os estudos arturianos, os contos \textit{Kulhwch e Owein}
e \textit{O sonho de Rhonabwy}, com notórios paralelos na cultura escrita
medieval de expressões francesa e alemã. De acordo com Helmut Birkhan, o
primeiro conto, por denotar afastamento em relação à morfologia clássica das
novelas de cavalaria centro-medievais, é decisivo para equacionar a “questão
dos Mabinogion”. Esse escrito apresenta uma senha de especulação sobre o
possível itinerário de apropriação pelo qual teriam transitado os
\textit{Mabiogion}, se não influenciado pela recepção continental das
narrativas arturianas. Apesar de presentes aos contos celtas, Artur e seus
cavaleiros são referidos por caracteres diversos daqueles assinalados nos
\textit{romans} de Chrétien de Troyes.

\textit{Kulhwch e Owein }traz ainda outra especificidade ante as novelas
centro-medievais, qual seja, um tema arcaico, conforme o qual as aquisições de
Owein não se devem a suas habilidades heróicas, mas à ação coletiva da corte, o
que permite a aparição do Rei Artur como um \textit{primus inter pares} quanto
aos cavaleiros dessa corte, destacando-se por seus atos de heroísmo. Os
\textit{romans} da tradição continental apresentam um Artur que se aparta das
batalhas e se vê ameaçado quando ocupa a posição de protagonista. Com efeito,
os escritos novelescos arturianos revestem-se de um caráter de rito de
iniciação à cavalaria, vez que a aventura é “distribuída” por personagens que
agem solitárias, aumentando seu valor por meio de seus atos. 

Além dos \textit{Mabinogion}, outros escritos de antiga tradição celta insular
apresentam referências ao Rei Artur, como o \textit{Livro Negro de Carmathen}
(\textit{Das Schwarze Buch von Carmathen}), que data de cerca do ano 1000
(portanto pré-normando), em que o monarca se faz acompanhar de Key, figurando
ambos como campeões de Hexen, ocasião em que teriam conhecido um gato
gigantesco maravilhoso. O mesmo livro relata uma pugna, nos montes que
circundam Edimburgo, entre os dois heróis e homens cinocéfalos. Da mesma forma,
em outro conto gaélico, \textit{O saque do inframundo} (\textit{Preideu
Annwvyn}), narra-se a imersão do Rei Artur no Além céltico (a Ilha dos Mortos,
Avalon), de onde teria trazido um caldeirão mágico e sua espada maravilhosa
\textit{Caledvwlch,} depois denominada \textit{Excallibur}, que havia estado
sob a tutela de nove virgens no supramundo (\textit{Oberwelt})\footnote{ O nome
\textit{Excallibur} aparece em uma novela inglesa de fins do século \versal{XIII},
denominada \textit{Arthour and Merlin}. Patrick Ford propõe que
\textit{Caledvwlch }deriva do galês \textit{caled} (“duro, forte”) e
\textit{vwlch }ou \textit{bwlch} (“ponta”). Já outro autor, Heinrich Zimmer,
preconiza que a fonte de referência para a espada de Artur seria
\textit{Caladbolg}, a espada da personagem-título do poema holandês
\textit{Fergus}. Em algumas versões, como \textit{Morte Arthure}, o Rei possui
duas espadas, \textit{Clarent} (\textit{Guerra}) e \textit{Claris}
(\textit{Peace}). O nome Excallibur, por sua vez, advém de outra versão das
narrativas arturianas, conforme a qual a espada originária do Rei,
\textit{Calliburnus}, teria sido partida em duas em uma batalha contra um
cavaleiro anônimo que guardava uma fonte. Aconselhado por Merlin, Artur lança a
espada partida no lago onde habita a Dama do Lago, que lhe restitui uma nova
espada, forjada a partir dos fragmentos da anterior, portanto, uma espada
\textit{ex Calliburnu}, daí se originando \textit{Excallibur}.\textit{ }Cf.
LITTLETON, Scott. MALCOR, Linda. \textit{From Scythia to Camelot}. A Radical
Reassessment of the Legends of King Arthur, The Knights of the Round Table and
The Holy Grail. New York: Routledge, 1994, p. 190. }. Tal narrativa foi
atribuída ao bardo galês do século \versal{VI} Taliesin, declamador na corte do rei
Urien (ou Urbgen), do reino celta escocês de Rheged. Em \textit{Bran, Filha de
Ll\^yr}, também se fala de uma expedição militar à Hibérnia (atual Irlanda),
comandada por Artur, com o fito de apossar-se de um caldeirão mágico. O
caldeirão de \textit{Annwvyn}, ao lado de inúmeros outros presentes nas
culturas celtas, e a cornucópia celta da fartura são prováveis antecessores do
Santo Graal. 

Outro exemplo da recorrência do tema arturiano entre os celtas provém de
\textit{A Estória da Condessa da Fonte}, em que se apresenta o herói Owein (ou
Ewein), filho do já mencionado Uryen de Rheged, localidade na Escócia austral,
esse último um dos três “reis benditos” (\textit{gwynderyn}) das Ilhas
Britânicas. O aludido conto também se atribui ao bardo da corte de Uryen,
Taliesin, sendo que, na \textit{Historia Britonum}, Nennius relata que o velho
rei (Uryen ou Urbgen) enfrentou o rei anglo Teodorico (que teria reinado entre
572 e 579 d.C.). Esse embate celta contra os anglos viu-se retratado no
\textit{Lamento} (\textit{Klagenlied}) de Taliesin quando Teodorico é
denominado “flamejante” (\textit{Flamddwyn}) e referido como
\textit{Theodoricus de Bernicia}, o Dietrich von Bern\footnote{ A mesma
descrição pode ser encontrada na canção de outro bardo celta coevo,
identificado como Llywarch Hen. Cf. BIRKHAN, Helmut. \textit{Op. Cit}., p. 21.
}. Essa personagem foi referida, ainda, na \textit{Canção dos Nibelungos}
(\textit{Nibelungenlied}), como o rei dos amelungos. 

Em outra fonte, \textit{O sonho de Ronabwy} (\textit{Breudwyt Ronabwy}),
desenha-se uma rivalidade entre Owein e Artur. Ambos são descritos disputando
um jogo de tabuleiro, enquanto seus guerreiros assassinavam-se reciprocamente
em uma contenda. Cada movimento efetuado no tabuleiro implicava um golpe no
campo de batalha. Interessa salientar que, em \textit{Peredur, Filho de
Evrawc}, o “Castelo das Maravilhas”, objeto da demanda do herói, observa-se
descrito como castelo do tabuleiro mágico de xadrez. 

Ainda em terras britânicas, os temas arturianos conheceram ampla difusão em
latim, em obras da Primeira Idade Média (séculos \versal{IV} a \versal{VIII}) como \textit{De
excidio et conquestu Britaniae}, do prelado galês Gildas (c. 504--570)\footnote{
Helmut Birkhan apresenta uma narrativa galesa de cerca de 1188, o
\textit{Itinerarium Kambriae}, atribuído a Giraldus Cambrensis, em que Artur
teria assassinado o irmão do próprio Gildas. O narrador semi-anônimo ainda se
refere, em Caerlon, a primeira corte do Rei Artur, à presença de um mago,
Myrddin, uma possível prefiguração do Mago Merlin. Giraldus estatui um vínculo
entre os videntes celtas de Gales e a vidente Cassandra, de Tróia, reverberando
a tendência messiânica de tais populações celtas, bem como seu desejo de
estabelecer uma mitologia das origens que os vinculasse aos troianos. Cf.
BIRKHAN, Helmut. \textit{Op. Cit}., p.p. 19, 24 e 25. }, que descreve a invasão
de hordas anglo-saxãs à Britânia romana e as tentativas de resistência da
população romano-bretã, sob a liderança de Artorius, a \textit{Historia
Britonum} (c. 800 d.C.), de Nennius, a \textit{Gesta regum anglorum }(1125), em
que William of Malmesbury apresenta Artur e seu sobrinho, Galwain, como
personagens históricas referidas à narrativa das origens da monarquia
britânica, confirmando suas virtudes guerreiras e denegando as expectativas
messiânicas acerca do retorno do rei da Ilha de Avalon. Ademais, em
\textit{Historia regum Britaniae }(1100--1155), que o erudito arturiano Volker
Mertens considera o “momento fundador” da tradição arturiana, Geoffrey of
Monmouth alude, a par das virtudes bélicas do herói, a sua generonisade,
datando sua ascensão ao trono de Logres aos 15 anos de idade, predicando-lhe o
mesmo estatuto de figura histórica atribuído a Carlos Magno. Este compilador
clamava ter escrito com base em \textit{auctoritates} como Nennius, o Venerável
Beda ou Gildas, a par de um livro escrito “em língua britânica”, que estaria
traduzindo, entregue pelo arquediácono Walter. O contributo fundamental de
Monmouth para a gesta mítica de Artur seria sua caracterização -- inaugural --
como conquistador galês contemporâneo do imperador romano do Oriente Leão \versal{I}
(457 a 474 d.C.). Algumas fontes não aventadas por Megale, porém citadas por
Littleton e Malcor, são a \textit{Historia Anglorum} (c. 1129), de Henry of
Hundingdon, que situa o reinado de Artur entre 527 e 530 d.C., e o
\textit{Chronicon Montis Sancti Michaelis in Periculo Maris}, que associa o rei
à data de 421 d.C. 

Também se encontram alusões à corte de Artur na \textit{Historia Ecclesiastica
Gentis Anglorum }(731), do Venerável Beda\footnote{ Cf. MEGALE, Heitor.
\textit{Op. Cit}., p.p. 30 e 31. }. Não por acaso, Helmut Birkhan assinala que,
na Alta Idade Média (séculos \versal{IX} a \versal{XI}), os habitantes de Gales estavam
plenamente convencidos da existência “histórica” do Rei Artur, a quem se
atribui um túmulo no elenco de sepulcros \textit{Beddeu}, constante do
\textit{Livro Negro de Camarthen}. 

Na Europa continental, o advento dos mitos arturianos, oralmente cultivados
desde o século VI nas Ilhas, deve-se, de acordo com a maioria dos estudiosos,
ao contato com os principados celtas britânicos a partir da Batalha de
Hastings. Os primeiros \textit{romans} em verso com estes motivos originaram-se
na Armórica (depois chamada Bretanha), sob a pluma do clérigo intermediário
Chrétien de Troyes, na segunda metade do século \versal{XII}. Esta primeira versão de
\textit{Perceval ou Le conte dou Graal} foi redigida sob os auspícios do conde
Felipe, de Flandres, e evidências indicam que a primeira versão continuadora,
anônima, guarda relações com a Burgúndia ou a Champanha em princípios do século
\versal{XIII}, atingindo a Picardia e a região de Paris apenas décadas mais tarde. 

A segunda versão, atribuída a Wauchier de Denain, provavelmente foi composta
para Joana, a neta do conde Felipe de Flandres, entre 1212 e 1244, para quem o
mesmo já havia dedicado alguns escritos, bem como algumas hagiografias a seu
tio, o conde de Namur. A terceira versão seria redigida por Manessier, tão
incógnito como Wauchier de Denain, possivelmente para a mesma destinatária,
como narrativa de legitimação de sua pretensão ao trono de Flandres,
questionada por um nobre que se pretendia seu pai. A corte senhorial dos condes
flamengos desenvolveu a tal grau o mecenato que Richard Barber incorre na
afirmação, algo temerária, de que o \textit{roman} redigido por Chrétien de
Troyes era reputado “propriedade da família governante”, em virtude das
associações dinásticas. A propósito, um manuscrito da terceira versão, sem o
prólogo e os versos finais, teria sido compilado para João \versal{II} de Avesne, que
clamava o trono de Flandres no século \versal{XIII}, com vinculações linhageiras
referidas ao imperador romano-germânico.

Por outro lado, o autor da quarta versão permite-se conhecer melhor, em função
de outros escritos que o notabilizaram, como o \textit{Roman de la Violette},
dedicado à condessa de Poitou, entre 1227 e 1229. Trata-se de Gerbert de
Montreuil, que Barber supõe frenquentador da corte régia em Paris e, de forma
muito indiciária, um ator social da cultura intermediária, que “parece ter tido
um pé em ambos mundos, clerical e aquele dos menestréis e dos jograis”, esses
últimos em estreito contato com a cultura popular. 

A par das obras completas \textit{Le Chevalier de la Charrette }(1179), que
apresenta a infâmia dos amores clandestinos do cavaleiro Lancelot e da rainha
Guinevere, e \textit{Eric et Enide}, uma ode às proezas guerreiras da cavalaria
e uma advertência para que os bravos cavaleiros não contraiam matrimônio, sob
pena de acovardar-se, Chrétien de Troyes falece antes de concluir
\textit{Perceval ou Le conte dou Graal}. A par das quatro continuações
mencionadas, houve ainda dois prólogos ao \textit{roman} em verso, ambos de
autoria anônima. O primeiro deles, menos extenso, identificado por Richard
Barber como \textit{O Prólogo de Bliocadran}, enfatiza a genealogia de Perceval
(Bliocadrian aqui figura como seu pai), a própria linhagem sagrada de
protetores do Santo Graal. Há uma notória insistência em aspectos negativos --
assim considerados sob a perspectiva de uma normativa eclesial -- da cavalaria,
como a propensão à guerra e suas formas ressignificadas, o torneio e a justa.
Trata-se de um alerta moralizante acerca dos perigos da guerra e de como a
demanda por glórias nos feitos de bravura redunda em traição aos ideais da
cavalaria cristã.

Já no segundo escrito, o \textit{Prólogo da Elucidação}, narram-se aventuras
preliminares ao conto do Santo Graal, em que ocorre o roubo de copas mágicas e
o estupro de virgens habitantes de uma floresta, nas imediações de poços. A
violência se dá por parte do rei Amangons e seus cavaleiros, assumindo Artur e
seus cavaleiros o dever de vingá-las. No mais, há uma clara repetição do enredo
apresentado por Chrétien de Troyes, com a distinção de que aqui o Santo Graal,
cuja aparência ecoa a primeira e anônima continuação a \textit{Perceval ou Le
conte dou Graal}, encontra-se bastante dessacralizado, sendo apenas mais uma
entre tantas aventuras cavaleirescas. O que esse Prólogo apresenta de inédito é
uma descrição dos sete ramos da estória da corte do rico Rei Pescador, que
aludem aos sete sacramentos da ortodoxia católica, definidos desde o século \versal{XII}
e ratificados no Quarto Concílio de Latrão (1215).

A denominada \textit{Vulgata da Matéria da Bretanha} representa a primeira
prosificação pela qual passou o espólio anterior em versos, ao redor de 1220.
Abrange a sequência narrativa das novelas \textit{Estoire de Merlin},
\textit{Estoire dou Graal, Lancelot du Lac }(\textit{roman} redigido em três
livros, que ocupa mais de metade desse primeiro ciclo), \textit{La Queste del
Saint Graal} e \textit{La Mort le roi Artu. }Com efeito, detectou-se que
\textit{Lancelot du Lac}, \textit{La Queste del Saint Graal} e \textit{La Mort
le roi Artu} foram redigidos antes de \textit{Estoire dou Graal} \textit{e
Estoire de Merlin}, cabendo a primazia cronológica ao primeiro. 

Como expõe Heitor Megale, a constituição plena do \textit{Ciclo da Vulgata}
exigia a redação das \textit{Suites} ao \textit{roman} sobre o Mago Merlin, com
as necessárias acomodações para tornar coerentes as novelas. Esse primeiro
ciclo de prosificação denominou-se também \textit{Ciclo do}
\textit{Lancelot-Graal}, o que desvela a fusão das massas narrativas
pertinentes ao Cavaleiro Lancelot, mais antiga, e ao Santo Graal, posterior. A
propósito, a narrativa relativa a Lancelot não figura no \textit{Ciclo da
Post-Vulgata}. O \textit{Ciclo do} \textit{Lancelot-Graal} conheceu incontáveis
cópias que geraram uma abundante tradição manuscrita no Ocidente europeu
medieval, o que atesta, à evidência, uma difusão ímpar, sem qualquer paralelo
conhecido, da \textit{Matéria da Bretanha }no universo medieval. No
\textit{Ciclo da Post-Vulgata}, a \textit{Estoire dou Graal} passa também a ser
referida como \textit{O Livro de José de Arimatéia}. Alguns autores referem-se
a \textit{Lancelot du Lac}, \textit{Queste del Saint Graal} e  \textit{La mort
le Roi Artu}, em conjunto, como \textit{Lancelot en prose}, apesar de outros
empregarem tal expressão apenas para designar o \textit{Lancelot du Lac}.
Observe-se que as expressões \textit{Ciclo da Vulgata} e \textit{Ciclo da
Post-Vulgata} devem-se à terminologia proposta pela estudiosa Fanny Bogdanow,
em seu ensaio \textit{The Romance of the Grail }(1966). 

O \textit{Ciclo da Vulgata}, conservado em 6 manuscritos dos 100 compilados
entre os séculos \versal{XIII} e \versal{XV}, foi identificado a um só autor ou compilador,
apesar da improbabilidade de se deverem todas as novelas a uma pena solitária.
Esse escriba seria o galês Gautier Map ou Walter Map, porém já há tempos é
denominado Pseudo-Map, pois já era falecido tal compilador quando da primeira
prosificação. O primeiro \textit{roman} a integrar esse primeiro ciclo de
prosificação da Matéria da Bretanha, \textit{Lancelot du Lac}, atribuído ao
suposto clérigo galês Gautier Map, foi compilado em francês. O escriba era
arquediácono da sé de Oxford e cortesão do Rei Henrique II, tendo falecido em
cerca de 1209. Scott Littleton e Linda Malcor assinalam que o compilador exibia
bons conhecimentos da geografia da região de Poitou, parcas noções sobre aquela
relativa ao sudeste da Bretanha e praticamente nenhuma acerca de Gales. Para os
mencionados autores, existiria um consenso entre os especialistas no
\textit{Ciclo da Vulgata}: a novela teria, efetivamente, sido escrita nas
cercanias de Poitou, em cerca de 1200--1210 d.C., combinando elementos de
\textit{Le Chevalier de la Charrette}, de Chrétien de Troyes, e de
\textit{Lanzelet}, do poeta Ulrich von Zatzikhoven, escrito entre 1194 e
1205\footnote{ Cf. LITTLETON, Scott. MALCOR, Linda. \textit{Op. Cit.}, p.p.
82--84. }. O(s) compilador(es) ocultou-se ou ocultaram-se sob seu nome para
atrair, em procedimento muito comum para a Idade Média, seu prestígio e a
aceitação futura de seus manuscritos. O que se pôde averiguar, posteriormente,
foi a possível autoria da \textit{Estoire dou Graal} e da \textit{Estoire de
Merlin}, atribuídas a Robert de Boron.

Já no segundo ciclo de prosificação, o do \textit{Pseudo-Boron}, houve uma
expressiva redução da matéria narrativa, com a eliminação daquela relativa a
Lancelot, ao passo que \textit{A Demanda do Santo Graal} e \textit{A Morte do
Rei Artur} foram acoplados em um único volume, reduzindo-se a matéria do
derradeiro. O \textit{Livro de José de Arimatéia} encerra praticamente o mesmo
conteúdo da versão primeira da \textit{Estoire dou Graal.} Essa segunda
prosificação, inicialmente atribuída a Robert de Boron, fez-se conhecer como
\textit{Ciclo do }\textit{Pseudo-Boron} ou da \textit{Post-Vulgata}. Richard
Barber, entretanto, pondera que a matéria narrativa concernente ao Graal pode
ter sido uma interpolação advinda de outro compilador. A última apresenta uma
introdução coerente ao principal \textit{roman}, \textit{A Demanda do Santo
Graal}. O mencionado historiador inglês propõe ainda outro argumento, afirmando
que o vínculo entre as novelas não elidiu completamente suas contradições. A
\textit{Estoire de Merlin} compunha o ciclo de prosificação de Rober de Boron
(\textit{Ciclo da Vulgata}), a que subjaz uma coerência narrativa entre as
novelas acerca de Artur e do Graal, o que confirma que o segundo ciclo de
prosificação ducentista deve-se mesmo a um Pseudo-Boron, como evidenciou Heitor
Megale. Todavia, uma \textit{Estoire dou Graal} também compõe o \textit{Ciclo
da Post-Vulgata}. Mais uma menção introdutória ao Graal ocorre em
\textit{Lancelot du Lac}, provavelmente fruto de outra interpolação tardia.
Nesse \textit{roman}, institui-se, pela primeira vez, a aventura cavaleiresca
para descobrir-se quem é o virtuoso cavaleiro digno do Graal\footnote{ Cf.
BARBER, Richard.\textit{ The Holy Grail}. Imagination and Belief. Cambridge:
Harvard University Press, 2004.,  p.p. 71 e 72. }. 

A explanação da gesta do Ciclo Arturiano, bem como de \textit{Tristan}, deveu-se
a autores do renome de Ferdinand Lot, Albert Pauphilet, Jean Frappier e
Alexandre Micha, porém competiu a Fanny Bogdanow a proposição de uma exegese
unitária do C\textit{iclo da Post-Vulgata,} como possível conversão da
\textit{Vulgata} em enredos mais breves e homogêneos. Consagra-se uma trilogia
de \textit{romans} que não se encontram representados em manuscritos, mas em
inúmeros fragmentos de duas naturezas, vestígios de códices despedaçados, ou
seções narrativas inseridas em manuscritos de outros ciclos, propensos à
transmissão oral.

No concernente ao texto de \textit{A Demanda do Santo Graal}, permaneceram
apenas três versões, vale citar, a bretã, a portuguesa e a alemã. Com relação a
uma possível versão inglesa, não se reteve qualquer exemplar original ou
posterior, porém sua existência poderia ser supostamente comprovada, para
alguns autores, de forma indireta, pela influência sobre a obra de Sir Thomas
Malory, \textit{Le Morte d’Arthur} (\textit{sic}), concluída em 1470 e impressa
em 1485, para o rei Henrique \versal{VII} da Inglaterra. Tal a hipótese aventada pela
estudiosa Elizabeth Bryan. Como leciona Richard Barber, o século \versal{XV} testemunhou
uma demanda por textos mais breves e coerentes sobre o rei Artur e o Santo
Graal, a que Malory responde com um desejo de conhecimento “verídico” de sua
“história”. Neste espectro, \textit{Le Morte d’Arthur} também pode ter sido
escrito com base em um poema inglês medieval, \textit{Morte Arthure}. No texto
quatrocentista, o Santo Graal, claramente uma relíquia, aparece no contexto da
desolação do reino de Logres, tema logo abandonado, no entanto. Em que pese o
fato de que Malory procura introduzir uma visão laica a respeito do Santo
Graal, sobretudo aludindo à Crucificação e ao \textit{Evangelho Apócrifo de
Nicodemos} (fins do século \versal{IV}), aqui o Santo Graal se apresenta
indissoluvelmente vinculado ao Mistério Eucarístico e à doutrina ortodoxa da
Transubstanciação.  Há, ainda, uma adaptção de \textit{A Demanda do Santo
Graal}  para o irlandês, denominada \textit{Logaireacht an tSoidigh Naomhtha},
bem como para o galês, \textit{Y Seint Greal.}

\section{A Demanda do Santo Graal alemã (Gral-Queste)}

Quanto ao universo cultural alemão, observe-se que  \textit{A Demanda do Santo
Graal} \textit{(Die Suche nach dem Gral) }e \textit{A Morte do Rei Artur (Der
Tod des Königs Artus)} compõem a terceira parte do Códice 147 da Biblioteca
Palatina Germânica de Heidelberg (\textit{Codex Palatinus Germanicus} 147). A
primeira parte do manuscrito contém uma adaptação do \textit{Lancelot du Lac}
do Primeiro Ciclo de Prosificação. Esses textos baseiam-se, essencialmente, no
\textit{Ciclo da Vulgata} ou \textit{Ciclo do Lancelot-Graal}, e não no
\textit{Ciclo da Post-Vulgata}, em que pese situar-se o início de sua
compilação na segunda metade do século \versal{XIII}\footnote{ Cf. ANDERSEN, Elizabeth
A. “The Reception of Prose: The \textit{Prosa-Lancelot}”. In: VVAA. \textit{The
Arthur of the Germans}. The Arthurian Legend in Medieval German and Dutch
Literature. Cardiff: University of Wales Press, 2000., p. 156. }. O manuscrito
alemão integral, também designado como \textit{Prosa-Lancelot} pelos
estudiosos, apresenta lacunas. O erudito que analisou e adaptou, para o alemão
contemporâneo, a versão de Heidelbeg de\textit{ A Demanda do Santo Graal},
Hans-Hugo Steinhoff, elucida que, no caso alemão, esse processo de prosificação
foi multissecular. A compilação de \textit{A Demanda do Santo Graal} data da
segunda metade do século XIII, mas a integralidade do códice 147 já aludido não
se configurou antes de 1455, tendo existido interpolações do século XVI. O
mencionado prelado adverte, entretanto, que o texto alemão apresenta variações
e especificidades que o afastam do \textit{corpus} bretão que embasou os
trabalhos de compilação\footnote{ STEINHOFF, Hans-Hugo. “Der deutsche Text”.
In: \textit{Die Suche nach dem Gral. }Heidelberg: Deutscher Klassiker Verlag,
2004.  }. Os estudiosos da versão alemã de \textit{A Demanda do Santo Graal},
por vezes referida como \textit{Gral-Queste}, supõem que a mesma não teria sido
adaptada para o alemão, diretamente, com base no texto homônimo francês, mas a
partir de um hipotético escrito que se teria compilado na região do Reno em
meados do século \versal{XII}, em alemão ou holandês\footnote{ Cf. JACKSON, W.H.
RANAWAKE, Silvia. \textit{Op. Cit}., p.p. 03 e 04. Cf. SPECKENBACH, Klaus.
“Prosa-Lancelot”. In: \textit{Interpretationen}. Mittelhochdeutsche Romane und
Heldenepen. Stuttgart: Reclam, 2007., p. 328. }. Com efeito, um manuscrito
holandês preservado, que narra as aventuras do cavaleiro Lancelot, apresenta
forte similaridade com a parte primeva do \textit{Prosa-Lancelot} alemão,
datada de cerca de 1250, correspondente ao \textit{Lancelot du Lac} do Ciclo da
Vulgata\footnote{ Cf. ANDERSEN, Elizabeth A. \textit{Op. Cit}., p. 156. }.
Elizabeth Andersen detectou uma lacuna de cerca de 1/10 da narrativa entre os
dois primeiros livros do \textit{Prosa-Lancelot}, se comparados aos escritos
franceses. Ainda assim, o livro \versal{I}, relativo a Lancelot, é três vezes mais longo
que as narrativas combinadas de \textit{A Demanda do Santo Graal }e \textit{A
Morte do Rei Artur}. A mesma autora supõe a eventual existência de uma versão
não cíclica de \textit{Lancelot}, que depois teria sido melhor desenvolvido no
\textit{Lancelot du Lac} do \textit{Ciclo do Pseudo-Map}. Sua hipótese se
fundamentou na percepção de que há muitas homologias entre o primeiro livro do
\textit{Prosa-Lancelot} e o texto francês, como o episódio da humilhação de
Lancelot ao adentrar a charrete, para resgatar Guinevere ou a assunção de
Galahad aos Céus, no final da \textit{Gral-Queste}\footnote{ \textit{Idem
ibidem}. Para Volker Mertens, desenha-se uma relação entre todo o
\textit{Prosa-Lancelot} alemão e o \textit{Le chevalier de la charrette}, de
Chrétien de Troyes. Cf. MERTENS, Volker. \textit{Op. Cit}., p. 147}. 

O corpus alemão integral parece ter sido dedicado aos condes de Hesse, Frederico
I, o Vitorioso (1425--1476), e Matilde de Rottenburg, cuja corte se localizava
em Heidelberg. Como os exemplares francês e português, a \textit{Demanda do
Santo Graal} alemã revela forte influência do pensamento cisterciense,
sobretudo referido ao monastério de Gottesthal, no ducado de Lemburgo\footnote{
Cf. ANDERSEN, Elizabeth A. \textit{Op. Cit}., p. 157. Entretanto, outro
estudioso do documento alemão, Klaus Speckenbach, afirma a inexistência do
espírito cruzadista próprio à Ordem de Cister quanto a Galahad. Cf.
SPECKENBACH, Klaus. \textit{Op. Cit}., p. 340. }. A propósito, convém observar
que os \textit{romans} arturianos alemães foram cultivados, principalmente,
pelas cortes senhoriais, raramente pelas principescas, interessadas aquelas na
difusão da imagem do cavaleiro laico enquanto campeão da justiça. Em tais
ambientes aristocráticos alemães, entrevia-se forte presença de mulheres
letradas, que conviviam com cavaleiros por vezes iletrados, tendo impulsionado
a propagação de escritos em vernáculo. Jackson e Ranawake salientam que os
motivos arturianos conheceram especial difusão nas regiões meridionais do Sacro
Império Romano-Germânico\footnote{ Cf. JACKSON, W. H. RANAWAKE, Silvia.
\textit{Op. Cit}., p. 06. }. 

O \textit{Codex Palatinus Germanicus} 147 gerou dez manuscritos copiados, a par
de uma edição impressa de 1576, tendo representado, ainda, um ponto de inflexão
nos escritos em prosa da tradição alemã. Em primeiro lugar, como destaca Volker
Mertens, o \textit{roman} intoduz, nos círculos letrados alemães, a exemplo do
francês e o flamengo, o hábito da leitura em pequenos grupos ou mesmo
individual, paralela à declamação que ainda ocorre, nas cortes, do conteúdo
desses escritos\footnote{ Cf. MERTENS, Volker. \textit{Op. Cit}., p. 146. }.
Antes do \textit{Prosa-Lancelot}, não havia, na tradição escrita alemã, a prosa
novelesca, e sim a bíblica e a jurídica. A familiaridade do compilador anônimo
com os códices jurídicos e sermões clericais evidencia-se na sintaxe textual
flexível deste \textit{Roman}, bem como na visível impregnação da
espiritualidade laica das ordens mendicantes do século \versal{XIII}, estando presente
uma perspectiva de História da Salvação, independente de qualquer contexto
específico de tensão política\footnote{ \textit{Idem}, p. 150. }.

\begin{bibliohedra}

\tit{BARBER}, Richard. \textit{ The Holy Grail}. Imagination and Belief. Cambridge:
Harvard University Press, 2004.

\tit{BIRKHAN}, Helmut. \textit{Keltische Erzählungen vom Kaiser Arthur}. Wien: Lit
Verlag, 2004.

\tit{FRANCO JR}, Hilário. “Meu, teu, nosso. Reflexões sobre o conceito de cultura
intermediária”. In: \textit{A Eva Barbada. }Ensaios de Mitologia Medieval, São
Paulo: Edusp, 1996.

\tit{JACKSON}, W. H. \versal{RANAWAKE}, Silvia A.(org.) \textit{The Arthur of the Germans}. The
Arthurian Legend in Medieval German and Dutch Literature. Avon: University of
Wales Press, 2000. 

\tit{LITTLETON}, Scott. \versal{MALCOR}, Linda. \textit{From Scythia to Camelot}. A Radical
Reassessment of the Legends of King Arthur, The Knights of the Round Table and
The Holy Grail. New York: Routledge, 1994.

\tit{MEGALE}, Heitor. \textit{A Demanda do Santo Graal}. Das origens ao Códice
Português. Cotia: Ateliê 

\tit{MERTENS}, Volker. \textit{Der deutsche Artusroman}. Stuttgart: Reclam,
2007Editorial, 2000.

\tit{SCHMITT}, Jean-Claude. \textit{Le corps, les rites, les rêves, le temps}: essais
d’anthropologie médiévale, Paris: Editions Gallimard, 2007.

\tit{RÉGNIER-BOHLER}, Danielle. “O amor cortesão”. In: \versal{LE GOFF}, Jacques. \versal{SCHMITT},
Jean-Claude. \textit{Dicionário Temático do Ocidente Medieval}, São Paulo:
Edusc, 2002.
\end{bibliohedra}


