\SVN $Id: PRETAS.tex 6321 2010-04-27 19:41:53Z oliveira $ 
\begin{resumopage}

\item[Johan August Strindberg] (Estocolmo, 1849--\textit{id.}, 1912) 
foi escritor, dramaturgo, pintor e fotógrafo sueco. Após concluir
seus estudos, dedica"-se à carreira de professor, ao mesmo tempo em que
estuda medicina. Mais tarde tenta lançar"-se como ator, mas em 1870 vai
estudar na universidade de Uppsala, onde começa a escrever. Dois anos
mais tarde interrompe os estudos por razões financeiras; passa a
trabalhar no jornal \textit{Dagens Nyheter} e, a seguir, na
Kungliga Bibliotek -- a Biblioteca Nacional da Suécia.
Em 1879, a publicação do livro \textit{Röda Rummet} (A sala vermelha) e
a encenação da peça \textit{Mäster Olof} trazem"-lhe o reconhecimento
merecido. Em 1882, o aparecimento de \textit{Det Nya Riket} (O novo
reino) -- obra de cunho realista, repleta de críticas às instituições
sociais vigentes na época -- rende"-lhe tantas críticas que o autor vê"-se
obrigado a deixar seu país natal. Strindberg muda"-se com a primeira
mulher, Siri von Essen, e os filhos para Paris e então para a Suíça. No
exterior, escreve uma parcela significativa de sua obra, ao mesmo tempo
em que luta contra graves problemas psicológicos. Em 1897, após
divorciar"-se de Frida Uhl, sua segunda esposa, a condição mental de
Strindberg -- já delicada na época -- deteriora"-se ainda mais. Em um
período de crise profunda, atormentado pela paranoia e por surtos
psicóticos, escreve o romance \textit{Inferno}. Após tornar à Suécia em
1897, casa"-se pela terceira vez, em 1901, com a atriz Harriet Bosse,
que lhe dá a filha Anne"-Marie. Nesse período, as leituras e crenças
pessoais de Strindberg influenciam seu estilo, que passa do realismo ao
expressionismo. Rechaçado pela Academia Sueca, que até 
hoje concede o Prêmio Nobel, Strindberg foi agraciado com uma 
distinção sem precedentes: o Prêmio Anti"-Nobel, uma arrecadação 
pública de dinheiro promovida por seus conterrâneos. 
No fim da vida, o autor instala"-se na Blå Tornet -- a “Torre azul” 
onde hoje funciona o museu em sua memória.
Strindberg morreu no dia 14 de maio de 1912, deixando como legado uma
vasta produção de grande valor literário -- entre elas, as peças
\textit{Senhorita Júlia}, \textit{A dança da morte}, \textit{O pai},
\textit{A caminho de Damasco} e \textit{A sonata espectral}, além dos
romances \textit{Inferno}, \textit{O~filho da criada}, \textit{Defesa
de um louco} e \textit{Gente de Hemsö}.

\item[Sagas] (1903) reúne contos que vão de breves peças morais a narrativas oníricas, 
passando por temas históricos, humorísticos e heróicos com igual desenvoltura, trazendo 
ora um tom fabulesco, ora um tom decididamente strindberguiano. Grande parte dessas histórias 
foram inspiradas por acontecimentos na vida do autor. Ainda que as tenha escrito para a filha Anne{}"-Marie, 
que à época tinha apenas um ano, Strindberg não deixou de preocupar{}"-se com seu aspecto literário, 
o que faz dessa obra uma leitura fascinante para todas as idades.

\item[Carlos Rabelo] traduziu a peça \textit{Camaradagem}, de Strindberg, adaptada por Eduardo Tolentino, 
para o Grupo Tapa (prêmio de melhor espetáculo da \textsc{apca}, 2006) e \textit{O amor é tão simples}, de Lars Norèn.

\item[Ivo Barroso] é poeta, tradutor e ensaísta com cerca de 40 livros publicados. 
Organizou para a Nova Aguilar \textit{Charles Baudelaire: poesia e prosa} (1995) e \textit{O corvo e suas traduções} (2000). 
Traduziu a obra completa de Arthur Rimbaud para a Topbooks e o teatro completo de T.~S.~Eliot para a Arx/Siciliano. 
Como poeta, publicou, em Portugal, \textit{Nau dos náufragos} (Minerva, 1981), \textit{Visitações de Alcipe} (Fundação das Casas de 
Fronteira e Alorna, 1992) e, no Brasil, \textit{A caça virtual e outros  poemas} (Record), finalista do prêmio Jabuti de 2001. 
De August Strindberg, traduziu \textit{Inferno} (Nova Fronteira, 1989).

\end{resumopage}

