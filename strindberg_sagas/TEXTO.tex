\SVN $Id: TEXTO.tex 6321 2010-04-27 19:41:53Z oliveira $ 
\chapter{Nos dias de verão}


\textsc{Nos dias de verão} -- quando nas terras do Norte o solo é fértil, o campo
se alegra, a fonte deságua, quando as flores do prado despontam e os
pássaros cantam -- foi então que a pomba veio da floresta e pousou em
frente à casa, onde a mãe de noventa anos estava de cama.

Durante vinte anos a velhinha esteve acamada e pela janela via tudo o
que acontecia na fazenda onde os dois filhos lavravam a terra. Mas via
o mundo e as pessoas de um modo bastante peculiar, pois as vidraças de
seu quarto estavam manchadas com todas as cores do arco"-íris; bastava
virar a cabeça para que tudo ficasse vermelho, amarelo, verde, azul ou
violeta. Se fosse um dia de inverno, de árvores cobertas pela geada,
como se vestissem folhas de prata, virava a cabeça no travesseiro e as
árvores tornavam"-se verdes; se fosse verão, via o campo dourado e o
céu azul, mesmo se, na verdade, o dia estivesse cinza. Assim, achava
que tinha poderes mágicos e nunca se aborrecia. Mas não era esse o
único poder das vidraças mágicas: elas também eram curvas, e podiam
aumentar ou diminuir o que se via lá fora.

Assim, quando o filho mais velho chegava, fazendo travessuras, gritando
pela fazenda afora, ela desejava"-o pequeno e bonzinho; num instante o
via assim. Quando os netos vinham com seus passinhos incertos,
imaginava o futuro deles, e -- um, dois, três! -- passavam em frente ao
vidro, e ela os via crescidos, adultos, verdadeiros gigantes.

Mas quando chegava o verão, a janela ficava aberta, pois a vidraça não
deixava entrever toda a beleza que havia lá fora. 
E na véspera de São João o dia ficava ainda mais lindo;
de sua cama ela contemplava os pastos e as campinas. Foi quando a pomba
voltou a cantar. Entoou canções melodiosas sobre Jesus Cristo e as
graças e a glória do reino dos céus, e deu boas"-vindas aos que sofrem
e aos que não mais suportam os dissabores da vida. 

A velhinha escutou, agradeceu muito mas recusou, pois naquele dia a
terra estava tão bela quanto o próprio céu, e isso lhe bastava.

Então a pomba sobrevoou o prado e alçou"-se em direção ao bosque do
morro, onde um fazendeiro cavava um poço. Estava dentro do buraco, a
sete palmos de fundura, como em seu próprio túmulo.

A pomba pousou em um abeto e cantou a alegria do reino dos céus, para
que o homem lá embaixo, que não podia ver nem céu nem mar nem prado,
ansiasse pelas alturas.

-- Não -- disse o fazendeiro. -- Preciso cavar o poço, senão faltará água
para os meus hóspedes de verão, e a infeliz senhora terá de partir com
sua filha. 

A pomba voou para a praia, onde o irmão do fazendeiro puxava a rede de
pesca, pousou num junco e cantou.

-- Não -- disse o irmão do fazendeiro. -- Tenho de levar comida para casa,
senão as crianças choram de fome.  Mais tarde, bem mais tarde\ldots{} O céu
pode esperar; primeiro a vida, depois a morte!

A pomba voou para a casa onde a jovem infeliz estava hospedada durante o
verão. Sentada na varanda, costurava à máquina. Sob o chapéu de feltro
vermelho, seu rosto era branco como um lírio, e uma papoula repousava em seus cabelos negros 
-- negros como um véu de luto.
Costurava um belo vestido para a filha usar no dia de São João. Sentada
no chão, a menina brincava com os retalhos.

-- Por que papai não vem para casa? 

Era justamente a dolorosa pergunta que a jovem mãe não podia responder,
e, na verdade, nem mesmo o pai, que apascentava sua tristeza em terra
estrangeira -- tristeza duas vezes maior que a da mãe.  

 A máquina de costura funcionava mal, mas dava ponto atrás de ponto, com
tantas agulhadas quanto um coração pode suportar antes de se esvair por
completo; e cada agulhada aproximava ainda mais as fibras -- que curioso! 

-- Quero ir para a cidade, mamãe! -- disse a menininha. -- Quero ver o sol!
Aqui é escuro demais.

-- Você verá o sol à tarde, filha.

Estava mesmo escuro nos rochedos desse lado da ilha, e a casa ficava em
meio a abetos negros, que encobriam a vista até o mar.

-- Também quero que você me compre muitos brinquedos, mamãe.

-- Filha, temos tão pouco dinheiro! -- respondeu a mãe baixando a cabeça
contra o peito.

Era verdade: o conforto se transformara em penúria; 
não havia criados no verão e tinha de fazer tudo sozinha. 

Mas quando viu o rosto entristecido da filha, pegou"-a no colo.

-- Dê um abraço na mamãe! -- disse ela.

A filha abraçou"-a.

-- Dê um beijo na mamãe!

O beijo veio daquela boquinha pequena e entreaberta, como a de um
filhote de passarinho; e ao ver os olhinhos azuis da filha, azuis como
as flores do linho, o rosto formoso da mãe refletiu a serena inocência
da menina e iluminou"-se como o rosto de uma criança feliz ao sol.

``Aqui não canto o reino dos céus'', pensou a pomba, ``mas vou servi"-las
no que puder.''

E voou para a cidade ensolarada, onde tinha um compromisso.

A tarde veio; a jovem mãe caminhava levando um cesto no braço e a filha
pela mão. Nunca estivera na cidade, mas sabia que ficava em direção ao
poente, do outro lado da ilha; um fazendeiro lhe havia dito que para
chegar era preciso passar por seis cercas e seis porteiras. 

E as duas seguiram adiante.

Primeiro andaram por um caminho cheio de pedras e raízes de árvores; a
criança teve de ser carregada, um peso considerável para a mãe. O
médico havia proibido a menina de forçar o pé esquerdo, que de tão
frágil corria o risco de crescer torto. 

A jovem mãe carregava seu doce fardo e gotas de suor perolavam seu
rosto, pois fazia calor na floresta.

-- Estou com sede, mamãe! -- queixava"-se a menininha.

-- Paciência, querida; vamos beber ao chegar.

Beijou os lábios secos da menina, e a criança sorriu, esquecendo a sede.


Mas o sol ardia na floresta, onde as folhas mal se mexiam. 

-- Tente andar um pouco -- disse a mãe, pondo a filha no chão.

Mas o pezinho não se firmava e a menina não pôde andar. 

-- Estou tão cansada, mamãe! -- exclamou a criança, sentando"-se para
chorar.

No chão cresciam lindas campânulas cor"-de"-rosa, que recendiam a
amêndoa; a menina jamais vira florezinhas como aquelas e voltou a
sorrir; a mãe recuperou as forças e seguiu adiante, carregando"-a nos
braços. 

Quando chegaram à primeira porteira, atravessaram"-na e puseram a
tranca de volta com cuidado. 

Foi quando se ouviu um forte relincho, e um cavalo solto atirou"-se na
frente delas, parou no meio do caminho e relinchou; logo vieram outros
relinchos da floresta -- da direita, da esquerda e logo de toda parte; o
chão tremeu, os galhos rangeram e as pedras estalaram. Desamparadas,
mãe e filha viram"-se em meio a uma tropa de cavalos.

A filha escondeu o rosto no peito da mãe; o coraçãozinho palpitava de
angústia, como um relógio.

-- Estou com tanto medo! -- sussurrou a menina.

-- Ó Deus do céu, nos ajude! -- implorou a mãe.

Um melro cantou entre os abetos, e eis que num instante os cavalos se
dispersaram; o silêncio voltou a reinar.

Chegaram à segunda porteira e puseram a tranca de volta.

Lá havia um campo cercado de árvores, onde o sol estava mais forte que
na floresta. Vistos de longe, os torrões acinzentados da encosta
pareciam mexer"-se -- eram um rebanho de ovelhas.

Ovelhas são animais mansos -- em particular os cordeiros; mas não se deve
brincar com o carneiro, besta indócil, que gosta de atacar quem não lhe
fez mal algum. Um carneiro apareceu no meio do caminho depois de saltar
por cima de um fosso. Abaixou a cabeça e recuou, preparando"-se para
investir.

-- Estou com tanto medo, mamãe! -- disse a pequena; seu coração palpitava.

-- Ó Deus misericordioso, nos ajude! -- suspirou a mãe, elevando uma
oração aos céus.

Lá repousava uma pequena cotovia, esvoaçando como uma borboleta. Quando
ela começou a cantar, o carneiro sumiu em meio aos torrões cinzentos.

Assim chegaram à terceira porteira. O chão começou a afundar, os pés
ficaram molhados; era um pântano. Os tufos de erva pareciam pequenos
túmulos com flores brancas -- flores de lã ou de algodão. Era preciso
cuidado para não afundar o pé na lama. Havia ali umas frutinhas pretas,
venenosas. A menina queria colhê"-las, mas a mãe não deixou; e a
menina ficou triste, pois não entendia o que era ``venenoso''.

Seguiram pelo caminho até se depararem com um manto branco em meio as
árvores; o sol se escondeu e elas foram envolvidas por uma terrível
escuridão branca. 

De dentro da nuvem saiu uma cabeça mugindo, uma cabeça com uma estrela
branca e dois chifres curvos; e depois mais cabeças, muitas cabeças,
que se aproximavam cada vez mais.

-- Estou com medo, mamãe! -- sussurrou a criança. -- Estou com medo! 

A mãe deu um passo para o lado e se encolheu entre dois montinhos no
pântano.

-- Ó Deus grande e misericordioso, tende piedade de nós! -- gritou do
fundo de sua alma.

E ouviu"-se o som do vento, que soprava forte pela floresta, vindo do
mar; as árvores se curvaram humildemente diante do grande espírito, um
jovem pinheiro se inclinou, e de sua copa sussurrou algo no ouvido da
mãe abandonada; e quando ela se agarrou a um dos galhos, o pinheiro se
endireitou e tirou"-as do pântano.

A névoa se dissipou, o sol voltou a brilhar e as duas se viram diante da
quarta porteira. A mãe, que havia perdido o chapéu, enxugava as
lágrimas da filha com seus cabelos negros. A pequena sorriu e o pobre
coração da mãe se encheu de alegria; esqueceu da dor e ganhou novas
forças para alcançar a quinta porteira. E finalmente um consolo para
seus olhos: avistou telhados vermelhos e bandeirolas; ao longo do
caminho cresciam arbustos de bola"-de"-neve e rosas selvagens, dois a
dois, como se estivessem de mãos dadas. 

A menina agora já conseguia andar, e enchia o cesto de flores para que a
boneca Lisa dormisse ali dentro na noite de São João e tivesse belos
sonhos.

Iam brincando pelo caminho, despreocupadas; para que chegassem bastava
cruzar um bosque repleto de bétulas. O caminho era íngreme, um pequeno
monte, e, quando chegaram lá em cima e dobraram à esquerda, encontraram
um touro no meio do caminho.

Era impossível fugir; a mãe, pesarosa, caiu de joelhos, pôs a filha no
chão e inclinou a cabeça sobre ela a fim de protegê"-la; os longos
cabelos desciam"-lhe pelo rosto como um véu negro e, de mãos postas,
fez uma prece silenciosa. De sua fronte a angústia fazia pingar gotas
de suor, vermelhas como gotas de sangue.

-- Ó Deus! -- implorou. -- Tire a minha vida, mas salve a pequena!

Então a mãe ouviu um ruflar de asas e, quando olhou para cima, viu uma
pomba branca voar em direção à cidade; e o touro desaparecera.

Encontrou a filha na beira do caminho, colhendo morangos silvestres,
vermelhos como gotas de sangue; e compreendeu então de onde estas haviam surgido.

Chegaram juntas à última porteira e entraram na cidadezinha.

A cidade brilhava ao sol, junto a uma enseada verde e à sombra de
grandes tílias e bordos; via"-se no monte uma igreja branca de
campanário vermelho, o presbitério com lilases, o correio com jasmins e
a casa do jardineiro à sombra de um enorme carvalho. Tudo estava muito
iluminado; as bandeirolas tremulavam ao vento e barquinhos margeavam as
praias e o ancoradouro. Percebia"-se que era véspera de São João. 

Mas lá não havia ninguém. Mãe e filha procuraram uma venda, onde a
menina pudesse beber água. 

Quando chegaram lá, a venda estava fechada.

-- Estou com tanta sede, mamãe! -- queixava"-se a filha.

Foram até o correio -- estava fechado.

-- Estou com tanta fome, mamãe!

A mãe não sabia o que dizer nem entendia por que tudo estava fechado num
dia de semana e não se avistava vivalma. Foi até a casa do jardineiro. 
Estava fechada, e um grande cachorro deitado em frente à porta.

-- Estou tão cansada, mamãe!

-- Eu também, minha filha, mas precisamos encontrar um lugar para beber
água.

E foram de casa em casa, mas tudo estava fechado; a menina não conseguia
mais andar, pois seu pezinho estava cansado e ela mancava. Quando a mãe
viu a criaturinha se curvando, também sentiu cansaço e sentou"-se à
beira do caminho, com a filha no colo. A menina adormeceu.

Uma pomba cantou no arbusto de lilás; cantou a alegria do reino dos céus
e a eterna tristeza e sofrimento do mundo.

A mãe olhou para a filha adormecida. O capuz cingia"-lhe o rostinho,
como as pétalas circundam o lírio. Sentiu que, nos braços, trazia o
reino dos céus. 

Mas a menina acordou e pediu água.

A mãe se calou.

-- Quero ir para casa, mamãe! -- queixou"-se a menina.

-- O mesmo caminho terrível de volta? Jamais! Prefiro ir para o mar --
respondeu a mãe.

-- Quero ir para casa!

A mãe se levantou. Viu ao longe jovens bétulas frondosas atrás de um
monte, e enquanto olhava, as árvores começaram a se mexer e a andar.
Percebeu que lá havia pessoas derrubando as bétulas para os
caramanchões do dia de São João; e pôs"-se nessa direção, à procura de
água.

No caminho, notou uma pequena casa com uma cerca verde e um portão
branco, aberto e convidativo. Cruzou o portão e entrou no jardim, onde
havia peônias e aqüilégias. Notou que todas as cortinas eram brancas e
estavam fechadas. Porém, uma janela na água"-furtada estava aberta, e
entre duas balsaminas se via acenar uma mão alva com um pequeno lenço
branco, como se acena para os que partem.

Entrou no quintal da casa; no meio do gramado alto havia uma coroa de
mirto verde com rosas brancas. Era grande demais para ser uma coroa de
noiva. 

Entrou na varanda e perguntou se havia alguém em casa. 

Como não houve resposta, entrou. No chão estava um caixão preto com pés
de prata, em meio a uma relva de flores. Dentro do caixão jazia uma
moça com uma coroa de noiva na cabeça.

As paredes do quarto eram de tábuas novas de pinho, com uma camada
simples de verniz, deixando os nós da madeira à mostra. Os nós ovais
pareciam olhos escuros, entreabertos. 

A menina percebeu as paredes estranhas e disse:  

-- Veja quantos olhos, mamãe!

Sim, havia todos os tipos de olhos: grandes e eloqüentes, sérios,
olhinhos brilhantes de criança, com um sorriso no canto, olhos
malvados, em que se via apenas a parte branca, olhos abertos, olhos
atentos, olhos que perscrutam o coração, e um grande e doce olho de
mãe, olhando amorosamente para a menina morta; dele escorria uma grande
lágrima de seiva de pinho que, com os raios do sol poente, irradiava
matizes vermelhos e verdes, como um diamante.

-- A moça está dormindo? -- perguntou a menina, quando por fim viu a noiva
morta.

-- Está, sim.

-- É uma noiva, mamãe?

-- Isso, uma noiva.

A mãe reconheceu"-a. Era a moça que se casaria na noite de São João
assim que o marinheiro voltasse para casa; mas quando o marinheiro
escreveu dizendo que voltaria apenas no outono, seu coração se partiu;
não queria esperar até o outono, quando as folhas caem e as tempestades
chegam. 

A mãe escutara e compreendera a canção da pomba.  Quando saiu, já sabia
que caminho tomar. 

Deixou o cesto pesado do lado de fora do portão e levou a filha nos
braços, dirigindo seus passos à próxima campina, última antes da praia.
Era um mar de flores, que sussurravam e murmuravam, colorindo a saia
branca com todos os matizes de pólen; abelhas, vespas e borboletas
esvoaçavam, voejando a seu redor, cantando numa única e brilhante nuvem
multicor. Então desceu com passos leves até a praia.

Quando chegou, viu na enseada um barco com as velas enfunadas, vindo em
direção ao ancoradouro; mas não havia ninguém no leme. Continuou
andando com flores até a cintura, nadando nas flores e em seus aromas,
até que a saia branca ficasse parecendo um campo florido, mas com cores
ainda mais belas.

Parou nos salgueiros da praia; entre o tronco e os galhos havia um
ninho, e quando a árvore balançou com o vento da tarde, embalando três
filhotes de passarinho, a menina quis acariciá"-los.

-- Não, minha filha -- disse a mãe; -- nunca mexa em um ninho.

Assim que as duas chegaram aos escolhos da praia, o barco branco aportou
aos seus pés, mas nele não havia ninguém. 

A mãe tomou a criança em seus braços e subiu a bordo. Logo depois o
barco deixou a enseada. 

Quando passaram pelo monte da igreja, os sinos começaram a soar com
alegria e entusiasmo. 

O barco deslizou pela enseada até sair pela baía, de onde se via o mar
aberto. 

A menina estava radiante de alegria; o mar estava azul e calmo. Já não
singravam as águas, mas um campo de flores de linho, que a menina
colhia esticando a mão.

As flores subiam e desciam como pequenas ondas murmurando nas bordas do
barco. Um campo infinito de linho se abria diante delas. Mas então um
nevoeiro branco as envolveu e elas ouviram o bramido de uma onda. Acima
do nevoeiro soou o canto da cotovia.

-- Como é que as cotovias cantam no mar? -- perguntou a menina.

-- O mar é tão verde que as cotovias tomam"-no por um campo -- respondeu
a mãe.

Então o nevoeiro se dispersou; o céu estava azul como um campo de linho
e a cotovia alçou vôo.

Mãe e filha avistaram uma ilha verdejante com praias de areia branca,
onde pessoas andavam de mãos dadas, vestidas de branco. Um pórtico
dourado brilhava com a luz do sol poente, fogueiras alvas ardiam sob as
oferendas, e sobre a ilha verdejante estendia"-se um arco"-íris verde
e rosa.

-- O que é isso, mamãe?

A mãe não sabia responder.

-- É o reino dos céus que a pomba cantou? O que é o reino dos céus,
mamãe?

-- É um lugar, minha filha, onde todas as pessoas são amigas -- respondeu
a mãe. -- Não há tristeza e tudo é paz.

-- Quero ir para lá, então! -- disse a menina.

-- Eu também quero, minha filha  -- disse a cansada, abandonada e sofrida mãe.

\oneside

\chapter{A grande peneira}
\vspace{0.9em}

\textsc{Era uma vez} uma enguia e seu filhote. Os dois estavam no fundo do mar,
ao lado de um ancoradouro para barcos a vapor, observando um menino que
preparava sua linha de pesca.

-- Olhe bem -- disse a enguia -- para aprender sobre a maldade e as
ciladas do mundo\ldots{} Olhe bem; ele tem um chicote na mão, e ali
arremessa a ponta! Está vendo a chumbada, que afunda com o anzol e a
minhoca? É isso o que você não pode pôr na boca, ou estará perdido! Mas
só as percas e os ruivos são tolos o bastante para cair nessa; agora
você já sabe.

A floresta de algas, com mexilhões e conchas, começou a balançar;
ouviu"-se um marulho e um rumor, e sobre as copas da floresta passou
uma baleia vermelha. Seu rabo lembrava um saca"-rolha.

-- É o barco a vapor! -- disse a enguia mais velha. -- Saia de perto!

Então se fez um barulho horrível lá cima. Com batidas e estampidos,
construíram uma ponte entre a terra e o barco em apenas dois segundos.
Mas era difícil vê"-la em meio a tanta fuligem e óleo. 

Havia algo muito pesado na ponte, que a fazia ranger; alguns homens
começaram a cantar:

-- Levanta aí! Vai que dá! Segura firme! Vamos lá! Ei,
acerta esse lado! Vamos, vamos!

Então aconteceu uma coisa indescritível. Primeiro, um barulho como o de
sessenta homens rachando lenha; depois, abriu"-se um vão na água que
foi até o fundo do mar, onde uma caixa escura caiu em meio a três
pedras. A caixa cantava, tocava, soava e ressoava, bem perto da enguia
e de seu filhote, que se mandaram para o fundo. 

Uma voz gritou lá de cima:

-- Três braças de água! Não tem jeito! Podem esquecer, não compensa
buscar uma carcaça velha dessas. Mesmo que a vendesse, eu não teria
dinheiro para pagar o conserto!

Era o inspetor de minas, que perdera seu piano no mar.

O silêncio voltou; a baleia vermelha foi embora com seu rabo em forma de
parafuso e fez"-se mais silêncio ainda. Quando o sol se pôs, o vento
começou a soprar; a caixa preta, em meio às algas, oscilava e batia
contra as pedras, soando a cada choque. Todos os peixes das redondezas
vieram ver e escutar. 

Quem chegou primeiro para examinar o piano foi a enguia; e como se viu
refletida naquela caixa, disse:

-- É um armário com espelho!

Uma conclusão bem lógica, e por isso todos repetiram:

-- É um armário com espelho!

Um cadoz veio farejar os candelabros que ainda estavam presos ao piano,
onde restavam uns tocos de vela. 

-- Serve para comer -- disse. -- Tomara que não seja uma isca.

Um grande bacalhau veio e pôs"-se em cima do pedal; o estrondo foi tão
alto que todos os peixes fugiram. 

Nesse dia, deram a exploração por encerrada.

À noite, uma chuva forte golpeou a caixa musical, que ressoou como
marreta de pedreiro, até o sol raiar. Quando a enguia voltou,
acompanhada por todos os outros peixes, aquela caixa se havia
transformado. 

A tampa estava aberta qual boca de tubarão; nunca viram dentes tão
grandes como aqueles; porém, de cada dois, um era preto. Todo o
mecanismo tinha inchado nos lados, feito uma fêmea com ovas; as tábuas
empenaram, o pedal apontava para frente como um pé chutando, os
candelabros cerravam"-se como punhos -- era medonho!

-- Está se abrindo! -- gritou o bacalhau, de barbatana em riste, pronto
para fugir.

-- Está se abrindo! -- gritaram todos. 

As tábuas soltaram"-se; a caixa se abriu, e deu para ver o que havia lá
dentro: um monte de coisa esquisita.

-- É uma armadilha! Não entrem aí! -- disse a enguia.

-- É um tear! -- disse o peixe"-espinho, que tece sua casa e entende de
costura. 

-- É uma grande peneira -- disse a perca, que costumava ficar sob um forno
de cal. 

Sim, era uma peneira! Mas dentro havia tantos alhos e bugalhos que
aquilo já não parecia mais o objeto usado para separar o cascalho. Eram
umas coisinhas parecidas com dedos do pé, vestidos com meia de lã; e
quando se mexiam, punham em movimento um pé com duzentos dedos
esqueléticos, que andava e andava sem nunca sair do lugar.

Era um corpo bem estranho. Por fim, parou de tocar; o esqueleto não
alcançava mais as cordas. Apenas movia"-se na água como se batesse em
uma porta com o nó dos dedos, pedindo entrada.

A música parou. No entanto, veio um cardume de peixes"-espinho que
passou por dentro da caixa, e quando tangeram as cordas com suas
caudas, fizeram"-nas soar outra vez, de um jeito diferente, pois agora
estavam em outra afinação.
\asterisc

Passado algum tempo, numa tarde de verão com céu cor"-de"-rosa, duas
crianças estavam sentadas no ancoradouro: um menino e uma menina. Não
pensavam em nada em particular; talvez em alguma travessura. De
repente, escutaram uma música lenta do fundo do mar e ficaram sérios.

-- Você ouviu?

-- Ouvi. O que é isso? Estão tocando escalas.

-- Não, é a serenata dos mosquitos.

-- Para mim, é a sereia.

-- O professor falou que sereia não existe.

-- O professor não sabe de nada.

-- É, mas escute!

Escutaram por um bom tempo e depois foram embora.

Um casal de banhistas recém"-chegados sentou"-se no ancoradouro; ele
olhou nos olhos dela, que refletiam o pôr"-do"-sol rosado e as praias
verdejantes. Escutaram uma música que lembrava o timbre de uma
harmônica de vidro, em novas tonalidades, que só eles haviam sonhado --
eles, que sonhavam com um mundo novo. Não lhes ocorreu procurar de onde
vinha a música, pois pensavam vir do fundo de suas almas.

Mas veio um casal de velhos banhistas que conhecia o truque; e tiveram o
prazer de dizer em voz alta:

-- É o piano afundado do inspetor de minas!

E era só aparecer novos banhistas que não sabiam do logro para ficarem
imaginando, alegrando"-se com a música desconhecida, até chegarem
banhistas experientes para os esclarecer. Era o fim da alegria.

Assim, a caixa sonora permaneceu ali por todo o verão, e os
peixes"-espinho ensinaram sua arte às percas, que se dedicaram a
aprimorá"-la. O piano tornou"-se um reservatório de percas para os
visitantes: os marinheiros armavam redes a seu redor, e certa vez um
zelador tentou pescar bacalhau ali. Jogou a linha com uma velha
chumbada e estava prestes a puxar, quando escutou um trinado em tom
menor e o anzol enganchou"-se. Puxou a linha com vagar e fisgou cinco
dedos com lã nas pontas; os ossos estalavam como um esqueleto. O
zelador ficou com medo e os atirou na água, embora soubesse o
que fosse aquilo.

A lua vermelha chegou; a água esquentou e todos os peixes nadavam mais
fundo para se refrescar. Com isso a música silenciou de novo. Mas
agosto trouxe o luar; organizaram"-se regatas. Dentro de um barco
branco vinham o inspetor de minas e sua mulher; os dois iam para lá e para cá,
devagar, com os filhos remando. Enquanto deslizavam sobre as  águas
escuras, de um brilho prata e um dourado difuso, escutaram música
embaixo do barco.

-- Ah! -- exclamou o inspetor de minas -- é a carcaça do nosso velho piano!

Mas ele calou"-se ao ver a mulher baixar a cabeça até o peito, como os
pelicanos aparecem nas ilustrações, tentando morder o próprio seio e
ocultar o rosto.

O velho piano e sua longa história reavivaram a distante lembrança da
primeira sala de jantar que o casal havia mobiliado; do primeiro filho
aprendendo a tocar piano e da tristeza nas noites longas, que sucumbia
apenas às notas tempestuosas que punham a casa a vibrar, mandando o
tédio embora e restabelecendo o humor, conferindo um novo brilho até
mesmo aos móveis\ldots{} Mas isso é outra história.
\asterisc

Quando o outono chegou, e com ele a primeira tempestade, vieram também
milhares e milhares de arenques que cruzaram a caixa a nado. Talvez
fosse aquele um concerto de adeus; andorinhas e gaivotas reuniram"-se
para escutar. Nessa noite, a caixa sonora viajou para o alto"-mar, o
que pôs fim a todos os prodígios.

\chapter{O dorminhoco}
\vspace{1em}

\textsc{O maestro Kreuzberg} adorava dormir pelas manhãs, tanto porque trabalhava
na orquestra às noites quanto porque bebia mais de um copo de cerveja
antes de se deitar. Várias vezes pensara em se levantar mais cedo, mas
não via razão para tal. Se procurava um conhecido pela manhã, ele
estava dormindo; se precisava depositar dinheiro no banco, estava
fechado; se ia à loja de música em busca de partituras, ainda não
abrira; e se precisava tomar o bonde, descobria que ainda não estava em
circulação. De manhã cedo ele não conseguia nem um coche, nem sua dose
de rapé e nem mesmo resolver suas pendências. Por isso continuou dormindo
pelas manhãs, como lhe aprazia.

O maestro também gostava de sol, de flores e de crianças; no entanto,
por causa dos seus delicados instrumentos musicais, as janelas de sua
casa não davam para o lado onde bate o sol, já que num quarto
ensolarado os instrumentos não mantêm a afinação. Assim, no começo de
abril, o maestro alugou um apartamento que dava para o lado norte.
Observou isso com atenção, porque levava uma bússola na correia do
relógio e de noite sabia localizar as sete estrelas da Ursa Maior.

E eis que chega a primavera. Fazia tanto calor que morar voltado para o
norte era uma bênção. O quarto ficava junto à sala de estar, onde o
maestro dormia num escuro de breu, graças às persianas; na sala não as
havia, porque não precisava.

O verão chegou e tudo ficou verde. O maestro bebera e comera no
Hasselbacken, e por isso dormiu muito e bem -- ainda mais que nesse dia
o teatro estava fechado.

O maestro Kreuzberg dormia como uma pedra, mas o quarto esquentou de tal
modo que algumas vezes ele acordou, ou pensou ter acordado. Houve uma
hora em que achou que o papel de parede estava em chamas, o que podia
ser efeito do borgonha que degustara; em seguida, sentiu um calor no
rosto, mas isso era com certeza o borgonha, e assim ele virou"-se e
voltou a dormir.

Levantou"-se às nove e meia, vestiu"-se e foi para a sala
refrescar"-se com o copo de leite frio que lhe era servido toda manhã.


Mas hoje a sala não estava nada fresca: estava quase quente, quente
demais. E o leite frio não estava frio, mas morno; desagradavelmente
morno.

O maestro não era rabugento, mas gostava de ter tudo em ordem. Assim,
tocou a campainha para chamar a velha Lovisa, e como não se irritava
antes de repetir a mesma coisa no mínimo cinqüenta vezes, dirigiu"-se
a ela num tom amigável, porém um tanto decidido quando a cabeça dela
apareceu na porta.

-- Lovisa -- disse ele --, a senhora me serviu leite morno.

-- Não, patrão! -- respondeu Lovisa. -- O leite estava frio, ficou aí e
esquentou.

-- Então a senhora acendeu o fogo, porque a sala está quente.

Mas não, Lovisa não acendera o fogo, e assim arrastou"-se de volta para
o seu canto, emburrada.

Ora, aquilo era apenas um copo de leite; mas quando o maestro passou os
olhos pela sala ele se entristeceu. Em um canto perto do piano ele
havia arrumado o altar da casa, formado por uma mesinha com dois
candelabros de prata, um grande retrato de uma mulher jovem e, em
frente ao retrato, uma taça de champanhe com borda dourada. 

Nessa taça, a taça do seu casamento -- o maestro agora era viúvo --, todo
dia havia uma rosa vermelha, uma lembrança e um tributo àquela que no
passado fora o sol de sua vida. Fosse inverno ou verão, sempre havia
uma rosa na taça; no inverno ela durava oito dias, se ele cortasse a
ponta do caule e pusesse um pouco de sal na água. E no dia anterior ele
pusera uma rosa recém"-colhida e hoje ela já estava murcha, seca,
morta, cabisbaixa. Era um mau sinal. Ele bem sabia o quão delicadas
eram essas flores, e sempre reparava em quais casas elas vicejavam e em
quais não. De vez em quando lembrava"-se da época em que a esposa era
viva, da rosa que ela sempre tinha sobre a mesa de costura, que às
vezes perdia o viço e murchava de repente. Observara que era justamente
nessa hora que o sol resolvia esconder"-se por trás das nuvens, que
se desfaziam em gotas de chuva ao primeiro sinal do trovão. Rosas gostam de
paz e palavras amorosas, e não toleram palavras duras. Também adoram
música, e de vez em quando o maestro Kreuzberg tocava para vê"-las 
se abrirem num sorriso.

Lovisa tinha uma disposição indócil e costumava resmungar enquanto
limpava a casa. Na cozinha, tinha dias de um mau humor terrível em que
os molhos queimavam e a comida toda ficava com um travo de desprazer
que o maestro logo notava; pois ele também era um instrumento sensível,
que percebia na alma o que outros não percebem.

Logo supôs que Lovisa matara a rosa, talvez resmungando perto da
pobrezinha ou esbarrando na taça, ou ainda bafejando maldade sobre a
flor, que não resistiria a uma coisa dessas. Tocou a campainha mais uma
vez e, quando Lovisa apareceu com a cabeça na porta, falou em um tom
que não chegava a ser agressivo, porém um pouco mais firme que da outra
vez:

-- O que a senhora fez com minha rosa, Lovisa?

-- Nada, meu bom patrão!

-- Nada? A senhora pensa que uma flor morre assim sem mais nem menos?
Veja como a taça está seca! A senhora jogou a água fora!

Como Lovisa era inocente, ela foi para a cozinha e chorou, porque ser
acusado injustamente é muito ruim.

O maestro Kreuzberg, que não agüentava ver ninguém chorando, esqueceu o
assunto e comprou outra rosa à tarde -- uma bem fresca, sem arame, pois
isso a esposa não tolerava. 

Depois deitou"-se de lado e tirou uma soneca; achou que o papel de
parede estava pegando fogo e que o travesseiro estava quente, mas
voltou a dormir. 

Quando na manhã seguinte saiu do quarto para a sala, a fim de fazer sua
oração no altar da casa -- ah, que dor! A rosa jazia desfolhada.

Teve o ímpeto de tocar a campainha, mas conteve"-se ao ver, meio
enrolada, a fotografia, a fotografia daquela que sua alma adorava,
caída junto à taça. 

Lovisa não faria isso! -- Na sua cabeça infantil, pensou: ``Ela que era
tudo para mim, minha consciência e minha musa\ldots{} será que já não gosta
mais de mim, está com raiva de mim? O que será que fiz?''

É claro que, examinando a consciência, o maestro lembrou"-se de
pequenos defeitos, que decidiu eliminar -- aos poucos, evidentemente.

Então pôs o retrato de volta na moldura e a rosa debaixo de uma redoma
de vidro; achava que isso ajudaria, mas era um tanto improvável. 

Partiu em seguida para uma viagem de oito dias; quando voltou era noite,
e foi direto para a cama. Como sempre, acordou, abriu um só olho e
imaginou que o lustre estivesse pegando fogo.

Mais tarde voltou a entrar na sala, que estava muito quente e parecia
abandonada. As cortinas haviam desbotado; a capa do piano também
perdera a cor; as partituras estavam em desalinho, e o
querosene do lustre tinha evaporado e pendia em gotas ameaçadoras sobre
um ornamento, onde as moscas dançavam. A água na jarra estava quente.

O mais triste de tudo, no entanto, era que a o retrato da esposa também
desbotara; estava amarelado como as folhas do outono. O maestro ficou
consternado. E quando ele ficava triste de verdade, sentava"-se ao
piano ou empunhava o violino enquanto\ldots{} 

E dessa vez sentou"-se meio indeciso ao piano, com o intuito de tocar a
sonata em mi menor -- a de Grieg, é claro. Era a sonata preferida da
esposa, a que ele julgava a melhor do mundo depois da sonata em ré
menor de Beethoven; não porque ré vem antes de mi, mas porque é assim e
pronto.

Hoje, porém, o piano não queria obedecer. Estava dissonante e resistia
ao maestro, tanto que ele achou que havia algum problema com seus dedos
e ouvidos. Mas não era nada disso. O piano estava simplesmente
desafinado, desafinadíssimo, embora houvesse passado há pouco tempo
pelas mãos hábeis do afinador. Parecia uma coisa sobrenatural, feitiço!

Pegou então o violino, que pode ser afinado. Mas quando precisou subir a
quinta, a cravelha não se mexeu; estava seca e emperrada. Quando o
maestro apertou com mais força, a corda arrebentou num estrépito e se
enrolou como pele de enguia seca.

Era mesmo um feitiço!

Quanto à fotografia ter desbotado, era o mais triste de tudo, e por isso
o altar foi coberto com um véu. 

E assim um véu toldou a beleza na vida do maestro; ele perdeu a
inspiração, ficou cismado e parou de sair à noite.

E foi se aproximando o auge do verão. A noite chegava cada vez mais
tarde; mas como as persianas estavam sempre fechadas, o maestro não
via diferença alguma. 

Por fim, certa noite -- na própria noite de São João -- o maestro acordou
com o relógio da sala batendo treze horas. Era apavorante, tanto por
este ser um número agourento como porque um relógio regrado não bate
treze horas. Kreuzberg não conseguiu pegar no sono outra vez e ficou
acordado, escutando. Um ruído seco veio da sala, como o estalar de um
móvel. Depois ele escutou passos furtivos e o relógio começou a bater;
e batia, batia, cinqüenta, cem vezes. Apavorante!

Um feixe de luz invadiu o quarto e projetou uma figura na parede; era a
estranha figura de uma cruz gamada, que vinha da porta da sala.
Portanto, lá havia uma luz acesa. Mas quem a acendera? Ouvia"-se 
barulho de copos, como se houvesse convidados, taças de champanhe, mas
nada de conversa. Mais um barulho, como o de alguém recolhendo as velas
de um barco, passando roupas, ou algo semelhante.

O maestro precisava sair para ver o que era; assim, entregando sua alma
na mão do todo"-poderoso, adentrou a sala.

Primeiro viu a touca de Lovisa desaparecer pela porta da cozinha, as
cortinas abertas e a mesa de jantar cheia de flores nas taças. Ah! Tão
cheia de flores quanto aquela vez, na noite de núpcias; a noite em que
trouxe a esposa para casa.

Foi quando o sol brilhou no seu rosto, em montes azuis e florestas
longínquas; fora o sol que havia iluminado a sala, pregando ao maestro
todas essas pequenas peças. Era o dia de seu aniversário e ele bendisse
o sol, que se levantara tão cedo para acordar o dorminhoco com suas
brincadeiras. E bendisse a memória daquela que fora o sol de sua
vida. Não era uma designação muito original, mas o maestro não
conseguia inventar nada melhor e gostava desse nome.

A rosa, no altar da casa, estava cheia de vida; tão cheia de vida como a
esposa do maestro Kreuzberg antes de se cansar da lida. Cansada, sim;
ela era frágil, e a vida lhe foi muito brutal com seus golpes e
choques. O maestro ainda se lembrava da voz lamentosa quando ela,
depois de passar a roupa e arrumar a casa, caía no sofá e exclamava:

-- Estou tão cansada!

A pobrezinha não pertencia a este mundo; fez apenas um papel temporário
e depois partiu.

O médico dissera uma vez que ela precisava tomar sol, mas naquela época
eles não tinham meios de providenciar isso, porque os apartamentos
ensolarados são mais caros.
\pagebreak
E agora ele tinha sol e nem se dava conta: estava banhado de sol, mas
era tarde demais. O verão estava quase acabando, e o sol mais uma vez  \EP[]
faria seu caminho; afastar"-se"-ia por um ano para depois voltar.
Como o mundo é estranho!

\chapter{As agruras do timoneiro}


\textsc{O timoneiro} conduzia um cúter que bordejava em frente ao último farol; o
sol de inverno tinha se posto há muito tempo e o cúter singrava em
alto"-mar, em meio a ondas bravias. Então o marinheiro da proa
assinalou: 

-- Veleiro a barlavento!

Um brigue orçava à frente, com a bandeira pedindo piloto hasteada; o navio
deveria ser conduzido ao porto.

-- Moço! -- gritou do leme o capitão. -- Será difícil alcançá"-los num mar
desses. Viktor, quando o ladearmos pela ré, a sotavento, você
poderá agarrar"-se à enxárcia sem grandes dificuldades\ldots{} Vamos lá! A
postos!

O cúter logo virou e se aproximou do brigue, que batia com o casco em
alguma coisa no fundo do mar.

-- Que estranho, eles não vão bracear! Vocês estão vendo alguma luz a bordo? 

-- Não! 

-- Nenhuma luz no mastro da proa.

-- A todo pano! Moço! Viktor!

O cúter veio com tudo. Viktor estava na balaustrada a barlavento e,
quando um vagalhão levantou o barco uma vez mais, agarrou"-se ao ovém
do brigue. Enquanto isso o cúter seguiu adiante, virou e rumou em
direção ao farol e à barra.

Viktor estava a meio caminho do poleame e desceu ofegante para o convés.
Quando chegou, foi logo para o leme, onde era o seu lugar. Pode"-se
imaginar seu espanto quando não achou ninguém ao timão! Chamou em volta
e não teve resposta.

-- Devem estar enchendo a cara lá dentro -- pensou, e foi até a porta da
cabine. Não havia ninguém. Foi adiante até a cozinha, ao castelo da
proa, e não encontrou vivalma. Compreendeu que a embarcação fora
abandonada, e supôs que estivesse fazendo água e prestes a naufragar.

Olhou à procura do cúter, que já desaparecera na escuridão. 

Não havia como conduzir o brigue até a margem; era inconcebível manejar
escotas e adriças junto com todo o massame e ao mesmo tempo cuidar do
leme. 

Não havia o que fazer senão ficar à deriva, embora o barco avançasse mar
adentro.

Claro que Viktor não ficou contente, mas um marinheiro deve estar
preparado para qualquer coisa; em breve outra embarcação passaria por
perto, e então bastaria fazer"-lhe um sinal de luz. Foi para a cozinha
à procura de fósforos e de um lampião. Embora o mar estivesse
encapelado, o barco não se mexia, o que deixou Viktor espantado. E o
que o espantou ainda mais, quando chegou ao mastro grande, foi perceber
que andava num piso de parquê, coberto por um longo tapete xadrez azul
e branco. Andou e andou, mas o tapete não acabava nunca, e logo perdeu
a cozinha de vista. Era assustador, e ao mesmo tempo divertido, por ser
uma novidade.

O tapete ainda não tinha acabado quando chegou à entrada de um passeio
repleto de lojas iluminadas. À sua esquerda havia uma balança e uma
máquina automática. Sem pensar, subiu na balança e pôs uma moeda. Como
sabia pesar oitenta quilos, sorriu quando o mostrador registrou apenas
oito. E pensou: ``Ou essa balança é maluca, ou cheguei a outro planeta
dez vezes maior ou menor que a Terra'', pois freqüentara a escola de
navegação e tivera aulas de astronomia.

Viktor logo veria que tipo de máquina era aquela! 

Quando a moeda caiu, abriu"-se uma portinhola, de onde uma carta caiu
na sua mão. Era um envelope branco, com um grande selo vermelho, que
Viktor não conseguiu ler; o que não fazia muita diferença, já que não
sabia quem era o remetente. Quebrou o selo e leu a assinatura, como se
costuma fazer. A carta dizia\ldots{} Bem, vamos saber disso mais tarde. Leu
a carta três vezes, com atenção, e guardou"-a no bolso da camisa, com
ar muito pensativo; muito pensativo \textit{mesmo!}

Seguiu pelo passeio, tomando cuidado para andar sempre pelo meio. Havia
lojas de todos os tipos imagináveis, mas não se via uma pessoa que
fosse -- nem fregueses, nem atendentes. Depois de andar por um tempo,
parou em frente a uma grande vitrine, onde várias conchas estavam
dispostas. Como a porta estava aberta, Viktor entrou. Prateleiras
repletas de conchas oriundas dos sete mares cobriam as paredes do chão
ao teto. Lá também não havia ninguém, mas Viktor enxergou anéis de
fumaça, que pareciam recém"-soprados por algum fumante bem"-humorado.
Viktor, que era um rapaz alegre, enfiou o dedo no anel e disse:

-- Rá! Agora sou noivo da senhorita Tabaco! 

Escutou um ruído estranho, como o de um relógio, mas não havia relógio
algum por perto.  Percebeu que o ruído vinha de um molho de chaves. Uma
delas parecia recém"-saída do caixa; as outras balançavam para frente
e para trás com os movimentos regulares de um pêndulo, e assim
permaneceram por alguns instantes. Então se fez silêncio, e quando tudo
ficou quieto ouviu"-se um sussurro, como o vento batendo nas roldanas
do navio ou o vapor passando por um cano estreito. Eram as conchas que
murmuravam, e como cada uma era de um tamanho, as notas de seus
murmúrios tinham alturas diferentes. Soava como uma orquestra de
murmúrios. Viktor, que nascera numa quinta"-feira e por isso entendia
o canto dos pássaros, ajustou sua concha da orelha para entender os
murmúrios. Logo ouviu as conchas dizerem:

-- Eu tenho o nome mais bonito -- disse uma --, porque me chamo
\textit{Strombus pespelicanus}.

-- Eu sou a mais bela! -- disse a concha púrpura, que se chama
\textit{Murex} e tem um sobrenome esquisito.

-- Eu canto melhor! -- disse a concha"-tigre, que se chama assim porque
parece uma pantera.

-- Silêncio, silêncio, silêncio! -- disse o caracol. -- Eu sou o que vende
mais, porque meu preço tem desconto nas férias de verão. As pessoas me
acham chato, mas assim mesmo precisam de mim. No inverno, porém, fico
na despensa dentro de um barril de couve.

-- Que turma horrível, que não pára de se gabar! -- disse Viktor; e, para
se distrair, pegou um livro que estava aberto sobre o balcão. Como
tinha olhos, viu que o livro estava aberto na página 240, e que o
capítulo 51 começava na página da esquerda. O capítulo tinha um verso
de Coleridge como epígrafe; seu tema atingiu Viktor como um raio. Com
as faces coradas e a respiração suspensa, pôs"-se a ler\ldots{} Ah, vamos
falar disso mais tarde; mas fiquem sabendo que não era nada sobre
conchas.

Todavia o lugar lhe agradou e Viktor sentou"-se -- porém, longe do
caixa, que era uma companhia perigosa. Começou a pensar naquelas
criaturas fantásticas que, como ele, viviam no mar; lá no fundo não faz
calor, mas assim mesmo as conchas suam e, ao suar cálcio, ganham uma
nova blusa. As conchas movimentavam"-se como minhocas: umas para a
direita e outras para a esquerda; mas isso era de se esperar, afinal
elas precisam se mexer e evidentemente não podem se mexer todas para o
mesmo lado.

Nesse momento uma voz vinda do interior da loja atravessou a cortina
verde:

-- Sim, que elas se mexem assim nós sabemos, mas o que não sabemos é que
o caracol do ouvido é um \textit{Helix}, e que os ossinhos do tímpano
parecem com o bicho que habita o \textit{Limnaeus stagnalis}, como
consta no livro.

Viktor logo percebeu que estava lidando com alguém que lia o pensamento
e respondeu em tom amigável, mas enérgico, sem demonstrar nenhum
espanto; ele também falou através da cortina:

-- Sei; mas \textit{por que }temos um \textit{Helix} dentro do ouvido,
isto o livro sabe melhor que o vendedor de conchas\ldots{}

-- Não sou um vendedor de conchas -- berrou a voz lá de dentro.

-- O que é então? -- berrou Viktor de volta.

-- Eu sou\ldots{} um \textit{troll}!

E ao mesmo tempo a cortina entreabriu"-se e apareceu uma cabeça tão
horrível que qualquer outro teria dado no pé. Mas Viktor sabia como
lidar com um \textit{troll}. Primeiro, olhou para o fornilho
incandescente do cachimbo, pois o \textit{troll} estava fumando;
fumando e soprando anéis de fumaça pela fresta da cortina. Quando um
anel aproximou"-se, Viktor pegou"-o com o dedo e atirou"-o de volta.

-- Sabe atirar anéis de fumaça! -- disse o \textit{troll}, com escárnio.

-- Mais ou menos -- respondeu Viktor.

-- E nem está com medo!

-- Um marinheiro não pode ter medo, senão nenhuma moça gosta dele.

-- O senhor que não tem medo, escute: vá um pouco mais adiante pelo
passeio e vamos ver se continua tão destemido.

Viktor já vira conchas o suficiente e aproveitou o ensejo para ir embora
sem parecer que estava fugindo; saiu da loja andando para trás, pois
sabia que jamais se vai embora dando as costas, sempre muito mais
vulneráveis que o peito. 

Continuou andando, seguindo o tapete azul e branco. O passeio não era
reto; fazia voltas e mais voltas, e nunca se via o final. O tempo todo
surgiam lojas novas, mas nada de gente. Não se viam sequer os donos das
lojas. Mas Viktor aprendera por experiência própria que eles não
ficavam na rua. 

Quando passou por uma loja de perfumes, que exalava o aroma de todas as
flores do campo e dos bosques, pensou: ``Comprarei um frasco de água
de colônia para minha noiva''. Dito e feito! A loja era bem parecida com
a loja de conchas, mas o cheiro era tão forte que Viktor sentiu dor de
cabeça e teve de sentar"-se. 

Havia um cheiro particularmente forte de amêndoa amarga que até dava 
um zumbido no ouvido, e também deixava um gosto bom de vinho de cereja na boca. 
Viktor, que não era de ficar indeciso, abriu sua caixinha de latão e 
serviu"-se de uma pitada de tabaco grosso para esclarecer as idéias e 
acabar com a dor de cabeça. Então bateu no balcão e chamou:

-- Olá! Alguém aí?

Não houve resposta. ``Posso muito bem cuidar disso eu mesmo'', pensou. Pôs
a mão direita no balcão, tomou impulso e saltou para o outro lado. Em
seguida abriu a cortina e olhou o interior da loja, onde viu uma imagem
que o deslumbrou por completo. Sobre uma longa mesa coberta por uma
toalha persa havia uma laranjeira com frutas e flores, de folhas tão
lisas que lembravam camélias. Copos de cristal lavrado, decorados com
flores perfumadas de todos os cantos do mundo, de jasmins a angélicas,
passando por violetas, lírios"-do"-vale, rosas e até mesmo alfazemas
estavam dispostos em fila ao longo da mesa. 

De um lado, meio encobertas
pela laranjeira, Viktor divisou pequenas mãos brancas saindo de um
vestido com mangas arregaçadas que seguravam um destilador de prata;
mas não conseguia ver o rosto da dama, que tampouco o via. Quando
percebeu que o vestido era amarelo e verde, soube tratar"-se de uma
feiticeira, pois amarela e verde é a larva da mariposa"-esfinge, que
consegue mudar de aparência. A parte traseira do corpo dessa mariposa
parece a parte dianteira, com um chifre como o do unicórnio, que ela
usa para assustar seus inimigos com o rosto falso enquanto come com o
que parece ser a parte traseira.

Viktor pensou: ``Lá vem confusão, mas ela que comece''. E fez muito bem,
se queremos que alguém fale, basta ficarmos calados.

-- É o senhor que está procurando trabalho para o verão? -- perguntou a
dama, aproximando"-se.

 -- Sim, sou eu -- disse Viktor; mas foi só para ter o que responder, pois
ele nunca pensava durante o inverno no que faria nas férias de verão. 

 A dama pareceu constrangida, mas era bela como o pecado e lançava um
olhar encantador para o timoneiro.

-- Não adianta tentar me enfeitiçar, porque sou noivo de uma boa moça! --
disse o timoneiro, olhando para ela entre o dedo médio e o anular, como
fazem as bruxas para se proteger dos juízes.

A dama era jovem e bela da cintura para cima, mas da cintura para baixo
era muito velha, como se feita de duas partes diferentes.

-- Deixe"-me dar uma olhada por aqui -- disse o timoneiro.

-- Pois não! -- respondeu a dama, abrindo uma porta ao fundo.

Os dois saíram pela passagem e foram parar num bosque de carvalhos.

-- É só atravessar a floresta e pronto! -- disse a dama. Pediu ao
timoneiro que fosse na frente, pois naturalmente não queria dar"-lhe
as costas.  

-- Dá para ver que o touro passa por aqui -- disse o timoneiro, que estava
atento.

-- Você tem medo do touro? -- perguntou a dama.

-- É o que veremos! -- respondeu Viktor.  

Passaram por outeiros e raízes de árvores, pântanos e terrenos
queimados, áreas desmatadas e fornos de carvão abandonados; de vez em
quando Viktor precisava olhar para trás e ver se a feiticeira ainda o
seguia, porque não conseguia escutar"-lhe os passos; e mesmo quando se
virava e a tinha diante de si, seus olhos continuavam a procurá"-la,
porque o vestido amarelo e verde a deixava quase invisível.

Chegaram enfim a uma clareira ou área desmatada no meio da floresta;
Viktor estava no meio do campo verde, e assim o touro veio em sua
direção como se o estivesse esperando. O touro era preto, tinha uma
estrela branca na testa e olhos injetados de sangue. 

Como não dava para fugir, era atacar ou defender"-se. Viktor olhou ao
redor e viu uma ripa de madeira recém"-cortada em forma de porrete.
Pegou"-a do chão e se posicionou.

-- É eu ou você! -- gritou ele. -- Um, dois, três!

Então a dança começou. O touro recuou como um navio, soltando vapor
pelas ventas e mexendo o rabo como uma hélice; e investiu com tudo. 

O chão tremia; e quando o porrete acertou o touro bem entre os olhos,
fez"-se um estampido como o de um tiro. Viktor pulou para o lado e o
touro seguiu correndo em frente. O cenário mudou: para o terror de
Viktor, o monstro correu para os limites da floresta, onde, com um
vestido claro, sua noiva apressava"-se de encontro ao noivo. 

Viktor gritou do fundo da alma:

-- Anna, suba em uma árvore! Lá vem o touro!

Correu atrás do monstro e golpeou"-lhe a pata traseira, \mbox{no lugar mais
frágil, para quebrar"-lhe um osso, se possível.} E com um esforço
sobre"-humano, fez o colosso desabar. Anna foi salva e o timoneiro a
tomou em seus braços.

-- Para onde vamos agora? -- perguntou ele. -- Para casa?

Quanto a perguntar de onde ela veio, isso nem lhe ocorreu, por motivos
que vamos conhecer mais tarde. 

Os noivos iam de mãos dadas pelo caminho, felizes com o inesperado
reencontro. Mas logo em seguida, Viktor parou de repente e disse:

-- Espere, preciso voltar e ver como está o pobre touro.

O rosto de Anna se transformou; os olhos injetaram"-se. Com uma
expressão feroz e maldosa ela disse apenas:

-- Vá! Eu espero.

O timoneiro fitou"-a com olhar triste ao perceber que a noiva faltara
com a verdade. Contudo, seguiu"-a. O andar dela estava diferente, e
Viktor começou a sentir frio em todo lado esquerdo do corpo. 

Após terem andado por um tempo, Viktor parou de novo.

-- Me dê sua mão -- disse ele. -- Não esta; a mão esquerda.

Viu que ela não usava o anel de noivado.

-- Onde está o anel? -- perguntou ele.

-- Perdi -- respondeu ela.

-- Você é e ao mesmo tempo não é minha Anna. Um espírito a possuiu.

Nesse momento a noiva olhou"-o de soslaio e Viktor percebeu que aquele
olhar não era humano, mas o olhar injetado do touro; e compreendeu
tudo.

-- Chega de blasfêmias, bruxa! -- disse; e cuspiu no rosto dela.

Então vocês tinham que ver! A falsa Anna mudou de forma e, fervendo de
ódio, adquiriu a cor amarelo"-esverdeada da bile; no instante
seguinte, um coelho preto pulou por cima de um pé de mirtilo e
desapareceu. 

Agora Viktor estava numa floresta erma, mas nem assim perdeu a coragem.
``Eu continuo sozinho'', pensou, ``mas caso
nos encontremos, sigo adiante rezando meu pai"-nosso, que me acompanha
por um longo caminho.'' 

Andou até avistar uma casa. Bateu na porta e foi recebido por uma
velhinha; perguntou se ela poderia hospedá"-lo. A velhinha respondeu
que sim, mas que não havia muito a oferecer; apenas um lugar mais ou
menos no sótão.

-- Pode ser de qualquer jeito. Preciso dormir!

Como os dois se entenderam, ele acompanhou a velhinha até o quarto no
sótão. Acima da cama havia um grande vespeiro, e a velhinha pediu
desculpas pelas intrusas.

-- Não tem problema; as vespas são como as pessoas: boas até que alguém
as irrite. Aqui também tem cobra?

-- Claro, tem algumas por aí.

-- Não tem problema; elas gostam do calor da cama e vamos nos dar bem.
São víboras ou cobras"-d’água? Não sou muito exigente com a companhia,
mas prefiro as cobras"-d’água!

A velhinha ficou muda quando viu o timoneiro preparando a cama, decidido
a dormir no quarto. 

Nesse momento escutou"-se um zumbido inquieto do lado de fora da
janela; era uma grande vespa que tentava entrar.

-- Deixe a coitada entrar! -- disse o timoneiro, abrindo a janela.

-- Não! Mate"-a! -- gritou a velhinha.

-- Por quê? Talvez ela tenha filhotes no vespeiro; se eu a matar, eles
vão ficar com fome, e eu terei de agüentar o choro! Eu não, muito
obrigado\ldots{} Entre, vespinha!

-- Ela pica! -- continuou a velha.

-- Não pica qualquer um; só as pessoas ruins\ldots{}

Ele abriu a janela. Entrou uma vespa do tamanho de um ovo de pomba;
zumbindo como a corda de um contrabaixo, e desapareceu vespeiro
adentro. Tudo ficou em silêncio. 

A velha saiu e o timoneiro caiu na cama. 

Quando ele desceu na manhã seguinte, não encontrou a velha; mas no único
banco da casa havia um gato preto, que ronronava. Como todos sabem, os
gatos estão condenados a ronronar porque são preguiçosos demais, e
alguma coisa eles precisam fazer.

-- Saia daí, gato! -- disse o timoneiro. -- Eu quero sentar.

Pegou o gato e o pôs dentro do forno a lenha. Mas esse não era um gato
qualquer: os pêlos das costas começaram"-lhe a faiscar e logo a lenha
pegou fogo.

-- Se sabe fazer fogo, também sabe fazer café -- disse o timoneiro.

Mas os gatos adoram contrariar os outros, e assim ele começou a espumar
de raiva e a cuspir para apagar o fogo.

Nesse mesmo instante o timoneiro escutou o baque de uma pá sendo posta
contra a parede da cabana. Quando olhou para fora, viu a velha ao lado
de uma cova que acabara de abrir no quintal.

-- Ah, é assim, velha?! Já cavou minha sepultura? -- perguntou ele.

Nesse instante a velha entrou na cabana. Quando percebeu que Viktor
estava são e salvo, ficou muito surpresa e disse"-lhe que ninguém
jamais saíra vivo daquele sótão e, por isso, de antemão já lhe havia
cavado uma sepultura. Como a velha enxergava mal, achou que o timoneiro
estava usando um cachecol estranho.

-- Já viu um cachecol como este? -- perguntou Viktor, passando a mão por
debaixo do queixo.

Uma cobra estava enrolada em seu pescoço, fazendo um belo nó com duas
manchas amarelas; eram os olhos, brilhando como pedras preciosas.

-- Mostre suas presas para a titia! -- disse o timoneiro.

E quando ele coçou a cabeça da cobra deu para ver as duas presas na
bocarra escancarada. 

A velha caiu no vestíbulo e exclamou:

-- Vejo que você recebeu e compreendeu minha carta. Você é um bom rapaz.

-- Sim, a sua carta automática -- disse o timoneiro, tirando a carta do
bolso. -- Esta eu vou emoldurar quando voltar para casa!

Sabem o que dizia na carta? Apenas isto: \textit{Mann muss sich nie
verbluffen lassen}, o que pode ser traduzido como: ``A sorte ajuda os
audazes''.
\asterisc

Anne"-Marie, que escutou sua mãe terminar a história assim, perguntou:

 -- Mas como foi que o timoneiro conseguiu passar do barco para o
passeio? E depois ele voltou, ou foi tudo apenas um sonho?

-- Isso eu conto outra hora, menina curiosa! -- disse a mãe.

-- Tinha também uns versos no livro\ldots{}

-- Que versos? Ah, sim, aqueles na loja de conchas\ldots{} Tinha até esquecido
deles! -- disse a mãe. -- Mas não se deve perguntar sobre essas coisas,
querida; é apenas uma fábula!

\chapter{Fotografia e filosofia}


\textsc{Era uma vez} um fotógrafo. Trabalhava muito: fazia perfis e retratos,
pose sentada ou corpo inteiro; sabia revelar e fixar, tingir, dourar e
copiar. Era um verdadeiro craque. Mas o fotógrafo nunca estava
satisfeito, pois era filósofo também; um grande filósofo e um inventor.
Tinha até uma teoria de que o mundo está às avessas. Isso é o que se vê
numa chapa fotográfica durante o processo da revelação: o lado direito
da pessoa se torna o esquerdo; o escuro transforma"-se em luz, a
sombra em claridade; o azul torna"-se preto e botões prateados ganham
a cor do ferro -- tudo às avessas. 

Tinha um sócio, um sujeito comum e cheio de peculiaridades: fumava o dia
inteiro, jamais se lembrava de fechar as portas; ao comer, levava à
boca a faca, e não o garfo; entrava em casa de chapéu; limpava as unhas
no meio do ateliê, e à noite tomava três copos de cerveja. Era um homem
cheio de defeitos!

O filósofo, ao contrário, não tinha defeito algum, e demonstrava certa
má vontade para com seu irmão imperfeito; queria se afastar dele e não
podia, os negócios os prendiam um ao outro. Por isso a má vontade do
filósofo começou a se transformar num ódio renitente; era horrível.

Quando a primavera chegou, era a hora de alugar uma casa para as férias
de verão. O sócio foi encarregado de arranjar tal lugar, e arranjou. Os
dois partiram num \mbox{sábado} à noite, no barco a vapor. O filósofo passou a
viagem toda sentado no deque, bebendo ponche. Era um tanto gorducho, e
várias moléstias o afligiam; sofria do fígado e, além disso, tinha um
problema nos pés -- talvez um reumatismo, ou algo do gênero. Ao fim do
trajeto, os dois desembarcaram no píer. 

-- É aqui? -- perguntou o filósofo.

-- Fica logo adiante -- respondeu o sócio.

Foram por um caminho coberto de raízes de árvore, interrompido na metade
por uma cerca, que era preciso pular. Então vinha um caminho de pedras.
O filósofo se queixava dos pés, mas esqueceu da dor quando precisaram
pular outra cerca. Daí em diante a trilha desaparecia; era preciso
passar por outeiros e abrir caminho por entre arbustos e galhos de
mirtilo.

Diante da terceira cerca havia um touro que avistou o filósofo e
perseguiu"-o até a quarta; isso rendeu"-lhe um banho de suor e
abriu"-lhe os poros. Depois da sexta cerca chegaram à casa. O filósofo
abriu a porta e em seguida saiu para a varanda.

-- Para que tantas árvores? -- perguntou. -- Elas encobrem a vista.

-- Protegem do vento do mar -- respondeu o sócio.

-- Isto aqui parece uma fazenda mesmo; estamos no meio do mato.

-- É saudável -- respondeu o sócio.

Os dois foram nadar. Mas, na acepção filosófica do termo, não havia
praia alguma; era apenas um fundo de lama e pedras, nada mais. Depois
do banho, o filósofo foi até a fonte beber um copo d’água. A água tinha
cor de ferrugem e gosto amargo. Estava insatisfeito, muito insatisfeito
com tudo. Não havia onde comprar carne; só se achava peixe.

O filósofo ficou mal"-humorado e sentou"-se debaixo de uma mamoneira
para reclamar. Mas não podia ir embora; o sócio voltara à cidade para
tomar conta dos negócios durante a ausência do colega.

Passadas seis semanas, o sócio reencontrou o filósofo.

No píer se encontrava um jovem atlético, com as bochechas rosadas e o pescoço
queimado. Era o filósofo, rejuvenescido e cheio de vida.

Saltou por cima de todas as cercas e afugentou o touro sozinho. Quando
chegaram à varanda, o sócio disse:

-- O senhor está com um aspecto ótimo. Como tem passado?

-- Bem -- disse o filósofo. -- Ótimo! As cercas acabaram com minha gordura,
as pedras massagearam meus pés, a lama deu"-me banhos medicinais
contra o reumatismo, a alimentação leve curou meu fígado, a floresta
curou meus pulmões e, imagine só, a água escura da fonte é rica em
ferro, justamente o que eu precisava.

-- Sim, seu filósofo -- disse o sócio. -- De uma chapa negativa faz"-se
uma positiva, em que a sombra se converte em luz. Se o senhor pegasse
uma chapa positiva de mim e se ocupasse dos defeitos que não tenho, o
senhor não me odiaria. Pense no seguinte: não me embriago, e por isso
posso tomar conta do ateliê; não roubo e nunca falo mal do senhor;
nunca reclamo nem crio caso; nunca sou deseducado com os fregueses;
levanto"-me bem cedo; limpo as unhas para manter o revelador limpo;
uso chapéu para que não caia cabelo nas chapas e fumo tabaco para
limpar o ar de vapores nocivos; deixo as portas entreabertas para o
barulho sair do ateliê; bebo cerveja à noite para não acabar no uísque;
e, para não me espetar com o garfo, prefiro comer com a faca.

-- O senhor é realmente um grande filósofo -- disse o fotógrafo. -- Sejamos
amigos, e assim havemos de \mbox{prosperar.}

\chapter{Meia folha de papel}


\textsc{A última leva} da mudança partira, e o inquilino, um jovem com véu de
luto no chapéu, vagava pelo apartamento mais uma vez, para
certificar"-se de que não esquecera nada. Não, não esquecera nada,
absolutamente nada; e saiu para o vestíbulo, decidido a não mais pensar
na vida que levou nesse apartamento. Mas lá, junto ao telefone, a
metade de uma folha de papel estava pregada com tachas. O papel estava
coberto por várias caligrafias; algumas eram muito elegantes, a tinta;
outras foram rabiscadas a lápis ou caneta vermelha. Aquela meia folha
de papel contava a bela história que se desenrolou no curto período de
dois anos; ali estava tudo que ele tentava esquecer: um pedaço de vida
humana, ocupando a metade de uma folha de papel.

Desprendeu a folha; era um desses papéis de rascunho amarelo"-claros,
que brilham. Colocou"-a em cima da cornija da lareira e, apoiado, leu
a folha. Primeiro estava escrito o nome dela: Alice. Para ele, o nome
mais belo do mundo, porque era o nome de sua noiva. E o número -- 15.~11.
Parecia o número de um salmo na igreja. Em seguida, ``banco''. Era o seu
trabalho, o sagrado trabalho que lhe provia o pão, o lar e a esposa,
sua razão de viver. Mas o banco havia falido, e o nome estava riscado;
ele se salvara em outro banco, após um curto período de muita
preocupação.

A hora chegou; a floricultura, o cocheiro -- era a época do noivado,
quando seus bolsos estavam cheios. A seguir, a loja de móveis, a loja
de tapetes, a casa montada. A empresa de transportes: o casal se muda.

A bilheteria da Ópera: 50.~50. Os dois são recém"-casados e vão a Ópera
aos domingos. Os melhores momentos juntos eram passados lá, em
silêncio, quando se imaginavam em meio à beleza e harmonia do outro
lado da cortina, naquele mundo de faz"-de"-conta.

A seguir vem um nome riscado. Um amigo que ganhou certo renome, não
soube lidar com o sucesso e caiu, desamparado; depois, precisou viajar
para longe. Como isso é delicado!

Aqui alguma novidade parece ter entrado na vida do casal. Está escrito
com uma caligrafia de mulher e a lápis: ``Parteira''. Mas que parteira? --
Sim, aquela de capa longa e com o rosto amigável e simpático que vinha
silenciosa, e nunca passava pela sala, preferindo ir até o quarto pelo
corredor. Abaixo do nome dela estava escrito ``Doutor L''.

Pela primeira vez surge o nome de um parente: ``Mamãe''. Era a sogra que,
sendo discreta, ausentara"-se para não incomodar os recém"-casados;
mas quando chamada, na hora da necessidade, veio de bom grado, pois
precisavam dela. 

Depois começa um grande rabisco azul e vermelho. Agência de emprego: a
criada se mudou, ou uma nova foi chamada. Farmácia. Hum! As coisas
pioram. A companhia de laticínios. Aqui um pedido de leite, livre de
micróbios.

Loja de especiarias, açougueiro\ldots{} A casa começa a ser gerida pelo
telefone, pois a patroa não está mais no seu lugar. Não. Ela está de
cama.

Não conseguiu ler o que vinha mais adiante, porque seus olhos começaram
a se turvar, como deve acontecer aos que se afogam quando tentam ver
através da água salgada. Estava escrito: ``funerária''. Já diz o
bastante! Um grande e um pequeno. Subentenda"-se: caixões. E em
parênteses estava escrito: ``de cinzas''. Depois mais nada! 
Cinzas às cinzas, e é assim mesmo. 

Apesar disso, pegou aquele papel amarelo e deu"-lhe um beijo. Depois,
colocou"-o no bolso da camisa. Em dois minutos, revivera dois anos de
sua vida. Quando saiu para a rua, não estava mais cabisbaixo; pelo
contrário: tinha a cabeça erguida como um homem feliz e orgulhoso, pois
sentia que, apesar de tudo, havia possuído o que há de mais belo.
Quantos infelizes nunca o possuíram!

\chapter{O herói e o bobo}


\textsc{Foi naquele entardecer} primaveril de 1880, dia que nós suecos jamais
esquecemos, pois o celebramos todos os anos. Em Blockhusudden, era a
mesma noite inesquecível.\footnote{ Dia do retorno da expedição
comandada por Adolf Nordenskiöld, a bordo do \textit{Vega} (24 de abril de 1880),
que descobriu, navegando ao norte da Sibéria, a passagem
noroeste, uma rota do Atlântico para o Pacífico. [N.~do T.]} Um casal
de velhos lavradores estava lá; pessoas humildes, que passaram juntas a
maior parte de suas vidas sofridas. Os dois olhavam fixamente para o
canal, que jazia naquela escuridão sob as estrelas lacrimejantes,
observando na penumbra um homem que armava algo indefinível no meio da
ponte; ficaram ali parados, por um longo tempo: ora contemplando a
escuridão do canal, ora admirando as luzes da cidade.

Por fim viram ao longe uma, duas, muitas lanternas; nesse momento os
velhos apertaram as mãos e, em silêncio, sob a luz das estrelas, deram
graças a Deus por ter lhes devolvido o filho que dividira a honra pela
heróica circunavegação da Ásia, e que foi chorado como morto por um
ano. Na verdade o filho não fora tão importante, mas fizera parte da
expedição, e agora iria jantar com o rei, receber uma insígnia e uma
nomeação que poderia lhe prover o sustento, já que o governo aprovara
uma pensão federal em dinheiro vivo para os navegantes.

As lanternas cresceram e se aproximaram; um pequeno barco a vapor vinha
rebocando uma grande barca escura, que parecia tão natural a curta
distância como qualquer outra grandeza pode parecer. 

O homem na ponte com a armação esquisita ficou visível ao acender um
fósforo.

-- O que será aquilo? -- disse o velho. -- Parecem grandes velas de
estearina.

O casal se aproximou para ver melhor.

-- Parece uma vara para estender redes de pesca -- disse a velha, que era
do litoral.

Foi quando\ldots{} Fsss! Chop! Cabum! E os velhos se viram em meio a fogo e
labaredas! Rumo às estrelas celestes alçavam"-se agora todos os fogos,
que lá no alto acendiam novas estrelas. Se um astrônomo as visse de um
observatório, acreditaria ver novas constelações na abóbada celeste.

A novidade chegou de fato ao céu e à terra no ano de 1880, quando
surgiram novas idéias em novas cabeças, nova luz e novas descobertas.
Com o trigo novo nasce também a erva daninha, mas esta deve estar
presente para oferecer umidade e sombra; e durante a colheita deve"-se
separá"-la do trigo. Mas ela deve estar lá, pois é \textit{parte}, tal
como o joio é parte do trigo.

O que tanto explodia eram esplêndidos fogos de artifício; e quando a
fumaça se desfez -- pois a fumaça é parte do fogo -- a pompa terminou. 

-- Seria divertido estar na cidade hoje à noite! -- disse a velha.

-- Não! -- respondeu o velho. -- Só iríamos atrapalhar, gente simples que
vai aonde não é chamada acaba ganhando a reputação de presunçosa.
Encontraremos o menino de manhã, quando ele se despedir da noiva, que
tem mais a ver com isso tudo do que nós.

O velho falou com juízo; e é de se esperar que os velhos tenham juízo,
porque se eles não tiverem, quem teria? 

E assim foram para a cidade!
\asterisc

Mas vejamos o que se passou com o filho! Era navegador; media a
profundidade do mar, a altitude da terra e o movimento do céu; podia
dizer que horas eram apenas olhando para o sol; e, quando olhava as
estrelas, sabia que distância haviam percorrido. Era um homem
formidável, e acreditava que tinha o céu e a terra nas mãos; que media
o tempo e acertava os ponteiros do relógio que marca a eternidade.
Depois de voltar e ser convidado para o jardim do rei e receber a
insígnia no paletó, sentia"-se acima dos outros e não se orgulhava
muito de seus pais pobres ou de sua noiva; e mesmo sem que ele dissesse
nada, eles percebiam. O filho também estava um pouco tenso, porque
tinha essa propensão.

Pois bem; as grandes festividades na capital haviam passado e o bairro
dos estudantes também queria homenagear os heróis que retornaram a
casa; e foram para lá. 

Os estudantes são pessoas excêntricas, que só lêem livros de doutores
oniscientes e por isso se acham mais espertos que os outros. Além
disso, são jovens, e portanto estouvados e cruéis.

Após o jantar, quando os velhos doutores terminassem os sábios e
respeitosos discursos para a glória dos navegantes proferidos durante o
jantar, os estudantes fariam uma parada festiva.

O navegador sentou"-se em uma sacada com sua noiva, do lado dos outros
grandes senhores; os sinos dobraram, ouviram"-se disparos de canhões,
toques de clarim e tambores. Bandeiras tremulavam e pessoas acenavam;
em seguida o desfile começou.

Primeiro entrou o navio, com marujos e tudo; depois morsas e ursos
polares com tudo o que têm direito, então vieram estudantes
fantasiados como os heróis. Até mesmo o grande Nordenskiöld estava lá,
com casaco de pele e óculos; pode"-se imaginar que a representação não
era muito respeitosa, e a honra de ser homenageado assim era um tanto
questionável; mas tudo bem. Ao menos a intenção era boa. Logo veio
mais um herói, e ainda um outro; todos estudantes fantasiados. 

O navegador veio por último. Não era bonito, mas isso um rapaz não
precisa ser, uma vez que seja um navegador hábil ou possua outro
talento qualquer. Mas foi assim que o representaram! Escolheram um
estudante feioso e careteiro para representá"-lo. Até aí era
aceitável, mas a natureza tinha lhe dado um braço mais curto que o
outro, o que os estudantes também representaram; coisa feia de se
fazer, pois ninguém tem culpa de uma deformidade dessas.

Mas quando o bobo que representava o navegador chegou em frente à
sacada, disse umas palavras com sotaque de Sk\r ane para ridicularizar o
navegador, que era de lá; foi uma grande bobagem, porque cada um fala
com o sotaque que aprendeu de sua mãe e isso deve ser motivo de
orgulho. 

A platéia riu somente por educação, já que todos estavam se divertindo
de graça; mas a noiva sentia o coração magoado -- o que era de se
esperar, pois não gostaria que fizessem troça de seu futuro marido. O
navegador ficou sombrio e mudo; para ele, toda alegria da festa
acabara. Mas não podia deixar transparecer, porque seria considerado
tolo, incapaz de entender uma brincadeira.

Foi quando veio o pior! O bobo dançava e fazia macaquices; começou a
representar com mímicas o nome do navegador -- o sobrenome que herdou do
pai e o nome que recebeu da mãe no batismo, tão sagrado para ele que
nunca quis mudá"-lo, embora fosse um pouco ostentoso.

Nesse instante o navegador quis se levantar para ir embora, mas a noiva
o segurou e ele permaneceu sentado.

Quando o desfile acabou e todos na sacada se levantaram, Nordenskiöld
dirigiu"-se  à noiva do navegador e pôs"-lhe a mão no ombro, dizendo
com um sorriso bondoso:

-- Eles têm um jeito bem esquisito de celebrar as grandezas neste país. E
nós temos de engolir!

De noite houve uma nova festa da qual o navegador também participou; mas
sua satisfação havia desaparecido, e sentia"-se pequeno pelo tanto que
riram dele. Sentia"-se ainda menor que o bobo, que se saiu bem como
brincalhão. O navegador estava desanimado, preocupado com o futuro e
inseguro. Por onde ele passava no grande jardim, via sua caricatura
feita pelo bobo, que estava em toda parte. Via seus defeitos aumentados
e, acima de tudo, seu orgulho e sua vaidade imitados; e o pior era que
seus pensamentos e suas inclinações mais recônditas haviam sido 
revelados.

Por três horas sofridas ele teve tempo de passar a limpo o livro de sua
consciência; o que homem nenhum ousara lhe dizer, o bobo havia
mostrado. De fato, é muito bom conhecer a si próprio -- Sócrates afirma
ser esse o maior dos bens; e ao fim da noite o navegador havia se
acalmado, reconhecido suas fraquezas e decidido melhorar.

Passou em frente a um grupo e ouviu uma voz atrás de uma sebe.

-- É extraordinário como o navegador amadureceu. É um homem muito
simpático!

O comentário fez"-lhe bem ao coração; mas o que o alegrou até o fundo
da alma foi uma frase da noiva:

-- Hoje você está muito gentil, e por isso mais bonito!

Ele, bonito? Era mesmo um milagre; e ainda que milagres não aconteçam
mais, precisava acreditar neste, pois sabia que era feio.

Por fim, Nordenskiöld  bateu na taça de ponche e fez um discurso, que
começou mais ou menos assim:

-- Quando um conquistador romano desfilava em triunfo, sempre vinha atrás
dele um escravo, gritando para o general enquanto o povo e o senado
celebravam"-no: ``Lembrai"-vos que sois apenas um homem!''. E ao lado
do carro do vencedor ia um bobo que fazia pouco caso da conquista com
afrontas, e desprezava"-o entoando canções de escárnio. Um velho e bom
costume, porque nada é mais perigoso a um homem do que julgar"-se um
deus, e nada desagrada mais os deuses do que a arrogância humana!  Meus
jovens amigos; a façanha que protagonizamos ao voltar para casa talvez
tenha sido exagerada; a embriaguez da vitória subiu"-nos à cabeça, e
por isso foi bom, hoje, vermo"-nos satirizados. Não que eu inveje o
papel do bobo ou me iluda acreditando em suas belas intenções; longe
disso. Mas em todo caso, agradeço a celebração um tanto peculiar com
que nos brindaram. Graças a ela, percebo que ainda tenho muitas
conquistas a fazer, mas sempre que a aspiração divina me tentar,
lembrarei que sou apenas um homem!

-- Bravo! -- gritou o navegador.

E a festa prosseguiu sem perturbações, com franca  alegria e animação.
Nem mesmo o bobo perturbou"-os, pois se encolhera de vergonha e
desaparecera. 

 Deixamos aqui o navegador e Nordenskiöld! Vejamos agora o que se passou
com o bobo.

O bobo, que estava à mesa durante o discurso do grande Nordenskiöld,
recebeu um olhar do navegador, um olhar daqueles que são como uma
flecha de fogo, capaz de incendiar uma fortaleza. E o bobo ficou fora
de si, e quando saiu pela noite, parecia  que suas roupas estavam em
chamas. Ele não era um homem bom. Os bobos e os carrascos com certeza
também são gente, mas não da melhor estirpe. O bobo também tinha muitos
defeitos e fraquezas, como todos nós; mas decidiu escondê"-los. E
aconteceu uma coisa singular: por ter imitado o navegador o dia
inteiro, e sob a influência da bebida, ele havia incorporado de tal
forma aquele papel que não conseguia mais sair dele; e como
representara os defeitos e as fraquezas do navegador, foi como se as
recebesse em si mesmo; e o tal olhar do navegador encheu"-lhe a alma
com estes defeitos como a vareta enche a arma de pólvora. O bobo estava
carregado com a personalidade do navegador, e por isso começou a berrar
e a contar vantagem quando saiu pela rua. Mas não teve sorte; um
policial apareceu e ordenou que ficasse quieto. O bobo deu uma resposta
engraçadinha com sotaque de Sk\r ane, imaginem! O guarda, que por acaso
era de lá, não gostou nem um pouco e levou o bobo para a cadeia. Ora,
os bobos têm a mesma dificuldade para entender coisas sérias que os
policiais têm para entender piadas; por isso, o bobo resistiu à prisão
com todas as suas forças. E o resultado disso foi que apareceu um
cassetete e\ldots{} Pá! Pá! Pá! 

E nisso ele escapou!

Pode parecer que foi castigo suficiente, mas não. O bobo, longe de se
endireitar, ficou ainda mais exasperado e partiu como um índio
\textit{sioux} parte para a guerra, em busca de vingança. 

Por algum motivo -- talvez acaso --, seus passos o levaram a Tullgatan,
até o bairro camponês. Fazendeiros e moleiros estavam sentados ao redor
de uma mesa no quintal, bebendo à saúde dos grandes homens ao lado de
uma lamparina. Quando avistaram o bobo, tomaram"-no pelo navegador e
se encheram de alegria quando ele, humilde, sentou"-se para brindar
com eles.

O espírito de soberba do navegador voou então para dentro da câmara de
pólvora do bobo e ele se incendiou. Gabou"-se de suas proezas: disse
que fora ele a conduzir a expedição; que se não tivesse ele próprio
medido a profundidade do mar, teriam encalhado; que se não tivesse
visto a posição das estrelas, nunca teriam voltado para casa.

E paff! O bobo levou um ovo bem na testa, e o moleiro disse:

-- O senhor navegador é um grande fanfarrão; disso já sabíamos, mas foi
ele quem escreveu no jornal que Nordenskiöld é um 
humboldt.\footnote{ Alexander von Humboldt (1769-1859), naturalista, geógrafo
e explorador alemão. [N.~da E.]} 

Foi quando a outra fraqueza do navegador tomou conta 
do bobo e ele disse esta inverdade:

-- Nordenskiöld é um vigarista.\footnote{ Trocadilho intraduzível com
Humboldt. Em sueco, \textit{humbug} é vigarista. [N.~da E.]} Aquilo foi a gota
d’água; os camponeses se insurgiram e amarraram o bobo a um saco cheio
de farinha, usando uma rédea. Maquiaram seu rosto com a mais fina
farinha de trigo e pintaram"-no com a cera derretida da lamparina.
Enquanto isso, um criado costurou as roupas do bobo no saco de farinha,
usando agulha e barbante. 

Mas não acabou aí; com um lampião na ponta de uma vara, o exército de
camponeses conduziu a carroça, o saco de farinha e o bobo rua acima,
até chegarem à praça central. Lá, exibiram o bobo para o povo, que ria.
Bem feito!

Quando o soltaram, o bobo saiu sozinho pela beira do caminho e
sentou"-se numa escadaria para chorar. Um marmanjo chorando! Era quase
de dar pena.

\chapter{A história de São Gotardo}


\textsc{No cantão suíço} de Uri -- um dos quatro cantões iniciais, o cantão de
Guilherme Tell e de Walter Fürst --, do lado norte do maciço de São
Gotardo, fica a cidadezinha de Göschenen; lá se fala alemão e moram
pessoas pacatas e amistosas que possuem o direito de decidir sobre seus
assuntos; é também a área que a floresta sagrada protege contra
avalanches de neve e desabamentos de terra. Lá se encontra essa
cidadezinha verdejante à beira de um riacho, que move a roda do moinho
e abriga as trutas. 

Era o entardecer de um sábado, quando o sino fez soar o ângelus e o povo
da aldeia se reuniu junto ao poço, debaixo da nogueira.
Aproximaram"-se o agente do correio, o prefeito e até o general, todos
em mangas de camisa e carregando foices nos ombros. Vieram lavar as
foices da colheita do dia, pois nesse lugar o trabalho é honrado, ainda
mais se feito com as próprias mãos. Os rapazes também chegaram com
foices, e meninas com baldes de leite; por último foram reunidas as
vacas típicas da região, uma raça gigante que dá vacas grandes como os
touros. A terra é fértil e abençoada, embora no lado norte de São
Gotardo não cresçam vinhas nem oliveiras, nem albízias e tampouco o
milho exuberante. A grama verde, os grãos dourados, as nogueiras
frondosas e a doçura da beterraba são a dádiva da região.

Sob um desfiladeiro do São Gotardo, ao lado do poço, encontrava"-se a
estalagem Cavalo de Ouro; e no jardim, ao redor de uma longa mesa
solitária, estavam sentados ao fim do dia de trabalho os homens
cansados que participaram da colheita; todos na mesma mesa, sem
hierarquia: o prefeito, o agente do correio, o general e também os
criados, o fabricante de chapéus de palha e seus empregados, o
sapateiro da aldeia, o mestre"-escola e todos os demais.

Falaram da colheita e da ordenha; juntos, cantaram canções que soam como
as tríades das trompas de pastoreio e as sinetas no pescoço das vacas.
As canções falavam da primavera e de suas alegrias puras, do verde
renovado e do azul da esperança.

E beberam da cerveja clara.

Depois os jovens se levantaram para jogar, lutar e saltar, porque no dia
seguinte haveria uma festa com competição de tiro, onde o que contava
era ser o mais habilidoso. E por isso nessa noite as trompas soaram o
toque de recolher mais cedo, para que ninguém chegasse maldormido e
sonolento para a festa, onde a honra da cidadezinha estaria em jogo.
\asterisc

O domingo animou"-se com o repicar dos sinos e a luz do sol; pessoas
com roupas festivas vieram das cidades vizinhas e todos pareciam
recém"-acordados de uma noite de sono revigorante. Quase todos os
rapazes trocaram a foice pela espingarda; meninas e mulheres casadas
lançavam olhares curiosos e animados para os rapazes, porque era para
servir à casa e ao lar que eles aprendiam a atirar; e o melhor atirador
teria a honra de abrir o baile com a moça mais bonita.

Chegou uma enorme carroça, puxada por quatro cavalos intrépidos ornados
com fitas e flores; toda a carroça era um único caramanchão; não se via
ninguém dentro, mas de lá vinham belas vozes agudas que entoavam
canções sobre a terra e o povo suíço, o mais belo e valente dos povos. 

Depois veio a procissão das crianças, de duas em duas, com as mãos
dadas, como se fossem bons amigos ou pequenas noivas e noivos.

Os sinos dobraram e todos subiram em direção à igreja.

Ao fim da missa, teve início a festa; e na pista de tiro, montada contra
o imenso desfiladeiro do São Gotardo, os disparos logo começaram. O
filho do agente de correio era o melhor atirador da aldeia e não havia
dúvida que ganharia o prêmio. Ele atirou sua série e com seis disparos
acertou o alvo quatro vezes.

Foi quando escutaram um chamado e um estalido no alto da montanha;
pedras e cascalho rolaram precipício abaixo, e os abetos da sagrada
floresta que os protegia envergaram"-se como durante as tempestades.
Em cima de um bloco de pedra, com a espingarda no ombro e agitando o
chapéu, apareceu o feroz caçador de camurças, Andrea. Vinha de Airolo,
a cidade italiana do cantão de Tessino, do outro lado da montanha.

-- Não entre na floresta! -- gritaram todos os atiradores.

Andrea não entendeu.

-- Não entre na floresta sagrada, ou a montanha vai desabar sobre nós! --
gritou o prefeito.

-- Pois que desabe! -- respondeu Andrea, descendo pelo desfiladeiro numa
velocidade impressionante. -- Aqui estou!

-- Chegou atrasado -- respondeu o prefeito.

-- Nunca chego atrasado -- replicou Andrea, que depois foi até a pista e
levou a espingarda seis vezes junto ao rosto, acertando seis vezes no
alvo.

Ele deveria ser o campeão, mas a confraria tinha leis próprias e não
gostava do povo escuro e latino que vivia do outro lado da montanha,
onde as vinhas cresciam e fiavam seda. Havia entre eles uma velha
inimizade e os tiros de Andrea não podiam ser considerados. Contudo,
ele foi até a moça mais bela, que era a própria filha do prefeito, e
com grande cortesia pediu para abrir o baile com ela.

A bela Gertrud corou, porque via Andrea com bons olhos; mas ela era
obrigada a recusar o convite.

O semblante de Andrea escureceu e, inclinando"-se para frente,
sussurrou no ouvido da moça, que ficou vermelha como sangue:

-- Você será minha, mesmo que eu tenha de esperar dez anos. Andei pela
montanha por oito horas para encontrar você, e por isso cheguei
atrasado. Mas da próxima vez vou chegar na hora, mesmo que tenha de
passar por dentro da montanha!

Quando a festa e o baile acabaram, os atiradores estavam sentados em
frente ao Cavalo Dourado, junto com Andrea; mas Rudi, o filho do agente
de correio, estava no banco mais alto, já que era o campeão de tiro --
segundo as regras, claro, mas Andrea era o campeão de fato. Rudi queria
começar uma discussão.

-- Pois é, Andrea -- disse ele. -- O senhor é um caçador formidável, mas
atirar em camurças não é a mesma coisa que acertá"-las.

-- Nunca errei um tiro -- respondeu Andrea.

-- Bem, quanto às camurças pode até ser. Mas todos já atiraram no anel de
Barba"-Roxa, e ninguém acertou! -- retrucou Rudi.

-- O que é o anel de Barba"-Roxa? -- perguntou um estrangeiro que nunca
viera antes a Göschenen.

-- Bom -- respondeu Rudi --, o senhor pode vê"-lo bem ali!

E apontou para um ponto acima do caminho na montanha, onde um grande
anel de cobre repousava, pendurado em um gancho. E continuou:

-- O imperador Frederico Barba"-Roxa costumava passar por este caminho
quando ia para a Itália; ele o cruzou seis vezes e foi coroado tanto em
Milão quanto em Roma. E como foi assim que ele se tornou o imperador
alemão"-romano, mandou colocar este anel na montanha do lado alemão,
como sinal de que casou a Alemanha com a Itália. 

A lenda diz que se o anel for retirado do gancho, esse casamento infeliz
estará anulado.

-- Vou anular esse casamento então -- disse Andrea. --, já que meus
antepassados libertaram minha infeliz Tessino dos tiranos de Schwütz,
Uri e Unterwalden.

-- O senhor não é suíço? -- perguntou o prefeito, de cara fechada.

-- Não, sou um italiano da Confederação Suíça.

Depois disso Andrea carregou sua espingarda com uma bala de ferro, mirou
e atirou. O anel deu um salto no gancho e caiu; o anel de Hohenstaufern
e Barba"-Roxa.

-- Viva a Itália livre! -- gritou Andrea, abanando o chapéu. Mas ninguém
mais se manifestou. Andrea pegou o anel e entregou"-o ao prefeito,
dizendo:

-- Guarde esse anel como uma lembrança minha e desse dia em que foram
injustos comigo.

Em seguida, foi até Gertrud e beijou"-lhe a mão. Depois, subiu pela
montanha e desapareceu; foi visto de novo entre as nuvens. Mas após um
instante Andrea reapareceu num lugar mais alto. Não era ele, mas sua
sombra engrandecida na nuvem; estava de cabeça erguida, ameaçador sobre
o vilarejo alemão.

-- Esse era o demônio em pessoa! -- disse o chefe de polícia.

-- Não, era um italiano! -- objetou o agente do correio.

-- Já que está tão tarde -- disse o prefeito --, vou contar um segredo de
governo que vai sair amanhã no jornal!

-- Escutem! Escutem!

-- Informaram pelo telégrafo que depois que o imperador francês foi
capturado em Sedan, os italianos expulsaram as tropas francesas de
Roma. Vittorio Emanuele marcha nesse instante rumo à capital.

-- Uma grande notícia. Acabaram os passeios romanos dos alemães. Andrea
já devia saber, pois parecia um tanto esnobe.

-- Devia saber de mais coisas ainda -- disse o prefeito.

-- O quê? O quê?

-- Logo os senhores verão!

E viram mesmo.
\asterisc

Um belo dia, senhores estrangeiros chegaram, trazendo instrumentos para
explorar a montanha; parecia que estavam procurando o anel de
Barba"-Roxa, pois foi para aquela direção que apontaram o binóculo,
olhando na bússola como se não soubessem onde era o norte e o sul.
Deram um grande jantar no Cavalo Dourado; até o prefeito compareceu. À
sobremesa se falou em milhões e milhões.

Passado algum tempo, demoliram o Cavalo Dourado; carregaram a igreja
para longe, parte por parte, e reconstruíram"-na em um lugar mais
afastado; demoliram metade do vilarejo, construíram acampamentos,
mudaram o curso do riacho e removeram a roda do moinho; fecharam a
fábrica e venderam os animais.

Vieram três mil trabalhadores de pele escura que falavam italiano. As
belas canções sobre a velha Suíça e as alegrias puras da primavera
silenciaram. Em seu lugar, ouviam"-se batidas dia e noite; trouxeram
picaretas até onde o anel de Barba"-Roxa costumava ficar e lá
começaram a construir um túnel por dentro da montanha.

Como bem se sabe, não há nenhuma dificuldade especial em perfurar uma
rocha; mas o plano era abrir dois buracos, um de cada lado, para que os
dois se encontrassem retos como uma régua no meio do caminho. E nisso
ninguém acreditava, pois eram quase 15 quilômetros de pedra. Quinze quilômetros!

-- Imagine se eles não se encontrarem! Vão ter de começar tudo de novo!

Mas o engenheiro"-chefe disse: eles vão se encontrar.

E Andrea, do lado italiano, acreditou no engenheiro"-chefe; porque,
como sabemos, era um rapaz de boa pontaria. Por isso entrou na equipe
de trabalhadores e se tornou o chefe dos operários.

Era o trabalho ideal para Andrea. Já não via a luz do sol, os verdes
prados nem a brancura dos Alpes. Mas pensava estar abrindo seu caminho
para encontrar Gertrud, aquele caminho através da montanha que ele, num
momento de grandiloqüência, prometera percorrer.

Por oito anos ficou trabalhando no escuro, esgotando"-se numa vida de
cão. Na maior parte do tempo usava pouca roupa, porque lá dentro fazia
um calor de trinta graus. Ora encontravam uma nascente de rio, e então
ele vivia em meio à água; ora um depósito de argila, e então vivia em
meio à lama. Quase sempre o ar era viciado; os antigos colegas caíam
por terra e os novos chegavam. Por fim Andrea também caiu e foi levado
para o hospital. Lá ficou obcecado com a idéia de que os dois túneis
jamais se encontrariam; e isso era o que mais o fazia sofrer. Nunca se
encontrariam!

Na sala também estavam acamados os trabalhadores de Uri que deliravam; e
quando voltavam a si, perguntavam sem trégua:

-- O senhor acha que vamos nos encontrar?

Jamais houvera um encontro tão esperado entre as duas cidades como o
encontro no interior da montanha. Eles sabiam que, caso se
encontrassem, séculos de inimizade cessariam e, reconciliados, os dois
povos se abraçariam uma vez mais.

Andrea se curou e voltou ao trabalho. Participou da greve de 1875;
atirou pedras, foi para a cadeia, mas saiu de novo.

No ano de 1877, Airolo, sua cidade natal, pegou fogo.

-- Agora queimei meus navios; não há mais volta. Preciso avançar! -- disse
ele.

No ano de 1879, o dia 19 de julho foi uma data triste. O
engenheiro"-chefe do túnel entrou na montanha para fazer medições e
cálculos; e enquanto estava lá dentro, teve um ataque e morreu. 
No meio do caminho! Deveriam tê"-lo sepultado lá mesmo, como
um faraó, na maior pirâmide de pedra que existe; e deveriam ter gravado
seu nome -- Favre -- no interior da montanha.

Mesmo assim os anos passaram. Andrea juntou dinheiro, experiência e
habilidade. Nunca visitava Göschenen, porém uma vez por ano ia até a
floresta sagrada e via a devastação, como a chamava. Ele nunca via
Gertrud nem escrevia para ela; não era preciso, porque vivia com ela no
pensamento, e sabia ter recebido sua simpatia.

No sétimo ano o prefeito morreu na miséria.

-- Que sorte que ele era pobre! -- pensou Andrea; mas este não é um
pensamento comum entre os genros.

No oitavo ano aconteceu uma coisa extraordinária. Andrea liderava os
operários dentro do túnel italiano e batia com sua picareta. A
atmosfera era sufocante, o pouco ar que havia estava viciado e seus
ouvidos zumbiam. Então ele ouviu umas batidinhas, parecidas com as de
uma larva que dá em árvores, chamada bicho"-carpinteiro.

-- Terá chegado a minha hora? -- pensou em voz alta.

-- É a hora! -- respondeu uma voz dentro, ou mesmo fora dele! Andrea ficou
assustado.

No dia seguinte ouviu mais uma vez as batidinhas, porém ainda mais
distintas, o que o levou a pensar no relógio que trazia no bolso.

Mas no dia seguinte, que era feriado, ele não escutou nada e achou que
fora apenas seu ouvido; ficou com medo e foi à missa. Lá, elevando
pensamentos calmos, lamentou a instabilidade de sua vida. A esperança o
abandonara; a esperança de poder viver o grande dia, a esperança de
receber o grande prêmio estipulado para a primeira picareta que
atravessasse a parede de pedra, a esperança de ganhar o amor de
Gertrud!

Apesar disso, na segunda"-feira estava de volta com sua picareta, mas
desanimado; não acreditava mais que encontrariam os alemães dentro da
montanha.

Golpeava e golpeava a rocha, mas sem pressa, lento como as batidas de
seu coração após a doença causada pelo túnel. De repente, ouviu um
barulho de tiro e um estrondo ensurdecedor, de dentro da montanha, do
outro lado.

E então Andrea percebeu que eles haviam se encontrado.

Caiu de joelhos e agradeceu a Deus; levantou"-se e voltou a bater. Não
parou para tomar o café"-da"-manhã nem o almoço; não fez pausas nem
ceou. Quando o braço direito adormeceu, usou o esquerdo. Ao mesmo
tempo, pensava no engenheiro"-chefe que perecera no meio do caminho;
cantou a canção dos três reis magos naquele forno incandescente, porque
o ar ardia ao redor dele, enquanto a água pingava na sua cabeça e os
pés se arrastavam na lama.

Às sete horas, no dia 28 de fevereiro de 1880, Andrea caiu de frente com
a cara no chão. Sua picareta voou e bateu contra a parede da montanha.
Gritos de alegria retumbantes vindos do outro lado o acordaram, e ele
compreendeu, compreendeu que haviam se encontrado, que este era seu
último momento de estafa e que dez mil liras lhe pertenciam.

Após um curto suspiro para o todo"-misericordioso, ele pôs a boca na
fenda feita com a picareta e sussurrou baixinho, para que não o
escutassem: ``Gertrud''; e em seguida berrou nove vivas para os alemães.
Às onze horas da noite escutou"-se um estrondoso ``vamos lá!'' do lado
italiano, e com um estampido como o dos canhões num cerco a parede
desabou. Alemães e italianos abraçaram"-se e \mbox{choraram}, os italianos se
beijavam, e todos caíram de joelhos cantando um \textit{Te Deum
laudamus}.

Foi um grande momento; e foi em 1880, mesmo ano em que Stanley voltou da
África e Nordenskiöld da expedição a bordo do \textit{Vega}.

Quando cessaram as canções de graça ao Eterno, um trabalhador do lado
alemão entregou aos italianos um pergaminho. Era um
texto em honra e memória do engenheiro"-chefe, Louis Favre. Andrea
seria o primeiro a atravessar o túnel, levando o panegírico e o nome do
engenheiro no pequeno trem da mina até Airolo. E foi isso que ele fez
com muita honra, sentado num vagonete empurrado pela locomotiva.

Foi um grande dia, e a noite não foi menor. Em Airolo beberam vinho,
vinho italiano, e soltaram fogos de artifício. Discursos para Louis
Favre, Stanley e Nordenskiöld, discursos para São Gotardo, a montanha
cheia de mistérios, que por milênios fora um muro separando a Alemanha
da Itália, o Norte do Sul. Motivo de afastamento, sem dúvida, mas
também de aproximação, porque o São Gotardo divide suas águas de modo
igual para o \textit{Rhein} alemão e para o \textit{Rhône} francês,
para o Mar do Norte e para o Mediterrâneo\ldots{}

-- E para o Adriático -- interrompeu um de Tessino. -- Tenha a bondade de
não se esquecer o Tessino, que deságua no maior rio da Itália, o grande
Pó.

-- Bravo! Melhor ainda! Viva o São Gotardo, a grande Alemanha, a Itália
livre e a nova França!

Foi uma grande noite de um grande dia.
\asterisc

\pagebreak
Na manhã seguinte Andrea apareceu no escritório do engenheiro. Estava
vestido com seu traje de caçador italiano, de penas no chapéu,
espingarda no ombro e bornal nas costas, com o rosto e as mãos limpos.

-- Então, não quer mais saber de túnel! -- disse o caixa da engenharia, ou
o ``homem do dinheiro'', como era chamado. -- Bem, ninguém vai
criticá"-lo por isso; ademais, agora só há vagas para pedreiros.
Portanto, às contas!

O homem do dinheiro abriu um livro, escreveu uma nota e contou dez mil
liras em ouro. Andrea assinou em cruz, pôs o ouro no bornal e se foi.

Subiu num vagão da obra e em dez minutos estava junto às ruínas do que
fora a divisa. Fogueiras ardiam dentro da montanha nos dois lados do
trilho; trabalhadores gritaram vivas para Andrea e acenaram"-lhe com
os chapéus. Foi uma maravilha! Dez minutos depois, estava do lado
alemão, e quando viu a luz do dia na abertura, o vagão parou e ele
desceu.

Andrea caminhou em direção à luz esverdeada e viu a cidadezinha de novo,
ensolarada e verdejante; estava reformada, branca, resplendente, mais
bela do que outrora. E quando passou na frente dos trabalhadores, eles
saudaram seu chefe. Andrea dirigiu seus passos diretamente para uma
pequena casa, e sob uma nogueira, ao lado das colméias, estava Gertrud,
serena, mais bela e mais doce, como se o estivesse esperando por esses
oito anos.

-- Aqui estou eu -- disse. -- Do jeito que eu disse que chegaria, através
da montanha! Você vem comigo para a minha terra?

-- Com você, vou a qualquer lugar.

-- Eu lhe dei meu anel; ainda o tem?

-- Tenho, sim.

-- Então vamos nesse instante. Não, não vá buscar suas coisas; não
precisa buscar nada!

E foram de mãos dadas, mas não tomaram o túnel.

-- Vamos pela montanha -- disse Andrea, dirigindo"-se ao velho caminho. --
Vim até você através da escuridão; agora quero viver ao seu lado, e por
você, na luz!

\chapter{Quando a andorinha pousou no espinheiro}


\textsc{Quem estiver} no ancoradouro onde os barcos a vapor atracam e olhar para
o lago vê, à esquerda, uma montanha coberta por um bosque jovem e
verdejante, e por detrás dessa um edifício em forma de aranha; no meio
da construção há um círculo de onde se projetam oito alas, como oito
pernas que saem do corpo de uma aranha. Quem entra nesse prédio não sai
quando quer; alguns permanecem lá pelo resto da vida. É o presídio.

No tempo de Oscar \textsc{i}, a montanha não era verde. Era cinza e fria; e lá
nada crescia, nem mesmo musgo ou amor"-perfeito, apesar de serem
plantas comuns na rocha lisa. Só havia pedra cinza e homens cinzentos
que pareciam de pedra, quebravam pedras, dinamitavam pedras, carregavam
pedras. Entre esses homens da idade da pedra havia um que parecia mais
petrificado que os outros. Ainda era jovem quando, durante o reinado de
Oscar \textsc{i}, foi preso por homicídio.

Condenaram"-no à prisão perpétua, e por isso suas roupas cinzentas
traziam bordadas as iniciais \textsc{p.~p.}\footnote{ Prisão perpétua. [N.~da E.]}

O prisioneiro quebrava pedras na montanha durante o inverno e o verão.
No inverno ele via o porto deserto e vazio; a ponte semicircular e as
estacas pareciam uma boca com uma fileira de dentes. Ele enxergava o
depósito de lenha, a estrebaria e as duas enormes tílias desfolhadas.
De vez em quando um trenó ou meninos de patins deslizavam em frente à
ilhota. Durante o resto do tempo, era tudo deserto e silencioso.

Quando chegava o verão, o lugar se alegrava. O porto orlava"-se de
barcos esplêndidos, recém"-pintados, com bandeiras tremulando ao
vento. As tílias verdejavam; e foi à sombra delas que o prisioneiro
sentara"-se quando menino, à espera do pai, maquinista de um dos mais
belos navios a vapor.

Há muitos anos ele não escutava o vento soprar nas árvores, pois não
havia nenhuma no rochedo; mas, na memória, o prisioneiro ainda trazia o
farfalhar das tílias de Riddarholmen, a única coisa da qual sentia
saudade.\footnote{ Abaixo do porto do Riddarholmen fica o palácio 
Wrangelska, diante do qual há duas enormes tílias. Strindberg nasceu
numa casa que não mais existe, no cais, onde seu pai teve por muitos
anos um escritório de barco a vapor. [N.~do T.]}

Num dia de verão, um barco a vapor passou pela ilhota e o prisioneiro
escutou um barulho de onda, um som que lembrava uma banda de metais;
viu rostos alegres, que se ensombreceram ao perceber os cinzentos
homens de pedra na montanha.

Nesse momento, o prisioneiro amaldiçoou o céu e a terra, o seu destino e
a crueldade dos homens. E todo ano a maldição se repetia; ele e seus
companheiros maldiziam e atormentavam"-se uns aos outros, dia e noite;
pois o crime divide, mas a desgraça congrega os sofredores.

No começo, a vida era de uma crueldade sem sentido, e os carcereiros
maltratavam os presos, com arbitrariedade e sem misericórdia.

Mas eis que um dia tudo mudou: a comida ficou melhor, o tratamento
se tornou mais humano, e cada preso podia dormir na sua cela. O próprio
rei afrouxara um pouco os grilhões dos condenados; mas como a
desesperança petrificara o coração desses infelizes, eles não
conseguiam sentir nada parecido com gratidão; e acharam que era
melhor dormir dividindo a cela, para ter com quem conversar à noite. 
Reclamavam da comida, das roupas e da vigilância, como antes.

Um belo dia todos os sinos da cidade repicaram, especialmente os de
Riddarholmen. O rei Oskar havia morrido, e os presos tiveram um dia
livre. Como puderam conversar uns com os outros, falaram de um plano de
fuga, de como iriam matar os guardas; falaram até do falecido rei, e
falaram mal dele.

-- Se o rei fosse justo, devia ter nos libertado -- disse um preso.

-- Ou prender todos os criminosos que estão à solta.

-- Nesse caso ele seria o diretor da prisão, porque todo o país viria
parar aqui dentro.

Os presos são assim mesmo: consideram que todos são criminosos e que
eles só foram presos porque deram azar.

Em um dia quente de verão, o homem de pedra andava pela praia e escutava
as badaladas para Oscar, o Gentil. Ele estava procurando escorpiões de
água doce e peixes"-espinho embaixo das pedras na praia, mas não havia
encontrado nenhum; e nas águas não se viam nem ruivos nem alburnetes, e
por isso não se avistava nenhuma gaivota ou andorinha"-do"-mar. O
prisioneiro sentiu a maldição que pairava sobre aquele lugar, que até
mesmo os peixes e os pássaros evitavam. E pensou mais uma vez no seu
destino. Perdera seu nome, o nome de batismo e o sobrenome.
Chamava"-se número 65. Um nome que se escrevia com algarismos em vez
de letras. Não era recenseado, não pagava imposto; não sabia a própria
idade. Deixou de ser uma pessoa; não estava mais vivo, mas também não
estava morto. Ele não era nada. Apenas uma coisa cinzenta que se mexia
na montanha com um sol terrível abrasando"-lhe as roupas e a cabeça
raspada, de onde outrora pendiam cachos, que aos sábados eram penteados
pela doce mão materna, com pente fino. Hoje, não podia sequer usar
boné, porque assim seria mais fácil fugir. E quando o sol
abrasava"-lhe a cabeça, o prisioneiro lembrava da história do profeta
Jonas, que ganhou um pé de mamona do Senhor, para sentar"-se na
sombra.

-- Mas o que ele recebeu depois! -- zombou o prisioneiro, que não
acreditava de modo algum na bondade.

Nesse instante, viu um enorme galho de bétula balançando ao sabor das
ondas. Ainda estava verde, com o caule branco, e podia ter caído de um
barco de passeio. O prisioneiro puxou o galho para a terra, sacudiu a
água e levou"-o até uma fenda na rocha, onde o pôs de pé, apoiado por
três pedras. E sentou"-se debaixo da bétula e escutou o vento soprando
devagar nas folhas, que recendiam à mais delicada resina. Depois de
ficar sentado por uns instantes na sombra, adormeceu.

E sonhou:

A montanha toda era um bosque verde com árvores lindas e flores
cheirosas. Os pássaros cantavam, abelhas e zangões zumbiam e borboletas
esvoaçavam. Sozinha, isolada, estava uma árvore que ele nunca tinha
visto; e era mais bela do que as outras, pois tinha tantos caules como
um arbusto e os galhos faziam curvas elegantes como um bordado. Embaixo
das folhas brilhantes, havia um passarinho preto e branco pousado num
galho. Parecia uma andorinha, mas não era.

E no sonho o prisioneiro entendia a língua dos pássaros, e portanto
escutou e entendeu um pouco do que o passarinho cantava. A canção era
assim: 
\begin{verse}
Lama, lama, lama lá!\\ 
Venha, venha, venha cá!\\ 
E lá, lá, lamentou!\\ 
Mas se, se, se salvou!\\
\end{verse}

A canção falava de lama, lástima e redenção; foi o que ele entendeu.

Mas o sonho prosseguiu. Agora, estava sozinho no rochedo, com o sol
abrasador, morrendo de fome e sede. Os companheiros haviam se afastado
e o ameaçavam de morte, porque ele não queria participar do plano para
incendiar o presídio. Os presos formaram uma turba que o expulsou da
montanha a pedradas, tão longe quanto agüentasse correr. E agora o
prisioneiro havia se deparado com um muro.

Saltá"-lo parecia impossível; desesperado, o prisioneiro resolveu
correr o mais rápido que pudesse e bater com a cabeça no muro, pondo
fim à própria vida.

Saiu correndo morro abaixo, e eis que no mesmo instante abriu"-se um
portão, o portão verde de um jardim e\ldots{} ele acordou.

Ao dar"-se conta da situação e perceber que o lindo bosque
limitava"-se ao galho de bétula, o prisioneiro ficou muito desapontado
e disse:

-- Se ao menos fosse um galho de tília!

E quando parou para escutar, teve a impressão de que a bétula cantava
forte demais; soava como alguém peneirando areia e pedrinhas, enquanto
uma tília entoa as notas aveludadas do coração.

No dia seguinte o galho de bétula estava murcho e dava pouca sombra.

No outro dia, as folhas estavam secas como folhas de papel e rangiam
como dentes. Por fim, havia uma grande vara de bétula na fenda, o que
fez o prisioneiro lembrar de sua infância.

Pensou de novo no pé de mamona do profeta e blasfemou quando o sol
atingiu sua cabeça.
\asterisc

O novo rei subiu ao trono, e o governo e a administração do país tomaram
novo rumo. Decidiram fazer um novo acesso aos barcos, e os presos foram
mandados de barcaça para dragar as águas.

Foi a primeira vez em tantos, tantos anos que ele pôde deixar o rochedo.
E mais uma vez viajou por água; viu muita coisa nova em sua cidade
natal; em especial, a estrada de ferro e o trem. Era abaixo da estação
que eles iriam dragar.

Começaram a tirar todas a sujeira que havia no fundo do lago. Apareceram
gatos afogados e sapatos velhos, gordura podre da fábrica de vela de
estearina, filamentos coloridos da tinturaria Mão Azul, cascas de
árvore do curtume e todas as misérias humanas, que as lavadeiras por
cem anos enxaguaram nas margens. O fedor de enxofre e amoníaco era tão
forte que apenas os prisioneiros suportavam"-no.

Quando a barcaça estava cheia, os presos se perguntaram onde toda aquela
sujeira seria descarregada. A resposta veio quando o barco tomou a
direção do rochedo onde viviam.

Lá descarregaram toda a pilha e jogaram"-na sobre a montanha. O ar logo
ficou empesteado; Os presos andavam no meio da imundície, sujando as
roupas, as mãos e o rosto.

-- É o inferno! -- disseram os prisioneiros.

Por dois anos dragaram e depositaram os detritos no rochedo, que por fim
desapareceu.

As neves do inverno caíam todo fim de outono, cobrindo a sujeira com um
grande tapete branco.

Quando a última primavera chegou e a neve derreteu, o mau cheiro já
havia sumido e o lodo começava a parecer com terra. Nesse verão acabou
a dragagem e nosso homem de pedra foi trabalhar na forja; assim,
passava o tempo no interior do rochedo. Mas um dia de outono ele se
esgueirou para fora, e lá presenciou uma maravilha.

Ervas cresciam no lodo. Ervas feias e gordas, claro. O que mais dava era
a planta conhecida como cânhamo"-aquático, que parece uma urtiga e tem
flores marrons -- o que é feio, porque as flores devem ser brancas,
amarelas, azuis ou vermelhas. Mas havia também urtigas de verdade, com
flores verdes, bardanas, azedas, cardos, erva"-sal; ervas das mais
feias, ardidas, espinhosas e fedorentas, de que as pessoas não gostam e
que dão em montes de lixo, nos terrenos baldios, no lodo.

-- Limpamos o lago e fomos pagos com lixo! -- disse o prisioneiro. -- Isso
lá é jeito de nos agradecer?

Até que um dia o levaram para um outro rochedo, onde iriam construir uma
fortaleza; voltou a trabalhar com pedra; pedra, pedra, pedra!

No outro rochedo, perdeu um olho; de vez em quando, era espancado. Ficou
tanto tempo lá que o novo rei morreu e o sucessor foi coroado. No dia
da coroação um preso receberia o indulto real e seria solto. O
prisioneiro que tivesse se comportado melhor e, além disso,
reconhecesse o mal que havia feito receberia o indulto. E de fato
recebeu! Os outros presos acharam tudo uma injustiça, porque no meio
deles quem se arrependia ``do que não fez'' era visto com desdém.

Anos se passaram. Nosso homem de pedra estava agora muito velho, e como
era incapaz de realizar trabalhos pesados, levaram"-no de volta ao
velho rochedo, onde costurava sacos.

Um dia o pastor parou ao lado do homem de pedra, que costurava.

-- E então? -- perguntou o pastor. -- O senhor não pretende sair daqui?

-- Como? -- retrucou o homem de pedra.

-- Ora, quando compreender que fez uma injustiça!

-- Se eu vir uma pessoa que faça mais do que o justo, vou acreditar que
fui injusto. Mas isso eu não verei nunca!

-- Mais do que justo, só a misericórdia. Espero que o senhor perceba isso
o quanto antes!

Um dia mandaram o homem de pedra abrir estradas na montanha, onde não
estivera por vinte anos. Mais uma vez, fazia um dia quente de verão, e
os barcos a vapor zumbiam com graça pelas redondezas, vistosos como
borboletas. Quando o velho prisioneiro saiu no promontório -- não viu rochedo algum,
mas um lindo bosque verdejante, onde as folhas cintilavam ao vento como
pequenas ondas no mar. Na praia havia bétulas brancas e altas, álamos
balançantes e amieiros.

Era como naquele sonho. Debaixo das árvores a grama cochichava e as		\EP[]
flores acenavam; os zangões voejavam junto às borboletas esvoaçantes.
Muitos passarinhos \mbox{cantavam}, mas ele não compreendia o canto deles e
assim soube que não estava sonhando.

A montanha amaldiçoada se tornara uma bênção, e ele não podia deixar de
pensar no profeta e no pé de mamona.

-- Isso é o perdão, é a misericórdia! -- disse alguém dentro dele; ou uma
voz, um pressentimento, chame como quiser.

Quando um barco a vapor passou por lá, os rostos não se ensombreceram;
iluminaram"-se com a vista do belo verde; sim, pensou até que alguém
havia acenado, como se costuma fazer ao passar por um lugar de
veraneio.

O prisioneiro seguiu por um caminho sob árvores farfalhantes. Com
certeza não havia nenhuma tília, mas não ousava desejar uma tília para
que as bétulas não se transformassem em galhos secos, açoites; uma
lição que aprendera.

Enquanto seguia por um caminho coberto de folhas, viu um muro branco lá
adiante, com uma portão verde de madeira. Alguém estava tocando um
instrumento; não era um órgão, pois soava mais alegre e mais ligeiro.
Acima do muro via"-se um belo telhado, e uma bandeira azul e amarela
tremulava ao vento.

Acima do mesmo muro, uma bola colorida subiu, caindo em seguida;
pequenas vozes ternas riam, e o barulho de pratos e taças indicava que
estavam pondo a mesa.

Aproximou"-se do portão e viu\ldots{} Lilases floresciam, e embaixo deles
puseram a mesa; crianças brincavam, tocavam e cantavam.

-- É o paraíso! -- disse a voz dentro dele.

O velho ficou um bom tempo olhando; tanto que desabou de
cansaço, de fome, de sede, de todas as misérias da vida.

O portão abriu"-se e uma menininha vestida de branco saiu. Na mão,
trazia uma bandeja de prata com uma taça de vinho, o vinho mais rubro
que o velho jamais vira. A garotinha parou em frente a ele e disse:

-- Venha, meu bom velho; tome uma taça de vinho!

O velho aceitou e bebeu. Era vinho de gente rica, que vem de longe, dos
países ensolarados; e tinha o sabor doce da boa vida, em seus melhores
momentos.

-- É a misericórdia! -- disse sua própria voz velha e abatida. -- Mas,
menina, quanta ingenuidade! Se você soubesse quem eu sou, não teria me
oferecido vinho. Sabe quem eu sou?

-- Sei, claro. O senhor é um prisioneiro! -- respondeu a menina.

-- Sabia! E mesmo assim\ldots{} É a misericórdia.

Quando o velho homem de pedra voltou, não era mais de pedra; algo havia
começado a germinar em sua alma.

Ao passar por um precipício, viu uma árvore com muitos caules, como um
arbusto. Era o espinheiro"-cerval, a mais linda de todas as árvores;
mas o velho não sabia disso. Na árvore esvoaçava um passarinho
inquieto, preto e branco como uma andorinha, que as pessoas chamam de
andorinha"-das"-árvores, ainda que seu nome seja outro. Pousou entre
as folhas e por um longo tempo cantou com grande tristeza, mas imensa
doçura:
\begin{verse}
E lá, lá, lamentou!\\
Mas se, se, se salvou!
\end{verse}

Tudo era como no sonho; e nesse instante o velho compreendeu o que a
andorinha queria dizer.

\chapter[Os segredos do celeiro de fumo]{Os segredos do\break celeiro de fumo}


\textsc{Era uma vez} uma moça na Ópera. Era tão bela que as pessoas viravam"-se
na rua para olhá"-la, e cantava como poucos. O maestro e o compositor
ofereceram"-lhe seus reinos e corações. A moça aceitou os reinos, mas
os corações tiveram de esperar.

Era uma estrela de brilho inigualável! Passava de carruagem pela rua e
acenava para seu retrato, pregado em todas as vitrines das livrarias. A
fama cresceu e ela saiu em papéis de carta, sabonetes e cigarreiras.
Por fim, penduraram seu retrato no \textit{foyer} do teatro entre os
imortais; depois disso ficou muito arrogante.

Um dia ela estava no ancoradouro, diante de um mar de águas encapeladas
e fortes correntezas. O maestro, é claro, estava a seu lado, assim como
uma porção de jovens; a bela brincava com uma rosa que todos queriam
ganhar, mas só a ganharia quem a pegasse. Ela jogou a rosa longe, nas
ondas, e os jovens seguiram"-na apenas com os olhos; mas o maestro
atirou"-se ao mar, nadou nas ondas como uma gaivota, e logo tinha a
rosa entre os lábios.

No ancoradouro, ouviu"-se aplauso; e o maestro, que estava no mar, viu
nos olhos da bela que era amado. Mas quando tentou nadar de volta à
margem, não conseguia sair do lugar. O refluxo impelia"-o para trás;
mas ela, que estava na margem, não percebeu. Julgou que era um gracejo,
e por isso riu. Ele, que sentia a iminência da morte, levou a mal
aquele riso, que não era um riso bondoso; sentiu uma aguilhoada no
coração e com isso o seu amor se acabou. Mesmo assim, conseguiu sair do
mar, com as mãos sangrando por ter se arranhado no ancoradouro.

-- Minha mão é sua -- disse a bela.

-- Não a quero -- respondeu o maestro, deu as costas e se foi.

Foi um crime de lesa"-majestade contra a beleza, e por isso ele tinha de
morrer. O que levou o maestro a perder seu lugar, somente as pessoas do
teatro sabem; elas é que entendem dessas coisas. O maestro tinha uma
posição bem firme e levou dois anos para ser derrubado. Mas por fim
caiu; e quando a bela derrubou seu benfeitor, triunfou e ficou ainda
mais arrogante -- tanto que começou a dar na vista. O público viu que o
coração por trás da maquiagem era mau, e por isso deixou de se comover
com seu canto e acreditar em suas lágrimas e sorrisos.

Ela percebeu e ficou amargurada. Mas ainda era a rainha do teatro e
sufocava todos que desejavam crescer, deixando os jornais
ceifarem"-nos. Perdeu o favor do público, mas o poder, mais do que
nunca, era seu. E como agora era rica, poderosa e satisfeita, a vida
lhe era prazerosa; e as pessoas que se sentem bem não permanecem
magras; elas tendem a engordar. De fato, ficou um pouco gorducha.
Começou tão devagar e de mansinho que nem percebeu, até que já era
tarde. Ploft! A queda é sempre rápida, mas dessa vez a velocidade foi
espantosa. E a tortura a que ela se submeteu não ajudou em nada. Tinha
a mesa mais apetitosa da cidade e precisava passar fome; só que quanto
mais fome passava, mais engordava. 
\pagebreak
Em um ano a cantora estava fora do jogo e teve o cachê diminuído. Em
dois estava quase esquecida e substituída por outra moça, mais jovem.
No terceiro ano foi demitida, e então alugou um quarto de sótão.

-- Ela tinha uma gordura antinatural -- disse o encenador para o ponto.

-- O problema não era a gordura, era a arrogância! -- disse o ponto.
\asterisc

Estava sentada no sótão, olhando uma grande plantação embaixo. Havia
também um celeiro de fumo, do qual gostava por não ter nenhuma janela
de onde as pessoas pudessem observá"-la. Debaixo das telhas havia
ninhos de pardais, e ela nunca via pendurarem fumo, já que ali não se
plantava fumo algum. Passou o verão assim, olhando para o celeiro e
imaginando qual seria sua utilidade. As portas estavam trancadas com
grandes cadeados e ninguém aparecia entrando ou saindo -- supôs que o
celeiro guardasse segredos, mas só foi descobri"-los mais tarde.

Da fama passada, restavam"-lhe dois fios aos quais se apegava e pelos
quais vivia: eram seus papéis de destaque, Carmen e Aída, que ainda
estavam vagos por falta de uma sucessora. Na memória do público ainda
estava gravada a sua interpretação, que fora brilhante. Pois bem,
chegou agosto e reacenderam"-se os lampiões; o teatro reabriria em
breve.

A cantora estava sentada à janela, olhando para o celeiro lá embaixo,
que fora pintado de vermelho e recebera um novo telhado. Um homem
atravessou a plantação de batatas a passos largos; trazia uma grande
chave enferrujada. Abriu o celeiro e entrou. Chegaram mais dois homens,
que a cantora julgou reconhecer; e logo desapareceram no interior do
celeiro.

-- Isso está ficando interessante!

Após uns instantes os três rapazes saíram carregando umas coisas grandes
e estranhas que pareciam estrados de cama ou divisórias; junto à porta
viraram as divisórias e as colocaram com a frente inclinada contra a
parede do celeiro, deixando entrever que era um forno de sala, mas
pintado -- e mal pintado. Em seguida, via"-se a porta de uma casa de
campo, talvez a porta da cabana de um caçador. Depois um bosque, uma
janela e uma biblioteca.

 Eram os cenários do teatro. E após alguns instantes ela reconheceu a
roseira de \textit{Fausto}.

Era o depósito de cenários da Ópera; e ao lado dessa roseira ela própria
cantara ``\textit{Faites"-lui mes aveux}''.\footnote{ Ária de \textit{mezzo} do
3º ato da ópera \textit{Fausto}, de Charles Gounod. [N.~da E.]} Quando
entendeu que montariam \textit{Fausto}, sentiu uma dor no seu pobre
coração, mas achou um consolo: que ela não fazia o papel principal,
Marguerite.

-- Deixe estar esse Fausto! Mas não mexam com Carmen ou Aída, senão eu
morro!

E ficava lá, sentada à janela, acompanhando as mudanças no repertório;
duas semanas antes de os jornais anunciarem, ela já sabia o que
montariam na Ópera. E claro que isso era muito divertido! Viu \textit{O
franco"-atirador} ser puxado com o desfiladeiro do lobo e tudo; viu
\textit{O navio fantasma} com o barco e o mar, \textit{Tannhäuser},
\textit{Lohengrin} e muitos outros.

 Mas por fim chegou o dia em que o inevitável aconteceu. Os rapazes
arrastaram para fora um cenário (um deles se chamava Lindkvist e ela se
lembrava dele cuidando da maquinaria), e apareceu uma praça espanhola.
O cenário estava de través e assim ela não conseguia ver direito o que
era, mas um dos rapazes foi virando lentamente a armação; quando
terminou, a cantora viu a parte de trás, que sempre é feia. E ali
estava escrito, em grandes letras pretas que apareciam uma de cada vez,
devagar, como a torturá"-la; ali estava escrito de modo irrevogável e
nítido: \textsc{c}-\textsc{a}-\textsc{r}-\textsc{m}-\textsc{e}-\textsc{n}. 
Era a \textit{Carmen!}

-- Agora eu morro! -- exclamou a cantora.

Mas a coitada não morreu, nem mesmo quando montaram \textit{Aída}, e com
isso o nome dela foi apagado da memória das pessoas, das vitrines das
livrarias, dos papéis de carta, e por fim até seu retrato desapareceu,
misteriosamente, do \textit{foyer} do teatro. Não conseguia entender
como as pessoas esquecem tão rápido; era inexplicável. E lamentou"-se
como quem lamenta um morto: a cantora, a celebrada, estava mesmo morta!

Um dia ela passeava sozinha por uma rua deserta. No fim da rua havia um
depósito de lixo. Ela deteve"-se, sem nenhum motivo especial, mas viu
o suficiente de sua aniquilação: em meio ao lixo havia um papel de
carta com seu retrato, vestida de Carmen. Saiu dali chorando por
dentro; entrou numa rua transversal, onde a vitrine de uma pequena
livraria a fez parar, pois estava acostumada a parar diante dessas
vitrines e procurar seu retrato. Mas o retrato não estava lá. Em vez
disso, havia um cartaz onde, contra sua vontade, a cantora leu essas
notáveis palavras: ``A face do Senhor paira acima dos que fazem o mal,
para que sua memória suma da terra''. Os que fazem o mal! Era por isso
que sua recordação havia sido apagada. Era por isso que todos a haviam
esquecido.

-- Mas o mal não pode transformar"-se novamente em bem? Já não sofri o
bastante? -- lamentava.

Dirigiu"-se então para a floresta, onde não havia ninguém. Enquanto
andava desesperada, abatida, humilhada, viu outro homem sozinho vindo
em sua direção. Com os olhos, ele perguntou silenciosamente se poderia
cumprimentá"-la. Era o maestro, mas seu olhar não demonstrava censura
nem condescendência. Pelo contrário: expressava admiração, surpresa e
carinho.

-- Hanna, como você está esbelta e elegante! -- disse o maestro.

Ela se olhou e viu que era verdade; a tristeza havia queimado o excesso
de carne arrogante e ela estava mais bela do que antes.

--E não envelheceu nada. Parece até mais jovem!

Foram as primeiras palavras gentis que ouvia em muito tempo. E como
vieram de um homem a quem ela fizera um mal tão grande, Hanna
compreendeu o valor da bondade; e disse isto para o maestro.

-- Hanna, e a sua voz? Continua a mesma? -- perguntou, ele que não
suportava ouvir elogios.

-- Não sei! -- choramingou ela.

--Apareça amanhã no salão de ensaio\ldots{} Sim, na Ópera, comigo; vamos ver.
Voltei a trabalhar lá\ldots{}

A cantora foi, recuperou"-se e reergueu"-se. O público a havia
perdoado e esquecido, esquecido todo o mal; e ela é hoje uma cantora
muito aclamada, mais até do que antes.

Essa é uma história muito edificante, não?

\chapter{Jubal sem identidade}


\textsc{Houve uma vez} um rei chamado João Sem Terra; o motivo é fácil de
adivinhar. Mas houve também um grande cantor chamado Jubal Sem
Identidade, e o motivo é o que essa história vai narrar.

Klang era o sobrenome que recebera do pai, um soldado; e havia música
neste sobrenome.\footnote{ \textit{Klang}: som. [N.~do T.]} Da natureza, recebera uma grande força de vontade, que
era como uma barra de ferro em sua espinha, um bem de grande valia para
enfrentar os dissabores da vida. Criança ainda, enquanto aprendia a
falar, Klang não dizia \textit{ele} para referir"-se a si mesmo, como os
outros menininhos; dizia logo \textit{eu}.

-- Você não tem \textit{eu}! -- diziam os mais velhos.

Quando cresceu um pouco, expressava seus desejos com \textit{eu quero},
e escutava dos mais velhos: 

-- Você não tem vontades! -- ou então: -- Você é novo demais para querer!

Foi uma insensatez da parte do soldado, mas ele não entendia bem dessas
coisas, pois era soldado e aprendera a querer apenas o que o comando
queria.

O jovem Klang achava estranho que ``não tivesse vontades'' quando tinha
vontades tão fortes, mas tudo ficou na mesma.

 Um dia, quando já estava grandinho, o pai lhe perguntou:

-- O que você quer ser quando crescer?

O menino não sabia; havia deixado de sentir vontades, uma vez que era
proibido. Na verdade gostaria de fazer algo relacionado a música, mas
não tinha coragem de dizer, por achar que não deixariam. Então
respondeu como um filho obediente: 

-- Não quero nada.

-- Então vai engarrafar vinho! -- disse o pai.

Não se sabe se o pai disse isto porque conhecia um engarrafador de vinho
ou porque o vinho exercia certa atração sobre ele. Fosse como fosse, o
jovem Klang foi trabalhar em uma adega, onde não se saiu mal.

O cheiro de lacre vermelho e de vinho francês eram muito bons lá
embaixo, e havia grandes salas com tetos abobadados, como igrejas.
Quando estava junto ao barril e o vinho tinto jorrava, Klang se
alegrava e começava a cantarolar todo tipo de canções que já ouvira.

O patrão, que vivia no vinho, gostava de canto e alegria, e deixava o
jovem continuar; o canto soava muito bem sob as abóbadas. E quando
alcançava as notas mais agudas de \textit{Im tiefen Keller
sitz` ich hier}, acorriam mais clientes à adega, o que agradava o patrão.

Um belo dia apareceu um caixeiro"-viajante, ex"-cantor do teatro.
Quando escutou Klang, ficou tão encantado que o convidou para uma festa
à noite.

Lá jogaram boliche, comeram lagostins temperados com endro e beberam
ponche; mas, acima de tudo, cantaram. Entre brindes e saideiras, quando
deixaram de lado as formalidades, o caixeiro disse:

-- Por que você não vai para o teatro?

-- O quê? -- disse Klang. -- Você acha que consigo?

-- Não fale assim. Você precisa dizer \textit{eu quero}! Assim
conseguirá.

Foi uma nova lição, pois desde os três anos o jovem Klang não empregara
as palavras \textit{eu} e \textit{quero}.

Na época, não ousava querer ou almejar coisa alguma e pediu para não ser
mais tentado.

Mas o caixeiro"-viajante voltou muitas vezes, na companhia de grandes
cantores. A tentação tornou"-se grande demais, e Klang tomou sua
decisão numa noite em que foi aplaudido por um famoso professor.

Despediu"-se do patrão, e com uma taça na mão agradeceu seu amigo
caixeiro"-viajante, que lhe devolveu a autoconfiança e a força de
vontade; ``à vontade, esta barra de ferro na espinha, que mantêm os
homens eretos para que não caiam de quatro''. E Klang jamais esqueceria
o amigo que o ensinou a acreditar em si mesmo.

E então foi se despedir do pai e da mãe.

-- Eu quero ser cantor! -- disse ele, com uma voz que ecoou por toda a
casa.

O pai procurou o açoite e a mãe chorou; mas de nada adiantou.

-- Não se perca de si mesmo, filho! -- foram as últimas palavras da mãe.
\asterisc

O jovem Klang conseguiu dinheiro para viajar ao estrangeiro, onde
aprendeu a cantar de acordo com as regras e, passados alguns anos,
tornou"-se um grande cantor. Ganhou dinheiro e arranjou seu próprio
empresário.

Nosso amigo Klang estava no auge da carreira e podia dizer tanto \textit{eu
quero} quanto \textit{eu ordeno}. Mas passou a ocupar demasiado espaço com
esse \textit{eu}, e não tolerava outras pessoas a seu redor. Não se negava
nada, e tampouco renegava sua personalidade. Quando estava prestes a
voltar para casa, o empresário convenceu"-o de que Klang não era um
nome conveniente, agora que era um cantor lírico; precisava de um nome
elegante, de preferência estrangeiro, como de costume.

``O Grande'' travou uma luta consigo mesmo, pois mudar de nome não lhe
parecia lá muito bom; daria a impressão de estar renegando pai e mãe, o
que podia pegar mal.

Mas, como era esse o costume, assim o fez. 

Procurou na Bíblia para achar o nome certo, porque é lá que os bons
nomes estão.

E quando achou Jubal, filho de Lamec, que inventou os instrumentos,
decidiu adotá"-lo. Era um bom nome e queria dizer ``trombone'' em
hebraico. E como o empresário era inglês, quis que Klang adotasse o
título de \textit{mister}; e foi o que ele fez. Mr.~Jubal, portanto.

Tudo isso era um tanto inócuo, um costume; mas o estranho mesmo foi que,
com o novo nome, Klang transformou"-se em outra pessoa. Esqueceu sua
vida passada, e Mr.~Jubal se sentia como um inglês nato, falava com
sotaque britânico, deixou crescer suíças e usava colarinho alto; sim,
as roupas xadrez passaram a fazer parte de sua aparência da mesma forma
que a casca faz parte da árvore; tornou"-se austero e cumprimentava
com um olho só; nunca se virava se um conhecido o chamasse na rua e
sempre ficava em pé no meio do bonde.

Ele próprio mal se reconhecia!

Entretanto, voltara para casa e era um grande cantor da Ópera. 
Representava reis, profetas, heróis da liberdade e demônios; quando
tinha um papel para estudar, era tão bom ator que acreditava ser o
personagem que estava representando.

Um dia andava na rua enquanto era um demônio em algum palco, mas também
era Mr.~Jubal.

Então ouviu alguém gritar atrás de si: 

-- Klang! 

Não se virou, é claro, pois um inglês não faria isso; além disso, não se
chamava mais Klang.

Só que a voz chamou por Klang mais uma vez, e na frente dele apareceu o
amigo caixeiro"-viajante, com um olhar curioso, perguntando num tom
tímido e amigável:

-- Klang, não é você?

Mr.~Jubal foi possuído pelo demônio; e, mostrando todos os dentes com a
boca escancarada, como se estivesse emitindo uma nota de peito por uma
cavidade do crânio, berrou: 

-- Não! 

Então o amigo reconheceu"-o e foi embora. Mas era um homem esclarecido
e entendia muita a respeito das coisas da vida, das pessoas e de si
mesmo, e por isso não ficou magoado nem surpreso. 

Porém, Mr.~Jubal achou que sim, e quando escutou essas palavras dentro
da cabeça: ``antes do galo cantar, tu me negarás três vezes'', fez como
Pedro; foi até um vestíbulo e chorou amargamente. Fê"-lo em
pensamento, mas o demônio ria dentro do seu coração.

Depois desse dia, o que mais fazia era rir; do bem e do mal, da tristeza
e da vergonha, de tudo e de todos. 

Graças a um jornal, seu pai e sua mãe sabiam muito bem quem era 
Mr.~Jubal, mas nunca iam à Ópera, pois achavam que era um lugar onde havia
cavalos e malabaristas, e não queriam ver o filho envolvido nisso.

Mr.~Jubal era agora o maior dos cantores, e, de certo modo, deixara de
lado uma boa parte da sua personalidade; mas conservou a vontade.

Foi quando seu dia chegou! Havia uma mocinha no balé que tinha o dom de
enfeitiçar os homens, e Jubal caiu sob seu feitiço. Ficou tão
enfeitiçado que perguntou se ele podia ser dela\ldots{} (Claro que a
pergunta era se \textit{ela} queria ser dele, mas não se pode dizer uma
coisa dessas.)

-- Você pode ser meu -- disse a feiticeira -- se eu tiver\ldots{}

-- Você terá tudo o que quiser! -- respondeu Jubal.

A moça acreditou em sua palavra e os dois se casaram. Primeiro
ensinou"-a a cantar e tocar; depois ela ganhou tudo que queria. Mas
como era uma feiticeira, queria tudo que ele não queria, e assim, aos
poucos, ficou com a vontade dele na palma da mão.

Até que um belo dia a sra.~Jubal tornou"-se uma grande cantora -- tão
grande que, quando o público gritava Jubal, queriam dizer a senhora e
não o senhor.

Jubal queria se reerguer, mas não à custa de sua mulher; e por isso, não
conseguiu.

Começavam a apagá"-lo da memória, a esquecê"-lo.

O brilhante círculo de amigos que Mr.~Jubal reunira em sua casa na época
de solteiro reunia"-se agora na sua casa, em torno da senhora Jubal, a
quem chamavam simplesmente Jubal.

Ninguém olhava para o \textit{mister}, ninguém bebia com ele; se
tentasse falar, ninguém lhe dava ouvidos. Era como se não existisse, e
sua esposa era tratada como se fosse solteira. Mr.~Jubal ficou sozinho,
e sozinho ia aos cafés. 

Uma noite foi a um café em busca de companhia. Estava disposto a
confraternizar com qualquer um; bastava que fosse uma pessoa.

Viu então seu velho amigo caixeiro"-viajante, sozinho e entediado; e
pensou: ``Eis o velho Lundberg, a pessoa que procuro!'' Foi até a mesa
cumprimentá"-lo, mas o rosto do amigo se transformou, e adotou uma
expressão tão terrível que Jubal teve de perguntar:

-- O senhor não é o Lundberg?

-- Sou!

-- E não me reconhece? Jubal?

-- Não!

-- E não reconhece Klang, seu velho amigo?

-- Não, ele morreu há muito tempo!

Foi quando Jubal entendeu que, de certo modo, estava morto; e saiu.

No dia seguinte despediu"-se da Ópera e virou professor titular de
canto.

Viajou para o estrangeiro e permaneceu fora por muitos anos. A tristeza
-- lamentava"-se como quem lamenta um morto -- e a amargura fizeram"-no
envelhecer depressa.

Mas gostou disso, pois desse modo o fim chegaria mais rápido. Só que
Jubal não envelhecia tão depressa quanto gostaria; por isso, arranjou
uma peruca branca com longos cachos. Ela o fazia sentir"-se bem, pois
deixava"-o irreconhecível até mesmo a si próprio.

Andava pela calçada com passos lentos e mãos para trás, pensativo;
parecia estar procurando ou esperando alguém. Se alguém o olhasse nos
olhos, não veria neles brilho algum; se alguém tentasse travar
conhecimento com ele, falava coisas sem sentido; nunca dizia
\textit{eu}, nunca \textit{eu acho que} -- apenas \textit{parece que}.
Perdera sua identidade; descobriu isso um dia enquanto fazia a barba.
Ensaboou o rosto e foi com a navalha para o espelho. Olhou e olhou no
quarto por detrás dele, mas não enxergou o próprio rosto. Compreendeu o
que havia acontecido. E foi tomado por um forte anseio de
reencontrar"-se. Dera o melhor de si à esposa, que ficara com a sua
vontade; assim, decidiu procurá"-la.

Quando voltou a seu país e andou pelas ruas da cidade com a peruca
branca, ninguém o reconheceu; mas um músico que havia estado na Itália
disse na rua: ``É um maestro!''

E num instante Jubal sentiu"-se um grande compositor. Comprou papel
pautado e começou a escrever uma partitura, ou seja, escreveu um monte
de notas longas e curtas na pauta, a maioria para os violinos, claro,
algumas para as madeiras e o resto para os metais. Em seguida, mandou a
partitura ao conservatório, mas ninguém conseguia executá"-la porque
aquilo não era nada, não passava de um amontoado de notas.

Um dia, andando pela rua, encontrou um pintor que estivera em Paris. 

-- Lá vai um modelo! -- disse o pintor.

Jubal escutou, e pensou que era de fato um modelo, porque acreditava em
tudo que diziam dele, visto que não sabia quem nem o que era.
Quando se lembrou da esposa, que havia ficado com a sua personalidade,
decidiu procurá"-la. Foi o que ele fez, mas ela havia se casado de
novo com um barão e viajado para longe.

Cansou"-se de procurar; e como todos os homens cansados, sentiu saudade
de seus dias de infância, de sua mãe. Sabia que ela enviuvara e estava
morando em um casebre no alto de uma montanha, e foi para lá.

-- A senhora não se lembra de mim? -- perguntou.

-- Como o senhor se chama? -- perguntou a mãe.

-- A senhora não sabe o nome do seu filho?

-- Meu filho se chamava Klang, mas o senhor se chama Jubal. Não conheço
ninguém com esse nome.

-- A senhora está me renegando!

-- Como você renegou a si mesmo e à sua mãe.

-- Por que vocês me tiraram a vontade, quando eu era criança?

-- Você entregou sua vontade a uma mulher.

-- Não tive escolha; se não fosse assim, não teria me casado. Mas por que
me diziam que eu não tinha vontades?

-- Era seu pai quem dizia isso, filho querido. Mas ele não entendia bem
dessas coisas. Perdoe"-o, agora; o coitado morreu. Mas em todo caso,
crianças não devem ter vontade; os adultos, sim.

-- Veja como a senhora conseguiu definir bem, mãe! Crianças não devem
ter, e adultos sim.

-- Escute, Gustav -- Disse a mãe. -- Gustav Klang\ldots{}

Eram seus dois nomes; e quando os escutou, voltou a ser quem ele era.
Todos os papéis, reis e demônios, maestros e modelos dissiparam"-se
como fumaça, e ele voltou a ser apenas o filho de sua mãe. Pôs a cabeça
em seu colo e disse:

-- Agora eu quero morrer! Quero morrer!

\chapter[Os Elmos Dourados de \r Alleberg]{Os Elmos Dourados\break de \r Alleberg}


\textsc{Anders nasceu em Falbygden} e, na sua juventude, marchou campo e reino
afora com seu cajado e seu fardo. Um dia ele achou que era melhor
marchar com fuzil e as roupas gastas da coroa, e portanto alistou"-se
em Västgöta"-Dal. De lá mandaram"-no a Estocolmo, para servir na guarda.

O amigo Kask\footnote{ \textit{Kask}: capacete; como os soldados eram
conhecidos. [N.~da E.]}, como agora era chamado, ganhou um dia livre, que
aproveitou para visitar o Skansen\footnote{ Museu de Estocolmo, a céu
aberto. [N.~do T.]}. Porém, quando chegou ao portão, viu que não tinha os
cinqüenta centavos do bilhete de entrada e teve que ficar de fora.
Olhou ao longo da cerca e pensou: ``Pois então vou dar a volta; deve
haver uma escada de acesso. Em último caso, pulo a cerca''.

O sol se punha enquanto ele avançava ao longo do pântano, junto à
montanha. A cerca parecia cada vez mais alta, e no lado de dentro havia
canto e música. Kask andou e andou, rodeando a cerca, mas não viu
nenhuma escada e a cerca desapareceu no interior da floresta de
nogueiras; ficou cansado e sentou numa encosta, descascando nozes. Eis
que aparece um esquilo abanando o rabo:

-- Largue minhas nozes! -- disse o esquilo.

-- Largo se você me mostrar onde fica a escada -- respondeu Kask.

-- É bem ali -- respondeu o esquilo.

Saiu pulando, com o soldado logo atrás, mas Kask perdeu"-o de vista.
Veio então um porco"-espinho farfalhando entre as folhas.

-- Venha comigo -- disse Kask. -- Vamos procurar a escada.

-- Não, obrigado! -- disse o porco"-espinho, e grunhiu uma coisa
incompreensível.

Mas o porco"-espinho seguiu"-o mesmo assim.

Então veio a cobra; era uma cobra muito elegante, mas tinha a língua
presa e fazia contorcionismos.

-- Sssigam"-me -- disse ela. -- Vou mossstrar onde figa a ezgada!

-- Sigo! -- exclamou Kask.

-- Masss comporte"-ssse; tenha güidado para não piszar em mim! Eu gozdo
das bezoasss bem"-educadasss e elegantesss!

-- Soldados não são lá muito elegantes -- disse Kask.  -- Seja como for,
não estou de botas.

-- Pise nela ou será picado, com toda essa elegância! -- disse o
porco"-espinho.

A cobra ergueu a cabeça e se retorceu. 

-- Olhe lá, comporte"-se! -- disse o porco"-espinho aos berros. -- Não
sou assim tão elegante, mas ao menos mostro os meus espinhos!

O porco"-espinho deu cabo da cobra e desapareceu.

Agora Kask estava sozinho na floresta, e se arrependeu de ter desprezado
o bichinho espinhento.

Já era noite, mas o luar atravessava a copa das bétulas; não se ouvia o
menor ruído. O soldado teve a impressão de ver uma grande mão amarela acenando, para
frente e para trás. Aproximando"-se, descobriu que era uma folha de
bordo; os bordos costumam gesticular assim com os dedos, sem que
saibamos o que eles querem dizer. Enquanto Kask observava a folha,
escutou um álamo, que tremia:

-- Ai, que frio! Estou com os pés molhados -- disse o álamo. -- E muito,
muito assustado!

-- Assustado por quê? -- perguntou o soldado.

-- Há um duende dentro da caverna.

O soldado entendeu o que o bordo queria dizer, e logo viu o duende
sentado dentro da caverna, cozinhando mingau.

-- E quem vem a ser o senhor? -- perguntou o duende.

-- Estou servindo no Vässkötta"-Dal; e você?

-- Estou em \r Alleberg -- disse o duende.

-- Mas \r Alleberg fica em Västergylln! -- replicou o soldado.

-- Agora estamos morando aqui -- disse o duende.

-- Mentira! -- respondeu o soldado, pegando no cabo da panela e jogando o
mingau no fogo. -- Vamos ver essa toca -- disse, e entrou na caverna.

Lá dentro estava um gigante em frente a uma grande fogueira. No meio das
chamas havia uma haste de ferro.

-- Bom dia! -- disse o soldado, estendendo a mão.

-- Bom dia para você também! -- respondeu o gigante, estendendo a haste de
ferro em brasa.

Kask apertou o ferro, que chiou:

-- Que aperto de mão caloroso! Qual é o seu nome?

-- Sou o gigante Svensk.\footnote{ \textit{Svensk}: sueco. [N.~do T.]} 

-- Foi um aperto de mão à moda sueca; agora vejo que estou em \r Alleberg!
Os Elmos Dourados ainda estão adormecidos aqui?

-- Silêncio, silêncio! -- disse o gigante, brandindo o atiçador. -- Já que
está servindo em Vässkötta"-Dal, deixarei que os veja se você vencer
meu desafio.

-- Se quiser discutir com um compatriota, tudo bem! Mas livre"-se
primeiro desse atiçador.

-- Pois então, Kask, quero que me conte a história da Suécia enquanto
fumo este cachimbo; assim poderá ver os Elmos Dourados. Toda a história
da Suécia!

-- Darei  um jeito, embora eu não tenha sido nenhum sabichão na escola de
oficiais. Vou precisar de um tempinho para lembrar.

-- E há uma condição: você não pode chamar nenhum rei pelo nome, senão
eles ficam com raiva lá dentro; e \textit{quando eles ficam com raiva},
você sabe\ldots{}

-- Assim fica bem difícil; mas acenda o cachimbo que vou começar. Aqui
está o fogo!

O soldado coçou a cabeça por uns instantes e em seguida começou:

-- Um, dois, três! A Suécia surgiu por volta de 1161: um reino, um rei e
um arcebispo. Já é o bastante?

-- Não -- disse Svensk. -- É muito pouco. Continue.

-- Então assim: no ano de 1359, o povo sueco estava formado, porque
reuniram um parlamento que continuou até 1866, apesar de algumas
interrupções.

-- Já que você é soldado -- disse Svensk --, aproveite e fale um pouco
sobre as guerras.

-- Há só duas guerras que devem ser mencionadas, e as duas culminaram em
períodos de paz. Uma é Brömsebro, em 1645, quando ganhamos Härjedalen,
Jämtland e Gottland; a outra é Roeskilde, em 1658, quando ganhamos
Sk\r ane, Halland, Blekinge e Bohuslän. E assim acaba a história sueca; o
resto não passou de pancadarias.

-- Bom, e as constituições?

-- Tivemos uma ditadura de 1680 até 1718 e então liberdade até 1789,
quando a ditadura voltou. Depois Adlersparre liderou a revolução de
1809 e Hans Järta escreveu a constituição que perdura até hoje. E isso
é tudo; já terminou o cachimbo?

-- Hum! -- disse o gigante -- Foi por pouco. Agora você verá os Elmos
Dourados!

O velho gigante levantou"-se com esforço e adentrou a caverna, seguido
pelo soldado.

-- Silêncio! -- disse o gigante, apontando um cavaleiro com elmo dourado,
dormindo junto a uma porta escavada na rocha. Nesse instante Kask
tropeçou e bateu o ferro do salto de sua bota contra uma pedra, que
soltou faíscas. O cavaleiro acordou de imediato, como se houvesse
dormido na guarda, e gritou:

-- Já está na hora?

-- Ainda não! -- respondeu o gigante.

O cavaleiro de elmo dourado sentou"-se e voltou a dormir. 

O gigante abriu a porta de pedra e o soldado viu"-se diante de uma
grande sala. Uma mesa interminável estendia"-se bem no meio do salão,
e na penumbra avistava"-se uma reunião magnífica de cavaleiros com
elmos dourados, cada um em sua cadeira, com a coroa de ouro no
espaldar. Na ponta da mesa estava sentado um homem cuja cabeça
erguia"-se um pouco acima da dos outros e cuja barba lhe descia até a
cintura, como a de Moisés e a de Isaías; levava na mão um cajado.
Todos pareciam dormir; mas não era nem o sono noturno que nos restaura
as forças, nem o sono que chamam de sono eterno.

-- Comporte"-se bem -- disse o gigante --, e poderá presenciar o encontro
anual.

Apertou uma grande pedra preciosa na parede da caverna e mil chamas se
acenderam. Os Elmos Dourados acordaram:

-- Quem se aproxima? -- perguntou o homem com barba de profeta.

-- Svensk! -- respondeu o gigante.

-- Belo nome! -- disse Gustav Erikson Vasa; sim, era o próprio! -- Quanto tempo
se passou?

-- Desde o nascimento de Cristo, 1903 anos.

-- O tempo passa; mas e vós? Prosperastes? Sois ainda um único país, um
único povo?

-- Somos! Mas depois de Gustavo \textsc{i} o país cresceu, com Jämtland,
Härjedalen e Gottland.

-- Quem conquistou essas regiões?

-- Foi na época da rainha Cristina, mas o conquistador foi o regente.

-- E depois?

-- Depois conquistamos Sk\r ane, Halland, Blekinge e Bohuslän.

-- São os vales! Quem os conquistou?

-- Karl Gustav \textsc{x}.

-- E depois?

--Depois mais nada.

-- Isso é tudo?

Alguém bateu na mesa.

-- Erik, o Sagrado, pede a palavra! -- disse Gustav Vasa.

-- Erik Jedvardson é meu nome, e nunca fui Sagrado. Posso perguntar ao
Svensk que fim levou minha Finlândia?

-- A Finlândia voltou a ser da Rússia por requerimento da aliança de
Anjala e, depois da paz de Fredrikshamn, em 1809, os finlandeses
prestaram homenagem ao tsar.

Gustav Adolf \textsc{ii} pediu a palavra.

-- E os países bálticos? -- perguntou ele.

-- Voltaram a seus donos -- respondeu Svensk.

-- E o imperador ainda existe?

-- Existem dois: um em Berlim e outro em Viena.

-- Dois habsburgos?

-- Não, um habsburgo e um hohenzollare, e chamam isso de Alemanha
unificada, segundo Bismarck.

-- Inacreditável! Os católicos do norte da Alemanha foram convertidos?

-- Não! Os católicos são maioria no parlamento do norte da Alemanha, e o
imperador de Berlim interfere no \textit{quorum} do colégio de cardeais
para poder decidir a eleição do papa.

-- Então o papa ainda existe?

-- Claro que existe, embora um tenha morrido há pouco.

-- E o que o Brandemburguês quer em Roma?

-- Não se sabe; uns dizem que quer se tornar imperador romano"-alemão de
fé protestante.

-- Um imperador sincrético como Johan Georg da Saxônia sonhou! Não quero
escutar mais. Os caminhos da Providência são misteriosos; e nós,
mortais, o que somos? Cinzas e pó!

Karl \textsc{xii} pede a palavra.

-- Svensk, pode me dizer como anda a Polônia?

-- A Polônia não existe mais. Está divida.

-- Dividida? E a Rússia?

-- A Rússia celebrou recentemente a fundação de São Petersburgo e o
prefeito de Estocolmo participou da procissão.

-- Como prisioneiro?

-- Não, como convidado. As nações são hoje relativamente amigas, tanto
que na China, recentemente, um corpo do exército francês ficou de bom
grado sob as ordens de um general alemão.

-- Deve ter saído caro! Hoje se é amigo dos inimigos?

-- Sim, estamos impregnados do espírito cristão e até existe um tribunal
permanente da paz em Haia.

-- Tribunal de quê?

-- Tribunal da paz.

-- Meu tempo passou mesmo! Seja o que Deus quiser.

O rei baixou a viseira do elmo e não falou mais nada.

Karl \textsc{xii} pediu a palavra.

-- Então, Svensk, como andam as finanças da velha Suécia?

-- É difícil dizer, acho que não entendem muito de contabilidade. Mas uma
ou duas coisas são certas: metade do território sueco está penhorada no
estrangeiro por um valor de quase 300 milhões.

-- Ah, Santo Deus!

-- E a dívida dos municípios chega a quase 200 milhões.

-- Duzentos?

-- Entre 1881 e 1885, 146 mil suecos emigraram.

-- Não quero escutar mais nada!

Gustav Vasa bateu o cajado na mesa.

-- O que pude compreender é que o país está mal conduzido. Sois
mandriões, preguiçosos, invejosos, irresponsáveis; lentos na hora da
resolução, e rápidos na decisão de não fazer nada. Mas Svensk, como
anda minha igreja e meus pastores?

-- Os pastores da igreja são agricultores e leiteiros; os bispos recebem
trinta mil coroas de salário e juntam dinheiro, como antes da recessão
de Väster\r as, e no geral são quase todos hereges\ldots{} ou
livre"-pensadores, como se diz hoje em dia. Espera"-se de novo algum
tipo de reforma, o que acabará acontecendo.

-- Como?! Como?! E essa música, esse canto aqui em cima, o que é?

-- É o Skansen, um parque onde colecionaram lembranças patrióticas, como
alguém que, pressentindo o fim, escreve o testamento e junta lembranças
do passado; isso demonstra respeito aos antepassados, mas nada além
disso.

-- O que nós percebemos nesse encontro anual é que o trabalho e os feitos
dos antepassados foram engolidos pelo curso do tempo; enquanto uns
emergem, outros se afundam. E nós ficamos sentados aqui, como sombras
de nós mesmos, e para vós que viveis não devemos ser mais que isso\ldots{}
 Apaguem as luzes!

O gigante Svensk apagou as luzes e saiu, seguido pelo soldado, a quem
pediu que subisse em uma espécie de \mbox{gaiola.}

-- Se falar disso -- disse o gigante --, você será infeliz.

-- Entendido -- respondeu Kask. -- Não vou esquecer! Mas em todo caso,
pense: encheram o bolso com a velha Suécia e depois a espezinharam no
estrangeiro! Se isso for mesmo verdade, é uma tragédia!

O motor fez \textit{pá!} e o elevador subiu, levando o soldado até o
Skansen. E lá estava Kask, ao pôr"-do"-sol, no instante em que o
sinal tocou em H\r asjöstapeln, quando Gustav Vasa adentrou Estocolmo cercado pelos homens de \mbox{Dalarna.}

\chapter[Borboleta Azul encontra a flor de ouro]{Borboleta Azul encontra\break a flor de ouro}


\textsc{Certa vez} um homem rico foi para uma ilha pobre e se apaixonou pelo lugar.
Por quê, não saberia dizer, mas estava apaixonado; talvez a ilha
parecesse com uma memória perdida de infância, ou com um belo sonho.

Comprou a ilha, construiu uma bela casa e plantou lindas árvores,
arbustos e flores dos mais diversos tipos. Na beira do mar, tinha seu
próprio ancoradouro, com barcos brancos e um mastro; carvalhos grandes
como igrejas davam sombra à sua casa e ventos frescos alisavam os
prados verdejantes. Tinha mulher, filhos, criados, animais; tinha tudo,
mas faltava"-lhe uma coisa; uma coisa simples, porém a mais a
importante de todas, que sequer lhe ocorrera: água potável. Mandou
cavar poços e explorar a montanha, mas só encontraram água escura e
salgada. Depois de filtrada, a água tornava"-se clara como o cristal,
mas permanecia salgada. Uma tristeza!

Nessa época apareceu um homem agraciado por Deus, que teve sorte em
todas suas empreitadas e era um dos homens mais famosos do mundo.
Lembramos de como bateu sua bengala de diamante na montanha e, como
Moisés, fez com que da rocha brotasse água. Cavariam a pedra, portanto,
com pontas de diamante, como haviam cavado outras montanhas e
encontrado água em todas elas. Mas tentaram aqui e ali e em outros cem,
mil, dez mil lugares, mas só encontravam água salgada. Com certeza,
essa ilha não era abençoada, e o homem rico aprendeu que não se
consegue tudo com dinheiro; nem mesmo um gole de água
fresca, quando a água não aparece.

O rico ficou pensativo e a vida não lhe sorria mais. Entretanto, o
mestre"-escola da ilha começou a pesquisar em livros antigos e chamou
um velho sábio que andava com uma vara de vedor, mas isso não ajudou em
nada. O pastor, que era ainda mais sábio, reuniu as crianças da escola
um dia e ofereceu uma recompensa para quem encontrasse uma erva chamada
saxífraga"-dourada, que indica o lugar de uma fonte. 

-- Ela tem flores como as da alquemila e folhas como as da
sanícula"-dos"-montes, também conhecida como quebra"-pedra. É como
se tivesse pó de ouro nas folhas superiores. Lembrem"-se disso!

-- Flores como as da alquemila e folhas como as da
sanícula"-dos"-montes -- repetiram as crianças, que saíram floresta e
campo afora em busca da flor de ouro.

Nenhuma criança achou a flor. É verdade que um menininho voltou à casa
com um ramo de trovisco, que tem mesmo um pouco de dourado em cima; mas
era uma erva daninha, não o que procuravam. Assim, as crianças se
cansaram da busca.

Havia uma menininha que ainda não ia à escola; seu pai era dragão da
cavalaria e tinha um pequeno sítio, e era mais pobre do que rico. Seu
único tesouro era a filhinha, conhecida no vilarejo pelo lindo nome de
Borboleta Azul, pois sempre andava com uma blusinha de mangas largas da
cor do céu, que esvoaçavam quando ela se mexia. Aliás, a borboleta azul
é uma pequena borboleta, que no auge do verão aparece nos gramados;
suas asas lembram as pétalas da flor do linho -- uma flor do linho alada
e com antenas, onde ficam os filamentos coloridos.

A Borboleta Azul -- a filha do oficial de cavalaria -- era uma criança
fora do comum; falava pouco e com muita sabedoria. Ninguém sabia de
onde tirava as palavras. Todos gostavam dela, inclusive os animais:
galinhas e bezerros seguiam"-na, e Borboleta Azul não tinha medo de
acariciar nem mesmo o touro. Muitas vezes, passeava sozinha, por um
tempo, depois voltava; e quando lhe perguntavam onde estivera, não
sabia responder. Mas vinha cheia de histórias para contar, sobre as
coisas estranhas que vira, os senhores e senhoras que encontrara e
sobre o que lhe haviam dito. O oficial de cavalaria deixava"-a passear
assim, pois acreditava que algo a protegia.
\asterisc

Um dia, pela manhã, Borboleta Azul saiu a passear. Com passinhos curtos,
atravessou campinas e pastos; e ia cantando para si mesma canções que
ninguém conhecia, porque as inventava na hora. O sol matutino brilhava
com tanta vontade que parecia ter rejuvenescido; o ar parecia
revigorado, como após uma boa noite de sono; e o orvalho evaporava,
refrescando"-lhe o rostinho com essa umidade tão saudável. Quando
entrou na floresta, a menina encontrou um velho vestido de verde.

-- Bom dia, Borboleta Azul! -- disse o velho. -- Sou o jardineiro"-chefe
de Solglänta. Venha, vou lhe mostrar minhas flores!

-- É uma honra grande demais para mim. -- Respondeu a menina.

-- Não; você nunca maltratou uma planta.

E assim foram juntos e chegaram à praia. Lá havia uma bela pontezinha
que dava acesso a uma ilhota, para onde seguiram.

Era um jardim! E lá havia de tudo, grande e pequeno; tudo ordenado como
em um livro.

O jardineiro morava numa casa construída com árvores sempre"-verdes
ainda enraizadas: pinheiros, abetos e zimbros com galhos; o chão era
coberto de arbustos e plantas sempre"-verdes. Musgo e líquen cresciam
nas frinchas do piso para fazer o acabamento; empetros, uvas"-ursinas
e linéias serviam de tábuas. O teto era feito de trepadeiras: vinha
virgem, madressilva, clematite e hera; e as fibras eram tão
entrelaçadas que nem uma gota de chuva passava. No pátio havia
colméias, onde em vez de abelhas moravam borboletas. Quando elas saíam
todas juntas era uma imagem linda!

-- Não gosto de importunar as abelhas -- disse o velho. -- E além do mais,
são feias; parecem grãos de café felpudos, e ainda por cima picam como
víboras.

Os dois saíram do jardim.

-- Agora você poderá estudar com a cartilha da natureza e aprender os
segredos das flores e os sentimentos das ervas. Mas você não pode
perguntar nada, só escutar e responder\ldots{} Preste atenção, menina: aqui
nesta pedra cinzenta cresce uma coisa que parece um papel cinza; é a
primeira coisa que aparece quando a pedra fica úmida. A pedra cria
bolor, e esse bolor se chama líquen. Aqui tem dois tipos: esse parece
um chifre de rena, e é chamado mesmo líquen de rena. É o alimento
mais importante da rena.  O outro se chama líquen da Islândia e
parece\ldots{} parece o quê?

-- Parece um pulmão, porque é isso que o livro de ciências diz.

-- Sim, quando visto na lupa, parece com os alvéolos pulmonares, e por
isso as pessoas aprenderam a usá"-lo para as doenças do peito. Porém,
quando o líquen na pedra junta húmus, nasce o musgo. Os musgos têm um
tipo simples de flor e produzem sementes; parecem cristais de geada,
mas você verá que eles lembram a urze, as coníferas e todo o resto,
porque as plantas são todas da mesma família. Este musgo de parede
lembra um abeto, mas tem um pericarpo como o da papoula, só que mais
simples. No musgo logo cresce a urze, e se você a observar com uma lupa
mais potente, verá que, aumentada, ela vira outra planta, cujo nome
em latim é \textit{epilobium}, ou rododendro; mas a urze é
igualzinha ao olmeiro, que, por sua vez, nada mais é do que uma urtiga
grande.  O tapete de húmus está pronto, e nessa terra nasce tudo. O
homem apropriou"-se de certas plantas, mas a própria natureza indicou
quais deveriam ser usadas e como. O mesmo se dá com os enfeites e cores
que as flores receberam para mostrar aos insetos onde está o mel. Você
mesma já viu a espiga de centeio que os padeiros penduram do lado de
fora da padaria, como um letreiro. Se você observar o linho, o mais
proveitoso dos vegetais, verá que ele próprio ensinou as pessoas a
fiar. Olhando dentro da flor se vêem os filamentos coloridos enrolados
em torno do estilete como a linha em torno do carretel. Para se
exprimir com mais clareza, a natureza deixou uma erva parasita, chamada
enredadeira, enroscar"-se em torno da planta inteira, por todos
os lados, como a lançadeira no tear. O mais incrível é que não foi uma
pessoa quem primeiro se deu conta de que o linho podia ser fiado -- foi
uma borboleta fiandeira, que com as folhas e sua própria seda faz
roupas de cama e véus para o berço dos filhotes. Mas depois que o linho
começou a ser cultivado, ela foi esperta e logo se adequou aos novos
tempos, para que seus filhos estivessem aptos a voar antes da colheita.
E as ervas medicinais, você nem imagina! Veja a grande papoula,
vermelha como o fogo, como a febre e a loucura! No fundo dessa flor há
uma cruz negra; este é o rótulo com que os farmacêuticos identificam os
venenos. No meio da cruz tem um vaso romano com sulcos. Basta raspar
esses sulcos para que escorra um remédio que pode matar se mal usado;
mas, se bem usado, invoca o irmão bondoso da morte, o sono. Sim, a
natureza é sábia e generosa. E agora veremos a saxífraga"-dourada\ldots{}

Aqui o velho fez uma pausa, para ver se a menina estava curiosa. Mas ela
não estava.

-- Vamos encontrar a flor de ouro!

Mais uma pausa! Mas não; apesar da pouca idade, Borboleta Azul sabia
calar"-se.

-- Vamos ver a flor de ouro que tem flores como as da alquemila e folhas
como as da quebra"-pedra. Estes são os sinais que a distinguem, e ela
indica o lugar de uma fonte. A alquemila traz orvalho e água nas
folhas, sendo ela própria uma fonte pequenina e clara; e a
quebra"-pedra vai perfurando a rocha. O que a flor de ouro mostra a
quem quiser ver é que, sem montanha, não há fonte, não importa o quão
longe a montanha estiver. Essa flor cresce aqui na ilha; e vou lhe
dizer onde, porque você é uma menina bondosa. Das suas mãozinhas a água
virá para refrescar a alma seca desse homem rico, e graças a você essa
ilha será abençoada. Vá em paz, minha filha. Quando entrar no bosque de
nogueiras, você verá uma tília prateada à esquerda; debaixo dela está
uma cobra"-de"-vidro, que não é perigosa. A cobra vai mostrar"-lhe o
caminho até a flor de ouro. E antes de ir, dê um beijo aqui neste
velho\ldots{} mas só se quiseres. 

Borboleta Azul aproximou os lábios do velho e o beijou. E num instante
seu rosto transformou"-se; ele ainda estava lá, em frente à menina,
mas cinqüenta anos mais moço.

-- Fui beijado por uma criança e ganhei a juventude! -- disse o
jardineiro. -- Você não precisa agradecer por nada. Adeus!

A menina foi até o bosque de nogueiras, onde a tília prateada tocava e
os zangões cantavam em meio às flores. A cobra"-de"-vidro estava
mesmo ali, mas parecia um pouco enfezada.

-- Aí está a Borboleta Azul que vai colher a flor de ouro -- disse a
serpente. -- Mas para encontrá"-la, existem três condições: não
fofocar, não mentir nem bisbilhotar. Siga em frente e você vai
encontrar a flor. 

Ela foi em frente e encontrou uma senhora.

-- Bom dia -- disse a senhora. -- Você por acaso não esteve em Solglänta,
na casa do jardineiro?

-- Bom dia, senhora -- respondeu a menina, seguindo adiante.

-- Pelo menos não é fofoqueira! -- disse a senhora.

Em seguida, encontrou um cigano.

-- Aonde vai? -- perguntou.

-- Vou em frente -- respondeu a menina.

-- Então não mente! -- disse o cigano.

Depois ela encontrou uma carroça de leite, mas não conseguia entender
por que o cavalo estava sentado dentro da carroça e o cocheiro estava
preso às rédeas, puxando a carga.

-- Agora vou aprontar! -- disse o cocheiro, e disparou; tanto que o cavalo
caiu no fosso\ldots{} -- Vou regar o centeio -- voltou a dizer ele, abrindo um
tarro de leite para regar a lavoura.

Borboleta Azul achou muito estranho, mas não olhou e seguiu adiante.

-- Também não é bisbilhoteira -- disse o cocheiro.

A menina estava agora ao pé da montanha; os raios de sol passavam por
entre as copas das aveleiras e caíam sobre uma flor, que brilhava como
o ouro mais puro. Era a saxífraga"-dourada! A menina viu como um veio
d’água saía da flor e escorria montanha abaixo, em direção ao terreno
do homem rico. Ela se ajoelhou e colheu três flores, escondeu"-as no
avental e voltou a casa, para os braços do pai.

O oficial pôs o capacete, o sabre e o uniforme; pai e filha foram até o
pastor. Então os três seguiram até a casa do homem rico.

-- Borboleta Azul encontrou a flor de ouro! -- disse o pastor, entrando
pela porta da sala. -- Agora vamos todos ficar ricos nesta cidade! Vamos
transformá"-la em um balneário!

E fizeram o balneário; vieram barcos a vapor e negociantes; o vilarejo
ganhou hospedaria e correio, hospital e farmácia. No verão, o ouro
jorrava na cidade; eis a lenda da saxígrafa"-dourada, capaz de trazer o ouro.
