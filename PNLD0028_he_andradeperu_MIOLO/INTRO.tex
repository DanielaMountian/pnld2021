\chapterspecial{Vida e obra de Mário de Andrade}{}{Rodrigo Jorge Ribeiro Neves}


\setlength{\epigraphwidth}{.45\textwidth}
\epigraph{Na rua Aurora eu nasci\\ Na aurora de minha vida\\ E numa aurora
cresci.\\[5pt] No largo do Paiçandu\\ Sonhei, foi luta renhida,\\ Fiquei pobre
e me vi nu.\\[5pt] Nesta Rua Lopes Chaves\\ Envelheço, e envergonhado.\\ Nem
sei quem foi Lopes Chaves.\\[5pt] Mamãe! me dá essa lua,\\ Ser esquecido e
ignorado\\ Como esses nomes de rua.}

\section{Sobre o autor}

\noindent{}Publicados em \emph{Lira
paulistana}, em 1945, esses versos traçam uma espécie de síntese lírica
biocartográfica da vida de Mário de Andrade, embora não seja tarefa
simples descrever em poucas linhas um indivíduo tão múltiplo e diverso
como ele, ``trezentos, trezentos e cinquenta'', como atestara em outro
de seus poemas mais famosos.

Mário Raul Morais de Andrade teve a sua aurora no dia 9 de outubro de
1893, na cidade de São Paulo, no número 320 da rua que carrega
simbolicamente a projeção e a permanência de sua produção intelectual e
artística. Desde a infância, demonstrou talento para a música,
destacando"-se como exímio pianista, o que o levou a ser matriculado no
Conservatório Dramático e Musical de São Paulo ao completar dezoito
anos. Como autodidata, dedicou"-se também ao estudo da literatura e de
outras artes, mas foi por meio da poesia que se lançou como escritor,
atividade à qual se dedicou intensamente até o fim da vida.

Seu primeiro livro foi \emph{Há uma gota de sangue em cada poema}
(1917), publicado sob o pseudônimo de Mário Sobral. Nos textos de
estreia, ainda estão presentes as influências das tendências estéticas
da virada do século \textsc{xix} ao \textsc{xx}, como o simbolismo e o parnasianismo,
embora estejam esboçadas as questões que permeariam toda a sua obra
posterior. Poucos anos depois, Mário de Andrade se engajou no
modernismo, movimento que, em sua fase inicial, se oporia radicalmente a
essas estéticas anteriores, influenciado pelas vanguardas artísticas
europeias.

Em 1922, Mário publicou o livro de poemas \emph{Pauliceia desvairada},
sua obra"-manifesto, no mesmo ano em que trabalhava na organização de um
dos eventos mais importantes da vida intelectual e cultural brasileira
no século \textsc{xx}, a Semana de Arte Moderna, ocorrida entre os dias 11 e 18
de fevereiro de 1922, no Theatro Municipal de São Paulo, e que contou
ainda com a participação de artistas como Oswald de Andrade, Anita
Malfatti e Heitor Villa"-Lobos. Com o ensaio \emph{A escrava que não é
Isaura} (1925), Mário lançou mais um manifesto, mas, desta vez, por meio
de uma reflexão mais séria sobre a poesia moderna no Brasil.

No mesmo decênio, Mário de Andrade começou a se embrenhar pelos caminhos
da prosa de ficção. Como contista, publicou o livro \emph{Primeiro
andar} (1926), mas foram seus romances \emph{Amar, verbo intransitivo}
(1927) e \emph{Macunaíma, o herói sem nenhum caráter} (1928) que o
destacaram como prosador. Também se dedicou à crítica de artes
plásticas, de música e de literatura, além de ter se dedicado a
importantes estudos sobre a cultura popular nacional.

Como missivista, Mário de Andrade foi um dos mais prolíficos
intelectuais de seu tempo. A correspondência do escritor paulista é
volumosa e diversa tanto em sua dimensão numérica quanto no que diz
respeito aos temas e interlocutores envolvidos no diálogo epistolar, de
escritores a pintores, de folcloristas a políticos. As cartas de Mário
não são apenas os bastidores da intimidade dos seus correspondentes, mas
também espaço de memória da formação e transformação da vida cultural,
intelectual e política do Brasil no século \textsc{xx}.

Em 22 de fevereiro de 1945, um infarto no miocárdio abreviou sua vida.
No entanto, a importância de sua obra, já reconhecida por seus
contemporâneos, se ampliou ainda mais com o passar dos anos, sendo hoje
objeto de estudos de variadas perspectivas críticas e suscitando um
interesse cada vez maior dos leitores.

\section{Sobre a obra}

Este livro reúne alguns contos de Mário de Andrade, apresentados na
mesma ordem em que estão dispostos nos livros dos quais foram extraídos.
Os textos são, respectivamente, de \emph{Primeiro andar} (1926),
\emph{Os contos de Belazarte} (1934) e \emph{Contos novos} (1947). Os
títulos citados trazem contos que expõem, mobilizam e discutem aspectos
caros à concepção artística de Mário de Andrade, apesar de ele não ser
tão referenciado por essa faceta, uma das várias que assumiu. Não
adotamos, como critério exclusivo, a escolha do que se costuma julgar
como ``melhores'', mas o que consideramos representativo dentro da
proposta de cada um dos livros e dos conceitos mais amplos de Mário
sobre arte, cultura e sociedade.

\emph{Primeiro andar} faz parte dos seus escritos da juventude. O livro
foi publicado pela Casa Editora Antonio Tisi, em 1926, e depois pela
Editora Piratininga em 1932, mas sem nenhuma alteração nos textos, não
constituindo propriamente uma segunda edição, que viria apenas em 1943.
Naquele ano, no planejamento de suas \emph{Obras completas} para a
Livraria Martins Editora, publicadas apenas em 1960, Mário incluiu
\emph{Primeiro andar} em uma reunião intitulada \emph{Obra imatura}, em
que também constam \emph{Há uma gota de sangue em cada poema} e \emph{A
escrava que não é Isaura}. Os contos selecionados para esta coletânea
são ``Conto de Natal'' (1914 {[}1943{]}),\footnote{Entre parênteses, o
  primeiro ano indica a data de escrita do conto; entre colchetes, em
  ordem cronológica, as datas das modificações feitas pelo escritor.}
``História com data'' (1921 {[}1943{]}), ``Galo que não cantou'' (1918
{[}1943{]}) e ``Briga das pastoras'' (1939 {[}1943{]}). Neles, o
escritor se lança em experimentações de procedimentos e limites da
linguagem e da estrutura do próprio gênero, percorrendo vários temas,
como quem ainda está investigando as possibilidades da narrativa moderna
e buscando se afastar dos modelos tradicionais.

As histórias de \emph{Os contos de Belazarte} nasceram a partir das
\emph{Crônicas de Malazarte}, publicadas entre outubro de 1923 e julho
1924 no periódico \emph{América Brasileira}, em que o narrador e
personagem Belazarte, espécie de \emph{alter ego} do escritor, surge
pela primeira vez. Em sua primeira edição, de 1934, o livro saiu com o
título \emph{Belazarte: contos}, pela Editora Piratininga. O título
atual observa o plano deixado por Mário de Andrade para suas \emph{Obras
completas}, em que \emph{Os contos de Belazarte} figurariam como o
quinto volume da coleção. Para esta edição, coletamos as seguintes
histórias: ``O besouro e a rosa'' (1923 {[}1925,1943{]}), ``Jaburu
malandro'' (1924 {[}1934,1943--1944{]}), ``Caim, Caim e o resto'' (1924
{[}1934,1943--1944{]}) e ``Piá não sofre? Sofre'' (1926
{[}1934,1943--1944{]}). ``Belazarte me contou'' é o recurso narrativo com
que o escritor inicia cada uma das histórias do livro. Mário de Andrade
aprofunda as experiências com a linguagem, incorporando a oralidade na
narração e na fala das personagens.

Por fim, temos \emph{Contos novos}, publicação póstuma. Quando retornou
a São Paulo, em 1941, depois de uma temporada morando no Rio de Janeiro,
Mário começou a trabalhar em contos produzidos entre 1920 e 1930, cuja
organização recebeu os títulos provisórios de \emph{Contos piores} e
\emph{Contos redivivos.} Infelizmente, o contista faleceu antes de
arrematar este e outros trabalhos. Assim, \emph{Contos novos} teve sua
primeira edição publicada como o volume 17 de suas \emph{Obras
completas}, pela Livraria Martins Editora, a partir dos planos deixados
pelo escritor. Reunimos os seguintes contos nesta edição: ``Primeiro de
maio'', ``O poço'' (S. Paulo, 26--\textsc{xii}--42 {[}Terceira versão{]}), ``O peru
de Natal'' (Versão definitiva, agosto, 1938--1942) e ``Nelson'' (São
Paulo, 12--\textsc{iv}--43 --- 15--\textsc{iv}--43; versão nova do final, 21--\textsc{iv}43). %junto mesmo IV43?
Neste
último livro de narrativas curtas, temos um contista mais amadurecido em
relação aos pressupostos estéticos e narrativos do gênero, com a
linguagem revitalizada, a exploração da dimensão psicológica e a
presença de uma dramaticidade própria na constituição das histórias e no
conflito das personagens.

Para o estabelecimento do texto, cotejamos as primeiras edições dos
livros selecionados para esta coletânea com os respectivos volumes das
\emph{Obras completas}, \emph{Obra imatura}, editora Agir, de 2009,
\emph{Primeiro andar}, Com"-Arte/Edusp, de 2018, \emph{Os contos de
Belazarte}, editora Agir, de 2008, e \emph{Contos novos}, Nova
Fronteira, 2011. A grafia foi atualizada conforme o Novo Acordo
Ortográfico da Língua Portuguesa, mas mantendo as idiossincrasias da
escrita de Mário de Andrade, elemento indissociável de seu projeto
literário.

\section{Sobre o gênero}