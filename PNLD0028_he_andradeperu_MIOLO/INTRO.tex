\chapter{Vida e obra de Mário de Andrade}

\begin{flushright}
\textsc{rodrigo jorge ribeiro neves}
\end{flushright}


\setlength{\epigraphwidth}{.45\textwidth}
\epigraph{Na rua Aurora eu nasci\\ Na aurora de minha vida\\ E numa aurora
cresci.\\[5pt] No largo do Paiçandu\\ Sonhei, foi luta renhida,\\ Fiquei pobre
e me vi nu.\\[5pt] Nesta Rua Lopes Chaves\\ Envelheço, e envergonhado.\\ Nem
sei quem foi Lopes Chaves.\\[5pt] Mamãe! me dá essa lua,\\ Ser esquecido e
ignorado\\ Como esses nomes de rua.}

\section{Sobre o autor}

\noindent{}Publicados em \emph{Lira
paulistana}, em 1945, esses versos traçam uma espécie de síntese lírica
biocartográfica da vida de Mário de Andrade, embora não seja tarefa
simples descrever em poucas linhas um indivíduo tão múltiplo e diverso
como ele, ``trezentos, trezentos e cinquenta'', como atestara em outro
de seus poemas mais famosos.

Mário Raul Morais de Andrade teve a sua aurora no dia 9 de outubro de
1893, na cidade de São Paulo, no número 320 da rua que carrega
simbolicamente a projeção e a permanência de sua produção intelectual e
artística. Desde a infância, demonstrou talento para a música,
destacando"-se como exímio pianista, o que o levou a ser matriculado no
Conservatório Dramático e Musical de São Paulo ao completar dezoito
anos. Como autodidata, dedicou"-se também ao estudo da literatura e de
outras artes, mas foi por meio da poesia que se lançou como escritor,
atividade à qual se dedicou intensamente até o fim da vida.

Seu primeiro livro foi \emph{Há uma gota de sangue em cada poema}
(1917), publicado sob o pseudônimo de Mário Sobral. Nos textos de
estreia, ainda estão presentes as influências das tendências estéticas
da virada do século \textsc{xix} ao \textsc{xx}, como o simbolismo e o parnasianismo,
embora estejam esboçadas as questões que permeariam toda a sua obra
posterior. Poucos anos depois, Mário de Andrade se engajou no
modernismo, movimento que, em sua fase inicial, se oporia radicalmente a
essas estéticas anteriores, influenciado pelas vanguardas artísticas
europeias.

Em 1922, Mário publicou o livro de poemas \emph{Pauliceia desvairada},
sua obra"-manifesto, no mesmo ano em que trabalhava na organização de um
dos eventos mais importantes da vida intelectual e cultural brasileira
no século \textsc{xx}, a Semana de Arte Moderna, ocorrida entre os dias 11 e 18
de fevereiro de 1922, no Theatro Municipal de São Paulo, e que contou
ainda com a participação de artistas como Oswald de Andrade, Anita
Malfatti e Heitor Villa"-Lobos. Com o ensaio \emph{A escrava que não é
Isaura} (1925), Mário lançou mais um manifesto, mas, desta vez, por meio
de uma reflexão mais séria sobre a poesia moderna no Brasil.

No mesmo decênio, Mário de Andrade começou a se embrenhar pelos caminhos
da prosa de ficção. Como contista, publicou o livro \emph{Primeiro
andar} (1926), mas foram seus romances \emph{Amar, verbo intransitivo}
(1927) e \emph{Macunaíma, o herói sem nenhum caráter} (1928) que o
destacaram como prosador. Também se dedicou à crítica de artes
plásticas, de música e de literatura, além de ter se dedicado a
importantes estudos sobre a cultura popular nacional.

Como missivista, Mário de Andrade foi um dos mais prolíficos
intelectuais de seu tempo. A correspondência do escritor paulista é
volumosa e diversa tanto em sua dimensão numérica quanto no que diz
respeito aos temas e interlocutores envolvidos no diálogo epistolar, de
escritores a pintores, de folcloristas a políticos. As cartas de Mário
não são apenas os bastidores da intimidade dos seus correspondentes, mas
também espaço de memória da formação e transformação da vida cultural,
intelectual e política do Brasil no século \textsc{xx}.

Em 22 de fevereiro de 1945, um infarto no miocárdio abreviou sua vida.
No entanto, a importância de sua obra, já reconhecida por seus
contemporâneos, se ampliou ainda mais com o passar dos anos, sendo hoje
objeto de estudos de variadas perspectivas críticas e suscitando um
interesse cada vez maior dos leitores.

\section{Sobre a obra}

Este livro reúne alguns contos de Mário de Andrade, apresentados na
mesma ordem em que estão dispostos nos livros dos quais foram extraídos.
Os textos são, respectivamente, de \emph{Primeiro andar} (1926),
\emph{Os contos de Belazarte} (1934) e \emph{Contos novos} (1947). Os
títulos citados trazem contos que expõem, mobilizam e discutem aspectos
caros à concepção artística de Mário de Andrade, apesar de ele não ser
tão referenciado por essa faceta, uma das várias que assumiu. Não
adotamos, como critério exclusivo, a escolha do que se costuma julgar
como ``melhores'', mas o que consideramos representativo dentro da
proposta de cada um dos livros e dos conceitos mais amplos de Mário
sobre arte, cultura e sociedade.

\emph{Primeiro andar} faz parte dos seus escritos da juventude. O livro
foi publicado pela Casa Editora Antonio Tisi, em 1926, e depois pela
Editora Piratininga em 1932, mas sem nenhuma alteração nos textos, não
constituindo propriamente uma segunda edição, que viria apenas em 1943.
Naquele ano, no planejamento de suas \emph{Obras completas} para a
Livraria Martins Editora, publicadas apenas em 1960, Mário incluiu
\emph{Primeiro andar} em uma reunião intitulada \emph{Obra imatura}, em
que também constam \emph{Há uma gota de sangue em cada poema} e \emph{A
escrava que não é Isaura}. Os contos selecionados para esta coletânea
são ``Conto de Natal'' (1914 {[}1943{]}),\footnote{Entre parênteses, o
  primeiro ano indica a data de escrita do conto; entre colchetes, em
  ordem cronológica, as datas das modificações feitas pelo escritor.}
``História com data'' (1921 {[}1943{]}), ``Galo que não cantou'' (1918
{[}1943{]}) e ``Briga das pastoras'' (1939 {[}1943{]}). Neles, o
escritor se lança em experimentações de procedimentos e limites da
linguagem e da estrutura do próprio gênero, percorrendo vários temas,
como quem ainda está investigando as possibilidades da narrativa moderna
e buscando se afastar dos modelos tradicionais.
Alguns acabaram servindo, inclusive, como ponto de partida
para outras obras, como ``História com data''. Além da intertextualidade
com um dos clássicos do conto nacional, o livro \emph{Histórias sem
data}, de Machado de Assis, ele também traz questões que são
aprofundadas em \emph{Macunaíma}, como a ideia de que o fragmentário e o
desencaixado pode constituir a identidade.

As histórias de \emph{Os contos de Belazarte} nasceram a partir das
\emph{Crônicas de Malazarte}, publicadas entre outubro de 1923 e julho
1924 no periódico \emph{América Brasileira}, em que o narrador e
personagem Belazarte, espécie de \emph{alter ego} do escritor, surge
pela primeira vez. Em sua primeira edição, de 1934, o livro saiu com o
título \emph{Belazarte: contos}, pela Editora Piratininga. O título
atual observa o plano deixado por Mário de Andrade para suas \emph{Obras
completas}, em que \emph{Os contos de Belazarte} figurariam como o
quinto volume da coleção. Para esta edição, coletamos as seguintes
histórias: ``O besouro e a rosa'' (1923 {[}1925,1943{]}), ``Jaburu
malandro'' (1924 {[}1934,1943--1944{]}), ``Caim, Caim e o resto'' (1924
{[}1934,1943--1944{]}) e ``Piá não sofre? Sofre'' (1926
{[}1934,1943--1944{]}). ``Belazarte me contou'' é o recurso narrativo com
que o escritor inicia cada uma das histórias do livro. Mário de Andrade
aprofunda as experiências com a linguagem, incorporando a oralidade na
narração e na fala das personagens e retratando os conflitos de classe que se
aprofundam com a modernização do país.

Por fim, temos \emph{Contos novos}, publicação póstuma. Quando retornou
a São Paulo, em 1941, depois de uma temporada morando no Rio de Janeiro,
Mário começou a trabalhar em contos produzidos entre 1920 e 1930, cuja
organização recebeu os títulos provisórios de \emph{Contos piores} e
\emph{Contos redivivos.} Infelizmente, o contista faleceu antes de
arrematar este e outros trabalhos. Assim, \emph{Contos novos} teve sua
primeira edição publicada como o volume 17 de suas \emph{Obras
completas}, pela Livraria Martins Editora, a partir dos planos deixados
pelo escritor. Reunimos os seguintes contos nesta edição: ``Primeiro de
maio'', ``O poço'' (S. Paulo, 26--\textsc{xii}--42 {[}Terceira versão{]}), ``O peru
de Natal'' (Versão definitiva, agosto, 1938--1942) e ``Nelson'' (São
Paulo, 12--\textsc{iv}--43 --- 15--\textsc{iv}--43; versão nova do final, 21--\textsc{iv}43). %junto mesmo IV43?
Neste
último livro de narrativas curtas, temos um contista mais amadurecido em
relação aos pressupostos estéticos e narrativos do gênero, com a
linguagem revitalizada, a exploração da dimensão psicológica e a
presença de uma dramaticidade própria na constituição das histórias e no
conflito das personagens.

Para o estabelecimento do texto, cotejamos as primeiras edições dos
livros selecionados para esta coletânea com os respectivos volumes das
\emph{Obras completas}, \emph{Obra imatura}, editora Agir, de 2009,
\emph{Primeiro andar}, Com"-Arte/Edusp, de 2018, \emph{Os contos de
Belazarte}, editora Agir, de 2008, e \emph{Contos novos}, Nova
Fronteira, 2011. A grafia foi atualizada conforme o Novo Acordo
Ortográfico da Língua Portuguesa, mas mantendo as idiossincrasias da
escrita de Mário de Andrade, elemento indissociável de seu projeto
literário.

\section{Sobre o gênero}

\begin{quote}
O conto é, do ângulo dramático, unívoco, univalente. [\ldots]
Etimológicamente preso à linguagem teatral,
``drama'' significava ``ação''. E com o tempo passou a designar
toda peça destinada à representação. Na época romântica, dado o
princípio da fusão de gêneros, entendia-se por drama o misto de
tragédia e comédia. Transferido para a prosa de ficção, o termo
``drama'' entrou a significar ``conflito'', ``atrito''. Nesse caso,
``ação'' ``cortflito'' se tonaram equivalentes, uma vez que toda
ação pressupõe conflito, e este, promove a ação, ou por meio dela
se manifesta; em suma, ambos se implicam mutuamente.

O conto é, pois, uma narrativa unívoca, univalente: constitui
uma \textit{unidade dramática}, uma \textit{célula dramática}, visto gravitar ao
redor de um só conflito, um só drama, uma só ação. Caracteriza-se,
assim, por conter \textit{unidade de ação}, tomada esta como a sequência de atos praticados pelos protagonistas, ou de acontecimentos de
que participam. A ação pode ser externa, quando as personagens se
deslocam no espaço e no tempo, e interna, quando o conflito se
localiza em sua mente.\footnote{\textsc{moisés}, Massaud. \textit{A criação literária}. São Paulo: Cultrix, 2006, p.\,40.}
\end{quote}

Partindo da definição de Massaud Moisés sobre o conto, evidencia"-se a principal característica desse gênero literário: a unidade de conflito, condensada em ações que se completam em um único enredo. Ao conto, ainda seguindo Moisés, aborrecem as divagações e os excessos, pois há uma concentração de efeitos e pormenores essenciais, em sua brevidade, para o bom funcionamento do conto.
Cada construção, cada palavra nesse gênero tem sua razão de existir, pois integra a economia global da narrativa.

Apesar da brevidade de sua forma, o conto desdobra"-se em muitas direções e implicações, e o faz a partir de elementos restritos: a unidade dramática, como já mencionada, assim como a presença de poucas personagens e a limitação espacial e temporal. Um ótimo exemplo é o conto ``Missa do galo'', de Machado de Assis, em que o narrador, Nogueira, conta a sua experiência de uma única noite na companhia de sua hospedeira, D.\,Conceição. Apesar de unidade temporal (a noite que antecede a Missa do galo), espacial (uma sala na casa de D.\,Conceição) e da redução dramática, basicamente, à interação entre duas personagens, Conceição e Nogueira, esse conto desdobra"-se em muitas direções. A companhia de Conceição desperta a sexualidade de Nogueira, e seu impacto é tão profundo que o narrador relembra aos leitores esse acontecimento de sua juventude. As intenções da anfitriã, narradas e, logo, distorcidas pela memória de Nogueira, também são ambíguas, levantando as mais diversas questões e interpretações.

Como reflete o escritor argentino Julio Cortázar, o conto consegue, de forma muito concisa, despertar ``uma realidade infinitamente mais vasta que a do seu mero argumento'', influindo ``em nós com uma força que nos faria suspeitar da modéstia do seu conteúdo aparente, da brevidade do seu texto''.\footnote{\textsc{CORTÁZAR}, Julio. \textit{Valise de cronópio}. São Paulo: Editora Perspectiva, 2008, p.\,155.}

Apesar da aparente banalidade do argumento, o conto abre essa possibilidade de desenvolver o tema em profundidade, em contraposição à aparente concisão narrativa. Realiza plenamente, assim, o que Cortázar define como o gênero do conto:

\begin{quote}
Um escritor argentino, muito amigo do boxe, dizia"-me que nesse combate que se travra entre um texto apaixonante e o leitor, o romance ganha sempre por pontos, enquanto que o conto deve ganhar por \textit{knock"-out}. É verdade, na medida em que o romance acumula progressivamente seus efeitos no leitor, enquanto que um bom conto é incisivo, mordente, sem trégua desde as primeiras frases. Não se entenda isto demasiado literalmente, porque o bom contista é um boxeador muito astuto, e muitos dos seus golpes iniciais podem parecer pouco eficazes quando, na realidade, estão minando já as resistências mais sólidas do adversário.
Tomem os senhores qualquer grande conto que seja de sua preferência, e analisem a primeira página. Surpreender"-me"-ia se encontrassem elementos gratuitos, meramente decorativos. O contista sabe que não pode proceder acumulativamente, que não tem o tempo por aliado; seu único recurso é trabalhar em profundidade, verticalmente, seja para cima ou para baixo do espaço literário.\footnote{Ibid., p.\,152.}
\end{quote}

No caso específico de Mário de Andrade, além de executar à perfeição esses preceitos do conto, acrescentando"-lhes traços estilísticos típicos do modernismo, 
suas narrativas curtas revestem"-se de ainda mais importância pela
valorização que promovem da cultura popular e na
exposição das contradições da modernização. Eles apresentam um retrato
multifacetado do Brasil e de sua formação sociocultural, sem deixar de
destacar seus problemas. Os contos são atuais pela sua linguagem e pela
maneira como expõem a diversidade e as desigualdades que formam o país.
A importância da cultura popular para a compreensão da nossa identidade
também merece destaque.

Além disso, ao valorizar o ritmo da oralidade na linguagem escrita e a cultura
popular, os contos de Mário de Andrade dialogam com a produção dos
principais escritores do século \textsc{xx}, como Jorge Amado e Guimarães Rosa. O
tema dos conflitos de classe e dos contrastes regionais do país pode
encontrar ressonância em autores contemporâneos como Marcelino Freire,
Maria Valéria Rezende e Luiz Ruffato.

Como analisa o crítico Anatol Rosenfeld, a busca de Mário de Andrade por essa oralidade na escrita era, igualmente, uma busca pela sua própria identidade através da procura de uma identidade nacional. Para isso, o escritor foge da petrificação da língua, a forma fixa e estratificada no espírito coletivo que é apenas aparência, disfarce dessa identidade que o escritor intenta explorar pela linguagem.

Em seus \textit{Contos novos}, por exemplo, pode"-se observar, na interpretação de Rosenfeld, uma unidade temática que perpassa todas as narrativas do volume: a cisão da subjetividade, o homem disfarçado, desdobrado entre o ser e a aparência.
Para recriar essa separação contraditória do homem, Mário utiliza"-se largamente da recriação da própria língua:

\begin{quote}
O próprio abrasileiramento da língua é parte dessa reconstrução, na medida em que representa linguisticamente a busca do autêntico; mas na medida em que é uma estilização cuidadosamente elaborada, partilha também os fingimentos, tornando"-se a máscara do genuíno. A intenção da sinceridade implica sempre a ``segunda intenção''.\footnote{\textsc{rosenfeld}, Anatol. ``Mário e o cabotinismo''. In: \textit{Texto/Contexto I}. São Paulo: Perspectiva, 1996, p.\,194.}
\end{quote}

É o diálogo entre essas duas intenções, ou duas sinceridades, que Rosenfeld chama de o ``cabotinismo'' de Mário de Andrade: uma obra que, de um lado, quer transmitir a ``paisagem profunda'' do autor, os motivos que impelem o artista à criação, e, de outro, trabalha essa verdade primeira no nível artesanal e da elaboração de suas possibilidades de comunicação. É a partir dsse espírito coletivo, exterior,
que o escritor molda sua interioridade. Para Mário de Andrade, no entanto, essas ``máscaras'' com que se enxergava a realidade interior não a afetava, ao contrário, fazia parte da totalidade da subjetividade e da sinceridade poética.

Na dimensão dessa sinceridade total, nas palavras de Rosenfeld, dessa virtude nietzschiana da verdade subjetiva, é que ``se subentende o seu empenho heroico por uma sintética língua falada"-escrita'', capaz de abraçar amorosamente todas as regiõe do Brasil''.\footnote{Ibid., p. 192.}

\section*{Sobre nossa equipe}

Rodrigo Ribeiro Neves é crítico literário e pesquisador, com doutorado em Estudos de Literatura e mestrado em Letras pela Universidade Federal Fluminense (\textsc{uff}). Foi pesquisador visitante na Princeton University, nos \textsc{eua}, e bolsista da Fundação Casa de Rui Barbosa. Atuou como docente de literatura brasileira na Universidade Federal Fluminense (\textsc{uff}) e na Universidade Federal do Rio de Janeiro (\textsc{ufrj}). Desenvolveu pesquisa de pós"-doutorado no Instituto de Estudos Brasileiros da Universidade de São Paulo (\textsc{ieb"-usp}) e na Universidad de Alcalá, na Espanha.

