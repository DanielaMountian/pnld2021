\chapter{Introdução}

Os contos que compõem esta antologia foram extraídos de três obras de
Mário de Andrade: \emph{Primeiro andar} (1926), \emph{Os contos de
Belazarte} (1934) e \emph{Contos novos} (1947). O objetivo é oferecer ao
leitor um panorama dos contos do autor, selecionando textos que são
considerados momentos decisivos de sua produção e cujos temas são
característicos de sua obra, como os limites da linguagem, a
incorporação da oralidade na escrita, a cultura popular, a dimensão
psicológica das personagens, o conflito de classes e a busca por uma
identidade nacional.

``Conto de Natal'', ``História com data'', ``Galo que não cantou'' e
``Briga das pastoras'' foram extraídos de \emph{Primeiro Andar}.
Trata"-se de textos em que Mário de Andrade experimenta (no sentido do
\emph{laboratório} ou da \emph{oficina}) os procedimentos e limites da
linguagem e da estrutura do próprio gênero e lança hipóteses de longo
alcance em sua obra, como a ideia de que o fragmentário pode constituir
a identidade.

Os contos ``O besouro e a rosa'', ``Jaburu malandro'', ``Caim, Caim e o
resto'' e ``Piá não sofre? Sofre'' pertencem a \emph{Os contos de
Belazarte}. Neles, a frase ``Belazarte me contou'' é o recurso narrativo
com que o escritor inicia cada uma das histórias. Aqui, as experiências com a
linguagem são aprofundadas, por meio da incorporação da oralidade na
narração e na fala das personagens e do retrato dos conflitos de classe.

Dos \emph{Contos Novos} vieram ``Primeiro de maio'', ``O poço'', ``O
peru de Natal'' e ``Nelson'' textos mais maduros, de linguagem renovada,
em que Mário de Andrade explora a dimensão psicológica e a dramaticidade
na constituição das histórias e no conflito das personagens.

Finalmente, vale dizer que Mário de Andrade continua atual para os
leitores do século \textsc{xxi}. Do ponto de vista formal, sua valorização do
ritmo da oralidade na linguagem escrita dialoga com a produção dos
principais escritores do século \textsc{xx}, como Jorge Amado e Guimarães Rosa, e
ainda segue dando frutos. Em termos temáticos, no conjunto dos contos,
observa"-se um retrato multifacetado do Brasil, sua diversidade e sua
formação sociocultural, sem deixar de destacar seus problemas e
contradições, em debate que está longe de ter se encerrado.

\part{\textsc{o peru de natal e outros contos}}

\chapter{Conto de Natal}

\hfill{}\emph{a Joaquim A. Cruz}\bigskip


\noindent{}Seriam porventura dez horas da noite\ldots{}

Desde muitos dias os jornais vinham polindo a curiosidade pública,
estufados de notícias e reclamos de festa. O Clube Automobilístico dava
o seu primeiro grande baile. Tinham vindo de Londres as marcas do
cotilhão e corria que as prendas seriam de sublimado gosto e valor. Os
restaurantes anunciavam orgíacos revelhões de Natal. Os grêmios
carnavalescos agitavam"-se.

Seriam porventura dez horas da noite quando esse homem entrou na praça
Antônio Prado. Trazia uma pequena mala de viagem. Chegara sem dúvida de
longe e denunciava cansaço e tédio. Sírio ou judeu? Magro, meão na
altura, dum moreno doentio, abria admirativamente os olhos molhados de
tristeza e calmos como um bálsamo. Barba dura sem trato. Os lábios
emoldurados no crespo dos cabelos moviam"-se como se rezassem. O ombro
direito mais baixo que o outro parecia suportar forte peso e quem lhe
visse as costas das mãos notara duas cicatrizes como feitas por balas.
Fraque escuro, bastante velho. Chapéu gasto dum negro oscilante.

Desanimava. Já se retirara de muitos hotéis sempre batido pela mesma
negativa: --- Que se há de fazer! Não há mais quarto!

Alcançada a praça o judeu estacou. Pôs no chão a maleta e recostado a um
poste mirou o vaivém. O povo comprimia"-se. Erravam maltrapilhos aos
grupos conversando alto. Os burgueses passavam esmerados no trajar. No
ambiente iluminado dos automóveis esplendiam os peitilhos e as carnes
desnudadas, e, aos cachos, as mulheres da vida roçavam pela multidão,
bamboleando"-se, olhos pintados, lábios incrustados de carmim. Boiando no
espaço estrias de odores sensuais.

O homem olhava e olhava. Parecia admiradíssimo.

Por várias vezes fez o gesto de tirar o chapéu mas a timidez dolorosa
gelava"-lhe o movimento. Continuava a olhar.

--- Vais ao baile do Clube?

--- Não arranjei convite. Você vai?

--- Onde irás hoje?

--- Como não! Toda São Paulo estará lá.

--- Ao \emph{réveillon} do Hotel Sportsman.

--- Vamos ao Trianon!

--- Por que não vens comigo à casa dos Marques? Há lá um
\emph{souper"-rose}.\footnote{} %footnote vazia

--- Impossível.

--- Por quê?

--- Não Posso. Vou ter com a Amélia.

--- Ah\ldots{}

Tirando respeitoso o chapéu, o oriental dirigiu"-se por fim ao homem que
dissera ``ir ter com a Amélia'' e perguntou"-lhe com uma voz suave como
os olhos --- caíam"-lhe os cabelos pelas orelhas, pelo colarinho:

--- O senhor vai sem dúvida para o seu lar\ldots{}

Decerto um louco. Não, bêbedo apenas. O outro deu de ombros.
Descartou"-se:

--- Não.

--- Mas\ldots{} e o senhor poderia informar"-me\ldots{} não é hoje noite de
Natal?\ldots{}

--- Parece ­­--- e sorria. --- Estamos a 24 de dezembro.

--- Mas\ldots{}

O homem da Amélia tocara no chapéu e partira.

Desolação, no sacudir lento da cabeça. Agarrando a maleta o judeu
recomeçou a andar. Tomou pela rua de São Bento, venceu o último gomo da
rua Direita, atingiu o Viaduto. A vista era maravilhosa. À direita,
empinando sobre o parque fundo, o Clube Automobilístico arreado de
lâmpadas de cor. A mole do edifício entrajada pelo multicolorido da
eletricidade parecia um enorme foco de luz branca. Do outro lado do
viaduto na esplanada debruava a noite o perfil dum teatro.

O judeu perdia"-se na visão do espetáculo. Aproximava"-se do largo espaço
da esplanada onde no asfalto silencioso escorregava outro cortejo de
autos. Cada carro guardava outra mulher risonha a suportar toda a
riqueza no pescoço. Feixes de operários estacados aqui e além. O rutilar
daqueles monumentos, o anormal da comemoração batendo na pele angulosa
dos vilões fazia explodir uma faísca de admiração e cobiça. Toda a
população dos bairros miseráveis despejara"-se no centro. Viera divertir"-se. Sim: divertir"-se.

O sírio entrou por uma rua escura que entestava com o teatro.
Incomodava"-o a maleta. Num momento, unindo"-se a uma casa em construção,
deixou cair o trambolho entre dois suportes de andaime. Partiu ligeiro,
atirando as pernas para frente, como pessoa a quem chamam atrás e não
quer ouvir.

Obelisco. E na subida vagarosa, lido numa placa de esquina: rua da
Consolação. Aqui o alarido já se espraiava discreto na surdo"-mudez das
moradias adormecidas.

Subiu pela rua. De repente parou diante da porta. Bateu e esperou.
Acolheu"-o uma criada de voz áspera:

--- Por que não tocou a campainha? Não tem olhos? Que quer?

--- Desculpe. Queria falar com o dono da casa\ldots{}

--- Não tem ninguém. Foram na festa.

Partiu de novo. Mais adiante animou"-se a bater outra vez. Nem criada. E
na aspiração de encontrar uma família em casa, batia agora de porta em
porta. Desesperação febril. Persistência de poeta. Uma vez a família
estava. Que divino prazer lhe paga o esforço! Mas o chefe não podia
aparecer. Lamentações lá dentro. Alguém está morrendo. Deus o leve!

Mais ou menos uma hora, depois de ter subido toda a rua, o judeu
desembocou na avenida. A faixa tremente da luz talhava"-a pelo meio mas
dos lados as árvores escureciam o pavimento livre das calçadas. Entre
jardins onde a vegetação prolongava sombra e frescor, as vivendas
enramadas de trepadeiras, como bacantes, dormindo. Sono mortuário.
Apenas ao longe gritava um edifício qualquer num acervo de luzes. O
judeu parou. O pó caiara"-lhe as botinas e a beirada das calças. O
cansaço rasgara"-lhe ruga funda sob os olhos e os lábios sempre
murmurantes pendiam"-lhe da boca secos e abertos. O pergaminho rofo do
rosto polira"-se de suor. Limpando"-se descuidado, recomeçou a andar muito
rápido para o lado das luzes.

Atravessados quase em carreiras vários quarteirões chegou ao trecho
iluminado. Era uma praça artificial construída ao lado da avenida.
Alguns degraus davam acesso à praia dos ladrilhos, onde passeavam pares
muito unidos. Sob ósseos caramanchões de cimento armado agrupavam"-se em
redor da cerveja homens de olhares turvos, bocas fartas. Entre o zum"-zum
da multidão brincavam nas brisas, moderadas pela distância, melodias
moles de danças. Por toda a parte a mesma alegria fulgindo na luz.

Daquele miradouro via"-se a cidade irrequietamente estirada sobre colinas
e vales de surpresa. Os revérberos confundiam"-se na claridade ambiente e
nos longes recortados um grande halo mascarava de santa a Pauliceia.
Apoteose.

Mas o judeu mal reparou nos enfeites com que o homem recamara aquela
página da terra. Olhava apenas a multidão, perscrutava todos os olhares.
Procuraria alguém?\ldots{} Quase que corria no meio dos passeantes ora
afastando"-se ao contato de uns ora atirando"-se para outros como que
reconhecendo. Desiludia"-se entretanto e procurava mais, procurava
debatendo"-se na turbamulta. Enfim desanimado partiu de novo. Ao descer
os degraus do miradouro notou duas escadinhas conducentes ao subsolo.
Espiou. Outro restaurante! Fugiu para a rua. A fila imóvel dos autos.
Corrilhos de motoristas e a guizalhante frase obscena. Passou. Ia
afundar"-se de novo no deserto da avenida. Mudou de resolução. Retornou
de novo para a luz. Era um espelho de suor. Caíra"-lhe o chapéu para o
lado e uma longa mecha de cabelos oscilava"-lhe na fronte como um
pêndulo. Os motoristas repararam nele. Riram"-se. Houve mesmo um prelúdio
de vaia. Nada ouviu. Entrou de novo no miradouro. Desceu os degraus. Um
negrinho todo vermelho quis recusar"-lhe a entrada. O oriental
imobilizou"-o com o olhar. Entrou. Percorreu os compartimentos. O mesmo
desperdício de luz e mais as flores, os tapetes\ldots{} Bem"-estar! Numa
antítese à brancura reta das paredes o sensualismo de couros
almofadados. E o salão nobre. E a orgia escancarada.

Todo o recinto era branco. Dispostas a poucos metros das paredes as
colunas apoiavam o teto baixo no qual os candelabros plagiavam a luz
solar. Esgalgos espelhos no entremeio das portas fenestradas eram como
olhos em pasmo imóvel. As flores feminilizavam colunas e alampadários,
poluíam seu odor misturando"-o à emanação das carnes suarentas e nessa
decoração de fantasia apinhava"-se comendo e bebendo sorrindo e cantando
uma comparsaria heterogênea.

Bem na frente do judeu sentados em torno duma mesa estavam dois homens e
uma mulher. Falavam língua estranha cheia de acentos guturais. Seriam
ingleses\ldots{} Os homens louros e vermelhos denunciavam a proporção
considerável da altura pelo esguio dos torsos e dos membros mas a
perfeição das casacas dava"-lhes à figura um alto quê de aristocracia.

A mulher era profundamente bela. Trajava preto. Gaze. A fazenda
envolvia"-lhe a plasticidade das ancas e das pernas, dando a impressão de
que o busto saísse duma caligem. O vestido como que terminava na
cintura. Um tufo de tules brancos subia sem propriamente encobrir até
parte dos seios, prendendo"-se ao ombro esquerdo por um rubim. Sobre a
perfeição daquele corpo a cabeça era outra perfeição. Na brancura
multicor da pele queimava uma boca louca rindo alto. As narículas quase
vítreas palpitavam voluptuárias como asas de pombas. Os olhos eram da
maior fascinação no arqueado das sobrancelhas, na ondulação das
pálpebras, no verde das pupilas más. E colmava o esplendor uma cabeleira
de pesadas ondas castanhas.

Já tonta, meneando o corpo, estendendo os braços virgens de joias sobre
a toalha, oferecia"-se à contemplação abusiva da luz. E era também no
alaranjado de sua carnadura que os dois ingleses apascentavam os
olhares.

Em torno de todas as mesas, como refrão do prazer rico repetia"-se a
mesma tela: homens rudes acossados pelo desejo, mulheres incastas
perfeitas maravilhosas.

Do outro lado do salão a orquestra vibrou. Ritmo de dança, lento
brutesco. Balançaram dois ou três pares num círculo subitamente vazio.
Um dos ingleses e a mulher de preto puseram"-se a dançar. Inteiramente
abraçada pelo homem ela jungia"-se a ele, agarrava"-se"-lhe de tal jeito
que formavam um corpo só. Ondulavam na cadência da música: ora partiam
céleres como numa fuga, parando longamente depois como num espasmo. Ora
se afastavam um do outro num requebro, ora mais se uniam e o braço
esquerdo dela rastejava como um crótalo no dorso negro da casaca.
Dançavam com os sentidos e a mulher na ascensão do calor e da volúpia,
mostrava na juntura esquerda dos lábios um começo de língua.

O judeu continuava a olhar. Seguia os pares no baloiço do tango,
esforçando"-se por disfarçar com a imobilidade a excitação interior. Mas
seus olhos chispavam. Mas juntas nas costas tremiam"-lhe as mãos mordidas
pelos dedos.

Enfim vibrados os últimos acordes os dançarinos pararam. A inglesa
seguida pelo parceiro, arrebentando os olhares que lhe impediam a
passagem, viera sentar"-se. Incrível! O judeu bufando enterrara o chapéu
na cabeça, abrira o fraque com tal veemência que os botões saltaram e
tirando dum bolso interno uma trífida correia de couro fustigara a
espádua da mulher. Tal fora a energia da relhada que o sangue
imediatamente brotava no vergão enquanto a infeliz uivava ajoelhando. O
golpe arrebentara a gaze junto ao ombro. Seio lunar!

Mas o judeu malhava indiferente todas as formosuras.

Um primeiro imenso espanto paralisou a reação daqueles bêbedos. O
fustigador derribando cadeiras e mesas atravessava os renques de
pusilânimes, cortava caras braços nus. Tumulto. Balbúrdia dissonante. O
mulherio berrava. Os homens temendo serem atingidos pela correia do
louco fugiam dele na impiedosa comicidade das casacas. Arremessavam"-lhe
de longe copos e garrafas. Mas ele percorria em alargados passos o
salão, castigando todos com furor. Onde a correia assentava negrejava um
sulco, chispava um uivo.

Nos primeiros segundos\ldots{} Depois, açulados pelo número, os homens já se
expunham mais aos golpes na esperança de bater e derrubar. O círculo
apertava"-se. O oriental teve de defender"-se. Vendo junto à parede um
amontoado de mesas saltou sobre ele. Abandonara o chicote, empunhara uma
cadeira, esbordoava com ela os que procuravam aproximar"-se. Impossível
atingi"-lo. Seus braços moviam"-se agílimos tonteando cabeças, derreando
mãos.

As mulheres agrupadas à distância reagiam também. As taças pratos copos
atirados por elas sem nenhuma direção acertavam nos alampadários cujos
focos arrebentavam com fofos estampidos soturnos. As luzes apagadas
esmoreciam a nitidez do salão e as sombras enlutavam o espaço, diluindo
os corpos numa semiobscuridade pavorosa.

Mais gente que acorria. Os passeantes do miradouro atulhando as portadas
saboreavam em meio susto a luta. Os motoristas procuravam roubar
bebidas. A polícia telefonava pedindo reforços.

Mas o oriental já começava a arquejar. Seus lábios grunhiam
entrechocantes. Uma garrafa acertara"-lhe na fronte. O chapéu saltando da
cabeça descobriu na empastada desordem das madeixas a rachadura
sangrando. O sangue carminava"-lhe o rosto, cegara"-lhe o olho esquerdo,
entrava"-lhe na boca e, escorrendo pelo hissope da barba, espirrava sobre
a matilha gotas quentes.

Afinal alguém consegue agarrar"-lhe a perna. Puxa"-o com força. Ele tomba
batendo"-se. Todos tombam sobre ele. Ninguém lhe perdoa a desforra. Os
que estão atrás levantam os punhos inofensivos para o alto esperando a
vez. Desapareceu. O molho de homens.

Chega a polícia. A autoridade só com muita luta usando força, livra o
mísero. No charco de champanha sangue vidros estilhaçados ele jaz
expirante pernas unidas, braços estendidos para os lados, olhos fixos no
alto, como querendo perfurar as traves do teto e espraiar"-se na
claridade fosca da antemanhã.

Levaram"-no entre insultos.

\asterisc

Todo jornal comentava o caso no dia seguinte. O público lia, rebolcado
no inédito do escândalo, as invenções idiotas, as mentiras sensacionais
dos noticiaristas.

Entanto, nas múltiplas edições dos diários, relegado às derradeiras
páginas, repetia"-se o estribilho perdido que ninguém leu. Homessa!
curioso\ldots{} Um guarda"-noturno achara rente a uma casa em construção uma
pequena mala de viagem. Aberta na mais próxima delegacia, encontraram
nela entre roupas usadas e de preço pobre uma tabuinha com dizeres
apagados, quatro grandes cravos carcomidos pela ferrugem e uma coroa
feita com um trançado de ramos em que havia nódoas de sangue velho e



\chapter{História com data}

\hfill{}\emph{a Antônio V. de Azevedo}\bigskip

\noindent{}Agitação desusada no hospital. Telefonemas e telefonemas. A todo
instante chegavam automóveis particulares. Numa das salas a cena difícil
das pessoas que perderam alguém. As lágrimas já cansadas paravam pouco a
pouco nos olhos de irmãos tias e da srª Figueiredo Azoé mãe do ``infeliz
rapaz''.\footnote{\emph{Jornal do Comércio,~}14 de fevereiro de 1931.
  {[}Nota \textsc{ma}{]}.} Entrelaçavam"-se na penumbra do aposento soluços
desritmados, suspiros frases vulgares de consolo.

--- Viverá.

--- Tenha esperança, minha amiga.

--- Meu filho\ldots{} meu filho!

--- Sossegue!

--- Quanto tempo!\ldots{} desespero!\ldots{}

--- Tome um pouco de café.

--- Não.

--- Tome!

--- Não quero.

--- Tome\ldots{} Reabilita.

O pai acabou tomando o café. Telefonemas e telefonemas. A todo instante
chegavam automóveis particulares.

Tratava"-se de Alberto de Figueiredo Azoé 25 anos aviador, descendente
duma das mais antigas famílias do Jardim América. Nessa manhã de 13
abrira asas no Caudron.\footnote{Referência ao avião Caudron G.3,
  construído pelos irmãos Gaston e René Caudron, fundadores de
  fabricante francesa de aeronaves que operou entre 1909 e 1933. Foi o
  primeiro a realizar um loop em acrobacia aérea, em 1913.} Ao realizar
uma acrobacia a pouca altura o motor não funcionara a tempo. O avião se
espatifara na rua Jaguaribe a 20 metros do Hospital. Pronto socorro.
Telefone. E fortificados pelo pedido da família os três grandes
cirurgiões tomaram conta do ``imprudente moço''.\footnote{\emph{Estado de
  S. Paulo}, 14 de fevereiro de 1931. {[}Nota \textsc{ma}{]}.}

Era ainda um desses exemplos do que Gustavo Le Bon chamou a ``ironia dos
desastres''.\footnote{G. Le Bon:~\emph{La psychologie du hasard}, p. 836.
  Alcan. {[}Nota \textsc{ma}{]}.} Nenhuma lesão no corpo. Apenas um estilhaço de
motor esmigalhara parte do cérebro do ``arrojado aviador''.\footnote{\emph{Gazeta},
  14 de fevereiro de 1931. {[}Nota \textsc{ma}{]}.} Transportaram"-no ainda
vivendo pra sala das operações.

Dois médicos perplexos:

--- Morre. É inútil.

--- Morre.

O terceiro curioso inventivo. Riquíssimo subconsciente.

Um homem pobre ultrapassando talvez os quarenta anos morria duma lesão
cardíaca no hospital. Ninguém que o chorasse. Linda morte.

O terceiro operador falou. Repulsas. Risadas. O terceiro operador mesmo
sorrindo insistiu com mais energia.

--- \ldots{} Por que não?! Ele morre mesmo. O outro morre fatalmente, sem
lesão alguma no cérebro. Poderemos salvar ao menos um. Vocês parecem
estar ainda no tempo do doutor Carrel\ldots{} E Chimiuwsky, com o coração?\ldots{}
Tenta"-se!

Depois deu de ombros e derrubou a cinza do charuto. Houve perguntas para
fora da sala de operações. As freiras correram. Transportes.

A madre superiora abriu a porta da sala de visitas. A ansiosa
interrogação dos olhos, das mãos de todos. A comovente interrogação das
lágrimas da srª Figueiredo Azoé.

--- Vai tudo bem. A operação acabou agora. Dr.\,Xis garante a salvação.

Pouco depois o dr.\,Ípsilon amigo da família apareceu. Rodearam"-no
puxaram"-no. Ensurdeceram"-no de perguntas.

--- Sossegue, dona Clotilde. O caso é gravíssimo, não posso negar mas a
operação foi bem. Alberto é forte, perdeu pouco sangue\ldots{} Fizemos. Uma
trepanação\ldots{} Esperemos que se salve\ldots{}

Pendiam"-lhe dos óculos umas vergonhas hesitantes.

Em trêmula sequela a mãe, o pai, os irmãos foram ver de longe Alberto a
dormir. Depois o dr.\,Xis exigiu o afastamento da família até a cura do
rapaz. A comoção, explicava, provocada pela revivescência das imagens
poderia causar até a morte\footnote{Vide a totalidade dos romances do
  séc. \textsc{xix}. {[}Nota \textsc{ma}{]}.} ou no mínimo uma idiotia de primeiro grau.
Quanto a qualquer possível lesão que o mecanismo cerebral apresentasse
sempre seria tempo de ``constatá"-la'' (sic).

O período da morte passou. Alberto convalescia rápido.

Nada quase falava. Beijava comovido a mão da freira que o tratava. Tinha
lágrimas de gratidão para os médicos.

Fato curioso registrado pelas freiras é que à medida que Alberto sarava
o dr.\,Xis tornava"-se mais e mais inquieto. Agitação contínua. Cóleras
sem razão. Perguntas esquisitas que espantavam a enfermeira. Se o doente
ia tão bem! Passava os dias mirando as próprias mãos. Nada de anormal.

Mas o dr.\,Xis sentado à cabeceira do rapaz. Que dedicação! O sr.\,Felisberto Azoé ouvi que pretendia presenteá"-lo com um cheque de 40
contos (quarenta contos de réis). E tão dedicado quanto inflexível. Nada
de permitir que a família se aproximasse do moço. Por uma das janelas do
hospital apenas o viam passear agora pelo braço do dr.\,Xis nos pátios de
sol.

O dr.\,Ípsilon é que esfregava as mãos satisfeitíssimo. Um dia perguntara
a Alberto:

--- Lembras"-te de mim?

O outro chorando lhe beijara a mão:

--- Lembro sim senhor.

Desde então o dr.\,Ípsilon esfregava as mãos satisfeitíssimo.

--- O nosso trabalho foi admirável. Quando o comunicarmos à Sociedade de
Medicina e Cirurgia creio que o mundo inteiro se espantará. Dona
Clotilde, seu filho está salvo!

E ao dr.\,Xis três vezes por dia:

--- Não começaste ainda o relatório?

--- Espere.

Alberto estava bom. Caminhava por si.

Dr.\,Xis estava mal. Hesitava.

Um dia no entanto encontrou o moço gesticulando suecamente.\footnote{Referência
  à ginástica sueca.} Sorriu. Alberto parara a ginástica.

--- Seu doutor, já estou bom. Queria sair.

--- Sairás breve. Agora vamos dar uma volta pelo jardim.

Alberto caminhava firme alegre. O dr.\,Xis seguia"-o lateralmente um pouco
atrás. Na aleia de trânsito junto à porta um automóvel. Alberto parou
olhando a máquina. Caminhou para ela. Sentou"-se no lugar do motorista. A
máquina moveu"-se rápida habilíssima. Fez a volta do gramado e descansou
no ponto de partida.\footnote{~Ebbinghaus:~\emph{Der Gedachtnis und der
  Musk el}, p. 777. Schmidt und Gunther. {[}Nota \textsc{ma}{]}.} Dr.\,Xis
acendeu o charuto.

--- Sabes guiar automóvel?

--- \ldots{}sei?\ldots{} murmurou espantadíssimo.

Depois de olhar muito as pernas vago quase sorrindo Alberto murmurou:

--- Parece que espichei, seu doutor!

Era curiosa a agitação do dr.\,Xis. Dedos de gelatina. Até deixou cair o
charuto.

--- Não é nada. Voltemos.

--- Não começaste ainda o relatório?

--- Vais dizer ao sr.\,Azoé que lhe levo o filho amanhã. Que a casa
esteja como sempre sem modificação alguma.

E o dr.\,Xis fez o barbeiro entrar no quarto do rapaz.

--- Vai fazer"-te a barba\ldots{}

Alberto sentou no lugar que lhe indicavam. O barbeiro trabalhou entre
dois silêncios.

--- Agora vem lavar o rosto no quarto pegado. O lavatório de lá é maior.

No quarto de Alberto o dr.\,Xis fizera substituir o lavatório por uma
mesa onde se depusera bacia e jarro.

Alberto foi. Ao inclinar"-se para lavar o rosto viu"-se refletido no
espelho. Parou: Depois, quase a gritar horrorizado guardando os olhos no
braço:

--- Não!

Imediatamente o médico se fechara por dentro com o rapaz.

--- Não\ldots{} Não sou!\ldots{}

Entressorria medroso. Depois começou a chorar. Dr.\,Xis seguia"-lhe os
movimentos, Alberto voltou ao espelho. Fugiu dele apavorado. Quis
partir. Foi esconder"-se no corpo do dr.\,Xis como uma virgem.

--- Quem é, seu doutor!\ldots{} Quem é esse homem\ldots{}

--- Sossega, meu rapaz. Sou eu.

--- Não, o outro!

--- Estamos sós. Vem comigo!

Atraía"-o para o espelho. Alberto com lindas forças venceu o médico.

--- Não quero!

--- Sossega, Alberto!!

--- Alberto?\ldots{} quem é Alberto!

--- És tu.

--- Eu!\ldots{} Não! deve ser o outro\ldots{} o moço!\ldots{}

Apalpava"-se desesperado. Os olhos giragiravam no limite das órbitas,
infantis como num esforço para ver o rosto a que pertenciam.

--- Acalma"-te. Qual é teu nome então?

--- \ldots{} o outro\ldots{} Não! Eu\ldots{} eu sou José!

Dr.\,Xis aguentou a custo o golpe. Ficou gelo. Voltando do espavento:
Acalma"-te e escuta. És Alberto.

--- Não! Sou José!

--- Escuta primeiro, já disse! Estiveste muito doente ouviste? Segue bem
o que te digo. És Alberto de Figueiredo Azoé. És aviador. Tua mãe é dona
Clotilde de Figueiredo Azoé ouviste? Caíste do aeroplano. Quebraste a
cabeça. Fizemos uma operação muito difícil. Por isso estás assim como
quem não se lembra. Pensas que és outro. Mas tu és Alberto de Figueiredo
Azoé. Vamos, repete o teu nome!

--- Alberto de Figueiredo Azoé\ldots{}

--- Sou eu que te digo, ouviste bem? Teu médico. Que te salvou da morte.
És filho do sr.\,Felisberto Azoé teu pai. És aviador. Não te lembras\ldots{}
És muito rico.

Alberto, Alberto ou José? escutava. O médico parou observando"-o.
Desenhou"-se um sorriso mal feito nos lábios do moço. Sacudiu a cabeça
desolado. Apertava as faces com mãos desesperadas. Não sentia\footnote{Ribot:~\emph{Pathologie
  frénétique des changements de personnalité}, Alcan p. 83. {[}Nota
  \textsc{ma}{]}.} Alberto de Figueiredo Azoé.

--- Agora estás mais calmo. Vem ver o teu rosto no espelho.

--- Não, seu doutor! pelo amor de Deus! faz \emph{favore\ldots{} no}!!

Empuxado, reagia quase com grito.

--- Vem! Quero que sejas Alberto de Figueiredo Azoé.\footnote{Bergson:~\emph{Le
  règne de la volonté}, Garnier p. 135; W. James:~\emph{The Irradiations
  of Wish}, Century Co. pp. 14 e 15. {[}Nota \textsc{ma}{]}.}

--- Não! \emph{non ancora!\ldots{} Io\ldots{}}

Parou indeciso. Escutou as últimas palavras que saltitavam fugitivas no
aposento. O doutor:

--- \emph{Lei parla italiano?}

--- \emph{Si! Sono proprio d'Italia!\ldots{} ma\ldots{}} não\ldots{} Não!

As palavras saíam perturbadas com acento inverídico de quem não sabe
falar italiano. De boca desacostumada a pronunciar o italiano.

--- Descansa. Vamos pro teu quarto.

E lá:

--- Deita"-te. Fico a teu lado. Pensa bem, Alberto: tua cabeça ainda está
doente pelo choque. Perdeste a memória. Só te lembras de coisas de que
ouviste falar.\footnote{Ribot: op. cit., p. 249. {[}Nota \textsc{ma}{]}.} Pensa
bem no que te digo: és Alberto Figueiredo Azoé. Amanhã verás teus pais e
irmãos de que não te lembras. Deves conhecê"-los ouviste? Sofrerão muito
se te mostrares esquecido. Pensa agora em tudo isto. Não és José ouviste
bem?! responde que estás ouvindo, acreditando\ldots{} Responde, Alberto!\ldots{}

Alberto ou José moveu lábios sem frase abúlico.

E o doutor sentado à cabeceira do moço falou e continuou falando. Meia
hora depois inda remoía persuasões. Alberto adormecera entre elas. Duas
fundas rugas penduradas das abas do nariz guardavam como parênteses as
frases que aquela boca falaria e não lhe pertenceriam.

Às 17 horas acordaram"-no para o jantar. Comeu bem. Era pequeno o
abatimento. O dr.\,Xis quando ambos sós tentou a experiência:

--- Alberto!

--- Que é?

O médico sorriu agradecido. Aproximou"-se. Pôs"-lhe sob os olhos o livro
aberto e apontou para as letras.

--- Conheces isto?

--- Como não!\ldots{} são letras.

--- Sabes ler?

Os olhos de Alberto fixaram mais as letras, correram fácil e exatamente
pelas linhas. Espantado o moço murmurou como se perguntasse:

--- Não?\ldots{}

E voltou a seguir as linhas do papel numa ânsia de reconhecimento. Dr.\,Xis fê"-lo sentar"-se junto à mesa. Deu"-lhe o lápis.

--- Escreve.

Sobre as folhas esparsas o moço traçou a princípio firme, com letra
esportiva:

\emph{Rose mon chouchou 120 cavalos Part Alberto 30 record Rose"-Roice
mon chouchou Caudron Grevix}\footnote{Célebre boxista senegalês da
  época. V.~\emph{La vie au grand air}, dezembro 1932. {[}Nota \textsc{ma}{]}.}
\emph{mon choudron\ldots{}}

Dr.\,Xis arrancou"-lhe o lápis da mão.

Às 20 horas deu"-lhe uma beberagem. Alberto adormeceu. Foi transportado,
assim dormindo, para casa.

--- Minha senhora, seu filho sarou. Mas a lesão foi muito grave\ldots{} Ficou
com a memória um tanto perturbada\ldots{}

--- Meu filho está louco!

--- Sossegue. Não se trata de loucura. Apenas a memória\ldots{} Abandono
parcial de memória. Mas sara. Sarará! É preciso aos poucos incutir"-lhe
no espírito quem ele é. Por um fenômeno que\ldots{} se dá frequentemente
nesses traumatismos acredita ser outra pessoa\ldots{} Naturalmente cuja
história o impressionou.

--- Meu filho!

O pranto necessário.

--- Afirmo"-lhe que sara. Devemos aos poucos reeducá"-lo. Esqueceu"-se um
pouco por exemplo\ldots{} de ler. Mas a memória voltará. É preciso que tudo
se passe como antigamente.

--- Meu pobre filho! Naturalmente nem se lembra de sua mãe!\ldots{}

--- Minha senhora, descanse em mim! O quarto dele está pronto?

--- Sim. Não alteramos nada.

--- É preciso fazer"-lhe reviver os costumes antigos\ldots{}

--- Era eu que ia acordá"-lo sempre quando ele não se (soluço) levantava
muito cedo para ir nadar\ldots{} Levava o café para ele\ldots{}

--- Pois a senhora continuará a levar"-lhe o café. Irá acordá"-lo amanhã.
Estarei aqui. Não: prefiro passar a noite aqui, nalgum quarto pegado ao
dele, não tem?

--- Não tem.

--- Pois terá a bondade de ordenar que me deixem uma poltrona junto da
porta. Dormirei nela.

--- Doutor! quanta bondade!\ldots{} Doutor\ldots{}

Alberto dormia sossegadamente.

Às nove horas do dia seguinte a senhora Figueiredo Azoé num penteador
muito roxo acordou o médico. O sobressalto do dr.\,Xis espantou"-a:

--- Que é!

--- Desculpe, doutor. Apareço assim porque era assim que ia acordá"-lo.
Alberto gostava de roxo\ldots{}

--- Fez bem.

--- Geralmente acordava às nove\ldots{} Já são oito e três quartos\ldots{} Trago o
café\ldots{}

Num arranco de desesperada aventura o médico largou:

--- Pois vamos!

Entraram. Ela entreabriu uma das janelas. O raio curioso esquadrinhou o
aposento.

--- Era assim mesmo que ele dormia.

O rapaz tirara a coberta leve que lhe tinham posto sobre o corpo e de
pernas abertas pousando a cabeça num dos braços era como um lutador
cansado.

--- Alberto! Alberto!\ldots{}

O ``digno sucessor de Edu Chaves''\footnote{\emph{Diário Popular}, 22 de
  março de 1931. {[}Nota \textsc{ma}{]}} se moveu mole, abriu os olhos. Consertou
a posição dormindo outra vez. Dona Clotilde estava com medo do filho.
Venceu"-se:

--- Alberto!\ldots{} Sou eu! Tua mãe\ldots{}

Parava indecisa. Esforçava"-se por repetir as frases costumeiras. Não se
lembrava. Tudo agora lhe parecia tão artificial, tão inexato!

--- São horas\ldots{} Trago o café!!

O moço resmungou inconsciente. Abriu os olhos acordado. O reflexo do
espelho iluminava o corpo da ``ilustre dama''.\footnote{Cigarra, 20 de
  dezembro de 1929. Sob uma fotografia da Liga das Senhoras Católicas.
  {[}Nota \textsc{ma}{]}} Alberto sorriu"-lhe como sempre e murmurou o eterno:

--- Ora, mamãe!\ldots{}

Escutou"-se atraído. E fixou mais a mulher. E num pulo sentou"-se na cama.
Dona Clotilde recuou amedrontada. Dr.\,Xis aproximou"-se.

--- Bom"-dia, Alberto.

Agarrado ao médico, doído, pedindo proteção:

--- Seu doutor!

--- Sou eu, Alberto.

--- Alberto!?\ldots{}

--- Sim: Alberto. Esta senhora é tua mãe.

--- Não tenho mãe!\ldots{}

--- Esta senhora é tua mãe. Lembra"-te do que te disse ontem, Alberto.
Estiveste doente! Esqueceste!

--- Não, seu Doutor! Quero ir s'embora! vamos!

E no espelho da guarda"-casacas viu um moço quase conhecido agarrado ao
doutor. Olhou para este. Procurou"-lhe em torno\ldots{} Encontrou suas
próprias, não, mãos longas musculosas agarradas ao paletó do médico.
Começou a chorar todo infeliz.

A senhora Azoé chorava também, sem naturalidade uma das mãos ocupada com
a xícara. O duplo sofrimento das mães! Sofrem a dor dos filhos e a sua
dor de mães! Como se não lhes bastassem as deformações prematuras e o
castigo luminoso dos partos como outros tantos pelicanos\ldots{}

O dr.\,Xis procurou dar fim à cena. Ia pronunciar o ``sossega, meu
rapaz'' mas reparou que já dissera essa frase muitas vezes e mudou:

--- Acalme"-se, Alberto! Precisas acostumar"-te à tua nova situação. Não
te recordas porque estiveste doente.

O médico falava dificilmente agora. Devido ao caso do ``sossega'', sem
querer, contra a vontade mesmo começara a policiar a própria fala. Em
vez de ``lembras'' corrigira para ``recordas''. Foi Alberto que terminou
a situação cansado de reagir:

--- Não me lembro de nada disso tudo que seu doutor está dizendo\ldots{} Eu
não tinha\ldots{} mãe. Sou José\ldots{} Eu me lembro de mim sozinho (aqui fazia
esforços de rugas para lembrar). Em criança fiz viagem\ldots{} Tinha um homem
com um dente na boca que fumava um cachimbo fedido\ldots{} não me lembro!\ldots{}
O homem com uma ferida sarada parecia de navalha na cara\ldots{} Outro homem
dizia que era meu tio\ldots{} Meu tio e minha tia\ldots{} Depois na colônia\ldots{} Eu
fugi mocinho\ldots{}

A senhora Figueiredo Azoé soluçava alto.

--- \ldots{} Minha mãe\ldots{}?

E José, não, Alberto, Alberto ou José? queria lembrar sofria. Muita
coisa nos olhos nas mãos que dizia que parecia que era assim
mesmo.\footnote{Ribot:~\emph{Les reconnaissances musculaires}, Alcan p.
  101. {[}Nota \textsc{ma}{]}.} Mas se sabia que não era assim!

--- Alberto, estás martirizando tua mãe. Cala"-te! Contas alguma história
que te impressionou. Sossega, meu\ldots{} Veste"-te. Estou aqui!

Alberto cedeu como quem cede para o aniquilamento.

Desceu da cama pela direita onde moravam as chinelas. Abriu as torneiras
do lavatório. Lavou"-se. Penteou"-se. Foi buscar as meias limpas na gaveta
exata. E calçava as calças depois as botinas depois pôs a camisa o
colarinho a gravata\ldots{} Parava às vezes indeciso, outras envergonhado de
saber\ldots{} Então era preciso que o doutor lhe desse as calças\ldots{} e depois
o colarinho\ldots{} Alberto continuava maquinalmente entregue à dura sorte
feliz.

--- Estás vendo como te lembras?\ldots{} Se fosses esse outro como saberias
onde estavam as meias as botinas?\ldots{} Agora precisas de paciência
ouviste? Irás de novo aprendendo o que esquecestes, verás.

Alberto procurava qualquer coisa. Devia ser o paletó\ldots{} Assim ao menos
pensava o dr.\,Xis dando"-lhe o paletó. Alberto vestiu"-o. Exausto foi
tirar duma gaveta a escova de roupas. Esfregou vivamente as calças,
unicamente as calças como se o paletó não merecesse limpeza. Depois
jogou a escova sobre a cama e abrindo o guarda"-roupa tirou o pijama de
seda roxa. Começou a vesti"-lo sobre o paletó. Parou percebendo o engano.
Envergonhado olhou o médico. Guardou o pijama de novo.

--- Agora, Alberto, vais ver teus irmãos, teu pai, Felisberto Azoé.

Ao saírem do aposento houve do outro lado da galeria um esvoaçar
fugitivo de saias passos que desciam a escada. Alberto olhava
desconfiado para o dr.\,Xis. A família estava toda no \emph{hall}.
Impaciência irreprimível em cada olhar. Talvez dor. Aquela reunião
tantas pessoas o criado que espiava\ldots{} O moço sentiu"-se em terra
estranha. Fez um movimento de recuo.

--- Teu pai, Alberto. Não te lembras? tua irmã, teus irmãos\ldots{}

--- Seu doutor, vamos embora!\ldots{}

Apertava a mão do operador. Criança a proteger"-se. E baixinho dolorido:

--- Não\ldots{} não\ldots{} Não lembro!\ldots{} sou o outro\ldots{} sou\ldots{}

--- Cala"-te, Alberto! Já te disse que não és o outro! Esta é a tua
família\ldots{} teu pai\ldots{}

Alberto chorava sem largar o médico. A família chorava. O dr.\,Xis\ldots{} Mas
o rapaz levantara a cabeça resolvido. Cessaram"-lhe as lágrimas.

--- Vamos embora! Não fico mais aqui!

--- Sossega, Alberto. É tua famíl\ldots{}

--- Não é minha família! Sou o outro. Sou José! Quero ir embora!!

--- Ir para onde, então!

--- Para casa!

--- Aonde?

--- Para minha casa, com a Amélia. Minha mulher\ldots{} rua Barbosa\ldots{} Quero
ir!

E procurava alguma coisa. Dirigiu"-se enfim para a porta que dava no
jardim interior. O médico alcançou"-o.

--- Espera um pouco. Mando buscar tua mulher. Verás que a não conheces.
Espera!

--- Quero ir com Amélia!\footnote{Convém notar que esta Amélia não é a
  mesma do ``Conto de Natal''. {[}Nota \textsc{ma}{]}}

--- Escuta, Alberto, estou falando! Já disse que mando buscar essa
Amélia! Vais esperar. Esperas comigo, não te deixo. Rua Barbosa\ldots{} que
número?

--- Rua Barbosa\ldots{} não tem número. Última casa da direita.

Ninguém sabia onde era a rua Barbosa.

--- Onde fica a rua Barbosa, Alberto?

--- Na Lapa\ldots{} Atrás da fábrica de louças. Um dos Azoés partiu rápido.

Alberto esperava impaciente. Parecia não ver ninguém. Andava pela sala.
Sentava"-se. Erguia"-se. Reparava em todos francamente. Depois
envergonhava"-se. Vinha para junto do médico. Um momento, com gestos
largos cheios de liberdade sentou"-se na grande cadeira preguiçosa.
Assobiou dum modo especial. Logo os latidos dum cão. E o enorme policial
apareceu. Que festas para o dono! Alberto quis reconhecê"-lo. Seus lábios
juntaram"-se abriram"-se como querendo dizer um nome\ldots{} Teve medo daquele
cão. Quis erguer"-se. Defendeu"-se.

--- É Dempsey, meu filho!

--- Tirem esse cachorro! Me morde!\ldots{}

Foi preciso tirar Dempsey dali. E daí em diante os uivos do cão
compassando as cenas.

Trinta minutos depois o automóvel voltava. Luís fez entrar a mulata
forte com as mãos gretadas pela aspereza das águas no ofício de lavar.
Entrou olhando sem medo. Saudou consertando o xale preto.

--- Conheces, Alberto? É Amélia.

Alberto correu para ela. Segurou fortemente o braço da admirada.

--- Vamos embora, Amélia! Não fico aqui!

--- Largue de mim, moço!

--- Sou eu, teu homem!\ldots{} José\ldots{}

--- Meu homem morreu na Santa Casa\ldots{} Deus Nosso Senhor Jesus Cristo lhe
tenha!

--- Não morreu! Sarei! Sou eu, José!

Amélia recuou amedrontada:

--- Esse moço está doido, credo!

Alberto agarrava desesperado raivoso suplicante:

--- Não me deixe aqui! Estão caçoando de mim\ldots{} Sou José!

O dr.\,Xis que se aproxima toma um soco no peito.

--- Me largue, moço! Que é isso agora!

--- Amélia, não te lembras! Me leve!\ldots{} Teu\ldots{}

--- Me largue já disse! Meu pobre José está no Araçá! Foi então para
isso que me cham\ldots{} ahm\ldots{} me largue!

Debatia"-se nas mãos do rapaz. Dois fortes a lutar. Esfregavam"-se na
parede junto à porta.

--- Tirem esse moço daqui. Eu grito! Socorro!

Acudiram. O sr.\,Azoé o médico os rapazes. Alberto não largava a mulata.
Desenvencilhou"-se repentinamente do irmão que o agarrara por trás, moveu
o cacho de gente, empurrou"-o para o centro da sala. Correu para a porta.
Fechou"-a. E olhou todos com olhos duplicados da loucura de resolução.

--- Não queres me levar, desgraçada! Eu conto tudo! assassina!\ldots{} me
leva?\ldots{}

Amélia resoluta armara"-se dum vaso onde uma palmeirinha lutava por
viver. Que saudades do aclive aquoso sempre verde, onde junto das irmãs
e das avencas faceiras escutava noite e dia o reboo pluvial da cascata!
Nas tardes, quando o céu arcoirisado\ldots{}

--- Segurem o moço que eu atiro!\ldots{} atiro mesmo\ldots{} se ele vier outra
vez\ldots{}

A senhora Figueiredo Azoé levantou"-se diante do filho, como a estátua do
devotamento e do sacrifício, protegendo"-o. O sr.\,Azoé os rapazes lutando
com a lavadeira.

--- Ah!\ldots{} (rascante) É assim? Não queres me levar, desgraçada!\ldots{} Vou
para a correição\ldots{} Mas tens de ir também. Não ficas com o Júlio, já
sei! Ela matou! Assassina! Matou os dois filhos\ldots{} Quando nasceram.
Matou os dois filhos! Não queríamos crianças\ldots{} Ela enterrou no quintal.
Em Moji. O outro antes de nascer. Assassina! Vou parar na cad\ldots{}

--- Cachorro!

O vaso, desviado, se espatifou no meio da sala. Coitada palmeirinha!

--- Prendam ela!\ldots{} \emph{Figlia dun cane} (cão)! É verdade\ldots{}
\emph{lo}\ldots{} juro!\ldots{}

--- É mentira! Não conheço esse homem!

--- Prendam! Assassina!\ldots{} No jardim perto da escada\ldots{}

--- Não!\ldots{} não conheço!\ldots{} Não me prendam! não fiz nada!\ldots{} Foi ele que
quis\ldots{} Perdão!\ldots{} Não conheço esse moço\ldots{} nunca vi\ldots{} Foi o outro, foi
José que quis\ldots{} Perdão!

--- Fui eu! mas foi ela também!

Atirou"-se sobre a mulata. Ela voltou"-lhe uma punhada na cara. Alberto
desviou com gesto grácil de boxista.\footnote{Woodworth --~\emph{Gesture
  and Will}~-- Macmillan \& Co. p. 88. {[}Nota \textsc{ma}{]}.} Atracaram"-se de
novo. Ela dilacerou"-lhe a mão com os dentes. Prendam!\ldots{} Sujo!
Maldito!\ldots{} Foi um custo. Assassina! Com o barulho os criados, o
motorista acorreram. Prendam! Ela também!\ldots{} Me largue!\ldots{} Braços
punhos. Embrulho. Barulho. Foi difícil. Afinal os homens conseguiram
separar os dois. Amélia liberta fugiu por uma porta. Desapareceu. A
cólera de Alberto, Alberto ou José? foi tremenda. Berrava termos
repetidos numa língua infame. Socava os que o prendiam. Machucara
fortemente um dos irmãos. Depois diminuiu a resistência pouco a pouco.
Suor frio lhe irisava a fronte. A palidez. E desmaiou.

O esforço para livrá"-lo do desmaio continuava\ldots{} A campainha tocou. Um
repórter. Mandado embora. Depois do desmaio a prostração. A campainha
tocou. Outro repórter. Mandado embora. A campainha tocou. O primeiro
repórter insistia. Mandado embora. Desordem. Criados comentando\ldots{}
Automóvel de prontidão. O motorista lia desatento uma passagem do
romance em folhetos \emph{A filha do enforcado}. O conde de Vareuse,
devido a velho ódio de família fora enforcado por um sobrinho. Apenas o
filho corcunda de Jacquot fiel criado do sr.\,de Plessis amigo íntimo do
conde presenciara o assassínio. Aconteceu porém que justamente na noite
do delito Germaine a filha do conde era roubada por uns ciganos
espanhóis. Isto se deu no reinado de Carlos V. Germaine tinha nesse
tempo apenas cinco anos. Ora o corcundinha irmão de leite do sobrinho
assassino hesitava ainda em contar o que vira quando é roubado também
pelos ciganos. Mas ele não conhecia Germaine. O procurador, ou coisa que
o valha, da imensa fortuna do conde de Vareuse, mestre Leonard, vendo a
condessa viúva enlouquecer com a perda da filha concebe um plano
diabólico. Apossa"-se da personalidade do conde de Vareuse com o qual
muito se parecia más línguas davam"-no mesmo como filho postiço do velho
pai do conde ainda vivo mas cego e paralítico numa velha propriedade no
Languedoc. O procurador pois apossa"-se de todos os papéis do conde e
muda"-se para a Inglaterra onde se domicilia. Atinge logo uma das mais
fulgurantes posições na elite londrina. Casa"-se com a filha de Lord
Chaney\footnote{Não confundir com Lon Chaney. {[}Nota \textsc{ma}{]}.} e tem
desta uma filha. Passam"-se doze anos. O filho do assassino do conde
então com vinte e três anos brilhantíssima inteligência parte numa
comissão diplomática para a Rússia. É nesse instante justamente que a
condessa de Vareuse que o procurador mandara para a casa duns antigos
apaniguados seus na Boêmia recobra a razão ao ouvir um lindo moço de
seus vinte anos mais ou menos e que aparentava grande riqueza e sangue
puro, viajante recém"-chegado na aldeia entoar uma balada. Ora o
interessantíssimo do caso é que essa balada fora composta pela própria
condessa, exímia tocadora de harpa que porém não a revelara a ninguém.
(A balada) Somente cantarolava"-a às vezes para adormentar a filha, que
era doentia e sofria de insônias. E se a condessa jamais cantava perto
de qualquer pessoa essa balada, era porque dizia a própria história
dela. Tratava"-se duma moça que se deixara levar pelos encantos dum
estudante e que diante da impossibilidade de casar com o namorado pois
era de grã nobreza (a condessa) entregara"-se voluntariamente a ele num
assomo de paixão. Nasceu um filho que a família encobrira e fizera
desaparecer. Nesse tempo Germaine com o corcundinha desesperadamente
apaixonado por ela conseguiram livrar"-se das garras dos ciganos e fugir
para a Itália num navio de vela pertencente a mercadores marselheses. No
mesmo navio seguia também um rapaz nobre italiano que fora chamado
urgentemente a Nápoles onde uma terrível conspiração se organizava entre
os membros duma sociedade secreta indiana, os Treze Irmãos da Pantera
Vermelha, para assassinar Carlos V. Ora o príncipe Lotti que tal era o
nome do moço viajante a bordo da Reine Marie estava disposto a se
dedicar pelo rei por gratidões de família que não interessam aqui. Eis
que a Reine Marie é atacada por piratas tunisianos. Prestes a
entregar"-se já. O príncipe defendia Germaine heroicamente tendo ao lado
o fiel Jean o corcundinha. Mas surge a todo pano velejando uma fragata
de guerra francesa. Fogem os piratas. A maruja da Reine Marie canta
vitória. Germaine e o príncipe Lotti pois que a guerra lhes revelou o
mútuo amor estão abraçados ouvindo as últimas palavras de Jean
agonizante. Jean que durante toda a vida se calara por não criar um
sentimento de ódio na alma pura de Germaine pretendendo ele só vingá"-la
mais tarde vê"-se obrigado agora a revelar tudo o que sabe. O príncipe
Lotti e Germaine ainda trêmulos de horror vão para bordo do navio de
guerra francês onde os recebe justamente quem! o filho do assassino do
pai de Germaine, o jovem diplomata que por desfastio se partira para a
Rússia por caminho que a fantasia aconselhava. Mas imediatamente o filho
do assassino concebe infinito amor por Germaine. Esta, o príncipe e o
filho do assassino descem em Gênova. E justamente para a hospedaria onde
vão está a condessa de Vareuse e o filho. No momento em que Germaine é
perseguida pelo filho do assassino e surge o irmão para defendê"-la um
criado vem conversar com o motorista.

--- Vamos almoçar. É quase meio"-dia.

O dr.\,Xis, que dedicação! sempre ao lado do doente.

Falara"-lhe longamente, persuasivamente. Contou"-lhe então toda a
aventura. Era a última esperança: dizer tudo. O dr.\,Xis disse tudo: o
desastre, a operação, a substituição de cérebros e descreveu"-lhe por fim
a fortuna dele, José, cérebro de José, agora moço rico feliz\ldots{}

Alberto abandonado sobre o leito como que ouvia e aceitava. Muito calmo.
Quando o operador parou maior momento Alberto ou José abanou a cabeça.

--- Não\ldots{} Sou José. Quando eu\ldots{} o outro agora me lembro estava
morrendo fiz uma promessa para S. Vito de contar tudo se salvasse. Estou
vivo. Sinto que estou vivo\ldots{} Mudei\ldots{} Não! não sou eu!\ldots{} Este não!\ldots{}
Sou o outro!\ldots{} Sou o outro!\ldots{} Sou o criminoso!\ldots{} Este é inocente!\ldots{}
não matou meus dois filhos\ldots{} Foi o outro, eu, José\ldots{} \emph{Dio!}\ldots{}

Soluçava horrorizado desesperado. Neste momento o dr.\,Xis viu o rosto do
dr.\,Xis refletido no espelho. Era um homem de trinta anos no máximo.
Ardido aventureiro mas trazia nos lábios abertos em pétalas de rosa
qualquer coisa dessa sensualidade que faz ser bom, ser nobre e
sentimental. Perturbado por esses vinhos parecia ao médico que os raios
da luz elétrica formavam na superfície do espelho uma grade de prisão.
Por trás da grade um moço. Inocente?\ldots{} Criminoso?\ldots{} Tão linda a
operação! mas o cérebro é que sente\ldots{} que manda\footnote{Lombroso:~\emph{Criminologia
  degli irresponsabili}. t. 11, p. 240; F. Treves, Milano. {[}Nota
  \textsc{ma}{]}.} mas o corpo\ldots{} aviador\ldots{} avião\ldots{} memória muscular o
incidente do automóvel\ldots{} é melhor\ldots{} \textsc{é melhor}!\ldots{} sim, é melhor.
Acaba"-se duma vez\ldots{}

E o dr.\,Xis pôde tirar os olhos do dr.\,Xis porque firmara a decisão.
Telefonou para o aeródromo. Mandou ordens ao motorista.

--- Como vai?\ldots{}

--- Alb\ldots{} ele está calmo agora.

--- O doutor precisa tomar alguma coisa\ldots{} Vinte e duas horas já\ldots{}

--- Aceito um café\ldots{} café bem forte.

--- Não quer uma almofada? doutor\ldots{} Passar mais uma noite assim! Como
lhe poderemos pagar tanta dedicação!\ldots{}

--- Não fale nisso, minha senhora. Quero muito bem Alberto\ldots{} Estimo"-o
muito (aos arrancos) muito mesmo\ldots{} como\ldots{} Porque, minha senhora, na
minha profissão há momentos maravilhosos\ldots{} Sentir"-se diante dum homem
moço ainda que morrerá por certo\ldots{} e confiante orgulhoso diante da
fatalidade\ldots{} combatê"-la\ldots{} vencê"-la pela inteligência\ldots{} oh! como eu o
amo\ldots{} minha senhora\ldots{} como a filho!\ldots{} sim, perdão, como se fosse meu
filho também!\ldots{}

E escarninhas brilhantes alegres lépidas fugiram dos olhos do dr.\,Xis as
duas primeiras lágrimas da sua cirurgia.

--- Amanhã tentarei uma prova\ldots{} uma prova decisiva! A senhora verá!
Alb\ldots{} ele já aceita o que eu digo\ldots{} As roupas de aviador estão aqui?

--- Guardava"-as no aeródromo\ldots{}

--- Está bem.

O dr.\,Xis inflexivelmente mau para consigo escrevendo passeando fumando
contou o tempo até seis da manhã.

--- Acorda\ldots{} meu rapaz!

Como no dia antecedente Alberto se vestiu mais ou menos bem. Começava
sempre certo e firme. Depois invariavelmente na continuação dos gestos
parava indeciso. José não sabia onde estavam as botinas. Indicava"-as o
``imprudente e glorioso cientista''.\footnote{\emph{Gazeta}~desse dia, 22
  de março de 1931. {[}Nota \textsc{ma}{]}.} Alberto continuava certo e firme.

--- Seu doutor, vamos embora!

--- Vamos!

O auto esperava à porta.

--- Para o aeródromo.

O caudron de Alberto, 120 cavalos, riscava uma sombra de avantesma na
relva aguda do prado. O mecânico esperava. José admirado deixou"-se
vestir. Menos admirado deixou"-se sentar no aeroplano. As mãos ágeis
hábeis manobraram a máquina. O mecânico impulsionava a hélice lustrosa.
O dr.\,Xis entrava para o lugar do passageiro\ldots{} O caudron deslizou subiu
numa linha oblíqua macia\ldots{} Os dois ``ilustres representantes da ciência
e do esporte paulista''\footnote{\emph{Correio Paulistano}, 23 de março
  de 1931. {[}Nota \textsc{ma}{]}.} foram se espedaçar muito longe nos campos
vazios.

\section*{nota}

Este conto é plagiado do ``Avatara'' de Teófilo Gautier que eu
desconheceria até hoje sem a bondade do amigo que me avisou do plágio.
Mas como geralmente acontece no Brasil o plágio é melhor que o original.
Quanto a Germaine conseguiu casar com o príncipe Lotti depois de mais
vinte e três fascículos a quinhentos réis cada.

\chapter{Galo que não cantou}

\hfill{}\emph{a Rubens de Moraes}\bigskip

\noindent{}Arlindo Teles --- Telinho como lhe chamavam no lar --- casara"-se aos 25
anos. Era lá das bandas de Pinda onde seus pais tinham vivido como
agregados de parentes de riqueza e mais inteligente atividade. Órfão sem
mais recursos que mãos inermes e vontade bruxuleante, viera para São
Paulo onde lhe prometiam casa e dinheiro. Lendo correntemente as
palavras mais comuns da língua, escrevendo letra gorda em que se
confessava toda a candura da alma escassa, multiplicando somando
subtraindo porventura até os bilhões desembarcou na capital ao aproximar
dos vinte anos. Fugindo à orfandade e penúria acolhia"-se como os pais à
asa dum tio --- aliás menos galinha"-mãe que usurário e aproveitador.

Entrando para o escritório do comerciante lobrigou sem grande labor nem
fadigas a epiderme da escrituração mas em geral só aproveitavam dele os
donaires da caligrafia e a rijeza das pernas. Copiava cartas, levava a
correspondência ao correio, dava recados percebendo cinquenta por mês.
Que bom rapaz!

Caseiro moderado no comer e econômico, indo nos raros acessos de
liberalidade até a xícara de café paga ao amigo (retribuição de
gentileza mais notável) soube cair nas graças de viúva mais ou menos
rica. Dona Cremildes vendo"-o simples são mansueto e herdeiro de outros
méis deu"-lhe gostosamente a filha.

Telinho jamais folheara um desses livros de retórica que ensinam as
artes do bem escrever portanto mais que provável sua total ignorância da
força e da eloquência das antíteses mas tinha sem dúvida algum refolho
de alma onde ardesse a lâmpada da intuição artística pois ignaro das
antíteses tinha por elas intuitivo amor. Provou"-o casando com Jacinta. E
Jacinta era a antítese de Telinho. Alta e magra machucava o espaço com
as anfractuosidades dos membros e no rosto empinava o recorte bélico dum
nariz sem fim mui digno de figurar entre as setas do deus menino.

Camões num soneto imortal conta que do magano deus acoitado nuns olhos
recebeu as feridas incruentas da paixão. Raimundo Correia não menos foi
ferido pelo deus magano desta vez ajudado por dois lábios sonorosos.
Jacinta se soubera de poética e teogonias lançara como Schiller sobre a
morte dos deuses, especialmente de Eros, lástimas e imprecações. E
desejara pulsassem mais no Telinho além do gosto artístico pelas
antíteses os estos da inspiração pois bem pudera surgir do nariz dela as
galas cavalheirescas dum soneto a primor.

Mas não foi preciso seta nem deus. Jacinta mais prática, sabendo
instintivamente talvez que o começado em verso não termina em bênçãos
matrimoniais utilizou"-se do nariz para uma operação piscativa: fisgou
com ele o Telinho. Assim o que troveiros de antanho levariam dez anos e
quatorze versos para cantar realizou com um só verbo, recendente além
disso do aperitivo fartum de bagres e piracanjubas: pescou o Telinho.

Jacinta não expandia bondade. Tinha porém vago encanto. Talvez o encanto
de toda virgem moça. Talvez dos cabelos negros crespos.

Telinho era a antítese dela. Baixo e gordo. Não se lhe via sequer o
desenho da carcaça enluvada que estava na gelatina da carne mole. No
meio da cara aberta mal se arrebitava como ponto timidamente róseo o
embrião dum nariz. Além disso os cabelinhos pardos já lhe começavam a
rarear no cocuruto oval.

Casaram em manhã de neblina com missa prédica lágrimas de mãe ironias de
assistentes almofadas para ajoelhar, meninas de filó carregando a cauda
da nubente. Nubente simpática receosa, quase bela nesse dia.

Depois de nove meses um filho; no fim de outro ano outro filho; terceiro
ano terceiro filho\ldots{} Mas é preciso miudear o que aconteceu antes dos
filhos.

Arlindo ao casar não trabalhava mais no escritório do tio, copiando
cartas e fazendo somas pequeninas pouco maiores que a mesada. Com a
especial piedade da futura sogra associara"-se a um primo também moço e
necessitado de arrumar vida. No vazio da antiga rua do Rosário (o caso
foi na caudinha do século passado, ali por 1899) montaram os dois sócios
uma loja de duas portas. Loja muito síria onde aos olhos do transeunte
se expunham fazendas vistosas de nenhum luxo e muitos chiquismos de
armarinho burguês.

Sentou"-lhes Fortuna ao portal e Telinho pôde trazer para casa no
primeiro mês de lua a então gorda bolada de duzentos mil"-réis. E com ela
bem embrulhado no jornal um esquisito vidro de cheiro: \emph{Cuir de
Russie}.

Extasiara"-se à contemplação daquelas letras francesas que deviam contar
tantas coisas doces e finas. \emph{Russie}!\ldots{} E murmurava à portuguesa:
Rú"-ssi"-e\ldots{} Decerto queria dizer Rússia que sabia ser um país muito
longe por detrás do mar, acamado de neve devastado por ursos brancos,
Nossa Senhora!\ldots{} Telinho não era positivamente burro, aprendia até
relativamente rápido, que habilidade para trabalhos caseiros! arranjar o
pé do sofá, endireitar a luz, torcer lã para os cordões de sapatinhos
mas --- coisa comum a toda gente que imagina --- sempre acreditara que a
Rússia era habitada por fortes negros desnudos de olhos em brasa.
Consequência talvez de antítese aterrorizada dos ursos brancos\ldots{} Fossem
agora dizer"-lhe que lourejava naquelas terras de setentrião a mais negra
de todas as raças brancas!\ldots{} Sorriria incrédulo. Continuaria a povoar a
estepe nívea com os negros da própria imaginativa. Não sei se teria
razão. Mas correm mundo assim tantas rússias!

Entrando em casa antegozava a alegria da mulher. Como ficaria linda
cheirando a Cuír de Rússie!\ldots{} Encontrou"-a sentada à cama no quarto
remendando meias.

--- Imagine o que trago para você, hoje! gritou da porta com as mãos
escondidas.

--- É o dinheiro da loja.

--- Não! Falo de presente.

--- Então você não trouxe o dinheiro da loja!

--- Já está aqui. Falo\ldots{}

--- Quanto?

--- Duzentos.

--- É melhor eu guardar. Você é tão palerma, vai perder tudo e daí sim!

--- Mas posso precisar, Jacinta!

--- Ora essa! fique com uns cinco mil"-réis. Quando precisar mais, pede.
Me dê o dinheiro.

E as notas lá foram parar numa gavetinha de cômoda. Bem fechada.

Arlindo ficara muito aborrecido. Desagradava"-lhe viver sem dinheiro.
Necessitava de o ter no bolso. Tranquilizava"-se com ele como criança que
no meio da multidão caminha destemerosa segurando a mão de pessoa mais
velha. Tinha frequentemente ideias destas: Se cair e me machucar posso
tomar um tílburi e ir para casa. Se sentir sede posso tomar uma
limonada\ldots{}

Mas era perder a esperança. Lá estavam os mil"-réis bem dobradinhos numa
caixa escarlata de veludo numa gaveta fechada. E a chave dessa gaveta
pendia da cintura de mulher enérgica. Desistiu por esta vez de conservar
a mensalidade jurando intimamente que no próximo 31 guardaria custasse o
que custasse o dinheiro. E afinal, pensou rápido, Jacinta tinha razão:
não precisava de tanto cobre.

--- Que é isso que você traz aí?

Era o presente. Apagou"-se"-lhe a sombra. Num jeito rápido escondendo o
embrulho:

--- Adivinhe!

--- Já sei: é sabão.

--- Não senhora, não adivinhou! e desmanchava"-se numa gargalhada.

Mas Jacinta já se apropriara do embrulho e descobrira o vidro.

--- Ah, é perfume\ldots{} Obrigada\ldots{} Quanto custou?

--- Nada, então! Tirei lá da casa.

--- Mas vocês podem tirar assim qualquer coisa da loja?!

--- Posso, ora sebo! Pois tudo aquilo não é meu?!

Jacinta deixou"-se levar pelo suasivo da resposta. Até o marido entrara
com duas partes para a sociedade!\ldots{}

--- Então por que não trouxe uns lenços? Seria mais proveitoso\ldots{} Ou
meias\ldots{} Você fura demais suas meias! Ande com mais cuidado. Nunca vi
homem mais arara: vive a dar topadas\ldots{}

No fim do segundo mês Telinho trouxe as meias. No terceiro os lenços. Já
então a mulher se acostumara a lhe determinar o que traria para casa.
Por fim nem pedia os lenços hoje e esperar trinta dias para pedir as
meias: eram meias e lenços na mesma ocasião.

Por delicadeza o primo, rapaz tímido e de humildes descorajadas ambições
não descontara o vidro de cheiro dos lucros de Telinho. Por timidez
invencível e até se é possível vergonha, pelo outro ou do outro,
continuou a fazê"-lo com lenços meias metros de chita e mesmo a peça de
morim. E como para si por quase idiota honestidade não se dispunha a
praticar o mesmo vinha para casa furioso, desabafando com a esposa ao
jantar. Depois, passada a erupção, planejava mil meios de terminar com
aquelas meias. Preparava"-se para falar ao primo, decorava períodos,
incomodava"-se com as entrelinhas da frase --- não fosse molestar o sócio!
--- construía toda uma arenga pejada de desculpas e, quando no dia
seguinte por volta das nove horas Telinho entrava na loja, sorridente e
honesto com ``bons"-dias'' tão virginais tão dúlcidos, lá da caixa o
outro entressaindo das cifras respondia miúdo à saudação, pigarreava e
não dizia nada. Há timidezes intransponíveis. Delas estava rico o primo
de Arlindo.
É ordem das sociedades que uns ajoelhem para que os outros lhes subam às
costas.

Arlindo, que lhe honre isto a consciência, por leviana vaidade apenas
dissera poder tirar tudo da loja como se fosse apenas seu. Ao mando da
mulher, com receio e desgosto se apropriara dos lenços. Pouco a pouco
porém esvaíram"-se os receios e com a maior naturalidade excluía do bem
comum o que a mulher desejava. Já nos fins da sociedade --- isto cinco
anos mais tarde --- nem esperava o último dia do mês: em qualquer tempo
tirava rendas e sabonetes.

Os negócios no entanto iam bem. Arlindo falava com orgulho na ``minha
loja'' e nela passava os dias sentado à frente do balcão, chapéu
sagrando a calva incipiente, uma perna horizontalmente dobrada sobre a
outra, olhando o vaivém de fora e a tutear os fregueses. O primo lá da
mesa das cifras olhava"-o de quando em quando com luzes de ira nos olhos,
mastigando a caneta. Mas Telinho não sabia fazer coisa alguma. Era
natural que agisse assim. Quando o freguês era de alta roda Telinho o
servia, ele próprio. E continuava levando cartas ao Correio\ldots{}

Em casa espraiava"-se a abundância, não a felicidade da paz. Jacinta
tomara aos poucos conta de tudo. Esse tudo compreendia Arlindo. O
espírito prático e ambicioso dela, inteligente e arguto dava"-lhe sobre o
marido real ascendência. Aconselhava"-o nos projetos da loja, rescindia
contratos, imaginava outros, dirigia passos e mãos do marido,
indagava"-lhe das horas gastas fora, proibia"-lhe a saída, regrava"-lhe os
acessos de afeição. Arlindo tentou protestar. Jacinta em dura voz
protestou contra o protesto. E Telinho entregou as ventas a argolar.
Conto como foi.

Nos primeiros tempos quando não redarguia às imposições da esposa
consolava"-se pensando consigo que eram os primeiros tempos. Fazia tão
pouco que estavam casados!\ldots{} Melhor deixasse passar mais um mês ou
dois\ldots{} Então sim: mostraria que naquela casa quem cantava era o galo.
Passados dois meses mais um terceiro se escoou. E assim até que veio
pontual o período da gravidez. Era preciso deixá"-la sossegada, coitada!
ia sofrer tanto!\ldots{} Madama Assunta recomendara descanso. Depois do parto
ele tomaria conta da casa\ldots{} Coitada!\ldots{}

Aos olhos de Telinho vinham sorrir lágrimas enternecidas. Vivia num
encantamento. Teria um filho que era dele! Aquela manifestação de
masculinidade abrandava o afelear do grilhão. Mas a cada nova intimação
ou ordem da mulher mais lhe engrossava no espírito a ideia de após o
parto manobrar sozinho as difíceis rédeas do lar. Ora sebo! a casa era
dele. Quem mandava era ele. Depois do parto!\ldots{}

Afinal, velha manhã de terça"-feira um quinze de março, arranhou a casa o
choro da criatura nova. Telinho por mais de três horas ajoelhado junto à
cama olhos imersos nos lençóis, guardando entre os dedos a mão gelada da
mulher, derramara mais suores e lágrimas que fonte de cordilheira.
Sofreu muito mesmo. Aniquilou"-se. Emagreceu. E quando depois junto ao
leito onde pálida Jacinta e Jacintinha repousavam conversava com a sogra
e a parteira, confessou escorrendo as mãos trementes pelo ápice da
barriga que sofrera as mesmas dores da mulher. Parecia"-lhe que
Jacintinha nascera unicamente dele. Ai, dor!\ldots{}

Por uma semana adorou Jacinta e filha. Não saiu do quarto. Não trocou
roupa. Não se lavou. Quando lá surgiu na loja outra vez, os negócios
corriam uniformemente bem. Quis inteirar"-se de tudo. Agora sim,
precisava trabalhar muito. Tinha um nobre dever a cumprir. E foi
interrompendo a série de contas e de notas com litanias de peripécias e
rosários gozosos às graças de Jacintinha que chegou ao fim das
informações do primo. Não compreendera não ouvira coisa alguma. Estava
bem! Estava tudo muito bem! Mas ansiara por chegar até a loja porque
deixá"-la assim\ldots{} sem ele! E tinha de voltar para casa. Como irá
passando Jacintinha, minha filha?!\ldots{} Jacintinha!\ldots{}

Jacinta logo se tirara da cama e pusera"-se com ímpeto de mãe nova a
tratar da criança. Ninguém tinha licença de mudar uma fralda, só ela.
Lavava com segura habilidade a pequenina, curava"-lhe o umbigo, espargia
montanhas de pó no corpinho rubro, vestia"-o no azul das flanelas,
perfumava"-lhe a \emph{Cuir de Russie} as longas mantas debruadas de seda
e adormentava a filha, com o calor das palavras mais profundamente
maternais. Quando Telinho na ponta dos pés entrava da loja perguntando
com os olhos por Jacintinha, Jacinta recomendava"-lhe silêncio com tanta
veemência que às vezes chegava a acordar a criança. Então quem recebia
as admoestações ásperas e sofria"-lhe a descomponenda? Era, quem mais?
Telinho o só culpado.

E Telinho recordou"-se que chegara enfim o tempo de assumir a direção do
lar. Mas Jacinta ainda estava tão fraca! Contrariá"-la? Podia influir no
leite. Mais tarde, quando Jacintinha tivesse uns duzentos dias\ldots{} Então
sim! Havia de mostrar que não era nenhum pai"-gonçalo. Fazia mais aquele
sacrifício pela filha, Jacintinha!\ldots{}

E os dias passavam. Telinho inteirava"-se cada vez mais de sua
subalternidade. A consciência da própria fraqueza acirrava"-lhe por tal
forma a sensibilidade que duas ou três vezes por dia era todo cóleras
abafadas. E tudo ia tão bem! O progresso exterior e suas galas e mais
suas facilidades contrastavam dolorosamente com a decadência de alma em
que se via. Rebaixado assim! Não! E nos próprios atos caseiros em que
quem manda é a mulher Telinho via espezinhamentos à sua dignidade de
senhor e macho. Desconhecia gradações e meios"-termos, ignorava
compensações. O objeto disposto por ele e descolocado pela mulher
parecia"-lhe degradante humilhação. Mas sofreria muito com isso?
Tornara"-se apenas irritadiço, pensava um pouco mais, perdera uma quarta
parte do bom"-humor. Da completa paz anterior passara a viver vida de
gozos intermitentes em que os eclipses parciais da calma dilatavam"-se
apenas o eco dum pensamento ou dum gesto. O que mais lhe doía era se
alguma pessoa presenciava uma dessas ocasiões em que lhe cumpria
obedecer às ordens da esposa. Então remordia"-se interiormente e jurava
que aquilo havia de acabar. Mas saía dali já plácido já Telinho
sorridente referindo ao primeiro que encontrava, com a serenidade duma
importância repleta de si mesma que fizera a mulher executar tal ação, a
tal parte a mandara, aconselhara"-a por esta forma quando na verdade ele
era o executante, ele o mandado, ele o aconselhado. E num andantino
discreteava com a voz das coisas importantes sobre o quão difícil é
mandar e agir na sociedade quando se é negociante reto, nobre esposo e
legítimo pai.

Posso assegurar que a sogra era sua maior defensora, a única mesmo
contra as investidas de Jacinta. Desde o princípio talvez mesmo antes do
princípio com a sagacidade divinatriz das mães inteirara"-se dona
Cremildes da orientação que tomaria o lar da filha. Não lhe desagradou
ser mãe de tal segura energia. Vieram porém horas de pensar mais
acertado e concordara que a posição dum marido não era positivamente
aquela. Morando na mesma casa foi contínua testemunha das vergonhas do
Telinho. Vendo ocasião de intervir falou à filha. É porém a mais
verdadeira das verdades que Jacinta de nada se apercebera. Nem planejara
mandar. Mandava porque era índole sua mandar. Sem preconceitos. Sem
raciocínios. Mandava porque reconhecendo"-se inconscientemente superior e
mais forte sabia\ldots{} intuitivamente agora, que é dos superiores e dos
fortes mandar e não obedecer, dirigir e não aconselhar"-se. As
advertências maternas serviram só para que se inteirasse duma verdade
que até então não percebera e apressar"-lhe assim a conquista do lar e
dos negócios da família. Entre os negócios da família Jacinta incluiu
Telinho. Viu"-se o marido. Verificou a própria superioridade,
acostumou"-se a chamar Telinho de bobo, desdenhou dele e mandou. Agora se
uma dessas mulheres se põe a pensar em coisas de amor\ldots{} Mas Jacinta era
fria. O enclausurar"-se numa virgindade perpétua fora para ela o menos
pesado entre os exílios. Casara porque é costume casar. Pouco lhe
importava a escolha deste ou daquele. A virilidade do seu caráter fazia
dela homem entre homens. Todavia veremos que chegada a ocasião não se
esqueceu de ser inteiramente mulher.

Corriam os duzentos dias. Foi a época de maior exacerbação nos brios de
Telinho. Tempo em que existiu um pouco menos para o gosto de viver. Nem
repetia mais a promessa de retrucar às ordens da mulher ``quando
Jacintinha tivesse duzentos dias''. Remastigava em silêncio a erva ruim
da humilhação. Tinha azias de alma. Vencer ou\ldots{} Era preciso vencer. Nem
se passaram os duzentos dias.

Foi num lerdo crepúsculo de outubro. Depois do jantar. Jacinta, Telinho,
dona Cremildes. Com o palito a oscilar nos dentes espaçados, Arlindo
chegara à janela. Corria no roxo da tarde o hálito refrescante duma
brisa. Deram"-lhe vontades de sair. Foi buscar o chapéu.

--- Aonde você vai?

--- Sair um pouco. Já volto.

--- Você não vai, não.

Àquela proibição Telinho jogou o olhar à sogra na esperança dela não ter
ouvido e poder arrepiar caminho mas nos olhos que dona Cremildes mandara
à filha sobejavam para ele além da notícia nua de seu apoucamento firmes
alianças para uma revolta. Sentiu que estava no fim dos duzentos dias.
Exagitadamente após tantos duzentos dias de canga, perguntou o seu
primeiro, seu primeiríssimo ``por quê?''. Tão inesperada era a audácia
que Jacinta levantou os olhos para ele numa legítima admiração. Sem de
pronto compreender o fato, tal a anormalidade deste, numa baianada de
quem quer assegurar"-se de seus próprios músculos, escondendo uma
possível razão redarguiu como aço frio:

--- Porque não quero!

--- Mas a senhora não tem querer!

--- Quem é que não tem querer?! gritou ela alçando"-se na expressão
descomposta de fúria. Nos seus lábios trementes de cólera viera
sorrateira depor a sua linha de irresolução uma perplexidade.

--- A senhora! Quem é o marido nesta casa?

--- É você? e casquinhou: Bobo!

A situação abria"-se por tal forma extraordinária que Jacinta perdia a
faculdade de raciocínio. Se não, veria na figura oscilante de Telinho
não cólera e resolução mas covardia apavorida e idiota. E as risadas que
pretendera dar sair"-lhe"-iam sonoras e reais. Telinho em meio já daquele
Gólgota resolvera duma vez crucificar"-se. Berrou cuspindo o palito para
frente:

--- Bobo, se quiser mas hei de mandar! Veremos quem pode mais nesta casa
ou acabamos com isto! Ora sebo\ldots{} O que é que a senhora pensa! Se me
deixei mandar algum tempo foi\ldots{} foi\ldots{} por delicadeza! Mas agora quero
mandar e hei de mandar! Ora sebo!

Estava cada vez mais encorajado porque a réplica tardava. Enterrara
desmesuradamente o chapéu na cabeça como que se coroando rei daquele
Congo. Rebusnava com voz de matraca as frases primogênitas de energia
que nunca proferira na vida. Ebriou. Delirou no prazer de ouvir"-se e
ver"-se forte, senhor de sua e mais vontades. E pouco frequentado nos
requebros da oratória, valorizando os interstícios da fala com ora"-sebos
ofensivos fustigou os ouvidos de Jacinta com relhadas nunca ouvidas
mesmo dos lábios calejados dum orador pé de boi.

Jacinta viu"-se perdida. Lembrou"-se que era mulher. Golfou soluços de
assobio e murmúrios de trespasse e soltando um ``ai, mamãe!'' de
abandono e infelicidade estirou"-se no sofá largadamente para com mais
conforto representar a imagem da mulher batida.

Telinho então magnífico ante a queda da esposa. Descortinara afinal em
si mesmo uma força que não possuía. Tresvariava na vitória narcisando"-se
nos seus muques irreais. Discursou muito tempo. Aludes de objurgatórias
e inúteis afrontas ao amaro refrão de suspiros que eram uivos e chorares
gritados. E saiu.

Saiu atarantado. Na placidez monástica do pôr de sol andou sozinho.
Homem maluco? a dar punhadas na brisa\ldots{} Julgava tropeçar a cada passo
em Jacintas inermes. Paravam para vê"-lo. Riam. Só muito depois já na
doçura da noite Telinho começou a pensar.

As horas passavam rapidamente impiedosamente. As ruas estavam cheias de
famílias que findo o jantar vinham para o relento em busca de assunto.
Telinho embaralhava"-se no ruído das ruas com a ansiedade do que encontra
lenitivo. O barulho dos passos o tóróró dos carros o crepitar das
risadas a grita das crianças impedia"-lhe quase recordar. Dava graças a
Deus. Pensava acelerado. E a explosão com que delirara em frente da
mulher assumia agora proporções de tragédia. Insultara. Batera? Matara
talvez. Que esgotamento!\ldots{}

No largo do Palácio havia música. Arlindo ouviu música, muita música,
toda a música. Não escutava, deixava"-se assombrar na atoarda dos bombos
e dos saxofones. Maior porém era o chinfrim que lhe ia n'alma.
Encharcou"-se de multidão. Que consolo perder"-se na turba móvel
nulificado ignoto reassumindo as proporções de nada que sempre lhe
tinham ido tão bem!

Mas a música cessou. As famílias partiram. Pouco a pouco as últimas
janelas fecharam as pálpebras cansadas. As ruas adormeceram. Era preciso
voltar\ldots{} As brisas numa reviravolta feminina traziam uma névoa gelada e
úmida.

Telinho sentiu a nostalgia do leito. Sempre quente, sempre cômodo!
Também: que necessidade tivera de fazer tal sarseiro!\footnote{Salseiro.
  A substituição da letra ``l'' pela ``r'' busca reproduzir o som do
  dialeto caipira.} Podia ter saído sem dizer nada\ldots{} Uma zanguinha à
toa\ldots{} Passava logo. E ela tivera razão, olhe aí o tempo! E não:
insultara a esposa\ldots{} batera\ldots{} E agora por causa da raiva intempestiva
era aquilo: Jacinta muito bem adormecida nos cobertores e ele a se
molhar de neblina tonto de sono\ldots{} Burro!

Andou mais. Saudades de Jacintinha!\ldots{} As horas passavam lentamente
impiedosamente\ldots{} Chorou. Enraiveceu"-se. Vão pro diabo! Crescia"-lhe a
impaciência e o frio. Não voltaria!\ldots{} Resolveu voltar. Entraria muito
manso. Podia até ficar na sala de jantar dormindo no sofá\ldots{} Bem que lhe
sorrira a ideia de ir dormir num hotel mas\ldots{} e dinheiro?

Quando defrontou a rua em que morava tomou"-se de tal pavor que rodou
para trás. Decididamente não voltava! Havia de esperar sozinha toda a
noite. Bem feito! Experimentasse o que é falta de homem! Na casa dele
quem cantava era o galo\ldots{} Não voltaria! Se voltasse ela veria nisso\ldots{}
atchim!\ldots{} uma prova de submissão\ldots{} Isso nunca!
Nunca!\dotfill{}

\noindent{}\dotfill{}

\noindent{}\dotfill{}atchim\dotfill{}

\noindent{}\dotfill{}

Quando Telinho com mil paciências de ladrão deu volta à chave o relógio
avisava o silêncio que eram as duas da madrugada. Agradou"-lhe isso: as
badaladas ressoando disfarçavam o trreque da fechadura. Entrou na sala
de jantar. Doce calor na casa toda! Que melhor cama que um sofá!
Doer"-lhe"-ia talvez o corpo mas era uma noite só\ldots{} Depois\ldots{}

--- Telinho!

--- Que é!

--- Venha dormir na cama, seu bobo! Você vai passar a noite no
sofá?\ldots{}

Telinho foi.

\chapter{Briga das pastoras}

Chegáramos à sobremesa daquele meu primeiro almoço no engenho e embora
eu não tivesse a menor intimidade com ninguém dali, já estava
perfeitamente a gosto entre aquela gente nordestinamente boa,
impulsivamente generosa, limpa de segundos pensamentos. E eu me pus
falando entusiasmado nos estudos que vinha fazendo sobre o folclore
daquelas zonas, o que já ouvira e colhera, a beleza daquelas melodias
populares, os bailados, e a esperança que punha naquela região que ainda
não conhecia. Todos me escutavam muito leais, talvez um pouco
longínquos, sem compreender muito bem que uma pessoa desse tanto valor
às cantorias do povo. Mas concordando com efusão, se sentindo
satisfeitamente envaidecidos daquela riqueza nova de sua terra, a que
nunca tinham atentado bem.

Foi quando, estávamos nas vésperas do Natal, da ``Festa'' como dizem por
lá, sem poder supor a possibilidade de uma rata, lhes contei que ainda
não vira nenhum pastoril, perguntando se não sabiam da realização de
nenhum por ali.

--- Tem o da Maria Cuncau, estourou sem malícia o Astrogildo, o filho
mais moço, nos seus treze anos simpáticos e atarracados, de ótimo
exemplar ``cabeça chata''.

Percebi logo que houvera um desarranjo no ambiente. A srª dona Ismália,
mãe do Astrogildo, e por sinal que linda senhora de corpo antigo, olhara
inquieta o filho, e logo disfarçara, me respondendo com firmeza
exagerada:

--- Esses brinquedos já estão muito sem interesse por aqui\ldots{} (As duas
moças trocavam olhares maliciosos lá no fundo da mesa, e Carlos, a
esperança da família, com a liberdade dos seus vinte e dois anos, olhava
a mãe com um riso sem ruído, espalhado no rosto). Ela porém continuava
firme: pastoril fica muito dispendioso, só as famílias é que faziam\ldots{}
antigamente. Hoje não fazem mais\ldots{}

Percebi tudo. A tal de Maria Cuncau certamente não era ``família'' e não
podia entrar na conversa. Eu mesmo, com a maior naturalidade, fui
desviando a prosa, falando em bumba meu boi, cocos, e outros assuntos
que me vinham agora apenas um pouco encurtados pela preocupação de
disfarçar. Mas o senhor do engenho, com o seu admirável, tão nobre
quanto antediluviano cavanhaque, até ali impassível à indiscrição do
menino, se atravessou na minha fala, confirmando que eu deveria estar
perfeitamente à vontade no engenho, que os meus estudos haviam
naturalmente de me prender noites fora de casa, escutando os
``coqueiros'', que eu agisse com toda a liberdade, o Carlos havia de me
acompanhar. Tudo sussurrado com lentidão e uma solicitude suavíssima que
me comoveu. Mas agora, com exceção do velho, o malestar se tornara
geral. A alusão era sensível e eu mesmo estava quase estarrecido, se
posso me exprimir assim. Por certo que a Maria Cuncau era pessoa de
importância naquela família, não podia imaginar o que, mas
garantidamente não seria apenas alguma mulher perdida, que causasse
desarranjo tamanho naquele ambiente.

Mas foi deslizantemente lógico todos se levantarem pois que o almoço
acabara, e eu senti dever uma carícia à srª dona Ismália, que não podia
mais evitar um certo abatimento naquele seu mutismo de olhos baixos.
Creio que fui bastante convincente, no tom filial que pus na voz pra lhe
elogiar os maravilhosos pitus, porque ela me sorriu, e nasceu entre nós
um desejo de acarinhar, bem que senti. Não havia dúvida: Maria Cuncau
devia ser uma tara daquela família, e eu me amaldiçoava de ter falado em
pastoris. Mas era impossível um carinho entre mim e a dona da casa,
apenas conhecidos de três horas; e enquanto o Carlos ia ver se os
cavalos estavam prontos para o nosso passeio aos partidos de cana,
fiquei dizendo coisas meio ingênuas, meio filiais à srª dona Ismália,
jurando no íntimo que não iria ao pastoril da Maria Cuncau. E como num
momento as duas moças, ajudando a criadinha a tirar a mesa, se acharam
ausentes, não resisti mais, beijei a mão da srª dona Ismália. E fugi
para o terraço, lhe facilitando esconder as duas lágrimas de uma
infelicidade que eu não tinha mais direito de imaginar qual.

O senhor do engenho examinava os arreios do meu cavalo. Lhe fiz um aceno
de alegria e lá partimos, no arranco dos animais fortes, eu, o Carlos, e
mais o Astrogildo num petiço atarracado e alegre que nem ele. A mocidade
vence fácil os malestares. O Astrogildo estava felicíssimo, no orgulho
vitorioso de ensinar o homem do sul, mostrando o que era boi, o que era
carnaúba; e das próprias palavras do mano, Carlos tirava assunto pra
mais verdadeiros esclarecimentos. Maria Cuncau ficara pra trás,
totalmente esquecida.

Foram três dias admiráveis, passeios, noites atravessadas até quase o
``nascer da bela aurora'', como dizia a toada, na conversa e na escuta
dos cantadores da zona, até que chegou o dia da Festa. E logo a imagem
da Maria Cuncau, cuidadosamente escondida aqueles dias, se impôs
violentamente ao meu desejo. Eu tinha que ir ver o pastoril de Maria
Cuncau. O diabo era o Carlos que não me largava, e embora já
estivéssemos amigos íntimos e eu sabedor de todas as suas aventuras na
zona e farras no Recife, não tinha coragem de tocar no assunto nem meios
pra me desvencilhar do rapaz. Nas minhas conversas com os empregados e
cantadores bem que me viera uma vontadinha de perguntar quem era essa
Maria Cuncau, mas se eu me prometera não ir ao pastoril da Maria Cuncau!
por que perguntar!\ldots{} Tinha certeza que ela não me interessava mais, até
que, com a chegada da Festa, ela se impusera como uma necessidade fatal.
Bem que me sentia ridículo, mas não podia comigo.

Foi o próprio Carlos quem tocou no assunto. Delineando o nosso programa
da noite, com a maior naturalidade deste mundo, me falou que, depois do
Bumba que viria dançar de tardinha na frente da casa"-grande, daríamos um
giro pelas rodas de coco, fazendo hora pra irmos ver o pastoril da Maria
Cuncau. Olhei"-o e ele estava simples, como se não houvesse nada. Mas
havia. Então falei com minha autoridade de mais velho:

--- Olhe, Carlos, eu não desejava ir a esse pastoril. Me sinto muito
grato à sua gente que está me tratando como não se trata um filho, e
faço questão de não desagradar a\ldots{} a ninguém.

Ele fez um gesto rápido de impaciência:

--- Não há nada! isso é bobagem de mamãe!\ldots{} Maria Cuncau parece que\ldots{}
Depois ninguém precisa saber de nada, nós voltamos todos os dias tarde
da noite, não voltamos?\ldots{} Vamos só ver, quem sabe se lhe interessa\ldots{}
Maria Cuncau é uma velha já, mora atrás da ``rua'', num mocambo,
coitada\ldots{}

E veio a noitinha com todas as suas maravilhas do Nordeste. Era uma
noite imensa, muito seca e morna, lenta, com aquele vaguíssimo ar de
tristeza das noites nordestinas. O bumba meu boi, propositalmente
encurtado pra não prender muito a gente da casa"-grande, terminara lá
pela meia"-noite. A srª dona Ismália se recolhera mais as filhas e a
raiva do Astrogildo que teimava em nos acompanhar. O dono da casa desde
muito que dormia, indiferente àquelas troças em que, como lhe escapara
numa conversa, se divertira bem na mocidade. Retirado o grande lampião
do terraço, estávamos sós, Carlos e eu. E a imensa noite. O pessoal do
engenho se espalhara. Os ruídos musicais se alastravam no ar imóvel. Já
desaparecera nalguma volta longe do caminho, o rancho do boi que
demandava a rua, onde ia dançar de novo o seu bailado até o raiar do
dia. Um ``chama'' roncava longíssimo, talvez nalgum engenho vizinho,
nalguma roda de coco. As luzes se acendiam espalhadas como estrelas,
eram os moradores chegando em suas casas pobres. E de repente, lá para
os lados do açude onde o massapê jazia enterrado mais de dois metros no
areão, desde a última cheia, depois de uns ritmos debulhados de ganzá,
uma voz quente e aberta, subira noite em fora, iniciando um coco bom de
sapatear.

\begin{verse}
Olê, rosêra,\\
Murchaste a rosa!\ldots{}
\end{verse}

Era sublime de grandeza. A melancolia da toada, viva e ardente, mas
guardando um significado íntimo, misterioso, quase trágico de desolação,
casava bem com a meiga tristeza da noite.

\begin{verse}
Olê, rosêra,\\
Murchaste a rosa!\ldots{}
\end{verse}

E as risadas feriam o ar, os gritos, o coco pegara logo animadíssimo,
aquela gente dançava, sapateava na dança, alegríssima, o coro ganhava
amplidão no entusiasmo, as estrelas rutilavam quase sonoras, o ar morno
era quase sensual, tecido de cheiros profundos. E era estranhíssimo.
Tudo cantava, Cristo nascia em Belém, se namorava, se ria, se dançava, a
noite boa, o tempo farto, o ano bom de inverno, vibrava uma alegria
enorme, uma alegria sonora, mas em que havia um quê de intensamente
triste. E um solista espevitado, com uma voz lancinante, própria de
aboiador, fuzilava sozinho, dilacerando o coro, vencendo os ares,
dominando a noite:

\begin{verse}
Vô m'imbora, vô m'imbora\\
Pá Paraíba do Norte!\ldots{}
\end{verse}

E o coro, em sua humanidade mais serena:

\begin{verse}
Olê, rosêra,\\
Murchaste a rosa!\ldots{}
\end{verse}

Nós caminhávamos em silêncio, buscando o pastoril e Maria Cuncau. Minha
decisão já se tornara muito firme pra que eu sentisse qualquer espécie
de remorso, havia de ver a Maria Cuncau. E assim liberto, eu me
entregava apenas, com delícias inesquecíveis, ao mistério, à grandeza,
às contradições insolúveis daquela noite imensa, ao mesmo tempo alegre e
triste, era sublime. E o próprio Carlos, mais acostumado e bem mais
insensível, estava calado. Marchávamos rápido, entregues ao fascínio
daquela noite da Festa.

A rua estava iluminada e muita gente se agrupava lá, junto à casa de
alguém mais importante, onde o rancho do boi bailava, já em plena
representação outra vez. Entre duas casas, Carlos me puxando pelo braço,
me fez descer por um caminhinho cego, tortuoso, que num aclive forte,
logo imaginei que daria nalgum riacho. Com efeito, num minuto de descida
brusca, já mais acostumados à escuridão da noite sem lua, pulávamos por
umas pedras que suavemente desfiavam uma cantilena de água pobre. Era
agora uma subida ainda mais escura, entre árvores copadas, junto às
quais se erguiam como sustos, uns mocambos fechados. Um homem passou por
nós. E logo, pouco além, surgiu por trás dum dos mocambos uma luz forte
de lampião batendo nos chapéus e cabeleiras de homens e mulheres
apinhados juntos a uma porta. Era o mocambo de Maria Cuncau.

Chegamos, e logo aquela gente pobre se arredou, dando lugar para os dois
ricos. Num relance me arrependi de ter vindo. Era a coisa mais
miserável, mais degradantemente desagradável que jamais vira em minha
vida. Uma salinha pequeníssima, com as paredes arrimadas em mulheres e
crianças que eram fantasmas de miséria, de onde fugia um calor de forno,
com um cheiro repulsivo de sujeira e desgraça. Dessa desgraça horrível,
humanamente desmoralizadora, de seres que nem sequer se imaginam
desgraçados mais. Cruzavam"-se no teto uns cordões de bandeirolas de
papel de embrulho, que se ajuntavam no fundo da saleta, caindo por
detrás da lapinha mais tosca, mais ridícula que nunca supus. Apenas
sobre uma mesa, com três velinhas na frente grudadas com seu próprio
sebo na madeira sem toalha, um caixão de querosene, pintado no fundo com
uns morros muito verdes e um céu azul claro cheio de estrelas
cor"-de"-rosa, abrigava as figurinhas santas do presépio, minúsculas, do
mais barato bricabraque imaginável.

O pastoril já estava em meio ou findava, não sei. Dançando e cantando,
aliás com a sempre segura musicalidade nordestina, eram nove mulheres,
de vária idade, em dois cordões, o cordão azul e o encarnado da
tradição, com mais a Diana ao centro. O que cantavam, o que diziam não
sei, com suas toadas sonolentas, de visível importação urbana, em que a
horas tantas julguei perceber até uma marchinha carioca de carnaval.

Mas eu estava completamente desnorteado por aquela visão de miséria
degradada, perseguido de remorsos, cruzado de pensamentos tristes,
saudoso da noite fora. E arrependido. Tanto mais que a nossa aparição
ali trouxera o pânico entre as mulheres. Se antes já trejeitavam sem
gosto, no monótono cumprimento de um dever, agora que duas pessoas
``direitas'' estavam ali, seus gestos, suas danças, se desmanchavam na
mais repulsiva estupidez. Todas seminuas com uns vestidos quase trapos,
que tinham sido de festas e bailes muito antigos, e com a grande faixa
azul ou encarnada atravessando do ombro à cintura, braços nus, os colos
magros desnudados, em que a faixa colorida apertava a abertura dos seios
murchos. Mais que a Diana central, rapariguinha bem tratada e nova, quem
chamava a atenção era a primeira figura do cordão azul. Seu vestido fora
rico há vinte anos atrás, todo inteirinho de lantejoulas brilhantes, que
ofuscavam contrastando com os outros vestidos opacos em suas sedinhas
ralas. Essa a Maria Cuncau, dona do pastoril e do mocambo.

Fora, isto eu sube depois, a moça mais linda da Mata, filha de um
morador que voltara do sul casado com uma italiana. Dera em nada (e aqui
meu informante se atrapalhou um bocado) porque um senhor de engenho,
naquele tempo ainda não era senhor de engenho não, a perdera. Tinha
havido facadas, o pai, o João Cuncau morrera na prisão, ela fora
mulher"-dama de celebridade no Recife, depois viera pra aquela miséria de
velhice em sua terra, onde pelo menos, de vez em quando, às escondidas,
o senhor de engenho, dinheiro não mandava não, que também já tinha pouco
pra educar os filhos, mas enfim sempre mandava algum carneiro pra ela
vender ou comer.

Maria Cuncau, assim que nos vira, empalidecera muito sob o vermelho das
faces, obtido com tinta de papel de seda. Mas logo se recobrara, erguera
o rosto, sacudindo pra trás a violenta cabeleira agrisalhada, ainda
voluptuosa, e nos olhava com desafio. Rebolava agora com mais cuidado,
fazendo um esforço infinito pra desencantar, do fundo da memória, as
graças antigas que a tinham celebrizado em moça. E era sórdido. Não se
podia sequer supor a sua beleza falada, não ficara nada. A não ser
aquele vestido de lantejoulas rutilantes, que pendiam, num ruidinho
escarninho, enquanto Maria Cuncau malhava os ossos curtos, frágil,
baixinha, olhos rubescentes de alcoolizada, naquele reboleio de pastora.

Quando dei tento de mim é que a coisa acabara, com uns fracos aplausos
em torno e as risadas altas dos homens. As pastoras se dispersavam na
sala, algumas vinham se esconder no sereno, passando por nós de olhos
baixos, encabuladíssimas. Carlos, bastante inconsciente, examinava
sempre os manejos da Diana moça, na sua feroz animalidade de rapaz. Mas
eu lhe tocava já no braço, queria partir, me livrar daquele ambiente sem
nenhum interesse folclórico, e que me repugnava pela sordidez. Maria
Cuncau, que, fingindo conversar com as mulheres da sala, enxugava muito
a cara, nos olhando de soslaio, adivinhou minha intenção. Se dirigiu
francamente pra nós e convidou, meio apressada mas sem nenhuma timidez,
com decisão:

--- Os senhores não querem adorar a lapinha!\ldots{}

Decerto era nisso que todas aquelas mulheres pensavam porque num segundo
vi todas as pastoras me olhando na sala e as que estavam de fora se
chegando à janelinha pra me examinar. Percebi logo a finalidade do
convite, quando cheguei junto da lapinha, enquanto o Carlos se atrasava
um pouco, tirando um naco desajeitado de conversa com a Diana. Os outros
assistentes também desfilavam junto ao presépio, parece que rezavam
alguma coisa, e alguns deixavam escorregar qualquer níquel num pires
colocado bem na frente do Menino"-Deus. Fingi contemplar com muito
respeito a lapinha, mas na verdade estava discutindo dentro comigo
quanto daria. Já não fora pouco o que o rancho do boi me levara, e aliás
as pessoas da casa"-grande estavam sempre me censurando pelo muito que eu
dava aos meus cantadores. Puxei a carteira, decidido a deixar uns vinte
mil"-réis no pires. Seria uma fortuna entre aqueles níqueis magriços em
que dominava uma única rodela mais volumosa de cruzado. Porém, se
ansiava por sair dali, estava também muito comovido com toda aquela
miséria, miséria de tudo. A Maria Cuncau então me dava uma piedade tão
pesada, que já me seria difícil especificar bem se era comiseração se
era horror.

Sinto é maltratar os meus leitores. Este conto que no princípio parecia
preparar algum drama forte, e já está se tornando apenas uma esperança
de dramazinho miserável, vai acabar em plena mesquinharia. Quando puxei
a carteira, decidido a dar vinte mil"-réis, a piedade roncou forte, tirei
com decisão a única nota de cinquenta que me restava da noite e pus no
pires. Todos viram muito bem que era uma nota, e eu já me voltava pra
partir, encontrando o olho de censura que o Carlos me enviava. O mal foi
um mulatinho esperto, não sei se sabia ler ou conhecia dinheiro, que
estava junto de mim, me devorando os gestos, extasiado. Não pôde se
conter, casquinou uma risada estrídula de comoção assombrada, e apenas
conseguiu ainda agarrar com a mão fechada a enorme palavra"-feia que
esteve pra soltar, gritou:

--- Pó\ldots{} cincoentão!

Foi um silêncio de morte. Eu estava desapontadíssimo, ninguém me via,
ninguém se movia, as pastoras todas estateladas, com os olhos fixos no
pires. Carlos continuava parado, esquecido da Diana que também não o via
mais, olhava o pires. E ele sacudia de leve o rosto para os lados, me
censurando.

--- Vamos, Carlos.

E nos dirigimos para a porta da saída. Mas nisto, aquela pastora do
cordão encarnado que estava mais próxima da lapinha, num pincho agílimo
(devia estar inteiramente desvairada pois lhe seria impossível fugir),
abrindo caminho no círculo apertado, alcançou o pires, agarrou a nota,
enquanto as outras moedinhas rolavam no chão de terra socada. Mas Maria
Cuncau fora tão rápida como a outra, encontrara de peito com a fugitiva,
foi um baque surdo, e a luta muda, odienta, cheia de guinchos entre as
duas pastoras enfurecidas. Nós nos voltáramos aturdidos com o caso e a
multidão devorava a briga das pastoras, também pasma, incapaz de
socorrer ninguém. E aqueles braços se batiam, se agarravam, se
entrelaçavam numa briga chué, entre bufidos selvagens, até que Maria
Cuncau, mordendo de fazer sangue o punho da outra, lhe agarrou a nota,
enfiou"-a fundo no seio, por baixo da faixa azul apertada. A outra agora
chorava, entre borbotões de insultos horríveis.

--- É da lapinha! que Maria Cuncau grunhia, se encostando na mesa,
esfalfada. É da lapinha!

Os homens já se riam outra vez com caçoadas ofensivas, e as pastoras se
ajuntando, faziam dois grupos em torno das briguentas, consolando,
buscando consertar as coisas.

Partimos apressados, sem nenhuma vontade ainda de rir nem conversar,
descendo por entre as árvores, com dificuldade, desacostumados à
escureza da noite. Já estávamos quase no fim da descida, quando um ruído
arrastado de animal em disparada, cresceu por trás de nós. Nem bem eu me
voltara que duas mãos frias me agarraram pela mão, pelo braço, me
puxavam, era Maria Cuncau. Baixinha, magríssima, naquele esbulho
grotesco de luz das lantejoulas, cabeça que era um ninho de cabelos
desgrenhados\ldots{}

--- Moço! ôh moço!\ldots{} me deixa alguma nota pra mim também, aquela é da
lapinha!\ldots{} eu preciso mais! aquela é da lapinha, moço!

Aí, Carlos perdeu a paciência. Agarrou Maria Cuncau com aspereza,
maltratando com vontade, procurando me libertar dela:

--- Deixe de ser sem"-vergonha, Maria Cuncau! Vocês repartem o dinheiro,
que história é essa de dinheiro pra lapinha! largue o homem, Maria
Cuncau!

--- Moço! me dá uma nota pra\ldots{} me largue, seu Carlos!

E agora se estabelecia uma verdadeira luta entre ela e o Carlos
fortíssimo, que facilmente me desvencilhara dela.

--- Carlos, não maltrate essa coitada\ldots{}

--- Coitada não! me largue, seu Carlos, eu mordo!\ldots{}

--- Vá embora, Maria Cuncau!

--- Olha, esta é pra\ldots{}

--- Não! não dê mais não! faço questão que\ldots{}

Porém Maria Cuncau já arrancara o dinheiro da minha mão e num salto pra
trás se distanciara de nós, olhando a nota. Teve um risinho de desprezo:

--- Vôte! só mais vinte!\ldots{}

E então se aprumou com orgulho, enquanto alisava de novo no corpo o
vestido desalinhado. Olhou bem fria o meu companheiro:

--- Dê lembrança a seu pai.

Desatou a correr para o mocambo.

\chapter{O besouro e a rosa}


Belazarte me contou:

Não acredito em bicho maligno mas besouro, não sei não. Olhe o que
sucedeu com a Rosa\ldots{} Dezoito anos. E não sabia que os tinha. Ninguém
reparara nisso. Nem dona Carlotinha nem dona Ana, entretanto já
velhuscas e solteironas, ambas quarenta e muito. Rosa viera pra
companhia delas aos sete anos quando lhe morreu a mãe. Morreu ou deu a
filha que é a mesma coisa que morrer. Rosa crescia. O português adorável
do tipo dela se desbastava aos poucos das vaguezas físicas da infância.
Dez anos, quatorze anos, quinze\ldots{} Afinal dezoito em maio passado. Porém
Rosa continuava com sete, pelo menos no que faz a alma da gente. Servia
sempre as duas solteironas com a mesma fantasia caprichosa da antiga
Rosinha. Ora limpava bem a casa, ora mal. Às vezes se esquecia do
paliteiro no botar a mesa pro almoço. E no quarto afagava com a mesma
ignorância de mãe de brinquedo a mesma boneca, faz quanto tempo nem sei!
lhe dera dona Carlotinha no intuito de se mostrar simpática. Parece
incrível, não? porém nosso mundo está cheio desses incríveis: Rosa
mocetona já, era infantil e de pureza infantil. Que as purezas como as
morais são muitas e diferentes\ldots{} Mudam com os tempos e com a idade da
gente\ldots{} Não devia ser assim, porém é assim, e não temos que discutir.
Mas com dezoito anos em 1923, Rosa possuía a pureza das crianças dali\ldots{}
pela batalha do Riachuelo mais ou menos\ldots{} Isso: das crianças de 1865.
Rosa\ldots{} que anacronismo!

Na casinha em que moravam as três, caminho da Lapa, a mocidade dela se
desenvolvera só no corpo. Também saía pouco e a cidade era pra ela a
viagem que a gente faz uma vez por ano quando muito, finados chegando.
Então dona Ana e dona Carlotinha vestiam seda preta, sim senhor! botavam
um sedume preto barulhando que era um desperdício. Rosa acompanhava as
patroas na cassa mais novinha, levando os copos"-de"-leite e as avencas
todas da horta. Iam no Araçá aonde repousava a lembrança do capitão
Fragoso Vale, pai das duas tias. Junto do mármore raso dona Carlotinha e
dona Ana choravam. Rosa chorava também, pra fazer companhia. Enxergava
as outras chorando, imaginava que carecia chorar também, pronto!
chororó\ldots{} abria as torneirinhas dos olhos pretos pretos, que ficavam
brilhando ainda mais. Depois visitavam comentando os túmulos
endomingados. Aquele cheiro\ldots{} Velas derretidas, famílias bivacando,
afobação encrencada pra pegar o bonde\ldots{} que atordoamento meu Deus! A
impressão cheia de medos era desagradável.

Essa anualmente a viagem grande da Rosa. No mais: chegadas até a igreja
da Lapa algum domingo solto e na Semana Santa. Rosa não sonhava nem
matutava. Sempre tratando da horta e de dona Carlotinha. Tratando da
janta e de dona Ana. Tudo com a mesma igualdade infantil que não implica
desamor não. Nem era indiferença, era não imaginar as diferenças, isso
sim. A gente bota dez dedos pra fazer comida, dois braços pra varrer a
casa, um bocadinho de amizade pra fulano, três bocadinhos de amizade pra
sicrano que é mais simpático, um olhar pra vista bonita do lado com o
espigão de Nossa Senhora do Ó numa pasmaceira lá longe, e de sopetão,
zás! bota tudo no amor que nem no campista pra ver se pega uma cartada
boa. Assim é que fazemos\ldots{} A Rosa não fazia. Era sempre o mesmo bocado
de corpo que ela punha em todas as coisas: dedos braços vista e boca.
Chorava com isso e com o mesmo isso tratava de dona Carlotinha.
Indistinta e bem varridinha. Vazia. Uma freirinha. O mundo não existia
pra\ldots{} qual freira! santinha de igreja perdida nos arredores de Évora.
Falo da santinha representativa que está no altar, feita de massa
pintada. A outra, a representada, você bem sabe: está lá no céu não
intercedendo pela gente\ldots{} Rosa si carecesse intercedia. Porém sem saber
por quê. Intercedia com o mesmo pedaço de corpo dedos braços vista e
boca sem mais nada. A pureza, a infantilidade, a pobreza"-de"-espírito se
vidravam numa redoma que a separava da vida. Vizinhança? Só a casinha
além, na mesma rua sem calçamento, barro escuro, verde de capim livre. A
viela era engulida num rompante pelo chinfrim civilizado da rua dos
bondes. Mas já na esquina a vendinha de seu Costa impedia Rosa de entrar
na rua dos bondes. E seu Costa passava dos cinquenta, viúvo sem filhos,
pitando num cachimbo fedido. Rosa parava ali. A venda movia toda a
dinâmica alimentar da existência de dona Ana, de dona Carlotinha e dela.
E isso nas horas apressadas da manhã, depois de ferver o leite que o
leiteiro deixava muito cedo no portão.

Rosa saudava as vizinhas da outra casa. De longe em longe parava um
minuto conversando com a Ricardina. Porém não tinha assunto, que que
havia de fazer? partia depressa. Com essas despreocupações de viver e de
gostar da vida, como é que podia reparar na própria mocidade! não podia.
Só quem pôs reparo nisso foi o João. De primeiro ele enrolava os dois
pães no papel acinzentado e atirava o embrulho na varanda. Batia pra
saberem e ia"-se embora tlindliirim dlimdlrim, na carrocinha dele. Só
quando a chuva era de vento, esperava com o embrulho na mão.

--- Bom"-dia.

--- Bom"-dia.

--- Que chuva.

--- Um horror.

--- Até amanhã. 

--- Até amanhã.

Porém duma feita, quando embrulhava os pães na carrocinha, percebeu Rosa
que voltava da venda. Esperou muito naturalmente, não era nenhum
malcriado não. O sol dava de chapa no corpo que vinha vindo. Foi então
que João pôs reparo na mudança da Rosa, estava outra. Inteiramente
mulher com pernas bem delineadas e dois seios agudos se contando na
lisura da blusa, que nem rubi de anel dentro da luva. Isto é\ldots{} João não
viu nada disso, estou fantasiando a história. Depois do século dezenove
os contadores parece que se sentem na obrigação de esmiuçar com
sem"-vergonhice essas coisas. Nem aquela cor de maçã camoesa amorenada
limpa\ldots{} Nem aqueles olhos de esplendor solar\ldots{} João reparou apenas que
tinha um malestar por dentro e concluiu que o malestar vinha da Rosa.
Era a Rosa que estava dando aquilo nele não tem dúvida. Alastrou um riso
perdido na cara. Foi"-se embora tonto, sem nem falar bom"-dia direito. Mas
daí em diante não jogou mais os pães no passeio. Esperava que a Rosa
viesse buscá"-los das mãos dele.

--- Bom"-dia.

--- Bom"-dia. Por que não atirou?

--- É\ldots{} Pode sujar.

--- Até amanhã.

--- Até amanhã, Rosa!

Sentia o tal de malestar e ia"-se embora.

João era quase uma Rosa também. Só que tinha pai e mãe, isso ensina a
gente. E talvez por causa dos vinte anos\ldots{} De deveras chegara nessa
idade sem contato de mulher, porém os sonhos o atiçavam, vivia mordido
de impaciências curtas. Porém fazia pão, entregava pão e dormia cedo.
Domingo jogava futebol no Lapa Atlético. Quando descobriu que não podia
mais viver sem a Rosa, confessou tudo pro pai.

--- Pois casa, filho. É rapariga boa, não é?

--- É, meu pai. 

--- Pois então casa! A padaria é tua mesmo\ldots{} não tenho mais filhos\ldots{} E
si a rapariga é boa\ldots{}

Nessa tarde dona Ana e dona Carlotinha recebiam a visita envergonhada do
João. Que custo falar aquilo! Afinal quando elas adivinharam que aquele
mocetão, manco na fala porém sereno de gestos, lhes levava a Rosa, se
comoveram muito. Se comoveram porque acharam o caso muito bonito, muito
comovente. E num instante repararam também que a criadinha estava ũa
mocetona já. Carecia se casar. Que maravilha, Rosa se casava! Havia de
ter filhos! Elas seriam as madrinhas\ldots{} Quase se desvirginavam no gozo
de serem mães dos filhos da Rosinha. Se sentiam até abraçadas, apertadas
e, cruz credo! faziam cada pecadão na inconsciência\ldots{}

--- Rosa!

--- Senhora?

--- Venha cá!

--- Já vou, sim senhora!

Ainda não sabiam si o João era bom mas parecia. E queriam gozar o
encafifamento de Rosa e do moço, que maravilha!

Apertados nos batentes da porta relumearam dezoito anos fresquinhos.

--- Rosa, olhe aqui. O moço veio pedir você em casamento.

--- Pedir o que!\ldots{}

--- O moço diz que quer casar com você.

Rosa fizera da boca uma roda vermelha. Os dentes regulares muito
brancos. Não se envergonhou. Não abaixou os olhos. Rosa principiou a
chorar. Fugiu pra dentro soluçando. Dona Carlotinha foi encontrar ela
sentada na tripeça junto do fogão. Chorava gritadinho, soluçava aguçando
os ombros, desamparada.

--- Rosa, que é isso! Então é assim que se faz!? Si você não quer, fale!

--- Não! Dona Carlotinha, não! Como é que vai ser! Eu não quero largar
da senhora!\ldots{} 

Dona Carlotinha ponderou, gozou, aconselhou\ldots{} Rosa não sabia pra onde
ir si casasse, Rosa só sabia tratar de dona Carlotinha\ldots{} Rosa pôs"-se a
chorar alto. Careceu tapar a boca dela, salvo seja! pra que o moço não
escutasse, coitado! Afinal dona Ana veio saber o que sucedia, morta de
curiosidade.

João ficou sozinho na sala, não sabia o que tinha acontecido lá dentro,
mas porém adivinhando que lhe parecia que a Rosa não gostava dele.

Agora sim, estava mesmo atordoado. Ficou com vergonha da sala, de estar
sozinho, não sei, foi pegando no chapéu e saindo num passo de
boi"-de"-carro. Arredondava os olhos espantado. Agora percebia que gostava
mesmo da Rosa. A tábua dera uma dor nele, o pobre!

Foi tarde de silêncio na casa dele. O pai praguejou, ofendeu a menina.
Depois percebendo que aquilo fazia mal ao filho se calou. No dia
seguinte João atirou o pão no passeio e foi"-se embora. Lhe dava de
sopetão uma coisa esquisita por dentro, vinha lá de baixo do corpo
apertando, quase sufocava e a imagem da Rosa saía pelos olhos dele
trelendo com a vida indiferente da rua e da entrega do pão. Graças a
Deus que chegou em casa! Mas era muito sem letras nem cidade pra
cultivar a tristeza. E Rosa não aparecia pra cultivar o desejo\ldots{} No
domingo ele foi um zagueiro estupendo. Por causa dele o Lapa Atlético
venceu. Venceu porque derrepentemente ela aparecia no corpo dele e lhe
dava aquela vontade, isto é, duas vontades: a\ldots{} já sabida e outra, de
esquecimento e continuar dominando a vida\ldots{} Então ele via a bola,
adivinhava pra que lado ela ia, se atirava, que lhe incomodava agora de
levar pé na cara! quebrar a espinha! arrebentasse tudo! morresse! porém
a bola não havia de entrar no gol. João naturalmente pensava que era por
causa da bola.

Rosa quando viu que não deixava mesmo dona Ana e dona Carlotinha teve um
alegrão. Cantou. Agora é que o besouro entra em cena\ldots{} Rosa sentiu uma
calma grande. E não pensou mais no João. 

--- Você se esqueceu do paliteiro outra vez!

--- Dona Ana, me desculpe!

Continuou limpando a casa ora bem ora mal. Continuou ninando a boneca de
louça. Continuou.

Essa noite muito quente, quis dormir com a janela aberta. Rolava
satisfeita o corpo nu dentro da camisola, e depois dormiu. Um besouro
entrou. Zzz, zzz, zzzuuuuuummmm, pá! Rosa dormida estremeceu à sensação
daquelas pernas metálicas no colo. Abriu os olhos na escureza. O besouro
passeava lentamente. Encontrou o orifício da camisola e avançava pelo
vale ardente entre morros. Rosa imaginou ũa mordida horrível no peito,
sentou"-se num pulo, comprimindo o colo. Com o movimento, o besouro se
despegara da epiderme lisa e tombara na barriga dela, zzz tzzz\ldots{} tz.
Rosa soltou um grito agudíssimo. Caiu na cama se estorcendo. O bicho
continuava descendo, tzz\ldots{} Afinal se emaranhou tzz"-tzz, estava preso.
Rosa estirava as pernas com endurecimentos de ataque. Rolava. Caiu.

Dona Ana e dona Carlotinha vieram encontrá"-la assim, espasmódica, com a
espuma escorrendo do canto da boca. Olhos esgazeados relampejando que
nem brasa. Mas como saber o que era! Rosa não falava, se contorcendo.
Porém dona Ana orientada pelo gesto que a pobre repetia, descobriu o
bicho. Arrancou"-o com aspereza, aspereza pra livrar depressa a moça. E
foi uma dificuldade acalmá"-la\ldots{} Ia sossegando sossegando\ldots{} de repente
voltava tudo e era tal"-e"-qual ataque, atirava as cobertas rosnava, se
contorcendo, olhos revirados, uhm\ldots{} Terror sem fundamento, bem se vê.
Nova trabalheira. Lavaram ela, dona Carlotinha se deu ao trabalho de
acender fogo pra ter água morna que sossega mais, dizem. Trocaram a
camisola, muita água com açúcar\ldots{}

--- Também por que você deixou janela aberta, Rosa\ldots{}

Só umas duas horas depois tudo dormia na casa outra vez. Tudo não. Dois
olhos fixando a treva, atentos a qualquer ressaibo perdido de luz e aos
vultos silenciosos da escuridão. Rosa não dorme toda a noite. Afinal
escuta os ruídos da casa acordando. Dona Ana vem saber. Rosa finge
dormir, desarrazoadamente enraivecida. Tem um ódio daquela coroca! Tem
nojo de dona Carlotinha\ldots{} Ouve o estalo da lenha no fogo. Escuta o
barulho do pão atirado contra a porta do passeio. Rosa esfrega os dedos
fortemente pelo corpo. Se espreguiça. Afinal levantou.

Agora caminha mais pausado. Traz uma seriedade nunca vista ainda, na
comissura dos lábios. Que negrores nas pálpebras! Pensa que vai
trabalhar e trabalha. Limpa com dever a casa toda, botando dez dedos pra
fazer a comida, botando dois braços pra varrer, botando os olhos na mesa
pra não esquecer o paliteiro. Dona Carlotinha se resfriou. Pois Rosa lhe
dá uma porção de amizade. Prepara chás pra ela. Senta na cabeceira da
cama, velando muito, sem falar. As duas velhas olham pra ela
ressabiadas. Não a reconhecem mais e têm medo da estranha. Com efeito
Rosa mudou, é outra Rosa. E uma rosa aberta. Imperativa, enérgica. Se
impõe. Dona Carlotinha tem medo de lhe perguntar se passou bem a noite.
Dona Ana tem medo de lhe aconselhar que descanse mais. E sábado porém
podia lavar a casa na segunda"-feira\ldots{} Rosa lava toda a casa como nunca
lavou. Faz uma limpeza completa no próprio quarto. A boneca\ldots{} Rosa lhe
desgruda os últimos crespos da cabeça, gesto frio. Afunda um olho dela,
portuguesmente, à Camões. Porém pensa que dona Carlotinha vai sentir. A
gente nunca deve dar desgostos inúteis aos outros, a vida é já tão cheia
deles!\ldots{} pensa. Suspira. Esconde a boneca no fundo da canastra.

Quando foi dormir teve um pavor repentino: dormir só!\ldots{} E si ficar
solteira! O pensamento salta na cabeça dela assim, sem razão. Rosa tem
um medo doloroso de ficar solteira. Um medo impaciente, sobretudo
impaciente, de ficar solteira. Isso é medonho! \textsc{é uma vergonha}! 

Se vê bem que nunca tinha sofrido, a coitada! Toda a noite não dormiu.
Não sei a que horas a cama se tornou insuportavelmente solitária pra
ela. Se ergue. Escancara a janela, entra com o peito na noite,
desesperadamente temerária. Rosa espera o besouro. Não tem besouros essa
noite. Ficou se cansando naquela posição, à espera. Não sabia o que
estava esperando. Nós é que sabemos, não? Porém o besouro não vinha
mesmo. Era uma noite quente\ldots{} A vida latejava num ardor de estrelas
pipocantes imóveis. Um silêncio!\ldots{} O sono de todos os homens, dormindo
indiferentes, sem se amolar com ela\ldots{} O cheiro de campo requeimado
endurecia o ar que parara de circular, não entrava no peito! Não tinha
mesmo nada na noite vazia. Rosa espera mais um poucadinho. Desiludida,
se deita depois. Adormece agitada. Sonha misturas impossíveis. Sonha que
acabaram todos os besouros desse mundo e que um grupo de moças caçoa
dela zumbindo: Solteira! às gargalhadas. Chora em sonho.

No outro dia dona Ana pensa que carece passear a moça. Vão na missa.
Rosa segue na frente e vai namorar todos os homens que encontra. Tem de
prender um. Qualquer. Tem de prender um pra não ficar solteira. Na venda
de seu Costa, Pedro Mulatão já veio beber a primeira pinga do dia. Rosa
tira uma linha pra ele que mais parece de mulher"-da"-vida. Pedro Mulatão
sente um desejo fácil daquele corpo, e segue atrás. Rosa sabe disso.
Quem é aquele homem? Isso não sabe. Nem que soubesse do vagabundo e
beberrão, é o primeiro homem que encontra, carece agarrá"-lo sinão morre
solteira. Agora não namorará mais ninguém. Se finge de inocente e
virgem, riquezas que não tem mais\ldots{} Porém é artista e representa. De
vez em quando se vira pra olhar. Olhar dona Ana. Se ri pra ela nesse
riso provocante que enche os corpos de vontade.

Na saída da missa outro olhar mais canalha ainda. Pedro Mulatão pára na
venda. Bebe mais e trama coisas feias. Rosa imagina que falta açúcar, só
pra ir na venda. É Pedro que traz o embrulho, conversando. Convida"-a pra
de"-noite. Ela recusa porque assim não casará. Isso pra ele é
indiferente: casar ou não casar\ldots{} Irá pedir.

Desta vez as duas tias nem chamam Rosa, homem repugnante não? Como
casá"-la com aqueles trinta"-e"-cinco anos!\ldots{} No mínimo, de trinta"-e"-cinco
pra quarenta. E mulato, amarelo pálido já descorado\ldots{} pela pinga, Nossa
Senhora!\ldots{} Desculpasse, porém a Rosa não queria casar. Então ela
aparece e fala que quer casar com Pedro Mulatão. Elas não podem
aconselhar nada diante dele, despedem Pedro. Vão tirar informações. Que
volte na quinta"-feira.

As informações são as que a gente imagina, péssimas. Vagabundo, chuva,
mau"-caráter, não serve não. Rosa chora. Há de casar com Pedro Mulatão e
si não deixarem, ela foge. Dona Ana e dona Carlotinha cedem com a morte
na alma.

Quando o João soube que a Rosa ia casar, teve um desespero na barriga.
Saiu tonto, pra espairecer. Achou companheiros e se meteu na caninha.
Deixaram ele por aí, sentado na guia da calçada, manhãzinha, podre de
bebedeira. O rondante fez ele se erguer.

--- Moço, não pode dormir nesse lugar não! Vá pra sua casa!

Ele partiu, chorando alto, falando que não tinha a culpa. Depois deitou
no capim duma travessa e dormiu. O sol o chamou. Dor"-de"-cabeça, gosto
rúim na boca\ldots{} E a vergonha. Nem sabe como entra em casa. O estrilo do
pai é danado. Que insultos! seu filho disto, seu não"-sei"-que"-mais,
palavras feias que arrepiam\ldots{} Ninguém imaginaria que homem tão bom
pudesse falar aquelas coisas. Ora! todo homem sabe bocagens, é só ter
uma dor desesperada que elas saem. Porque o pai de João sofre deveras.
Tanto como a mãe que apenas chora. Chora muito. João tem repugnância de
si mesmo. De"-tarde quando volta do serviço, a Carmela chama ele na
cerca. Fala que João não deve de beber mais assim, porque a mãe chorou
muito. Carmela chora também. João percebe que si beber outra vez, se
prejudicará demais. Jura que não cai noutra, Carmela e ele suspiram se
olhando. Ficam ali.

Ia me esquecendo da Rosa\ldots{} Conto o resto do que sucedeu pro João um
outro dia. Prepararam enxoval apressado pra ela, menos de mês. Ainda na
véspera do casamento, dona Carlotinha insistiu com ela pra que mandasse
o noivo embora. Pedro Mulatão era um infame, até gatuno, Deus me perdoe!
Rosa não escutou nada. Bateu o pé. Quis casar e casou. Meia que sentia
que estava errada porém não queria pensar e não pensava. As duas
solteironas choraram muito quando ela partiu casada e vitoriosa, sem uma
lágrima. Dura.

Rosa foi muito infeliz.

\chapter{Jaburu malandro}

Belazarte me contou:

Pois é\ldots{} tem vidas assim, tão bem preparadinhas, sem surpresa\ldots{} São
ver gaveta arranjada, com que facilidade você tira a cueca até no
escuro, mesmo que ela esteja no fundo! Mas vem um estabanado, revira
tudo, que"-dê cueca? --- Maria, você não viu a minha cueca listrada de
azul? --- Está aí mesmo, seu dotoire! --- Não está! Já procurei, não
está\ldots{} E é um custo a gente encontrar a cueca. Você se lembra do João?
Ara, se lembra! o padeiro que gostava da Rosa, aquela uma que casou com
o mulato\ldots{} Pois quando contei o caso, falei que o João não era homem
educado pra estar cultivando males de amor\ldots{} Sofreu uns pares de dias,
até bebeu, se lembra? e encontrou a Carmela que principiou a consolá"-lo.
Não durou muito se consolou. Os dois passavam uma porção de vinte
minutos ali na cerca, falando nessas coisas corriqueiras que alimentam
amor de gente pobre.

Ora a Carmela\ldots{} será que ela gostava mesmo do João? Difícil de saber.
Era moça bonita, isso era, desses tipos de italiana que envelhecem muito
cedo, isto é, envelhecem não, engordam, ficam chatas, enjoativas. Porém
nos dezenove, que gostosura! Forte, um pouco baixa, beiços tão
repartidinhos no centro, um trevo encarnado! Cabelo mais preto nem de
brasileira! Porém o sublime era a pele, com todos os cambiantes do
rosado, desde o róseo"-azul do queixo com as veinhas de cá pra lá
sapecas, até o rubro esplendor ao lado dos olhos, querendo extravasar
pela fronte nos dias de verão brabo. Filha de italiano já se sabe\ldots{}

Mas Carmela não tinha a ciência das outras moças italianas daqui.
Pudera, as outras saíam todo santo dia, frequentavam as oficinas de
costura, as mais humildes estavam nos curtumes, na fiação, que
acontecia? Se acostumavam com a vida. Não tinha homem que não lhes
falasse uma graça ou no mínimo olhasse pra elas daquele jeito que ensina
as coisas. Ficavam sabendo logo de tudo e até segredavam imoralidades
umas pras outras, nos olhos. Ficavam finas, de tanta grosseria que
escutavam. A grosseria vinha, pam! batia nelas. Geralmente caía no chão.
Poucas, em comparação ao número delas, muito poucas se abaixavam pra
erguer a grosseria. Essas se perdiam, as pobres! Si não casavam na
Polícia, o que era uma felicidade rara, davam nas pensões.

Nas outras a grosseria relava apenas, escorregando pro chão. Mas o
choque desbastava um pouco essa crosta inútil de inocência que reveste a
gente no começo. Ficavam sabendo, se acostumavam facilmente com o manejo
da vida e escolhiam depois o rapaz que mais lhes convinha, seleção
natural. Casavam e o destino se cumpria. De chiques e aladas, viravam
mães anuais; filho na barriga, filho no peitume, filho agarrado na
perna. Domingo iam passear na cidade, espandongadas, cabelo caindo na
cara. Não tinha importância, não. Os trabalhadores o que queriam era mãe
pros oito a doze filhos do destino.

Carmela não. Vizinhava com a padaria em casa própria. O pai afinal tinha
seus cobres de tanta ferradura ordinária que passara adiante, e tanta
roda e varal consertados. E, fora as duas menores que nem na escola inda
iam, o resto eram filhos, meia"-dúzia, gente bem macha trabalhando numa
conta. Dois casados já. Só um ninguém sabia dele, talvez andasse pelas
fazendas\ldots{} Sei que fora visto uma vez em Botucatu. Era o defeito físico
da família. Si o nome dele caía na conversa, a gente só escutava os
palavrões que o pai dizia, porca la miséria. Restava a metade de
meia"-dúzia, menores que Carmela, treze, quatorze e dezesseis anos, que
seguiam o caminho bom dos mais velhos.

Assim florescentes, todos imaginaram de comum acordo que Carmela não
carecia de trabalhar. Deram um estadão pra ela, bonita! O pai olhava a
filha e sentia uma ternura diferente. Pra esvaziar a ternura, comprava
uma renda, sapatos de pelica alvinha, fitas, coisas assim.

Padeiro portuga e ferreiro italiano, de tanta vizinhança, ficaram
amigos. Quando o Serafino Quaglia viu que a filha pendia pro João,
gostou bem. Afinal, padaria instalada e afreguesada não é coisa que a
gente despreze numa época destas\ldots{} Porém a história é que Carmela,
sequestrada assim da vida, apesar de ter na família uma ascendência que
a fazia dona em casa, possuía coração que não sabia de nada. O João era
simpático, era. Forte, com os longos braços dependurados, e o bigode
principiando, não vê que galego larga bigode!\ldots{} Carmela gostou do João.
Quando pediu pra ele que não bebesse mais, João se comoveu. Principiou
sentindo Carmela. As entrevistas na cerca tornaram"-se diárias. Precisão
não havia, ninguém se opunha, e um entrava na casa do outro sem
cerimônia, mas é sempre assim porém\ldots{} Não carece a gente ser de muitos
livros, nem da alta, pra inventar a poesia das coisas, amor sempre
despertou inspiração\ldots{} Ora você há de convir que aqueles encontros na
cerca tinham seu encanto. Pra eles e pros outros. Ali estavam mais sós,
não tinham irmãos em roda. Pois então podiam passar muitos minutos sem
falar nada, que é a milhor maneira de fazer vibrar o sentimento. Os que
passavam viam aquele par tão bonito, brincando com a trepadeira, tirando
lasca do pau seco\ldots{} Isso reconciliava a gente com a malvadeza do mundo.

--- Sabe!\ldots{} a Carmela anda namorando com o João!

--- Sai daí, você\ldots{} Vem contar isso pra mim!\ldots{} Pois se até fui eu que
descobri primeiro! 

Pam!\ldots{} Pam!\ldots{} Pam!\ldots{} Pam!\ldots{} Pampampam!\ldots{} toda a gente correu na
esquina pra ver. O carro vinha a passo.

\begin{quote}\parindent=0em
\textsc{grande circo bahia}

dos irmão Garcias!

Hoje! Serata de estrea! Cachorros e maccacos sabios!

Irmãos Fô"-Hi equilibristas!

Grandes números de actração mundial!

Apresentação de toda a Compania!

Todos os dias novas estreias!

O homem Cobra. Malunga, o elephante sabio!

Terminará a função a grande pantommima

\textsc{os saltheadores da calabria}

Tres palhaços e o tony Come Mosca

Evohé! Todos ao Grande Circo Bahia! Hoje!

(Esquina da rua Guaicurús)

Só 2\$000 --- Cadeiras a Quatro

Imposto a cargo do respeitável Publico!

Eviva!
\end{quote}

O circo Bahia vinha tirar um pouco o bairro da rotina do cinema. Pam!
Pam!\ldots{} Pam!\ldots{} Lá seguia o carro de anúncio entre desejos. Carmela foi
contar pro João que ela ia com os três fratelos. João vai também.

O circo estava cheio. Pipoca! Amendoim torrado!\ldots{} Batat'assat'ô
furrn!\ldots{} Vozinha amarela: Nugá! nugá! nugá!\ldots{} Dentadura na escureza:
Baleiro!\ldots{} Balas de coco, chocolate, canela!\ldots{} E a banda sarapintando
de saxofone a noite calma. Estrelas. Foram pras cadeiras, Carmela
alumeando de boniteza. O circo não vinha pobre nem nada!

--- Todos os números são bons, hein! Eu volto! Você?

Come"-Mosca quis espiar a caixa tão grande toda de lantejoulas, verde e
amarela, que os araras traziam pro centro do picadeiro, prendeu o pé
debaixo dela. Foi uma gargalhada com o berro que ele deu.

--- Volto também.

Música. O reposteiro escarlate se abriu. O artista veio correndo lá de
dentro, com um malhô todo de lantejoulas, listrado de verde e amarelo.
Era o Homem Cobra. Fez o gesto em curva, braços no ar, deformação do
antigo beijo pro público\ldots{} é pena\ldots{} tradição que já vai se perdendo\ldots{}
Tipo esquisito o Homem Cobra\ldots{} esguio! esguio. Assim de malhô, então,
era ver uma lâmina. Tudo lantejoula menos a cabeça, até as mãos! Feio
não era não. Esse gênero de brasileiro quase branco já, bem pálido.
Cabelo liso, grosso, rutilando azul. O nariz não é chato mais, mesmo
delicado de tão pequeninho. Aliás a gente só via os olhos, puxa! negros,
enormes! aumentados pelas olheiras. Tomavam a cara toda. Carmela sentiu
uma admiração. E um malestar. Pressentimento não era, nem curiosidade\ldots{}
malestar.

O número causou sensação. Já pra trepar na caixa só vendo o que o Homem
Cobra fez! caiu no tapete, uma perna foi se arrastando caixa arriba, a
outra, depois o corpo, direitinho que nem cobra! até que ficou em cima.
Parecia que nem tinha osso, de tão deslocado. Fez coisas incríveis! dava
nós com as pernas, ficava um embrulhinho em cima da caixa\ldots{} Palmas de
toda a parte. Depois a música parou, era agora! Ergueu o corpo numa
curva, barriga pro ar, pés e mãos nos cantos da caixa. Vieram os irmãos
Garcias, de casaca, e o Dr.\,Cerquinho tão conhecido, médico do bairro.
--- Olha o doutor Cerquinho! --- O doutor Cerquinho!\ldots{} Homem tão bom,
consultas a três milréis\ldots{} Quando não podia pagar, não fazia mal,
ficava pra outra vez. Os irmãos Garcias puxavam a cabeça do Homem Cobra,
houve um estalo no bombo da música e a cabeça pendeu deslocada,
balanceando. Trrrrrrrrr\ldots{} tambor. A cabeça principiou girando. Trrr\ldots{}
Meu Deus! girava rapidíssimo! Trrrrr\ldots{} ``Chega! Chega!'' toda a gente
gritavam. Trrrrr\ldots{} Foi parando. Os irmãos Garcias endireitaram a cabeça
dele e o Dr.\,Cerquinho ajudou. Quando acabaram, o moço levantou meio
tonto, se rindo. Foi uma ovação. Não sei quantas vezes ele veio lá de
dentro agradecer. Os olhos vinham vindo, vinham vindo, aquele gesto de
beijo deformado, partiu. As palmas recomeçavam. Carmela pequititinha,
agarrada no João, que calor delicioso pra ele! Virou"-se, deu um beijo de
olhos nela, francamente, sem"-vergonha nenhuma, apesar de tanto pessoal
em roda.

--- Coitado não?

--- Batuta!

No dia seguinte deu"-se isto: estavam almoçando quando a porta se abriu,
Pietro! Era um ingrato, era tudo o que você quiser, mas era filho. Foi
uma festa. Tanto tempo, como é que viera sem avisar! como estava grande!
Pois fazem seis anos já!

--- Meu pai desculpa\ldots{}

O velho resmungou, porém o filho estava bem vestido, não era vagabundo,
não pense, estudara. Sabia música e viera dirigindo a banda do circo,
foi um frio. O velho desembuchou logo o que pensava de gente de circo.
Então Pietro meio que zangou"-se, estavam muito enganados! olhem: a moça
que anda na bola é mulher do equilibrista, a amazona se casara com o
Garcia mais velho, Dolores, uma uruguaia. Gente honesta, até os dois
japoneses. Todos espantados.

--- Meu pai, o senhor vai comigo lá no circo pra ver como todos são
direitos. Eu mesmo, si não casei até agora é porque nesta vida, hoje
aqui, amanhã não se sabe onde, inda não encontrei moça de minha
simpatia. E você, Carmela?

Ela sorriu, baixando o rosto, orgulhosa de já ter encontrado.

--- Temos coisa, não? Por que não foram no circo ontem? É!\ldots{} Pois não
vi não! Também estava uma enchente!\ldots{} Trouxe entrada pra vocês hoje.

Conversa vai, conversa vem, caiu sobre o Homem Cobra. Afinal não é que o
número fosse mais importante que os outros não, até os irmãos de Carmela
tinham preferido outras artistas, principalmente o de dezesseis, falando
sempre que a dançarina, filha"-da"-mãe! botava o pé mais alto que a
cabeça. Os outros tinham gostado mais da pantomima. Porém da pantomima,
Carmela só enxergara, só seguira os gestos heróicos, maquinais, do chefe
dos salteadores, aquele moreno pálido, esguio, flexível, e os grandes
olhos. Quando morreu com o tiro do polícia bersagliere, retorcendo no
chão que até parecia de deveras, Carmela teve ``uma'' dó. Sem saber,
estava torcendo pra que os salteadores escapassem.

--- O Almeidinha\ldots{} Está aí! um rapaz excelente! é do norte. Toda a
gente gosta dele. Faz todas aquelas maravilhas, você viu como ele
representa, pois não tem orgulho nenhum não, pau pra toda obra. Serve de
arara sem se incomodar\ldots{} Até foi convidado pra fazer parte duma
companhia dramática, uma feita, em Vitória do Espírito Santo, mas não
aceitou. É muito meu amigo\ldots{}

Carmela fitou o irmão, agradecida.

Afinal, pra encurtar as coisas, você logo imagina que o pai de Pietro
foi se acostumando fácil com o ofício do filho. Aquilo dava uma grande
ascendência pra ele, sobre a vizinhança\ldots{} Quando no intervalo, o Pietro
veio trazer o Garcia mais velho pra junto da família, venceu o pai. Todo
mundo estava olhando pra eles com desejo. Conhecer o dono dum circo tão
bom!\ldots{} já era alguma coisa. O João, esse teve só prazer. Fora
companheiro de infância do Pietro, este mais velho. Já combinaram um
encontro pro dia seguinte de"-tarde. Pietro mostrará tudo lá dentro, João
queria ver. 

E que Pietro apareça também lá na padaria\ldots{} Os pais ficariam contentes
de ver ele já homem, ah, meu caro, tempo corre!\ldots{}

No dia seguinte de"-tardinha, João já estava meio tonto com as
apresentações. Afinal, no picadeiro vazio, foram dar com o Almeidinha
assobiando. Endireitava o nó duma corda.

--- Boas"-tardes. Desculpe, estou com a mão suja.

Sorria. Tinha esse rosto inda mal desenhado das crianças, faltava
perfil. Quando se ria, eram notas claras sem preocupação. Distraído,
Nossa Senhora! ``Meidinha, você me arranja esta meia, a malha fugiu\ldots{}''
Almeidinha puxava a malha da meia, assobiando. ``Meidinha, dá comida pro
Malunga, faz favor, tenho de ir buscar os bilhetes.'' Lá ia o Almeidinha
assobiando, dar comida pro Malunga. Então carregar a filhinha da
Dolores, dez meses, não havia como ele, a criança adormecia logo com o
assobio doce, doce. E conversava tão delicado! João teve um entusiasmo
pelo Almeida. E quando, na noite seguinte, o Homem Cobra recebendo
aplausos, fez pra ele aquele gesto especial de intimidade, João
sentiu"-se mais feliz que o rei Dom Carlos. Safado rei dão Carlos\ldots{}

Carmela tanto falava, Pietro tanto insistiu, que o velho Quaglia recebeu
o Almeida em casa mas muito bem. Em dez minutos de conversa, o moço já
era estimado por todos. Carmela não pôde ir na cerca, já se vê, tinha
visita em casa. João que entrasse, pois não conhecia o Almeida também!

E, vamos falando logo a verdade, o Homem Cobra, assim com aquele jeito
indiferente, agarrou tendo uma atenção especial pra Carmela. Ninguém
percebia porque, afinal, a Carmela estava quase noiva do João.

Nunca mulher nenhuma tivera uma atenção especial pro Meidinha, Carmela
era a primeira. Ele percebeu. Só ele, porque os outros sabiam que ela
estava quase noiva do João. E tem coisas que só mesmo entre dois se
percebem. Carmela dum momento pra outro, você já sabe o que é a gente se
tornar criminoso, ficara hábil. Mesma habilidade no Meidinha, que fazia
tudo o que ela fazia primeiro. Até o caso da flor passou despercebido,
também quem é que percebe uma sempreviva destamanho! O certo é que
de"-noite o Homem Cobra trabalhou com ela entre as lantejoulas. Só olho
com vontade de ver é que enxerga uma pobre florzinha no meio de tanto
brilho artificial.

Era uma hora da madrugada, noite inteiramente adormecida no bairro da
Lapa, quando o esguio passou assobiando pela rua. Carmela, não sei que
loucura deu nela, acender luz não quis, podiam ver, saltou da cama, e,
com o casaquinho de veludo nas costas, entreabriu a janela. Abriu"-a.
Esperou. O esguio voltava, mãos nos bolsos, assobiando. Vendo Carmela
emudeceu. Essas casas de gente meia pobre são tão baixas\ldots{} Tocou no
chapéu passando.

--- Psiu\ldots{}

Se chegou.

--- Boa"-noite.

--- Safa! A senhora ainda não foi dormir!

--- Estava. Mas escutei o senhor, e vim.

--- Noite muito bonita\ldots{}

--- É.

--- Bom, boas"-noites.

--- Já vai\ldots{} Fique um pouco\ldots{}

Ele botara as costas na parede, mãos sempre nos bolsos. Olhava a rua,
com vontade de ir"-se embora decerto. Carmela é que trabalhou:

--- Vi a flor no seu peito.

--- Viu?

--- Fiquei muito agradecida.

--- Ora.

--- Por que o senhor botou a flor, hein?\ldots{} Podiam perceber!

Almeida se virou, muito admirado:

--- O que tinha que vissem!

--- É! tinha muita coisa, sabe! 

Ele tirara as mãos dos bolsos. Se encostara de novo na janela, e olhava
pro chão, brincando o pé numas folhinhas, a mão descansava ali do
peitoril. Carmela já conhecia a doçura das mãos dadas com o João, de
manso guardou a do moço entre as ardentes dela. Meidinha encarou"-a
inteiramente, se riu. Virou"-se duma vez e retribuiu o carinho pondo a
mão livre sobre as de Carmela.

--- As mãos da senhora estão queimando, safa!

E não pararam mais de se olhar e se sorrir. Porém os artistas, mesmo
ignorantes de vida, sabem tantas coisas por profissão\ldots{} não durou
muito, Carmela e o Meidinha trocaram o beijo n° I. Então ele partiu.

Estaria zangada?\ldots{} Aquela frieza decidiu o João: pedia a moça nessa
noite mesmo. Mas, e foi bom sinão a história ficava mais feia, não sei o
que deu nele de ir falar com ela primeiro. Cerca? era lugar aonde
Carmela não chegava desde a quarta"-feira. João mandou Sandro chamá"-la.
Que estava muito ocupada, não podia vir. O que seria!\ldots{} pois si não
tinha feito nada!\ldots{} resolveu entrar, não era homem pra complicações.
Porém a moça nem respondeu aos olhares dele. Pietro é que se divertiu
com a rusga, até fez uma caçoadinha. João teve um deslumbramento,
gostou. Mas Carmela ficou toda azaranzada. Desenhou um muxoxo de desdém
e foi pra dentro. Não sabia bem por quê, porém de repente principiou a
chorar. Veio a mãe ralhando com Pietro, onça da vida. E verdade que dona
Lina não sabia o que se passara, viu a filha chorando e deu razão à
filha. João, quando soube que a namorada estava chorando, teve um
pressentimento horrível, pressentimento de que, meu Deus!\ldots{}
pressentimento sem mais nada. Entrou em casa tonto, chegou"-se pra janela
sem pensamento, e ficou olhando a rua. Cada bonde, carroça que passava,
eram vulcões de poeira. Ar se manchando, que nem cara cheia de panos. O
jasmineiro da frente, e mesmo do outro lado da rua, por cima do muro, os
primeiros galhos das árvores tudo avermelhado. Não vê que Prefeitura se
lembra de vir calçar estas ruas! é só asfalto pras ruas vizinhas dos
Campos Elíseos\ldots{} Gente pobre que engula poeira dia inteirinho!

Si jantou, João nem percebeu. Depois caiu uma noite insuportável sem ar.
João na janela. Os pais, vendo ele assim, se puseram a amá"-lo. Doente
não estava, pois então devia de ser algum desgosto\ldots{} Carmela. Não podia
ser outra coisa. Mas o que teria sucedido! E afinal, gente pobre tem
também suas delicadezas, perguntaram de lado, o filho respondeu ``não''.
Consolar não sabiam. Nem tinham de que, ele embirrava negando. Então
puseram"-se a amar.

É assim que o amor se vinga do desinteresse em que a gente deixa ele. A
vida corre tão sossegada, ninguém não bota reparo no amor. Ahn\ldots{} é
assim, é!\ldots{} esperem que hão"-de ver!\ldots{} o amor resmunga. E fica
desimportante no lugarzinho que lhe deram. De repente a pessoa amada,
filho, mulher, qualquer um, sofre, e é então, quando mais a gente carece
de força pra combater o mal, é então que o amor reaparece, incomodativo,
tapando caminho, atrapalhando tudo, ajuntando mais dores a esta vida já
de si tão difícil de ser vivida.

Assim foi com os pais do João. O filho sofria, isso notava"-se bem\ldots{}
Pois careciam de calma, da energia acumulada em anos e anos de
trabalheira que endurece a gente\ldots{} Em vez: viram que uma outra coisa
também se fora ajuntando, crescendo sem que eles reparassem, e era
enorme agora, guaçu, macota, gigantesca! amavam o João! adoravam o João!
Como era engraçado, todo fechadinho, olho fechado, mãozinha fechada,
logo depois de nascido!\ldots{} os choros, noites sem dormir, o primeiro riso
enfim, balbucios, primeiro dente, a roupinha de cetineta cor"-de"-rosa, a
Rosa que não quisera casar com ele, e escola, as doenças, as sovas, a
primeira comunhão, o trabalho, a bondade, a força, o futebol, os olhos,
aqueles braços dependurados, meu Deus! todos os dias: o João!\ldots{} Si
tivessem vivido esse amor dia por dia, se compreende: agora só tinham
que amar aquele sofrimento do instante, isso inda cristão aguenta. Mas
os dias tinham passado sem que dessem tento do amor, e agora, por uma
causa que não sabiam, por causa daqueles cotovelos afincados na janela,
daquele queixo dobrando o pulso largo, olhar abrindo pra noite sem
resposta, vinha todo aquele amor grande de dias mil multiplicados por
mil. Amaram com desespero, desesperados de amor.

Quando João viu os vizinhos partindo pro circo, nem discutiu a verdade
do peito: vou também. Pegou no chapéu. Pra mãe ele se riu como si fosse
possível enganar mãe.

--- Vou pro circo\ldots{} Divertir um bocado.

Depois do que se passara, ir junto dela também era sem"-vergonhice,
procurou companheiros na arquibancada.

--- Ué! você não vai junto da Carmela?

--- Não me amole mais com essa carcamana!

--- Brigaram!

--- Não me amole, já disse!

Mas ver circo, quem é que podia ver circo num atarantamento daqueles! O
Homem Cobra com a sempreviva no peito. Gestos, olhares inconvenientes
não fez nenhum que se apontasse, João porém descobriu tudo. A gente não
pode culpar o Meidinha, não sabia que o outro gostava de Carmela. Um
moço pode estar sentado junto dũa moça sem ser pra namorar\ldots{}

Nessa noite o assobio chamou duas pessoas na janela. Bater, arrebentar
com aquele chicapiau desengonçado! confesso que o João espiando, matutou
nisso. Depois imaginou milhor, Carmela era dona do seu nariz e se tinha
que fazer das suas, antes agora! aprendia a ver adonde ia caindo, livra!
são todas umas galinhas. E bastava. Foi pra cama aparentemente
sossegado. Porém que"-dê sono! vinha de sopetão aquela vontade de ver,
tinha que espiar mesmo. Não podia enxergar bem, parece que se
beijavam\ldots{} ôh, que angústia na barriga!\ldots{} 

Afinal foi preciso partir, e o Meidinha andou naquele passo coreográfico
dos flexíveis. Ali mesmo na esquina distraiu"-se, o assobio contorcido
enfiou no ouvido da noite um maxixe acariocado. Carmela\ldots{} você imagine
que noites!

Convenhamos que o costume é lei grande. João mal entredormiu ali pelas
três horas, pois às quatro e trinta já estava de pé. Pesava a cabeça,
não tem dúvida, mas tinha que trabalhar e trabalhou. Botou o cavalo na
carrocinha perfumada com pão novo e tlim\ldots{} tlrintintim\ldots{} lá foi numa
festança de campainha, tirando um por um os prisioneiros das camas. São
cinco horas, padeiro passou.

--- É! circo, circo toda noite!\ldots{} Pois agora não vai mais!

Também agora pouco se amolava que a mãe proibisse espetáculo. Gozar
mesmo, só gozou na primeira noite. Depois, um poder de inquietações, de
vontades, remorsos, remorsos não, duvidinhas\ldots{} tomavam todo o tempo do
espetáculo e ela não podia mais se divertir.

Dona Lina tinha razão. Quando Carmela apareceu, o irregular do corado,
manchas soltas, falavam que isso não é vida que se dê pra uma rapariga
de dezenove anos. Pelos olhos ninguém podia pensar isso porque brilhavam
mais ainda. Estavam caindo pros lados das faces num requebro doce,
descansado, de pessoa feliz. Não digo mais linda, porém, assim, a
boniteza de Carmela se\ldots{} se humanizara. Isso: perdera aquele
convencional de pintura, pra adquirir certa violência de malvadez. Não
sei si por causa de olhar Carmela, ou por causa da pantomima, a gente se
punha matutando sobre os salteadores da Calábria. Não havia razão pra
isso, os pais dela eram gente dos arredores de Gênova\ldots{}

João, outro dia hei"-de contar o que sentiu e o que sucedeu pra ele,
agora só me lembro dele ainda porque foi o primeiro a ver chegar o
Almeida de"-tardinha. Veio, já se sabe, mãos nos bolsos, assobio no meio
da boca, bamboleando saltadinho no passo miúdo de cabra. Tinha pés de
borracha na certa, João tremeu de ódio. Pegou no chapéu, foi até muito
longe caminhando. O mal não é a gente amar\ldots{} O mal é a gente vestir a
pessoa amada com um despropósito de atributos divinos, que chegam a
triplicar às vezes o volume do amor, o que se dá? Uma pessoa natural é
fácil da gente substituir por outra natural também, questão de sair uma
e entrar outra\ldots{} Porém a que sai do nosso peito é amor que sofre de
gigantismo idealista, e não se acha outra de tanta gordura pra botar
logo no lugar. Por isso fica um vazio doendo, doendo\ldots{} Então a gente
anda cada estirão a pé\ldots{} Aquilo dura bastante tempo, até que o vazio,
graças aos ventinhos da boca"-da"-noite, se encha de pó. Se encha de pó.

Estamos no fim. São engraçadas essas mães\ldots{} Proíbem circo, obrigam as
meninas a ir cedo pra cama, pensam que deitar é dormir. Aliás, esta é
mesmo uã das fraquezas mais constantes dos homens\ldots{} Geralmente nós não
visamos o mal, visamos o remédio. Daí trinta por cento de desgraças que
podiam ser evitadas, trinta por cento é muito, vinte. Carmela entra na
conta. Também como é que dona Lina podia imaginar que quem está numa
cama não dorme? não podia. Mas nem bem o assobio vinha vindo pra lá da
esquina, já Carmela estava de pé. Beijo principiou. Até quando ela
retirava um pouco a cara pro respiro de encher, ele espichava o pescoço,
vinha salpicar beijos de guanumbi nos lábios dela. Sempre olhando muito,
percorrendo, parecia por curiosidade, a cara dela. Mas os beijos
grandes, os beijos engulidos, era a diabinha que dava. Ele se deixava
enlambuzar. Mestra e discípulo, não? Aquela inocentinha que não
trabalhava nas fábricas, quem que havia de dizer!\ldots{} Eis a inocência no
que dá: não vê que moça aprendida trocava o João pelo Homem Cobra\ldots{} Si
este penetrasse no quarto, creio que nenhum gesto de recusa encontraria
no caminho, Carmela estava louca. Só a loucura explica uma loucura
dessas. Mas até os desejos se cansam porém, a horas tantas ela sentiu"-se
exausta de amor. Puseram"-se a conversar. Meidinha, mãos nos bolsos,
encostara as costas na parede e olhava o chão. Carmela o incomodava com
a cobra aderente do abraço, rosto contra rosto. E perdidas, umas frases
de intimidade. Ela gemendo:

--- Eu gosto tanto de você!

--- Eu também.

Engraçado a ambiguidade das respostas elípticas! Gostava de quem? da
namorada ou dele mesmo?\ldots{}

--- Você trabalhou hoje?

--- Trabalhei. Vamos dar uma pantomima nova. Eu faço o violeiro do
Cubatão, venha ver.

--- Querido!

Beijo.

\textls[-15]{--- É verdade! não se vê mais o João\ldots{} É parente de você, é?}

--- Parente? Deus te livre! deu um muxoxo. Não sei onde anda. Não gosto
dele!

Silêncio. Carmela sentiu um instinto vago de arranjar as coisas. Afinal,
o caso dela se tornara uma dessas gavetas reviradas, aonde a gente não
encontra a cueca mais. Continuou:

--- Ele queria casar comigo, mas porém não gosto dele, é bobo. Só com
você que hei"-de casar!

Meidinha estava olhando o chão. Ficou olhando. Depois se virou manso e
encarou a bonita. Os olhos dele, grandes, inda mais grandes, enguliram
os da moça, contemplava. Contemplava embevecido. Carmela pousou nesses
beiços entreabertos o incêndio úmido dos dela. Meidinha agora deixava os
olhos caírem duma banda. Abraçados assim de frente, Carmela descansou o
queixo no ombro do moço, e respirava sossegada o aroma de vida que vinha
subindo da nuca dele. Ele sempre de olhos grandes, mais grandes ainda,
caídos dum lado, perdidos na escureza do quarto indiferente.

--- A gente há de ser muito feliz, não me incomodo que você trabalhe no
circo\ldots{} Irei aonde você for. Si papai não quiser, eu fujo. Uhm\ldots{}

Até conseguiu beijar o pescoço dele atrás. O Meidinha\ldots{} os lábios dele
mexiam, mas não falavam porém. Uma impressão de surpresa vibrou"-lhe os
músculos da cara de repente. Foi"-se esvaindo, não, foi descendo pros
beiços que ficaram caídos, com dor. Duramente uma energia lhe ajuntou
quase as sobrancelhas. Acalmou. Veio o sorriso. Tirou Carmela do ombro.
Na realidade era o primeiro gesto de posse que fazia, segurou a cabeça
dela. Contemplou"-a. Riu pra ela.

--- Vou embora. É muito tarde\ldots{}

Enlaçou"-a. Beijou"-lhe a boca ardentemente e tornou a beijar. Carmela
sentiu uma felicidade, que si ela fosse dessas lidas nos livros, dava
recordação pra vida inteira. Ficou imóvel, vendo ele se afastar. Assobio
não se escutou.

No dia seguinte, que"-dele o Homem Cobra?

--- Vocês não viram o Meidinha, gente!

--- Pois não dormiu em casa!

--- Não dormiu não!

--- Decerto alguma farra\ldots{}

--- Que o que!\ldots{}

\textls[-5]{Que"-dele o Almeida? Só de"-tarde, alguns grupos sabiam na Lapa que o
Homem Cobra embarcara não sei pra onde, o Abraão é que contava. Tinham
ido juntos, no primeiro bonde ``Anastácio'' da madrugada. Vendo o outro
de mala, indagou:}

--- Vai viajar!

--- Vou.

--- Deixa o circo!

--- Deixo.

--- Pra sempre é!

O Homem Cobra olhara pra ele, parecendo zangado.

--- Não tenho que lhe dar satisfações.

Virou a cara pro bairro trepando das Perdizes.

De repente, vocês não imaginam, principiou a assobiar, alegre! um
assobio de apito, nunca vi assobiar tão bem! Trabalho na Avenida
Tiradentes\ldots{} fui seguindo ele. Entrou na estação da Sorocabana.

--- Era o milhor número do circo\ldots{}

A essa hora já tivera tempo quente na casa dos Quaglias, Pietro levara a
notícia. Carmela abriu uma boca que não tinha; ataque, gente do povo não
sabe ter, caiu numa choradeira de desespero, só vendo! descobriram tudo.
Não que ela contasse, porém era muito fácil de adivinhar. Soluçava
gritando, querendo sair pra rua, chamando pelo Meidinha. Tiveram certeza
duma calúnia exagerada, pavorosa, que só o tempo desmentiu. O velho
Quaglia perdeu a cabeça duma vez, desancou a filha que não foi vida.
Carmela falava berrado que não era o que imaginavam\ldots{} mas só mesmo
quando não teve mais força misturada com a dor, é que o velho parou.
Parou pra ficar chorando que nem bezerro. Pietro andava fechando porta,
fechando quanta janela encontrava, pra ninguém de fora ouvir, mas boato
corre ninguém sabe como, as paredes têm ouvidos\ldots{} E língua muito
leviana, isso é que é. Os rapazes principiaram olhando pra Carmela dum
jeito especial, e ficavam se rindo uns pros outros. Até propostas lhe
fizeram. E ninguém mais não quis casar com ela. E só se vendo como ela
procurava!\ldots{} Uma verdadeira\ldots{} nem sei o que!

Até que ficou\ldots{} não"-sei"-o"-quê de verdade. E sabe inda por cima o que
andaram espalhando? Que quem principiou foi o irmão dela mesmo, o tal da
dançarina\ldots{} Porém coisa que não vi, não juro. E falo sempre que não
sei.

Só sei que Carmela foi muito infeliz.

\chapter{Caim, Caim e o resto}

Belazarte me contou:

Talvez ninguém reparasse, nem eles mesmo, porém foi sim, foi depois
daquela noite, que os dois começaram brigando por um nada. Dois manos
brigando desse jeito, onde se viu! E dantes tão amigos\ldots{} Pois foi
naquela noite. Sentados um a par do outro, olhavam a quermesse. O leilão
estava engraçado. O Sadresky dera três milréis por um cravo da Flora,
êta mulatinha esperta! Também com cada olhão de jabuticaba rachada,
branco e preto luzindo melado, ver suco de jabuticaba mesmo\ldots{} onde
estará ela agora? até com seu doutor Cerquinho!\ldots{}

--- Você foi pagar a conta pra ele, Aldo?

--- Já.

Contemplavam o povo entrançado no largo. Seguiam um, seguiam outro,
pensando só com os olhos. Nem trocavam palavra, não era preciso mais: se
conheciam bem por dentro. De repente viraram"-se um pro outro como pra
espiar onde que o mano olhava. Aldo fixou Tino. Tino não quis retirar
primeiro os olhos. Olho que não pestaneja cansa logo, fica ardendo que
nem com areia e pega a relampear. Quatro fuzis, meu caro, quatro fuzis
de raiva. Nem raiva, era ódio já. Aldo fez assim um jeito de muxoxo pro
magricela do irmão, riu com desprezo. Tino arreganhou o focinho como
gato assanhado. 

Se separaram. Aldo foi falar com uns rapazes, Tino foi falar com outros.
As vinte"-e"-duas horas tudo se acabava mesmo\ldots{} voltaram pra casa. Mas
cada qual vinha numa calçada. Braço a torcer é que nenhum não dava, não
vê! Dentro do quarto brigaram. Por um nadinha, questão de roupa na
guarda da cama. Dona Maria veio saber o que era aquilo espantada. Foi
uma discussão temível.

Da discussão aos murros não levou três dias. E por quê? Ninguém sabia. A
verdade é que a vida mudou pra aqueles três. Inútil a mãe chorar, se
lamentar, até insultando os filhos. Quê! nem si o defunto marido
estivesse inda vivo!\ldots{} Pegou fogo e a vida antiga não voltava mais.

E dantes tão irmãos um do outro!\ldots{} Aldo até protegia Tino que era
enfezado, cor escura. Herdara o brasileiro do pai, aquela cor caínha que
não dava nada de si e uns musculinhos que nem o trabalho vivo de
pedreiro consertava. Quando tirava fora a camisa pra se lavar no sábado,
qual! mesmo de camisa e paletó, as espáduas pousavam sobre o dorso curvo
como duas asas fechadas.

\textls[-15]{E era mesmo um anjo o Tino, tão quietinho! humilde, talhado pra
sacristão. Cantava com voz fraca muito bonita, principalmente a Mamma
mia num napolitano duvidoso de bairro da Lapa. Quando depois da janta,
fazendo algum trabalhinho, lá dentro ele cantava, Aldo junto da janela
sentia"-se orgulhoso si algum passante parava escutando. Si o tal não
parava, Aldo punha este pensamento na cachola: ``Esse não gosta de
música\ldots{} estúpido.'' Que alguém não apreciasse a voz do Tino, isso Aldo
não podia pensar porque adorava o mano.}

\textls[-31]{Era bem forte, puxara mais a mãe que o pai. Só que a gordura materna se
transformava em músculos no corpo vermelho dele. Pois então, percebendo
que os outros abusavam do Tino, não deixava mais que o irmão se
empregasse isolado, estavam sempre juntos na construção da mesma casa.
Ganhavam bem.} 

Naquela casinha do bairro da Lapa, a vida era de paraíso. Dona Maria
lavava o que não dava o dia. O defunto marido, uma pena morrer tão cedo!
fora assinzinho\ldots{} Homem, até fora bom, porque isso de beber no sábado,
quem que não bebe!\ldots{} Paciência, lavando também se ganha. Além disso,
logo os filhos tão bonzinhos principiaram trabalhando. Si a Lina fosse
viva\ldots{} que bonita!\ldots{} Felizmente os filhos a consolavam. Lhe entregavam
todo o dinheiro ganho. Gente pobre e assim é raro.

\textls[-25]{--- Meus filhos, mas vocês podem precisar\ldots{} Então tomem.}

\textls[-15]{Aqueles dois dez mil"-réis duravam quase o mês inteirinho. Fumar não
fumavam. Uma guaraná 110 domingo, de vez em quando a entrada no Recreio
ou no Carlos Gomes recentemente inaugurado, nos dias dos filmes com
muito anúncio.}

Mas no geral os manos passavam os descansos junto da mãe. No verão iam
pra porta, aquelas noites mansas, imensas da Lapa\ldots{} Pião, tlão,
tralharão, tão, plão, plãorrrrr\ldots{} bonde passava. E o silêncio. A casa
ficava um pouco apartada, sem vizinhos paredes"-meias. Na frente, do
outro lado da rua, era o muro da fábrica, tal"-e"-qual uma cinta de couro
separando a terra da noite esbranquiçada pela neblina. Chaminés. A
cinquenta metros outras casas. O cachorro latia, uau, uau\ldots{} uau\ldots{}

--- Pedro diz que vai deixar o emprego.

Silêncio.

--- Vamos no jogo domingo, Tino?

\textls[-20]{--- Não vale a pena, o Palestra vai perder. Bianco não joga.}

--- Mas Amílcar.

--- Você com seu Amílcar!

Silêncio. Tino não queria ir.

--- E tanto pessoal, Aldo\ldots{}

--- Você quer, a gente vai cedo.

Silêncio. Aldo acabava fazendo a vontade do irmão.

Às vezes também algum camarada vinha conversar. 

Agora? até já se comenta. Mãe que descomponha, que insulte\ldots{} Mais chora
que descompõe, a coitada! Lá estão os dois discutindo, ninguém sabe por
quê. De repente, tapas. E Tino não apanha mais que o outro, não pense, é
duma perversidade inventiva extraordinária. O irmão acaba sempre
sofrendo mais do que ele. Aldo é mais forte e por isso naturalmente mais
saranga. Porém paciência se esgota um dia, e quando se esgotava era cada
surra no irmão! Tino ficava com a cara vermelha de tanta bofetada. Um
pouco tonto dos socos. Aldo porém tinha sempre ũa mordida, ũa
alfinetada, coisa assim com perigo de arruinar. Os estragos da briga
duravam mais tempo nele.

Não se falavam mais. E agora cada qual andava num emprego diferente. O
mais engraçado é que quando um ia no cinema o outro ia também. Sempre
era o Tino que espiava Aldo sair, saía atrás.

Nunca iam à missa. De religião só tirar o chapéu quando passavam pela
porta das igrejas. Por que tiravam não sabiam, tinham visto o pai fazer
assim e muita gente fazia assim, faziam também, costume. Isso mesmo
quando não estavam com algum companheiro que era fachista e anticlerical
porque lera no Fanfulla. Então passavam muito indiferentes, mãos nos
bolsos talvez. E não sentiam remorso algum.

Pois nesse domingo foram à N.\,S. da Lapa outra vez. Agora que estavam
maus filhos, maus irmãos, enfim maus homens, davam pra ir na missa!
Quando a reza acabou ficaram ali, no adro da igreja meia construída,
cada um do seu lado, já sabe. Tino à esquerda da porta, Aldo à direita.
Toda a gente foi saindo e afinal tudo acabou. Ficaram apenas alguns
rapazes proseando.

Aldo voltou pra casa com uma tristeza, Tino com outra que, você vai ver,
era a mesma. Até se sentiram mais irmãos por um minuto. Minuto e meio.
Desejos de voltar à vida antiga\ldots{} Era só cada um chegar até no meio da
rua, pronto: se abraçavam chorando, ``Fratello!\ldots{}'' Que paz viria
depois! Mas, e o desespero, então? onde que leva? Reagiram contra o
sentimento bom. Uma raiva do irmão\ldots{} Uma raiva iminente do irmão. Dali,
iam só procurar o primeiro motivo e agora que tinham mais essa tristeza
por descarregar, temos tapa na certa.

\textls[-15]{Chegaram em casa e dito"-e"-feito: brigaram medonhamente. Porca la
miséria, dava medo! Se engalfinharam mudos. Aldo, subia o sangue no
rosto dele, tinha os olhos que nem fogaréu. Derrubou o mano, agarrou o
corpo do outro entre os joelhos e páa! Tino se ajeitando, rilhava os
dentes, muito pálido, engulindo tunda numa conta. A janela estava
aberta\ldots{} Dona Maria no quintal, não sei si ouviu, pressentiu com
certeza, coitada! era mãe\ldots{} ia entrar. Porém teve que saudar primeiro a
conhecida que vinha passando no outro lado da rua. Até quis botar um
riso na boca pra outra não desconfiar.}

--- Sabe, dona Maria, a conhecida gritava de lá, a Teresinha vai casar!
Com o Alfredo.

--- Ahn\ldots{}

--- Pois é. De repente. Bom, até logo.

--- Té"-logo.

O soco parou no ar, inútil, os dois manos se olharam. Viram muito bem
que não havia mais razão pra brigas agora. Não havia mesmo, deviam ser
irmãos outra vez. A felicidade voltava na certa e aquele sossego
antigo\ldots{} O soco seguiu na trajetória, foi martelar na testa do Tino,
peim! seco, seco. Tino com um jeito rápido, histérico, não sei como,
virou um bocado entre as pernas de Aldo. Conseguiu com as mãos livres
agarrar o pulso do outro. Encolheu"-se todinho em bola e mordeu onde
pôde, que dentada! Aldo puxou a mão desesperado, pleque! sofreu com o
estralo do dedo que não foi vida. Mas por ver sangue é que cegou.

--- Morde agora, filho"-da"-puta!

Na garganta. Apertou. Dona Maria entrava. 

--- Meu filho!

--- Morde agora!

\textls[-15]{Tino desesperado buscava com as mãos alargar aquele nó, sufocava.
Encontrou no caminho a mão do outro e uma coisa pendente, meia solta,
molhada, agarrou. E num esforço de última vida, puxou pra ver se abria a
tenaz que o enforcava. Dona Maria não conseguia separar ninguém. Tino
puxou, que eu disse, e de repente a mão dele sem mais resistência riscou
um semicírculo no ar. Foi bater no chão aberta ensanguentada, atirando
pra longe o dedo arrancado de Aldo.}

--- Morde agora!

Tino se inteiriçando. Abriu com os dentes uma risada lateral, até corara
um pouco. Dona Maria chegava só ao portãozinho, gritando. Não podia ir
mais além, lhe dava aquela curiosidade amorosa, entrava de novo. Tino se
inteiriçando. Ela saía outra vez:

--- Socorro! meu filho!

Meu Deus, era domingo! entrava de novo. Batia com os punhos na cabeça.
Pois batesse forte com um pau na cabeça do Aldo! Mas quem disse que ela
se lembrava de bater!

--- Socorro! meu filho morre!

Entrava. Saía. Às vezes dava umas viravoltas, até parecia que estava
dançando\ldots{} Balancez, tour, era horrível.

O primeiro homem que acorreu já chegou tarde. E só três juntos afinal
conseguiram livrar o morto das mãos do irmão. Aldo como que
enlouquecera, olho parado no meio da testa, boca aberta com uns
resmungos ofegantes.

Levaram ele preso. Dona Maria é que nem sei como não enlouqueceu de
verdade. Berrava atirada sobre o cadáver do filho, porém quando o outro
foi"-se embora na ambulância, até bateu nos soldados. Foram brutos com
ela. Esses soldados da Polícia são assim mesmo, gente mais ordinária que
há! ũa mãe\ldots{} compreende"-se que tivesse atos inconscientes! pois
tivessem paciência com ela! Que paciência nem mané paciência! em vez,
davam cada empurrão na pobre\ldots{}

--- Fique quieta, mulher, sinão levo você também!

Fecharam a portinhola e a sereia cantou numa fermata de ``Addio'' rumo
da correição.

Seguiu"-se toda a miséria do aparelho judiciário. Solidão. Raciocínio. O
julgamento. Aldo saiu livre. Pra que vale um dedo perdido? Caso de
legítima defesa complicada com perturbação de sentidos, é lógico, art.
32, art. 27 § 40\ldots{} A medicina do advogadinho salvou o réu.

Recomeçou no trabalho. Muito silencioso sempre, sossegado, parecia bom.
Às vezes parava um pouco o gesto como que refletindo. Afinal todos na
obra acabaram esquecendo o passado e Aldo encontrou simpatias.
Camaradagens até. Não: camaradagem não, porque não dava mais que duas
palavras pra cada um. Mas muitos operários simpatizavam com ele. São
coisas que acontecem, falavam, e a culpa fora do mano, a prova é que
Aldo saíra livre. E o dedo.

Mas o caso não terminou. Um dia Aldo desapareceu e nem semana depois
encontraram ele morto, já bem podrezinho, num campo. Quem seria? Procura
daqui, procura dali, a Polícia de São Paulo, você sabe, às vezes é
feliz, acabaram descobrindo que o assassino era o marido da Teresinha.

E por que, agora! Ninguém não sabia. A pobre da Teresinha é que chorava
agarrada nos dois filhinhos imaginando por que seria que o marido matara
esse outro. De que se lembrava muito vagamente, é capaz que dancei com
ele numa festa? Mas não lembrava bem, tantos moços\ldots{} E não pertencera
ao grupinho dela. Mas que o Alfredo era bom, ela jurava.

--- Meu marido está inocente! repetia cem vezes inúteis por dia.

O Alfredo gritava que fora provocado, que o outro o convidara pra irem
ver uma casa, não sei o quê! pra irem ver um terreno, e de repente se
atirara sobre ele quando atravessavam o campo\ldots{} Então pra que não veio
contar tudo logo! Em vez: continuou tranquilo indo no serviço todo santo
dia, muito satisfeito\ldots{}, que ``fascínora''! Toda a gente estava contra
ele, o Aldo tão quieto!\ldots{}

O advogado devassou a série completa dos argumentos de defesa própria. E
lembrou com termos convincentes que o Alfredo era bom. Afinal
vinte"-e"-dois anos de honestidade e bom"-comportamento provam alguma
coisa, senhores jurados! E a Teresinha com as duas crianças ali,
chorosa\ldots{} Grupo comovente. O maior, de quinze meses, procurava enfiar o
caracaxá vermelho na boca da mãe. Não brinque com essa história de
isolar sempre que falo em mãe, o caso é triste. Pois tudo inútil, o
criminoso estava com todos os dedos. Foi condenado a nem sei quantos
anos de prisão.

A Terezinha lavava roupa, costurava, mas qual! com filho de ano e pouco
e outro mamando, trabalhava mal. E, parece incrível! inda por cima com a
mãe nas costas, velha, sem valer nada\ldots{} Si ao menos soubesse aonde que
estavam esses irmãos pelas fazendas\ldots{} Mas não ajudariam, estou certo
disso, uns desalmados que nunca deram sinal de si\ldots{} Então desesperava,
ralhava com a mãe, dava nos pequenos que era uma judiaria.

A sogra, essa quando chegava até o porão da nora, trazia ũa esmola entre
pragas, odiava a moça. Adivinhava muito, com instinto de mãe, e odiava a
moça. Amaldiçoava os netos. Os dez milréis sobre um monte de insultos
ficavam ali atirados, aviltantes, relumeando no escuro. Teresinha pegava
neles, ia comprar coisas pra si, pros filhos, como ajudavam! Ainda
sobrava um pouco pra facilitar o pagamento do aluguel no mês seguinte.
Mas não lhe mitigavam a desgraça.

Também lhe faziam propostas, que inda restavam bons pedaços de mulher no
corpo dela. Recusava com medo do marido ao sair da prisão, um assassino,
credo!

Teresinha era muito infeliz.

\chapter{Piá não sofre? Sofre}

Belazarte me contou:

Você inda está lembrado da Teresinha? aquela uma que assassinou dois
homens por tabela, os manos Aldo e Tino, e ficou com dois filhos quando
o marido foi pra correição?\ldots{} Parece que o sacrifício do marido tirou o
mau"-olhado que ela tinha: foi desinfeliz como nenhuma, porém ninguém
mais assassinou por causa dela, ninguém mais penou. Só que o Alfredo lá
ficou no palácio chique da Penitenciária, ruminando os vinte anos de
prisão que a companheira fatalizada tinha feito ele engulir. Injustiça,
amargura, desejo\ldots{} tantas coisas que muito bucho não sabe digerir com
paciência, resultado: o Alfredo teve uma dessas indigestões tamanhas de
desespero que ficou dos hóspedes mais incômodos da Penitenciária.
Ninguém gostava dele, e o amargoso atravessava o tempo do castigo num
areião difícil e sem fim de castiguinhos. Estou perdendo tempo com ele.

A Teresinha sofria, coitada! ainda semiboa no corpo e com a pabulagem de
muitos querendo intimidades com ela ao menos por uma noite paga.
Recusou, de primeiro pensando no Alfredo gostado, em seguida pensando no
Alfredo assassino. Estava já no quase, porém vinha sempre aquela ideia
do Alfredo saindo da correição com uma faca nova pra destripá"-la. E a
virtude se conservava num susto frio, sem nenhum gosto de existir.
Teresinha voltava pra casa com uma raiva desempregada, que logo
descarregava na primeira coisa mais frouxa que ela. Enxergava a mãe
morrendo em pé por causa da velhice temporã, pondo cinco minutos pra
recolher uma ceroula do coaral, pronto: atirava a trouxa de roupa"-suja
na velha:

\textls[-25]{--- A senhora é capaz que vai dormir com a ceroula na mão!}

Entrava. Podia"-se chamar de casa aquilo! Era um rancho de tropeiro onde
ninguém não mora, de tão sujo. Dois aspectos de cadeira, a mesa, a cama.
No assoalho havia mais um colchão, morado pelas baratas que de"-noite
dançavam na cara da velha o torê natural dos bichinhos desta vida.

\textls[-5]{No outro quarto ninguém dormia. Ficou feito cozinha dessa família
passando muitas vezes dois dias sem fósforo acendido. Porque fósforo
aceso quer dizer carvão no fogãozinho portátil e algum desses alimentos
de se cozinhar. E muitas vezes não havia alimento de se cozinhar\ldots{} Mas
isso não fazia mal pro dicionário da Teresinha e da mãe, fogareiro não
estava ali? E o dicionário delas dera pra aqueles estreitos metros
cúbicos de ar mofado o nome estapafúrdio de cozinha.}

Nessa espécie de tapera a moça vivia com a mãe e o filhote de sobra. De
sobra em todos os sentidos, sim. Sobrava porque afinal amor pra
Teresinha, meu Deus! vivendo entre injustiças de toda a sorte, desejando
homem pro corpo e não tendo, se esquecendo do Alfredo gostado pelo
Alfredo ameaçando e já com morte na consciência\ldots{} E só tendo na mão
consolada pela água pura, ceroulas, calças, meias com mais de sete dias
de corpo suado\ldots{} E além do mais, odiando uns fregueses sempre devendo a
semana retrasada\ldots{} Tudo isso a Teresinha aguentava. E pra tampar duma
vez todos os vinhos do amor, inda por cima chegava a peste da sogra
amaldiçoada, odiada mas desejada por causa dos dez milréis deixados
mensalmente ali. A figlia dum cane vinha, emproada porque tinha de seu
aí pra uns trinta contos, nem sei, e desbaratava com ela por um
nadinha. 

Podia ter amor ũa mulher já feita, com trinta anos de seca no prazer,
corpo cearense e alma ida"-se embora desde muito!\ldots{} E o Paulino, faziam
já quase quatro anos, dos oito meses de vida até agora, que não sabia o
que era calor de peito com seio, dois braços apertando a gente, uma
palavra ``figliuolo mio'' vinda em cima dessa gostosura, e a mesma boca
enfim se aproximando da nossa cara, se ajuntando num chupão leve que faz
bulha tão doce, beijo de nossa mãe\ldots{}

Paulino sobrava naquela casa.

E sobrava tanto mais, que o esperto do maninho mais velho, quando viu
que tudo ia mesmo por água a baixo, teve um anjo"-da"-guarda caridoso que
depositou na língua do felizardo o micróbio do tifo. Micróbio foi pra
barriguinha dele, agarrou tendo filho e mais filho a milhões por hora, e
nem passaram duas noites, havia lá por dentro um footing tal da
microbiada marchadeira, que o asfaltinho das tripas se gastou. E o
desbatizado foi pro limbo dos pagãos sem culpa. Sobrou Paulino.

É lógico que ele não podia inda saber que estava sobrando assim tanto
neste mundo duro, porém sabia muito bem que naquela casa não sobrava
nada pra comer. Foi crescendo na fome, a fome era o alimento dele. Sem
pôr consciência nos mistérios do corpo, ele acordava assustado. Era o
anjo\ldots{} que anjo"-da"-guarda! era o anjo da malvadeza que acordava Paulino
altas horas pra ele não morrer. O desgraçadinho abria os olhos na
escuridão cheirando rúim do quarto, e inda meio que percebia que estava
se devorando por dentro. De primeiro ele chorava.

--- Stá zito, guaglion!

Que ``stá zito'' nada! Fome vinha apertando\ldots{} Paulino se levantava nas
pernas de arco, e balanceando chegava afinal junto à cama da mãe.
Cama\ldots{} A cama grande ela vendeu quando esteve uma vez com a corda na
garganta por causa do médico pedindo aquilo ou vinte bagarotes pela cura
do pé arruinado. Deu os vinte vendendo a cama. Cortou o colchão pelo
meio e botou a metade sobre aqueles três caixões. Essa era a cama.

Teresinha acordava da fadiga com a mãozinha do filho batendo na cara
dela. Ficava desesperada de raiva. Atirava a mão no escuro, acertasse
onde acertasse, nos olhos, na boca"-do"-estômago, pláa!\ldots{} Paulino rolava
longe com uma vontade legítima de botar a boca no mundo. Porém o corpo
lembrava duma feita em que a choradeira fizera o salto do tamanco vir
parar mesmo na boca dele, perdia o gosto de berrar. Ficava choramingando
tão manso que até embalava o sono da Teresinha. Pequenininho, redondo,
encolhido, talqualmente tatuzinho de jardim.

O sofrimento era tanto que acabava desprezando os pinicões da fome,
Paulino adormecia de dor. De madrugada, o tempo esfriando acordava o
corpo dele outra vez. Meio esquecido, Paulino espantava de se ver
dormindo no assoalho, longe do colchão da vó. Estava com uma dorzinha no
ombro, outra dorzinha no joelho, outra dorzinha na testa, direito no
lugar encostado no chão. Percebia muito pouco as dorzinhas, por causa da
dor guaçu do frio. Engatinhava medroso, porque a escureza estava já toda
animada com as assombrações da aurora, abrindo e fechando o olho das
frestas. Espantava as baratas e se aninhava no calor ilusório dos ossos
da avó. Não dormia mais.

Afinal, ali pelas seis horas, já familiarizado com a vida por causa dos
padeiros, dos leiteiros, dos homens cheios de comidas que passavam lá
longe, um calor custoso nascia no corpo de Paulino. Porém a mãe também
já estava acordando com as bulhas da vida. Sentada, vibrando com a
sensualidade matinal que bota a gente louco de vontade, a Teresinha
quase se arrebentava, apertando os braços contra a peitaria, o ventre e
tudo, forçando tanto uma perna contra a outra que sentia uma dor nos
rins. Nascia nela esse ódio impaciente e sem destino, que vem da muita
virtude conservada a custo de muita miséria, virtude que ela mesma
estava certa, mais dia menos dia tinha de se acabar. Procurava o
tamanco, dando logo o estrilo com a mãe, ``si não sabia que não era mais
hora de estar na cama'', que fosse botar água na tina, etc.

Então Paulino, antes das duas mulheres, abandonava o calor nascente do
corpo. Ia já rondar a cozinha porque estava chegando o momento mais
feliz da vida dele: o pedaço de pão. E que domingo pra Paulino quando,
porque um freguês pagou, porque a sogra apareceu, coisa assim, além do
pão, bebiam café com açúcar!\ldots{} Chupava depressa, queimando a língua e
os beicinhos brancos, aquela água quente, sublime de gostosa por causa
duma pitadinha de café. E saía comer o pão lá fora.

Na frente da casa não, era lá que ficavam a torneira, as tinas e o
coaral. As mulheres estavam fazendo suas lavagens de roupa e era ali na
piririca: briga e descompostura o tempo todo. Quem pagava era o reinação
do Paulino. Acabava sempre com um pão mal comido e algum cocre de inhapa
bem no alto do coco, doendo fino.

Deixou de ir para lá. Abria a porta só encostada da cozinha, descia o
degrau, ia correcorrendo se rir pra alegria do frio companheiro, por
entre os tufos de capim e as primeiras moitas de carrapicho. Esse
matinho atrás da casa era a floresta. Ali Paulino curtia as penas sem
disfarce. Sentado na terra ou dando com o calcanhar nos olhos dos
formigueiros, principiava comendo. De repente quase caía levantando a
perninha, ai! do chão, pra matar a saúva ferrada no tornozelinho de
bico. Erguia o pão caído e recomeçava o almoço, achando graça no
requetreque que a areia ficada no pão, ganzava agora nos dentinhos dele.

Mas não esquecia da saúva não. Pão acabado, surgia, distraindo a fome
nova, o guerreiro crila. Procurava uma lasca de pau, ia caçar formigas
no matinho. Afinal, matinho não muito pequeno porque dava atrás na
várzea, e não havia sinão um lembrete de cerca fechando o terreno. Mas
nunca Paulino penetrou na várzea que era grande demais pra ele. Lhe
bastava aquele matinho gigante, sem planta com nome, onde o sol mais
preguiça nunca deixava de entrar.

\textls[-25]{Graveto em punho lá ia em busca de saúva. As formiguinhas menores, não
se importava com elas não. Só arremetia contra saúva. Quando achava uma,
perseguia"-a paciente, rompendo entre os ramos entrançados dos arbustos,
donde muitas vezes voltava com a mão, a perna ardendo por ter relado
nalgum mandarová. Trazia a saúva pro largo e levava horas brincando com
a desgraçadinha, até a desgraçadinha morrer.}

Quando ela morria, o sofrimento recomeçava pra Paulino, era fome. O sol
já estava alto, porém Paulino sabia que só depois das fábricas apitarem
havia de ter feijão com arroz nos tempos ricos, ou novo pedaço de pão
nos tempos felizmente mais raros. Batia uma fome triste nele que outra
saúva combatida não conseguia distrair mais. Banzava na desgraça,
melancolizado com a repetição do sofrimento cotidiano. Sentava em
qualquer coisa, descansando a bochecha na mão, cabeça torcidinha, todo
penaroso. Afinal, nalguma sombra rendada, aprendeu a dormir de fome.
Adormecia. Sonhava não. As moscas vinham lhe bordando de asas e zumbidos
a boquinha aberta, onde um resto de adocicado ficou. Paulino dormindo
fecha de repente os beiços caceteados, se mexe, abre um pouco as
perninhas encolhidas e mija quente em si.

\textls[-25]{Sono curto. Acordou muito antes das fábricas apitarem. Mastigou a boca
esfomeada, recolheu com a língua os sucos perdidos nos beiços.
Requetreque de areia e uma coisinha meia doce no paladar. Tirou com a
mão pra ver o que era, eram duas moscas. Moscas sim, porém era meio
adocicado. Tornou a botar as moscas na língua, chupou o gostinho delas,
enguliu.}

\textls[-20]{Foi assim o princípio dum disfarce da fome por meio de todas as coisas
engulíveis do matinho. Não tardou muito e virou ``papista'' como se diz:
trocou a caça das saúvas pelos piqueniques de terra molhada. Comer
formiga então\ldots{} Junto dos montinhos dos formigueiros encostava a cara
no chão com a língua pronta. Quando formiga aparecia, Paulino largava a
língua hábil, grudava nela a formiga, e a esfregando no céu"-da"-boca
sentia um redondinho infinitesimal. Punha o redondinho entre os dentes,
trincava e engulia o guspe ilusório. E que ventura si topava com alguma
correição! De gatinhas, com o fiofó espiando as nuvens, lambia o chão
tamanduamente. Apagava uma carreira viva de formiga em três tempos.}

\textls[-18]{Nessa esperança de matar a fome, Paulino foi descendo a coisas nojentas.
Isto é, descendo, não. Era incapaz de pôr jerarquia no nojo, e até o
último comestível inventado foi formiga. Porém não posso negar que uma
vez até uma barata\ldots{} Agarrou e foi"-se embora mastigando, mais inocente
que vós, filhos dos nojos. Porém, compreende"-se: eram alimentos que não
davam sustância nenhuma. Fábrica apitava e o arroz"-com"-feijão vinha
achar Paulino empanturrado de ilusões, sem fome. Pegou aniquilando,
escurecendo que nem dia de inverno.}

Teresinha não reparava. O buçal da virtude estava já tão gasto que
via"-se o momento da moça desembestar livre, vida fora. Foi o tempo em
que tapa choveu por todas as partes de Paulino cegamente, caísse onde
caísse. Quando ela vinha pra casa já escutava a companhia do Fernandez,
carroceiro. Era um mancebo de boa tradição, desempenado, meio lerdo,
porém com muita energia. Devia de ter vinte"-e"-cinco anos, si tinha! e se
engraçou pela envelhecida, quem quiser saiba por que. Buçal arrebentou.
Quando ele pôde carregar a trouxa pra ela, veio até a casa, entrou que
nem visita, e Teresinha ofereceu café e consentimento. A velha, sujando
a língua com os palavrões mais incompreensíveis, foi dormir na cozinha
com Paulino espantado.

Em todo caso a bóia milhorou, e o barrigudinho conheceu o segredo da
macarronada. Só que tinha muito medo do homem. Fernandez fizera uma
festinha pra ele na primeira aparição, e quando saiu do quarto de"-manhã
e beberam café todos juntos, Paulino confiado foi brincar com a perna
comprida do homem. Mas tomou com um safanão que o fez andar de orelha
murcha um tempo.

\textls[-10]{E lógico que a sogra havia de saber daquilo, soube e veio. Teresinha
muito fingida falou bom"-dia pra ela e a mulatona respondeu com duas
pedras na mão. Porém agora Teresinha não carecia mais da outra e
refricou, assanhada feito irara. Bateboca tremendo! Paulino nem tinha
pernas pra abrir o pala dali, porque a velha apontava pra ele, falando
``meu neto'' que mais ``meu neto'' sem parada. E mandava que Teresinha
agora se arranjasse, porque não estava pra sustentar cachorrice de
italiana acueirada com espanhol. Teresinha secundava gritando que
espanhol era muito mais milhor que brasileiro, sabe! sua filha de negro!
mãe de assassino! Não careço da senhora, sabe! mulata! mulatona! mãe de
assassino!}

--- Mãe de assassino é tu, sua porca! Tu que fez meu filho sê infeliz,
maldiçoada do diabo, carcamana porca!

--- Saia já daqui, mãe de assassino! A senhora nunca se amolou com seu
neto, agora vem com prosa aí! Leve seu neto si quiser!

--- Pois levo mesmo! coitadinho do inocente que não sabe a mãe que tem,
sua porca! porca!

Suspendeu Paulino esperneando, e lá se foi batendo salto, ajeitando o
xale de domingo, por entre as curiosas raras do meidia. Inda virou,
aproveitando a assistência, pra mostrar como era boa:

--- Escute! Vocês agora, não pago mais aluguel de casa pra ninguém,
ouviu! Protegi você porque era mulher de meu filho desgraçado, mas não
tou pra dar pouso pra égua, não!

Mas a Teresinha, louca de ódio, já estava olhando em torno pra encontrar
um pau, alguma coisa que matasse a mulatona. Esta achou milhor partir
duma vez, triunfante ploque ploque.

\textls[-15]{Paulino ia ondulando por cima daquelas carnes quentes. Chorava
assustado, não tendo mais noção da vida, porque a rua nunca vista, muita
gente, aquela mulher estranha e ele sem mãe, sem pão, sem matinho, sem
vó\ldots{} não sabia mais nada! meu Deus! como era desgraçado! Teve um medo
pavoroso no corpinho azul. Inda por cima não podia chorar à vontade
porque reparara muito bem, a velha tinha um sapatão com salto muito
grande, pior que tamanco. Devia de ser tão doído aquele salto batendo no
dentinho, rasgando o beiço da gente\ldots{} E Paulino horrorizado enfiava
quase as mãozinhas na boca, inventando até bem artisticamente a função
da surdina.}

--- Pobre de meu neto!

Com a mão grande e bem quente pegou na cabecinha dele, ajeitando"-a no
pescoço de borracha. Carregado gostoso naqueles braços bons, com o xale
dando inda mais quentura pra gente ser feliz\ldots{} E a velha olhou pra ele
com olhos de piedade confortante\ldots{} Meu Deus! que seria aquilo tão
gostoso!\ldots{} É assomo de ternura, Paulino. A velha apertou"-o no peito
abraçando, encostou a cara na dele, e depois deu beijos, beijos,
revelando pro piá esse mistério maior.

Paulino quis sossegar. Pela primeira vez na vida o conceito de futuro se
alargou até o dia seguinte na ideia dele. Paulino sentiu que estava
protegido, e no dia seguinte havia de ter café"-com"-açúcar na certa. Pois
a velha não chegara a boca ajuntada bem na cara dele e não dera aquele
chupão que barulhava bom? Dera. E a ideia de Paulino se encompridou até
o dia seguinte, imaginando um canecão do tamanho da velha, cheinho de
café"-com"-açúcar. Foi se rir pras duas lágrimas piedosas dela, porém bem
no meio da gota apareceu uma botina que foi crescendo, foi crescendo e
ficou com um tacão do tamanho da velha. Paulino reprincipiou chorando
baixo, que nem nas noites em que o acalanto da manha embalava o sono da
Teresinha.

--- Ara! também agora basta de chorar! Ande um pouco, vamos!

O salto da botina encompridou enormemente e era a chaminé do outro lado
da rua. O pranto de Paulino parou, mas parou engasgado de terror.
Chegaram. 

Esta era uma casa de verdade. Entrava"-se no jardinzinho com flor, que
até dava vontade de arrancar as semprevivas todas, e, subida a
escadinha, havia uma sala com dois retratos grandes na parede. Um homem
e uma mulher que era a velha. Cadeiras, uma cadeira grande cabendo muita
gente nela. Na mesinha do meio um vaso com uma flor cor"-de"-rosa que
nunca murchou. E aquelas toalhinhas brancas nas cadeiras e na mesa, que
devia distrair a gente cortando tantas bolotinhas\ldots{}

O resto da casa assombrou desse mesmo jeito o despatriado. Depois
apareceram mais duas moças muito lindas, que sempre viveram de saia
azul"-marinho e blusa branca. Olharam duras pra ele. Aqueles quatro olhos
negros desceram lá do alto e tuque! deram um cocre na alma de Paulino.
Ele ficou tonto, sem movimento, grudado no chão.

Daí foi uma discussão terrível. Não sei o que a velha falou, e uma das
normalistas respondeu atravessado. A velha asperejou com ela falando no
``meu neto''. A outra respondeu gritando e uma tormenta de ``meu neto''
e ``seu neto'' relampagueou alto sobre a cabeça de Paulino. A história
foi piorando. Quando não teve mais agudos pras três vozes subirem, a
velha virou um bofete na filha da frente, e a outra fugindo escapou de
levar com a colher bem no coco.

\textls[-5]{A invenção de Paulino não podia ajuntar mais terrores. E o engraçado é
que o terror pela primeira vez despertou mais a inteligência dele. O
conceito de futuro que fazia pouco atingira até o dia seguinte, se
alongou, se alongou até demais, e Paulino percebeu que entre raivas e
maus"-tratos havia de passar agora o dia seguinte inteiro e o outro dia
seguinte e outro, e nunca mais haviam de parar os dias seguintes assim.
É lógico: sem ter a soma dos números, mais de três mil anos de dias
seguintes sofridos, se ajuntaram no susto do piá.}

--- Vá erguer aquela colher!

As metades do arco se moveram ninguém sabe como, Paulino levantou a
colher do chão que deu pra velha. Ela guardou a colher e foi lá dentro.
A varanda ficou vazia. Estava tudo arranjado, e as sombras da tarde
rápida entravam apagando as coisas desconhecidas. Só a mesa do centro
inda existia nitidamente, riscada de vermelho e branco. Paulino foi se
encostar na perna dela. Tremia de medo. Chiava um cheiro gostoso lá
dentro, e da sombra da varanda um barulhinho monótono, tique"-taque,
regulava as sensações da gente. Paulino sentou no chão. Uma calma grande
foi cobrindo o pensamento aniquilado: estava livre do tacão da velha.
Ela não era que nem a mãe não. Quando tinha raiva não atirava botina,
atirava uma colher levinha, brilhando de prateada. Paulino se encolheu
deitado, encostando a cabeça no chão. Estava com um sono enorme de tanto
cansaço nos sentimentos. Não havia mais perigo de receber com tamanco no
dentinho, a mulatona só atirava aquela colher prateada. E Paulino
ignorava se colher doía muito, batendo na gente. Adormeceu bem calmo.

--- Levante! que é isso agora! Como esse menino deve ter sofrido,
Margot! Olhe a magreira dele!

--- Pudera! com a mãe na gandaia, festando dia e noite, você queria o
que, então!

--- Margot\ldots{} você sabe bem certo o que quer dizer puta, hein? Eu acho
que a gente pode falar que Paulino é filho"-da"-puta, não?

Se riram.

--- Margot!

--- Senhora!

--- Mande Paulino aqui pra dar comida pra ele!

--- Vá lá dentro, menino!

As pernas de arco balançaram mais rápidas. Uma cozinha em que a gente
não podia nem se mexer. A velha boa inda puxou o capacho da porta com o
pé:

--- Sente aí e coma tudo, ouviu!

Era arroz"-com"-feijão. A carne, Paulino viu com olho comprido ela
desaparecer na porta da varanda. Menino de quatro anos não come carne,
decerto imaginou a velha, meia em dificuldades sempre com a educação das
filhas.

E a vida mudou de misérias pra Paulino, mas continuou a sempre
miserável. Bóia milhorou muito e não faltava mais, porém Paulino estava
sendo perseguido pelos vícios do matinho. Nunca mais a mulatona teve
daqueles assomos de ternura do primeiro dia, era uma dessas cujo
mecanismo de vida não difere muito do cumprimento do dever. Aquele beijo
fora sincero, mas apenas dentro das convenções da tragédia. Tragédia
acabara e com ela a ternura também. E no entanto ficara muito em Paulino
a saudade dos beijos\ldots{}

\textls[-10]{Quis se chegar pras moças porém elas tinham raiva dele, e podendo,
beliscavam. Assim mesmo a mais moça, que era uma curiosa do apá virado e
nunca tirava as notas de Margot na escola, Nininha, é que tomara pra si
dar banho no Paulino. Quando chegava no sábado, o pequeno meio espantado
e muito com medo de beliscão, sentia as carícias dum rosto lindo em fogo
se esfregando no corpinho dele. Acabava sempre aquilo, a menina com uma
raiva bruta, vestindo depressa a camisolinha nele, machucando, ``fica
direito, peste!'' pronto: um beliscão que doía tanto, meu Deus!}

Paulino descia a escada da cozinha, ia muito jururu pelo corredorzinho
que dava no jardim da frente, puxava com esforço o portão sempre
encostado, sentava, punha a mão na bochecha, cabecinha torcida pro lado
e ficava ali, vendo o mundo passar.

E assim, entre beliscões e palavras duras que ele não entendia nada,
``menino fogueto'', ``filho de assassino'', ele também passava feito o
mundo: magro escuro terroso, cada vez se aniquilando mais. Mas o que que
havia de fazer? Bebia o café e já falavam que fosse comer o pão no
quintal sinão, porco! sujava a casa toda. Ia pro quintal, e a terra
estava tão úmida, era uma tentação danada! Nem ele punha reparo que era
uma tentação porque nenhum cocre, nenhuma colherada, o proibira de comer
terra. Treque"-trrleque, mastigava um bocadinho, engulia, mastigava outro
bocadinho, engulia. E ali pelas dez horas sempre, com a pressa das
normalistas assombrando a calma da vida, tinha que assentar naquele
capacho pinicando, tinha que engulir aquele feijão"-com"-arroz num fastio
impossível\ldots{}

--- Minha Nossa Senhora, esse menino não come! Ói só com que cara ele
olha pra comida! Pra que que tu suja a cara de terra desse jeito, hein,
seu porcalhão!

Paulino assustava, e o instinto fazia ele engulir em seco esperando a
colherada nunca vinda. Porém desta vez a velha tivera uma iluminação no
mecanismo:

--- Será que!\ldots{} Você anda comendo terra, não! Deixe ver!

Puxou Paulino pra porta da cozinha, e com aquelas duas mãos enormes,
queimando de quentes:

--- Abra a boca, menino!

\textls[-40]{E arregaçava os beiços dele. Terra nos dentinhos, na gengiva.}

--- Abra a boca, já falei!

E o dedo escancarava a boquinha terrenta, língua aparecendo até a raiz,
todinha da cor do barro. A sova que Paulino levou nem se conta!
Principiou com o tapa na boca aberta, que até deu um som engraçado, bóo!
e não posso falar como acabou de tanta mistura de cocre beliscão
palmadas. E palavreado, que afinal pra criancinha é tabefe também.

Então é que principiou o maior martírio de Paulino. Dentro da casa,
nenhuma queria que ele ficasse, tinha mesmo que morar no quintal. Antes
do pão porém, já vinha uma sova de ameaças, tão dura, palavra"-de"-honra:
Paulino descia a escadinha completamente abobado, sentindo o mundo bater
nele. E agora?\ldots{} Pão acabou e a terra estava ali toda oferecida
chamando. Mas aquelas três beliscadoras não queriam que ele comesse a
terra gostosa\ldots{} Oh tentação pro pobre santantoninho! queria comer e não
podia. Podia, mas depois lá vinha de hora em hora o dedão da velha
furando a boquinha dele\ldots{} Como?\ldots{} Não como?\ldots{} Fugia da tentação,
subia a escadinha, ficava no alto sentado, botando os olhos na parede
pra não ver. E a terra sempre chamando ali mesmo, boa, inteirinha dele,
cinco degraus fáceis em baixo\ldots{}

Felizmente não sofreu muito não. Três dias depois, não sei si brincou na
porta com os meninos de frente, apareceu tossindo. Tosse aumentou, foi
aumentando, e afinal Paulino escutou a velha falar, fula de
contrariedade, que era tosse"-de"-cachorro. Si haviam de levar o menino no
médico, em vez, vamos dar pra ele o xarope que dona Emília ensinou. Nem
xarope de dona Emília, nem os cinco milréis ficados no boticário mais
chué do bairro sararam o coitadinho. Tinha mesmo de esperar a doença, de
tanto não encontrando mais sonoridade pra tossir, ir"-se embora sozinha.

O coitado nem bem sentia a garganta arranhando, já botava as mãozinhas
na cabeça, inquieto muito! engulindo apressado pra ver se passava. Ia
procurando parede pra encostar, vinha o acesso. Babando, olho babando,
nariz babando, boca aberta não sabendo fechar mais, babando numa conta.
O coitadinho sentava no lugar onde estava, fosse onde fosse porque sinão
caía mesmo. Cadeira girava, mesa girava, cheiro de cozinha girava.
Paulino enjoado atordoado, quebrado no corpo todo.

--- Coitado. Olhe, vá tossir lá fora, você está sujando todo o chão, vá!

Ele arranjava jeito de criar força no medo, ia. Vinha outro acesso, e
Paulino deitava, boca beijando a terra mas agora sem nenhuma vontade de
comer nada. Um tempo estirado passava. Paulino sempre na mesma posição.
Corpo nem doía mais, de tanto abatimento, cabeça não pensando mais, de
tanto choque aguentado. Ficava ali, e a umidade da terra ia piorar a
tosse e havia de matar Paulino. Mas afinal aparecia uma forcinha, e
vontade de levantar. Vai levantando. Vontade de entrar. Mas podia sujar
a casa e vinha o beliscão no peitinho dele. E não valia de nada mesmo,
porque mandavam ele pra fora outra vez\ldots{} 

\textls[-10]{Era de"-tarde, e os operários passavam naquela porção de bondes\ldots{} enfim
divertia um bocado pelo menos os olhos ramelosos. Paulino foi sentar no
portão da frente. A noite caía agitando vida. Um ventinho poento de
abril vinha botar a mão na cara da gente, delicado. O sol se agarrando
na crista longe da várzea, manchava de vermelho e verde o espaço
fatigado. Os grupos de operários passando ficavam quase negros contra a
luz. Tudo estava muito claro e preto, incompreensível. Os monstros
corriam escuros, com moços dependurados nos estribos, badalando uma
polvadeira vermelha na calçada. Gente, mais monstros e os cavalões nas
bonitas carroças.}

Nesse momento a Teresinha passou. Vinha nuns trinques, só vendo! sapato
amarelado e meia roseando uma perna linda mostrada até o joelho. Por
cima um vestido azul claro mais lindo que o céu de abril. Por cima a
cara da mamãe, que beleza! com aquele cabelo escuro fazendo um birote
luzido, e os bandós azulando de napolitano o moreno afogueado pelas
cores de Paris.

Paulino se levantou sem saber, com uma burundanga inexplicável de
instintos festivos no corpo, ``Mamma!'' que ele gritou. Teresinha virou
chamada, era o figliuolo. Não sei o que despencou na consciência dela,
correu ajoelhando a sedinha na calçada, e num transporte, machucando bem
delicioso até, apertou Paulino contra os peitos cheios. E Teresinha
chorou porque afinal das contas ela também era muito infeliz. Fernandez
dera o fora nela, e a indecisa tinha moçado duma vez. Vendo Paulino
sujo, aniquiladinho, sentiu toda a infelicidade própria, e meia que
desacostumou de repente da vida enfeitada que andava levando, chorou.

Só depois é que sofreu pelo filho, horroroso de magro e mais frágil que
a virtude. Decerto estava sofrendo com a mulatona da avó\ldots{} Um segundo
matutou levar Paulino consigo. Porém, escondendo de si mesma o
pensamento, era incontestável que Paulino havia de ser um trambolho pau
nas pândegas. Então olhou a roupinha dele. De fazenda boa não era, mas
enfim sempre servia. 

\textls[-5]{Agarrou nesse disfarce que apagava a consciência, ``meu filho está bem
tratado'', pra não pensar mais nele nunca mais. Deu um beijo na boquinha
molhada de gosma ainda, procurou engulir a lágrima, ``figliuolo'', não
foi possível, apertou muito, beijou muito. Foi"-se embora arranjando o
vestido.}

Paulino de"-pezinho, sem um gesto, sem um movimento, viu afinal lá longe
o vestido azul desaparecer. Virou o rostinho. Havia um pedaço de papel
de embrulho, todo engordurado, rolando engraçado no chão. Dar três
passos pra pegá"-lo\ldots{} Nem valia a pena. Sentou de novo no degrau. As
cores da tarde iam cinzando mansas. Paulino encostou a bochecha na
palminha da mão e meio enxergando, meio escutando, numa indiferença
exausta, ficou assim. Até a gosma escorria da boca aberta na mão dele.
Depois pingava na camisolinha. Que era escura pra não sujar.

\chapter{O ladrão}

--- Pega!

O berro, seria pouco mais de meia"-noite, crispou o silêncio no bairro
dormido, acordou os de sono mais leve, botando em tudo um arrepio de
susto. O rapaz veio na carreira desabalada pela rua.

--- Pega!

Nos corpos entrecortados, ainda estremunhando na angústia indecisa,
estalou nítida, sangrenta, a consciência do crime horroroso. O rapaz
estacara numa estralada de pés forçando pra parar de repente, sacudiu o
guarda estatelado:

--- Viu ele!

O polícia inda sem nexo, puxando o revólver:

--- Viu ele?

---P\ldots{}

Não perdeu tempo mais, disparou pela rua, porque lhe parecera ter
divisado um vulto correndo na esquina de lá. O guarda ficou sem saber o
que fazia, porém da mesma direção do moço já chegavam mais dois homens
correndo. O guarda eletrizado gritou:

--- Ajuda! e foi numa volada ambiciosa na cola do rapaz.

--- Pega! Pega! os dois perseguidores novos secundaram sem parar.
Alcançaram o moço na outra esquina, se informando com um retardatário
que só àquelas horas recolhia.

---\ldots{} é capaz que deu a volta lá em baixo\ldots{}

No cortiço, a única janela de frente se abriu, inundando de luz a
esquina. O retardatário virou"-se para os que chegavam:

--- Não! Voltem por aí mesmo! Ele dobrou a esquina lá de baixo! Fique
você, moço, vigiando aqui! Seu guarda, vem comigo!

Partiu correndo. Visivelmente era o mais expedito, e o grupo obedeceu,
se dividindo na carreira. O rapaz desapontara muito por ter de ficar
inativo, ele!

Justo ele que viera na frente!\ldots{} No ar umedecido, o frio principiou
caindo vagarento. Na janela do cortiço, depois de mandar pra cama o
homem que aparecera atrás dela, uma preta satisfeita de gorda assuntava.
Viu que a porta do 26 rangia com meia luz e os dois Moreiras saíram por
ela, afobados, enfiando os paletós. O Alfredinho até derrubou o chapéu,
voltou pra pegar, hesitou, acabou tomando a direção do mano.

O guarda com o retardatário, já tinham dobrado a esquina lá de baixo.
Uma ou outra janela acordava numa cabeça inquieta, entre agasalhos.
Também os dois perseguidores que tinham voltado caminho, já dobravam a
outra esquina. Mas foi a preta, na calma, quem percebeu que o quarteirão
fora cercado.

--- Então decerto ele escondeu no quarteirão mesmo.

O rapaz que só esperava um pretexto pra seguir na perseguição, deitou na
carreira. Parou.

--- A senhora então fique vigiando! Grite se ele vier!

E se atirou na disparada, desprezando escutar o ``Eu não! Deus te
livre!'' da preta, se retirando pra dentro porque não queria história
com o cortiço dela não. Pouco depois dos Moreiras, virada a esquina de
baixo, o rapaz alcançou o grupo dos perseguidores, na algazarra. Um dos
manos perguntava o que era. E o moço:

--- Pegaram!

--- Safado\ldots{} ele\ldots{}

--- Deixa de lero"-lero, seu guarda! assim ele escapa!

Aliás fora tudo um minuto. Vinha mais gente chegando.

--- O que foi?

--- Eu vou na esquina de lá, senão ele escapa outra vez!

--- Vá mesmo! Olha, vá com ele, você, para serem dois. Seu guarda! o
senhor é que pode pular no jardim!

--- Mas é que\ldots{}

--- Então bata na casa, p\ldots{}

O polícia inda hesitou um segundo, mas de repente encorajou:

--- Vam'lá!

Foram. Foi todo o grupo, agora umas oito pessoas. Ficou só o velho que
já não podia nem respirar da corridinha. Os dois manos, meio irritados
com a insignificância deles a que ninguém esclarecera o que havia,
ficaram também, castigando os perseguidores com ficarem. Lá no escuro do
ser estavam desejando que o ladrão escapasse, só pra o grupo não
conseguir nada. Um garoto de rua estava ali rente, se esfregando tremido
em todos, abobalhado de frio. Um dos Moreiras se vingou:

--- Vai pra casa, guri!\ldots{} de repente vem um tiro\ldots{}

--- Será que ele atira mesmo! perguntou o baita que chegava.

E o velho:

--- Tá claro! Quando o Salvini, aquele um que sufocou a mulher no Bom
Retiro, ficou cercado\ldots{}

Mas de súbito o apito do guarda agarrou trilando nos peitos, em firmatas
alucinantes. Todos recuaram, virados pro lado do apito. Várias janelas
fecharam.

O grupo estacara em frente de umas casas, quase no meio do quarteirão.
Eram dois sobradinhos gêmeos, paredes"-meias, na frente e nos lados
opostos os canteiros de burguesia difícil. Os perseguidores trocavam
palavras propositalmente em voz muito alta. O homem decerto ficava
amedrontado com tanta gente. Se entregava, era inútil lutar\ldots{} Em qual
das casas bater? O que vira o fugitivo pular no jardinzinho, quem sabe
um dos rapazes guardando a esquina, não estava ali pra indicar. Aliás
ninguém pusera reparo em quem falara. Os mais cuidadosos, três, tinham
se postado na calçada fronteira, junto ao portão entreaberto, bom pra
esconder. Se miravam ressabiados, com um bocado de vergonha. Mas um
sorrindo:

--- Tenho família.

--- Idem.

--- Pode vir alguma bala\ldots{}

--- Eu me armei, por via das dúvidas!

Quase todas as janelas estavam iluminadas, botando um ar de festa
inédito na rua. Saía mais gente encapuçada nas portas, coleção morna de
pijamas comprados feitos, transbordando pelos capotes mal vestidos. O
guarda estava tonto, sustentando posição aos olhos do grupo que dependia
dele. Mas lá vinham mais dois polícias correndo. Aí o guarda apitou com
entusiasmo e foi pra bater numa das casas. Mas da janela da outra jorrou
de chofre no grupo uma luz, todos recuaram. Era uma senhora, ainda se
abotoando.

--- Que é! que foi que houve, meu Deus!

--- Dona, acho que entrou um homem na sua casa que\ldots{}

--- Ai, meu Deus!

--- \ldots{} a gente veio\ldots{}

--- Nossa Senhora! meus filhos!

Desapareceu na casa. De repente escutou"-se um choro horrível de criança
lá dentro. Um segundo todos ficaram petrificados. Mas era preciso salvar
o menino, e à noção de ``menino'' um ardor de generosidade inflamou
todos. Avançaram, que pedir licença nem nada! uns pulando a gradinha,
outros já se ajudando a subir pela janela mesmo, outros forçando a
porta.

Que se abriu. A senhora apareceu, visão de pavor, desgrenhada, com as
três crianças. A menina, seus oito anos, grudada na saia da mãe, soltava
gritos como se a tivessem matando. A decisão foi instantânea, a imagem
da desgraça virilizara o grupo. A italiana de uma das casas operárias
defronte, vira tudo, nem se resguardava: veio no camisolão, abriu com
energia passagem pelos homens, agarrou a menina nos braços, escudando"-a
com os ombros contra tiros possíveis, fugira pra casa. Um dos homens
imitando a decidida agarrara outra criança, e empurrando a senhora com o
menorzinho no colo, levara tudo se esconder na casa da italiana. Os
outros se dividiram. Barafustaram pela casa aberta, alguns forçaram num
átimo a porta vizinha, tudo fácil de abrir, donos em viagem, a casa se
iluminou toda. Veio um gritando na janela do sobrado:

--- Por trás não fugiu, o muro é alto!

--- Ói lá!

Era a mocetona duma das casas operárias fronteiras, a ``vanyti"-case'' de
metalzinho esmaltado na mão, largara de se empoar, apontando. Toda a
gente parou estarrecida, adivinhando um jeito de se resguardar do
facínora. Olharam pra mocetona. Ela apontava no alto, aos gritos. Era no
telhado. Um dos cautelosos, não se enxergava bem por causa das árvores,
criou coragem, se abaixou e pôde ver. Deu um berro, avisando:

--- Está lá!

E veio feito uma bala, atravessando a rua, se resguardar na casa onde
empoleirara o ladrão. Os dois comparsas dele o imitaram. As janelas em
frente se fecham rápidas, bateu uma escureza sufocante. E os polícias, o
rapaz, todos tinham corrido pra junto do homem que vira, se escondendo
com ele, sem saber do que, de quem, a evidência do perigo independendo
já das vontades. Mas logo um dos polícias reagindo, sacudiu o
horrorizado, fazendo"-o voltar a si, perguntando gritado, com raiva. E a
raiva contra o cauteloso dominou o grupo. Ele enfim respondeu:

--- Eu também vi\ldots{} (mal podia falar) no telhado\ldots{}

--- Dissesse logo!

--- Está no telhado!

--- Vá pra casa, medroso!

--- Medroso não!

O rapaz atravessou a rua correndo, pra ver se enxergava ainda. O grupo
estourou de novo pelas duas casas a dentro.

--- Ele não tem pra onde pular!

--- Coitado!

--- Que cuidado! ele que venha!

--- Falei ``coitado''\ldots{}

Nos quintais dos fundos mais gente inspecionava o telhado único das
casas gêmeas. Não havia por onde fugir. E a caça continuava sanhuda. Os
dois sobrados foram esmiuçados, quarto por quarto, não houve
guarda"-roupa que não abrissem, examinaram tudo. Nada.

--- Mas não há nada! um falou.

--- Quem sabe se entrou no forro?

--- Entrou no forro!

--- Tem clarabóia?

O rapaz, do outro lado da rua, examinara bem. Na parte de frente do
telhado, positivamente o homem não estava mais. Algumas janelas se
entreabriram de novo, medrosas, riscando luz nas calçadas.

--- Pegaram?

---P\ldots{}

Mas alguém lhe segurara o braço, virou com defesa.

--- Meu filho! olhe a sua asma! Deixe, que os outros pegam! Está tão
frio!\ldots{}

O rapaz, deu um desespero nele, a assombração medonha da asma\ldots{} Foi
vestindo maquinalmente o sobretudo que a mãe trouxera.

--- Olha!\ldots{} ah, não é\ldots{} Também não sei pra que o prefeito põe tanta
árvore na rua!

--- Mas afinal o que foi, hein? perguntaram alguns, chegados tarde
demais pra se apaixonarem pelo caso.

--- Eu nem não sei!\ldots{} diz"-que estão pegando um ladrão.

--- Vamos pra casa, filhinho!\ldots{}

\ldots{} aquele fantasma da sufocação, peito chiando noite inteira, nem podia
mais nada\ldots{} Virou com ódio pro sabetudo:

--- Quem lhe contou que é ladrão?

Brotou em todos a esperança de alguma coisa pior.

--- O que é, hein?

A pergunta vinha da mulher sem nenhum prazer. O rapaz olhou"-a, aquele
demônio da asma\ldots{} deu de ombros, nem respondeu. Ele mesmo nem sabia
certo, entrara do trabalho, apenas despira o sobretudo, ainda estava
falando com a mãe já na cama, pedindo a bênção, quando gritaram
``Pega!'' na rua. Saíra correndo, vira o guarda não muito longe, um
vulto que fugia, fora ajudar. Mas aquele demônio medonho da asma\ldots{} O
anulou uma desesperança rancorosa. Entre os dentes:

--- Desgraçado\ldots{}

Foi"-se embora. De raiva. A mãe mal o pôde seguir, quase correndo, feliz!
feliz por ganhar o filho àquela morte certa.

Agora a maioria dos perseguidores saíra na rua. Nem no interior do
telhado encontraram o homem. Como fazer?

--- Ficou gente no quintal, vigiando?

--- Chi! tem pra uns deiz decidido lá!

Era preciso calma. Lá na janela da mocetona operária começara uma bulha
desgraçada. Os irmãos mais novos estavam dando um baile, nela, primeiro
insultando, depois caçoando que ela nem não tinha visto nada, só medo.
Ela jurava que sim, se apoiava no medroso que enxergara também, mas ele
não estava mais ali tinha ido embora, danado de o chamarem medroso,
esses bestas! A mocetona gesticulava, com o metalzinho da
``vanity"-case'' brilhando no ar. Afinal acabou atirando com a caixinha
bem na cara do irmão próximo e feriu. Veio a mãe, veio o pai, precisou
vir mais gente, que os irmãos cegados com a gota de sangue queriam
massacrar a mocetona.

Organizou"-se uma batida em regra, eram uns vinte. As demais casas
vizinhas estavam sendo varejadas também, quem sabe\ldots{} Alguns foram"-se
embora que tinha muita gente, não eram necessários mais. Mas paravam
pelas janelas, pelas portas, respondendo. Nascia aquela vontade de
conversar, de comentar, lembrar casos. Era como se se conhecessem
sempre.

--- Te lembra, João, aquele bebo no boteco da\ldots{}

--- Nem me!\ldots{}

Não encontraram nada nas casas e todos vieram saindo para as calçadas
outra vez. Ninguém desanimara, no entanto. Apenas despertara em todos
uma vontade de alívio, todos certos que o ladrão fugira, estava longe,
não havia mais perigo pra ninguém.

O guarda conversava pabulagem, bem distraído num grupo, do outro lado da
rua. Veio chegando, era a vergonha do quarteirão, a mulher do português
das galinhas. Era uma rica, linda com aqueles beiços largos, enquanto o
Fernandes quarentão lá partia no ``Ford'' passar três, quatro dias na
granja de Santo André. Ela, quem disse ir com ele! Chegava o entregador
da ``Noite'', batia, entrava. Ela fazia questão de não ter criada, comia
de pensão, tão rica! Vinha o mulato da marmita, pois entrava! E depois
diz"-que vivia sempre com doença, chamando cada vez era um médico novo,
desses que ainda não têm automóvel. Até o padeirinho da tarde, que tinha
só\ldots{} quinze? dezesseis anos? entrava, ficava tempo lá dentro.

O jornaleiro negava zangado, que era só pra conversar, senhora boa, mas
o entregadorzinho do pão não dizia nada, ficava se rindo, com sangue até
nos olhos, de vergonha gostosa.

Foi um silêncio carregado, no grupo, assim que ela chegou. As duas
operárias honestas se retiraram com fragor, facilitando os homens. Se
espalhou um cheiro por todos, cheiro de cama quente, corpo ardente e
perfumado recendente. Todos ficaram que até a noite perdera a umidade
gélida. De fato, a neblininha se erguera, e a cada uma janela que
fechava, vinha pratear mais forte os paralelepípedos uma calma elevada
de rua.

Vários grupos já não tinham coesão possível, bastante gente ia dormir.
Por uma das janelas agora, pouco além das duas casas, se via um moço
magro, de cabelo frio escorrendo, num pijama azul, perdido o sono,
repetindo o violino. Tocava uma valsa que era boa, deixando aquele gosto
de tristeza no ar.

Nisto a senhora não pudera mais consigo, muito inquieta com a casa
aberta em que tantas pessoas tinham entrado, apareceu na porta da
italiana. Esta insistia com a outra para ficar dormindo com ela, a
senhora hesitava, precisava ir ver a casa, mas tinha medo, sofria muito,
olhos molhados, sem querer.

A conversava vantajosa do grupo da portuguesa parou com a visão triste.
E o guarda, sem saber que era mesmo ditado pela portuguesa, heróico se
sacrificou. Destacou"-se do grupo insaciável, foi acompanhar a senhora (a
portuguesa bem que o estaria admirando), foi ajudar a senhora mais a
italiana a fechar tudo. Até não havia necessidade dela dormir na casa da
outra, ele ficava guardando, não arredava pé. E sem querer, dominado
pelos desejos, virou a cara, olhou lá do outro lado da calçada a
portuguesa fácil. Talvez ela ficasse ali conversando com ele, primeiro
só conversando, até de"-manhã\ldots{}

Alguns dos perseguidores, agrupados na porta da casa, tinham se
esquecido, naquela conversa apaixonada, o futebol do sábado. Se
afastaram, deixando a dona entrar com o guarda. Olharam"-na com piedade
mas sorrindo, animando a coitada. Nisto chegou com estalidos seu Nitinho
e tudo se resolveu. Seu Nitinho era compadre da senhora, muito amigo da
família, morava duas quadras longe. Viera logo com a espingarda
passarinheira dos domingos, proteger a comadre. Dormiria na casa também,
ela podia ficar no seu bem"-bom com os filhos, salva com a proteção. E a
senhora mais confiante entrou na casa.

--- É, não há nada.

Foi um alívio em todos. A italiana já trazia as crianças se rindo,
falando alto, gesticulando muito, insistindo na oferta do leite. Pois a
italiana assim mesmo conseguiu vencer a reserva da outra, e invadiu a
cozinha, preparando um café. A lembrança do café animou todos. Os
perseguidores se convidaram logo, com felicidade. Só o pobre do guarda,
mais uma vez sacrificado, não pôde com o sexo, foi se reunir ao grupo da
portuguesa.

Eis que a valsa triste acabou. Mas da sombra das árvores em frente, umas
quatro ou cinco pessoas, paralisadas pela magnitude da música, tinham
por alegria, só por pândega, pra desopilar, pra acabar com aquela
angústia miúda que ficara, nem sabiam! tinham\ldots{} enfim, pra fazer com
que a vida fosse engraçada um segundo, tinham arrebentado em aplausos e
bravos. E todos, com os aplausos, todos, o grupo da portuguesa, a
mocetona com os manos já mansos, os perseguidores da porta, dois ou três
mais longe, todos desataram na risada. Só o violinista não riu. Era a
primeira consagração. E o peitinho curto dele até parou de bater.

Soaram duas horas num relógio de parede. Os que tinham relógio,
consultaram. Um galo cantou. O canto firme lavou o ar e abriu o orfeão
de toda a galaria do bairro, uma bulha encarnada radiando no céu lunar.
O violinista reiniciara a valsa, porque tinham ido pedir mais música a
ele. Mas o violino, bem correto, só sabia aquela valsa mesmo. E a valsa
dançava queixosa outra vez, enchendo os corações.

--- Eu numa varsa dessa, mulher comigo, eu que mando!

E olhou a portuguesa bem nos olhos. Ela baixou os dela, puros,
umedecendo os lábios devagar. Os outros ficaram com ódio da declaração
do guarda lindo, bem arranjado na farda. Se sentiram humilhados nos
pijamas reles, nos capotes mal vestidos, nos rostos sujos de cama.
Todos, acintosamente, por delicadeza, ocultando nas mãos cruzadas ou
enfiadas nos bolsos, a indiscrição dos corpos. A portuguesa, em êxtase,
divinizada, assim violentada altas horas, por sete homens, traindo pela
primeira vez, sem querer, violentada, o marido da granja.

Na porta da casa, a italiana triunfante distribuía o café. Um momento
hesitou, olhando o guarda do outro lado da rua. Mas nisto fagulhou uma
risadinha em todos lá no grupo, decerto alguma piada sem"-vergonha, não!
não dava café ao guarda! Pensou na última xícara, atravessou
teatralmente a rua olhando o guarda, ele ainda imaginou que a xícara era
pra ele. E a italiana entrou na casa dela levando o café para o marido
na cama, dormindo porque levantava às quatro, com o trabalho em
Pirituba.

Foi um primeiro mal"-estar no grupo da portuguesa: todos ficaram com
vontade de beber um café bem quentinho. Se ela convidasse\ldots{} Ela bem
queria mas não achava razão. O guarda se irritou, qual! não tinha
futuro! assim com tanta gente ali\ldots{} Perdera o café. Ainda inventou ir
até a casa, saber se a senhora não precisava de nada. Mas a italiana
olhara pra ele com tanta ofensa a xícara bem agarrada na mão, que um
pudor o esmagou. Ficou esmagado, desgostoso de si, com um princípio de
raiva da portuguesa. De raiva, deu um trilo no apito e se foi, rondando
os seus domínios.

Os perseguidores tinham bebido o café, já agora perfeitamente repostos
em suas consciências\ldots{} Lhes coçava um pouco de vergonha na pele, tinham
perseguido quem? \ldots{} Mas ninguém não sabia. Uns tinham ido atrás dos
outros levados pelos outros, seria ladrão?\ldots{}

--- Bem vou chegando.

--- É. Não tem mais nada.

Boa"-noite, boa"-noite\ldots{}

E tudo se dispersou. Ainda dois mais corajosos acompanharam a portuguesa
até a porta dela, na esperança nem sabiam do quê. Se despediram
delicados, conhecedores de regras, se contando os nomes próprios, seu
criado. Ela, fechava a porta, perdidos os últimos passos além, se apoiou
no batente, engolindo silêncio. Ainda viria algum, pegava nela,
agarrava\ldots{} Amarrou violentamente o corpo nos braços, duas lágrimas
rolaram insuspeitas. Foi deitar sem ninguém.

A rua estava de novo quase morta, janelas fechadas. A valsa acabara o
bis. Sem ninguém. Só o violinista estava ali, fumando, fumegando muito,
olhando sem ver, totalmente desamparado, sem nenhum sono, agarrado a não
sei que esperança de que alguém, uma garota linda, um fotógrafo, um
milionário disfarçado, lhe pedisse pra tocar mais uma vez. Acabou
fechando a janela também.

Lá na outra esquina do outro quarteirão, ficara um último grupinho de
três, conversando. Mas é que lá passava bonde.

\bigskip

\hfill{}(1930--1941--1942)

\chapter{Primeiro de maio}

No grande dia Primeiro de Maio, não eram bem seis horas e já o 35 pulara
da cama, afobado. Estava bem disposto, até alegre, ele bem afirmara aos
companheiros da Estação da Luz que queria celebrar e havia de celebrar.
Os outros carregadores mais idosos meio que tinham caçoado do bobo,
viesse trabalhar que era melhor, trabalho deles não tinha feriado. Mas o
35 retrucava com altivez que não carregava mala de ninguém, havia de
celebrar o dia deles. E agora tinha o grande dia pela frente.

Dia dele\ldots{} Primeiro quis tomar um banho pra ficar bem digno de existir.
A água estava gelada, ridente, celebrando, e abrira um sol enorme e frio
lá fora. Depois fez a barba. Barba era aquela penuginha meio loura, mas
foi assim mesmo buscar a navalha dos sábados, herdada do pai, e se
barbeou. Foi se barbeando. Nu só da cintura pra cima por causa da mamãe
por ali, de vez em quando a distância mais aberta do espelhinho refletia
os músculos violentos dele, desenvolvidos desarmoniosamente nos braços,
na peitaria, no cangote, pelo esforço quotidiano de carregar peso. O 35
tinha um ar glorioso e estúpido. Porém ele se agradava daqueles músculos
intempestivos, fazendo a barba.

\textls[-15]{Ia devagar porque estava matutando. Era a esperança dum turumbamba
macota, em que ele desse uns socos formidáveis nas fuças dos polícias.
Não teria raiva especial dos polícias, era apenas a ressonância vaga
daquele dia. Com seus vinte anos fáceis, o 35 sabia, mais da leitura dos
jornais que de experiência, que o proletariado era uma classe oprimida.
E os jornais tinham anunciado que se esperava grandes ``motins'' do
Primeiro de Maio, em Paris, em Cuba, no Chile, em Madri.}

\textls[-15]{O 35 apressou a navalha de puro amor. Era em Madri, no Chile que ele não
tinha bem lembrança se ficava na América mesmo, era a gente dele. Uma
piedade, um beijo lhe saía do corpo todo, feito proteção sadia de macho,
ia parar em terras não sabidas, mas era a gente dele, defender,
combater, vencer\ldots{} Comunismo? \ldots{} Sim, talvez fosse isso. Mas o 35 não
sabia bem direito, ficava atordoado com as notícias, os jornais falavam
tanta coisa, faziam tamanha mistura de Rússia, só sublime ou só
horrenda, e o 35 infantil estava por demais machucado pela experiência
pra não desconfiar, o 35 desconfiava. Preferia o turumbamba porque não
tinha medo de ninguém, nem do Carnera, ah, um soco bem nas fuças dum
polícia\ldots{} A navalha apressou o passo outra vez. Mas de repente o 35 não
imaginou mais em nada por causa daquele bigodinho de cinema que era a
melhor preciosidade de todo o seu ser. Lembrou aquela moça do
apartamento, é verdade, nunca mais tinha passado lá pra ver se ela
queria outra vez, safada! Riu.}

Afinal o 35 saiu, estava lindo. Com a roupa preta de luxo, um nó errado
na gravata verde com listinhas brancas e aqueles admiráveis sapatos de
pelica amarela que não pudera sem comprar. O verde da gravata, o amarelo
dos sapatos, bandeira brasileira, tempos de grupo escolar\ldots{} E o 35
comoveu num hausto forte, querendo bem o seu imenso Brasil, imenso
colosso gigante, foi andando depressa, assobiando. Mas parou de sopetão
e se orientou assustado. O caminho não era aquele, aquele era o caminho
do trabalho.

Uma indecisão indiscreta o tornou consciente de novo que era o Primeiro
de Maio, ele estava celebrando e não tinha o que fazer. Bom, primeiro
decidiu ir na cidade pra assuntar alguma coisa. Mas podia seguir por
aquela direção mesmo, era uma volta, mas assim passava na Estação da Luz
dar um bom"-dia festivo aos companheiros trabalhadores. Chegou lá,
gesticulou o bom"-dia festivo, mas não gostou porque os outros riram
dele, bestas. Só que em seguida não encontrou nada na cidade, tudo
fechado por causa do grande dia Primeiro de Maio. Pouca gente na rua.
Deviam de estar almoçando já, pra chegar cedo no maravilhoso jogo de
futebol escolhido pra celebrar o grande dia. Tinha mas era muito
polícia, polícia em qualquer esquina, em qualquer porta cerrada de bar e
de café, nas joalherias, quem pensava em roubar! nos bancos, nas casas
de loteria. O 35 teve raiva dos polícias outra vez.

E como não encontrasse mesmo um conhecido, comprou o jornal pra saber.
Lembrou de entrar num café, tomar por certo uma média, lendo. Mas a
maioria dos cafés estavam de porta cerrada e o 35 mesmo achou que era
preferível economizar dinheiro por enquanto, porque ninguém não sabia o
que estava pra suceder. O mais prático era um banco de jardim, com
aquele sol maravilhoso. Nuvens? umas nuvenzinhas brancas, ondulando no
ar feliz. Insensivelmente o 35 foi se encaminhando de novo para os lados
do Jardim da Luz. Eram os lados que ele conhecia, os lados em que
trabalhava e se entendia mais. De repente lembrou que ali mesmo na
cidade tinha banco mais perto, nos jardins do Anhangabaú. Mas o Jardim
da Luz ele entendia mais. Imaginou que a preferência vinha do Jardim da
Luz ser mais bonito, estava celebrando. E continuou no passo em férias.

Ao atravessar a estação achou de novo a companheirada trabalhando.
Aquilo deu um mal"-estar fundo nele, espécie não sabia bem, de
arrependimento, talvez irritação dos companheiros, não sabia. Nem
quereria nunca decidir o que estava sentindo já\ldots{} Mas disfarçou bem,
passando sem parar, se dando por afobado, virando pra trás com o braço
ameaçador, ``Vocês vão ver!\ldots{}'' Mas um riso aqui, outro riso acolá, uma
frase longe, os carregadores companheiros, era tão amigo deles, estavam
caçoando. O 35 se sentiu bobo, impossível recusar, envilecido. Odiou os
camaradas.

\textls[-15]{Andou mais depressa, entrou no jardim em frente, o primeiro banco era a
salvação, sentou"-se. Mas dali algum companheiro podia divisar ele e
caçoar mais, teve raiva. Foi lá no fundo do jardim campear banco
escondido. Já passavam negras disponíveis por ali. E o 35 teve uma ideia
muito não pensada, recusada, de que ele também estava uma espécie de
negra disponível, assim. Mas não estava não, estava celebrando, não
podia nunca acreditar que estivesse disponível e não acreditou. Abriu o
jornal. Havia logo um artigo muito bonito, bem pequeno, falando na
nobreza do trabalho, nos operários que eram também os ``operários da
nação'', é isso mesmo. O 35 se orgulhou todo comovido. Se pedissem pra
ele matar, ele matava roubava, trabalhava grátis, tomado dum sublime
desejo de fraternidade, todos os seres juntos, todos bons\ldots{} Depois
vinham as notícias. Se esperavam ``grandes motins'' em Paris, deu uma
raiva tal no 35. E ele ficou todo fremente, quase sem respirar,
desejando ``motins'' (devia ser turumbamba) na sua desmesurada força
física, ah, as fuças de algum\ldots{} polícia? polícia. Pelo menos os safados
dos polícias.}

Pois estava escrito em cima do jornal: em São Paulo a Polícia proibira
comícios na rua e passeatas, embora se falasse vagamente em motins
de"-tarde no Largo da Sé. Mas a polícia já tomara todas as providências,
até metralhadoras, estavam em cima do jornal, nos arranha"-céus,
escondidas, o 35 sentiu um frio. O sol brilhante queimava, banco na
sombra? Mas não tinha, que a Prefeitura, pra evitar safadez dos
namorados, punha os bancos só bem no sol. E ainda por cima era aquela
imensidade de guardas e polícias vigiando que nem bem a gente punha a
mão no pescocinho dela, trilo. Mas a Polícia permitiria a grande reunião
proletária, com discurso do ilustre Secretário do Trabalho, no magnífico
pátio interno do Palácio das Indústrias, lugar fechado! A sensação foi
claramente péssima. Não era medo, mas por que que a gente havia de ficar
encurralado assim! é! É pra eles depois poderem cair em cima da gente,
(palavrão)! Não vou! não sou besta! Quer dizer: vou sim! desaforo!
(palavrão), socos, uma visão tumultuária, rolando no chão, se machucava
mas não fazia mal, saíam todos enfurecidos do Palácio das indústrias,
pegavam fogo no Palácio das Indústrias, não! a indústria é a gente,
``operários da nação'', pegavam fogo na igreja de São Bento mais próxima
que era tão linda por ``drento'', mas pra que pegar fogo em nada! (O 35
chegara até a primeira comunhão em menino\ldots{}), é melhor a gente não
pegar fogo em nada; vamos no Palácio do Governo, exigimos tudo do
Governo, vamos com o general da Região Militar, deve ser gaúcho, gaúcho
só dá é farda, pegamos fogo no palácio dele. Pronto. Isso o 35
consentiu, não porque o tingisse o menor separatismo (e o aprendido no
grupo escolar?) mas nutria sempre uma espécie de despeito por São Paulo
ter perdido na revolução de 32. Sensação aliás quase de esporte, questão
de Palestra"-Coríntians, cabeça inchada, porque não vê que ele havia de
se matar por causa de uma besta de revolução diz"-que democrática, vão
``eles''!\ldots{} Se fosse o Primeiro de Maio, pelos menos\ldots{} O 35 percebeu
que se regava todo por ``drento'' dum espírito generoso de sacrifício.
Estava outra vez enormemente piedoso, morreria sorrindo, morrer\ldots{} Teve
uma nítida, envergonhada sensação de pena. Morrer assim tão lindo, tão
moço. A moça do apartamento\ldots{}

Salvou"-se lendo com pressa, ôh! os deputados trabalhistas chegavam agora
às nove horas, e o jornal convidavam (sic) o povo pra ir na Estação do
Norte (a estação rival, desapontou) pra receber os grandes homens. Se
levantou mandado, procurou o relógio da torre da Estação da Luz, ora!
não dava mais tempo! quem sabe se dá!

\textls[-5]{Foi correndo, estava celebrando, raspou distraído o sapato lindo na
beira de tijolo do canteiro (palavrão), parou botando um pouco de guspe
no raspão, depois engraxo, tomou o bonde pra cidade, mas dando uma
voltinha pra não passar pelos companheiros da Estação. Que alvoroço por
dentro, ainda havia de aplaudir os homens. Tomou o outro bonde pro Brás.
Não dava mais tempo, ele percebia, eram quase nove horas quando chegou
na cidade, ao passar pelo Palácio das Indústrias, o relógio da torre
indicava nove e dez, mas o trem da Central sempre atrasa, quem sabe?
bom: às quatorze horas venho aqui, não perco, mas devo ir, são nossos
deputados no tal de congresso, devo ir. Os jornais não falavam nada dos
trabalhistas, só falavam dum que insultava muito a religião e exigia
divórcio, o divórcio o 35 achava necessário (a moça do apartamento\ldots{}),
mas os jornais contavam que toda a gente achava graça no homenzinho
``Vós, burgueses'', e toda a gente, os jornais contavam, acabaram se
rindo do tal do deputado. E o 35 acabou não achando mais graça nele.
Teve até raiva do tal, um soco é que merecia. E agora estava torcendo
pra não chegar com tempo na Estação.}

Chegou tarde. Quase nada tarde, eram apenas nove e quinze. Pois não
havia mais nada, não tinha aquela multidão que ele esperava, parecia
tudo normal. Conhecia alguns carregadores dali também e foi perguntar.
Não, não tinham reparado nada, decerto foi aquele grupinho que parou na
porta da Estação, tirando fotografia. Aí outro carregador conferiu que
eram os deputados sim, porque tinham tomado aqueles dois sublimes
automóveis oficiais. Nada feito.

Ao chegar na esquina o 35 parou pra tomar o bonde, mas vários bondes
passaram. Era apenas um moço bem"-vestidinho, decerto à procura de
emprego por aí, olhando a rua. Mas de repente sentiu fome e se reachou.
Havia por dentro, por ``drento'' dele um desabalar neblinoso de ilusões,
de entusiasmo e uns raios fortes de remorso. Estava tão desagradável,
estava quase infeliz\ldots{} Mas como perceber tudo isso se ele precisava não
perceber!\ldots{} O 35 percebeu que era fome.

Decidiu ir a"-pé pra casa, foi a"-pé, longe, fazendo um esforço penoso
para achar interesse no dia. Estava era com fome, comendo aquilo
passava. Tudo deserto, era por ser feriado, Primeiro de Maio. Os
companheiros estavam trabalhando, de vez em quando um carrego, o mais
eram conversas divertidas, mulheres de passagem, comentadas, piadas
grossas com as mulatas do jardim, mas só as bem limpas mais caras, que
ele ganhava bem, todos simpatizavam logo com ele, ora por que que hoje
me deu de lembrar aquela moça do apartamento!\ldots{} Também: moça morando
sozinha é no que dá. Em todo caso, pra acabar o dia era um ideia ir lá,
com que pretexto?\ldots{} Devia ter ido em Santos, no piquenique da
Mobiliadora, doze paus convite, mas o Primeiro de Maio\ldots{} Recusara,
recusara repetindo o ``não'' de repente com raiva, muito interrogativo,
se achando esquisito daquela raiva que lhe dera. Então conseguiu
imaginar que esse piquenique monstro, aquele jogo de futebol que
apaixonava eles todos, assim não ficava ninguém pra celebrar o Primeiro
de Maio, sentiu"-se muito triste, desamparado. É melhor tomo por esta
rua. Isso o 35 percebeu claro, insofismável que não era melhor, ficava
bem mais longe. Ara, que tem! Agora ele não podia se confessar mais que
era pra não passar na Estação da Luz e os companheiros não rirem dele
outra vez. E deu a volta, deu com o coração cerrado de angústia
indizível, com um vento enorme de todo o ser soprando ele pra junto dos
companheiros, ficar lá na conversa, quem sabe? trabalhar\ldots{} E quando a
mãe lhe pôs aquela esplêndida macarronada celebrante sobre a mesa, o 35
foi pra se queixar ``Estou sem fome, mãe''. Mas a voz lhe morreu na
garganta.

Não eram bem treze horas e já o 35 desembocava no parque Pedro \textsc{ii} outra
vez, à vista do Palácio das Indústrias. Estava inquieto mas modorrento,
que diabo de sol pesado que acaba com a gente, era por causa do sol. Não
podia mais se recusar o estado de infelicidade, a solidão enorme,
sentida com vigor. Por sinal que o parque já se mexia bem agitado.
Dezenas de operários, se via, eram operários endomingados, vagueavam,
por ali, indecisos, ar de quem não quer. Então nas proximidades do
palácio, os grupos se apinhavam, conversando baixo, com melancolia de
conspiração. Polícias por todo lado.

O 35 topou com o 486, grilo quase amigo, que policiava na Estação da
Luz. O 486 achara jeito de não trabalhar aquele dia porque se pensava
anarquista, mas no fundo era covarde. Conversaram um pouco de entusiasmo
semostradeiro, um pouco de primeiro de maio, um pouco de ``motim''. O
486 era muito valentão de boca, o 35 pensou. Pararam bem na frente do
Palácio das Indústrias que fagulhava de gente nas sacadas, se via que
não eram operários, decerto os deputados trabalhistas, havia até moças,
se via que eram distintas, todos olhando para o lado do parque onde eles
estavam.

Foi uma nova sensação tão desagradável que ele deu de andar quase
fugindo, polícias, centenas de polícias, moderou o passo como quem
passeia. Nas ruas que davam pro parque tinha cavalarias aos grupos,
cinco, seis escondidos na esquina, querendo a discrição de não ostentar
força e ostentando. Os grilos ainda não faziam mal, são uns (palavrão)!
O palácio dava ideia duma fortaleza enfeitada, entrar lá dentro, eu!\ldots{}
O 486 então, exaltadíssimo, descrevia coisas piores, massacres horrendos
de ``proletários'' lá dentro, descrevia tudo com a visibilidade dos
medrosos, o pátio fechado, dez mil proletários no pátio e os polícias lá
em cima nas janelas, fazendo pontaria na maciota.

Mas foi só quando aqueles três homens bem vestidos, se via que não eram
operários, se dirigindo aos grupos vagueantes, falaram pra eles em voz
alta: ``Podem entrar! não tenham vergonha! podem entrar!'' com voz de
mandando assim na gente\ldots{} O 35 sentiu medo franco. Entrar ele! Fez como
os outros operários: era impossível assim soltos, desobedecer aos três
homens bem vestidos, com voz mandando, se via que não eram operários.
Foram todos obedecendo, se aproximando das escadarias, mas o maior
número longe da vista dos três homens, torcia caminho, iam se espalhar
pelas outras alamedas do parque, mais longe.

Esses movimentos coletivos de recusa, acordaram a covardia do 35. Não
era medo, que ele se sentia fortíssimo, era pânico. Era um puxar
unânime, uma fraternidade, era carícia dolorosa por todos aqueles
companheiros fortes tão fracos que estavam ali também pra\ldots{} pra
celebrar? pra\ldots{} O 35 não sabia mais pra quê. Mas o palácio era
grandioso por demais com as torres e as esculturas, mas aquela porção de
gente bem vestida nas escadas enxergando ele (teve a intuição violenta
de que estava ridiculamente vestido), mas o enclausuramento na casa
fechada, sem espaço de liberdade, sem ruas abertas pra avançar, pra
correr dos cavalarias, pra brigar\ldots{} E os polícias na maciota,
encarapitados nas janelas, dormindo na pontaria, teve ódio do 486,
idiota medroso! De repente o 35 pensou que ele era moço, precisava se
sacrificar: se fizesse um modo bem 40 visível de entrar sem medo no
palácio, todos haviam de seguir o exemplo dele. Pensou, não fez. Estava
tão opresso, se desfibrara tão rebaixado naquela mascarada de
socialismo, naquela desorganização trágica, o 35 ficou desolado duma
vez. Tinha piedade, tinha amor, tinha fraternidade, e era só. Era uma
sarça ardente, mas era sentimento só. Um sentimento profundíssimo,
queimando, maravilhoso, mas desamparado, mas desamparado. Nisto vieram
uns cavalarias, falando garantidos:

--- Aqui ninguém não fica não! a festa é lá dentro, me'rmão! no parque
ninguém não pára não!

Cabeças"-chatas\ldots{} E os grupos deram de andar outra vez, de cá para lá,
riscando no parque vasto, com vontade, com medo, falando baixinho,
mastigando incerteza. Deu um ódio tal no 35, um desespero tamanho,
passava um bonde, correu, tomou o bonde sem se despedir do 486, com ódio
do 486, com ódio do primeiro de maio, quase com ódio de viver.

O bonde subia para o centro mais uma vez. Os relógios marcavam quatorze
horas, decerto a celebração estava principiando, quis voltar, dava muito
tempo, três minutos pra descer a ladeira, teve fome. Não é que tivesse
fome, porém o 35 carecia de arranjar uma ocupação senão arrebentava. E
ficou parado assim, mais de uma hora, mais de duas horas, no largo da
Sé, diz"-que olhando a multidão.

Acabara por completo a angústia. Não pensava, não sentia mais nada. Uma
vagueza cruciante, nem bem sentida, nem bem vivida, inexistência
fraudulenta, cínica, enquanto o primeiro de maio passava. A mulher de
encarnado foi apenas o que lhe trouxe de novo à lembrança a moça do
apartamento, mas nunca que ele fosse até lá, não havia pretexto, na
certa que ela não estava sozinha. Nada. Havia uma paz, que paz sem cor
por dentro\ldots{}

\textls[-5]{Pelas dezessete horas era fome, agora sim, era fome. Reconheceu que não
almoçara quase nada, era fome, e principiou enxergando o mundo outra
vez. A multidão já se esvaziava, desapontada, porque não houvera nem uma
briguinha, nem uma correria no largo da Sé, como se esperava. Tinha
claros bem largos, onde os grupos dos polícias resplandeciam mais. As
outras ruas do centro, essas então quase totalmente desertas. Os cafés,
já sabe, tinham fechado, com o pretexto magnânimo de dar feriado aos
seus ``proletários'' também.}

E o 35 inerme, passivo, tão criança, tão já experiente da vida, não
cultivou vaidade mais: foi se dirigindo num passo arrastado para a
Estação da Luz, pra os companheiros dele, esse era o domínio dele. Lá no
bairro os cafés continuavam abertos, entrou num, tomou duas médias,
comeu bastante pão com manteiga, exigiu mais manteiga, tinha um fraco
por manteiga, não se amolava de pagar o excedente, gastou dinheiro,
queria gastar dinheiro, queria perceber que estava gastando dinheiro,
comprou uma maçã bem rubra, oitocentão! foi comendo com prazer até os
companheiros. Eles se ajuntaram, agora sérios, curiosos, meio inquietos,
perguntando pra ele. Teve um instinto voluptuoso de mentir, contar como
fora a celebração, se enfeitar, mas fez um gesto só, (palavrão),
cuspindo um muxoxo de desdém pra tudo.

Chegava um trem e os carregadores se dispersaram, agora rivais, colhendo
carregos em porfia. O 35 encostou na parede, indiferente, catando com
dentadinhas cuidadosas os restos da maçã, junto aos caroços. Sentia"-se
cômodo, tudo era conhecido velho, os choferes, os viajantes. Surgiu um
farrancho que chamou o 22. Foram subir no automóvel mas afinal, depois
de muita gritaria, acabaram reconhecendo que tudo não cabia no carro.
Era a mãe, eram as duas velhas, cinco meninos repartidos pelos colos e o
marido. Tudo falando: ``Assim não serve não! As malas não vão não!'' Aí
o chofer garantiu enérgico que as malas não levava, mas as maletas elas
``não largavam não'', só as malas grandes que eram quatro. Deixaram elas
com o 22, gritaram a direção e partiram na gritaria. Mais cabeça"-chata,
o 35 imaginou com muita aceitação.

O 22 era velhote. Ficou na beira da calçada com aquelas quatro malas
pesadíssimas, preparou a correia, mas coçou a cabeça.

--- Deixe que te ajudo, chegou o 35.

E foi logo escolhendo as duas malas maiores, que ergueu numa só mão, num
esforço satisfeito de músculos. O 22 olhou pra ele, feroz, imaginando
que 35 propunha rachar o galho. Mas o 35 deu um soco só de pândega no
velhote, que estremeceu socado e cambaleou três passos. Caíram na risada
os dois. Foram andando.

\bigskip

\hfill{}(1934--1942)

\chapter{O poço}

Ali pelas onze horas da manhã o velho Joaquim Prestes chegou no
pesqueiro. Embora fizesse força em se mostrar amável por causa da visita
convidada para a pescaria, vinha mal"-humorado daquelas cinco léguas de
fordinho cabritando na estrada péssima. Aliás o fazendeiro era de pouco
riso mesmo, já endurecido por setenta e cinco anos que o mumificavam
naquele esqueleto agudo e taciturno.

O fato é que estourara na zona a mania dos fazendeiros ricos adquirirem
terrenos na barranca do Mogi pra pesqueiros de estimação. Joaquim
Prestes fora dos que inventaram a moda, como sempre: homem cioso de suas
iniciativas, meio cultivando uma vaidade de família --- gente escoteira
por aqueles campos altos, desbravadora de terras. Agora Joaquim Prestes
desbravava pesqueiros na barranca fácil do Mogi. Não tivera que
construir a riqueza com a mão, dono de fazendas desde o nascer,
reconhecido como chefe, novo ainda. Bem rico, viajado, meio sem
quefazer, desbravava outros matos.

Fora o introdutor do automóvel naquelas estradas, e se o município agora
se orgulhava de ser um dos maiores produtores de mel, o devia ao velho
Joaquim Prestes, primeiro a se lembrar de criar abelhas ali. Falando o
alemão (uma das suas ``iniciativas'' goradas na zona) tinha uma
verdadeira biblioteca sobre abelhas. Joaquim Prestes era assim.
Caprichosíssimo, mais cioso de mando que de justiça, tinha a idolatria
da autoridade. Pra comprar o seu primeiro carro fora à Europa, naqueles
tempos em que os automóveis eram mais europeus que americanos. Viera uma
``autoridade'' no assunto. E o mesmo com as abelhas de que sabia tudo.
Um tempo até lhe dera de reeducar as abelhas nacionais, essas ``porcas''
que misturavam o mel com a samora. Gastou anos e dinheiro bom nisso,
inventou ninhos artificiais, cruzou as raças, até fez vir abelhas
amazônicas. Mas se mandava nos homens e todos obedeciam, se viu obrigado
a obedecer às abelhas que não se educaram um isto. E agora que ninguém
falasse perto dele numa inocente jeteí, Joaquim Prestes xingava. Tempo
de florada no cafezal ou nas fruteiras do pomar maravilhoso, nunca mais
foi feliz. Lhe amargavam penosamente aquelas mandassaias, mandaguaris,
bijuris que vinham lhe roubar o mel da Apis Mellifica.

E tudo o que Joaquim Prestes fazia, fazia bem. Automóveis tinha três.
Aquela marmon de luxo pra o levar da fazenda à cidade, em compras e
visitas. Mas como fosse um bocado estreita para que coubessem à vontade,
na frente, ele choferando e a mulher que era gorda (a mulher não podia
ir atrás com o mecânico, nem este na frente e ela atrás) mandou fazer
uma ``rolls"-royce'' de encomenda, com dois assentos na frente que
pareciam poltronas de hol, mais de cem contos. E agora, por causa do
pesqueiro e da estrada nova, comprara o fordinho cabritante, todo dia
quebrava alguma peça, que o deixava de mau"-humor.

Que outro fazendeiro se lembrara mais disso! Pois o velho Joaquim
Prestes dera pra construir no pesqueiro uma casa de verdade, de tijolo e
telha, embora não imaginasse passar mais que o claro do dia ali, de medo
da maleita. Mas podia querer descansar. E era quase uma casa"-grande se
erguendo, quarto do patrão, quarto pra algum convidado, a sala vasta, o
terraço telado, tela por toda a parte pra evitar pernilongos. Só
desistiu da água encanada porque ficava um dinheirão. Mas a casinha, por
detrás do bangalô, até era luxo toda de madeira aplainada, pintadinha de
verde pra confundir com os mamoeiros, os porcos de raça por baixo (isso
de fossa nunca!) e o vaso de esmalte e tampa. Numa parte destocada do
terreno, já pastavam no capim novo quatro vacas e o marido, na espera de
que alguém quisesse beber um leitezinho caracu. E agora que a casa
estava quase pronta, sua horta folhuda e uns girassóis na frente,
Joaquim Prestes não se contentara mais com a água da geladeira, trazida
sempre no forde em dois termos gordos, mandara abrir um poço.

Quem abria era gente da fazenda mesmo, desses camaradas que entendem um
pouco de tudo. Joaquim Prestes era assim. Tinha dez chapéus
estrangeiros, até um panamá de conto de réis, mas as meias, só usava
meias feitas pela mulher, ``pra economizar'' afirmava. Afora aqueles
quatro operários ali, que cavavam o poço, havia mais dois que lá estavam
trabucando no acabamento da casa, as marteladas monótonas chegavam até à
fogueira. E todos muito descontentes, rapazes de zona rica e bem servida
de progresso, jogados ali na ceva da maleita. Obedeceram, mandados, mas
corroídos de irritação.

Só quem estava maginando que enfim se arranjara na vida era o vigia,
esse caipira da gema, bagre soma dos alagados do rio, maleiteiro eterno
a viola e rapadura, mais a mulher e cinco famílias enfezadas. Esse
agora, se quisesse tinha leite, tinha ovos de legornes finas e horta de
semente. Mas lhe bastava imaginar que tinha. Continuava feijão com
farinha, e a carne"-seca do domingo.

Batera um frio terrível esse fim de julho, bem diferente dos invernos
daquela zona paulista, sempre bem secos nos dias claros e solares, e as
noites de uma nitidez sublime perfeitas pra quem pode dormir no quente.
Mas aquele ano umas chuvas diluviais alagavam tudo, o couro das
carteiras embolorava no bolso e o café apodrecia no chão.

No pesqueiro o frio se tornara feroz, lavado daquela umidade maligna
que, além de peixe, era só o que o rio sabia dar. Joaquim Prestes e a
visita foram se chegando pra fogueira dos camaradas, que logo
levantaram, machucando chapéu na mão, bom"-dia, bom"-dia. Joaquim tirou o
relógio do bolso, com muita calma, examinou bem que horas eram. Sem
censura aparente, perguntou aos camaradas se ainda não tinham ido
trabalhar.

Os camaradas responderam que já tinham sim, mas que com aquele tempo
quem aguentava permanecer dentro do poço continuando a perfuração!
Tinham ido fazer outra coisa, dando uma mão no acabamento da casa.

--- Não trouxe vocês aqui pra fazer casa.

Mas que agora estavam terminando o café do meio"-dia. Espaçavam as
frases, desapontados, principiando a não saber nem como ficar de pé.
Havia silêncios desagradáveis. Mas o velho Joaquim Prestes impassível,
esperando mais explicações, sem dar sinal de compreender nem de
desculpar ninguém. Tinha um era o mais calmo, mulato desempenado,
fortíssimo, bem escuro na cor. Ainda nem falara. Mas foi esse que acabou
inventando um jeito humilhante de disfarçar a culpa inexistente, botando
um pouco de felicidade no dono. De repente contou que agora ainda ficara
mais penoso o trabalho porque enfim já estava minando água. Joaquim
Prestes ficou satisfeito, era visível, e todos suspiraram de alívio.

--- Mina muito?

--- A água vem de com força, sim senhor.

--- Mas percisa cavar mais.

--- Quanto chega?

--- Quer dizer, por enquanto dá pra uns dois palmo.

--- Parmo e meio, Zé.

O mulato virou contrariado para o que falara, um rapaz branco
enfezadinho, cor de doente.

--- Ocê marcou, mano\ldots{}

--- Marquei sim.

--- Então com mais dois dias de trabalho tenho água suficiente.

Os camaradas se entreolharam. Ainda foi o José quem falou:

--- Quer dizer\ldots{} a gente nem não sabe, tá uma lama\ldots{} O poço tá fundo,
só o mano que é leviano pode descer\ldots{}

--- Quanto mede?

--- Quarenta e cinco palmo.

--- Papagaio! escapou da boca de Joaquim Prestes. Mas ficou muito mudo,
na reflexão. Percebia"-se que ele estava lá dentro consigo, decidindo uma
lei. Depois meio que largou de pensar, dando todo o cuidado lento em
fazer o cigarro de palha com perfeição. Os camaradas esperavam, naquele
silêncio que os desprezava, era insuportável quase. O rapaz não
conseguiu se aguentar mais, como que se sentia culpado de ser mais leve
que os outros. Arrancou:

--- Por minha causa não, Zé, que eu desço bem.

José tornou a se virar com olhos enraivecidos pro irmão. Ia falar, mas
se conteve enquanto outro tomava a dianteira.

--- Então ocê vai ficar naquela dureza de trabalho com essa umidade!

--- Se a gente pudesse revezar inda que bem\ldots{} murmurou o quarto, também
regularmente leviano de corpo mas nada disposto a se sacrificar. E
decidiu:

--- Com essa chuvarada a terra tá mole demais, e se afunda!\ldots{} Deus te
livre\ldots{}

Aí José não pôde mais adiar o pressentimento que o invadia e protegeu o
mano:

--- `cê besta, mano! e sua doença!\ldots{}

A doença, não se falava o nome. O médico achara que o Albino estava
fraco do peito. Isso de um ser mulato e o outro branco, o pai espanhol
primeiro se amigara com uma preta do litoral, e quando ela morrera,
mudara de gosto, viera pra zona da Paulista casar com moça branca. Mas a
mulher morrera dando à luz o Albino, e o espanhol, gostando mesmo de
variar, se casara mas com a cachaça. José, taludinho, inda aguentou"-se
bem na orfandade, mas o Albino, tratado só quando as colonas vizinhas
lembravam, Albino comeu terra, teve tifo, escarlatina, desinteria,
sarampo, tosse comprida. Cada ano era uma doença nova, e o pai até
esbravejava nos janeiros: ``Que enfermedade le falta, caramba!'' e bebia
mais. Até que desapareceu pra sempre.

Albino, nem que fosse pra demonstrar a afirmativa do irmão, teve um
acesso forte de tosse. E Joaquim Prestes:

--- Você acabou o remédio?

--- Inda tem um poucadinho, sim sinhô.

Joaquim Prestes mesmo comprava o remédio do Albino e dava, sem descontar
no ordenado. Uma vidraça que o rapaz quebrara, o fazendeiro descontou os
três mil e quinhentos do custo. Porém montava na marmon, dava um pulo
até a cidade só pra comprar aquele fortificante estrangeiro, ``um
dinheirão!'' resmungava. E eram mesmo dezoito milréis.

Com a direção da conversa, os camaradas perceberam que tudo se arranjava
pelo milhor. Um comentou:

--- Não vê que a gente está vendo se o sol vem e seca um pouco, mode o
Albino descer no poço.

Albino, se sentindo humilhado nessa condição de doente, repetiu
agressivo:

--- Por isso não que eu desço bem! já falei\ldots{}

José foi pra dizer qualquer coisa mas sobresteve o impulso, olhou o mano
com ódio. Joaquim Prestes afirmou:

--- O sol hoje não sai.

O frio estava por demais. O café queimando, servido pela mulher do
vigia, não reconfortava nada, a umidade corroía os ossos. O ar sombrio
fechava os corações. Nenhum passarinho voava, quando muito algum pio
magoado vinha botar mais tristeza no dia. Mal se enxergava o aclive da
barranca, o rio não se enxergava. Era aquele arminho sujo da névoa, que
assim de longe parecia intransponível.

A afirmação do fazendeiro trouxera de novo um som apreensivo no
ambiente. Quem concordou com ele foi o vigia chegando. Só tocou de leve
no chapéu, foi esfregar forte as mãos, rumor de lixa, em cima do fogo.
Afirmou baixo, com voz taciturna de afeiçoado àquele clima ruim:

--- Peixe hoje não dá.

Houve silêncio. Enfim o patrão, o busto dele foi se erguendo
impressionantemente agudo, se endireitou rijo e todos perceberam que ele
decidira tudo. Com má vontade, sem olhar os camaradas, ordenou:

--- Bem\ldots{} é continuar todos na casa, vocês estão ganhando.

A última reflexão do fazendeiro pretendera ser cordial. Mas fora
navalhante. Até a visita se sentiu ferida. Os camaradas mais que
depressa debandaram, mas Joaquim Prestes:

--- Você me acompanhe, Albino, quero ver o poço.

Ainda ficou ali dando umas ordens. Havia de tentar uma rodada assim
mesmo. Afinal jogou o toco do cigarro na fogueira, e com a visita se
dirigiu para a elevação a uns vinte metros da casa, onde ficava o poço.

Albino já estava lá, com muito cuidado retirando as tábuas que cobriam a
abertura. Joaquim Prestes, nem mesmo durante a construção, queria que
caíssem ``coisas'' na água futura que ele iria beber. Afinal ficaram só
aquelas tábuas largas, longas, de cabreúva, protegendo a terra do
rebordo do perigo de esbarrondar. E mais aquele aparelho primário, que
``não era o elegante, definitivo'', Joaquim Prestes foi logo explicando
à visita, servindo por agora pra descer os operários no poço e trazer
terra.

--- Não pise aí, nhô Prestes! Albino gritou com susto.

Mas Joaquim Prestes queria ver a água dele. Com mais cuidado, se
acocorou numa das tábuas do rebordo e firmando bem as mãos em duas
outras que atravessavam a boca do poço e serviam apenas pra descanso da
caçamba, avançou o corpo pra espiar. As tábuas abaularam. Só o viram
fazer o movimento angustiado, gritou:

--- Minha caneta!

Se ergueu com rompante e sem mesmo cuidar de sair daquela bocarra
traiçoeira, olhou os companheiros, indignado:

--- Essa é boa!\ldots{} Eu é que não posso ficar sem a minha caneta"-tinteiro!
Agora vocês hão de ter paciência, mas ficar sem minha caneta é que eu
não posso! têm que descer lá dentro buscar! Chame os outros, Albino! e
depressa! que com o barro revolvido como está, a caneta vai afundando!

Albino foi correndo. Os camaradas vieram imediatamente, solícitos,
ninguém sequer lembrava mais de fazer corpo mole nem nada. Pra eles era
evidente que a caneta"-tinteiro do dono não podia ficar lá dentro. Albino
já tirava os sapatões e a roupa. Ficou nu num átimo da cintura pra cima,
arregaçou a calça. E tudo, num átimo, estava pronto, a corda com o nó
grosso pro rapaz firmar os pés, afundando na escureza do buraco. José
mais outro, firmes, seguravam o cambito. Albino com rapidez pegou na
corda, se agarrou nela, balanceando no ar. José olhava, atento:

--- Cuidado, mano\ldots{}

--- Vira.

--- Albino\ldots{}

---Nhô?

--- \ldots{} veja se fica na corda pra não pisar na caneta. Passe a mão de
leve no barro\ldots{}

--- Então é melhor botar um pau na corda pra fincar os pés.

--- Qual, mano! vira isso logo!

José e o companheiro viraram o cambito, Albino desapareceu no poço. O
sarilho gemeu, e à medida que a corda se desenrolava o gemido foi
aumentando, aumentando, até que se tornou num uivo lancinante. Todos
estavam atentos, até que se escutou o grito de aviso do Albino, chegado
apenas uma queixa até o grupo. José parou o manejo e fincou o busto no
cambito.

Era esperar, todos imóveis. Joaquim Prestes, mesmo o outro camarada
espiavam, meio esquecidos do perigo da terra do rebordo esbarrondar.
Passou um minuto, passou mais outro minuto, estava desagradabilíssimo.
Passou mais tempo, José não se conteve. Segurando firme só com a mão
direita o cambito, os músculos saltaram no braço magnífico, se inclinou
quanto pôde na beira do poço:

--- Achoooou!

Nada de resposta.

--- Achou, manoooo!\ldots{}

Ainda uns segundos. A visita não aguentara mais aquela angústia, se
afastara com o pretexto de passear. Aquela voz de poço, um tom surdo,
ironicamente macia que chegava aqui em cima em qualquer coisa parecia
com um ``não''. Os minutos passavam, ninguém mais se aguentava na
impaciência. Albino havia de estar perdendo as forças, grudado naquela
corda, de cócoras, passando a mão na lama coberta de água.

--- José\ldots{}

--- Nhô. Mas atentando onde o velho estava, sem mesmo esperar a ordem,
José asperejou com o patrão: --- Por favor, nhô Joaquim Prestes, sai
daí, terra tá solta!

Joaquim Prestes se afastou de má vontade. Depois continuou:

--- Grite pro Albino que pise na lama, mas que pise num lugar só. José
mais que depressa deu a ordem. A corda bambeou. E agora, aliviados, os
operários entreconversavam. O magruço, que sabia ler no jornal da
vendinha da estação, deu de falar, o idiota, no caso do ``Soterrado de
Campinas''. O outro se confessou pessimista, mas pouco, pra não
desagradar o patrão. José mudo, cabeça baixa, olho fincado no chão,
muito pensando. Mas a experiência de todos ali, sabia mesmo que a
caneta"-tinteiro se metera pelo barro mole e que primeiro era preciso
esgotar a água do poço. José ergueu a cabeça, decidido:

--- Assim não vai não, nhô Joaquim Prestes, percisa secar o poço.

Aí Joaquim Prestes concordou. Gritaram ao Albino que subisse. Ele ainda
insistiu uns minutos. Todos esperavam em silêncio, irritados com aquela
teima do Albino. A corda sacudiu, chamando. José mais que depressa
agarrou o cambito e gritou:

--- Pronto!

A corda enrijou retesada. Mesmo sem esperar que o outro operário o
ajudasse, José com músculos de amor virou sozinho o sarilho. A mola deu
aquele uivo esganado, assim virada rápido, e veio uivando, gemendo.

--- Vocês me engraxem isso, que diabo!

Só quando Albino surgiu na boca do poço o sarilho parou de gemer. O
rapaz estava que era um monstro de lama. Pulou na terra firme e tropeçou
três passos, meio tonto. Baixou muito a cabeça sacudida com estertor
purrr! agitava as mãos, os braços, pernas, num halo de lama pesada que
caía aos ploques no chão. Deu aquele disfarce pra não desapontar:

--- Puta"-frio!

Foi vestindo, sujo mesmo, com ânsia, a camisa, o pulôver esburacado, o
paletó. José foi buscar o seu próprio paletó, o botou silencioso na
costinha do irmão. Albino o olhou, deu um sorriso quase alvar de
gratidão. Num gesto feminino, feliz, se encolheu dentro da roupa,
gostando.

Joaquim Prestes estava numa exasperação terrível, isso via"-se. Nem
cuidava de disfarçar para a visita. O caipira viera falando que a mulher
mandava dizer que o almoço do patrão estava pronto. Disse um ``Já vou''
duro, continuando a escutar os operários. O magruço lembrou buscarem na
cidade um poceiro de profissão. Joaquim Prestes estrilou. Não estava pra
pagar poceiro por causa duma coisa à toa! que eles estavam com má
vontade de trabalhar! esgotar poço de pouca água não era nenhuma áfrica.
Os homens acharam ruim, imaginando que o patrão os tratara de negros. Se
tomaram dum orgulho machucado. E foi o próprio magro, mais independente,
quem fixou José bem nos olhos, animando o mais forte, e meio que
perguntou, meio que decidiu:

--- Bamo!.

Imediatamente se puseram nos preparos, buscando o balde, trocando as
tábuas atravessadas por outras que aguentassem peso de homem. Joaquim
Prestes e a visita foram almoçar.

Almoço grave, apesar o gosto farto do dourado. Joaquim Prestes estava
árido. Dera nele aquela decisão primária, absoluta de reaver a
caneta"-tinteiro hoje mesmo. Pra ele, honra, dignidade, autoridade não
tinha gradação, era uma só: tanto estava no custear a mulher da gente
como em reaver a caneta"-tinteiro. Duas vezes a visita, com ares de quem
não sabe perguntou sobre o poceiro da cidade. Mas só o forde podia ir
buscar o homem e Joaquim Pretes, agora que o vigia afirmara que não dava
peixe, tinha embirrado, havia de mostrar que, no pesqueiro dele, dava.
Depois que diabo! os camaradas haviam de secar o poço, uns palermas!
Estava numa cólera desesperada. Botando a culpa nos operários, Joaquim
Prestes como que distrai a culpa de fazê"-los trabalhar injustamente.

Depois do almoço chamou a mulher do vigia, mandou levar café aos homens,
porém que fosse bem quente. Perguntou se não havia pinga. Não havia
mais, acabara com a friagem daqueles dias. Deu de ombros. Hesitou. Ainda
meio que ergueu os olhos pra visita, consultando. Acabou pedindo
desculpa, ia dar uma chegadinha até o poço pra ver o que os camaradas
andavam fazendo. E não se falou mais em pescaria.

Tudo trabalhava na afobação. Um descia o balde. Outro, com empuxões
fortes na corda, afinal conseguia deitar o balde lá no fundo pra água
entrar nele. E quando o balde voltava, depois de parar tempo lá dentro,
vinha cheio apenas pelo terço, quase só lama. Passava de mão em mão pra
ser esvaziado longe e a água não se infiltrar pelo terreno do rebordo.
Joaquim Prestes perguntou se a água já diminuíra. Houve um silêncio
emburrado dos trabalhadores. Afinal um falou com rompante:

--- Quá!\ldots{}

Joaquim Prestes ficou ali, imóvel, guardando o trabalho. E ainda foi o
próprio Albino, mais servil, quem inventou:

--- Se tivesse duas caçamba\ldots{}

Os camaradas se sobressaltaram, inquietos, se entreolhando. E aquele
peste de vigia lembrou que a mulher tinha uma caçamba em casa, foi
buscar. O magruço, ainda mais inquieto que os outros, afiançou:

--- Nem com duas caçambas não vai não! é lama por demais! tá minando
muito\ldots{}

Aí o José saiu do seu silêncio torvo pra pôr as coisas às claras:

--- De mais a mais, duas caçamba percisa ter gente lá dentro, Albino não
desce mais.

--- Que que tem, Zé! deixa de história! Albino meio que estourou.

De resto o dia aquentara um bocado, sempre escuro, nuvens de chumbo
tomando o céu todo. Nenhum pássaro. Mas a brisa caíra por volta das
treze horas, e o ar curto deixava o trabalho aquecer os corpos movidos.
José se virara com tanta indignação para o mano, todos viram: mesmo com
desrespeito pelo velho Joaquim Prestes, o Albino ia tomar com um
daqueles cachações que apanhava quando pegado no truco ou na pinga. O
magruço resolveu se sacrificar, evitando mais aborrecimento. Interferiu
rápido:

--- Nós dois se reveza, José! Desta eu que vou.

O mulato sacudiu a cabeça, desesperado, engolindo raiva. A caçamba
chegava e todos se atiraram aos preparativos novos. O velho Joaquim
Prestes ali, mudo, imóvel. Apenas de vez em quando aquele jeito lento de
tirar o relógio e consultar a claridade do dia, que era feito uma
censura tirânica, pondo vergonha, quase remorso naqueles homens.

E o trabalho continuava infrutífero, sem cessar. Albino ficava o quanto
podia lá dentro, e as caçambas, lentas, naquele exasperante ir e vir. E
agora o sarilho deu de gritar tanto que foi preciso botar graxa nele,
não se suportava aquilo. Joaquim Prestes mudo, olhando aquela boca de
poço. E quando Albino não se aguentava mais o outro magruço o revezava.
Mas este depois da primeira viagem, se tomara dum medo tal, se fazia
lerdo de propósito, e era recomendações a todos, tinha exigências. Já
por duas vezes falara em cachaça.

Então o vigia lembrou que o japonês da outra margem tinha cachaça à
venda. Dava uma chegadinha lá, que o homem também sempre tinha algum
trairão de rede, pegado na lagoa.

Aí Joaquim Prestes se destemperou por completo. Ele bem que estava
percebendo a má vontade de todos. Cada vez que o magruço tinha que
descer eram cinco minutos, dez, mamparreando, se despia lento. Pois até
não se lembrara de ir na casinha e foi aquela espera insuportável pra
ninguém! (E o certo é que a água minava mais forte agora, livre da muita
lama. O dia passava. E uma vez que o Albino subiu, até, contra o jeito
dele, veio irritado, porque achara o poço na mesma.)

Joaquim Prestes berrava, fulo de raiva. O vigia que fosse tratar das
vacas, deixasse de invencionice! Não pagava cachaça pra ninguém não,
seus imprestáveis! Não estava pra alimentar manha de cachaceiro!

Os camaradas, de golpe, olharam todos o patrão, tomados de insulto,
feridíssimos, já muito sem paciência mais. Porém Joaquim Prestes ainda
insistia, olhando o magruço:

--- É isso mesmo!\ldots{} Cachaceiro!\ldots{} Dispa"-se mais depressa! cumpra o seu
dever!\ldots{}

E o rapaz não aguentou o olhar cutilante do patrão, baixou a cabeça, foi
se despindo. Mas ficara ainda mais lerdo, ruminando uma revolta
inconsciente, que escapava na respiração precipitada, silvando surda
pelo nariz. A visita percebendo o perigo, interveio. Fazia gosto de
levar um pescado à mulher, se o fazendeiro permitisse, ele dava um pulo
com o vigia lá no tal de japonês. E irritado fizera um sinal ao caipira.
Se fora, fugindo daquilo, sem mesmo esperar o assentimento de Joaquim
Prestes. Este mal encolheu os ombros, de novo imóvel, olhando o trabalho
do poço.

Quando mais ou menos uma hora depois, a visita voltou ao poço outra vez,
trazia afobada uma garrada de caninha. Foi oferecendo com felicidade aos
camaradas, mas eles só olharam a visita assim meio de lado, nem
responderam. Joaquim Prestes nem olhou, e a visita percebeu que tinha
sucedido alguma coisa grave. O ambiente estava tensíssimo. Não se via o
Albino nem o magruço que o revezava. Mas não estavam ambos no fundo do
poço, como a visita imaginou.

Minutos antes, poço quase seco agora, o magruço que já vira um bloco de
terra se desprender do rebordo, chegada a vez dele, se recusara descer.
Foi meio minuto apenas de discussão agressiva entre ele e o velho
Joaquim Prestes, desce, não desce, e o camarada, num ato de desespero se
despedira por si mesmo, antes que o fazendeiro o despedisse. E se fora,
dando as costas a tudo, oito anos de fazenda, curtindo uma tristeza
funda, sem saber. E Albino, aquela mansidão doentia de fraco, pra evitar
briga maior, fizera questão de descer outra vez, sem mesmo recobrar
fôlego. Os outros dois, com o fantasma próximo de qualquer coisa mais
terrível, se acovardaram. Albino estava no fundo do poço.

Agora o vento soprando, chicoteava da gente não aguentar. Os operários
tremiam muito, e a própria visita. Só Joaquim Prestes não tremia nada,
firme, olhos fincados na boca do poço. A despedida do operário o
despeitara ferozmente, ficara num deslumbramento horrível. Nunca
imaginara que num caso qualquer o adversário se arrogasse a iniciativa
de decidir por si. Ficara assombrado. Por certo que havia de mandar
embora o camarada, mas que este se fosse por vontade própria, nunca
pudera imaginar. A sensação do insulto estourara nele feito uma
bofetada. Se não revidasse era uma desonra, como se vingar!\ldots{} Mas só as
mãos se esfregando lentíssimas, denunciavam o desconcerto interior do
fazendeiro. E a vontade reagia com aquela decisão já desvairada de
conseguir a caneta"-tinteiro, custasse o que custasse. Os olhos do velho
engoliam a boca do poço, ardentes, com volúpia quase. Mas a corda já
sacudia outra vez, agitadíssima agora, avisando que o Albino queria
subir. Os operários se afobaram. Joaquim Prestes abriu os braços, num
gesto de desespero impaciente.

--- Também Albino não parou nem dez minutos!

José ainda lançou um olhar de imploração ao chefe, mas este não
compreendia mais nada. Albino apareceu na boca do poço. Vinha agarrado
na corda, se grudando nela com terror, como temendo se despegar.
Deixando o outro operário na guarda do cambito, José com muita
maternidade ajudava o mano. Este olhava todos, cabeça de banda decepando
na corda, boca aberta. Era quase impossível lhe aguentar o olho abobado.
Como que não queria se desagarrar da corda, foi preciso o José, ``sou
eu, mano'', o tomar nos braços, lhe fincar os pés na terra firme. Aí
Albino largou da corda. Mas com o frio súbito do ar livre, principiou
tremendo demais. O seguraram pra não cair. Joaquim Prestes perguntava se
ainda tinha água lá em baixo.

--- Fa\ldots{}Fa\ldots{}

Levou as mãos descontroladas à boca, na intenção de animar os beiços
mortos. Mas não podia limitar os gestos mais, tal o tremor. Os dedos
dele tropeçavam nas narinas, se enfiavam pela boca, o movimento
pretendido de fricção se alargava demais e a mão se quebrava no queixo.
O outro camarada lhe esfregava as costas. José estava tão triste\ldots{}
Enrolou, com que macieza! a cabeça do maninho no braço esquerdo, lhe pôs
a garrafa na boca:

--- Beba, mano.

Albino engoliu o álcool que lhe enchera a boca. Teve aquela reação
desonesta que os tragos fortes dão. Afinal pôde falar:

--- Farta\ldots{} é só tá"-tá seco.

Joaquim Prestes falava manso, compadecido, comentando inflexível:

--- Pois é, Albino: se você tivesse procurado já, decerto achava.
Enquanto isso a água vai minando.

--- Se eu tivesse uma lúiz\ldots{}

--- Pois leve.

José parou de esfregar o irmão. Se virou pra Joaquim Prestes. Talvez nem
lhe transparecesse ódio no olhar, estava simples. Mandou calmo, olhando
o velho nos olhos:

--- Albino não desce mais.

Joaquim Prestes ferido desse jeito, ficou que era a imagem descomposta
do furor. Recuou um passo na defesa instintiva, levou a mão ao revólver.
Berrou já sem pensar:

--- Como não desce!

--- Não desce não. Eu não quero.

Albino agarrou o braço do mano mas toma com safanão que quase cai. José
traz as mãos nas ancas, devagar, numa calma de morte. O olhar não
pestaneja, enfiando no do inimigo. Ainda repete, bem baixo, mas
mastigando:

--- Eu não quero não sinhô.

Joaquim Prestes, o mal pavoroso que terá vivido aquele instante\ldots{} A
expressão do rosto dele se mudara de repente, não era cólera mais, boca
escancarada, olhos brancos, metálicos, sustentando olhar puro, tão
calmo, do mulato. Ficaram assim. Batia agora uma primeira escureza do
entardecer. José, o corpo dele oscilou milímetros, o esforço moral foi
excessivo. Que o irmão não descia estava decidido, mas tudo mais era uma
tristeza em José, uma desolação vazia, uma semiconsciência de culpa
lavrada pelos séculos.

Os olhos de Joaquim Prestes reassumiam uma vibração humana. Afinal
baixaram, fixando o chão. Depois foi a cabeça que baixou, de súbito,
refletindo. Os ombros dele também foram descendo aos poucos. Joaquim
Prestes ficou sem perfil mais. Ficou sórdido.

--- Não vale a pena mesmo\ldots{}

Não teve a dignidade de aguentar também com a aparência externa da
derrota. Esbravejou:

--- Mas que diacho, rapaz! vista saia!

Albino riu, iluminando o rosto agradecido. A visita riu pra aliviar o
ambiente. O outro camarada riu, covarde. José não riu. Virou a cara,
talvez para não mostrar os olhos amolecidos. Mas ombros derreados,
cabeça enfiada no peito, se percebia que estava fatigadíssimo. Voltara a
esfregar maquinalmente o corpo do irmão, agora não carecendo mais disso.
Nem ele nem os outros, que o incidente espantara por completo qualquer
veleidade do frio.

Quer dizer, o caipira também não riu, ali chegado no meio da briga pra
avisar que os trairões, como Joaquim Prestes exigia, devidamente limpos
e envoltos em sacos de linho alvo, esperavam para partir. Joaquim
Prestes rumou pro forde. Todos o seguiram. Ainda havia nele uns restos
de superioridade machucada que era preciso enganar. Falava ríspido,
dando a lei com lentidão:

--- Amanhã vocês se aprontem. Faça frio não faça frio mando o poceiro
cedo. E\ldots{} José\ldots{}

Parou, voltou"-se, olhou firme o mulato:

--- \ldots{} doutra vez veja como fala com seu patrão.

Virou, continuou, mais agitado agora, se dirigindo ao forde. Os mais
próximos ainda o escutaram murmurar consigo: ``\ldots{} não sou nenhum
desalmado\ldots{}''

Dois dias depois o camarada desapeou da besta com a caneta"-tinteiro.
Foram lévá"-la a Joaquim Prestes que, sentado à escrivaninha, punha em
dia a escrita da fazenda, um brinco. Joaquim Prestes abriu o embrulho
devagar. A caneta vinha muito limpa, toda arranhada. Se via que os
homens tinham tratado com carinho aquele objeto, meio místico, servindo
pra escrever sozinho. Joaquim Prestes experimentou mas a caneta não
escrevia. Ainda a abriu, examinou tudo, havia areia em qualquer frincha.
Afinal descobriu a rachadura.

--- Pisaram na minha caneta! brutos\ldots{}

Jogou tudo no lixo. Tirou da gaveta de baixo uma caixinha que abriu.
Havia nela várias lapiseiras e três canetas"-tinteiro. Uma era de ouro.

\chapter{O peru de Natal}

O nosso primeiro Natal de família, depois da morte de meu pai acontecida
cinco meses antes, foi de consequências decisivas para a felicidade
familiar. Nós sempre fôramos familiarmente felizes, nesse sentido muito
abstrato da felicidade: gente honesta, sem crimes, lar sem brigas
internas nem graves dificuldades econômicas. Mas, devido principalmente
à natureza cinzenta de meu pai, ser desprovido de qualquer lirismo, duma
exemplaride incapaz, acolchoado no medíocre, sempre nos faltara aquele
aproveitamente da vida, aquele gosto pelas felicidades materiais, um
vinho bom, uma estação de águas, aquisição de geladeira, coisas assim.
Meu pai fora de um bom errado, quase dramático, o puro sangue dos
desmancha"-prazeres.

Morreu meu pai, sentimos muito, etc. Quando chegamos nas proximidades do
Natal, eu já estava que não podia mais pra afastar aquela memória
obstruente do morto, que parecia ter sistematizado pra sempre a
obrigação de uma lembrança dolorosa em cada gesto mínimo da família. Uma
vez que eu sugerira a mamãe a ideia dela ir ver uma fita no cinema, o
que resultou foram lágrimas. Onde se viu ir ao cinema, de luto pesado! A
dor já estava sendo cultivada pelas aparências, e eu, que sempre gostara
apenas regularmente de meu pai, mais por instinto de filho que por
espontaneidade de amor, me via a ponto de aborrecer o bom do morto.

Foi decerto por isto que me nasceu, esta sim, espontaneamente, a ideia
de fazer uma das minhas chamadas ``loucuras''. Essa fora aliás, e desde
muito cedo, a minha esplêndida conquista contra o ambiente familiar.
Desde cedinho, desde os tempos de ginásio, em que arranjava regularmente
uma reprovação todos os anos; desde o beijo às escondidas, numa prima,
aos dez anos, descoberto por Tia Velha, uma detestável de tia; e
principalmente desde as lições que dei ou recebi, não sei, duma criada
de parentes: eu consegui no reformatório do lar e na vasta parentagem, a
fama conciliatória de ``louco''. ``É doido, coitado!'' falavam. Meus
pais falavam com certa tristeza condescendente, o resto da parentagem
buscando exemplo para os filhos e provavelmente com aquele prazer dos
que convencem de algum superioridade. Não tinham doidos entre os filhos.
Pois foi o que me salvou, essa fama. Fiz tudo o que a vida me apresentou
e o meu ser exigia para se realizar com integridade. E me deixaram fazer
tudo, porque eu era doido, coitado. Resultou disso uma existência sem
complexos, de que não posso me queixar um nada.

Era costume sempre, na família, a ceia de Natal. Ceia reles, já se
imagina: ceia tipo meu pai, castanhas, figos, passas, depois da Missa do
Calo. Empanturrados de amêndoas e nozes (quanto discutimos os três manos
por causa dos quebra"-nozes\ldots{}), empanturrados de castanhas e monotonias,
a gente se abraçava e ia pra cama. Foi lembrando isso que arrebentei com
uma das minhas ``loucuras'':

--- Bom, no Natal, quero comer peru.

Houve um desses espantos que ninguém não imagina. Logo minha tia
solteirona e santa, que morava conosco, advertiu que não podíamos
convidar ninguém por causa do luto.

--- Mas quem falou de convidar ninguém! Essa mania\ldots{} Quando é que a
gente já comeu peru em nossa vida! Peru aqui em casa é prato de festa,
vem toda essa parentada do diabo\ldots{}

--- Meu filho, não fale assim\ldots{}

--- Pois falo, pronto!

E descarreguei minha gelada indiferença pela nossa parentagem infinita,
diz"-que vinda de bandeirantes, que bem me importa! Era mesmo o momento
pra desenvolver minha teoria de doido, coitado, não perdi a ocasião. Me
deu de sopetão uma ternura imensa por mamãe e titia, minhas duas mães,
três com minha irmã, as três mães que sempre me divinizaram a vida. Era
sempre aquilo: vinha aniversário de alguém e só então faziam peru
naquela casa. Peru era prato de festa: uma imundície de parentes já
preparados pela tradição, invadiam a casa por causa do peru, das
empadinhas e dos doces. Minhas três mães, três dias antes já não sabiam
da vida senão trabalhar, trabalhar no preparo de doces e frios
finíssimos de bem feitos, a parentagem devorava tudo e inda levava
embrulhinhos pros que não tinham podido vir. As minhas três mães mal
podiam de exaustas. Do peru, só no enterro dos ossos, no dia seguinte, é
que mamãe com titia inda provavam um naco de perna, vago, escuro,
perdido no arroz alvo. E isso mesmo era mamãe quem servia, catava tudo
pro velho e pros filhos. Na verdade ninguém sabia de fato o que era peru
em nossa casa, peru resto de festa.

Não, não se convidava ninguém, era um peru pra nós, cinco pessoas. E
havia de ser com duas farofas, a gorda com os miúdos, e a seca,
douradinha, com bastante manteiga. Queria o papo recheado só com a
farofa gorda, em que havíamos de ajuntar ameixa preta, nozes e um cálice
de xerez, como aprendera na casa da Rose, muito minha companheira. Está
claro que omiti onde aprendera a receita, mas todos desconfiaram. E
ficaram logo naquele ar de incenso assoprado, se não seria tentação do
Dianho aproveitar receita tão gostosa. E cerveja bem gelada, eu garantia
quase gritando. É certo que com meu ``gostos'', já bastante afinados
fora do lar, pensei primeiro num vinho bom, completamente francês. Mas a
ternura por mamãe venceu o doido, mamãe adorava cerveja.

Quando acabei meus projetos, notei bem, todos estavam felicíssimos, num
desejo danado de fazer aquela loucura em que eu estourara. Bem que
sabiam, era loucura sim, mas todos se faziam imaginar que eu sozinho é
que estava desejando muito aquilo e havia jeito fácil de empurrarem pra
cima de mim a \ldots{} culpa de seus desejos enormes. Sorriam se
entreolhando, tímidos como pombas desgarradas, até que minha irmã
resolveu o consentimento geral:

--- É louco mesmo!\ldots{}

Comprou"-se o peru, fez"-se o peru, etc. E depois de uma Missa do Galo bem
mal rezada, se deu o nosso mais maravilhoso Natal. Fora engraçado: assim
que me lembrara de que finalmente ia fazer mamãe comer peru, não fizera
outra coisa aqueles dias que pensar nela, sentir ternura por ela, amar
minha velhinha adorada. E meus manos também, estavam no mesmo ritmo
violento de amor, todos dominados pela felicidade nova que o peru vinha
imprimindo na família. De modo que, ainda disfarçando as coisas, deixei
muito sossegado que mamãe cortasse todo o peito do peru. Um momento
aliás, ela parou, feito fatias um dos lados do peito da ave, não
resistindo àquelas leis de economia que sempre a tinham entorpecido numa
quase pobreza sem razão.

--- Não senhora, corte inteiro! só eu como tudo isso!

Era mentira. O amor familiar estava por tal forma incandescente em mim,
que até era capaz de comer pouco, só pra que os outros quatro comessem
demais. E o diapasão dos outros era o mesmo. Aquele peru comido a sós,
redescobria em cada um o que a quotidianidade abafara por completo,
amor, paixão de mãe, paixão de filhos. Deus me perdoe mas estou pensando
em Jesus\ldots{} Naquela casa de burgueses bem modestos, estava se realizando
um milagre digno do Natal de um Deus. O peito do peru ficou inteiramente
reduzido a fatias amplas.

--- Eu que sirvo!

``É louco, mesmo!'' pois por que havia de servir, se sempre mamãe
servira naquela casa! Entre risos, os grandes pratos cheios foram
passados pra mim e principiei uma distribuição heróica, enquanto mandava
meu mano servir a cerveja. Tomei conta logo dum pedaço admirável da
``casca'', cheio de gordura e pus no prato. E depois vastas fatias
brancas. A voz severizada de mamãe cortou o espaço angustiado com que
todos aspiravam pela sua parte no peru:

--- Se lembre de seus manos, Juca!

Quando que ela havia de imaginar, a pobre! que aquele era o prato dela,
da Mãe, da minha amiga maltratada, que sabia da Rose, que sabia meus
crimes, a que eu só lembrava de comunicar o que fazia sofrer! O prato
ficou sublime.

--- Mamãe, este é o da senhora! Não! não passe não!

Foi quando ela não pôde mais com tanta comoção e principiou chorando.
Minha tia também, logo percebendo que o novo prato sublime seria o dela,
entrou no refrão das lágrimas. E minha irmã, que jamais viu lágrima sem
abrir a torneirinha também, se esparramou no choro. Então principiei
dizendo muitos desaforos pra não chorar também, tinha dezenove anos\ldots{}
Diabo de família besta que via peru e chorava! coisas assim. Todos se
esforçavam por sorrir, mas agora é que a alegria se tornara impossível.
É que o pranto evocara por associação a imagem indesejável de meu pai
morto. Meu pai, com sua figura cinzenta, vinha pra sempre estragar nosso
Natal. Fiquei danado.

Bom, principiou"-se a comer em silêncio, lutuosos, e o peru estava
perfeito. A carne mansa, de um tecido muito tênue boiava fagueira entre
os sabores das farofas e do presunto, de vez em quando ferida,
inquietada e redesejada, pela intervenção mais violenta da ameixa preta
e o estorvo petulante dos pedacinhos de noz. Mas papai sentado ali,
gigantesco, incompleto, uma censura, uma chaga, uma incapacidade. E o
peru, estava tão gostoso, mamãe por fim sabendo que peru era manjar
mesmo digno do Jesusinho nascido.

Principiou uma luta baixa entre o peru e o vulto de papai. Imaginei que
gabar o peru era fortalecê"-lo na luta, e, está claro, eu tomara
decididamente o partido do peru. Mas os defuntos têm meios visguentos,
muito hipócritas de vencer: nem bem gabei o peru a imagem de papai
cresceu vitoriosa, insuportavelmente obstruidora.

--- Só falta seu pai\ldots{}

Eu nem comia, nem podia mais gostar daquele peru perfeito, tanto que me
interessava aquela luta entre os dois mortos. Cheguei a odiar papai. E
nem sei que inspiração genial, de repente me tornou hipócrita e
político. Naquele instante que hoje me parece decisivo da nossa família,
tomei aparentemente o partido de meu pai. Fingi, triste:

--- E mesmo\ldots{} Mas papai, que queria tanto bem a gente, que morreu de
tanto trabalhar pra nós, papai lá no céu há de estar contente\ldots{}
(hesitei, mas resolvi não mencionar mais o peru) contente de ver nós
todos reunidos em família.

E todos principiaram muito calmos, falando de papai. A imagem dele foi
diminuindo, diminuindo e virou uma estrelinha brilhante do céu. Agora
todos comiam o peru com sensualidade, porque papai fora muito bom,
sempre se sacrificara por nós, fora um santo que ``vocês, meu filhos,
nunca poderão pagar o que devem a seu pai'', um santo. Papai virara
santo, uma contemplação agradável, uma inestorvável estrelinha do céu.
Não prejudicava mais ninguém, puro objeto de contemplação suave. O único
morto ali era o peru, dominador, completamente vitorioso.

Minha mãe, minha tia, nós, todos alagados de felicidade. Ia escrever
``felicidade gustativa'', mas não era só isso não. Era uma felicidade
maiúscula, um amor de todos, um esquecimento de outros parentescos
distraidores do grande amor familiar. E foi, sei que foi aquele primeiro
peru comido no recesso da família, o início de um amor novo,
reacomodado, mais completo, mais rico e inventivo, mais complacente e
cuidadoso de si. Nasceu de então uma felicidade familiar pra nós que,
não sou exclusivista, alguns a terão assim grande, porém mais intensa
que a nossa me é impossível conceber.

Mamãe comeu tanto peru que um momento imaginei, aquilo podia lhe fazer
mal. Mas logo pensei: ah, que faça! mesmo que ela morra, mas pelo menos
que uma vez na vida coma peru de verdade!

A tamanha falta de egoísmo me transportara o nosso infinito amor\ldots{}
Depois vieram umas uvas leves e uns doces, que lá na minha terra levam o
nome de ``bem"-casados''. Mas nem mesmo este nome perigoso se associou à
lembrança de meu pai, que o peru já convertera em dignidade, em coisa
certa, em culto puro de contemplação.

Levantamos. Eram quase duas horas, todos alegres, bambeados por duas
garrafas de cerveja. Todos iam deitar, dormir ou mexer na cama, pouco
importa, porque é bom uma insônia feliz. O diabo é que a Rose, católica
antes de ser Rose, prometera me esperar com uma champanha. Pra poder
sair, menti, falei que ia a uma festa de amigo, beijei mamãe e pisquei
pra ela, modo de contar onde é que ia e fazê"-la sofrer seu bocaco. As
outras duas mulheres beijei sem piscar. E agora, Rose!\ldots{}

\chapter{Nelson}

--- Você conhece?

--- Eu não, mas contaram ao Basílio o caso dele.

O indivíduo chamava a atenção mesmo, embora não mostrasse nada de
berrantemente extraordinário. Tinha um ar esquisito, ar antigo, que
talvez lhe viesse da roupa mal talhada. Mas que por certo derivava da
cara também, encardida, de uma palidez absurda, quase artificial, como a
cara enfarinhada dos palhaços. Olhos pequenos, claros, à flor da pele,
quase que apenas aquela mancha cinzenta, vaga, meio desaparecendo na
brancura sem sombra do rosto.

Deu uma olhadela disfarçada, bem de tímido, assuntando o ambiente mal
iluminado do bar. Ainda hesitou, numa leve ondulação de recuo, mas
acabou indo sentar no outro lado da sala vazia. Percebeu se acalmar e
depôs as duas mãos, uma agarrando a outra, sobre a toalha. Mas como se
arrependeu de mostrá"-las, retirou"-as rápido pra debaixo da mesa. Se
lembrou de repente que não tirara o chapéu, estremeceu, quis sorrir,
disfarçando a encabulação. Mas corou muito, tirou num gesto brusco o
chapéu, escondeu"-o no banco em que sentara, ao mesmo tempo que lançava
novo olhar furtivo, muito angustiado, meio implorante, aos rapazes. E
estes fingiram que não o examinavam mais, envergonhados da curiosidade.

--- Não parece brasileiro\ldots{}

--- Diz"-que é. Mora só, numa daquelas casinhas térreas da alameda do
Triunfo, perto de mim. Ele mesmo faz a comida dele\ldots{}

Parou, gozando o interesse que causava. Era desses vaidosos que não
contam sem martirizar o ouvinte com pausas de efeito, perguntas de
adivinhação, detalhes sem eira nem beira. Continuou: --- ``Vocês todos
sabem onde que ele faz as compras dele!\ldots{}'' Nova pausa. Os rapazes se
mexeram impacientes. Um arrancou:

--- Você garante que ele é brasileiro, enfim você sabe ou não sabe
alguma coisa sobre ele!

--- Eu sei a história dele completinha!\ldots{} --- Olhou lento, imperial os
três amigos. Sorriu. --- Mas, puxa! que lerdeza de vocês!\ldots{} Eu disse
que ele mora no Triunfo, pertinho de mim \ldots{} Então vocês não são capazes
de imaginar onde ele compra as coisas!\ldots{}

--- Ora, desembucha logo, Alfredo! que diabo de mania essa!\ldots{}

Diva passava levando dois duplos escuros. Era visível que ambos
pertenciam ao desconhecido, pois não havia mais ninguém no bar.
Recebendo os duplos o homem ficou envergonhado, tornou a corar forte,
mandando outro olho de relance aos rapazes. Falou qualquer coisa à
garçonette que ficou esperando. Então ele emborcou o primeiro chope com
sofreguidão, bebeu tudo duma vez só, entregando o copo à moça. E Diva se
retirou, sorrindo ao ``muito obrigado''quente que o homem lhe dizia.

Os rapazes voltavam pensativos aos seus chopes, o desconhecido era de
fato um sujeito extravagante\ldots{} Alfredo aproveitou a preocupação de
todos, pra retomar importância. Mas agora ``desembucha'' mais rápido.

--- Pois ele compra tudo no Basílio, e o Basílio é que sabe a história
dele bem. Põe tamanha confiança no vendeiro que até pede pra ele fazer
compra na cidade, camisa, roupa de baixo\ldots{} Diz"-que foi até bastante
rico. Ele é de Mato Grosso, possuía uma fazenda de criar no sul do
Estado, não tinha parente nenhum depois que a mãe morreu. De vez em
quando atravessava a fronteira que ficava ali mesmo, dava uma chegada em
Assunção que é a capital do Paraguai\ldots{}

--- Não sabia! pensei que era Campinas!

--- \ldots{} ia lá só pra farrear, vivendo naquele jejum da fazenda\ldots{} ---
Achou graça em si mesmo e quis tirar mais efeito: --- Em Assunção
desjejuava a valer. Mas um dia acabou trazendo uma paraguaia pra
fazenda, com ele. Era uma moça lindíssima e ele tinha paixão por ela,
dava tudo pra ela. Trabalhava e era pra ela; ia na cidade por um dia,
imaginem pra que!\ldots{} voltava carregado de presentes muito caros. Mesmo
na fazenda ela só arrastava seda. Mas que ela merecesse, merecia porque
também gostava muito dele e os dois viviam naquele amor. Mas a maior
besteira dele, isso dava um doce se vocês imaginassem.

Quis parar, mas um dos companheiros percebendo asperejou irritado:

--- Não dê o doce, e continue, Alfredo!

--- Pois acabou passando a fazenda com gado e tudo e ainda umas casa que
tinha em Cuiabá, passou tudo para o nome dela, porque ela já fizera
operação, mocinha, e não podia ter filho que herdasse. Não sei se vocês
sabem:\ldots{} mesmo casada no juiz, se não tivesse filho e ele morresse, ela
não herdava um isto. E agora é que estou vendo que o Basílio não me
informou se eles eram casados, amanhã mesmo vou saber\ldots{}

--- Mas\ldots{} me diga uma coisa, Alfredo: isso interessa pro caso!

--- Quer dizer\ldots{} interessar sempre interessa\ldots{} Mas afinal aquela vida
era chata pra moça tão bonita que não podia ser vista nem apreciada por
ninguém, não durou muito ela principiou entristecendo. Ele vinha e
perguntava, porém ela sempre respondia que não tinha nada e virava o
rosto pra não dar demonstração que estava chorando. Ele fez tudo.
Comprou uma vitrola, comprou um rádio e a casa se encheu de polcas
paraguaias. Depois até principiou aprendendo o guarani com ela, o
castelhano já falava muito bem. Era que ele imaginou ficar mais tempo
junto da moça, em vez de passar o dia inteiro no campo, cuidando do
gado.

--- Mas também que sujeito mais besta --- interrompeu um dos rapazes
irritado. --- Ele era rico, não era?

--- Era\ldots{}

--- Pois então porque não ia fazer uma viagem!

--- Pois fez, mas aí é que foi a causa de tudo. Eles resolveram ir
passear em Assunção, se divertiram tanto que passaram dois meses lá.
Quando voltaram ela até parecia outra, de tão alegre outra vez, e
fizeram projeto de todos os anos ir passear assim, se divertindo com os
outros, o amor é que não havia meios de afrouxar. Já antes da viagem, no
tempo da tristeza, ele assinara uma porção de revistas, até
norte"-americanas, pra ver se ela se distraía, ela nem olhava pras
figuras. Pois agora de volta na fazenda adivinhem pra o que ela deu!\ldots{}

--- Ora, deu pra ler as revistas!

---Não!

--- Deu pra ficar triste outra vez.

---Não!

--- Se acostumou\ldots{}

---Não!

--- Ora foi ver se você estava na esquina, ouviu!

Os rapazes estavam totalmente desinteressados da história do Alfredo. Um
deles olhou o homem, de quem a garçonette se aproximava outra vez,
levando mais um chope. O homem, percebendo a moça, retirou brusco as
mãos que descansavam na mesa, uma sobre a outra. Novo olhar angustiado
aos rapazes.

--- Parece que ele tem qualquer coisa na mão esquerda, o rapaz avisou
interessado. Não! não virem agora que ele está olhando pra cá, mas nem
bem Diva ia chegando com o chope, ele escondeu a mão. Diva!

A moça veio se chegando, familiar.

--- Mais chope. Diga uma coisa\ldots{} chegue mais pra cá.

A moça chegou contrafeita, depois de uma leve hesitação. Ela sabia que
iam lhe falar do desconhecido, e quando o rapaz perguntou o que o homem
tinha na mão, ela quase gritou um ``Nada!'' agressivo. E como o rapaz
procurasse agarrá"-la pelo braço, ainda perguntando se o homem não tinha
um defeito qualquer, ela se desvencilhou irritada, murmurando ``Não!'',
``Não sei!'', partiu confusa. O contador interrompido pretendeu
readquirir importância, afirmando apressado:

--- É uma cicatriz medonha, não queiram saber! Foi numa briga, parece
que até ele perdeu um dedo, só que isso eu não sei como foi, o
Basílio\ldots{}

O quarto rapaz, que se conservara calado, olhando com uma espécie de
riso o sabetudo, murmurou vingativo:

--- Eu sei.

--- Você sabe!

--- Quer dizer: sei\ldots{} Sei o que me contaram. É o polegar que ele
perdeu. Parece que nem é só o polegar que falta, mas quase toda a carne
do braço, é tudo repuxado, sem pele\ldots{} Foi piranha que comeu.

--- Safa!

--- Eu não sei bem\ldots{} tudo no detalhe. Como o Alfredo, eu não sei\ldots{} Foi
na Coluna Prestes\ldots{} nem tenho certeza se ele estava com o exército ou
com os revolucionários. Devia ser com estes porque ele era rapaz, se vê
que não tem trinta anos.

--- Isso não! garanto que já passa dos quarenta.

--- Você está doido!

--- Não\ldots{} --- arrancou o Alfredo, meio contra a vontade. --- Isso eu
também sei garantido que ele é novo ainda, o Basílio viu a caderneta
dele\ldots{} Tem vinte e sete, vinte e oito anos.

--- Mas conta como foi a piranha.

---\ldots{} diz"-que estava em Mato Grosso, um grupinho perseguido pelos
contrários, desgarrado, pra uns nove homens quando muito. Tinham se
arranchado na casinha dum caboclo que ficava perto dum rio, quando o
inimigo deu lá, era de noite. Foi aquele tiroteio feroz, eles dentro da
casa, os outros no cerco. Quando viram que não se aguentavam mais, a
munição estava acabando, decidiram furar pra banda do rio, onde o bote
do caboclo estava amarrado na maromba\ldots{}

--- O que é maromba?

--- É assim um estrado grande, pra servir de chão dos bois, quando o rio
enche.

--- Qual! tudo isso é história! pois você não vê logo que os policiais
já deviam estar tomando conta do bote!

--- Você está com despeito de eu saber, quer me atrapalhar à toa: pois é
isso mesmo! Deixe eu acabar, você vai ver. Já era de madrugadinha, mas
estava escuro ainda. De repente eles deram uma descarga juntos, e saíram
embolados, frechando pro rio. Ainda conseguiram passar, que os\ldots{}
contrários, eu não falei que era polícia que cercava! enfim, os\ldots{}
outros, só tinha dois amoitados no caminhinho que levava ao porto, se
acovardaram. Eles passaram na volada, gritando, desceram o barranco aos
pulos, mas quando chegaram lá tinha pra uns dez, de tocaia, na maromba.
Se atracaram uns com os outros, e esse um aí se abraçou com um inimigo e
os dois rolaram no rio, afundando. Bem, mas quando voltaram à tona,
sempre grudados um no outro, lutando, o diabo é que tinham vindo parar
bem debaixo \ldots{} não sei se vocês sabem\ldots{} lá, por causa de enchente,
eles usam construir um cais flutuante pra embarcar e desembarcar. O
desse porto por sinal que era bem feito e mais grande, porque era por
ali que a estrada do governo atravessava o rio: uma espécie de caixão
grande bem chato, feito de pranchões. Pois justo debaixo disso que os
dois vieram surgir e já estavam desesperados de vontade de respirar, não
se aguentavam mais. Por cima era aquele barulhão de gente brigando, o
caixão sacudia muito, mais outros caíam n'água\ldots{} Os dois não queriam,
decerto nem podiam se largar, mas não sei como foi, se uma das pranchas
da parte inferior estava podre e cedeu, ou se havia o buraco mesmo\ldots{}
sei é que num balanço que o caixão fez com os homens que brigavam em
cima deles, esse um ali sentiu que ia saindo fora d'água e pôde
respirar. Mas estava com a cabeça enforcada dentro do caixão chato, até
batendo no plano dos pranchões de cima, parece que estou vendo! quem me
contou foi o Querino do gás. Mas ele respirou fundo, foi ganhando
consciência e percebeu que os músculos do adversário afrouxavam. Se ele
largasse, o outro afundava, ia sair lá mais no largo e denunciava o
esconderijo dele, apertou mais. Por cima o inferno já estava diminuindo,
o caixão sacudia menos, paravam com a gritaria dos insultos. Afinal ele
percebeu que os inimigos tinham dominado a situação, eram muito mais
numerosos. Um que mandava nos outros, dava ordens, afirmava que faltavam
dois do grupo inimigo, um era ele, está claro. A manhã principiava
branqueando o rio. Procuravam no largo pra ver se tinha alguém nadando.
Alguns foram mandados percorrer o matinho ralo da margem. Dois outros,
no bote, se metiam pelas canaranas pra ver se descobriam os fugitivos.
Foi quando deram pela falta de um chamado Faustino, gritavam ``Faustino!
Faustiiiino!'', e ele percebeu que tinha matado um sujeito chamado
Faustino. Mas quem disse largar o cadáver que agarrava pelo gasnete com
a mão esquerda. O corpo era capaz que boiasse, saindo de baixo do
caixão, haviam de desconfiar. Na margem e na maromba ao lado, o pessoal
se acalmava, era um dia claro. Não tinham achado nem os fugitivos nem
Faustino, vinham contando os que voltavam da procura. Então o chefe
mandou que dois ficassem de vigia na maromba, e o resto dos
perseguidores foram lá na casa do caipira ver se faziam um café. Ele
estava quase vestido, calça cáqui, botas. Mas não tivera tempo de vestir
o dólmã, com a surpresa do ataque, e a camisa tinha se rasgado muito,
justo no braço esquerdo que estava dentro d'água, agarrando o corpo do
Faustino. Fazia já algum tempo que ele vinha percebendo uns estremeções
esquisitos na cara do morto, pois súbito sentiu uma ferroada na mão. O
rio não era de muita piranha, mas tinha alguma sim. Outra ferroada mais
forte e logo ele conferiu que era piranha mesmo, não havia mais dúvida.
E acudia cada vez mais piranha, o que ele não aguentou! As piranhas
mordiam, arrancavam pedacinhos da mão dele e depois do braço também, mas
ele ali, sem se mover. Lá em cima na maromba as duas sentinela
conversavam na calma. Ele percebeu, ia desfalecer na certa, porque já
quase nem se aguentava mais, vista turvando. Então, com muito cuidado,
muita lentidão pra os vigias não repararem, cuidou de enfiar mais que a
mão direita, o braço inteiro no buraco dos pranchões porque assim, se
desmaiasse, pelo menos ficava enganchado ali. Foi quando perdeu os
sentidos. Até fica difícil garantir que perdeu os sentidos ou não
perdeu, nem ele sabe, nem sabe o tempo que passou. Só que as forças
acabaram cedendo, teve um momento em que ele foi chamado à consciência
porque estava engolindo água, sem ar, se afogando. Mesmo fraco como
estava, bracejou, voltou à tona, se agarrou nas canaranas, conseguiu
chegar num chão mais firme e então desmaiou de verdade. Quando voltou a
si, o sol estava bem alto já, devia ser pelo meio do dia. Os inimigos já
tinham ido"-se embora. Então o pobre, ainda ajuntando um resto de força
que possuía, conseguiu se arrastar até próximo da casa do caboclo.
Quando este voltou, mais a mulher, lá dum vizinho longe onde tinham se
refugiado, encontraram o homem estendido no terreiro, moribundo.
Trataram dele. É o que sei\ldots{} o Querino é que anda contando porque até
eu vi, isso eu vi, ele conversando animado com esse homem, porque andou
vários dias indo na casa dele pra fazer uma instalação de gás. Ele
acabou sarando mas diz"-que ficou meio amalucado\ldots{} Se não ficou, parece.

Olharam o homem. Ele já estava no quarto ou quinto duplo, já agora como
inteiramente esquecido de mais ninguém. Tinha o queixo no peito, se
derreara no banco, olhando fixamente o chope escuro. A mão direita
inquieta tamborilava sobre a mesa, mas a esquerda se escondera
preventivamente no bolso da calça. Um dos rapazes se lembrou do caso que
o Alfredo estava contando.

--- Safa! mas que caso mais diferente do do Alfredo!

Mas este, ríspido:

--- Nnnnão\ldots{} deve ser o mesmo\ldots{}

--- Mas o que foi que sucedeu com a mulher?

--- \ldots{} Nnnnão tem importância.

--- Ora, deixa de besteira! Alfredo! que sujeito mais complicado, você!

--- Não tenho nada de complicado não! Essa história de piranha comer
braço de gente, eu nunca soube. O Basílio também me falou que o homem
era de Mato Grosso, leu na caderneta de identidade\ldots{} Mas ele ficou meio
tantã não foi por causa de piranha não, foi a paraguaia. Quando ela
voltou curada pra fazenda, como eu dizia, ela até às vezes acompanhava o
marido a cavalo no campo, mas quando no geral ficava em casa, ficava
ali, rádio aberto, lendo a quantidade de romances policiais e os outros
livros que trouxera da cidade. E não tinha semana que um peão não
trouxesse aquela quantidade de revistas que vinham do correio. Pois um
dia, quando ele chegou em casa, a mulher estava fechada no quarto e não
quis abrir a porta. Ele bateu, chamou de todo jeito, ela gritava que não
amolasse, até que ele perdeu a paciência e ameaçou arrombar a porta. Daí
ela abriu e se percebia que tinha chorado muito. Olhou pra ele com ódio
e gritou:

--- O que você me quer! me deixa!

E coisa assim. Ele estava assombrado, perguntava, ela não respondia, foi
no terraço e se atirou na rede, chorando feito louca. Mas isso?\ldots{} ele
que nem tocasse de leve nela com a mão, ela fugia o corpo como se ele
fosse uma cobra. Não valeu carinho, não valeu queixa: ela estava muda,
longe dele, olhando ele com ódio, e de repente falou que queria ir
embora pra terra dela. Ele não podia entender, foi discutir, mas ela
agarrou dando uns gritos, que ia"-se embora mesmo, que não ficava mais
ali parecia uma doida, saltou da rede, desceu a escadinha do terraço e
deitou correndo pelo pasto, como indo embora pro Paraguai. Foi um custo
trazer ela pra casa, agarrada. Ele muito triste fazia tudo pra acalmar,
jurava que no outro dia mesmo partiam pra Assunção, ela berrava que não!
que havia de ir sozinha e não queria saber mais dele. Ninguém dormiu
naquela casa. A moça acabou se fechando no quarto outra vez. Ele não
quis insistir mais imaginando que o passar da noite havia de acalmar
aquela crise. Puxou uma cadeira e sentou bem na frente da porta,
esperando. Não dormiu nada. Mas também a moça não dormiu, não vê! Toda a
noite ele escutou ela remexendo coisas, era gaveta que abria, que
fechava, móvel arrastando, coisas jogadas no chão.

Diva acabara de levar mais um chope ao homem. Veio se abraçar a um dos
rapazes, perguntando se não pagavam um aperitivo. Dois dos rapazes se
ajeitaram no banco em que estavam, cedendo o lugarzinho no meio onde ela
se espalhou, encostando muito logo nos dois, pra ver se ao menos um
mordia a isca. O homem do bar mesmo sem chamarem, muito acostumado, veio
servir o vermute.

---\ldots{} bem, mas como eu estava contando, no dia seguinte, ainda nem
ficara bastante claro, que a paraguaia abriu a porta do quarto. Vinha
simples, até estava ridícula e bem feia com aquele rosto transtornado,
num vestidinho caseiro, o mais usado, e uma trouxinha de roupa debaixo
do braço. E falou dura que ia"-se embora. Foi tudo em vão e esse homem\ldots{}

--- Que homem? Diva perguntou meio inquieta.

--- Esse que está bebendo chope escuro.

--- Santa Maria! mas será que vocês não podem deixar o pobre do homem em
paz!

--- Fica quieta aí, Diva!

--- Mas\ldots{}

--- Tome seu vermute.

Diva se acomodou de má vontade, irritada, enquanto o contador
continuava:

--- Pois ele gostava tanto da paraguaia que acabou cedendo, imaginando
que aquilo havia de passar se ela partisse como estava exigindo. Mandou
um próprio acompanhá"-la. Depois ele ia atrás, Assunção é pequena, e o
camarada ia industriado pra ficar por lá, seguindo a moça de longe. E
ela foi embora, só, com a trouxinha, sem uma despedida, sem olhar pra
trás. Quando ele foi pra entrar no quarto quase nem se podia andar lá
dentro, tudo aos montes jogado no chão. Os vestidos estavam
estraçalhados de propósito, picados devagar com a tesourinha de unha. As
joias arrebentadas, pedras caras, até o brilhante grande do anel, fora
do aro, relumeando na greta do assoalho. E os livros, os objetos, as
meias de seda, até as roupas dele, ela não poupou nada. E não tinha
levado absolutamente nada. Até a roupa de cama, também picada com a
tesourinha, não sobrara nada sem estrago. Mas agora é que vocês vão se
assombrar!\ldots{} Só bem por cima dos dois travesseiros grandes, amontoados
de propósito no meio da cama, um por cima do outro, tinha um livro. Esse
não estava estragado como os outros. Imaginem que\ldots{} bom, pra encurtar:
era simplesmente uma ``História do Paraguai'' em espanhol, desses livros
resumidos que a gente estudou no grupo. Folheando o livro, ele descobriu
justamente na última página do capítulo que falava da guerra com o
Brasil, está claro que tudo cheio de mentiras horríveis, ele descobriu
naquela letrona dela que mal sabia assinar o nome: ``Infames!''

--- Quem que era infame?

--- Safa, Diva, sua gente mesmo!

--- Que ``minha gente''?

--- Os brasileiros, Diva!

--- Eu não sou brasileira!

O rapaz sorrindo acarinhou os cabelos louros, frios dela. O contador ia
comentando:

--- Foi por causa da guerra do Paraguai\ldots{} O homem ficou feito doido,
não podia mais passar sem ela, se botou atrás da moça, porém ela não
houve meios de ceder. E pra não ser mais incomodada, acabou
desaparecendo de Assunção, ninguém sabe para onde. Foi uma trapalhada
dos dianhos vocês nem imaginam, porque a fazenda, as propriedades não
eram mais dele, e ela nunca reclamou nada, desapareceu pra sempre. Até
andaram falando que ela suicidou"-se, porque continuava apaixonadíssima
pelo brasileiro, apesar. Mas isto nunca se conseguiu tirar a limpo. Ele
é que vendeu o gado e ficou viajando por todo o sul, sempre com pensão
na amante. Quando foi da revolução de 30, se meteu na revolução, sem
gosto, sem acreditar em nada, só porque era revolução contra o Brasil.
Diz"-que ele ia ficando maníaco, odiava o Brasil e dava razão pra Solano
Lopes que foi quem declarou a guerra do Paraguai contra nós. Afinal
conseguiu vender a fazenda e as casas de Cuiabá, mas dizem que na casa
onde ele mora não tem nada. Só que ele prega na parede tudo quanto é
notícia ofendendo o Brasil.

--- Ah, não! isso não deve ser verdade senão o Querino me contava!

--- Por que que só o Querino é que há de saber!

--- Ele entrou vários dias na casa pra instalar o gás, já falei!

--- Uhm\ldots{}

Diva não se conteve mais, arrancou:

--- Tudo isso é uma mentira muito besta! Por que vocês não conversam
noutra coisa!

--- Você conhece ele, é?

--- Diva hesitou.

---\ldots{} nnnão. Mas ele sempre vem aqui.

--- Você já foi com ele?

--- Não, ele quis. Mas falou que eu desculpasse, é muito mais delicado
que vocês todos juntos, sabem!

--- Isso de delicadeza\ldots{} Deve ser é algum viciado, vá ver que não é
outra coisa.

A garçonette ficou indignada. Se ergueu com brutalidade.

--- Arre que vocês também são uns\ldots{} Ia insultar, enojada, mas se
lembrou que era garçonette: Por favor, não olhem tanto pra ele assim!
Ele vai sair\ldots{}

De fato, o homem estava mexendo exagitadamente em dinheiro. Diva foi pra
junto dele, achando jeito, com o corpo, de o esconder da curiosidade dos
rapazes. Fingia procurar troco. Olhou"-o com esperança tristonha:

--- Por que o senhor não toma mais um chope\ldots{} Está quente hoje\ldots{}

Ele estremeceu muito, devorou"-a com os olhos angustiados:

--- Por que a senhora quer que eu tome mais chope hoje! Seis não é a
minha conta de sempre! Estavam falando de mim naquela mesa, não!

E foi saindo muito rápido, escorraçado, sem olhar ninguém, sem esperar
resposta nem troco. Era incontestável que fugia.

Na rua andava com muita pressa, apenas hesitante nas esquinas que
acabava dobrando sempre, procurando desnortear perseguidores invisíveis.
Afinal, seis quarteirões longe, parou brusco. Estava ofegante, suava
muito na noite abafada. Olhou em torno e não tinha ninguém.
Certificou"-se ainda se ninguém o perseguia, mas positivamente não havia
pessoa alguma na rua morta, era já bem mais de uma hora da manhã. Enfim
tirava a mão esquerda do bolso e enxugava com algum sossego o suor do
rosto. A mão era mesmo repugnante de ver, a pele engelhada, muito
vermelha e polida. E assim, justamente por ser o polegar que faltava, a
mão parecia um garfo, era horrível.

Depois de se enxugar, olhou o relógio"-pulseira e tornou a esconder a mão
no bolso. Voltou a caminhar outra vez, e agora andava em passo normal,
sem mais pressa nenhuma. Aos poucos foi se engolfando lá nos próprios
pensamentos, o rosto readquiriu uma seriedade sombria enquanto o passo
se mecanizava. Tomou aquele seu jeito de enfiar o queixo no pescoço,
cabeça baixa, parecia numa concentração absoluta. Algum raro transeunte
que passava, ele nem dava tento mais. As vezes fazia gestos pequenos,
gestos mínimos, argumentando, houve um instante em que sorriu. Mas se
recobrou imediatamente, olhando em volta, apreensivo. Não estava ali
ninguém pra lhe surpreender o riso --- e era aquele sorriso quase esgar,
apenas uma linha larga, vincando uma porção de rugas na face lívida.

Mas decerto perseverara o receio de que o pudessem descobrir sorrindo:
principiou caminhando mais depressa outra vez. Lá na esquina em frente
despontavam alguns rapazes que vinham da noite de sábado, conversando
alto. O homem pretendeu parar, hesitou. Acabou atravessando apenas a
rua, tomando o outro passeio pra não topar de frente com os rapazes.
Enfim chegara na alameda do Triunfo. Três quarteirões mais longe devia
ser a casa onde morava, pelo que afirmara o Alfredo. Na esquina era o
botequim de seu Basílio, que estava fechando. O português chegou na
última porta ainda entreaberta, pediu licença aos três operários, fechou
a porta com um ``boa"-noite'' malcriado. Mas os operários estavam mais
falantes com a cerveja do sábado, chegaram até à beira da calçada e se
deixaram ficar ali mesmo, naquela conversa.

O homem vinha chegando e aos poucos diminuía o andar, observando a
manobra do botequim. Diminuiu o passo mais, dando tempo a que os
operários se afastassem. Afinal parou. Os três homens tinham ficado ali
conversando, e ele estacou, olhou pra trás, pretendendo voltar caminho,
talvez. Depois ficou imóvel, aproveitando o tronco da árvore, disposto a
esperar. Dali espiava os operários sem ser visto. Lhe dava aquela
inquietação subitânea, voltava"-se rápido. Parecia temer que alguém
viesse pela calçada e o apanhasse escondido ali. Mas a rua estava
deserta, não passava mais ninguém.

A situação durava assim pra mais de um quarto de hora e os operários não
davam mostra de partir. O homem esperando sempre, só que a impaciência
crescia nele. Olhava a todo instante o relógio, como se tivesse hora
marcada, olhos pregados nos três vultos da esquina. Falavam alto, a
conversa chegava até junto dele, uma conversa qualquer. Agora vinha lá
do lado oposto da alameda, o rondante, na indiferença, bem pelo meio da
rua, batendo o tacão da botina, no despoliciamento proverbial desta
cidade. O guarda, fosse pelo que fosse, ao menos pra mostrar força
diante da gente na cerveja, resolveu enticar com os operários. E parou
na esquina também, olhando franco os homens, rolando o bastão no pulso.
Os operários nem se deram por achados.

De longe, meio esquecido do esconderijo, o homem agora imóvel, devorava
a cena, olhos escancarados sem piscar. O guarda, vendo que os operários
não se intimidavam com a presença dele, resolveu fazer uma demonstração
de autoridade. Se dirigiu calmo aos homens, que pararam a conversa,
esperando o que o polícia ia falar. O homem chegou a sair com o corpo
todo de trás do tronco, na ânsia de escutar o que o guarda dizia. Mas
este falava baixo, resolvido a principiar pelo conselho, paternal.
Nasceu uma troca de palavras mas pequena, acabou logo, porque os
operários não estavam pra discutir com um rondante ranzinza. Resolveram
obedecer. Aliás era tarde mesmo. Foram"-se embora, ainda conversando mais
alto de propósito, forçando a voz, só porque o guarda falara que eles
estavam acordando quem dormia nas casas. O polícia percebeu, ficou com
raiva, mas também não estava muito disposto a se incomodar, que afinal
os operários eram três, bem fortes. Ficou olhando, mãos na cinta,
ameaçador, quando os três já estavam bem longe, sacudiu a cabeça
agressiva e dobrou a esquina, continuando o seu fingimento de ronda,
batendo tacão.

O homem se viu só. Houve um relaxamento de músculos pelo corpo dele, os
ombros caíram, veio o suspiro de alívio. Reprincipiou a andar
devagarinho, calmo outra vez. Na esquina ainda parou, espiando se o
guarda ia longe. Nem sombra de guarda mais. Atravessou mais rápido a
rua, passou pelo boteco do português, e agora andava com precaução,
tirando o molho volumoso de chaves do bolso. Chegado em frente duma
porta, foi disfarçadamente se dirigindo para a beira da calçada. Parou
sobre a guia, aproveitando a sombra da árvore pra se esconder. Virou os
olhos para um lado e outro, examinando a alameda. Num momento, se
dirigiu quase num pulo para a porta, abriu"-a, deslizou pela abertura,
fechou a porta atrás de si, dando três voltas à chave.
