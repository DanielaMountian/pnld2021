\chapter{Vida e obra de Mário de Andrade}

\begin{flushright}
\textsc{cristiane rodrigues de souza}
\end{flushright}
\bigskip

\section{Sobre o autor}

\noindent{}Mário de Andrade tudo amava. O modernista, em carta de 14 de setembro de
1940, escrita à discípula e amiga Oneyda Alvarenga, retoma consideração
feita por Manuel Bandeira: ele, ``querendo me definir pra me
compreender, uma vez, me disse: --- `Você\ldots{} você tem um amor que não é o
amor do sexo, não é nem mesmo o amor dos homens, nem da humanidade\ldots{}
você tem o \emph{amor do todo}!'\,''\footnote{\textsc{andrade}, Mário de.
  \emph{Cartas:} Mário de Andrade, Oneyda Alvarenga. Organização
  e notas de Oneyda Alvarenga. São Paulo: Duas Cidades, 1983, p. 271.
  (grifos nossos)}. A definição do amigo já aparecia em carta a Osório
de Oliveira, escrita em primeiro de agosto de 1934. Nela, Mário afirma
ao intelectual português que a expressão ``amor do todo'' designa a
atitude de receber, em si, traços que formam outros seres, paisagens e
coisas. Lembra ainda que sentir esse amor leva a mimetizar o diferente,
ou seja, a transformar"-se, em parte, no outro.

\begin{quote}
Mas esse amor do todo, que é verdadeiro, que é incontestável {[}\ldots{}{]},
se trata em verdade daquele mimetismo sublime com que a gente, em busca
de mais um amigo, em busca de mais um amor, só guarda aparentemente de
si mesmo aquilo em que a gente coincide com o que ama, e desenvolve essa
parte, e a materializa em procedimentos variadíssimos. {[}\ldots{}{]}. É
volúpia, é amor, é este inatraiçoável prazer de ser com mais alguém, ou
mesmo com mais alguma coisa {[}\ldots{}{]}\footnote{\textsc{andrade}, Mário de. Cartas de Mário de Andrade (Organização de Osório de Oliveira).
  \emph{Atlântico: revista luso"-brasileira}, n. 2, s/d, p. 2.}.
\end{quote}

O modernista marcado pela disposição de tudo amar acredita ser a obra
artística também impulsionada pelo amor, como fica claro em \emph{O
banquete}, publicação póstuma apresentada inicialmente de maneira
fracionada na \emph{Folha da Manhã}, entre maio, 1943, e fevereiro,
1945. Nele, com adesão ao personagem Pastor Fido, Mário de Andrade
mostra ser a criação da obra de arte conduzida por aquela ``faculdade
principalíssima da inteligência'' que o escritor define como a ``virtude
com que a Fé se confunde, {[}\ldots{}{]} {[}ou como{]} Charitas, vermelha,
incendiada de amor!''\footnote{\textsc{andrade}, Mário de. \emph{O banquete.}
  Introdução de Jorge Coli e Luiz Carlos da Silva Dantas. São Paulo:
  Duas Cidades, 1989, p. 58--9.}. Assim, percebemos que o amor, para
Mário, além de ser manifestação de \emph{Eros} e de \emph{Philia}, pode
ser compreendido como Charitas ou \emph{agapé}, que, de acordo com o
filósofo André Comte"-Sponville, por não possuir carências, leva ao
sacrifício pelo outro --- ``é o amor `tornado permanente e crônico,
estendido à universalidade dos homens'\,''\footnote{\textsc{comte"-sponville},
  André. \emph{Pequeno tratado das grandes virtudes.} Trad. de Eduardo
  Brandão. São Paulo: Martins Fontes, 1995, p. 297--306.}. Dessa forma, o
modernista, ligado a Charitas, ``tudo sacrific{[}a{]}''\footnote{\textsc{andrade},
  Mário de. \emph{Poesias completas.} Vol. 1. Edição de texto apurado,
  anotada e acrescida de documentos por Tatiana Longo Figueiredo e Telê
  Ancona Lopez. Rio de Janeiro: Nova Fronteira, 2013, p. 540.}, ao
buscar compreender o próprio eu e o país em que vive.

O eu que tudo ama também é o eu que se dispersa em fragmentos, como
aparece nos versos do modernista que abrem \emph{Remate de males}
(1930), em que o sujeito se assume ``trezentos {[}\ldots{}{]}
trezentos"-e"-cinquenta''\footnote{Idem, Ibidem, p. 295.}, identificado ao
boi sacrifical de nossas expressões culturais. No ato de acolher em si a
diversidade de seu país, o poeta se desdobra em elementos díspares que o
definem --- as caiçaras da paisagem do Norte do país, os Pirineus da
nossa herança europeia e os espelhos"-vitrais, daquele que veio do imenso
rio ---, como aparece ainda, em versos de ``O carro da miséria'' (1945,
publicação póstuma).

\medskip

\begin{verse}
Não sou mais eu nunca fui eu decerto\\
Aos pedaços me vim --- eu caio! --- aos pedaços disperso\\
Projetado em vitrais nos joelhos nas caiçaras\\
Nos Pireneus em pororoca prodigiosa\\
Rompe a consciência nítida: \textsc{eu tudoamo}\footnote{Idem, Ibidem, p. 473.}.
\end{verse}

\medskip

O percurso intelectual de Mário de Andrade também é marcado pelo
movimento de tudo amar, como se pode perceber por meio da diversidade de
sua produção, que se abre em diferentes direções e que busca servir
amorosamente o país. Como se sabe, Mário de Andrade, além de poeta,
cronista, autor de textos de ficção, com olhar artístico voltado ainda
para a música e a fotografia, foi também crítico e teórico, em diversas
áreas. Assim o crítico literário e de outras manifestações artísticas,
era ainda estudioso da música, das artes plásticas, da arquitetura, de
questões em torno da cultura afro"-brasileira e das manifestações
populares, além de correspondente incansável, em contato com nomes
importantes do cenário da época. Atuou como professor no Conservatório
Dramático e Musical de São Paulo e na Universidade do Distrito Federal,
no Rio de Janeiro. Idealizador do Serviço do Patrimônio Histórico e
Artístico Nacional, o \textsc{sphan}, foi ainda diretor do Departamento de
Cultura da cidade São Paulo, realizando importantes projetos que se
filiavam a um pensamento democrático.

\section{Sobre a obra}

Na poesia, especificamente, o olhar que tudo ama aparece nos diferentes
livros, na medida em que, por meio da palavra poética, o eu lírico quer
compreender a si mesmo e também o outro --- seu país ---, percebendo"-o
como parte do próprio ser fragmentado. Em \emph{Pauliceia desvairada}
(1922), por exemplo, o eu lírico percorre a cidade moderna, apontando
suas contradições e sentindo"-as em si, já que é tupi, mas tange o
alaúde, instrumento europeu. Depois, o poeta volta seu olhar para o
interior do Brasil, como em poemas de \emph{Clã do jabuti} (1927),
atualizando em versos a cultura popular. No livro está ainda ``Carnaval
carioca'' (1924), em que o dilema de ser um e ser outro ao mesmo tempo
--- o intelectual paulistano que procura sentir em si o ritmo solto da
festa popular --- dita o ritmo dos versos. Essa entrega ao outro aparece
também em ``Poemas da negra'', de \emph{Remate de males} (1930),
em que o sujeito lírico deseja a mulher que, misturada à paisagem
nordestina, acena com outro modo de ser, revelador do que está além da
aparência das coisas. Em ``Girassol da madrugada'', do ``Livro azul'',
parte do volume \emph{Poesias} (1941), no contato amoroso, o eu e o
outro se encontram, já que o poeta vê a si mesmo refletido no
olhar"-lagoa do ser amado, aproximando"-se, assim, do estado de
contemplação artística e de um entendimento de mundo que se assemelha
àquele encontrado por Mário de Andrade, em suas leituras de textos da
filosofia oriental. Em outros poemas, o movimento de tudo amar se liga
ao engajamento político de Mário de Andrade. Em ``A meditação sobre o
Tietê'', por exemplo, de \emph{Lira paulistana} (1945, publicação
póstuma), além de retomar seu percurso de intelectual e de pensar acerca
da própria criação literária, o poeta vivencia o ``amor do todo'', na
medida em que se identifica com o rio --- ``Me sinto o pai Tietê! ôh
força dos meus sovacos!''\footnote{Idem, Ibidem, p. 534.} ---, num
movimento mimético que se amplia e recebe em si não apenas a experiência
de pessoas de diversas camadas sociais que se misturam às águas, mas
também a herança dolorosa de séculos de civilização, já que o rio
oprimido pelas margens é conduzido pelos donos da vida, numa clara
referência ao contexto político opressor da época.

Assim, o conhecimento da poesia de Mário de Andrade, por meio desta
antologia, leva a perceber a experiência profunda do sujeito poético que
vivencia os dilemas do país que se moderniza ao mesmo tempo em que
apresenta, por meio da expressão de seu povo, o primitivo procurado
pelas vanguardas europeias. Conhecê"-la é fundamental para que se perceba
a configuração e o amadurecimento da arte moderna no Brasil, já que o
poeta"-pesquisador, preocupado em compreender a si mesmo e o seu país,
construiu obra que marcou a literatura brasileira, iluminando, em meio
ao modernismo, os caminhos de sua geração, sendo revisitado, até os dias
atuais, pelos poetas e leitores da contemporaneidade. Dessa forma,
percorrer sua obra poética, em seu diálogo com a tradição e com as
inovações modernistas, significa analisar o modo como o autor de
\emph{Macunaíma} se insere no painel da literatura, modificando"-a. Vale
a oportunidade de testemunhar a entrega do eu à paisagem e ao outro, que
reconhece como parte de si mesmo, reveladora de sua consciência
artística.

%\subsection{nota}

\section{Sobra o gênero}

Desde cedo, Mário de Andrade demonstrou talento para a música, o que se evidencia em sua percepção para o ritmo e a musicalidade da poesia.

Seu primeiro livro foi \emph{Há uma gota de sangue em cada poema}
(1917), que ele assinou com o pseudônimo de Mário Sobral. Nele, ainda
estão presentes as influências do simbolismo e do parnasianismo, mas já
aparecem nesse livro um pouco da sua ``cara'' como escritor modernista.
Poucos anos depois, ele se engajou no modernismo, movimento que, no
começou, se opôs radicalmente a essas estéticas anteriores, influenciado
pelas vanguardas artísticas europeias.

Em 1922, Mário publicou o livro de poemas \emph{Pauliceia desvairada},
sua obra-manifesto, no mesmo ano em que trabalhava na organização de um
dos eventos mais importantes da vida intelectual e cultural brasileira
no século \textsc{xx}, a Semana de Arte Moderna, ocorrida entre os dias 11 e 18
de fevereiro de 1922, no Theatro Municipal de São Paulo, e que contou
ainda com a participação de artistas como Oswald de Andrade, Anita
Malfatti e Heitor Villa"-Lobos.

Considerado um dos livros inaugurais do modernismo, \emph{Pauliceia desvairada}
já coloca algumas linhas de força que guiariam toda a composição poética de Mário de Andrade: desde sua capa, que na primeira edição remetia ao arlequim do teatro popular italiano, coloca"-se a estrutura arlequinal de seus poemas, formados por pares antitéticos tal qual a roupa do arlequim, dois triângulos opostos que formam o losango da veste teatral. Ao dedicar o livro a si mesmo, Mário também está fazendo esse movimento dual: coloca"-se como mestre e discípulo de si próprio, rompe com sua obra prematura, de traços mais parnasianos, ao duplicar sua subjetividade e inagurar sujeito cindido.

Como analisa Peter Bürger em \textit{Teoria da vanguarda}, uma das características da arte moderna de vanguarda é a montagem, que pressupõe a fragmentação da realidade, assim afastada de um todo coerente. Nos versos de \textit{Pauliceia}, por exemplo, percebe"-se como a simultaneidade e sobreposição de vozes pretende dar conta da caótica experiência urbana, em que reflexões do eu"-lírico colam"-se a cenas urbanas, vozes e burburinhos vindos dos bondes e bares.
Assim, o poeta procurava uma poesia que, ao mesmo tempo, fosse expressão individual, estruturada estéticamente e com função social.

O terceiro livro de poemas de Mário de Andrade, \textit{Losango cáqui}, remete igualmente à composição poliédrica dessa experiência moderna, composta de contradições e pelo dilaceramento do sujeito moderno, que não tem mais unidade.
Se o poeta é arlequinal, não menos o é a sua cidade, concerto polifônico de vozes e raças. A mistura racial evidencia"-se mais fortemente em seu próximo livro, \textit{Clã do jabuti}, que originalmente chamar"-se"-ia \textit{As três raças}.
Coloca"-se aqui, com mais força, a preocupação nacional de Mário, do que é a nacionalidade e a identidade brasileira. Com o ``clã'' do título o poeta remente a algo familiar, antigo, totêmico. Já o ``jabuti'' sintetiza a mistura nacional de diversos regionalismos pois, além de rementer novamente aos losangos arlequinais, os quelônios do jabuti figuram como a mistura de culturas e raças que formam o Brasil. Essa preocupação pode ser vista no aspecto formal dos poemas, que recriam a forma musical das cantigas populares, como se o livro fosse uma grande partitura da nação.

Para terminar esse breve panorama da inserção da obra de Mário de Andrade no gênero poético, abordado aqui em suas características modernas e vanguardistas, focalizam"-se mais dois livros que ajudam a entender a evolução do pensamento e da estética do poeta, \textit{Livro azul} e \textit{Lira paulistana}.
Decepcionado com os rumos políticos do país na década de 1930, percebemos o poeta, em \textit{Livro azul}, virar"-se para a paz amazônica, em uma atitude menos entusiasta e na procura de uma posição mais sábia. 
Se formos pensar com Theodor Adorno, o desligamento aparente do poeta com o social, em mergulho subjetivo, evidencia a própria denúncia da fissura com o real, em negação do social tal qual encontra"-se constituído.

O primeiro poema de \textit{Livro azul}, ``Rito do irmão pequeno'', demonstra esse embate entre o impulso de agir e o recolhimento interno, em busca de uma sabedoria. O irmão pequeno pode ser visto como a própria criança com todo seu instinto, seu impulso à ação, primitivo, vivaz, a vida que aflora em ímpeto. Já o eu"-lírico, através do poema, parece se esforçar para pacificar esse irmão pequeno e sua força de vontade de agir no mundo, limando"-lhe a crença no progresso.
Há uma invocação do tempo mítico, que relativiza o tempo e o espaço, para inaugurar um movimento em direção interior, a um tempo sem história que se integra ao ritmo lento dos processos da natureza. Nesse movimento, funda"-se um gesto sacrificial: o abandono da agitação da vida urbana, do progresso, pelo gesto ritual, aniquilando o eu, a subjetividade e o próprio Brasil, que pode ser alegorizado na figura dos bois que aparecem no poema.

Já em \textit{Lira paulistana}, em um movimento decrescente de decepção com os rumos do país, Mário de Andrade assume uma posição mais combativa, configurando talvez o livro mais político do poeta. Sem mais uma perspectiva utópica, o livro pode ser visto como uma reescrita da \textit{Pauliceia desvairada} pelo inverso, abandonando o encanto brincalhão com a modernidade e incorporando a luta de classes --- como se o poeta estivesse ansioso por repensar o que fez e pensou para sua nação. O tom do livro é mais sério, claro, objetivo e pessimista, a ponto de o crítico literário Antonio Candido afirmar que o poema que encerra o volume, ``A meditação sobre o Tietê'', funcionar como um testamento de Mário de Andrade, que observa seus projetos afundando nas águas do rio.

Essa perspectiva combativa e pessimista pode ser vista em outros poemas da obra aqui reunidos, como ``Garoa do meu São Paulo'', em que às redondilhas sonoras e musicadas do versos parece corresponder a triste melodia que a chuva produz ao cair sobre a cidade. Toda a riqueza e as particularidades da nação que o poeta vislumbrava anteriormente são, aqui, reduzidas pela cidade ao branco e rico, pois a cidade parece a tudo homogenizar em detrimento do pobre e preto. O conflito com a dominação estrangeira também parece figurar na contraposição de São Paulo e Londres.
A garoa da cidade, então, parece revestir"-se de caráter histórico, é a própria ideologia que anuvia os olhos do poeta e precisa ser descortinada para se enxergar as verdadeiras pessoas e particularidades da cidade.

Em ``A catedral de São Paulo'', a modernidade conservadora do país é espelhada no reflexo do poeta na Catedral da Sé de São Paulo, construída entre 1913--1954.
Marco do modernismo brasileiro, a igreja é contraditoriamente influenciada pelo gótico europeu, revelando as estruturas arcaicas e medievais da pretensa modernidade brasileira. 
Estruturado como uma cantiga medieval, o poema simula a própria arquitetura gótica, com refrões que ecoam como os repiques do sino da catedral. É, no entanto, um poema profundamente moderno ao apontar para uma subjetividade fraturada e cindida. Essa ambiguidade do eu"-lírico, portanto, pode ser vista como um pastiche dessa subjetividade medieval, forjada falsamente pela modernidade tal qual a catedral, reflexo da má formação da subjetividade nacional.

Já em ``Quando eu morrer quero ficar'', observamos a alma sem corpo do eu"-lírico, condenada às mesmas tormentas que fora em vida, identificada com cada pedaço da cidade. O eu"-lírico, quase como um boi esquartejado de uma festividade do Bumba"-meu"-boi, só se produz pelo próprio esquartejamento.

Por fim, em ``Agora eu quero cantar'', a temática da luta de classes é colocada de forma mais evidente. Composto em redondilha maior, o poema é quase como a \textit{via crucis} do operário pobre, que canta sua vida miserável do nascimento à morte --- vale lembrar o diálogo que João Cabral de Melo Neto estabelece com esse poema em sua \textit{Morte e vida severina}.
O poema é todo estruturado em uma negatividade crescente, em que o eu"-lírico, como uma espécie de aedo, canta narrativamente a história de Pedro, que aparece como destinado à morte, acabando a vida da mesma forma que a começou.
Todas as etapas de sua vida acabam com um bordão negativo (``um sono bruto o prostrou''), revelando essa circularidade da miséria que não faz progredir.

Ao final do poema, o eu"-lírico finalmente se descola de Pedro ao esboçar uma reação exasperada diante da alienação do operário.
Coloca"-se, assim, o abismo entre o trabalhador e o intelectual, que pode expressar a revolta que o primeiro não sente. Ao cabo, a terra sonhada por Pedro é sua própria tumba, desmistificando os caminhos da evolução e do progresso civilizacional.
Ressalta"-se que Mário de Andrade escreveu esses versos no momento em que o país se industrializava e encarava sua modernização.


%Mas devemos considerar que, na sociedade de consumo, a categoria do novo não é nenhuma categoria substancial, mas uma categoria aparente. O que ela descreve não é a natureza das mercadorias, mas a forma exterior impressa artificialmente nelas (o novo nas mercadorias é a embalagem) p. 140
