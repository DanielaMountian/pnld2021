\chapterspecial{Vida e obra}{}{Cristiane Rodrigues de Souza}

\section{Sobre o autor}

Mário de Andrade tudo amava. O modernista, em carta de 14 de setembro de
1940, escrita à discípula e amiga Oneyda Alvarenga, retoma consideração
feita por Manuel Bandeira: ele, ``querendo me definir pra me
compreender, uma vez, me disse: --- `Você\ldots{} você tem um amor que não é o
amor do sexo, não é nem mesmo o amor dos homens, nem da humanidade\ldots{}
você tem o \emph{amor do todo}!'\,''\footnote{\textsc{andrade}, Mário de.
  \emph{Cartas:} Mário de Andrade, Oneyda Alvarenga. Organização
  e notas de Oneyda Alvarenga. São Paulo: Duas Cidades, 1983, p. 271.
  (grifos nossos)}. A definição do amigo já aparecia em carta a Osório
de Oliveira, escrita em primeiro de agosto de 1934. Nela, Mário afirma
ao intelectual português que a expressão ``amor do todo'' designa a
atitude de receber, em si, traços que formam outros seres, paisagens e
coisas. Lembra ainda que sentir esse amor leva a mimetizar o diferente,
ou seja, a transformar"-se, em parte, no outro.

\begin{quote}
Mas esse amor do todo, que é verdadeiro, que é incontestável {[}\ldots{}{]},
se trata em verdade daquele mimetismo sublime com que a gente, em busca
de mais um amigo, em busca de mais um amor, só guarda aparentemente de
si mesmo aquilo em que a gente coincide com o que ama, e desenvolve essa
parte, e a materializa em procedimentos variadíssimos. {[}\ldots{}{]}. É
volúpia, é amor, é este inatraiçoável prazer de ser com mais alguém, ou
mesmo com mais alguma coisa {[}\ldots{}{]}\footnote{\textsc{andrade}, Mário de. Cartas de Mário de Andrade (Organização de Osório de Oliveira).
  \emph{Atlântico: revista luso"-brasileira}, n. 2, s/d, p. 2.}.
\end{quote}

O modernista marcado pela disposição de tudo amar acredita ser a obra
artística também impulsionada pelo amor, como fica claro em \emph{O
banquete}, publicação póstuma apresentada inicialmente de maneira
fracionada na \emph{Folha da Manhã}, entre maio, 1943, e fevereiro,
1945. Nele, com adesão ao personagem Pastor Fido, Mário de Andrade
mostra ser a criação da obra de arte conduzida por aquela ``faculdade
principalíssima da inteligência'' que o escritor define como a ``virtude
com que a Fé se confunde, {[}\ldots{}{]} {[}ou como{]} Charitas, vermelha,
incendiada de amor!''\footnote{\textsc{andrade}, Mário de. \emph{O banquete.}
  Introdução de Jorge Coli e Luiz Carlos da Silva Dantas. São Paulo:
  Duas Cidades, 1989, p. 58--9.}. Assim, percebemos que o amor, para
Mário, além de ser manifestação de \emph{Eros} e de \emph{Philia}, pode
ser compreendido como Charitas ou \emph{agapé}, que, de acordo com o
filósofo André Comte"-Sponville, por não possuir carências, leva ao
sacrifício pelo outro --- ``é o amor `tornado permanente e crônico,
estendido à universalidade dos homens'\,''\footnote{\textsc{comte"-sponville},
  André. \emph{Pequeno tratado das grandes virtudes.} Trad. de Eduardo
  Brandão. São Paulo: Martins Fontes, 1995, p. 297--306.}. Dessa forma, o
modernista, ligado a Charitas, ``tudo sacrific{[}a{]}''\footnote{\textsc{andrade},
  Mário de. \emph{Poesias completas.} Vol. 1. Edição de texto apurado,
  anotada e acrescida de documentos por Tatiana Longo Figueiredo e Telê
  Ancona Lopez. Rio de Janeiro: Nova Fronteira, 2013, p. 540.}, ao
buscar compreender o próprio eu e o país em que vive.

O eu que tudo ama também é o eu que se dispersa em fragmentos, como
aparece nos versos do modernista que abrem \emph{Remate de males}
(1930), em que o sujeito se assume ``trezentos {[}\ldots{}{]}
trezentos"-e"-cinquenta''\footnote{Idem, Ibidem, p. 295.}, identificado ao
boi sacrifical de nossas expressões culturais. No ato de acolher em si a
diversidade de seu país, o poeta se desdobra em elementos díspares que o
definem --- as caiçaras da paisagem do Norte do país, os Pirineus da
nossa herança europeia e os espelhos"-vitrais, daquele que veio do imenso
rio ---, como aparece ainda, em versos de ``O carro da miséria'' (1945,
publicação póstuma).

\medskip

\begin{verse}
Não sou mais eu nunca fui eu decerto\\
Aos pedaços me vim --- eu caio! --- aos pedaços disperso\\
Projetado em vitrais nos joelhos nas caiçaras\\
Nos Pireneus em pororoca prodigiosa\\
Rompe a consciência nítida: \textsc{eu tudoamo}\footnote{Idem, Ibidem, p. 473.}.
\end{verse}

\medskip

O percurso intelectual de Mário de Andrade também é marcado pelo
movimento de tudo amar, como se pode perceber por meio da diversidade de
sua produção, que se abre em diferentes direções e que busca servir
amorosamente o país. Como se sabe, Mário de Andrade, além de poeta,
cronista, autor de textos de ficção, com olhar artístico voltado ainda
para a música e a fotografia, foi também crítico e teórico, em diversas
áreas. Assim o crítico literário e de outras manifestações artísticas,
era ainda estudioso da música, das artes plásticas, da arquitetura, de
questões em torno da cultura afro"-brasileira e das manifestações
populares, além de correspondente incansável, em contato com nomes
importantes do cenário da época. Atuou como professor no Conservatório
Dramático e Musical de São Paulo e na Universidade do Distrito Federal,
no Rio de Janeiro. Idealizador do Serviço do Patrimônio Histórico e
Artístico Nacional, o \textsc{sphan}, foi ainda diretor do Departamento de
Cultura da cidade São Paulo, realizando importantes projetos que se
filiavam a um pensamento democrático.

\section{Sobre a obra}

Na poesia, especificamente, o olhar que tudo ama aparece nos diferentes
livros, na medida em que, por meio da palavra poética, o eu lírico quer
compreender a si mesmo e também o outro --- seu país ---, percebendo"-o
como parte do próprio ser fragmentado. Em \emph{Pauliceia desvairada}
(1922), por exemplo, o eu lírico percorre a cidade moderna, apontando
suas contradições e sentindo"-as em si, já que é tupi, mas tange o
alaúde, instrumento europeu. Depois, o poeta volta seu olhar para o
interior do Brasil, como em poemas de \emph{Clã do jabuti} (1927),
atualizando em versos a cultura popular. No livro está ainda ``Carnaval
carioca'' (1924), em que o dilema de ser um e ser outro ao mesmo tempo
--- o intelectual paulistano que procura sentir em si o ritmo solto da
festa popular --- dita o ritmo dos versos. Essa entrega ao outro aparece
também em ``Poemas da negra'', de \emph{Remate de males} (1930),
em que o sujeito lírico deseja a mulher que, misturada à paisagem
nordestina, acena com outro modo de ser, revelador do que está além da
aparência das coisas. Em ``Girassol da madrugada'', do ``Livro azul'',
parte do volume \emph{Poesias} (1941), no contato amoroso, o eu e o
outro se encontram, já que o poeta vê a si mesmo refletido no
olhar"-lagoa do ser amado, aproximando"-se, assim, do estado de
contemplação artística e de um entendimento de mundo que se assemelha
àquele encontrado por Mário de Andrade, em suas leituras de textos da
filosofia oriental. Em outros poemas, o movimento de tudo amar se liga
ao engajamento político de Mário de Andrade. Em ``A meditação sobre o
Tietê'', por exemplo, de \emph{Lira paulistana} (1945, publicação
póstuma), além de retomar seu percurso de intelectual e de pensar acerca
da própria criação literária, o poeta vivencia o ``amor do todo'', na
medida em que se identifica com o rio --- ``Me sinto o pai Tietê! ôh
força dos meus sovacos!''\footnote{Idem, Ibidem, p. 534.} ---, num
movimento mimético que se amplia e recebe em si não apenas a experiência
de pessoas de diversas camadas sociais que se misturam às águas, mas
também a herança dolorosa de séculos de civilização, já que o rio
oprimido pelas margens é conduzido pelos donos da vida, numa clara
referência ao contexto político opressor da época.

Assim, o conhecimento da poesia de Mário de Andrade, por meio desta
antologia, leva a perceber a experiência profunda do sujeito poético que
vivencia os dilemas do país que se moderniza ao mesmo tempo em que
apresenta, por meio da expressão de seu povo, o primitivo procurado
pelas vanguardas europeias. Conhecê"-la é fundamental para que se perceba
a configuração e o amadurecimento da arte moderna no Brasil, já que o
poeta"-pesquisador, preocupado em compreender a si mesmo e o seu país,
construiu obra que marcou a literatura brasileira, iluminando, em meio
ao modernismo, os caminhos de sua geração, sendo revisitado, até os dias
atuais, pelos poetas e leitores da contemporaneidade. Dessa forma,
percorrer sua obra poética, em seu diálogo com a tradição e com as
inovações modernistas, significa analisar o modo como o autor de
\emph{Macunaíma} se insere no painel da literatura, modificando"-a. Vale
a oportunidade de testemunhar a entrega do eu à paisagem e ao outro, que
reconhece como parte de si mesmo, reveladora de sua consciência
artística.

%\subsection{nota}

Além de assinalar o percurso da poesia modernista de Mário de Andrade,
desde 1922, marcado, como vimos, pelo ``amor do todo'', esta antologia
se organiza ainda de maneira a estabelecer conjuntos de textos que dão a
ver traços essenciais de cada livro de poemas de Mário de Andrade,
escolhidos, levando"-se em conta a alta qualidade formal que apresentam.

Sua elaboração tem base nos resultados de pesquisas que realizei, nos
últimos 20 anos, acerca da poesia de Mário de Andrade, em instituições
como a \textsc{unesp} de Araraquara, a Faculdade de Filosofia, Letras e Ciências
Humanas da \textsc{usp} e o Instituto de Estudos Brasileiros, também da
Universidade de São Paulo, com o apoio de seus docentes, pesquisadores e
funcionários, contando com fomento da \textsc{fapesp}.

No intuito de assegurar a qualidade do livro, no baseamos nos dois
volumes das \emph{Poesias completas}, publicados pela Nova Fronteira, em
2013, sob os cuidados de Tatiana Longo Figueiredo e de Telê Ancona
Lopez, tendo em vista o rigoroso e incansável trabalho de pesquisa que
realizaram, com olhares atentos a manuscritos, à correspondência, às
anotações marginais nos livros da biblioteca de Mário de Andrade, assim
como a edições publicadas em vida pelo autor, apresentando um texto
apurado, capaz de revelar o projeto estético da poesia de Mário de
Andrade.

\section{Sobra o gênero}
