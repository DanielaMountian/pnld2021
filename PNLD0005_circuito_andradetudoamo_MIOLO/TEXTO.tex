\chapter{Apresentação}

\begin{flushright}
\textsc{cristiane rodrigues de souza}
\end{flushright}
\bigskip

\noindent{}Essa antologia traz os poemas mais representativos dos principais livros que formam a trajetória poética de Mário de Andrade.
Além de assinalar o percurso da poesia modernista de Mário de Andrade,
desde 1922, marcado pelo ``amor do todo'', esta antologia
se organiza ainda de maneira a estabelecer conjuntos de textos que dão a
ver traços essenciais de cada livro de poemas de Mário de Andrade,
escolhidos, levando"-se em conta a alta qualidade formal que apresentam.

Sua elaboração tem base nos resultados de pesquisas que realizei, nos
últimos 20 anos, acerca da poesia de Mário de Andrade, em instituições
como a \textsc{unesp} de Araraquara, a Faculdade de Filosofia, Letras e Ciências
Humanas da \textsc{usp} e o Instituto de Estudos Brasileiros, também da
Universidade de São Paulo, com o apoio de seus docentes, pesquisadores e
funcionários, contando com fomento da \textsc{fapesp}.

Levando em consideração o conjunto da obra poética de Mário de Andrade,
destacam"-se nesta antologia alguns temas, imagens e linguagens que
caracterizam a obra desse autor. Chama a atenção o projeto arrojado de
abarcar, nos poemas, as culturas brasileiras e suas expressões
distintas, como se observa, por exemplo, no itinerário que vai da São
Paulo ``arlequinal'' da \emph{Pauliceia desvairada} aos ``Dois Poemas
Acreanos'', do \emph{Clã do jabuti}, passando pelo ``Carnaval Carioca''
e pelo ``Noturno de Belo Horizonte''. Os muitos tempos, lugares e povos
do Brasil se encontram nos poemas desta antologia, em cujas formas estão
inscritas falas e saberes populares, que lhes imprimem ritmos e
paisagens diversas. Paralelamente, a perspectiva modernista de
renovação da criação artística brasileira também participa da
composição desses textos, na ruptura com as formas fixas dos parnasianos
e na celebração da liberdade formal e dos temas urbanos, cotidianos,
especialmente da cidade de São Paulo. O resultado é um conjunto
fascinante, heterogêneo e coerente, matizado e harmonioso, que abriu
portas para muitos poetas que vieram depois.

No intuito de assegurar a qualidade do livro, no baseamos nos dois
volumes das \emph{Poesias completas}, publicados pela Nova Fronteira, em
2013, sob os cuidados de Tatiana Longo Figueiredo e de Telê Ancona
Lopez, tendo em vista o rigoroso e incansável trabalho de pesquisa que
realizaram, com olhares atentos a manuscritos, à correspondência, às
anotações marginais nos livros da biblioteca de Mário de Andrade, assim
como a edições publicadas em vida pelo autor, apresentando um texto
apurado, capaz de revelar o projeto estético da poesia de Mário de
Andrade.

\part{\textsc{pauliceia desvairada}}

\chapter{Inspiração}
\openany

{\setlength{\epigraphwidth}{.45\textwidth}
\epigraph{Onde até na força do verão havia\\
tempestades de ventos e frios de\\
crudelíssimo inverno}{\textsc{fr.\,luís de sousa}}}


\begin{verse}
São Paulo! comoção de minha vida\ldots{}\\
Os meus amores são flores feitas de original!\ldots{}\\
Arlequinal!\ldots{} Traje de losangos\ldots{} Cinza e ouro\ldots{}\\
Luz e bruma\ldots{} Forno e inverno morno\ldots{}\\
Elegâncias sutis sem escândalos, sem ciúmes\ldots{}\\
Perfumes de Paris\ldots{} Arys!\\
Bofetadas líricas no Trianon\ldots{} Algodoal!\ldots{}

São Paulo! comoção de minha vida\ldots{}\\
Galicismo a berrar nos desertos da América!
\end{verse}

\chapter{O trovador}

\begin{verse}
Sentimentos em mim do asperamente\\
dos homens das primeiras eras\ldots{}\\
As primaveras de sarcasmo\\
intermitentemente no meu coração arlequinal\ldots{}\\
Intermitentemente\ldots{}\\
Outras vezes é um doente, um frio\\
na minha alma doente como um longo som redondo\ldots{}\\
Cantabona! Cantabona!\\
Dlorom\ldots{}

Sou um tupi tangendo um alaúde!
\end{verse}

\chapter{Rua de São Bento}

\begin{verse}
Triângulo.

Há navios de vela para os meus naufrágios!\\
E os cantares da uiara rua de São Bento\ldots{}

Entre estas duas ondas plúmbeas de casas plúmbeas,\\
as minhas delícias das asfixias da alma!\\
Há leilão. Há feira de carnes brancas. Pobres arrozais!\\
Pobres brisas sem pelúcias lisas a alisar!\\
A cainçalha\ldots{} A Bolsa\ldots{} As jogatinas\ldots{}

Não tenho navios de vela para mais naufrágios!\\
Faltam"-me as forças! Falta"-me o ar!\\
Mas qual! Não há sequer um porto morto!\\
-- Can you dance the tarantella? -- Ach! ya.\\
São as califórnias duma vida milionária\\
numa cidade arlequinal\ldots{}

O Clube Comercial\ldots{} A Padaria Espiritual\ldots{}\\
Mas a desilusão dos sombrais amorosos\\
põe majoration temporaire, 100\% \textsuperscript{nt}!\ldots{}

Minha Loucura, acalma"-te!\\
Veste o water"-proof dos tambéns!\\
Nem chegarás tão cedo\\
à fábrica de tecidos dos teus êxtases;\\
telefone: Além, 3991\ldots{}\\
Entre estas duas ondas plúmbeas de casas plúmbeas,\\
vê, lá nos muito"-ao"-longes do horizonte,\\
a sua chaminé de céu azul!
\end{verse}

\chapter{Paisagem nº 1}

\begin{verse}
Minha Londres das neblinas finas!\\
Pleno verão. Os dez mil milhões de rosas paulistanas.\\
Há neve de perfumes no ar.\\
Faz frio, muito frio\ldots{}\\
E a ironia das pernas das costureirinhas\\
parecidas com bailarinas\ldots{}\\
O vento é como uma navalha\\
nas mãos dum espanhol. Arlequinal!\ldots{}\\
Há duas horas queimou sol.\\
Daqui a duas horas queima sol.

Passa um São Bobo, cantando, sob os plátanos,\\
um tralalá\ldots{} A guarda"-cívica! Prisão!\\
Necessidade a prisão\\
para que haja civilização?\\
Meu coração sente"-se muito triste\ldots{}\\
Enquanto o cinzento das ruas arrepiadas\\
dialoga um lamento com o vento\ldots{}

Meu coração sente"-se muito alegre!\\
Este friozinho arrebitado\\
dá uma vontade de sorrir!\\
E sigo. E vou sentindo,\\
à inquieta alacridade da invernia,\\
como um gosto de lágrimas na boca\ldots{}
\end{verse}

\chapter{Ode ao burguês}

\begin{verse}
Eu insulto o burguês! O burguês"-níquel,\\
o burguês"-burguês!\\
A digestão bem feita de São Paulo!\\
O homem"-curva! o homem"-nádegas!\\
O homem que sendo francês, brasileiro, italiano,\\
é sempre um cauteloso pouco"-a"-pouco!

Eu insulto as aristocracias cautelosas!\\
Os barões lampiões! os condes Joões! os duques zurros!\\
que vivem dentro de muros sem pulos;\\
e gemem sangues de alguns milréis fracos\\
para dizerem que as filhas da senhora falam o francês\\
e tocam o \emph{Printemps} com as unhas!

Eu insulto o burguês"-funesto!\\
O indigesto feijão com toucinho, dono das tradições!\\
Fora os que algarismam os amanhãs!\\
Olha a vida dos nossos setembros!\\
Fará sol? Choverá? Arlequinal!\\
Mas à chuva dos rosais\\
o êxtase fará sempre sol!

Morte à gordura!\\
Morte às adiposidades cerebrais!\\
Morte ao burguês"-mensal!\\
ao burguês"-cinema! Ao burguês"-tílburi!\\
Padaria Suíça! Morte viva ao Adriano!\\
``-- Ai, filha, que te darei pelos teus anos?\\
-- Um colar\ldots{} -- Conto e quinhentos!!!\\
Mas nós morremos de fome!''

Come! Come"-te a ti mesmo, oh! gelatina pasma!\\
Oh! purée de batatas morais!\\
Oh! cabelos nas ventas! oh! carecas!\\
Ódio aos temperamentos regulares!\\
Ódio aos relógios musculares! Morte e infâmia!\\
Ódio à soma! Ódio aos secos e molhados!\\
Ódio aos sem desfalecimentos nem arrependimentos,\\
sempiternamente as mesmices convencionais!\\
De mãos nas costas! Marco eu o compasso! Eia!\\
Dois a dois! Primeira posição! Marcha!\\
Todos para a Central do meu rancor inebriante!

Ódio e insulto! Ódio e raiva! Ódio e mais ódio!\\
Morte ao burguês de giolhos,\\
cheirando religião e que não crê em Deus!\\
Ódio vermelho! Ódio fecundo! Ódio cíclico!\\
Ódio fundamento, sem perdão!

Fora! Fu! Fora o bom burguês!\ldots{}
\end{verse}

\chapter{O Domador}

\begin{verse}
Alturas da Avenida. Bonde 3.\\
Asfaltos. Vastos, altos repuxos de poeira\\
sob o arlequinal do céu ouro"-rosa"-verde\ldots{}\\
As sujidades implexas do urbanismo.\\
Filets de manuelino. Calvícies de Pensilvânia.\\
Gritos de goticismo.\\
Na frente o tram da irrigação,\\
onde um sol bruxo se dispersa\\
num triunfo persa de esmeraldas, topázios e rubis\ldots{}\\
Lânguidos boticellis a ler Henry Bordeaux\\
nas clausuras sem dragões dos torreões\ldots{}

Mário, paga os duzentos réis.\\
São cinco no banco: um branco,\\
um noite, um ouro,\\
um cinzento de tísica e Mário\ldots{}\\
Solicitudes! Solicitudes!

Mas\ldots{} olhai, oh meus olhos saudosos dos ontens\\
esse espetáculo encantado da Avenida!\\
Revivei, oh gaúchos paulistas ancestremente!\\
e oh cavalos de cólera sanguínea!\\
Laranja da China, laranja da China, laranja da China!\\
Abacate, cambucá e tangerina!\\
Guardate! Aos aplausos do esfuziante clown,\\
heroico sucessor da raça heril dos bandeirantes,\\
passa galhardo um filho de imigrante,\\
louramente domando um automóvel!
\end{verse}

\chapter{Noturno}

\begin{verse}
Luzes do Cambuci pelas noites de crime\ldots{}\\
Calor!\ldots{} E as nuvens baixas muito grossas,\\
feitas de corpos de mariposas,\\
rumorejando na epiderme das árvores\ldots{}

Gingam os bondes como um fogo de artifício,\\
sapateando nos trilhos,\\
cuspindo um orifício na treva cor de cal\ldots{}

Num perfume de heliotrópios e de poças\\
gira uma flor"-do"-mal\ldots{} Veio do Turquestã;\\
e traz olheiras que escurecem almas\ldots{}\\
Fundiu esterlinas entre as unhas roxas\\
nos oscilantes de Ribeirão Preto\ldots{}

-- Batat'assat'ô furnn!\ldots{}

Luzes do Cambuci pelas noites de crime!\ldots{}\\
Calor\ldots{} E as nuvens baixas muito grossas,\\
feitas de corpos de mariposas,\\
rumorejando na epiderme das árvores\ldots{}

Um mulato cor de ouro,\\
com uma cabeleira feita de alianças polidas\ldots{}\\
Violão! ``Quando eu morrer\ldots{}'' Um cheiro pesado de baunilhas\\
oscila, tomba e rola no chão\ldots{}\\
Ondula no ar a nostalgia das Baías\ldots{}

E os bondes passam como um fogo de artifício,\\
sapateando nos trilhos,\\
ferindo um orifício na treva cor de cal\ldots{}\\

-- Batat'assat'ô furnn!\ldots{}

Calor!\ldots{} Os diabos andam no ar\\
corpos de nuas carregando\ldots{}\\
As lassitudes dos sempres imprevistos!\\
e as almas acordando às mãos dos enlaçados!\\
Idílios sob os plátanos!\ldots{}\\
E o ciúme universal às fanfarras gloriosas\\
de saias cor"-de"-rosa e gravatas cor"-de"-rosa!\ldots{}

Balcões na cautela latejante, onde florem Iracemas\\
para os encontros dos guerreiros brancos\ldots{} Brancos?\\
E que os cães latam nos jardins!\\
Ninguém, ninguém, ninguém se importa!\\
Todos embarcam na Alameda dos Beijos da Aventura!\\
Mas eu\ldots{} Estas minhas grades em girândolas de jasmins,\\
enquanto as travessas do Cambuci nos livres\\
da liberdade dos lábios entreabertos!\ldots{}\\
Arlequinal! Arlequinal!\\
As nuvens baixas muito grossas,\\
feitas de corpos de mariposas,\\
rumorejando na epiderme das árvores\ldots{}\\
Mas sobre estas minhas grades em girândolas de jasmins,\\
o estelário delira em carnagens de luz,\\
e meu céu é todo um rojão de lágrimas!\ldots{}

E os bondes riscam como um fogo de artifício,\\
sapateando nos trilhos,\\
jorrando um orifício na treva cor de cal\ldots{}

-- Batat'assat'ô furnn!\ldots{}
\end{verse}

\chapter{Tu}

\begin{verse}
Morrente chama esgalga,\\
mais morta inda no espírito!\\
Espírito de fidalga,\\
que vive dum bocejo entre dois galanteios\\
e de longe em longe uma chávena da treva bem forte!

Mulher mais longa\\
que os pasmos alucinados\\
das torres de São Bento!\\
Mulher feita de asfalto e de lamas de várzea,\\
toda insultos nos olhos,\\
toda convites nessa boca louca de rubores!

Costureirinha de São Paulo,\\
ítalo"-franco"-luso"-brasílico"-saxônica,\\
gosto dos teus ardores crepusculares,\\
crepusculares e por isso mais ardentes,\\
bandeirantemente!

Lady Macbeth feita de névoa fina,\\
pura neblina da manhã!\\
Mulher que és minha madrasta e minha irmã!\\
Trituração ascencional dos meus sentidos!\\
Risco de aeroplano entre Moji e Paris!\\
Pura neblina da manhã!

Gosto dos teus desejos de crime turco\\
e das tuas ambições retorcidas como roubos!\\
Amo"-te de pesadelos taciturnos,\\
Materialização da Canaã do meu Poe!\\
Never more!

Emílio de Menezes insultou a memória do meu Poe\ldots{}

Oh! Incendiária dos meus aléns sonoros!\\
tu és o meu gato preto!\\
Tu te esmagaste nas paredes do meu sonho!\\
este sonho medonho!\ldots{}

E serás sempre, morrente chama esgalga,\\
meio fidalga, meio barregã,\\
as alucinações crucificantes\\
de todas as auroras de meu jardim!
\end{verse}

\chapter{Paisagem nº 3}

\begin{verse}
Chove?\\
Sorri uma garoa cor de cinza,\\
muito triste, como um tristemente longo\ldots{}\\
A casa Kosmos não tem impermeáveis em liquidação\ldots{}\\
Mas neste largo do Arouche\\
posso abrir meu guarda"-chuva paradoxal,\\
este lírico plátano de rendas mar\ldots{}

Ali em frente\ldots{} -- Mário, põe a máscara!\\
-- Tens razão, minha Loucura, tens razão.\\
O rei de Tule jogou a taça ao mar\ldots{}

Os homens passam encharcados\ldots{}\\
Os reflexos dos vultos curtos\\
mancham o petit"-pavé\ldots{}\\
As rolas da Normal\\
esvoaçam entre os dedos da garoa\ldots{}\\
(E se pusesse um verso de Crisfal\\
No \emph{De Profundis}?\ldots{})\\
De repente\\
um raio de Sol arisco\\
risca o chuvisco ao meio.
\end{verse}

\chapter[As Enfibraturas do Ipiranga]{As Enfibraturas do Ipiranga \subtitulo{(Oratório profano)}}


\epigraph{O, woe is me\\
To have seen what I have seen, see what I see!}{\textsc{shakespeare}}

\section*{Distribuição das vozes:}

\begingroup\parindent=0em
\textsc{Os Orientalismos Convencionais} -- (escritores e demais
artífices elogiáveis) -- Largo, imponente coro afinadíssimo de sopranos,
contraltos, barítonos, baixos.

\textsc{As Senectudes Tremulinas} -- (milionários e burgueses) -- Coro
de sopranistas.

\textsc{Os Sandapilários Indiferentes} -- (operariado, gente pobre) --
Barítonos e baixos.

\textsc{As Juvenilidades Auriverdes} -- (nós) -- Tenores, sempre
tenores! Que o diga Walter von Stolzing!

\textsc{Minha Loucura} -- Soprano ligeiro. Solista.

Acompanhamento de orquestra e banda.
\endgroup

\medskip

\begin{hangparas}{.15in}{1}
Local de execução: a esplanada do Teatro Municipal. Banda e orquestra
colocadas no terraplano que tomba sobre os jardins. São perto de cinco
mil instrumentistas dirigidos por maestros\ldots{} vindos do estrangeiro.
Quando a solista canta há silêncio orquestral -- salvo nos casos
propositadamente mencionados. E, mesmo assim, os instrumentos que então
ressoam, fazem"-no a contragosto dos maestros. Nos coros dos
\textsc{Orientalismos Convencionais} a banda junta"-se à orquestra. É um
\emph{tutti} formidando.

Quando cantam \textsc{As Juvenilidades Auriverdes} (há
naturalmente falta de ensaios) muitos instrumentos silenciam. Alguns
desafinam. Outros partem as cordas. Só aguentam o \emph{rubato}
lancinante violinos, flautas, clarins, a bateria e mais borés e maracás.

\textsc{Os Orientalismos Convencionais} estão nas janelas e terraços do
Teatro Municipal. \textsc{As Senectudes Tremulinas} disseminaram"-se
pelas sacadas do Automóvel Clube, da Prefeitura, da Rôtisserie, da
Tipografia Weisflog, do Hotel Carlton e mesmo da Livraria Alves, ao
longe. \textsc{Os Sandapilários Indiferentes} berram do Viaduto do Chá.
Mas \textsc{As Juvenilidades Auriverdes} estão embaixo, nos parques do
Anhangabaú, com os pés enterrados no solo. \textsc{Minha Loucura} no
meio delas.
\end{hangparas}

\section*{Na Aurora do Novo Dia}

\section*{prelúdio}

As caixas anunciam a arraiada. Todos os 550.000 cantores concertam
apressadamente as gargantas e tomam fôlego com exagero, enquanto os
borés, as trompas, o órgão, cada timbre por sua vez, entre largos
silêncios reflexivos, enunciam, sem desenvolvimento, nem harmonização o
tema: ``\emph{Utilius est saepe et securius quod homo non habeat multas
consolationes in hāc vitā}''.

E começa o oratório profano, que teve por nome

\section*{as enfibraturas do ipiranga}

\section*{As Juvenilidades Auriverdes}

\hfill{}(pianíssimo)

\begin{verse}
Nós somos as Juvenilidades Auriverdes!\\
As franjadas flâmulas das bananeiras,\\
as esmeraldas das araras,\\
os rubis dos colibris,\\
os lirismos dos sabiás e das jandaias,\\
os abacaxis, as mangas, os cajus\\
almejam localizar"-se triunfantemente,\\
na fremente celebração do Universal!\ldots{}\\
Nós somos as Juvenilidades Auriverdes!\\
As forças vivas do torrão natal,\\
as ignorâncias iluminadas,\\
os novos sóis luscofuscolares\\
entre os sublimes das dedicações!\ldots{}\\
Todos para a fraterna música do Universal!\\
Nós somos as Juvenilidades Auriverdes!
\end{verse}

\section*{Os Sandapilários Indiferentes}

\hfill{}(num estampido preto)

\begin{verse}
Vá de rumor! Vá de rumor!\\
Esta gente não nos deixa mais dormir!\\
Antes \emph{E lucevan le stelle} de Puccini!\\
Oh! pé de anjo, pé de anjo!\\
Fora! Fora o que é de despertar!

\hfill{}(A orquestra num crescendo cromático de contrabaixos anuncia\ldots{})
\end{verse}

\section*{Os Orientalismos Convencionais}

\begin{verse}
Somos os Orientalismos Convencionais!\\
Os alicerces não devem cair mais!\\
Nada de subidas ou de verticais!\\
Amamos as chatezas horizontais!\\
Abatemos perobas de ramos desiguais!\\
Odiamos as matinadas arlequinais!\\
Viva a Limpeza Pública e os hábitos morais!\\
Somos os Orientalismos Convencionais!

Deve haver Von Iherings para todos os tatus!\\
Deve haver Vitais Brasis para os urutus!\\
Mesmo peso de feijão em todos os tutus!\\
Só é nobre o passo dos jabirus!\\
Há estilos consagrados para os Pacaembus!\\
Que os nossos antepassados foram homens de truz!\\
Não lhe bastam velas? Para que mais luz!

Temos nossos coros só no tom de dó!\\
Para os desafinados, doutrina de cipó!\\
Usamos capas de seda, é só escovar o pó!\\
Diariamente à mesa temos mocotó!\\
Per omnia saecula saeculorum moinhos terão mó!\\
Anualmente de sobrecasaca, não de paletó,\\
vamos visitar o esqueleto de nossa grande Avó!\\
Glória aos Iguais! Um é todos! Todos são um só!\\
Somos os Orientalismos Convencionais!
\end{verse}

\section*{As Juvenilidades Auriverdes}

\begin{flushright}
(perturbadas com o fabordão,\\
recomeçam mais alto, incertas)
\end{flushright}

\begin{verse}
Magia das alvoradas entre magnólias e rosas\ldots{}\\
Apelos do estelário visível aos alguéns\ldots{}\\
-- Pão de Ícaros sobre a toalha estática do azul!\\
Os tuins esperanças das nossas ilusões!\\
Suaviloquências entre as deliquescências\\
dos sáfaros, aos raios do maior solar!\ldots{}\\
Sobracemos as muralhas! Investe com os cardos!\\
Rasga"-te nos acúleos! Tomba sobre o chão!\\
Hão"-de vir valquírias para os olhos"-fechados!\\
Anda! Não pares nunca! Aliena o duvidar\\
e as vacilações perpetuamente!
\end{verse}

\section*{As Senectudes Tremulinas}

\begin{flushright}
(tempo de minuete)
\end{flushright}

\begin{verse}
Quem são estes homens?\\
Maiores menores\\
Como é bom ser rico!\\
Maiores menores\\
Das nossas poltronas\\
Maiores menores\\
olhamos as estátuas\\
Maiores menores\\
do signor Ximenes\\
-- o grande escultor!

Só admiramos os célebres\\
e os recomendados também!\\
Quem tem galeria\\
terá um Bouguereau!\\
Assinar o Lírico?\\
Elegância de preceito!\\
Mas que paulificância\\
Maiores menores\\
o \emph{Tristão e Isolda}!\\
Maiores menores

Preferimos os coros\\
dos Orientalis"-\\
mos Convencionais!\\
Depois os sanchismos\\
(Ai! gentes, que bom!)\\
da alta madrugada\\
no largo do Paiçandu!

Alargar as ruas\ldots{}\\
E as Instituições?\\
Não pode! Não pode!

Maiores menores\\
Mas não há quem diga\\
Maiores menores\\
quem são esses homens\\
que cantam do chão?

\hfill{}(a orquestra súbito emudece, depois duma grande gargalhada de timbales)
\end{verse}

\section*{Minha Loucura}

\begin{flushright}
(recitativo e balada)
\end{flushright}

\begin{verse}
Dramas da luz do luar no segredo das frestas\\
perquirindo as escuridões\ldots{}\\
A traição das mordaças!\\
E a paixão oriental dissolvida no mel!\ldots{}

Estas marés da espuma branca\\
e a onipotência intransponível dos rochedos!\\
Intransponivelmente! Oh!\ldots{}\\
A minha voz tem dedos muito claros\\
que vão roçar nos lábios do Senhor;\\
mas as minhas tranças muito negras\\
emaranharam"-se nas raízes do jacarandá\ldots{}

Os cérebros das cascatas marulhantes\\
e o benefício das manhãs serenas do Brasil!

\hfill{}(grandes glissandos de harpa)

Estas nuvens da tempestade branca\\
e os telhados que não deixam a chuva batizar!\\
Propositadamente! Oh!\ldots{}\\
Os meus olhos têm beijos muito verdes\\
que vão cair às plantas do Senhor;\\
mas as minhas mãos muito frágeis\\
apoiaram"-se nas faldas do Cubatão\ldots{}

Os cérebros das cascatas marulhantes\\
e o benefício das manhãs solenes do Brasil

\hfill{}(notas longas de trompas)

Estas espigas da colheita branca\\
e os escalrachos roubando a uberdade!\\
Enredadamente! Oh!\ldots{}\\
Os meus joelhos têm quedas muito crentes\\
que vão bater no peito do Senhor;\\
mas os meus suspiros muito louros\\
entreteceram"-se com a rama dos cafezais\ldots{}

Os cérebros das cascatas marulhantes\\
e o benefício das manhãs gloriosas do Brasil!

\hfill{}(harpas, trompas, órgão)
\end{verse}

\section*{As Senectudes Tremulinas}

\hfill{}(iniciando uma gavota)

\begin{verse}
Quem é essa mulher!\\
É louca, mas louca\\
pois anda no chão!
\end{verse}

\section*{As Juvenilidades Auriverdes}

\hfill{}(num crescendo fantástico)

\begin{verse}
Ódios, invejas, infelicidades!\ldots{}\\
Crenças sem Deus! Patriotismos diplomáticos!\\
Cegar!\\
Desvalorização das lágrimas lustrais!\\
Nós não queremos mascaradas! E ainda menos\\
cordões \emph{Flor"-do"-abacate} das superfluidades!\\
Os tumultos da luz!\ldots{} As lições dos maiores!\ldots{}\\
E a integralização da vida no Universal!\\
As estradas correndo todas para o mesmo final!\ldots{}\\
E a pátria simples, una, intangivelmente\\
partindo para a celebração do Universal!\\
Ventem nossos desvarios fervorosos!\\
Fulgurem nossos pensamentos dadivosos!\\
Clangorem nossas palavras proféticas\\
na grande profecia virginal!\\
Somos as Juvenilidades Auriverdes!\\
A passiflora! o espanto! a loucura! o desejo!\\
Cravos! mais cravos para nossa cruz!
\end{verse}

\section*{Os Orientalismos Convencionais}

\hfill{}(\emph{Tutti.} O crescendo é resolvido

\hfill{}numa solene marcha fúnebre)

\begin{verse}
Para que cravos? Para que cruzes?\\
Submetei"-vos à metrificação!\\
A verdadeira luz está nas corporações!\\
Aos maiores: serrote; aos menores: o salto\ldots{}\\
E a glorificação das nossas ovações!

{[}\ldots{}{]}
\end{verse}

\section*{As Juvenilidades Auriverdes}

\hfill{}(loucos, sublimes, tombando exaustos)

\begin{verse}
Seus\dotfill{}!!!\\
(A maior palavra feia que o leitor conhecer)\\
Nós somos as Juvenilidades Auriverdes!\\
A passiflora! o espanto!\ldots{} a loucura! o desejo!\ldots{}\\
Cravos!\ldots{} Mais cravos\ldots{} para\ldots{} a nossa\ldots{}
\end{verse}

Silêncio. Os \textsc{Orientalismos Convencionais}, bem como as
\textsc{Senectudes Tremulinas} e os \textsc{Sandapilários Indiferentes}
fugiram e se esconderam, tapando os ouvidos à grande, à máxima \textsc{verdade}.
A orquestra evaporou"-se, espavorida. Os \emph{maestri} sucumbiram. Caiu
a noite, aliás; e na solidão da noite das mil estrelas as
\textsc{Juvenilidades Auriverdes}, tombadas no solo, chorando, chorando
o arrependimento do tresvario final.

{[}\ldots{}{]}

\partepigraph{\emph{para Anita Malfatti}}{}
\part[Losango cáqui]{\textsc{losango cáqui}\\ou\\\textsc{afetos militares de mistura com\\ os porquês de eu saber alemão}\\(1926)}
\removeepigraph

\chapter[\textsc{i}\ \ \ \ ``Meu coração estrala'']{I}

\begin{verse}
Meu coração estrala.\\
Esse lugar"-comum inesperado: Amor.

\quad\quad\quad{}Na trajetória rápida do bonde\ldots{}\\
\quad\quad\quad\quad{}De Sant'Ana à cidade.\\
\quad\quad\quad\quad\quad{}Da Terra à Lua\\
\quad\quad\quad\quad\quad\quad{}Júlio Verne\\
\quad\quad\quad\quad\quad{}Atravessei o núcleo dum cometa?\\
\quad\quad\quad\quad{}Me sinto vestido de luzes estranhas\\
\quad\quad\quad\quad{}E da inquietação fulgurante da felicidade.

Aqueles olhos matinais sem nuvens\ldots{}\\
Meu coração estrala.

No entanto dia intenso apertado.\\
\quad\quad\quad\quad{}Fui buscar minha farda.\\
\quad\quad\quad\quad{}Choveu.\\
\quad\quad\quad\quad{}Visita espanto\\
\quad\quad\quad\quad{}Discussões estéticas.\\
\quad\quad\quad\quad{}Automóvel confidencial.\\
\quad\quad\quad\quad{}Os cariocas perderam o matche.\\
\quad\quad\quad\quad{}Eta paulistas!

Mas aqueles olhos matinais sem nuvens\ldots{}\\
Meu refrão!

E penso nela, unicamente penso em mim.\\
Amo todos os amores de S. Paulo\ldots{} do Brasil.\\
Eu sou a Fama de cem bocas\\
Pra beijar todas as mulheres do mundo!\\
Hoje é Suburra nos meus braços abraços frementes amor!\\
Minha Loucura, acalma"-te.\\
\ldots{} Muitos dias de exercícios militares\ldots{}\\
\quad\quad\quad\quad{}Previsões tenebrosas\ldots{}\\
\quad\quad\quad\quad\quad\quad{}Revoluções futuras\ldots{}\\
Perspectiva de escravo cáqui, pardacento, fardacento\ldots{}\\

Meu coração estrala.\\
Amor!\ldots{}
\end{verse}

\chapter[\textsc{iv}\ \ ``Soldado"-raso da República'']{\textsc{iv}}

\begin{verse}
Soldado"-raso da República.\\
Quarto Batalhão de Caçadores aquartelado em Sant'Ana.\\
\hfill{}Rogai por nós!\\
\qquad\qquad\qquad\qquad\qquad\qquad\qquad{}Valha"-me Deus!\\
\quad\qquad\qquad{}Todo vibro de ignorâncias militares.\\
\qquad\quad{}\ldots{} O calcanhar direito se levanta,\\
\qquad\quad{}Corpo inclinado pra frente\ldots{}

A marcha rompe.

\qquad\quad\quad{}Marcha, soldado,\\
\qquad\quad\quad{}Cabeça de papel,\\
\qquad\quad\quad{}Soldado relaxado\\
\qquad\quad\quad{}Vai preso pro quartel\ldots{}
\end{verse}

\chapter[\textsc{ix}\ \ ``Careço de marchar cabeça levantada'']{\textsc{ix}}

\begin{verse}
Careço de marchar cabeça levantada\\
Olhar altivo pra frente\ldots{}

Mas eu queria olhar à esquerda\ldots{}\\
\qquad\qquad\qquad\qquad{}Bonita casa colonial\\
\qquad\qquad\qquad\qquad{}Cheinha mesmo de paisagem!

``-- Olhar altivo pra frente!''

O meu tenente\\
Não aprecia as casas coloniais.

Porém o meu olhar blefa o tenente.\\
Olhou altivo pra frente\\
E batendo no quepe do soldado da frente\\
Fez esquerda"-volver\\
E meigamente espiou a casa colonial.
\end{verse}

\chapter[\textsc{xxi}\ \ \textsc{A menina e a cantiga}]{\textsc{xxi}\\\textsc{A menina e a cantiga}}


\begin{verse}
\ldots{} trarilarára\ldots{} traríla\ldots{}

A meninota esganiçada magriça com a saia voejando\\
por cima dos joelhos em nó vinha meia dançando cantando no crepúsculo escuro. Batia compasso com a varinha na poeira da calçada.

\ldots{} trarilarára\ldots{} traríla\ldots{}

De repente voltou"-se pra negra velha que vinha trôpega atrás, enorme trouxa de roupas na cabeça:\\
-- Qué mi dá, vó?\\
-- Naão.

\ldots{} trarilarára\ldots{} traríla\ldots{}
\end{verse}

\chapter[\textsc{xxxi}\ \ \textsc{Cabo Machado}]{\textsc{xxxi}\\\textsc{Cabo Machado}}

\begin{verse}
Cabo Machado é cor"-de"-jambo,\\
Pequeninho que nem todo brasileiro que se preza.\\
Cabo Machado é moço bem bonito.\\
É como se a madrugada andasse na minha frente.\\
Entreabre a boca encarnada num sorriso perpétuo\\
Adonde alumia o sol de ouro dos dentes\\
Obturados com um luxo oriental.

Cabo Machado marchando\\
É muito pouco marcial.\\
Cabo Machado é dançarino, sincopado,\\
Marcha vem"-cá"-mulata.\\
Cabo Machado traz a cabeça levantada\\
Olhar dengoso pros lados.\\
Segue todo rico de joias olhares quebrados\\
Que se enrabicharam pelo posto dele\\
E pela cor"-de"-jambo.

Cabo Machado é delicado, gentil.\\
Educação francesa mesureira.\\
Cabo Machado é doce que nem mel\\
E polido que nem manga"-rosa.\\
Cabo Machado é bem o representante duma terra\\
Cuja Constituição proíbe as guerras de conquista\\
E recomenda cuidadosamente o arbitramento.\\
Só não bulam com ele!\\
Mais amor menos confiança!\\
Cabo Machado toma um jeito de rasteira\ldots{}

Mas traz unhas bem tratadas\\
Mãos transparentes frias,\\
Não rejeita o bom"-tom do pó"-de"-arroz.\\
Se vê bem que prefere o arbitramento.\\
E tudo acaba em dança!\\
Por isso cabo Machado anda maxixe.

Cabo Machado\ldots{} bandeira nacional!
\end{verse}

\chapter[\textsc{xxxiii}\ \ ``Meu gozo profundo ante a manhã sol'']{\textsc{xxxiii}}

\epigraph{Prazeres e dores prendem a alma no corpo como com um prego. Tornam"-a
corporal\ldots{} Consequentemente é impossível a ela chegar pura nos
Infernos.}{\textsc{platão}}

\begin{verse}
Meu gozo profundo ante a manhã sol\\
\qquad\qquad\qquad\qquad{}a vida carnaval\ldots{}\\
\qquad\qquad\qquad\qquad{}Amigos\\
\qquad\qquad\qquad{}Amores\\
\qquad\qquad{}Risadas\\
Os piás imigrantes me rodeiam pedindo retratinhos de artistas de cinema,
desses que vêm nos maços de cigarros.

Me sinto a Assunção de Murilo!

Já estou livre da dor\ldots{}\\
Mas todo vibro da alegria de viver.

\quad{}Eis porque minha alma inda é impura.
\end{verse}

\chapter[\textsc{xxxvii}\ \ ``Te gozo!\ldots{}'']{\textsc{xxxvii}}

\begin{verse}
Te gozo!\ldots{}\\
E bem humanamente, rapazmente.

Mas agora esta insistência em fazer versos sobre ti\ldots{}
\end{verse}

\chapter[\textsc{xxxix}\ \ \textsc{Parada}]{\textsc{xxxix}\\\textsc{Parada} \subtitulo{(7 de setembro de 1922)}}

\begin{verse}
``-- Colunas de pelotões por quatro!''

\textsc{o desfile principia.}

O refle rombudo da soldadesca marchando\\
Mansamente se embainha na Avenida.

``-- Olhe a conversão!''\\
\qquad\quad Conversão de S. Paulo\ldots{}\\
Todos convergem pra esquerda.\\
Lá está Bilac estreando a fatiota de bronze.\\
\qquad\quad Pátria latejo em ti\ldots{}\\
Meu Brasilzinho do coração!\\
A alma da gente drapeja no espaço cinzento.\\
Os mil milhões de rosas paulistanas.

Moça bonita!\\
Muitas moças.\\
Conhecidos.\\
``-- Troque o passo!''\\
Gi, Taco, Maria, que lindos os três!\\
Máquinas cinematográficas.\\
\qquad\qquad\qquad My Boy.\\
\qquad\qquad Não posso me rir.\\
Olhar altivo pra frente\ldots{}

Na minha frente\\
O cabo mais descabido deste mundo.\\
Rua Augusta curiosa.\\
Todas as ruas transversais espiando curiosas\\
Trepadas em trincheiras de automóveis.\\
Sorveteiro.\\
Moça bonita!\\
Palmas.\\
Grade dos escoteiros perfilados.\\
Cunhãs, velhas corocas debruçadas\ldots{}\\
\qquad\quad\qquad\quad\qquad\quad\qquad\quad\qquad\quad Brutas!\\
No parapeito das cabeças infantis.\\
As famílias dos mitras nos castelos roqueiros\\
Apresentam armas em negligé.\\
\qquad\quad\qquad\qquad Zero uniforme.

Este cabo caminha em contratempo,\\
\qquad\qquad\qquad\quad Cinco por quatro,\\
\qquad\qquad Tal e qual Boieldieu na \emph{Dama branca}\\
\qquad ``Viens, gentille dame''\ldots{}\\
\quad\emph{Zortzico} de Albeniz\ldots{}\\
Esculhamba toda a marcha!

Moça bonita!\\
``-- Olhe o Mário de Andrade!''\\
Se enganou, moça.\\
\qquad\qquad\qquad Onde estarei?\\
\qquad\qquad\qquad Ela não veio com certeza\ldots{}\\
\qquad\qquad\qquad Que bem me importa!\\
\qquad\qquad\qquad Saiba a cidade de S. Paulo\\
\qquad\qquad\qquad Que nela vive um homem feliz!\\
``-- Olhe a cadência!''\\
\qquad\qquad\textsc{o trianon vai passar}\\
Palmas.\\
O tenente gesticula com a espada\\
E todos olham pra direita em continência.\\
Músicas.\\
Ovação.\\
Trinta carinhas adoráveis.\\
Esta família sorocaba\ldots{}\\
Tudo procissiona em meus olhos um"-dois\ldots{}\\
\qquad\qquad\qquad\qquad\qquad\quad Árvores,\\
\qquad\qquad\qquad\qquad\qquad\quad O preto,\\
\qquad\qquad\qquad\qquad\qquad\quad Beiço vermelho tapa o resto.\\
\qquad\qquad\qquad\qquad\qquad\quad Moça bonita!\\
\qquad\qquad\qquad\qquad\qquad\quad Músicas.\\
\qquad\qquad\qquad\qquad\qquad\quad Cornetas.\\
\qquad\qquad\qquad\qquad\quad Cornacas.\\
\qquad\qquad\qquad\quad Bengalós.\\
\qquad No alto dum palanquim\\
\qquad Sua Excia. o marajá de Khajurao.\\
O sr. presidente do Estado não gosta de Modernismo\ldots{}

Olha pra mim!\\
``-- Fora de forma!\\
Quarenta dias de prisão!\ldots{}''\\
\qquad\qquad\qquad\quad Oh, minhas alucinações!\\
Moça bonita!\\
Palmas.\\
Passou o palanquim.\\
Serenamente continuou sua jornada\\
Sua Excia. o marajá de Khajurao.\\
E os diademas de pérolas luzentes\\
Nos risos das favoritas.\\
Toneladas de moças bonitas!\\
\qquad\quad ``-- Viva o Brasil!''\\
\qquad\qquad\quad ``-- Viva o Quarto Batalhão de Caçadores!''\\
Risos.\\
Sorveteiro"-sorveteiro.\\
Acerte o passo, cabo!\\
Um senhor três filhas gordas,\\
\qquad\qquad\qquad\quad Colares falsos,\\
\qquad\qquad\qquad Terra"-roxa,\\
\qquad\qquad\quad Guaratinguetá,\\
\qquad\qquad Tabatinguera,\\
\qquad\quad Oblivion!\\
\qquad Oblivion\ldots{}

Está acabando a preocupação.\\
Braço dói.\
A Avenida escampou.\\
Não tem mais moça bonita.\\
Quedê as palmas?\\
Não existo.\\
Não marcho.\\
Muito longe\\
Nos cafundós penumbristas de Santo Amaro\\
O vácuo badalando badalando\ldots{}\\
Eco dentro de mim.\\
Não tem mais Independência do Brasil.\\
Olhos defuntos.\\
Ninguém.\\
Nada.\\
Pra que tanto tambor?\\
O braço nem dói mais.\\
Cheiros de almoços mayonnaises.\\
Sol crestado nas nuvens que nem PÃO.\\
\qquad\qquad Kennst du das Land\\
\qquad\qquad Wo die Zitronen blühen?\ldots{}\\
\qquad\qquad\qquad\qquad  Assombrações desaparecidas.\\
\qquad\qquad\qquad\qquad\qquad  O mundo não existe.\\
\qquad\qquad\qquad\qquad\qquad\qquad Não existo.\\
\qquad\qquad\qquad\qquad\qquad\qquad Não sou.

\qquad\qquad\qquad\quad\textsc{ciclização}

\qquad\qquad\qquad\qquad\qquad Alô?\ldots{}\\
Dava dez milréis por um copo de leite.
\end{verse}

\part{\textsc{clã do jabuti}}

\chapter[Carnaval carioca]{\textsc{carnaval carioca} \subtitulo{(1923)}}

\hfill\emph{a Manuel Bandeira}

\begin{verse}
A fornalha estrala em mascarados cheiros silvos\\
Bulhas de cor bruta aos trambolhões,\\
Cetins sedas cassas fundidas no riso febril\ldots{}\\
Brasil!\\
Rio de Janeiro!\\
Queimadas de verão!\\
E ao longe, do tição do Corcovado a fumarada das nuvens pelo céu.

Carnaval\ldots{}\\
Minha frieza de paulista,\\
Policiamentos interiores,\\
Temores da exceção\ldots{}\\
E o excesso goitacá pardo selvagem!\\
Cafrarias desabaladas\\
Ruínas de linhas puras\\
Um negro dois brancos três mulatos, despudores\ldots{}\\
O animal desembesta aos botes pinotes desengonços\\
No heroísmo do prazer sem máscaras supremo natural.

Tremi de frio nos meus preconceitos eruditos\\
Ante o sangue ardendo povo chiba frêmito e clangor.\\
Risadas e danças\\
Batuques maxixes\\
Jeitos de micos piricicas\\
Ditos pesados, graça popular\ldots{}\\
Ris? Todos riem\ldots{}

O indivíduo é caixeiro de armarinho na Gamboa.\\
Cama de ferro curta por demais,\\
Espelho mentiroso de mascate\\
E no cabide roupas lustrosas demais\ldots{}\\
Dança uma joça repinicada\\
De gestos pinchando ridículos no ar.\\
Corpo gordo que nem de matrona\\
Rebolando embolado nas saias baianas,\\
Braço de fora, pelanca pulando no espaço\\
E no decote cabeludo cascavéis saracoteando\\
Desritmando a forçura dos músculos viris.\\
Fantasiou"-se de baiana,\\
\qquad\qquad\qquad\qquad\quad A Baía é boa terra\ldots{}\\
\qquad\qquad\qquad\qquad\qquad\qquad\qquad\qquad\quad Está feliz.

Entoa à toa a toada safada\\
E no escuro da boca banguela\\
O halo dos beiços de carmim.\\
Vibrações em redor.\\
Pinhos gargalhadas assobios\\
Mulatos remelexos e boduns.\\
Palmas. Pandeiros. -- Aí, baiana!\\
\qquad\qquad\qquad\qquad\quad Baiana do coração!\\
Serpentinas que saltam dos autos em monóculos curiosos,\\
Este cachorro espavorido,\\
Guarda"-civil indiferente.\\
Fiscalizemos as piruetas\ldots{}\\
Então só eu que vi?\\
Risos. Tudo aplaude. Tudo canta:\\
\quad\quad{}-- Aí, baiana faceira,\\
\quad\quad{}Baiana do coração!\\
Ele tinha nos beiços sonoros beijando se rindo\\
Uma ruga esquecida uma ruga longínqua\\
Como esgar duma angústia indistinta ignorante\ldots{}\\
Só eu pude gozá"-la.\\
E talvez a cama de ferro curta por demais\ldots{}

Carnaval\ldots{}\\
A baiana se foi na religião do Carnaval\\
Como quem cumpre uma promessa.\\
Todos cumprem suas promessas de gozar.\\
Explodem roncos roucos trilos tchique"-tchiques\\
E o falsete enguia esguia rabejando pelo aquário multicor.\\
Cordões de machos mulherizados,\\
Ingleses evadidos da pruderie,\\
Argentinos mascarando a admiração com desdéns superiores\\
Degringolando em lenga"-lenga de milonga,\\
Polacas de indiscutível índole nagô,\\
Yankees fantasiados de norte"-americanos\ldots{}\\
Coiozada emproada se aturdindo turtuveando\\
Entre os carnavalescos de verdade\\
Que pererecam pararacas em derengues meneios cantigas, chinfrim de gozar!

Tem outra raça ainda.\\
O mocinho vai fuçando o manacá naturalizado espanhola.\\
Ela se deixa bolinar na multidão compacta.\\
\qquad\qquad\qquad\qquad\qquad\qquad\quad Por engano.\\
Quando aproximam dos polícias\\
Como ela é pura conversando com as amigas!\\
Pobre do moço olhando as fantasias dos outros,\\
Pobre do solitário com chapéu caicai nos olhos!\\
Naturalmente é um poeta\ldots{}

Eu mesmo\ldots{} Eu mesmo, Carnaval\ldots{}\\
Eu te levava uns olhos novos\\
Pra serem lapidados em mil sensações bonitas,\\
Meus lábios murmurejando de comoção assustada\\
Haviam de ter puríssimo destino\ldots{}\\
É que sou poeta\\
E na banalidade larga dos meus cantos\\
Fundir"-se"-ão de mãos dadas alegrias e tristuras, bens e males,\\
Todas as coisas finitas\\
Em rondas aladas sobrenaturais.

Ânsia heroica dos meus sentidos\\
Pra acordar o segredo de seres e coisas.\\
Eu colho nos dedos as rédeas que param o infrene das vidas,\\
Sou o compasso que une todos os compassos,\\
E com a magia dos meus versos\\
Criando ambientes longínquos e piedosos\\
Transporto em realidades superiores\\
A mesquinhez da realidade.\\
Eu bailo em poemas, multicolorido!\\
Palhaço! Mago! Louco! Juiz! Criancinha!\\
Sou dançarino brasileiro!\\
Sou dançarino e danço! E nos meus passos conscientes\\
Glorifico a verdade das coisas existentes\\
Fixando os ecos e as miragens.\\
Sou um tupi tangendo um alaúde\\
E a trágica mixórdia dos fenômenos terrestres\\
Eu celestizo em euritmias soberanas,\\
Ôh encantamento da Poesia imortal!\ldots{}\\
Onde que andou minha missão de poeta, Carnaval?\\
Puxou"-me a ventania,\\
Segundo círculo do Inferno,\\
Rajadas de confetes\\
Hálitos diabólicos perfumes\\
Fazendo relar pelo corpo da gente\\
Semíramis Marília Helena Cleópatra e Francesca.\\
Milhares de Julietas!\\
Domitilas fantasiadas de cow"-girls,\\
Isoldas de pijama bem francesas,\\
Alzacianas portuguesas holandesas\ldots{}\\
\qquad\qquad\qquad\qquad\qquad\qquad\quad Geografia!\\
Êh liberdade! Pagodeira grossa! É bom gozar!\\
Levou a breca o destino do poeta,\\
Barreei meus lábios com o carmim doce dos dela\ldots{}

Teu amor provinha de desejos irritados,\\
Irritados como os morros do nascente nas primeiras horas da manhã.\\
Teu beijo era como o grito da araponga,\\
Me alumiava atordoava com o golpe estridente viril.\\
Teu abraço era como a noite dormida na rede\\
Que traz o dia de membros moles mornos de torpor.\\
Te possuindo eu me alimentei com o mel dos guarupus,\\
Mel ácido, mel que não sacia,\\
Mel que dá sede quando as fontes estão muitas léguas além,\\
Quando a soalheira é mais desoladora\\
E o corpo mais exausto.

Carnaval\ldots{}\\
Porém nunca tive intenção de escrever sobre ti\ldots{}\\
Morreu o poeta e um gramofone escravo\\
Arranhou discos de sensações\ldots{}
\end{verse}

\medskip
\section*{I}

\begin{verse}
Embaixo do Hotel Avenida em 1923\\
Na mais pujante civilização do Brasil\\
Os negros sambando em cadência.\\
Tão sublime, tão áfrica!\\
A mais moça bulcão polido ondulações lentas lentamente\\
Com as arrecadas chispando raios glaucos ouro na luz peluda de pó.\\
Só as ancas ventre dissolvendo"-se em vaivéns de ondas em cio.\\
Termina se benzendo religiosa talqualmente num ritual.

E o bombo gargalhante de tostões\\
Sincopa a graça da danada.
\end{verse}

\medskip
\section*{II}

\begin{verse}
Na capota franjada com xale chinês\\
Amor curumim abre as asas de ruim papelão.\\
Amor abandonou as setas sem prestígio\\
E se agarra na cinta fecunda da mãe.\\
Vênus Vitoriosa emerge de ondas crespas serpentinas,\\
De ondas encapeladas por mexicanos e marqueses cavalgando autos perseguidores.\\
-- Quero ir pra casa, mamãe!\\

Amor com medo dos desejos\ldots{}
\end{verse}

\medskip
\section*{III}

\begin{verse}
O casal jovem rompendo a multidão.\\
O bando de mascarados de supetão em bofetadas de confetes na mulher.\\
-- Olhe só a boquinha dela!\\
-- Ria um pouco, beleza!\\
-- Come do meu!\\
O marido esperou (com paciência) que a esposa se desvencilhasse do bando de máscaras\\
E lá foram rompendo a multidão.\\
Ela apertava femininamente contra o seio o braço protetor do\\
Esposo.\\
Do esposo recebido ante a imponência catedrática da Lei\\
E as bênçãos invisíveis -- extraviadas? -- do Senhor\ldots{}

Meu Deus\ldots{}\\
Onde que jazem tuas atrações?\\
Pra que lados de fora da Terra\\
Fugiu a paz das naves religiosas\\
E a calma boa de rezar ao pé da cruz?\\
Reboa o batuque.\\
São priscos risadas\\
São almas farristas\\
Aos pinchos e guinchos\\
Cambeteando na noite estival.\\
Pierrots"-fêmeas em calções mais estreitos que as pernas,\\
\qquad{}Gambiarras iluminadas!\\
Oblatas de confetes no ar,\\
Incenso e mirra marca Rodo nacional\\
Açulam raivas de gozar.

O cabra enverga fraque de cetim verde no esqueleto.\\
Magro magro asceta de longos jejuns dificílimos.\\
Jantou gafanhotos.\\
E gesticula fala canta.\\
Prédicas de meu Senhor\ldots{}\\
Será que vai enumerar teus pecados e anátemas justos?\\
A boca vai florir em bênçãos e perdões\ldots{}

Porém de que lados de fora da Terra\\
Falam agora as tuas prédicas?\\
Quedê teus padres?\\
Quedê teus arcebispos purpurinos?\\
Quedele o tempo\\
Sem fraque de cetim verde no esqueleto\\
Agarrava a contar as parábolas lindas\\
De que os padres não se lembram mais?\\
Por onde pregam os Sumés de meu Senhor?\\
Aqueles a quem deixaste a tua Escola\\
Fingem ignorar que gostamos de parábolas lindas,\\
E todos nos pusemos sapeando histórias de pecado\\
Porque não tinha mais histórias pra escutar\ldots{}

Senhor! Deus bom, Deus grande sobre a terra e sobre o mar,\\
Grande sobre a alegria e o esquecimento humano,\\
Vem de novo em nosso rancho, Senhor!\\
Tu que inventaste as asas alvinhas dos anjos\\
E a figura batuta de Satanás;\\
Tu, tão humilde e imaginoso\\
Que permitiste Isis guampuda nos templos do Nilo,\\
Que indicaste a bandeira triunfal de Dionísio pros gregos\\
E empinaste Tupã sobre os Andes da América\ldots{}

Aleluia!\\
Louvemos o Criador com os sons dos saxofones arrastados,\\
Louvemo"-Lo com os salpicos dos xilofones nítidos!\\
Louvemos o Senhor com os riscos dos recorrecos e os estouros do tam"-tam,\\
Louvemo"-Lo com a instrumentarada crespa do jazz"-band!\\
Louvemo"-Lo com os violões de cordas de tripa e as cordeonas imigrantes,\\
Louvemo"-Lo com as flautas dos choros mulatos e os cavaquinhos das serestas ambulantes!\\
Louvemos O que permanece através das festanças virtuosas e dos gozos ilegítimos!\\
Louvemo"-Lo sempre e sobre tudo! Louvemo"-Lo com todos os instrumentos e todos os
ritmos!\ldots{}

Vem de novo em nosso rancho, Senhor!\\
Descobrirei no colo dengoso da Serra do Mar\\
Um derrame no verde mais claro do vale,\\
Arrebanharei os cordões do carnaval\\
E pros carlitos marinheiros gigoletes e arlequins\\
Tu contarás de novo com tua voz que é ver o leite\\
Essas histórias passadas cheias de bons samaritanos,\\
Dessas histórias cotubas em que Madalena atapetava com os cabelos o teu chão\ldots{}

\ldots{} pacapacapacapão!\ldots{} pacapão! pão! pão!\ldots{}

Pão e circo!\\
Roma imperial se escarrapacha no anfiteatro da Avenida.\\
Os bandos passam coloridos,\\
Gesticulam virgens,\\
Semivirgens,\\
Virgens em todas as frações\\
Num desespero de gozar.

Homens soltos\\
Mulheres soltas\\
Mais duas virgens fuxicando o almofadinha\\
Maridos camaradas\\
Mães urbanas\\
Meninos\\
Meninas\\
Meninos\\
O de dois anos dormindo no colo da mãe\ldots{}\\
-- Não me aperte!\\
\qquad\qquad -- Desculpe, Madama!\\
Falsetes em desarmonia\\
Coros luzes serpentinas serpentinas\\
Coriscos coros caras colos braços serpentinas serpentinas\\
Matusalém cirandas Breughel\\
\qquad\qquad\qquad\qquad\qquad -- Diacho!\\
Sambas bumbos guizos serpentinas serpentinas\ldots{}\\
E a multidão compacta se aglomera aglutina mastiga em aproveitamentos brincadeiras asfixias desejadas delírios sardinhas desmaios\\
Serpentinas serpentinas coros luzes sons\\
E sons!\\

\quad\quad\quad\quad\quad{}\textsc{yayá, fruta"-do"-conde},\\
\quad\quad\quad\quad\quad{}\textsc{castanha"-do"-pará!}\ldots{}

\quad\quad\quad\quad\quad\quad\quad{}Yayá, fruta"-do"-conde,\\
\quad\quad\quad\quad\quad\quad\quad{}Castanha"-do"-Pará!\ldots{}

O préstito passando.\\
Bandos de clarins em cavalos fogosos.\\
Utiaritis aritis assoprando cornetas sagradas.\\
Fanfarras fanfarrãs\\
\qquad\qquad\qquad{}fenferrens\\
\qquad\qquad\qquad\qquad{}finfirrins\ldots{}\\
\qquad\qquad\qquad\qquad\qquad{}Forrobodó de cuia!\\
Vitória sobre a civilização! Que civilização?\ldots{} É Baco!

É Baco num carro feito de ouro e de mulheres\\
E dez parelhas de bestas imorais.\\
Tudo aplaude guinchos berros,\\
E sobre o Etna de loucuras e pólvoras\\
Os Tenentes do Diabo.\\
Alegorias, críticas, paródias\\
Palácios bestas do fundo do mar,\\
Os aluguéis se elevam\ldots{}\\
\quad{}Os senhorios exigentes\ldots{}\\
\quad\quad{}Cães! infames! malditos!\ldots{}

\ldots{} Eu enxerguei com estes meus olhos que inda a terra há"-de comer\\
Anteontem as duas mulheres se fantasiando de lágrimas.\\
A mais nova amamentava o esqueletinho.\\
Quatro barrigudinhos sem infância,\\
Os trastes sem conchego\\
No lar"-de"-todos da rua\ldots{}\\
O solzão ajudava a apoteose\\
Com o despejo das cores e calores\ldots{}

Segue o préstito numa via"-látea de esplendores.\\
Presa num palanquim de ônix e pórfiro\ldots{}\\
Ôta, morena boa!\\
Os olhos dela têm o verde das florestas,\\
Todo um Brasil de escravos"-banzo sensualismos,\\
Índios nus balanceando na terra das tabas,\\
Cauim curare caxiri\\
Cajás\ldots{} Ariticuns\ldots{} Pele de sol!\\
Minha vontade por você serpentinando\ldots{}

O préstito se vai.

Os Blocos se amontoam me afastando de você\ldots{}\\
Passa o Flor de Abacate,\\
Passa o Miséria e Fome, o Ameno Resedá\ldots{}\\
O préstito se vai\ldots{}

Você também se foi rindo pros outros,\\
Senhora dona ingrata\\
Coberta de ouro e prata\ldots{}

Esfuzios de risos\ldots{}\\
\quad\quad\quad{}Arrancos de metais\ldots{}\\
O schlschlsch monótono das serpentinas\ldots{}

Monótono das serpentinas\ldots{}

E a surpresa do fim: fadiga de gozar\ldots{}

Claros em torno da gente.\\
Bolas de fitas de papel rolando pelo chão.\\
Manchas de asfalto.\\
Os corpos adquirem de novo as sombras deles.\\
Tem lugares no bar.\\
As árvores pousam de novo no chão graciosas ordenadas,\\
Os palácios começam de novo subindo no céu\ldots{}

Quatro horas da manhã.\\
Nos clubes nas cavernas\\
Inda se ondula vagamente no maxixe.\\
Os corpos se unem mais.\\
Tem cinzas na escureza indecisa da arraiada.\\
Já é quarta"-feira no Passeio Público.\\
Numa sanha final\\
Os varredores carnavalizam as brisas da manhã\\
Com poeiras perfumadas e cromáticas.\\
Peri triste sentou na beira da calçada.\\
O carro"-chefe dos Democráticos\\
Sem a falação do estandarte\\
Sem vida, sem mulheres\\
Senil buscando o barracão.\\
Democraticamente\ldots{}

Aurora\ldots{} Tchim! Um farfalhar de plumas áureas no ar.\\
E as montanhas que nem tribos de guaianás em rapinas de luz\\
Com seus cocares de penas de tucano.

O poeta se debruça no parapeito de granito.\\
A rodelinha de confete cai do chapéu dele,\\
Vai saracotear ainda no samba mole das ondas.

Então o poeta vai deitar.

Lentamente se acalma no país das lembranças\\
A invasão furiosa das sensações.\\
O poeta sente"-se mais seu.\\
E puro agora pelo contato de si mesmo\\
Descansa o rosto sobre a mão que escreverá.

Lhe embala o sono\\
A barulhada matinal de Guanabara\ldots{}\\
Sinos buzinas clácsons campainhas\\
Apitos de oficinas\\
Motores bondes pregões no ar,\\
Carroças na rua, transatlânticos no mar\ldots{}\\
É a cantiga"-de"-berço.\\
E o poeta dorme.\\

O poeta dorme sem necessidade de sonhar.
\end{verse}

\chapter[\textsc{coordenadas}\\Sambinha \smallskip]{\textsc{coordenadas}\\(1924)}

\hfill\emph{a Couto de Barros}

\section*{Sambinha}

\begin{verse}
Vêm duas costureirinhas pela rua das Palmeiras.\\
Afobadas, braços dados, depressinha,\\
Bonitas, Senhor! que até dão vontade pros homens da rua.\\
As costureirinhas vão explorando perigos\ldots{}\\
Vestido é de seda.\\
Roupa"-branca é de morim.

Falando conversas fiadas\\
As duas costureirinhas passam por mim.\\
-- Você vai?\\
\qquad\qquad{}-- Não vou não!\\
Parece que a rua parou pra escutá"-las.\\
Nem os trilhos sapecas\\
Jogam mais bondes um pro outro.\\
E o sol da tardinha de abril\\
Espia entre as pálpebras crespas de duas nuvens.\\
As nuvens são vermelhas.\\
A tardinha é cor"-de"-rosa.

Fiquei querendo bem aquelas duas costureirinhas\ldots{}\\
Fizeram"-me peito batendo\\
Tão bonitas, tão modernas, tão brasileiras!\\
Isto é\ldots{}\\
Uma era ítalo"-brasileira.\\
Outra era áfrico"-brasileira.\\
Uma era branca.\\
Outra era preta.
\end{verse}

\chapter[Noturno de Belo Horizonte]{Noturno de Belo Horizonte \subtitulo{(1924)}}

\begin{flushright}
\emph{a Elísio de Carvalho}
\end{flushright}

\begin{verse}
Maravilha de milhares de brilhos vidrilhos,\\
Calma do noturno de Belo Horizonte\ldots{}\\
O silêncio fresco desfolha das árvores\\
E orvalha o jardim só.\\
Larguezas.\\
Enormes coágulos de sombra.\\
O polícia entre rosas\ldots{}\\
\qquad{}Onde não é preciso, como sempre\ldots{}\\
Há uma ausência de crimes\\
Na jovialidade infantil do friozinho.\\
Ninguém.\\
O monstro desapareceu.\\
Só as árvores árvores do mato"-virgem\\
Pendurando a tapeçaria das ramagens\\
Nos braços cabindas da noite.

Que luta pavorosa entre floresta e casas\ldots{}\\
Todas as idades humanas\\
Macaqueadas por arquiteturas históricas\\
Torres torreões torrinhas e tolices\\
Brigaram em nome da?\\
Os mineiros secundam em coro:\\
-- Em nome da civilização!\\
Minas progride.\\
Também quer ter também capital moderníssima também\ldots{}\\
Pórticos gregos do Instituto de Rádio\\
Onde jamais Empédocles entrará\ldots{}\\
O Conselho Deliberativo é manuelino,\\
Salão sapiente de Manuéis"-da"-hora\ldots{}\\
Arcos românicos de São José\\
E a catedral que pretende ser gótica\ldots{}\\
Pois tanto esquecimento da verdade!\\
A terra se insurgiu.

O mato invadiu o gradeado das ruas,\\
Bondes sopesados por troncos hercúleos,\\
Incêndio de Cafés,\\
Setas inflamadas,\\
Comboio de trânsfugas pro Rio de Janeiro,\\
A ramaria crequenta cegando as janelas\\
Com a poeira dura das folhagens\ldots{}\\
Aquele homem fugiu.\\
A imitação fugiu.\\
Clareiras do Brasil, praças agrestes!\ldots{}\\
Paz.

O mato vitorioso acampou nas ladeiras.\\
Suor de resinas opulentas.\\
Grupos de automóveis:\\
Baitacas e jandaias do rosal.\\
E o noturno apagando na sombra o artifício e o defeito\\
Adormece\\
Como um sonho mineiro.\\
Tem festas do Tejuco pelo céu!\\
As estrelas baralham"-se num estardalhaço de luzes.\\
O sr. barão das Catas"-Altas\\
Reúne todas as constelações\\
Pra fundir uma baixela de mundos\ldots{}\\
Bulício de multidões matizadas\ldots{}\\
Emboabas, carijós, espanhóis de Felipe \textsc{iv}\ldots{}\\
Tem baianos redondos\ldots{}\\
Dom Rodrigo de Castel Branco partirá!\ldots{}\\
Lumeiro festival\ldots{} Gritos\ldots{} Tocheiros\ldots{}\\
O Triunfo Eucarístico abala chispeando\ldots{}\\
Os planetas comparecem em pessoa!\\
Só as magnólias -- que banzo dolorido! --\\
As carapinhas fofas polvilhadas\\
Com a prata da Via"-Látea\\
Seguem pra igreja do Rosário\\
E pro jongo de Chico"-Rei\ldots{}

Estrelas árvores estrelas\\
E o silêncio fresco da noite deserta.\\
Belo Horizonte desapareceu\\
Transfigurada nas recordações.

\ldots{} Minas Gerais, fruta paulista\ldots{}\\
Ouvi que tem minas ocultas por cá\ldots{}\\
Mas ninguém mais conhece Marcos de Azeredo,\\
Quedê os roteiros de Robério Dias?\\
\qquad\qquad\quad{}Prata\\
\qquad\quad{}Diamantes cascateantes\\
Esmeraldas esmeraldas esperanças!\ldots{}

Não são esmeraldas, são turmalinas, bem se vê:

A casinha de taipa a beira"-rio.\\
Canoa abicada na margem,\\
A bruma das monções,\\
Mais nada.\\
Os galhos lavam matinalmente os cabelos\\
Na água barrenta indiferente.\\
As ondas sozinhas do Paraíba\\
Morrem avermelhadas mornas cor"-de"-febre.\\
E a febre\ldots{}

Não sejamos muito exigentes.\\
Todos os países do mundo\\
Tem os seus Guaicuis emboscados\\
No sossego das ribanceiras dolentes.\\
As carneiradas ficavam pra trás\ldots{}\\
O trem passava apavorado.\\
Só parou muito longe na estação\\
Pra que os romeiros saudassem\\
Nosso Senhor da Boa"-Viagem.

Ele ficava imóvel na beira dos trilhos\\
Amarrado à cegueira.\\
Trazia só os molambos necessários\\
Como convém aos santos e\\
Aos avarentos.\\
Porém o netinho corria junto das janelas dos vagões\\
Com o chapéu do cego na mão.\\
Quando a esmola caía -- com que triunfo! -- o menino gritava:\\
-- Pronto! Mais uma!\\
Então lá do seu mundo\\
Nosso Senhor abençoava:\\
-- Boa viagem.

Examina a carne do teu corpo.\\
Apesar da perfeição das estradas"-de"-ferro\\
E da inflexível providência dos horários,\\
Encontros descarrilamentos mortes\ldots{}\\
Pode ser!\ldots{}\\
As esmolas tombavam.\\
-- Pronto! Mais uma!\\
-- Boa viagem.

Minas Gerais de assombros e anedotas\ldots{}\\
Os mineiros pintam diariamente o céu de azul\\
Com os pincéis das macaúbas folhudas.\\
Olhe a cascata lá!\\
Súbita bombarda.\\
Talvez folha de arbusto,\\
Ninho de teneném que cai pesado,\\
Talvez o trem, talvez ninguém\ldots{}\\
As águas se assustaram\\
E o estouro dos rios começou.

Vão soltos pinchando rabanadas pelos ares,\\
Salta aqui salta corre viravolta pingo grito\\
Espumas brancas alvas\\
Fluem bolhas bolas,\\
Itoupavas altas\ldots{}\\
Borbulham bulhando em murmúrios churriantes\\
Nas bolsas brandas largas das enseadas lânguidas\ldots{}\\
De supetão fosso.\\
\qquad\qquad\qquad\quad Mergulho.\\
\qquad\qquad\qquad\qquad\qquad\quad Uivam tombando.\\
Desgarram serra abaixo.\\
Rio das Mortes\\
Paraopeba\\
Paraibuna,\\
Mamotes brancos\ldots{}\\
E o Araçuí de Fernão Dias\ldots{}\\
Barafustam vargens fora\\
Até acalmarem muito longe exânimes\\
Nas polidas lagoas de cabeça pra baixo.

Rio São Francisco o marrueiro dos matos\\
Partiu levando o rebanho pro norte\\
Ao aboio das águas lentamente.\\
A barcaça que ruma pra Juazeiro\\
Desce ritmada pelos golpes dos remeiros.\\
Na proa, o olhar distante a olhar,\\
Matraca o dançador:

\qquad\quad{}``Meu pangaré arreado,\\
\qquad\quad{}Minha garrucha laporte,\\
\qquad\quad{}Encostado no meu bem\\
\qquad\quad{}Não tenho medo da morte.\\
\qquad\qquad\quad{}Ah!\ldots{}''

Um grande Ah!\ldots{} aberto e pesado de espanto\\
Varre Minas Gerais por toda a parte\ldots{}\\
Um silêncio repleto de silêncio\\
Nas invernadas, nos araxás,\\
No marasmo das cidades paradas\ldots{}\\
Passado a fuxicar as almas,\\
Fantasmas de altares, de naves douradas\\
E dos palácios de Mariana e Vila Rica\ldots{}\\
\qquad\quad{}Isto é: Ouro Preto.\\
E o nome lindo de São José d'El Rei mudado num odontológico Tiradentes\ldots{}\\
Respeitemos os mártires.

Calma do noturno de Belo Horizonte\ldots{}\\
As estrelas acordadas enchem de Ahs!\ldots{} ecoantes o ar.\\
O silêncio fresco despenca das árvores.\\
Veio de longe, das planícies altas,\\
Dos cerrados onde o guaxe passa rápido\ldots{}\\
Vvvvvvv\ldots{} passou.\\
Passou tal qual o fausto das paragens de ouro velho\ldots{}\\
Minas Gerais, fruta paulista\ldots{}\\
Fruta que apodreceu.

Frutificou mineira! Taratá!\\
Há também colheitas sinceras!\\
Milharais canaviais cafezais insistentes\\
Trepadeirando morro acima.\\
Mas que chãos sovinas como o mineiro"-zebu!\\
Dizem que os baetas são agarrados\ldots{}\\
Não percebi, graças a Deus!\\
Na fazenda do Barreiro recebem opulentamente.\\
Os pratos nativos são índices de nacionalidade.\\
Mas no Grande Hotel de Belo Horizonte servem à francesa.\\
Et bien! Je vous demande un toutou!\\
Venha a batata"-doce e o torresmo fondant!\\
Carne"-de"-porco não!\\
O médico russo afirma que na carne"-de"-porco andam micróbios de loucura\ldots{}\\
Basta o meu desvairismo!\\
E os pileques\\
\qquad\qquad\quad{}quase pileques\\
\qquad\qquad\qquad\qquad{}salamaleques\\
\qquad\qquad\qquad\qquad\qquad{}da caninha de manga!\ldots{}

Taratá! Quero a couve mineira!\\
Minas progride!
Mãos esqueléticas de máquinas britando minérios,\\
As estradas"-de"-ferro estradas"-de"-rodagem\\
Serpenteiam teosoficamente fecundando o deserto\ldots{}

Afinal Belo Horizonte é uma tolice como as outras.\\
São Paulo não é a única cidade arlequinal.\\
E há vida há gente, nosso povo tostado.\\
O secretário da Agricultura é novo!\\
Fábricas de calçados\\
Escola de Minas no palácio dos Governadores,\\
Na Casa dos Contos não tem mais poetas encarcerados,\\
Campo de futebol em Carmo da Mata,\\
Divinópolis possui o melhor chuveiro do mundo,\\
As cunhãs não usam mais pó de ouro nos cabelos,\\
Os choferes avançam no bolso dos viajantes,\\
Teatro grego d'El Rei\\
Onde jamais Eurípedes será representado\ldots{}\\
Ninguém mais para nas pontes, Critilo,\\
Novidadeirando sobre damas casadas.\\
Tenho pressa! Ganhemos o dia!\\
Progresso! Civilização!\\
As plantações pendem maduras.\\
\quad{}O morfético ao lado da estrada esperando automóveis\ldots{}\\
Cheiro fecundo de vacas,\\
Pedreiras feridas,\\
Eletricidade submissa\ldots{}\\
Minas Gerais sáxea e atualista\\
Não resumida às estações"-termais!\\
Gentes do Triângulo Mineiro, Juiz de Fora!\\
Força das xiriricas das florestas e cerrados!\\
Minas Gerais, fruta paulista!\ldots{}

Alegria da noite de Belo Horizonte!\\
Há uma ausência de males\\
Na jovialidade infantil do friozinho.\\
Silêncio brincalhão salta das árvores,\\
Entra nas casas desce as ruas paradas\\
E se engrossa agressivo na praça do Mercado.\\
Vento florido roda pelos trilhos.\\
Vem de longe, das grotas pré"-históricas\ldots{}\\
Descendo as montanhas\\
Fugiu dos despenhadeiros assombrados do Rola"-Moça\ldots{}

Estremeção brusco de medo.\\
Pavor.\\
Folhas chorosas de eucaliptos.\\
Sino bate.\\
Ninguém.\\
A solidão angustiosa dos píncaros\ldots{}\\
A paz chucra, ressabiada, das gargantas da montanha\ldots{}


\quad\quad\quad{}A serra do Rola"-Moça\\
\quad\quad\quad{}Não tinha esse nome não\ldots{}\\
\quad\quad\quad{}Eles eram do outro lado,\\
\quad\quad\quad{}Vieram na vila casar.\\
\quad\quad\quad{}E atravessaram a serra,\\
\quad\quad\quad{}O noivo com a noiva dele\\
\quad\quad\quad{}Cada qual no seu cavalo.

\quad\quad\quad{}Antes que chegasse a noite\\
\quad\quad\quad{}Se lembraram de voltar.\\
\quad\quad\quad{}Disseram adeus pra todos\\
\quad\quad\quad{}E se puseram de novo\\
\quad\quad\quad{}Pelos atalhos da serra\\
\quad\quad\quad{}Cada qual no seu cavalo.

\quad\quad\quad{}Os dois estavam felizes,\\
\quad\quad\quad{}Na altura tudo era paz.\\
\quad\quad\quad{}Pelos caminhos estreitos\\
\quad\quad\quad{}Ele na frente ela atrás.\\
\quad\quad\quad{}E riam. Como eles riam!\\
\quad\quad\quad{}Riam até sem razão.

\quad\quad\quad{}A serra do Rola"-Moça\\
\quad\quad\quad{}Não tinha esse nome não.

\quad\quad\quad{}As tribos rubras da tarde\\
\quad\quad\quad{}Rapidamente fugiam\\
\quad\quad\quad{}E apressadas se escondiam\\
\quad\quad\quad{}Lá embaixo nos socavões\\
\quad\quad\quad{}Temendo a noite que vinha.

\quad\quad\quad{}Porém os dois continuavam\\
\quad\quad\quad{}Cada qual no seu cavalo,\\
\quad\quad\quad{}E riam. Como eles riam!\\
\quad\quad\quad{}E os risos também casavam\\
\quad\quad\quad{}Com as risadas dos cascalhos\\
\quad\quad\quad{}Que pulando levianinhos\\
\quad\quad\quad{}Da vereda se soltavam\\
\quad\quad\quad{}Buscando o despenhadeiro.

\quad\quad\quad{}Ah, Fortuna inviolável!\\
\quad\quad\quad{}O casco pisara em falso.\\
\quad\quad\quad{}Dão noiva e cavalo um salto\\
\quad\quad\quad{}Precipitados no abismo.\\
\quad\quad\quad{}Nem o baque se escutou.\\
\quad\quad\quad{}Faz um silêncio de morte.\\
\quad\quad\quad{}Na altura tudo era paz\ldots{}\\
\quad\quad\quad{}Chicoteando o seu cavalo,\\
\quad\quad\quad{}No vão do despenhadeiro\\
\quad\quad\quad{}O noivo se despenhou.

\quad\quad\quad{}E a serra do Rola"-Moça\\
\quad\quad\quad{}Rola"-Moça se chamou.

Eu queria contar as histórias de Minas\\
Aos brasileiros do Brasil\ldots{}

Filhos do Luso e da melancolia,\\
Vem, gente de Alagoas e de Mato Grosso,\\
De norte e sul homens fluviais do Amazonas e do rio Paraná\ldots{}\\
E os fluminenses salinos\\
E os guascas e os paraenses e os pernambucanos\\
E os vaqueiros de couro das caatingas\\
E os goianos governados por meu avô\ldots{}\\
Teutos de Santa Catarina,\\
Retirantes de língua seca,\\
Maranhenses paraibanos e do Rio Grande do Norte e do Espírito Santo\\
E do Acre, irmão caçula,\\
Toda a minha raça morena!\\
Vem, gente! vem ver o noturno de Belo Horizonte!\\
Sejam comedores de pimenta\\
Ou de carne requentada no dorso dos pigarços petiços,\\
Vem, minha gente!\\
Bebedores de guaraná e de açaí,\\
Chupadores do chimarrão,\\
Pinguços cantantes, cafezistas ricaços,\\
Mamíferos amamentados pelos cocos de Pindorama,\\
Vem, minha gente, que tem festas do Tejuco pelo céu!\\
Bárbara Heliodora desgrenhada louca\\
Dizendo versos desce a rua do Pará\ldots{}\\
Quem conhece as ingratidões de Marília?\\
Juro que foi Nosso Senhor Jesus Cristo Ele mesmo\\
Que plantou a sua cruz no adro das capelas da serra!\\
Foi Ele mesmo que d'El Rei\\
Esculpiu as imagens dos seus santos\ldots{}\\
E há histórias também pros que duvidam de Deus\ldots{}\\
\end{verse}

%\begin{hangparas}{.25in}{1}
\parbox{\textwidth}{
O coronel Antônio de Oliveira Leitão era casado com dona Branca Ribeiro
do Alvarenga, ambos de orgulhosa nobreza vicentina. Porém nas tardes de
Vila Rica a filha deles abanava o lenço no quintal\ldots{} -- ``Deve ser a
algum plebeu, que não há moços nobres na cidade\ldots{}'' E o descendente de
cavaleiros e capitães"-mores não quer saber de mésalliances. O coronel
Antônio de Oliveira Leitão esfaqueou a filha. Levaram"-no preso à Baía
onde foi decapitado. Pois dona Branca Ribeiro do Alvarenga reuniu todos
os cabedais. Mandou construir com eles uma igreja para que Deus
perdoasse as almas pecadoras do marido e da filha.}
%\end{hangparas}

\begin{verse}
Meus brasileiros lindamente misturados,\\
Se vocês vierem nessa igreja dos Perdões\\
Rezem três ave"-marias ajoelhadas\\
Pros dois desinfelizes.\\
Creio que a moça não carece muito delas\\
Mas ninguém sabe onde estará o coronel\ldots{}\\
Credo!

Mas não há nada como histórias pra reunir na mesma casa\ldots{}\\
Na Arábia por saber contar histórias\\
Uma mulher se salvou\ldots{}\\
A Espanha estilhaçou"-se numa poeira de nações americanas\\
Mas sobre o tronco sonoro da língua do ão\\
Portugal reuniu 22 orquídeas desiguais.\\
Nós somos na Terra o grande milagre do amor!

Que vergonha se representássemos apenas contingência de defesa\\
Ou mesmo ligação circunscrita de amor\ldots{}\\
Porém as raças são verdades essenciais\\
E um elemento de riqueza humana.\\
As pátrias têm de ser uma expressão de Humanidade.

Separadas na guerra ou na paz são bem pobres\\\
Bem mesquinhos exemplos de alma\\
Mas compreendidas juntas num amor consciente e exato\\
Quanta história mineira pra contar!

Não prego a guerra nem a paz, eu peço amor!\\
Eu peço amor em todos os seus beijos,\\
Beijos de ódio, de cópula ou de fraternidade.\\
Não prego a paz universal e eterna, Deus me livre!\\
Eu sempre contei com a imbecilidade vaidosa dos homens\\
E não me agradam os idealistas.\\
E temo que uma paz obrigatória\\
Nos fizesse esquecer o amor\\
Porque mesmo falando de relações de povo e povo\\
O amor não é uma paz\\
E é por amor que Deus nos deu a vida\ldots{}\\
O amor não é uma paz, bem mais bonito que ela,\\
Porque é um completamento!\ldots{}

Nós somos na Terra o grande milagre do amor!\\
E embora tão diversa a nossa vida\\
Dançamos juntos no carnaval das gentes,\\
Bloco pachola do ``Custa mas vai!''

E abre alas que Eu quero passar!\\
Nós somos os brasileiros auriverdes!\\
As esmeraldas das araras\\
Os rubis dos colibris\\
Os abacaxis as mangas os cajus\\
Atravessam amorosamente\\
A fremente celebração do Universal!

Que importa uns falem mole descansado\\
Que os cariocas arranhem os erres na garganta\\
Que os capixabas e paroaras escancarem as vogais?\\
Que tem se o quinhentos réis meridional\\
Vira cinco tostões do Rio pro Norte?\\
Juntos formamos este assombro de misérias e grandezas,\\
Brasil, nome de vegetal!\ldots{}

O bloco fantasiado de histórias mineiras\\
Move"-se na avenida de seis renques de árvores\ldots{}\\
O sol explode em fogaréus\ldots{}\\
O dia é frio sem nuvens, de brilhos vidrilhos\ldots{}\\
Não é dia! Não tem sol explodindo no céu!\\
É o delírio noturno de Belo Horizonte\ldots{}\\
Não nos esqueçamos da cor local:\\
Itacolomi\ldots{} \emph{Diário de Minas}\ldots{} Bondes do Calafate\ldots{}\\
E o silêncio\ldots{} sio\ldots{} sio\ldots{} quiriri\ldots{}

Os seres e as coisas se aplainam no sono.\\
Três horas.\\
A cidade oblíqua\\
Depois de dançar os trabalhos do dia\\
Faz muito que dormiu.

Seu corpo respira de leve o aclive vagarento das ladeiras.\\
De longe em longe gritam solitários brilhos falsos\\
Perfurando o sombral das figueiras:\\
Berenguendens berloques ouropéis de Oropa consagrada\\
Que a goiana trocou pelas pepitas de ouro fino.\\
Dorme Belo Horizonte.\\
Seu corpo respira de leve o aclive vagarento das ladeiras\ldots{}\\
Não se escuta sequer o ruído das estrelas caminhando\ldots{}\\
Mas os poros abertos da cidade\\
Aspiram com sensualidade com delícia\\
O ar da terra elevada.\\
Ar arejado batido nas pedras dos morros,\\
Varado através da água trançada das cachoeiras,\\
Ar que brota nas fontes com as águas\\
Por toda a parte de Minas Gerais.
\end{verse}

\chapter[\textsc{o ritmo sincopado}\\Toada do Pai"-do"-Mato]{\textsc{o ritmo sincopado}\\(1923--1926)}


\begin{flushright}
\emph{a Tarsila}
\end{flushright}

\section*{Toada do Pai"-do"-Mato\break (Índios~Parecis)}


\begin{verse}
A moça Camalalô\\
Foi no mato colher fruta.\\
A manhã fresca de orvalho\\
Era quase noturna.\\
\qquad-- Ah\ldots{}\\
Era quase noturna\ldots{}

Num galho de tarumã\\
Estava um homem cantando.\\
A moça sai do caminho\\
Pra escutar o canto.\\
\qquad-- Ah\ldots{}\\
Ela escuta o canto\ldots{}

Enganada pelo escuro\\
Camalalô fala pro homem:\\
Ariti, me dá uma fruta\\
Que eu estou com fome.\\
\qquad-- Ah\ldots{}\\
Estava com fome\ldots{}

O homem rindo secundou:\\
-- Zuimaalúti se engana,\\
Pensa que sou ariti?\\
Eu sou Pai"-do"-Mato.

Era o Pai"-do"-Mato!
\end{verse}

\pagebreak
\addcontentsline{toc}{chapter}{Tostão de chuva}
\section*{Tostão de chuva}

\begin{verse}
Quem é Antônio Jerônimo? É o sitiante\\
\qquad\quad{}Que mora no Fundão\\
Numa biboca pobre. É pobre. Dantes\\
Inda a coisa ia indo e ele possuía\\
\qquad\quad{}Um cavalo cardão.\\
Mas a seca batera no roçado\ldots{}\\
Vai, Antônio Jerônimo um belo dia\\
Só por debique de desabusado\\
Falou assim: ``Pois que nosso padim\\
Pade Ciço que é milagreiro, contam,\\
Me mande um tostão de chuva pra mim!''\\
Pois então nosso ``padim'' padre Cícero\\
Coçou a barba, matutando e disse:\\
``Pros outros mando muita chuva não,\\
Só dois vinténs. Mas pra Antônio Jerônimo\\
\qquad\quad{}Vou mandar um tostão''.\\
No outro dia veio uma chuva boa\\
Que foi uma festa pros nossos homens\\
E o milho agradeceu bem. Porém\\
No Fundão veio uma trovoada enorme\\
Que num átimo virou tudo em lagoa\\
E matou o cavalo de Antônio Jerônimo.\\
\qquad\quad{}Matou o cavalo.
\end{verse}

\pagebreak
\addcontentsline{toc}{chapter}{Moda da cadeia de Porto Alegre}
\section*{Moda da cadeia de Porto Alegre}

\begin{flushright}
\emph{a Mário Pedrosa}
\end{flushright}

\begin{verse}
Dona Rita amouxa em casa\\
Uma porção da riqueza\\
Que o marido, que Deus tenha!\\
Por amor dela ajuntou.\\
A riqueza de que falo\\
É cobres, porque dos filhos\\
Só um mocinho não gorou.

Apesar dessa família\\
Já grande, em pleno viçor,\\
Quando ela pensa em gatunos\\
Corre pela espinha dela\\
Uma friagem de horror.

Também não tem na cidade\\
Correição de segurança\\
Adonde gatuno que entra\\
Perde pra sempre a esperança\\
De outra vez ir gatunar.\\
Dona Rita passa as noites\\
Sem dormir, sem descansar.\\
Qualquer barulhinho a pobre\\
Levanta, vai assuntar.

Pois então ela resolve,\\
Gasta mas gasta pra bem:\\
Faz construir uma cadeia\\
Que mais segura não tem\\
Por este grande Brasil.

Era mesmo um casarão\\
Alvo que nem tabatinga,\\
Com tanta grade tamanha\\
Que apertava o coração.\\
Toda a gente ia passear\\
Lá no largo da Cadeia\\
Mas porém se espera um preso\\
Pra estreia da correição.

Agora o filho entra tarde.\\
Dona Rita sossegada\\
Costura, pesponta meias\\
Enquanto sono não vem.\\
Só de pensar na cadeia\\
Dona Rita dorme bem.

Foi então que numa festa\\
Já quase de"-manhãzinha\\
O filho de dona Rita\\
Botou seis tiros no peito\\
De outro moço, rival dele\\
Nuns negócios de paixão.

Estrearam a correição.\\
Dona Rita não foi ver.

Definha que não definha,\\
Durou uns pares de meses,\\
Afinal veio a morrer.

Falam também que de"-noite\\
O carcereiro rondando\\
Escuta pelo caminho\\
O choro de dona Rita\\
Gemendo devagarzinho\ldots{}

Mas isso de assombração\\
Só quem vê é que acredita\ldots{}
\end{verse}

\chapter[\textsc{dois poemas acreanos}\\\textsc{i}\ \ Descobrimento]{\textsc{dois poemas acreanos}}

\begin{flushright}
\emph{a Ronald de Carvalho}
\end{flushright}

\section*{I \break Descobrimento}

\begin{verse}
Abancado à escrivaninha em São Paulo\\
Na minha casa da rua Lopes Chaves\\
De supetão senti um friúme por dentro.\\
Fiquei trêmulo, muito comovido\\
Com o livro palerma olhando pra mim.

Não vê que me lembrei que lá no norte, meu Deus! muito longe de mim,\\
Na escuridão ativa da noite que caiu,\\
Um homem pálido, magro, de cabelo escorrendo nos olhos,\\
Depois de fazer uma pele com a borracha do dia,\\
Faz pouco se deitou, está dormindo.

Esse homem é brasileiro que nem eu\ldots{}
\end{verse}

\partepigraph{Quid, homo, ineptam sequeris laetitiam.}{\textsc{séc.\,xi}}
\part{\textsc{remate de males}}
\removeepigraph


\chapter[Eu sou trezentos\ldots{}]{Eu sou trezentos\ldots{} \subtitulo{(7 de junho de 1929)}}

\begin{verse}
Eu sou trezentos, sou trezentos"-e"-cinquenta,\\
As sensações renascem de si mesmas sem repouso,\\
Ôh espelhos, ôh Pireneus! ôh caiçaras!\\
Se um deus morrer, irei no Piauí buscar outro!

Abraço no meu leito as melhores palavras,\\
E os suspiros que dou são violinos alheios;\\
Eu piso a terra como quem descobre a furto\\
Nas esquinas, nos táxis, nas camarinhas seus próprios beijos!

Eu sou trezentos, sou trezentos"-e"-cinquenta,\\
Mas um dia afinal me encontrarei comigo\ldots{}\\
Tenhamos paciência, andorinhas curtas,\\
Só o esquecimento é que condensa,\\
E então minha alma servirá de abrigo.
\end{verse}

\chapter[Danças]{Danças \subtitulo{(1924)}}


\begin{flushright}
\emph{a Dona Baby Guilherme de Almeida}
\end{flushright}

\section*{I}

\begin{verse}
Quem dirá que não vivo satisfeito! Eu danço!

Dança a poeira no vendaval.\\
Raios solares balançam na poeira.\\
Calor saltita pela praça\\
\qquad\qquad\qquad\quad pressa\\
\qquad\qquad\qquad\quad apertos\\
\qquad\qquad\qquad\quad automóveis\\
\qquad\qquad bamboleios\\
\qquad Pinchos ariscos de gritos\\
Bondes sapateando nos trilhos\ldots{}

A moral não é roupa diária!

Sou bom só nos domingos e dias"-santos!

Só nas meias o dia"-santo é quotidiano!\\
\qquad\qquad\qquad\quad Vida\\
\qquad\qquad\qquad\quad arame\\
\qquad\qquad\qquad\quad crimes\\
\qquad\qquad\qquad\quad \emph{quidam}\\
\qquad\qquad\qquad\quad cama e pança!\\
\qquad\qquad Viva a dança!\\
\qquad\qquad Dança viva!\\
\qquad Vivedouro de alegria!\\
Eu danço!\\
Mãos e pés, músculos, cérebro\ldots{}\\
\qquad Muito de indústria me fiz careca,\\
\qquad Dei um salão aos meus pensamentos!\\
\qquad\qquad\qquad\qquad Tudo gira,\\
\qquad\qquad\qquad\qquad Tudo vira,\\
\qquad\qquad\qquad\qquad Tudo salta,\\
\qquad\qquad\qquad\qquad Samba,\\
\qquad\qquad\qquad\qquad Valsa,\\
\qquad\qquad\qquad\qquad Canta,\\
\qquad\qquad\qquad\qquad Ri!

Quem foi que disse que não vivo satisfeito?\\
\textsc{eu danço}!
\end{verse}

\pagebreak
\section*{VI}

\begin{verse}
Parceiro, tu sabes a dança do ventre\\
Mas eu vou te ensinar dança melhor.\\
Olha: a Terra é uma bola.\\
\qquad\qquad\quad\quad A bola gira.\\
\qquad\qquad Gira o universo.\\
\qquad Os homens giram também.

Tudo é girar, tudo é rodar.

\quad Sofres acaso de amor sem volta?\\
\quad Porque paraste no teu amor!\\
\qquad\quad Choras que os outros não te compreendem?\\
\qquad\quad Fala francês que te entenderão!\\
\qquad\qquad\quad Morres, duvidas, pensas?\ldots{} -- Parceiro,\\
Tu só conheces a dança do ventre,\\
A dança do ombro é muito melhor!
\end{verse}

\pagebreak
\section*{IX}

\begin{verse}
\textsc{eu danço!}

Eu danço manso, muito manso,\\
Não canso e danço,\\
Danço e venço,\\
\qquad Manipanso\ldots{}\\
\qquad\quad Só não penso\ldots{}

Quando nasci eu não pensava e era feliz\ldots{}

Quando nasci eu já dançava,\
Dançava a dança da criança,\\
\qquad Surupango da vingança\ldots{}

Dança do berço:\\
\qquad Sim e Não\ldots{}\\
Dança do berço:\\
\qquad Não e Sim\ldots{}\\
\qquad\quad A vida é assim\ldots{}\\
E eu sou assim.

\ldots{} ela dançava porque tossia\ldots{}\\
\qquad Outros dançam de soluçar\ldots{}\\
\qquad\quad Eu danço manso a dança do ombro\ldots{}\\
\qquad\qquad\quad Eu danço\ldots{} Não sei mais chorar!\ldots{}
\end{verse}

\chapter[\textsc{tempo da maria}\\\textsc{i}\ \ \ \ Moda do corajoso]{\textsc{tempo da maria} \subtitulo{(1926)}}

\begin{flushright}
a \emph{Dona Eugênia Álvaro Moreira}
\end{flushright}

\section*{I\break Moda do corajoso}

\begin{verse}
Maria dos meus pecados,\\
Maria, viola de amor\ldots{}

Já sei que não tem propósito\\
Gostar de donas casadas,\\
Mas quem que pode com o peito!\\
Amar não é desrespeito,\\
Meu amor terá seu fim.\\
Maria há"-de ter um fim.

Quem sofre sou eu, que importa\\
Pros outros meu sofrimento?\\
Já estou curando a ferida.\\
Se dando tempo pro tempo\\
Toda paixão é esquecida.\\
Maria será esquecida.

Que bonita que ela é!\ldots{} Não\\
Me esqueço dela um momento!\\
Porém não dou cinco meses,\\
Acabarão as fraquezas\\
E a paixão será arquivada.\\
Maria será arquivada.

Por enquanto isso é impossível.\\
O meu corpo encasquetou\\
De não gostar senão de uma\ldots{}\\
Pois, pra não fazer feiura,\\
Meu espírito sublima\\
O fogo devorador.\\
Faz da paixão uma prima,\\
Faz do desejo um bordão,\\
E encabulado ponteia\\
A malvadeza do amor.

Maria, viola de amor!\ldots{}
\end{verse}

\pagebreak
\addcontentsline{toc}{chapter}{\textsc{iv}\ \ Lenda das mulheres de peito chato}
\section*{\textsc{iv}\break Lenda das mulheres de peito chato}

\begin{verse}
Macunaíma, Maria,\\
Viajando por essas terras\\
Com os dois manos, encontrou\\
Uma cunhã tão formosa\\
Que era um pedaço de dia\\
Na noite do mato"-virgem.\\
Macunaíma, Maria,\\
Gostou da moça bonita.\\
Porém ela era casada,\\
E jamais não procedia\\
Que nem as donas de agora,\\
Que vivem mais pelas ruas\\
Do que na casa em que moram;\\
Vivia só pro marido\\
E os filhos do seu amor,\\
Fiava, tecia o fio,\\
Pescava, e março chegado,\\
Mexendo o corpo gostoso,\\
Ela fazia a colheita\\
Do milho de beira"-rio.\\
Que bonita que ela é!\ldots{} Bom.\\
Macunaíma, Maria,\\
Não pôde seguir, ficou.\\
Que que havia de fazer!\\
Amar não é desrespeito,\\
Falou pra ela e ela se riu.\\
Então lhe subiu do peito\\
A escureza da paixão,\\
E o apaixonado cegou.\\
Pegou nela, mas a moça\\
Possuía essa grande força\\
Que é a força de querer bem:\\
Forceja que mais forceja,\\
Até deu nele! Não doeu.\\
Macunaíma, Maria,\\
Largou da moça.\\
\qquad\qquad\qquad{}Ôh, meu Deus!

Como estava contrariado!\\
Pois um moço que ama então\\
Não tem direito de amar!\\
Tem, Maria, tem direito!\\
Te juro que tem direito!\\
Macunaíma fez bem!\\
O amor dele era tão nobre\\
Ver o do outro que casou.\\
Casar é uma circunstância\\
Que se dá, que não se dá,\\
Porém amar é a constância,\\
Porta num, se abanca, e o pobre\\
Tem que lhe matar a fome,\\
Dar cama pra ele dormir.\\
Macunaíma, Maria,\\
Era como eu brasileiro,\\
E em todas as moradias\\
Que se erguem no chão quentinho\\
Do nosso imenso Brasil,\\
Não tem uma que não tenha\\
Um quarto"-de"-hóspedes pronto!\\
Pobre do Macunaíma,\\
Não tem culpa de penar!\\
Foi brasileiro, amor veio,\\
Ele teve que hospedar!

-- Eu te amo, (que ele falava)\\
Moça linda! Você tem\\
Esse risco de urucum\\
Na beira do olhar somente\\
Pra não ver quem te quer bem!\\
Olhos de jabuticaba!\\
Colinho de cujubim!\ldots{}\\
Te adoro como se adora\\
Com doçura e com paixão!\\
Maria\ldots{} Vamos embora!\\
(Que ele falava pra moça)\\
Eu quero você pra mim!

Bom. O coitado, Maria,\\
De tanta contrariedade,\\
Pôs reparo que é impossível\\
Se ser feliz neste mundo,\\
Em plena infelicidade\ldots{}\\
Se vingou. Tinha ali perto\\
Dois cachos de bananeira.\\
Cortou deles\ldots{} você sabe,\\
Os mangarás pendurados,\\
Que de tão arroxeados\\
Têm mesmo a cor da paixão.\\
Lá no Norte chamam isso\\
De ``filhotes da banana'',\\
E a bananeira dá fruta\\
Uma vez, não dá mais não\ldots{}\\
Macunaíma, Maria,\\
Pegou na moça, arrancou\\
Os peitinhos emproados\\
Do colo de cujubim,\\
Pendurou no lugar deles\\
Os filhotes da paixão.

Por isso essa moça dura,\\
De quem nós todos nascemos,\\
Tem o colo que nem de homem,\\
De achatado que ficou.\\
E hoje as donas são assim\ldots{}

Adianta a lenda que a moça\\
Ficou feia\ldots{} Não sei não\ldots{}
\end{verse}

\pagebreak
\addcontentsline{toc}{chapter}{\textsc{vi}\ \ Louvação da tarde \smallskip}
\section*{\textsc{vi}\break Louvação da tarde}

\begin{verse}
Tarde incomensurável, tarde vasta,\\
Filha de sol já velho, filha doente\\
De quem despreza as normas da eugenia,\\
Tarde vazia, dum rosado pálido,\\
Tarde tardonha e sobretudo tarde\\
Imóvel\ldots{} quase imóvel: é gostoso\\
Com o papagaio louro do ventinho\\
Pousado em minha mão, pelas ilhotas\\
Dos teus perfumes me perder, rolando\\
Sobre a desabitada rodovia.\\
Só tu me desagregas, tarde vasta,\\
Da minha trabalheira. Sigo livre,\\
Deslembrado da vida, lentamente,\\
Com o pé esquecido do acelerador.\\
E a maquininha me conduz, perdido\\
De mim, por entre cafezais coroados,\\
Enquanto meu olhar maquinalmente\\
Traduz a língua norte"-americana\\
Dos rastos dos pneumáticos na poeira.\\
O doce respirar do forde se une\\
Aos gritos pontiagudos das graúnas,\\
Aplacando meu sangue e meu ofego.\\
São murmúrios severos, repetidos,\\
Que me organizam todo o ser vibrante\\
Num método sadio. Só no exílio\\
De teu silêncio, os ritmos maquinares\\
Sinto, metodizando, regulando\\
O meu corpo. E talvez meu pensamento\ldots{}

Tarde, recreio de meu dia, é certo\\
Que só no teu parar se normaliza\\
A onda de todos os transbordamentos\\
Da minha vida inquieta e desregrada.\\
Só mesmo distanciado em ti, eu posso\\
Notar que tem razão"-de"-ser plausível\\
Nos trabalhos de ideal que vou semeando\\
Atabalhoadamente sobre a Terra.\\
Só nessa vastidão dos teus espaços,\\
Tudo o que gero e mando, e que parece\\
Tão sem destino e sem razão, se ajunta\\
Numa ordem verdadeira\ldots{} Que nem gado,\\
Pelo estendal do jaraguá disperso,\\
Ressurge de tardinha e, enriquecido\\
Ao aboio sonoro dos campeiros,\\
Enriquece o criador com mil cabeças\\
No circo da mangueira recendente\ldots{}

Tarde macia, pra falar verdade:\\
Não te amo mais do que a manhã, mas amo\\
Tuas formas incertas e estas cores\\
Que te maquilham o carão sereno.\\
Não te prefiro ao dia em que me agito,\\
Porém contigo é que imagino e escrevo\\
O rodapé do meu sonhar, romance\\
Em que o Joaquim Bentinho dos desejos\\
Mente, mente, remente impávido essa\\
Mentirada gentil do que me falta.\\
Um despropósito de perfeições\\
Me cerca e, em grata sucessão de casos,\\
Vou com elas vivendo uma outra vida:

\ldots{} Toda dor física azulou\ldots{} Meu corpo,\\
Sem artritismos, faringites e outras\\
Específicas doenças paulistanas,\\
Tem saúde de ferro. Às intempéries\\
Exponho as ondas rijas dos meus músculos,\\
Sem medo. Praquê medo!\ldots{} Regulares,\\
Mais regulares do que os meus, os traços\\
Do meu rosto me fazem desejado\\
Mais facilmente que na realidade\ldots{}\\
Já não falo por ela não, por essa\\
Em cujo perfil duro jaz perdida\\
A independência do meu reino de homem\ldots{}\\
Que bonita que ela é!\ldots{} Qual!\ldots{} Nem por isso.\\
Não sonho sonhos vãos. A realidade,\\
Mais esportiva de vencer, me ensina\\
Esse jeito viril de ir afastando\\
Dos sonhos vesperais os impossíveis\\
Que fazem a quimera, e de que a vida\\
É nua, friorentamente nua.\\
Não a desejo não\ldots{} Viva em sossego\\
Essa que sendo minha, nos traria\\
Uma vida de blefe, arrebatada\\
Por mais estragos que deslumbramentos.\\
Isto, em bom português, é amor platônico\ldots{}\\
Quá! quá! quá!\ldots{} Desejemos só conquistas!\\
Um poder de mulheres diferentes,\\
Meninas"-de"-pensão, costureirinhas,\\
Manicuras, artistas, datilógrafas,\\
Brancaranas e louras sem escândalo,\\
Desperigadas\ldots{} livro de aventuras\\
Dentro do qual secasse a imagem da outra,\\
Que nem folha de malva, que nem folha\\
De malva\ldots{} da mais pura malva perfumada!\ldots{}

Livre dos piúns das doenças amolantes,\\
Com dinheiro sobrando, organizava\\
As poucas viagens que desejo\ldots{} Iria\\
Viajar todo esse Mato Grosso grosso,\\
Danado guardador da indiada feia,\\
E o Paraná verdinho\ldots{} Ara, se acaso\\
Tivesse imaginado no que dava\\
A Isidora, não vê que ficaria\\
Na expectativa pança em que fiquei!\\
Revoltoso banzando em viagens tontas,\\
Ao menos o meu sul conheceria,\\
Pampas forraginosos do Rio Grande\\
E praias ondejantes do Iguaçu\ldots{}\\
Tarde, com os cobres feitos com teu ouro,\\
Paguei subir pelo Amazonas\ldots{} Mundos\\
Desbarrancando, chãos desbarrancados,\\
Aonde no quiriri do mato brabo\\
A terra em formação devora os homens\ldots{}\\
Este refrão dos meus sentidos\ldots{} Nada\\
Matutarei mais sem medida, ôh tarde,\\
Do que esta pátria tão despatriada!

Vibro! Vibro. Mas constatar sossega\\
A gente. Pronto, sosseguei. O forde\\
Recomeça tosando a rodovia.\\
``Nosso ranchinho assim tava bom\ldots{}'' Sonho\ldots{}\\
Já sabe: desejando sempre\ldots{} Um sítio,\\
Colonizado, sem necessidade\\
De japoneses nem de estefanóderis\ldots{}\\
Que desse umas quatorze mil arrobas\ldots{}\\
Já me bastava. Gordas invernadas\\
Pra novecentos caracus bem\ldots{}

\qquad\qquad\qquad\qquad\qquad\qquad\quad Tarde,\\
Careço de ir voltando, estou com fome.\\
Ir pra um quarto"-de"-banho hidroterápico\\
Que fosse a peça de honra deste rancho,\\
Aonde também, faço questão, tivesse\\
Dois ou três quartos"-de"-hóspedes\ldots{} Isto é,\\
De hóspedes não, de amigos\ldots{} Esta casa\\
É sua\ldots{} Entre\ldots{} Se abanque\ldots{} Mande tudo\ldots{}\\
Não faça cerimônia\ldots{} Olha, de"-noite\\
Teremos Hindemith e Villa"-Lobos!\\
Que bom! possuir um aparelho de\\
Radiotelefonia tão perfeito\\
Que pegasse New York e Buenos Aires!\ldots{}\\
Tarde de meu sonhar, te quero bem!\\
Deixa que nesta louvação, se lembre\\
Essa condescendência puxapuxa\\
De teu sossego, essa condescendência\\
Tão afeiçoável ao desejo humano.\\
De"-dia eu faço, mas de"-tarde eu sonho.\\
Não és tu que me dás felicidade,\\
Que esta eu crio por mim, por mim somente,\\
Dirigindo sarado a concordância\\
Da vida que me dou com o meu destino.\\
Não marco passo não! Mas se não é\\
Com desejos sonhados que me faço\\
Feliz, o excesso de vitalidade\\
Do espírito é com eles que abre a válvula\\
Por onde escoa o inútil excessivo;\\
Pois afastando o céu de junto à terra,\\
Tarde incomensurável, me permites,\\
Qual jaburus"-moleques de passagem,\\
Lançar bem alto nos espaços essa\\
Mentirada gentil do que me falta.

Ciao, tarde, estou chegando. É quase noite.\\
Todo o céu já cinzou. Dependurada\\
Na rampa do terreiro a gaiolinha\\
Branca da máquina ``São Paulo'' inda arfa,\\
As tulhas de café desentulhando.\\
Pelo ar um lusco"-fusco brusco trila,\\
Serelepeando na baixada fria.\\
Bem no alto do espigão, sobre o pau seco,\\
Ver um carancho, se empoleira a lua,\\
-- Condescendente amiga das metáforas\ldots{}
\end{verse}

\chapter[Poemas da negra]{Poemas da negra \subtitulo{(1929)}}

\begin{flushright}
\emph{a Cícero Dias}
\end{flushright}

\section*{I}

\begin{verse}
Não sei por que espírito antigo\\
Ficamos assim impossíveis\ldots{}

A lua chapeia os mangues\\
Donde sai um favor de silêncio\\
E de maré.\\
És uma sombra que apalpo\\
Que nem um cortejo de castas rainhas.\\
Meus olhos vadiam nas lágrimas.\\
Te vejo coberta de estrelas,\\
Coberta de estrelas,\\
Meu amor!

Tua calma agrava o silêncio dos mangues.
\end{verse}

\pagebreak
\section*{II}

\begin{verse}
Não sei se estou vivo\ldots{}\\
Estou morto.

Um vento morno que sou eu\\
Faz auras pernambucanas.\\
Rola rola sob as nuvens\\
O aroma das mangas.\\
Se escutam grilos,\\
Cricrido contínuo\\
Saindo dos vidros.\\
Eu me inundo de vossas riquezas!\\
Não sou mais eu\ldots{}\\

Que indiferença enorme\ldots{}
\end{verse}

\pagebreak
\section*{III}

\begin{verse}
Você é tão suave,\\
Vossos lábios suaves\\
Vagam no meu rosto,\\
Fecham meu olhar.

Sol"-posto.

É a escureza suave\\
Que vem de você,\\
Que se dissolve em mim.

Que sono\ldots{}

Eu imaginava\\
Duros vossos lábios,\\
Mas você me ensina\\
A volta ao bem.
\end{verse}

\pagebreak
\section*{IV}

\begin{verse}
Estou com medo\ldots{}\\
Teu beijo é tão beijo,\\
Tua inocência é dura,\\
Feita de camélias.

Ôh, meu amor,/\\
Nós não somos iguais!\\
Tu me proíbes\\
Beber água após\ldots{}

Eu volto à calma\\
E não te vejo mais.
\end{verse}

\pagebreak
\section*{V}

\begin{verse}
Lá longe no sul,\\
Lá nos pés da Argentina,\\
Marulham temíveis os mares gelados,\\
Não posso fazer mesmo um gesto!

Tu me adivinhas, meu amor,\\
Porém não queres ser escrava!

Flores!\\
Apaixonadamente meus braços desgalham"-se,\\
Flores!\\
Flores amarelas do pau"-d'arco secular!\\
Eu me desgalho sobre teu corpo manso,\\
As flores estão caindo sobre teu corpo manso,\\
Te cobrirei de flores amarelas!

Apaixonadamente\\
Eu me defenderei!
\end{verse}

\pagebreak
\section*{VI}

\begin{verse}
Quando\\
Minha mão se alastra\\
Em vosso grande corpo,\\
Você estremece um pouco.

É como o negrume da noite\\
Quando a estrela Vênus\\
Vence o véu da tarde\\
E brilha enfim.

Nossos corpos são finos,\\
São muito compridos\ldots{}\\
Minha mão relumeia\\
Cada vez mais sobre você.

E nós partimos adorados\\
Nos turbilhões da estrela Vênus!\ldots{}
\end{verse}

\pagebreak
\section*{VII}

\begin{verse}
Não sei porque os tetéus gritam tanto esta noite\ldots{}\\
Não serão talvez nem mesmo os tetéus.\\
Porém minha alma está tão cheia de delírios\\
Que faz um susto enorme dentro do meu ser.

Estás imóvel.\\
És feito uma praia\ldots{}\\
Talvez estejas dormindo, não sei.

Mas eu vibro cheinho de delírios,\\
Os tetéus gritam tanto em meus ouvidos,\\
Acorda! ergue ao menos o braço dos seios!\\
Apaga o grito dos tetéus!
\end{verse}

\pagebreak
\section*{VIII}

\begin{verse}
Nega em teu ser primário a insistência das coisas,\\
Me livra do caminho.

Colho mancheias de meus olhares,\\
Meu pensamento assombra mundos novos,\\
E eu desejava estar contigo\ldots{}

Há vida por demais neste silêncio nosso!\\
Eu próprio exalo fluidos leves\\
Que condensam"-se em torno\ldots{}\\
Me sinto fatigantemente eterno!

Ah, meu amor,\\
Não é minha amplidão que me desencaminha,\\
Mas a virtuosidade\ldots{}
\end{verse}

\pagebreak
\section*{IX}

\begin{verse}
Na zona da mata o canavial novo\\
É um descanso verde que faz bem;\\
É uma suavidade pousar a vista\\
Na manteiga e no pelo dos ratos;\\
No mais matinal perfume francês\\
A gente domina uma dedicação;\\
Apertando os dedos no barro mole\\
Ele escorre e foge,\\
E o corpo estremece que é um prazer\ldots{}

Mas você é grave sem comparação.
\end{verse}

\pagebreak
\section*{X}

\begin{verse}
Há o mutismo exaltado dos astros,\\
Um som redondo enorme que não para mais.\\
Os duros vulcões ensanguentam a noite,\\
A gente se esquece no jogo das brisas,\\
A jurema perde as folhas derradeiras\\
Sobre Mestre Carlos que morreu.\\
Dir"-se"-ia que os ursos\\
Mexem na sombra do mato\ldots{}\\
A escureza cai sobre abelhas perdidas.\\
Um potro galopa.\\
Ponteia uma viola\\
De sertão.

Nós estamos de pé,\\
Nós nos enlaçamos,\\
Somos tão puros,\\
Tão verdadeiros\ldots{}\\
Ôh, meu amor!\\
O mangue vai refletir os corpos enlaçados!\\
Nossas mãos já partem no jogo das brisas,\\
Nossos lábios se cristalizam em sal!\\
Nós não somos mais nós!\\
Nós estamos de pé!\\
Nós nos amamos!
\end{verse}


\pagebreak
\section*{XI}

\begin{verse}
Ai momentos de físico amor,\\
Ai reentrâncias de corpo\ldots{}\\
Meus lábios são que nem destroços\\
Que o mar acalanta em sossego.

A luz do candeeiro te aprova,\\
E\ldots{} não sou eu, é a luz aninhada em teu corpo\\
Que ao som dos coqueiros do vento\\
Farfalha no ar os adjetivos.
\end{verse}

\pagebreak
\section*{XII}

\begin{verse}
Lembrança boa,\\
Carrego comigo tua mão.

O calor exausto\\
Oprime estas ruas\\
Que nem a tua boca pesada.\\
As igrejas oscilam\\
Por cima dos homens de branco,\\
E as sombras despencam inúteis\\
Das botinas, passo a passo.

O que me esconde\\
É o momento suave\\
Com que as casas velhas\\
São róseas, morenas,\\
Na beira do rio.

Dir"-se"-ia que há madressilvas\\
No cais antigo\ldots{}\\
Me sinto suavíssimo de madressilvas\\
Na beira do rio.
\end{verse}


\chapter[\textsc{marco de viração}\\Improviso do mal da América]{Marco de viração}

\begin{flushright}
\emph{a José Bento Faria Ferraz}
\end{flushright}


\section*{Improviso do mal da América\break (fevereiro~de~1928)}

\begin{verse}
Grito imperioso de brancura em mim\ldots{}

Êh coisas de minha terra, passados e formas de agora,\\
Êh ritmos de síncopa e cheiros lentos de sertão,\\
Varando contracorrente o mato impenetrável do meu ser\ldots{}\\
Não me completam mais que um balango de tango,\\
Que uma reza de indiano no templo de pedra,\\
Que a façanha do chim comunista guerreando,\\
Que prantina de piá, encastoado de neve, filho de lapão.

São ecos. Mesmos ecos com a mesma insistência filtrada\\
Que ritmos de síncopa e cheiro do mato meu.\\
Me sinto branco, fatalizadamente um ser de mundos que nunca vi.\\
Campeio na vida a jacumã que mude a direção destas igaras fatigadas\\
E faça tudo ir indo de rodada mansamente\\
Ao mesmo rolar de rio das aspirações e das pesquisas\ldots{}\\
Não acho nada, quase nada, e meus ouvidos vão escutar amorosos\\
Outras vozes de outras falas de outras raças, mais formação, mais forçura.\\
Me sinto branco na curiosidade imperiosa de ser.

Lá fora o corpo de São Paulo escorre vida ao guampaço dos arranha"-céus,\\
E dança na ambição compacta de dilúvios de penetras.\\
Vão chegando italianos didáticos e nobres;\\
Vai chegando a falação barbuda de Unamuno\\
Emigrada pro quarto"-de"-hóspedes acolhedor da Sulamérica;\\
Bateladas de húngaros, búlgaros, russos se despejam na cidade\ldots{}\\
Trazem vodka no sapiquá de veludo,\\
Detestam caninha, detestam mandioca e pimenta,\\
Não dançam maxixe, nem dançam catira, nem sabem amar suspirado.

E de"-noite monótonos reunidos na mansarda, bancando conspiração,\\
As mulheres fumam feito chaminés sozinhas,\\
Os homens destilam vícios aldeões na catinga;\\
E como sempre entre eles tem sempre um que manda sempre em todos,\\
Tudo calou de supetão, e no ar amolegado da noite que sua\ldots{}\\
-- Coro? Onde se viu agora coro a quatro vozes, minha gente! --\\
São coros, coros ucranianos batidos ou místicos, Sehensucht d'além"-mar!\\
Home\ldots{} Sweet home\ldots{} Que sejam felizes aqui!

Mas eu não posso, não, me sentir negro nem vermelho!\\
De certo que essas cores também tecem minha roupa arlequinal,\\
Mas eu não me sinto negro, mas eu não me sinto vermelho,\\
Me sinto só branco, relumeando caridade e acolhimento,\\
Purificado na revolta contra os brancos, as pátrias, as guerras, as posses, as
preguiças e ignorâncias!\\
Me sinto só branco agora, sem ar neste ar"-livre da América!\\
Me sinto só branco, só branco em minha alma crivada de raças!
\end{verse}

\pagebreak
\addcontentsline{toc}{chapter}{Momento}
\section*{Momento\break (16~de~setembro~de~1928)}

\begin{verse}
Deve haver aqui perto uma roseira florindo,\\
Não sei\ldots{} sinto por mim uma harmonia,\\
Um pouco da imparcialidade que a fadiga traz consigo.

Olho pra minhas mãos. E uma ternura perigosa\\
Me faz passar a boca sobre elas, roçando,\\
(De certo é alguma rosa\ldots{})\\
Numa ternura que não é mais perigosa não, é piedade paciente.\\
As rosas\ldots{} Os milhões de rosas paulistanas\ldots{}\\
Já tanto que enxerguei minhas mãos trabalhando,\\
E tapearem por brinquedo umas costas de amigo,\\
Se entregarem pra inimigo, erguerem dinheiro do chão\ldots{}\\
Uma feita meus dedos pousaram nuns lábios,\\
Nesse momento eu quis ser cego!\\
Ela não quis beijar a ponta dos meus dedos,\\
Beijou as mãos, apaixonadamente, em submissão\ldots{}\\
Ela beijou o pó das minhas mãos\ldots{}\\
O mesmo pó que já desce na rosa nem bem ela se abre.\\
Deve haver aqui perto uma roseira florindo\ldots{}\\
Que harmonia por mim\ldots{} Que parecença com jardim\ldots{}\\
O meu corpo está são\ldots{} Minha alma foi"-se embora\ldots{}\\
E me deixou.
\end{verse}

\pagebreak
\addcontentsline{toc}{chapter}{Pela noite de barulhos espaçados\ldots{} \smallskip}
\section*{Pela noite de barulhos espaçados\ldots{}\break (junho~de~1929)}

\begin{verse}
Pela noite de barulhos espaçados,\\
Neste silêncio que me livra do momento\\
E acentua a fraqueza do meu ser fatigadíssimo,\\
Eu me aproximo de mim mesmo\\
No espanto ignaro com que a gente se chega pra morte.

Meu espírito ringe cruzado por dores sem nexo,\\
Numa dor unida, tão violentamente física,\\
Que me sinto feito um joelho que dobrasse.\\
A luz excessiva do estúdio desmancha a carícia do objeto,\\
Um frio de vento vem que me pisa tal qual um contato,\\
Tudo me choca, me fere, uma angústia me leva,\\
Estou vivendo ideias que por si já são destinos\\
E não escolho mais minhas visões.

A aparência é de calma, eu sei. Dir"-se"-ia que as nações vivem em paz\ldots{}\\
Há um sono exausto de repouso em tudo,\\
E uma cega esperança, cantando benditos, esmola\\
Em favor dos homens algum bem que não virá\ldots{}\\
Me sinto joelho. Há um arrependimento vasto em mim.\\
Eu digo que os séculos todos\\
Se atrasaram propositalmente no caminho,\\
Me esperaram, e puxo"-os agora como boi fatal.\\
Me sinto culpado de milhões de séculos desumanos\ldots{}\\
Milhões de séculos desumanos me fizeram, fizeram"-te, irmão;\\
E pela noite de barulhos espaçados\\
Não quero escutar o conselho que desce dos arranha"-céus do norte!\\
Eu sei que teremos um tempo de horror mais fecundo\\
Que as rapsódias da força e do dinheiro!

Será que nem uma arrebentação\ldots{}\\
Os postos isolados das cidades\\
Se responderão em alarmas raivacentos,\\
Saídos das casas iguais e da incúria dos donos da vida.\\
Havemos de ver muitos manos passando a fronteira,\\
Haverá pão grátis muito duvidoso,\\
As salas de improviso se encherão de discussões apaixonadas,\\
Mortas no dia seguinte em desastres que não sei quais.\\
Será tempo de esforço caudaloso,\\
Será humano e será também terribilíssimo\ldots{}\\
Só há"-de haver mulheres que não serão mais nossas mulheres.\\
Os piás hão"-de estar sem confiança catalogados na fila,\\
E os homens morrerão violentamente\\
Antes que chegue o tempo da velhice.
\end{verse}

\chapter[\textsc{poemas da amiga}\\I\ \ \ ``A tarde se deitava nos meus olhos'']{\textsc{poemas da amiga}\\(1929--1930)}

\begin{flushright}
\emph{a Jorge de Lima}
\end{flushright}

\section*{I}

\begin{verse}
A tarde se deitava nos meus olhos\\
E a fuga da hora me entregava abril,\\
Um sabor familiar de até"-logo criava\\
Um ar, e, não sei porque, te percebi.

Voltei"-me era apenas tua lembrança.\\
Estavas longe, doce amiga; e só vi no perfil da cidade\\
O arcanjo forte do arranha"-céu cor"-de"-rosa\\
Mexendo asas azuis dentro da tarde.
\end{verse}


\pagebreak
\addcontentsline{toc}{chapter}{\textsc{ii}\ \ ``Se acaso a gente se beijasse uma vez só\ldots{}''}
\section*{II}

\begin{verse}
Se acaso a gente se beijasse uma vez só\ldots{}\\
Ontem você estava tão linda\\
Que o meu corpo chegou.

Sei que era um riacho e duas horas de sede,\\
Me debrucei, não bebi.\\
Mas estou até agora desse jeito,\\
Olhando quatro ou cinco borboletas amarelas,\\
Dessas comuns, brincabrincando no ar.\\
Sinto um rumor\ldots{}
\end{verse}

\pagebreak
\addcontentsline{toc}{chapter}{\textsc{iii}\ \ ``Agora é abril, ôh minha doce amiga''}
\section*{III}

\begin{verse}
Agora é abril, ôh minha doce amiga,\\
Te reclinaste sobre mim, como a verdade,\\
Fui virar, fundeei o rosto no teu corpo.

Nos dominamos pondo tudo no lugar.\\
O céu voltou a ser por sobre a terra,\\
As laranjeiras ergueram"-se todas de"-pé\\
E nelas fizemos cantar um primeiro sabiá.

Mas a paisagem logo foi"-se embora\\
Batendo a porta, escandalizadíssima.
\end{verse}

\pagebreak
\addcontentsline{toc}{chapter}{\textsc{ix}\ \ ``Vossos olhos são um mate costumeiro''}
\section*{IX}

\begin{verse}
Vossos olhos são um mate costumeiro.\\
Vossas mãos são conselhos que é indiferente seguir.\\
Gosto da vossa boca donde saem as palavras isoladas\\
Que jamais não ouvi.\\
Porém o que eu adoro sobretudo é vosso corpo\\
Que desnorteia a vida e poupa as restrições.

Ôh, doce amiga! vossos castos espelhos de aurora\\
Despejam sobre mim paisagens e paisagens\\
Em que passeio feito um rei sem povo,\\
Cortejado por noruegas, caponetes e caminhos,\\
-- Os caminhos incompetentes que jamais não me conduzirão a alguém!\ldots{}\\
\end{verse}


\partepigraph{\emph{a Murilo Miranda}}{}
\part{\textsc{a costela do grã cão}}
\removeepigraph

\chapterspecial{Reconhecimento de Nêmesis}{(março de 1926)}{}

\begin{verse}
Mão morena dele pousa\\
No meu braço\ldots{} Estremeci.\\
Sou eu quando era guri\\
Esse garoto feioso.\\
Eu era assim mesmo\ldots{} Eu era\\
Olhos e cabelos só.\\
Tão vulgar que fazia dó.\\
Nenhuma fruta não viera\\
Madurando temporã.\\
Eu era menino mesmo,\\
Menino\ldots{} Cabelos só,\\
Que à custa de muita escova\\
E de muita brilhantina,\\
Me ondulavam na cabeça\\
Que nem sapé na lagoa\\
Se vem brisando a manhã.

É gente que não compreendo\\
Os saudosos do passado,\\
Nem os gratos\ldots{} Relembrança\\
Porta muito raramente\\
Nos olhos dos ocupados.\\
Por isso enxergo sem gosto\\
A casa da minha infância,\\
Casão meio espandongado\\
Onde meu pai se acabou.\\
Só mesmo o que é bem de agora\\
Possui direito de lágrima,\\
Sofrer\ldots{} pois sim, mas lutando\\
Pela replanta brotando,\\
Sofrer sim, mas porém nunca\\
Sofrer puxando memória\\
Pelo café que secou.

No entanto quando sucede\\
Mais braba a vileza humana\\
Arranhar na minha porta,\\
Não sei porque o curumim\\
Que eu já fui, surge e se bota\\
Assim rentinho de mim.\\
Será que é um anjo"-da"-guarda?\ldots{}\\
Não sei não\ldots{} Creio que não.\\
Ele faz que não me enxerga,\\
Que não me conhece\ldots{} Mão\\
Morena sempre pousando\\
No meu ombro, aluada muito!\\
Até o menino inteirinho\\
É que nem cousa perdida\\
E não dá tento de si.\\
Possui a vida sem vida\\
Das sombras. É assombração.

Remexe por todo o quarto,\\
Não desloca nenhum traste,\\
Se vê bem que não faz parte\\
Do grupo dos meus amigos\ldots{}\\
Volta"-e"-meia vem e pousa\\
No meu braço a mão morena\ldots{}\\
É um silêncio atravessando\\
O corpo manso das cousas.

Eu também se o reconheço\\
É só porque sofro agreste,\\
E embora grudando a vista\\
No livro, eu faça de conta\\
Que não reparo no tal,\\
Minha alma espia o menino\\
Enquanto a vista devora\\
Uma sopa de aletria\\
Feita de letras malucas.\\
Mas ele não vai"-se embora,\\
E o vulto do curumim,\\
Sem piedade, me recorda\\
A minha presença em mim.

Só isso. E por causa disso\\
Não posso fugir de mim!\\
Não posso ser como os outros!\\
Riso não pega de enxerto,\\
Ser mau carece raiz\ldots{}\\
E confessando que sofro,\\
Não sei se é pela coragem,\\
Mas tenho como uma aragem\\
E fico bem mais feliz.\\
Menino, tu me recordas\\
A minha presença em mim!

\ldots{} A primeira vez que veio,\\
Tive uma alegria enorme,\\
Gostei de ver que já era\\
Bem mais taludo e mais forte\\
Que em pequeno e que possuía\\
Uma alma aquecida pelo\\
Fogo humano do universo.\\
Segunda vez me irritou.\\
Fui covarde, fui perverso,\\
Peguei no tal, lhe ensinei\\
A indecente dança"-do"-ombro.\\
Não quis saber, foi"-se embora.\\
E quando não o vi mais,\\
Sozinho, me arrependi.\\
A terceira vez é agora\\
E eu\ldots{} não sei\ldots{} não gosto dele\\
Mas não quero que o rapaz\\
Me deixe sozinho aqui.\\
Não danço mais dança"-do"-ombro!\\
Eu reconheço que sofro!

Ah! malvadeza brutaça\\
Dos indivíduos humanos,\\
Dos humanos desta praça!\\
Ah! homens filhos"-da"-puta,\\
Gente bem ruim, bem odiando,\\
Homens bem homens, grandiosos\\
Na sua inveja acordada!\\
Grandiosos na força bruta,\\
Na estupidez desvelada!\\
Que heroísmo sem inocência,\\
O do sujeito esquecendo\\
Do remorso e da consciência!\\
Ôh! força reta, bem homem,\\
De ser talqualmente os mares,\\
E os movimentos do mundo!\\
Perversidades solares\\
Da magrém! ser matapau!\\
Sucuri, raio, minuano!\\
Forçura destes humanos,\\
Iguais na perversidade,\\
Iguais na imbecilidade,\\
Na calúnia, iguais no ciúme!\ldots{}\\
Conscientemente implacáveis!\\
Imperiais no riso mau!\ldots{}\\
Ota, cabra demográfico,\\
Jornaleiro do azedume,\\
Secreção de baço podre,\\
Alma em que a sífilis deu!\\
Burrice gorda, indiscreta,\\
Veneranda\ldots{} \emph{Homo imbecillis},\\
Invejado pelo poeta\ldots{}\\
Viva piolho"-de"-galinha!\\
Êh! homem, bosta de Deus!

Menino, sai! Eu te odeio,\\
Menino assombrado, feio,\\
Menino de mim, menino,\\
Menino trelento, que enches\\
Com teus silêncios puríssimos\\
A bulha dos meus desejos,\\
Que nem a calma da tarde\\
Vence a bulha da cidade\ldots{}\\
Menino mau, que me impedes\\
De entrar também pro recheio\\
Das estatísticas\ldots{} sai!\\
Menino vago, sem nome,\\
Que me embebes inteirinho\\
Nesta amargura visguenta\\
Pelos homens! pelos homens!\ldots{}

Puxa! rapazes, minha alma,\\
Comprida que não se acaba,\\
Está negra tal"-e"-qual\\
Fruta seca de goiaba!\\
Meus olhos tão gostadores\\
Nem têm mais gosto de olhar!

E pela primeira vez\\
O murmurejo natal\\
Desta vida está sem graça,\\
E eu só desejo uma calma\\
Que apagasse até meus ais!\\
Tudo amarga porque os homens\\
Me amargaram por demais!\\
Uma tristeza profunda,\\
Uma fadiga profunda,\\
E até, miseravelmente,\\
O projeto inconfessável\\
De parar\ldots{}

\qquad\qquad\qquad\qquad\qquad Menino, sai!\\
Você é o estranho periódico\\
Que me separa do ritmo\\
Unânime desta vida\ldots{}\\
E o que é pior, você relembra\\
Em mim o que geralmente\\
Se acaba ao primeiro sopro:\\
Você renova a presença\\
De mim em mim mesmo\ldots{} E eu sofro.

É tarde. Vamos dormir.\\
Amanhã escrevo o artigo,\\
Respondo cartas, almoço,\\
Depois tomo o bonde e sigo\\
Para o trabalho\ldots{} Depois\ldots{}\\
Depois o mesmo\ldots{} Depois,\\
Enquanto fora os malévolos\\
Se preocupam com ele,\\
Vorazes feito caprinos,\\
Nesta rua Lopes Chaves\\
Terá um homem concertando\\
As cruzes do seu destino.
\end{verse}

\chapterspecial{Toada}{(1932)}{}

\begin{verse}
No outro lado da cidade,\\
Não sei o quê, foi o vento,\\
O vento me dispersou.

Viajei por terras estranhas\\
Entre flores espantosas,\\
Tive coragem pra tudo\\
No outro lado da cidade,\\
Sem tomar cuidado em mim.\\
Passeava com tais perícias,\\
Punha girafas na esquina,\\
Quantos milagres na viagem,\\
Meu coração de ninguém!\\
E pude estar sem perigo\\
Por entre aconchegos pagos,\\
Em que o carinho mais velho\\
Inda guardava agressão.\\
Busquei São Paulo no mapa,\\
Mas tudo, com cara nova,\\
Duma tristeza de viagem,\\
Tirava fotografia\ldots{}\\
E o meu cigarro na tarde\\
Brilhava só, que nem Deus.\\
Fiquei tão pobre, tão triste\\
Que até meu olhar fechou.

No outro lado da cidade\\
O vento me dispersou.
\end{verse}

\chapter[\textsc{grã cão do outubro}\\\textsc{ii}\ \ Os gatos]{\textsc{grã cão do outubro}}
\hedramarkboth{os gatos}{}

\section*{II\break Os gatos}

\bigskip

\subsection{\textsc{(a)}\break (14 de outubro de 1933)}

\begin{verse}
Que beijos que eu dava\ldots{}\\
Não tigre, vossa boca é mesmo que um gato\\
Imitando tigre.\\
Boca rajada, boca rasgada de listas,\\
De preto, de branco,\\
Boca hitlerista,\\
Vossa boca é mesmo que um gato.

Nas paredes da noite estão os gatos.\\
Têm garras, têm enormes perigos\\
De exércitos disfarçados,\\
Milhares de gatos escondidos por detrás da noite incerta.\\
Irão estourar por aí de repente,\\
Já estão com mil rabos além de São Paulo,\\
Nem sei mais se são as fábricas que miam\\
Na tarde desesperada.

Penso que vai chover sobre os amores dos gatos.\\
Fugirão?\ldots{} e só eu no deserto das ruas,\\
Oh incendiária dos meus aléns sonoros,\\
Irei buscando a Vossa boca,\\
Vossa boca hitlerista,\\
Vossa boca mais nítida que o amor,\\
Ai, que beijos que eu dava\ldots{}\\
Guardados na chuva\ldots{}\\
Boiando nas enxurradas\\
Nosso corpo de amor\ldots{}\\
Que beijos, que beijos que eu dou!

Vamos enrolados pelas enxurradas\\
Em que boiam corpos, em que boiam os mortos,\\
Em que vão putrefatos milhares de gatos\ldots{}\\
Das casas cai mentira,\\
Nós vamos com as enxurradas,\\
Com a perfeita inocência dos fenômenos da terra,\\
Voluptuosamente mortos,\\
Os sem ciência mais nenhuma de que a vida\\
Está horrenda, querendo ser, erguendo os rabos\\
Por trás da noite, em companhia dos milhões de gatos verdes.
\end{verse}

\medskip

\subsection{\textsc{(b)}\break(15 de outubro de 1933)}


\begin{verse}
Me pus amando os gatos loucamente,\\
Ôh China!\\
Mas agora porém não são gatos tedescos,\\
Tudo está calmo em plena liberdade,\\
Se foram as volúpias e as perversões tão azedas,\\
Eu sou cravo, tu és rosa,\\
Tu és minha rosa sincera,\\
És odorante, és brasileira à vontade,\\
Feito um prazer que chega todo dia.

Mas eu te cresço em meu desejo,\\
Ai, que vivo arrasado de notícias!\\
Murmurando com medo ao teu ouvido:\\
Ôh China! ôh minha China!\ldots{}

Tu te gastas sob o meu peso bom,\\
Teus lábios estão alastrados na abertura do reconhecimento,\\
Teus olhos me olham, me procuram todo\ldots{}

Mas eu insisto em meu castigo, ôh China.

Como um gato chinês criado através de séculos de posse e de aproveitamentos,\\
Para meu gozo só, pra meu enfeite só de mim,\\
Pra mim, pra mim, tu foste feita, ôh China!\\
Estou te saboreando, és gato china que apanhei vagamundo na rua,\\
Ôh China! ôh minha triste China,\\
Estarei pesando, te fazendo pesar sem motivo,\\
Estou\ldots{} estava, ôh minha triste sina,\\
Até que fui guardar nos teus cabelos perdidos\\
Lágrima que não pude sem chorar.
\end{verse}

\pagebreak
\addcontentsline{toc}{chapter}{\textsc{iv}\ \ Poema tridente}
\section*{IV\break Poema tridente \break (outubro de 1933)}

\begin{verse}
Vosso corpo seria encontrado nos desertos.\\
Sois tão linda\ldots{} você é a Lei!\\
Você é tão mal contrária a essas mil leis humanas\\
Que avançam cegas insensíveis sobre o horror\ldots{}\\
Você é tal"-e"-qual, bem polida,\\
Sem erros, cadencial.

Ôh besta fera maldita,\\
Você é mas é um braço esfomeado terminando em faísca de gládio,\\
Caindo aqui, varrendo além,\\
Voando, cego braço, aterrissando no meio das turbas,\\
Matando gente, depredando gente, inventando orfanatos,\\
Bandos de caravanas de leprosos,\\
Exílios pra judeus, pra paulistas, pra estudantada cubana,\\
Eu te amo de um amor educado no inferno!\\
Te mordo no peito até o sangue escorrer\\
Me dando socos, chorando, chamando de bruto, de cão,\\
O Grã Cão é o Mildiabo educado sozinho no inferno!

Nos debatemos, o braço esfomeado braceja,\\
Golpeia aqui, matou centenas de operários,\\
Queima cafezais, trigais, canaviais, desocupados,\\
Quebra os museus grandiosos,\\
Usa a lei de fugir pra estudantada cubana.

E no esforço sobrosso colhendo com o gládio o subsolo da Europa,\\
Abaixo os tiranos! abaixo Afonso \textsc{xiii}!\\
O mar fez maremoto, e convulsivos\\
Nos odiando no mesmo abraço confundidos,\\
Eleitos, desesperados na febre de amar,\\
Jorramos em lucilações fantásticas tremendas,\\
Todo o nosso ardor vai se esgotar na seiva!\\
Você é lindíssima! É polida e cadencial feito uma lei!\\
Mas eu sou o Grã Cão que te marquei um bocado com o crime dos mundos!\\
E agora nem de perdão carecemos\\
No mesmo abraço desaparecidos.
\end{verse}

\pagebreak
\addcontentsline{toc}{chapter}{\textsc{v}\ \ Dor}
\section*{V\break Dor\break (15 de outubro de 1933)}

\begin{verse}
A cidade está mais agitada a meidia.\\
As ruas devastam minha virgindade\\
E os cidadãos talvez marquem encontro nos meus lábios.\\
Minha boca é o peixe macho e derramo núcleos de amor pelas ruas.\\
Que irão fecundar os ovários da vida algum dia.

Eu venho das altas torres, venho dos matos alagados,\\
Com meus passos conduzidos pelo fogo do Grã Cão!\\
Mas pra viver na cidade de São Paulo escondi na corrente de prata\\
A inútil semente do milho, a maniva,\\
E enroupei de acerba seda o arlequinal do meu dizer\ldots{}

E agora apontai"-me, janelas do Martinelli,\\
Calçadas, ruas, ruas, ladeiras rodantes, viadutos,\\
Onde estão os judeus de consciência lívida?\\
Os tortuosos japoneses que flertam São Paulo?\\
Os ágeis brasileiros do Nordeste? os coloridos?\\
Onde estão os coloridos italianos? onde estão os turcomanos?\\
Onde estão os pardais, madame la Françoise,\\
Ergo, ego, Ega, égua, água, iota, calúnia e notícias,\\
Balouçantes nas marquesas dos roxos arranha"-céus?\ldots{}

Não vos trago a fala de Jesus nem o escudo de Aquiles,\\
Nem a casinha pequenina ou a sombra do jatobá.\\
Tudo escondi no caminho da corrente de prata.\\
Mas eu venho das altas torres trazido ao facho do Grã Cão,\\
Lábios, lábios para o encontro em que cantareis fatalmente,\\
Ameaçados pela fome que espia detrás da coxilha,\\
A dor, a caprichosa dor desocupada que desde milhões de existências\\
Busca a razão de ser.
\end{verse}

\chapterspecial{Quarenta anos}{(27 de dezembro de 1933)}{}

\begin{verse}
A vida é para mim, está se vendo,\\
Uma felicidade sem repouso;\\
Eu nem sei mais se gozo, pois que o gozo\\
Só pode ser medido em se sofrendo.

Bem sei que tudo é engano, mas sabendo\\
Disso, persisto em me enganar\ldots{} Eu ouso\\
Dizer que a vida foi o bem precioso\\
Que eu adorei. Foi meu pecado\ldots{} Horrendo

Seria, agora que a velhice avança,\\
Que me sinto completo e além da sorte,\\
Me agarrar a esta vida fementida.

Vou fazer do meu fim minha esperança,\\
Oh sono, vem!\ldots{} Que eu quero amar a morte\\
Com o mesmo engano com que amei a vida.
\end{verse}

\chapterspecial{Momento}{(abril de 1937)}{}

\begin{verse}
O vento corta os seres pelo meio.\\
Só um desejo de nitidez ampara o mundo\ldots{}\\
Faz sol. Fez chuva. E a ventania\\
Esparrama os trombones das nuvens no azul.

Ninguém chega a ser um nesta cidade,\\
As pombas se agarram nos arranha"-céus, faz chuva.\\
Faz frio. E faz angústia\ldots{} É este vento violento\\
Que arrebenta dos grotões da terra humana\\
Exigindo céu, paz e alguma primavera.
\end{verse}

\chapterspecial{Brasão}{(10 de dezembro de 1937)}{}

\begin{verse}
Vem a estrela dos treze bicos,\\
Brasil, Coimbra, Guiné, Catalunha,\\
E mais a Bruges inimaginável\\
E a decadência dos Almeidas.

E sobre a estrela dos treze bicos\\
Pesa um coração mole\\
De prata coticada trezemente,\\
Em cujo campo há"-de inscrever"-se\\
``Eu sou aquele que veio do imenso rio''.

E sobre o campo do meu coração,\\
Todo em zarcão ardendo,\\
Há em ouro a arca de Noé com vinte"-e"-nove bichos blau,\\
E a jurema esfolhando as folhas derradeiras\\
Sobre Mestre Carlos, o meu grande sinal.

E a seguir a trombeta, essa trombeta\\
Insiste pela Catalunha,\\
Mas desta vez eu que escolhi!\\
Ôh, meus amigos,\\
Perdão pelos séculos pesados de cicatrizes infinitas,\\
Perdão por todas as sabedorias,\\
Pela esfera armilar das conquistas insanas!\\
Essa trombeta eu que escolhi, toda de prata,\\
Com treze línguas de fogo na assustadora boca,\\
E a inscrição ``Que"-dele eles?'',\\
Eles, os bandeirantes\ldots{}\\
E falta o boi Paciência, o boi que pertence a Armida.\\
Traz por guampas os cornos da luna\\
E um peitoral de turmalinas.\\
Mas esse vem no outro coração mole,\\
Não se mostra a ninguém.\\
O boi Paciência serão treze preguiças assustadas,\\
No porto do imenso rio esperando,\\
Esperando pelos treze caminhos\\
Das mil cavernas das quarentas mil perguntas.

Ai, que eu vou me calar agora,\\
Não posso, não posso mais!
\end{verse}

\chapterspecial{Canção}{(Rio de Janeiro, 22 de dezembro de 1940)}{}


\begin{verse}
\ldots{} de árvores indevassáveis\\
De alma escusa sem pássaros\\
Sem fonte matutina\\
Chão tramado de saudades\\
À eterna espera da brisa,\\
Sem carinhos\ldots{} como me alegrarei?

Na solidão solitude,\\
Na solidão entrei.

Era uma esperança alada,\\
Não foi hoje mas será amanhã,\\
Há"-de ter algum caminho\\
Raio de sol promessa olhar\\
As noites graves do amor\\
O luar a aurora o amor\ldots{} que sei!

Na solidão solitude,\\
Na solidão entrei,\\
Na solidão perdi"-me\ldots{}

O agouro chegou. Estoura\\
No coração devastado\\
O riso da mãe"-da"-lua,\\
Não tive um dia! uma ilusão não tive!\\
Ternuras que não me viestes\\
Beijos que não me esperastes\\
Ombros de amigos fiéis\\
Nem uma flor apanhei.

Na solidão solitude,\\
Na solidão entrei,\\
Na solidão perdi"-me,\\
Nunca me alegrarei.
\end{verse}

\part{\textsc{livro azul}}

\chapterspecial{Rito do irmão pequeno}{(1931)}{}

\begin{flushright}
\emph{a Manuel Bandeira}
\end{flushright}

\section*{I}

\begin{verse}
Meu irmão é tão bonito como o pássaro amarelo,\\
Ele acaba de nascer do escuro da noite vasta!\\
Meu irmão é tão bonito como o pássaro amarelo,\\
Eu sou feito um ladrão roubado pelo roubo que leva,\\
Neste anseio de fechar o sorriso da boca nascida\ldots{}

Gentes, não creiam não que em meu canto haja sequer um reflexo de vida!\\
Ôh não! antes será talvez uma queixa de espírito sábio,\\
Aspiração do fruto mais perfeito,\\
Ou talvez um derradeiro refúgio para minha alma humilhada\ldots{}

Me deixem num canto apenas, que seja este canto somente,\\
Suspirar pela vida que nasceria apenas do meu ser!\\
Porque meu irmão pequeno é tão bonito como o pássaro amarelo,\\
E eu quisera dar pra ele o sabor do meu próprio destino\\
A projeção de mim, a essência duma intimidade incorruptível!\ldots{}
\end{verse}

\medskip
\section*{II}

\begin{verse}
Vamos caçar cotia, irmão pequeno,\\
Que teremos boas horas sem razão,\\
Já o vento soluçou na arapuca do mato\\
E o arco"-da"-velha já engoliu as virgens.

Não falarei uma palavra e você estará mudo,\\
Enxergando na ceva a Europa trabalhar;\\
E o silêncio que traz a malícia do mato,\\
Completará o folhiço, erguendo as abusões.

E quando a fadiga enfim nos livrar da aventura,\\
Irmão pequeno, estaremos tão simples, tão primários,\\
Que os nossos pensamentos serão vastos,\\
Graves e naturais feito o rolar das águas.
\end{verse}

\medskip
\section*{III}

\begin{verse}
Irmão pequeno, sua alma está adejando no seu corpo,\\
E imagino nas borboletas que são efêmeras e ativas\ldots{}\\
Não é assim que você colherá o silêncio do enorme sol branco,\\
O ferrão dos carapanãs arde em você reflexos que me entristecem.

Assim você preferirá visagens, o progresso\ldots{}\\
Você não terá paz, você não será indiferente,\\
Nem será religioso, você\ldots{} ôh você, irmão pequeno,\\
Vai atingir o telefone, os gestos dos aviões,\\
O norte"-americano, o inglês, o arranha"-céu!\ldots{}

Venha comigo. Por detrás das árvores, sobrado dos igapós,\\
Tem um laguinho fundo onde nem medra o grito do cacauê\ldots{}\\
Junto à tocaia espinhenta das largas vitórias"-régias,\\
Boiam os paus imóveis, alcatifados de musgo úmido, com calor\ldots{}

Matemos a hora que assim mataremos a terra e com ela\\
Estas sombras de sumaúmas e violentos baobás,\\
Monstros que não são daqui e irão se arretirando.\\
Matemos a hora que assim mataremos as sombras sinistras,\\
Esta ambição de morte, que nos puxa, que nos chupa,\\
Guia da noite,\\
Guiando a noite que canta de uiara no fundo do rio.
\end{verse}

\medskip
\section*{IV}

\begin{verse}
Deixa pousar sobre nós dois, irmão pequeno,\\
A sonolência desses enormes passados;\\
E mal se abra o descuido ao rolar das imagens,\\
A chuva há"-de cair, auxiliando as enchentes.

Sob a jaqueira no barranco ao pé da sombra\\
As pedras e as raízes sossegadas apodrecem.\\
Havemos de escutar o som da fruta caindo n'água,\\
E perceber em toda essa fraca indigência,\\
A luminosa vaga imperecível lentidão.
\end{verse}

\medskip
\section*{V}

\begin{verse}
Há o sarcástico predomínio das matérias\\
Com seu enorme silêncio sufocando os espíritos do ar\ldots{}\\
Será preciso contemplá"-las, e a paciência,\\
Irmão pequeno, é que entreabre as melhores visões.

Nos dias em que o sol exorbita esse branco\\
Que enche as almas e reflete branqueando a solidão da ipueira,\\
Havemos de sacrificar os bois pesados.\\
O sangue lerdo escorre das marombas sobre a água do rio,\\
E catadupa reacendido o crime das piranhas.

Só isso deixará da gente o mundo tão longínquo\ldots{}\\
As nossas almas se afastam escutando o segredo parvo,\\
E o branco penetra em nós que nem a inexistência incomparável.
\end{verse}

\medskip
\section*{VI}

\begin{verse}
Chora, irmão pequeno, chora,\\
Porque chegou o momento da dor.\\
A própria dor é uma felicidade\ldots{}

Escuta as árvores fazendo a tempestade berrar.\\
Valoriza contigo bem estes instantes\\
Em que a dor, o sofrimento, feito vento,\\
São consequências perfeitas\\
Das nossas razões verdes,\\
Da exatidão misteriosíssima do ser.

Chora, irmão pequeno, chora,\\
Cumpre a tua dor, exerce o rito da agonia.\\
Porque cumprir a dor é também cumprir o seu próprio destino:\\
É chegar àquela coincidência vegetal\\
Em que as árvores fazem a tempestade berrar,\\
Como elementos da criação, exatamente.
\end{verse}

\medskip
\section*{VII}

\begin{verse}
O acesso já passou. Nada trepida mais e uma acuidade gratuita\\
Cria preguiças nos galhos, com suas cópulas lentíssimas.\\
Volúpia de ser a blasfêmia contra as felicidades parvas do homem\ldots{}\\
São deuses\ldots{}\\
Mas nós blefamos esses deuses desejosos de futuro,\\
Nós blefamos a punição europeia dos pecados originais.

Ouça. Por sobre o mato, encrespado nas curvas da terra,\\
Por aí tudo, o calor anda em largado silêncio,\\
Ruminando o murmulho do rio, como um frouxo cujubim.

Na vossa leve boca o suspiro gerou uma abelha.\\
É o momento, surripiando mel pras colmeias da noite incerta.
\end{verse}

\medskip
\section*{VIII}

\begin{verse}
O asilo é em pleno mato, cercado de troncos negros\\
Em que a água deixa um ólio eterno e um som,\\
Só uma picada fere a terra e leva ao porto,\\
Onde entre moscas jaz uma pele de uiara a secar.

As maqueiras se abanam com lerdeza,\\
Enquanto à voz do cotcho uma toada se esvai.\\
Ela foi embora e nós ficamos. Não há nada.\\
Nem a inquieta visão dessa curiosidade que se foi.
\end{verse}

\medskip
\section*{IX}

\begin{verse}
A cabeça desliza com doçura,\\
E nas pálpebras entrecerradas\\
Vaga uma complacência extraordinária.

É pleno dia. O ar cheira a passarinho.\\
O lábio se dissolve em açúcares breves,\\
O zumbido da mosca embalança de sol.\\
\ldots{} Assurbanipal\ldots{}\\
A alma, à vontade,\\
Se esgueira entre as bulhas gratuitas,\\
Deixa a felicidade ronronar.

Vamos, irmão pequeno, entre palavras e deuses,\\
Exercer a preguiça, com vagar.
\end{verse}

\medskip
\section*{X}

\begin{verse}
A enchente que cava margem,\\
Roubou os barcos do porto,\\
A água brota em nosso joelho\\
Delícias de solidão.

Trepados na castanheira\\
Viveremos sossegados\\
Enquanto a terra for mar;\\
Pauí"-Pódole virá\\
Nas horas de Deus trazer\\
A estrela, a umidade, o aipim.

E quando a terra for terra,\\
Só nós dois, e mais ninguém,\\
De mim nascerão os brancos,\\
De você, a escuridão.
\end{verse}

\chapterspecial{Girassol da madrugada}{(1931)}{}

\begin{flushright}
\emph{a \textsc{r.\,g.}}\footnote[*]{Em carta de 17 de junho de 1941, Mário de
  Andrade afirma: ``Você vai ter uma surpresa desagradável, mas tive
  mesmo que mudar definitivamente a dedicatória do ``\textsc{Girassol da
  Madrugada''}. Tenha paciência mas não posso mesmo dedicar esse poema
  senão a quem o inspirou. Tanto mais que se puser o R. G. das iniciais,
  há duas cartas minhas a amigos que poderão futuramente identificar
  essas letras. Não sei ainda se porei as iniciais ou deixo o poema sem
  dedicatória. Mas decididamente não posso dedicar esses versos a outra
  pessoa, me causa transtorno psicológico muito desagradável'' (V.
  \textsc{antelo}, Raúl, org. \emph{Cartas de Mário de Andrade a Murilo Miranda
  (1934--1945)}. Rio de Janeiro: Nova Fronteira, 1981, p. 84).}
\end{flushright}

\section*{I}

\begin{verse}
De uma cantante alegria onde riem"-se as alvas uiaras\\
Te olho como se deve olhar, contemplação,\\
E a lâmina que a luz tauxia de indolências\\
É toda um esplendor de ti, riso escolhido no céu.

Assim. Que jamais um pudor te humanize. É feliz\\
Deixar que o meu olhar te conceda o que é teu,\\
Carne que é flor de girassol! sombra de anil!\\
Eu encontro em mim mesmo uma espécie de abril\\
Em que se espalha o teu sinal, suave, perpetuamente.
\end{verse}

\medskip
\section*{II}

\begin{verse}
Diga ao menos que nem você quer mais desses gestos traiçoeiros\\
Em que o amor se compõe feito uma luta;\\
Isso trará mais paz, porquanto o caminho foi longo,\\
Abrindo o nosso passo através dos espelhos maduros.

Você não diz, porém o vosso corpo está delindo no ar,\\
Você apenas esconde os olhos no meu braço e encontra a paz na escuridão..\\
A noite se esvai lá fora serena sobre os telhados,\\
Enquanto o nosso par aguarda, soleníssimo,\\
Radiando luz, nesse esplendor dos que não sabem mais pra onde ir.
\end{verse}

\medskip
\section*{III}

\begin{verse}
Se o teu perfil é puríssimo, se os teus lábios\\
São crianças que se esvaecem no leite,\\
Se é pueril o teu olhar que não reflete por detrás,\\
Se te inclinas e a sombra caminha na direção do futuro:

Eu sei que tu sabes o que eu nem sei se tu sabes,\\
Em ti se resume a perversa e imaculada correria dos fatos,\\
És grande por demais para que sejas só felicidade!\\
És tudo o que eu aceito que me sejas\\
Só pra que o sono passe, e me acordares\\
Com a aurora incalculavelmente mansa do sorriso.
\end{verse}

\medskip
\section*{IV}

\begin{verse}
Não abandonarei jamais de"-noite as tuas carícias,\\
De"-dia não seremos nada e as ambições convulsivas\\
Nos turbilhonarão com as malícias da poeira\\
Em que o sol chapeará torvelins uniformes.

E voltarei sempre de"-noite às tuas carícias,\\
E serão búzios e bumbas e tripúdios invisíveis\\
Porque a Divindade muito naturalmente virá.\\
Agressiva Ela virá sentar em nosso teto,\\
E seus monstruosos pés pesarão sobre nossas cabeças,\\
De"-noite, sobre nossas cabeças inutilizadas pelo amor.
\end{verse}

\medskip
\section*{V}

\begin{verse}
Teu dedo curioso me segue lento no rosto\\
Os sulcos, as sombras machucadas por onde a vida passou.\\
Que silêncio, prenda minha\ldots{} Que desvio triunfal da verdade,\\
Que círculos vagarosos na lagoa em que uma asa gratuita roçou\ldots{}

Tive quatro amores eternos\ldots{}\\
O primeiro era a moça donzela,\\
O segundo\ldots{} eclipse, boi que fala, cataclisma,\\
O terceiro era a rica senhora,\\
O quarto és tu\ldots{} E eu afinal me repousei dos meus cuidados.
\end{verse}

\medskip
\section*{VI}

\begin{verse}
Os trens"-de"-ferro estão longe, as florestas e as bonitas cidades,\\
Não há senão Narciso entre nós dois, lagoa,\\
Já se perdeu saciado o desperdício das uiaras,\\
Há só meu êxtase pousando devagar sobre você.

Ôh que pureza sem impaciência nos calma\\
Numa fragrância imaterial, enquanto os dois corpos se agradam,\\
Impossíveis que nem a morte e os bons princípios.\\
Que silêncio caiu sobre a vossa paisagem de excesso dourado!\\
Nem beijo, nem brisa\ldots{} Só, no antro da noite, a insônia apaixonada\\
Em que a paz interior brinca de ser tristeza.
\end{verse}

\medskip
\section*{VII}

\begin{verse}
A noite se esvai lá fora serena sobre os telhados\\
Num vago rumor confuso de mar e asas espalmadas,\\
Eu, debruçado sobre vossa perfeição, num cessar ardentíssimo,\\
Agora pouso, agora vou beber vosso olhar estagnado, ôh minha lagoa!

Eis que ciumenta noção de tempo, tropeçando em maracás,\\
Assusta guarás, colhereiras e briga com os arlequins,\\
Vem chegando a manhã. Porém, mais compacta que a morte,\\
Para nós é a sonolenta noite que nasce detrás das carícias esparsas.

Flor! flor!\ldots{}\\
\qquad\qquad Graça dourada!\ldots{}\\
\qquad\qquad\qquad\qquad\qquad\quad Flor\ldots{}
\end{verse}

\chapterspecial{O grifo da morte}{(1933)}{}

\begin{flushright}
\emph{a Lúcio Rangel}
\end{flushright}

%I -- ``Milhões de rosas''
\section*{I}

\begin{verse}
Milhões de rosas\\
Para esta grave\\
Melancolia,\\
Milhões de rosas,\\
Milhões de castigos\ldots{}

Milhões de castigos,\\
Imperfeita grávida,\\
Quem foi? foi o vento\\
Que fez"-te imperfeita,\\
Milhões de aratacas!

A toca fendeu\\
Para esta grave\\
Melancolia,\\
Milhões de castigos,\\
Milhões de aratacas\ldots{}

Salta o bicho roxo.\\
Depois ficou ruim,\\
Depois ficou roxo,\\
Depois ficou ruim,\\
Depois ficou roxo,\\
Ruim"-roxo, ruim"-roxo,\\
Milhões de bandeiras!

Os camisas pretas,\\
Os camisas pardas,\\
Os camisas roxas,\\
Ruim"-roxo, ruim"-roxo,\\
Milhões de bandeiras!\\
Milhões de castigos!\\
Quem foi? foi a rosa\\
Dos ventos da amarga\\
Desesperança\ldots{}

Ei"-vem a morte\\
-- ruim"-roxo\ldots{} --\\
Consoladora\ldots{}\\
Milhões de rosas,\\
Milhões de castigos\ldots{}
\end{verse}

\pagebreak
%IV -- ``}Quando o rio Madeira''
\section*{IV}

\begin{verse}
Quando o rio Madeira\\
Fica inavegável,\\
A corredeira clara\\
Junto ao trem"-de"-ferro\\
Vai rasa entre as pedras\\
Da margem deserta,\\
Suspensa no charco\\
Imenso da morte.

A claridade vasta\\
Guasca Mato Grosso,\\
Filtrada da nuvem\\
Que de tão exausta\\
Se apoia na crista\\
De espuma do rio.

O calor mais branco\\
Esturrica as pedras\\
E tange o Grão Chaco\\
Pros altos dos Andes,\\
Onde as almas planam\\
Sem fecundidade,\\
Na terra sem mal,\\
Sem fecundidade.
\end{verse}

\pagebreak
%V -- ``}Silêncio monótono,''
\section*{V}

\begin{verse}
Silêncio monótono,\\
Calma serenata\\
Na monotonia,\\
A alma sem tristeza\\
Pouco a pouco vai\\
Desabrochando\\
O instante do lago.

\qquad\emph{Morte, benfeitora morte,}\\
\qquad\emph{Eu vos proclamo }\\
\qquad\emph{Benfeitora, ôh morte! }\\
\qquad\emph{Benfeitora morte! }\\
\qquad\emph{Morte, morte\ldots{}}

Se escuta no fundo\\
A sombra das águas\\
-- calma serenata\ldots{} --\\
Se depositando\\
Para nunca mais.
\end{verse}

\partepigraph{\emph{a Carlos Lacerda}}{}
\part[o carro da miséria]{\textsc{o carro da miséria}\\(24 de dezembro de 1930, 11 de outubro de 1932, 26 de dezembro de 1943)}
\removeepigraph

%I -- ``O que que vêm fazer pelos meus olhos tantos barcos''
\section*{I}

\begin{verse}
O que que vêm fazer pelos meus olhos tantos barcos\\
Lenços rompendo adeuses presentinhos\\
Charangas na terra"-roxa das estações um grito\\
Um grito não um gruto\\
Que me faz esquecer a miséria do mundo pão pão\ldots{}

O que que vem fazer na minha boca um beijo\\
A mulher da Bolívia agarrando\\
Um penacho de viúvas restritas\\
Restritas não restrutas\\
Que o papagallo repassa e põe na vida\ldots{}

Ah\ldots{} caminhos caminhos caminhos errados de séculos\ldots{}\\
Me sinto o Pai Tietê. Dos meus sovacos\\
Saem fantasmas bonitões pelos caminhos\\
Penetrando o esplendor falso da América.

Dei"-vos minas de ouro vós me dais mineiros!\\
Glória a Cícero nas vendinhas alterosas\\
Com a penugem dos pensamentos sutis\\
Feito ninho de guaxe\\
O passado atrapalha os meus caminhos\\
Não sou daqui venho de outros destinos\\
Não sou mais eu nunca fui eu decerto\\
Aos pedaços me vim -- eu caio! -- aos pedaços disperso\\
Projetado em vitrais nos joelhos nas caiçaras\\
Nos Pireneus em pororoca prodigiosa\\
Rompe a consciência nítida: \textsc{eu tudoamo.}

Ora vengan los zabumbas\\
Tudoamarei! Morena eu te tudoamo!\\
Destino pulha alma que bem cantaste\\
Maxixa agora samba o coco\\
E te enlambuza na miséria nacionar.
\end{verse}

\pagebreak
%II -- ``Meu baralho dois ouros''
\section*{II}

\begin{verse}
\qquad\qquad\quad Meu baralho dois ouros\\
\qquad\qquad\quad Eu não quero mais jogar\\
\qquad\qquad\quad Meu baralho dois ouros\\
\qquad\qquad\quad Eu não quero mais jogar.

E diz o prinspo\\
Sangue"-azul louro perneta\\
Ontem me deu na veneta\\
Fui na venda pra jogar\\
Joguei no sangue\\
Companheiro de aventura\\
Mas o sangue se depura\\
Está na moda depurar.

\qquad\qquad\quad Meu baralho dois ouros\\
\qquad\qquad\quad Eu não quero mais jogar.

E diz o sangue\\
Rebolando a raça fina\\
Tintinabulem tintinas\\
Que eu vou jogar no ariano\\
Mai' não me assustem\\
Que num mês viro paulista\\
Ganho bem suspendo a crista\\
E tenho quatrocentos anos.

\qquad\qquad\quad Meu baralho dois ouros\\
\qquad\qquad\quad Eu não quero mais jogar.

Diz o ariano\\
Deixe de parte seu mano\\
Você fede a veterano\\
Da rabolução de julho\\
Tava danado\\
Com a sonhança desses pestes\\
Que juguei no Júlio Prestes\\
Mas quem deu foi o Getúlio.

\qquad\qquad\quad Meu baralho dois ouros\\
\qquad\qquad\quad Eu não quero mais jogar.

E diz o Júlio\\
Sou o mês nublado e frio\\
Que lava a bunda no rio\\
E economiza sabão\\
Fui trapaceado\\
Tanto heroísmo tanto estralo\\
Que arrisquei tudo em São Paulo\\
Mas quem deu foi a treição.

\qquad\qquad\quad Meu baralho dois ouros\\
\qquad\qquad\quad Eu não quero mais jogar.

Diz a treição\\
Navegando na água turva\\
Vá pela sombra e na curva\\
Apite que nem buzina\\
E foi"-se embora\\
Tão elegante e gentil\\
Que joguei no meu Brasil\\
Mas quem deu foi a Argentina!

\qquad\qquad\quad Ai meu baralho dois ouros\\
\qquad\quad Eu não quero nunca mais jogar!\\
\quad Vou seguindo no cortejo\\
E vira o coco Sinhá!
\end{verse}

\pagebreak
%V -- ``Plaff! chegou o Carro da Miséria''
\section*{V}

\begin{verse}
Plaff! chegou o Carro da Miséria\\
Do carnaval intaliano!

Tia Miséria vem vestida de honour\\
Cor de cobre do tempo\\
Atrás dela recolhendo guspe\\
O caronel o ginaral o gafetão\\
O puro o heroico o bem"-intencionado\\
Fio da usina brasilera\\
Requebra o povo de Colombo.

Tia Miséria vai se ajeita\\
E tira o peido da miséria.

Mármores estralam rebentados\\
Vento sulão barrendo as chamas\\
Contorce os pinheiros machados\\
Zine o espaço carpideira\\
Arrancando os cabelos\\
Dos luminosos magistrais\\
E à luz dos raios que te partam\\
Colhida pelos vendavais\\
Faz bilboquê com a bolinha do mundo\\
A cibalização cristã.
\end{verse}

\pagebreak
%XI -- ``Enquanto isso os sabichões discutem''
\section*{XI}

\begin{verse}
Enquanto isso os sabichões discutem\\
Se doce"-de"-abobra não dá chumbo pra canhão.
\end{verse}

\pagebreak
%XIV -- ``Vou"-me embora vou"-me embora''
\section*{XIV}

\begin{verse}
Vou"-me embora vou"-me embora\\
Vou"-me embora pra Belém\\
Vou colher cravos e rosas\\
Volto a semana que vem

Vou"-me embora paz da terra\\
Paz da terra repartida\\
Uns têm terra muita terra\\
Outros nem pra uma dormida

Não tenho onde cair morto\\
Fiz gorar a inteligência\\
Vou reentrar no meu povo\\
Reprincipiar minha ciência

Vou"-me embora vou"-me embora\\
Volto a semana que vem\\
Quando eu voltar minha terra\\
Será dela ou de ninguém.
\end{verse}

\part{\textsc{lira paulistana}}

\addcontentsline{toc}{chapter}{``Minha viola bonita''}
\chapter*{}

\begin{verse}
Minha viola bonita,\\
Bonita viola minha,\\
Cresci, cresceste comigo\\
\qquad Nas Arábias.

Minha viola namorada,\\
Namorada viola minha,\\
Cantei, cantaste comigo\\
\qquad Em Granada.

Minha viola ferida,\\
Ferida viola minha,\\
O amor fugiu para leste\\
Na borrasca.

Minha viola quebrada,\\
Raiva, anseios, lutas, vida,\\
Miséria, tudo passou"-se\\
\qquad Em São Paulo.
\end{verse}


\addcontentsline{toc}{chapter}{``Garoa do meu São Paulo''}
\chapter*{}

\begin{verse}
Garoa do meu São Paulo,\\
-- Timbre triste de martírios --\\
Um negro vem vindo, é branco!\\
Só bem perto fica negro,\\
Passa e torna a ficar branco.

Meu São Paulo da garoa,\\
-- Londres das neblinas finas --\\
Um pobre vem vindo, é rico!\\
Só bem perto fica pobre,\\
Passa e torna a ficar rico.

Garoa do meu São Paulo,\\
-- Costureira de malditos --\\
Vem um rico, vem um branco,\\
São sempre brancos e ricos\ldots{}

Garoa, sai dos meus olhos.
\end{verse}

\addcontentsline{toc}{chapter}{``Ruas do meu São Paulo''}
\chapter*{}

\begin{verse}
Ruas do meu São Paulo,\\
Onde está o amor vivo,\\
Onde está?

Caminhos da cidade,\\
Corro em busca do amigo,\\
Onde está?

Ruas do meu São Paulo,\\
Amor maior que o cibo,\\
Onde está?

Caminhos da cidade,\\
Resposta ao meu pedido,\\
Onde está?

Ruas do meu São Paulo,\\
A culpa do insofrido,\\
Onde está?

Há"-de estar no passado,\\
Nos séculos malditos,\\
Aí está.
\end{verse}

\addcontentsline{toc}{chapter}{``O bonde abre a viagem''}
\chapter*{}

\begin{verse}
O bonde abre a viagem,\\
No banco ninguém,\\
Estou só, stou sem.

Depois sobe um homem,\\
No banco sentou,\\
Companheiro vou.

O bonde está cheio,\\
De novo porém\\
Não sou mais ninguém.
\end{verse}

\addcontentsline{toc}{chapter}{``O céu claro tão largo, cheio de calma na tarde''}
\chapter*{}

\begin{verse}
O céu claro tão largo, cheio de calma na tarde,\\
É ver uma criança adormecida\\
Baixando as pálpebras sem pensamento\\
Sobre um mundo que ainda não viveu.

Luzes suaves e certas, luzes até nas sombras,\\
Doçura homens estão mais longe,\\
São apenas recordações mansas pousando\\
Num sentimento sem temor.

Os ruídos se amaciam quase envelhecidos,\\
Doçura chão é vagarento,\\
O ar se esquece. A tensão do insofrido se abranda\\
Como a firmeza das continuações.

Eu te guardo, homem do meu caminho\ldots{}\\
Ôh espelhos, Pireneus, caiçaras insistentes,\\
Porque não sereis sempre assim!\\
Abril\ldots{}
\end{verse}

\addcontentsline{toc}{chapter}{``Tua imagem se apaga em certos bairros''}
\chapter*{}

\begin{verse}
Tua imagem se apaga em certos bairros,\\
Mas tua dor rasga nos ares,\\
Não me deixa dormir.

Ôh, Gilda, Oneida, Tarsila, me fechem a boca,\\
Tapem meus olhos e meus ouvidos,\\
Para que a glória do insofrido\\
Volte a cantar Minas Gerais!

A tua dor se dispersa nos ares,\\
Mas tua imagem suando ao dia inútil\\
Me impede até de chorar.

Eu vou"-me embora, vou"-me embora,\\
Fazer \emph{weekend} em Santo Amaro,\\
Repartir em vãs alegrias\\
Meu desejo vão de esquecer!

Só isso levas, coração.
\end{verse}

\addcontentsline{toc}{chapter}{``A catedral de São Paulo''}
\chapter*{}

\begin{verse}
A catedral de São Paulo\\
Por Deus! que nunca se acaba\\
-- Como minha alma.

É uma catedral horrível\\
Feita de pedras bonitas\\
-- Como minha alma.

A catedral de São Paulo\\
Nasceu da necessidade.\\
-- Como minha alma.

Sacro e profano edifício,\\
Tem pedras novas e antigas\\
-- Como minha alma.

Um dia há"-de se acabar,\\
Mas depois se destruirá\\
-- Como o meu corpo.

E a alma, memória triste,\\
Por sobre os homens arisca,\\
Sem porto.
\end{verse}

\addcontentsline{toc}{chapter}{``Agora eu quero cantar''}
\chapter*{}

\begin{verse}
Agora eu quero cantar\\
Uma história muito triste\\
Que nunca ninguém cantou,\\
A triste história de Pedro,\\
Que acabou qual principiou.

Não houve acalanto. Apenas\\
Um guincho fraco no quarto\\
Alugado. O pai falou,\\
Enquanto a mãe se limpava:\\
-- É Pedro. E Pedro ficou.\\
Ela tinha o que fazer,\\
Ele inda mais, e outro nome\\
Ali ninguém procurou,\\
Não pensaram em Alcibíades,\\
Floriscópio, Ciro, Adrasto,\\
Quedê tempo pra inventar!\\
-- É Pedro. E Pedro ficou.

Pedrinho engatinhou logo\\
Mas muito tarde falou;\\
Ninguém falava com ele,\\
Quando chorava era surra\\
E aprendeu a emudecer.\\
Falou tarde, brincou pouco,\\
Em breve a mãe ajudou.\\
Nesse trabalho insuspeito\\
Passou o dia, e nem bem\\
A noite escura chegou,\\
Como única resposta\\
Um sono bruto o prostrou.

Por trás do quarto alugado\\
Tinha uma serra muito alta\\
Que Pedro nunca notou,\\
Mas num dia desses, não\\
Se sabe porque, Pedrinho\\
Para a serra se voltou:\\
-- Havia de ter, decerto,\\
Uma vida bem mais linda\\
Por trás da serra, pensou.

Sineta que fere ouvido,\\
Vida nova anunciou;\\
Que medo ficar sozinho,\\
Sem pai, mesmo longínquo, sem\\
Mãe, mesmo ralhando, tanta\\
Piazada, ele sem ninguém\ldots{}

Pedro foi para um cantinho,\\
Escondeu o olho e chorou.\\
Mas depois foi divertido,\\
Aliás prazer misturado,\\
Feito de comparação.\\
O menino roupa"-nova\\
Pegava tudo o que a mestra\\
Dizia, ele não pegou!\\
Porque!\ldots{} Mas depois de muito\\
Custo, a coisa melhorou.

Ele gostava era da\\
História Natural, os\\
Bichos, as plantas, os pássaros,\\
Tudo entrava fácil na\\
Cabecinha mal penteada,\\
Tudo Pedro decorou.\\
Havia de saber tudo!\\
Se dedicar! descobrir!\\
Mas já estava bem grandinho\\
E o pai da escola o tirou.\\
Ah que dia desgraçado!\\
E quando a noite chegou,\\
Como única resposta\\
Um sono bruto o prostrou.

Por trás da escola de Pedro\\
Tinha uma serra bem alta\\
Que o menino nunca olhou;\\
Logo no dia seguinte\\
Quando a oficina parou,\\
Machucado, sujo, exausto,\\
Pedrinho a escola rondou.\\
E eis que de repente, não\\
Se sabe porque, Pedrinho\\
Para a serra se voltou:\\
-- Havia de ter por certo\\
Outra vida bem mais linda\\
Por trás da serra! pensou.

Vida que foi de trabalho,\\
Vida que o dia espalhou,\\
Adeus, bela natureza,\\
Adeus, bichos, adeus, flores,\\
Tudo o rapaz, obrigado\\
Pela oficina, largou.\\
Perdeu alguns dentes e antes,\\
Pouco antes de fazer quinze\\
Anos, na boca da máquina\\
Um dedo Pedro deixou.\\
Mas depois de mês e pico\\
Ao trabalho ele voltou,\\
E quando em frente da máquina,\\
Pensam que teve ódio? Não!\\
Pedro sentiu alegria!\\
A máquina era ele! a máquina\\
Era o que a vida lhe dava!\\
E Pedro tudo perdoou.

Foi pensando, foi pensando,\\
E pensou que mais pensou,\\
Teve uma ideia, veio outra,\\
Andou falando sozinho,\\
Não dormiu, fez experiência,\\
E um ano depois, num grito,\\
Louca alegria de amor,\\
A máquina aperfeiçoou.\\
O patrão veio amigável\\
E Pedro galardoou,\\
Pôs ele noutro trabalho,\\
Subiu um pouco o ordenado:\\
-- Aperfeiçoe esta máquina,\\
Caro Pedro! e se afastou.

Era um cacareco de\\
Máquina! e lá, bem na frente,\\
Bela, puxa vida! bela,\\
A primeira namorada\\
De Pedro, nas mãos dum outro,\\
Bela, mais bela que nunca,\\
Se mexendo trabalhou\\
O dia inteiro. Nem bem\\
A noite negra chegou,\\
O rapaz desiludido\\
Um sono bruto prostrou.

Por trás da fábrica havia\\
Uma serra bem mais baixa\\
Que Pedro nunca enxergou,\\
Porém no dia seguinte\\
Chegando pra trabalhar,\\
Não se sabe porque, Pedro\\
Para a serra se voltou:\\
-- Havia de ter, decerto,\\
Uma vida bem mais linda\\
Por trás da serra, pensou.

Ôh, segunda namorada,\\
Flor de abril! cabelo crespo,\\
Mão de princesa, corpinho\\
De vaca nova\ldots{} Era vaca.\\
Aquele riso que faz\\
Que ri, nunca me enganou\ldots{}\\
Caiu nos braços de quem?\\
Caiu nos braços de todos,\\
Caiu na vida e acabou.

Com a terceira namorada,\\
Na primeira roupa preta,\\
Pedro de preto casou.\\
E logo vieram os filhos,\\
Vieram doenças\ldots{} Veio a vida\\
Que tudo, tudo aplainou.\\
Nada de horrível, não pensem,\\
Nenhuma desgraça ilustre\\
Nem dores maravilhosas,\\
Dessas que orgulham a gente,\\
Fazendo cegos vaidosos,\\
Tísicos excepcionais,\\
Ou formando Aleijadinhos,\\
Beethovens e heróis assim:\\
Pedro apenas trabalhou.\\
Ganhou mais, foi subindinho,\\
Um pão de terra comprou.\\
Um pão apenas, três quartos\\
E cozinha, num subúrbio\\
Que tudo dificultou.\\
Menos tempo, mais despesa,\\
Terra fraca, alguma pera,\\
Emprego lá na cidade,\\
Escola pra filho, ofício\\
Pra filho, um num choque de\\
Trem, inválido ficou.

-- Sono! único bem da vida!\ldots{}

Foi essa frase sem força,\\
Sem História Natural,\\
Sem máquina, sem patente\\
De invenção, que por derradeiro\\
Pedro na vida inventou.\\
E quando remoendo a frase,\\
A noite preta chegou,\\
Pedro, Pedrinho, José,\\
Francisco, e nunca Alcibíades,\\
Um sono bruto anulou.

Por trás da morada nova\\
Não tinha serra nenhuma,\\
Nem morro tinha, era um plano\\
Devastado e sem valor,\\
Mas um dia desses, sempre\\
Igual ao que ontem passou,\\
Pedro, João, Manduca, não\\
Se sabe porque, Antônio\\
Para o plano se voltou:\\
-- Talvez houvesse, quem sabe,\\
Uma vida bem mais calma\\
Além do plano, pensou.

Havia, Pedro, era a morte,\\
Era a noite mais escura,\\
Era o grande sono imenso;\\
Havia, desgraçado, havia\\
Sim, burro, idiota, besta,\\
Havia sim, animal,\\
Bicho, escravo sem história,\\
Só da História Natural!\ldots{}

Por trás do túmulo dele\\
Tinha outro túmulo\ldots{} Igual.
\end{verse}


\addcontentsline{toc}{chapter}{``Na rua Aurora eu nasci''}
\chapter*{}

\begin{verse}
Na rua Aurora eu nasci\\
Na aurora de minha vida\\
E numa aurora cresci.

No largo do Paiçandu\\
Sonhei, foi luta renhida,\\
Fiquei pobre e me vi nu.

Nesta rua Lopes Chaves\\
Envelheço, e envergonhado\\
Nem sei quem foi Lopes Chaves.

Mamãe! me dá essa lua,\\
Ser esquecido e ignorado\\
Como esses nomes da rua.
\end{verse}


\addcontentsline{toc}{chapter}{``Quando eu morrer quero ficar''}
\chapter*{}

\begin{verse}
Quando eu morrer quero ficar,\\
Não contem aos meus inimigos,\\
Sepultado em minha cidade,\\
\qquad\qquad Saudade.

Meus pés enterrem na rua Aurora,\\
No Paiçandu deixem meu sexo,\\
Na Lopes Chaves a cabeça\\
\qquad\qquad Esqueçam.

No Pátio do Colégio afundem\\
O meu coração paulistano:\\
Um coração vivo e um defunto\\
\qquad\qquad Bem juntos.

Escondam no Correio o ouvido\\
Direito, o esquerdo nos Telégrafos,\\
Quero saber da vida alheia,\\
\qquad\qquad Sereia.

O nariz guardem nos rosais,\\
A língua no alto do Ipiranga\\
Para cantar a liberdade.\\
\qquad\qquad Saudade\ldots{}

Os olhos lá no Jaraguá\\
Assistirão ao que há"-de vir,\\
O joelho na Universidade,\\
\qquad\qquad Saudade\ldots{}

As mãos atirem por aí,\\
Que desvivam como viveram,\\
As tripas atirem pro Diabo,\\
Que o espírito será de Deus.\\
\qquad\qquad Adeus.
\end{verse}

\chapterspecial{A meditação sobre o Tietê}{(30 de novembro de 1944 a\\ 12 de fevereiro de 1945)\footnote[*]{Mário de Andrade falece em 25 de fevereiro de 1945.}}{}


\begin{verse}
\qquad\qquad\qquad\qquad\qquad Água do meu Tietê,\\
\qquad\qquad\qquad\qquad\qquad Onde me queres levar?\\
\qquad\qquad\qquad\qquad\qquad -- Rio que entras pela terra\\
\qquad\qquad\qquad\qquad\qquad E que me afastas do mar\ldots{}
\end{verse}

\begin{verse}
É noite. E tudo é noite. Debaixo do arco admirável\\
Da Ponte das Bandeiras o rio\\
Murmura num banzeiro de água pesada e oliosa.\\
É noite e tudo é noite. Uma ronda de sombras,\\
Soturnas sombras, enchem de noite tão vasta\\
O peito do rio, que é como se a noite fosse água,\\
Água noturna, noite líquida, afogando de apreensões\\
As altas torres do meu coração exausto. De repente\\
O ólio das águas recolhe em cheio luzes trêmulas,\\
É um susto. E num momento o rio\\
Esplende em luzes inumeráveis, lares, palácios e ruas,\\
Ruas, ruas, por onde os dinossauros caxingam\\
Agora, arranha"-céus valentes donde saltam\\
Os bichos blau e os punidores gatos verdes,\\
Em cânticos, em prazeres, em trabalhos e fábricas,\\
Luzes e glória. É a cidade\ldots{} É a emaranhada forma\\
Humana corrupta da vida que muge e se aplaude.\\
E se aclama e se falsifica e se esconde. E deslumbra.\\
Mas é um momento só. Logo o rio escurece de novo,\\
Está negro. As águas oliosas e pesadas se aplacam\\
Num gemido. Flor. Tristeza que timbra um caminho de morte.\\
É noite. E tudo é noite. E o meu coração devastado\\
É um rumor de germes insalubres pela noite insone e humana.

Meu rio, meu Tietê, onde me levas?\\
Sarcástico rio que contradizes o curso das águas\\
E te afastas do mar e te adentras na terra dos homens,\\
Onde me queres levar?\ldots{}\\
Por que me proíbes assim praias e mar, por que\\
Me impedes a fama das tempestades do Atlântico\\
E os lindos versos que falam em partir e nunca mais voltar?\\
Rio que fazes terra, húmus da terra, bicho da terra,\\
Me induzindo com a tua insistência turrona paulista\\
Para as tempestades humanas da vida, rio, meu rio!\ldots{}

Já nada me amarga mais a recusa da vitória\\
Do indivíduo, e de me sentir feliz em mim.\\
Eu mesmo desisti dessa felicidade deslumbrante,\\
E fui por tuas águas levado,\\
A me reconciliar com a dor humana pertinaz,\\
E a me purificar no barro dos sofrimentos dos homens.\\
Eu que decido. E eu mesmo me reconstituí árduo na dor\\
Por minhas mãos, por minhas desvividas mãos, por\\
Estas minhas próprias mãos que me traem,\\
Me desgastaram e me dispersaram por todos os descaminhos,\\
Fazendo de mim uma trama onde a aranha insaciada\\
Se perdeu em cisco e pólen, cadáveres e verdades e ilusões.

Mas porém, rio, meu rio, de cujas águas eu nasci,\\
Eu nem tenho direito mais de ser melancólico e frágil,\\
Nem de me estrelar nas volúpias inúteis da lágrima!\\
Eu me reverto às tuas águas espessas de infâmias,\\
Oliosas, eu, voluntariamente, sofregamente, sujado\\
De infâmias, egoísmos e traições. E as minhas vozes,\\
Perdidas do seu tenor, rosnam pesadas e oliosas,\\
Varando terra adentro no espanto dos mil futuros,\\
À espera angustiada do ponto. Não do meu ponto final!\\
Eu desisti! Mas do ponto entre as águas e a noite,\\
Daquele ponto leal à terrestre pergunta do homem,\\
De que o homem há"-de nascer.

Eu vejo, não é por mim, o meu verso tomando\\
As cordas oscilantes da serpente, rio.\\
Toda a graça, todo o prazer da vida se acabou.\\
Nas tuas águas eu contemplo o Boi Paciência\\
Se afogando, que o peito das águas tudo soverteu.\\
Contágios, tradições, brancuras e notícias,\\
Mudo, esquivo, dentro da noite, o peito das águas, fechado, mudo,\\
Mudo e vivo, no despeito estrídulo que me fustiga e devora.

Destino, predestinações\ldots{} meu destino. Estas águas\\
Do meu Tietê são abjetas e barrentas,\\
Dão febre, dão a morte decerto, e dão garças e antíteses.\\
Nem as ondas das suas praias cantam, e no fundo\\
Das manhãs elas dão gargalhadas frenéticas,\\
Silvos de tocaias e lamurientos jacarés.\\
Isto não são as águas que se beba, conhecido, isto são\\
Águas do vício da terra. Os jabirus e os socós\\
Gargalham depois morrem. E as antas e os bandeirantes e os ingás,\\
Depois morrem. Sobra não. Nem sequer o Boi Paciência\\
Se muda não. Vai tudo ficar na mesma, mas vai!\ldots{} e os corpos\\
Podres envenenam estas águas completas no bem e no mal.

Isto não são águas que se beba, conhecido! Estas águas\\
São malditas e dão morte, eu descobri! e é por isso\\
Que elas se afastam dos oceanos e induzem à terra dos homens,\\
Paspalhonas. Isto não são águas que se beba, eu descobri!\\
E o meu peito das águas se esborrifa, ventarrão vem, se encapela\\
Engruvinhado de dor que não se suporta mais.

Me sinto o Pai Tietê! ôh força dos meus sovacos!\\
Cio de amor que me impede, que destrói e fecunda!\\
Nordeste de impaciente amor sem metáfóras,\\
Que se horroriza e enraivece de sentir"-se\\
Demagogicamente tão sozinho! Ôh força!\\
Incêndio de amor estrondante, enchente magnânima que me inunda,\\
Me alarma e me destroça, inerme por sentir"-me\\
Demagogicamente tão só!

A culpa é tua, Pai Tietê? A culpa é tua\\
Se as tuas águas estão podres de fel\\
E majestade falsa? A culpa é tua\\
Onde estão os amigos? onde estão os inimigos?\\
Onde estão os pardais? e os teus estudiosos e sábios, e\\
Os iletrados?\\
Onde o teu povo? e as mulheres! dona Hircenuhdis Quiroga!\\
E os Prados e os crespos e os pratos e os barbas e os gatos e os línguas\\
Do Instituto Histórico e Geográfico, e os mu"-\\
seus e a Cúria, e os senhores chantres reverendíssimos,\\
Celso nihil estate varíolas gide memoriam,\\
Calípedes flogísticos e a Confraria Brasiliense e \emph{Clima}\\
E os jornalistas e os trustkistas e a Light e as\\
Novas ruas abertas e a falta de habitações e\\
Os mercados?\ldots{} E a tiradeira divina de Cristo!\ldots{}

Tu és Demagogia. A própria vida abstrata tem vergonha\\
De ti em tua ambição fumarenta.\\
És demagogia em teu coração insubmisso.\\
És demagogia em teu desequilíbrio anticéptico\\
E antiuniversitário.\\
És demagogia. Pura demagogia.\\
Demagogia pura. Mesmo alimpada de metáforas.\\
Mesmo irrespirável de furor na fala reles:\\
Demagogia.\\
Tu és enquanto tudo é eternidade e malvasia:\\
Demagogia.\\
Tu és em meio à (crase) gente pia:\\
Demagogia.\\
És tu jocoso enquanto o ato gratuito se esvazia:\\
Demagogia.\\
És demagogia, ninguém chegue perto!\\
Nem Alberto, nem Adalberto nem Dagoberto\\
Esperto Ciumento Peripatético e Ceci\\
E Tancredo e Afrodísio e também Armida\\
E o próprio Pedro e também Alcibíades,\\
Ninguém te chegue perto, porque tenhamos o pudor,\\
O pudor do pudor, sejamos verticais e sutis, bem\\
Sutis!\ldots{} E as tuas mãos se emaranham lerdas,\\
E o Pai Tietê se vai num suspiro educado e sereno,\\
Porque és demagogia e tudo é demagogia.

Olha os peixes, demagogo incivil! Repete os carcomidos peixes!\\
São eles que empurram as águas e as fazem servir de alimento\\
Às areias gordas da margem. Olha o peixe dourado sonoro,\\
Esse um é presidente, mantém faixa de crachá no peito,\\
Acirculado de tubarões que escondendo na fuça rotunda\\
O perrepismo dos dentes, se revezam na rota solene,\\
Languidamente presidenciais. Ei"-vem o tubarão"-martelo\\
E o lambari"-spitfire. Ei"-vem o boto"-ministro.\\
Ei"-vem o peixe"-boi com as mil mamicas imprudentes,\\
Perturbado pelos golfinhos saltitantes e as tabaranas\\
Em zás"-trás dos guapos Pêdêcês e Guaporés.\\
Eis o peixe"-baleia entre os peixes muçuns lineares,\\
E os bagres do lodo oliva e bilhões de peixins japoneses;\\
Mas é asnático o peixe"-baleia e vai logo encalhar na margem,\\
Pois quis engolir a própria margem, confundido pela facheada.\\
Peixes aos mil e mil, como se diz, brincabrincando\\
De dirigir a corrente, com ares de salva"-vidas.\\
E lá vem por debaixo e por de"-banda os interrogativos peixes\\
Internacionais, uns rubicundos sustentados de mosca,\\
E os espadartes a trote chique, esses são espadartes! e as duas\\
Semanas Santas se insultam e odeiam, na lufa"-lufa de ganhar\\
No bicho o corpo do Crucificado. Mas as águas,\\
As águas choram baixas num murmúrio lívido, e se difundem\\
Tecidas de peixe e abandono, na mais incompetente solidão.

Vamos, Demagogia! eia! sus! aceita o ventre e investe!\\
Berra de amor humano impenitente,\\
Cega, sem lágrima, ignara, colérica, investe!\\
Um dia hás"-de ter razão contra a ciência e a realidade,\\
E contra os fariseus e as lontras luzidias.\\
E contra os guarás e os elogiados. E contra todos os peixes.\\
E também os mariscos, as ostras e os trairões fartos de equilíbrio e\\
Pundhonor.\\
\quad\quad\quad\quad{}Pum d'honor.\\
\hfill{}Quedê as Juvenilidades Auriverdes!\\
Eu tenho medo\ldots{} Meu coração está pequeno, é tanta\\
Essa demagogia, é tamanha,\\
Que eu tenho medo de abraçar os inimigos,\\
Em busca apenas dum sabor,\\
Em busca dum olhar,\\
Um sabor, um olhar, uma certeza\ldots{}

É noite\ldots{} Rio! meu rio! meu Tietê!\\
É noite muito!\ldots{} As formas\ldots{} Eu busco em vão as formas\\
Que me ancorem num porto seguro na terra dos homens.\\
É noite e tudo é noite. O rio tristemente\\
Murmura num banzeiro de água pesada e oliosa.\\
Água noturna, noite líquida\ldots{} Augúrios mornos afogam\\
As altas torres do meu exausto coração.\\
Me sinto esvair no apagado murmulho das águas.\\
Meu pensamento quer pensar, flor, meu peito\\
Quereria sofrer, talvez (sem metáfora) uma dor irritada\ldots{}\\
Mas tudo se desfaz num choro de agonia\\
Plácida. Não tem formas nessa noite, e o rio\\
Recolhe mais esta luz, vibra, reflete, se aclara, refulge,\\
E me larga desarmado nos transes da enorme cidade.

Se todos esses dinossauros imponentes de luxo e diamante,\\
Vorazes de genealogias e de arcanos,\\
Quisessem reconquistar o passado\ldots{}\\
Eu me vejo sozinho, arrastando sem músculo\\
A cauda do pavão e mil olhos de séculos,\\
Sobretudo os vinte séculos de anticristianismo\\
Da por todos chamada Civilização Cristã\ldots{}\\
Olhos que me intrigam, olhos que me denunciam,\\
Da cauda do pavão, tão pesada e ilusória.\\
Não posso continuar mais, não tenho, porque os homens\\
Não querem me ajudar no meu caminho.\\
Então a cauda se abriria orgulhosa e reflorescente\\
De luzes inimagináveis e certezas\ldots{}\\
Eu não seria tão somente o peso deste meu desconsolo,\\
A lepra do meu castigo queimando nesta epiderme\\
Que encurta, me encerra e me inutiliza na noite,\\
Me revertendo minúsculo à advertência do meu rio.\\
Escuto o rio. Assunto estes balouços em que o rio\\
Murmura num banzeiro. E contemplo\\
Como apenas se movimenta escravizada a torrente,\\
E rola a multidão. Cada onda que abrolha\\
E se mistura no rolar fatigado é uma dor. E o surto\\
Mirim dum crime impune.

Vem de trás o estirão. É tão soluçante e tão longo,\\
E lá na curva do rio vêm outros estirões e mais outros,\\
E lá na frente são outros, todos soluçantes e presos\\
Por curvas que serão sempre apenas as curvas do rio.\\
Há"-de todos os assombros, de todas as purezas e martírios\\
Nesse rolo torvo das águas. Meu Deus! meu\\
Rio! como é possível a torpeza da enchente dos homens!\\
Quem pode compreender o escravo macho\\
E multimilenar que escorre e sofre, e mandado escorre\\
Entre injustiça e impiedade, estreitado\\
Nas margens e nas areias das praias sequiosas?\\
Elas bebem e bebem. Não se fartam, deixando com desespero\\
Que o resto do galé aquoso ultrapasse esse dia,\\
Pra ser represado e bebido pelas outras areias\\
Das praias adiante, que também dominam, aprisionam e mandam\\
A trágica sina do rolo das águas, e dirigem\\
O leito impassível da injustiça e da impiedade.\\
Ondas, a multidão, o rebanho, o rio, meu rio, um rio\\
Que sobe! Fervilha e sobe! E se adentra fatalizado, e em vez\\
De ir se alastrar arejado nas liberdades oceânicas,\\
Em vez se adentra pela terra escura e ávida dos homens,\\
Dando sangue e vida a beber. E a massa líquida\\
Da multidão onde tudo se esmigalha e se iguala,\\
Rola pesada e oliosa, e rola num rumor surdo,\\
E rola mansa, amansada imensa eterna, mas\\
No eterno imenso rígido canal da estulta dor.

Porque os homens não me escutam! Por que os governadores\\
Não me escutam? Por que não me escutam\\
Os plutocratas e todos os que são chefes e são fezes?\\
Todos os donos da vida?\\
Eu lhes daria o impossível e lhes daria o segredo,\\
Eu lhes dava tudo aquilo que fica pra cá do grito\\
Metálico dos números, e tudo\\
O que está além da insinuação cruenta da posse.\\
E se acaso eles protestassem, que não! que não desejam\\
A borboleta translúcida da humana vida, porque preferem\\
O retrato a ólio das inaugurações espontâneas,\\
Com béstias do operário e do oficial, imediatamente inferior,\\
E palminhas, e mais os sorrisos das máscaras e a profunda comoção,\\
Pois não! Melhor que isso eu lhes dava uma felicidade deslumbrante\\
De que eu consegui me despojar porque tudo sacrifiquei.\\
Sejamos generosíssimos. E enquanto os chefes e as fezes\\
De mamadeira ficassem na creche de laca e lacinhos,\\
Ingênuos brincando de felicidade deslumbrante:\\
Nós nos iríamos de camisa aberta ao peito,\\
Descendo verdadeiros ao léu da corrente do rio,\\
Entrando na terra dos homens ao coro das quatro estações.

Pois que mais uma vez eu me aniquilo sem reserva,\\
E me estilhaço nas fagulhas eternamente esquecidas,\\
E me salvo no eternamente esquecido fogo de amor\ldots{}\\
Eu estalo de amor e sou só amor arrebatado\\
Ao fogo irrefletido do amor.\\
\ldots{} eu já amei sozinho comigo; eu já cultivei também\\
O amor do amor, Maria!\\
E a carne plena da amante, e o susto vário\\
Da amiga, e a confidência do amigo\ldots{} Eu já amei\\
Contigo, Irmão Pequeno, no exílio da preguiça elevada, escolhido\\
Pelas águas do túrbido rio do Amazonas, meu outro sinal.\\
E também, ôh também! na mais impávida glória\\
Descobridora da minha inconstância e aventura,\\
Desque me fiz poeta e fui trezentos, eu amei\\
Todos os homens, odiei a guerra, salvei a paz!\\
E eu não sabia! Eu bailo de ignorâncias inventivas,\\
E a minha sabedoria vem das fontes que eu não sei!\\
Quem move meu braço? Quem beija por minha boca?\\
Quem sofre e se gasta pelo meu renascido coração?\\
Quem? senão o incêndio nascituro do amor?\ldots{}\\
Eu me sinto grimpado no arco da Ponte das Bandeiras,\\
Bardo mestiço, e o meu verso vence a corda\\
Da caninana sagrada, e afina com os ventos dos ares, e enrouquece\\
Úmido nas espumas da água do meu rio,\\
E se espatifa nas dedilhações brutas do incorpóreo Amor.

Por que os donos da vida não me escutam?\\
Eu só sei que eu não sei por mim! sabem por mim as fontes\\
Da água, e eu bailo de ignorâncias inventivas.\\
Meu baile é solto como a dor que range, meu\\
Baile é tão vário que possui mil sambas insonhados!\\
Eu converteria o humano crime num baile mais denso\\
Que estas ondas negras de água pesada e oliosa,\\
Porque os meus gestos e os meus ritmos nascem\\
Do incêndio puro do amor\ldots{} Repetição. Primeira voz sabida, o Verbo.\\
Primeiro troco. Primeiro dinheiro vendido. Repetição logo ignorada.\\
Como é possível que o amor se mostre impotente assim\\
Ante o ouro pelo qual o sacrificam os homens,\\
Trocando a primavera que brinca na face das terras,\\
Pelo outro tesouro que dorme no fundo baboso do rio!

É noite! é noite!\ldots{} E tudo é noite! E os meus olhos são noite!\\
Eu não enxergo sequer as barcaças na noite.\\
Só a enorme cidade. E a cidade me chama e pulveriza,\\
E me disfarça numa queixa flébil e comedida,\\
Onde irei encontrar a malícia do Boi Paciência\\
Redivivo. Flor. Meu suspiro ferido se agarra,\\
Não quer sair, enche o peito de ardência ardilosa,\\
Abre o olhar, e o meu olhar procura, flor, um tilintar\\
Nos ares, nas luzes longe, no peito das águas,\\
No reflexo baixo das nuvens.

São formas\ldots{} Formas que fogem, formas\\
Indivisas, se atropelando, um tilintar de formas fugidias\\
Que mal se abrem, flor, se fecham, flor, flor, informes, inacessíveis,\\
Na noite. E tudo é noite. Rio, o que eu posso fazer!\ldots{}\\
Rio, meu rio\ldots{} mas porém há"-de haver com certeza\\
Outra vida melhor do outro lado de lá\\
Da serra! E hei"-de guardar silêncio!\\
O que eu posso fazer!\ldots{} hei"-de guardar silêncio\\
Deste amor mais perfeito do que os homens?\ldots{}

Estou pequeno, inútil, bicho da terra, derrotado.\\
No entanto eu sou maior\ldots{} Eu sinto uma grandeza infatigável!\\
Eu sou maior que os vermes e todos os animais.\\
E todos os vegetais. E os vulcões vivos e os oceanos,\\
Maior\ldots{} Maior que a multidão do rio acorrentado,\\
Maior que a estrela, maior que os adjetivos,\\
Sou homem! vencedor das mortes, bem"-nascido além dos dias,\\
Transfigurado além das profecias!

Eu recuso a paciência, o boi morreu, eu recuso a esperança.\\
Eu me acho tão cansado em meu furor.\\
As águas apenas murmuram hostis, água vil mas turrona paulista\\
Que sobe e se espraia, levando as auroras represadas\\
Para o peito dos sofrimentos dos homens.\\
\ldots{} e tudo é noite. Sob o arco admirável\\
Da Ponte das Bandeiras, morta, dissoluta, fraca,\\
Uma lágrima apenas, uma lágrima,\\
Eu sigo alga escusa nas águas do meu Tietê.
\end{verse}


\partepigraph{\emph{a Liddy Chiaffarelli}}{}
\part{\textsc{café}}
\removeepigraph


\part*{\textsc{café}\\\textsc{tragédia coral em três atos}\\\textsc{o poema}\\
(São Paulo, Natal de 1942)}


\chapter{Primeiro ato -- primeira cena}


\section*{PORTO PARADO}

\hfill\parbox{150pt}{(A cena representa o interior
de um armazém de café, no cais.
Os estivadores na entressombra)}

\section*{I\break CORAL DO QUEIXUME}

\emph{Os Estivadores}:

\begin{verse}
Minha terra perdeu seu porte de grandeza\ldots{}\\
O café que alevanta os homens apodrece\\
Escravizado pela ambição dos gigantes da mina do ouro.\\
A planta nobre, o grão civilizador\\
Que jamais recusou a sua recompensa\\
Nada mais vale, nada mais.\\
Que farei agora que o café não vale mais!

Essa força grave da terra era também a minha força.\\
Ela era verde e me ensinava o futuro.\\
Ela era encarnada e audaciosa\\
Era negra e aquentava o meu coração.\\
Foi ela que deu à minha terra o seu porte de grandeza\\
E hoje nada mais vale, nada mais.

Café!\ldots{} Café!\ldots{} Eu exclamo a palavra sagrada\\
Café!\ldots{} O seu fruto me trazia o calor no coração\\
Era o cheiro da minha paz, o gosto do meu riso\\
E agora ele me nega o pão.\\
Que farei agora que o café não vale mais!

Porte de grandeza, odor da minha terra, força da minha vida,\\
Que farei agora que pra mim não vales mais!
\end{verse}

\hfill\parbox{150pt}{(Os estivadores se encostam nas pilhas
de sacas de café, desanimados. Um grupo
deles, no chão, está jogando baralho.)}

{[}\ldots{}{]}

\chapter{Segundo ato -- primeira cena}

\section*{«CÂMARA"-BALÉ»}

\hfill\parbox{240pt}{(A cena representa o recinto duma câmara de
deputados. Junto à ribalta a mesa da
presidência, anfiteatro das bancadas em
seguida, e no fundo, a meia altura, as galerias do público.)}

{[}\ldots{}{]}

\section*{III\break A ENDEIXA DA MÃE}

\hfill\parbox{240pt}{
(Mas sucedeu que com o grito do homem irritado, as galerias principiaram
se manifestando ainda tímidas. Vem um ritmo batido de vaia, murmurando
num terceiro plano sonoro ``Café, café, café -- Café, café, café'' um
sem"-número de vezes. O presidente bate o sino. Todos reprovam muito,
escandalizados, a falta de educação das galerias, com aquele povinho,
numa bagunça ainda discreta, em que se escutam, espirradas num
\emph{stretto} surdo, frases como ``Vá carregar piano!''; ``Téréré não
resolve!''; ``Isso é conversa pra boi dormir!''; ``Desgraça pouca é
bobagem!''; ``Deixa de lero"-lero!''. Durante a baguncinha o Deputado
Cinza aproveitou pra entrar no recinto da Câmara. Entrada espetacular,
porque ele vem puxando a Mãe. Ela se assusta com o ambiente, quer fugir.
O Deputado Cinza ordena que ela fale, ela implora partir, ele insiste. E
a Mãe, se vendo mesmo perdida, no medo, no susto, meio que delira.)
}

\bigskip

\emph{A Mãe}:

\begin{verse}
\ldots{} Depois que o grão apodreceu no galho\\
A miséria chegou com seus dias compridos\\
E as noites curtas por demais que a fome acorda.\\
Nunca mais o meu filho fugiu da horta\\
Amassando na boca as alfaces.

Os peitos das mães já secaram\\
Caíram as cercas das hortas\\
Vendeu"-se a vaca, fugiu o sabiá dos pomares\\
E muitos homens jazem podres\\
Nos botequins de beira"-estrada\\
Nos armazéns do cais vazio\\
Nas grunhas do conluio da noite.

Falai se há dor que se compare à minha!\ldots{}

Nos caminhos da noite pressaga\\
Os infelizes vêm chegando, vêm chegando\\
Conduzidos pela estrela da cidade.\\
São todos os que abafaram o sonho, meninos\\
Todos os que só amaram no susto e no arrependimento da procriação\\
Os que se viram já velhos sem ter o que recordar.\\
São os famintos, são os rotos, são os escravos,\\
São os mil e um cativos da vida, em procissão.

Falai!\ldots{}\\
Falai se há dor que se compare à minha!\ldots{}

No avanço lerdo dos bois\\
Os infelizes vêm chegando, vêm chegando.\\
A sentinela avançada de serra"-acima\\
Se erriça toda de estátuas, de espantalhos, de estafermos doentes\\
Movidos pelo rito da esmola e do furto.\\
Acaso não vedes que o ponteiro está chegando na hora?\\
As estátuas comungarão fatais no crime hediondo\\
Acaso não vedes que o ponteiro chega na hora do crime hediondo?\\
Os peitos da Mãe se enrijarão no escudo seco de aço\\
Ruirão por milagre os muros, ruirão fortalezas e forças\\
A guerra vai passar com seu rancho de peste e de morte\\
Varrendo tudo na batucada infernal.

Falai!\ldots{} Falai!\ldots{}\\
Falai se há dor que se compare à minha!\ldots{}

Ôh gigantes da mina do ouro\\
Ôh anões subterrâneos da servidão\\
Ôh magnatas e seus poetas laureados, galões e galinhas,\\
Pastéis, pastores, professores, jornalistas e genealogistas,\\
Furta"-cores camiseiros e pontapezeiros,\\
Ôh melancias e melaços, burros borras, borrachas, molhos pardavascos\\
Ôh grandavascos e vendidavascos\\
O vosso peito ladrilhado com pedrinhas diamantes\\
É concho e vazio feito a bexiga do Mateus\\
Monstros tardios sem olhos sem beijo sem mãos\\
O que fizestes do sentido da vida!\\
Ôh vós gigantes da mina e vós anões subterrâneos\\
Falai!\\
O que fizestes, o que fizestes do sentido da vida!\ldots{}

\textsc{eu sou aquela que disse:}\\
Raça culpada, a vossa destruição está próxima!\\
Já o pato bravo avoou na escuridão da noite\\
E as gaivotas gritam no alarma lunar da praia!\\
Pois não vedes que os seres do campo e da rua\\
Estão se aquerenciando no malhadouro da praça\\
Já indiferentes ao chamamento passivo do ninho!\\
Raça culpada, a vossa destruição está próxima!\\
A aurora feito um gato verde se assanha por trás da cidade\\
E rompe antes do dia as barras triunfais do dia!
\end{verse}

\hfill\parbox{230pt}{
(Só que pelo meio da endeixa o povo das galerias não se conteve mais. E
enquanto os deputados, não querendo escutar as verdades que a Mãe estava
clamando, reencetavam a dançinha e a cantarola da embolada anterior, o
povo estourou numa bagunçona desesperada. Gritam em vozes amontoadas, em
sanha: ``Canalhas! Vendidos!''; ``Infames! Malditos!''; ``A raiva
incendiou meu desejo! Não quero mais dormir!''; ``Nasceu a tigre dos
caminhos! Eu baterei na porta dos gigantes!''; ``Um chefe! Um chefe!'';
``Ele não para de crescer! Ele está rutilando por trás da cidade!'';
``Café, café, café -- Café, café, café!''. Os polícias estão
chanfalhando o povo das galerias. Levam a Mãe presa. Os deputados
dançandinho sempre.)

(Pano com estrondo)
}

\pagebreak

\section*{SEGUNDA CENA\break O ÊXODO}

\hfill\parbox{150pt}{
(Na estaçãozinha do trem"-de"-ferro. Vêm chegando os colonos, respondendo
ao apelo da cidade. Primeiro chegam os solteiros, rapazes, garotas.
Estão esperançados, quase brincalhões. Confiantes de viver na cidade
terrível.)
}

{[}\ldots{}{]}

\section*{III\break CORAL DA VIDA}

\hfill\parbox{150pt}{
(Agora vêm chegando os casais. Estão fatigados e ardentes. Sérios. Aos
pares. Os solteiros logo se afinam com os recém"-chegados que também se
arrancham por aí na espera do trem. Há como que uma intensificação
ardente de vida em todos. A tarde está se avermelhando.)
}

\emph{Casados e Solteiros} (coral misto a quatro vozes):

\begin{verse}
Cafezal grande na calma fatigada da tarde\ldots{}

Uns homens de fala vagarenta e de nariz furão\\
Conquistaram estas paisagens, os chãos mais felizes da terra\\
Para sobre eles plantar o oceano da esmeralda\\
E eu vim à chama vermelha do grão pequenino.

Porém no princípio dos chãos está postada a cidade terrível\\
Grandiosa e carrancuda, histórica e completa\\
Cheia de passado e futuro, inimiga cinzenta do estranho,\\
Dona das sete doenças irascíveis do frio.\\
No seu rumor resmungam as animosidades desconfiadas\\
Dos seus bueiros brota o sentimento da solidão.\\
A cidade terrível repudiou o mar facílimo\\
E se escanchou grimpada no penedo mais alto de serra"-acima\\
Gritando a todos o seu gélido e agressivo Quem vem lá!

\quad\quad\emph{Eco}, fora de cena:

\quad\quad\quad\quad{}-- Quem vem láááaa!\ldots{}


Mas eu penetrei na cidade inimiga e os meus pés não queriam andar de saudade\\
E a Terrível riu seu riso de garoa pervertida\\
E me fez punir as sete provas.\\
Ela me fez passar pelas sete provas da promissão.

A primeira foi obedecer mas eu me opus.\\
A segunda foi mandar e então eu obedeci.\\
A terceira foi sonhar mas eu me equilibrei num pé só e não dormi.\\
A quarta e a quinta foram roubar e matar\\
Mas eu, cheio da fragilidade, beijei de mãos abertas.\\
A sexta, a mais infamante de todas, foi ignorar.\\
Mas eu, chorando, provei o pó amargo da rua e lembrei.\\
Então a cidade insidiosa, cheia de música e festa,\\
Passou a mão de bruma nos meus olhos, me convidando a esquecer.\\
Mas eu com uma rosa roubada na abertura da camisa\\
Gritei no eco do mundo: Eu sou!

\quad\quad\emph{Eco}, fora de cena:

\quad\quad\quad\quad{}-- Eu soooooou!\ldots{} Eu sooooooooooooou!\ldots{}

Pois então a cidade se fez mãe e eu descansei nela uma noite e um dia.\\
Ela é a mãe do trabalho, mãe do pensamento,\\
Ela é a mãe carinhosa do lar fechadinho bem quente\\
E nas suas noites graves todos dormem sem sonhar.

Só na lucidez do seu frio ácido\\
Só nela se pode beber o vinho generoso de corpo grosso\\
Só nela é permitido bailar sem vertigem\\
Só nela é possível querer sem miragem\\
Só nela, feiosa e leal, se erriça na boca do homem\\
O sal da verdade da hora\\
Sem se tornar salobro à glória do passado.

E depois que eu descansei a noite e o dia\\
A cidade me levou para os chãos mais felizes da terra\\
Onde tudo é carícia no seio dos morros mansos\\
Onde o calor é ouro no dia coroado por noites de prata.

Ôh cafezal! cafezal grande na mágoa sangrenta da tarde\\
Gosto de um tempo acabado, será permitido sonhar?\ldots{}

Raça culpada, raça envilecida maldita,\\
Os gigantes da mina com os seus anões ensinados\\
Traíram a cidade e os chãos felizes.\\
E tudo foi, tudo será desilusão constante\\
Enquanto não nascer do enxurro da cidade\\
O Homem Zangado, o herói do coração múltiplo,\\
O justiçador moreno, o esmurrador com mil punhos\\
Amassando os gigantes da mina e peidando para os anões.

O urro da tempestade acorda no seio alarmado do horizonte\\
De cada planta o cafezal destila o veneno grosso do ódio.\\
Em cada mão comichona a volúpia da morte.\\
O meu passo deixou rastro de sangue no caminho,\\
O céu se embebedou de sangue, o meu suor cheira sangue.

O herói vingador já nasceu do enxurro das cidades.\\
Ele é todo encarnado, tem mil punhos, o olhar implacável\\
Todo ele comichona impaciente no desejo voluptuoso da morte.\\
Neste instante ele está vestindo a armadura de ouro e prata\\
O seu chapéu de aba larga é levantado na frente\\
Ele tem uma estrela de verdade bem na testa\\
Ele tem um corisco no sapato\\
E um coração humano no lugar do coração.
\end{verse}

\hfill\parbox{150pt}{
(Só um largo listrão encarnado marca a fimbria do horizonte, no longe.
Escureceu muito, e o chefe de estação precisou pendurar uma lâmpada
sobre o anúncio que trouxe. E o anúncio avisa: ``Trem de Segunda Classe
-- Não haverá mais''. O silêncio abatido abafa os corações.)
}

{[}\ldots{}{]}

\pagebreak
\section*{TERCEIRO ATO\break O DIA NOVO}

\hfill\parbox{150pt}{
(É o pátio de um cortiço, num arrabalde da cidade, convulsionada pela
revolução. Todo o pano de fundo é tomado pela descrição da cidade, com o
centro urbano longe, um amontoado de arranha"-céus. Um esgalho da
revolução vem se aproximando do bairro pobre. Passa um homem fugindo na
carreira pela rua, atrás do muro do cortiço, no fundo. Na parte da casa
operária que se enxerga dum lado, na boca da cena, junto à mãe
inquietíssima, a meninota displicente, conseguiu ligar o rádio.)
}

\section*{I\break 1º PARLATO DO RÁDIO}

\hfill\parbox{150pt}{
(Na saturnidade apreensiva da orquestra ainda pobre, arrepiada de frases
inquietas, o rádio explode.)
}

\emph{O Rádio}:

-- Alô! alô!\ldots{} Alô! alô!\ldots{} Prezados ouvintes, alô"-alô!\ldots{} O
Rádio é nosso! O Rádio acaba de cair em nossas mãos! urraaa!\ldots{} Alô!
alô!\ldots{} A revolução está prestes a se tornar vitoriosa!\ldots{} Prezados
ouvintes! patriotas devotados desta grande terra vilipendiada! já
tomamos todas as estações de\ldots{} Também! alô! alô! estou recebendo
notícias! alô!\ldots{} alô!\ldots{} Urraaaaaaa! tomaram"-se os Correios e
Telégrafos! Os Correios e Telégrafos!\ldots{} Tomaram"-se os Correios e
Telégrafos!\ldots{} Ainda se luta com violência no Bairro Dourado mas a
vitória há"-de ser nossa, guardem os rádios ligados! Prezados ouvintes!
Estou recebendo notícias, não desliguem o rádio!\ldots{} Vamos agora executar
a valsa \emph{Perfil duro}, enquanto esperamos notícias\ldots{}

\hfill{}(A mulher impaciente fecha o rádio.)

{[}\ldots{}{]}

\pagebreak
\section*{VII\break GRANDE CORAL DE LUTA}

\hfill\parbox{150pt}{
(Há incêndios lá pelo centro urbano manchando de vermelho o ambiente. Um
situacionista na fuga, pulou o muro e veio se abrigar nas saias das
mulheres, mas as furiosas o estraçalharam, é aquela posta informe de
sangue. O clarão esplêndido duma bomba cega por um segundo, caiu o muro
do cortiço, a luta se generaliza em pleno palco. As mulheres entram
nela.)
}

\emph{Todos}:

\begin{verse}
É guerra! É guerra!\\
É revolução!\\
É de parte a parte\\
Fogo na nação!
\end{verse}

%\includegraphics[width=4.51458in,height=0.77639in]{media/image1.jpeg}

\begin{quote}
(Textos e música tradicionais no Brasil.)
\end{quote}

\section*{VIII\break O RÁDIO DA VITÓRIA}

O \emph{Rádio} (rapidíssimo, gritadíssimo):

-- Alô! alô!\ldots{} Vitória! \textsc{vi"-tó"-riaaaa}!\ldots{} O Bairro Dourado caiu! caiu!
os gigantes morreram! Alô! Patriotas! Patriotas! o presidente
suicidou"-se"-o"-Quegê"-se"-entregou, se entregou! os anões se converteram à
grande causa pública! a vitória é completa! Vi"-tó"-ria! \textsc{vi"-tó"-ria}!\ldots{}
\textsc{vi"-tóóóóó"-ria!}

\hfill\parbox{150pt}{
(A meninota fatigadinha, desinteressada fecha o rádio e vai dormir. Que
durma sossegada e viva dias novos melhores.)
}

{[}\ldots{}{]}

\section*{IX\break HINO DA FONTE DA VIDA}

\hfill\parbox{150pt}{
(Durante o Rádio da Vitória principiam entrando pelo pátio, fugindo
desvairados deputados, gente chique, que caem por aí mortos. Nisto,
ferocíssima, inteiramente irracional, desgrenhada, o rosto horrendo de
volúpia sanhuda, entra correndo a Mãe. Está rasgada, um seio à mostra,
nas mãos uma bandeira enorme, vermelha"-e"-branca. Entra correndo, pula a
posta sangrenta do soldado estraçalhado. E canta, estática, na
apoteose.)
}

A \emph{Mãe} (solo) e todo o coral misto:

\begin{verse}
Eu sou a fonte da vida\\
Do meu corpo nasce a terra\\
Na minha boca floresce\\
A palavra que será.

\textsc{eu sou aquele que disse:}\\
Os homens serão unidos\\
Se a terra deles nascida\\
For pouso a qualquer cansaço.

Eu odeio os que amontoam\\
Eu odeio os esquecidos\\
Que não provam deste vinho\\
Sanguíneo das multidões.

É deles que nasce a guerra\\
E são a fonte da morte\\
Eu sou a fonte da vida:\\
Força, amor, trabalho, paz.

E se a força esmorecer\\
E se o amor se dispersar\\
E se o trabalho parar\\
E a paz for gozo de poucos

\textsc{eu sou aquele que disse}:\\
Eu sou a fonte da vida\\
Não conta o segredo aos grandes\\
E sempre renascerás.

\textsc{força}!\ldots{} \textsc{amor}!\ldots{} \textsc{trabalho}!\ldots{} \textsc{paz}!\ldots{}

\hfill(Pano)
\end{verse}

\part{\textsc{poemas esparsos e póstumos}}

\chapter{\textsc{anteriores ao modernismo}}

\section{O primeiro poema\footnote[*]{Nota de Mário de Andrade: ``Este
  poema\ldots{} surrealista foi inventado, com a respectiva música, em
  criança'' (\textsc{andrade}, Mário de. \emph{Poesias completas}. 2 v. Edição de
  texto apurado, anotada e acrescida de documentos por Tatiana Longo
  Figueiredo e Telê Ancona Lopez. Rio de Janeiro: Nova Fronteira, 2013,
  p. ).}}

\begin{verse}
Fio"-ri de"-la"-pá\\
Ge"-ni trans"-fé"-li gú"-i"-di nus pi"-gór"-di\\
Ge"-ni"-trans fe"-li gu"-i"-nór"-di\\
Ge"-ni
\end{verse}

\pagebreak
\section{Soneto}

\begin{verse}
Tanta lágrima hei já, senhora minha,\\
Derramado dos olhos sofredores,\\
Que se foram com elas meus ardores\\
E a ânsia de amar que de teus dons me vinha.

Todo o pranto chorei. Todo o que eu tinha,\\
Caiu"-me ao peito cheio de esplendores,\\
E em vez de aí formar terras melhores,\\
Tornou minha alma sáfara e maninha.

E foi tal o chorar por mim vertido,\\
E tais as dores, tantas as tristezas\\
Que me arrancou do peito vossa graça,

Que de muito perder, tudo hei perdido!\\
Não vejo mais surpresas nas surpresas\\
E nem chorar sei mais, por mor desgraça!
\end{verse}

\pagebreak
\section{O retrato}

\begin{verse}
O meu peito é uma sala de castelo,\\
Sala deserta, sala muda, fria,\\
Sem um riso de flor, sem a alegria\\
Duma açafata ou de algum pajem belo.

Solenidade e poeira. Fora o dia,\\
Festa do azul, do róseo, do amarelo,\\
Mas dentro apenas o ligeiro anelo\\
Da luz mortiça duma fresta esguia.

Paredes guarnecidas de veludo,\\
Alfaias, móveis, almadraques, tudo\\
Cobre a penumbra com seu olho absorto.

E há o retrato do meu antepassado,\\
O meu eu de criança abandonado,\\
O meu primeiro coração, já morto.
\end{verse}

\pagebreak
\section{Epitalâmio}

\begin{verse}
O alto fulgor desta paixão insana\\
Há"-de cegar os nossos corações\\
E deserdados da esperança humana\\
Palmilharemos por escuridões\ldots{}

Não mais te orgulharás da soberana\\
Beleza! e eu, minhas cálidas canções\\
Não mais dedilharei com mão ufana\\
Na harpa de luz das minhas ilusões!\ldots{}

Pela realização que ora se ultima\\
Vai apagar"-se em breve o alto fulgor\\
Que te inflama e ilumina o meu desejo\ldots{}

Como no último verso a última rima,\\
Eu deporei, sem gozo e sem calor,\\
Meu derradeiro beijo no teu beijo!
\end{verse}

\pagebreak
\section{A culpa}

\begin{verse}
Não tem ninguém por si, ninguém que o estime.\\
Percebe em todos natural repulsa.\\
Sorri. Ninguém sorri. E, à dor que o oprime,\\
Sai"-lhe a risada, esgar, torta e convulsa.

Jamais pratica um mal, jamais um crime\\
Dentro em seu peito encontra abrigo e pulsa:\\
Mas vai, sem ter um ombro a que se arrime,\\
De coração sem eco, de alma avulsa.

Desde que assim se viu perdeu a calma,\\
Busca em ânsia um amigo e ao grande mundo\\
Só vê desertos a florir abrolhos\ldots{}

Até para chorar, no fundo da alma,\\
Precisou de cavar um poço fundo\\
Onde ecoassem os prantos dos seus olhos.
\end{verse}

\pagebreak
\section{Minha epopeia}

\subsection*{I}

\begin{verse}
Na mocidade corajosa e forte,\\
Abandonando as ilusões vadias,\\
Através de verões e de invernias,\\
Sem desconfianças, afrontando a morte,

Calmo, sorrindo sobranceiro à sorte,\\
Transpondo pessimismos e heresias,\\
Fui em busca de ti, que me sorrias\\
Na mocidade corajosa e forte.

Viajei a vida, o mar que desengana\\
E vagalhões de lágrimas enrista\\
Contra os assomos da esperança humana.

Voltei, das horas juvenis ao coro,\\
Trazendo como esplêndida conquista,\\
Teu coração, meu velocino de ouro.
\end{verse}

\medskip
\subsection*{II}

\begin{verse}
Voltei, como Jasão voltando de Argos,\\
Trazia o teu amor, mas o perdi,\\
E hoje, a primeira vez curtindo amargos\\
Pesares, sinto"-me infeliz sem ti.

Tantos foram os males e os embargos!\\
Tantos naufrágios pelo mar sofri!\\
Mas pude enfim, sobre os meus ombros largos\\
Carregando o teu corpo, vir aqui\ldots{}

E agora, despojado, sem futuro,\\
Sinto às espáduas muito mais pesar\\
A indiferença do horizonte escuro\ldots{}

Oh voltas da fortuna singular!\\
Como Jasão pude vencer, seguro,\\
Mas, como Eneias, fujo a cambalear!
\end{verse}

\medskip
\subsection*{III}

\begin{verse}
E, como Ulisses, vou partir à toa.\\
Nalguma landa regelada, ao fundo\\
Do hemisfério do sul, onde infecundo\\
Não brilha o sol que dentre as brumas coa,

Batalhador sem pré, meditabundo,\\
Com o gelo que se racha e se esboroa,\\
Buscarei levantar outra Lisboa,\\
Maior! maior! a capital do mundo!

Dominando meu reino, a defendê"-lo,\\
Subirão as mil torres da cidade\\
Maldita, lacerando o céu traidor\ldots{}

E morrerei, sem prantos, sobre o gelo\\
Vendo os mil punhos da infelicidade\\
Erguidos contra a vida e contra o amor!\ldots{}
\end{verse}

\pagebreak
\section{Eterna presença\footnote[*]{Nota de Mário de Andrade: ``O soneto
  \textsc{xxviii} de Cláudio Manuel da Costa (ed. Garnier) repete quase a mesma
  ideia, terminando `Tão longe dela estou e estou tão perto'. Gosto
  muito de Cláudio Manuel, talvez seja isso'' (\textsc{andrade}, Mário de, op.
  cit., p. ).}}

\begin{verse}
Este feliz desejo de abraçar"-te,\\
Pois que tão longe tu de mim estás,\\
Faz com que te imagine em toda a parte\\
Visão, trazendo"-me ventura e paz.

Vejo"-te em sonho, sonho de beijar"-te;\\
Vejo"-te sombra, vou correndo atrás;\\
Vejo"-te nua, oh branco lírio de arte,\\
Coroando"-me a existência de rapaz\ldots{}

E com ver"-te e sonhar"-te, esta lembrança\\
Geratriz, esta mágica saudade,\\
Dá"-me a ilusão de que chegaste enfim;

Sinto alegrias de quem pede e alcança\\
E a enganadora força de, em verdade,\\
Ter"-te, longe de mim, juntinho a mim.
\end{verse}

\chapter{\textsc{poemas em jornais e revistas}}

\section[Obsessão]{Obsessão\break(1921)\footnote[*]{Recorte de jornal sem identificação, de
  acordo com Telê Ancona Lopez e Tatiana Longo Figueiredo (\textsc{andrade},
  Mário de, op. cit., p. ).}}


\begin{verse}
\ldots{} Na noite boca aberta num bafo rescaldo de mato\\
Escravos cabindas bum"-bum bailam\ldots{}

Lentos, lânguidos, olhos palermas, dentes brancos,\\
Escravos cabindas batendo umbigadas\ldots{}

Vem do escuro da noite o convite carnal das sovacas,\\
Os negros remexem ardentes batendo umbigadas\ldots{}

Os negros resfolegam fungando batendo umbigadas\ldots{}\\
Primeiros corpos fugindo na sombra da noite\ldots{}

Os negros fungando rolando na sombra da noite\ldots{}\\
Últimos arrancos do samba bum"-bum banzo\\
No bodum grosso dos corpos largados\ldots{}
\end{verse}

\pagebreak
\section[Franzina]{Franzina\footnote[*]{Publicado em \emph{La Paga}, nº 1, revista
  italiana de São Paulo, de acordo com Telê Ancona Lopez e Tatiana Longo
  Figueiredo (\textsc{andrade}, Mário de, op. cit., p. ).}}

\begin{verse}
Franzina,\\
Estrangeira, londrina,\\
Sobre os ombros a névoa do organdi\ldots{}\\
Reaparecida em minha sensação!

Estávamos os dois quase juntos, juntinhos,\\
\qquad\quad Povo\\
\qquad\quad Parque Antártica\\
Insulados na multidão erva de campo indiferente.\\
Era gostoso estar assim unidos\\
\qquad\qquad\qquad\qquad\qquad\qquad esquecidos\ldots{}\\
Qual o teu nome? o meu?\\
Seguindo a bola.\\
\qquad\qquad\qquad Campeonato\\
\qquad\qquad\quad\textsc{apea}\\
\qquad\qquad Taça\\
Os dois apaixonados pelo jogo.\\
\qquad\qquad\qquad\qquad\qquad Por nós.

Falta muito?\\
-- Dez minutos.\\
-- Meu Deus!\\
-- É agora!\\
E foi. Bianco avançou demais; Guariba\ldots{} não; Netinho centrou Mário caiu, mas Formiga emendou e a bola\ldots{}\\
\quad\quad\quad\quad\quad\quad{}Friedenreich!\\
\quad\quad\quad\quad\quad{}Goal!\\
\quad\quad\quad\quad{}Delírio"-vinho!\\
\quad\quad\quad{}Alegria bacante!\\
\quad\quad{}As Grandes Dionisíacas!\\
\quad{}Elaphebolion em dezembro!\\
Alle"-goak, goak, goak!\ldots{}

Olhaste"-me brasileira\\
\quad\quad\quad\quad{}Paulistano\\
Com duas lágrimas nas hortências dos teus olhos;\\
E teu ombro apoiou"-se no meu peito de rapaz\ldots{}\\
Asa de pomba! asa de pomba de organdi!\ldots{}

Franzina,\\
Reapareces"-me agora na lembrança,\\
Doce como a pálpebra que se fecha para o sonho\ldots{}\\
Ai! saudade de amor!\\
Ai! sublime tortura!\\
Ai! memória de peito comovido\\
Onde poisa macia uma asa de mulher!\ldots{}
\end{verse}

\pagebreak
\section[Losangos arlequinais]{Losangos arlequinais\break(Sonetos condensados)\footnote[*]{Publicado em \textsc{O mês modernista}, em 2 de janeiro de 1926, seção do jornal carioca \emph{A
  Noite}, como lembram Telê Ancona Lopez e Tatiana Longo Figueiredo
  (\textsc{andrade}, Mário de, op. cit., p. ).}}

\subsection*{Bar}

\begin{verse}
Aperitivo em que há um ressaibo\\
Do sírio em frente. Palmas. Almas\\
Penadas de garçons. Fagote.\\
Abas largas. Quedê o cowboy?\ldots{}
\end{verse}

\medskip
\subsection*{Futebóler}

\begin{verse}
Pan moderno de calça curta.\\
(O inglês mudou Sirinx em bola)\\
Tu não tens cornos mas tens corners,\\
Único orgulho do Brasil!
\end{verse}

\medskip
\subsection*{Telefone}

\begin{verse}
Catleya Nigra das paredes.\\
O botão longo pende murcho.\\
Guarda o minuto dos encontros\ldots{}\\
-- Alô, as horas, faz favor?
\end{verse}


\medskip
\subsection*{Luz elétrica}

\begin{verse}
Milagre! a usina distribui\\
Sol em ampolas. Josué. Segue\\
Perfeito o dia, quem diria\\
Que é uma linda noite de luar!
\end{verse}


\medskip
\subsection*{Cabaré}

\begin{verse}
Gâmbias"-machuchos e maxixes\\
Chochos, machos mochos, mocinhos,\\
Amor fino, morfina, amorfo\\
Nó de corpos\ldots{} ``\textsc{grupo escolar}''.
\end{verse}

\medskip
\subsection*{Mulher}

\begin{verse}
Eterna novidade: Rendas.\\
Jóias, despesas, cremes, crimes,\\
Espelhos, rouge, pós, perfumes,\\
Pernas, peles, plumas\ldots{} Amor?
\end{verse}

\medskip
\subsection*{Rui Barbosa}

\begin{verse}
Gênio genioso, andor brasílico\\
Nas procissões anti"-germânicas,\\
Errou bastante na política,\\
\textsc{mas nunca errou no português!}
\end{verse}

\medskip
\subsection*{Sífilis}

\begin{verse}
Si Filis te ama, aceita o beijo\\
Si filhos tens, dá"-lhes paterno\\
Tua gloriosa e única herança\\
A Rosa Rubra universal!
\end{verse}

\medskip
\subsection*{Mário de Andrade}

\begin{verse}
Lará, leré, lirí, loró,\\
Lurú. Serei mesmo só isso?\ldots{}\\
Mas\ldots{} si l'histoire vous embête\\
Je pourrai la recommencer\ldots{}
\end{verse}

\parbox{\textwidth}{
Post"-scriptum -- Bem considerando os sonetos já lidos valem bem os 50
bagarotes porém como me simpatizo com os leitores da \emph{Noite}
resolvo mandar mais um de graça.}


\bigskip
\subsection*{Alegria\ldots{} de choro}

\begin{verse}
Sábio grosso em ciência magriça,\\
Tardonho egresso de Sankara,\\
Por integrar"-se no Infinito\\
Ficou o Infinitesimal!
\end{verse}

\pagebreak
\section[Seção livre]{Seção livre\break Comunicação urgente\footnote[*]{Publicado em \emph{Terra roxa e
  outras terras}, a.1, nº 5. São Paulo, 27 de
  abril de 1926, p. 5, de acordo com Telê Ancona Lopez e Tatiana Longo
  Figueiredo (\textsc{andrade}, Mário de, op. cit., p. ).}}

\medskip

\begingroup\centering\parbox{165pt}{
Devido a vários jornalistas de São Paulo e o sr. Tasso da Silveira do
Rio de Janeiro terem afirmado que não sou poeta e devido a terem
afirmado o contrário os srs. Martim Damy, Sérgio Milliet e Martins de
Almeida, pra tranquilizar o público e evitar futuros equívocos
históricos venho comunicar e jurar solenemente \textsc{que sou poeta.}

\hfill{}São Paulo, 20 de abril de 1926.

\hfill{}Mário de Andrade}\endgroup

\pagebreak
\section[A morte que ri!]{}

\begin{verse}
A morte que ri!\ldots{}\footnotemark\\
a morte que baila!\\
a gostosura de se morrer bastante\\
mas não é muito\\
como no Carnaval!\\
a espécie desonrosa de morte que\\
consiste em fingir que ainda vive, Lorca!

Lorca! Lorca!\\
Cadáver espedaçado de Lorca! Sombra de Lorca!\\
E vós todos, sacrificados da inteligência!

Espectros de crimes! Fantasmas da opressão\\
Descei nestes céus demasiadamente agradáveis!\\
Ocultai a alegria entorpecente da nossa Via"-Láctea!\\
Lorca! Sombra de Lorca!\\
Esconde no teu cadáver despedaçado a glória fácil do Cruzeiro!\\
Fica no seu lugar!
\end{verse}

\footnotetext{O poema aparece como parte do artigo \textsc{lorca},
  \textsc{pobre de nós}!, publicado em \emph{Leitura} (a. 2, nº 15, Rio de
  Janeiro, fevereiro de 1944), como afirmam Telê Ancona Lopez e Tatiana
  Longo Figueiredo (\textsc{andrade}, Mário de, op. cit., p. ).}

\chapter{\textsc{poemas na correspondência}}

\section[Noturno n° 4]{Noturno n° 4\break(janeiro de 1924)\footnote[*]{``Poema enviado a Anita Malfatti na carta de 3 de janeiro de 1924, autógrafo a tinta preta em folha de papel
  branco pautado, filigranado, tendo no anverso o poema \textsc{Noturno
  n° 3} (V. \textsc{batista}, Marta Rossetti, org. \emph{Op. cit.}, p. 72--73;
  Arquivo Anita Malfatti, \textsc{ieb"-usp})'', de acordo com as organizadoras das
  \emph{Poesias completas} (\textsc{andrade}, Mário de, op. cit., p. ).}}


\begin{verse}
Com este calor quem dormirá!\ldots{}

A escuridão acumulou"-se em minha rua\\
E encapuça a cabeça alemã dos lampiões\ldots{}

Eu preciso de alguém\ldots{}\\
Meus olhos varrem a escuridão.\\
Mas somente o calor a se mexer\\
Sob a vigilância implacável das estrelas.\\
Dir"-se"-ia que os burgueses dormem\ldots{}\\
\qquad\qquad\qquad Casais suados\\
\qquad\qquad\qquad Virgens vazias\\
\qquad\qquad\qquad Crianças descobertas\ldots{}\\
O que mais me comove é pensar nos solteirões.\\
Os solteirões mastigam o silêncio,\\
Os solteirões roncam e viram de lado na cama\\
Ofegando em silvos malcheirosos\ldots{}\\
\quad\qquad\qquad\qquad\qquad -- São sonhos imorais.

A noite hesita em seguir para a frente.\\
De repente deitou"-se nas hortênsias.

E eu velo.\\
Eu velo o sono dos burgueses\\
Condescendentemente.
\end{verse}

\pagebreak
\section[Reza de fim de ano (5º Noturno)]{Reza de fim de ano (5º Noturno)\break(31 de dezembro de 1923 -- 1º de janeiro de 1924)\footnote{``O poema, logo ao final da
  primeira versão, foi postado para Anita Malfatti em 3 de janeiro de
  1924 (V. \textsc{batista}, Marta Rossetti, org. \emph{Op. cit.}, p. 67--72;
  Arquivo Anita Malfatti, \textsc{ieb"-usp}), de acordo com Telê Ancona Lopez e
  Tatiana Longo Figueiredo (\textsc{andrade}, Mário de, op. cit., p. ).}}

\begin{verse}
Senhor, é 31\ldots{}\\
Deixei a noite lá.\\
\qquad\quad Pauliceia alegre farrista,\\
\qquad\quad Sacudida em fordes dodges\ldots{}\\
\qquad\qquad\qquad\qquad\qquad Riscos de disparadas.\\
\qquad\qquad\qquad 23 no relógio do Correio.\\
\qquad\qquad\qquad Yes, we have no bananas\ldots{}\\
\qquad\quad Mas ninguém se incomoda!\\
\qquad\quad Pauliceia dança empetecada de cinemas e confeitarias\ldots{}\\
Tive raiva de mim\\
Senhor, por que meus olhos se abrem só pro que é gozo?\\
Delícia destas cargas de automóveis!\\
Entrar na guanabara dos bares\\
Levando nos braços uma colorida,\\
Pedir champanha para dois!\\
Senhor, muito obrigado por me teres feito pobre!\\
Tenho pena de mim.\\
Mas a culpa será dos meus desejos?\\
Por que a eletricidade só ilumina fêmeas e prazeres?\\
Não vejo mais as torres! Não vejo mais as torres!\\
E unicamente nos livros que falam de arquitetura\\
Ainda se encontram teus campanários, Senhor,\\
Assustados, agrupados que nem urus durante a chuvarada.\\
No entanto eu penso na arquitetura maravilhosa da tua Trindade.\\
Pauliceia se tornou indiferente pra mim.

Senhor, é 31\ldots{}\\
Eu te agradeço este ano que me deste.\\
Que o novo seja igual ao que passou\ldots{}\\
\qquad\qquad\qquad\qquad\quad Alegrias bombásticas\\
\qquad\qquad\qquad\qquad\quad Sofrimentos redundantes\\
\qquad\qquad\qquad\qquad\qquad\qquad\quad Retumbantes\\
\qquad\qquad\qquad\qquad\qquad\quad Samba\\
\qquad\qquad\qquad\qquad\qquad Villa"-Lobos\\
\qquad\qquad\qquad\qquad\quad Meus amigos de Paris\\
\qquad\qquad Águas"-fortes de Seewald\\
\qquad\qquad Águas"-fortes de Chagall\\
\qquad\qquad\qquad\qquad\qquad Sangue\\
\qquad\qquad\qquad\qquad\quad Pranto e riso\\
\qquad\qquad\quad Quanta coisa! Quanta coisa!\\
Shakespeare disse que a experiência é uma joia\\
Sou rei todo enfeitado de joias colossais!

\qquad\qquad \textsc{apito}\\
\qquad Mais apitos.\\
\qquad São milhares!\\
Gritos batuques inferneira\\
A barulheira entra no quarto de repente\\
As sensações desordenadas me puxam.\\
\qquad\qquad\qquad Bala de estalo\\
\qquad\qquad Arrebento em risos convulsos\\
\qquad\qquad Não tem tréguas!\\
\qquad Ano Novo!\\
Vaias alegrias apitos fuzilando no ar\\
Não posso mais escrever! Não posso mais sonhar!\\
Meus amigos estão ceiando no Esplanada.\\
\qquad\qquad Imoralidades engraçadas\\
\qquad\qquad Gargalhadas\\
\qquad\qquad Anedotas\\
\qquad\qquad Danças\\
Eu só mastigo umas hesitações\ldots{}\\
Senhor, não posso mais!\\
Me liberta de Ti! Me liberta de Ti!\ldots{}

Senhor, eu não sou mau\ldots{}\\
Sou tal e qual os outros homens desta Terra,\\
Bom e mau, mau e bom no mesmo tempo.\\
Senhor, se a gente odeia também sabe amar.\\
A gente peca sem que nenhum se esqueça de se arrepender.\\
Só Tu me compreenderás estes últimos versos\\
Porque me deste aquela ideia de que os padres falam da tua Justiça,\\
Pautando"-a pela vaidosa justiça nossa.

Senhor, o ano novo principiou.\\
Nada te peço, não.\\
És Pai, deves saber melhor do que careço.\\
Mas não deixes de pôr nos teus presentes\\
Um pouco mais de virtudes cristãs.\\
Eu queria ter a energia do Cubismo\\
E a força também\ldots{}\\
E que nunca me baste a mim mesmo, Senhor!\\
Me assustam as solidões orgulhosas.\\
Quero andar junto dos outros pra compreender o equilíbrio do corpo,\\
Quero muito pedir pra penetrar no segredo da esmola.\\
Faze de mim um Jesusinho pesadíssimo!\\
Que a humanidade seja um São Cristovão pra mim!

Mas porém acredito que os meus pedidos estão todos errados\\
Pois só Tu sabes o que faz falta pra mim\ldots{}\\
Prefiro antes te dar os meus valores\\
Eu sou muito mais rico do que Tu\\
Porque posso te dar minha alma unicamente pra Ti,\\
Tu não podes dar a tua só pra mim!\\
Eu te preparei o maior dos presentes,\\
Um presente que nunca poderás me dar\ldots{}\\
\textsc{senhor eu te ofereço o meu pecado!}\\
\qquad\qquad\qquad\qquad\qquad\qquad Blasfêmias,\\
\qquad\qquad\qquad\qquad\qquad\qquad Volúpias\\
\qquad\qquad\qquad\qquad\qquad\qquad Raivas\\
\qquad\qquad\qquad\qquad\qquad\qquad Desesperos\ldots{}

Não careço de os enumerar mais,\\
Devem estar bem gravados em Ti,\\
Sangrando pelas tuas feridas incompreensíveis\\
E ainda posso te dar mais\\
Porque te dou meu Arrependimento!

Mais um ano vai principiar\ldots{}\\
Estou tranquilo, bem feliz.\\
Ainda apitos de fábricas retardatárias\ldots{}\\
\qquad\qquad\qquad\qquad\quad Time is money\ldots{}\\
\qquad Deviam ser expulsas desta cidade industrial.\\
Alguns sinos ondulantes já se trançam\\
Nos filamentos verticais dos apitos\ldots{}\\
São tuas missas e teus cânticos\ldots{}\\
Sinos\ldots{} Bimbalhadas\ldots{} Os Votivos\ldots{}\\
As preces subidas\ldots{} As graças vertidas\ldots{}\\
Vós tereis o bronze\ldots{}\\
\qquad Vós tereis a minha lembrança\ldots{}\\
\qquad\qquad Vós tereis a minha coleção de águas"-fortes\ldots{}\\
Minhas ideias se baralham.\\
Estou que nem Cendrars no fim de \emph{Pâques}.\\
Não tenho mais vontade de rezar\ldots{}\\
Quereria beber alguma coisa\ldots{}\\
Vou sair.\\
Um whisky retempera\\
\qquad Forte\\
\qquad Sem água\ldots{}\\
Duas sandwiches\ldots{}\\
Me esquecia: Senhor, o pão nosso de cada dia nos dai hoje\ldots{}
\end{verse}

\chapter{\textsc{em manuscritos}}

\section{Viola quebrada\footnote[*]{O manuscrito, com letra e música, está
  guardado no Arquivo Mário de Andrade, \textsc{ieb"-usp}, como lembram Telê
  Ancona Lopez e Tatiana Longo Figueiredo (\textsc{andrade}, Mário de, op. cit.,
  p. ).}}

\begin{verse}
Quando da brisa\\
No açoite\\
A frô"-da"-noite\\
Se acurvô\\
Fui s'incontrá\\
Cum a Maroca,\\
Meu amô,\\
Eu tive n'arma\\
Um choque duro\\
Quando ao muro\\
Já no escuro\\
Meu oiá\\
Andô buscando\\
A cara dela\\
E num achô!

Minha viola gemeu\\
Meu coração estremeceu,\\
Minha viola quebrou\\
Teu coração me deixou!

Minha Maroca\\
Arresorveu\\
Por gosto seu\\
Me abandoná\\
Pruque os fadista\\
Nunca sabe\\
Trabaiá.\\
Isso é bestera\\
Que das frô\\
Que bria e chera\\
A noite intera\\
Vem apois\\
As fruita\\
Que dá gosto\\
Saboriá!

Minha viola gemeu\\
Meu coração estremeceu,\\
Minha viola quebrou\\
Teu coração me deixou!

Pru causa dela\\
Sô rapaiz\\
Munto capaiz\\
De trabaiá\\
E as noite intera\\
Os dia intero\\
Capiná;\\
Eu ser carpi\\
Pruque a minha alma\\
Está arada\\
Arroteada\\
Capinada\\
Com as foiçada\\
Dessa luiz\\
Do seu oiá!

Minha viola gemeu\\
Meu coração estremeceu,\\
Minha viola quebrou\\
Teu coração me deixou!
\end{verse}

\pagebreak
\section{Canção marinha\footnote[*]{Como afirmam Telê Ancona Lopez e
  Tatiana Longo Figueiredo, ``Canção marinha'' é poema ``musicado por
  Marcelo Tupinambá, para canto e piano; n\textsuperscript{o} 2 na 2ª
  série das \emph{Canções brasileiras}'' (\textsc{andrade}, Mário de, op. cit.,
  p. ).}}

\begin{verse}
Tantos peixinhos nas águas,\\
Tanta escama dos peixinhos!\\
No coração tantas mágoas,\\
Das mágoas tantos espinhos.

A alegria que aparece\\
Foge depois, nem se vê;\\
Também a onda brota, cresce,\\
morre sem saber por quê.

Por mais que a tristeza escondas,\\
Nunca a escondes muito bem,\\
É como o sulco das ondas,\\
Surge ora aqui, ora além.

Prazer que mais se deseja\\
Quanto mais custoso e incerto,\\
Espuma branca que alveja\\
Mais de longe que de perto!

Na vida só brotam mágoas\\
No mar bolhas entre escolhos,\\
Há olhos de luz nas águas\\
Há luzes de água nos olhos.
\end{verse}

\pagebreak
\section[Cântico]{Cântico\break(São Paulo, 6 de maio de 1933)\footnote[*]{Em nota, Telê Ancona Lopez e Tatiana Longo
  Figueiredo lembram que o poema está ``na caderneta \textsc{Poemas}, no
  manuscrito de \emph{Lira paulistana} (Arquivo Mário de Andrade,
  \textsc{ieb"-usp})'' (\textsc{andrade}, Mário de, op. cit., p. ).}}

\begin{verse}
Os tufões vieram de todas as partes\\
Devastando as cidades, desgalhando a esperança\\
Secando a mais última gota das águas do amor\\
Eu não imaginava que a minha alma\\
Estivesse (tão) desamparada assim\\
As grandes nuvens tomaram toda a vastidão do céu\\
E o sol se divorciou para sempre deste mundo vão.\\
Oh morte! Teu desejo implacável como uma polvadeira seca\\
Me deforma a vida em redor\\
E pousa áspera, sem limites, consentida\\
Sobre a minha alma desamparada\\
Sobre mim.
\end{verse}

\pagebreak
\section[Nova canção de dixie]{Nova canção de dixie\break(São Paulo, 25 de janeiro de 1944)\footnote[*]{Telê Ancona Lopez e Tatiana Longo
  Figueiredo afirmam: ``Manuscrito no Arquivo Mário de Andrade, \textsc{ieb"-usp}.
  Publicação póstuma, como \textsc{Um poema inédito de Mário de Andrade/
  Nova canção de Dixie} (\emph{Correio Paulistano}. São Paulo, 24 de
  fevereiro de 1946)'' (\textsc{andrade}, Mário de, op. cit., p. ).}}


\begin{verse}
Kenst du das Land\\
Où fleurit l'oranger?\ldots{}\\
É a terra maravilhosa\\
Nascida duma barquinha\\
``Flor de Maio'' se chamava,\\
Onde tudo o que é bom dava,\\
Que tudo o que é rico tinha\ldots{}

Lá quem queira gozar goza\\
Com toda a felicidade,\\
É só passear pela rama,\\
É só não ser tagarela:\\
É a terra maravilhosa,\\
Parece com a liberdade\\
Pois já tem a estátua dela.

É a terra dos plutocratas,\\
Palácios de cem andares,\\
Você sai se faz questão,\\
Mas pode ficar nos ares,\\
É só apertar o botão,\\
Que recebe tudo em latas\\
Pela quarta dimensão.

\qquad No. I'll never never be\
\qquad In Colour Line Land.

Mas porque tanta esquivança!\\
Lá tem Boa Vizinhança\\
Com prisões de ouro maciço;\\
Lá te darão bem bom lanche\\
E também muito bom linche,\\
Mas se você não é negro\\
O que você tem com isso!

\qquad No. I'll never never be\\
\qquad In Colour Line Land.

É a terra maravilhosa\\
Chamada do Amigo Urso,\\
Lá ninguém não cobra entrada\\
Se a pessoa é convidada.\\
Depois lhe dão com discurso\\
Abraço tão apertado\\
Que você morre asfixiado,\\
Feliz de ser estimado.

\qquad No! I'll never never be\\
\qquad In Colour Line Land.
\end{verse}

\pagebreak
\section[Esboço \textsc{vi}]{Esboço \textsc{vi}\break(São Paulo, 21 de junho de 1944)\footnote[*]{Como lembram Telê Ancona Lopez e Tatiana
  Longo Figueiredo, o texto está ``na caderneta \textsc{Poemas}, no
  manuscrito de \emph{Lira paulistana} (Arquivo Mário de Andrade,
  \textsc{ieb"-usp})'' (\textsc{andrade}, Mário de, op. cit., p. ).}}

\begin{verse}
Cheiro de bueiros\\
Gosto de esgotos\\
Olhos dormentes\\
Unhas dementes,\\
Merda.

Bafo de mofos\\
Choro de enxurros\\
Crianças sujas\\
Barbas corujas\\
Merda.

O sr. Comissário\\
Da Comissão\\
Do Controle\\
Da Cooperação\\
Nacional\\
Sem igual\\
Patrocinadora\\
Do Racionamento\\
Do Renascimento\\
Da Pátria Sagrada\\
Da Pátria Amada\\
Salve Salve\\
Merda Merda\\
Que os pobres deserda\\
Na infelicidade\\
Maiores Menores\\
Uns da mão direita\\
Outros só da esquerda\\
Merda.
\end{verse}

\chapter{\textsc{traduções feitas por mário de andrade}}

\section{Premier nocturne\footnote[*]{De acordo com Telê Ancona Lopez e
  Tatiana Longo Figueiredo, o manuscrito do poema está no Arquivo Mário
  de Andrade, \textsc{ieb"-usp}. Como lembram as estudiosas, ``\textsc{ma} recriou, em
  francês, poemas seus. Sua tradução de \textsc{Noturno} e
  \textsc{Paisagem nº 4,} de \emph{Pauliceia desvairada}; de \textsc{São
  Pedro,} publicado na revista \emph{Klaxon}, assim como
  \textsc{Momento}, que saiu em \emph{A Idea ilustrada}, foram
  difundidos em ``\textsc{Território da tradução}'', número especial de
  \emph{Remate de Males: revista do Departamento de Teoria Literária da
  Unicamp} (n\textsuperscript{\emph{o}} 4; Campinas, dez. 1984, p.
  17--18), com notas de Iumma M. Simon, Alexandre Eulálio e Charlotte
  Galves, esta responsável pela revisão, à qual a {[}\ldots{}{]} edição de
  \emph{Poesias completas} recorreu'' (\textsc{andrade}, Mário de, op. cit., p.
  ).}}

\begin{verse}
Reverbères du Camboucy par cette nuit de crime\ldots{}\\
Chaleur! Et la nue basse, grosse, épaisse,\\
faite de corps de papillons nocturnes,\\
rasant l'epiderme des arbres\ldots{}

Les trams se dandinent, comme un feu d'artifice,\\
secoués par les rails,\\
crachant un trou dans les ténèbres blanches\ldots{}

Dans un parfum d'héliotrope et de boue\\
passe une fleur"-du"-mal\ldots{}\\
Elle est venue du Tourkestan.\\
Elle a des yeux cernés, qu'obscurcissent les âmes\ldots{}\\
On dit que l'argent se fond entre ses ongles violets\\
dans les tripôts de Ribeirão Preto.

-- Batat'assat'ô fourn!\ldots{}

Reverbères du Camboucy par cette nuit de crime!\ldots{}\\
Chaleur\ldots{} Et la nue basse, grosse, épaisse,\\
faite de corps de papillons nocturnes,\\
rasant l'épiderme des arbres\ldots{}

Un mulâtre doré,\\
à la chevelure faite d'anneaux pollis\ldots{}\\
Guitarre!\ldots{} ``Quand je mourrai\ldots{}''\\
Une pesante odeur de vanilles oscille, tombe et roule dans la rue.\\
Vague dans l'air la nostalgie des pays chauds.

Et toujours les trams comme un feu d'artifice,\\
secoués par les rails,\\
creusant un trou dans les ténèbres blanches\ldots{}

-- Batat'assat'ô fourn!\ldots{}

Chaleur\ldots{} Les diables voltigent dans l'espace\\
portant des corps de femmes nues.\\
O, les lassitudes des toujours imprévus,\\
et les âmes se reveillant aux mains des enlacés!\ldots{}\\
Idylles à l'ombre des platanes!\\
Et l'universelle envie à la gloire fanfaronne\\
des jupes roses et des cravates roses\ldots{}

Balcons prudents où fleurissent les jeunes Iracemas\\
pour les ardeurs des hommes blancs\ldots{}\\
Est"-ce qu'ils sont vraiment blancs, ces européens?\ldots{}\\
Et que les chiens hurlent aux amants\ldots{}\\
Qu'importe!\\
Tous marchent dans l'allée des Baisers de l'Aventure!\\
Mais moi\ldots{} Mais ces grilles en arabesques de jasmins qui m'emprisonnent,\\
tandis que les impasses du quartier sont ouvertes\\
à la liberté des lèvres qui se cherchent\ldots{}

Arlequinale! Arlequinale!\\
Et la nue basse, grosse, épaisse,\\
faite de corps de papillons nocturnes,\\
rasant l'épiderme des arbres\ldots{}\\
Mais sur ces grilles en arabesques de jasmins qui m'emprisonnent\\
les étoiles délirent en des carnages de lumière,\\
et mon ciel est tout en fusées de larmes!\ldots{}

\ldots{} et toujours les trams, comme un feu d'artifice,\\
secoués par les rails,\\
perforant un trou dans les ténèbres blanches\ldots{}

-- Batat'assat'ô fourn!\ldots{}
\end{verse}

\hfill(Mário de Andrade\textsc{. Noturno,} \emph{Pauliceia desvairada}\textsc{)}

\pagebreak
\section{Paysage n° 4\footnote[*]{Telê Ancona Lopez e Tatiana Longo
  Figueiredo afirmam que o texto do manuscrito encontrado no Arquivo
  Mário de Andrade do \textsc{ieb"-usp} foi ``publicado no número especial
  `\textsc{Território da tradução'}; loc. cit., p. 19--20)'' (\textsc{andrade},
  Mário de, op. cit., p. ).}}

\begin{verse}
Les camions rôdent, les tomberaux rôdent,\\
les rues se déroulent rapides,\\
rumeur sourde et rauque, craquements, fracas\ldots{}\\
Et le choeur large de l'or des sacs pleins de café!

Toutes les voies se dirigent vers le cri anglais de la São Paulo Railway\ldots{}\\
Mais voici les vents des désillusions!\\
Le café est en baisse!\\
Des cracs, des ménaces, des audaces superfines.\\
Les richards se cachent dans leurs plantations\\
\qquad\qquad\qquad\qquad\qquad\qquad caféiers\\
Où sont les hommes qui gouvernent?\\
Mais le Brésil croise ses bras au loin\ldots{}\\
O les indifférences maternelles!\ldots{}\\

Et les camions rôdent, les tomberaux rôdent,\\
les rues se déroulent rapides,\\
rumeur sourde et rauque, craquements, fracas\ldots{}\\
Et le choeur large de l'or des sacs pleins de café!

Lutter!\\
La victoire pour ceux qui luttent seuls!\\
Les drapeaux, les clairons des magazins débordant de café\ldots{}\\
Frapper le premier!\\
Et le Brésil croise ses bras, au loin..

Mais le sacre viendra, avec nos propres mains!\\
Vous, hommes qui gouvernez, en arrière!\\
Victoire!\\
Mettons des colliers de dents ennemies!\\
Couronnons"-nous avec le café mûr!\\
Taratá!\ldots{}\\
Et persiflons le monde entier!

O! Ce suprème orgueuil d'être paulistement!
\end{verse}

\hfill(Mário de Andrade\textsc{. Paisagem nº 4,} \emph{Pauliceia desvairada}\textsc{)}
