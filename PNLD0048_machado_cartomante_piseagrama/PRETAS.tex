\textbf{Monteiro Lobato} \lipsum[1]

\textbf{Um suplício moderno e outros contos} \lipsum[2]

\textbf{Ieda Lebensztayn} é crítica literária, pesquisadora e ensaísta,
preparadora e revisora de livros. Mestre em Teoria Literária e
Literatura Comparada e doutora em Literatura Brasileira pela
Universidade de São Paulo. Fez dois pós-doutorados: no Instituto de
Estudos Brasileiros, \textsc{ieb-usp}, sobre a correspondência de
Graciliano Ramos (Fapesp 2010/12034-9); e na Biblioteca Brasiliana
Mindlin / Faculdade de Filosofia, Letras e Ciências Humanas,
\textsc{bbm/fflch-usp}, a respeito da recepção literária de Machado de
Assis (\textsc{cnp}q 166032/2015-8). Autora de \emph{Graciliano Ramos e
a} Novidade\emph{: o astrônomo do inferno e os meninos impossíveis} (São
Paulo: Hedra, 2010). Organizou, com Thiago Mio Salla, os livros
\emph{Cangaços} e \emph{Conversas}, de Graciliano Ramos, publicados em
2014 pela Record. E, com Hélio Guimarães, os dois volumes de
\emph{Escritor por escritor: Machado de Assis segundo seus pares} --
1908-1939; 1939-2008 (São Paulo: Imprensa Oficial do Estado de São
Paulo, 2019). Colabora no caderno ``Aliás'' de \emph{O Estado de S.
Paulo}.


