\chapter[Muito mais terrível: a vida de Harriet Jacobs]{Muito mais terrível \subtitulo{A vida de Harriet Jacobs}}

\section{sobre a autora}

\noindent{}Nascida em Edenton, na Carolina do Norte, por volta do outono de 1813, Harriet Ann Jacobs era filha de Elijah Knox e Delilah Horniblow. Harriet já nasceu na
condição de escravizada devido ao princípio do \textit{partus sequitur ventrem},
que estabelecia que filhos de escravos já nasceriam em condição de
escravidão. Ainda criança, Harriet foi submetida ao seu dono legal, Dr.\,James Norcom.
Esse senhor de escravos foi abusivo com a garota desde a
infância dela, durante a qual eram frequentes violações e torturas. Para
tentar escapar das mãos de Norcom, Harriet se uniu ao advogado Samuel
Sawyer, com quem teve dois filhos, Joseph e Louisa, ambos também
nascidos em condição de escravizados.


Viveu a tragédia do cativeiro até, em 1835, principiar uma vida em fuga que, após sete anos de percalços, terminou por levá"-la ao Norte em 1842. Primeiro se estabeleceu na Filadélfia, seguindo depois para a cidade de Nova Iorque, e foi finalmente em Boston que Jacobs pôde produzir \textit{Incidentes da vida de uma escrava} que, sem deixar de se inserir no \textit{corpus} dos relatos da escravidão norte-americana, guarda uma singularidade: é pioneiro e inspirador das autobiografias femininas, e joga luz nos horrores que eram partilhados apenas entre as mulheres. “A escravidão é terrível para os homens”, escreve a autora, “mas é muito mais terrível para as mulheres”: Jacobs convive, antes e depois da fuga, com o perverso sistema de assédio e coação sexual contra o qual as escravas pouco podiam lutar. Transmitindo brilhantemente uma vida em prosa crua e seca, \textit{Incidentes} adiciona camadas de complexidade ao horror da escravatura.

Através de sua narrativa, o leitor tem uma aproximação de dentro, em perspectiva única, do que foi o escravismo sulista norte-americano, que às vésperas da abolição da escravatura contava com 4 milhões de escravizados em seus campos de trabalho. Nesse relato o leitor vai encontrar a descrição do universo familiar de Jacobs e suas resistências diárias, mas também a exploração do trabalho e a violência simbólica, física e sexual cometida pelos senhores brancos. Seu testemunho sugere que o objetivo da escravidão negra nas Américas era análogo à utopia autoritária do capital no século \textsc{xxi}: desumanizar o ser humano até reduzi"-lo à condição inanimada e sedutora de uma mercadoria.

Durante todo o seu período de liberdade, Harriet Jacobs foi extremamente ativa no movimento
abolicionista norte"-americano e participou de um grupo, associado ao
jornal \textit{North Star}, de Frederick Douglass, que defendia a liberdade dos negros.
Harriet Ann Jacobs viveu 84 anos, vindo a falecer na cidade de Washington no ano de 1897.


\section{sobre a obra}

\emph{Incidentes da vida de uma escrava, escritos por ela mesma} é a
autobiografia da ex"-escravizada norte"-americana Harriet Ann Jacobs,
publicada originalmente no ano de 1861. A obra representou um marco na
tradição de escrita biográfica sobre a temática da escravidão nos
Estados Unidos.

Quase dez anos antes de seu \textit{Incidentes}, Harriet Jacobs já publicava a sua primeira obra, uma carta destinada ao jornal
\emph{The New York Tribune} no ano de 1853. Esse escrito inicial foi
uma resposta ao ensaio de Julia Tyler, que defendia com veemência a
escravidão. Além de diversas outras cartas publicadas por Harriet, sua
principal obra foi, sem dúvida, a presente autobiografia, autorizada para publicação no ano de 1861 pela editora Thayer and Eldridge. A obra narra toda a sua vitoriosa luta contra a escravidão e em defesa da liberdade.

Durante muito tempo, os relatos e documentos sobre a escravidão
estadunidense vinham a partir de falas e escritos predominantemente
masculinos, fossem da parte dos senhores ou mesmo dos escravizados.
Dessa maneira, \emph{Incidentes da vida de uma escrava} rompe com um
paradigma, trazendo"-nos a perspectiva de uma mulher
escravizada no sul dos Estados Unidos do século \textsc{xix}.

Harriet Ann Jacobs é a primeira mulher negra a escrever uma
autobiografia sobre a sua condição de escravizada e uma das primeiras a
relatar, de forma precisa, a diferença entre ser homem e ser mulher no
sistema escravista dos Estados Unidos. Ela conta como era viver diante
de um senhor de escravos que praticava diversas violências físicas e
emocionais. Revela também toda a dificuldade de ser mãe dentro de um
contexto de escravidão. O livro deu origem a um intenso debate político,
histórico e social acerca da feminidade negra nos Estados Unidos.

Os dois temas de maior destaque no livro de Jacobs, meninice e maternidade, evidenciam essa diferença em relação ao homem em situação de escravidão.
Como escreve a pesquisadora Kellie Carter Jackson, professora do Wellesley College,

\begin{quote}
O próprio título do
livro de Jacobs no original, \emph{Incidents in the Life of Slave Girl}
(``incidentes na vida de uma menina escrava''), informa ao leitor que a
narrativa coloca a meninice no centro. Para as meninas escravizadas, a
inocência esmorece rápido, quando não é completamente erradicada. ``A
moça escrava é criada em uma atmosfera de medo e libidinosidade'',
escreve Jacobs. Em uma de suas passagens mais descritivas, ela discute o
que significa para uma menina ter sua inocência arrancada: ``Mesmo a
criancinha, acostumada a atender
sua senhora e os filhos, aprende antes dos doze anos de idade por que
sua senhora odeia esse escravo ou aquele''. Ela reconhece que as meninas
escravizadas entendiam até quando suas mães eram o motivo para a fúria e
o ciúme das senhoras.\footnote{\textsc{jackson}, Kellie Carter. ``Introdução''. In: \textsc{jacobs}, Harriet. \textit{Incidentes da vida de uma escrava}. São Paulo: Hedra, 2020, p.24--25.}
\end{quote}

Sua preocupação com a família e com um lar seguro, onde pudesse ser provedora, protetora e progenitora também particulariza esse relato em relação ao \textit{corpus} das narrativas de escravos homens. Jacobs perdeu a mãe muito jovem, com apenas seis anos, e desde então estabeleceu um vínculo ainda mais forte com sua avó, tia Martha. Foi em sua casa, especificamente em seu sótão, que a escritora pôde fugir do cerco de Norcom e encontrar um lugar seguro para cuidar dos filhos. Percebe"-se como seu próprio círculo afetivo rondava figuras de mulheres fortes e determinadas.

Para Kellie Jackson, esses são traços que distinguem a narrativa de Jacobs das demais narrativas de escravos:

\begin{quote}
A meninice e a maternidade são histórias que Frederick Douglass, William
Wells Brown e Solomon Northup podiam testemunhar, mas nunca viver.
\emph{Incidentes da vida de uma escrava} é uma obra especial e uma
perspectiva séria sobre o terror da escravidão. Douglass pode usar sua
narrativa para recontar a famosa briga com o Sr.\,Covey, na qual derrota
fisicamente o homem que tentava domá"-lo, mas Jacobs não tem essa opção.
Simplesmente rechaçar a raiva de Norcom a coloca sob um perigo terrível.
Ainda assim, a obra de Jacobs é corajosa. Durante toda a narrativa, ela
resiste ativamente ao poder de Norcom de transformar sua feminidade e
sua maternidade em arma contra ela mesma.\footnote{Ibidem, p.\,27.}
\end{quote}

Sua autobiografia --- da infância à vida adulta, passando pelas tempestades da adolescência e da juventude --- tem como pano de fundo a vida no sul dos Estados Unidos, região
fortemente estruturada sob o sistema de \textit{plantantion}, produzindo
arroz, tabaco, açúcar e algodão. Este, aliás, era o principal produto
norte"-americano no período, em razão da crescente industrialização do
setor têxtil na Europa. Desse modo, podemos perceber que a
industrialização europeia gerou consequências mundiais; no caso dos
Estados Unidos, intensificou a escravidão e a prolongou por muitos anos.

Cerca de 4 milhões de escravos viviam nas fazendas dos estados sulistas,
sendo que quase metade dessa população era feminina. O número de
escravizadas foi um fator essencial para a solidificação do regime
escravista no país por dois motivos: em primeiro lugar, o tráfico
negreiro no Atlântico Norte mingou muito mais rapidamente do que seu par
no Atlântico Sul, tornando"-se diminuto na primeira metade do século \textsc{xix};
em segundo lugar, deve"-se destacar que o recém"-nascido seguia a condição
escravizada de sua mãe. Portanto, se a mãe fosse escrava, seu filho,
consequentemente, também o seria.

Por esses motivos, o corpo das mulheres escravizadas era o grande motor
da escravatura norte"-americana. As meninas da faixa etária entre 12 e 15
anos eram as mais suscetíveis aos abusos físicos, sexuais e psicológicos
por parte dos seus senhores. Também era nessa idade que as escravas eram
tidas, por seus proprietários, como apropriadas para a reprodução,
conforme Jacobs comprova ao relatar as principais brutalidades sofridas
por ela.


A obra \emph{Incidentes da vida de uma escrava}, assim, ajuda a
compreender o terrível cotidiano escravista estadunidense e mostra como
a violência e os abusos mexiam com a mente e as emoções dos escravizados
e escravizadas. Tais sofrimentos podem ser equiparados às dores físicas
dos castigos e punições. A autobiografia possibilita observar também
como o abuso sexual de mulheres negras escravizadas era algo cotidiano.

Sobre essa questão, a autora assinala como as meninas negras
escravizadas perdiam muito cedo sua inocência. Frequentemente, ela
afirma que a beleza das meninas poderia ser algo que se voltava contra
elas: a beleza, segundo ela, era a maior maldição para uma menina negra
escravizada nos Estados Unidos do século \textsc{xix}.

A partir de Harriet Ann Jacobs, outras mulheres buscaram também relatar
suas experiências no contexto do escravismo e, graças também às suas
narrativas, é possível enveredar por um rico campo de pesquisa histórica
sobre a trajetória dos negros nos Estados Unidos.

\section{Sobre o gênero}

O gênero do relato, em linhas fundamentais, define"-se como a narrativa em que um sujeito, inscrito em um determinado tempo histórico, debruça"-se sobre fatos, descrições e interpretações desse momento histórico no qual vive. Para o historiador francês Paul Veyne, o relato histórico segue uma forma similar à forma tradicional de escrever história, seguindo um \textit{continuum} espaço"-temporal.

Apesar dessa relativa unidade, o relato, considerado como uma forma de fazer história,
é parcial e subjetivo, pois não consegue apreender a globalidade dos acontecimentos, apenas
aquilo que está ao alcance do narrador e, mesmo isso, não de uma forma pura, mas filtrado pela sua subjetividade e pelos objetivos de seu relato.
Para Veyne, estaríamos assim quase próximos do romance:

\begin{quote}
A história é uma narrativa de eventos: todo o resto resulta disso. Já que é, de fato, uma narrativa, ela não faz reviver esses eventos, assim como tampouco o faz o romance; o vivido, tal como ressai das mãos do historiador, não é o dos atores; é uma narração,
o que permite evitar alguns falsos problemas. Como o romance, a
história seleciona, simplifica, organiza, faz com que um século
caiba numa página.\footnote{\textsc{veyne}, Paul. \textit{Como se escreve a história}. Brasília: Editora Universidade de Brasília, 1999, p.\,18.}
\end{quote}

Seguindo nessa linha de pensamento, podemos observar, com o historiador francês Marc Bloch, que o relato é apenas um ``vestígio'' da história, um pequeno pedaço do factual que, pela pena de um narrador, pôde"-se cristalizar no tempo e ser transmitido a gerações posteriores, sendo apenas uma das infinitas possibilidades de apreensão e compreensão de determinados fenômenos:

\begin{quote}
Quer se trate das ossadas
emparedadas nas muralhas da Síria, de uma palavra cuja forma ou emprego revele um
costume, de um relato escrito pela testemunha de uma cena antiga [ou recente], o que
entendemos efetivamente por documentos senão um ``vestígio'' quer dizer, a marca,
perceptível aos sentidos, deixada por um fenômeno em si mesmo impossível de captar?\footnote{\textsc{bloch}, Marc. \textit{Apologia da história}. Rio de Janeiro: Zahar, 2002, p.\,73.}
\end{quote}

No caso deste \textit{Incidentes da vida de uma escrava}, evidencia"-se seu caráter de relato pois Jacobs estava circunscrito a um determinado momento da escravidão norte"-americana e pôde transmitir à posteridade o que viu e vivenciou nessa circunstância.
É um relato, dado o seu caráter histórico, mas também uma autobiografia, pois perpassa toda a trajetória da autora, da separação dos pais para ser escravizado à emancipação no Norte.
Dentro do gênero biográfico, a narrativa ainda se inscreve na categoria, muito popular nos Estados Unidos, das ``narrativas de escravos''
(\textit{slave narratives}), autobiografias de ex"-escravizados focadas nas histórias de liberdade que detalhavam a reação de seus autores à escravidão e seus caminhos para a liberdade. No Estados Unidos, surgiram em princípios da década de 1820, quando escritores começaram a compor obras contra a emergente ficção regionalista do Sul, que representava a vida dos brancos e negros nas fazendas escravistas como um jardim das delícias.

Com a publicação de várias autobiografias de ex"-escravizados que fugiram do cativeiro --- como as de William Wells Brown e Frederick Douglass, e os relatos do \textsc{fwp} --- as narrativas de escravos consolidaram"-se como um gênero literário em meados do século \textsc{xx}, quando começaram a despontar mais narrativas, e passaram gradativamente a ser aceitas pelos historiadores e acadêmicos como peças fundamentais para entender a escravidão nas Américas.

Na definição do historiador William L.\,Andrews, professor da Universidade da Carolina do Norte, as ``narrativas de escravos'' podem ser entendidas como ``qualquer relato da vida, ou uma parte importante da vida, de um fugitivo ou ex"-escravo, escrito ou relatado oralmente pelo próprio escravo''.\footnote{Disponível em \emph{http://nationalhumanitiescenter.org/tserve/freedom/1609-1865/essays/slavenarrative.htm}.}

A narrativa de Jacobs, no entanto, tem a particularidade de ter sido a primeira narrativa de escravo escrita por uma mulher. Segundo Kellie Jackson, sua obra foi fundamental pois, além de ampliar a compreensão da posição da mulher na sociedade patriarcal escravocrata, ela incentivou o aparecimento de outros relatos de mulheres viveram em situação de escravidão:

\begin{quote}
Jacobs foi a primeira mulher negra a escrever sobre suas
experiências na escravidão, mas não a última. Sua obra convenceu outras a contar suas
histórias, que por sua vez levaram os estudiosos a escrever sobre as
inúmeras experiências de mulheres escravizadas. O cânone das narrativas
das mulheres cativas inclui, hoje, a obra de Louisa Piquet, que
sofreu por anos como concubina do seu senhor, com quem teve quatro
filhos. Também conhecemos Bethany Veney, que viveu até os 103 anos e
escreveu sobre sua vida como escrava no Vale do Shenandoah, no estado da
Virgínia. Também somos gratos aos historiadores que recuperaram as
histórias traumáticas de mulheres como Celia, que matou seu senhor após
sofrer abusos sexuais constantes, ou de Margaret Garner, que matou uma
filha e tentou matar os outros três para impedir que fossem
reescravizados.\footnote{\textsc{jackson}, op.\,cit.,\,p.27--28.}
\end{quote}

Antes do aparecimento dessas narrativas, no entanto, o que prevalece é a narrativa do homem escravizado.
A primeira narrativa de escravo que se tem história é a 
\textit{Interesting Narrative of the Life of Olaudah Equiano, or Gustavus Vassa, the African}
(1789), em que Olaudah Equiano narra a sua trajetória de vida desde sua infância na África Ocidental, passando pelo tráfico transatlântico, sua situação de cativo, e encerrando"-se com sua liberdade e o sucesso financeiro como um cidadão britânico.
A história de Equiano, publicada na Inglaterra, chocou o público ao revelar os horrores da escravidão e balançou o senso"-comum da época que via na escravidão uma bem aceita e estabelecida instituição socioeconômica inglesa.

Já nos Estados Unidos da América, a primeira narrativa de escravo foi a \emph{Life of William Grimes, the Runaway Slave, Written by Himself}, de 1825.
Com seu relato, William Grimes revelou aos leitores do Norte o verdadeiro terror do cativeiro nos estados do Sul e as injustiças raciais da Nova Inglaterra.
Essas autobiografias que deram início ao gênero eram, na análise de Schermerhorn, 
``um híbrido de gêneros diversos, incluindo narrativas de cativeiro, literatura de protesto, confissão religiosa e relatos de viagem''.\footnote{\textsc{schermerhorn}, Calvin, op.\,cit., p.\,9.}

No entanto, como ressalta Schermerhorn, a maioria dessas biografias foram publicadas com auxílio editorial de brancos e para públicos brancos.  “Assim, desde os primeiros momentos da autobiografia negra na América, domina a pressuposição de que o narrador negro precisa de um leitor branco para completar o seu texto, para construir uma
hierarquia de significância abstrata referente ao simples
conjunto dos seus fatos, para oferecer uma presença onde
antes havia apenas um `Negro', uma ausência escura.”\footnote{\textsc{andrews}, William L. \textit{To Tell A Free Story: The First Century of Afro-American Autobiography, 1760–1865}, 1988, p.\,32--33 \textit{apud} \textsc{schermerhorn}, Calvin, op.\,cit., p.\,9--10.}

Desse conflito entre um gênero escrito por negros no meio de uma sociedade branca e racista, desponta outra característica das ``narrativas de escravo'', a modelação de sua história para ser bem"-aceita pelo público leitor branco:

\begin{quote}
O testemunho era a pedra fundamental da autobiografia de um ex-escravizado, e um dos principais desafios artísticos enfrentados pelos autores negros foi como atrair a
simpatia do leitor para que enxergassem as cenas de subjugação da maneira apresentada. Muitas vezes, isso significava aceitar boa parte da cultura anglo-americana como
normativa, sugerindo que a civilidade dos brancos seria a
razão por que era inaceitável a crueldade dos escravizadores. Os americanos civilizados não deveriam tolerar o barbarismo da escravidão. Para os escritores negros, isso
significou atenuar a importância das tradições, culturas, religiões e idiomas da África e dar preferência aos valores e tradições dos povos descendentes de europeus. Significou atenuar o radicalismo e a militância que caracterizou líderes descendentes de escravizados como Toussaint ou Jean-Jacques Dessalines no Haiti.\footnote{Ibidem, p.\,10.}
\end{quote}

