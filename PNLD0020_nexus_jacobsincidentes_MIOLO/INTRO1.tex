\part{\textsc{para saber mais}}

\chapter*{Uma Mulher Escravizada em uma Sociedade Escravista Patriarcal} %Muito mais terrível: A vida de Harriet Jacobs
\addcontentsline{toc}{chapter}{Uma Mulher Escravizada em uma Sociedade Escravista Patriarcal, \emph{por Kellie Carter Jackson}}
\hedramarkboth{Para saber mais}{}

\begin{flushright}
\versal{KELLIE CARTER JACKSON}\\\vspace{-3pt}
\versal{WELLESLEY COLLEGE}
\end{flushright}



%\section{Uma Mulher Escravizada em uma Sociedade~Escravista~Patriarcal}

\section{A Historiografia de Harriet Jacobs}

Sem dúvida alguma, o trabalho da incansável Jean Fagan Yellin sobre
abolicionistas negras faz dela a maior a autoridade sobre a vida de
Jacobs. \emph{Harriet Jacobs: A Life}, a premiada primeira biografia
extensa de Jacobs, junto com a coleção dos documentos da família Jacobs
que editou, oferecem aos leitores uma perspectiva detalhada sobre a vida
da autora durante e após a escravidão. A edição revolucionária de Yellin
de \emph{Incidentes da vida de uma escrava} também acabou com as
suspeitas de que a obra de ``Linda Brent'' seria fruto da abolicionista
branca Lydia Maria Child, sua editora. É graças ao trabalho de Yellin
que os estudiosos sobre o tema concordam e reconhecem que Jacobs foi a
única autora da sua narrativa. Além disso, Yellin enfatiza a agência,
força de vontade, coragem e honestidade de Jacobs. Assim, Jacobs não
deve ser lida como um complemento para Frederick Douglass, e sim como um
par à sua altura.

As memórias de Jacobs foram apenas o início dos seus esforços em prol
dos afro"-americanos. Durante toda a Guerra Civil, ela defendeu os
interesses de homens, mulheres e crianças cativos e
recém"-emancipados. Em 1863, ela fundou uma escola livre administrada por
negros, a Jacobs School, em Alexandria, Virgínia. Dois anos depois,
Jacobs se mudou para Savannah, Geórgia, com sua filha Louisa, onde
seriam representantes da Sociedade Auxiliadora dos Libertos da Nova
Inglaterra. Lá, as duas ensinaram ex"-cativos, e se recusaram a ir
embora quando ameaçadas por sulistas brancos. Jacobs acreditava na
liberdade coletiva e nos direitos de todos os americanos negros. Até sua
morte, em 1897, ela continuou a trabalhar em associações beneficentes e
movimentos de reforma pela educação dos afro"-americanos. Yellin leva
seus leitores além da narrativa, apresentando uma vida de luta política e
serviço à comunidade.

Além disso, a respeitada estudiosa da história das mulheres negras Nell
Irvin Painter argumenta que \emph{Incidentes da vida de uma escrava}
ilustra três ideias fundamentais. Primeiro, o livro mostra como a
violência e o trauma atingiam os escravizados. A violência era a
gravidade para as pessoas em cativeiro, pois
mantinha as famílias negras presas sob o jugo dos seus donos. Da
separação ao abuso sexual, todas as armas possíveis foram empregadas
para manter homens, mulheres e crianças em estado constante de medo,
exploração e humilhação. Os efeitos da escravidão não foram apenas
físicos, foram emocionais e psicológicos também. Jacobs escreve
frequentemente sobre os ataques verbais que sofreu. A violência e a
ameaça da violência cercavam e prendiam todas as pessoas que trabalhavam
na fazenda. Segundo, Painter afirma que a obra de Jacobs tenta fazer com
que o leitor abandone a ideia do ``negrinho feliz''. Os fazendeiros
adoravam promover mitos de que os escravizados estavam todos contentes com a
sua sina na vida. Jacobs nos garante que nada poderia estar mais
distante da verdade, e que a escravidão também não era uma instituição
civilizadora. Ela dedica parte do seu livro ao esforço de desmentir esse
mito para os nortistas e de ilustrar que a escravidão roubava dos
escravizados sua felicidade e sua humanidade. Terceiro, Painter acredita
que Jacobs oferece aos seus leitores uma de suas afirmações mais
importantes, a saber, que as mulheres negras em cativeiro não podem ser
comparadas ou ``julgada{[}s{]} pelos mesmos padrões'' que as livres,
especialmente as brancas.\footnote{Nell Irvin Painter, ``Introduction,''
  \emph{Incidents in the Life of a Slave Girl by Harriet Jacobs},
  (London: Penguin Books, 2000), \versal{\versal{IX}--\versal{X}}.} Em termos de abuso sexual, a
crítica Saidiya Hartman argumenta que o consentimento era impossível
para as mulheres escravas. Era impossível se defender da violência
sexual cometida contra elas. Tanto legal quanto socialmente, Hartman
afirma que as mulheres negras eram consideradas inestupráveis. Assim, a
virtude e a castidade eram características que jamais poderiam ser
associadas às mulheres negras.

Painter também afirma que a ``narrativa tem uma perspectiva de gênero
consciente, é totalmente feminista e critica a escravidão por corromper
a moral e as famílias de todos que entravam em contato com ela, fossem
eles ricos ou pobres, negros ou brancos''. Durante o livro, vemos como
Jacobs não consegue realizar os ideais morais e vitorianos de feminidade
e pureza. Contudo, nem Jacobs nem qualquer mulher negra da sua época
teria sido capaz de viver esses ideais impossíveis em cativeiro. Painter
afirma corretamente que a autobiografia de Jacobs ``alicerça a análise
da feminidade negra'' como algo independente e incomparável àquela das
mulheres brancas.\footnote{Nell Irvin Painter, ``Introduction,''
  \emph{Incidents in the Life of a Slave Girl by Harriet Jacobs},
  (London: Penguin Books, 2000), ix; Ver também Hazel Carby,
  \emph{Reconstructing Womanhood: The Emergence of the Afro"-American
  Woman Novelist} (New York: Oxford University Press, 1987), 39,45--61.}

Mais recentemente, Tera Hunter argumenta em \emph{Bound in Wedlock:
Slave and Free Black Marriage in the Nineteenth Century} que ninguém
fala com mais eloquência sobre a antítese entre escravidão, casamento e
paternidade do que os próprios escravizados. Hunter afirma que mesmo
para fugitivos famosos, como William e Ellen Craft, o casamento
e a paternidade não existiam sem liberdade.\footnote{William e Ellen Craft foram escravos em Macon, Geórgia, e fugiram
  para a liberdade em 1848. Ellen se disfarçou de senhora de escravos
  branca, graças ao seu tom de pele claro, enquanto William, seu marido,
  fingiu ser seu escravo. Eles escreveram sua própria narrativa de fuga,
  publicada sob o título \emph{Running a Thousand Miles for Freedom: The
  Escape of William and Ellen Craft from Slavery.}} Em outras palavras,
a liberdade era essencial para viver esses relacionamentos, algo que o
leitor também pode compreender ao ler algumas das entrevistas da \versal{WPA} com
ex"-escravizados, publicadas nesta série Narrativas da Escravião da Editora
Hedra. Na escravidão, os votos de casamento muitas vezes eram recitados
na forma ``até que sejamos separados''. Jacobs nunca casou com o Sr.\,Sands, mas o arranjo entre os dois não era incomum. Na verdade, Hunter
afirma que, enquanto muitos casamentos entre brancos não eram
sancionados por autoridades civis ou religiosas, essas uniões eram
amplamente reconhecidas pela sociedade e não estavam sujeitas a ``forças
externas de destruição'', um fato amplamente ignorado pelos estudiosos
sob o discurso falacioso de que os casamentos dos escravizados podem
ser comparados com os da elite branca. Hunter também
afirma que um dos maiores obstáculos emocionais que os
escravizados enfrentavam eram as ``iliberdades da escravidão e dos casamentos''.
Perante a insegurança esmagadora da sua situação, alguns casais
de cativos não conseguiam reconciliar o ato de ``casar"-se'' ou de ter
filhos em cativeiro. A fuga era a única solução para criar e proteger a
unidade familiar negra.\footnote{Tera Hunter, \emph{Bound in Wedlock:
  Slave and Free Black Marriage in the Nineteenth Century} (Cambridge:
  Belknap, a division of Harvard University Press, 2017), 12--13.}%; See  also Ed
  Foi por isso que lutaram Elijah e Delilah, sem nunca
conquistar, em vida, para seus próprios filhos.

Além das vidas dos escravizados, os estudiosos também estudam sua
linguagem.\footnote{Ver Margaret Washington, "From Motives of Delicacy":
  Sexuality and Morality in the Narratives of Sojourner Truth and
  Harriet Jacobs." \emph{The Journal of African American History} 92,
  no. 1 (2007): 57--73.} Durante toda a narrativa, Jacobs é cuidadosa com
o uso da linguagem. Ela recorre a eufemismos para disfarçar ou
neutralizar suas experiências. Dado seu senso de decoro, Jacobs queria
que os leitores lessem nas entrelinhas. Até mesmo o título,
\emph{Incidentes}, sugere experiências cotidianas inofensivas. Sem
dúvida alguma, Jacobs está dividindo com os leitores seus traumas, mas
ela não se prende aos momentos mais violentos. Nada em sua narrativa é
hiperbólico ou exagerado. Culturalmente, teria sido vergonhoso para uma
mulher divulgar essas experiências. Lydia Maria Child escreveu que ``a
própria Jacobs admitiu que determinados detalhes de sua vida deveriam
ser sussurrados `ao pé do ouvido de uma amiga muito
querida'".\footnote{Ibid., 57.} Ainda assim, Jacobs apresenta uma
expressão ao mesmo tempo ponderada e poderosa do feminismo perante o
abuso sexual.

Não é por falta de confiança que Jacobs camufla suas palavras; ela
estava tentando evitar a crença negativa e generalizada de que as
mulheres negras demonstravam comportamentos ilícitos de natureza sexual.
Amigos avisaram Jacobs que o excesso de candura sobre sua vida poderia
levá"-la a ser desprezada e rejeitada em público. A historiadora Margaret
Washington argumenta que ``enquanto não temos motivo algum para duvidar
de Jacobs\ldots{} ter \emph{dois} relacionamentos com \emph{dois} homens
brancos estava \emph{longe} de ser uma imagem neutra para o público
nortista'', algo que até poucos círculos abolicionistas estariam
dispostos a aceitar.\footnote{Ibid., 67.} Jacobs não queria causar
escândalo. Ela estava trabalhando na missão difícil de revelar sua vida
e, simultaneamente, manter sua própria respeitabilidade.

%\section{Meninice e Maternidade da~Perspectiva~Escrava}

\section{Entendendo Harriet Jacobs Hoje}

Jacobs foi a primeira mulher negra a escrever sobre suas
experiências na escravidão, mas não a última. Sua obra convenceu outras a contar suas
histórias, que por sua vez levaram os estudiosos a escrever sobre as
inúmeras experiências de mulheres escravizadas. O cânone das narrativas
das mulheres cativas inclui, hoje, a obra de Louisa Piquet, que
sofreu por anos como concubina do seu senhor, com quem teve quatro
filhos. Também conhecemos Bethany Veney, que viveu até os 103 anos e
escreveu sobre sua vida como escrava no Vale do Shenandoah, no estado da
Virgínia. Também somos gratos aos historiadores que recuperaram as
histórias traumáticas de mulheres como Celia, que matou seu senhor após
sofrer abusos sexuais constantes, ou de Margaret Garner, que matou uma
filha e tentou matar os outros três para impedir que fossem
reescravizados.

No mundo dos movimentos Black Lives Matter e \#Metoo nos Estados Unidos,
ou da Lei Maria da Penha no Brasil, a obra de Jacobs jamais foi tão
relevante. Ela dá voz àquelas que foram vitimizadas e silenciadas. Seu
livro é um apelo à humanidade, a reconhecer o poder inegável das
mulheres negras na escravidão e na liberdade. A obra de Jacobs
estabelece um precedente para como as mulheres negras de toda a história
podem resistir e reagir a assédio sexual, ataques verbais e violência.
As mulheres negras têm uma rica tradição de protesto contra os seus
opressores. De Jacobs a Ida B. Wells, Rosa Parks, Anita Hill e Tarana
Burke, as mulheres negras usaram coletivamente suas vozes para falar
sobre seus ``incidentes'' traumáticos.\footnote{Ida B. Wells foi uma
  ativista e jornalista, líder intrépida e de uma cruzada nacional
  contra os linchamentos. Rosa Parks foi uma ativista que começou sua
  carreira investigando casos de abuso sexual no sul dos \versal{EUA}. Sua
  história, e a de muitas outras mulheres corajosas, é discutida em
  \emph{At the Dark End of the Street: Black Women, Rape, and
  Resistance --- A New History of the Civil Rights Movement from Rosa
  Parks to the Rise of Black Power} (New York: Vintage Press, 2010), de
  Danielle McGuire. Anita Hill é a professora universitária e advogada
  que acusou Clarence Thomas, juiz da Suprema Corte dos \versal{EUA}, de assédio
  sexual. Tarana Burke é a fundadora do \emph{\#metoo}, um movimento
  internacional contra o abuso e o assédio sexual.} Não é surpresa que
as fundadoras do movimento Black Lives Matter sejam três mulheres negras
jovens: Alicia Garza, Patrisse Cullors e Opal Tometi.

A obra de Jacobs é ao mesmo tempo pessoal e política. Ela ilumina o que
significa ser mulher e mãe na escravidão. Encorajo os leitores a
pensarem criticamente sobre os perigos e as experiências que as jovens
negras enfrentavam no cativeiro. Como é ser uma mãe escravizada? Como devemos
entender o custo físico e emocional de ser uma mulher escravizada? Como
a resistência à opressão se manifesta nesta narrativa? São todas
questões que merecem ser investigadas. Encorajo meus leitores a
refletirem profundamente sobre o retrato da comunidade de Jacobs e sobre
o que estava em jogo. Encorajo meus leitores a pensarem sobre os limites
do cativeiro. Por fim, encorajo os leitores a pensarem sobre sua própria
capacidade de produzir mudanças, mesmo que em espaços pequenos e
isolados. Todos temos algo a aprender. E todos podemos levar a sério os
incidentes nas vidas das meninas negras; não apenas pelo que é terrível,
mas para mudar o que sabemos e como o sabemos.
