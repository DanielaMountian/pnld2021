\chapter{Apresentação}

\emph{Pai contra mãe e outros contos} é uma compilação dos
contos dos livros \emph{Várias histórias} (1896), \emph{Páginas
recolhidas} (1899) e \emph{Relíquias de casa velha} (1906), de Machado
de Assis (1839--1908). Trata"-se dos três últimos volumes de contos
publicados pelo escritor, sendo que a obra de 1906 trazia também outros
escritos, como o soneto ``A Carolina''. Ou seja: a antologia apresentada
reúne narrativas da maturidade do contista. \emph{Várias histórias}
inclui diversos contos justamente antológicos de Machado, como ``A
cartomante'', ``Uns braços'', ``Um homem célebre'', ``A causa secreta'',
``Trio em lá menor'', ``O enfermeiro'', ``Conto de escola'', ``Um
apólogo'', ``D.\,Paula'', ``O cônego ou metafísica do estilo''. Os
célebres ``O caso da vara'', ``Missa do galo'' e ``Ideias de canário'',
entre outros, são de \emph{Páginas recolhidas}. E ``Pai contra mãe'',
que intitula a antologia, saiu originalmente em \emph{Relíquias de casa
velha}.

Os contos são exemplares dos temas, motivos e da forma criada por
Machado de Assis, em que se fundem e se confundem a apologia ao
individualismo e a crítica social, como se analisa, em especial no
livro, em ``Pai contra mãe''. Em suas ambiguidades e caráter
polissêmico, deixam ver as contradições da ordem social e humana.

Em ``Obras"-primas desconhecidas do conto brasileiro'', entrevista ao
crítico literário Otto Maria Carpeaux, publicada em ``Letras e Artes'',
suplemento de \emph{A Manhã}, do Rio de Janeiro, a 10 de abril de 1949,
Graciliano Ramos observa que Machado de Assis é grande escritor, e antes
contista do que romancista. Ao apontar que os romances machadianos são
misturas de aforismos, crônicas, ensaios, meditações e sobretudo de
contos, destaca dois aqui presentes: ``Magníficos contos, aliás, pois
Machado é grande nesse gênero, maior entre os brasileiros. Como
contista, o autor do `Trio em lá menor' e da `Causa secreta' seria
grande em qualquer língua, você não acha?''. Este último foi o conto de
Machado escolhido por Graciliano para a antologia que organizou nos anos
1940, seleção de exemplares do gênero das várias regiões do país.

A descrição do personagem Garcia, de ``A causa secreta'', nos faz pensar
no escritor maduro dos contos aqui reunidos, cuja marca é a análise de
caracteres que nos abre a compreensão: ``Este moço possuía, em gérmen, a
faculdade de decifrar os homens, de decompor os caracteres, tinha o amor
da análise, e sentia o regalo, que dizia ser supremo, de penetrar muitas
camadas morais, até apalpar o segredo de um organismo''. A capacidade de
codificar pessoas e seus comportamentos é das características mais
pronunciadas da obra de Machado de Assis e se faz presente especialmente
nos contos, em que a exigência de concisão aguçou o talento do escritor.
Se considerarmos que não há personagens fora do tempo e do espaço, e que
estas condições, ao lado dos temperamentos pessoais, acabam por delinear
formas de pensar e agir, então também podemos dizer que o autor destes
contos é um intérprete do Brasil do século \textsc{xix}. Daí à atualidade basta
um salto: dadas as diferenças de superfície, Machado de Assis soube ver
e dar a ver elementos menos detectáveis na superfície da experiência de
ser brasileiro, que se manifestam ainda hoje em nossa sociedade.

O leitor encontrará nestes contos, portanto, não apenas um rico panorama
da melhor narrativa curta de Machado de Assis, mas também da paisagem
social, política e cultural do Brasil daquele período, que pode servir,
se lido com atenção, para compreender contradições atuais de nosso país.


\chapter{Nota dos organizadores}

\emph{Pai contra mãe e outros contos} realiza uma compilação dos livros
de contos \emph{Várias histórias} (1896)\footnote{Fazem parte de
  \emph{Várias histórias} os contos ``A cartomante''; ``Entre santos'';
  ``Uns braços''; ``Um homem célebre''; ``A desejada das gentes''; ``A
  causa secreta''; ``Trio em lá menor''; ``Adão e Eva''; ``O
  enfermeiro''; ``O diplomático''; ``Mariana''; ``Conto de escola'';
  ``Um apólogo''; ``D. Paula''; ``Viver!''; ``O cônego ou metafísica do
  estilo''.}, \emph{Páginas recolhidas} (1899)\footnote{De \emph{Páginas
  recolhidas} fazem parte os contos ``O caso da vara''; ``O
  dicionário''; ``Um erradio''; ``Eterno!''; ``Missa do galo''; ``Ideias
  do canário''; ``Lágrimas de Xerxes''; ``Papéis velhos''.} e
\emph{Relíquias de casa velha} (1906)\footnote{Fazem parte de
  \emph{Relíquias de casa velha} os contos ``Pai contra mãe''; ``Maria
  Cora''; ``Marcha fúnebre''; ``Um capitão de voluntários''; ``Suje"-se,
  gordo!''; ``Umas férias''; ``Evolução''; ``Pílades e Orestes'';
  ``Anedota do \emph{Cabriolet}''.}, de Joaquim Maria Machado de Assis
(1839-1908).

Como forma de analisar alguns temas que julgamos fundamentais em meio ao
universo machadiano -- temas que, de um modo ou de outro (e sempre com o
látego da ironia em riste), perpassam os contos deste volume --,
remetemos os leitores ao ensaio ``Os inimigos do homem serão as pessoas
de sua própria casa: crítica e apologia sociais em `Pai contra mãe'",
de nossa autoria, que desponta como estudo preliminar.

Por fim, apresentamos aos leitores os fragmentos que despontam como
advertências/prefácios dos livros de contos ora reunidos de Machado de
Assis. São eles:

\pagebreak

\textbf{(i)} ``Advertência'' a \emph{Várias histórias: }
\begin{quote}
\emph{Mon ami, faisons toujours des contes. (\ldots{}) Le temps se passe, et
le conte de la vie s'achève, sans qu'on s'en aperçoive. }
\end{quote}

\begin{flushright}
Diderot\footnote{``Meu amigo, escrevamos sempre contos. (\ldots{}) O tempo
  passa, e o conto da vida chega ao fim, sem nos darmos conta disso''
  (tradução livre dos organizadores). Citação da obra \emph{Salon}
  (1767), do filósofo e escritor francês Denis Diderot (1713-1784).}
 \end{flushright}

\begin{quote}
As várias histórias que formam este volume foram escolhidas entre outras
e podiam ser acrescentadas, se não conviesse limitar o livro às suas
trezentas páginas. É a quinta coleção que dou ao público. As palavras de
Diderot que vão por epígrafe no rosto desta coleção servem de desculpa
aos que acharem excessivos tantos contos. É um modo de passar o tempo.
Não pretendem sobreviver como os do filósofo. Não são feitos daquela
matéria nem daquele estilo que dão aos de Mérimée\footnote{Prosper
  Mérimée (1803-1970), escritor, dramaturgo e historiador francês.} o
caráter de obras"-primas e colocam os de Poe\footnote{Edgar Allan Poe
  (1809-1849), escritor, poeta e crítico literário estadunidense.} entre
os primeiros escritos da América. O tamanho não é o que faz mal a este
gênero de histórias, é naturalmente a qualidade; mas há sempre uma
qualidade nos contos, que os torna superiores aos grandes romances, se
uns e outros são medíocres: é serem curtos.
\end{quote}

\bigskip

\textbf{(ii)} ``Prefácio'' de \emph{Páginas recolhidas: }

\begin{quote}
\emph{Quelque diversité d'herbes qu'il y ayt, tout s'enveloppe sous le
nom de salade. }
\end{quote}

\begin{flushright}
Montaigne, Essais, liv. \versal{I}, cap. \versal{XLVI}\footnote{``Por maior que seja a
  diversidade das verduras, tudo se torna compreensível sob o nome de
  salada''. Citação dos ensaios do escritor e pensador francês Michel de
  Montaigne (1533-1592).}
\end{flushright}

\begin{quote}
Montaigne explica pelo seu modo dele a variedade deste livro. Não há que
repetir a mesma ideia, nem qualquer outro lhe daria a graça da expressão
que vai por epígrafe. O que importa unicamente é dizer a origem destas
páginas. Umas são contos e novelas, figuras que vi ou imaginei, ou
simples ideias que me deu na cabeça reduzir a linguagem. Saíram primeiro
nas folhas volantes do jornalismo, em data diversa, e foram escolhidas
dentre muitas, por achar que ainda agora possam interessar. Também vai
aqui ``Tu só, tu, Puro Amor'', comédia escrita para as festas
centenárias de Camões\footnote{``Tu, só tu, puro amor'' é um verso da
  obra \emph{Os Lusíadas} (1572; \versal{III}, 119:1), do poeta português Luís
  Vaz de Camões (1524-1580).} e representada por essa ocasião.
Tiraram"-se dela cem exemplares numerados que se distribuíram por algumas
estantes e bibliotecas. Uma análise da correspondência de Renan com sua
irmã Henriqueta\footnote{Ernest Renan (1823-1892), filólogo, crítico
  literário e historiador francês, e Henriette Renan (1811-1861), sua
  irmã.}, e um debuxo do nosso antigo Senado foram dados na
\emph{Revista Brasileira}, tão brilhantemente dirigida pelo meu ilustre
e prezado amigo José Veríssimo\footnote{José Veríssimo (1857-1916),
  editor, escritor e crítico literário brasileiro.}. Sai também um
pequeno discurso, lido quando se lançou a primeira pedra da estátua de
Alencar\footnote{José de Alencar (1829-1877), escritor brasileiro.}.
Enfim, alguns retalhos de cinco anos de crônica na \emph{Gazeta de
Notícias} que me pareceram não destoar do livro, seja porque o objeto
não passasse inteiramente, seja porque o aspecto que lhe achei ainda
agora me fale ao espírito. Tudo é pretexto para recolher folhas amigas.
\end{quote}

\bigskip

\textbf{(iii)} ``Advertência'' a \emph{Relíquias de casa velha: }

\begin{quote}
Uma casa tem muitas vezes as suas relíquias, lembranças de um dia ou de
outro, da tristeza que passou, da felicidade que se perdeu. Supõe que o
dono pense em as arejar e expor para teu e meu desenfado. Nem todas
serão interessantes, não raras serão aborrecidas, mas, se o dono tiver
cuidado, pode extrair uma dúzia delas que mereçam sair cá fora.
Chama"-lhe à minha vida uma casa, dá o nome de relíquias aos inéditos e
impressos que aqui vão, ideias, histórias, críticas, diálogos, e verás
explicados o livro e o título. Possivelmente não terão a mesma suposta
fortuna daquela dúzia de outras nem todas valerão a pena de sair cá
fora. Depende da tua impressão, leitor amigo, como dependerá de ti a
absolvição da má escolha.
\end{quote}