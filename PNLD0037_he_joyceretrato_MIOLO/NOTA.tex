\chapter[Nota do tradutor]{Nota do tradutor}

\vspace*{2em}

\noindent \textsc{Escrevo esta nota} em um tom mais pessoal por sentir"-me desobrigado 
(dada a abundância de exegeses sobre a obra de James Joyce) de contribuir 
com o que não passaria de um modesto filete entre tributários mais ricos.

Ocorreu"-me no entanto que eu poderia mencionar um aspecto da produção literária 
que é muitas vezes ofuscado em favor da análise acadêmica, imprescindível 
quanto esta seja: que \textit{Um retrato do artista quando jovem} é antes 
de mais nada sobre a vida.  Em menor ou maior medida, ele é na verdade sobre você. 
Tomando emprestado o termo que Stephen Dedalus usa 
para se referir a Horácio, trata"-se de ``páginas humanas'', sobre vidas 
humanas reais, de pessoas com problemas e preocupações reais, que existiram por algum tempo, viveram, acumularam experiência 
e então partiram. Dores e alegrias reais, toda uma época com seu 
entorno afetivo e cultural, preservada com amor meticuloso pelo artista.

É por isso que os problemas técnicos desta tradução, interessantes 
como tenham sido, não me parecem tão instigantes quanto a questão 
sobre a quem interessa um romance sobre a formação da alma de um artista. 
Muito embora pessoas inteligentes e sensíveis de todo tipo possam se deleitar com a riqueza de conteúdo e forma que o 
\textit{Retrato} oferece, não tenho dúvidas de que o drama e o conflito central 
deste romance falem mais alto e mais perto às jovens almas de artistas em formação, 
uma vez que um dos confortos da literatura é o prazer do reconhecimento.

Numa época em que os artistas se encontram mais apartados do fórum da 
consciência pública do que jamais estiveram e em que (para usar outra
expressão cara a Stephen Dedalus) ``a palavra está no mercado'', creio 
que seria apenas adequado se o \textit{Retrato} fosse também espelho para 
os que tem que começar a partir de escombros — os jovens artistas do 
país, a quem dedico esta tradução --- e que talvez achem bom uso para as 
armas empregadas por Stephen em sua jornada: ``o silêncio, o exílio e a astúcia''.

\thispagestyle{empty}

\medskip

\hfill\textit{Elton Mesquita}

