\chapter{Vida e obra de James Joyce}

\section{Sobre o autor}

\noindent{}Nascido a 2 de fevereiro de 1882 no Condado de Dublin, Irlanda, James Joyce foi um dos mais importantes escritores modernos. Filho mais velho dos dez filhos de John Stanislaus Joyce (1849--1931) e Mary Murray Joyce (1859--1903), estudou em importantes colégios jesuítas e depois se formou no University College de Dublin. Aos vinte anos o jovem escritor foi morar em Paris, para estudar medicina, mas logo passou a escrever versos e desenvolver um ``sistema estético'', cujos desdobramentos podem ser observados nesse \textit{Stephen Herói}.

Retornou à Irlanda no ano seguinte, em 1903, para assistir sua mãe gravemente doente. Desse ano data os primeiros manuscritos de seu \emph{Um retrato do artista}. 
Em 1904 conheceu Nora Barnacle, uma jovem de Galway, com quem logo se envolveu amorosamente.
Pensando em lecionar inglês para sobreviver, convenceu"-a a sair com ele da Irlanda.
Passaram alguns meses em Pula, atualmente na Croácia, e depois se mudaram para Triste, Itália, onde viveram até 1915, tendo dois filhos, um menino e uma menina.

Seu primeiro livro, poemas organizados sob o título \textit{Música de câmara}, foi publicado em 1907 em Londres. Seu livro de contos \textit{Dublinenses} saiu sete anos depois. Em seguida vieram o romance \textit{Um retrato do artista quando jovem} (1916) e a peça de teatro \textit{Os exilados} (1918).
Com o ingresso da Itália na Primeira Guerra Mundial, Joyce se mudou com a família para Zurique, onde viveu até 1919. Após um breve retorno a Triste, decide se mudar para Paris, onde pensava poder trabalhar melhor no seu romance \textit{Ulysses}, iniciado em 1914.
O livro veio a lume na data de seu aniversário, 2 de fevereiro, em 1922 em Paris.

Apesar de problemas oculares e a preocupação com sua filha Lucia, que padecia de uma doença mental, trabalhou com afinco em seu \textit{Finnegans Wake}, que conseguiu publicar em 1939.
Dessa época data seu contato com jovens escritores e admiradores como Samuel Beckett.
Após o início da Segunda Guerra Mundial, consegue uma permissão para retornar a Zurique, em dezembro de 1940, onde logo depois morreu, em 13 de janeiro do ano seguinte.


\section{Sobre a obra}

``Era uma vez e foi muito bom dessa vez que'' James Joyce ousou remontar
o labirinto existencial para, nas asas de Dédalo, libertar"-se rumo ao
êxtase poético. Stephen Dedalus, alter ego do autor nessa narrativa
semiautobiográfica, deixa a infância na casa dos pais para frequentar o
internato do Colégio Clongowes. A instrução católica inicia"-se num
castelo habitado por vultos do passado e segue sob tutela jesuíta até a
faculdade. O garoto dá pinotes pelos pátios, saltando entre lembranças,
desventuras, dogmas, orações, música e ciência, enquanto o futuro
artista constrói o conhecimento de si mesmo. O jovem Stephen sente a
revolução sensual e descobre a adoração, não à Maria, mas à moça da
vizinhança. A obra \textit{Retrato do artista quando jovem}, romance de
formação publicado em 1916, narra as experiências da infância à
juventude do protagonista aprendiz que forjarão sua vida e sua arte. A
crise do ego e a ruptura com as vozes repressoras e contraditórias
familiares, religiosas e pátrias são determinantes nessa jornada. A
conversão à religião ``arte'', com as epifanias profanas e a revelação
profética do Belo transformam Stephen de criatura em criador. E Joyce em
extraordinário.

James Joyce começou a escrever o livro \textit{Retrato do artista} quando jovem,
aos 22 anos de idade, e terminou 10 anos depois. Podem ser apreciadas
ainda frescas as observações do protagonista Stephen Dedalus. A formação
ortodoxa alicerçou os conhecimentos e a personalidade do jovem Stephen.
Seus traços questionadores foram catalisados pelas forças incoerentes e
castradoras das instituições irlandesas na virada do século \textsc{xx}. O
impulso para a libertação do artista foi proporcional à profundidade do
apreendido, incluindo o cerne filosófico. Sensível, inteligente e
introspectivo, Stephen revolucionou as ideias tradicionais, as demandas
do corpo e do espírito, recriando"-as, a partir de sua genialidade, na
dimensão poética. As memórias sobre pormenores do mundo interior e
exterior, sob as lentes do incomum, ganham o fluxo dinâmico e revelador
da alma humana. Recontado em estilo que amadurece conforme o
desenvolvimento do protagonista, o livro começa como um conto de fadas e
termina na forma de diário, quando o narrador já não é mais necessário.
Stephen perde a fé religiosa, mas encontra um caminho norteador para
Joyce e nós, seus leitores.

\subsection{Stephen Dedalus}

O nome do protagonista combina mitos de origem cristã e grega. Estevão
foi o primeiro mártir cristão a ser perseguido por sua fé. Já Stephen de
James Joyce foi excluído por seus pares por não ser devoto do
catolicismo.

O sobrenome Dedalus relembra a história de Dédalo e seu filho Ícaro, que
foram condenados à prisão no labirinto.
Como nota a professora e pesquisadora Beatriz Kopschitz Bastos:

\begin{quote}
O nome de Stephen Dedalus, o personagem central, é uma referência ao
mito grego de Dédalo, obreiro astucioso, artífice lendário de Atenas,
cujas estátuas podiam mover"-se. Receoso de que seu sobrinho, Talos,
pudesse ultrapassá"-lo em engenhosidade, Dédalo decidiu lançá"-lo do alto
da Acrópole. Condenado, fugiu para Creta, onde construiu o Labirinto
para o rei Minos. Após ajudar na fuga de Teseu, Dédalo acabou confinado
no próprio Labirinto, do qual tentou escapar construindo asas de cera
para si mesmo e para seu filho, Ícaro. Ícaro voou muito alto e suas
asas foram derretidas pelo Sol, levando à sua queda. O romance de Joyce
narra o confinamento de Stephen e, de forma oblíqua, do próprio Joyce,
nas redes de sua vida; seu recurso final, assim como o de Dédalo para
livrar"-se do confinamento do Labirinto, é, literalmente, fugir, deixar
o país, e, metaforicamente, voar rumo à liberdade, rumo ao
desconhecido, com as asas da arte. O caminho para as artes
desconhecidas já é apontado na epígrafe do livro, extraída das
\textit{Metamorfoses}, de Ovídio: “\textit{Et ignotas animum dimittit
in artes}” (“E ele aplica a mente às artes ocultas”). E na última linha
do romance, quando está prestes a partir, e assim assumindo o papel de
filho espiritual, Stephen evoca a proteção de Dédalo: “Velho pai, velho
artífice, ampara"-me agora e sempre em sã jornada”.\footnote{\textsc{bastos}, Beatriz Kopschitz. ``Introdução''. In: \textsc{joyce}, James. \textit{Um retrato do artista quando jovem}. São Paulo: Hedra, 2013, p.\,10--11.}
\end{quote}

Stephen, próximo à narrativa grega, vê a necessidade de escapar de Dublin, e voar em direção à iluminação filosófica e à liberdade artística, a contragosto da família e
dos clérigos. Ao ir em direção ao ``pecado'', ele morre para a fé
católica.

\subsection{Em um conto de fadas}

Sr.\,Dedalus, de rosto peludo, vê o filho por trás do vidro e conta"-lhe
esta história:

\begin{quote}
Once upon a time and a very good time it was there was a moocow coming
down along the road and this moocow that was coming down along the road
met a nice's little boy named baby tuckoo.
\end{quote}

O ambiente mágico do conto de fadas --- contação oral, palavras
distorcidas pelo vocabulário infantil, a presença de elementos
sensoriais, o conforto saudoso da infância --- aparece revelando as ilhas
de conhecimento que constituem a consciência da criança aos 3 anos de
idade.

A vacamumu [\textit{moocow}] encontra o neném tuquim [\textit{tuckoo}]
pela estrada. A palavra ``tuckoo''
pode corresponder a ``tuck'', no sentido de embrulhar, isto é, bebê
embrulhado. Uma lenda do folclore irlândes refere"-se a uma vaca branca
sobrenatural que carrega as crianças para uma ilha, onde há um reino
mágico. Nesse local, as crianças são instruídas e transformam"-se em
heróis. No retorno, a comunidade recebe"-as com admiração. Outra história
que dá significado à vacamumu vem da mitologia grega. Pasífae, esposa do
rei Minos, apaixonou"-se por um touro divino. Dédalo, um artífice
grandioso, construiu um simulacro de vaca oco, dentro do qual Pasífae
pôde consumar a cópula com o animal. Dessa união nasceu o Minotauro.
Dédalo construiu o labirinto e também foi a ele condenado. Pasífae, para
retribuir o favor pregresso, ajuda Dédalo a fugir com seus conhecimentos
em feitiçaria. O encontro do bebê ``a ser desembrulhado'' com a vacamumu
pode significar simbolicamente o despertar para a sabedoria, para a
sensualidade, e em última análise, para a busca da vocação artística, ao
se libertar dos controles e padrões estabelecidos.

A canção preferida de Stephen é sobre rosas selvagens:

\begin{verse}
Oh, a rosa selvagem floresce\\
Lá no cantinho verde.
\end{verse}

Claramente, a rebelião e a liberdade brotam no seu interior do menininho.

\subsection{Não servirei}

O jovem Stephen percebe desde cedo a incoerências sociais: a conduta
adotada pelas personagens contradiz as normas pregadas; e a repreensão a
certos valores que, verdadeiros que são, deveriam ser acolhidos. Nesse
sentido, sente"-se desajustado e caminha para a dolorosa alienação social.

A formação católica de Stephen tem base familiar e escolar --- do
internato à universidade. No entanto, experiências decepcionantes
precoces contribuem para sua futura apostasia. Entre elas, a crítica à
igreja pelo pai e amigos na noite de Natal; a afirmação de Dante (Tita)
de que os católicos esmagaram Parnell com a morte; a traição do reitor,
ao acolher Stephen após a injusta punição na aula de latim, mas,
tardiamente, rir ao lembrar do incidente.

Ainda menino, é punido ao revelar algo de caráter emocional --- sua
intenção de casamento com Eileen. Na adolescência, o amadurecimento
sexual intensifica o peso da censura às pulsões. A iniciação sexual
demonstra uma revolta do corpo contra as regras impostas, embora ainda
inseguro e com medo das punições infernais descritas pelo Padre Arnall.
Por meio do fluxo de consciência, o leitor acompanha o conteúdo
consciente e subconsciente da mente do garoto, podendo aprofundar"-se em
suas questões existenciais.

Uma associação simbólica entre amor carnal e arte é estabelecida.
Convencido de seguir sua vocação artística, após uma gestação de anos,
Joyce, na pele de Stephen, opta por romper com as amarras espirituais e
culturais que ameaçam sua carreira literária:

\begin{quote}
Não servirei àquilo em que não acredito mais quer isso se chame minha
família, minha terra natal ou minha Igreja; e procurarei me expressar
por meio de uma certa forma de vida ou de arte tão livremente quanto
possa e tão totalmente quanto possa, usando em minha defesa as únicas
armas que me permito usar: o silêncio, o exílio e a astúcia.
\end{quote}

\subsection{Bous Stephanoumenos}

O bebê tuquim transforma"-se em Bous Stephanoumenos, o aluno de alma"-boi.
Há uma alusão a Tomás de Aquino, o boi mudo da Sicília.
Joyce, vale lembrar, lera o filósofo escolástico e \textit{De Anima}, de
Aristóteles, que ajudaram"-lhe a descobrir e desenvolver seus estilos
que compõem o \textit{Retrato}.

Como analisa Beatriz Bastos, o estilo desse \textit{Retrato} acompanha os desenvolvimentos de seu protagonista, em uma clara influência dos princípios artísticos adquiridos de Aristóteles e Tomás de Aquino:

\begin{quote}
A autodescoberta de Stephen e de sua missão como artista desenrola"-se
paralelamente à evolução da própria escrita do livro. [\ldots] Cada capítulo é
escrito em estilo que corresponde ao estágio da vida em que o
personagem se encontra: a escrita imita sua psicologia interior. Da
infância, passando pela adolescência, até a juventude, enquanto Stephen
toma consciência de sua missão como artista, a prosa do livro progride
e evolui em estilo. A força do \textit{Retrato} está exatamente na
riqueza da variedade de estilos e na densidade da organização
simbólica, em um texto que combina a sequência de episódios conectados
cronologicamente com a intervenção de momentos de epifania e clímax
triunfal. Enquanto o primeiro capítulo é escrito praticamente em estilo
realista, com algum uso de monólogo interior, a partir do segundo 
intensificam"-se as reações interiores do protagonista aos
acontecimentos internos; nos capítulos subsequentes predomina cada vez
mais a técnica do fluxo de consciência, que antecipa a prosa
característica do próximo romance de Joyce, sua obra"-prima modernista:
\textit{Ulisses}. A paixão pelo pensamento de Stephen no
\textit{Retrato} dá lugar à centralidade da própria atividade do pensar
em \textit{Ulisses}, publicado em 1922.\footnote{Ibidem, p.\,11.}
\end{quote}

Além disso, o sistema ortodoxo determinou o caráter e a personalidade de Stephen. De
Tomás de Aquino, Stephen também empresta a fundamentação metafísica. Para o
jovem, a apreensão do belo se dá a partir do sujeito, transmutando a
concepção da beleza original tomista. Quando a imagem estética é
originalmente concebida na imaginação do artista, ocorre o êxtase
estético, um estado espiritual de encantamento do coração.

De Gabriele D'Annunzio, Stephen apreendeu a concepção de epifania, em
que o poeta recebe uma manifestação espiritual da beleza ou uma
revelação, e deve se tornar seu mensageiro. Pode ser compara ao instante
em que o véu se rasga, fazendo com que a personagem e, por consequência,
o leitor alcancem um novo nível de consciência.

De Gustave Flaubert, o futuro poeta capta como se deve dar a criação,
comparando o artista ao criador, que brinca de Deus, um poeta"-profeta
divino.

Stephen Dedalus serve agora à arte e adora a Beleza, das quais se tornou
criador e profeta. Está pronto para partir!

Em seu diário escreve a Dédalo:
``Velho pai, velho artífice, mantém"-me, agora e sempre, em boa forma''.


\section{Sobre o gênero}

Este \textit{Um retrato do artista
quando jovem}, pode ser classificado no gênero do romance de formação, tradução
de \textit{Bildungsroman}, principal tradição narrativa da língua alemã.
Seu surgimento remonta ao século \textsc{xviii}, momento em que a razão iluminista
rompia com a mentalidade fatalista da Igreja católica.
Alguns autores desse período histórico são Goethe, Voltaire, Rousseau, Locke, Diderot e Montesquieu que, apesar de suas diferenças ideológicas e estilísticas, tinham em comum
uma visão mais cética, empírica e materialista, valorizando a razão em detrimento do obscurantismo da Igreja e dos desmandos dos reis absolutistas.

Nascendo nesse contexto de intensas transformações sociais, políticas, econômicas e religiosas, e de renovação da produção cultural, o romance de formação, invertendo a importância conferida à transcendência na clássica metafísica cristã, foca"-se no homem, coloca"-o como centro autônomo de seu próprio mundo. Não por acaso, é também o momento de consolidação da burguesia, que acompanha um movimento de individualização do sujeito e crença no progresso. Na análise do escritor e crítico literário Caio Gagliardi, a juventude é o símbolo do romance de formação por excelência: representa o anseio pela modernidade, personifica a instabilidade da sociedade capitalista industrial e do próprio indivíduo, assim como de seu meio social, com o dinanismo das classes e dos valores.

No romance de formação, revela"-se em detalhes o desenvolvimento físico,
psicológico, moral, social, estético, e/ou político de uma personagem,
ao percorrer as fases de sua vida: infância, adolescência, idade adulta
e maturidade. O gênero, iniciado com Goethe em seu \textit{Os
sofrimentos do jovem Werther}, tem uma de suas variações no
Romance do artista, em que o aprendiz, de acordo com as circunstâncias,
tem a oportunidade de desenvolver as concepções de sua arte e a sua
criação. A convergência dos atributos citados faz da saga de Stephen
Dedalus, na busca pelo autoconhecimento nas esferas da vida e da arte,
um romance de formação do artista.


Segundo Gagliardi, ``a característica predominante do
\textit{Bildungsroman} é o tema de que
trata: o desenvolvimento interior de um protagonista criança ou jovem
mediante suas dificuldades de interação com os demais e o meio em que vive''.\footnote{\textsc{gagliardi}, Caio. ``Introdução''. In: \textsc{pompeia}, Raul. \textit{O Ateneu}. São Paulo: Hedra, 2020, p.\,27.} Segundo o crítico, essa característica se evidencia ao se constatar que antes do século \textsc{xix} nunca houve, na literatura, um enfoque tão grande no protagonista em formação.

O enfoque no desdobramento interior de um personagem, no entanto, não faz do romance de formação um gênero subjetivo ou introspecto, alerta Gagliardi. 
Para o escritor, o \textit{Bildungsroman} está atrelado à realidade histórica, que tem um papel fundamental na transformação do indivíduo. ``Se o autoconhecimento define"-se
pela interação do indivíduo com a sociedade, o caráter subjetivo do
texto não apaga sua outra dimensão, objetiva. Realidade íntima e
histórica devem manter"-se em contato. Daí uma narrativa difícil de
definir: que nem é idealista, [\ldots] nem realista [\ldots]. Por outro ângulo, trata"-se de uma narrativa que se aproxima da \textit{autobiografia} (na focalização tanto externa 
quanto interna do protagonista, isto é, no desenrolar das experiências vividas e incorporadas)''.\footnote{Ibidem, p.\,31.}.

Na conjunção desses aspectos, subjetivos, históricos e autobiográficos, Gagliardi localiza o texto de James Joyce:

\begin{quote}
Em termos estilísticos, \textit{O retrato do
artista quando jovem} (1916), de James Joyce, é uma
narrativa nitidamente moderna, isto é, autoconsciente de seus
procedimentos de escrita, e revela a evolução do
\textit{Bildungsroman}. Também considerado o
romance autobiográfico de Joyce, narra episódios traumáticos referentes
à sua infância, tal como o castigo de palmatória, que o jovem
protagonista sofre de um bedel inábil, desconfiado de que o menino (que
havia perdido os óculos) não fazia sua tarefa em sala por mera
preguiça. O ensino jesuíta é o responsável pela impressionante cultura
clássica, pelo conhecimento de línguas, mas também pelos fantasmas de
Joyce. No romance, Stephen Dedalus é um autor"-menino que procura
libertar"-se da educação recebida e dos dilemas da pátria para
encontrar seu próprio caminho como
escritor.\footnote{Ibidem, p.\,29--30.}
\end{quote}

Além de sua filiação ao romance de formação, esse \textit{O reatro do artista quando jovem}
aproxima"-se muito, pela técnica e pelos assuntos, de seu livro de contos \textit{Dublinenses}, como nota Beatriz Bastos:

\begin{quote}
O \textit{Retrato} começou a ser publicado em episódios na prestigiada revista
literária londrina \textit{The Egoist} em 1914, mesmo ano da publicação da
coleção de contos \textit{Dublinenses.} Aproxima"-se desta em aspectos como
assunto e técnica narrativa, especialmente no caso do último conto do volume,
“Os mortos”, que já exibe parte do caráter de biografia lírica que é o
\textit{Retrato.} A publicação em livro deu"-se em 1916, ironicamente o
ano do Levante de Páscoa, um dos momentos de maior apelo emocional na história
da Irlanda, extremamente significativo na iconografia do nacionalismo
revolucionário irlandês que Joyce rejeitava.\footnote{\textsc{bastos}, Beatriz, op.\,cit., p.\,10.}
\end{quote}

\begin{bibliohedra}

\tit{ELLMANN}, Richard.  \textit{James Joyce.}  Oxford: Oxford University
Press, 1959.

\tit{Fargnoli}, A.~Nicholas \& \textsc{gillespie}, Michael Patrick. 
\textit{Critical Companion to James Joyce: A Literary Reference to His Life and Work}.
New York: Facts on File, 2006.

\tit{LEVIN}, Harry.  \textit{James Joyce.}  Revised, augmented edition.  New
York: James Laughlin, 1960.

\tit{SPENCER}, Theodore.  Introduction.  \textit{Stephen Hero: James Joyce}. 
A New Edition, incorporating the Additional Manuscript Pages in the Yale
University Library and the Cornell University Library, eds. John J.
Slocum and Herbert Cahoon.  New Directions Series.  New York: James
Laughlin, 1963.

\end{bibliohedra}