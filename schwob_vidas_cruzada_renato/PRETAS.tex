\SVN $Id: PRETAS.tex 8289 2011-02-09 20:08:44Z bruno $
\begin{resumopage}

\item[Marcel Schwob] (Chaville, 1867--Paris, 1905) foi ficcionista, ensaísta e tradutor francês. 
Com formação intelectual erudita, ocupou lugar de destaque nos meios literários parisienses
nos anos 1890, tendo convivido intimamente com escritores como Paul Claudel,
Guy de Maupassant, Jules Renard e Alfred Jarry, entre outros. Traduziu autores
latinos como Luciano de Samósata, Catulo e Petrônio, mas tinha especial
predileção por escritores de língua inglesa, como Defoe, Stevenson, Meredith e
Whitman. Entre suas obras mais importantes estão \textit{Cœur double}
(Coração duplo, 1891), \textit{Le Roi au masque d’or} (O rei
da máscara de ouro, 1892), \textit{Le Livre de Monelle} (O livro de
Monelle, 1894), \textit{La Croisade des enfants} (A cruzada das
crianças, 1896) e \textit{Vies imaginaires} (Vidas imaginárias, 1896).

\item[A cruzada das crianças] (1896) tem como ponto de partida as crônicas medievais do século \versal{XIII} 
sobre um grupo de crianças alemãs e francesas que teriam se reunido em torno de um jovem profeta para
marchar rumo a Jerusalém. A narrativa é composta de oito relatos que trazem
pontos de vista independentes sobre o acontecimento.

\item[Vidas imaginárias] (1896) reúne narrativas que têm como protagonistas personagens históricos
mais ou menos conhecidos. Schwob reconstitui à sua maneira a trajetória de
filósofos, escritores, escravos, soldados, piratas e criminosos, seja a partir
de biografias já existentes ou documentação histórica, seja a
partir de fontes literárias.
        
\item[Dorothée de Bruchard] é graduada em Letras pela Universidade Federal
de Santa Catarina (\textsc{ufsc}), e mestre em Literatura Comparada pela University of
Nottingham, Inglaterra. Entre 1993 e 2001, dirigiu a Editora Paraula, dedicada à
publicação de clássicos em edições bilíngues. Atualmente é tradutora e responsável pelo 
site e publicações do Escritório do Livro (www.escritoriodolivro.com.br). 

\item[Marcelo Jacques de Moraes] é professor de Literatura Francesa da
Universidade Federal do Rio de Janeiro, pesquisador do CNPq e tradutor. Tem
doutorado em Letras Neolatinas (\versal{UFRJ}, 1996) e pós-doutorado em Literatura
Francesa. É coeditor da revista \textit{Alea: Estudos Neolatinos} desde 1999. 
Publicou diversos artigos em revistas e livros no Brasil e no exterior. 

\end{resumopage}

















