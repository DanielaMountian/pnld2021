\part{\textsc{crônicas}}

%\part{Autobiografia}

\chapter[Maio]{Maio\footnote[*]{Publicado no jornal \emph{Gazeta da Tarde} em 04/05/1911.}}

Estamos em maio, o mês das flores, o mês sagrado pela poesia. Não é sem
emoção que o vejo entrar. Há em minha alma um renovamento; as ambições
desabrocham de novo e, de novo, me chegam revoadas de sonhos. Nasci sob
o seu signo, a treze, e creio que em sexta"-feira; e, por isso, também à
emoção que o mês sagrado me traz, se misturam recordações da minha
meninice.

Agora mesmo estou a lembrar"-me que, em 1888, dias antes da data áurea,
meu pai chegou em casa e disse"-me: a lei da abolição vai passar no dia
de teus anos. E de fato passou; e nós fomos esperar a assinatura no
largo do Paço.

Na minha lembrança desses acontecimentos, o edifício do antigo paço,
hoje repartição dos Telégrafos, fica muito alto, um
\emph{sky"-scraper}\footnote{Palavra inglesa que significa
  ``arranha"-céu''. Na época em que Lima Barreto escreveu, não eram
  muitos os prédios existentes no Rio de Janeiro e nem tão altos como os
  de hoje em dia. Para conhecer um pouco dos edifícios e da paisagem
  urbana da cidade do Rio, consultar o site
  \emph{www.brasilianafotografica.bn.br}.};
e lá de uma das janelas eu vejo um homem que acena para o povo.

Não me recordo bem se ele falou e não sou capaz de afirmar se era mesmo
o grande Patrocínio\footnote{Trata"-se se de José Carlos do Patrocínio,
  mais conhecido por José do Patrocínio. Foi um dos mais importantes
  membros do movimento abolicionista, que lutava pelo fim da escravidão
  no Brasil. De origem humilde, conseguiu se formar farmacêutico, mas
  fez carreira na imprensa, como jornalista republicano e também como
  grande orador.}.

Havia uma imensa multidão ansiosa, com o olhar preso às janelas do velho
casarão. Afinal a lei foi assinada e, num segundo, todos aqueles
milhares de pessoas o souberam. A princesa veio à janela. Foi uma
ovação: palmas, acenos com lenço, vivas\ldots{}

Fazia sol e o dia estava claro. Jamais, na minha vida, vi tanta alegria.
Era geral, era total; e os dias que se seguiram, dias de folganças e
satisfação, deram"-me uma visão da vida inteiramente festa e harmonia.

Houve missa campal, no Campo de São Cristóvão. Eu fui também com meu
pai; mas pouco me recordo dela, a não ser lembrar"-me que, ao assisti"-la,
me vinha aos olhos a ``Primeira Missa'', de Vitor Meireles\footnote{Referência
  ao quadro ``Primeira Missa no Brasil'', pintado pelo brasileiro Victor
  Meirelles, sob a técnica óleo sobre tela. Suas dimensões são 270 x 357
  cm e demandou cerca de dois anos de trabalho --- entre 1858 e 1860. A
  primeira apresentação do quadro ocorreu no aclamado Salão de Paris, no
  ano de 1861. Meirelles representa neste quadro aquele que foi
  considerado o primeiro evento realizado pela Igreja Católica na terra
  recém conquistada pelos portugueses e que viria a ser conhecida como
  Brasil. A missa foi celebrada por Frei Francisco Henrique de Coimbra,
  no dia 26 de abril de 1500. Os relatos estão presentes nas anotações
  de Pero Vaz de Caminha e serviram de inspiração para Victor Meirelles
  compor a obra.}. Era como se o Brasil tivesse sido descoberto outra
vez\ldots{} Houve o barulho de bandas de músicas, de bombas e
girândolas, indispensável aos nossos regozijos; e houve também préstitos
cívicos. Anjos despedaçando grilhões, alegrias toscas passaram
lentamente pelas ruas. Construíram"-se estrados para bailes populares;
houve desfile de batalhões escolares e eu me lembro que vi a Princesa
Imperial, na porta da atual Prefeitura, cercada de filhos, assistindo
àquela fieira de numerosos soldados desfiar devagar. Devia ser de tarde,
ao anoitecer.

Ela me parecia loura, muito loura, maternal, com um olhar doce e
apiedado. Nunca mais a vi e o Imperador\footnote{Dom Pedro \textsc{ii}, aclamado
  o segundo Imperador do Brasil, só assumiu de fato o trono em 1841. Sua
  filha, a princesa Isabel, foi quem assinou a ``Lei Áurea'' em 13 de
  maio de 1888. Dom Pedro \textsc{ii} governou o Brasil até o dia 15 de Novembro
  de 1889, quando a Monarquia veio a ser derrubada para a proclamação da
  República.} nunca vi, mas me lembro dos seus carros, aqueles enormes
carros dourados, puxados por quatro cavalos, com cocheiros montados e um
criado à traseira.

Eu tinha então sete anos e o cativeiro não me impressionava. Não lhe
imaginava o horror; não conhecia a sua injustiça. Eu me recordo, nunca
conheci uma pessoa escrava. Criado no Rio de Janeiro, na cidade, onde já
os escravos rareavam, faltava"-me o conhecimento direto da vexatória
instituição, para lhe sentir bem os aspectos hediondos.

Era bom saber se a alegria que trouxe à cidade a lei da abolição foi
geral pelo país. Havia de ser, porque já tinha entrado na consciência de
todos a injustiça originária da escravidão.

Quando fui para o colégio, um colégio público, à Rua do Resende, a
alegria entre a criançada era grande. Nós não sabíamos o alcance da lei,
mas a alegria ambiente nos tinha tomado.

A professora, Dona Teresa Pimentel do Amaral, uma senhora muito
inteligente, a quem muito deve o meu espírito, creio que nos explicou a
significação da coisa; mas com aquele feitio mental de criança, só uma
coisa me ficou: livre! livre!

Julgava que podíamos fazer tudo que quiséssemos; que dali em diante não
havia mais limitação aos propósitos da nossa fantasia.

Parece que essa convicção era geral na meninada, porquanto um colega
meu, depois de um castigo, me disse: ``Vou dizer a papai que não quero
voltar mais ao colégio. Não somos todos livres?''.

Mas como ainda estamos longe de ser livres! Como ainda nos enleamos nas
teias dos preceitos, das regras e das leis!

Dos jornais e folhetos distribuídos por aquela ocasião, eu me lembro de
um pequeno jornal, publicado pelos tipógrafos da Casa
Lombaerts\footnote{A Casa Lombaerts \& Cia foi uma importante livraria e
  editora criada no Rio de Janeiro, em 1853, pelo belga Jean"-Baptiste
  Lombaerts e depois continuada por seu filho, Henri Gustave Lombaerts.
  A empresa funcionou até o ano de 1904, quando entrou em dificuldades
  financeiras e foi adquirida pela maior livraria da época, a Francisco
  Alves.}. Estava bem impresso, tinha umas vinhetas
elzevirianas\footnote{Um tipo de letra bastante usada na imprensa da
  época, também conhecida como ``caracteres euzevirianos''}, pequenos
artigos e sonetos. Desses, dois eram dedicados a José do Patrocínio e o
outro à Princesa. Eu me lembro, foi a minha primeira emoção poética a
leitura dele. Intitulava"-se ``Princesa e Mãe'' e ainda tenho de memória
um dos versos:

``Houve um tempo, senhora, há muito já passado\ldots{}''

São boas essas recordações; elas têm um perfume de saudade e fazem com
que sintamos a eternidade do tempo.

Oh! O tempo! O inflexível tempo, que como o Amor, é também irmão da
Morte, vai ceifando aspirações, tirando presunções, trazendo desalentos,
e só nos deixa na alma essa saudade do passado às vezes composta de
coisas fúteis, cujo relembrar, porém, traz sempre prazer.

Quanta ambição ele não mata! Primeiro são os sonhos de posição: com os
dias e as horas e, a pouco e pouco, a gente vai descendo de ministro a
amanuense; depois são os do Amor --- oh! como se desce nesses! Os de
saber, de erudição, vão caindo até ficarem reduzidos ao bondoso
Larousse\footnote{Referência a Pierre Larousse (1817--1875), pedagogo
  e enciclopedista francês. Foi o criador do \emph{Grand dictionnaire
  universal du \textsc{xix} siècle} (Grande dicionário universal do século \textsc{xix}),
  mais conhecido como a Grande Enciclopédia Larousse Cultural.
  Originalmente publicada na França a partir de 1863, se espalhou por
  boa parte do ocidente.}. Viagens\ldots{} Oh! As viagens! Ficamos a
fazê"-las nos nossos pobres quartos, com auxílio do Baedecker\footnote{Karl
  Baedeker (1801--1859) foi um dos livreiros pioneiros na publicação
  de guias de viagens. Tornou"-se mundialmente famoso após a publicação
  do ``Guia de Viagem Baedeker'', que proporcionava aos turistas um
  conjunto de informações completas sobre transportes, alojamentos,
  postos turísticos, etc. Hoje em dia essas informações são facilmente
  encontradas através da internet e dos incontáveis guias de turismos
  existentes. Na época, foi uma grande invenção.} e outros livros
complacentes.

Obras, satisfações, glórias, tudo se esvai e se esbate. Pelos trinta
anos, a gente que se julgava Shakespeare, está rente que não passa de um
``Mal das Vinhas''\footnote{O mesmo que ``Zé Ninguém'' ou ``João
  Ninguém''.} qualquer; tenazmente, porém, ficamos a viver, ---
esperando, esperando\ldots{} o quê? O imprevisto, o que pode acontecer
amanhã ou depois. Esperando os milagres do tempo e olhando o céu vazio
de Deus ou Deuses, mas sempre olhando para ele, como o filósofo
Guyau\footnote{Trata"-se do filósofo francês Jean"-Marie Guyau (1854--1888). Seu livro ``A arte do ponto de vista sociológico'' exerceu \label{guyau}
  grande influência em Lima Barreto, principalmente no que se diz
  respeito ao tema da ``função da literatura na sociedade''.}.

Esperando, quem sabe se a sorte grande ou um tesouro oculto no quintal?

E maio volta\ldots{} Há pelo ar blandícias e afagos; as coisas ligeiras
têm mais poesia; os pássaros como que cantam melhor; o verde das
encostas é mais macio; um forte fluxo de vida percorre e anima
tudo\ldots{}

O mês augusto e sagrado pela poesia e pela arte, jungido eternamente à
marcha da Terra, volta; e os galhos da nossa alma que tinham sido
amputados --- os sonhos, enchem"-se de brotos muito verdes, de um claro e
macio verde de pelúcia, reverdecem mais uma vez, para de novo perderem
as folhas, secarem, antes mesmo de chegar o tórrido dezembro.

E assim se faz a vida, com desalentos e esperanças, com recordações e
saudades, com tolices e coisas sensatas, com baixezas e grandezas, à
espera da morte, da doce morte, padroeira dos aflitos e
desesperados\ldots{}



\chapter[Esta minha letra]{Esta minha letra\footnote[*]{Publicado no jornal \emph{Gazeta da Tarde} em 28/06/1911.}}

A minha letra é um bilhete de loteria. Às vezes ela me dá muito, outras
vezes tira"-me os últimos tostões da minha inteligência. Eu devia esta
explicação aos meus leitores, porque, sob a minha responsabilidade, tem
saído cada coisa de se tirar o chapéu. Não há folhetim\footnote{Na época
  em que Lima Barreto escrevia, os jornais ainda dedicavam um espaço
  para a literatura. Tal espaço era o \textit{Folhetim} do jornal.
  Muitos livros que conhecemos hoje em dia foram publicados ``em série''
  nos folhetins dos jornais. Daí a nomenclatura ``romance de folhetim''
  ser muito utilizada na época. O mais famoso livro de Lima Barreto,
  ``Triste fim de Policarpo Quaresma'', foi publicado nos folhetins do
  \emph{Jornal do Comércio,} do Rio de Janeiro, no segundo semestre de
  1911. O autor se refere a alguma parte desse folhetim que, por causa
  de sua péssima letra, levou à confusão entre a palavra `grandeza' e
  `pândega'.} em que não venham coisas extraordinárias. Se, às vezes,
não me põe mal com a gramática, põe"-me em hostilidade com o bom senso e
arrasta"-me a dizer coisas descabidas. Ainda no último folhetim, além de
um ou dois períodos completamente truncados e outras coisas, ela levou à
compreensão dos meus raros leitores --- grandeza --- quando se tratava
de pândega; num artigo que publiquei há dias na \emph{Estação Teatral},
este então totalmente empastelado\footnote{O jornal \emph{A Estação
  Teatral} circulou no Rio de Janeiro entre 1910 e 1912. Era voltado
  para as artes do Teatro, Música, Pintura e Cinema (que começava a
  nascer no Brasil). Lima Barreto colaborou com esse periódico
  escrevendo alguns artigos sobre teatro e pintura. Chegou a publicar
  uma peça de teatro, ``Casa de Poetas --- Comédia em um Ato'', na edição
  de 27 de maio de 1911. O artigo ao qual o autor se refere, muito
  provavelmente, chama"-se ``Qualquer coisa'', publicado na edição de 24
  de junho de 1911. \emph{A Estação Teatral} deixou de circular no
  inicio de 1912, por falta de recursos financeiros. Era costume na
  época dizer que um jornal foi ``empastelado'' quando entrava em
  falência ou quando era fechado pela polícia, algo bastante corriqueiro
  naquele tempo.}, havia coisas do arco"-da"-velha.

Aqui já saiu um folhetim meu, aquele que eu mais estimo, ``Os galeões do
México'', tão truncado, tão doido, que mais parecia delírio que coisa de
homem são de espírito. Tive medo de ser recolhido ao hospício\ldots{}

Que ela me levasse a incorrer na crítica gramatical da terra, vá; mas
que me leve a dizer coisas contra a clara inteligência das coisas,
contra o bom senso e o pensar honesto e com plena consciência do que
estou fazendo! E não sei a razão por que a minha letra me trai de
maneira tão insólita e inesperada. Não digo que sejam os tipógrafos ou
os revisores; eu não digo que sejam eles que me fazem escrever ``a
exposição de palavras sinistras'' quando se tratava de ``exposição de
projetos sinistros''. Não, não são eles, absolutamente não são eles. Nem
eu. É a minha letra.

Estou nesta posição absolutamente inqualificável, original e pouco
classificável: um homem que pensa uma coisa, quer ser escritor, mas a
letra escreve outra coisa e asnática. Que hei de fazer?

Eu quero ser escritor, porque quero e estou disposto a tomar na vida o
lugar que colimei. Queimei os meus navios; deixei tudo, tudo, por essas
coisas de letras.

Não quero aqui fazer a minha biografia; basta, penso eu, que lhes diga
que abandonei todos os caminhos, por esse das letras; e o fiz
conscientemente, superiormente, sem nada de mais forte que me desviasse
de qualquer outra ambição; e agora vem essa coisa de letra, esse último
obstáculo, esse premente pesadelo, e não sei que hei de fazer!

Abandonar o propósito; deixar a estrada desembaraçada a todos os gênios
explosivos e econômicos de que esses Brasis e os políticos nos
abarrotam?

É duro fazê"-lo, depois de quase dez anos de trabalho, de esforço
contínuo e --- por que não dizer? --- de estudo, sofrimento e
humilhações. Mude de letra, disse"-me alguém.

É curioso. Como se eu pudesse ficar bonito, só pelo fato de querer.

Ora, esse meu conselheiro é um dos homens mais simples que eu conheço.
Mudar de letra! Onde é que ele viu isso? Com certeza ele não disse isso
ao Sr. Alcindo Guanabara\footnote{Alcindo Guanabara (1865--1918) foi
  um dos mais importantes jornalistas da época.}, cuja letra é famosa
nos jornais. Que o fizesse, com certeza, ele não diria ao Sr. Machado de
Assis também. O motivo é simples: o Sr. Alcindo é o chefe, é príncipe do
jornalismo, é deputado; e Machado de Assis era grande chanceler das
letras, homem aclamado e considerado; ambos, portanto, não podiam mudar
de letra; mas eu, pobre autor de um livreco, eu que não sou nem doutor
em qualquer história --- eu, decerto, tenho o dever e posso mudar de
letra.

Outro conselheiro (são sempre pessoas a quem faço reclamações sobre os
erros) disse"-me: escreva em máquina\footnote{Referência à máquina de
  escrever, praticamente inexistente hoje em dia.}. Ponho de parte o
custo de um desses desgraciosos aparelhos, e lembro aqui os senhores que
aquilo é fatigante, cansa muito e obrigava"-me ao trabalho nauseante de
fazer um artigo duas vezes: escrever a pena e passar a limpo em máquina.

O mais interessante é que a minha letra, além de me ter emprestado uma
razoável estupidez, fez"-me arranjar inimigos. Não tenho a indiferença
que toda a gente tem pelos inimigos; se não tenho medo, não sou neutro
diante deles; mas isso de ter inimigos só por causa da letra, é de
espantar, é de mortificar.

Já não posso entrar na revisão e nas oficinas aqui da casa. Logo na
entrada percebo a hostilidade muda contra mim e me apavoro. Se fosse no
cenáculo do Garnier\footnote{Essa história começa com a migração para o
  Brasil, em 1844, do livreiro francês Baptiste"-Louis Garnier.
  Habilidoso comerciante de livros, Garnier impulsionou o mercado
  livreiro ao instalar no Rio de Janeiro uma sede da famosa livraria
  francesa. Com seu falecimento, em 1893, seu irmão, Hippolyte Garnier,
  que jamais veio ao Brasil, assumiu a direção comercial da empresa e
  enviou um sobrinho para cuidar da matriz brasileira. Em 1901, foi
  inaugurada uma nova e magnifica sede da livraria Garnier, que
  rapidamente virou o grande ponto de encontro (cenáculo) de escritores
  e jornalistas. Lima Barreto escreveu um artigo quando da morte de
  Hippolyte Garnier, em 1911, que se chama ``O Garnier Morreu'',
  publicada no jornal \emph{Gazeta da Tarde}, em 07 de agosto de 1911.}
ou em outro qualquer, seria bom; se fosse mesmo no salão literário do
Coelho Neto\footnote{Henrique Maximiano Coelho Neto. Nasceu no estado do
  Maranhão em 1864, mas viveu boa parte de sua vida no Rio de Janeiro.
  Um dos mais célebres escritores brasileiros do início do século \textsc{xx}.
  Escreveu dezenas de romances, além de contos, peças teatrais, artigos
  jornalísticos. Faleceu no Rio de Janeiro em 1934. Lima Barreto não
  gostava da literatura de Coelho Neto e contra ele escreveu inúmeros
  textos. Um excelente livro que analisa a relação literária entre os
  dois escritores é o ``Lima Barreto e Coelho Neto: um Fla"-Flu
  literário'', escrito pelo pesquisador Mauro Rosso.}, eu ficaria
contente; entre aqueles homens simples, porém, com os quais eu não
compito em nada, é para a gente julgar"-se um monstro, um peste, um
flagelo. E tudo isso por quê? Por causa da minha letra. Desespero
decididamente.

De manhã, quando recebo a~\emph{Gazeta}~ou outra publicação em que haja
coisas minhas, eu me encho de medo, e é com medo que começo a ler o
artigo que firmo com a responsabilidade do meu humilde nome. A
continuação da leitura é então um suplício. Tenho vontade de chorar, de
matar, de suicidar"-me; todos os desejos me passam pela alma e todas as
tragédias vejo diante dos olhos. Salto da cadeira, atiro o jornal ao
chão, rasgo"-o; é um inferno.

Eu não sei se todos nos jornais têm boa caligrafia. Certamente, hão de
ter e os seus originais devem chegar à tipografia quase impressos. Nas
letras, porém, não é assim.

Eu não cito autores, porque citar autores só se pode fazer aos ilustres,
e seria demasia eu me pôr em paralelo com eles, mesmo sendo em negócio
de caligrafia. Deixo"-os de parte e só quero lembrar os que escreveram
grandes obras, belas, corretas, até ao ponto em que as coisas humanas
podem ser perfeitas. Como conseguiram isso?

Não sei; mas há de haver quem o saiba e espero encontrar esse alguém
para explicar"-me.

De tal modo essa questão de letra está implicando com o meu futuro que
eu já penso em casar"-me. Hão de surpreender"-se em ver estas duas coisas
misturadas: boa letra e casamento. O motivo é muito simples e vou
explicar a gênese da associação com toda a clareza de detalhes.

Foi um dia destes. Eu vinha de trem muito aborrecido porque saíra o meu
folhetim todo errado. O aspecto desordenado dos nossos subúrbios ia se
desenrolando aos meus olhos; o trem se enchia da mais fina flor da
aristocracia dos subúrbios. Os senhores com certeza não sabiam que os
subúrbios têm uma aristocracia.

Pois têm. É uma aristocracia curiosa, em cuja composição entrou uma
grande parte dos elementos médios da cidade inteira: funcionários de
pequena categoria, chefes de oficinas, pequenos militares, médicos de
fracos rendimentos, advogados sem causa etc.

Iam entrando com a ``morgue'' que caracteriza uma aristocracia de tal
antiguidade e tão fortes rendimentos, quando uma moça, carregada de
lápis, penas, réguas, cadernos, livros, entrou também e veio sentar"-se a
meu lado.

Não era feia, mas não era bela. Tinha umas feições miúdas, um triste
olhar pardo de fraco brilho, uns cabelos pouco abundantes, um colo
deprimido e pouco cheio. Tudo nela era pequenino, modesto; mas era,
afinal, bonitinha, como lá dizem os namorados.

Olhei"-a com o temor com que sempre olho as damas e continuei a mastigar
as minhas mágoas.

Num dado momento, ela puxou um dos muitos cadernos que trazia, abriu"-o,
dobrou"-o e pôs"-se a ler. Que não me levem a mal o \emph{Binóculo} e a
\emph{Nota Chic}\footnote{O \emph{Binóculo} era uma coluna do jornal
  \emph{Gazeta de Notícias}, do Rio de Janeiro. Criada pelo jornalista
  Figueiredo Pimentel e que circulou entre 1907 e 1914. Tratava"-se do
  que conhecemos hoje como ``coluna social'', na qual aparecem
  reportagens sobre as `celebridades'. O público mais cativo da coluna
  era o feminino. Lima Barreto também implicava e criticava muito os
  jornalistas e as matérias que eram escritas para o \emph{Binóculo},
  por considerar que só tratavam de coisas fúteis. A \emph{Nota
  ``chic''} (hoje escreve"-se chique), também era uma coluna da
  \emph{Gazeta de Notícias}, menor e menos importante do que o
  \emph{Binóculo} e que consistia em parabenizar alguma pessoa ou evento
  que aconteceu. Basicamente seguia esse esquema: ``A nota `chic' de
  hoje é incontestavelmente a estreia do primeiro tenor da companhia
  Marchetti --- Alessandrini, que nos aparecerá numa opereta nova para o
  Rio --- ``Valsa do Amor'', do maestro Ziehrer''. (trecho extraído da
  edição de 29 de junho de 1910).} e não deitem por isso excomunhão
sobre mim! Sei bem que não é de boa educação ler o que os outros estão
lendo ao nosso lado; mas não me contive e deitei uma olhadela, tanto
mais (notem bem os senhores do~\emph{Binóculo}~e da~\emph{Nota Chic})
que, me pareceu, a moça o fazia para ralar"-me de inveja ou encher"-me de
admiração por ela.

Tratava"-se de álgebra, e as mulheres têm pela matemática uma fascinação
de ídolo inacessível. Foi, portanto, para mostrar"-me que ela o ia
atingindo que desdobrou o caderno; ou então para dizer"-me sem palavras:
Veja, você, seu homem! Você anda de calças, mas não sabe isso\ldots{} Ela se
enganava um pouco.

Mas\ldots{} como dizia: olhei o caderno e o que vi, meu Deus! Uma letra, um
cursivo irrepreensível, com todos os tracinhos, com todas as filigranas.
Os ``tt'' muito bem traçados --- uma maravilha!

Ah! pensei eu. Se essa moça se quisesse casar comigo, como eu não seria
feliz? Como diminuiriam os meus inimigos e as tolices que são escritas
por minha conta? Copiava"-me os artigos e\ldots{}

Quis namorá"-la, mas não sei namorar, não só porque não sei, como também
porque tenho consciência da minha fealdade. Fui, pois, tão canhestro,
tão tolo, tão inábil, que ela nem percebeu. Um namoro de\ldots{} caboclo.

Seria, casar"-me com ela, uma solução para esse meu problema da letra,
mas nem este mesmo eu posso encontrar e tenho que aguentar esse meu
inimigo, essa traição que está nas minhas mãos, esse abutre que me
devora diariamente a fraca reputação e apoucada inteligência.

\chapter[Elogio da morte]{Elogio da morte\footnote[*]{Publicado no jornal \emph{A.\,B.\,C.} em 19/10/1918.}}

Não sei quem foi que disse que a Vida é feita pela Morte. É a destruição
contínua e perene que faz a vida.

A esse respeito, porém, eu quero crer que a Morte mereça maiores
encômios.

É ela que faz todas as consolações das nossas desgraças; é dela que nós
esperamos a nossa redenção; é ela a quem todos os infelizes pedem
socorro e esquecimento.

Gosto da Morte porque ela é o aniquilamento de todos nós; gosto da Morte
porque ela nos sagra. Em vida, todos nós só somos conhecidos pela
calúnia e maledicência, mas, depois que Ela nos leva, nós somos
conhecidos (a repetição é a melhor figura de retórica), pelas nossas
boas qualidades.

\textls[-30]{É inútil estar vivendo, para ser dependente dos outros; é inútil estar
vivendo para sofrer os vexames que não merecemos.}

A vida não pode ser uma dor, uma humilhação de contínuos e burocratas
idiotas; a vida deve ser uma vitória. Quando, porém, não se pode
conseguir isso, a Morte é que deve vir em nosso socorro.

A covardia mental e moral do Brasil não permite movimentos de
independência; ela só quer acompanhadores de procissão, que só visam
lucros ou salários nos pareceres. Não há, entre nós, campo para as
grandes batalhas de espírito e inteligência. Tudo aqui é feito com o
dinheiro e os títulos. A agitação de uma ideia não repercute na massa e
quando esta sabe que se trata de contrariar uma pessoa poderosa, trata o
agitador de louco.

Estou cansado de dizer que os malucos foram os reformadores do mundo.

Le Bon\footnote{Gustave Le Bon (1841--1931) foi um importante pensador
  social francês. Influenciou bastante o pensamento de Lima Barreto,
  principalmente através de seus trabalhos sobre psicologia social, como
  o clássico ``Psicologia das Massas'', de 1895. O livro citado pelo
  autor, ``La civilisation des Arabes'' (A civilização dos Árabes), de
  1884, é um importante estudo de Le Bon acerca da contribuição daquela
  civilização para o pensamento filosófico, matemático, cultural, etc. e
  que também traz algumas reflexões sobre o profeta do islamismo, Maomé.}
dizia isto a propósito de Maomé, nas suas \emph{Civilisation des
arabes,} com toda a razão; e não há chanceler falsificado e secretária
catita que o possa contestar.

São eles os heróis; são eles os reformadores; são eles os iludidos; são
eles que trazem as grandes ideias, para melhoria das condições da
existência da nossa triste Humanidade.

Nunca foram os homens de bom senso, os honestos burgueses ali da esquina
ou das secretárias \emph{chics} que fizeram as grandes reformas no
mundo.

Todas elas têm sido feitas por homens, e, às vezes
mesmo mulheres, tidos por doidos.

A divisa deles consiste em não ser panurgianos\footnote{Adjetivo formado
  a parir do nome Panúrgio. Trata"-se de um personagem do livro
  ``Gangântua e Pantagruel'' escrito pelo francês Rabelais (1494--1553). Na história, Panúrgio é humilhado pelo personagem Dindenault
  durante uma viagem de barco. Para se vingar decide comprar um carneiro
  de Dindenault, que era um comerciante desse animal e trazia vários na
  embarcação. Panúrgio, de posse do carneiro, atira"-o ao mar. Todos os
  demais carneiros de Dindenault seguem aquele que foi arremessado do
  barco. A expressão ``carneiros de Panúrgio'' é utilizada no sentido de
  criticar as pessoas que apresentam esse comportamento de rebanho, que
  seguem o ``líder'' sem reflexão. Lima Barreto, em vez de utilizar a
  expressão ``carneiros de Panúrgio'', preferiu adjetiva"-la usando a
  palavra panurgiano. Para ele, os doidos são por excelência pessoas não
  panurgianas, ou seja, jamais agem como bando de carneiros ou boiada.}
e seguir a opinião de todos, por isso mesmo podem ver mais longe do que
os outros.

Se nós tivéssemos sempre a opinião da maioria, estaríamos ainda no
Cro"-Magnon\footnote{Referência à região localizada no sudoeste da
  França, conhecida por Cro"-Magnon onde, em 1868, uma equipe de
  arqueólogos encontrou algumas ossadas consideradas de ancestrais do
  \emph{homo sapiens}. São conhecidos como ``homens de Cro"-Magnon'' e
  teriam vivido a cerca de 35 mil anos atrás. Muitos pesquisadores
  atribuem aos ``homens de Cro"-Magnon'' os vestígios conhecidos como
  ``pinturas rupestres'' encontradas em vários lugares da Europa.} e não
teríamos saído das cavernas.

O que é preciso, portanto, é que cada qual respeite a opinião de
qualquer, para que desse choque surja o esclarecimento do nosso destino,
para própria felicidade da espécie humana.

Entretanto, no Brasil, não se quer isto. Procura"-se abafar as opiniões,
para só deixar em campo os desejos dos poderosos e prepotentes.

Os órgãos de publicidade por onde se podiam elas revelar são fechados e
não aceitam nada que os possa lesar.

Dessa forma, quem, como eu nasceu pobre e não quer ceder uma linha da
sua independência de espírito e inteligência, só tem que fazer elogios à
Morte.

Ela é a grande libertadora que não recusa os seus benefícios a quem lhe
pede. Ela nos resgata e nos leva à luz de Deus.

Sendo assim, eu a sagro, antes que ela me sagre na minha pobreza, na
minha infelicidade, na minha desgraça e na minha honestidade.

Ao vencedor, as batatas!\footnote{Nessa última frase da crônica, Lima
  Barreto faz referência direta ao livro ``Quincas Borba'', do escritor
  brasileiro Machado de Assis (1839--1908), escrito em folhetins na
  revista \emph{A Estação}, entre os anos 1886 e 1891, depois publicado
  em livro neste mesmo ano, pela livraria Garnier. ``Ao vencedor as
  batatas'' tornou"-se uma das mais célebres frases da literatura
  brasileira e diz respeito à filosofia do Humanitismo, desenvolvida
  pelo personagem Quincas Borba, primeiro no romance ``Memórias Póstumas
  de Brás Cubas'' (1881) e depois no próprio romance que leva o nome do
  personagem.}

%\part{Cotidiano e vida nos subúrbios}

\chapter[A polícia suburbana]{A polícia suburbana\footnote[*]{Publicado no jornal \emph{Correio da Noite} em 28/12/1914.}}

Noticiam os jornais que um delegado inspecionando, durante uma noite
destas, algumas delegacias suburbanas, encontrou"-as às moscas,
comissários a dormir e soldados a sonhar.

Dizem mesmo que o delegado inspetor surripiou objetos para pôr mais à
mostra o descaso dos seus subordinados.

Os jornais, com aquele seu louvável bom senso de sempre, aproveitaram a
oportunidade para reforçar as suas reclamações contra a falta de
policiamento nos subúrbios.

Leio sempre essas reclamações e pasmo. Moro nos subúrbios há muitos anos
e tenho o hábito de ir para a casa alta noite.

Uma vez ou outra encontro um vigilante noturno, um policial e muito
poucas vezes é"-me dado ler notícias de crimes nas ruas que atravesso.

A impressão que tenho é de que a vida e a propriedade daquelas paragens
estão entregues aos bons sentimentos dos outros e que os pequenos furtos
de galinhas e coradouros não exigem um aparelho custoso de patrulhas e
apitos.

Aquilo lá vai muito bem, todos se entendem livremente e o Estado não
precisa intervir corretivamente para fazer respeitar a propriedade
alheia.

Penso mesmo que, se as coisas não se passassem assim, os vigilantes,
obrigados a mostrar serviço, procurariam meios e modos de efetuar
detenções e os notívagos, como eu, ou os pobres"-diabos que lá procuram
dormida, seriam incomodados, com pouco proveito para a lei e para o
Estado.

Os policiais suburbanos têm toda a razão. Devem continuar a dormir.
Eles, aos poucos, graças ao calejamento do ofício, se convenceram de que
a polícia é inútil.

Ainda bem.



\chapter[O «muambeiro»]{O «muambeiro»\footnote[*]{Publicado na revista \emph{Careta} em 07/08/1915.}}

Quando saio de casa e vou à esquina da Estrada Real de Santa Cruz,
esperar o bonde, vejo bem a miséria que vai por este Rio de Janeiro.

Moro há mais de 10 anos naquelas paragens e não sei por que os humildes
e os pobres têm"-me na conta de pessoa importante, poderosa, capaz de
arranjar empregos e solver dificuldades.

Pergunta"-me um se deve assentar praça na Brigada, pois há oito meses não
trabalha no seu ofício de carpinteiro; pergunta"-me outro se deve votar
no Senhor Fulano; e, às vezes mesmo, consultam"-me sobre casos
embaraçosos. Houve um matador de porcos que pediu a minha opinião sobre
este caso curioso: se devia aceitar dez mil"-réis para matar o cevado do
capitão M., o que lhe dava trabalho por três dias, com a salga e o
fabrico de linguiças; ou se devia comprar o canastra por cinquenta
mil"-réis e revendê"-lo aos quilos pela redondeza. Eu, que nunca fui
versado em coisas de matadouro, olhei os Órgãos\footnote{Referência à
  Serra dos Órgãos, localizada na região serrana do Rio de Janeiro. Hoje
  em dia encontra"-se sob uma área de proteção ambiental, onde funciona o
  Parque Nacional da Serra dos Órgãos.} ainda fumarentos nestas manhãs
de cerração e pensei que o meu destino era ser vigário de uma pequena
freguesia.

Ultimamente, na esquina, veio ao meu encontro um homem com quem
conversei alguns minutos. Ele me contou a sua desdita com todo o vagar
de popular.

Era operário não sei de que ofício; ficara sem emprego, mas, como tinha
um pequeno sítio lá para as bandas do Timbó e algumas economias, não se
atrapalhou em começo. As economias foram"-se, mas ficou"-lhe o sítio, com
as suas laranjeiras, com as suas tangerineiras, as suas bananeiras,
árvore de futuro com a qual o Senhor Cincinato Braga\footnote{Cincinato
  Cesar da Silva Braga (1864--1953), advogado e político paulista. Foi
  Deputado Federal por São Paulo e muito influente na vida republicana
  do país. Participou ativamente da chamada ``política de valorização do
  café'', na época o principal produto de exportação do Brasil. Lima
  Barreto fazia severas críticas, através da imprensa, em relação à
  ``política de valorização'', daí a tom de ironia com o qual menciona o
  nome do político e financista paulista.}, depois de salvar o café, vai
salvar o Brasil. Notem bem: depois.

Este ano foi particularmente abundante em laranjas e o nosso homem teve
a feliz ideia de vendê"-las. Vendo, porém, que os compradores na porta
não lhe davam o preço devido, tratou de valorizar o produto, mas sem
empréstimo a 30\%.

Comprou um cesto, encheu"-o de laranjas e saiu a gritar:

--- Vai laranja boa! Uma a vintém!

Foi feliz e pelo caminho apurou uns dois mil"-réis. Quando, porém, chegou
a Todos os Santos\footnote{Bairro localizado no subúrbio da cidade do
  Rio de Janeiro, onde Lima Barreto morou boa parte da vida.}, saiu"-lhe
ao encontro a lei, na pessoa de um guarda municipal:

--- Que dê a licença?

--- Que licença?

--- Já sei, intimou o guarda. Você é ``muambeiro''. Vamos para a agência.

Tomaram"-lhe o cesto, as laranjas, o dinheiro e, a muito custo,
deixaram"-no com a roupa do corpo.

Eis aí como se protege a pomicultura.



\chapter[A carroça dos cachorros]{A carroça dos cachorros\footnote[*]{Publicado na revista \emph{Careta} em 20/09/1919.}}

Quando de manhã cedo, saio da minha casa, triste e saudoso da minha
mocidade que se foi fecunda, na rua eu vejo o espetáculo mais engraçado
desta vida.

Amo os animais e todos eles me enchem do prazer da natureza.

Sozinho, mais ou menos esbodegado, eu, pela manhã desço a rua e vejo.

O espetáculo mais curioso é o da carroça dos cachorros. Ela me lembra a
antiga caleça dos ministros de Estado\footnote{Caleça ou caleche era um
  tipo de carruagem luxuosa puxada por cavalos, muito utilizada pelos
  membros da família real e por ministros na época do império.}, tempo
do império, quando eram seguidas por duas praças de cavalaria de
polícia.

Era no tempo da minha meninice e eu me lembro disso com as maiores
saudades.

--- Lá vem a carrocinha! --- dizem.

E todos os homens, mulheres e crianças se agitam e tratam de avisar os
outros.

Diz Dona Marocas a Dona Eugênia:

--- Vizinha! Lá vem a carrocinha! Prenda o Jupi!

E toda a ``avenida'' se agita e os cachorrinhos vão presos e escondidos.
Esse espetáculo tão curioso e especial mostra bem de que forma profunda
nós homens nos ligamos aos animais.

Nada de útil, na verdade, o cão nos dá; entretanto, nós o amamos e nós o
queremos.

Quem os ama mais, não somos nós os homens; mas são as mulheres e as
mulheres pobres, depositárias por excelência daquilo que faz a
felicidade e infelicidade da humanidade --- o Amor.

São elas que defendem os cachorros das praças de polícia e dos guardas
municipais; são elas que amam os cães sem dono, os tristes e desgraçados
cães que andam por aí à toa.

Todas as manhãs, quando vejo semelhante espetáculo, eu bendigo a
humanidade em nome daquelas pobres mulheres que se apiedam pelos cães.

A lei, com a sua cavalaria e guardas municipais, está no seu direito em
persegui"-los; elas, porém, estão no seu dever em acoitá"-los.



\chapter[Atribulações de um vendeiro]{Atribulações de um vendeiro\footnote[*]{Publicado na revista \emph{Careta} em 27/09/1919.}}

Não há coisa mais interessante de que observar uma venda.

Toda a gente tem mais ou menos escrúpulo em entrar em uma delas. Já o
tive também; mas, agora, é um prazer e meu agrado.

Aposentado e satisfeito da vida, logo, nas primeiras horas, a minha
satisfação é visitá"-las na minha redondeza.

Não me arrependo com isso, porquanto muito observo e adivinho.

Tem meu amigo, o senhor Carlos Ventura, um excelente camarada e
discípulo --- o Alípio.

Prego"-lhe todas as doutrinas subversivas que me vêm à cabeça; e ele me
ouve e medita.

Estou emprazado até para ser servente de pedreiro, desde que ele seja
chefe da empreitada.

O que quero é que ele me dê o atestado de que eu fui de alguma forma
trabalhador manual.

Entretanto, a coisa que mais curiosidade me provoca, é a atitude dos
vendedores das casas chamadas atacadistas.

São em geral rapazes finos, limpos e agradáveis. Chegam, sempre com um
jornal, mesureiros e delicados e dizem:

--- ``Seu'' Ventura, cá estou! Tenho um arroz finíssimo, Iguape. O senhor
quer?

O Senhor Ventura diz que não quer e o homem insiste. Daqui a instantes,
lá vem outro, também fino, delicado e elegante:

--- ``Seu'' Ventura, eu me lembro do senhor.

--- Por quê?

--- A razão é simples. Tenho uma banha com pouca água.

São todos assim e os varejistas ou, como chama o povo, taverneiros, têm
que atendê"-los de hora em hora. Já lhes deram um nome muito engraçado;
são os tocadores de realejo.

Não há mais essa espécie de necessitados; mas ficaram os caixeiros
vendedores para substitui"-los e atribular os varejistas, os vendeiros,
que, às vezes, são a providência de muita gente pobre.

Entretanto, não são eles só perseguidos por esses rapazes vendedores. Há
um flagelo maior; são os almofadinhas do comissariado. Conhecidos são
eles por afinadores de piano.

Merecem muito esse apelido, mesmo não trazendo o diapasão.

Para eles não há necessidade disso. Basta para a sua economia política a
tabela do comissariado e por parte dos afinadores do comissariado e as
ordens do Vieira Souto\footnote{Luiz Raphael Viera Souto (1849--1922),
  Engenheiro Civil e Matemático, ficou bastante conhecido por suas
  atividades industriais e pelo ímpeto modernizador. Entre suas
  principais atividades estão a presidência da Comissão encarregada da
  construção do Porto do Rio de Janeiro e a direção da construção da
  importante avenida Beira Mar.}.

Um pobre taverneiro sofre de ambos lados, isto é, por parte dos
afinadores do comissariado e por parte dos tocadores de realejo. Eis aí
a fortuna de um taverneiro.



\chapter[Atribulações de um autor]{Atribulações de um autor\footnote[*]{Publicado na revista \emph{Careta} em 10/09/1921.}}

--- Vou deixar a literatura.

--- Por quê?

--- Porque ela nada rende, senão desgostos: além dos que provêm
propriamente dela, há outros.

--- Quais são?

--- Você não imagina como sou assediado no bairro modesto em que moro. As
crianças me pedem livros de ``histórias'', os marmanjos querem cartas
para namoradas; as moças querem versos; os velhos me perguntam se tenho
\emph{O judeu errante}\footnote{Referência ao livro \emph{Le juif errant}
  (O judeu errante), do escritor francês Eugene Sue (1804--1857),
  publicado em 1845 e que teve enorme sucesso na França e em diversos
  outros países, incluindo o Brasil. A história do ``judeu errante'' é
  um misto de lenda e mito, que se criou em torno do episódio da
  ``Paixão de Cristo'', a partir de um judeu que teria supostamente
  zombado de Jesus Cristo, quando percorria o caminho da crucificação. O
  judeu teria recebido uma maldição perpétua --- caminharia errante pela
  Terra até a volta de Jesus, no final dos tempos. Desse ponto da lenda,
  várias outras versões da história surgiram, a partir da Idade Média
  até período do romantismo. Nessas versões, posteriores ao século \textsc{xvi},
  o ``judeu errante'' recebeu o nome de Ahasverus e geralmente aparece
  nas narrativas com um viés antissemita.} ou \emph{Os doze pares de
França}\footnote{Muito provavelmente, Lima Barreto está se referindo a
  algum livro que traz a história dos ``doze pares de França'', outra
  história que tem várias versões a partir de uma matriz bastante
  antiga, neste caso, a ``Canção de Rolando'', do século \textsc{viii}, que narra
  as batalhas travadas entre o guerreiro Rolando, fiel ao Imperador
  Carlos Magno, e os povos do norte da África, conhecido como
  ``mouros''. Os ``doze pares'' fazem referência aos guerreiros da elite
  das tropas do imperador, que foram mortos durante uma batalha. A
  história também é conhecida como ``A história de Carlos Magno e os
  doze pares de França'', que teve muita influência nas composições de
  cordel no nordeste do Brasil.}.

--- Ora, bolas!

--- Não se ria você; é a pura verdade --- garanto a você! Já me pediram
até uma cantiga de carnaval para um ``rancho''\footnote{Os
  \emph{ranchos} e \emph{cordões} foram os primeiros grupos organizados
  a desfilar nos dias de Carnaval. São os precursores dos blocos de
  carnaval que, nos dias de hoje, principalmente no Rio de Janeiro e de
  Salvador, chegam a contar com milhões de pessoas.} de moças\ldots{} Veja
você só!

--- Que é que você disse?

--- Que não sabia fazer versos, sobretudo os de\ldots{} carnaval.

Os dois amigos conversavam numa sala pobre de casa pobre, cuja única
riqueza eram livros. O que acaba de falar, repetiu:

--- Vou deixar a literatura; é um aborrecimento\ldots{}

--- Mesmo que você a deixe, eles não acreditarão e continuarão a
perseguir você.

--- Você tem razão. Qual o remédio?

--- É você mudar"-se de bairro, ir para outro extremo da cidade.

--- A ideia é boa, mas a despesa que tenho que fazer é grande.

--- Não há dúvida; mas é preciso, meu caro.

--- De resto, ainda por cima, sou perseguido pelos poetas incipientes.
Eles me invadem a casa, com os seus poemas e novelas; convidam"-me para
isso e para aquilo; e, quando lhes dou uma opinião sincera, zangam"-se e
me desfeiteiam. Um inferno, Deus dos Céus!

--- Por tudo isso, passam os autores célebres --- fez o outro rindo"-se.

--- Mas, eu não sou célebre. Se ainda fosse um acadêmico sisudo --- vá! ---
mas não sou mais nem menos que um autor pobre, modesto e simples.

--- E por isso mesmo é que eles procuram você. Se você fosse um autor
grave, bem posto, que pesasse as palavras e os gestos, eles não
procurariam; mas você não é assim, menino\ldots{} o que se há de fazer?

--- Há coisa mais séria que lamento: são os livros de valor que recebo, e
sobre os quais não tenho tempo de dar notícia. São tantos que não me é
possível atendê"-los logo; e os seus autores hão de julgar que não o faço
por descaso, desdém ou orgulho. Enfim: essa literatura é para mim um
tormento. Vou deixá"-la.

--- Qual o quê! Você só a deixará com a Morte.

--- O queixoso olhou para o céu, através da janela do aposento, e disse
com mágoa:

--- Talvez nem assim\ldots{}



\chapter[O caso do mendigo]{O caso do mendigo\footnote[*]{Publicado na revista \emph{Careta} em 20/03/1920.}}

Os jornais anunciaram, entre indignados e jocosos, que um mendigo, preso
pela polícia, possuía em seu poder valores que montavam à respeitável
quantia de seis contos e pouco\footnote{A moeda corrente na época em que
  Lima Barreto escreveu esse texto era o \emph{Conto de Réis} ou
  simplesmente \emph{Réis}. Vem justamente dai o nome de nossa moeda
  atual, o \emph{Real}. O \emph{Réis} era composto a partir de unidades
  decimais, ou seja, 500 \emph{réis}, 1000 \emph{réis,} 2000
  \emph{réis}, etc. Quando se chegava à quantia de um milhão de
  \emph{réis} (1:000\$000 réis) adotava"-se a nomenclatura
  \emph{Conto}. O mendigo que figura como personagem nessa crônica
  estava com uma quantia de um pouco mais de seis milhões de \emph{réis}
  (``seis contos e pouco''). Se convertermos essa quantidade em nossa
  moeda atual, o mendigo estaria com aproximadamente duzentos e
  cinquenta mil reais! (R\$ 250.000,00). No site
  \emph{acervo.estadao.com.br} é
  possível encontrar um conversor de valores, a partir do qual pode"-se
  fazer as conversões dos valores antigos em estimativas próximas dos
  valores atuais.}.

Ouvi mesmo comentários cheios de raiva a tal respeito. O meu amigo \textsc{x},
que é o homem mais esmoler desta terra, declarou"-me mesmo que não dará
mais esmolas. E não foi só ele a indignar"-se. Em casa de família de
minhas relações, a dona da casa, senhora compassiva e boa, levou a tal
ponto a sua indignação, que propunha se confiscasse o dinheiro ao cego
que o ajuntou.

Não sei bem o que fez a polícia com o cego. Creio que fez o que o Código
e as leis mandam; e, como sei pouco das leis e dos códigos, não estou
certo se ela praticou o alvitre lembrado pela dona da casa de que já
falei.

O negócio fez"-me pensar e, por pensar, é que cheguei a conclusões
diametralmente opostas à opinião geral.

O mendigo não merece censuras, não deve ser perseguido, porque tem todas
as justificativas a seu favor. Não há razão para indignação, nem
tampouco para perseguição legal ao pobre homem.

Tem ele, em face dos costumes, direito ou não a esmolar? Vejam bem que
eu não falo de leis; falo dos costumes. Não há quem não diga: sim.
Embora a esmola tenha inimigos, e dos mais conspícuos, entre os quais,
creio, está M. Bergeret\footnote{Trata"-se do personagem \emph{Monsieur}
  Bergeret, professor, mestre de conferências na faculdade de letras,
  figura central no romance do escritor francês Anatole France (1844--1924), intitulado ``M. Bergeret à Paris'' (Monsenhor Bergeret em
  Paris), de 1901. No livro, há uma cena em que o professor Bergeret e
  sua filha, Pauline, encontram alguns mendigos na rua. A cena apresenta
  um comentário bastante preconceituoso --- e irônico --- por parte
  narrador do livro, que expressa também a `visão de mundo' de M.
  Bergeret. Anatole France foi um dos escritores mais populares no
  Brasil na época de Lima Barreto e também muito admirado pelo escritor
  brasileiro. Existem boas traduções para o português dos livros de
  Anatole France.}, ela ainda continua a ser o único meio de
manifestação da nossa bondade em face da miséria dos outros. Os séculos
a consagraram; e, penso, dada a nossa defeituosa organização social, ela
tem grandes justificativas. Mas não é bem disso que eu quero falar. A
minha questão é que, em face dos costumes, o homem tinha direito de
esmolar. Isto está fora de dúvida.

Naturalmente ele já o fazia há muito tempo, e aquela respeitável quantia
de seis contos talvez represente economias de dez ou vinte anos.

Há, pois, ainda esta condição a entender: o tempo em que aquele dinheiro
foi junto. Se foi assim num prazo longo, suponhamos dez anos, a coisa é
assim de assustar? Não é. Vamos adiante.

Quem seria esse cego antes de ser mendigo? Certamente um operário, um
homem humilde, vivendo de pequenos vencimentos, tendo às vezes falta de
trabalho; portanto, pelos seus hábitos anteriores de vida e mesmo pelos
meios de que se servia para ganhá"-la, estava habituado a economizar. É
fácil de ver por quê. Os operários nem sempre têm serviço constante. A
não ser os de grandes fábricas do Estado ou de particulares, os outros
contam que, mais dias, menos dias, estarão sem trabalhar, portanto sem
dinheiro; daí lhes vem a necessidade de economizar, para atender a essas
épocas de crise.

Devia ser assim o tal cego, antes de o ser. Cegando, foi esmolar. No
primeiro dia, com a falta de prática, o rendimento não foi grande; mas
foi o suficiente para pagar um caldo no primeiro frege que encontrou, e
uma esteira na mais sórdida das hospedarias da Rua da Misericórdia. Esse
primeiro dia teve outros iguais e seguidos; e o homem se habituou a
comer com duzentos réis e a dormir com quatrocentos; temos, pois, o
orçamento do mendigo feito: seiscentos réis (casa e comida) e, talvez,
cem réis de café; são, portanto, setecentos réis por dia.

Roupa, certamente, não comprava: davam"-lhe. É bem de crer que assim
fosse, porque bem sabemos de que maneira pródiga nós nos desfazemos dos
velhos ternos.

Está, portanto, o mendigo fixado na despesa de setecentos réis por dia.
Nem mais, nem menos; é o que ele gastava. Certamente não fumava e muito
menos bebia, porque as exigências do ofício haviam de afastá"-lo da
``caninha''. Quem dá esmola a um pobre cheirando a cachaça? Ninguém.

Habituado a esse orçamento, o homenzinho foi se aperfeiçoando no ofício.
Aprendeu a pedir mais dramaticamente, a aflautar melhor a voz; arranjou
um cachorrinho, e o seu sucesso na profissão veio.

Já de há muito que ganhava mais do que precisava. Os níqueis caíam, e o
que ele havia de fazer deles? Dar aos outros? Se ele era pobre, como
podia fazer? Pôr fora? Não; dinheiro não se põe fora. Não pedir mais? Aí
interveio uma outra consideração.

Estando habituado à previdência e à economia, o mendigo pensou lá
consigo: há dias que vem muito; há dias que vem pouco, sendo assim, vou
pedindo sempre, porque, pelos dias de muito, tiro os dias de nada.
Guardou. Mas a quantia aumentava. No começo eram só vinte mil"-réis; mas,
em seguida foram quarenta, cinquenta, cem. E isso em notas, frágeis
papéis, capazes de se deteriorarem, de perderem o valor ao sabor de uma
ordem administrativa, de que talvez não tivesse notícia, pois, era cego
e não lia, portanto. Que fazer, em tal emergência, daquelas notas?
Trocar em ouro? Pesava, e o tilintar especial dos soberanos, talvez
atraísse malfeitores, ladrões. Só havia um caminho: trancafiar o
dinheiro no banco. Foi, o que ele fez. Estão aí um cego de juízo e um
mendigo rico.

Feito o primeiro depósito, seguiram"-se a este outros; e, aos poucos,
como hábito é segunda natureza, ele foi encarando a mendicidade não mais
como um humilhante imposto voluntário, taxado pelos miseráveis aos ricos
e remediados; mas como uma profissão lucrativa, lícita e nada
vergonhosa.

Continuou com o seu cãozinho, com a sua voz aflautada, com o seu ar
dorido a pedir pelas avenidas, pelas ruas comerciais, pelas casas de
famílias, um níquel para um pobre cego. Já não era mais pobre; o hábito
e os preceitos da profissão não lhe permitiam que pedisse uma esmola
para um cego rico.

O processo por que ele chegou a ajuntar a modesta fortuna de que falam
os jornais, é tão natural, é tão simples, que, julgo eu, não há razão
alguma para essa indignação das almas generosas.

Se ainda continuasse a ser operário, nós ficaríamos indignados se ele
tivesse juntado o mesmo pecúlio? Não. Por que então ficamos agora?

É porque ele é mendigo, dirão. Mas é um engano. Ninguém mais que um
mendigo tem necessidade de previdência. A esmola não é certa; está na
dependência da generosidade dos homens, do seu estado moral psicológico.
Há uns que só dão esmolas quando estão tristes, há outros que só dão
quando estão alegres e assim por diante. Ora, quem tem de obter meios de
renda de fonte tão incerta, deve ou não ser previdente e econômico?

Não julguem que faço apologia da mendicidade. Não só não faço como não a
detrato.

Há ocasiões na vida que a gente pouco tem a escolher; às vezes mesmo
nada tem a escolher, pois há um único caminho. É o caso do cego. Que é
que ele havia de fazer? Guardar. Mendigar. E, desde que da sua
mendicidade veio"-lhe mais do que ele precisava, que devia o homem fazer?
Positivamente, ele procedeu bem, perfeitamente de acordo com os
preceitos sociais, com as regras da moralidade mais comezinha e atendeu
às sentenças do \emph{Bom homem Ricardo}, do falecido Benjamin
Franklin\footnote{Benjamin Franklin (1706--1790). Um dos mais
  importantes dentre os pensadores e cientistas dos Estados Unidos da
  América. O \emph{Bom homem Ricardo} faz referência ao almanaque ``Poor
  Richard's Almanac'' (Almanaque do pobre Ricardo), uma série de
  publicações que trouxe enorme popularidade a seu autor, principalmente
  pelos provérbios de cunho financeiro e moral, tais como ``\emph{Um
  tostão poupado é um tostão ganhado}'' ou o mundialmente famoso
  ``\emph{tempo é dinheiro}''. As ideias de Franklin influenciaram a
  criação dos quadrinhos do \emph{Tio Patinhas}, que fez bastante
  sucesso no Brasil, principalmente através do desenho animado ``Duck
  Tales'', da \emph{Disney}, exibido nos anos 1980 e 90.}.

As pessoas que se indignaram com o estado próspero da fortuna do cego,
penso que não refletiram bem, mas, se o fizerem, hão de ver que o homem
merecia figurar no \emph{Poder da vontade}, do conhecidíssimo
Smiles\footnote{Samuel Smiles (1812--1904) um dos mais famosos e lidos
  escritores escoceses, além de grande e aclamado palestrante e orador.
  Foi o criador do gênero que conhecemos hoje como ``literatura de
  autoajuda''. Seu livro mais popular, ``Self"-Help'' (Autoajuda), de
  1859, foi um enorme sucesso na Inglaterra e principalmente nos Estados
  unidos da América. As ideias de Smiles encontraram muitos adeptos
  entre os estadunidenses, pois já havia uma tradição bastante forte que
  remontava, principalmente, aos escritos de Benjamin Franklin e dos
  pregadores protestantes. Outros livros como ``Character'' (Caráter),
  de 1871, ``Thrift'' (Poupança), de 1875, ``Duty'' (Dever), de 1880 e
  ``Life and Labour'' (Vida e Trabalho), de 1887, ajudaram a solidificar
  a carreira do escritor. O livro ``O poder da vontade: caráter,
  comportamento e perseverança'' foi publicado no Brasil no ano de 1880,
  pela editora Garnier.}.

De resto, ele era espanhol, estrangeiro, e tinha por dever voltar rico.
Um acidente qualquer tirou"-lhe a vista, mas lhe ficou a obrigação de
enriquecer. Era o que estava fazendo, quando a polícia foi perturbá"-lo.
Sinto muito; e são meus desejos que ele seja absolvido do delito que
cometeu, volte à sua gloriosa Espanha, compre uma casa de campo, que
tenha um pomar com oliveiras e a vinha generosa; e, se algum dia, no
esmaecer do dia, a saudade lhe vier deste Rio de Janeiro, deste Brasil
imenso e feio, agarre em uma moeda de cobre nacional e leia o
ensinamento que o governo da República dá\ldots{} aos outros, através
dos seus vinténs: ``A economia é a base da prosperidade''.



\chapter[Os enterros de Inhaúma]{Os enterros de Inhaúma\footnote[*]{Publicado na revista \emph{Careta} em 26/08/1922.}}

Certamente há de ser impressão particular minha não encontrar no
cemitério municipal de Inhaúma aquele ar de recolhimento, de resignada
tristeza, de imponderável poesia do Além, que encontro nos outros.
Acho"-o feio, sem compunção, com um ar morno de repartição pública; mas
se o cemitério me parece assim, e não me interessa, os enterros que lá
vão ter, todos eles, aguçam sempre a minha atenção quando os vejo
passar, pobres ou não, a pé ou em coche"-automóvel\footnote{Coche era um
  dos muitos tipos de veículos não motorizados da época, normalmente
  puxados por animais, cavalos, burro ou bestas, assim como as
  carruagens, carroças, caleças, etc.}.

A pobreza da maioria dos habitantes dos subúrbios ainda mantém neles
esse costume rural de levar a pé, carregados a braços, os mortos
queridos.

É um sacrifício que redunda num penhor de amizade em uma homenagem das
mais sinceras e piedosas que um vivo pode prestar a um morto.

Vejo"-os passar e calculo que os condutores daquele viajante para tão
longínquas paragens, já andaram alguns quilômetros e vão carregar o
amigo morto, ainda durante cerca de uma légua. Em geral assisto a
passagem desses cortejos fúnebres na Rua José Bonifácio, canto da
Estrada Real. Pela manhã gosto de ler os jornais num botequim que há por
lá. Vejo os Órgãos\footnote{Sobre os ``Órgãos'', ver nota 21.}, quando
as manhãs estão límpidas, tintos com a sua tinta especial de um profundo
azul"-ferrete e vejo uma velha casa de fazenda que se ergue bem próximo,
no alto de uma meia laranja, passam carros de bois, tropas de mulas com
sacas de carvão nas cangalhas, carros de bananas, pequenas manadas de
bois, cujo campeiro cavalga atrás sempre com o pé direito embaralhado em
panos.

Em certos instantes, suspendo mais demoradamente a leitura do jornal, e
espreguiço o olhar por sobre o macio tapete verde do capinzal intérmino
que se estende na minha frente.

Sonhos de vida roceira me vêm; suposições do que aquilo havia sido,
ponho"-me a fazer. Índios, canaviais, escravos, troncos, reis, rainhas,
imperadores --- tudo isso me acode à vista daquelas coisas mudas que em
nada falam do passado.

De repente, tilinta um elétrico\footnote{Eram chamados de ``elétricos''
  os bondes movidos à eletricidade e que circulavam por sobre trilhos.
  Ainda hoje é possível fazer um passeio de bonde no Rio de Janeiro, no
  famoso \emph{Bondinho de Santa Teresa}, uma das muitas atrações
  turísticas da cidade.}, buzina um automóvel, chega um caminhão
carregado de caixas de garrafas de cerveja; então, todo o bucolismo do
local se desfaz, a emoção das priscas eras em que os coches de Dom João
\textsc{vi}\footnote{Dom João \textsc{vi} (1767--1826) chegou ao Brasil junto com a
  Família Real Portuguesa, em 1808, depois de terem deixado Portugal sob
  a ameaça dos exércitos franceses de Napoleão Bonaparte. Dom João já
  era o príncipe regente do reino português desde 1799, em virtude das
  complicações na saúde mental de sua mãe, Dona Maria \textsc{i} (que foi
  apelidada de ``D. Maria, a louca). Com a morte da mãe, assumiu
  oficialmente o Reino de Portugal, Brasil e Algarves, em 1816. A vinda
  da família real para o Brasil se constitui como um dos fatos mais
  importante de nossa história. O retorno de D.\,João \textsc{vi} a Portugal
  ocorreu em 1821, fato que, entre outros, acelerou o processo de
  Independência do Brasil e a consequente coroação de seu filho, D.\,Pedro \textsc{i}, como nosso primeiro monarca. A figura de D.\,João \textsc{vi} era muito
  marcante no Rio de Janeiro, principalmente pelo hábito de passear
  pelos arredores da cidade. Lima Barreto sempre demonstrou simpatia
  pelo monarca.} transitavam por ali, esvai"-se e ponho"-me a ouvir o
retinir de ferro malhado, uma fábrica que se constrói bem perto.

Vem, porém, o enterro de uma criança; e volto a sonhar.

São moças que carregam o caixão minúsculo; mas assim mesmo, pesa.
Percebo"-o bem, no esforço que fazem.

Vestem"-se de branco e calçam sapatos de salto alto. Sopesando o esquife,
pisando o mau calçamento da rua, é com dificuldade que cumprem a sua
piedosa missão. E eu me lembro que ainda têm de andar tanto! Contudo,
elas vão ficar livres de um suplício; é o do calçamento da Rua do
Senador José Bonifácio. É que vão entrar na Estrada Real; e, naquele
trecho, a prefeitura só tem feito amontoar pedregulhos, mas tem deixado
a vetusta via pública no estado de nudez virginal em que nasceu. Isto há
anos que se verifica.

Logo que as portadoras do defunto pisam o barro unido do velho trilho,
adivinho que elas sentem um grande alívio dos pés à cabeça. As
fisionomias denunciam. Atrás, seguem outras moças que as auxiliarão bem
depressa, na sua tocante missão de levar um mortal à sua última morada
neste mundo; e, logo após, graves cavalheiros de preto, com o chapéu na
mão, carregando palmas de flores naturais, algumas com aspecto
silvestre, e baratas e humildes coroas artificiais fecham o cortejo.

Este calçamento da Rua Senador José Bonifácio, que deve datar de uns
cinquenta anos é feito de pedacinhos de seixos mal ajustados e está
cheio de depressões e elevações imprevistas. É mau para os defuntos; e
até já fez um ressuscitar.

Conto"-lhes. O enterro era feito em coche puxado por muares. Vinha das
bandas do Engenho Novo, e tudo corria bem. O carro mortuário ia na
frente, ao trote igual das bestas. Acompanhavam"-no seis ou oito caleças,
ou meias caleças, com os amigos do defunto. Na altura da estação de
Todos os Santos, o cortejo deixa a Rua Arquias Cordeiro e toma
perpendicularmente, à direita, a de José Bonifácio. Coche e caleças
põem"-se logo a jogar como navios em alto mar tempestuoso. Tudo dança
dentro deles. O cocheiro do carro fúnebre mal se equilibra na boleia
alta. Oscila da esquerda para a direita e da direita para a esquerda,
que nem um mastro de galera debaixo de tempestade braba. Subitamente,
antes de chegar aos ``Dois Irmãos'', o coche cai num caldeirão, pende
violentamente para um lado; o cocheiro é cuspido ao solo, as correias
que prendem o caixão ao carro, partem"-se, escorregando a jeito e vindo
espatifar"-se de encontro às pedras; e --- oh! terrível surpresa! do
interior do esquife, surge de pé --- lépido, vivo, vivinho, o defunto que
ia sendo levado ao cemitério a enterrar. Quando ele atinou e coordenou
os fatos não pôde conter a sua indignação e soltou uma maldição:
``Desgraçada municipalidade de minha terra que deixas este calçamento em
tão mal estado! Eu que ia afinal descansar, devido ao teu relaxamento
volto ao mundo, para ouvir as queixas da minha mulher por causa da
carestia da vida, de que não tenho culpa alguma; e sofrer as
impertinências do meu chefe Selrão, por causa das suas hemorroidas,
pelas quais não me cabe responsabilidade qualquer! Ah! Prefeitura de uma
figa, se tivesses uma só cabeça havias de ver as forças das minhas
munhecas! Eu te esganava, maldita, que me trazes de novo à vida!''

A este fato, eu não assisti, nem ao menos morava naquelas paragens,
quando aconteceu; mas pessoas dignas de toda a confiança me garantem a
autenticidade dele. Porém, um outro muito interessante aconteceu com um
enterro quando eu já morava por elas, e dele tive notícias frescas, logo
após o sucedido, por pessoas que nele tomaram parte.

Tinha morrido o Felisberto Catarino, operário, lustrador e empalhador
numa oficina de móveis de Cascadura\footnote{Bairro localizado na zona
  norte da cidade do Rio de Janeiro.}. Ele morava no Engenho de
Dentro\footnote{Outro bairro da zona norte do Rio, próximo a Cascadura.},
em casa própria, com razoável quintal, onde havia, além de alguns pés de
laranjeiras, uma umbrosa mangueira, debaixo da qual, aos domingos,
reunia colegas e amigos para bebericar e jogar a bisca\footnote{Jogo de
  cartas.}.

Catarino gozava de muita estima, tanto na oficina como na vizinhança.

Como era de esperar, o seu enterro foi muito concorrido e feito a pé,
com um denso acompanhamento. De onde ele morava, até ao cemitério de
Inhaúma, era um bom pedaço; mas os seus amigos a nada quiseram atender:
Resolveram levá"-lo mesmo a pé. Lá fora, e no trajeto, por tudo que era
botequim e taverna por que passavam, bebiam o seu trago. Quando o
caminho se tornou mais deserto até os condutores do esquife deixavam"-no
na borda da estrada e iam à taverna ``desalterar''. Numa das últimas
etapas do itinerário, os que carregavam, resolveram de mútuo acordo
deixar o pesado fardo para os outros e encaminharam"-se sub"-repticiamente
para a porta do cemitério. Tanto estes como os demais --- é de toda a
conveniência dizer --- já estavam bem transtornados pelo álcool. Outro
grupo concordou fazer o mesmo que tinham feito os carregadores dos
despojos mortais de Catarino; um outro, idem; e, assim, todo o
acompanhamento dividido em grupos, tomou o rumo do portão do
campo"-santo, deixando o caixão fúnebre com o cadáver de Catarino dentro
abandonado à margem da estrada.

Na porta do cemitério, cada um esperava ver chegar o esquife pelas mãos
de outros que não as deles; mas nada de chegar. Um, mais audaz, após
algum tempo de espera, dirigindo"-se a todos os companheiros, disse bem
alto:

--- Querem ver que perdemos o defunto?

--- Como? perguntaram os outros, a uma voz.

--- Ele não aprece e estamos todos aqui, refletiu o da iniciativa.

--- É verdade, fez outro.

Alguém então aventou:

--- Vamos procurá"-lo. Não seria melhor?

E todos voltaram sobre os seus passos, para procurar aquela agulha em
palheiro\ldots{}

Tristes enterros de Inhaúma! Não fossem essas tintas pinturescas e
pitorescas de que vos revestis de quando em quando de quanta reflexão
acabrunhadora não havíeis de sugerir aos que vos veem passar; e como não
convenceríeis também a eles que a maior dor desta vida não é
morrer\ldots{}\footnote{Interessante notar que Lima Barreto muda a
  \emph{pessoa gramatical} do texto neste último parágrafo. Passa a usar
  o pronome \emph{vós}, ou seja, a \emph{segunda pessoa do plural}, para
  se referir aos enterros de Inhaúma. A modulação dos verbos para a
  segunda pessoa do plural, no caso \emph{revestis, havíeis} e
  \emph{convenceríeis}, indica uma intensão de exaltação e respeito por
  parte do autor/narrador do texto. O uso poético do pronome ``vós'' foi
  uma constante durante muito tempo, sobretudo em sua função vocativa (Ó
  vós, bondoso e inebriante amor\ldots{}), mas também no tratamento
  dispensado a autoridades das mais diversas instituições. É de se notar
  a data em que foi publicada esta crônica --- 26/08/1922. Lima Barreto
  faleceu em 01/11/1922, pouco mais de dois meses após a saída da
  crônica.}



\chapter[Queixa de defunto]{Queixa de defunto\footnote[*]{Publicado na revista \emph{Careta} em 20/03/1920.}}

Antônio da Conceição, natural desta cidade, residente que foi em vida na
Boca do Mato, no Méier\footnote{Bairro da zona norte da cidade do Rio.},
onde acaba de morrer, por meios que não posso tornar público, mandou"-me
a carta abaixo que é endereçada ao prefeito. Ei"-la:

\begin{quote}
``Ilustríssimo e Excelentíssimo Senhor Doutor Prefeito do Distrito
Federal\footnote{Na época dessa crônica, o estado do Rio de Janeiro era
  então a capital do Brasil, por isso a denominação de Distrito Federal
  que a cidade do Rio de Janeiro também recebia.}. Sou um pobre homem
que em vida nunca deu trabalho às autoridades públicas nem a elas fez
reclamação alguma. Nunca exerci ou pretendi exercer isso que se chama os
direitos sagrados de cidadão. Nasci, vivi e morri modestamente, julgando
sempre que o meu único dever era ser lustrador de móveis e admitir que
os outros os tivessem para eu lustrar e eu não.

Não fui republicano\footnote{Os \emph{republicanos} formavam um grupo
  político que ganhou maior visibilidade e influência a partir do ano de
  1870. Foram eles os grandes propagandistas da República e críticos da
  Monarquia. Muitos \emph{republicanos} eram também abolicionistas, que
  lutavam pelo fim da escravidão. Em 1888 veio a Abolição e em 1889 a
  proclamação da República. A grande força dos \emph{republicanos} era o
  Partido Republicano Paulista (\textsc{prp}) e, principalmente, a imprensa.
  Mesmo com a proclamação, em 15 de novembro de 1889, os
  \emph{republicanos} não tiveram vida fácil. A reação pelo retorno da
  monarquia sempre foi uma constante ao longo dos primeiros anos da vida
  republicana.}, não fui florianista\footnote{Eram chamados de
  \emph{florianistas} os adeptos e seguidores do vice"-presidente da
  república, o Marechal Floriano Peixoto (1839--1895), que assumiu a
  presidência após a renúncia do Marechal Deodoro da Fonseca, em 1891.
  Floriano foi o segundo presidente da fase republicana do Brasil, entre
  1891 e 1894.}, não fui custodista\footnote{A renúncia do Marechal
  Deodoro da Fonseca, em 23 de novembro de 1891, provocou uma enorme
  crise política. Muitos acreditavam e defendiam que deveria haver novas
  eleições, pois não tinham sido decorridos dois anos da posse do
  presidente. O vice"-presidente, Marechal Floriano Peixoto, deveria
  cumprir um mandato temporário até que um novo presidente fosse eleito.
  A Constituição, em seu artigo nº 42, estipulava eleições em caso de
  vacância no cargo antes de decorridos dois anos da posse. Floriano e
  seus partidários não aceitavam tal solução e o impasse se fez. As
  tensões explodiram definitivamente em setembro de 1893, quando uma
  rebelião em alguns navios da Marinha ganhou proporções alarmantes. O
  conflito reuniu quase que totalmente a Marinha brasileira em torno da
  liderança do almirante Custódio José de Melo, Ministro da Marinha e da
  Guerra, que foi o líder da chamada Revolta da Armada. Contra os
  revoltosos ficou o Exército e apenas algumas unidades da Marinha. O
  Rio de Janeiro viveu meses de uma guerra cruel e sangrenta. A batalha
  ganhou um caráter bastante personalista, centrada nas figuras do
  Marechal Floriano versus o Almirante Custódio, daí a ideia de
  \emph{custodista}, que aparece na crônica, referente aos partidários
  do almirante Custódio de Melo.}, não fui hermista\footnote{Referência
  aos partidários e apoiadores do General Hermes da Fonseca (1855--1923), sobrinho de Deodoro da Fonseca e que, como o tio, foi
  presidente do Brasil entre os anos de 1910 e 1914.}, não me meti em
greves, nem coisa alguma de reivindicações e revoltas, mas morri na
santa paz do Senhor quase sem pecados e sem agonia.

Toda a minha vida de privações e necessidades era guiada pela esperança
de gozar depois de minha morte no sossego, uma calma de vida que não sou
capaz de descrever, mas que pressenti pelo pensamento, graças à
doutrinação das seções católicas dos jornais.

Nunca fui ao espiritismo, nunca fui aos `bíblias'\footnote{Forma um
  tanto quanto pejorativa com a qual os evangélicos eram chamados na
  época.}, nem a feiticeiros, e apesar de ter tido um filho que penou
dez anos nas mãos dos médicos, nunca procurei macumbeiros nem médiuns.

Vivi uma vida santa e obedecendo às prédicas do Padre André do Santuário
do Sagrado Coração de Maria, em Todos os Santos, conquanto as não
entendesse bem por serem pronunciadas com toda a eloquência em galego ou
vasconço\footnote{O \emph{galego} é uma língua muito parecida com o
  português. Já o termo \emph{vasconço} pode ser uma forma pejorativa de
  se referir ao idioma falado entre os povos bascos, que habitam o País
  Basco, região localizada entre o norte da Espanha e o sudoeste da
  França. O basco (também chamado de \emph{vasco}) é o idioma falado por
  cerca de 30\% daquela população. \emph{Vasconço} ganhou conotação de
  língua difícil, complicada, incompreensível.}.

Segui"-as, porém, com todo o rigor e humildade, e esperava gozar da mais
dúlcida paz depois de minha morte. Morri afinal um dia destes. Não
descrevo as cerimônias porque são muito conhecidas e os meus parentes e
amigos deixaram"-me sinceramente porque eu não deixava dinheiro algum. É
bom meu caro Senhor Doutor Prefeito, viver na pobreza, mas muito melhor
é morrer nela. Não se levam para a cova maldições dos parentes e amigos
deserdados; só carregamos lamentações e bênçãos daqueles a quem não
pagamos mais a casa.

Foi o que aconteceu comigo e estava certo de ir direitinho para o Céu,
quando, por culpa do Senhor e da Repartição que o Senhor dirige, tive
que ir para o inferno penar alguns anos ainda.

Embora a pena seja leve, eu me amolei, por não ter contribuído para ela
de forma alguma. A culpa é da Prefeitura Municipal do Rio de Janeiro que
não cumpre os seus deveres, calçando convenientemente as ruas. Vamos ver
por quê. Tendo sido enterrado no cemitério de Inhaúma e vindo o meu
enterro do Méier, o coche e o acompanhamento tiveram que atravessar em
toda a extensão a Rua José Bonifácio, em Todos os Santos.

Esta rua foi calçada há perto de cinquenta anos a macadame e nunca mais
foi o seu calçamento substituído. Há caldeirões de todas as
profundidades e largura, por ela afora. Dessa forma, um pobre defunto
que vai dentro do caixão em cima de um coche que por ela rola, sofre o
diabo. De uma feita um até, após um trambolhão do carro mortuário,
saltou do esquife, vivinho da silva, tendo ressuscitado com o susto.

Comigo não aconteceu isso, mas o balanço violento do coche machucou"-me
muito e cheguei diante de São Pedro cheio de arranhaduras pelo corpo. O
bom do velho santo interpelou"-me logo:

--- Que diabo é isto? Você está todo machucado! Tinham"-me dito que você
era bem comportado --- como é então que você arranjou isso? Brigou depois
de morto?

Expliquei"-lhe, mas não me quis atender e mandou que me fosse purificar
um pouco no inferno.

Está aí como, meu caro Senhor Doutor Prefeito, ainda estou penando por
sua culpa, embora tenha tido vida a mais santa possível. Sou, etc.,
etc''.
\end{quote}

Posso garantir a fidelidade da cópia e aguardar com paciência as
providências da municipalidade.



\chapter[As enchentes]{As enchentes\footnote[*]{Publicado no jornal \emph{Correio da Noite} em 19/01/1915.}}

As chuvaradas de verão, quase todos os anos, causam no nosso Rio de
Janeiro, inundações desastrosas.

Além da suspensão total do tráfego, com uma prejudicial interrupção das
comunicações entre os vários pontos da cidade, essas inundações causam
desastres pessoais lamentáveis, muitas perdas de haveres e destruição de
imóveis.

De há muito que a nossa engenharia municipal se devia ter compenetrado
do dever de evitar tais acidentes urbanos. Uma arte tão ousada e quase
tão perfeita, como é a engenharia, não deve julgar irresolvível tão
simples problema.

O Rio de Janeiro, da avenida, dos \emph{squares}, dos freios elétricos,
não pode estar à mercê de chuvaradas, mais ou menos violentas, para
viver a sua vida integral.

Como está acontecendo atualmente, ele é função da chuva. Uma vergonha!
Não sei nada de engenharia, mas, pelo que me dizem os entendidos, o
problema não é tão difícil de resolver como parece fazerem constar os
engenheiros municipais, procrastinando a solução da questão.

O Prefeito Passos\footnote{Francisco Pereira Passos (1836--1913),
  Engenheiro Civil, foi prefeito da cidade do Rio de Janeiro entre os
  anos de 1902 e 1906. Sob sua administração, a cidade passou por uma de
  suas maiores reformas. Centenas de casarões antigos, cortiços,
  pequenos prédios foram derrubados; grandes avenidas foram abertas e
  palácios luxuosos foram construídos. Tempo de modernização da cidade,
  que tinha no \emph{slogan} ``O Rio civiliza"-se'' sua grande
  propaganda. Tal período ficou conhecido como ``bota"-abaixo''. Lima
  Barreto foi um feroz crítico dessas reformas, principalmente pela
  falta de cuidados com as pessoas mais pobres e sobretudos a população
  descendente dos ex"-escravos, obrigada a abandonar as áreas centrais da
  cidade em direção aos subúrbios e principalmente aos morros e
  encostas. Foi justamente nessa época que o processo de favelização do
  Rio de Janeiro começou.}, que tanto se interessou pelo embelezamento
da cidade, descurou completamente de solucionar esse defeito do nosso
Rio. Cidade cercada de montanhas e entre montanhas, que recebe
violentamente grandes precipitações atmosféricas, o seu principal
defeito a vencer era esse acidente das inundações. Infelizmente, porém,
nos preocupamos muito com os aspectos externos, com as fachadas, e não
com o que há de essencial nos problemas da nossa vida urbana, econômica,
financeira e social.



\chapter[Coisas do jogo do «Bicho»]{Coisas do jogo do «Bicho»\footnote[*]{Publicado no periódico \emph{Livros Novos} em abril de 1919.}}

Há alguns anos, mantendo estreitas relações com o proprietário de uma
tipografia, na Rua da Alfândega, tinha ocasião de passar por ela toda a
tarde, demorar"-me, a fazer isto ou aquilo, nas mais das vezes conversar
unicamente.

Aos poucos fui tomando conhecimento com o pessoal; e, em breve, era de
todos camarada. A tipografia do meu amigo tinha a especialidade de
imprimir jornais de ``bicho'' e ele mesmo editava um --- \emph{O Talismã}
--- que desapareceu.

Era tão rendosa essa parte de sua indústria tipográfica que ele
destacava um único tipógrafo para executá"-la. O encarregado dessa obra,
além de compor os jornais, redigia"-os também, com o cuidado
indispensável em tais jornais"-oráculos de por, sob este ou aquele
disfarce de seções, de chapinhas, de palpite deste ou daquela, todos
vinte e cinco animais da rifa do Barão.

Conversando mais detalhadamente com o tipógrafo dos jornalecos de bicho,
ele me deu inúmeras informações sobre os seus periódicos ``zoológicos''.
O Bicho, o mais famoso e conhecido, dava de lucro, na média, 50 mil"-réis
por dia, quase o subsídio diário de um deputado naquele tempo; \emph{A
Mascote} e \emph{O Talismã}, se não forneciam um lucro tão avultado,
rendiam mais, por mês, que os vencimentos de um chefe de seção de
secretaria, naqueles anos, a regular por aí assim, em setecentos e
poucos mil"-réis.

Solicitado pelas informações do jornalista ``animaleiro'', pus"-me a
observar, nas vendas da minha vizinhança que, pela manhã, o tipo de
compras era este: um tostão de café, um ou dois de açúcar e um Bicho ou
Mascote.

O tipógrafo tinha razão e ele mesmo se encarregava de fortificar a minha
convicção do império excepcional que o ``Jogo do Jardim'' exercia sobre
a população carioca.

Mostrou"-me pacotes de cartas de toda a sorte de gente, o que se via pela
redação, de senhoras de todas as condições, de homens em todas as
posições.

Li algumas. Todas ressumavam esperança na sua clarividência
transcendente para dizer o bicho, a dezena e a centena que dariam à
tarde deste ou qualquer dia; algumas eram agradecidas e se alongavam em
palavras efusivas, em oferecimentos, por terem os missivistas acertado
com o auxílio dos ``palpites'' do Doutor Bico"-Doce.

Lembro"-me de uma assinada por certa adjunta de um colégio municipal, no
Engenho de Dentro, que convidava o pobre tipógrafo, já meio tuberculoso,
a ir almoçar ou jantar com ela e a família. Recordo"-me ainda do nome da
moça, mas não o ponho aqui, por motivos fáceis de adivinhar.

O prestígio da letra de forma, do jornal, e o mistério de que se cercava
o ``palpitador'', operavam sobre as imaginações de uma forma
verdadeiramente inacreditável. Julgavam"-no capaz de adivinhar de fato o
número a ser premiado na ``Loteria'', ou, no mínimo, apalavrado com os
homens dela, podendo, portanto, saber de antemão os algarismos da
felicidade.

Apesar da relutância do redator"-tipógrafo de tão curiosos exemplares da
nossa imprensa cotidiana, eu pude conseguir algumas cartas, das quais
uma, por me parecer a mais típica e mostrar de que maneira a situação
desesperada de um pobre homem, pode reforçar a fé no ``Jogo do Bicho'',
como salvação, e a ingênua crença de que o redator do jornaleco de
palpites seria capaz de indicar o número a ser premiado, eu a transcrevo
aqui, tal e qual, só omitindo a assinatura e o número da residência do
signatário.

É um documento humano de impressionar e de comover, por todos os
aspectos. Ei"-lo:

(Clichê de um envelope selado e subscritado: ``Ilmo. Dig.mo. Snr.
Bico"-Doce Muito Dig.mo Redator"-Chefe do Jornal O ``Talismã'' Rua da
Alfândega n.0 182 Sobrado'').

\begin{quote}
``Ilmo. Sr. Dr. Bico"-Doce. --- Rio de Janeiro, 20-12-911. --- Em primeiro
que tudo muito estimarei que esta inesperada Carta vos encontrem Com
perfeita Saúde Conjuntamente a toda vossa família, e possa fruir os mais
esplendorosos gozos.

Enquanto eu minha família, vamos passando uma vida penosa. Senhor, vós
que Sois um bondoso, vós que Sois tão caridoso, vós que Deus dotou Com
tanta doçura, e que Sois uma alma bem formada!\ldots{} tenha Compaixão
deste pobre Sofredor que a 2 anos está desempregado, e que neste longo
período, posso vos dizer que tenho passado os dias bem acerbos, e estou
tão oberado, Com o Quitandeiro, Padeiro, Peixeiro, etc., etc e só neste
ou deve 200\$; o meu senhorio já está Com a cara ruvinhaso comigo, peço
dinheiro emprestado, e Compro todos os dias: ``Mascote'', ``Bicho'' e o
``Talismã'' e nunca Sou capaz de acertar em um Bicho ou numa Dezena que
me liberte deste jugo que tanto tem me morteficado o espírito e já me
acho desanimado da Sorte que me tem Sido tão tirana.

Pois bem, em nome de Deus vos peço Dê"-me uma Dezena ou Centena num
destes dias em que a Natureza lhe der inspiração, porque aos Espíritos
bem formados ele proteje a fim de poderem espalhar com aqueles menos
favorecidos as Sorte, pode Ser que Se vós Condoer"-se das minhas misérias
em breve me libertarei desta vergonha que estou passando, pois um pobre
que deve 1:OOO\$600, e sem poder pagar, é muito triste vergonhoso.

E Se vós me libertar deste jugo pode Contar que eu saberei reconhecer o
meu bemfeitor, terá vós um Criado para qualquer Serviço que estiver em
minhas fracas força e me apresentarei diante da vossa nobre pessoa, e
poder utilizar"-se em qualquer mister que vos aprouver.

Deus que lhe queira ajudar, Deus lhe dê Saúde e felicidade pra Si e toda
vossa família, e lhe dê boa inspiração e força para minorar as aflições
dos pobres. --- F\ldots{}, Vosso humilde Criado e Obrigado, Rua Senador
Pompeu\ldots{}

Aqui aguardo a vossa proteção. --- O vosso Abedê\ldots{}''\footnote{Reparem
  que Lima Barreto mantém os supostos erros gramaticais e de escrita
  (\emph{morteficado}, \emph{proteje}) e a linguagem não muito
  articulada da pessoa que teria escrito a carta. Este detalhe reforça a
  ideia segundo a qual o jogo"-do"-bicho, desde seus primórdios, tem como
  clientela especial, mas não a única, a gente pobre e necessitada.}
\end{quote}

Não era só em cartas que se revelava total e poderosa fé de pessoas de
todas as condições nos poderes divinatórios do Dr. Bico"-Doce,
redator"-tipógrafo de \emph{O Talismã}. Nas visitas, também. Ele as
recebia a toda a hora do dia e de pessoas de todos os sexos e idade.

Havia uma senhora de Paquetá, bem trajada, joias, plumas, etc., que não
vinha ao Rio que não fosse ao Dr. Bico"-Doce para obter de viva voz, um
palpite, na centena. Se ganhava, era certo, além do fervoroso
agradecimento, uma gratificação qualquer.

A mais curiosa e assustadora visita que ele recebeu, foi a de um
capoeira da Saúde, um valentão, de chapéu de abas largas, calças
bombachas, e navalha a adivinhar"-se nas algibeiras ou em qualquer dobra
das roupas. O valente falou ao Dr. Bico"-Doce meio amigável e meio
ameaçador. É fácil de supor a atrapalhação do ``profeta"-bicheiro''. Para
sair"-se da entalação, indicou uma centena qualquer e safou"-se logo,
temendo não acertar e levar uns pescoções.

O bicho deu e a centena também. O destemido não teve o prazer de dar"-lhe
a propina em mão, mas a deixou com um colega do Dr. Bico"-Doce que a
entregou no dia seguinte.

--- Felizmente, dizia"-me o pobre jornalista do \emph{Talismã}, o homem
não quis voltar mais.

Plutarco\footnote{Historiados, filósofo e principalmente biógrafo,
  Plutarco nasceu em Queroneia, uma cidade localizada na Grécia, por
  volta do ano 45 de nossa era. Sua obra principal chama"-se ``Vidas
  Paralelas'', onde o escritor traçou diversas biografias de grandes e
  importantes nomes para a história. As biografias, normalmente, são
  estruturadas em pares; primeiro um biografado grego, depois um romano,
  além de uma terceira parte comparando os dois.}, ou outro qualquer,
conta que Alexandre\footnote{Alexandre \textsc{iii}, Alexandre Magno ou
  Alexandre, o Grande, imperador da Macedônia, nasceu por volta de 354
  a.C. Foi aluno do filósofo grego Aristóteles. Destacou"-se na história
  antiga por ser um dos maiores, senão o maior, dos conquistadores
  militares. Fundou várias cidades e territórios por onde difundia a
  cultura grega. Além de ter preparado o caminho para as futuras
  relações entre as civilizações grega e romana. Foi um dos biografados
  de Plutarco. A \emph{Vida de Alexandre} é uma das mais importantes
  biografias do livro ``Vidas Paralelas''. O outro biografado foi o
  romano Júlio César (100--44 a.C.), primeiro Imperador de
  Roma.}, nas vésperas de morrer, distribuiu o seu império entre os seus
generais. Um deles lhe perguntou: Que te fica, general? O macedônio logo
respondeu: A esperança.

Ai de nós se não fosse assim, mesmo quando a Esperança é representada
pelo jogo do bicho e o palpite de um humilde tipógrafo como o Dr.
Bico"-Doce, que, normalmente, mal ganha para a sua vida! A
Esperança\ldots{} O povo afirma que quem espera sempre alcança. Será
verdade? Parece que aí a voz do povo não é a voz de Deus\ldots{}



\chapter[Feiras e «mafuás»]{Feiras e «mafuás»\footnote[*]{Publicado no jornal \emph{Gazeta de Notícias} em 28/07/1921.}}

Ainda é do tempo da minha meninice, as barraquinhas que se armavam no
Campo de Santana\footnote{Localizado no centro da cidade do Rio de
  Janeiro, hoje é constituído de um parque e uma praça --- Praça da
  República. É considerado um dos lugares históricos mais importantes
  daquela cidade, referência durante o período Colonial e o Império. Sua
  importância histórica também se deve ao movimento republicano e à
  Proclamação da República, que acontecera naquele local.}, no largo em
frente ao Quartel General, aí pelo mês de junho, por ocasião das festas
tradicionais deste mês. Eram as barraquinhas de Santo Antônio ou de
Sant'Ana, não me lembro ao certo o nome popular que tinham; mas sei bem
que os poderes públicos do tempo toleravam essa espécie de feira,
alicerçada em toscas roletas, porque os empresários pretextavam que a
renda dela era destinada a acabar as obras da matriz de Sant'Ana, na
antiga Rua das Flores.

Os virtuosos jornais da época sempre implicaram com tal coisa. Chamavam
e apostrofavam contra a desenfreada jogatina que havia naquelas
barraquinhas.

De fato, as prendas eram tiradas à sorte, que corria numa espécie de
roleta, pinguelim, ou que outro nome tenha, com um certo número de
bilhetes de baixo custo. Em geral, eram aves de galinheiro: perus,
galinhas, patos, marrecos, etc.; mas havia outras sortes: leitões,
carneiros, cabras, rendas, potes, fitas, etc.

Fui lá várias vezes, em menino; e a lembrança que dessa curiosa feira
tenho, é muito esbatida, diluída. Lembro"-me bem dos bichos, e das
barracas de tábuas, metim, sarrafos, iluminadas por toscos e fumarentos
lampiões de querosene, que bem se pareciam com aqueles elementares a que
as cozinheiras chamam ``vagabundos''.

Veio a República, e logo as novas autoridades acabaram com aquela
folgança de mês. A República chegou austera e ríspida. Ela vinha armada
com a Política Positiva, de Comte\footnote{\textls[-10]{Referência ao livro
  \emph{Politique Positive}, do filósofo francês Auguste Comte (1798--1857), um dos mais influentes pensadores do século \textsc{xix}, com sua
  filosofia denominada \emph{Positivismo}. Ver mais a respeito na nota
  141.}}, e com os seus complementos: Um sabre e uma carabina. Esta, ela
deixou no descanso; mas o espadagão, o sabre, ela pôs no seu escudo.
Quem quiser que o veja; eu --- posso lhes garantir --- que já estou
cansado de vê"-lo nos tímpanos dos portões dos edifícios públicos, ou,
então, funcionando aqui, ali ou acolá, por estes Brasis em fora. Quando
opera assim, a carabina o segue.

Não se pode dizer muito mal do positivismo. Ele trouxe vantagens à nossa
cultura e às nossas instituições políticas.

Quanto à cultura, o comtismo republicano, com todos os seus exageros
dogmáticos, mostrou bem que toda aquela que não se baseava no estudo da
ciência, tendo por princípio a matemática, era inane e não valia nada.

De resto, apesar das orgulhosas pretensões a cavalo de batalha, como que
ele leva, nas suas últimas consequências, a um perfeito ceticismo
científico.

De modo que, hoje, qualquer um de nós, sem cultura especial alguma, mas
que ouviu o senhor Teixeira Mendes e leu"-lhe as brochuras de capa verde,
está logo disposto a sorrir, quando vê doutores enxadristas ou
arcandristas empertigarem"-se, encherem as bochechas e dizer com toda a
suficiência, a qualquer propósito: ``porque a ciência''; a ``ciência
afirma'' e outras frases que tais.

Em nome da religião têm"-se praticado muito crimes; em nome da arte
têm"-se justificado muitas sem"-vergonhices; mas, atualmente, é a ciência
que justifica crimes e também os assaltos aos minguados orçamentos do
país.

Mas, como dizia, a República severamente acabou com aquela folia de
barraquinhas, no campo de Santana, por ocasião do mês de São João. Se
não me engano elas foram se asilar no adro da igreja de Santana, na rua
das Flores.

Também aí, creio que por causa dos rolos e conflitos, a polícia acabou
com elas. A matriz ainda está por finalizar, e isto há talvez cinquenta
anos.

É pena, porque, segundo parece, o seu projeto é do operoso Manuel de
Araújo Porto Alegre\footnote{Manuel de Araújo Porto Alegre (1806--1879), nascido no Rio Grande do Sul, foi um importante pintor,
  arquiteto, historiador de arte, professor e escritor. Responsável pela
  execução de diversas obras arquitetônicas no tempo do Império.
  Recebeu, em 1874, do imperador D.\,Pedro \textsc{ii}, o título de Barão de Santo
  Ângelo.}, Barão de Santo Ângelo.

Muito pouca gente, atualmente, se há de lembrar das antigas
``barraquinhas'' do campo. Eu mesmo já me havia esquecido delas, quando,
há pouco, me vieram à lembrança, por causa de coisas congêneres, que,
presentemente, há pelos subúrbios. Não sei se há mais; conheço, porém,
duas: uma no Méier, em benefício das obras da igreja do Sagrado Coração
de Maria; e outra no Engenho de Dentro, em benefício da construção da
respectiva matriz.

O povo chama tais coisas de ``mafuás''. Não atino qual seja a origem
desse termo. Há quem diga que é corruptela do francês; \emph{ma foi} ---
``minha fé''. Não sei se é isso; mas a etimologia não vem ao caso. Seja
como for, ``mafuá'' é coisa pitoresca. Funciona aos domingos e é a
festa, o passeio domingueiro, por excelência, do povo dos subúrbios.
Toda aquela humilde gente que lá se acantona da melhor maneira possível,
fustigada pelo látego da vida, toda a semana, encontra no domingo de
``mafuá'' um derivativo da alegria e consolação para as suas mágoas,
necessidades e tormentos morais.

Nas tardes em que eles funcionam, os bondes mastodônticos da Light
chegam nas proximidades deles, apinhados de passageiros de outros
subúrbios , onde não os há; e despejam uma multidão, que se vai colear
por entre as barracas, sob a luz firme dos focos elétricos, ao compasso
de uma charanga rouca e estridente, a espaços, olhando avidamente para
aqueles objetos tentadores das barracas piedosas, na sua primeira
tentação.

\textls[-10]{O branco domina o vestuário das mulheres; e o cinzento, os ternos dos
homens. Nas barracas há de tudo. Há leitões, há carneiros, há galinhas,
há cabritos, há chapéus, há bengalas; mas a barraca mais procurada é
aquela em que se extraem por sorte frascos e perfumes. Nela, a
competente ``roda'' gira o dobro de vezes mais do que as outras em que
se vendem coisas mais úteis e proveitosas. Há de haver quem censure.}

Convém, porém, recordar que não é só de pão que vive o homem, e o
\emph{Cântico dos Cânticos}, do
grande Salomão\footnote{O \emph{Cântico dos Cânticos} é um dos livros
  que constam tanto na ``Bíblia Hebraica'' (Tanakh) e no ``Antigo
  Testamento'' cristão. É o único dos textos bíblicos que trazem o tema
  do amor erótico. Algumas fontes dizem ter sido escrito pelo rei
  Salomão, filho de Davi e Bate"-Sabá, e que teria sido o terceiro a
  governar Israel.}, rescende mirra, a aloés, a sândalo, a nardo\ldots{} Lá
está a \emph{Bíblia} sagrada: ``Quem é esta, que sobe pelo deserto, como
uma varinha de fumo composta de aromas de mirra, e de incenso, e de toda
a casta de polvilhos odoríferos?'' E, por todo aquele poema bíblico,
perpassa um sopro apaixonado de perfumes capitosos\ldots{}

Em outra parte do mesmo venerável livro, no ``Reis'', narrando os
presentes que a rainha de Sabá ofereceu ao magnifico Rei Salomão, a
santa obra frisa muito bem esse capítulo dos perfumes: ``Deu, pois, ao
rei cento e vinte talentos d'ouro, e infinitos aromas, e pedra
preciosas: desde então não trouxeram a Jerusalém tantos aromas, como os
que a rainha de Sabá deu ao Rei Salomão''.

Se o sábio Salomão e a curiosíssima rainha de Sabá eram assim tão amigos
de inebriantes aromas, porque os modestos suburbanos, de ambos os sexos
e de todas as cores, não o podem ser também a seu modo e conforme as
suas fortunas? ``Homo sum\ldots{}''\footnote{Lima Barreto apenas inicia a
  famosa frase do filósofo romano Terêncio (185--159 a.C.):
  \emph{homo sum et nihil humani a me alienum}. Em tradução para o
  português: ``sou humano e nada do que é humano me é estranho''.
  Trata"-se de um princípio de solidariedade e um alerta contra a
  indiferença entre uma pessoa e outra.}

Em compensação, essas barracas, não a da rainha de Sabá, nem a do Rei
Salomão, mas as de perfumes dos ``mafuás'' suburbanos, são as que pagam
mais caro arrendamento às irmandades respectivas.

\textls[-15]{Essa suburbana folgança domingueira acaba cedo, às dez horas da noite;
e, então, é de ver"-se o desfile daquela gente, a maioria cheia de
decepções, mas uma boa parte carregando despretensiosamente patos,
perus, galinhas e leitões que grunhem, enquanto galinhas e galos, mais
adiante, cacarejam.}

Nos bondes e nos trens, quase sempre, há questões com os condutores,
quando estes descobrem ``mafuenses'', carregando de contrabando um pato
ou uma galinha. Há o que eles chamam o ``lelê'': ``Para o bonde! Salta!
Não salta! Toca esta joça!''. Afinal, o contrabandista apeia, sobraçando
o animal em penas, pois o jornal rompera"-se e é difícil encontrar outro,
naquelas últimas. Apesar de acabar cedo, tem acontecido muitas vezes que
certos felizardos, aquinhoados com sortes de galinhas e perus,
demoram"-se pelo caminho: e, ao tomarem rumo de casa, em ruas escuras e
desertas, são presos por uma patrulha excepcional, que os toma como
ladrões de galinheiros familiares, e levam"-nos para a delegacia.

A ideia eclesiástica dos ``mafuás'' não deixou de ter a sua repercussão
nas concepções sociais dos nossos atuais dirigentes. O parentesco do
``mafuá'' com a ``feiras livre'' da superintendência é evidente, é
axiomático.

Há, entretanto, pequenas diferenças; o ``mafuá'' é à noite e a ``feira
livre'' é pela manhã; nesta não há ``rodas'', não há ``pinguelim'',
naquela há. E é só; no mais, ambos têm um constante ar de família. É a
mesma coisa, com idêntico aspecto da festa popular, pretexto para
passeios e discretos namoros; e, tanto num, como na outra, vende"-se de
tudo, como diz o povo. É mais outra semelhança. Não afianço; mas espero
ver todos dois, daqui em breve, transformados em ``carnaval'', com
bisnagas, confetes, serpentinas, ranchos e blocos.

O princípio da feira livre é totalmente socialista. Ela visa eliminar o
intermediário entre o consumidor e o produtor, mas, para os seus sábios
iniciadores, o intermediário só é o varejista, escorchado de impostos,
coisas que não acontece com o da feira, que quase não paga nada de
emolumentos ao governo.

Há dias, fui a uma delas, na minha vizinhança; e surpreendi"-me, vendo aí
vender"-se açúcar. Cheguei e fiz menção de comprar. Pedi um quilo. O
homem passou"-me um pacote. Perguntei então:

--- O senhor não tem balança?

--- Para que? Isto aqui não é feira livre? É livre!

Concordei e, ao pagar o pacote, indaguei do industrial:

--- De onde é este açúcar?

--- Não sei\ldots{} Isto é: é de Pernambuco ou de Campos. Por quê?

--- Pensei que fosse da sua usina, aqui, nos arredores.

Afastei"-me, deixando o homem espantado, com o espanto de quem acaba de
falar com um maluco.

Mais adiante, parei, em face de um mercador que oferecia uma singular
mercadoria. Eram sapatinhos de criança, toalhas de crochê, toucas,
rendas de bilros, etc. Entre estas últimas havia algumas lindas, bem
acabadas, de um desenho feliz, que bem podia rivalizar com aquelas que o
Rei Alberto trouxe nos porões do ``São Paulo''\footnote{``São Paulo''
  era o nome de um dos mais importantes navios de guerra da Marinha
  Brasileira na época.} e são fabricadas em Bruxelas.\footnote{Em
  setembro de 1920, o Brasil recebeu a visita da realeza belga
  representada pela Rainha Elizabeth, o Rei Alberto I e o filho
  Leopoldo. Era a primeira vez que acontecia a visita de uma família
  real europeia ao Brasil republicano. O evento tomou conta do
  noticiário e foi aclamado pela população onde quer que as autoridades
  aparecessem.}

Conquanto o mercador tivesse uma catadura de domador de feras, eu me
animei a dirigir"-lhe a palavra, nestes termos:

É o senhor mesmo quem faz essas lindas coisas de moças prendadas?

--- Que pergunta! Não; é minha mulher!

--- Ah!

Encaminhei"-me até um sujeito que vendia pintassilgo, canários e
papa"-capins, lembrando"-me Hércules fiando aos pés de Ônfale\footnote{Hércules
  é o nome que o herói da mitologia grega, Héracles, recebeu na
  mitologia romana. De acordo com uma das versões do mito, Hércules
  havia sido castigado por ter matado um homem, e seu castigo foi o de
  ter sido vendido como escravo para a rainha Ônfale. Como escravo,
  Hércules era submetido a diversos trabalhos e em muitas ocasiões
  trocava de papel social com Ônfale. Ela se vestia com roupagens de
  pele de leão e ele com roupas femininas; ela simulava caça e ele fiava
  (costurava) aos pés de sua ama.}.

A feira estava no seu auge. Dos bondes desciam moças e senhoras aos
magotes. Todas bem vestidas e agasalhadas convenientemente. Os
automóveis chegavam buzinando. Vi descer dele gente que não era
positivamente suburbana. Tinham vindo, certamente, do Leme ou Ipanema.

A modesta burguesia suburbana olhava esse pessoal que se diverte, com
susto e, ao mesmo tempo, com estranha curiosidade.

Dentre ele, salientava"-se uma guapa rapariga, morena, um tanto fanada,
mas, assim mesmo, ainda bela com as suas viciosas olheiras roxas.

O cavalheiro que a acompanhava, apesar de ter passado a noite em claro,
como era fácil de perceber, tinha ainda o colarinho imaculado e
reluzente. Pelos dedos, pelo peito, anéis e brilhantes; a mulher também,
tendo ainda por cima brincos, colares e braceletes.

Acompanhei"-os com jeito. Dirigiram"-se para a tenda do tal domador de
feras, que vendia artefatos da feminina indústria doméstica. O domador,
vendo a dama, tornou mais firme seu mau olhar esperto. A rapariga disse
então para o cavalheiro que ia ao lado:

--- Jaime, compra aqueles sapatinhos cor"-de"-rosa e aquela touca branca,
com rendas.

--- Para que, Benvinda? --- perguntou ele, surpreendido.

--- Para pôr na Zezé.

--- Que Zezé? --- expectorou o rapaz.

--- Tolo! A boneca, a minha boneca, que está lá no quarto. Não a tens
visto?

--- Bem, --- rematou o tal Jaime; e comprou as prendas domésticas do
domador de leões, tigres e panteras.

Está aí uma coisa que não é lá muito suburbana: comprar toucas e
sapatinhos para pôr em bonecas. Se fosse para bebês de carne e osso\ldots{}

\chapter[No «mafuá» dos padres]{No «mafuá» dos padres\footnote[*]{Publicado na revista \emph{Careta} em 11/10/1919.}}

O meu amigo doutor João Ribeiro\footnote{João Batista Ribeiro de Andrade
  (1860--1934), amigo de Lima Barreto, jornalista, crítico literário,
  historiador. Foi um dos líderes das Comissões do Dicionário e da
  Gramática, que funcionavam na Academia Brasileira de Letras. Lima
  Barreto faz menção à vasta erudição do amigo nesta crônica.} ainda não
me pôde explicar o que quer dizer ``mafuá''.

Apesar disso, eu, pela boca do povo, sei que, mais ou menos, tal termo
exprime uma barafunda de homens e mulheres de todas as condições.

Não quero contribuir para o dicionário de brasileirismos da Academia;
mas o que aprendo ensino.

Ouvi esse termo de ``mafuá'' no Engenho de Dentro, para designar umas
barraquinhas que os padres tinham lá feitas.

Era, como lá diziam, o ``mafuá'' dos padres. Eles fazem um leilão de
prendas, por intermédio de moças mais ou menos decotadas.

Aprecio o aspecto e, domingo último, com o meu amigo Modestino
Kanto\footnote{Modestino Kanto (1889--1967) foi um dos principais
  escultores brasileiros da primeira metade do século \textsc{xx}.}, fomos até
lá.

Havia, como já disse, um leilão de prendas e numa das barracas estava em
leilão um carneiro.

Surgiram logo como disputantes o Oscar Saião, soldado de linha de tiro,
e o João do Norte, alferes de polícia. Ambos queriam dar o carneiro à
crioula Candinha, que não se importava com nenhum deles, mas tinha um
grande interesse pelo carneiro.

Saião grita:

--- Dou mil e quinhentos.

Logo João do Norte berra:

--- Ofereço mil e setecentos.

A disputa ia assim, quando aparece o Raul Soares, reservista naval, e dá
um lance maior:

--- Levo por dois mil"-réis.

Todos ficaram atônitos inclusive a Candinha, que logo se embeiçou pelo
reservista, vendo a sua liberalidade.

Entretanto, estava ela enganada, porquanto, dentro em pouco, chegava o
voluntário de manobras Kalogheras e cobria o preço do reservista naval
Raul Soares:

--- O carneiro me fica por dois mil e quinhentos.

Foi o diabo por aí, porquanto Candinha ficou logo apaixonada pelo
voluntário de manobras.

O fato é que se engalfinharam e foi preciso a presença do almirante
Epitácio, sendo os feridos pensados anteriormente pelo doutor Ângelo
Tavares, clínico muito conhecido no Méier.

O delegado não prendeu Candinha; mas examinou o carneiro.

Era falso e só dizia --- ``mé'' --- porque tinha uma gaita no bucho.


\chapter[Paulino e o «mafuá»]{Paulino e o «mafuá»\footnote[*]{Publicado na revista \emph{Careta} em 11/03/1922.}}

--- Você não sabe, Segadas, este negócio de ``mafuá'' é um flagelo.

--- Como? É uma coisa religiosa, abençoada pela Igreja\ldots{}

--- Eu conto a você: Moro em Todos os Santos, há muitos anos.
Naturalmente todos nós, os da minha família, terão conhecimentos e
relações. Damo"-nos e trocamos favores. Até aí está muito direito, porque
isso é cristão e humano. Há meses minha irmã recolheu à nossa casa um
pequeno muito pobre e humilde. Embora seja eu nominalmente chefe da
casa, não fui consultado nem cheirado; mas concordei que era obra de
caridade e uma obra de caridade não se censura.

O pequeno veio tímido; mas, com os dias e graças à bondade com que era
tratado, ganhou confiança e ficou sendo o ai"-jesus da casa. Níquel daqui
e níquel dacolá, ele ajuntava para ir ao cinema no Méier. Não havia mal
nisto e nós gostávamos até do prurido de economia que ele denunciava.

Eu cá dizia comigo: ``É assim que se começa; ele economiza para o cinema
e mais tarde economizará para os filhos. A economia é a base da
prosperidade, sentença que está nos vinténs e, por estar em tão reles
moeda, ninguém a cumpre''.

Neste último carnaval, porém, tive prova que essa história de economia é
uma parvoíce. Contra a economia estão armadas uma porção de alçapões que
a simplicidade do povo não vê e cai neles com uma facilidade assombrosa.

Um deles é o ``mafuá''.

Paulino, sem consentimento de minha irmã, sai de casa para o ``mafuá''
do Engenho de Dentro. E preciso notar que ele é uma criança de nove
anos.

O ``mafuá'' estava em sessão permanente. Funcionava dia e noite; tinha
até um salão de baile, cuja entrada custava simplesmente mil"-réis. O
povo o chamara, ao salão, ``parque das cabras'' --- não sei por quê.

Pois Paulino lá foi e jogou os dois mil e tantos que tinha, no
``jaburu'' do tal ``mafuá''.

--- Daí? --- perguntei.

--- Daí é que o jogo não deve ser permitido a menores, mesmo que se trate
de edificação de igrejas.



\chapter[Coisas de «mafuá»]{Coisas de «mafuá»\footnote[*]{Publicado na revista \emph{Careta} em 22/01/1921.}}

--- Mas, onde esteve você, Jaime?

--- Onde estive?

--- Sim; onde você esteve?

--- Estive no ``xadrez''.

--- Como?

--- Por causa de você.

--- Por minha causa? Explique"-se, vá!

--- Desde que você se meteu como barraqueiro do imponente Bento,
consultor técnico do ``mafuá'' do padre A, que o azar me persegue.

--- Então eu havia de deixar de ganhar uns ``cobres''?

--- Não sei; a verdade, porém, é que essas relações entre você, Bento e
``mafuá'' trouxeram"-me urucubaca. Não se lembra você da questão do pau?

--- Isto foi há tanto tempo!\ldots{} Demais o Capitão Bento nada tinha a ver
com o caso. Ele só pagou para derrubar a arvore; mas você\ldots{}

--- Vendi o pau, para lenha, é verdade. Uma coisa à toa de que você fez
um ``lelé'' medonho e, por causa, quase nós brigamos.

--- Mas o capitão não tinha nada com o caso.

--- À vista de todos, não; mas foi o azar dele que envenenou a questão.

--- Qual, azar! qual nada! O capitão tem os seus ``quandos'' e não há
negócios que se meta, que não lhe renda bastante.

--- Isto é para ele; mas, para os outros que se metem com ele, sempre a
roda desanda.

--- Comigo não se tem dado isso.

--- Como, não?

--- Sim. Tenho ganho ``algum'' --- como posso me queixar?

--- Grande coisa! O dinheiro que ele te dá, não serve pra nada. Mal vem,
logo vai.

--- A culpa é minha que o gasto; mas do que não é minha culpa --- fique
você sabendo --- é que você tenha sido metido no ``xadrez''.

--- Pois foi. Domingo, anteontem, não fui ao ``mafuá'' de você?

--- Meu, não! É do padre ou da irmandade.

--- De você, do padre, da irmandade, do Bento ou de quem quer que seja, o
certo é que lá fui e caí na asneira de jogar na tua barraca.

--- Homessa! Você foi até feliz!\ldots{} Tirou uma galinha! Não foi?

--- Tirei --- é verdade; mas a galinha do ``mafuá'' foi que me levou a
visitar o ``xadrez''.

--- Qual o quê!

--- Foi, Pena! Eu não tirei a ``indrômita'' à última hora?

--- Tirou; e não vi você mais.

--- Tentei passá"-la ao Bento, por três mil"-réis, como era costume; mas
ele não quis aceitar.

--- Por força! A galinha já tinha sido resgatada três ou quatro vezes,
não ficava bem\ldots{}

--- A questão, porém, não é essa. Comprei \emph{A Noite}\footnote{Um
  importante jornal da época.}\emph{,} embrulhei nela a galinha e tomei
o bonde para Madureira\footnote{Bairro do subúrbio da zona norte do Rio
  de Janeiro.}. No meio da viagem, o bicho começou a cacarejar. Tentei
acalmar o animal; ele, porém, não estava pelos autos e continuou:
``crá"-crá"-cá, cró"-cró"-có''. Os passageiros caem na gargalhada; e o
condutor me põe fora do bonde e, tenho eu que acabar a viagem a pé.

--- Até aí\ldots{}

--- Espere. O papel estava despedaçado e, também, para maior comodidade,
resolvi carregar a galinha pelos pés. Ia assim, quando me surge pela
frente a ``canoa'' dos agentes. Suspeitaram da proveniência da galinha;
não quiseram acreditar que eu a tivesse tirado do ``mafuá''. E, sem mais
aquela, fui levado para o distrito e metido no xadrez, como ladrão de
galinheiros. Iria para a ``central'', para a colônia, se não fosse ter
aparecido o caro Bernadino que me conhecia, e afiançou que eu não era
vasculhador de quintais, à alta hora da noite.

--- Mas que tem isso com o ``mafuá''?

--- Muita coisa: vocês deviam fazer a coisa clara; dar logo o dinheiro de
prêmio e não galinhas, bodes, carneiros, patos e outros bicharocos que,
carregados alta noite, fazem a polícia tome um qualquer por ladrão\ldots{}

Eis aí.


%\part{Futebol, o tal jogo de pontapés}

\chapter[Uma partida de \emph{football}]{Uma partida de \emph{football}\footnote[*]{Publicado na revista \emph{Careta} em 04/10/1919.}}

Das coisas elegantes que as elegâncias cariocas podem fornecer ao
observador imparcial, não há nenhuma tão interessante como uma partida
de \emph{football}.

É um espetáculo da maior delicadeza em que a alta e a baixa sociedade
cariocas revelam a sua cultura e educação.

Num círculo romano, com imperadores, retiários\footnote{Retiário ou
  reciário era um tipo ou categoria de gladiador da época do Império
  Romano.}, vestais e outros sacerdotes e sacerdotisas, não se poderiam
presenciar aspectos tão interessantes, coisas tão inéditas como nas
nossas arenas de jogo dos pontapés na bola.

Os gladiadores eram raramente homens de grande beleza física e muito
menos intelectual; os nossos jogadores de \emph{football}, porém, são
excelentes modelos, em que o crânio alongado e pontiagudo dá um remate
de beleza aos seus membros inferiores que muito lembram certos
ancestrais do homem.

O senhor Coelho Neto\footnote{Sobre o escritor Coelho Neto, ver nota 16.},
a quem muito admiro, já fez a apologia desses apolos\footnote{Referência
  a Apolo, um dos principais deuses da mitologia grega, cuja beleza era
  cultuada como um grande ideal a ser alcançado pelos mortais.}, com a
força de sua erudição em coisas gregas. Não há, portanto, nos nossos
hábitos, fato mais agradável do que assistir uma partida de
bolapé\footnote{Na época em que a crônica foi escrita, o futebol ainda
  era uma novidade no Brasil, por isso as tentativas de tradução direta
  do termo inglês \emph{football} por ``bolapé'', por exemplo, mas o
  normal era a grafia em inglês, com algumas variações, como
  \emph{foot"-ball}.}.

As senhoras que assistem, merecem então todo o nosso respeito. Elas se
entusiasmam de tal modo que esquecem todas as conveniências. São as
chamadas ``torcedoras'' e o que é mais apreciável nelas, é o
vocabulário.

Rico no calão, veemente e colorido, o seu fraseado só pede meças ao dos
humildes carroceiros do cais do porto. Poderia dar alguns exemplos, mas
tinha que os dar em sânscrito\footnote{Sânscrito é uma das línguas mais
  antigas a que se tem conhecimento, falada em algumas regiões da Índia.}.
Em português ou mesmo em latim, eles desafiariam a honestidade: e é, por
um, que me abstenho de toda e qualquer citação elucidativa.

O que há, porém, de mais interessante nessas festanças esportivas, é o
final. Sendo um divertimento ou passatempo, elas acabam sempre em rolo e
barulho. Por tal preço, não vale a pena a gente divertir"-se. É o que me
parece\footnote{Essa crônica é uma das muitas que Lima Barreto escreveu
  contra o futebol. O escritor não gostava desse esporte, tanto pela
  violência quanto, principalmente, pelo racismo praticado pelas equipes
  e clubes oficiais. Lima Barreto chegou a criar, em 1919, \emph{A Liga
  contra o Football}, da qual faziam parte alguns intelectuais como
  Antônio Noronha Santos e Carlos Sussekind de Mendonça; este último
  escreveu um pequeno trabalho, em 1922, intitulado ``O esporte está
  deseducando a mocidade brasileira''. É importante notar, a partir
  dessas informações, o forte teor de ironia presente nessa crônica.}.



\chapter[O Haroldo]{O Haroldo\footnote[*]{Publicado na revista \emph{Careta} em 04/12/1920.}}

Haroldo Hartings era descendente de dinamarqueses e seu pai tinha
enriquecido no comércio de fazendas\footnote{\emph{Fazenda} era o nome
  mais utilizado para panos e tecidos, principalmente os utilizados para
  confecção de roupas.}. A sua loja, com um simples armarinho de
arrabalde\footnote{Loja de \emph{armarinhos} é um tipo de comércio
  voltado para várias espécies de utensílios domésticos, papelaria etc.
  \emph{Arrabalde} significa lugar afastado do centro, localizado na
  periferia ou subúrbio.}, com o tempo, aumentara e a sua riqueza
crescera com a prosperidade da loja.

Tendo vários filhos e filhas, destinou o Haroldo a ser doutor. O pequeno
formou"-se em direito e foi uma grande festa na casa do lojista, quando
ele recebeu o grau. Houve banquete, discursos e baile. O palacete
resplandeceu de luzes e os convidados estavam vestidos com o máximo de
rigor e luxo.

Um diz ao pai:

--- O Haroldo vai fazer uma bela carreira. O seu curso foi distinto.

Um outro dizia à mãe:

--- Seu filho vai ser um grande jurista. É o que dizem os lentes dele.

Assim falavam todos, gabando o talento do rapaz.

O pai, muito contente com tais prognósticos, montou um escritório
soberbo para o doutor Haroldo, numa das ruas centrais da cidade.

Os jornais, quase todos, tiveram a renda aumentada com as publicações de
anúncios do novel advogado.

Os clientes, porém, não apareciam e um amigo a quem ele se prestou a
advogar"-lhe uma causa, conseguiu perdê"-la redondamente.

Foi uma decepção para o velho mercador. Matutou, matutou muito e disse
ao filho:

--- Você não dá para advogado; o melhor é você ir para os Estados Unidos
estudar eletricidade.

--- Pois não, papai.

Doido por isso estava ele. Fez"-se de malas e embarcou para New York.

Lá, não estudou eletricidade, nem coisa alguma; tratou de pandegar à
vontade.

Voltou e nunca se propôs a montar a mais simples campainha elétrica.

Vendo, entretanto, que muitos dos seus antigos colegas ganhavam nome do
foro, nas letras e em outras atividades, teve desejo de ser também
notável.

Mas, em quê?

Não tinha jeito para a advocacia, como já experimentara; não tinha jeito
para o professorado, pois, após a formatura, não abrira mais livro; não
tinha jeito para as letras, embora toda a gente tenha.

Como havia de ser?

Pensou muito e lembrou que, em New York, tinha demonstrado certa
habilidade para o jogo de \emph{football}.

Fez"-se apóstolo desse jogo de pontapés e, graças à sua fortuna, em
breve, era uma celebridade nele.



\chapter[O ideal]{O ideal\footnote[*]{Publicado na revista \emph{Careta} em 25/09/1915.}}

Assim que Irene soube que a sua amiga Inês se havia casado, imaginou que
o tivesse feito com um grande poeta, uma jovem notabilidade.

Irene estava em Paris há muitos anos e raramente se correspondia com sua
amiga, de forma que não podia fazer um juízo certo de quem fosse o
marido de Inês.

Entretanto, sabia das ideias de casamento de sua antiga colega.

No colégio em que ambas cursaram, quando tratavam desse assunto
palpitante para o coração das moças --- o casamento --- era hábito de Inês
dizer à amiga:

--- Eu hei de me casar com um grande poeta.

Ao que a amiga respondia:

--- Esta gente não serve para marido; são estroinas, volúveis\ldots{}

--- Qual! Nem todos\ldots{} E mesmo que assim seja, eu quero que o meu nome
corra mundo junto ao nome do meu marido\ldots{}

Moça feita, Inês sempre se interessou por essas coisas de letras e
seguia todos os poetas que surgiam, com vagar, com ardor, e uma ingênua
admiração.

Conferência deste ou daquele não era anunciada que lá não estivesse; aos
salões da literatura elegante e decorativa, estava sempre presente.

Muitos esperavam dela uma literata e houve um ironista que a crismou
mesmo de próxima futura poetisa ou\ldots{} romancista.

Tudo isso fez ver à sua amiga Irene que ela se houvesse casado com um
jovem poeta de grande talento.

Aconteceu que o marido dessa última, com medo dos azares da
guerra\footnote{Referência à Primeira Guerra Mundial, que durou entre
  julho de 1914 há novembro de 1918.}, deixasse a sua residência em
Paris e viesse para o Rio.

Logo que as duas se avistaram, Irene imediatamente perguntou pressurosa:

--- Já vi que o teu marido é um grande poeta.

--- Não; é campeão do \emph{football}.



\chapter[Macaquitos]{Macaquitos\footnote[*]{Publicado na revista \emph{Careta} em 23/10/1920.}}

Um jornal ou semanário de Buenos Aires, quando uma \emph{équipe}
brasileira de \emph{football}, de volta do Chile, onde fora disputar um
campeonato internacional, por lá passou, pintou"-a como macacos.

A cousa passou desapercebida, devido ao atordoamento das festas do Rei
Alberto\footnote{Sobre a visita do Rei Alberto ao Brasil, ver nota 60.};
mas, se assim não fosse, estou certo de que haveria irritação em todos
os ânimos.

Precisamos nos convencer de que não há nenhum insulto em chamar"-nos de
macacos. O macaco, segundo os zoologistas, é um dos mais adiantados
exemplares da série animal; e há mesmo competências que o fazem, senão
pai, pelo menos primo do homem. Tão digno ``totem'' não nos pode causar
vergonha.

A França, isto é, os franceses são tratados de galos e eles não se
zangam com isto; ao contrário: o galo gaulês, o \emph{chantecler}, é
motivo de orgulho para eles.

Entretanto, quão longe está o galo, na escala zoológica, do macaco! Nem
mamífero é!

Quase todas as nações, segundo lendas e tradições, têm parentesco ou se
emblemam com animais. Os russos nunca se zangaram por chamá"-los de ursos
brancos; e o urso não é um animal tão inteligente e ladino como o
macaco.

Vários países, como a Prússia e a Áustria, põem nas suas bandeiras
águias; entretanto, a águia, desprezando a acepção pejorativa que tomou
entre nós, não é lá animal muito simpático.

A Inglaterra tem como insígnias animais o leopardo e o unicórnio.
Digam"-me agora os senhores: o leopardo é um animal muito digno?

A Bélgica tem leões ou leão nas suas armas; entretanto, o leão é um
animal sem préstimo e carniceiro. O macaco --- é verdade --- não tem
préstimo; mas é frugívoro, inteligente e parente próximo do homem.

Não vejo motivos para zanga, nessa história dos argentinos chamar"-nos de
macacos, tanto mais que, nas nossas histórias populares, nós
demonstramos muita simpatia por esse endiabrado animal.



%\part{Nossa Política e nossos políticos }

\chapter[«Encrencas» nacionais]{«Encrencas» nacionais\footnote[*]{Publicado na revista \emph{Careta} em 04/12/1920.}}

Este Brasil é o país das ``encrencas''. Não se conhece no mundo nação
mais ceia de atrapalhações do que esta.

Todo o ano aparece uma e elas se somam sem que qualquer seja resolvida.

Sobem presidentes, entram ministros, elegem"-se deputados e senadores,
criam"-se repartições e comissões e elas continuam de pé.

Não sei para que há tantos sábios e doutores, no Brasil, se eles não dão
solução a tais ``encrencas''!

Por exemplo: o \emph{déficit}. Temos entre nós --- não há dúvida alguma
--- financeiros de valor, como o Senhor Cincinato Braga\footnote{Sobre
  Cincinato Braga, ver nota 23.}, o Senhor Antônio Carlos\footnote{Antônio
  Carlos Ribeiro de Andrada (1870--1946), foi um dos mais importantes
  políticos mineiros no período da Primeira República. Ocupou cargos de
  Deputado Federal, Senador, Governador de Minas Gerais, Ministro da
  Fazenda. É sobre este último cargo que Lima Barreto faz alusão durante
  a crônica. Considerado grande especialista em finanças públicas,
  Antônio Carlos ocupou a pasta da Fazenda (hoje, chamada de Ministério
  da Economia), entre 1917 e 1918, com grande expectativa em relação ao
  controle do \emph{déficit} público, ou seja, à solução do problema em
  que os gastos do governo superam as receitas (os ganhos), o que de
  fato não ocorreu.}, o Senhor Nuno de Andrade\footnote{Nuno Ferreira de
  Andrade (1851--1922) fez sua carreira como médico e pesquisador. Foi
  uma das principais referências em saúde pública no final do Império.
  Com a instauração da República, além de exercer importantes cargos na
  área da saúde, especializou"-se, na Europa, em Finanças e Políticas
  Públicas, tendo atuado em alguns setores do Ministério da Fazenda,
  entre os anos 1912--1913.} e outros muitos --- como é que não acabam
com ele?

É uma pergunta muito natural que um leigo como eu, em matéria de
finanças, faz logo e os entendidos deviam responder cabalmente.

As tais secas do Norte são outra atrapalhação. O Brasil possui
celebridades em engenharia hidráulica e, apesar de há tantos anos
tratarem elas de acabá"-las e se haver gasto rios de dinheiro, até hoje
não deram cabo do flagelo que continua a dizimar milhares de pessoas de
tempos em tempos.

E o caso do Lloyd\footnote{A Companhia de Navegação \emph{Lloyd
  Brasileiro} foi um grande conglomerado de empresas de navios
  comerciais e de passageiros, que operavam tanto nacionalmente, quanto
  internacionalmente, principalmente para a exportação de produtos para
  a Europa e o transporte de imigrantes para nossas terras. A empresa,
  no início, era considerada de capital misto, ou seja, empresários e o
  setor público mantinham aquele empreendimento gigantesco. O
  \emph{Lloyd Brasileiro} foi à falência no final de 1899, tendo sido
  adquirido, em leilão, pelo Banco do Brasil, passando a ser considerada
  patrimônio nacional. A empresa, no entanto, nunca conseguiu gerar os
  lucros que seus dirigentes e responsáveis sempre previam, para
  justificar os investimentos estatais cada vez maiores, daí a crítica
  de Lima Barreto. A companhia funcionou até a década de 1990, quando
  entrou no programa de desestatização do governo de Fernando Henrique
  Cardoso.}?

Há não sei quantos anos vários alvitres são apresentados para que ele dê
renda. Mudam os administradores e ele contia a ser um sorvedouro de
dinheiros públicos.

Por que será que não se encontra um doutor que lhe dê remédio? É
incrível que não haja!

Uma outra ``encrenca'' nacional é o tal do ``café''.

De quando em quando, trata"-se de valorizá"-lo. Fazem"-se comissões,
empréstimos vultuosos e nunca ele fica valorizado de vez. É uma pedra de
Sísifo\footnote{Referência ao ``mito de Sísifo'', um dos mais famosos da
  mitologia grega. Muitas pessoas conhecem a história daquele homem que
  é obrigado a empurrar uma rocha gigante, em formato esférico, por uma
  montanha acima e, chegando lá, a enorme pedra regressa montanha abaixo
  e o homem começa a caminhada novamente, infinitamente, pois se trata
  de uma penitência eterna. Um trabalho que não tem fim é um ``trabalho
  de Sísifo''. Na mitologia grega, Sísifo era um ``mortal'', rei da
  cidade de Corinto, que tinha como diferencial dos outros mortais uma
  sagacidade enorme, capaz de enganar os próprios deuses. E foi
  justamente por ter enganado um dos mais poderosos deuses do Olimpo ---
  Júpiter --- que teve seu castigo de passar a velhice empurrando aquela
  enorme pedra para cima da montanha, eternamente.}.

O carvão nacional também faz parte das complicações nacionais. Dizem uns
que ele presta, dizem outros que ele não presta para nada. O governo tem
gasto um dinheirão com ele; entretanto uma partida do mesmo que estava
no cais do porto, quando há dias houve lá um incêndio, foi a única coisa
que não pegou fogo, conforme noticiaram os jornais.

Belo carvão!

Muitas outras ``encrencas'' nacionais há por aí que deixo de citar por
brevidade.



\chapter[País rico]{País rico\footnote[*]{Publicado na revista \emph{Careta} em 08/05/1920.}}

Não há dúvida alguma que o Brasil é um país muito rico. Nós que nele
vivemos não nos apercebemos bem disso; e até, ao contrário, o supomos
muito pobre, pois a toda hora e a todo instante, estamos vendo o governo
lamentar"-se que não faz isto ou não faz aquilo por falta de verba.

Nas ruas da cidade, nas mais centrais até, andam pequenos vadios, a
cursar a perigosa universidade da calaçaria das sarjetas, aos quais o
governo não dá destino, o os mete num asilo, num colégio profissional
qualquer, porque não tem verba, não tem dinheiro. É o Brasil rico\ldots{}

Surgem epidemias pasmosas, a matar e a enfermar milhares de pessoas, que
vêm mostrar a falta de hospitais na cidade, a má localização dos
existentes. Pede"-se à construção de outros bem situados; e o governo
responde que não pode fazer porque não tem verba, não tem dinheiro. E o
Brasil é um país rico\ldots{}

Anualmente cerca de duas mil mocinhas procuram uma escola anormal ou
anormalizada, para aprender disciplinas úteis. Todos observam o caso e
perguntam:

--- Se há tantas moças que desejam estudar, por que o governo não aumenta
o número de escolas a elas destinadas?

O governo responde:

--- Não aumento porque não tenho verba, não tenho dinheiro.

E o Brasil é um país rico, muito rico\ldots{}

As notícias que chegam das nossas guarnições fronteiriças são
desoladoras. Não há quartéis; os regimentos de cavalaria não têm
cavalos, etc., etc.

--- Mas que faz o governo, raciocina Brás Bocó, que não constrói quartéis
e não compra cavalhadas?

O doutor Xisto Beldroegas, funcionário respeitável do governo acode
logo:

--- Não há verba; o governo não tem dinheiro.

--- E o Brasil é um país rico; e tão rico é ele, que apesar de não cuidar
dessas coisas que vim enumerando, vai dar trezentos contos para alguns
latagões irem ao estrangeiro divertir"-se com os jogos de bola como se
fossem crianças de calças curtas, a brincar nos recreios dos colégios.

O Brasil é um país rico\ldots{}


\chapter[A política republicana]{A política republicana\footnote[*]{Publicado no jornal \emph{A.B.C.} \textsc{em} 19/10/1918.}}

Não gosto, nem trato de política. Não há assunto que mais me repugne do
que aquilo que se chama habitualmente política. Eu a encaro, como todo o
povo a vê, isto é, um ajuntamento de piratas mais ou menos diplomados
que exploram a desgraça e a miséria dos humildes.

Nunca quereria tratar de semelhante assunto, mas a minha obrigação de
escritor leva"-me a dizer alguma coisa a respeito, a fim de que não
pareça que há medo em dar, sobre a questão, qualquer opinião.

No Império\footnote{Período da história brasileira correspondente entre
  os anos de 1822 e 1889.}, apesar de tudo, ela tinha alguma grandeza e
beleza. As fórmulas eram mais ou menos respeitadas; os homens tinham
elevação moral e mesmo, em alguns, havia desinteresse.

Não é mentira isto, tanto assim, que muitos que passaram pelas maiores
posições morreram pobríssimos e a sua descendência só tem de fortuna o
nome que recebeu.

O que havia neles, não era a ambição de dinheiro. Era, certamente, a de
glória e de nome; e, por isso mesmo, pouco se incomodariam com os
proventos da ``indústria política''.

A República, porém, trazendo tona dos poderes públicos, a borra do
Brasil, transformou completamente os nossos costumes administrativos e
todos os ``arrivistas'' se fizeram políticos para enriquecer.

Já na Revolução Francesa\footnote{A Revolução francesa compreende o
  período de profundas mudanças sócio"-políticas, econômicas e culturais
  ocorridas entre os anos 1789 e 1799. Marcou o fim do absolutismo na
  França, a declaração dos direitos universais sob o lema liberdade,
  igualdade e fraternidade. Um grande evento histórico que repercutiu
  por toda a Europa e em boa parte do mundo.} a coisa foi a mesma.
Fouché\footnote{Joseph Fouché (1759--1820) foi um dos principais nomes
  político na França do período pós Revolução. Ficou conhecido
  principalmente pela inteligência de ocasião e pelas mudanças radicais
  em seus posicionamentos políticos, o que lhe valeu a alcunha de ``o
  perfeito traidor'' por parte de Napoleão Bonaparte.}, que era um
pobretão, sem ofício nem benefício, atravessando todas as vicissitudes
da Grande Crise, acabou morrendo milionário.

Como ele, muitos outros que não cito aqui para não ser fastidioso.

Até este ponto eu perdoo toda a espécie de revolucionários e
derrubadores de regimes; mas o que não acho razoável é que eles queiram
modelar todas as almas na forma das suas próprias.

A República no Brasil é o regime da corrupção. Todas as opiniões devem,
por esta ou aquela paga, ser estabelecidas pelos poderosos do dia.
Ninguém admite que se divirja deles e, para que não haja divergências,
há a ``verba secreta'', os reservados deste ou daquele Ministério e os
empreguinhos que os medíocres não sabem conquistar por si e com
independência.

A vida, infelizmente, deve ser uma luta; e quem não sabe lutar, não é
homem.

A gente do Brasil, entretanto, pensa que a existência nossa deve ser a
submissão aos Acácios e Pachecos, para obter ajudas de custo e
sinecuras.

Vem disto a nossa esterilidade mental, a nossa falta de originalidade
intelectual, a pobreza da nossa paisagem moral e a desgraça que se nota
no geral da nossa população.

Ninguém quer discutir; ninguém quer agitar ideias; ninguém quer dar a
emoção íntima que tem da vida e das coisas. Todos querem ``comer''.

``Comem'' os juristas, ``comem'' os filósofos, ``comem'' os médicos,
``comem'' os advogados, ``comem'' os poetas, ``comem'' os romancistas,
``comem'' os engenheiros, ``comem'' os jornalistas: o Brasil é uma vasta
``comilança''.

Esse aspecto da nossa terra para quem analisa o seu estado atual, com
toda a independência de espírito, nasceu"-lhe depois da República.

Foi o novo regime que lhe deu tão nojenta feição para os seus homens
públicos de todos os matizes.

Parecia que o Império reprimia tanta sordidez nas nossas almas.

Ele tinha a virtude da modéstia e implantou em nós essa mesma virtude;
mas, proclamada que foi a República, ali, no Campo de Santana, por três
batalhões, o Brasil perdeu a vergonha e os seus filhos ficaram capachos,
para sugar os cofres públicos, desta ou daquela forma.

Não se admite mais independência de pensamento ou de espírito. Quando
não se consegue, por dinheiro, abafa"-se.

É a política da corrupção, quando não é a do arrocho.

Viva a República!


\chapter[Os cortes]{Os cortes\footnote[*]{Publicado no jornal \emph{A. B.C.} em 14/12/1914.}}

Nos momentos em que a pátria fica a níqueis\footnote{O mesmo que ``fica
  na miséria''.}, a Câmara e o Senado, isto é, os senhores senadores e
os senhores deputados, lembram"-se logo de diminuir o número de
funcionários públicos.

Não digo que se não possa fazê"-lo; a tal respeito, não tenho opinião.

Diminuí"-los ou não, mesmo que eu entre no corte, é para mim
absolutamente indiferente.

Noto, porém, que as duas casas do congresso não se lembram, de forma
alguma, do que se passa nelas.

Toda a gente sabe que a Câmara e o Senado, têm cada qual uma secretaria,
um serviço de redação de debates, uma legião de auxiliares, de contínuos
e serventes, e que esse cardume de empregos aumenta de ano para ano.
Porque o congresso não começa cortando nas respectivas secretarias, para
dar exemplo?

Nesse ponto não se toca, não se diz nada e os empregados do executivo
são os mais culpados do déficit.

É uma verdadeira injustiça, tanto mais que os funcionários da Câmara e
do Senado têm, quase sempre, além de bons ordenados legais,
consideráveis gratificações, sob este ou aquele pretexto.

O povo diz que macaco não olha para o seu rabo; os parlamentares só
olham para os dos outros.

Não se lembram que, de quando em quando, vão criando lugares nas suas
secretarias, absolutamente desnecessários, tão"-somente para atender a
impulsos de coração.

Homo sum\ldots{}\footnote{Sobre esta frase, ver nota 58}

Certamente os senhores devem saber que, antigamente, os atuais diretores
de secretarias eram chamados oficiais"-maiores.

Pois bem: a Câmara tem na sua secretaria um diretor, um vice"-diretor ou
dois, e um oficial"-maior.

Não é fácil mostrar assim o rol de empregados em duplicata ou triplicata
que há por lá. Os regulamentos não falam claro; é preciso combiná"-los
com indicações, com autorizações camarárias e é trabalho que sempre
reputei e reputo enfadonho.

O Diário Oficial foi feito para não ser lido e o congresso não tem mais
direitos a melhores atenções.

A observação aí fica, e, enquanto ela quiser imitar qualquer das famosas
``secretarias da comissão tal'' legisladores extraconstitucionais e
sobremodo empertigados nas suas funções, penso, dizia, que os abnegados
pais da pátria devem meditar sobre o fato.

Não é só o poder executivo o grande plantador de sinecuras; o
legislativo colabora na plantação, na colheita; e, na sua própria seara
faz das suas.

Cá e lá, más fadas há; e não é a última vez que torto ri"-se do aleijado.



\chapter[O novo manifesto]{O novo manifesto\footnote[*]{Publicado no jornal \emph{Correio da Noite} em 16/01/1915.}}

Eu também sou candidato a deputado. Nada mais justo. Primeiro: eu não
pretendo fazer coisa alguma pela pátria, pela família, pela humanidade.

Um deputado que quisesse fazer qualquer coisa dessas ver"-se"-ia bambo,
pois teria, certamente, os duzentos e tantos espíritos dos seus colegas
contra ele.

Contra as suas ideias levantar"-se"-iam duas centenas de pessoas do mais
profundo bom senso.

Assim, para poder fazer alguma coisa útil, não farei coisa alguma, a
não ser receber o subsídio.

Eis aí em que vai consistir o máximo da minha ação parlamentar, caso o
preclaro eleitorado sufrague o meu nome nas urnas.

Recebendo os três contos mensais, darei mais conforto à mulher e aos
filhos, ficando mais generoso nas facadas aos amigos.

Desde que minha mulher e os meus filhos passem melhor de cama, mesa e
roupas, a humanidade ganha. Ganha, porque, sendo eles parcelas da
humanidade, a sua situação melhorando, essa melhoria reflete sobre o
todo de que fazem parte.

Concordarão os nossos leitores e prováveis eleitores, que o meu
propósito é lógico e as razões apontadas para justificar a minha
candidatura são bastante ponderosas.

De resto, acresce que nada sei da história social, política e
intelectual do país; que nada sei da sua geografia; que nada entendo de
ciências sociais e próximas, para que o nobre eleitorado veja bem que
vou dar um excelente deputado.

Há ainda um poderoso motivo, que, na minha consciência, pesa para dar
este cansado passo de vir solicitar dos meus compatriotas atenção para o
meu obscuro nome.

Ando mal vestido e tenho uma grande vocação para elegâncias.

O subsídio, meus senhores, viria dar"-me elementos para realizar essa
minha velha aspiração de emparelhar"-me com a deschanelesca\footnote{O
  mesmo que extravagante.} elegância do senhor Carlos Peixoto\footnote{Carlos
  Peixoto de Melo Filho (1871--1917), advogado e político mineiro. Foi
  deputado estadual e federal e teve bastante destaque na política
  brasileira durante a presidência de Rodrigues Alves (1902--1906).}.

Confesso também que, quando passo pela Rua do Passeio e outras do
Catete, alta noite, a minha modesta vagabundagem é atraída para certas
casas cheias de luzes, com carros e automóveis à porta, janelas com
cortinas ricas, de onde jorram gargalhadas femininas, mais ou menos
falsas.

Um tal espetáculo é por demais tentador, para a minha imaginação; e, eu
desejo ser deputado para gozar esse paraíso de Maomé sem passar pela
algidez da sepultura.

Razões tão ponderosas e justas, creio, até agora, nenhum candidato
apresentou, e espero da clarividência dos homens livres e orientados o
sufrágio do meu humilde nome, para ocupar uma cadeira de deputado, por
qualquer Estado, província, ou emirado, porque, nesse ponto, não faço
questão alguma.

Às urnas.


\chapter[Um candidato]{Um candidato\footnote[*]{Publicado na revista \emph{Careta} em 03/04/1915.}}

Um dia destes, encontrei um senhor vestido de ganga\footnote{Um tipo de
  tecido bastante resistente, normalmente azul, semelhante ao jeans.},
com um chapéu de sol, de cabo de volta, rosto tostado pelo sol, que me
perguntou qual o bonde próprio para ir ao morro da Graça\footnote{Localizado
  no bairro de Laranjeiras, próximo ao centro da cidade do Rio de
  Janeiro.}. Ouço muito falar em tal morro, mas não sei se há bonde
próprio para lá ir. Pelo que tenho ouvido dizer, creio mesmo que o
veículo mais próprio são os nossos pés e devemos subi"-lo, como os
devotos sobem os degraus da Penha\footnote{Referência aos 382 degraus
  que levam até à Igreja da Penha, localizada no bairro da Penha, na
  cidade do Rio de Janeiro. Muitos devotos ainda têm o costume de subir
  de joelhos a escadaria, normalmente para agradecer uma graça alcançada
  ou pagar alguma promessa.}: de joelhos.

É punição bastante digna quando se está diante de Deus, mas pouco
decente quando se trata de ir à presença de qualquer mortal por mais
poderoso que seja\footnote{Muito provavelmente, Lima Barreto estaria se
  referindo ao Palacete pertencente ao Senador Pinheiro Machado (1851--1915). General do Exército, gaúcho, foi uma figura das mais
  importantes, poderosas e controversas da política brasileira no início
  da Primeira República. Sua propriedade, o ``Palacete do Pinheiro
  Machado'', era local de encontro de políticos, fazendeiros poderosos,
  generais. Muitos acordos, alianças políticas e candidaturas foram
  tramadas nas reuniões do Morro da Graça. Trata"-se, como é de costume,
  de uma alusão com forte teor irônico de Lima Barreto.}.

Disse ao matuto isto e ele não se espantou.

Continuou a conversar comigo e notei que aquele candidato à vítima do
conto do vigário era um espertalhão de marca. Contou"-me que era
candidato a deputado. Perguntei:

--- Foi diplomado?

--- Fui.

--- Por que junta?

--- Pela minha.

--- É a legal?

--- A minha junta é sempre legal.

Gostei imensamente do critério de legalidade do futuro legislador e
indaguei:

--- Que pretende fazer na Câmara?

--- Eu? Não pretendo fazer nada, mesmo porque não entendo desses negócios
de ``falação'', nem de ``leizes,'' nem de ``franciú''. Olhe, menino,
quer que lhe conte as coisas tais quais são? Ouça.

Acendeu um enorme cigarro e emendou:

--- Eu tinha alguns cobres e sempre trabalhei nas ``inleição'' pelo
senador Chavantes. Trabalho há muitos anos. Ultimamente perdi muito
dinheiro na ``campista''. Apareceu lá pela terra o demônio de um turco
que ganhava a mais não poder. Quero tampar o rombo e só achei um
caminho; fazer"-me deputado por meu distrito.

--- Teve grande votação?

--- Que votação, menino! Isso é lá preciso!\ldots{} Eu tenho atas e atas,
livros e livros\ldots{} Não preciso mais nada\footnote{As fraudes eleitorais
  eram frequentes e corriqueiras, além de amplamente reconhecidas
  durante o período da Primeira República. Lima Barreto sempre denunciou
  tal prática.}.

--- Mas o seu rival talvez tivesse eleitores e\ldots{}

--- Que eleitores! Pra quê? Eleitores são as assinaturas dos mesários e
arranjei um espanhol que faz ``elas'' tão bem como cada um deles.

--- E o reconhecimento?

--- É disso que vou tratar com o general. Vou lhe dizer que, se ele
arranjar o meu reconhecimento, ponho até cabresto e barbicacho.

--- Então vai ao morro da Graça?

--- Vou.

--- A pé e subi"-lo"-á de joelhos?

--- Não. ``Inté'' o morro, vou de ``intumove'', mas para a casa do
``home'' subo de quatro. ``Inté'' logo, moço.



\chapter[O café]{O café\footnote[*]{Publicado na revista \emph{Careta} em 26/06/1915..}}

Tenho ouvido dizer que o café é a maior fortuna do Brasil\footnote{Desde
  os anos 1880 o café se transformou no principal produto de exportação
  do Brasil e o responsável por boa parte das receitas dos governos. Ao
  longo da Primeira República houve algumas ações governamentais no
  sentido de socorrer as crises que se abatiam sobre os fazendeiros e
  barões do café. Lima Barreto veio à imprensa inúmeras vezes criticar a
  chamada ``politica de valorização do café'', que consistia na compra
  de milhões de sacas de café pelo próprio governo. Muitas vezes
  toneladas de café eram queimadas, para manter o nível competitivo do
  preço do café brasileiro no mercado mundial.}; que ele, quase
unicamente, contribui para a riqueza orçamentária da nossa pátria. São
coisas que andam por aí afirmadas pelos jornais, sobretudo pelos de São
Paulo.

Pus"-me a verificar e estudar as coisas com todo o método. A história não
me pareceu assim tão inconcussa como parece.

Sempre tenho ouvido dizer que quem tem dinheiro, dá dinheiro e não pede
dinheiro.

O tal café, porém, só leva a pedir dinheiro. Como é que ele é riqueza do
Brasil?

Não se abre um jornal, governista, neutro ou oposicionista, que não se
encontre uma lamúria, uma ``facada'' da lavoura de café!

Um dia é porque os preços estão baixos; outro dia porque o câmbio
baixou; outro é porque não pode ser exportado e assim por diante.

Urge que se tenha uma explicação satisfatória para um tal estado de
coisas.

A economia política, oficial ou oficiosa, deve declarar por que razão
tal riqueza anda sempre na pobreza; por que razão São Paulo, o estado
mais rico do Brasil, que vive do café, é por isso mesmo o mais rico.

Não sou economista, nem financista, nem juristinista, mas um tal fato me
causa pasmo e assombra"-me.

Estou, portanto, no meu pleno direito, de pedir aos sábios das
escrituras explicação para esses milagres da natureza.



\chapter[O rico mendigo]{O rico mendigo\footnote[*]{Publicado na revista \emph{Careta} em 24/07/1915.}}

Não sei como vos conte a coisa. A história passou"-se em sonho, creio eu.

Sonhei uma noite destas que tinha encontrado na rua um senhor cheio de
brilhantes, cheio de roupas, bengala de castão de ouro, botinas das mais
finas, que me estendeu a mão:

--- Uma esmola, pelo amor de Deus!

Admirei"-me de tal fato, espantei"-me e lhe dei a esmola. Ia seguir o meu
caminho, quando o mendigo bem"-vestido me chamou e disse"-me:

--- Venha cá, por favor.

Voltei e ele me convidou a ir a uma confeitaria. Houve da minha parte
novo espanto. Como é que o homem me pedia uma esmola, a mim, de recursos
reduzidos, cheio de ``encrencas'' na vida, e, minutos após, convidava"-me
a beber em uma confeitaria? Fui ao bar mais próximo e ele, sem mais
delongas, explicou"-se:

--- Deve o senhor admirar"-se de que eu, bem"-vestido, com joias, com
bengala de luxo, com um Patek\footnote{Famosa marca de relógios. Na
  época, os relógios ficavam nos bolsos das calças e eram puxados por um
  cordão. Com o surgimento dos relógios de pulso, os antigos relógios de
  bolso foram saindo de uso.} no bolso, lhe tivesse pedido uma esmola.
Eu lhe explico.

Fez uma pausa, sorvemos alguns goles de cerveja e continuou:

--- Sou rico e digo isto a todo o mundo. Moro em uma grande casa, tenho
lindos e caros móveis, tenho alfaias, tenho carros, tenho numerosa
criadagem, tenho um banheiro que é uma verdadeira terma romana e custeio
tudo isto sem o menor esforço; mas peço esmolas.

--- Por quê?

--- Porque quero ganhar mais e mais. Peço até aos meus irmãos mais
pobres, mesmo àqueles que vivem com dificuldades. Quero sempre ter mais,
ganhar mais, para proclamar a todos a minha riqueza; e as esmolas me
servem para as despesas miúdas. Às vezes até, elas me proporcionam
especulações felizes.

--- Mas quem é o senhor?

--- Não sabe? Eu sou o Café.



%\part{Violência contra as mulheres}

\chapter[A lei]{A lei\footnote[*]{Publicado no jornal \emph{Correio da Noite} em 07/01/1915.}}

Este caso da parteira merece sérias reflexões que tendem a interrogar
sobre a serventia da lei.

Uma senhora, separada do marido, muito naturalmente quer conservar em
sua companhia a filha; e muito naturalmente também não quer viver
isolada e cede, por isto ou aquilo, a uma inclinação amorosa.

O caso se complica com uma gravidez e para que a lei, baseada em uma
moral que já se findou, não lhe tire a filha, procura uma conhecida, sua
amiga, a fim de provocar um aborto de forma a não se comprometer.

Vê"-se bem que na intromissão da ``curiosa'' não houve nenhuma espécie de
interesse subalterno, não foi questão de dinheiro. O que houve foi
simplesmente camaradagem, amizade, vontade de servir a uma amiga, de
livrá"-la de uma terrível situação.

Aos olhos de todos, é um ato digno, porque, mais do que o amor, a
amizade se impõe.

Acontece que a sua intervenção foi desastrosa e lá vem a lei, os
regulamentos, a polícia, os inquéritos, os peritos, a faculdade e
berram: você é uma criminosa! você quis impedir que nascesse mais um
homem para aborrecer"-se com a vida!

Berram e levam a pobre mulher para os autos, para a justiça, para a
chicana, para os depoimentos, para essa via"-sacra da justiça, que talvez
o próprio Cristo não percorresse com resignação.

A parteira, mulher humilde, temerosa das leis, que não conhecia,
amedrontada com a prisão, onde nunca esperava parar, mata"-se.

Reflitamos, agora; não é estúpida a lei que, para proteger uma vida
provável, sacrifica duas? Sim, duas porque a outra procurou a morte para
que a lei não lhe tirasse a filha. De que vale a lei?

\chapter[Não as matem]{Não as matem\footnote[*]{Publicado no jornal \emph{Correio da Noite} em 27/01/1915.}}

Esse rapaz que, em Deodoro\footnote{Bairro localizado na zona oeste da
  cidade do Rio de Janeiro.}, quis matar a ex"-noiva e suicidou"-se em
seguida, é um sintoma da revivescência de um sentimento que parecia ter
morrido no coração dos homens: o domínio, \emph{quand même}\footnote{\textls[-25]{Expressão
  francesa que quer dizer ``mesmo assim'', que pode servir tanto como
  reforço de uma afirmação, quanto como oposição. Neste caso, o uso que
  Lima Barreto faz vai no sentido de reafirmar a ideia de dominação.}},
sobre a mulher.

O caso não é único. Não há muito tempo, em dias de carnaval, um rapaz
atirou sobre a ex"-noiva, lá pelas bandas do Estácio\footnote{Bairro da
  região central do município do Rio de Janeiro.}, matando"-se em
seguida. A moça com a bala na espinha, veio morrer, dias após, entre
sofrimentos atrozes.

Um outro, também, pelo carnaval, ali pelas bandas do ex"-futuro Hotel
Monumental, que substituiu com montões de pedras o vetusto Convento da
Ajuda\footnote{O Convento da Ajuda foi uma importante instituição de
  freiras, construído entre 1745 e 1750 e foi demolido em 1911, para dar
  lugar ao Hotel Monumental, que passou por grandes dificuldades para
  ser construído.}, alvejou a sua ex"-noiva e matou"-a.

Todos esses senhores, parece, não sabem o que é a vontade dos outros.

Eles se julgam com o direito de impor o seu amor ou o seu desejo a quem
não os quer. Não sei se se julgam muito diferentes dos ladrões à mão
armada; mas o certo é que estes não nos arrebatam senão o dinheiro,
enquanto esses tais noivos assassinos querem tudo que é de mais sagrado
em outro ente, de pistola na mão.

O ladrão ainda nos deixa com vida, se lhe passamos o dinheiro; os tais
passionais, porém, nem estabelecem a alternativa: a bolsa ou a vida.
Eles, não; matam logo.

Nós já tínhamos os maridos que matavam as esposas adúlteras; agora temos
os noivos que matam as ex"-noivas.

De resto, semelhantes cidadãos são idiotas. É de supor que, quem quer
casar, deseje que a sua futura mulher venha para o tálamo conjugal com a
máxima liberdade, com a melhor boa"-vontade, sem coação de espécie
alguma, com ardor até, com ânsia e grandes desejos; como e então que se
castigam as moças que confessam não sentir mais pelos namorados amor ou
coisa equivalente?

Todas as considerações que se possam fazer, tendentes a convencer os
homens de que eles não têm sobre as mulheres domínio outro que não
aquele que venha da afeição, não devem ser desprezadas.

Esse obsoleto domínio à valentona, do homem sobre a mulher, é coisa tão
horrorosa, que enche de indignação.

O esquecimento de que elas são, como todos nós, sujeitas, a influências
várias que fazem flutuar as suas inclinações, as suas amizades, os seus
gostos, os seus amores, é coisa tão estúpida, que, só entre selvagens
deve ter existido.

Todos os experimentadores e observadores dos fatos morais têm mostrado a
inanidade de generalizar a eternidade do amor.

Pode existir, existe, mas, excepcionalmente; e exigi"-la nas leis ou a
cano de revólver, é um absurdo tão grande como querer impedir que o sol
varie a hora do seu nascimento.

Deixem as mulheres amar à vontade.

Não as matem, pelo amor de Deus!

\chapter[Lavar a honra, matando?]{Lavar a honra, matando?\footnote[*]{Publicado no jornal \emph{Lanterna} em 28/01/1918.}}

Dentre as muitas coisas engraçadas que me têm acontecido, uma delas é
ter sido jurado, e mais de uma vez. Da venerável instituição, eu tenho
notas que me animo qualificá"-las de judiciosas e um dia, desta ou
daquela maneira, hei de publicá"-las.

Antes de tudo, declaro que não tenho sobre o júri a opinião dos
jornalistas honestíssimos, nem tampouco a dos bacharéis pedantes. Sou de
opinião que ela deve ser mantida, ou por outra, voltar ao que foi. A
lei, pela sua generosidade mesmo, não pode prever tais e quais casos, os
aspectos particulares de tais e quais crimes; e só um tribunal como o
júri, sem peias de praxistas, de autoridades jurídicas, etc., pode
julgar com o critério muito racional e concreto da vida que nós vivemos
todos os dias, desprezando o rigor abstrato da lei e os preconceitos dos
juristas.

A massa dos jurados é de uma mediocridade intelectual pasmosa, mas isto
não depõe contra o júri, pois nós sabemos de que força mental são a
maioria dos nossos juízes togados.

A burrice nacional julga que deviam ser os formados a compor unicamente
o júri. Há nisso somente burrice, e às toneladas. Nas muitas vezes em
que servi no tribunal popular, tive como companheiros doutores de todos
os matizes. Com raras exceções, todos eles eram excepcionalmente idiotas
e os mais perfeitos eram os formados em direito.

Todos eles estavam no mesmo nível mental que o senhor Ramalho, oficial
da Secretaria da Viação; que o senhor Sá, escriturário da Intendência;
que o senhor Guedes, contramestre do Arsenal de Guerra. Podem objetar
que esses doutores todos exerciam cargos burocráticos. É um engano.
Havia"-os que ganhavam o seu pão dentro das habilidades fornecidas pelo
canudo e eram bem tapados.

Não há país algum em que, tirando"-se à sorte os nomes de doze homens, se
encontrem dez de inteligentes; e o Brasil que tem os seus expoentes
intelectuais no Aluísio de Castro\footnote{Aluísio de Castro (1881--1959), médico, professor e poeta. Foi eleito para a Academia
  Brasileira de Letras em 1917. Lima Barreto implicava com a suposta
  sabedoria do médico.} e no Miguel Calmon\footnote{Miguel Calmon du Pin
  e Almeida (1879--1935) foi um engenheiro e político baiano, que fez
  carreira no Rio de Janeiro, principalmente como Ministro dos
  Transportes e da Agricultura. Lima Barreto também implicava com a
  suposta sabedoria deste político.}, não pode fazer exceção à regra.

O júri, porém, não é negócio de inteligência. O que se exige de
inteligência é muito pouco, está ao alcance de qualquer um. O que se
exige lá é força de sentimento e firmeza de caráter, e isto não há lata
doutoral que dê. Essas considerações vêm ao bico da pena, ao ler que o
júri mais uma vez absolveu um marido que matou a mulher, sobre o
pretexto de ser ela adúltera.

Eu julguei um crime destes e foi das primeiras vezes que fui sorteado e
aceito. O promotor era o doutor Cesário Alvim, que já é juiz de direito.
O senhor Cesário Alvim fez uma acusação das mais veementes e perfeitas
que eu assisti no meu curso de jurado. O senhor Evaristo de Morais
defendeu, empregando o seu processo predileto de autores, cujos livros
ele leva para o tribunal, e referir"-se a documentos particulares que, da
tribuna mostra aos jurados. A mediocridade de instrução e inteligência
dos jurados fica sempre impressionada com as coisas do livro; e o doutor
Evaristo sabe bem disto e nunca deixa de recorrer ao seu predileto
processo de defesa.

Mas\ldots{} Eu julguei um uxoricida\footnote{Termo jurídico para
  designar o assassinato de uma mulher marido ou ex"-marido.}. Entrei no
júri com reiterados pedidos de sua própria mãe, que me foi procurar por
toda a parte. A minha firme opinião era condenar o tal matador conjugal.
Entretanto a mãe\ldots{} Durante a acusação, fiquei determinado a
mandá"-lo para o xilindró\ldots{} Entretanto a mãe\ldots{} A defesa do
doutor Evaristo de Morais não me abalou\ldots{} Entretanto a mãe\ldots{}
Indo para a sala secreta, tomar café, o desprezo que um certo Rodrigues,
campeão de réu, demonstrava por mim, mais alicerçou a minha convicção de
que devia condenar aquele estúpido marido\ldots{} Entretanto a
mãe\ldots{} Acabando os debates, Rodrigues queria lavrar a ata, sem
proceder a votação dos quesitos. Protestei e disse que não a assinaria
se assim procedessem. Rodrigues ficou atônito, os outros confabularam
com ele. Um veio ter a mim, indagou se eu era casado, disse"-lhe que não
e ele concluiu: ``É por isso. O senhor não sabe o que são essas coisas.
Tomem nota desta\ldots{}''. Afinal cedi\ldots{} A mãe\ldots{} Absolvi o
imbecil marido que lavou a sua honra, matando uma pobre mulher que tinha
todo o direito de não amá"-lo, se o amou, algum dia, e amar um outro
qualquer\ldots{} Eu me arrependo profundamente.



\chapter[Os matadores de mulheres]{Os matadores de mulheres\footnote[*]{Publicado no jornal \emph{Lanterna} em 18/03/1918.}}

Preocupações de outras ordens, não me têm permitido escrever sobre
coisas diárias; mas este caso de Niterói\footnote{Município vizinho à
  cidade do Rio de Janeiro, localizado do outro lado da Bahia da
  Guanabara, que pode ser acessada hoje em dia pela ponte Rio"-Niterói,
  mas que na época de Lima Barreto só era acessível por barco.}, caso do
Filadelfo Rocha, fez"-me voltar de novo à imprensa quotidiana.

Eu não me cansarei nunca de protestar e de acusar esses vagabundos
matadores de mulheres, sobretudo, como no caso presente, quando não têm
nem a coragem do seu crime.

Eu conheço este Filadelfo desde tenente. Sou funcionário da Secretaria
da Guerra há quinze anos. Ele nunca passou de um tarimbeiro vulgar,
feito oficial pelo Floriano\footnote{Referência ao presidente Floriano
  Peixoto.}. De bajulação em bajulação, foi subindo, até que, com a sua
máxima bajulação ao senhor Hermes da Fonseca\footnote{Sobre Hermes da
  Fonseca, ver nota 47.} foi levado a ser comandante da polícia de
Niterói.

Ele é quase analfabeto, sem nenhuma inteligência, nunca fez o mínimo
esforço mental; entretanto, agora, coberto pelo opróbrio de um
assassinato, insinua que o fez porque o seu rival era um simples
funileiro. Mas onde foi Filadelfo encontrar superioridade suficiente
para julgar"-se mais do que o tal bombeiro? Este Filadelfo ignorante,
bajulador, que eu via pelos corredores de Ministério da Guerra a pegar
na casaca deste ou daquele graúdo, para não comandar as suas praças, é,
por acaso, alguma coisa?

Com essa tatuagem de galões\footnote{Galões eram chamados os distintivos
  que os militares usavam nas fardas.}, eles querem fazer das suas,
matando as mulheres a torto e a direito. Eu me refiro simplesmente a
semelhantes sujeitos. E digo isso, não por covardia, mas em atenção à
verdade.

Por exemplo: este senhor Faceiro que, ontem ou anteontem, matou a
mulher, porque teve a franca, a franca franqueza orgulhosa de dizer que
a sua gravidez era do seu amor e não dele, não me merece a mínima
piedade; mas há tantos outros que eu estimo\ldots{} Adiante.

A mulher não é propriedade nossa e ela está no seu pleno direito de
dizer donde lhe vêm os filhos.

Mas a questão não é esta. Eu falava do Filadelfo, do pequenino
Filadelfo, a quem eu queria dizer simplesmente que nem ao menos ele teve
ou tem coragem do seu crime. Espécie de Mendes Tavares\footnote{Mendes
  Tavares, médico e sanitarista, ficou marcado na época pelo
  envolvimento em um assassinato, do qual foi mentor e executante. O
  caso ocorreu no dia 14 de outubro de 1911 e ficou conhecido como a
  \emph{Tragédia da Avenida}. Mendes Tavares tinha fama de sedutor de
  mulheres e não foram poucos os casos em que se envolveu em intrigas
  amorosas. Uma delas acabou no assassinato do Comandante Lopez da Cruz,
  oficial da Marinha. Mendes teve um longo caso com a esposa de Lopez,
  quando este se encontrava no Paraguai. A traição acabou sendo
  descoberta. O marido, como era costume na época, denunciou o caso à
  polícia e desafiou Mendes Tavares para um duelo com revolver. O médico
  não aceitou o desafio, mas tramou o assassinato de Lopez da Cruz. A
  emboscada ocorreu na Avenida Rio Branco, num sábado à tarde. Mendes
  Tavares estava acompanhado por mais três homens. Dispararam vários
  tiros na vítima, que morreu no local. A notícia tomou conta dos
  noticiários cariocas durante muito tempo.}!

Basta.



\chapter[Mais uma vez]{Mais uma vez\footnote[*]{Publicado no jornal \emph{A. B. C.} em 1920.}}

Este recente crime da Rua da Lapa\footnote{Rua localizada no centro da
  cidade do Rio de Janeiro.} traz de novo à tona essa questão do
adultério da mulher e seu assassinato pelo marido.

Na nossa hipócrita sociedade parece estabelecido como direito, e mesmo
dever do marido, o perpetrá"-lo.

Não se dá isto nesta ou naquela camada, mas de alto a baixo.

Eu me lembro ainda hoje que, numa tarde de vadiação, há muitos anos, fui
parar com o meu amigo, já falecido Ari Toom, no necrotério, no largo do
Moura\footnote{Localizado na região central da cidade do Rio de Janeiro.
  Recebeu esse nome em 1794 por ocasião da chegada ao Brasil de um
  regimento militar vindo da cidade de Moura, em Portugal. Em princípio
  chamava"-se largo de Moura. Neste lugar ficava a forca pública, onde
  eram executados os condenados à morte. Em 1873 foi construído o
  necrotério municipal, do qual Lima Barreto faz alusão.} por aquela
época.

Uma rapariga --- nós sabíamos isso pelos jornais --- creio que espanhola,
de nome Combra, havia sido assassinada pelo amante e, suspeitava"-se, ao
mesmo tempo \emph{marquereau}\footnote{Palavra francesa que significa
  cafetão, homem que vive da prostituição de mulheres.} dela, numa casa
da Rua de Sant'Ana.

O crime teve a repercussão que os jornais lhe deram e os arredores do
necrotério estavam povoados da população daquelas paragens e das
adjacências do Beco da Música\footnote{Localizado no centro da cidade do
  Rio. Recebeu esse nome também por conta do regimento vindo de Moura,
  pois os músicos da banda marcial lá ensaiavam e tocavam.} e da Rua da
Misericórdia\footnote{Localizado no centro da cidade, foi uma das
  primeiras ocupações construídas no Rio de Janeiro, ainda no século
  \textsc{xvi}.}, que o Rio de Janeiro bem conhece. No interior da
\emph{morgue}\footnote{Palavra francesa que significa necrotério.} era a
frequência algo diferente sem deixar de ser um pouco semelhante à do
exterior, e, talvez mesmo, em substância igual, mas muito bem vestida.
Isto quanto às mulheres --- bem entendido!

Ari ficou mais tempo a contemplar os cadáveres. Eu saí logo. Lembro"-me
só do da mulher que estava vestida com um corpete e tinha só a saia de
baixo. Não garanto que estivesse calçada com as chinelas, mas me parece
hoje que estava. Pouco sangue e um furo bem circular no lado esquerdo,
com bordas escuras, na altura do coração.

Escrevi --- cadáveres --- pois o amante"-cáften\footnote{Hoje escreve"-se
  cafetão.} se havia suicidado após matar a Combra --- o que me havia
esquecido de dizer.

Como ia contando, vim para o lado de fora e pus"-me a ouvir os
comentários daquelas pobres \emph{pierreuses}\footnote{Termo francês
  usado para se referir às prostitutas na época.} de todas as cores,
sobre o fato.

Não havia uma que tivesse compaixão da sua colega da aristocrática
classe. Todas elas tinham objurgatórias terríveis, condenando"-a,
julgando o seu assassínio coisa bem feita; e, se fossem homens, diziam,
fariam o mesmo --- tudo isto entremeado de palavras do calão obsceno
próprias para injuriar uma mulher. Admirei"-me e continuei a ouvir o que
diziam com mais atenção. Sabem por que eram assim tão severas com a
morta?

Porque a supunham casada com o matador e ser adúltera.

Documentos tão fortes como este não tenho sobre as outras camadas da
sociedade; mas, quando fui jurado e, tive por colegas os médicos da
nossa terra, funcionários e doutos de mais de três contos e seiscentos
mil"-réis de renda anual como manda a lei sejam os juízes de fato
escolhidos, verifiquei que todos pensavam da mesma forma que aquelas
maltrapilhas \emph{rôdeuses}\footnote{Termo francês que designava as
  prostitutas.} do Largo do Moura.

Mesmo eu --- já contei isto alhures --- servi num conselho de sentença que
tinha de julgar um uxoricida e o absolvi. Fui fraco, pois a minha
opinião, se não era falhe comer alguns anos de cadeia, era manifestar
que havia, e no meu caso completamente incapaz de qualquer conquista, um
homem que lhe desaprovava a barbaridade do ato. Cedi a rogos e, até,
alguns partidos dos meus colegas de sala secreta.

No caso atual, neste caso da Rua da Lapa, vê"-se bem como os defensores
do criminoso querem explorar essa estúpida opinião de nosso povo que
desculpa o uxoricídio quando há adultério, e parece até impor ao marido
ultrajado dever de matar a sua ex"-cara metade.

Que um outro qualquer advogado explorasse essa abusão bárbara da nossa
gente, vá lá; mas que o Senhor Evaristo de Morais\footnote{Antônio
  Evaristo de Moraes (1871--1939) foi um jurista e advogado carioca,
  muito conhecido na época por participar de casos emblemáticos e de
  grande repercussão social. Era um grande orador e seus discursos
  ficaram marcados em inúmeras sessões dos tribunais da época. Publicou
  uma quantidade enorme de livros, a maioria da área criminal, mas
  também de História, Psicologia Social, além de quase duas centenas de
  artigos em jornais e revistas da época. Envolveu"-se também na
  política, tentando por duas vezes se eleger como deputado, mas não
  conseguiu. No entanto, fez parte do governo formado por Getúlio
  Vargas, a partir de 1930, e colaborou no então recém"-criado Ministério
  do Trabalho, principalmente na formulação da primeira legislação
  trabalhista daquele governo.}, cuja ilustração, cujo talento e cujo
esforço na vida me causam tanta admiração, endosse, mesmo
profissionalmente, semelhante doutrina é que me entristece. O liberal, o
socialista Evaristo, quase anarquista, está me parecendo uma dessas
engraçadas feministas Brasil, gênero professora Daltro\footnote{Leolinda
  de Figueiredo Daltro (1860--1935), uma das pioneiras no movimento
  pela reivindicação da cidadania plena para as mulheres. Em 1910 fundou
  o Partido Republicano Feminino, numa época em que as mulheres não
  tinham sequer o direito ao voto (que só veio em 1932). Militante pelos
  direitos das mulheres, incomodou muito os conservadores e avessos à
  participação feminina na política e no funcionalismo público. Entre os
  críticos da professora Daltro, como era conhecida, estava Lima
  Barreto.}, que querem a emancipação da mulher unicamente para exercer
sinecuras do governo e rendosos cargos políticos; mas que, quando se
trata desse absurdo costume nosso de perdoar os maridos assassinos de
suas mulheres, por isso ou aquilo, nada dizem e ficam na moita.

A meu ver, não há degradação maior para a mulher do que semelhante
opinião quase geral; nada a degrada mais do que isso, penso eu.
Entretanto\ldots{}

Às vezes mesmo, o adultério é o que se vê e o que não se vê são outros
interesses e despeitos que só uma análise mais sutil podia revelar
nesses lagos.

No crime da Rua da Lapa, o criminoso, o marido, o interessado no caso,
portanto, não alegou quando depôs sozinho que a sua mulher fosse
adúltera; entretanto, a defesa, lemos nos jornais, está procurando
``justificar'' que ela o era.

O crime em si não me interessa, senão no que toca à minha piedade por
ambos; mas, se houvesse de escrever um romance, e não é o caso,
explicaria, ainda me louvando nos jornais, a coisa de modo talvez
satisfatório.

Não quero, porém, escrever romances e estou mesmo disposto a não
escrevê"-los mais, se algum dia escrevi um, de acordo com os cânones da
nossa crítica; por isso guardo as minhas observações e ilusões para o
meu gasto e para o julgamento da nossa atroz sociedade burguesa, cujo
espírito, cujos imperativos da nossa ação na vida animaram, o que parece
absurdo, mas de que estou absolutamente certo, o protagonista do
lamentável drama da rua da Lapa.

Afastei"-me do meu objetivo, que era mostrar a grosseria, a barbaridade
desse nosso costume de achar justo que o marido mate a mulher adúltera
ou que a crê tal.

Toda a campanha para mostrar a iniquidade de semelhante julgamento não
será perdida; e não deixo passar vaza que não diga algumas toscas
palavras, condenando"-o.

Se a coisa continuar assim, em breve, de lei costumeira, passará a lei
escrita e retrogradamos às usanças selvagens que queimavam e enterravam
vivas as adúlteras.

Convém, entretanto, lembrar que, nas velhas legislações, havia casos de
adultério legal. Creio que Sólon e Licurgo\footnote{Sólon (640--560 a.C.) foi um dos principais legisladores do período clássico da
  Grécia Antiga. Licurgo, sobre o qual não existem fontes seguras em
  relação à época em que viveu, também se destacou como um grande
  legislador de outra cidade"-estado, Esparta.} os admitia; creio mesmo
ambos. Não tenho aqui o meu Plutarco\footnote{Sobre Plutarco, ver nota
  52.}. Seja, porém, como for, não digo que todos os adultérios são
perdoáveis. Pior do que o adultério é o assassinato; e nós queremos
criar uma espécie dele baseado na lei.

%\part{Humor}

\chapter[A cartomante]{A cartomante\footnote[*]{Publicado no \emph{Jornal das moças} em 10/07/1919.}}

Não havia dúvida que naqueles atrasos e atrapalhações de sua vida,
alguma influência misteriosa preponderava. Era ele tentar qualquer
coisa, logo tudo mudava. Esteve quase para arranjar"-se na Saúde Pública;
mas, assim que obteve um bom ``pistolão''\footnote{\textls[-20]{Chamava"-se
  \emph{pistolão} a pessoa que tinha influência nas instituições
  públicas e que conseguia arranjar cargos para parentes e conhecidos.
  Ter um \emph{pistolão} significava conseguir bons cargos em
  instituições públicas, que na época não exigiam concursos para funções
  mais baixas na hierarquia.}}, toda a política mudou.

Se jogava no bicho, era sempre o grupo seguinte ou o anterior que dava.
Tudo parecia mostrar"-lhe que ele não devia ir para adiante. Se não
fossem as costuras da mulher, não sabia bem como poderia ter vivido até
ali.

Há cinco anos que não recebia vintém de seu trabalho. Uma nota de dois
mil"-réis, se alcançava ter na algibeira por vezes, era obtida com
auxílio de não sabia quantas humilhações, apelando para a generosidade
dos amigos.

Queria fugir, fugir para bem longe, onde a sua miséria atual não tivesse
o realce da prosperidade passada; mas, como fugir? Onde havia de buscar
dinheiro que o transportasse, a ele, a mulher e aos filhos?

\textls[-20]{Viver assim era terrível! Preso à sua vergonha como a uma calceta, sem
que nenhum código e juiz tivessem condenado, que martírio! A certeza,
porém, de que todas as suas infelicidades vinham de uma influência
misteriosa, deu"-lhe mais alento. Se era ``coisa feita'', havia de haver
por força quem a desfizesse.}

Acordou mais alegre e se não falou à mulher alegremente era porque ela
já havia saído. Pobre de sua mulher! Avelhantada precocemente,
trabalhando que nem uma moura, doente, entretanto a sua fragilidade
transformava"-se em energia para manter o casal. Ela saía, virava a
cidade, trazia costuras, recebia dinheiro, e aquele angustioso lar ia se
arrastando, graças aos esforços da esposa.

Bem! As coisas iam mudar! Ele iria a uma cartomante e havia de descobrir
o que e quem atrasavam a sua vida. Saiu, foi à venda e consultou o
jornal. Havia muitos videntes, espíritas, teósofos anunciados; mas
simpatizou com uma cartomante, cujo anúncio dizia assim: ``Madame Dadá,
sonâmbula, extralúcida, deita as cartas e desfaz toda espécie de
feitiçaria, principalmente a africana. Rua etc.''.

Não quis procurar outra; era aquela, pois já adquirira a convicção de
que aquela sua vida vinha sendo trabalhada pela mandinga de algum preto
mina, a soldo do seu cunhado Castrioto, que jamais vira com bons olhos o
seu casamento com a irmã.

Arranjou, com o primeiro conhecido que encontrou, o dinheiro necessário,
e correu depressa para a casa de Madame Dadá.

O mistério ia desfazer"-se e o malefício ser cortado. A abastança
voltaria à casa; compraria um terno para o Zezé, umas botinas para
Alice, a filha mais moça; e aquela cruciante vida de cinco anos havia de
lhe ficar na memória como passageiro pesadelo.

Pelo caminho tudo lhe sorria. Era o sol muito claro e doce, um sol de
junho; eram as fisionomias risonhas dos transeuntes; e o mundo, que até
ali lhe aparecia mau e turvo, repentinamente lhe surgia claro e doce.

Entrou, esperou um pouco, com o coração a lhe saltar do peito. O
consulente saiu e ele foi afinal à presença da pitonisa. Era sua mulher.



\chapter[A pescaria]{A pescaria\footnote[*]{Publicado na revista \emph{Careta} em 13/08/1921.}}

O Jorge era, apesar de boêmio, um bom chefe de família.

A sua mulher que lhe sabia cavalheiro e bom marido não se importava
absolutamente com as suas extravagâncias.

Eles viviam na maior paz e harmonia. Chegasse ele às dez, às onze ou às
quatro horas da madrugada, a recepção era a mais cordial possível.

Um dia pela semana santa, isto é, na quinta"-feira da Paixão, Jorge
chegou em casa e disse à mulher:

--- Eugênia, amanhã vou pescar e você me acorde cedo.

Dona Eugênia recebeu a recomendação com todo o carinho e, no dia
seguinte, logo pela manhã, pela madrugada, despertou o marido.

Jorge saiu lépido e contente com o prazer que ia dar à cara metade.

Em chegando ao primeiro botequim, porém, abancou e pôs"-se a beber.

Comer e beber, a questão é começar; e ele tinha começado e continuou.

Quando chegou aí pelo meio dia, lembrou"-se da pescaria que tinha
prometido à mulher.

--- Como havia de ser? pensou ele de si para si.

A canoa e os companheiros já deviam ter partido, e precisava levar os
peixes.

Entrou em uma confeitaria e comprou camarões, postas de peixe, siris,
ostras, etc. Tomou o bonde e foi para casa. Entregou os embrulhos à
mulher e foi dormir, tão cansado estava da pescaria.

Às cinco horas da tarde, Dona Eugênia veio"-lhe despertar:

--- Jorge! Jorge! Vem jantar.

Ele ergueu"-se e foi para a sala de refeições.

Quando lá chegou e viu aqueles primores de confeitaria, perguntou à
mulher:

--- Que diabo é isso?

Estamos em piquenique?

A mulher acudiu:

--- Isto é a pescaria que tu fizeste!



\chapter[Rocha, o guerreiro]{Rocha, o guerreiro\footnote[*]{Publicado na revista \emph{Careta} em 19/08/1922.}}

Este Rocha, quando nasceu, os pais levaram"-lhe à pia\footnote{O mesmo
  que batizar.} com o intuito de que ele fosse general.

Ele era descendente de gente brava que se tinha batido valentemente nos
campos do Paraguai\footnote{Referência à Guerra do Paraguai, o maior
  conflito armado da história da América do Sul. Durou desde 1864 até
  1870 e envolveu o Brasil, Argentina e Uruguai (que formavam a Tríplice
  Aliança) contra o Paraguai.}.

Quiseram meter"-lhe na Escola Militar, mas não houve meio de Rocha
aprender aritmética.

A mãe, como ele dizia ter vocação militar e ser neto ou bisneto do barão
de Jacutinga, herói da guerra do Paraguai, resolveu requerer"-lhe praça,
a fim de ser reconhecido cadete.

Rocha era medroso que nem um jacu; e quando foi metido num regimento da
Quinta da Boa Vista\footnote{Região localizada na zona norte da cidade
  do Rio de Janeiro, no bairro de São Cristóvão.}, ao primeiro trote,
pôs"-se a chorar. Queixou"-se ao capitão de sua companhia, a quem pediu
licença para escrever à sua velha mãe. O capitão, atendendo que ele não
tinha nenhuma enfibratura para a vida militar, consentiu.

A velha veio, e ele, mal ela chegou, disse"-lhe:

--- Mamãe, quero mamar.

Está aí em que deu o general futuro --- o Rocha das Arábias.



\chapter[O Gambá]{O Gambá\footnote[*]{Publicado na revista \emph{Careta} em 08/01/1921.}}

Eram muito amigos, o Jaime e o Pena. Jaime morava nos fundos da chácara
de um tio, num barracão de madeira; e ambos trabalhavam, como operários,
em uma mesma oficina suburbana.

Saíam juntos, e era raro o dia em que não se encontrassem, opôs o
jantar, no botequim do Marques, para beber alguma coisa, indo, certas
vezes, mais além.

Eram bem comportados, e as suas ``festanças'' acabavam para sempre em
ordem.

Pena morava perto de Jaime, mas, naquele dia, este convidou o amigo para
ir dormir no seu barracão, ou melhor: barraco, como se vai agora
chamando tais ``palacetes'' da nossa opulenta pobreza de país opulento.

A coisa deu"-se de maneira que vai ser aqui contada.

Era sábado, começo de mês e dia de ``mafuá''.

Chama"-se isto, nos subúrbios, uma nova edição das ``barraquinhas'' de
Santo Antônio, que, antigamente, em frente ao quartel general, se
armavam, para vender ``sortes'' de leitões, perus, etc., e cujo lucro
era em favor de não sei qual irmandade religiosa. Tinham as antigas
``barraquinhas'' uma época marcada --- era em junho; os ``mafuás'',
porém, funcionavam todo o ano, em dias santificados, feriados, sábados e
domingos.

Naquele sábado, dia, portanto, de ``mafuá'', após o jantarem ``à la
gondaça'' numa casa de pasto, os dois amigos andaram perambulando pelos
``mafuás''. Já haviam tratado dormir juntos, para caçar, na chácara do
tio de Jaime, um gambá que teimava em dar cabo dos pintos e frangos.

Às armadilhas de toda a ordem, o bicho tinha resistido. Não havia meio
de cair em laço, em alçapão, em coisa alguma.

Pena aventou a ideia da ``cachaça''.

--- Gambá --- dizia ele --- é havido por parati. Nós compramos um litro da
``branquinha'', deitamo"-lo numa vasilha, numa cumbuca qualquer; o gambá
vem atraído pelo cheiro da ``pinga''. Bebe à beça e cai numa bebedeira
de não poder voltar para o ninho. De manhã, logo cedo, um de nós vai até
o lugar e temos o petisco para almoçar domingo.

O outro concordou e só pôs uma dúvida:

--- Tu sabes moquear ``elas''?

--- Como não sei! Vais ver\ldots{}

Assim fizeram. No sábado, logo depois de deixarem a oficina, compraram
um litro de aguardente, levaram"-no para o barraco, jantaram na casa de
pasto e puseram"-se a passear.

Andaram de botequim para ``mafuá'' e de ``mafuá'' para botequim.

No começo, foram ``lambadas''; mas, ao chegar a hora da virtuosa
temperança policial, entraram pela cerveja afora.

Ambos tinham recebido a quinzena; o sacrifício, portanto, não era grande
para casa um.

Jaime, aí pelas onze horas, propôs irem para casa; Pena acedeu. Aquele
já estava bem ``trincado'', mas, assim mesmo, ao chegar em casa, não se
esqueceu de deitar o litro de ``cachaça'' num longo prato, pouco fundo,
de barro vidrado, para, com ela, deitar mão no gambá. Pena estava em
melhor estado.

Sorveu um pouco do ``malavo'' e os dois ainda rebateram o que já haviam
bebido por aí, com duas talagadas boas.

Jaime deitou"-se e pegou no sono logo. Pena, porém, estranhando a cama,
não conciliou o sono imediatamente.

Um dado momento, teve vontade de beber, correu ao litro; mas estava
totalmente vazio.

Como havia de ser? Acudiu"-lhe uma ideia: ir tirar uma ``golada'' do
recipiente destinado a apanhar o animalejo, pela embriaguez.

Agarrou numa xícara pequena, saiu mansamente e foi até o lugar em que
estava o prato com aguardente.

Se bem que não fosse de luar, a noite estava linda, segura e
profusamente estrelada.

Pena despediu"-se do parati com grande desgosto.

Daí a minutos, voltava com a tal xicarazinha.

Tomou outro gole, furtado ao gambá.

Em meio do caminho, deu"-lhe de novo vontade de beber a insidiosa
cachaça.

Voltou a terceira vez e pôs"-se a pensar: para que gambá?

É um bicho imundo, fedorento. Demais, podemos pegá"-lo a cacete.

Fico aqui e dou cabo dessa deliciosa ``branquinha''

Assim fez.

No dia seguinte, Jaime deu por falta do amigo no barracão. Talvez
tivesse ido embora, imaginou.

Lembrou"-se do gambá e da cachaça. Correu até lá; e --- oh! surpresa ---
encontrou um enorme gambá de dois pés e duas mãos, roncando ao sol alto.
Era o Pena.



\chapter[Pedra \& Moskowa]{Pedra \& Moskowa\footnote[*]{Publicado na revista \emph{Careta} em 24/01/1920.}}

Os dois boêmios de tempos já distantes, H. Pedra e Pedro Moskowa, um dia
se encontraram e foram tomar café --- uma infâmia, verdadeira perfumaria!

Puseram"-se, no botequim, a conversar e a palestrar.

Veio a conversa recair sobre a arte de ``morder''\footnote{``Morder''
  era uma gíria usada na época para a prática de pedir dinheiro
  emprestado a amigos e colegas.}.

Pedro, que era o mais inteligente, disse, num dado momento, ao H. Pedra:

--- Pedra, nós somos uns tolos.

--- Por quê?

--- Não sabemos ``morder'' cientificamente.

--- Não te compreendo.

--- Eu te explico.

--- Vá lá.

--- Nós dispensamos os nossos esforços, quando tudo nos ensina que
devemos conjugá"-los, articulá"-los, encadeá"-los em benefício comum.

--- Como é então?

--- Olha: vamos organizar uma lista das pessoas ``mordíveis''. Tu ficas
com uma e eu com outra. Num dia, mordo eu; noutro dia, tu. Ando do
almoço, dividimos a féria\footnote{Chamava"-se féria o nome dado à
  quantidade de dinheiro que um trabalhador recebia em um dia de
  trabalho. Neste caso, o ``trabalho'' seria o de ``morder'' os amigos.};
e, antes do jantar, também. Queres?

--- Aceito. Mas no domingo?

--- Cada um tem liberdade de ação, mas o melhor é não morder nenhuma
pessoa da lista.

--- Por quê?

--- Pode acontecer que nós ambos mordamos, num mesmo domingo, uma delas;
e, no dia seguinte, tanto eu quanto tu estaremos atrapalhados para
fazê"-la ``sangrar''. Aceitas?

--- Está feito.

Combinado isto, os dois organizaram a relação das pessoas conspícuas que
podiam merecer a honra das suas ``facadas'' e puseram em prática os fins
de sua curiosa associação, que, se fosse registrada na Junta Comercial,
teria de girar sob a firma Pedra \& Moskowa.

Pedra ``mordia'' nas segundas, e Moskowa, nas terças; e assim por
diante, alternando"-se.

Nas horas marcadas, dividiam irmãmente a féria, sem que nenhum
``refundisse'' um níquel, isto é, sonegasse"-o ao outro.

Um dia, porém, em uma confeitaria, Pedra viu que o dr. F. C. era da
lista, puxava uma nota graúda, para pagar um vermute que tomara no
balcão. Não era o seu dia, mas não se conteve e deu o bote. A vítima
sangrou e H. Pedra, que recebera uma ``forquilha'' (2\$000)\footnote{Dois
  mil réis.}, tratou de refestelar"-se num angu do Bernardinho; no largo
da Sé.

Moskowa, que não sabia da coisa, quando encontrou o doutor F. C. foi
cumprir a sua obrigação; qual não foi o seu espanto, porém, quando ele
lhe disse:

--- ``Seu'' Moskowa, hoje não é seu dia, pois já dei ao Pedra.



\chapter[Percalços da farda]{Percalços da farda\footnote[*]{Publicado no \emph{Jornal das Moças} em 27/11/1919.}}

O coronel M. morava em uma rua transversal à dos Voluntários\footnote{Referência
  à rua Voluntários da Pátria, localizada no bairro de Botafogo, na zona
  sul da cidade do Rio de Janeiro. Até hoje é considerada uma das
  principais ruas daquele bairro. O nome faz homenagem aos Batalhões dos
  Voluntários da Pátria, que se formavam para engrossar as fileiras do
  Exército Brasileiro, que estava enfrentando muitas dificuldades
  durante a Guerra do Paraguai, ocorrida entre os anos 1864--1870.} há
tempos e, durante muitos anos, foi pessoa conceituada no lugar pela sua
lisura e ainda por cima por ser coronel da briosa\footnote{``Briosa''
  era um termo informal utilizado para nomear a Guarda Nacional
  Brasileira, que teve seu auge no período do Império.}, que naquelas
épocas era muito honrada e respeitada.

Tinha boa casa, vasta e antiga, habitação de outros tempos, um jardim na
frente e uns restos de chácara nos fundos.

Esquecia"-me de dizer que era negociante não sei de quê; mas era
negociante forte. Além das criadas de interior, cozinheira, arrumadeira,
copeira, tinha também empregados para o jardim e para a chácara.

Aqueles nacionais; e estes eram espanhóis ou portugueses. Durante muito
tempo, el1e foi o mais exato dos patrões; e todos os começos de mês ele
satisfazia os salários dos seus fâmulos com a máxima pontualidade e sem
o menor desconto. Era o melhor dos patrões; e as raparigas quebravam a
louça e perdiam peças de roupa impunemente, enquanto os homens da
chácara e jardim perdiam ferramentas e vendiam os produtos respectivos
como se deles fossem donos. Esta fartura comera muito; mas vieram
negócios, e M. perdeu muito arriscando até o negócio, que ficou sob o
regime da concordata.

Esperava que, com o tempo, solvesse os seus compromissos e voltasse à
fartura antiga. Reduziu as despesas, mas não o pôde fazer totalmente,
porquanto perderia de todo o crédito; é preciso sempre apresentar e
conservar alguns disfarces, entre os quais o chacareiro, pois a chácara
era a menina dos seus olhos.

Enquanto foi o antigo, ele se atrasava alguns meses, não causando a
menor ira ao velho serviçal; mas, adoecendo este, despediu"-se para ir
para a terra. Estava velho e queria morrer lá. M. despediu"-se do velho
Manuel cheio de mágoa, mas não havia remédio.

Admitiu em substituição um espanhol a quem durante alguns meses pagou
pontualmente. Um mês, porém, não pôde pagar; no seguinte, também não,
mas pelo terceiro o espanhol não continuou e despediu"-se amuado. Voltou
ao mês seguinte, e o Iglesias não se conteve com a negativa:

--- Ou tu me dás ou por \emph{Dios\ldots{}}

E fechou a mão cerrada de raiva. Vendo a atitude do espanhol, o coronel
disse:

--- Espera aí que já volto.

O campeador esperou enquanto M dirigia"-se ao interior da casa.

Não tardou muito que de novo visse"-o, fardado de ponto em branco de
coronel, brandindo um espadagão:

--- Então, seu cão?

O espanhol, não esperando por essa, recuou, tirou um revólver, despejou
sobre M., que quase morreu dos ferimentos.



\part{\textsc{contos}}


\chapter[O homem que sabia javanês]{O homem que sabia javanês\footnote[*]{Publicado na edição de 20 de abril de 1911 no jornal \emph{Gazeta da Tarde}.}}

Em uma confeitaria, certa vez, ao meu amigo Castro, contava eu as
partidas que havia pregado às convicções e às respeitabilidades, para
poder viver.

Houve mesmo, uma dada ocasião, quando estive em Manaus, em que fui
obrigado a esconder a minha qualidade de bacharel, para mais confiança
obter dos clientes, que afluíam ao meu escritório de feiticeiro e
adivinho. Contava eu isso.

O meu amigo ouvia"-me calado, embevecido, gostando daquele meu Gil
Blas\footnote{Referência ao personagem do livro ``História de Gil Blas
  de Santillana'', escrito pelo francês Alain"-René Lesage, durante cerca
  de vinte anos (1715--1735). Gil Blas é um tipo de herói conhecido
  como \emph{pícaro}, de longa tradição na literatura espanhola. Gil
  Blas, na qualidade de personagem"-pícaro, vive uma vida de aventuras,
  sem residência fixa, aplicando diversas artimanhas para conseguir
  sobreviver.} vivido, até que, em uma pausa da conversa, ao esgotarmos
os copos, observou a esmo:

--- Tens levado uma vida bem engraçada, Castelo!

--- Só assim se pode viver\ldots{} Isto de uma ocupação única: sair de casa a
certas horas, voltar a outras, aborrece, não achas? Não sei como me
tenho aguentado lá, no consulado!

--- Cansa"-se; mas, não é disso que me admiro. O que me admira, é que
tenhas corrido tantas aventuras aqui, neste Brasil imbecil e
burocrático.

--- Qual! Aqui mesmo, meu caro Castro, se podem arranjar belas páginas
de vida. Imagina tu que eu já fui professor de javanês!

--- Quando? Aqui, depois que voltaste do consulado?

--- Não; antes. E, por sinal, fui nomeado cônsul por isso.

--- Conta lá como foi. Bebes mais cerveja?

--- Bebo. Mandamos buscar mais outra garrafa, enchemos os copos, e
continuei:

--- Eu tinha chegado havia pouco ao Rio estava literalmente na miséria.
Vivia fugido de casa de pensão em casa de pensão, sem saber onde e como
ganhar dinheiro, quando li no \emph{Jornal do Comércio} o anúncio
seguinte:

``Precisa"-se de um professor de língua javanesa. Cartas etc.''

Ora, disse cá comigo, está ali uma colocação que não terá muitos
concorrentes; se eu capiscasse quatro palavras, ia apresentar"-me. Saí do
café e andei pelas ruas, sempre a imaginar"-me professor de javanês,
ganhando dinheiro, andando de bonde e sem encontros desagradáveis com os
``cadáveres''. Insensivelmente dirigi"-me à Biblioteca Nacional. Não
sabia bem que livro iria pedir; mas, entrei, entreguei o chapéu ao
porteiro, recebi a senha e subi. Na escada, acudiu"-me pedir a Grande
Encyclopédie\footnote{Referência à ``Enciclopédia: dicionário racional
  das ciências, artes e profissões'', publicada na França na segunda
  metade do século \textsc{xviii} pelos filósofos iluministas Jean le Rond
  d'Alembert, em parceria com Denis Diderot. Considerada uma das
  primeiras e o modelo a partir do qual todas as demais Enciclopédias
  foram criadas no Ocidente.}, letra J, a fim de consultar o artigo
relativo a Java e a língua javanesa. Dito e feito. Fiquei sabendo, ao
fim de alguns minutos, que Java era uma grande ilha do arquipélago de
Sonda, colônia holandesa, e o javanês, língua aglutinante do grupo
malaio"-polinésio, possuía uma literatura digna de nota e escrita em
caracteres derivados do velho alfabeto hindu.

A Enciclopédia dava"-me indicação de trabalhos sobre a tal língua malaia
e não tive dúvidas em consultar um deles. Copiei o alfabeto, a sua
pronunciação figurada e saí.

Andei pelas ruas, perambulando e mastigando letras. Na minha cabeça
dançavam hieróglifos; de quando em quando consultava as minhas notas;
entrava nos jardins e escrevia estes calungas na areia para guardá"-los
bem na memória e habituar a mão a escrevê"-los.

À noite, quando pude entrar em casa sem ser visto, para evitar
indiscretas perguntas do encarregado, ainda continuei no quarto a
engolir o meu ``a"-b"-c'' malaio, e, com tanto afinco levei o propósito
que, de manhã, o sabia perfeitamente.

Convenci"-me que aquela era a língua mais fácil do mundo e saí; mas, não
tão cedo que não me encontrasse com o encarregado dos aluguéis dos
cômodos:

--- Senhor Castelo, quando salda a sua conta?

Respondi"-lhe então eu, com a mais encantadora esperança:

--- Breve\ldots{} Espere um pouco\ldots{} Tenha paciência\ldots{} Vou ser nomeado
professor de javanês, e\ldots{} Por aí o homem interrompeu"-me:

--- Que diabo vem a ser isso, senhor Castelo?

Gostei da diversão e ataquei o patriotismo do homem:

--- É uma língua que se fala lá pelas bandas do Timor. Sabe onde é?

Oh! alma ingênua! O homem esqueceu"-se da minha dívida e disse"-me com
aquele falar forte dos portugueses:

--- Eu cá por mim, não sei bem; mas ouvi dizer que são umas terras que
temos lá para os lados de Macau. E o senhor sabe isso, senhor Castelo?

Animado com esta saída feliz que me deu o javanês, voltei a procurar o
anúncio. Lá estava ele. Resolvi animosamente propor"-me ao professorado
do idioma oceânico. Redigi a resposta, passei pelo Jornal e lá deixei a
carta. Em seguida, voltei à biblioteca e continuei os meus estudos de
javanês. Não fiz grandes progressos nesse dia, não sei se por julgar o
alfabeto javanês o único saber necessário a um professor de língua
malaia ou se por ter me empenhado mais na bibliografia e história
literária do idioma que ia ensinar.

Ao cabo de dois dias, recebia eu uma carta para ir falar ao doutor
Manuel Feliciano Soares Albernaz, barão de Jacuecanga, à Rua Conde de
Bonfim, não me recordo bem que número. É preciso não te esqueceres que
entrementes continuei estudando o meu malaio, isto é, o tal javanês.
Além do alfabeto, fiquei sabendo o nome de alguns autores, também
perguntar e responder ``como está o senhor?'' --- e duas ou três regras
de gramática, lastrado todo esse saber com vinte palavras do léxico. Não
imaginas as grandes dificuldades com que lutei, para arranjar os
quatrocentos réis da viagem! É mais fácil --- podes ficar certo ---
aprender o javanês\ldots{}

Fui a pé. Cheguei suadíssimo; e, com maternal carinho, as anosas
mangueiras, que se perfilavam em alameda diante da casa do titular, me
receberam, me acolheram e me reconfortaram. Em toda a minha vida, foi o
único momento em que cheguei a sentir a simpatia da natureza\ldots{}

Era uma casa enorme que parecia estar deserta; estava maltratada, mas
não sei por que me veio pensar que nesse mau tratamento havia mais
desleixo e cansaço de viver que mesmo pobreza. Devia haver anos que não
era pintada. As paredes descascavam e os beirais do telhado, daquelas
telhas vidradas de outros tempos, estavam desguarnecidos aqui e ali,
como dentaduras decadentes ou malcuidadas.

Olhei um pouco o jardim e vi a pujança vingativa com que a tiririca e o
carrapicho tinham expulsado os tinhorões e as begônias. Os crótons
continuavam, porém, a viver com a sua folhagem de cores mortiças. Bati.
Custaram"-me a abrir. Veio, por fim, um antigo preto africano, cujas
barbas e cabelo de algodão davam à sua fisionomia uma aguda impressão de
velhice, doçura e sofrimento.

Na sala, havia uma galeria de retratos: arrogantes senhores de barba em
colar se perfilavam enquadrados em imensas molduras douradas, e doces
perfis de senhoras, em bandós, com grandes leques, pareciam querer subir
aos ares, enfunadas pelos redondos vestidos à balão; mas, daquelas
velhas coisas, sobre as quais a poeira punha mais antiguidade e
respeito, a que gostei mais de ver foi um belo jarrão de porcelana da
China ou da Índia, como se diz. Aquela pureza da louça, a sua
fragilidade, a ingenuidade do desenho e aquele seu fosco brilho de luar,
diziam"-me a mim que aquele objeto tinha sido feito por mãos de criança,
a sonhar, para encanto dos olhos fatigados dos velhos desiludidos\ldots{}

Esperei um instante o dono da casa. Tardou um pouco. Um tanto trôpego,
com o lenço de alcobaça na mão, tomando veneravelmente o simonte de
antanho, foi cheio de respeito que o vi chegar. Tive vontade de ir"-me
embora. Mesmo se não fosse ele o discípulo, era sempre um crime
mistificar aquele ancião, cuja velhice trazia à tona do meu pensamento
alguma coisa de augusto, de sagrado. Hesitei, mas fiquei.

--- Eu sou, avancei, o professor de javanês, que o senhor disse
precisar.

--- Sente"-se, respondeu"-me o velho. O senhor é daqui, do Rio?

--- Não, sou de Canavieiras.

--- Como? fez ele. Fale um pouco alto, que sou surdo,

--- Sou de Canavieiras, na Bahia, insisti eu.

--- Onde fez os seus estudos?

--- Em São Salvador.

--- Em onde aprendeu o javanês? indagou ele, com aquela teimosia
peculiar aos velhos. Não contava com essa pergunta, mas imediatamente
arquitetei uma mentira. Contei"-lhe que meu pai era javanês. Tripulante
de um navio mercante, viera ter à Bahia, estabelecera"-se nas
proximidades de Canavieiras como pescador, casara, prosperara e fora com
ele que aprendi javanês.

--- E ele acreditou? E o físico?
perguntou meu amigo, que até então me ouvira calado.

--- Não sou, objetei, lá muito diferente de um javanês. Estes meus
cabelos corridos, duros e grossos e a minha pele basané\footnote{Palavra
  francesa que significa bronzeada.} podem dar"-me muito bem o aspecto de
um mestiço de malaio\ldots{} Tu sabes bem que, entre nós, há de tudo: índios,
malaios, taitianos, malgaches, guanches, até godos. É uma comparsaria de
raças e tipos de fazer inveja ao mundo inteiro.

--- Bem, fez o meu amigo, continua.

--- O velho, emendei eu, ouviu"-me atentamente, considerou demoradamente
o meu físico, pareceu que me julgava de fato filho de malaio e
perguntou"-me com doçura:

--- Então está disposto a ensinar"-me javanês?

--- A resposta saiu"-me sem querer:

--- Pois não.

--- O senhor há de ficar admirado, aduziu o barão de Jacuecanga, que eu,
nesta idade, ainda queira aprender qualquer coisa, mas\ldots{}

--- Não tenho que admirar. Têm"-se visto exemplos e exemplos muito
fecundos\ldots{}

--- O que eu quero, meu caro senhor\ldots{}

--- Castelo, adiantei eu.

--- O que eu quero, meu caro senhor Castelo, é cumprir um juramento de
família. Não sei se o senhor sabe que eu sou neto do conselheiro
Albernaz, aquele que acompanhou Pedro I, quando abdicou.\footnote{Dom
  Pedro I, o primeiro Imperador do Brasil, abdicou (desistiu) do Trono
  no dia 07 de abril de 1831. O sucessor legal do Trono, Dom Pedro \textsc{ii},
  tinha apenas cinco anos. Esse fato gerou uma enorme crise política na
  recente nação brasileira. O país passou a ser governado pelas
  Regências, período conhecido como Período Regencial, que durou até
  1840, quando foi decretada a maioridade antecipada de Pedro \textsc{ii}, que se
  tornou Imperador antes de completar quinze anos. Esse episódio ficou
  conhecido como o ``golpe da maioridade''.} Voltando de Londres, trouxe
para aqui um livro em língua esquisita, a que tinha grande estimação.
Fora um hindu ou siamês que dera a ele, em Londres, em agradecimento a
não sei que serviço prestado por meu avô. Ao morrer meu avô, chamou meu
pai e lhe disse:

``Filho, tenho este livro aqui, escrito em javanês. Disse"-me quem me deu
que ele evita desgraças e traz felicidades para quem o tem. Eu não sei
nada ao certo. Em todo o caso, guarda"-o; mas, se queres que o fado que
me deitou o sábio oriental se cumpra, faze com que teu filho o entenda,
para que sempre a nossa raça seja feliz''.

Meu pai, continuou o velho barão, não acreditou muito na história;
contudo, guardou o livro. Às portas da morte, ele entregou"-me e disse"-me
o que prometera ao pai. Em começo, pouco caso fiz da história do livro.
Deitei"-o a um canto e fabriquei minha vida. Cheguei até a esquecer"-me
dele; mas, de uns tempos a esta parte, tenho passado por tanto desgosto,
tantas desgraças têm caído sobre a minha velhice que me lembrei do
talismã da família. Tenho que o ler, que o compreender, se não quero que
os meus últimos dias anunciem o desastre da minha posteridade; e, para
entendê"-lo, é claro que preciso entender o javanês. Eis aí.

Calou"-se e notei que os olhos do velho se tinham orvalhado. Enxugou
discretamente os olhos e perguntou"-me se queria ver o tal livro.
Respondi"-lhe que sim. Chamou o criado, deu"-lhe as instruções e
explicou"-me que perdera todos os filhos, sobrinhos, só lhe restando uma
filha casada, cuja prole, porém, estava reduzida a um filho, débil de
corpo e de saúde frágil e oscilante.

Veio o livro. Era um velho calhamaço, um in"-quarto antigo, encadernado
em couro, impresso em grandes letras, em um papel amarelado e grosso.
Faltava a folha do rosto e por isso não se podia ler a data da
impressão. Tinha ainda umas páginas de prefácio, escritas em inglês,
onde li que se tratava das histórias do príncipe Kulanga, escritor
javanês de muito mérito.

Logo informei disso o velho barão que, não percebendo que eu tinha
chegado aí pelo inglês, ficou tendo em alta consideração o meu saber
malaio. Estive ainda folheando o cartapácio, à laia de quem sabe
magistralmente aquela espécie de vasconço, até que afinal contratamos as
condições de preço e de hora, comprometendo"-me a fazer com que ele lesse
o tal alfarrábio antes de um ano.

Dentro em pouco, dava a minha primeira lição, mas o velho não foi tão
diligente quanto eu. Não conseguia aprender a distinguir e a escrever
nem sequer quatro letras. Enfim, com metade do alfabeto levamos um mês e
o senhor barão de Jacuecanga não ficou lá muito senhor da matéria:
aprendia e desaprendia.

A filha e o genro (penso que até aí nada sabiam da história do livro)
vieram a ter notícias do estudo do velho; não se incomodaram. Acharam
graça e julgaram a coisa boa para distraí"-lo.

Mas com o que tu vais ficar assombrado, meu caro Castro, é com a
admiração que o genro ficou tendo pelo professor de javanês. Que coisa
Única! Ele não se cansava de repetir: ``É um assombro! Tão moço! Se eu
soubesse isso, ah! onde estava!''.

O marido de dona Maria da Glória
(assim se chamava a filha do barão), era desembargador, homem
relacionado e poderoso; mas não se pejava em mostrar diante de todo o
mundo a sua admiração pelo meu javanês. Por outro lado, o barão estava
contentíssimo. Ao fim de dois meses, desistira da aprendizagem e
pedira"-me que lhe traduzisse, um dia sim outro não, um trecho do livro
encantado. Bastava entendê"-lo, disse ele; nada se opunha que outrem o
traduzisse e ele ouvisse. Assim evitava a fadiga do estudo e cumpria o
encargo.

Sabes bem que até hoje nada sei de javanês, mas compus umas histórias
bem tolas e impingi"-as ao velhote como sendo do crônicon.

Como ele ouvia aquelas bobagens!\ldots{} Ficava estático, como se estivesse a
ouvir palavras de um anjo. E eu crescia aos seus olhos!

Fez"-me morar em sua casa, enchia"-me de presentes, aumentava"-me o
ordenado. Passava, enfim, uma vida regalada. Contribuiu muito para isso
o fato de vir ele a receber uma herança de um seu parente esquecido que
vivia em Portugal. O bom velho atribuiu a coisa ao meu javanês; e eu
estive quase a crê"-lo também.

Fui perdendo os remorsos; mas, em todo o caso, sempre tive medo que me
aparecesse pela frente alguém que soubesse o tal patuá malaio. E esse
meu temor foi grande, quando o doce barão me mandou com uma carta ao
visconde de Caruru, para que me fizesse entrar na diplomacia. Fiz"-lhe
todas as objeções: a minha fealdade, a falta de elegância, o meu aspecto
tagalo.

--- ``Qual! retrucava ele. Vá, menino; você sabe javanês!''

Fui. Mandou"-me o visconde para a Secretaria dos Estrangeiros com
diversas recomendações. Foi um sucesso.

O diretor chamou os chefes de secção: ``Vejam só, um homem que sabe
javanês --- que portento!''.

Os chefes de secção levaram"-me aos oficiais e amanuenses e houve um
destes que me olhou mais com ódio do que com inveja ou admiração. E
todos diziam: ``Então sabe javanês? É difícil? Não há quem o saiba
aqui!''.

O tal amanuense, que me olhou com ódio, acudiu então: ``É verdade, mas
eu sei canaque. O senhor sabe?''. Disse"-lhe que não e fui à presença do
ministro.

A alta autoridade levantou"-se, pôs as mãos às cadeiras, concertou o
\emph{pince"-nez}\footnote{\emph{Pince"-nez} é um termo francês para um
  modelo de óculos utilizado por séculos, até a invenção das armações
  com haste, no início do século \textsc{xx}. O \emph{pince"-nez} não tem haste e
  as lentes são presas nos aros e estes ligados por uma ``ponte''
  flexível, que faz pressão sobre o nariz. A palavra se encontra
  aportuguesada para pincenê.} no nariz e perguntou: ``Então, sabe
javanês?''. Respondi"-lhe que sim; e, à sua pergunta onde o tinha
aprendido, contei"-lhe a história do tal pai javanês. ``Bem, disse"-me o
ministro, o senhor não deve ir para a diplomacia; o seu físico não se
presta\ldots{} O bom seria um consulado na Ásia ou Oceania. Por ora, não há
vaga, mas vou fazer uma reforma e o senhor entrará. De hoje em diante,
porém, fica adido ao meu ministério e quero que, para ano, parta para
Bale\footnote{Hoje em dia se escreve Bali. Trata"-se de uma ilha
  pertencente a Indonésia.}, onde vai representar o Brasil no Congresso
de Linguística. Estude, leia o Hovelacque\footnote{Abel Hovelacque (1843--1896), foi um importante linguista francês. Seus estudos sobre a
  linguagem estão reunidos principalmente no livro ``La Linguistique''
  (A Linguística), de 1887. Também participou da política do tempo, como
  um dos fundadores do Partido Radical, Conselheiro Municipal de Paris e
  Deputado.}, o Max Müller\footnote{O Alemão Friedrich Max Müller (1823--1900) estudou por toda sua vida a história das religiões,
  principalmente as orientais. Se destacou também como linguista e
  estudioso da mitologia.}, e outros!''.

Imagina tu que eu até aí nada sabia de javanês, mas estava empregado e
iria representar o Brasil em um congresso de sábios.

O velho barão veio a morrer, passou o livro ao genro para que o fizesse
chegar ao neto, quando tivesse a idade conveniente e fez"-me uma deixa no
testamento.

Pus"-me com afã no estudo das línguas malaio"-polinésicas; mas não havia
meio! Bem jantado, bem"-vestido, bem dormido, não tinha energia
necessária para fazer entrar na cachola aquelas coisas esquisitas.
Comprei livros, assinei revistas: \emph{Revue Anthropologique et
Linguistique, Proceedings of the English"-Oceanic Association, Archivo
Glottologico Italiano}, o diabo, mas nada! E a minha fama crescia.

Na rua, os informados apontavam"-me, dizendo aos outros: ``Lá vai o
sujeito que sabe javanês''. Nas livrarias, os gramáticos consultavam"-me
sobre a colocação dos pronomes no tal jargão das ilhas de Sonda. Recebia
cartas dos eruditos do interior, os jornais citavam o meu saber e
recusei aceitar uma turma de alunos sequiosos de entenderem o tal
javanês.

A convite da redação, escrevi, no \emph{Jornal do Comércio} um artigo de
quatro colunas sobre a literatura javanesa antiga e moderna\ldots{}

--- Como, se tu nada sabias?, interrompeu"-me o atento Castro.

--- Muito simplesmente: primeiramente, descrevi a ilha de Java, com o
auxílio de dicionários e umas poucas publicações de geografias, e depois
citei a mais não poder.

--- E nunca duvidaram?, perguntou"-me ainda o meu amigo.

--- Nunca. Isto é, uma vez quase fico perdido. A polícia prendeu um
sujeito, um marujo, um tipo bronzeado que só falava uma língua
esquisita. Chamaram diversos intérpretes, ninguém o entendia. Fui também
chamado, com todos os respeitos que a minha sabedoria merecia,
naturalmente. Demorei"-me em ir, mas fui afinal. O homem já estava solto,
graças à intervenção do cônsul holandês, a quem ele se fez compreender
com meia dúzia de palavras holandesas. E o tal marujo era javanês ---
uf!

Chegou, enfim, a época do congresso, e lá fui para a Europa. Que
delícia! Assisti à inauguração e às sessões preparatórias.
Inscreveram"-me na secção do tupi"-guarani e eu abalei para Paris. Antes,
porém, fiz publicar no \emph{Mensageiro de Bali} o meu retrato, notas
biográficas e bibliográficas. Quando voltei, o presidente pediu"-me
desculpas por me ter dado aquela seção; não conhecia os meus trabalhos e
julgara que, por ser eu americano brasileiro, me estava naturalmente
indicada a seção do tupi"-guarani. Aceitei as explicações e até hoje
ainda não pude escrever as minhas obras sobre o javanês, para lhe
mandar, conforme prometi.

Acabado o congresso, fiz publicar extratos do artigo do \emph{Mensageiro
de Bali}, em Berlim, em Turim e Paris, onde os leitores de minhas obras
me ofereceram um banquete, presidido pelo senador Gorot. Custou"-me toda
essa brincadeira, inclusive o banquete que me foi oferecido, cerca de
dez mil francos, quase toda a herança do crédulo e bom barão de
Jacuecanga.

Não perdi meu tempo nem meu dinheiro. Passei a ser uma glória nacional
e, ao saltar no Cais Pharoux\footnote{O Cais Pharoux foi um importante
  porto de embarque e desembarque de pequenas embarcações na cidade do
  Rio de Janeiro. Recebeu esse nome por conta da proximidade que ficava
  do Hotel Pharoux, de propriedade do francês Luís Adolfo Pharoux,
  considerado o primeiro grande hotel de luxo do Brasil. Desembarcar e
  embarcar no Cais Pharoux era símbolo de importância e distinção para
  as pessoas na época.}, recebi uma ovação de todas as classes sociais e
o presidente da República, dias depois, convidava"-me para almoçar em sua
companhia.

Dentro de seis meses fui despachado cônsul em Havana, onde estive seis
anos e para onde voltarei, a fim de aperfeiçoar os meus estudos das
línguas da Malaia, Melanésia e Polinésia.

--- É fantástico, observou Castro, agarrando o copo de cerveja.

--- Olha: se não fosse estar contente, sabes que ia ser?

--- Que?

--- Bacteriologista eminente. Vamos?

--- Vamos.



\chapter[A nova Califórnia]{A nova Califórnia\footnote[*]{Publicado na edição de março de 1911 da \emph{Revista Americana}.}}

\section*{Capítulo i}

\noindent{}Ninguém sabia donde viera aquele homem. O agente do Correio pudera
apenas informar que acudia ao nome de Raimundo Flamel, pois assim era
subscrita a correspondência que recebia. E era grande. Quase
diariamente, o carteiro lá ia a um dos extremos da cidade, onde morava o
desconhecido, sopesando um maço alentado de cartas vindas do mundo
inteiro, grossas revistas em línguas arrevesadas, livros, pacotes\ldots{}

Quando Fabrício, o pedreiro, voltou de um serviço em casa do novo
habitante, todos na venda perguntaram"-lhe que trabalho lhe tinha sido
determinado.

--- Vou fazer um forno, disse o preto, na sala de jantar.

Imaginem o espanto da pequena cidade de Tubiacanga, ao saber de tão
extravagante construção: um forno na sala de jantar! E, pelos dias
seguintes, Fabrício pôde contar que vira balões de vidros, facas sem
corte, copos como os da farmácia --- um rol de coisas esquisitas a se
mostrarem pelas mesas e prateleiras como utensílios de uma bateria de
cozinha em que o próprio diabo cozinhasse.

O alarme se fez na vila. Para uns, os mais adiantados, era um fabricante
de moeda falsa; para outros, os crentes e simples, um tipo que tinha
parte com o tinhoso.

Chico da Tirana, o carreiro, quando passava em frente da casa do homem
misterioso, ao lado do carro a chiar, e olhava a chaminé da sala de
jantar a fumegar, não deixava de persignar"-se e rezar um ``credo'' em voz
baixa; e, não fora a intervenção do farmacêutico, o subdelegado teria
ido dar um cerco a casa daquele indivíduo suspeito, que inquietava a
imaginação de toda uma população.

Tomando em consideração as informações de Fabrício, o boticário Bastos
concluíra que o desconhecido devia ser um sábio, um grande químico,
refugiado ali para mais sossegadamente levar avante os seus trabalhos
científicos.

Homem formado e respeitado na cidade, vereador, médico também, porque o
doutor Jerônimo não gostava de receitar e se fizera sócio da farmácia
para mais em paz viver, a opinião de Bastos levou tranquilidade a todas
as consciências e fez com que a população cercasse de uma silenciosa
admiração a pessoa do grande químico, que viera habitar a cidade.

De tarde, se o viam a passear pela margem do Tubiacanga, sentando"-se
aqui e ali, olhando perdidamente as águas claras do riacho, cismando
diante da penetrante melancolia do crepúsculo, todos se descobriam e não
era raro que às ``boas noites'' acrescentassem ``doutor''. E tocava muito o
coração daquela gente a profunda simpatia com que ele tratava as
crianças, a maneira pela qual as contemplava, parecendo apiedar"-se de
que elas tivessem nascido para sofrer e morrer.

Na verdade, era de se ver, sob a doçura suave da tarde, a bondade de
Messias com que ele afagava aquelas crianças pretas, tão lisas de pele e
tão tristes de modos, mergulhadas no seu cativeiro moral, e também as
brancas, de pele baça, gretada e áspera, vivendo amparadas na necessária
caquexia dos trópicos.

Por vezes, vinha"-lhe vontade de pensar qual a razão de ter Bernardin de
Saint"-Pierre gasto toda a sua ternura com Paulo e Virgínia\footnote{\emph{Paul
  et Virginie} (Paulo e Virgínea) foi um dos romances mais famosos do
  final do século \textsc{xviii}. Escrito em 1788 pelo francês Jacques"-Henri
  Bernardin de Saint Pierre (1737--1814), o livro continuou como um
  dos mais lidos na Europa ao longo de quase todo o século \textsc{xix}.} e
esquecer"-se dos escravos que os cercavam\ldots{} Em poucos dias a admiração
pelo sábio era quase geral, e não o era unicamente porque havia alguém
que não tinha em grande conta os méritos do novo habitante. Capitão
Pelino, mestre"-escola e redator da \emph{Gazeta de Tubiacanga}, órgão
local e filiado ao partido situacionista, embirrava com o sábio. ``Vocês
hão de ver, dizia ele, quem é esse tipo\ldots{} Um caloteiro, um aventureiro
ou talvez um ladrão fugido do Rio''.

A sua opinião em nada se baseava, ou antes, baseava"-se no seu oculto
despeito vendo na terra um rival para a fama de sábio de que gozava. Não
que Pelino fosse químico, longe disso; mas era sábio, era gramático.
Ninguém escrevia em Tubiacanga que não levasse bordoada do Capitão
Pelino, e mesmo quando se falava em algum homem notável lá no Rio, ele
não deixava de dizer: ``Não há dúvida! O homem tem talento, mas escreve:
`um outro', `de resto'\ldots{}'' E contraía os lábios como se tivesse engolido
alguma cousa amarga.

Toda a vila de Tubiacanga acostumou"-se a respeitar o solene Pelino, que
corrigia e emendava as maiores glórias nacionais. Um sábio\ldots{}

Ao entardecer, depois de ler um pouco o Sotero, o Candido de Figueiredo
ou o Castro Lopes\footnote{Francisco Sotero dos Reis (1800--1871),
  brasileiro, Antônio Pereira Cândido de Figueiredo (1846--1925),
  português e Antônio de Castro Lopes (1827--1901), também brasileiro,
  foram importantes estudiosos da língua portuguesa. Escreveram livros
  que se tornaram referência, principalmente para o estudo e compreensão
  da gramática da língua portuguesa.}, e de ter passado mais uma vez a
tintura nos cabelos, o velho mestre"-escola saía vagarosamente de casa,
muito abotoado no seu paletó de brim mineiro, e encaminhava"-se para a
botica do Bastos a dar dois dedos de prosa. Conversar é um modo de
dizer, porque era Pelino avaro de palavras, limitando"-se tão"-somente a
ouvir. Quando, porém, dos lábios de alguém escapava a menor incorreção
de linguagem, intervinha e emendava. ``Eu asseguro, dizia o agente do
Correio, que\ldots{}''. Por aí, o mestre"-escola intervinha com mansuetude
evangélica: ``Não diga `asseguro' Senhor Bernardes; em português é
garanto''.

E a conversa continuava depois da emenda, para ser de novo interrompida
por uma outra. Por essas e outras, houve muitos palestradores que se
afastaram, mas Pelino, indiferente, seguro dos seus deveres, continuava
o seu apostolado de vernaculismo. A chegada do sábio veio distraí"-lo um
pouco da sua missão. Todo o seu esforço voltava"-se agora para combater
aquele rival, que surgia tão inopinadamente.

Foram vãs as suas palavras e a sua eloquência: não só Raimundo Flamel
pagava em dia as suas contas, como era generoso --- pai da pobreza --- e
o farmacêutico vira numa revista de específicos seu nome citado como
químico de valor.

\section*{Capítulo ii}

Havia já anos que o químico vivia em Tubiacanga, quando, uma bela manhã,
Bastos o viu entrar pela botica adentro. O prazer do farmacêutico foi
imenso. O sábio não se dignara até aí visitar fosse quem fosse e, certo
dia, quando o sacristão Orestes ousou penetrar em sua casa, pedindo"-lhe
uma esmola para a futura festa de Nossa Senhora da Conceição, foi com
visível enfado que ele o recebeu e atendeu.

Vendo"-o, Bastos saiu de detrás do balcão, correu a recebê"-lo com a mais
perfeita demonstração de quem sabia com quem tratava e foi quase em uma
exclamação que disse:

--- Doutor, seja bem"-vindo.

O sábio pareceu não se surpreender nem com a demonstração de respeito do
farmacêutico, nem com o tratamento universitário. Docemente, olhou um
instante a armação cheia de medicamentos e respondeu:

--- Desejava falar"-lhe em particular, Senhor Bastos.

O espanto do farmacêutico foi grande. Em que poderia ele ser útil ao
homem, cujo nome corria mundo e de quem os jornais falavam com tão
acendrado respeito? Seria dinheiro? Talvez\ldots{} Um atraso no pagamento das
rendas, quem sabe? E foi conduzindo o químico para o interior da casa,
sob o olhar espantado do aprendiz que, por um momento, deixou a ``mão''
descansar no gral, onde macerava uma tisana qualquer.

Por fim, achou ao fundo, bem no fundo, o quartinho que lhe servia para
exames médicos mais detidos ou para as pequenas operações, porque Bastos
também operava. Sentaram"-se e Flamel não tardou a expor:

--- Como o senhor deve saber, dedico"-me à química, tenho mesmo um nome
respeitado no mundo sábio\ldots{}

--- Sei perfeitamente, doutor, mesmo tenho disso informado, aqui, aos
meus amigos.

--- Obrigado. Pois bem: fiz uma grande descoberta, extraordinária. . .

Envergonhado com o seu entusiasmo, o sábio fez uma pausa e depois
continuou:

--- Uma descoberta\ldots{} Mas não me convém, por ora, comunicar ao mundo
sábio, compreende?

--- Perfeitamente.

--- Por isso precisava de três pessoas conceituadas que fossem
testemunhas de uma experiência dela e me dessem um atestado em forma,
para resguardar a prioridade da minha invenção\ldots{} O senhor sabe: há
acontecimentos imprevistos e\ldots{}

--- Certamente! Não há dúvida!

--- Imagine o senhor que se trata de fazer ouro\ldots{}

--- Como? O quê?, fez Bastos, arregalando os olhos.

--- Sim! Ouro!, disse, com firmeza, Flamel.

--- Como?

--- O senhor saberá, disse o químico secamente. A questão do momento são
as pessoas que devem assistir à experiência, não acha?

--- Com certeza, é preciso que os seus direitos fiquem resguardados,
porquanto\ldots{}

--- Uma delas, interrompeu o sábio, é o senhor; as outras duas, o Senhor
Bastos fará o favor de indicar"-me.

O boticário esteve um instante a pensar, passando em revista os seus
conhecimentos e, ao fim de uns três minutos, perguntou:

--- O Coronel Bentes lhe serve? Conhece?

--- Não. O senhor sabe que não me dou com ninguém aqui.

--- Posso garantir"-lhe que é homem sério, rico e muito discreto.

--- E religioso? Faço"-lhe esta pergunta, acrescentou Flamel logo, porque
temos que lidar com ossos de defunto e só estes servem\ldots{}

--- Qual! E quase ateu\ldots{}

--- Bem! Aceito. E o outro? Bastos voltou a pensar e dessa vez
demorou"-se um pouco mais consultando a sua memória\ldots{} Por fim, falou:

--- Será o Tenente Carvalhais, o coletor, conhece?

--- Como já lhe disse\ldots{}

--- E verdade. E homem de confiança, sério, mas\ldots{}

--- Que é que tem?

--- É maçom.\footnote{Maçom é o termo que identifica as pessoas que
  pertencem à maçonaria. Surgida na Europa, no período da Idade Média, a
  maçonaria por muito tempo foi considerada uma das organizações mais
  secretas do mundo. Seus membros não podem contar para os não maçons o
  conteúdo do que foi discutido nas reuniões e encontros. A maçonaria
  teve um papel importante nos bastidores da política desde pelo menos a
  Revolução Francesa, por contar com a participação de pessoas
  influentes neste meio. Um bom resumo das principais características da
  maçonaria pode ser encontrado a partir desse link:
  \emph{https://super.abril.com.br/historia/o"-que"-e"-e"-como"-surgiu"-a"-maconaria/}.}

--- Melhor.

--- E quando é?

--- Domingo. Domingo, os três irão lá em casa assistir à experiência e
espero que não me recusarão as suas firmas para autenticar a minha
descoberta.

--- Está tratado.

Domingo, conforme prometeram, as três pessoas respeitáveis de Tubiacanga
foram à casa de Flamel, e, dias depois, misteriosamente, ele desaparecia
sem deixar vestígios ou explicação para o seu desaparecimento.

\section*{Capítulo iii}

Tubiacanga era uma pequena cidade de três ou quatro mil habitantes,
muito pacífica, em cuja estação, de onde em onde, os expressos davam a
honra de parar. Há cinco anos não se registrava nela um furto ou roubo.
As portas e janelas só eram usadas\ldots{} porque o Rio as usava.

O único crime notado em seu pobre cadastro fora um assassinato por
ocasião das eleições municipais; mas, atendendo que o assassino era do
partido do governo, e a vítima da oposição, o acontecimento em nada
alterou os hábitos da cidade, continuando ela a exportar o seu café e a
mirar as suas casas baixas e acanhadas nas escassas águas do pequeno rio
que a batizara.

Mas, qual não foi a surpresa dos seus habitantes quando se veio a
verificar nela um dos repugnantes crimes de que se tem memória! Não se
tratava de um esquartejamento ou parricídio; não era o assassinato de
uma família inteira ou um assalto à coletoria; era cousa pior, sacrílega
aos olhos de todas as religiões e consciências: violavam"-se as
sepulturas do ``Sossego'', do seu cemitério, do seu campo"-santo.

Em começo, o coveiro julgou que fossem cães, mas, revistando bem o muro,
não encontrou senão pequenos buracos. Fechou"-os; foi inútil. No dia
seguinte, um jazigo perpétuo arrombado e os ossos saqueados; no outro,
um carneiro e uma sepultura rasa. Era gente ou demônio. O coveiro não
quis mais continuar as pesquisas por sua conta, foi ao subdelegado e a
notícia espalhou"-se pela cidade.

A indignação na cidade tomou todas as feições e todas as vontades. A
religião da morte precede todas e certamente será a última a morrer nas
consciências. Contra a profanação, clamaram os seis presbiterianos do
lugar --- os bíblicos, como lhes chama o povo; clamava o agrimensor
Nicolau, antigo cadete, e positivista do rito Teixeira Mendes\footnote{O
  ``Apostolado Positivista'' foi uma associação criada no Rio de Janeiro
  no final do século \textsc{xix} e que tinha como missão a difusão da filosofia
  positivista no Brasil. O positivismo foi um dos mais importantes
  movimentos do pensamento filosófico do início do século \textsc{xx} no
  Ocidente. Teve como seu fundador o filósofo francês Augusto Comte
  (1798--1857). No Brasil, o positivismo se desenvolveu principalmente
  no Rio de Janeiro e no Rio Grande do Sul. Seus principais expoentes
  foram Raimundo Teixeira Mendes (1855--1927), Miguel Lemos (1854--1917) e Benjamin Constant (1836--1891). A doutrina positivista teve
  forte influência nos meios políticos e militares no início do século
  \textsc{xx}. O lema do positivismo --- Ordem e Progresso --- acabou indo parar na
  bandeira do Brasil. Há um documentário sobre o positivismo no Brasil,
  chamado ``A última Religião'', disponível no YouTube através do link:
  \emph{https://www.youtube.com/watch?v=aHpG"-cr1eMg}.}; clamava o Major
Camanho, presidente da Loja Nova Esperança; clamavam o turco Miguel
Abudala, negociante de armarinho, e o céptico Belmiro, antigo estudante,
que vivia ao deus"-dará, bebericando parati nas tavernas.

A própria filha do engenheiro residente da estrada de ferro, que vivia
desdenhando aquele lugarejo, sem notar sequer os suspiros dos
apaixonados locais, sempre esperando que o expresso trouxesse um
príncipe a desposá"-la, a linda e desdenhosa Cora não pôde deixar de
compartilhar da indignação e do horror que tal ato provocara em todos do
lugarejo. Que tinha ela com o túmulo de antigos escravos e humildes
roceiros? Em que podia interessar aos seus lindos olhos pardos o destino
de tão humildes ossos? Porventura o furto deles perturbaria o seu sonho
de fazer radiar a beleza de sua boca, dos seus olhos e do seu busto nas
calçadas do Rio?

Decerto, não; mas era a Morte, a Morte implacável e onipotente, de que
ela também se sentia escrava, e que não deixaria um dia de levar a sua
linda caveirinha para a paz eterna do cemitério. Aí Cora queria os seus
ossos sossegados, quietos e comodamente descansando num caixão bem feito
e num túmulo seguro, depois de ter sido a sua carne encanto e prazer dos
vermes\ldots{}

O mais indignado, porém, era Pelino. O professor deitara artigo de
fundo, imprecando, bramindo, gritando: ``Na estória do crime, dizia ele,
já bastante rica de fatos repugnantes, como sejam: o esquartejamento de
Maria de Macedo, o estrangulamento dos irmãos Fuoco, não se registra um
que o seja tanto como o saque às sepulturas do `Sossego'\,''.

E a Vila vivia em sobressalto. Nas faces não se lia mais paz; os
negócios estavam paralisados; os namoros suspensos. Dias e dias por
sobre as casas pairavam nuvens negras e, à noite, todos ouviam ruídos,
gemidos, barulhos sobrenaturais\ldots{} Parecia que os mortos pediam
vingança\ldots{}

O saque, porém, continuava. Toda noite eram duas, três sepulturas
abertas e esvaziadas de seu fúnebre conteúdo. Toda a população resolveu
ir em massa guardar os ossos dos seus maiores. Foram cedo, mas, em
breve, cedendo à fadiga e ao sono, retirou"-se um, depois outro e, pela
madrugada, já não havia nenhum vigilante. Ainda nesse dia o coveiro
verificou que duas sepulturas tinham sido abertas e os ossos levados
para destino misterioso.

Organizaram então uma guarda. Dez homens decididos juraram perante o
subdelegado vigiar durante a noite a mansão dos mortos.

Nada houve de anormal na primeira noite, na segunda e na terceira; mas,
na quarta, quando os vigias já se dispunham a cochilar, um deles julgou
lobrigar um vulto esgueirando"-se por entre a quadra dos carneiros.
Correram e conseguiram apanhar dois dos vampiros. A raiva e a
indignação, até aí sopitadas no animo deles, não se contiveram mais e
deram tanta bordoada nos macabros ladrões, que os deixaram estendidos
como mortos.

A notícia correu logo de casa em casa e, quando, de manhã, se tratou de
estabelecer a identidade dos dois malfeitores, foi diante da população
inteira que foram neles reconhecidos o Coletor Carvalhais e o Coronel
Bentes, rico fazendeiro e presidente da Câmara. Este último ainda vivia
e, a perguntas repetidas que lhe fizeram, pôde dizer que juntava os
ossos para fazer ouro e o companheiro que fugira era o farmacêutico.

Houve espanto e houve esperanças. Como fazer ouro com ossos? Seria
possível? Mas aquele homem rico, respeitado, como desceria ao papel de
ladrão de mortos se a cousa não fosse verdade!

Se fosse possível fazer, se daqueles míseros despojos fúnebres se
pudesse fazer alguns contos de réis, como não seria bom para todos eles!

O carteiro, cujo velho sonho era a formatura do filho, viu logo ali
meios de consegui"-la. Castrioto, o escrivão do juiz de paz, que no ano
passado conseguiu comprar uma casa, mas ainda não a pudera cercar,
pensou no muro, que lhe devia proteger a horta e a criação. Pelos olhos
do sitiante Marques, que andava desde anos atrapalhado para arranjar um
pasto, pensou logo no prado verde do Costa, onde os seus bois
engordariam e ganhariam forças\ldots{}

Às necessidades de cada um, aqueles ossos que eram ouro viriam atender,
satisfazer e felicitá"-los; e aqueles dois ou três milhares de pessoas,
homens, crianças, mulheres, moços e velhos, como se fossem uma só
pessoa, correram à casa do farmacêutico.

A custo, o subdelegado pôde impedir que varejassem a botica e conseguir
que ficassem na praça, à espera do homem que tinha o segredo de todo um
Potosi\footnote{Referência à cidade de Potosí, na Bolívia, uma das mais
  antigas da América do Sul e que teve muita importância no período
  colonial daquele país, por conta de suas minas de prata.}. Ele não
tardou a aparecer. Trepado a uma cadeira, tendo na mão uma pequena barra
de ouro que reluzia ao forte sol da manhã, Bastos pediu graça,
prometendo que ensinaria o segredo, se lhe poupassem a vida. ``Queremos
já sabê"-lo,'' gritaram. Ele então explicou que era preciso redigir a
receita, indicar a marcha do processo, os reativos --- trabalho longo
que só poderia ser entregue impresso no dia seguinte. Houve um murmúrio,
alguns chegaram a gritar, mas o subdelegado falou e responsabilizou"-se
pelo resultado.

Docilmente, com aquela doçura particular às multidões furiosas, cada
qual se encaminhou para casa, tendo na cabeça um único pensamento:
arranjar imediatamente a maior porção de ossos de defunto que pudesse.

O sucesso chegou à casa do engenheiro residente da estrada de ferro. Ao
jantar, não se falou em outra coisa. O doutor concatenou o que ainda
sabia do seu curso, e afirmou que era impossível. Isto era alquimia,
coisa morta: ouro é ouro, corpo simples, e osso é osso, um composto,
fosfato de cal. Pensar que se podia fazer de uma coisa outra era
``besteira''. Cora aproveitou o caso para rir"-se petropolimente da
crueldade daqueles botocudos; mas sua mãe, Dona Emitia, tinha fé que a
coisa era possível.

À noite, porém, o doutor percebendo que a mulher dormia, saltou a janela
e correu em direitura ao cemitério; Cora, de pés nus, com as chinelas
nas mãos, procurou a criada para irem juntas à colheita de ossos. Não a
encontrou, foi sozinha; e Dona Emília, vendo"-se só, adivinhou o passeio
e lá foi também.

E assim aconteceu na cidade inteira.

O pai, sem dizer nada ao filho, saía; a mulher, julgando enganar o
marido, saía; os filhos, as filhas, os criados --- toda a população, sob
a luz das estrelas assombradas, correu ao satânico rendez"-vous\footnote{Palavra
  francesa que significa ``encontro''.} no ``Sossego''.

E ninguém faltou. O mais rico e o mais pobre lá estavam. Era o turco
Miguel, era o professor Pelino, o doutor Jerônimo, o Major Camanho,
Cora, a linda e deslumbrante Cora, com os seus lindos dedos de
alabastro, revolvia a sânie das sepulturas, arrancava as carnes, ainda
podres agarradas tenazmente aos ossos e deles enchia o seu regaço até
ali inútil. Era o dote que colhia e as suas narinas, que se abriam em
asas rosadas e quase transparentes, não sentiam o fétido dos tecidos
apodrecidos em lama fedorenta\ldots{}

A desinteligência não tardou a surgir; os mortos eram poucos e não
bastavam para satisfazer a fome dos vivos. Houve facadas, tiros,
cachações. Pelino esfaqueou o turco por causa de um fêmur e mesmo entre
as famílias questões surgiram.

Unicamente, o carteiro e o filho não brigaram. Andaram juntos e de
acordo e houve uma vez que o pequeno, uma esperta criança de onze anos,
até aconselhou ao pai: ``Papai vamos aonde está mamãe; ela era tão
gorda\ldots{}''

De manhã, o cemitério tinha mais mortos do que aqueles que recebera em
trinta anos de existência. Uma única pessoa lá não estivera, não matara
nem profanara sepulturas: fora o bêbedo Belmiro.

Entrando numa venda, meio aberta, e nela não encontrando ninguém,
enchera uma garrafa de parati e se deixara ficar a beber sentado na
margem do Tubiacanga, vendo escorrer mansamente as suas águas sobre o
áspero leito de granito --- ambos, ele e o rio, indiferentes ao que já
viram, mesmo à fuga do farmacêutico, com o seu Potosi e o seu segredo,
sob o dossel eterno das estrelas.



\chapter[Como o «homem» chegou]{Como o «homem» chegou\footnote[*]{Publicado em duas partes, nas edições de abril e março de 1915 da \emph{Revista Águia}.}}

\epigraph{Deus está morto; a sua piedade pelos homens matou"-o.}{\textsc{nietzsche}}

\section*{\textsc{i}}

\noindent{}A polícia da república, como toda a gente sabe, é paternal e compassiva
no tratamento das pessoas humildes que dela necessitam; e sempre, quer
se trate de humildes, quer de poderosos, a velha instituição cumpre
religiosamente a lei. Vem"-lhe daí o respeito que aos políticos os seus
empregados tributam e a procura que ela merece desses homens, quase
sempre interessados no cumprimento das leis que discutem e votam.

O caso que vamos narrar não chegou ao conhecimento do público,
certamente devido a pouca atenção que lhe deram os repórteres; e é pena,
pois, se assim não fosse, teriam nele encontrado pretexto para clichés
bem macabramente mortuários que alegrassem as páginas de suas folhas
volantes.

O delegado que funcionou na questão talvez não tivesse notado o grande
alcance de sua obra; e tanto isso é de admirar quanto as consequências
do fato concordam com luxuriantes sorites de um filósofo sempre capaz de
sugerir, do pé para a mão, novíssimas estéticas aos necessitados de
apresentá"-las ao público bem informado.

Sabedores de acontecimento de tal monta, não nos era possível deixar de
narrá"-lo com alguma minudência, para edificação dos delegados passados,
presentes e futuros.

Naquela manhã, tinha a delegacia um movimento desusado. Passavam"-se
semanas sem que houvesse uma simples prisão, uma pequena admoestação. A
circunscrição era pacata e ordeira. Pobre, não havia furtos; sem
comércio, não havia gatunos; sem indústria, não havia vagabundos, graças
à sua extensão e aos capoeirões que lá havia; os que não tinham
domicílio arranjavam"-no facilmente em chocas ligeiras sobre chãos de
outros donos mal conhecidos.

Os regulamentos policiais não encontravam emprego; os funcionários do
distrito viviam descansados e, sem desconfiança, olhavam a população do
lugarejo. Compunha"-se o destacamento de um cabo e três soldados; todos
os quatro, gente simples, esquecida de sua condição de sustentáculos do
Estado.

O comandante, um cabo gordo que falava arrastando a voz, com a cantante
preguiça de um carro de bois a chiar, habitava com a família um rancho
próximo e plantava ao redor melancias, colhendo"-as de polpa bem rosada e
doce, pelo verão inflexível da nossa terra. Um dos soldados tecia redes
de pescaria, chumbava"-as com cuidado para dar cerco às tainhas; e era de
vê"-las saltar por cima do fruto de sua indústria com a agilidade de
acrobatas, agilidade surpreendente naqueles entes sem mãos e pernas
diferenciadas. Um outro camarada matava o ócio pescando de caniço e
quase nunca pescava crocorocas, pois diante do mar, da sua infinita
grandeza, distraía"-se, lembrando"-se das quadrinhas que vinha compondo em
louvor de uma beleza local.

Tinham também os inspetores de polícia essa concepção idílica, e não se
aborreciam no morno vilarejo. Conceição, um deles, fabricava carvão e os
plantões os faziam junto às caieiras, bem protegidas por cruzes toscas
para que o tinhoso não entrasse nelas e fabricasse cinza em vez do
combustível das engomadeiras. Um seu colega, de nome Nunes, aborrecido
com o ar elísico daquela delegacia, imaginou quebrá"-lo e lançou o jogo
do bicho. Era uma cousa inocente: o mínimo da pule, um vintém; o máximo,
duzentos réis, mas, ao chegar a riqueza do lugar, aí pelo tempo do caju,
quando o sol saudoso da tarde dourava as areias e os frutos amarelos e
vermelhos mais se intumesciam nos cajueiros frágeis, jogavam"-se pules de
dez tostões.

Vivia tudo em paz; o delegado não aparecia. Se o fazia de mês em mês, de
semestre em semestre, de ano em ano, logo perguntava: houve alguma
prisão? Respondiam alvissareiros: não, doutor; e a fronte do doutor se
anuviava, como se sentisse naquele desuso do xadrez a morte próxima do
Estado, da Civilização e do Progresso.

De onde em onde, porém, havia um caso de defloramento e este era o
delito, o crime, a infração do lugarejo. Um crime, uma infração, um
delito muito próprio do Paraíso, que o tempo, porém, levou a ser julgado
pelos policiais, quando, nas primeiras eras das nossas origens bíblicas,
o fora pelo próprio Deus.

Em geral, os inspetores por eles mesmos resolviam o caso; davam paternos
conselhos suasórios e a lei sagrava o que já havia sido abençoado pelas
prateadas folhas das imbaúbas, nos capoeirões cerrados.

Não quis, porém, o delegado deixar que os seus subordinados liquidassem
aquele caso. A paciente era filha do Sambabaia, chefe político do
partido do Senador Melaço; e o agente era eleitor do partido contrário a
Melaço. O programa do partido de Melaço era não fazer coisa alguma e o
do contrário tinha o mesmo ideal; ambos, porém, se diziam adversários de
morte e essa oposição, refletindo"-se no caso, embaraçada sobremodo o
subdelegado.

Interrogado, confessara"-se o agente pronto a reparar o mal; e, desde há
muito, a paciente dera a tal respeito a sua indispensável opinião. A
autoridade, entretanto, hesitava, por causa da incompatibilidade
política do casal. As audiências se sucediam e aquela era já a quarta.
Estavam os soldados atônitos com tanta demora, provinda de não saber bem
o delegado se, unindo mais uma vez o par, não iria o caso desgostar
Melaço e mesmo o seu adversário Jati --- ambos Senadores poderosos,
aquele do governo e este da oposição; e, desgostar qualquer dele punha
em perigo o seu emprego porque, quase sempre entre nós, a oposição passa
a ser governo e o governo oposição instantaneamente. O consentimento dos
rapazes não bastava ao caso; era preciso, além, uma reconciliação ou uma
simples adesão política.

Naquela manhã, o delegado tomava mais uma vez o depoimento do agente,
inquirindo"-o desta forma:

--- Já se resolveu?

--- Pois não, doutor. Estou inteiramente a seu dispor\ldots{}

--- Não é bem ao meu. Quero saber se o senhor tem tenção?

--- De que, doutor? De casar? Pois não, doutor.

--- Não é de casar\ldots{} Isto já sei\ldots{} E\ldots{}

--- Mas de que deve ser então, doutor?

--- De entrar para o partido do doutor Melaço.

--- Eu sempre, doutor, fui pelo doutor Jati. Não posso\ldots{}

--- Que tem uma coisa com a outra? O senhor divide o seu voto: a metade
dá para um e a outra metade para outro. Está aí!

--- Mas como?

--- Ora! O senhor saberá arranjar as coisas da melhor forma; e, se o
fizer com habilidade, ficarei contente e o senhor será feliz, porquanto
pode arranjar tanto com um como com outro, conforme andar a política no
próximo quatriênio, um lugar de guarda dos mangues.

--- Não há vaga, doutor.

--- Qual! Há sempre vaga, meu caro. O Felizardo não se tem querido
alistar, não nasceu aqui, é de fora, é ``estrangeiro''; e, dessa maneira,
não pode continuar a fiscalizar os mangues. E vaga certa. O senhor adere
ou antes: divide a votação?

--- Divido então\ldots{}

Por aí, um dos inspetores veio avisar de que o guarda civil de nome Hane
lhe queria falar. O doutor Cunsono estremeceu. Era cousa do chefe, do
geral lá de baixo; e, de relance, viu o seu hábil trabalho de harmonizar
Jati e Melaço perdido inteiramente, talvez por causa de não ter, naquele
ano, efetuado sequer uma prisão. Estava na rua, suspendeu o
interrogatório e veio receber o visitador com muita angústia no coração.
Que seria?

--- Doutor, foi logo dizendo o guarda, temos um louco.

Diante daquele caso novo, o delegado quis refletir, mas logo o guarda
emendou:

--- O doutor Silly\ldots{}

Era assim o nome do ajudante do geral inacessível; e dele, os delegados
têm mais medo do que do chefe supremo todo poderoso. Hane continuou:

--- O doutor Silly mandou dizer que o senhor o prendesse e o enviasse à
Central.

Cunsono pensou bem que esse negócio de reclusão de loucos é por demais
grave e delicado e não era propriamente da sua competência fazê"-lo, a
menos que fossem sem eira nem beira ou ameaçassem a segurança pública.
Pediu a Hane que o esperasse e foi consultar o escrivão. Este
serventuário vivia ali de mau humor. O sossego da delegacia o aborrecia,
não porque gostasse da agitação pela agitação, mas pelo simples fato de
não perceber emolumentos ou quer que seja, tendo que viver de seus
vencimentos. Aconselhou"-se com ele o delegado e ficou perfeitamente
informado do que dispunham a lei e a praxe. Mas Silly\ldots{}

Voltando à sala, o guarda reiterou as ordens do auxiliar, contando
também que o louco estava em Manaus. Se o próprio Silly não o mandava
buscar, elucidou o guarda, era porque competia a Cunsono deter o
``homem'', porquanto a sua delegacia tinha costas do oceano e de Manaus se
vinha por mar.

--- É muito longe, objetou o delegado.

O guarda teve o cuidado de explicar que Silly já vira a distancia no
mapa e era bem reduzida: obra de palmo e meio. Cunsono perguntou ainda:

--- Qual a profissão do ``homem''?

--- É empregado da delegacia fiscal.

--- Tem pai?

--- Tem.

Pensou o delegado que competia ao pai o pedido de internação, mas o
guarda adivinhou"-lhe o pensamento e afirmou:

--- Eu conheço muito e meu primo é cunhado dele.

Estava já Cunsono irritado com as objeções do escrivão e desejava servir
a Silly, tanto mais que o caso desafiava a sua competência policial. A
lei era ele; e mandou fazer o expediente.

Após o que, tratou Cunsono de ultimar o enlace de Melaço e Jati, por
intermédio do casamento da filha do Sambabaia. Tudo ficou assentado da
melhor forma; e, em pequena hora, voltava o delegado para as ruas onde
não policiava, satisfeito consigo mesmo e com a sua tríplice obra, pois
não convém esquecer a sua caridosa intervenção no caso do louco de
Manaus.

Tomava a condução que devia trazer à cidade, quando a lembrança do meio
de transporte do dementado lhe foi presente. Ao guarda"-civil, ao
representante de Silly na zona, perguntou por esse instante:

--- Como há de vir o ``sujeito''?

O guarda, sem atender diretamente à pergunta, disse:

--- É\ldots{} É, doutor; ele está muito furioso.

Cunsono pensou um instante, lembrou"-se dos seus estudos e acudiu:

--- Talvez um couraçado\ldots{} O ``Minas Gerais''\footnote{Referência ao navio
  de guerra ``Minas Gerais'' que, junto com outro, de nome ``São
  Paulo'', eram as duas embarcações com mais poder militar com que
  contava a Marinha do Brasil na época.} não serve? Vou requisitá"-lo.

Hane, que tinha prática do serviço e conhecimento dos compassivos
processos policiais, refletiu:

--- Doutor: não é preciso tanto. O ``carro"-forte'' basta para trazer o
``homem''.

Concordou Cunsono e olhou as alturas um instante sem notar as nuvens que
vagavam sem rumo certo, entre o céu e a terra.

\section*{\textsc{ii}}

Silly, o doutor Silly, bem como Cunsono, graças à prática que tinham do
oficio, dispunham da liberdade dos seus pares com a maior facilidade.
Tinham substituído os graves exames íntimos provocados pelos deveres de
seus cargos, as perigosas responsabilidades que lhes são próprias, pelo
automático ato de uma assinatura rápida. Era um contínuo trazer um
oficio, logo, sem bem pensar no que faziam, sem lê"-lo até, assinavam e
ia com essa assinatura um sujeito para a cadeia, onde ficava aguardando
que se lembrasse de retirá"-lo de lá a sua mão distraída e ligeira.

Assim era; e foi sem dificuldade que atendeu ao pedido de Cunsono no que
toca ao carro"-forte. Prontamente deu as ordens para que fosse fornecida
a seu colega a masmorra ambulante, pior do que masmorra, do que
solitária, pois nessas prisões sente"-se ainda a algidez da pedra, alguma
coisa ainda de meiguice de sepultura, mas ainda assim meiguice; mas, no
tal carro feroz, é tudo ferro, há inexorável antipatia do ferro na
cabeça, ferro nos pés, aos lados uma igaçaba de ferro em que se vem
sentado, imóvel, e para a qual se entra pelo próprio pé. É blindada e
quem vai nela, levado aos trancos e barrancos de seu respeitável peso e
do calçamento das vias públicas, tem a impressão de que se lhe quer
poupar a morte por um bombardeio de grossa artilharia para ser empalado
aos olhos de um sultão. Um requinte de potentado asiático.

Essa prisão de Calistenes\footnote{Referência a Calístenes de Olinto,
  que viveu aproximadamente entre os anos 360 a.C. a 328 a.C. Foi um
  historiador grego, sobrinho do filósofo Aristóteles. Alguns estudos
  apontam um fim trágico para Calístenes, que foi condenado à prisão por
  não ter se ajoelhado diante da presença do imperador Alexandre, o
  Grande. Algumas versões da história sustentam que sua morte tenha
  ocorrido por fome, tão severa se tornou a sua pena.}, blindada,
chapeada, couraçada, foi posta em movimento; e saiu, abalando o
calçamento, a chocalhar ferragens, a trovejar pelas ruas afora em busca
de um inofensivo.

O ``homem'', como dizem eles, era um ente pacato, lá dos confins de
Manaus, que tinha a mania da Astronomia e abandonara, não de todo, mas
quase totalmente, a terra pelo céu inacessível. Vivia com o pai velho
nos arrabaldes da cidade e construíra na chácara de sua residência um
pequeno observatório, onde montou lunetas que lhe davam pasto à inocente
mania. Julgando insuficientes o olhar e as lentes, para chegar ao
perfeito conhecimento da Aldebarã longínqua, atirou"-se ao cálculo, à
inteligência pura, à Matemática e a estudar com afinco e fúria de um
doido ou de um gênio.

Em uma terra inteiramente entregue à chatinagem e à veniaga, Fernando
foi tomando a fama de louco, e não era ela sem algum motivo. Certos
gestos, certas despreocupações e mesmo outras manifestações mais
palpáveis pareciam justificar o julgamento comum; entretanto, ele vivia
bem com o pai e cumpria os seus deveres razoavelmente. Porém, parentes
oficiosos e outros longínquos aderentes entenderam curá"-lo, como se se
curassem assomos de alma e anseios de pensamento.

Não lhes vinha tal propósito de perversidade inata, mas de estultice
congênita, juntamente com a comiseração explicável em parentes. Julgavam
que o ser descompassado envergonhava a família e esse julgamento era
reforçado pelos cochichos que ouviam de alguns homens esforçados por
parecerem inteligentes.

O mais célebre deles era o doutor Barrado, um catita do lugar, cheiroso
e apurado no corte das calças. Possuía esse doutor a obsessão das coisas
extraordinárias, transcendentes, sem"-par, originais; e, como sabia
Fernando simples e desdenhoso pelos mandões, supôs que ele, com esse
procedimento, censurava Barrado por demais mesureiro com os magnates.
Começou, então, Barrado a dizer que Fernando não sabia astronomia; ora,
este último não afirmava semelhante coisa. Lia, estudava e contava o que
lia, mais ou menos o que aquele fazia nas salas, com os ditos e opiniões
dos outros.

Houve quem o desmentisse; teimava, no entanto, Barrado no propósito.
Entendeu também de estudar uma astronomia e bem oposta à de Fernando: a
astronomia do centro da Terra. O seu compêndio favorito era \emph{A
morgadinha} \emph{de Valflor}\footnote{``A morgadinha de Valflor'' é uma
  peça de teatro criada pelo escritor português Manuel Joaquim Pinheiro
  Chagas (1841--1845). Sua estreia ocorreu em abril de 1869 e logo se
  tornou um grande sucesso de público.} e os livros auxiliares: \emph{A
dama de Monsoreau}\footnote{Referência ao romance do escritor francês
  Alexandre Dumas (1802--1860), lançado em 1846.} e \emph{O rei dos
grilhetas}, numa biblioteca de Herschell. \footnote{Referência ao
  romance escrito pelo francês François"-Pierre"-Ernest Capendu (1825--1868), datado de 1861. Capendu, ao lado de Alexandre Dumas, foi um dos
  escritores mais populares de sua época até meados do século \textsc{xx}. Seus
  livros eram muito publicados no Brasil e alcançavam bastante sucesso.
  Lima Barreto usa sua ironia nesta passagem, ao apontar que os estudos
  de astronomia do doutor Barrado eram adquiridos em obras romanescas
  populares.}

Com isto, e cantando, e espalhando que Fernando vivia nas tascas com
vagabundos, auxiliado pelo poeta Machino, o jornalista Cosmético e o
antropologista Tucolas, que fazia sábias mensurações nos crânios das
formigas, conseguiu remover os simplórios parentes de Fernando, e foi
bastante que, de parente para conhecido, de conhecido para Hane, de
Hane, para Silly e Cunsono, as coisas se encadeassem e fosse obtida a
ordem de partida daquela fortaleza couraçada, roncando pelas ruas,
chocalhando ferragens, abalando calçadas, para ponto tão longínquo.

Quando, porém, o carro chegou à praça mais próxima, foi que o cocheiro
lembrou"-se de que não lhe tinham ensinado onde ficava Manaus. Voltou e
Silly, com a energia de sua origem britânica, determinou que fretassem
uma falua e fossem a reboque do primeiro paquete.

Sabedor do caso e como tivesse conhecimento de que Fernando era desafeto
do poderoso chefe político Sofonias, Barrado que, desde muito, lhe
queria ser agradável, calou o seu despeito, apresentou"-se pronto para
auxiliar a diligência. Esse chefe político dispunha de um prestígio
imenso e nada entendia de astronomia; mas, naquele tempo, era a ciência
da moda e tinham em grande consideração os membros da Sociedade
Astronômica, da qual Barrado queria fazer parte.

Sofonias influía nas eleições da Sociedade, como em todas as outras, e
podia determinar que Barrado fosse escolhido. Andava, portanto, o doutor
captando a boa vontade da potente influência eleitoral, esperando obter,
depois de eleito, o lugar de Diretor Geral das Estrelas de Segunda
Grandeza.

Não é de estranhar, pois, que aceitasse tão árdua incumbência e, com
Hane e carrião, veio até à praia; mas não havia canoa, caíque, bote,
jangada, catraia, chalana, falua, lancha, calunga, poveiro, peru,
macacuano, pontão, alvarenga, saveiro que os quisesse levar a tais
alturas.

Hane desesperava, mas o companheiro, lembrando"-se dos seus conhecimentos
de Astronomia, indicou um alvitre:

--- O carro pode ir boiando.

--- Como, doutor? É de ferro\ldots{} muito pesado, doutor!

--- Qual o quê! O ``Minas'', o ``Aragón'', o ``São Paulo'' não boiam?
Ele vai, sim!

--- E os burros?

--- Irão a nadar, rebocando o carro.

Curvou"-se o guarda diante do saber do doutor e deixou"-lhe a missão
confiada, conforme as ordens terminantes que recebera.

A calistênica entrou pela água adentro, consoante as ordens promanadas
do saber de Barrado e, logo que achou água suficiente, foi ao fundo com
grande desprezo pela hidrostática do doutor. Os burros, que tinham
sempre protestado contra a física do jovem sábio, partiram os arreios e
salvaram"-se; e graças a uma poderosa cábrea, pôde a almanjarra ser salva
também.

Havia poucos paquetes para Manaus e o tempo urgia. Barrado tinha ordem
franca de fazer o que quisesse. Não hesitou e, energicamente, fez
reparar as avarias e tratou de embarcar num paquete todo o trem, fosse
como fosse.

Ao embarcá"-lo, porém, surgiu uma dúvida entre ele e o pessoal de bordo.
Teimava Barrado que o carro merecia ir para um camarote de primeira
classe, teimavam os marítimos que isso não era próprio, tanto mais que
ele não indicava o lagar dos burros. Era difícil essa questão da
colocação dos burros.

Os homens de bordo queriam que fossem para o interior do navio; mas,
objetava o doutor:

--- Morrem asfixiados, tanto mais que são burros e mesmo por isso.

De comum acordo, resolveram telegrafar a Silly para resolver a curiosa
contenda. Não tardou viesse a resposta, que foi clara e precisa:
``Burros sempre em cima. Silly''.

Opinião como esta, tão sábia e tão verdadeira, tão cheia de filosofia e
sagacidade da vida, aliviou todos os corações e abraços fraternais foram
trocados entre conhecidos e inimigos, entre amigos e desconhecidos.

A sentença era de Salomão e houve mesmo quem quisesse aproveitar o
apotegma para construir uma nova ordem social.

Restava a pequena dificuldade de fazer entrar o carro para o camarote do
doutor Barrado. O convés foi aberto convenientemente, teve a sala de
jantar mesas arrancadas e o bendengó ficou no centro dela, em exposição,
feio e brutal, estúpido e inútil, como um monstro de museu.

O paquete moveu"-se lentamente em demanda da barra. Antes fez uma doce
curva, longa, muito suave, lentamente, como se, ao despedir"-se,
cumprimentasse reverente a beleza da Guanabara. As gaivotas voavam
tranquilas, cansavam"-se, pousavam na água --- não precisavam de terra\ldots{}

A cidade sumia"-se vagarosamente e o carro foi atraindo a atenção de
bordo.

--- O que vem a ser isto?

Diante da almanjarra, muitos viajantes murmuravam protestos contra a
presença daquele estafermo ali; outras pessoas diziam que se destinava a
encarcerar um bandoleiro da Paraíba; outras que era um salva"-vidas; mas,
quando alguém disse que aquilo ia acompanhando um recomendado de
Sofonias, a admiração foi geral e imprecisa.

Um oficial disse:

--- Que construção engenhosa!

Um médico afirmou:

--- Que linhas elegantes!

Um advogado refletiu:

--- Que soberba criação mental!

Um literato sustentou:

--- Parece um mármore de Fídias\footnote{Referência a Fídias, um dos
  maiores escultores da Grécia antiga.}!

Um sicofanta berrou:

--- É obra mesmo de Sofonias! Que republicano!

Uma moça adiantou:

--- Deve ter sons magníficos!

Houve mesmo escala para dar ração aos burros, pois os mais graduados se
disputavam a honraria. Um criado, porém, por ter passado junto ao
monstro e o olhado com desdém, quase foi duramente castigado pelos
passageiros. O ergástulo ambulante vingou"-se do serviçal; durante todo o
trajeto perturbou"-lhe o serviço.

Apesar de ir correndo a viagem sem mais incidentes, quis ao meio dela
Barrado desembarcar e continuá"-la por terra. Consultou, nestes termos,
Silly: ``Melhor carro ir terra faltam três dedos mar alonga caminho''; e
a resposta veio depois de alguns dias: ``Não convém desembarque embora
mais curto carro chega sujo. Siga''.

Obedeceu e o meteorito, durante duas semanas, foi objeto da adoração do
paquete. Nos últimos dias, quando um qualquer dos passageiros dele se
acercava, passava"-lhe pelo dorso negro a mão espalmada com a contrição
religiosa de um maometano ao tocar na pedra negra da Caaba.

Sofonias, que nada tinha com o caso, não teve nunca notícia dessa
tocante adoração.

\section*{\textsc{iii}}

Muito rica é Manaus, mas, como em todo o Amazonas, nela é vulgar a moeda
de cobre. É um singular traço de riqueza que muito impressiona o
viajante, tanto mais que não se quer outra e as rendas do Estado são
avultadas. O Eldorado não conhece o ouro, nem o estima.

Outro traço de sua riqueza é o jogo. Lá, não é divertimento nem vício: é
para quase todos profissão. O valor dos noivos, segundo dizem, é
avaliado pela média das paradas felizes que fazem, e o das noivas pelo
mesmo processo no tocante aos pais.

Chegou o navio a tão curiosa cidade quinze dias após fazendo uma plácida
viagem, com o fetiche a bordo. Desembarcá"-lo foi motivo de absorvente
cogitação para o doutor Barrado. Temia que fosse de novo ao fundo, não
porque o quisesse encaminhá"-lo por sobre as águas do rio Negro; mas,
pelo simples motivo de que, sendo o cais flutuante, o peso do carrião
talvez trouxesse desastrosas consequências para ambos, cais e carro.

O capataz não encontrava perigo algum, pois desembarcavam e embarcavam
pelos flutuantes volumes pesadíssimos, toneladas até.

Barrado, porém, que era observador, lembrava"-se da aventura do rio, e
objetou:

--- Mas não são de ferro.

--- Que tem isso? fez o capataz.

Barrado, que era observador e inteligente, afinal compreendeu que um
quilo de ferro pesa tanto quanto um de algodão; e só se convenceu
inteiramente disso, como observador que era, quando viu o ergástulo em
salvamento, rolando pelas ruas da cidade.

Continuou a ser ídolo e o doutor agastou"-se deveras porque o governador
visitou a caranguejola, antes que ele o fizesse. Como não tivesse
completas as instruções para detenção de Fernando, pediu"-as a Silly. A
resposta veio num longo telegrama, minucioso e elucidativo. Devia
requisitar força ao governador, arregimentar capangas e não desprezar as
balas de alteia. Assim fez o comissário. Pediu uma companhia de
soldados, foi às alfurjas da cidade catar bravos e adquirir uma
confeitaria de alteia. Partiu em demanda do ``homem'' com esse trem de
guerra; e, pondo"-se cautelosamente em observação, lobrigou os óculos do
observatório, donde concluiu que a sua força era insuficiente. Normas
para o seu procedimento requereu a Silly. Vieram secas e peremptórias:
``Empregue também artilharia''.

De novo pôs"-se em marcha com um parque do Krupp\footnote{Referência ao
  canhão Krupp, muito utilizado pelas forças armadas no período.}.
Desgraçadamente, não encontrou o homem perigoso. Recolheu a expedição a
quartéis; e, certo dia, quando de passeio, por acaso, foi parar a um
café do centro comercial. Todas as mesas estavam ocupadas; e só em uma
delas havia um único consumidor. A esta ele sentou"-se. Travou por
qualquer motivo conversa com o mazombo; e, durante alguns minutos,
aprendeu com o solitário alguma coisa.

Ao despedirem"-se, foi que ligou o nome à pessoa, e ficou atarantado sem
saber como proceder no momento. A ação, porém, lhe veio prontamente; e,
sem dificuldade, falando em nome da lei e da autoridade, deteve o
pacífico ferrabrás em um dos bailéus do cárcere ambulante.

Não havia paquete naquele dia e Silly havia recomendado que o trouxessem
imediatamente. ``Venha por terra'', disse ele; e Barrado, lembrado do
conselho, tratou de segui"-lo. Procurou quem o guiasse até ao Rio, embora
lhe parecesse curta e fácil a viagem. Examinou bem o mapa e, vendo que a
distância era de palmo e meio, considerou que dentro dela não lhe cabia
o carro. Por este e aquele, soube que os fabricantes de mapas não têm
critério seguro: era fazer uns muito grandes, ou muito pequenos,
conforme são para enfeitar livros ou adornar paredes. Sendo assim, a tal
distância de doze polegadas bem podia esconder viagem de um dia e mais.

Aconselhado pelo cocheiro, tomou um guia e encontrou"-o no seu antigo
conhecido Tucolas, sabedor como ninguém do interior do Brasil, pois o
palmilhara à cata de formigas para bem firmar documentos às suas
investigações antropológicas.

Aceitou a incumbência o curioso antropologista de himenópteros,
aconselhando, entretanto, a modificação do itinerário.

--- Não me parece, senhor Barrado, que devamos atravessar o Amazonas.
Melhor seria, senhor Barrado, irmos até a Venezuela, alcançar as Guianas
e descermos, senhor Barrado.

--- Não teremos rios a atravessar, Tucolas?

--- Homem! Meu caro senhor, eu não sei bem; mas, senhor Barrado me
parece que não, e sabe por quê?

--- Por quê?

--- Por quê? Porque este Amazonas, senhor Barrado, não pode ir até lá,
ao Norte, pois só corre de oeste para leste\ldots{}

Discutiram assim sabiamente o caminho; e, à proporção que manifestava o
seu profundo trato com a geografia da América do Sul, mais Tucolas
passava a mão pela cabeleira de inspirado.

Achou que os conselhos do doutor eram justos, mas temia as surpresas do
carrião. Ora, ia ao fundo, por ser pesado; ora, sendo pesado, não fazia
ir ao fundo frágeis flutuantes. Não fosse ele estranhar o chão
estrangeiro e pregar"-lhe alguma peça? O cocheiro não queria também ir
pela Venezuela, temia pisar em terra de gringos e encarregou"-se da
travessia do Amazonas --- o que foi feito em paz e salvamento, com a
máxima simplicidade.

Logo que foi ultimada, Tucolas tratou de guiar a caravana. Prometeu que
o faria com muito acerto e contentamento geral, pois aproveitá"-la"-ia,
dilatando as suas pesquisas antropológicas aos moluscos dos nossos rios.
Era sábio naturalista, e antropologista, e etnografista da novíssima
escola do conde de Gobineau\footnote{Arthur de Gobineau (1816--1882)
  foi um escritor e filósofo francês. Ganhou notoriedade após publicar,
  em 1855, o livro ``Ensaio sobre a desigualdade das raças humanas'',
  que defendia a tese segundo a qual a miscigenação, ou seja, a mistura
  de raças, era responsável pela degeneração de um povo, tanto física
  quanto intelectualmente. É considerado um precursor do racismo
  científico. Sua teoria ganhou muitos adeptos no Brasil e Lima Barreto
  foi um grande opositor a estas ideias.}, novidade de uns sessenta anos
atrás; e, desde muito, desejava fazer uma viagem daquelas para completar
os seus estudos antropológicos nas formigas e nas ostras dos nossos
rios.

A viagem correu maravilhosamente durante as primeiras horas. Sob um sol
de fogo, o carro solavancava pelos maus caminhos; e o doente, à míngua
de não ter onde se agarrar, ia ao encontro de uma e outra parede de sua
prisão couraçada. Os burros, impelidos pelas violentas oscilações dos
varais, encontravam"-se e repeliam"-se, ainda mais aumentando os ásperos
solavancos da traquitana; e o cocheiro, na boleia, oscilava de lá para
cá, de cá para lá, marcando o compasso da música chocalhante daquela
marcha vagarosa.

Na primeira venda que passaram, uma dessas vendas perdidas, quase
isoladas, dos caminhos desertos, onde o viajante se abastece e os
vagabundos descansam de sua errância pelos descambados e montanhas, o
encarcerado foi saudado com uma vaia: ó maluco! ó maluco!

Andava Tucolas distraído a fossar e cavoucar, catando formigas; e, mal
encontrava uma mais assim, logo examinava bem o crânio do inseto,
procurava"-lhe os ossos componentes, enquanto não fazia uma mensuração
cuidadosa do ângulo de Camper ou mesmo de Cloquet\footnote{Ângulo de
  Camper ou Plano de Camper é uma medida anatômica realizada no crânio
  humano, que mede desde a borda inferior das narinas até o lóbulo da
  orelha correspondente ao mesmo lado da face. Já o ângulo de Cloquet é
  a medida que vai desde a ponta do lábio superior até o lóbulo da
  orelha. As teorias do racismo científico se baseavam nas medidas dos
  crânios das ``diferentes raças'' que supostamente constituíam os seres
  humanos. A partir do formato do crânio era possível ``medir'' o nível
  de desenvolvimento de um ser humano, tanto físico quanto mental. De
  acordo com tais teorias, a raça negra seria aquela mais degenerada e
  inferior, e a raça ariana a mais evoluída e superior. A mestiçagem era
  responsável pela degeneração, o que resultaria em maior propensão para
  o crime, a delinquência, o vício em álcool ou em jogos. Essa técnica
  era conhecida como \emph{antropomorfia} e foi muito utilizada na
  medicina criminal e psiquiátrica do período compreendido entre a
  segunda metade do século \textsc{xix} até as primeiras décadas do século \textsc{xx}.
  Mais uma vez, Lima Barreto está fazendo uma crítica às teorias do
  racismo científico.}. Barrado, cuja preocupação era ser êmulo do Padre
Vieira\footnote{Referência ao padre Antônio Vieira (1608--1697), que
  viveu a maior parte de sua vida no Brasil, onde foi membro da
  Companhia de Jesus e atuou na catequização dos indígenas. Além de suas
  funções como padre, Vieira se destacou como grande orador, conhecedor
  profundo e divulgador da língua portuguesa clássica no Brasil. Por
  isso, mais uma vez, a ironia de Lima Barreto ao apresentar o doutor
  Barrado como êmulo (rival) do Padre Vieira.}, aproveitara o tempo para
firmar bem as regras de colocação de pronomes, sobretudo a que manda que
o ``que'' atraia o pronome complemento.

E assim andando foi o carro, após dias de viagem, até chegar a uma
aldeia pobre, à margem de um rio, onde chalanas e naviecos a vapor
tocavam de quando em quando.

Cuidaram imediatamente de obter hospedagem e alimentação no lugarejo. O
cocheiro lembrou o ``homem'' que traziam. Barrado, a respeito, não tinha
com segurança uma norma de proceder. Não sabia mesmo se essa espécie de
doentes comia e consultou Silly, por telegrama. Respondeu"-lhe a
autoridade, com a energia britânica que tinha no sangue, que não era do
regulamento retirar aquela espécie de enfermos do carro, o ``ar'' sempre
lhes fazia mal. De resto, era curta a viagem e tão sábia recomendação
foi cegamente obedecida.

Em pequena hora, Barrado e o guia sentavam"-se à mesa do professor
público, que lhes oferecera de jantar. O ágape ia fraternal e alegre,
quando houve a visita da Discórdia\footnote{Nesse sentido, Lima Barreto
  faz referência à Discórdia, como deusa, que na mitologia grega é
  conhecida por Éris.}, a visita da Gramática.

O ingênuo professor não tinha conhecimento do pichoso saber gramatical
do doutor Barrado e expunha candidamente os usos e costumes do lugar com
a sua linguagem roceira:

--- Há aqui entre nós muito pouco caso pelo estudo, doutor. Meus filhos
mesmo e todos quase não querem saber de livros. Tirante este defeito,
doutor, a gente quer mesmo o progresso.

Barrado implicou com o ``tirante'' e o ``a gente'', e tentou ironizar.
Sorriu e observou:

--- Fala"-se mal, estou vendo.

O matuto percebeu que o doutor se referia a ele. Indagou mansamente:

--- Por que o doutor diz isso?

--- Por nada, professor. Por nada!

--- Creio, aduziu o sertanejo, que, tirante eu, o doutor aqui não falou
com mais ninguém.

Barrado notou ainda o ``tirante'' e olhou com inteligência para Tucolas,
que se distraía com um naco de tartaruga.

Observou o caipira, momentaneamente, o afã de comer do antropologista e
disse, meigamente:

--- Aqui, a gente come muito isso. Tirante a caça e a pesca, nós
raramente temos carne fresca.

A insistência do professor sertanejo irritava sobremaneira o doutor
inigualável. Sempre aquele ``tirante'', sempre o tal ``a gente, a gente,
a gente'' --- um falar de preto mina!

O professor, porém, continuou a informar calmamente:

--- A gente aqui planta pouco, mesmo não vale a pena. Felizardo do
Catolé plantou uns leirões de horta, há anos, e quando veio o calor e a
enchente\ldots{}

--- É demais! É demais! exclamou Barrado.

Docemente, o pedagogo indagou:

--- Por quê? Por quê, doutor?

Estava o doutor sinistramente raivoso e explicou"-se a custo:

--- Então, não sabe? Não sabe?

--- Não, doutor. Eu não sei, fez o professor, com segurança e
mansuetude.

Tucolas tinha parado de saborear a tartaruga, a fim de atinar com a
origem da disputa.

--- Não sabe, então, rematou Barrado, não sabe que até agora o senhor
não tem feito outra coisa senão errar em português?

--- Como, doutor?

--- É ``tirante'', é ``a gente, a gente, a gente''; e, por cima de tudo,
um solecismo!

--- Onde, doutor?

--- Veio o calor e a chuva --- é português?

--- É, doutor, é, doutor!

Veja o doutor João Ribeiro\footnote{Sobre João Ribeiro, ver nota 62.}!
Tudo isso está lá. Quer ver?

O professor levantou"-se, apanhou sobre a mesa próxima uma velha
gramática ensebada e mostrou a respeitável autoridade ao sábio doutor
Barrado. Sem saber desdéns simular, ordenou:

--- Tucolas, vamo"-nos embora.

--- E a tartaruga? diz o outro.

O hóspede ofereceu"-a, o original antropologista embrulhou"-a e saiu com o
companheiro. Cá fora, tudo era silêncio e o céu estava negro. As
estrelas pequeninas piscavam sem cessar o seu olhar eterno para a Terra
muito grande. O doutor foi ao encontro da curiosidade recalcada de
Tucolas:

--- Vê, Tucolas, como anda o nosso ensino? Os professores não sabem os
elementos de gramática, e falam como negros de senzala.

--- Senhor Barrado, julgo que o senhor deve a esse respeito chamar a
atenção do ministro competente, pois me parece que o país, atualmente,
possui um dos mais autorizados na matéria.

--- Vou tratar, Tucolas, tanto mais que o Semicas é amigo do Sofonias.

--- Senhor Barrado, uma coisa\ldots{}

--- Que é?

--- Já falou, senhor Barrado, a meu respeito com o senhor Sofonias?

--- Desde muito, meu caro Tucolas. Está à espera da reforma do museu e
tu vais para lá direitinho. É o teu lugar.

--- Obrigado, senhor Barrado. Obrigado.

A viagem continuou monotonamente. Transmontaram serras, vadearam rios e,
num deles, houve um ataque de jacarés, dos quais se salvou Barrado
graças à sua pele muito dura. Entretanto, um dos animais de tiro perdeu
uma das patas dianteiras e mesmo assim conseguiu pôr"-se a salvo na
margem oposta.

Sarou"-lhe a ferida não se sabe como e o animal não deixou de acompanhar
a caravana. Às vezes, distanciava"-se; às vezes, aproximava"-se; e sempre
a pobre alimária olhava longamente, demoradamente, aquele forno
ambulante, manquejando sempre, impotente para a carreira, e como se se
lastimasse de não poder auxiliar eficazmente o lento reboque daquela
almanjarra pesadona.

Em dado momento, o cocheiro avisa Barrado de que o ``homem'' parecia
estar morto; havia até um mau cheiro indicador. O regulamento não
permitia a abertura da prisão e o doutor não quis verificar o que havia
de verdade no caso. Comia aqui, dormia ali, Tucolas também e os burros
também --- que mais era preciso para ser agradável a Sofonias? Nada, ou
antes: trazer o ``homem'' até ao Rio de Janeiro. As doze polegadas da
sua cartografia desdobravam"-se em um infinito número de quilômetros.
Tucolas que conhecia o caminho, dizia sempre: estamos a chegar, senhor
Barrado! Estamos a chegar! Assim levaram meses andando, com o burro
aleijado a manquejar atrás do ergástulo ambulante, olhando"-o docemente,
cheio de piedade impotente.

Os urubus crocitavam por sobre a caravana, estreitavam o voo, desciam
mais, mais, mais, até quase debicar no carro"-forte. Barrado punha"-se
furioso a enxotá"-los a pedradas; Tucolas imaginava aparelhos para
examinar a caixa craniana das ostras de que andava à caça; o cocheiro
obedecia.

Mais ou menos assim, levaram dois anos e foram chegar à aldeia dos
Serradores, margem do Tocantins.

Quando aportaram, havia na praça principal uma grande disputa, tendo por
motivo o preenchimento de uma vaga na Academia dos Lambrequins\footnote{Lambrequim
  normalmente é um item decorativo utilizado em beirais e telhados.
  Assume o aspecto de uma franja, se visto por fora das residências.
  Mais uma ironia de Lima Barreto, mostrando que o doutor Barrado é
  fortemente atraído por alguma disputa, não importa qual, desde que
  resulte uma vaga em alguma ``Academia''.}.

Logo que Barrado soube do que se tratava, meteu"-se na disputa e foi
gritando lá a seu jeito e sacudindo as perninhas:

--- Eu também sou candidato! Eu também sou candidato!

Um dos circunstantes perguntou"-lhe a tempo, com toda a paciência:

--- Moço: o senhor sabe fazer lambrequins?

--- Não sei, não sei, mas aprendo na academia e é para isso que quero
entrar.

A eleição teve lugar e a escolha recaiu sobre um outro mais hábil no uso
da serra que o doutor recém"-chegado.

Precipitou"-se por isso a partida e o carro continuou a sua odisseia, com
o acompanhamento do burro, sempre a olhá"-lo longamente, infinitamente,
demoradamente, cheio de piedade impotente.

Aos poucos os urubus se despediram; e, no fim de quatro anos, o carrião
entrou pelo Rio adentro, a roncar pelas calçadas, chocalhando duramente
as ferragens, com o seu manco e compassivo burro a manquejar"-lhe à
sirga.

Logo que foi chegado, um hábil serralheiro veio abri"-lo, pois a
fechadura desarranjara"-se devido aos trancos e às intempéries da viagem,
e desobedecia à chave competente. Silly determinou que os médicos
examinassem o doente, exame que, mergulhados numa atmosfera de
desinfetantes, foi feito no necrotério público.

Foi este o destino do enfermo pelo qual o delegado Cunsono se interessou
com tanta solicitude.


\chapter[Hussein Bem"-Áli Al"-Bálec e Miquéias Habacuc]{Hussein Bem"-Áli Al"-Bálec\\ e Miquéias Habacuc \subtitulo{Conto argelino}\footnote[*]{Publicado na coletânea de contos \emph{Histórias e Sonhos}, de 1920.}}
\hedramarkboth{Hussein Bem"-Áli Al"-Bálec\ldots}{}

\hfill\emph{Ao senhor Cincinato Braga}\bigskip

\noindent{}Antes da conquista francesa, havia, na Argélia, uma família composta de
um velho pai doente e seis filhos varões. Desde muito que o pai, devido
aos achaques da idade, não se entregava diretamente aos trabalhos da sua
lavoura; mas, sempre que o seu estado de saúde lhe permitia, tinha o
cuidado de correr as suas terras com plantações, que eram de tâmaras,
alfa, oliveiras, laranjeiras, havendo somente uma parte que era
destinada à criação de ovelhas, cabras e bezerros. As plantações e a
criação estavam entregues a cinco dos seus filhos, pois o mais velho,
ele o tinha mandado ao Cairo, para estudar profundamente, na respectiva
universidade, a lei do Profeta e vir a ser um ulemá digno e sábio no
Corão\footnote{Ulemá é um título recebido pelos estudiosos dos textos e
  da tradição da religião muçulmana, que tem Maomé como seu grande
  Profeta. Um tipo de teólogo.}.

Áli Bálec Al"-Bálec era o nome desse filho do velho árabe e esteve de
fato no Cairo; mas, bem depressa, abandonou o estudo das santas leis de
Alá e do Profeta e procurou a sociedade dos infiéis.

Foi ter nas suas aventuras à Grécia, onde se demorou muito tempo e
adquiriu dos gregos muitos hábitos, costumes e vícios. Não se pode em
confiança dizer que os atuais sejam bem netos dos antigos; mas são
aparentados. A finura e sagacidade dos últimos para abstrações
filosóficas, para especulações científicas, para a análise dos
sentimentos e paixões, do que dão provas as suas obras de filosofia, as
suas criações científicas e as suas grandes obras literárias, empregam
nos nossos dias os atuais na mercancia, no tráfico, no escambo, em que
sempre procuram, com a máxima habilidade e sabedoria enganar não só os
estrangeiros, como os seus próprios patrícios.

No Oriente, só há um traficante que não seja enganado pelo grego: é o
armênio. Diz"-se mesmo lá: o judeu é enganado pelo grego, mas o armênio
engana ambos.

Os turcos, de onde em onde, matam estes últimos aos milheiros, não tanto
por motivos religiosos, mas por ódio do comprador cavalheiresco, do
homem leal e crédulo, que se vê enganado despudoradamente, e sente que
não há, no outro que o ludibriou, nenhum princípio de honra, de
lealdade, de honestidade, que as relações entre os homens o exigem.

Áli Bálec Al"-Bálec, apesar de ser muçulmano, foi atraído para o meio dos
gregos e, com eles, aprendeu as suas espertezas, maroscas e habilidades
para enganar os outros.

E assim foi que ele andou fora da casa paterna, fazendo o escambo dos
mares do Levante, indo de Alexandria para Constantinopla, daí para Jafa,
deste porto para Salônica, desta cidade para Corfu\footnote{Regiões
  localizadas no Oriente Médio e nas proximidades do Mar Mediterrâneo.},
perlustrando todos aqueles mares azuis, cheios de história, de lenda, de
sangue e piratas, comerciando e mesmo pirateando quando a ocasião se lhe
oferecia.

Ao saber da morte do pai, vendeu logo a faluca\footnote{Uma pequena
  embarcação a vela utilizada no mar Mediterrâneo, em sua parte
  oriental.} que possuía e correu a receber a herança. Coube"-lhe uma
grande data de terra, coberta de pés de tâmaras, enquanto os irmãos
tinham as suas cultivadas com alfa, com laranjeiras, oliveiras e um
mesmo recebeu a sua parte em terrenos de pastagens magras, onde pasciam
rebanhos enfezados de ovelhas e cabras.

Todos, porém, ficaram contentes com a partilha e iam vivendo.

Áli Bálec Al"-Bálec trouxera como sua mulher uma israelita que renegara o
Talmude pelo Corão\footnote{Ou seja, trocou a religião judaica, que tem
  o Talmude como um de seus principais livros religiosos, pela religião
  muçulmana, representada nos escritos presentes no Corão.}, mas, apesar
disso, tinha o maior desprezo pelos muçulmanos, aos quais considerava
grosseiros, convencendo de tal coisa o marido a ponto dele não dar mais
importância aos seus próprios irmãos.

Logo ao voltar ainda os atendia e os visitava; mas a mulher lhe dizia
sempre:

--- Esses teus irmãos são uns brutos! Parecem mochos! Uns bobos! Que
sandálias! O pano das suas chéchias\footnote{Um chapéu tradicional usado
  pelos seguidores do islamismo.} é barato e sempre está sujo! Deixa"-os
lá!

Aos poucos, devido aos conselhos de sua mulher, Salisa, da sua
insistência, ele deixou de procurar os irmãos, fez"-lhes má cara, embora
os filhos deles viessem de quando em quando, à casa do tio, para ver o
primo Hussein, que se ia criando mais pérfido que o pai e mais orgulhoso
que a mãe.

Em pouco, Áli ficou inteiramente convencido da sua imensa superioridade
sobre os seus humildes e resignados irmãos.

Por ter na sua sala um tapete de Esmirna\footnote{Esmirna é uma
  importante cidade da Turquia, famosa pela qualidade de seus tapetes.},
serem as suas armas de aço de Damasco\footnote{Capital da Síria.},
tauxiadas de ouro, julgava os seus manos, que se tinham habituado à
simplicidade e à modéstia, como inferiores, iguais aos das tribos negras
que viviam para além do deserto. Julgando"-os assim, esquecia"-se que,
enquanto ele viajava, enquanto ele aprendia aquelas coisas finais, os
irmãos plantavam, ceifavam e colhiam, para ele aprender.

Além disso, Áli, como falasse alguns patoás levantinos\footnote{O autor
  se refere a algumas línguas da região conhecida como Levante, na
  região asiática composta por países como Israel, Jordânia, Síria,
  Palestina, Líbano.}, julgava"-se muito mais que todos os do
vilaiete\footnote{Pequena vila.} e também, por possuir joias de ouro e
pedras caras, valendo muitas piastras\footnote{Um tipo de moeda comum,
  que circulava em países como Egito, Líbano, Síria, entre outros.},
imaginava que tudo podia.

Por esse tempo, chegaram os franceses e o caide\footnote{O termo
  \emph{caide} significa um título hierárquico e administrativo em
  algumas regiões da Ásia central. É um tipo de chefe militar ou de
  algum território.} apelou para todos, a fim de socorrer o
bei\footnote{\emph{Bei} era a denominação que recebia o rei, na época em
  que o Império Turco"-Otomano dominava toda aquela região da Ásia. O
  termo \emph{bei} deixou de ser usado e em seu lugar a designação
  \emph{sultão} passou a ser utilizada. No entanto, em muitas
  localidades o \emph{bei,} assim como o \emph{caide}, continuaram a ser
  utilizados para identificar os líderes e chefes de localidades e
  territórios. Os franceses, no final do século \textsc{xix}, se lançaram à
  colonização de diversos países daquela região. Esse período histórico
  ficou conhecido como neocolonialismo e foi um dos principais
  responsáveis pelo início da Primeira Guerra Mundial, em 1914.} com
homens e valores. Áli ofereceu uma das joias do seu tesouro e quase por
isso foi empalado. O joalheiro do palácio verificou que as joias eram
inteiramente falsas e, vindo o bei a saber disso, tomou a coisa como
afronta e mandou castigar severamente o doador.

Salisa, sua mulher, ficou, ao conhecer a notícia, no mais completo
desespero, não porque o marido estivesse em risco de vida, mas pelo fato
que a fortuna representada por aquelas joias não era mais que fumaça.

Áli foi solto e jurou que havia de enriquecer de novo. Aceitou sem
resistência a dominação francesa e, com alegria, viu que essa dominação
trazia uma grande alta para as tâmaras que o seu terreno produzia
prodigiosamente.

Seus irmãos, a seu exemplo, aceitaram os francos e continuaram na sua
modéstia, observando muito religiosamente as leis do Corão.

Áli, já habituado, em pouco se misturou com os infiéis a quem vendia as
tâmaras por bom preço e gastava o grosso do rendimento que ia tendo em
bebidas, apesar da proibição do Corão, em orgias com os oficiais e
funcionários franceses. Construiu um palácio que ele pretendia parecido
com aquele do grande califa Harum Al"-Raxid, em Bagdá, conforme é
descrito no livro de histórias da princesa Xerazade\footnote{Referência
  ao livro ``As mil e uma noites'', de autoria desconhecida, escrito em
  árabe e que passou a circular na região do Oriente Médio e sul da Ásia
  por volta dos séculos \textsc{viii} e \textsc{ix}. Trata"-se de uma série de histórias
  contadas pela princesa Xerazade. No começo do século \textsc{xviii} surgiu uma
  tradução para o francês e então as histórias das ``Mil e uma noites''
  passaram a fazer parte da cultura letrada do ocidente.}.

Vendo que as tâmaras eram muito procuradas pelos francos que, por elas,
pagavam bom dinheiro, por toda a parte começaram a plantar tâmaras; os
irmãos de Áli, porém, não quiseram fazer tal, pois sabiam por
experiência de seu pai, que, desde que houvesse muitas tâmaras para
vender e, não se precisando desse fruto para o nosso comer diário, não
era possível que muita gente as quisesse comprar tão caro. Abundando
tinham que vendê"-las mais barato, para atingir e provocar os compradores
mais pobres.

Continuaram com a sua alfa, as suas laranjeiras, a pascer os seus
rebanhos, sem nenhuma inveja do irmão que parecia rico e os desprezava.

Os seus sobrinhos, de quando em quando, iam às terras do tio e ele, por
ostentação, por vaidade e para mostrar riqueza, lhes dava uma libra
turca e as crianças voltavam para casa dos pais, dizendo:

--- Tio Áli é que é gente! Tem tudo! Como ele é rico, por Alá!

Os seus pais respondiam:

--- Cada um se deve conformar com o que Alá lhe dá! É bom que prospere,
pois tem família\ldots{} Deus é Deus e Maomé é seu profeta.

Veio a morrer Áli, quando as tâmaras começaram a cair de preço.
Herdou"-lhe os bens, além da mulher, o seu único filho Hussein Ben"-Áli
Al"-Bálec que tinha todos os defeitos do pai aumentados com os de sua
mãe.

Era vaidoso, presunçoso, ávido, desprezando os parentes, para os quais
era somítico e avaro, desprezando"-os como se fossem animais imundos e
tidos em maldição pelas Leis do Profeta. Com os franceses, entretanto,
era mais pródigo do que o pai e fingia ter as suas maneiras e usos.

Nas gazetas que começaram a aparecer em Argel, Hussein Ben"-Áli Al"-Bálec
era gabado e, apesar das leis do Corão proibirem a reprodução da figura
humana, uma delas lhe publicou o retrato. As tâmaras começaram a descer;
e, como Hussein tivesse notícias que, duas léguas próximas, um outro
muçulmano possuía uma grande plantação delas, começou a pensar que era
esta que fazia descer o preço das suas.

Em Argel\footnote{Argel era uma das formas de se pronunciar da Argélia,
  país do norte da África, principal colônia francesa no período do
  neocolonialismo.}, sobretudo no vilaiete de Hussein, personificam"-se
sempre os fenômenos e a sutileza que um plantador de tâmaras não pode
bem conhecer, apesar de raça árabe, o filigranado das induções da
economia política\ldots{}

Imaginou logo destruir a plantação e mesmo toda aquela que aparecesse na
redondeza. Supôs de bom alvitre ir com alguns homens e queimar os
coqueiros. O dono certamente queixar"-se"-ia ao caide, às autoridades
francas; e seria uma complicação. Homem de expedientes, lembrou"-se de
conseguir do capitão francês da guarnição, Al"-Durand ou Al"-Burhant, a
destruição do plantio rival. Habitualmente, fez"-se amigo do
rume\footnote{Rume era um termo genérico utilizado pelos portugueses
  para se referir aos turcos ou os habitantes da região do norte da
  África e do Oriente Médio.}, encheu"-o de presentes, de festas, de
bebidas, pois seguia o exemplo de seu pai nesse tocante; e o ``cão do
cristão'' se fez afinal seu amigo. Um dia, depois de uma festa, o
militar, que pisava indignamente a terra onde estavam os ossos do seu
pai, após muitas queixas de Áli, apiedado do árabe, apressou"-se em ir à
plantação do vizinho e castigá"-lo. Assim fez, com os seus soldados e os
ferozes serviçais de Hussein. Houve queixa; o capitão foi punido; mas o
saas\footnote{Medida utilizada pelos árabes, algo equivalente a uma
  saca.} de tâmaras não subiu nem meio \emph{gourde}\footnote{Um tipo de
  unidade monetária. Hoje em dia é o nome da moeda oficial do Haiti.}.

As suas finanças iam de mal a pior, a casa magnífica ia dando mostras de
ruína e os seus móveis e alfaias deterioravam"-se com o tempo. Sua mãe
não cessava de censurar"-lhe pelas faltas que não lhe cabiam. Ela, com
aquela arrogância muito sua e inveja também muito sua, repreendia"-o:

--- Vês: as tâmaras caem de preço e tu não tomas providência alguma. Os
meus não são assim\ldots{} Mas tens o sangue de teu pai\ldots{} É verdade que teus
tios estão vendendo alfa, oliveiras, gado e laranjas e ganham\ldots{} Se tu
não fizeres esforço algum, ficarás como eles, uns macacos a viver em
tocas e a dormir em pelegos de carneiro\ldots{} Xmed, o teu segundo tio,
ganhou duzentas piastras em azeitonas e ficou contente. Queres ser como
ele?

--- Que hei de fazer, mãe?

--- Pensa; e não fiques aí a chorar como mulher. Saul chorou? Davi
chorou? Só o Deus dos cristãos chorou: Jeová não ama o choro. Ele ama a
guerra e o combate, até o extermínio. Lê os livros, os que foram os meus
e os teus que são também agora os meus. Lembra"-te de Débora e de
Judite\footnote{Personagens presentes na Bíblia cristã.} e eram
mulheres!

Hussein Ben"-Áli Al"-Bálec não podia dormir com a impressão das palavras
de sua mãe. O saas de tâmaras continuava a descer de \emph{gourde} em
\emph{gourde}, e ele só se lembrava de Áli, de Ornar, de todos aqueles
de sua raça que as tinham levado em meio século, do Ganges ao
Ebro\footnote{Ganges, o maior rio da Índia e um dos maiores do mundo. Já
  o Ebro, o maior rio da Espanha. Por essa citação percebe"-se o tamanho
  do esforço que os antepassados de Hussein empreenderam para construir
  uma pequena fortuna, que ele ia, a exemplo do pai, aniquilando
  completamente.}. Mas o saas de tâmaras parecia não temer aquelas
sombras augustas e ferozes. Descia sempre.

Certo dia, apareceu"-lhe um homem que queria falar a sua mãe, Salisa. Era
o irmão dela, Miqueias Habacuc. A irmã e o sobrinho acolheram muito bem
tão próximo parente e lhe falaram na baixa das tâmaras que os
atormentava. Miqueias, que era homem esperto em negócios, disse para o
sobrinho:

--- Filho de minha irmã, tens meu sangue, mas não a minha fé nos livros
santos da sinagoga; mas teus avós Isaque, Baruque, Daniel, Azaf, Etã,
Zabulon, Neftali e tantos outros mandam que eu te auxilie nesse transe
da tua vida que é preciosa a eles e a mim, pois ela é deles e também
minha. Portanto, tais forem os presentes que tu me fizeres, eu posso
purificar"-me de ter socorrido um ente que não é de Israel. Dize"-o que o
rabino me perdoará.

Hussein ficou de pensar e, à noite, conferenciou com sua mãe Salisa.

--- Filho, dá"-lhe alguns cequins turcos\footnote{Nome de uma antiga
  moeda, fabricada na Itália, mas que circulava pela região do Oriente
  Médio e Mediterrâneo.} e aquelas joias falsas que quase custaram a
morte de teu pai. Porque --- ouve bem --- o conselho dele pode ser
falaz.

Despertando Miqueias, logo Hussein foi ter com ele e propôs"-lhe o
escambo. O israelita, ao ver as joias, nem olhou mais os cequins. Ficou
com os olhinhos fosforescentes de tigre na escuridão. Era como se fosse
dar um salto de felino. Contou então ao sobrinho como devia proceder.

--- Tu que tens o sangue de minha avó Micaia, que era da tribo de
Jeroboão, e de Azarela, que era da casa de Leedã, ouve, comprarás todas
as tâmaras que houver na redondeza, mesmo antes de amadurecerem, ficando
elas nos pés. Quando for época de colhê"-las, colhê"-las"-ás todas,
guardando em surrões nos armazéns de tua casa e não venderás senão
quando te oferecerem um lucro que dê a fartar para gastares\ldots{}

--- Tio amado e sábio: elas não apodrecerão?

--- Não importa. As poucas ``medidas'' em que isto acontecer darão
prejuízo, mas tu marcarás o lucro de modo que o cubras. Hussein Ben"-Áli
Al"-Bálec descansou um instante a cabeça sobre o peito, depois a ergueu
de repente e exclamou:

--- Falas com a sabedoria do Profeta, Miqueias Habacuc. Que Alá seja
contigo!

Miqueias Habacuc, filho de Uriel de Sepetai, não se quis demorar mais e
partiu despedindo"-se da irmã Salisa e do sobrinho Hussein Ben"-Áli
Al"-Bálec com lágrimas nos olhos, canastras pesadas com os cequins turcos
e as joias falsas com que o sobrinho lhe pagara o seu profundo conselho
de economia política hebraica.

Hussein fez o que lhe foi aconselhado; e as tâmaras começaram a ter mais
oferta de preço. Vendeu"-as com grande lucro no primeiro ano; no segundo,
se sentia uma certa resistência no mercado, ele as reteve em grande
parte; mas, no terceiro ano, ele teve que comprar a produção e viu que
ia aumentando o estoque do que se pode chamar de valorização das
tâmaras. Viu bem que se continuasse a comprar a produção, ficaria com
ele demasiado aumentado, a sua fortuna comprometida e que fez? Cedeu. As
tâmaras começaram a descer \emph{gourde} a \emph{gourde}. Teve uma ideia
que um sargento francês lhe indicou. Vendo que elas encalhavam nos seus
armazéns e os pedidos cresciam lentamente; vendo, pouco a pouco, os seus
coquinhos perdendo o valor, alugou alguns gritadores que berrassem, nas
ruas de Argel, a guerreira:

--- Vivam as tâmaras! Não há coisa melhor que as tâmaras de Hussein Bem
Áli Al"-Bálec!

Nas gazetas, ele pagava anúncios das suas tâmaras, mas não vendia mais
que dantes. Deu"-as de graça e, como toda coisa dada de graça, elas só
agradavam desse modo.

Em se tratando de vendê"-las, nada! Os surrões de tâmaras aumentavam nos
seus armazéns, pois teimava em comprá"-las e guardá"-las, para que elas
não viessem afinal a não valer nada.

O tapete de Esmirna que o pai lhe deixara desfiava"-se, empenhou as armas
preciosas, também a herança do pai, para comprar mais sacas de tâmaras.
Comprou um tapete falso e umas armas vagabundas de um cabila\footnote{Os
  \emph{cabilas}, ou \emph{cabildas} constituem um povo habitante da
  Cabila, região montanhosa localizada a noroeste da Argélia.} mais
vagabundo ainda, para pôr no lugar das antigas preciosidades. Os outros
plantadores, que se tinham limitado a colher e vender, iam vivendo das
suas modestas plantações; ele, Hussein Ben"-Áli Al"-Bálec, corria para a
ruína certa.

Foi por aí que, novamente, lhe apareceu Miqueias Habacuc, seu tio, homem
hábil e esperto nos negócios. Hussein ficou espantado, mas o tio lhe
disse:

--- Rebento da minha querida irmã, pelo Deus de Abraão, de Israel e de
Jacó, não te amedrontes: vendi as joias por um bom preço a um grego, com
o que ganhei duas coisas: dinheiro e a glória de ter enganado um cão
dessa espécie. Mas, pelo Eterno! Esta ideia de pagar"-me o conselho em
joias falsas não é tua\ldots{} Isto tem dedo de pessoa inteiramente da minha
raça de Mardoc e Malaquias\ldots{} Isto é de minha irmã! Não foi tua mãe
quem\ldots{}

--- Foi. E que fizeste do dinheiro, tio amado da minha alma; socorro da
minha vida?

--- Emprestei"-o aos turcos com bons juros e quando os cobrei, quase me
esfolaram. Muito tem sofrido a raça de Israel; mas o que sofri deles,
nem contar te posso --- ó descendente do grande Al"-Bálec, companheiro de
Muça --- conquistador das Espanhas!\footnote{No início do século \textsc{viii},
  aproximadamente entre 710 a 730, vários grupos da região do norte da
  África, principalmente seguidores do islamismo, conquistaram a
  península ibérica, região em que hoje se encontram Portugal e Espanha.
  Neste primeiro movimento de expansão árabe na península ibérica, um
  dos principais responsáveis pela conquista do território que hoje
  conhecemos como Espanha, foi justamente o líder Muça Ibne Noçair.}

Acabava de dizer estas palavras, quando entra no aposento em que estavam
Salisa, a feroz Judite, a eloquente Débora --- que, ao dar com o irmão,
se põe em prantos, exclamando:

--- Irmão do coração, sábio Miqueias! Tu que descendes como eu de
Micaia, da tribo de Jeroboão, e de Azarela, que era da casa de Leedã,
salva"-me pelo nosso Deus de Abraão, de Israel e de Jacó --- salva"-me!

E a feroz Judite e eloquente Débora chorou não a sua dor, nem a dos
outros, mas o dinheiro que se sumia.

Contou, então, Hussein ao tio, como a ruína se aproximava; como a
valorização das tâmaras\footnote{Importante notar que Lima Barreto
  também está contando, de maneira crítica, a história do Brasil,
  justamente na época da chamada ``valorização do café'', no período da
  Primeira República (1889--1930). Ver mais a respeito consultando as
  notas 23 e 93. Vale a pena indicar que o conto é ``dedicado'' ao
  Senhor Cincinato Braga, um dos grandes incentivadores de tal política.
  Trata"-se de uma ``dedicatória irônica'', visto que Lima Barreto foi um
  dos grandes críticos dessas maracutaias econômicas.}, no começo dando
tão bom resultado, viera a acabar, no fim, em desastre completo.

O velho Miqueias, filho de Uriel de Sepetai, coçou as barbas hirsutas;
os seus olhinhos luziram naquele quadro de pelos cerdosos; depois,
faiscando"-os malignamente, perguntou ao sobrinho:

--- Com que dinheiro tu, sobrinho meu; com que dinheiro fizeste a
operação?

Hussein disse"-lhe que fora com o dinheiro dele e o da sua mãe.

Miqueias Habacuc, judeu de Salônica, homem esperto e hábil em negócios,
sorriu com gosto e demora, dizendo após:

--- Tolo que és!

--- Por quê?

Habacuc assim falou de súbito, logo imediatamente à pergunta:

--- Que me darás em troca pela explicação?

--- A última bolsa de cequins de ouro que me resta.

--- És generoso e grande, sobrinho meu, filho de Salisa, minha irmã,
guarda"-a. Ganharemos mais. Fizeste mal em empregar o teu dinheiro e o da
tua mãe. Devias empregar o dos outros.

--- Como, tio Miqueias?

--- Tu não sabes, meu sobrinho, essas operações de câmbio e de banco. Eu
as sei. Nós agora vamos organizar a defesa das tâmaras, isto é, impedir
que especuladores reduzam à miséria e à desolação esta rica região do
Magreb\footnote{Magreb é uma das mais importantes regiões do noroeste da
  África e que abrange diversos países, Marrocos, Tunísia, Argélia,
  entre outros.}, como dizia o teu grande avô, Al"-Bálec. Vamos pedir
dinheiro aos seus habitantes, para que não morram de fome e não pereçam
à míngua por falta de trabalho.

--- Não me darão, tio.

--- Dar"-te"-ão, sobrinho do meu coração; dar"-te"-ão. Chama teus tios,
irmãos de teu pai, e os filhos, e convence"-os que devem dar as economias
que têm, em moeda, para poderes lutar com os que querem acabar com as
plantações de tâmaras do vilaiete. Dize"-lhes que se não o fizerem as
plantações morrerão, os habitantes fugirão, aqui ficará tudo deserto,
sem água e sem pastagens; e os bens deles nada valerão e serão também
eles obrigados a fugir, perdendo muito, senão tudo.

--- E em troca?

--- Tu lhes darás vales que vencerão juros e pagarás os vales em certo
prazo.

--- Mas\ldots{}

--- Nada objetes, meio do meu sangue de Sepetai, mas meu sobrinho
inteiramente. Não sabes o que é a cobiça; não sabes o que é querer ter
dinheiro sem trabalhar. Eles aceitarão na certa e, não sendo ricos em
breve precisarão de dinheiro. Eu vou pôr um ``bazar'' com o saco de
cequins de ouro que te resta e farei saber que desconte esses vales
teus, em dinheiro ou em mercadoria. O pouco dinheiro que tens atrairá o
deles, tu comprarás tâmaras, mas pagarás em vales que vencerão o juro de
dois por cento, mas que eu descontarei a vinte, trinta e mais por cento.

--- Se não quiserem descontar, tio que és sábio como o mais sábio dos
ulemás, como há de ser?

--- Tens o dinheiro dos teus parentes. Em começo, pagarás tudo em
dinheiro. Mas teus parentes, precisando de dinheiro, irão, como te
disse, procurar"-me. Eu os atenderei imediatamente. A fama correrá e
ninguém temerá receber os teus vales.

--- Compreendo. E as tâmaras?

--- Irás vendendo a bom preço e guardando o dinheiro, deixando que uma
grande parte apodreça. Tu viverás na pompa, na grandeza, e um belo dia,
em vez de eu descontar vales, adquiro"-os com ágio. Toda a gente quererá
os teus vales e encheremos as arcas de dinheiro.

--- E no fim, no pagamento, como será?

--- Marcarás um prazo longo, pela festa do Beirão, e daqui até lá
teremos tempo de agir.

Hussein Ben"-Áli Al"-Bálec empregou todas as lábias que lhe ensinou
Miqueias Habacuc. Seus tios e primos entregaram"-lhe as economias, pois
ficaram muito contentes que ele se lembrasse de defendê"-los, de impedir
a ser completa a miséria. Tio e sobrinho encheram os simplórios homens
de todos os afagos, de todas as blandícias, e iniciaram a defesa das
tâmaras, que era a própria defesa do vilaiete.

Um único não quis entregar as terras de pastagem. Foi o tio que herdara
as terras de pastagem. Dissera o velho:

--- As tâmaras não são do gosto de todo o mundo e as que se colhem são
de sobra para os que gostam delas. Hão de se as vender barato por força,
pois são demais.

Hussein Ben"-Áli Al"-Bálec, porém, deu início à sua obra de grande
eficácia para todo o vilaiete, ostentando uma riqueza, um luxo e uma
magnificência que reduziram, fascinaram a imaginação do povo do lugar e
das circunvizinhanças.

O seu palácio foi aumentado; as suas estrebarias ficaram cheias de
soberbos ginetes do Hedjaz, nas suas piscinas só corriam águas
perfumadas --- tudo ficou sendo um encanto no seu alcáçar e
dependências.

A fama de sua riqueza cor

ria por toda a parte e até, em Argel, a branca, a guerreira, seu nome
era falado. Dizia a boca do povo:

--- Se todos fossem como Hussein Ben"-Áli Al"-Bálec conquistaríamos todo o
Magreb, expulsando os rumes.

O seu crédito ficou sendo tal que todo o dinheiro que havia naquelas
terras entrou para as suas arcas. As tâmaras subiram de preço, de fato;
mas pouco.

Entretanto, enquanto vendia um terço, guardava dois. Miqueias Habacuc
exultava, com os descontos que fazia e com o dinheiro que era trazido
para as mãos do sobrinho.

Só a irmã, a feroz Salisa, temia o fim e perguntava ao irmão:

--- Como pagaremos tantos vales, se já gastamos o dinheiro deles e temos
mais tâmaras guardadas que vendidas?

--- Cala"-te, irmã que és minha. Aí é que está a minha grande sabedoria.

O dinheiro amoedado desapareceu e os vales de Hussein corriam como
moeda. No começo equivaliam ao seu valor em cequins; mas, bem depressa,
para se comprar com eles um saas de trigo, tinha"-se de gastar o duplo do
que se gastava antigamente. O povo começava a desconfiar, quando veio
rebentar a guerra de
Abdelcáder\footnote{Referência à
  guerra comandada por Abdelcáder (1808--1883), líder político e
  militar (emir) da cidade de Mascara, na Argélia. Abdelcáder foi um dos
  principais comandantes dos levantes argelinos contra a ocupação
  francesa, sobretudo nos anos de 1830. É referenciado como um grande
  mártir naquele país até os dias de hoje.}, emir de Mascara. Andava ele
precisando de homens e víveres. O emir, que sabia do prestígio de
Hussein naquele vilaiete, oferece"-lhe alguns milhares de libras turcas,
para que mandasse homens.

Miqueias, que sabe do caso, intervém, e propõe que o sobrinho aceite,
contanto que o emir lhe compre as tâmaras. O emir acede, paga as mil
libras turcas, compra as tâmaras de que não precisava.

E Hussein convence os parentes que devem partir para os
goums\footnote{Os \emph{goums}
  eram formados por grupos de guerrilheiros árabes que partiam em
  colunas e marchas pelas colinas e desertos. Chegavam a caminhar mais
  de cem quilômetros em uma semana.}. Para isso falou como um santo
marabuto.

Antes da festa do Beirão, época que era marcada para o vencimento dos
vales, fugia, com a mãe, a feroz Salisa, o tio Miqueias Habacuc, homem
hábil e esperto em negócios --- cheios todos de ouro, ricos de
apodrecer.

No vilaiete a população caiu na miséria, menos aquele tio de Hussein
Bem"-Áli Al"-Bálec, que não quis entrar na defesa das tâmaras.

Durante muito tempo, pastoreou as suas ovelhas e tosou os seus
carneiros. Os seus netos ainda hoje fazem a mesma coisa naquele lugarejo
argelino, onde as inocentes tamareiras, se não constituem objeto de
maldição, são tidas como simples árvores de adorno.



\chapter[A barganha]{A barganha\footnote[*]{Publicado na edição de junho de 1920 da \emph{Revista Souza Cruz}.}}

E o ``turco'', desde muito cedo, andava pelos subúrbios a mercar aqueles
coloridos registros de santos. Havia um São João Batista, com a sua
tanga, o seu bordão de pastor e o seu inocente carneiro que olhava doce
tudo o que via fora da estampa; havia um Cristo com o coração muito
rubro à mostra, coroado de espinhos, e os olhos revirados para o Céu que
naquele dia estava lindo, de um profundo azul cobalto; havia uma Ceia em
que Jesus presidia, mansueto e resignado, apesar de se saber traído, e
havia muitos outros santos e santas que o ``turco'' levava, alguns
enrolados, mas outros diante do seu peito arquejante das suas caminhadas
de humilde bufarinheiro\footnote{Vendedor ambulante. O ``turco'' é
  vendedor de quadros com figuras de santos.}, daquelas modestas
paragens da cidade.

E ele ia:

--- Compra, sinhor! Muita bonita!

Das casas, às vezes, lá saía uma mulher ou outra, de cores as mais
variadas, e indagava com desprezo:

--- Olá! O que é que você leva aí?

Miguel José parava, aproximava"-se da porteira e respondia:

--- Santa, sinhora! Muita bonita!

--- Que santos tem?

--- Muitas, sinhora. Tuda bonita.

Desenrolava os registros e a rapariga começava a examinar. De repente, à
vista de uma daquelas oleogravuras, ela gritava:

--- Leocádia! Leocádia!

Lá do interior da casa respondiam:

--- Que é?

A outra acudia:

--- Vem cá. Vem ver uma coisa.

Vinha uma outra rapariga e a que estava, recomendava, mostrando um dos
quadros do ``turco'':

--- Vê só como é lindo este Menino Jesus.

A outra examinava e concordava. O ``turco'' se animava e perguntava:

--- Não quer compra ele?

Uma delas ia ao encontro da pergunta do bufarinheiro:

--- Quanto é?

--- Barata, sinhora.

--- Quanto?

--- Dois mil"-réis.

--- Chi, meu Deus!

É caro, muito mesmo.

O pobre ambulante não fazia negócio algum; e continuava com a sua carga
sagrada a palmilhar aquelas ruas que são mais propriamente veredas.

Ainda se houvesse árvores, sombra que amaciasse aquela manhã quente,
embora linda e cristalina, o seu ofício seria suportável; mas não as
havia. Tudo era descampado e as ruas eram batidas pelo sol em chapa. Lá
ia ele. As calças ficavam"-lhe pelos tornozelos; o chapéu era de feltro,
mas não se sabia se era preto, azul, cinzento. Tinha todas as cores
próprias a chapéus dessa espécie. Em um pé calçava uma botina amarela;
em outro, um sapato preto.

--- Cumpra, sinhor! Coisa bonita de Deus! Cumpra.

Foi dizendo isto a um petulante crioulo, muito preto, de um preto fosco
e desagradável, cabeleira grande, gordurosa, repartida ao alto, e o
chapéu a dançar"-lhe em cima dela; foi dizendo isto a ele que lhe ia
acontecendo uma grande desgraça naquela manhã. O negro, ao ouvi"-lo,
chegou"-se muito junto ao ``turco'' e indagou com um ar autoritário:

--- Que é que você está dizendo?

O humilde armênio pensou logo que tratava com um soldado de polícia à
paisana, pois lhe parecia que, na terra em que estava, todos os pretos
são soldados e podem prender todos os armênios.

Com essa convicção, Miguel José respondeu cheio de respeito e
acatamento:

--- Dizia, sinhor: cumpra santo muita bonita.

O negro perfilou"-se todo, tomou uns ares judiciais ou policiais, chegou
o chapéu de palha para a testa e disse:

--- Você parece que não é civilizado.

--- Cumo, sinhor?

--- Sim, você é herege, inimigo de Nosso Senhor.

--- Não, sinhor.

O preto desarmou"-se um pouco de seus ares judiciais ou policiais,
tornou"-se mais suave, quis fazer de penetrante e sagaz. Perguntou:

--- Você come carne de porco?

E Miguel José olhou as montanhas pedregosas que ele via lá, longe,
esbatidas no azul profundo da manhã, ressaltando quase inteiramente na
ambiência translúcida do dia, e lembrou"-se da sua aldeia armênia, das
suas cabras, das suas ovelhas, dos seus porcos.

A sua fisionomia dura contraiu"-se um pouco e os seus olhos de carneiro
quiseram chorar de recordação, de sofrimento, de mágoa.

Ele se encheu todo de uma pesada tristeza; mas pôde responder:

--- Sim, senhor, eu coma.

--- Então você é cristão? insistiu o preto.

--- Sim, sinhor; diga a sinhor sou cristão.

--- Admira.

--- Por quê, sinhor?

--- Porque você diz ``vender'', ``comprar'' santos.

--- Cuma se diz então?

--- Troca"-se. Aprenda --- está ouvindo! É falta de respeito, é
sacrilégio dizer comprar ou vender santos. Aprendeu?

--- Sim, sinhor. Obrigada, sinhor.

E o crioulo se foi, deixando o pobre armênio arrasado por mais aquele
déspota que passava sobre a sua pobre raça; mas mesmo assim, continuou
na sua mercancia.

Lá se foi ele por aquelas ruas de tão caprichoso nivelamento que permite
as carroças que por lá se arriscam andarem no ar com burros e tudo. Lá
ia ele:

--- Cumpra, sinhor! Muita bonita.

Subia, descia ladeiras; parava nas portas; mas não fazia negócio algum.

Num pequeno campo, encontrou uma porção de crianças a empinar papagaios.
Parou um pouco para ver aquele divertimento interessante que as crianças
da sua terra não conheciam. Veio um pequenote:

--- Ó Zé! O que é que você leva aí?

--- Santo, menina. Pede mamãe compra uma.

--- Ora, esta! Lá em casa tem tanto santo --- para que mais um? Vende
ali, aos ``bíblias''.

Miguel José percebeu bem a malícia da criança, pois de uma feita caíra
na tolice de oferecer um registro a essa espécie de religiosos e se vira
atrapalhado. Não que o tivessem maltratado, mas um deles, baixinho, com
um \emph{pince"-nez} muito puro de vidros cristalinos, o levara para o
interior da casa, lera"-lhe uma porção de coisas de um livro e depois
quisera que ele se ajoelhasse e abandonasse os registros.

Noutra não cairia ele\ldots{}

Continuou o caminho, mas estava cansado. Ansiava por uma sombra, onde
repousasse um pouco. Havia muitas árvores, mas todas no interior das
casas, nas chácaras, nos quintais ou nos jardins. Uma assim pública, na
margem da rua, em terreno abandonado que o abrigasse aí, por uns dez
minutos, ele não encontrava.

E seria tão bom descansar assim fazendo o seu minguado almoço, para
continuar até à tarde a sua faina, vendo se ganhava pelo menos uns dez
ou cinco tostões de comissão com a venda daquelas coisas sagradas.

E continuou o seu caminho, tendo sempre exposta diante do peito a imagem
de Cristo, coroado de espinhos, a mostrar o coração muito rubro, com os
seus misericordiosos olhos a procurar o Céu, naquela manhã muito linda,
de um profundo azul"-cobalto\ldots{}

Afinal, achou uma mangueira, maltratada, cheia de ervas parasitas, a
crescer na borda do caminho, num terreno desocupado.

Sentou"-se, tirou da algibeira um naco de pão dormido, uma cebola e
pôs"-se a comer, olhando as montanhas pedroucentas que assomavam ao longe
e lhe faziam lembrar a terra natal. Ele não tinha nenhum nítido
pensamento sobre a vida, a natureza e a sociedade\ldots{}

Não tardou que se lhe viesse juntar um companheiro. Era também um
``volante'' como ele; mas a sua mercancia era outra, menos espiritual.
Vendia sardinhas, de que trazia um cesto cheio. Era um português, cheio
de saúde, de força, de audácia. Vinha suado, mais do que o armênio;
entretanto, não dava mostras de ter ressentimentos nem do sol nem da
dureza do seu ofício. O armênio olhou"-o com inveja e pensou de si para
si:

--- Como é que esse homem pode ser alegre, pode ter esperanças?

O português, sem auxílio, arriou o grande cesto na sombra e sentou"-se
também cheio de confiança e desembaraço. Foi logo dizendo:

--- Bons dias, patrício.

Miguel José fez uma voz sumida:

--- Bom dia, sinhor. O português, sem mais aquela, observou:

--- Qual senhor! Qual nada! Cá entre nós, é você pra baixo. Isto de
senhor é lá pros doutores, não é para nós que andamos aqui aos tombos. E
emendou comunicativo:

--- Que diabo --- ó patrício! --- que tu comes pra aí?

O ``turco'' disse"-lhe e o Manuel da Silva considerou:

--- Lá na minha terra, há quem goste disto; mas eu nunca me acostumei.
Cebola pra mim, só na comida. Numa bacalhoada, ah!\ldots{}

Miguel José continuava a mastigar sua cebola com pão, enquanto Manuel da
Silva contava a féria\footnote{Ver nora 125.}. Contada que ela foi,
disse bem alto:

--- Pela hora que é, as coisas não vão mal. Até o meio"-dia vendo tudo\ldots{}
Guardou o dinheiro na bolsa que tinha a tiracolo e perguntou subitamente
ao companheiro de acaso:

--- Você já vendeu muito hoje, patrício?

--- Nada, sinhor.

--- Está você a dar com o tal de senhor! Pergunto se você já vendeu
alguma coisa hoje, homem!

--- Nada.

--- O que é que você vende?

--- Santo, sinhor.

--- Santo?

--- Sim; santo.

--- Deixa ver isto, como é? --- fez o português curioso.

O armênio passou"-lhe os registros coloridos e o vendedor de sardinhas
pôs"-se a olhá"-los com espanto e deslumbramento artístico de aldeão
simplório. Achou tudo aquilo bonito: aquele Jesus, mostrando o coração;
São João, com o carneirinho; o Menino Jesus --- tudo muito lindo aos
seus olhos maravilhados de camponês cândido e enfeitiçado pelas coisas
do senhor vigário.

Refletiu de si para si: ``Coisas tão bonitas, se não as vendeu, é porque
este `turco' é mesmo burro. Comigo, já as tinha vendido, ganhado
dinheiro e ficado com algumas, pra pôr lá no quarto''.

Veio"-lhe uma ideia.

--- Patrício! Você quer fazer um negócio?

Os olhos de carneiro do armênio luziram mais forte e com mais esperança.

--- Qual é? perguntou ele.

--- Tenho ali na cesta cerca de vinte mil"-réis de sardinhas, vendidas a
duas por um vintém. Se você vendê"-las a vinte, ganha o dobro. Quer você
trocar estes santos pelo cesto de sardinhas?

Miguel José rapidamente pesou os prós e contras da operação comercial.
Sabia bem, por experiência própria, que a população, até as crianças, se
mostrava refratária à mercadoria espiritual de que ele era portador; e,
pelo que lhe vira ainda agora nas mãos, a do seu companheiro não se
portava da mesma forma.

Em se tratando de sardinhas, as coisas não corriam da mesma maneira como
no tocante a santos. Considerou bem e logo respondeu:

--- Tá feita, senhor.

Os dois se despediram e trocaram de carga. Miguel José voltou a passar
pelos mesmos lugares em que oferecera os registros, sem nenhum
resultado; mas, quando apregoou as sardinhas, não teve mãos a medir.
Vendeu"-as a vintém, então fez escambos de compensação e, de tal forma
correram"-lhe as coisas que, dentro de três horas, tinha vendido tudo,
podia pagar os registros à loja e lucrava cinco mil e tanto.

Manuel da Silva, o alegre português das sardinhas, saiu muito ancho com
os seus registros; mas não foi logo vendê"-los.

A frugalidade do ``turco'' tinha"-lhe dado uma fome extraordinária.
Procurou uma casa de pasto e comeu a fartar, acompanhado de um bom
martelo de verdasco\footnote{A expressão \emph{martelo} era utilizada
  como se utiliza hoje a palavra dose. \emph{Verdasco} é um tipo de
  vinho verde, de baixa qualidade e bastante ácido.}.

Bem alimentado, satisfeito, dispôs"-se a ``trocar'' o são João Batista,
Menino Jesus, correndo a sua freguesia de peixes e crustáceos.

Batia as portas:

--- Mamãe, dizia uma criança, está aí o seu Manuel.

A mãe perguntava lá de dentro:

--- Ele traz camarão?

--- Não, mamãe; quer vender santos.

--- Para que deu agora, seu Manuel!

Ora, vejam só! Vender santos. Diga a ele que não quero.

Dessa e de outra maneira, ele foi percorrendo em vão sua freguesia das
sardinhas, sem mercar uma única estampa religiosa.

A sua alegria matinal se ia e todo o seu desgosto se voltava terrível
contra ele mesmo. Não fora o ``turco'' que o embrulhara; fora ele mesmo
que propusera aquele negócio. Era castigo. Ia tão bem com as sardinhas,
para que fizera aquela barganha?

Andou até quase a noitinha e nada vendeu. Ao recolher"-se, ainda quis ver
as oleogravuras que o haviam deslumbrado.

Mirou uma, mirou outra e, olhando"-as firmemente, refletiu:

--- Se não fosse por faltar o respeito devido a Nosso Senhor Jesus
Cristo, que aí está, eu havia de dizer que tudo isso são coisas do diabo
que aquele ``turco'' me impingiu. Nunca mais! Tarrenego!



\chapter[Cló]{Cló\footnote[*]{Publicado na edição de 05 de março de 1918, do jornal \emph{A.B.C.}.}}

\hfill\emph{A Alexandre Valentim Magalhães}\bigskip

\noindent{}Devia ser já a terceira pessoa que lhe sentava à mesa.

Não lhe era agradável aquela sociedade com desconhecidos; mas que fazer
naquela segunda"-feira de Carnaval, quando as confeitarias têm todas as
mesas ocupadas e as cerimônias dos outros dias desfazem"-se,
dissolvem"-se?

Se as duas primeiras pessoas eram desajeitados sujeitos sem atrativos, o
terceiro conviva resgatava todo o desgosto causado pelos outros. Uma
mulher formosa e bem tratada é sempre bom ter"-se à vista, embora sendo
desconhecida, ou, talvez, por isso mesmo\ldots{}

Estava ali o velho Maximiliano esquecido, só moendo cismas, bebendo
cerveja, obediente ao seu velho hábito. Se fosse um dia comum, estaria
cercado de amigos; mas, os homens populares, como ele, nunca o são nas
festas populares. São populares a seu jeito, para os frequentadores das
ruas célebres, cafés e confeitarias, nos dias comuns; mas nunca para a
multidão que desce dos arrabaldes, dos subúrbios, das províncias
vizinhas, abafa aqueles e como que os afugenta. Contudo não se sentia
deslocado\ldots{}

A quinta garrafa já se esvaziara e a sala continuava a encher"-se e a
esvaziar"-se, a esvaziar"-se e a encher"-se. Lá fora, o falsete dos
mascarados em trote\footnote{Referência às cantigas de carnaval que
  chegavam até o bar (confeitaria) onde o personagem Maximiliano se
  encontrava. O \emph{falsete} é um tipo de entonação vocal que permite
  às pessoas com voz grossa, principalmente os homens, produzirem sons
  agudos, neste caso, dos homens ``mascarados'' que vinham desfilando em
  trote.}, as longas cantilenas dos cordões\footnote{Sobre os
  ``cordões'', ver nota 29.}, os risos e as músicas lascivas enchiam a
rua de sons e ruídos desencontrados e, dela, vinha à sala uma satisfação
de viver, um frêmito de vida e de luxúria que convidava o velho
professor a ficar durante mais tempo bebendo, afastando o momento de
entrar em casa.

E esse frêmito de vida e luxúria que faz estremecer a cidade nos três
dias de sua festa clássica, naquele momento, diminuía"-lhe muito as
grandes mágoas de sempre e, sobretudo, aquela teimosia e pequenina de
hoje. Ela o pusera assim macambúzio e isolado, embora mergulhado no
turbilhão de riso, de alegria, de rumor, de embriaguez e luxúria dos
outros, em segunda"-feira gorda. O ``jacaré'' não dera e muito menos a
centena\footnote{Referência ao ``jogo do bicho'', que tem o jacaré com
  um de seus animais da sorte.}. Esse capricho da sorte tirava"-lhe a
esperança de um conto e pouco\footnote{Um conto de réis. Sobre o
  dinheiro da época, ver nota 30.} --- doce esperança que se esvaía
amargosamente naquele crepúsculo de galhofa e prazer.

E que trabalho não tivera ele, doutor Maximiliano, para fazê"-la brotar
no seu peito, logo nas primeiras horas do dia! Que chusmas de
interpretações, de palpites, de exames cabalísticos! Ele bem parecia um
áugure romano que vem dizer ao cônsul se deve ou não oferecer
batalha\ldots{}\footnote{Os \emph{áugures} eram, na Grécia e Roma antigas, os
  adivinhos que previam o futuro e que participavam dos grupos que
  ofereciam conselhos aos cônsules, reis, imperadores e comandantes
  militares. Em vésperas de batalhas importantes, um \emph{áugure} era
  consultado para prever os possíveis resultados dos confrontos.}

Logo que ela lhe assomou aos olhos, como não lhe pareceu certo aquele
navegar precavido dentro do nevoento mar do Mistério, marcando rumo para
aquele ponto --- o ``jacaré'' --- onde encontraria sossego, abrigo, durante
alguns dias!

E agora, passado o nevoeiro, onde estava?\ldots{} Estava ainda em mar alto,
já sem provisões quase, e com débeis energias para levar o barco a
salvamento\ldots{} Como havia de comprar bisnagas, confetes, serpentinas,
alugar automóvel? E --- o que era mais grave --- como havia de pagar o
vestido de que a filha andava precisada, para se mostrar sábado próximo,
na rua do Ouvidor, em toda a plenitude de sua beleza, feita (e ele não
sabia como) da rija carnadura de Itália e de uma forte e exótica
exalação sexual\ldots{} Como havia de dar"-lhe o vestido?

Com aquele seu olhar calmo em que não havia mais nem espanto, nem
reprovação, nem esperança, o velho professor olhou ainda a sala tão
cheia, por aquelas horas, tão povoada e animada de mocidade, de talento
e de beleza. Ele viu alguns poetas conhecidos, quis chamá"-los, mas,
pensando melhor, resolveu continuar só.

O velho doutor Maximiliano não cansou de observar, um por um, aqueles
homens e aquelas mulheres, homens e mulheres cheios de vícios e aleijões
morais; e ficou um instante a pensar se a nossa vida total, geral, seria
possível sem os vícios que a estimulavam, embora a degradem também.

Por esse tempo, então, notou ele a curiosidade e a inveja com que um
grupo, de modestas meninas dos arrabaldes, examinava a
\emph{toilette}\footnote{Palavra francesa que significa o conjunto dos
  trajes, vestimentas, acessórios, principalmente os femininos. A
  palavra foi aportuguesada para \emph{toalete} e ganhou novos
  significados, como banheiro --- ir ao toalete.} e os ademanes das
mundanas presentes.

Na sua mesa, atraindo"-lhe os olhares, lá estava aquela formosa e famosa
Eponina\footnote{\emph{Epônima} é forma feminina do substantivo
  \emph{epônimo}, que significa o ato de nomear um local, coisa, época,
  um grupo humano, uma cidade e até uma nação. Normalmente um
  \emph{epônimo} é identificado por uma personalidade histórica ou por
  algum animal. O estado de São Paulo, por exemplo, recebeu o seu nome
  de forma \emph{eponímica}, ou seja, uma localidade geográfica que
  ganhou o nome de um dos primeiros escritores do novo testamento, Paulo
  de Tarso, que depois foi canonizado como São Paulo. Muitos times de
  futebol também fazem uso de \emph{epônimos} animais, como o ``galo'',
  que nomeia \emph{eponimicamente} o Atlético Mineiro, a ``raposa'' que
  é o \emph{epônimo} do Cruzeiro, o ``peixe'' para o Santos, etc. Nessa
  passagem, Lima Barreto está se referindo à alguma ``garota de
  programa'' --- \emph{a mais linda mulher pública da cidade}, tão
  deslumbrante que poderia servir de ``Epônima'' para alguma localidade
  do Rio de Janeiro.}, a mais linda mulher pública da cidade, produto
combinado das imigrações italiana e espanhola, extraordinariamente
estúpida, mas com um olhar de abismo, cheio de atrações, de promessas e
de volúpia.

E o velho lente olhava tudo aquilo pausadamente, com a sua indulgência
de infeliz, quando lhe veio o pensar na casa, naquele seu lar, onde o
luxo era uma agrura, uma dor, amaciada pela música, pelo canto, pelo
riso e pelo álcool.

Pensou, então, em sua filha, Clôdia --- a Cló, em família --- em cujo
temperamento e feitio de espírito havia estofo de uma grande
hetaira\footnote{``Estofo'' é um tipo de enchimento, normalmente de
  algodão, usado para acentos de cadeiras, sofás, etc., daí o nome
  ``estofados''. Lima Barreto usa a palavra no sentido de ``recheio''
  para uma \emph{grande hetaira}. Esse era o nome pelo qual eram
  conhecidas, na Grécia antiga, as prostitutas que frequentavam a elite
  da sociedade, muito famosas nas cortes, por isso também chamadas de
  cortesãs. Uma das mais famosas \emph{hetairas} da Grécia antiga
  chamava"-se Aspasia, da cidade de Mileto.}. Lembrou"-se com casta
admiração de sua carne veludosa e palpitante, do seu amor às danças
lúbricas, do seu culto à \emph{toilette} e ao perfume, do seu fraco
senso moral, do seu gosto pelos licores fortes; e, de repente e por
instantes, ele a viu coroada de hera, cobrindo mal a sua magnifica
nudez, com uma pele mosqueada, o ramo de tirso erguido, dançando,
religiosamente bêbeda, cheia de fúria sagrada de bacante: ``Evoé!
Baco!''\footnote{Toda essa passagem é descrita como se o velho professor
  Maximiliano estivesse visualizando sua filha Cló em uma cerimônia
  ritualística em homenagem ao deus grego Dioniso. Na mitologia romana,
  Dioniso recebeu o nome Baco. Era conhecido como a divindade das
  festas, das bebidas, da volúpia e sensualidade. As \emph{bacantes}
  ``adoradoras de baco'', eram as dançarinas dos rituais.}

E essa visão antiga lhe passou pelos olhos, quando a Eponina ergueu"-se
da mesa, tilintando as pulseiras e berloques caros, chamando muito a
atenção de Mme. Rego da Silva que, em companhia do marido e da sua
extremosa amiga Dulce, amante de ambos, no dizer da cidade, tomavam
sorvetes, numa mesa ao longe.

O doutor Maximiliano, ao ver aquelas joias e aquele vestido, voltou a
lembrar"-se de que o ``jacaré'' não dera; e refletiu, talvez com
profundeza, mas certo com muita amargura, sobre a má organização da
nossa sociedade. Mas não foi adiante e procurou decifrar o problema da
sua multiplicação em Cló, tão maravilhosa e tão rara. Como é que ele
tinha posto no mundo um exemplar de mulher assaz vicioso e delicado como
era a filha? De que misteriosa célula sua saíra aquela floração
exuberante de fêmea humana? Vinha dele ou da mulher? De ambos?

Ou de sua mulher só, daquela sua carne apaixonada e sedenta que
trepidava quando lhe recebia as lições de piano, na casa dos pais?

Não pôde, porém, resolver o caso. Aproximava"-se o doutor André, com o
seu rosto de ídolo peruano\footnote{Lima Barreto quer dizer com ``ídolo
  peruano'' as pequenas estátuas esculpidas em madeira e que
  representavam as divindades ou os reis que existiram antes da chegada
  dos conquistadores europeus. No caso do Peru, os espanhóis foram os
  conquistadores no início do século \textsc{xvi}. Os espanhóis destruíam os
  santuários onde se encontravam os \emph{ídolos}, que eram substituídos
  por cruzes, marcando assim o início da dominação da igreja católica.
  Alguns \emph{ídolos} frequentemente são descobertos por arqueólogos,
  como no caso do ``Ídolo inca de Pachamac'', que tem mais de dois
  metros de altura e teria sido venerado sete séculos antes da chegada
  dos espanhóis.}, duro, sem mobilidade alguma na fisionomia, acobreada,
onde o ouro do aro do \emph{pince}-\emph{nez} reluzia fortemente e
iluminava a barba cerdosa.

Era um homem forte, de largos ombros, musculoso, tórax saliente,
saltando; e, se bem tivesse as pernas arqueadas, era assim mesmo um belo
exemplar da raça humana.

Lamentava"-se que ele fosse um bacharel vulgar e um deputado obscuro. A
sua falta de agilidade intelectual, de maleabilidade, de ductílidade, a
sua fraca capacidade de abstração e débil poder de associar ideias não
impediam fosse ele deputado e bacharel. Ele seria rei,
estaria no seu quadro natural, não
na câmara, mas remando em ubás ou igaras nos nossos grandes rios ou
distendendo aqueles fortes arcos de iri que despejam frechas ervadas com
curare\footnote{Esta passagem mostra como Lima Barreto ironiza a
  pretensão dos brasileiros em seguir os padrões culturais que foram
  impostos pela colonização europeia --- Deputado, Bacharel, etc. O
  personagem André combinaria mais, por seus aspectos físicos --- rosto
  de ídolo peruano, pernas arqueadas --- com aqueles reis que viviam
  entre os povos ameríndios antes da chegada dos europeus, pois estaria
  ``no seu quadro natural'', ao invés de estar na câmara dos deputados.
  Interessante notar os termos tupi"-guarani usados pelo escritor: ubás
  (pequenas canoas); igaras (canoas feitas em apenas um único tronco de
  árvore); arcos de iri (arcos utilizados pelos povos originários da
  Amazônia, feitos a partir dos galhos da brejaúva, um tipo de palmeira
  encontrada naquela região, também conhecida como iri); ervadas com
  curare (curare é o nome de um composto venenoso feito a partir de
  várias ervas da flora amazônica, utilizado nas pontas das flechas).}.

Era o seu último amigo, entretanto o mais constante comensal de sua mesa
luculesca\footnote{Luculesca ou luculenta, significa uma coisa que
  esbanja luxo, esplendor, riqueza.}.

Deputado, como já ficou dito, e rico, representava, com muita galhardia
e liberalidade, uma feitoria\footnote{É de se notar, mais uma vez, a
  ironia do autor. Ele usa o termo \emph{feitoria}, que era uma
  localidade construída para servir de entreposto comercial entre as
  colônias e as metrópoles. As feitorias eram de responsabilidade dos
  feitores e muitas delas foram as matrizes de pequenas cidades e
  vilarejos. Com o avanço da colonização e depois da urbanização e
  estruturação das vilas e cidades, perderam a antiga importância que
  tinham.} mansa do Norte, nas salas burguesas; e, apesar de casado, a
filha do antigo professor, a lasciva Cló, esperava casar"-se com ele,
pela religião do Sol, um novo culto recentemente fundado por um
agrimensor ilustrado e sem emprego.

O velho Maximiliano nada de definitivo pensava sobre tais projetos; não
os aprovava, nem os reprovava. Limitava"-se a pequenas reprimendas sem
convicção, para que o casamento não fosse efetuado sem a bênção do
sacerdote do Sol ou de outro qualquer.

E se isto fazia, era para não precipitar as cousas; ele gostava dos
desdobramentos naturais e encadeados, das passagens suaves, das
inflexões doces, e detestava os saltos bruscos de um estado para o
outro.

--- Então, doutor, ainda por aqui? fez o rico parlamentar sentando"-se.

--- É verdade, respondeu"-lhe o velho. Estou fazendo o meu sacrifício,
rezando a minha missa\ldots{} É a quinta\ldots{} Que toma, doutor?

--- Um ``madeira''\footnote{Um tipo de vinho português bastante amargo.}\ldots{}
Que tal o Carnaval?

--- Como sempre.

E, depois, voltando"-se para o caixeiro\footnote{\emph{Caixeiro} era o
  nome que os trabalhadores do comércio recebiam. Hoje em dia usa"-se
  ``atendente'' para lojas e ``garçom'' para bares, lanchonetes e
  restaurantes.}:

--- Outra cerveja e um ``madeira'', aqui, para o doutor. Olha: leva a
garrafa.

O caixeiro afastou"-se, levando a garrafa vazia e o doutor André
perguntou:

--- Dona Isabel não veio?

--- Não. Minha mulher não gosta das segundas"-feiras de Carnaval. Acha"-as
desenxabidas\ldots{} Ficaram, ela e a Cló, em casa a se prepararem para o
baile á fantasia na casa dos Silvas\ldots{} Quer ir?

--- O senhor vai?

--- Não, meu caro senhor; do Carnaval, eu só gosto dessa barulhada da
rua, dessa música selvagem e sincopada de recos"-recos, de pandeiros, de
bombos, desse estridulo de fanhosos instrumentos de metais\ldots{} Até do
bombo gosto, mais nada! Essa barulhada faz"-me bem à alma. Não irei\ldots{}
Agora, se o doutor quer ir\ldots{} Cló vai de preta mina\footnote{O termo
  ``preta mina'' surgiu em função de uma localidade conhecido como Costa
  da Mina, que se estende pelo litoral dos países como o Benin, Togo e
  Nigéria, no continente africano. Neste lugar funcionava a feitoria
  ``São Jorge da Mina'', lugar de onde vieram muitos escravizados e
  escravizadas para o Brasil. Mais uma vez Lima Barreto toca em um lugar
  sensível, ao apresentar a fantasia de carnaval de Cló como a de uma
  ``preta mina''.}.

--- Deve"-lhe ficar muito bem\ldots{} Não posso ir; entretanto, irei à sua casa
para ver a sua senhora e a sua filha fantasiadas. O senhor devia também
ir\ldots{}

--- Fantasiado?

--- Que tinha?

--- Ora, doutor! eu ando sempre com a máscara no rosto.

E sorriu leve com amargura; o deputado pareceu não compreender e
observou:

--- Mas, a sua fisionomia não é tão decrépita assim\ldots{}

Maximiliano ia objetar qualquer cousa quando o caixeiro chegou com as
bebidas, ao tempo em que Mme. Rego da Silva e o marido levantaram"-se com
a pequena Dulce, amante de ambos, no dizer da cidade em peso.

O parlamentar olhou"-os bastante com o seu seguro ar de quem tudo pode.
Ouviu que ao lado diziam --- à passagem dos três: ménage à
trois\footnote{Termo francês que significa ``um trio de amantes''.}. A
sua simplicidade provinciana não compreendeu a maldade e logo dirigiu"-se
ao velho professor:

--- Jantam em casa?

--- Jantamos; e o doutor não quer jantar conosco?

--- Obrigado. Não me é possível ir hoje\ldots{} Tenho um compromisso sério\ldots{}
Mas fique certo que, antes de saírem, lá irei tomar um uisquezinho\ldots{} Se
me permite?

--- Oh! doutor! O senhor é nosso melhor amigo. Não imagina como todos lá
falam no senhor. Isabel levanta"-se a pensar no doutor André; Cló, essa,
nem se fala! Até o Caçula quando o vê, não late; faz"-lhe festas, não é?

--- Como isso me cumula de\ldots{}

--- Ainda há dias, Isabel me disse: Maximiliano, eu nunca bebi um
\emph{Chambertin}\footnote{Um tipo de vinho francês.} como esse que o
doutor André nos mandou\ldots{} O meu filho, o Fred, sabe até um dos seus
discursos de cor; e, de tanto repeti"-lo, creio que sei de memória vários
trechos dele.

A face rígida do ídolo, com grande esforço, abriu"-se um pouco; e ele
disse, ao jeito de quem quer o contrário:

--- Não vá agora recitá"-lo.

--- Certo que não. Seria inconveniente; mas não estou impedido de dizer,
aqui, que o senhor tem muita imaginação, belas imagens e uma forma
magnífica.

--- Sou principiante ainda, por isso não me fica mal aceitar o elogio e
agradecer a animação.

Fez uma pausa, tomou um pouco de vinho e continuou em tom conveniente:

--- O senhor sabe perfeitamente que espécie de força me prende aos
seus\ldots{} Um sentimento acima de mim, uma solicitação, alguma cousa a mais
que os senhores puseram na minha vida\ldots{}

--- Pois então, interrompeu cheio de comoção o doutor Maximiliano: à
nossa!

Ergueu o copo e ambos tocaram os seus, reatando o parlamentar a conversa
desta maneira:

--- Deu aula hoje?

--- Não. Desci para espairecer e ``cavar''\footnote{\emph{Cavar} ou fazer
  \emph{cavação} era um termo utilizado quando as pessoas estavam à
  procura de algum tipo de trabalho, um rendimento extra, ou mesmo pata
  conseguir amizades ou ``contatos'' influentes que ajudassem a
  conseguir tais objetivos.}. É dura esta vida\ldots{} ``cavar''! Como é triste
dizer"-se isto! Mas que se há de fazer? Ganha"-se uma miséria\ldots{} Um
professor com oitocentos mil"-réis o que é? Tem"-se a família,
representação\ldots{} uma miséria! Ainda agora, com tantas dificuldades, é
que Cló deu em tomar banhos de leite\ldots{}

--- Que ideia! Onde aprendeu isso?

--- Sei lá! Ela diz que tem não sei que propriedades, certas virtudes\ldots{}
O diabo é que tenho de pagar uma conta estupenda no leiteiro\ldots{} São
banhos de ouro, é que são! Jogo nos bichos\ldots{} Hoje tinha tanta fé no
``jacaré''\ldots{}

O caixeiro passava e ele recomendou:

--- Baldomero, outra cerveja. O doutor não toma mais um ``madeira''?

--- Vá lá. Ganhou, doutor?

--- Qual! E não imagina que falta me fez!

--- Se quer?\ldots{}

--- Por quem é, meu caro; deixe"-se disso! Então há de ser assim todo o
dia?

--- Que tem!\ldots{} Ora!\ldots{} Nada de cerimônias; é como se recebesse de um
filho\ldots{}

--- Nada disso\ldots{} Nada disso\ldots{}

Fingindo que não entendia a recusa, o doutor André foi retirando da
carteira uma bela nota, cujo valor nas algibeiras do doutor Maximiliano
fez"-lhe esquecer em muito a sua desdita no ``jacaré''.

O deputado ainda esteve um pouco; em breve, porém, se despediu,
reiterando a promessa de que iria até à casa do professor, para ver as
duas senhoras fantasiadas.

O doutor Maximiliano bebeu ainda uma cerveja e, acabada que foi a
cerveja, saiu vagarosamente um tanto trôpego.

A noite já tinha caído de há muito. Era já noite fechada. Os cordões e
os bandos carnavalescos continuavam a passar, rufando, batendo, gritando
desesperadamente. Homens e mulheres de todas as cores --- os alicerces do
país --- vestidos de meia, canitares\footnote{Coroas de penas, penachos.}
e enduapes\footnote{Um tipo de tanga, utilizada pelos índios tupinambá,
  com penas multicoloridas de pássaros.} de penas multicores, fingindo
índios, dançavam na frente ao som de uma zabumbada africana, tangida com
fúria em instrumentos selvagens, roufenhos, uns, estridentes, outros. As
danças tinham luxuriosos requebros de quadris, uns caprichosos trocar de
pernas, umas quedas imprevistas.

Aqueles fantasiados tinham guardado na memória muscular velhos gestos
dos avoengos, mas não mais sabiam coordená"-los nem a explicação deles.
Eram restos de danças guerreiras ou religiosas dos selvagens de onde a
maioria deles provinha, que o tempo e outras influências tinham
transformado em palhaçadas carnavalescas\ldots{}

Certamente, durante os séculos de escravidão, nas cidades, os seus
antepassados só se podiam lembrar daquelas cerimônias de suas aringas ou
tabas, pelo carnaval. A tradição passou aos filhos, aos netos, e estes
estavam ali a observá"-la com as inevitáveis deturpações.

Ele, o doutor Maximiliano, apaixonado amador de música, antigo professor
de piano, para poder viver e formar"-se, deteve"-se um pouco, para ouvir
aquelas bizarras e bárbaras cantorias, pensando na pobreza de invenção
melódica daquela gente. A frase, mal desenhada, era curta, logo cortada,
interrompida, sacudida pelos rufos, pelo ranger, pelos guinchos de
instrumentos selvagens e ingênuos. Um instante, ele pensou em continuar
uma daquelas cantigas, em completá"-la; e a ária\footnote{Um tipo de
  música orquestrada, erudita, composta para ser cantada por uma só voz
  e acompanhada por instrumentos. Pode ser também um andamento
  orquestral, sem canto, uma melodia sinfônica.} veio"-lhe inteira, ao
ouvido, provocando o antigo professor de música a fazer parar o
``Chuveiro de Ouro''\footnote{Referência ao ``Club Recreativo Chuveiro de
  Ouro'', cuja sede era localizada no bairro da Gávea. Foi um dos
  primeiros clubes a organizar festejos e cortejos de carnaval, com
  carros alegóricos, bandas e desfiles de fantasias. Muitos compositores
  de canções carnavalescas participavam de concursos para terem suas
  letras cantadas durante os dias de desfile, o que trazia muito
  prestígio aos letristas, a maioria anônimos.}, a fim de ensinar"-lhes,
aos cantores, o que a imaginação lhe havia trazido à cabeça naquele
momento.

Arrependeu"-se que tivesse fito gostar daquela barulhada; porém, o amador
de música vencia o homem desgostoso. Ele queria que aquela gente
entoasse um hino, uma cantiga, um canto com qualquer nome, mas que
tivesse regra e beleza. Mas --- logo imaginou --- para quê? Corresponderia a
música mais ou menos artística aos pensamentos íntimos deles? Seria
mesmo a expansão dos seus sonhos, fantasias e dores?\footnote{Note"-se,
  mais uma vez, que Lima Barreto trabalha as contradições entre a
  cultura ``importada'' --- pois a ária é um tipo de música erudita
  europeia, trazida para o Brasil pelos portugueses --- e as
  manifestações genuinamente populares e brasileiras.}

E, devagar, se foi indo pela rua em fora, cobrindo de simpatia toda a
puerilidade aparente daqueles esgares e berros, que bem sentia profundos
e próprios daquelas criaturas grosseiras e de raças tão várias, mas que
encontravam naquele vozerio bárbaro e ensurdecedor meio de fazer porejar
os seus sofrimentos de raça e de indivíduo e exprimir também as suas
ânsias de felicidade.

Encaminhou"-se direto para a casa. Estava fechada; mas havia luzes na
sala principal, onde tocavam e dançavam.

Atravessou o pequeno jardim, ouvindo o piano. Era sua mulher quem
tocava; ele o adivinhava pelo seu \emph{velouté}\footnote{Palavra
  francesa que significa ``tempero'', ``molho''. Aqui, o autor faz uso
  metafórico do termo, indicando que o professor Maximiliano reconheceu
  ser a esposa quem estava ao piano, pelo seu \emph{velouté}, por seu
  jeito próprio, seu ``tempero''.}, pela maneira de ferir as notas,
muito docemente, sem deixar quase perceber a impulsão que os dedos
levavam. Como ela tocava aquele tango! Que paixão punha naquela música
inferior!

Lembrou"-se então dos ``cordões'', dos ``ranchos'', das suas cantilenas
ingênuas e bárbaras, daquele ritmo especial a elas que também perturbava
sua mulher e abrasava sua filha. Por que caminho lhes tinha chegado ao
sangue e à carne aquele gosto, aquele pendor por tais músicas? Como
havia correlação entre elas e as almas daquelas duas mulheres?

Não sabia ao certo; mas viu em toda a sociedade complicados movimentos
de trocas e influências --- trocas de ideias e sentimentos, de
influências e paixões, de gostos e inclinações.

Quando entrou, o piano cessava e a filha descansava, no sofá, a fadiga
da dança lúbrica que estivera ensaiando com o irmão. O velho ainda ouviu
indulgentemente o filho dizer:

--- É assim que se dança nos Democráticos\footnote{Uma das mais antigas
  agremiações carnavalescas do Brasil, fundada em 1867 e até hoje em
  funcionamento. Na época em que se passa o conto de Lima Barreto, os
  bailes de carnaval do ``Club dos Democráticos'' eram os mais
  concorridos e badalados da sociedade carioca. Para saber um pouco mais
  sobre os Democráticos, acessar o site:
  \emph{https://www.clubedosdemocraticos.com.br/historia/}.}.

Cló, logo que o viu, correu a abraçá"-lo e, abraçada ao pai, perguntou:

--- André não vem?

--- Virá.

Mas, logo, em tom severo, acrescentou:

--- Que tem você com André?

--- Nada, papai; mas ele é tão bom\ldots{}

Quis Maximiliano ser severo; quis apossar"-se da sua respeitável
autoridade de pai de família; quis exercer o velho sacerdócio de
sacrificador aos deuses penates\footnote{Os ``penates'', na mitologia
  romana, são os deuses do lar.}; mas era céptico demais, duvidava, não
acreditava mais nem no seu sacerdócio nem no fundamento da sua
autoridade. Ralhou, entretanto, frouxamente:

--- Você precisa ter mais compostura, Cló. Veja que o doutor André é
casado e isto não fica bem.

A isto, todos entraram em explicações. O respeitável professor foi
vencido e convencido de que a afeição da filha pelo deputado era a cousa
mais inocente e natural deste mundo. Foram jantar. A refeição foi tomada
rapidamente. Fred, contudo, pôde dar algumas informações sobre os
préstitos carnavalescos do dia seguinte. Os Fenianos\footnote{Referência
  ao ``Club dos Fenianos'', outra importante agremiação carnavalesca do
  Rio de Janeiro. Na época já existiam disputas entre desfiles
  organizados pelos clubes ou organizações carnavalescas.} perderiam na
certa. Os Democráticos tinham gastado mais de sessenta contos e iriam
pôr na rua uma cousa nunca vista. O carro do estandarte, que era um
templo japonês, havia de fazer um ``bruto sucesso''. Demais, as mulheres
eram as mais lindas, as mais bonitas\ldots{} Estariam a Alice, a Charlotte, a
Lolita, a Cármen\ldots{}

--- Ainda toma muito cloral? perguntou Cló.

--- Ainda, retrucou o irmão; e emendou: vai ser uma lindeza, um triunfo,
à noite, com luz elétrica, nas ruas largas\ldots{}

E Cló, por instantes, mordeu os lábios, suspendeu um pouco o corpo e
viu"-se também, no alto de um daqueles carros, iluminada pelos
fogos"-de"-bengala, recebida com palmas, pelos meninos, pelos rapazes,
pelas moças, pelas burguesas e burgueses da cidade. Era o seu triunfo a
meta de sua vida; era a proliferação imponderável de sua beleza em
sonhos, em anseios, em ideias, em violentos desejos naquelas almas
pequenas, sujeitas ao império da convenção, da regra e da moral. Tomou a
cerveja, todo o copo de um hausto, limpou a espuma dos lábios e o seu
ligeiro buço surgiu lindo sobre os breves lábios vermelhos. Em seguida,
perguntou ao irmão:

--- E essas mulheres ganham?

--- Qual! Você não vê que é uma honra? respondeu"-lhe o irmão.

E o jantar acabou sério e familiar, embora a cerveja e o vinho não
tivessem faltado aos devotos de cada uma das duas bebidas.

Logo que a refeição acabou, talvez uns vinte minutos após, o doutor
André se fazia anunciar. Desculpou"-se com as senhoras; não pudera vir
jantar, questões políticas, uma conferência\ldots{} Pedia licença para
oferecer aquelas pequenas lembranças de Carnaval. Deu uma pequena caixa
a dona Isabel e uma maior à Cló. As joias saíram dos escrínios e
faiscaram orgulhosamente para todos os presentes deslumbrados. Para a
mãe, um anel; para a filha, um bracelete.

--- Oh, doutor! fez dona Isabel. O senhor está a sacrificar"-se e nós não
podemos consentir nisto\ldots{}

--- Qual, dona Isabel! São falsas, nada valem\ldots{} Sabia que dona Clódia ia
de ``preta mina'' e lembrei"-me trazer"-lhe este enfeite\ldots{}

Cló agradeceu sorridente a lembrança e a suave boca quis fixar
demoradamente o longo sorriso de alegria e agradecimento. E voltaram a
tocar. Dona Isabel pôs"-se ao piano e, como tocasse depois da sobremesa,
hora da melancolia e das discussões transcendentes, como já foi
observado, executou alguma cousa triste.

Chegava a ocasião de se prepararem para o baile à fantasia que os Silvas
davam. As senhoras retiraram"-se e só ficaram, na sala, os homens,
bebendo uísque. André, impaciente e desatento; o velho lente,
indiferente e compassivo, contando histórias brejeiras, com vagar e
cuidado; o filho, sempre a procurar caminho para exibir o seu saber em
cousas carnavalescas. A conversa ia caindo, quando o velho disse para o
deputado:

--- Já ouviu a Bamboula, de Gottschalk\footnote{Referência à ópera para
  piano Bamboula (1848) criada pelo pianista e compositor estadunidense
  Louis Moureau Gottschalk (1829--1869). Filho de uma mulher haitiana
  e de um comerciante inglês, nascido em New Orleans, na década de 1860
  já era reconhecido como um dos maiores pianistas e compositores do
  mundo. Faleceu justamente no Brasil, vitimado por complicações
  decorrentes de uma febre amarela.}, doutor?

--- Não\ldots{} Não conheço.

--- Vou tocá"-la.

Sentou"-se ao piano, abriu o álbum onde estava a peça e começou a
executar aqueles compassos de uma música negra de Nova Orleans, que o
famoso pianista tinha filtrado e civilizado.

A filha entrou, linda, fresca, veludosa, de pano da Costa ao ombro,
trunfa, com o colo inteiramente nu, muito cheio e marmóreo, separado do
pescoço modelado, por um colar de falsas turquesas. Os braceletes e as
miçangas tilintavam no peito e nos braços, a bem dizer totalmente
despidos; e os bicos de crivo da camisa de linho rendavam as raízes dos
seios duros que mal suportavam a alvíssima prisão onde estavam retidos.

Ainda pôde requebrar, aos últimos compassos da Bamboula, sobre as
chinelas que ocupavam a metade dos pés; e toda risonha sentou"-se por
fim, esperando que aquele Salomão de \emph{pince"-nez} de ouro lhe
dissesse ao ouvido:

``Os teus lábios são como uma fita de escarlate; e o teu falar é doce.
Assim como é o vermelho da romã partida, assim é o nácar das tuas faces;
sem falar no que está escondido dentro''.

O doutor Maximiliano deixou o tamborete do piano e o deputado, bem perto
de Clódia, se não falava como o rei Salomão à rainha de Sabá dilatava as
narinas para sorver toda a exalação acre daquela moça, que mais capitosa
se fazia dentro daquele vestuário de escrava desprezada.

A sala encheu"-se de outros convidados e a sessão de música veio a cair
na canção e na modinha. Fred cantou e Cló, instada pelo doutor André,
cantou também. O automóvel não tinha chegado; ela tinha tempo\ldots{}

Dona Isabel acompanhou; e a moça, pondo tudo o que havia de sedução na
sua voz, nos seus olhos pequenos e castanhos, cantou a ``Canção da Preta
Mina'':

--- Pimenta de cheiro, jiló, quibombô; Eu vendo barato, mi compra ioiô!

Ao acabar, era com prazer especial, cheia de dengues nos olhos e na voz,
com um longo gozo intimo que ela, sacudindo as ancas e pondo as mãos
dobradas pelas costas na cintura, curvava"-se para o doutor André e dizia
vagamente:

--- Mi compra ioiô!

E repetia com mais volúpia, ainda uma vez:

--- Mi compra ioiô!


\chapter[Clara dos Anjos]{Clara dos Anjos\footnote[*]{Publicado na coletânea de contos \emph{Histórias e Sonhos}, de 1920.}}

\hfill\emph{A Andrade Murici}\bigskip

\noindent{}O carteiro Joaquim dos Anjos não era homem de serestas e serenatas, mas
gostava de violão e de modinhas. Ele mesmo tocava flauta, instrumento
que já foi muito estimado, não o sendo tanto atualmente como outrora.
Acreditava"-se até músico, pois compunha valsas, tangos e acompanhamentos
para modinhas.

Aprendera a ``artinha'' musical na terra de seu nascimento, nos arredores
de Diamantina\footnote{Cidade de Minas Gerais, localizada no Vale do
  Jequitinhonha, muito próspera na época colonial.}, e a sabia de cor e
salteado; mas não safra daí.

Pouco ambicioso em música, ele o era também nas demais manifestações de
sua vida. Empregado de um advogado famoso, sempre quisera obter um
modesto emprego público que lhe desse direito à aposentadoria e ao
montepio\footnote{Um tipo de pensão, destinada à família de funcionários
  públicos, por ocasião da morte do funcionário ou invalidez.}, para a
mulher e a filha. Conseguira aquele de carteiro, havia quinze para vinte
anos, com o qual estava muito contente, apesar de ser trabalhoso e o
ordenado ser exíguo.

Logo que foi nomeado, tratou de vender as terras que tinha no local de
seu nascimento e adquirir aquela casita de subúrbio, por preço módico,
mas, mesmo assim, o dinheiro não chegara e o resto pagou ele em
prestações. Agora, e mesmo há vários anos, estava de plena posse dela.
Era simples a casa. Tinha dois quartos, um que dava para a sala de
visitas e outro, para a de jantar. Correspondendo a um terço da largura
total da casa, havia nos fundos um puxadito que era a cozinha. Fora do
corpo da casa, um barracão para banheiro, tanque, etc.; e o quintal era
de superfície razoável, onde cresciam goiabeiras maltratadas e um grande
tamarineiro copado.

A rua desenvolvia"-se no plano e, quando chovia, encharcava que nem um
pântano; entretanto, era povoada e dela se descortinava um lindo
panorama de montanhas que pareciam cercá"-la de todos os lados, embora a
grande distância. Tinha boas casas a rua. Havia até uma grande chácara
de outros tempos com aquela casa característica de velhas chácaras de
longa fachada, de teto acaçapado, forrada de azulejos até â metade do
pé"-direito, um tanto feia, é fato, sem garridice, mas casando"-se
perfeitamente com as anosas mangueiras, com as robustas jaqueiras e com
todas aquelas grandes e velhas árvores que, talvez, os que as plantaram,
não tivessem visto frutificar.

Por aqueles tempos, nessa chácara, se haviam estabelecido as
``bíblias''\footnote{Sobre os ``bíblias'', ver nota 48.}. Os seus
cânticos, aos sábados, quase de hora em hora, enchiam a redondeza. O
povo não os via com hostilidade, mesmo alguns humildes homens e pobres
raparigas simpatizavam com eles, porque, justificavam, não eram como os
padres que, para tudo, querem dinheiro.

Chefiava os protestantes um americano, Mr. Sharp, homem tenaz e cheio de
uma eloquência bíblica que devia ser magnífica em inglês; mas que, no
seu duvidoso português, se fazia simplesmente pitoresca. Era Sharp
daquela raça curiosa de \emph{yankees} que, de quando em quando, à luz
da interpretação de um ou mais versículos da Bíblia, fundam seitas
cristãs, propagam"-nas, encontram adeptos logo, os quais não sabem bem
por que foram para a nova e qual a diferença que há entre esta e a de
que vieram.

Fazia prosélitos e, quando se tratava de iniciar uma turma, os noviços
dormiam em barracas de campanha, erguidas no eirado da chácara ou entre
as suas velhas árvores maltratadas e desprezadas. As cerimônias
preparatórias duravam uma semana, cheia de cânticos divinos; e a velha
propriedade, com as suas barracas e salmodias, adquiria um aspecto
esquisito de convento ao ar livre de mistura com um certo ar de
acampamento militar.

Da redondeza, poucos eram os adeptos ortodoxos; entretanto, muitos lá
iam por mera curiosidade ou para deliciar"-se com a oratória de Mr.
Sharp.

Iam sem nenhuma repugnância, pois é próprio do nosso pequeno povo fazer
um extravagante amálgama de religiões e crenças de toda sorte, e
socorrer"-se desta ou daquela, conforme os transes de sua existência. Se
se trata de afastar atrasos de vida, apela para a feitiçaria; se se
trata de curar uma moléstia tenaz e resistente, procura o espírita; mas
não falem à nossa gente humilde em deixar de batizar o filho pelo
sacerdote católico, porque não há quem não se zangue: Meu filho ficar
pagão! Deus me defenda!

Joaquim não fazia exceção desta regra e sua mulher, a Engrácia, ainda
menos.

Eram casados há quase vinte anos, mas só tinham uma filha, a Clara. O
carteiro era pardo claro, mas com cabelo ruim, como se diz; a mulher,
porém, apesar de mais escura, tinha o cabelo liso.

Na tez, a filha puxava o pai; e no cabelo, à mãe. Na estatura, ficara
entre os dois. Joaquim era alto, bem alto, acima da média, ombros
quadrados; a mãe, não sendo muito baixa, não alcançava a média,
possuindo uma fisionomia miúda, mas regular, o que não acontecia com o
marido que tinha o nariz grosso, quase chato. A filha, a Clara, tinha
ficado em tudo entre os dois; média deles, era bem a filha de ambos.
Habituada às musicatas do pai, crescera cheia de vapores das modinhas e
enfumaçara a sua pequena alma de rapariga pobre com os dengues e a
melancolia dos descantes e cantarolas.

Com dezessete anos, tanto o pai como a mãe tinham por ela grandes
desvelos e cuidados. Mais depressa ia Engrácia à venda de ``seu''
Nascimento, buscar isto, ou aquilo, do que ela. Não que a venda de ``seu''
Nascimento fosse lugar de badernas; ao contrário: as pessoas que lá
faziam ``ponto'' eram de todo o respeito.

O Alípio, uma delas, era um tipo curioso de rapaz, que, conquanto pobre,
não deixava de ser respeitador e bem comportado. Tinha um aspecto de
galo de briga; entretanto, estava longe de possuir a ferocidade
repugnante desses galos malaios de apostas, não possuindo --- é preciso
saber --- nenhuma.

Um outro que aparecia sempre lá era um inglês, Mr. Persons, desenhista
de uma grande oficina mecânica das imediações. Quando saía do trabalho,
passava na venda, lá se sentava naqueles característicos tamboretes de
abrir e fechar, e deixava"-se ficar até ao anoitecer bebericando ou lendo
os jornais do senhor Nascimento. Silencioso, quase taciturno, pouco
conversava e implicava muito com quem o tratava por mister.

Havia lá também o filósofo Meneses, um velho hidrópico\footnote{Pessoa
  que sofre de hidropisia, uma doença que causa inchaço em partes do
  corpo, devido ao acúmulo de líquidos.}, que se tinha na conta de
sábio, mas que não passava de um simples dentista clandestino, e dizia
tolices sobre todas as cousas. Era um velho branco, simpático, com um
todo de imperador romano, barbas alvas e abundantes.

Aparecia, às vezes, o J. Amarante, um poeta, verdadeiramente poeta, que
tivera o seu momento de celebridade em todo o Brasil, se ainda não a
tem; mas que, naquela época, devido ao álcool e a desgostos íntimos, era
uma triste ruína de homem, apesar dos seus dez volumes de versos, dez
sucessos, com os quais todos ganharam dinheiro menos ele. Amnésico,
semi"-imbecilizado, não seguia uma conversa com tino e falava
desconexamente. O subúrbio não sabia bem quem ele era; chamava"-o muito
simplesmente --- o poeta.

Um outro frequentador da venda era o velho Valentim, um português dos
seus sessenta anos e pouco, que tinha o corpo curvado para diante,
devido ao hábito contraído no seu ofício de chacareiro que já devia
exercer há mais de quarenta. Contava ``casos'' e anedotas de sua terra,
pontilhando tudo de rifões portugueses do mais saboroso pitoresco.

Apesar de ser assim decente, Clara não ia à venda; mas o pai, em alguns
domingos, permitia que fosse com as amigas ao cinema do Méier ou Engenho
de Dentro, enquanto ele e alguns amigos ficavam em casa tocando violão,
cantando modinhas e bebericando parati.

Pela manhã, logo nas primeiras horas, os companheiros apareciam, tomavam
café, iam em seguida para o quintal, para debaixo do tamarineiro, jogar
a bisca, com o litro de cachaça ao lado; e ai, sem dar uma vista d'olhos
sobre as montanhas circundantes, nuas e empedrouçadas, deixavam"-se ficar
até à hora do ``ajantarado'' que a mulher e a filha preparavam.

Só depois deste é que as cantorias começavam. Certo dia, um dos
companheiros dominicais do Joaquim pediu"-lhe licença para trazer, no dia
do aniversário dele, que estava próximo, um rapaz de sua amizade, o
Júlio Costa, que era um exímio cantor de modinhas. Acedeu. Veio o dia da
festa e o famoso trovador apareceu. Branco, sardento, insignificante, de
rosto e de corpo, não tinha as tais melenas denunciadoras, nem outro
qualquer traço de capadócio\footnote{Adjetivo que recebiam os rapazes
  brigões, valentões, conquistadores de mulheres.}. Vestia"-se seriamente
com um apuro muito suburbano; sob a tesoura de alfaiate de quarta ordem.
A única pelintragem adequada ao seu mister que apresentava consistia em
trazer o cabelo repartido no alto da cabeça, dividido muito exatamente
pelo meio. Acompanhava"-o o violão. A sua entrada foi um sucesso.

Todas as moças das mais diferentes cores que, ai, a pobreza harmonizava
e esbatia, logo o admiraram. Nem César Bórgia\footnote{César Bórgia
  (1475--1507), filho do Papa Alexandre \textsc{vi}, foi um sacerdote e líder
  militar dos mais importantes de seu tempo.}, entrando mascarado, num
baile à fantasia dado por seu pai, no Vaticano, causaria tanta emoção.

Afirmavam umas para as outras:

--- É ele! É ele, sim!

Os rapazes, porém, não ficaram muito contentes com isto; e, entre eles,
puseram"-se a contar histórias escabrosas da vida galante do cantor de
modinhas.

Apresentado aos donos da casa e à filha, ninguém notou o olhar guloso
que deitou para os seios empinados de Clara.

O baile começou com a música de um ``terno'' de flauta, cavaquinho e
violão. A polca era a dança preferida e quase todos a dançavam com
requebros próprios de samba.

Num intervalo Joaquim convidou:

--- Por que não canta, ``seu'' Júlio?

--- Estou sem voz, respondeu ele.

Até ali, ele tinha tomado parte no ``remo''; e, repinicando as cordas, não
deixava de devorar com os olhos os bamboleios de quadris de Clarinha,
quando dançava. Vendo que seu pai convidara o rapaz, animou"-se a fazê"-lo
também:

--- Por que não canta, ``seu'' Júlio? Dizem que o senhor canta tão bem\ldots{}

Esse --- ``tão bem'' --- foi alongado maciamente. O cantador acudiu logo:

--- Qual, minha senhora! São bondades dos camaradas\ldots{}

Concertou a ``pastinha'' com as duas mãos, enquanto Clara dizia:

--- Cante! Vá!

--- Já que a senhora manda, disse ele, vou cantar.

Com todo o dengue, agarrou o violão, fez estalar as cordas e anunciou:

--- Amor e sonho.

E começou com uma voz muito alta, quase berrando, a modinha, para depois
arrastá"-la num tom mais baixo, cheio de mágoa e langor, sibilando os
``ss'', carregando os ``rr'' das metáforas horrendas de que estava cheia a
cantoria. A cousa era, porém, sincera; e mesmo as comparações
estrambóticas levantavam nos singelos cérebros das ouvintes largas
perspectivas de sonhos, erguiam desejos, despertavam anseios e visões
douradas. Acabou. Os aplausos foram entusiásticos e só Clarinha não
aplaudiu, porque, tendo sonhado durante toda a modinha, ficara ainda
embevecida quando ela acabou\ldots{}

Dias depois, vindo à janela por acaso --- era de tarde --- sem grande
surpresa, como se já o esperasse, Clara recebeu o cumprimento do cantor
magoado. Não pôs malícia na cousa, tanto assim que disse candidamente à
mãe:

--- Mamãe, sabe quem passou aí?

--- Quem?

--- ``Seu'' Júlio.

--- Que Júlio?

--- Aquele que cantou nos ``anos'' de papai.

A vida da casa, após a festança de aniversário do Joaquim, continuou a
ser a mesma. Nos domingos, aquelas partidas de bisca com o Eleutério,
servente da biblioteca, e com o Augusto, guarda municipal, acompanhadas
de copitos de cachaça, e o violão, à tarde. Não tardou que se viesse
agregar um novo comensal: era o Júlio Costa, o famoso modinheiro
suburbano, amigo íntimo do Augusto e seu professor de trovas.

Júlio quase nunca jantava, pois tinha sempre convites em todos os quatro
pontos cardeais daquelas paragens. Tomava parte nas partidas de bisca,
de parceirada, e pouco bebia. Apesar de não demorar"-se pela tarde
adentro, pôde ir cercando a rapariga, a Clara, cujos seios empinados,
volumosos e redondos o fascinavam extraordinariamente e excitavam a sua
gula carnal insaciável. Em começo foram só olhares que a moça, com os
seus úmidos olhos negros, grandes, quase cobrindo toda a esclerótica,
correspondia a furto e com medo; depois, foram pequenas frases,
galanteios, trocados às escondidas, para, afinal, vir a fatídica carta.

Ela a recebeu, meteu"-a no seio e, ao deitar"-se, leu"-a, sob a luz da
vela, medrosa e palpitante. A carta era a cousa mais fantástica, no que
diz respeito à ortografia e à sintaxe, que se pode imaginar; tinha,
porém, uma virtude: não era copiada do Secretário dos amantes, era
original. Contudo a missiva fez estremecer toda a natureza virgem de
Clara que, com a sua leitura, sentiu haver nela surgido alguma cousa de
novo, de estranho, até ali nunca sentida. Dormiu mal. Não sabia bem o
que fazer: se responder, se devolver. Viu o olhar severo do pai; as
recriminações da mãe. Ela, porém, precisava casar"-se. Não havia de ser
toda a vida assim como um cão sem dono\ldots{} Os pais viriam a morrer e ela
não podia ficar pelo mundo desamparada\ldots{} Uma dúvida lhe veio: ele era
branco; ela, mulata\ldots{} Mas que tinha isso? Tinham"-se visto tantos
casos\ldots{} Lembrou"-se de alguns\ldots{} Por que não havia de ser? Ele falava
com tanta paixão\ldots{} Ofegava, suspirava, chorava; e os seus seios duros
estouravam de virgindade e de ansiedade de amar\ldots{} Responderia; e assim
fez, no dia seguinte. As visitas de Costa tomaram"-se mais demoradas e as
cartas mais constantes. A mãe desconfiou e perguntou à filha:

--- Você está namorando ``seu'' Júlio, Clarinha?

--- Eu, mamãe! Nem penso nisso\ldots{}

--- Está, sim! Então não vejo?

A menina pôs"-se a chorar; a mãe não falou mais nisso; e Clara, logo que
pôde, mandou pelo Aristides, um molecote da vizinhança, uma carta ao
modinheiro, relatando o fato.

Júlio morava na estação próxima e a situação de sua família era bem
superior à sua namorada. O seu pai tinha um emprego regular na
prefeitura e era, em tudo, diferente do filho. Sisudo, grave, sério, ia
até a imponência grotesca do bom funcionário; e não seria capaz de
admitir que a namorada do filho dançasse na sua sala. Sua mulher não
tinha o ar solene do marido, era, porém, relaxada de modos e hábitos.
Comia com a mão, andava descalça, catava intrigas e ``novidades'' da
vizinhança; mas tinha, apesar disso, uma pretensão intima de ser grande
cousa, de uma grande família. Além do Júlio, tinha três filhas, uma das
quais já era adjunta municipal; e, das outras duas, uma estava na Escola
Normal e a mais moça cursava o Instituto de Música.

Tiravam muito ao pai, no gênio sobranceiro, no orgulho fofo da família;
e tinham ambição de casamentos doutorais. Mercedes, Adelaide e Maria
Eugênia, eram esses os nomes, não suportariam de nenhuma forma Clara
como cunhada, embora desprezassem soberbamente o irmão pelos seus maus
costumes, pelo seu violão, pelos seus plebeus galos de briga e pela sua
ignorância crassa.

Pequeno"-burguesas, sem nenhuma fortuna, mas, devido à situação do pai e
a terem frequentado escolas de certa importância, elas não admitiriam,
para Clara, senão um destino: o de criada de servir.

Entretanto, Clara era doce e meiga; inocente e boa, podia"-se dizer que
era muito superior ao irmão delas pelo sentimento, ficando talvez acima
dele pela instrução, conquanto fosse rudimentar, como não podia deixar
de ser, dada a sua condição de rapariga pobríssima. Júlio era quase
analfabeto e não tinha poder de atenção suficiente para ler o entrecho
de uma fita de cinematógrafo\footnote{Nessa época, o cinema estava
  apenas começando a se desenvolver. Nas salas de cinema as imagens eram
  projetadas por um aparelho chamado ``cinematógrafo'' e os filmes, na
  linguagem popular, eram chamados de ``fitas'', por conta das imagens
  ficaram registradas em rolos ou películas, formados por um material à
  base de tricetato de plástico, flexível e transparente. Nos jornais
  era comum aparecerem anúncios do tipo ``uma fita extraordinária'',
  para fazer alusão a algum filme. Com tempo, o termo ``fita'' passou a
  fazer parte da linguagem metafórica, servindo de alusão a determinadas
  situações complicadas ou até mesmo a escândalos que surgiam nos
  jornais. Expressões do tipo ``houve uma \emph{fita} extraordinária na
  sessão da Câmara dos Deputados'', eram correntes na imprensa da época.
  Hoje em dia, o termo ``fita'' é utilizado e conhecido como uma
  ``gíria''. Está muito presente nas letras de rap, como, por exemplo,
  no rap ``Eu sou 157'' dos Racionais \textsc{mc}'s, onde ouvimos uma parte que
  diz ``Não vou te por em \emph{fita} podre, aliado, a cena é essa, oh:
  fica ligado\ldots{}''.}. Muito estúpido, a sua vida mental se cifrava na
composição de modinhas delambidas, recheadas das mais estranhas imagens
que a sua imaginação erótica, sufocada pelas conveniências, criava,
tendo sempre perante seus olhos o ato sexual.

Mais de uma vez, ele se vira a braços com a polícia por causa de
defloramento e seduções de menores.

O pai, desde a segunda, recusara intervir; mas a mãe, dona Inês, a custo
de rogos, de choro, de apelo --- para a pureza de sangue da família,
conseguira que o marido, o capitão Bandeira, procurasse influenciar, a
fim de evitar que o filho casasse com uma negrinha de dezesseis anos, a
quem o Júlio ``tinha feito mal''.

Apesar de não ser totalmente má, os seus preconceitos junto à estreiteza
da sua inteligência não permitiram ao seu coração que agasalhasse ou
protegesse o seu infeliz neto. Sem nenhum remorso, deixou"-o por aí, à
toa, pelo mundo\ldots{}

O pai, desgostoso com o filho, largara"-o de mão; e quase não se viam.
Júlio vivia no porão da casa ou nos fundos da chácara onde tinha gaiolas
de galos de briga, o bicho mais hediondo, mais repugnantemente feroz que
é dado a olhos humanos ver. Era a sua indústria e o seu comércio, esse
negócio de galos e as suas brigas em rinhadeiros. Barganhava"-os,
vendia"-os, chocava as galinhas, apostava nas rinhas; e com o resultado
disso e com alguns cobres que a mãe lhe dava, vivia e obtinha dinheiro
para vestir"-se. Era o tipo completo do vagabundo doméstico, como há
milhares nos subúrbios e em outros bairros do Rio de Janeiro.

A mãe, sempre temendo que se repetissem os seus ajustes de contas com a
polícia, esforçava"-se sempre por estar ao corrente dos seus amores. Veio
a saber do seu último com a Clara e repreendeu"-o nos termos mais
desabridos. Ouviu"-a o filho respeitosamente, sem dizer uma palavra; mas,
julgou da boa política relatar, a seu modo, por carta, tudo à namorada.
Assim escreveu: ``Queridinha confesso"-te que ontem quando recebi a tua
carta minha mãe viu e fiquei tão louco que confessei tudo a mamãe que
lhe amava muito e fazia por você as maiores violências, ficaram todos
contra mim é a razão porque previno"-te que não ligues ao que lhe
disserem, por isso peço"-te que preze bem o meu sofrimento. Pense bem e
veja se estás resolvida a fazer o que lhe pedi na última cartinha.
Saudades e mais saudades deste infeliz que tanto lhe adora e não é
correspondido. O teu Júlio''.

Clara já estava habituada com a redação e ortografia do seu namorado,
mas, apesar de escrever muito melhor, a sua instrução era insuficiente
para desprezar um galanteador tão analfabeto. Ainda por cima, a sua
fascinação pelo modinheiro e a sua obsessão pelo casamento lhe tiravam
toda a capacidade critica que pudesse ter. A carta produziu o efeito
esperado por Júlio. Choro, palpitações, anseios vagos, esperanças
nevoentas, vislumbres de céus desconhecidos e encantados --- tudo isso
aquela carta lhe trouxe, além do halo de dedicação e amor por ela com
que Clara fez resplandecer, na imaginação, as pastinhas do violeiro. Daí
a dias, fez o prometido, isto é, deixou a janela do quarto aberta para
que ele entrasse no aposento. Repetiu a façanha quase todas as noites
seguidas, sem que ele se demorasse muito no quarto.

Nos domingos, aparecia, cantava e semelhava que entre ambos não havia
nada. Um belo dia, Clara sentiu alguma cousa de estranho no ventre.
Comunicou ao namorado. Qual! Não era nada, disse ele. Era, sim; era o
filho. Ela chorou, ele acalmou"-a, prometendo casamento. O ventre
crescia, crescia\ldots{}

O cantador de modinhas foi fugindo, deixou de aparecer a miúdo; e Clara
chorava. Ainda não lhe tinham percebido a gravidez. A mãe, porém, com
auxílio de certas intimidades próprias de mãe para filha, desconfiou e
pô"-la em confissão. Clara não pôde esconder, disse tudo; e aquelas duas
humildes mulheres choraram abraçadas diante do irremediável\ldots{} A filha
teve uma ideia:

--- Mamãe, antes da senhora dizer a papai, deixa"-me ir até à casa dele,
para falar com a sua mãe?

A velha meditou e aceitou o alvitre:

--- Vai!

Clara vestiu"-se rapidamente e foi. Recebida com altaneria por uma das
filhas, disse que queria falar à mãe de Júlio. Recebeu"-a esta
rispidamente; mas a rapariga, com toda a coragem e com sangue"-frio
difícil de crer, confessou"-lhe tudo, o seu erro e a sua desdita.

--- Mas o que é que você quer que eu faça?

--- Que ele se case comigo, fez Clara num só hausto.

--- Ora, esta! Você não se enxerga! Você não vê mesmo que meu filho não é
para se casar com gente da laia de você! Ele não amarrou você, ele não
amordaçou você\ldots{} Vá"-se embora, rapariga! Ora já se viu! Vá!

Clara saiu sem dizer nada, reprimindo as lágrimas, para que na rua não
lhe descobrissem a vergonha. Então, ela? Então ela não se podia casar
com aquele calaceiro, sem nenhum título, sem nenhuma qualidade superior?
Por quê?

Viu bem a sua condição na sociedade, o seu estado de inferioridade
permanente, sem poder aspirar a cousa mais simples a que todas as moças
aspiram. Para que seriam aqueles cuidados todos de seus pais? Foram
inúteis e contraproducentes, pois evitaram que ela conhecesse bem
justamente a sua condição e os limites das suas aspirações
sentimentais\ldots{} Voltou para casa depressa. Chegou; o pai ainda não
viera.

Foi ao encontro da mãe. Não lhe disse nada; abraçou"-a chorando. A mãe
também chorou e, quando Clara parou de chorar, entre soluços, disse:

--- Mamãe, eu não sou nada nesta vida.



\chapter[O Filho de Gabriela]{O Filho de Gabriela\footnote[*]{Publicado junto à primeira edição de \emph{Triste Fim de Policarpo Quaresma}, em 1915.}}

\hfill\emph{A Antônio Noronha Santos}

\setcounter{footnote}{-1}
{\setlength{\epigraphwidth}{.6\textwidth}
\epigraph{\emph{Chaque progrès, au fond, est un avortement\\
Mais l'échec même sert.}\footnotemark}{\textsc{guyau}\footnotemark}}

\setcounter{footnote}{1}
\footnotetext{Frase francesa que significa: ``Qualquer progresso, no fundo, é um fracasso. Mas o fracasso em si é proveitoso''.}
\setcounter{footnote}{2}
\footnotetext{Sobre o filósofo Guyau, ver nota 10, à página \pageref{guyau}.}

\noindent{}-- Absolutamente não pode continuar assim\ldots{} Já passa\ldots{} É
todo o dia! Arre!

--- Mas é meu filho, minh'ama.

--- E que tem isso? Os filhos de vocês agora têm tanto luxo. Antigamente,
criavam"-se à toa; hoje, é um deus nos acuda; exigem cuidados, têm
moléstias\ldots{} Fique sabendo: não pode ir amanhã!

--- Ele vai melhorando, Dona Laura; e o doutor disse que não deixasse de
levá"-lo lá, amanhã\ldots{}

--- Não pode, não pode, já lhe disse! O conselheiro precisa chegar cedo à
escola; há exames e tem que almoçar cedo\ldots{} Não vai, não senhora! A
gente tem criados pra que? Não vai, não!

--- Vou, e vou sim!\ldots{} Que bobagem!\ldots{} Quer matar o pequeno,
não é? Pois sim\ldots{} Está"-se ``ninando''\ldots{}

--- O que é que você disse, hein?

--- É isso mesmo: vou e vou!

--- Atrevida.

--- Atrevida é você, sua\ldots{} Pensa que não sei\ldots{}

Em seguida as duas mulheres se puseram caladas durante um instante: a
patroa --- uma alta senhora, ainda moça, de uma beleza suave e marmórea
--- com os lábios finos muito descorados e entreabertos, deixando ver os
dentes aperolados, muito iguais, cerrados de cólera; a criada agitada,
transformada, com faiscações desusadas nos olhos pardos e tristes. A
patroa não se demorou assim muito tempo. Violentamente contraída naquele
segundo a sua fisionomia repentinamente se abriu num choro convulsivo.

A injúria da criada, decepções matrimoniais, amarguras do seu ideal
amoroso, fatalidades de temperamento, todo aquele obscuro drama de sua
alma, feito de uma porção de coisas que não chegava bem a colher, mas
nas malhas das quais se sentia presa e sacudida, subiu"-lhe de repente à
consciência, e ela chorou.

Na sua simplicidade popular, a criada também se pôs a chorar,
enternecida pelo sofrimento que ela mesma provocara na ama.

E ambas, pelo fim dessa transfiguração inopinada, entreolharam"-se
surpreendidas, pensando que se acabavam de conhecer naquele instante,
tendo até ali vagas notícias uma da outra, como se vivessem longe, tão
longe, que só agora haviam distinguido bem nitidamente o tom de voz
próprio a cada uma delas.

No entendimento peculiar de uma e de outra, sentiram"-se irmãs na
desoladora mesquinhez da nossa natureza e iguais, como frágeis
consequências de um misterioso encadear de acontecimentos, cuja ligação
e fim lhes escapavam completamente, inteiramente\ldots{}

A dona da casa, à cabeceira da mesa de jantar, manteve"-se silenciosa,
correndo, de quando em quando, o olhar ainda úmido pelas ramagens do
atoalhado, indo, às vezes, com ele até à bandeira da porta defronte,
donde pendia a gaiola do canário, que se sacudia na prisão niquelada.

De pé, a criada avançou algumas palavras. Desculpou"-se inábil e
despediu"-se humilde.

--- Deixe"-se disso, Gabriela, disse Dona Laura. Já passou tudo; eu não
guardo rancor; fique! Leve o pequeno amanhã\ldots{} Que vai você fazer
por esse mundo afora?

--- Não senhora\ldots{} Não posso\ldots{} É que\ldots{}

E de um hausto falou com tremuras na voz:

--- Não posso, não minh'ama; vou"-me embora!

Durante um mês, Gabriela andou de bairro em bairro, à procura de
aluguel. Pedia lessem"-lhe anúncios, corria, seguindo as indicações, a
casas de gente de toda a espécie. Sabe cozinhar? perguntavam. --- Sim,
senhora, o trivial. --- Bem, e lavar? Serve de ama? --- Sim, senhora; mas
se fizer uma coisa, não quero fazer outra. --- Então, não me serve,
concluía a dona da casa. É um luxo\ldots{} Depois queixam"-se que não têm
aonde se empreguem\ldots{}

Procurava outras casas; mas nesta já estavam servidas, naquela o salário
era pequeno e naquela outra queriam que dormisse em casa e não trouxesse
o filho.

A criança, durante esse mês, viveu relegada a um canto da casa de uma
conhecida da mãe. Um pobre quarto de estalagem, úmido que nem uma
masmorra. De manhã, via a mãe sair; à tarde, quase à boca da noite,
via"-a entrar desconfortada. Pelo dia em fora, ficava num abandono de
enternecer. A hóspede, de longe em longe, olhava"-o cheia de raiva. Se
chorava aplicava"-lhe palmadas e gritava colérica: ``Arre diabo! A
vagabunda de tua mãe anda saracoteando\ldots{} Cala a boca, demônio!
Quem te fez, que te ature\ldots{}''.

Aos poucos, a criança torrou"-se de medo; nada pedia, sofria fome, sede,
calado. Enlanguescia a olhos vistos e sua mãe, na caça de aluguel, não
tinha tempo para levá"-lo ao doutor do posto médico. Baço, amarelado,
tinha as pernas que nem palitos e o ventre como o de um batráquio. A mãe
notava"-lhe o enfraquecimento, os progressos da moléstia e desesperava,
não sabendo que alvitre tomar. Um dia pelos outros, chegava em casa
semiembriagada, escorraçando o filho e trazendo algum dinheiro. Não
confessava a ninguém a origem dele; em outros mal entrava, beijava muito
o pequeno, abraçava"-o. E assim corria a cidade. Numa destas correrias
passou pela porta do conselheiro, que era o marido de Dona Laura. Estava
no portão, a lavadeira, parou e falou"-lhe; nisto, viu aparecer a sua
antiga patroa numa janela lateral. ``-- Bom dia minh'ama,'' --- ``Bom
dia, Gabriela. Entre.'' Entrou. A esposa do conselheiro perguntou"-lhe se
já tinha emprego; respondeu"-lhe que não. ``Pois olha, disse"-lhe a
senhora, eu ainda não arranjei cozinheira, se tu queres\ldots{}''

Gabriela quis recusar, mas Dona Laura insistiu.

Entre elas, parecia que havia agora certo acordo íntimo, um quê de mútua
proteção e simpatia. Uma tarde em que Dona Laura voltava da cidade, o
filho da Gabriela, que estava no portão, correu imediatamente para a
moça e disse"-lhe, estendendo a mão: ``a bênção''. Havia tanta tristeza
no seu gesto, tanta simpatia e sofrimento, que aquela alta senhora não
lhe pôde negar a esmola de um afago, de uma carícia sincera. Nesse dia,
a cozinheira notou que ela estava triste e, no dia seguinte, não foi sem
surpresa que Gabriela se ouviu chamar.

--- O Gabriela!

--- Minh'ama.

--- Vem cá.

Gabriela concertou"-se um pouco e correu à sala de jantar, onde estava a
ama.

--- Já batizaste o teu pequeno? perguntou"-lhe ela ao entrar.

--- Ainda não.

--- Por quê? Com quatro anos!

--- Por quê? Porque ainda não houve ocasião\ldots{}

--- Já tens padrinhos?

--- Não, senhora.

--- Bem; eu e o conselheiro vamos batizá"-lo. Aceitas?

Gabriela não sabia como responder, balbuciou alguns agradecimentos e
voltou ao fogão com lágrimas nos olhos.

O conselheiro condescendeu e cuidadosamente começou a procurar um nome
adequado. Pensou em Huáscar, Ataliba, Guatemozim; consultou dicionários,
procurou nomes históricos, afinal resolveu"-se por ``Horácio'', sem saber
por quê.

Assim se chamou e cresceu. Conquanto tivesse recebido um tratamento
médico regular e a sua vida na casa do conselheiro fosse relativamente
confortável, o pequeno Horácio não perdeu nem a reserva nem o enfezado
dos seus primeiros anos de vida. À proporção que crescia, os traços se
desenhavam, alguns finos: o corte da testa, límpida e reta; o olhar doce
e triste, como o da mãe, onde havia, porém, alguma coisa a mais --- um
fulgor, certas expressões particulares, principalmente quando calado e
concentrado. Não obstante, era feio, embora simpático e bom de ver.

Pelos seis anos, mostrava"-se taciturno, reservado e tímido, olhando
interrogativamente as pessoas e coisas, sem articular uma pergunta. Lá
vinha um dia, porém, que o Horácio rompia numa alegria ruidosa; punha"-se
a correr, a brincar, a cantarolar, pela casa toda, indo do quintal para
as salas, satisfeito, contente, sem motivo e sem causa.

A madrinha espantava"-se com esses bruscos saltos de humor, queria
entendê"-los, explicá"-los e começou por se interessar pelos seus
trejeitos. Um dia, vendo o afilhado a cantar, a brincar, muito contente,
depois de uma porção de horas de silêncio e calma, correu ao piano e
acompanhou"-lhe a cantiga, depois, emendou com uma ária qualquer. O
menino calou"-se, sentou"-se no chão e pôs"-se a olhar, com olhos
tranquilos e calmos, a madrinha, inteiramente delido nos sons que saíam
dos seus dedos. E quando o piano parou, ele ainda ficou algum tempo
esquecido naquela postura, com o olhar perdido numa cisma sem fim. A
atitude imaterial do menino tocou a madrinha, que o tomou ao colo,
abraçando"-o e beijando"-o, num afluxo de ternura, a que não eram
estranhos os desastres de sua vida sentimental.

Pouco depois a mãe lhe morria. Até então vivia numa semidomesticidade.
Daí em diante, porém, entrou completamente na família do Conselheiro
Calaça. Isso, entretanto, não lhe retirou a taciturnidade e a reserva;
ao contrário, fechou"-se em si e nunca mais teve crises de alegria.

Com sua mãe ainda tinha abandonos de amizade, efusões de carícias e
abraços. Morta que ela foi, não encontrou naquele mundo tão diferente,
pessoa a quem se pudesse abandonar completamente, embora pela madrinha
continuasse a manter uma respeitosa e distante amizade, raramente
aproximada por uma carícia, por um afago. Ia para o colégio calado,
taciturno, quase carrancudo, e, se, pelo recreio, o contágio obrigava"-o
a entregar"-se à alegria e aos folguedos, bem cedo se arrependia,
encolhia"-se e sentava"-se, vexado, a um canto. Voltava do colégio como
fora, sem brincar pelas ruas, sem traquinadas, severo e insensível.
Tendo uma vez brigado com um colega, a professora o repreendeu
severamente, mas o conselheiro, seu padrinho, ao saber do caso, disse
com rispidez: ``Não continue, hein? O senhor não pode brigar --- está
ouvindo?''

E era assim sempre o seu padrinho, duro, desdenhoso, severo em demasia
com o pequeno, de quem não gostava, suportando"-o unicamente em atenção à
mulher --- maluquices da Laura, dizia ele. Por vontade dele, tinha"-o
posto logo num asilo de menores, ao morrer"-lhe a mãe; mas a madrinha não
quis e chegou até a conseguir que o marido o colocasse num
estabelecimento oficial de instrução secundária, quando acabou com
brilho o curso primário.

Não foi sem resistência que ele acedeu, mas os rogos da mulher, que
agora juntava à afeição pelo pequeno uma secreta esperança no seu
talento, tanto fizeram que o conselheiro se empenhou e obteve.

Em começo, aquela adoção fora um simples capricho de Dona Laura; mas,
com o tempo, os seus sentimentos pelo menino foram ganhando importância
e ficando profundos, embora exteriormente o tratasse com um pouco de
cerimônia.

Havia nela mais medo da opinião, das sentenças do conselheiro, do que
mesmo necessidade de disfarçar o que realmente sentia, e pensava.

Quem a conheceu solteira, muito bonita, não a julgaria capaz de tal
afeição; mas, casada, sem filhos, não encontrando no casamento nada que
sonhara, nem mesmo o marido, sentiu o vazio da existência, a inanidade
dos seus sonhos, o pouco alcance da nossa vontade; e, por uma
reviravolta muito comum, começou a compreender confusamente todas as
vidas e almas, a compadecer"-se e a amar tudo, sem amar bem coisa alguma.
Era uma parada de sentimento e a corrente que se acumulara nela,
perdendo"-se do seu leito natural, extravasara e inundara tudo.

Tinha um amante e já tivera outros, mas não era bem a parte mística do
amor que procurara neles. Essa, ela tinha certeza que jamais podia
encontrar; era a parte dos sentidos tão exuberantes e exaltados depois
das suas contrariedades morais.

Pelo tempo em que o seu afilhado entrara para o colégio secundário, o
amante rompera com ela; e isto a fazia sofrer, tinha medo de não possuir
mais beleza suficiente para arranjar um outro como ``aquele''. E a esse
desastre sentimental não foi estranha a energia dos seus rogos junto ao
marido para admissão do Horácio no estabelecimento oficial.

O conselheiro, homem de mais de sessenta anos, continuava superiormente
frio, egoísta e fechado, sonhando sempre uma posição mais alta ou que
julgava mais alta. Casara"-se por necessidade decorativa. Um homem de sua
posição não podia continuar viúvo; atiraram"-lhe aquela menina pelos
olhos, ela o aceitou por ambição e ele por conveniência. No mais, lia os
jornais, o câmbio especialmente, e, de manhã passava os olhos nas
apostilas de sua cadeira --- apostilas por ele organizadas, há quase
trinta anos, quando dera as suas primeiras lições, moço, de vinte e
cinco anos, genial nas aprovações e nos prêmios.

Horácio, toda a manhã, ao sair para o colégio, lá avistava o padrinho
atarraxado na cadeira de balanço a ler atentamente o jornal: ``A bênção,
meu padrinho!'' --- ``Deus te abençoe'', dizia ele, sem menear a cabeça
do espaldar e no mesmo tom de voz com que pediria os chinelos à criada.

Em geral, a madrinha estava deitada ainda e o menino saía para o
ambiente ingrato da escola, sem um adeus, sem dar um beijo, sem ter quem
lhe reparasse familiarmente o paletó. Lá ia. A viagem de bonde, ele a
fazia humilde, espremido a um canto do veículo, medroso que seu paletó
roçasse as sedas de uma rechonchuda senhora ou que seus livros tocassem
nas calças de um esquelético capitão de uma milícia qualquer. Pelo
caminho, arquitetava fantasias; seu espírito divagava sem nexo. À
passagem de um oficial a cavalo, imaginava"-se na guerra, feito general,
voltando vencedor, vitorioso de ingleses, de alemães, de americanos e
entrando pela Rua do Ouvidor aclamado como nunca se fora aqui. Na sua
cabeça ainda infantil, em que a fraqueza de afetos próximos concentrava
o pensamento, a imaginação palpitava, tinha uma grande atividade,
criando toda a espécie de fantasmagorias que lhe apareciam como fatos
possíveis, virtuais.

Eram"-lhe as horas de aula um bem triste momento. Não que fosse vadio,
estudava o seu bocado, mas o espetáculo do saber, por um lado grandioso
e apoteótico, pela boca dos professores, chegava"-lhe tisnado e um quê
desarticulado. Não conseguia ligar bem umas coisas às outras, além do
que tudo aquilo lhe aparecia solene, carrancudo e feroz. Um teorema
tinha o ar autoritário de um régulo selvagem; e aquela gramática cheia
de regrinhas, de exceções, uma coisa cabalística, caprichosa e sem
aplicação útil.

O mundo parecia"-lhe uma coisa dura, cheia de arestas cortantes,
governado por uma porção de regrinhas de três linhas, cujo segredo e
aplicação estavam entregues a uma casta de senhores, tratáveis uns,
secos outros, mas todos velhos e indiferentes.

Aos seus exames ninguém assistia, nem por eles alguém se interessava;
contudo, foi sempre regularmente aprovado.

Quando voltava do colégio, procurava a madrinha e contava"-lhe o que se
dera nas aulas. Narrava"-lhe pequenas particularidades do dia, as notas
que obtivera e as travessuras dos colegas.

Uma tarde, quando isso ia fazer, encontrou Dona Laura atendendo a uma
visita. Vendo"-o entrar e falar à dona da casa, tomando"-lhe a bênção, a
senhora estranha perguntou: ``Quem é este pequeno?'' --- ``E meu
afilhado'', disse"-lhe Dona Laura. ``Teu afilhado? Ahn! sim! É o filho da
Gabriela\ldots{}''.

Horácio ainda esteve um instante calado, estatelado e depois chorou
nervosamente.

Quando se retirou observou a visita à madrinha:

--- Você está criando mal esta criança. Faz"-lhe muitos mimos, está lhe
dando nervos\ldots{}

--- Não faz mal. Podem levá"-lo longe.

E assim corria a vida do menino em casa do conselheiro.

Um domingo ou outro, só ou com um companheiro, vagava pelas praias,
pelos bondes ou pelos jardins. O Jardim Botânico era"-lhe preferido. Ele
e o seu constante amigo Salvador sentavam"-se a um banco, conversavam
sobre os estudos comuns, maldiziam este ou aquele professor. Por fim, a
conversa vinha a enfraquecer; os dois se calavam instantes. Horácio
deixava"-se penetrar pela flutuante poesia das coisas, das árvores, dos
céus, das nuvens; acariciava com o olhar as angustiadas colunas das
montanhas, simpatizava com o arremesso dos píncaros, depois deixava"-se
ficar, ao chilreio do passaredo, cismando vazio, sem que a cisma lhe
fizesse ver coisa definida, palpável pela inteligência. Ao fim,
sentia"-se como que liquefeito, vaporizado nas coisas era como se
perdesse o feitio humano e se integrasse naquele verde escuro da mata ou
naquela mancha faiscante de prata que a água a correr deixava na encosta
da montanha. Com que volúpia, em tais momentos, ele se via dissolvido na
natureza, em estado de fragmentos, em átomos, sem sofrimento, sem
pensamento, sem dor! Depois de ter ido ao indefinido, apavorava"-se com o
aniquilamento e voltava a si, aos seus desejos, às suas preocupações com
pressa e medo. --- Salvador, de que gostas mais, do inglês ou francês? ---
Eu do francês; e tu? --- Do inglês. --- Por quê? Porque pouca gente o
sabe.

A confidência saía"-lhe a contragosto, era dita sem querer. Temeu que o
amigo o supusesse vaidoso. Não era bem esse sentimento que o animava;
era uma vontade de distinção, de reforçar a sua individualidade, que ele
sentia muito diminuída pelas circunstâncias ambientes. O amigo não
entrava na natureza do seu sentimento e despreocupadamente perguntou: ---
Horácio, já assististe uma festa de São João? --- Nunca. --- Queres
assistir uma? --- Quero, onde? --- Na ilha, em casa de meu tio.

Pela época, a madrinha consentiu. Era um espetáculo novo; era um outro
mundo que se abria aos seus olhos. Aquelas longas curvas das praias, que
perspectivas novas não abriam em seu espírito! Ele se ia todo nas
cristas brancas das ondas e nos largos horizontes que descortinava.

Em chegando a noite, afastou"-se da sala. Não entendia aqueles folguedos,
aquele dançar sôfrego, sem pausa, sem alegria, como se fosse um castigo.
Sentado a um banco do lado de fora, pôs"-se a apreciar a noite, isolado,
oculto, fugido, solitário, que se sentia ser no ruído da vida. Do seu
canto escuro, via tudo mergulhado numa vaga semiluz. No céu negro, a luz
pálida das estrelas; na cidade defronte, o revérbero da iluminação; luz,
na fogueira votiva, nos balões ao alto, nos foguetes que espoucavam, nos
fogaréus das proximidades e das distâncias --- luzes contínuas,
instantâneas, pálidas, fortes; e todas no conjunto pareciam representar
um esforço enorme para espancar as trevas daquela noite de mistérios.

No seio daquela bruma iluminada, as formas das árvores boiavam como
espectros; o murmúrio do mar tinha alguma coisa de penalizado diante do
esforço dos homens e dos astros para clarear as trevas. Havia naquele
instante, em todas as almas, um louco desejo de decifrar o mistério que
nos cerca; e as fantasias trabalhavam para idear meios que nos fizessem
comunicar com o Ignorado, com o Invisível. Pelos cantos sombrios da
chácara pessoas deslizavam. Iam ao poço ver a sombra --- sinal de que
viveriam o ano; iam disputar galhos de arruda ao diabo; pelas janelas,
deixavam copos com ovos partidos para que o sereno, no dia seguinte,
trouxesse as mensagens do Futuro.

O menino, sentindo"-se arrastado por aquele frêmito de augúrio e
feitiçaria, percebeu bem como vivia envolvido, mergulhado, no
indistinto, no indecifrável; e uma onda de pavor, imensa e aterradora,
cobriu"-lhe o sentimento.

Dolorosos foram os dias que se seguiram. O espírito sacolejou"-lhe o
corpo violentamente. Com afinco estudava, lia os compêndios; mas não
compreendia, nada retinha. O seu entendimento como que vazava. Voltava,
lia, lia e lia e, em seguida, virava as folhas sofregamente,
nervosamente, como se quisesse descobrir debaixo delas um outro mundo
cheio de bondade e satisfação. Horas havia que ele desejava abandonar
aqueles livros, aquela lenta aquisição de noções e ideias, reduzir"-se e
anular"-se; horas havia, porém, que um desejo ardente lhe vinha de
saturar"-se de saber, de absorver todo o conjunto das ciências e das
artes. Ia de um sentimento para outro; e foi vã a agitação. Não
encontrava solução, saída; a desordem das ideias e a incoerência das
sensações não lhe podiam dar uma e cavavam"-lhe a saúde. Tornou"-se mais
flébil, fatigava"-se facilmente. Amanhecia cansado de dormir e dormia
cansado de estar em vigília. Vivia irritado, raivoso, não sabia contra
quem.

Certa manhã, ao entrar na sala de jantar, deu com o padrinho a ler os
jornais, segundo o seu hábito querido.

--- Horácio, você passe na casa do Guedes e traga"-me a roupa que mandei
consertar.

--- Mande outra pessoa buscar.

--- O que?

--- Não trago.

--- Ingrato! Era de esperar\ldots{}

E o menino ficou admirado diante de si mesmo, daquela saída de sua
habitual timidez.

Não sabia onde tinha ido buscar aquele desaforo imerecido, aquela tola
má"-criação; saiu"-lhe como uma coisa soprada por outro e que ele
unicamente pronunciasse.

A madrinha interveio, aplainou as dificuldades; e, com a agilidade de
espírito peculiar ao sexo, compreendeu o estado d'alma do rapaz.
Reconstituiu"-o com os gestos, com os olhares, com as meias palavras, que
percebera em tempos diversos e cuja significação lhe escapara no
momento, mas que aquele ato, desusadamente brusco e violento, aclarava
por completo. Viu"-lhe o sofrimento de viver à parte, a transplantação
violenta, a falta de simpatia, o princípio de ruptura que existia em sua
alma, e que o fazia passar aos extremos das sensações e dos atos.

Disse"-lhe coisas doces, ralhou"-o, aconselhou"-o, acenou"-lhe com a
fortuna, a glória e o nome.

Foi Horácio para o colégio abatido, preso de um estranho sentimento de
repulsa, de nojo por si mesmo. Fora ingrato, de fato; era um monstro. Os
padrinhos lhe tinham dado tudo, educado, instruído. Fora sem querer,
fora sem pensar; e sentia bem que a sua reflexão não entrara em nada
naquela resposta que dera ao padrinho. Em todo o caso, as palavras foram
suas, foram ditas com sua voz e a sua boca, e se lhe nasceram do íntimo
sem a colaboração da inteligência, devia acusar"-se de ser
fundamentalmente mau\ldots{}

Pela segunda aula, pediu licença. Sentia"-se doente, doía"-lhe a cabeça e
parecia que lhe passavam um archote fumegante pelo rosto.

--- Já, Horácio? perguntou"-lhe a madrinha, vendo"-o entrar.

--- Estou doente.

E dirigiu"-se para o quarto. A madrinha seguiu"-o. Chegado que foi,
atirou"-se à cama, ainda meio"-vestido.

--- Que é que você tem, meu filho?

--- Dores de cabeça\ldots{} um calor\ldots{}

A madrinha tomou"-lhe o pulso, assentou as costas da mão na testa e
disse"-lhe ainda algumas palavras de consolação: que aquilo não era nada;
que o padrinho não lhe tinha rancor; que sossegasse.

O rapaz, deitado, com os olhos semicerrados, parecia não ouvir;
voltava"-se de um lado para outro; passava a mão pelo rosto, arquejava e
debatia"-se. Um instante pareceu sossegar; ergueu"-se sobre o travesseiro
e chegou a mão aos olhos, no gesto de quem quer avistar alguma coisa ao
longe. A estranheza do gesto assustou a madrinha.

--- Horácio!\ldots{} Horácio!\ldots{}

--- Estou dividido\ldots{} Não sai sangue\ldots{}

--- Horácio, Horácio, meu filho!

--- Faz sol\ldots{} Que sol!\ldots{} Queima\ldots{}Árvores
enormes\ldots{} Elefantes\ldots{}

--- Horácio, que é isso? Olha; é tua madrinha!

--- Homens negros\ldots{} fogueiras\ldots{} Um se estorce\ldots{} Chi!
Que coisa!\ldots{} O meu pedaço dança\ldots{}

--- Horácio! Genoveva, traga água de flor\ldots{} Depressa, um
médico\ldots{} Vá chamar, Genoveva!

--- Já não é o mesmo\ldots{} é outro\ldots{} lugar, mudou\ldots{} uma
casinha branca\ldots{} carros de bois\ldots{} nozes\ldots{}
figos\ldots{} lenços\ldots{}

--- Acalma"-te, meu filho!

--- Ué! Chi! Os dois brigam\ldots{}

Daí em diante a prostração tomou"-o inteiramente. As últimas palavras não
saíam perfeitamente articuladas. Pareceu sossegar. O médico entrou,
tomou a temperatura, examinou"-o e disse com a máxima segurança:

--- Não se assuste, minha senhora. É delírio febril, simplesmente. Dê"-lhe
o purgante, depois as cápsulas, que, em breve, estará bom.



\chapter[O Pecado]{O Pecado\footnote[*]{Publicado na edição de agosto de 1924 da \emph{Revista Souza Cruz}.}}

Quando naquele dia São Pedro despertou, despertou risonho e de bom
humor. E, terminados os cuidados higiênicos da manhã, ele se foi à
competente repartição celestial buscar ordens do Supremo e saber que
almas chegariam na próxima leva.

Em uma mesa longa, larga e baixa, um grande livro aberto se estendia e
debruçado sobre ele, todo entregue ao serviço, um guarda"-livros punha em
dia a escrituração das almas, de acordo com as mortes que Anjos
mensageiros e noticiosos traziam de toda a extensão da terra. Da pena do
encarregado celeste escorriam grossas letras, e de quando em quando ele
mudava a caneta para melhor talhar um outro caráter caligráfico.

Assim, páginas ia ele enchendo, enfeitadas, iluminadas em os mais
preciosos tipos de letras. Havia, no emprego de cada um deles, uma certa
razão de ser e entre si guardavam tão feliz disposição que encantava o
ver uma página escrita do livro. O nome era escrito em bastardo, letra
forte e larga; a filiação em gótico, tinha um ar religioso, antigo, as
faltas, em bastardo e as qualidades em ronde arabescado\footnote{\emph{Bastardo,
  gótico e ronde} eram tipos de letras utilizadas pelos escrivães e
  monges copistas medievais, algo que conhecemos hoje como ``fontes'',
  disponíveis nos editores de texto dos computadores e celulares.}.

Ao entrar São Pedro, o escriturário do Eterno voltou"-se, saudou"-o e, à
reclamação da lista d'almas pelo Santo, ele respondeu com algum enfado
(enfado do ofício) que viesse à tarde buscá"-la.

Aí pela tardinha, ao findar a escrita, o funcionário celeste (um velho
jesuíta encanecido no tráfico de açúcar da América do Sul) tirava uma
lista explicativa e entregava a São Pedro a fim de se preparar
convenientemente para receber os ex"-vivos no dia seguinte.

Dessa vez ao contrário de todo o sempre, São Pedro, antes de sair, leu
de antemão a lista; e essa sua leitura foi útil, pois que se a não
fizesse talvez, dali em diante, para o resto das idades --- quem sabe? ---
o Céu ficasse de todo estragado. Leu São Pedro a relação: havia muitas
almas, muitas mesmo, delas todas, à vista das explicações apensas, uma
lhe assanhou o espanto e a estranheza. Leu novamente. Vinha assim:

P. L. C., filho de\ldots{}, neto de\ldots{}, bisneto de\ldots{} Carregador, quarenta e
oito anos. Casado. Casto. Honesto. Caridoso. Pobre de espírito. Ignaro.
Bom como São Francisco de Assis. Virtuoso como São Bernardo e meigo como
o próprio Cristo. E um justo.

Deveras, pensou o Santo Porteiro, é uma alma excepcional; com tão
extraordinárias qualidades bem merecia assentar"-se à direita do Eterno e
lá ficar, \emph{per saecula saeculoram}\footnote{Frase escrita em latim,
  que significa ``pelos séculos dos séculos''.}, gozando a glória perene
de quem foi tantas vezes Santo\ldots{}

--- E por que não ia? deu"-lhe vontade de perguntar ao seráfico burocrata.

--- Não sei, retrucou"-lhe este. Você sabe, acrescentou, sou mandado\ldots{}

--- Veja bem nos assentamentos. Não vá ter você se enganado. Procure,
retrucou por sua vez o velho pescador canonizado.

Acompanhado de dolorosos rangidos da mesa, o guarda"-livros foi folheando
o enorme Registro até encontrar a página própria, onde com certo esforço
achou a linha adequada e com o dedo afinal apontou o assentamento e leu
alto:

--- P. L. C., filho de\ldots{} neto de\ldots{} bisneto de\ldots{}Carregador. Quarenta e
oito anos. Casado. Honesto. Caridoso. Leal. Pobre de espírito. Ignaro.
Bom como São Francisco de Assis. Virtuoso como São Bernardo e meigo como
o próprio Cristo. E um justo.

Levando o dedo pela pauta horizontal e nas ``Observações'', deparou
qualquer coisa que o fez dizer de súbito:

--- Esquecia"-me\ldots{} Houve engano. E! Foi bom você falar. Essa alma é a de
um negro. Vai para o purgatório.



\chapter[O cemitério]{O cemitério\footnote[*]{Publicado na edição de maio de 1924 da \emph{Revista Souza Cruz}.}}

Pelas ruas de túmulos, fomos calados. Eu olhava vagamente aquela
multidão de sepulturas, que trepavam, tocavam"-se, lutavam por espaço, na
estreiteza da vaga e nas encostas das colinas aos lados. Algumas
pareciam se olhar com afeto, roçando"-se amigavelmente; em outras,
transparecia a repugnância de estarem juntas. Havia solicitações
incompreensíveis e também repulsões e antipatias; havia túmulos
arrogantes, imponentes, vaidosos e pobres e humildes; e, em todos,
ressumava o esforço extraordinário para escapar ao nivelamento da morte,
ao apagamento que ela traz às condições e às fortunas.

Amontoavam"-se esculturas de mármore, vasos, cruzes e inscrições; iam
além; erguiam pirâmides de pedra tosca, faziam caramanchéis\footnote{\emph{Caramanchel}
  ou \emph{caramanchão} é o nome dado a uma pequena construção,
  normalmente de madeira, coberto com folhas ou tenhas de barro, que
  serve de abrigo do sol ou da chuva.} extravagantes, imaginavam
complicações de matos e plantas --- coisas brancas e delirantes, de um
mau gosto que irritava. As inscrições exuberavam; longas, cheias de
nomes, sobrenomes e datas, não nos traziam à lembrança nem um nome
ilustre sequer; em vão procurei ler nelas celebridades, notabilidades
mortas; não as encontrei. E de tal modo a nossa sociedade nos marca um
tão profundo ponto, que até ali, naquele campo de mortos, mudo
laboratório de decomposição, tive uma imagem dela, feita
inconscientemente de um propósito, firmemente desenhada por aquele
acesso de túmulos pobres e ricos, grotescos e nobres, de mármore e
pedra, cobrindo vulgaridades iguais umas às outras por força estranha às
suas vontades, a lutar\ldots{}

Fomos indo. A carreta, empunhada pelas mãos profissionais dos
empregados, ia dobrando as alamedas, tomando ruas, até que chegou à boca
do soturno buraco, por onde se via fugir, para sempre do nosso olhar, a
humildade e a tristeza do contínuo da Secretaria dos Cultos.

Antes que lá chegássemos, porém, detive"-me um pouco num túmulo de
límpidos mármores, ajeitados em capela gótica, com anjos e cruzes que a
rematavam pretensiosamente.

Nos cantos da lápide, vasos com flores de biscuit e, debaixo de um
vidro, à nívea altura da base da capelinha, em meio corpo, o retrato da
morta que o túmulo engolira. Como se estivesse na Rua do Ouvidor, não
pude suster um pensamento mau e quase exclamei:

--- Bela mulher!

Estive a ver a fotografia e logo em seguida me veio à mente que aqueles
olhos, que aquela boca provocadora de beijos, que aqueles seios túmidos,
tentadores de longos contatos carnais, estariam àquela hora reduzidos a
uma pasta fedorenta, debaixo de uma porção de terra embebida de gordura.

Que resultados teve a sua beleza na terra? Que coisas eternas criaram os
homens que ela inspirou? Nada, ou talvez outros homens, para morrer e
sofrer. Não passou disso, tudo mais se perdeu; tudo mais não teve
existência, nem mesmo para ela e para os seus amados; foi breve,
instantâneo, e fugaz.

Abalei"-me! Eu que dizia a todo o mundo que amava a vida, eu que afirmava
a minha admiração pelas coisas da sociedade --- eu meditar como um
cientista profeta hebraico! Era estranho! Remanescente de noções que se
me infiltraram e cuja entrada em mim mesmo eu não percebera! Quem pode
fugir a elas?

Continuando a andar, adivinhei as mãos da mulher, diáfanas e de dedos
longos; compus o seu busto ereto e cheio, a cintura, os quadris, o
pescoço, esguio e modelado, as espáduas brancas, o rosto sereno e
iluminado por um par de olhos indefinidos de tristeza e desejos\ldots{}

Já não era mais o retrato da mulher do túmulo; era de uma, viva, que me
falava.

Com que surpresa, verifiquei isso.

Pois eu, eu que vivia desde os dezesseis anos, despreocupadamente,
passando pelos meus olhos, na Rua do Ouvidor, todos os figurinos dos
jornais de modas, eu me impressionar por aquela menina do cemitério! Era
curioso.

E, por mais que procurasse explicar, não pude.



\chapter[Milagre do Natal]{Milagre do Natal\footnote[*]{Publicado em anexo no volume \emph{Vida e Morte de M. J. Gonzaga de Sá}, em 1949.}}

O bairro do Andaraí\footnote{Bairro localizado na zona norte da cidade
  do Rio de Janeiro.} é muito triste e muito úmido. As montanhas que
enfeitam a nossa cidade, aí tomam maior altura e ainda conservam a densa
vegetação que as devia adornar com mais força em tempos idos. O tom
plúmbeo das árvores como que enegrece o horizonte e torna triste o
arrabalde.

Nas vertentes dessas mesmas montanhas, quando dão para o mar, este
quebra a monotonia do quadro e o sol se espadana mais livremente,
obtendo as coisas humanas, minúsculas e mesquinhas, uma garridice e uma
alegria que não estão nelas, mas que sê percebem nelas. As tacanhas
casas de Botafogo se nos afigura assim; as bombásticas ``vilas'' de
Copacabana, também; mas, no Andaraí, tudo fica esmagado pela alta
montanha e sua sombria vegetação.

Era numa rua desse bairro que morava Feliciano Campossolo Nunes, chefe
de secção do Tesouro Nacional, ou antes e melhor: subdiretor. A casa era
própria e tinha na cimalha este dístico pretensioso: ``Vila
Sebastiana''. O gosto da fachada, as proporções da casa não precisam ser
descritas: todos conhecem um e as outras. Na frente, havia um
jardinzinho que se estendia para a esquerda, oitenta centímetros a um
metro, além da fachada. Era o vão que correspondia à varanda lateral,
quase a correr todo o prédio. Campossolo era um homem grave, ventrudo,
calvo, de mãos polpudas e dedos curtos. Não largava a pasta de marroquim
em que trazia para a casa os papéis da repartição com o fito de não
lê"-los; e também o guarda"-chuva de castão de ouro e forro de seda.
Pesado e de pernas curtas, era com grande dificuldade que ele vencia os
dois degraus dos ``Minas Gerais'' da Light\footnote{Lima Barreto faz uma
  metáfora para se referir aos bondes elétricos da época, pois o ``Minas
  Gerais'' era o nome de um couraçado (navio de guerra). Os primeiros
  bondes elétricos foram instaurados no Rio de Janeiro pela empresa
  canadense ``Rio de Janeiro Tramway, Light and Power Company Limited'',
  ou simplesmente \emph{Rio Light}. Era comum na época a referência aos
  ``bondes da Light''. Para subir no bonde, as pessoas tinham que correr
  e fazer o movimento com a máquina em movimento, dai a dificuldade
  enfrentada por Campossolo, que além de estar sempre portando o
  guarda"-chuva e as pastas, tinha as pernas curtas.}, atrapalhado com
semelhantes cangalhas: a pasta e o guarda"-chuva de ``ouro''. Usava
chapéu de coco e cavanhaque.

Morava ali com sua mulher mais a filha solteira e única, a Mariazinha.

A mulher, Dona Sebastiana, que batizara a vila e com cujo dinheiro a
fizeram, era mais alta do que ele e não tinha nenhum relevo de
fisionomia, senão um artificial, um aposto\footnote{\emph{Aposto} é um
  termo utilizado em estudos gramaticais. Trata"-se de uma parte da frase
  ou oração cujo objetivo é dar ênfase, explicar melhor, anumerar ou
  dotar de qualidades algum segmento anterior. Por exemplo: ``Lima
  Barreto, que foi um dos maiores
  escritores brasileiros, não conseguiu ser eleito para a Academia
  Brasileira de Letras''. O trecho que aparece em destaque, entre
  vírgulas (\emph{que foi um dos maiores escritores brasileiros}),
  chama"-se o ``aposto'' da frase.}. Consistia num pequeno
\emph{pince"-nez} de aros de ouro, preso, por detrás da orelha, com
trancelim de seda. Não nascera com ele, mas era como se tivesse nascido,
pois jamais alguém havia visto Dona Sebastiana sem aquele adendo,
acavalado no nariz; fosse de dia, fosse de noite. Ela, quando queria
olhar alguém ou alguma cousa com jeito e perfeição, erguia bem a cabeça
e toda Dona Sebastiana tomava um entono de magistrado severo.

Era baiana, como o marido, e a única queixa que tinha do Rio cifrava"-se
em não haver aqui bons temperos para as moquecas, carurus e outras
comidas da Bahia, que ela sabia preparar com perfeição, auxiliada pela
preta Inácia, que, com eles, viera do Salvador, quando o marido foi
transferido para São Sebastião. Se se oferecia portador, mandava"-os
buscar; e quando aqui chegavam e ela preparava uma boa moqueca,
esquecia"-se de tudo, até que estava muito longe da sua querida cidade de
Tomé de Sousa.

Sua filha, a Mariazinha, não era assim e até se esquecera que por lá
nascera: cariocara"-se inteiramente. Era uma moça de vinte anos, fina de
talhe, poucas carnes, mais alta que o pai, entestando com a mãe, bonita
e vulgar. O seu traço de beleza eram os seus olhos de topázio com
estilhas negras. Nela, não havia nem invento, nem novidade como --- as
outras.

Eram estes os habitantes da ``Vila Sebastiana'', além de um molecote que
nunca era o mesmo. De dois em dois meses, por isso ou por aquilo, era
substituído por outro, mais claro ou mais escuro, conforme a sorte
calhava.

Em certos domingos, o senhor Campossolo convidava alguns dos seus
subordinados a irem almoçar ou jantar com eles. Não era um qualquer. Ele
os escolhia com acerto e sabedoria. Tinha uma filha solteira e não podia
pôr dentro de casa um qualquer, mesmo que fosse empregado de fazenda.

Aos que mais constantemente convidava, eram os terceiros escriturários
Fortunato Guaicuru e Simplício Fontes, os seus braços direitos na
secção. Aquele era bacharel em Direito e espécie de seu secretário e
consultor em assuntos difíceis; e o último chefe do protocolo da sua
secção, cargo de extrema responsabilidade, para que não houvesse
extravio de processos e se acoimasse a sua subdiretoria de relaxada e
desidiosa. Eram eles dois os seus mais constantes comensais, nos seus
bons domingos de efusões familiares. Demais, ele tinha uma filha a casar
e era bom que\ldots{}

Os senhores devem ter verificado que os pais sempre procuram casar as
filhas na classe que pertencem: os negociantes com negociantes ou
caixeiros; os militares com outros militares; os médicos com outros
médicos e assim por diante. Não é de estranhar, portanto, que o chefe
Campossolo quisesse casar sua filha com um funcionário público que fosse
da sua repartição e até da sua própria secção.

Guaicuru era de Mato Grosso. Tinha um tipo acentuadamente índio. Malares
salientes, face curta, mento largo e duro, bigodes de cerdas de javali,
testa fugidia e as pernas um tanto arqueadas. Nomeado para a alfândega
de Corumbá, transferira"-se para a delegacia fiscal de Goiás. Aí, passou
três ou quatro anos, formando"-se, na respectiva faculdade de Direito,
porque não há cidade do Brasil, capital ou não, em que não haja uma.
Obtido o título, passou"-se para a Casa da Moeda e, desta repartição,
para o Tesouro. Nunca se esquecia de trazer o anel de rubi\footnote{Era
  costume as pessoas formadas em Direito usarem um anel de pedras,
  normalmente rubi, como uma marca de distinção.}, à mostra. Era um
rapaz forte, de ombros largos e direitos; ao contrário de Simplício que
era franzino, peito pouco saliente, pálido, com uns doces e grandes
olhos negros e de uma timidez de donzela.

Era carioca e obtivera o seu lugar direitinho, quase sem
pistolão\footnote{Sobre o ``pistolão'', ver nota 120.} e sem nenhuma
intromissão de políticos na sua nomeação.

Mais ilustrado, não direi; mas muito mais instruído que Guaicuru, a
audácia deste o superava, não no coração de Mariazinha, mas no interesse
que tinha a mãe desta no casamento da filha. Na mesa, todas as atenções
tinha Dona Sebastiana pelo hipotético bacharel:

--- Por que não advoga? perguntou Dona Sebastiana, rindo, com seu
quádruplo olhar altaneiro, da filha ao caboclo que, na sua frente e a
seu mando, se sentavam juntos.

--- Minha senhora, não tenho tempo\ldots{}

--- Como não tem tempo? O Felicianinho consentiria --- não é Felicianinho?

Campossolo fazia solenemente:

--- Como não, estou sempre disposto a auxiliar a progressividade dos
colegas.

Simplício, à esquerda de Dona Sebastiana, olhava distraído para a
fruteira e nada dizia. Guaicuru, que não queria dizer que a verdadeira
razão estava em não ser a tal faculdade ``reconhecida'', negaceava:

--- Os colegas podiam reclamar.

Dona Sebastiana acudia com vivacidade:

--- Qual o que. O senhor reclamava, Senhor Simplício?

Ao ouvir o seu nome, o pobre rapaz tirava os olhos da fruteira e
perguntava com espanto:

--- O que, Dona Sebastiana?

--- O senhor reclamaria se Felicianinho consentisse que o Guaicuru
saísse, para ir advogar?

--- Não.

E voltava a olhar a fruteira, encontrando"-se rapidamente com os olhos de
topázio de Mariazinha. Campossolo continuava a comer e Dona Sebastiana
insistia:

--- Eu, se fosse o senhor ia advogar.

--- Não posso. Não é só a repartição que me toma o tempo. Trabalho em um
livro de grandes proporções.

Todos se espantaram. Mariazinha olhou Guaicuru; Dona Sebastiana levantou
mais a cabeça com \emph{pince"-nez} e tudo; Simplício que, agora,
contemplava esse quadro célebre nas salas burguesas, representando uma
ave, dependurada pelas pernas e fazendo \emph{pendant}\footnote{Palavra
  francesa que significa ``fazer par'', normalmente utilizada em
  referência a um objeto artístico que que forma par com outro.} com a
ceia do Senhor --- Simplício, dizia, cravou resolutamente o olhar sobre o
colega, e Campossolo perguntou:

--- Sobre o que trata?

--- Direito administrativo brasileiro.

Campossolo observou:

--- Deve ser uma obra de peso.

--- Espero.

Simplício continuava espantado, quase estúpido a olhar Guaicuru.
Percebendo isto, o mato"-grossense apressou"-se:

--- Você vai ver o plano. Quer ouvi"-lo?

Todos, menos Mariazinha, responderam, quase a um tempo só:

--- Quero.

O bacharel de Goiás endireitou o busto curto na cadeira e começou:

--- Vou entroncar o nosso Direito administrativo no antigo Direito
administrativo português. Há muita gente que pensa que no antigo regímen
não havia um Direito administrativo. Havia. Vou estudar o mecanismo do
Estado nessa época, no que toca a Portugal. Vou ver as funções dos
ministros e dos seus subordinados, por intermédio de letra"-morta dos
alvarás, portarias, cartas régias e mostrarei então como a engrenagem do
Estado funcionava; depois, verei como esse curioso Direito público se
transformou, ao influxo de concepções liberais; e, como ele transportado
para aqui com Dom João \textsc{vi}\footnote{Sobre D.\,João \textsc{vi}, ver nota 37.}, se
adaptou ao nosso meio, modificando"-se aqui ainda, sob o influxo das
ideias da Revolução.

Simplício, ouvindo"-o falar assim dizia com os seus botões: ``Quem teria
ensinado isto a ele?''.

Guaicuru, porém, continuava:

--- Não será uma seca enumeração de datas e de transcrição de alvarás,
portarias, etc. Será uma cousa inédita. Será cousa viva.

Por aí, parou e Campossolo com toda a gravidade disse:

--- Vai ser uma obra de peso.

--- Já tenho editor!

--- Quem é? perguntou o Simplício.

--- É o Jacinto. Você sabe que vou lá todo o dia, procurar livros a
respeito.

--- Sei; é a livraria dos advogados, disse Simplício sem querer sorrir.

--- Quando pretende publicar a sua obra, doutor? perguntou Dona
Sebastiana.

--- Queria publicar antes do Natal. Porque as promoções\footnote{Promoção,
  nesse contexto, significa subir de cargo na hierarquia do
  funcionalismo estatal.} serão feitas antes do Natal, mas\ldots{}

--- Então há mesmo promoções antes do Natal, Felicianinho?

O marido respondeu:

--- Creio que sim. O gabinete já pediu as propostas e eu já dei as minhas
ao diretor.

--- Devias ter"-me dito, ralhou"-lhe a mulher.

--- Essas cousas não se dizem às nossas mulheres; são segredos de Estado,
sentenciou Campossolo.

O jantar foi acabando triste, com essa história de promoções para o
Natal.

Dona Sebastiana quis ainda animar a conversa, dirigindo"-se ao marido:

--- Não queria que me dissesses os nomes, mas pode acontecer que seja o
promovido o doutor Fortunato ou\ldots{} O ``Seu'' Simplício, e eu
estaria prevenida para a uma ``festinha''.

Foi pior. A tristeza tornou"-se mais densa e quase calados tomaram café.

Levantaram"-se todos com o semblante anuviado, exceto a boa Mariazinha,
que procurava dar corda à conversa. Na sala de visitas, Simplício ainda
pôde olhar mais duas vezes furtivamente os olhos topazinos de
Mariazinha, que tinha um sossegado sorriso a banhar"-lhe a face toda; e
se foi. O colega Fortunato ficou, mas tudo estava tão morno e triste
que, em breve, se foi também Guaicuru.

No bonde, Simplício pensava unicamente em duas cousas: no Natal próximo
e no ``Direito'' de Guaicuru. Quando pensava nesta, perguntava de si
para si: ``Quem lhe ensinou aquilo tudo? Guaicuru é absolutamente
ignorante''. Quando pensava naquilo, implorava: ``Ah! Se Nosso Senhor
Jesus Cristo quisesse\ldots{}''.

Vieram afinal as promoções. Simplício foi promovido porque era muito
mais antigo na classe que Guaicuru. O Ministro não atendera a pistolões
nem a títulos de Goiás.

Ninguém foi preterido; mas, Guaicuru que tinha em gestação a obra de um
outro, ficou furioso sem nada dizer.

Dona Sebastiana deu uma consoada à moda do Norte. Na hora da ceia,
Guaicuru, como de hábito, ia sentar"-se ao lado de Mariazinha, quando
Dona Sebastiana, com \emph{pince"-nez} e cabeça, tudo muito bem erguido,
chamou"-o:

--- Sente"-se aqui a meu lado, doutor, aí vai sentar"-se o ``Seu''
Simplício.

Casaram"-se dentro de um ano; e, até hoje, depois de um lustro de casados
ainda teimam.

Ele diz:

--- Foi Nosso Senhor Jesus Cristo que nos casou.

Ela obtempera:

--- Foi a promoção.

Fosse uma cousa ou outra, ou ambas, o certo é que se casaram. É um fato.
A obra de Guaicuru, porém, é que até hoje não saiu\ldots{}



\chapter[Quase ela deu o «sim»]{Quase ela deu o «sim»\footnote[*]{Publicado em anexo no volume \emph{Vida e Morte de M. J. Gonzaga de Sá}, em 1949.}}

João Cazu era um moço suburbano, forte e saudável, mas pouco ativo e
amigo do trabalho.

Vivia em casa dos tios, numa estação de subúrbios, onde tinha moradia,
comida, roupa, calçado e algum dinheiro que a sua bondosa tia e madrinha
lhe dava para os cigarros.

Ele, porém, não os comprava; ``filava"-os'' dos outros. ``Refundia'' os
níqueis que lhe dava a tia, para flores a dar às namoradas e comprar
bilhetes de tômbolas\footnote{\emph{Tômbola} é um tipo de jogo de
  tabuleiro numerado, parecido com o bingo.}, nos vários
``mafuás''\footnote{Sobre os ``mafuás'', ver a crônica \emph{Feiras e
  ``mafuás''}, na seção \textsc{cotidiano e vida nos subúrbios}.}, mais ou menos
eclesiásticos, que há por aquelas redondezas.

O conhecimento do seu hábito de ``filar'' cigarros aos camaradas e
amigos, estava tão espalhado que, mal um deles o via, logo tirava da
algibeira um cigarro; e, antes de saudá"-lo, dizia:

---Toma lá o cigarro, Cazu.

Vivia assim muito bem, sem ambições nem tenções. A maior parte do dia,
especialmente a tarde, empregava ele, com outros companheiros, em dar
loucos pontapés, numa bola, tendo por arena um terreno baldio das
vizinhanças da residência dele ou melhor: dos seus tios e padrinhos.

Contudo, ainda não estava satisfeito. Restava"-lhe a grave preocupação de
encontrar quem lhe lavasse e engomasse a roupa, remendasse as calças e
outras peças do vestuário, cerzisse as meias, etc., etc.

Em resumo: ele queria uma mulher, uma esposa, adaptável ao seu jeito
descansado.

Tinha visto falar em sujeitos que se casam com moças ricas e não
precisam trabalhar; em outros que esposam professoras e adquirem a
meritória profissão de ``maridos da professora''; ele, porém, não
aspirava a tanto.

Apesar disso, não desanimou de descobrir uma mulher que lhe servisse
convenientemente.

Continuou a jogar displicentemente, o seu \emph{football} vagabundo e a
viver cheio de segurança e abundância com os seus tios e padrinhos.

Certo dia, passando pela porteira da casa de uma sua vizinha mais ou
menos conhecida, ela lhe pediu:

--- ``Seu'' Cazu, o senhor vai até à estação?

--- Vou, Dona Ermelinda.

--- Podia me fazer um favor?

--- Pois não.

--- É ver se o ``Seu'' Gustavo da padaria ``Rosa de Ouro'', me pode
ceder duas estampilhas de seiscentos réis. Tenho que fazer um
requerimento ao Tesouro, sobre coisas do meu montepio, com urgência,
precisava muito.

--- Não há dúvida, minha senhora.

Cazu, dizendo isto, pensava de si para si: ``É um bom partido. Tem
montepio, é viúva; o diabo são os filhos!''. Dona Ermelinda, à vista da
resposta dele, disse:

--- Está aqui o dinheiro.

Conquanto dissesse várias vezes que não precisava daquilo --- o dinheiro
--- o impenitente jogador de \emph{football} e feliz hóspede dos tios,
foi embolsando os nicolaus\footnote{\emph{Nicolau} era uma gíria usada
  na época para se referir a dinheiro.}, por causa das dúvidas.

Fez o que tinha a fazer na estação, adquiriu as estampilhas e voltou
para entregá"-las à viúva.

De fato, Dona Ermelinda era viúva de um contínuo ou cousa parecida de
uma repartição pública. Viúva e com pouco mais de trinta anos, nada se
falava da sua reputação.

Tinha uma filha e um filho que educava com grande desvelo e muito
sacrifício.

Era proprietária do pequeno \emph{chalet}\footnote{Palavra francesa
  aportuguesada para chalé, que significava na época uma casa pequena
  com um grande quintal.} onde morava, em cujo quintal havia laranjeiras
e algumas outras árvores frutíferas.

Fora o seu falecido marido que o adquirira com o produto de uma
``sorte'' na loteria; e, se ela, com a morte do esposo, o salvara das
garras de escrivães, escreventes, meirinhos, solicitadores e advogados
``mambembes'', devia"-o à precaução do marido que comprara a casa, em
nome dela.

Assim mesmo, tinha sido preciso a intervenção do seu compadre, o Capitão
Hermenegildo, a fim de remover os obstáculos que certos ``águias''
começavam a pôr, para impedir que ela entrasse em plena posse do imóvel
e abocanhar"-lhe afinal o seu chalézito humilde.

De volta, Cazu bateu à porta da viúva que trabalhava no interior, com
cujo rendimento ela conseguia aumentar de muito o módico, senão
irrisório montepio, de modo a conseguir fazer face às despesas mensais
com ela e os filhos.

Percebendo a pobre viúva que era o Cazu, sem se levantar da máquina,
gritou:

--- Entre, ``Seu'' Cazu.

Estava só, os filhos ainda não tinham vindo do colégio. Cazu entrou.

Após entregar as estampilhas, quis o rapaz retirar"-se; mas foi obstado
por Ermelinda nestes termos:

--- Espere um pouco, ``Seu'' Cazu. Vamos tomar café.

Ele aceitou e ambos se serviram da infusão da ``preciosa rubiácea'',
como se diz no estilo ``valorização''.

A viúva, tomando café, acompanhado com pão e manteiga, pôs"-se a olhar o
companheiro com certo interesse. Ele notou e fez"-se amável e galante,
demorando em esvaziar a xícara. A viuvinha sorria interiormente de
contentamento. Cazu pensou com os seus botões: ``Está aí um bom partido:
casa própria, montepio, renda das costuras; e além de tudo, há de
lavar"-me e consertar a roupa. Se calhou, fico livre das censuras da
tia\ldots{}''.

Essa vaga tenção ganhou mais corpo, quando a viúva, olhando"-lhe a
camisa, perguntou:

--- ``Seu'' Cazu, se eu lhe disser uma cousa, o senhor fica zangado?

--- Ora, qual, Dona Ermelinda?

--- Bem. A sua camisa está rasgada no peito. O senhor traz ``ela''
amanhã, que eu conserto ``ela''.

Cazu respondeu que era preciso lavá"-la primeiro; mas a viúva
prontificou"-se em fazer isso também. O \emph{player} dos pontapés,
fingindo relutância no começo, aceitou afinal; e doido por isso estava
ele, pois era uma ``entrada'', para obter uma lavadeira em condições
favoráveis.

Dito e feito: daí em diante, com jeito e manha, ele conseguiu que a
viúva se fizesse a sua lavadeira bem em conta.

Cazu, após tal conquista, redobrou de atividade no \emph{football},
abandonou os biscates e não dava um passo, para obter emprego. Que é que
ele queria mais? Tinha tudo\ldots{}

Na redondeza, passavam como noivos; mas não eram, nem mesmo namorados
declarados.

Havia entre ambos, unicamente um ``namoro de caboclo'', com o que Cazu
ganhou uma lavadeira, sem nenhuma exigência monetária e cultivava"-o
carinhosamente.

Um belo dia, após ano e pouco de tal namoro, houve um casamento na casa
dos tios do diligente jogador de \emph{football}. Ele, à vista da
cerimônia e da festa, pensou: ``Porque também eu não me caso? Porque eu
não peço Ermelinda em casamento? Ela aceita, por certo; e eu\ldots{}''.

Matutou domingo, pois o casamento tinha sido no sábado; refletiu segunda
e, na terça, cheio de coragem, chegou"-se à Ermelinda e pediu"-a em
casamento.

--- É grave isto, Cazu. Olhe que sou viúva e com dois filhos!

--- Tratava ``eles'' bem; eu juro!

--- Está bem. Sexta"-feira, você vem cedo, para almoçar comigo e eu dou a
resposta.

Assim foi feito. Cazu chegou cedo e os dois estiveram a conversar. Ela,
com toda a naturalidade, e ele, cheio de ansiedade e, apreensivo.

Num dado momento, Ermelinda foi até à gaveta de um móvel e tirou de lá
um papel.

--- Cazu --- disse ela, tendo o papel na mão --- você vai à venda e à
quitanda e compra o que está aqui nesta ``nota''. É para o almoço.

Cazu agarrou trêmulo o papelucho e pôs"-se a ler o seguinte:

\medskip

1 quilo de feijão \dotfill 600\,rs.

1/2 de farinha \dotfill 200\,rs.

1/2 de bacalhau \dotfill 1.200\,rs.

1/2 de batatas \dotfill 360\,rs.

Cebolas \dotfill 200\,rs.

Alhos \dotfill 100\,rs.

Azeite \dotfill 300\,rs.

Sal \dotfill 100\,rs.

Vinagre \dotfill 200\,rs.

\dotfill 3.260\,rs.

\smallskip

Quitanda:

Carvão \dotfill 280\,rs.

Couve \dotfill 200\,rs.

Salsa\dotfill 100\,rs.

Cebolinha \dotfill 100\,rs.

tudo: \dotfill 3.860\,rs.\medskip

Acabada a leitura, Cazu não se levantou logo da cadeira; e, com a lista
na mão, a olhar de um lado a outro, parecia atordoado, estuporado.

--- Anda Cazu, fez a viúva. Assim, demorando, o almoço fica
tarde\ldots{}

--- É que\ldots{}

--- Que há?

--- Não tenho dinheiro.

--- Mas você não quer casar comigo? É mostrar atividade meu filho! Dê os
seus passos\ldots{} Vá! Um chefe de família não se atrapalha\ldots{} É
agir!

João Cazu, tendo a lista de gêneros na mão, ergueu"-se da cadeira, saiu e
não mais voltou\ldots{}



\chapter[O Jornalista]{O Jornalista\footnote[*]{Publicado na edição de julho de 1921 da \emph{Revista Souza Cruz}.}}

\hfill\emph{A Ranulfo Prata}\bigskip

\noindent{}A cidade de Sant'Ana dos Pescadores fora em tempos idos uma cidadezinha
próspera. Situada entre o mar e a montanha que escondia vastas vargens
férteis, e muito próximo do Rio, os fazendeiros das planuras
transmontanas preferiam enviar os produtos de suas lavouras através de
uma garganta, transformada em estrada, para, por mar, trazê"-los ao
grande empório da Corte. O contrário faziam com as compras que aí
faziam.

Dessa forma, erguida à condição de uma espécie de entreposto de uma zona
até bem pouco fértil e rica, ela cresceu e tomou ares galhardos de
cidade de importância. As suas festas de igreja eram grandiosas e
atraíam fazendeiros e suas famílias, alguns tendo mesmo casas de recreio
apalaçadas nela. O seu comércio era por isso rico com o dinheiro que os
tropeiros lhe deixavam.

Veio, porém, a estrada de ferro e a sua decadência foi rápida. O
transporte das mercadorias de ``serra"-acima'' se desviou dela e os seus
sobrados deram em descascar como velhas árvores que vão morrer. Os
mercadores ricos a abandonaram e os galpões de tropa desabaram.
Entretanto, o sítio era aprazível, com as suas curtas praias alvas que
foram separadas por desabamentos de grandes moles de granito da montanha
verdejante do fundo do vilarejo, formando aglomerações de grossos
pedregulhos.

A gente pobre, após a sua morte, deu em viver de pescarias, pois o mar
ai era rumoroso e abundante de pescado de bom quilate.

Tripulando grandes canoas de voga, os seus pescadores traziam o produto
de sua humilde indústria, vencendo mil dificuldades, até
Sepetiba\footnote{Bairro litorâneo
  localizado na zona oeste da
  cidade do Rio de Janeiro.} e, daí, à Santa Cruz\footnote{Também
  localizado na zona oeste da cidade do Rio de Janeiro.}, onde ele era
embarcado em trem de ferro até ao Rio de Janeiro.

Os ricos de lá, além dos fabricantes de cal de marisco, eram os
taverneiros que, nessas vendas, como se sabe, vendem tudo, mesmo
casimiras e arreios, e são os banqueiros. Lavradores não havia e até
frutas iam do Rio de Janeiro.

As pessoas importantes eram o juiz de direito, o promotor, o escrivão,
os professores públicos, o presidente da Câmara e o respectivo
secretário. Este, porém, o Salomão Nabor de Azevedo, descendente dos
antigos Nabores de Azevedo de ``serra"-acima'' e dos Breves, ricos
fazendeiros, era o mais. Era o mais porque, além disto, se fizera o
jornalista popular do lugar.

A ideia de fundar \emph{O Arauto}, --- órgão dos interesses da cidade de
Sant'Ana dos Pescadores --- não fora dele e sim do promotor. Este veio a
perder o jornal, de um modo curioso. O doutor Fagundes, o tal de
promotor, começou a fazer oposição ao doutor Castro, advogado no lugar
e, no tempo, presidente da Câmara. Nabor não via com bons olhos aquele
e, certo dia, foi ao jornal e retirou o artigo do promotor e escreveu um
descabelado de elogios ao doutor Castro, porque ele tinha suas luzes,
como veremos. Resultado: Nabor, o nobre Nabor, foi nomeado secretário da
Câmara e o promotor perdeu a importância de melhor jornalista local, que
coube, daí por diante e para sempre, a Nabor.

Como já disse, este Nabor recebera luzes num colégio de padres de
Vassouras ou Valença\footnote{Cidades localizadas no interior --- região
  centro"-sul --- do estado do Rio de Janeiro.}, quando os pais eram
ricos. O seu saber não era lá grande; não passava de gramaticazinha
portuguesa, das quatro operações e umas citações históricas que
aprendera com Fagundes Varela\footnote{O poeta Luís Nicolau Fagundes
  Varela (1841--1875) ou simplesmente Fagundes Varela nasceu e viveu
  boa parte da vida no Rio de Janeiro. Foi um dos principais poetas do
  nosso período romântico, embora tenha morrido prematuramente, aos 33
  anos de idade. Teve fama de boêmio e andarilho e de principalmente ter
  passado os últimos anos de vida perambulando de fazenda em fazenda, no
  interior do estado do Rio.}, quando este foi hóspede de seus pais, em
cuja fazenda chegara, certa vez, de tarde, numa formidável carraspana e
em trajes de tropeiro, calçado de tamancos.

O poeta gostara dele e lhe dera algumas noções de letras. Lera o
Macedo\footnote{Referência a Joaquim Manuel de Macedo (1820--1882),
  escritor, jornalista, dramaturgo, deputado. Foi dos maiores escritores
  brasileiros daquele período, ao lado de José de Alencar.} e os poetas
do tempo, daí o seu pendor para cousas de letras e de jornalismo.

Herdou alguma cousa do pai, vendera a fazenda e viera morar em Sant'Ana,
onde tinha uma casa, também pela mesma herança. Casou aí com uma moça de
alguma pecúnia e vivia a fazer política e a ler os jornais da
Corte\footnote{Antes da Proclamação da República, a capital do Império,
  na cidade do Rio de Janeiro, também era conhecida pela denominação de
  Corte.}, que assinava. Deixou os romances e apaixonou"-se por José do
Patrocínio\footnote{José do Patrocínio (1853--1905) foi escritor,
  jornalista e um dos principais líderes do Movimento Abolicionista
  Brasileiro. Era filho de Maria Justina, uma mulher que viera
  escravizada do continente africano. Seu pai era cônego. Patrocínio
  começou suas atividades contra a escravidão e a monarquia ainda no
  início da década de 1870, escrevendo poemas para o jornal \emph{A
  República.} Já na década de 1880 propôs a criação da Confederação
  Abolicionista, que levava a campanha para as principais localidades do
  país.}, Ferreira de Meneses\footnote{José Ferreira de Meneses (1842--1881), não possui um registro seguro para seu nascimento. Segundo a
  pesquisadora Ana Flávia Magalhães, em sua Tese de Doutorado ``Fortes
  laços em linhas rotas'' (\textsc{unicamp}, 2014) o ano de nascimento de
  Ferreira de Meneses teria sido 1842. Assim como José do Patrocínio,
  era negro (\emph{Filho do preto Joaquim Ferreira}) e foi um dos
  grandes abolicionistas. Formou"-se em Direito, em São Paulo, na
  Academia do Largo de São Francisco. Foi grande amigo, desde a
  infância, de Fagundes Varela. Começou sua vida intelectual traduzindo
  e escrevendo algumas peças de teatro, na década de 1860, atividades
  que lhe trouxeram grande reconhecimento. Também teve uma atividade
  jornalística bastante intensa e produtiva.}, Joaquim Serra\footnote{Joaquim
  Maria Serra Sobrinho (1838--1888) nasceu em São Luís, no estado do
  Maranhão. Fez carreira como político, jornalista, homem de teatro,
  poeta. Viveu também no Rio de Janeiro, onde ocupou cadeira de Deputado
  por seu estado. Foi um dos mais contundentes abolicionistas e
  antiescravistas da propaganda pela Abolição.} e outros jornalistas dos
tempos calorosos da abolição. Era abolicionista, porque\ldots{} os seus
escravos ele os tinha vendido com a fazenda que herdara; e os poucos que
tinha em casa, dizia que não os libertava, por serem da mulher\footnote{É
  de se notar a nota crítica de Lima Barreto ao oportunismo do
  personagem Nabor, que se lançou no jornalismo abolicionista para se
  aproveitar da onda.}.

O seu abolicionismo, com a Lei de 13 de Maio\footnote{Sobre o 13 de
  Maio, ver nota 4.}, veio dar, naturalmente, algum prejuízo à
esposa\ldots{} Enfim, após a República\footnote{Referência aos anos de
  1888 (Abolição) e 1889 (Proclamação da República).} e a Abolição, foi
várias vezes subdelegado e vereador de Sant'Ana. Era isto, quando o
promotor Fagundes lembrou"-lhe a ideia de fundar um jornal na cidade.
Conhecia aquele a mania do último, por jornais, e a resposta confirmou a
sua esperança:

--- Boa ideia, ``Seu'' Fagundes! A ``estrela do Abraão'' (assim era
chamada Sant'Ana) não ter um jornal! Uma cidade como esta, pátria de
tantas glórias, de tão honrosas tradições, sem essa alavanca do
progresso que é a imprensa, esse fanal que guia a humanidade --- não é
possível!

--- O diabo, o diabo\ldots{} fez Fagundes.

--- Por que o diabo, Fagundes?

--- E o capital?

--- Entro com ele.

O trato foi feito e Nabor, descendente dos Nabores de Azevedo e dos
famigerados Breves, entrou com o cobre; e Fagundes ficou com a direção
intelectual do jornal. Fagundes era mais burro e, talvez, mais ignorante
do que Nabor; mas este deixava"-lhe a direção ostensiva porque era
bacharel. \emph{O Arauto} era semanal e saía sempre com um artiguete
laudatório do diretor, à guisa de artigo de fundo, umas composições
líricas, em prosa, de Nabor, aniversários, uns mofinos anúncios e os
editais da Câmara Municipal. Às vezes, publicava certas composições
poéticas do professor público. Eram sonetos bem quebrados e bem
estúpidos, mas que eram anunciados como ``trabalhos de um puro
parnasiano que é esse Sebastião Barbosa, exímio educador e glória da
nossa terra e da nossa raça''.

Às vezes, Nabor, o tal dos Nabores de Azevedo e dos Breves, honrados
fabricantes de escravos, cortava alguma coisa de valia dos jornais do
Rio e o jornaleco ficava literalmente esmagado ou inundado.

Dentro do jornal, reinava uma grande rivalidade latente entre o promotor
e Nabor. Cada qual se julgava mais inteligente por decalcar ou pastichar
melhor um autor em voga.

A mania de Nabor, na sua qualidade de profissional e jornalista moderno,
era fazer de \emph{O Arauto} um jornal de escândalo; de altas
reportagens sensacionais, de enquetes com notáveis personagens da
localidade, enfim, um jornal moderno; a de Fagundes era a de fazê"-lo um
cotidiano doutrinário, sem demasias, sem escândalos --- um \emph{Jornal
do Comércio} de Sant'Ana dos Pescadores, a ``Princesa'' de ``O Seio de
Abraão'', a mais formosa enseada do Estado do Rio.

Certa vez, aquele ocupou três colunas do grande órgão (e achou pouco),
com a narração do naufrágio da canoa de pescaria --- ``Nossa Senhora do
Ó'', na praia da Mabombeba. Não morrera um só tripulante.

Fagundes censurou"-lhe:

--- Você está gastando papel à"-toa!

Nabor retrucou"-lhe:

--- É assim que se procede no Rio com os naufrágios sensacionais.
Demais: quantas colunas você gastou com o artigo sobre o direito de
cavar ``tariobas''\footnote{Um tipo de molusco apreciado em localidades
  litorâneas do Nordeste e do Sudeste.} nas praias.

--- É uma questão de marinhas e acrescidos; é uma questão de direito.

Assim, viviam aparentemente em paz, mas, no fundo, em guerra surda.

Com o correr dos tempos, a rivalidade chegou ao auge e Nabor fez o que
fez com Fagundes. Reclamou este e o descendente dos Breves
respondeu"-lhe:

--- Os tipos são meus; a máquina é minha; portanto, o jornal é meu.

Fagundes consultou os seus manuais e concluiu que não tinha direito à
sociedade do jornal, pois não havia instrumento de direito bastante
hábil para prová"-la em juízo; mas, de acordo com a lei e vários
jurisconsultos notáveis, podia reclamar o seu direito aos honorários de
redator"-chefe, à razão de 1:800\$000. Ele o havia sido quinze anos e
quatro meses; tinha, portanto, direito a receber 324 contos, juros de
mora e custas.\footnote{Sobre a unidade monetária da época, o
  \emph{conto de réis}, ver nota 26.}

Quis propor a causa, mas viu que a taxa judicial ia muito além das suas
posses. Abandonou o propósito; e Nabor, o tal dos Azevedo e dos Breves,
um dos quais recebera a visita do Imperador, numa das suas fazendas, na
da Grama, ficou único dono do jornal.

Dono do grande órgão, tratou de modificar"-lhe o feitio carranca que lhe
imprimira o pastrana do Fagundes. Fez inquéritos com o sacristão da
irmandade; atacou os abusos das autoridades da Capitania do Porto;
propôs, a exemplo de Paris, etc., o estabelecimento do exame das
amas"-de"-leite, etc., etc. Mas, nada disso deu retumbância a seu jornal.

Certo dia, lendo a notícia de um grande incêndio no Rio, acudiu"-lhe a
ideia de que se houvesse um em Sant'Ana, podia publicar uma notícia de
``escacha''\footnote{Notícia de \emph{escacha}, do verbo
  \emph{escachar}, significa abrir caminho à força.}, no seu jornal, e
esmagar o rival --- \emph{O Baluarte} --- que era dirigido pelo promotor
Fagundes, o antigo companheiro e inimigo.

Como havia de ser? Ali, não havia incêndios, nem mesmo casuais. Esta
palavra abriu"-lhe um clarão na cabeça e completou"-lhe a ideia. Resolveu
pagar a alguém que atacasse fogo no palacete do doutor Gaspar, seu
protetor, o melhor prédio da cidade. Mas, quem seria, se tentasse pagar
a alguém? Mas\ldots{} esse alguém se fosse descoberto denunciá"-lo"-ia,
por certo. Não valia a pena\ldots{} Uma ideia! Ele mesmo poria fogo no
sábado, na véspera de sair o seu hebdomadário --- \emph{O Arauto}. Antes
escreveria a longa notícia com todos os ``ff'' e ``rr''. Dito e feito. O
palácio pegou fogo inteirinho no sábado, alta noite; e de manhã, a
notícia saía bem feitinha. Fagundes, que já era Juiz Municipal, logo viu
a criminalidade de Nabor. Arranjou"-lhe uma denúncia"-processo e o grande
jornalista Salomão Nabor de Azevedo, descendente dos Azevedos, do Rio
Claro, e dos Breves, reis da escravatura, foi parar na cadeia, pela sua
estupidez e vaidade.



\chapter[O Número da Sepultura]{O Número da Sepultura\footnote[*]{Publicado na edição de março de 1921 da \emph{Revista Souza Cruz}.}}

Que podia ela dizer, após três meses de casada, sobre o casamento?

Era bom? Era mau?

Não se animava a afirmar nem uma cousa, nem outra. Em essência,
``aquilo'' lhe parecia resumir"-se em uma simples mudança de casa.

A que deixara não tinha mais nem menos cômodos do que a que viera
habitar; não tinha mais ``largueza''; mas a ``nova'' possuía um
jardinzito minúsculo e uma pia na sala de jantar.

Era, no fim de contas, a diminuta diferença que existia entre ambas.

Passando da obediência dos pais, para a do marido, o que ela sentia, era
o que se sente quando se muda de habitação.

No começo, há nos que se mudam, agitação, atividade; puxa"-se pela ideia,
a fim de adaptar os móveis à casa ``nova'' e, por conseguinte, eles, os
seus recentes habitantes também; isso, porém, dura poucos dias.

No fim de um mês, os móveis já estão definitivamente ``ancorados'', nos
seus lugares, e os moradores se esquecem de que residem ali desde poucos
dias.

Demais, para que ela não sentisse, profunda modificação, no seu viver,
advinda com o casamento, havia a quase igualdade de gênios e hábitos de
seu pai e seu marido.

Tanto um como outro, eram corteses com ela; brandos no tratar, serenos,
sem impropérios, e ambos, também, meticulosos, exatos e metódicos. Não
houve, assim, abalo algum, na sua transplantação de um lar para outro.

Contudo, esperava, no casamento alguma cousa de inédito até ali, na sua
existência de mulher: uma exuberante e contínua satisfação de viver.

Não sentiu, porém, nada disso.

O que houve de particular na sua mudança de estado, foi insuficiente
para lhe dar uma sensação nunca sentida da vida e do mundo. Não percebeu
nenhuma novidade essencial\ldots{}

Os céus cambiantes, com o rosado e dourado de arrebóis, que o casamento
promete a todos, moços e moças; ela não os vira. O sentimento de inteira
liberdade, com passeios, festas, teatros, visitas --- tudo que se contém
para as mulheres, na ideia de casamento, durou somente a primeira semana
de matrimônio.

Durante ela, ao lado do marido, passeara, visitara, fora a festas, e a
teatros; mas assistira todas essas cousas, sem muito se interessar por
elas, sem receber grandes ou profundas emoções de surpresa, e ter sonhos
fora do trivial da nossa mesquinha vida terrestre. Cansavam"-na até!

No começo, sentia alguma alegria e certo contentamento; por fim, porém,
veio o tédio por elas todas, a nostalgia da quietude de sua casa
suburbana, onde vivia à \emph{négligé}\footnote{Expressão francesa que
  significa à vontade, sem preocupações, tranquilamente.} e podia
sonhar, sem desconfiar que os outros lhe pudessem descobrir os devaneios
crepusculares de sua pequenina alma de burguesa, saudosa e enfumaçada.

Não era raro que também ocorresse saudades da casa paterna, provocadas
por aquelas chinfrinadas de teatros ou cinematográficas. Acudia"-lhe, com
indefinível sentimento, a lembrança de velhos móveis e outros pertences
familiares da sua casa paterna, que a tinham visto desde menina. Era uma
velha cadeira de balanço de jacarandá; era uma leiteira de louça,
pintada de azul, muito antiga; era o relógio sem pêndula, octogonal,
velho também; e outras bugigangas domésticas que, muito mais fortemente
do que os móveis e utensílios adquiridos recentemente, se haviam gravado
na sua memória.

Seu marido era um rapaz de excelentes qualidades matrimoniais, e não
havia, no nebuloso estado d'alma de Zilda, nenhum desgosto dele ou
decepção que ele lhe tivesse causado.

Morigerado, cumpridor exato dos seus deveres, na secção de que era chefe
seu pai, tinha todas as qualidades médias, para ser um bom chefe de
família, cumprir o dever de continuar a espécie e ser um bom diretor de
secretaria ou repartição outra, de banco ou de escritório comercial.

Em compensação, não possuía nenhuma proeminência de inteligência ou de
ação. Era e seria sempre uma boa peça de máquina, bem ajustada, bem
polida e que, lubrificada convenientemente, não diminuiria o rendimento
daquela, mas que precisava sempre do motor da iniciativa estranha, para
se pôr em movimento.

Os pais de Zilda tinham aproximado os dois; a avó, a quem a moça
estimava deveras, fizera as insinuações de praxe; e, vendo ela que a
coisa era do gosto de todos, por curiosidade mais do que por amor ou
outra cousa parecida, resolveu"-se a casar com o escriturário de seu pai.
Casaram"-se, viviam muito bem. Entre ambos, não havia a menor rusga, a
menor desinteligência que lhes toldasse a vida matrimonial; mas não
existia também, como era de esperar, uma profunda e constante
penetração, de um para o outro e vice"-versa, de desejos, de sentimentos,
de dores e alegrias.

Viviam placidamente numa tranquilidade de lagoa, cercada de altas
montanhas, por entre as quais os ventos fortes não conseguiam penetrar,
para encrespar"-lhe as águas imotas.

A beleza do viver daquele novel casal, não era ter conseguido de duas
fazer uma única vontade; estava em que os dois continuassem a ser cada
um uma personalidade, sem que, entanto, encontrassem nunca motivo de
conflito, o mais ligeiro que fosse. Uma vez, porém\ldots{} Deixemos isso para
mais tarde\ldots{} O gênio e a educação de ambos muito contribuíam para
tal.

O marido, exato burocrata, era cordato, de temperamento calmo, ponderado
e seco que nem uma crise ministerial. A mulher era quase passiva e tendo
sido educada na disciplina ultra regrada e esmerilhadora de seu pai,
velho funcionário, obediente aos chefes, aos ministros, aos secretários
destes e mais bajuladores, às leis e regulamentos, não tinha assomos nem
caprichos, nem fortes vontades. Refugiava"-se no sonho e, desde que não
fosse multado, estava por tudo.

Os hábitos do marido eram os mais regulares e executados, sem a mínima
discrepância. Erguia"-se do leito muito cedo, quase ao alvorecer, antes
mesmo da criada, a Genoveva, levantar"-se da cama. Pondo"-se de pé, ele
mesmo coava o café e, logo que estava pronto, tomava uma grande xícara.

Esperando o jornal (só comprava um), ia para o pequeno jardim, varria"-o,
amarrava as roseiras e craveiros, nos espeques, em seguida, dava milho
às galinhas e pintos e tratava dos passarinhos.

Chegando o jornal, lia"-o meticulosamente, organizando, para uso do dia,
as suas opiniões literárias, científicas, artísticas, sociais e, também,
sobre a política internacional e as guerras que havia pelo mundo.

Quanto à política interna, construía algumas, mas não as manifestava a
ninguém, porque quase sempre eram contra o governo e ele precisava ser
promovido.

Às nove e meia, já almoçado e vestido, despedia"-se da mulher, com o
clássico beijo, e lá ia tomar o trem. Assinava o ponto, de acordo como
regulamento, isto é, nunca depois das dez e meia.

Na repartição, cumpria religiosamente os seus sacratíssimos deveres de
funcionário.

Sempre foi assim; mas, após o casamento, aumentou de zelo, a fim de pôr
a secção do sogro que nem um brinco, em questão de rapidez e presteza no
andamento e informações de papéis.

Andava pelas bancas dos colegas, pelos protocolos, quando o serviço lhe
faltava e se, nessa correição, topava com expediente em atraso, não
hesitava: punha"-se a ``desunhar''.

Acontecendo"-lhe isto, ao sentar"-se à mesa, para jantar, já em trajes
caseiros, apressava"-se em dizer a mulher:

--- Arre! Trabalhei hoje, Zilda, que nem o diabo!

--- Por quê?

--- Ora, por quê? Aqueles meus colegas são uma pinoia\ldots{}

--- Que houve?

--- Pois o Pantaleão não está com o protocolo dele, o da Marinha,
atrasado de uma semana? Tive que o pôr em dia\ldots{}

--- Papai foi quem te mandou?

--- Não; mas era meu dever, como genro dele, evitar que a secção que ele
dirige, fosse tachada de relaxada. Demais não posso ver expediente
atrasado\ldots{}

--- Então, esse Pantaleão falta muito?

--- Um horror! Desculpa"-se com estar estudando direito. Eu também
estudei, quase sem faltas.

Com semelhantes notícias e outras de mexericos sobre a vida íntima,
defeitos morais e vícios dos colegas, que ele relatava à mulher, Zilda
ficou enfronhada no viver da diretoria em que funcionava seu marido,
tanto no aspecto puramente burocrático, como nos da vida particular e
famílias dos respectivos empregados.

Ela sabia que o Calçoene bebia cachaça; que o Zé Fagundes vivia
amancebado com uma crioula, tendo filhos com ela, um dos quais com
concurso e ia ser em breve colega do marido; que o Feliciano Brites das
Novas jogava nos dados todo o dinheiro que conseguia arranjar, que a
mulher do Nepomuceno era amante do General T., com auxílio do qual ele
preteria todos nas promoções, etc., etc.

O marido não conversava com Zilda senão essas coisas da repartição; não
tinha outro assunto para palestrar com a mulher. Com as visitas e raros
colegas com quem discutia, a matéria da conversação eram coisas
patrióticas: as forças de terra e mar, as nossas riquezas naturais, etc.

Para tais argumentos tinha predileção especial e um especial orgulho em
desenvolvê"-los com entusiasmo. Tudo o que era brasileiro era primeiro do
mundo ou, no mínimo, da América do Sul. E --- ai! --- de quem o
contestasse; levava uma sarabanda que resumia nesta frase clássica:

--- É por isso que o Brasil não vai para adiante. O brasileiro é o maior
inimigo de sua pátria.

Zilda, pequena burguesa, de reduzida instrução e, como todas as
mulheres, de fraca curiosidade intelectual quando o ouvia discutir assim
com os amigos, enchia"-se de enfado e sono; entretanto, gostava das suas
alcovitices sobre os lares dos colegas\ldots{}

Assim ela ia repassando a sua vida de casada, que já tinha mais de três
meses feitos, na qual, para quebrar"-lhe a monotonia e a igualdade, só
houvera um acontecimento que a agitara, a torturara, mas, em
compensação, espantara por algumas horas o tédio daquele morno e plácido
viver. É preciso contá"-lo.

Augusto --- Augusto Serpa de Castro, tal era o nome de seu marido ---
tinha um ar mofino e enfezado; alguma cousa de índio nos cabelos muito
negros, corredios e brilhantes, e na tez acobreada. Seus olhos eram
negros e grandes, com muito pouca luz, mortiços e pobres de expressão,
sobretudo de alegria.

A mulher, mais moça do que ele uns cinco ou seis anos, ainda não havia
completado os vinte. Era de uma grande vivacidade de fisionomia, muito
móbil e vária, embora o seu olhar castanho claro tivesse, em geral, uma
forte expressão de melancolia e sonho interior. Miúda de feições,
franzina, de boa estatura e formas harmoniosas, tudo nela era a graça do
caniço, a sua esbelteza, que não teme os ventos, mas que se curva à
força deles com mais elegância ainda, para ciciar os queixumes contra o
triste fado de sua fragilidade, esquecendo"-se, porém, que é esta que o
faz vitorioso.

Após o casamento, vieram residir na Travessa das Saudades, na estação de
* * *

É uma pitoresca rua, afastada alguma cousa das linhas da Central, cheia
de altos e baixos, dotada de uma caprichosa desigualdade de nível, tanto
no sentido longitudinal como no transversal.

Povoada de árvores e bambus, de um lado e outro, correndo quase
exatamente de norte para sul, as habitações do lado do nascente, em
grande número, somem"-se na grota que ela forma, com o seu
desnivelamento; e mais se ocultam debaixo dos arvoredos em que os Cipós
se tecem.

Do lado do poente, porém, as casas se alteiam e, por cima das de
defronte, olham em primeira mão a Aurora, com os seus inexprimíveis
cambiantes de cores e matizes.

Como no fim do mês anterior, naquele outro, o segundo término de mês
depois do seu casamento, o bacharel Augusto, logo que recebeu os
vencimentos e conferiu as contas dos fornecedores, entregou o dinheiro
necessário à mulher, para pagá"-los, e também a importância do aluguel da
casa.

Zilda apressou"-se em fazê"-lo ao carniceiro, ao padeiro e ao vendeiro;
mas, o procurador do proprietário da casa em que moravam, demorou"-se um
pouco. Disso, avisou o marido, em certa manhã, quando ele lhe dava uma
pequena quantia para as despesas com o quitandeiro e outras miudezas
caseiras. Ele deixou o importe do aluguel com ela.

Havia já quatro dias que ele se havia vencido; entretanto, o preposto do
proprietário não aparecia.

Na manhã desse quarto dia, ela amanheceu alegre e, ao mesmo tempo
apreensiva.

Tinha sonhado; e que sonho!

Sonhou com a avó, a quem amava profundamente e que desejara muito o seu
casamento com Augusto. Morrera ela poucos meses antes de realizar"-se o
seu enlace com ele; mas ambos já eram noivos.

Sonhara a moça com o número da sepultura da avó --- 1724; e ouvira a voz
dela, da sua vovó, que lhe dizia: ``Filha, joga neste número!''

O sonho impressionou"-a muito; nada, porém, disse ao marido. Saído que
ele foi para a repartição, determinou à criada o que tinha a fazer e
procurou afastar da memória tão estranho sonho.

Não havia, entretanto, meios para conseguir isso. A recordação dele
estava sempre presente ao seu pensamento, apesar de todos os seus
esforços em contrário.

A pressão que lhe fazia no cérebro a lembrança do sonho, pedia uma
saída, uma válvula de descarga, pois já excedia a sua força de
contenção. Tinha que falar, que contar, que comunicá"-lo a alguém\ldots{}

Fez confidência do sucedido à Genoveva. A cozinheira pensou um pouco e
disse:

--- Nhanhã: eu se fosse a senhora arriscava alguma cousa no ``bicho''.

--- Que ``bicho'' é?

--- 24 é cabra; mas não deve jogar só por um lado. Deve cercar por todos
e fazer fé na dezena, na centena, até no milhar. Um sonho destes não é
por aí cousa à toa.

--- Você sabe fazer a lista?

--- Não, senhora. Quando jogo é o Seu Manuel do botequim quem faz
``ela''. Mas a vizinha, Dona Iracema, sabe bem e pode ajudar a senhora.

--- Chame ``ela'' e diga que quero lhe falar.

Em breve chegava a vizinha e Zilda contou"-lhe o acontecido.

Dona Iracema refletiu um pouco e aconselhou:

--- Um sonho desses, menina, não se deve desprezar. Eu, se fosse a
vizinha, jogava forte.

--- Mas, Dona Iracema, eu só tenho os oitenta mil"-réis para pagar a casa.
Como há de ser?

A vizinha cautelosamente respondeu:

--- Não lhe dou a tal respeito nenhum conselho. Faça o que disser o seu
coração; mas um sonho desses\ldots{}

Zilda que era muito mais moça que Iracema, teve respeito pela sua
experiência e sagacidade. Percebeu logo que ela era favorável a que ela
jogasse. Isto estava a quarentona da vizinha, a tal Dona Iracema, a
dizer"-lhe pelos olhos.

Refletiu ainda alguns minutos e, por fim, disse de um só hausto:

--- Jogo tudo.

E acrescentou:

--- Vamos fazer a lista --- não é Dona Iracema?

--- Como é que a senhora quer?

--- Não sei bem. A Genoveva é quem sabe.

E gritou, para o interior da casa:

--- O Genoveva! Genoveva! Venha cá, depressa!

Não tardou que a cozinheira viesse. Logo que a patroa lhe comunicou o
embaraço, a humilde preta apressou"-se em explicar:

--- Eu disse a nhanhã que cercasse por todos os lados o grupo, jogasse na
dezena, na centena e no milhar.

Zilda perguntou à Dona Iracema:

--- A senhora entende dessas cousas?

--- Ora! Sei muito bem. Quanto quer jogar?

--- Tudo! Oitenta mil"-réis!

--- É muito, minha filha. Por aqui não há quem aceite. Só se for no
Engenho de Dentro, na casa do ``Halavanca'', que é forte. Mas quem há de
levar o jogo? A senhora tem alguém?

--- A Genoveva.

A cozinheira, que ainda estava na sala, de pé, assistindo os
preparativos de tão grande ousadia doméstica, acudiu com pressa:

--- Não posso ir, nhanhã. Eles me embrulham e, se a senhora ganhar, a mim
eles não pagam. É preciso pessoa de mais respeito.

Dona Iracema, por aí, lembrou:

--- É possível que o Carlito tenha vindo já de Cascadura, onde foi ver a
avó\ldots{} Vai ver, Genoveva!

A rapariga foi e voltou em companhia do Carlito, filho de Dona Iracema.
Era um rapagão dos seus dezoito anos, espadaúdo e saudável.

A lista foi feita convenientemente; e o rapaz levou"-a ao ``banqueiro''.

Passava de uma hora da tarde, mas ainda faltava muito para as duas.
Zilda lembrou"-se então do cobrador da casa. Não havia perigo. Se não
tinha vindo até ali, não viria mais.

Dona Iracema foi para a sua casa; Genoveva foi para a cozinha e Zilda
foi repousar daqueles embates morais e alternativas cruciantes,
provocados pelo passo arriscado que dera. Deitou"-se já arrependida do
que fizera.

Se perdesse, como havia de ser? O marido\ldots{} sua cólera\ldots{} as
repreensões\ldots{} Era uma tonta, uma doida\ldots{} Quis cochilar um
pouco; mas logo que cerrou os olhos, lá viu o número --- 1724. Tomava"-se
então de esperança e sossegava um pouco da sua ânsia angustiosa.

Passando, assim, da esperança ao desânimo, prelibando a satisfação de
ganhar e antevendo os desgostos que sofreria, caso perdesse --- Zilda,
chegou até à hora do resultado, suportando os mais desencontrados
estados de espírito e os mais hostis ao seu sossego. Chegando o tempo de
saber ``o que dera'', foi até à janela. De onde em onde, naquela rua
esquecida e morta, passava uma pessoa qualquer. Ela tinha desejo de
perguntar ao transeunte o ``resultado'', mas ficava possuída de vergonha
e continha"-se.

Nesse ínterim, surge o Carlito a gritar:

--- Dona Zilda! Dona Zilda! A senhora ganhou, menos no milhar e na
centena.

Não deu um ``ai'' e ficou desmaiada no sofá da sua modesta sala de
visitas.

Voltou em breve a si, graças às esfregações de vinagre de Dona Iracema e
de Genoveva. Carlito foi buscar o dinheiro que subia a mais de dois
contos de réis. Recebeu"-o e gratificou generosamente o rapaz, a mãe dele
e a sua cozinheira, a Genoveva. Quando Augusto chegou, já estava
inteiramente calma. Esperou que ele mudasse de roupa e viesse à sala de
jantar, a fim de dizer"-lhe:

--- Augusto: se eu tivesse jogado o aluguel da casa no ``bicho'' você
ficava zangado?

--- Por certo! Ficaria muito e havia de censurar você com muita
veemência, pois que uma dona de casa não\ldots{}

--- Pois, joguei.

--- Você fez isto, Zilda?

--- Fiz.

--- Mas quem virou a cabeça de você para fazer semelhante tolice? Você
não sabe que ainda estamos pagando despesas do nosso casamento?

--- Acabaremos de pagar agora mesmo.

--- Como? Você ganhou?

--- Ganhei. Está aqui o dinheiro.

Tirou do seio o pacote de notas e deu"-o ao marido, que se tornara mudo
de surpresa. Contou as pelejas muito bem, levantou"-se e disse com muita
sinceridade, abraçando e beijando a mulher\ldots{}

--- Você tem muita sorte. É o meu anjo bom.

E todo o resto da tarde, naquela casa, tudo foi alegria.

Vieram Dona Iracema, o marido, o Carlito, as filhas e outros vizinhos.

Houve doces e cervejas. Todos estavam sorridentes, palradores; e o
contentamento geral só não desandou em baile, porque os recém"-casados
não tinham piano. Augusto deitou patriotismo com o marido de Iracema.

Entretanto, por causa das dúvidas, no mês seguinte, quem fez os
pagamentos domésticos foi ele próprio, Augusto em pessoa.



\chapter[Numa e a Ninfa]{Numa e a Ninfa\footnote[*]{Publicado em junho de 1911, no jornal \emph{Gazeta da Tarde}.}}

Na rua não havia quem não apontasse a união daquele casal. Ela não era
muito alta, mas tinha uma fronte reta e dominadora, uns olhos de visada
segura, rasgando a cabeça, o busto erguido, de forma a possuir não sei
que ar de força, de domínio, de orgulho; ele era pequenino, sumido,
tinha a barba rala, mas todos lhe conheciam o talento e a ilustração.
Deputado há bem duas legislaturas, não fizera em começo grande figura;
entretanto, surpreendendo todos, um belo dia fez um ``brilhareto'', um
lindo discurso tão bom e sólido que toda a gente ficou admirada de sair
de lábios que até então ali estiveram hermeticamente fechados.

Foi por ocasião do grande debate que provocou, na câmara, o projeto de
formação de um novo estado, com terras adquiridas por força de cláusulas
de um recente tratado diplomático.

Penso que todos os contemporâneos ainda estão perfeitamente lembrados do
fervor da questão e da forma por que a oposição e o governo se
digladiaram em torno do projeto aparentemente inofensivo. Não convém,
para abreviar, relembrar aspectos de uma questão tão dos nossos dias;
basta que se recorde o aparecimento de Numa Pompílio de Castro, deputado
pelo Estado de Sernambi, na tribuna da câmara, por esse tempo.

Esse Numa, que ficou, daí em diante, considerado parlamentar consumado e
ilustrado, fora eleito deputado, graças à influência do seu sogro, o
Senador Neves Cogominho, chefe da dinastia dos Cogominhos que, desde a
fundação da república, desfrutava empregos, rendas, representações, tudo
o que aquela mansa satrapia possuía de governamental e administrativo.

A história de Numa era simples. Filho de um pequeno empregado de um
hospital militar do Norte, fizera"-se, à custa de muito esforço, bacharel
em direito. Não que houvesse nele um entranhado amor ao estudo ou às
letras jurídicas. Não havia no pobre estudante nada de semelhante a
isso. O estudo de tais coisas era"-lhe um suplício cruciante; mas Numa
queria ser bacharel, para ter cargos e proventos; e arranjou os exames
de maneira mais econômica. Não abria livros; penso que nunca viu um que
tivesse relação próxima ou remota com as disciplinas dos cinco anos de
bacharelado. Decorava apostilas, cadernos; e, com esse saber mastigado,
fazia exames e tirava distinções.

Uma vez, porém, saiu"-se mal; e foi por isso que não recebeu a medalha e
o prêmio de viagem. A questão foi com o arsênico, quando fazia prova
oral de medicina legal. Tinha havido sucessivos erros de cópias nas
apostilas, de modo que Numa dava como podendo ser encontradas na
glândula tireoide dezessete gramas de arsênico, quando se tratam de
dezessete centésimos de miligrama.

Não recebeu distinção e o rival passou"-lhe a perna. O seu desgosto foi
imenso. Ser formado já era alguma coisa, mas sem medalha era incompleto!

Formado em direito, tentou advogar; mas, nada conseguindo, veio ao Rio,
agarrou"-se à sobrecasaca de um figurão, que o fez promotor de justiça do
tal Sernambi, para livrar"-se dele.

Aos poucos, com aquele seu faro de adivinhar onde estava o vencedor ---
qualidade que lhe vinha da ausência total de emoção, de imaginação, de
personalidade forte e orgulhosa ---, Numa foi subindo.

Nas suas mãos, a justiça estava a serviço do governo; e, como juiz de
direito, foi na comarca mais um ditador que um sereno apreciador de
litígios.

Era ele juiz de Catimbau, a melhor comarca do Estado, depois da capital,
quando Neves Cogominho foi substituir o tio na presidência de Sernambi.

Numa não queria fazer mediocremente uma carreira de justiça de roça.
Sonhava a câmara, a Cadeia Velha, a Rua do Ouvidor, com dinheiro nas
algibeiras, roupas em alfaiates caros, passeio à Europa; e se lhe
antolhou, meio seguro de obter isso, aproximar"-se do novo governador,
captar"-lhe a confiança e fazer"-se deputado.

Os candidatos à chefatura de polícia eram muitos, mas ele, de tal modo
agiu e ajeitou as coisas, que foi o escolhido.

O primeiro passo estava dado; o resto dependia dele. Veio a posse. Neves
Cogominho trouxera a família para o Estado. Era uma satisfação que dava
aos seus feudatários, pois havia mais de dez anos que lá não punha os
pés.

Entre as pessoas da família, vinha a filha, a Gilberta, moça de pouco
mais de vinte anos, cheia de prosápias de nobreza, que as irmãs de
caridade de um colégio de Petrópolis lhe tinham metido na cabeça.

Numa viu logo que o caminho mais fácil para chegar a seu fim era
casar"-se com a filha do dono daquela ``comarca'' longínqua do desmedido
império do Brasil.

Fez a corte, não deixava a moça, trazia"-lhe mimos, encheu as tias
(Cogominho era viúvo) de presentes; mas a moça parecia não atinar com os
desejos daquele bacharelinho baço, pequenino, feio e tão roceiramente
vestido. Ele não desanimou; e, por fim, a moça descobriu que aquele
homenzinho estava mesmo apaixonado por ela. Em começo, o seu desprezo
foi grande; achava até ser injúria que aquele tipo a olhasse; mas,
vieram os aborrecimentos da vida da província, a sua falta de festas, o
tédio daquela reclusão em palácio, aquela necessidade de namoro que há
em toda a moça, e ela deu"-lhe mais atenção.

Casaram"-se, e Numa Pompílio de Castro foi logo eleito deputado pelo
Estado de Sernambi.

Em começo, a vida de ambos não foi das mais perfeitas. Não que houvesse
rusgas; mas, o retraimento dela e a \emph{gaucherie}\footnote{Palavra
  francesa que significa pessoa ``embaraçada''.} dele toldavam a vida
íntima de ambos.

No casarão de São Clemente\footnote{Referência à rua São Clemente,
  importante e valorizada localidade do bairro de Botafogo.}, ele vivia
só, calado a um canto; e Gilberta, afastada dele, mergulhada na leitura;
e, não fosse um acontecimento político de certa importância, talvez a
desarmonia viesse a ser completa.

Ela lhe havia descoberto a simulação do talento e o seu desgosto foi
imenso porque contava com um verdadeiro sábio, para que o marido lhe
desse realce na sociedade e no mundo. Ser mulher de deputado não lhe
bastava; queria ser mulher de um deputado notável, que falasse, fizesse
lindos discursos, fosse apontado nas ruas.

Já desanimava, quando, uma madrugada, ao chegar da manifestação do
Senador Sofonias, naquele tempo o mais poderoso chefe da política
nacional, quase chorando, Numa dirigiu"-se à mulher:

--- Minha filha, estou perdido!\ldots{}

--- Mas que há, Numa?

--- Ele\ldots{} O Sofonias\ldots{}

--- Que tem? que há? por quê?

A mulher sentia bem o desespero do marido e tentava soltar"-lhe a língua.
Numa, porém, estava alanceado e hesitava, vexado em confessar a
verdadeira causa do seu desgosto. Gilberta, porém, era tenaz; e, de uns
tempos para cá, dera em tratar com mais carinho o seu pobre marido.
Afinal, ele confessou quase em pranto:

--- Ele quer que eu fale, Gilberta.

--- Mas, você fala\ldots{}

--- E fácil dizer\ldots{} Você não vê que não posso\ldots{} Ando
esquecido\ldots{} Há tanto tempo\ldots{} Na faculdade, ainda fiz um ou
outro discurso; mas era lá, e eu decorava, depois pronunciava.

--- Faz agora o mesmo\ldots{}

--- E\ldots{} Sim\ldots{} Mas, preciso ideias\ldots{} Um estudo sobre o
novo Estado! Qual!

--- Estudando a questão, você terá ideias\ldots{}

Ele parou um pouco, olhou a mulher demoradamente e lhe perguntou de
supetão:

--- Você não sabe aí alguma coisa de história e geografia do Brasil?

Ela sorriu indefinidamente com os seus grandes olhos claros, apanhou com
uma das mãos os cabelos que lhe caíram sobre a testa; e depois de ter
estendido molemente o braço meio nu sobre a cama, onde a fora encontrar
o marido, respondeu:

--- Pouco\ldots{} Aquilo que as irmãs ensinam; por exemplo: que o rio
São Francisco nasce na serra da Canastra.

Sem olhar a mulher, bocejando, mas já um tanto aliviado, o legislador
disse:

--- Você deve ver se arranja algumas ideias, e fazemos o discurso.

Gilberta pregou os seus grandes olhos na armação do cortinado, e ficou
assim um bom pedaço de tempo, como a recordar"-se. Quando o marido ia
para o aposento próximo, despir"-se, disse com vagar e doçura:

---Talvez.

Numa fez o discurso e foi um triunfo. Os representantes dos jornais, não
esperando tão extraordinária revelação, denunciaram o seu entusiasmo, e
não lhe pouparam elogios. O José Vieira escreveu uma crônica; e a glória
do representante de Sernambi encheu a cidade. Nos bondes, nos trens, nos
cafés, era motivo de conversa o sucesso do deputado dos Cogominhos:

--- Quem diria, hein? Vá a gente fiar"-se em idiotas. Lá vem um dia que
eles se saem. Não há homem burro --- diziam ---, a questão é querer\ldots{}

E foi daí em diante que a união do casal começou a ser admirada nas
ruas. Ao passarem os dois, os homens de altos pensamentos não podiam
deixar de olhar agradecidos aquela moça que erguera do nada um talento
humilde; e as meninas olhavam com inveja aquele casamento desigual e
feliz.

Daí por diante, os sucessos de Numa continuaram. Não havia questão em
debate na câmara sobre a qual ele não falasse, não desse o seu parecer,
sempre sólido, sempre brilhante, mantendo a coerência do partido, mas
aproveitando ideias pessoais e vistas novas. Estava apontado para
ministro e todos esperavam vê"-lo na secretaria do Largo do
Rossio\footnote{Localizado no centro da cidade do Rio.}, para que ele
pusesse em prática as suas extraordinárias ideias sobre instrução e
justiça.

Era tal o conceito de que gozava que a câmara não viu com bons olhos
furtar"-se, naquele dia, ao debate que ele mesmo provocara, dando um
intempestivo aparte ao discurso do Deputado Cardoso Laranja, o
formidável orador da oposição.

Os governistas esperavam que tomasse a palavra e logo esmagasse o
adversário; mas não fez isso.

Pediu a palavra para o dia seguinte e o seu pretexto de moléstia não foi
bem aceito.

Numa não perdeu tempo: tomou um tílburi, correu à mulher e deu"-lhe parte
da atrapalhação em que estava. Pela primeira vez, a mulher lhe pareceu
com pouca disposição de fazer o discurso.

--- Mas, Gilberta, se eu não o fizer amanhã, estou perdido!\ldots{} E o
ministério? Vai"-se tudo por água abaixo\ldots{} Um esforço\ldots{} E
pequeno\ldots{} De manhã, eu decoro\ldots{} Sim, Gilberta?

A moça pensou e, ao jeito da primeira vez, olhou o teto com os seus
grandes olhos cheios de luz, como a lembrar"-se, e disse:

--- Faço; mas você precisa ir buscar já, já, dois ou três volumes sobre
colonização\ldots{} Trata"-se dessa questão, e eu não sou forte. E
preciso fingir que se tem leituras disso\ldots{} Vá!

--- E os nomes dos autores?

--- Não é preciso\ldots{} O caixeiro sabe\ldots{} Vá!

Logo que o marido saiu, Gilberta redigiu um telegrama e mandou a criada
transmiti"-lo.

Numa voltou com os livros; marido e mulher jantaram em grande intimidade
e não sem apreensões. Ao anoitecer, ela recolheu"-se à biblioteca e ele
ao quarto.

No começo, o parlamentar dormiu bem; mas bem cedo despertou e ficou
surpreendido em não encontrar a mulher a seu lado. Teve remorsos. Pobre
Gilberta! Trabalhar até àquela hora, para o nome dele, assim
obscuramente! Que dedicação! E --- coitadinha! --- tão moça e ter que
empregar o seu tempo em leituras árduas! Que boa mulher ele tinha! Não
havia duas\ldots{} Se não fosse ela\ldots{} Ah! Onde estaria a sua
cadeira? Nunca seria candidato a ministro\ldots{} Vou fazer"-lhe uma
mesura, disse ele consigo. Acendeu a vela, calçou as chinelas e foi pé
ante pé até ao compartimento que servia de biblioteca.

A porta estava fechada; ele quis bater, mas parou a meio. Vozes
abaladas\ldots{} Que seria? Talvez a Idalina, a criada\ldots{} Não, não
era; era voz de homem. Diabo! Abaixou"-se e olhou pelo buraco da
fechadura. Quem era? Aquele tipo\ldots{} Ah! Era o tal primo\ldots{}
Então, era ele, era aquele valdevinos, vagabundo, sem eira nem beira,
poeta sem poesias, frequentador de chopes; então, era ele quem lhe fazia
os discursos? Por que preço?

Olhou ainda mais um instante e viu que os dois acabavam de beijar"-se. A
vista se lhe turvou; quis arrombar a porta; mas logo lhe veio a ideia do
escândalo e refletiu. Se o fizesse, vinha a coisa a público; todos
saberiam do segredo da sua ``inteligência'' e adeus câmara, ministério e
--- quem sabe? --- a presidência da república. Que é que se jogava ali? A
sua honra? Era pouco. O que se jogava ali eram a sua inteligência, a sua
carreira; era tudo! Não, pensou ele de si para si, vou deitar"-me.

No dia seguinte, teve mais um triunfo.


