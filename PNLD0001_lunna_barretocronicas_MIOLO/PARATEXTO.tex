\section{Sobre o autor}

Afonso Henriques de Lima Barreto, ou simplesmente Lima Barreto, nasceu
no simbólico dia 13 de maio de 1881 --- sete anos antes da Abolição
formal da escravatura no Brasil. Tanto sua avó materna quanto paterna
foram mulheres escravizadas. Sua mãe, Amália Augusta Pereira de
Carvalho, foi professora de primeiras letras em uma escola que
funcionava em sua própria casa. O pai de Lima Barreto, João Henriques de
Lima Barreto, conseguiu se destacar como um profissional importante no
ramo da imprensa, trabalhando como tipógrafo.

Lima Barreto, graça aos esforços dos pais, recebeu uma educação de ótima
qualidade, embora tenha perdido a mãe quando ainda tinha seis anos de
idade. Estudou nos melhores colégios do Rio de Janeiro e chegou a cursar
a Faculdade de Engenharia da Escola Politécnica, da qual abandonou sem
conseguir se formar. Tal abandono se deu em virtude de uma grave doença
mental que deixou o pai incapacitado, além de constantes reprovações em
algumas disciplinas na Faculdade. A necessidade de assumir a
responsabilidade financeira de sua casa --- com o pai doente e mais três
irmãos para cuidar --- pesou definitivamente para desistência do curso.

Lima Barreto prestou concurso para a Secretaria do Ministério da Guerra,
onde trabalhou até se aposentar por invalidez, com graves problemas
psicológicos devidos ao abuso do álcool, que o levou a mais de uma
internação no ``Hospício'', o Hospital Nacional de Alienados.

Como escritor foi um dos maiores de nossa literatura. Iniciou sua
carreira ainda como estudante da Escola Politécnica, publicando pequenos
textos nos jornais estudantis. E nunca mais parou. Escreveu romances,
contos, crônicas, artigos de jornal, ensaios de crítica literária.

Combateu de frente, com suas palavras e ideias, os desmandos dos
poderosos, denunciando a corrupção dos políticos, a ganância dos ricos,
a ingerência do Estado frente à miséria do povo, a violência contra as
mulheres e o racismo. Este último tema com especial envolvimento, pois,
como todo homem negro, sofreu muito com o preconceito expressou tal
sofrimento por quase toda sua obra.

Lima Barreto é um autor extremamente necessário nos dias de hoje; um
verdadeiro rebelde contra os inimigos do povo. Lendo alguns de seus
textos temos a impressão de que foram escritos por um autor nosso
contemporâneo, tanto pelo estilo quanto pelos problemas que procura
levantar. Desejamos aos leitores e leitoras um ótimo encontro com este
escritor tão essencial para se compreender o Brasil.

\section{Sobre o livro}

A presente Antologia reúne crônicas e contos de Lima Barreto. A grande
maioria desses textos foi publicada em jornais e revistas da época, que
posteriormente foram agrupados em livros e coletâneas. A seleção que a
leitora e o leitor têm em mãos foi dividida em duas partes --- Crônicas e
Contos.

A primeira parte, dedicada às crônicas, foi pensada de modo a reunir os
textos em pequenos conjuntos temáticos. Tais conjuntos vêm precedidos
por subtítulos. A ideia é apresentar o escritor de uma forma mais leve,
com textos curtos e linguagem fluída. Ao mesmo tempo, convidando a
leitora e o leitor a refletir sobre alguns dos principais temas que Lima
Barreto desenvolveu ao longo de sua vida, tanto na atividade
jornalística quanto em sua prosa de ficção, além de irem criando
intimidade com o estilo do autor.

Alguns temas que organizam essa primeira parte da Antologia são:
\textit{Autobiografia}; \textit{Cotidiano e vida nos subúrbios};
\textit{Futebol}; \textit{Nossa Política e nossos políticos};
\textit{Violência contra as mulheres} e \textit{Humor}.

A segunda parte da Antologia reúne o que de mais significativo Lima
Barreto escreveu sob o gênero conto. São deste escritor alguns dos
principais textos escritos, no Brasil, dentro desta modalidade de
narrativa curta: ``O homem que sabia javanês'', ``A nova Califórnia e
``Como o `homem' chegou'' podem ser considerados como grandes
realizações do conto brasileiro.

Nessa parte do livro, leitoras e leitores entrarão em contato com a
produção ficcional propriamente dita de Lima Barreto; algo para o qual
já vinham sendo preparados com a leitura das crônicas e, principalmente,
com o conjunto de textos selecionados na categoria \emph{Humor}.

Os contos não precisam ser lidos necessariamente na ordem em que se
encontram. Cada um concentra em si um universo particular daquilo que se
convencionou chamar de problemas e dilemas da condição humana; e podemos
dizer \emph{da condição humana em geral e brasileira em particular.}

O livro gerou um número grande de notas de rodapé, principalmente para a
contextualização de nomes de pessoas, localização geográfica,
informações sobre obras literárias e outras obras artísticas, termos
antigos ou específicos que não são mais utilizados hoje em dia, etc.
Mesmo assim, as e os estudantes encontrarão algumas palavras que vão
exigir uma consulta ao dicionário --- físico ou virtual. Essa é também
uma maneira de se habituar à leitura de textos mais antigos e
contemporâneos, para uma melhor compreensão da totalidade daquilo que
vai ser lido.

No final da Antologia preparamos outro texto mais extenso, no qual
aparecem algumas informações complementares sobre o autor, o tempo em
que viveu e uma abordagem sobre os dois gêneros literários que
escolhemos para compor o livro: Crônicas e Contos. Desejamos a todas e
todos os estudantes uma ótima leitura!

\section{Sobre o gênero}