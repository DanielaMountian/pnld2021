\chapter{Vida e obra de Lima Barreto}

\begin{flushright}
\textsc{alexandre rosa}
\end{flushright}

\noindent{}O volume que organizamos consiste num pequeno recorte na grande produção de
crônicas e contos de Lima Barreto, publicados quase que totalmente na imprensa
da época --- jornais e revistas. Nosso intuito foi o de aproximar o escritor
brasileiro ao público jovem, buscando textos que apresentem uma fácil
compreensão, leitura fluida e que dialoguem com as questões contemporâneas
importantes para essa faixa etária.  

A linguagem da crônica permitirá uma
leitura mais agradável, menos densa, embora necessitando de contextualização em
muitos casos. Isso, no entanto, pode ser um bom pretexto para a prática de
atividades incipientes de pesquisa textual e histórica. A organização das
crônicas em pequenos conjuntos temáticos teve como objetivo possibilitar a
realização de trabalhos e atividades continuadas, principalmente pelo fato de
apresentarem questões relevantes para os dias atuais.  A linguagem dos contos,
além de suscitar questões importantes para a formação cidadã das alunas e dos
alunos, possibilita a fruição do texto enquanto obra de arte, como ficção.
Vocês perceberão que muitos contos são desenvolvimentos mais ficcionais de
assuntos que o autor já havia trabalhado no nível da crônica, do jornalismo.    



\section{Sobre o autor}

Afonso Henriques de Lima Barreto, ou simplesmente Lima Barreto, nasceu
no simbólico dia 13 de maio de 1881 --- sete anos antes da Abolição
formal da escravatura no Brasil. Tanto sua avó materna quanto paterna
foram mulheres escravizadas. Sua mãe, Amália Augusta Pereira de
Carvalho, foi professora de primeiras letras em uma escola que
funcionava em sua própria casa. O pai de Lima Barreto, João Henriques de
Lima Barreto, conseguiu se destacar como um profissional importante no
ramo da imprensa, trabalhando como tipógrafo.

Lima Barreto, graça aos esforços dos pais, recebeu uma educação de ótima
qualidade, embora tenha perdido a mãe quando ainda tinha seis anos de
idade. Estudou nos melhores colégios do Rio de Janeiro e chegou a cursar
a Faculdade de Engenharia da Escola Politécnica, da qual abandonou sem
conseguir se formar. Tal abandono se deu em virtude de uma grave doença
mental que deixou o pai incapacitado, além de constantes reprovações em
algumas disciplinas na Faculdade. A necessidade de assumir a
responsabilidade financeira de sua casa --- com o pai doente e mais três
irmãos para cuidar --- pesou definitivamente para desistência do curso.

Lima Barreto prestou concurso e começou a trabalhar como amanuense na
Secretaria do Ministério da Guerra, uma repartição burocrática do Exército,
onde trabalhou até se aposentar por invalidez, com graves problemas
psicológicos devidos ao abuso do álcool, que o levou a mais de uma
internação no ``Hospício'', o Hospital Nacional de Alienados.

Como escritor foi um dos maiores de nossa literatura. Iniciou sua
carreira ainda como estudante da Escola Politécnica, publicando pequenos
textos nos jornais estudantis. No geral eram textos humorísticos e
satíricos, alguns dos quais tinham como personagens os próprios
professores da instituição. E nunca mais parou de escrever, compondo romances,
contos, crônicas, artigos de jornal, ensaios de crítica literária.

Seu primeiro livro, \emph{Recordações do escrivão Isaías Caminha}, foi
publicado em 1909. A partir daí, Lima Barreto começou a despertar o
incômodo de muitos setores da intelectualidade. Seu livro de estreia
apresenta uma forte denúncia contra o preconceito de cor, além de tecer
severas críticas ao jornalismo, à imprensa e à política da ``Primeira
República''.

De origem afrodescendente, experimentou na pele a discriminação racial e
os limites impostos pelo racismo para quem desejava, como ele, uma
carreira bem sucedida, primeiro como Engenheiro, depois como escritor e
jornalista. Sua literatura, bem como sua atividade jornalística, trazem
as marcas do inconformismo social, a revolta contra os preconceitos e
injustiças, principalmente contra o racismo, que combateu abertamente,
seja através de artigos para a imprensa ou de sua prosa de ficção.

Foi um dos primeiros escritores a perceber, junto a Euclides da Cunha,
autor de \emph{Os Sertões}, e de Graça Aranha, com o romance
\emph{Canaã}, ambos publicados em 1902, que o advento da Abolição (1888)
e a Proclamação da República (1889) não trouxeram nenhum tipo de alívio
para a população pobre, sobretudo para os descendentes de escravos.
Combateu a ``República'', que considerava um arranjo de poderosos e
politiqueiros de interesses mesquinhos, bem como tudo aquilo que ajudava
a sustentar ideologicamente um empreendimento que na prática não tinha
nada de democrático ou republicano. Muitos historiadores dizem ser
impossível compreender a essência da chamada ``Primeira República''
(1889--1930) sem estudar a obra de Lima Barreto.

O escritor apreendeu uma
sociedade que havia mudado muito rapidamente ao longo dos vinte
primeiros anos do século \textsc{xx}. As antigas elites do Império, compostas por
fazendeiros escravagistas, foram substituídas por uma nova elite cada
vez mais citadina, moderna e financista. Mesmo os chamados barões do
café compunham uma nova camada social, mais ligada ao desenvolvimento de
setores modernos, como ferrovias, portos, eletricidade, urbanismo,
arquitetura (palacetes, teatros, cinemas). No campo político, as
promessas de uma nova era, com a Proclamação da República, ficaram muito
aquém da realidade. O que se via era um conchavo de oligarquias se
alternando no poder, um acúmulo cada vez maior de riquezas, enquanto o
grosso da população vivia em condições semelhantes, senão piores, que na
época do Império.

Lima Barreto não só viveu tudo isso, mas compôs um retrato crítico e
fiel da época, sem deixar de se posicionar ao lado dos desfavorecidos.
Por isso, nunca participou como membro efetivo da ``elite literária'' da
época. Com fama de ``boêmio'' e ``beberrão'', passou por períodos de
internação em Hospitais psiquiátricos, dos quais temos algumas das mais
dilacerantes páginas testemunhais, conforme aparecem em \emph{O
cemitério dos vivos}, romance que deixou inacabado, e no \emph{Diário do
Hospício}, escrito nas épocas de internação.

Combateu de frente, com suas palavras e ideias, os desmandos dos
poderosos, denunciando a corrupção dos políticos, a ganância dos ricos,
a ingerência do Estado frente à miséria do povo, a violência contra as
mulheres e o racismo. Este último tema com especial envolvimento, pois,
como todo homem negro, sofreu muito com o preconceito e expressou tal
sofrimento por quase toda sua obra.

Assim, Lima Barreto é um autor extremamente necessário nos dias de hoje; um
verdadeiro rebelde contra os inimigos do povo. Lendo alguns de seus
textos temos a impressão de que foram escritos por um autor nosso
contemporâneo, tanto pelo estilo quanto pelos problemas que procura
levantar. Desejamos aos leitores e leitoras um ótimo encontro com este
escritor tão essencial para se compreender o Brasil.

\section{Sobre a obra}

%\subsection{Contexto histórico}

Como salientamos um pouco acima, as obras de Lima Barreto --- crônicas, contos,
novelas, romances, diários e memórias --- não são apenas relatos
instantâneos de uma realidade que se apresentava hostil ao escritor. Com
o tempo, a pecha de escritor ``desleixado'' foi ficando superada. Hoje
em dia, não há dúvidas em relação a um livro como \emph{Triste fim de
Policarpo Quaresma} figurar como um dos maiores de nossa literatura. E
mesmo livros que não foram bem recebidos na época de publicação,
\emph{Recordações do escrivão Isaías Caminha} ou \emph{Vida e Morte de
M. J. Gonzaga de Sá} (1919) cada vez mais são objeto de estudos e novas
interpretações. Não ficaram estagnados no tempo, como ficou a grande
maioria dos ``famosos'' e mais lidos escritores seus contemporâneos.

Lima Barreto escreveu numa época em que a literatura tinha um parentesco
fortíssimo com os jornais. Normalmente, antes de ganhar uma edição em
livro, os textos apareciam primeiramente nas edições dos jornais e
revistas. A atividade jornalística não havia se desenvolvido como hoje
em dia e boa parte do que era escrito nos jornais vinha das mãos dos
escritores literatos. A profissão de \emph{jornalista}, como nós a
conhecemos atualmente, estava dando os primeiros passos.

Conforme salientou o sociólogo Sérgio Miceli, em seu livro \emph{Poder,
Sexo e Letras na República Velha}: ``toda a vida intelectual era
dominada pela grande imprensa, que constituía a principal instância de
produção cultural da época e que fornecia a maioria das gratificações e
posições intelectuais.''\footnote{Sérgio Miceli. \emph{Poder, Sexo e
  Letras na República Velha}. São Paulo: Perspectiva, 1977, p.\,15.}

A principal atividade dos escritores era o jornalismo, onde conseguiam
uma renda extra ``cada vez mais indispensável'', além de, alcançarem
``salários melhores, sinecuras burocráticas e favores
diversos.''\footnote{Idem, p. 77.} Não faltavam nos principais jornais e
revistas da época seções destinadas à literatura, além do espaço cada
vez maior aberto aos escritores em diversas colunas dos jornais, ou até
mesmo cargos de redatores e diretores.

Jornais e revistas ilustradas foram os grandes veículos para os textos
de ficção, em suas mais diversas vertentes. Conforme observou Nelson
Werneck Sodré: ``os homens de letras buscarem no jornal o que não
encontravam no livro: notoriedade, em primeiro lugar; um pouco de
dinheiro, se possível.''\footnote{Nelson Werneck Sodré. \emph{A História
  da Imprensa no Brasil}. Rio de Janeiro: Civilização Brasileira, 1966,
  p. 297.}

Este processo aconteceu primeiro como estratégia por parte de alguns
donos de jornais para conseguirem mais leitores e assinantes, além de
diversificar o material jornalístico, sempre voltado para assuntos
políticos e administrativos.

Na metade do século \textsc{xix} a literatura já tinha ocupado um lugar de
destaque na imprensa brasileira. O gênero ``folhetim'' fazia parte dos
principais jornais da época. Junto ao folhetim, a crônica também foi se
desenvolvendo. Como demonstra Marcus Vinicius Soares, a crônica é
anterior ao folhetim e ``já fazia parte do universo jornalístico, sempre
acompanhada pelo epíteto designativo de sua matéria específica: crônica
dramática, crônica musical, crônica literária, etc.''\footnote{Marcus
  Vinicius Soares. \emph{João do Rio e a nova esfera da crônica no
  século \textsc{xx}}. São Paulo: Intermeios, 2016, p. 122.}

Já no século \textsc{xx}, o jornalismo passa a ser uma empresa especializada, com
diferenciação nas modalidades de texto. Na diagramação dos jornais,
criam"-se espaços dedicados à literatura. Surgem os `suplementos
literários', como sintoma desta nova relação estabelecida entre
jornalismo e literatura.


%No final da Antologia preparamos outro texto mais extenso, no qual
%aparecem algumas informações complementares sobre o autor, o tempo em
%que viveu e uma abordagem sobre os dois gêneros literários que
%escolhemos para compor o livro: Crônicas e Contos. Desejamos a todas e
%todos os estudantes uma ótima leitura!

\section{Sobre o gênero}


De estreitas relações com o jornalismo foi também o desenvolvimento do
gênero conto, em sua modalidade moderna, ou seja, escrita e voltada para
um público leitor. Nesse ambiente literário, no entanto, o conto também
era considerado algo esteticamente secundário, um ``treinamento'' para
narrativas de maior fôlego. Embora já tivéssemos desenvolvida no Brasil
uma tradição de contadores de histórias oriundos da cultura oral, o
conto literário ainda era visto como uma literatura de menor valor
artístico.

Tradicionalmente já tínhamos o hábito do ``causo'' e da ``anedota''; dos
indígenas conhecemos suas lendas, mitos e cosmologias; as escravas e
amas"-de"-leite povoaram o imaginário dos brancos com suas histórias
cheias de seres fantásticos, magias, feiticeiros, entidades do panteão
religioso, etc. Conforme demonstra a pesquisadora Sônia Brayner: ``a
oralidade do contar foi criando e embalando os embriões de personagens e
tramas mais tarde corporificados e desenvolvidos pela literatura
escrita''\footnote{Sônia Brayner. ``O conto de Machado de Assis''. In:
  \emph{O conto de Machado de Assis -- Antologia}. Rio de Janeiro:
  Civilização Brasileira, 1981, p. 06.}.

Foi neste contexto que Lima Barreto desenvolveu sua produção literária.
Incorporou em sua produção as novas técnicas expressivas, que surgiram
das relações entre literatura e jornalismo. E também nossa tradição
oral, que junto à dicção jornalística de sua prosa, consolidou uma forma
de escrita bastante moderna para a época. Percebemos muito bem esses
traços em narrativas como {\textit{O gambá}}, {\textit{A
Pescaria}}, {\textit{Rocha, o guerreiro}}, {\textit{O homem
que sabia javanês}}, que escolhemos para compor a \emph{Antologia},
entre outras tantas criações.

Além desse aspecto formal, como já mencionado, o autor se destacou por suas ideias políticas e sociais, escrevendo num momento de grande efervescência cultural
e modernização do país, principalmente de algumas capitais como Rio de
Janeiro, São Paulo, Belo Horizonte, Manaus, Belém, entre outras. Era o
tempo da \emph{belle époque}, cuja literatura representativa foi
definida por Antonio Candido nos seguintes termos:

\begin{quote}
\ldots{} era sobretudo uma conservação de formas cada vez mais vazias de
conteúdo; uma tendência a repisar soluções plásticas que, na sua
superficialidade, conquistaram por tal forma o gosto médio, que até hoje
representam para ele a boa norma literária. Uma literatura para a qual o
mundo exterior existia no sentido mais banal da palavra, e que por isso
mesmo se instalou num certo oficialismo graças, em parte, à ação
estabilizadora da Academia Brasileira. As letras, o público burguês e o
mundo oficial se entrosavam numa harmoniosa mediania.\footnote{\textsc{candido},
  Antonio. ``Literatura e cultura de 1900 a 1945''. In: \emph{Literatura
  e Sociedade}. Rio de Janeiro: Ouro sobre Azul, 2006, p. 125.}
\end{quote}

A literatura de Lima Barreto, no entanto, foi exatamente o oposto da
``literatura oficial'' praticada na \emph{belle époque} carioca. Ao
desobedecer às regras do bem escrever e do ``bom gosto literário'',
colocou a prosa de ficção brasileira em um outro patamar estético. Ele
mesmo definiu sua escrita como ``literatura militante'', um meio de
atuação política e estética.

Muitos estudiosos da literatura se renderam ao conceito de
``Pré"-modernismo'' para tentar dar conta das transformações estéticas
ocorridas naquele período, compreendido mais ou menos entre o último
quarto do século \textsc{xix} e as duas primeiras décadas do século \textsc{xx}. Trata"-se
de um momento em que as escolas literárias já não tinham tanto fôlego
quanto antes --- Romantismo, Parnasianismo, Simbolismo, Naturalismo --- e
cediam lugar a formas cada vez mais híbridas e inovadoras de composição.
As chamadas vanguardas artísticas e o surgimento de novas tecnologias,
como as revistas ilustradas, o jornalismo empresarial, o telégrafo, a
expansão da eletricidade, o rádio, o cinema, enfim, o próprio momento de
modernização do país (industrialização, urbanização, rápidas mudanças
nos costumes), todo esse arcabouço de novidades e invenções operou e
exigiu uma nova mentalidade estética, uma maneira outra de percepção da
realidade e dos costumes.

Muitos escritores, que na época da \emph{belle époque} fizeram um enorme
sucesso --- Coelho Neto, Figueiredo Pimentel, Afrânio Peixoto, Medeiros e
Albuquerque, para citarmos apenas alguns --- quase não são lembrados ou
conhecidos hoje em dia. Lima Barreto teve seu nome quase esquecido após
sua morte, ocorrida no dia 01 de novembro de 1922. Sua obra caiu num
limbo de quase três décadas e só voltou a ser reeditada graças aos
esforços de Francisco de Assis Barbosa que, além de escrever a grande
biografia do autor, \emph{A vida de Lima Barreto}, em 1952, também
coordenou a edição das \emph{Obras completas de Lima Barreto,} em 17
volumes, publicados em 1956 pela editora Brasiliense. Somente a partir
daí o nome e a obra do grande escritor carioca voltou a fazer parte do
patrimônio das letras brasileiras.

A crítica literária oscilou bastante até reconhecer em Lima Barreto um
escritor de primeiro escalão. Desde seus primeiros romances,
\emph{Recordações do escrivão Isaías Caminha} (1909), \emph{Triste fim
de Policarpo Quaresma} --- publicado primeiramente em folhetins, no
Jornal do Comércio, em 1911 e depois editado em livro em 1915 --- e
\emph{Numa e a Ninfa} (1915), o escritor foi apreciado ora como um
grande prosador e crítico da sociedade, ora como um artista ``menos
realizado''. Muitos estudiosos consideraram que Lima Barreto exagerou um
pouco no ``autobiografismo'', principalmente em seu romance de estreia,
como também no excesso de coloquialismos, num certo ``desleixo''
artístico, na irregularidade do estilo, que oscila entre uma linguagem
jornalística, sociológica, anárquica. Tudo isso foi encarado como
``incorreções da forma'', que sempre teve como exemplares os romances e
contos de Machado de Assis.

Lima Barreto foi um escritor de outra geração, uma após a do autor de
\emph{Memórias póstumas de Brás Cubas} (1881), ou seja, foi um escritor
do período republicano e pós"-abolicionista. Nesse sentido, tanto do
ponto de vista formal, quanto temático, Lima Barreto assumiu uma posição
renovadora, confrontadora, pois a literatura não podia mais ser um
privilégio das elites letradas, nem se abster dos problemas sociais
concretos pelos quais a população em sua grande maioria passava,
sobretudo os descendentes de escravos.

Além de cronista e contista, Lima Barreto foi um grande romancista e
considerado por muitos pesquisadores como um dos protótipos do
escritor"-jornalista. A atualidade de sua obra é algo a ser levado em
consideração, principalmente pelos temas que abordou,
impressionantemente presentes e atuantes nos dias de hoje, e pela forma
renovadora de sua literatura, que só muitas décadas depois de sua morte
teve o seu devido reconhecimento.

\subsection{teoria dos gêneros}

\begin{quote}
O conto é, do ângulo dramático, unívoco, univalente. [\ldots]
Etimológicamente preso à linguagem teatral,
``drama'' significava ``ação''. E com o tempo passou a designar
toda peça destinada à representação. Na época romântica, dado o
princípio da fusão de gêneros, entendia-se por drama o misto de
tragédia e comédia. Transferido para a prosa de ficção, o termo
``drama'' entrou a significar ``conflito'', ``atrito''. Nesse caso,
``ação'' ``cortflito'' se tonaram equivalentes, uma vez que toda
ação pressupõe conflito, e este, promove a ação, ou por meio dela
se manifesta; em suma, ambos se implicam mutuamente.

O conto é, pois, uma narrativa unívoca, univalente: constitui
uma \textit{unidade dramática}, uma \textit{célula dramática}, visto gravitar ao
redor de um só conflito, um só drama, uma só ação. Caracteriza-se,
assim, por conter \textit{unidade de ação}, tomada esta como a sequência de atos praticados pelos protagonistas, ou de acontecimentos de
que participam. A ação pode ser externa, quando as personagens se
deslocam no espaço e no tempo, e interna, quando o conflito se
localiza em sua mente.\footnote{\textsc{moisés}, Massaud. \textit{A criação literária}. São Paulo: Cultrix, 2006, p.\,40.}
\end{quote}

Partindo da definição de Massaud Moisés sobre o conto, evidencia"-se a principal característica desse gênero literário: a unidade de conflito, condensada em ações que se completam em um único enredo. Ao conto, ainda seguindo Moisés, aborrecem as divagações e os excessos, pois há uma concentração de efeitos e pormenores essenciais, em sua brevidade, para o bom funcionamento do conto.
Cada construção, cada palavra nesse gênero tem sua razão de existir, pois integra a economia global da narrativa.

Apesar da brevidade de sua forma, o conto desdobra"-se em muitas direções e implicações, e o faz a partir de elementos restritos: a unidade dramática, como já mencionada, assim como a presença de poucas personagens e a limitação espacial e temporal. Um ótimo exemplo é o conto ``Missa do galo'', de Machado de Assis, em que o narrador, Nogueira, conta a sua experiência de uma única noite na companhia de sua hospedeira, D.\,Conceição. Apesar de unidade temporal (a noite que antecede a Missa do galo), espacial (uma sala na casa de D.\,Conceição) e da redução dramática, basicamente, à interação entre duas personagens, Conceição e Nogueira, esse conto desdobra"-se em muitas direções. A companhia de Conceição desperta a sexualidade de Nogueira, e seu impacto é tão profundo que o narrador relembra aos leitores esse acontecimento de sua juventude. As intenções da anfitriã, narradas e, logo, distorcidas pela memória de Nogueira, também são ambíguas, levantando as mais diversas questões e interpretações.

Como reflete o escritor argentino Julio Cortázar, o conto consegue, de forma muito concisa, despertar ``uma realidade infinitamente mais vasta que a do seu mero argumento'', influindo ``em nós com uma força que nos faria suspeitar da modéstia do seu conteúdo aparente, da brevidade do seu texto''.\footnote{\textsc{cortázar}, Julio. \textit{Valise de cronópio}. São Paulo: Editora Perspectiva, 2008, p.\,155.}

Apesar da aparente banalidade do argumento, o conto abre essa possibilidade de desenvolver o tema em profundidade, em contraposição à aparente concisão narrativa. Realiza plenamente, assim, o que Cortázar define como o gênero do conto:

\begin{quote}
Um escritor argentino, muito amigo do boxe, dizia"-me que nesse combate que se travra entre um texto apaixonante e o leitor, o romance ganha sempre por pontos, enquanto que o conto deve ganhar por \textit{knock"-out}. É verdade, na medida em que o romance acumula progressivamente seus efeitos no leitor, enquanto que um bom conto é incisivo, mordente, sem trégua desde as primeiras frases. Não se entenda isto demasiado literalmente, porque o bom contista é um boxeador muito astuto, e muitos dos seus golpes iniciais podem parecer pouco eficazes quando, na realidade, estão minando já as resistências mais sólidas do adversário.
Tomem os senhores qualquer grande conto que seja de sua preferência, e analisem a primeira página. Surpreender"-me"-ia se encontrassem elementos gratuitos, meramente decorativos. O contista sabe que não pode proceder acumulativamente, que não tem o tempo por aliado; seu único recurso é trabalhar em profundidade, verticalmente, seja para cima ou para baixo do espaço literário.\footnote{Ibid., p.\,152.}
\end{quote}

Já sobre a crônica enquanto gênero literário, escreve o crítico literário Antonio Candido que ``é filha do jornal e da era da máquina, onde tudo acaba tão depressa. Ela não foi feita originalmente para o livro, mas para essa publicação efêmera que se compra num dia e no seguinte é usada para embrulhar um par de sapatos ou forrar o chão de cozinha''.\footnote{\textsc{cândido}, Antonio. ``A vida ao rés do chão''. In: \textit{Para gostar de ler: crônicas}, volume 5. São Paulo: Ática.}

Como se observa, para Antonio Candido esse gênero literário se caracteriza por sua aparição no jornal, revelando uma de suas características que mais a afasta do conto, com o que é constantemente confundida: a crônica aborda os acontecimentos do dia a dia, fatos banais, cotidianos, pequenos eventos e conversas observadas pelo cronista e plasmada, muitas vezes de forma irônica e sarcástica, nos jornais e periódicos de circulação regular.

Apesar de seu estreito vínculo com os acontecimentos contemporâneos, lastreado pela atualidade e imediaticidade dos jornais, a crônica não se limita a registrar a realidade tal qual uma notícia de jornal, pois, no que tem de brevidade na forma, ganha muitas vezes na distenção em profundidade do que aborda. Em uma crônica como ``O mineirinho'', por exemplo, de Clarice Lispector, o texto não se prende ao acontecimento imediato. Na crônica, a escritora narra uma notícia de jornal da época, o assassinato do bandido Mineirinho, morto com treze tiros pela polícia.
A violência e a brutalidade da execução, no entanto, demovem a escritora do mero relato estilizado. Ela alça sua crônica a indagações morais, éticas e existenciais que, ao cabo, não dizem respeito apenas ao Mineirinho ou ao ato de roubar e matar, mas a todos enquanto humanidade.

Sob a ótica do cronista, qualquer fato banal ganha relevância e passa a ter uma existência autônoma em relação ao seu desdobramento fatídico.
Nas palavras de Massaud Moisés, a crônica

\begin{quote}
move-se entre ser no e para o jornal,
uma vez que se destina a ser lida na folha diária ou na revista. Difere, porém,
da maneira substancialmente jornalística naquilo em que, apesar de fazer do
seu cotidiano o seu húmus permanente, não visa à mera informação: o seu
objetivo, confesso ou não, reside em transcender o dia a dia pela universalização das suas virtualidades latentes, objetivo esse via de regra minimizado
pelo jornalista de ofício. O cronista pretende-se não o repórter, mas o poeta ou
o ficcionista do cotidiano.\footnote{\textsc{moisés}, Massaud. \textit{A criação literária II}. São Paulo: Cultrix, 2006, p.\,104.}
\end{quote}

\section*{Sobre nossa equipe}

Alexandre Rosa é educador, escritor, bacharel e licenciado em Ciências Sociais pela Universidade de São Paulo. É Mestre em Literatura Brasileira pelo Instituto de Estudos Brasileiros -- \textsc{ieb/usp}, com pesquisa sobre os contos de Lima Barreto.
