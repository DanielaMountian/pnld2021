\chapter[O manuscrito de Joyce, por José Roberto O'Shea]{Apresentação}

\section{Sobre o autor}

Nascido a 2 de fevereiro de 1882 no Condado de Dublin, Irlanda, James Joyce foi um dos mais importantes escritores modernos. Filho mais velho dos dez filhos de John Stanislaus Joyce (1849--1931), aos vinte anos o jovem escritor foi morar em Paris, para estudar medicina. Retornou à Irlanda no ano seguinte, em 1903, para assistir sua mãe doente. Desse ano data o seu \emph{Um retrato do artista}. 
Em 1904 conhece Nora Barnacle, com quem ficaria toda a sua vida, e parte para a Croácia, passando pela Itália, Paris e Zurique, onde falece em 1941.


\section{Sobre a obra}

\textsc{Em 7 de janeiro} de 1904, James Joyce escreve no decorrer de um único dia
um conto ou ensaio autobiográfico cujo conteúdo mescla admiração e
ironia.  Por sugestão do irmão três anos mais jovem, Stanislaus, Joyce
intitula o texto “Um retrato do artista” e o envia aos editores da
recém"-criada revista literária \textit{Dana}.  Por meio de um processo
que levaria  cerca de uma década, o conto seria transformado em
\textit{Stephen Herói}, obra bastante extensa, posteriormente enxugada
em sua versão final sob o título de \textit{Um retrato do artista
quando jovem}.\footnote{ Ellmann, 149.}		


\subsection*{Características do manuscrito}

Em 1935, a livreira Sylvia Beach, responsável pela primeira edição de
\textit{Ulisses} (1922), põe à venda um manuscrito produzido pelo
próprio punho de Joyce com páginas numeradas --- de 519 a 902 --- que
constitui um fragmento da primeira versão de \textit{Um retrato do
artista quando Jovem}.  Na ocasião da venda, a livreira registra a
seguinte observação no catálogo: 

\begin{quote}
Quando o manuscrito foi devolvido,
após ter sido rejeitado pelo vigésimo editor, o autor atirou"-o ao fogo,
e Mrs.~Joyce ao resgatar essas páginas quase queimou as mãos.\footnote{
Citado em Spencer, 7--8. A tradução das citações originalmente em
língua inglesa é responsabilidade do autor do presente ensaio.} 
\end{quote}

Curiosamente, cabe registrar o fato de que nenhuma das páginas que
restaram do manuscrito apresenta sinais de fogo.\footnote{ Nicholas Fargnoli e Michael Gillespie 
esclarecem que, na realidade, o que Joyce, indignado com os problemas enfrentados na publicação 
de \textit{Dublinenses}, lançou ao fogo foi o manuscrito de \textit{Um retrato do artista quando jovem}, 
e as páginas não foram salvas por Nora, mas por Eileen, irmã de Joyce.}  Em todo caso, o
fragmento do manuscrito, cujo teor refere"-se apenas ao período
compreendido pelo último capítulo (o quinto) do \textit{Retrato}, é
adquirido em Paris, em 1938, pela biblioteca da Universidade de
Harvard.

Segundo a secretária de Joyce, o manuscrito em sua forma integral teria
cerca de mil páginas.\footnote{ Spencer, 8.} Embora dois terços das páginas
tenham desaparecido, o fragmento de 383 páginas, quase tão extenso
quanto o \textit{Retrato} em sua versão final, é editado por Theodore
Spencer sob o título de \textit{Stephen Hero} em 1944.  
Seis anos depois (1950), em Trieste, John J.~Slocum adquire
25 páginas adicionais do mesmo manuscrito, atualmente
arquivadas na Coleção Slocum, da biblioteca da Universidade de Yale,
páginas que, como resultado de uma colaboração com Herbert Cahoon, são
acrescentadas ao texto de Spencer.  Posteriormente, acirrando a
rivalidade entre instituições de ensino superior nos \textsc{eua}, outras cinco
páginas surgem no acervo da biblioteca da Universidade de Cornell.  As
páginas adicionais do manuscrito relatam a visita de Stephen Dedalus à
casa do padrinho, Mr.~Fulham, nos arredores de Mullingar, em Westmeath,
pouco tempo depois que o jovem iniciou os estudos na universidade, bem
como o encontro com a enigmática Miss Howard.\footnote{ O texto"-base
utilizado para a presente tradução de \textit{Stephen Hero} inclui a
totalidade das páginas adicionais (ver Referências).}


\subsection*{Características de «Stephen Herói»}

A despeito das páginas perdidas, o fragmento apresenta certa unidade,
além de reconhecida riqueza de detalhes.  Em \textit{Stephen Herói}
Joyce narra, topicalizando a figura do “herói”, Stephen Dedalus, seu
próprio conflito com a Igreja e com a família, sua atitude crítica
perante o nacionalismo irlandês, sua incursão na sexualidade e sua
defesa da individualidade e da arte.  O fragmento
retrata personagens e incidentes que nem sempre constam da versão final
e descreve o desenvolvimento da mente de Stephen de maneira bem mais
direta e menos elíptica do que ocorre no \textit{Retrato.}  Aquilo
que é sugerido obliquamente no \textit{Retrato} é explicitado em
\textit{Stephen Herói}, seja em relação às experiências narradas, às
ideias propostas, ou à caracterização dos personagens.

Vejamos uma ilustração do relato das experiências.  Enquanto em
\textit{Stephen Herói} a precária situação econômica da família, em
franca perda de status social, é corroborada pela presença “física” de
uma série de caminhões de mudança e ameaças de despejo, no		
\textit{Retrato} a mesma experiência é sintetizada num único episódio,
no qual Stephen tão"-somente chega em casa e ouve dos irmãos a notícia
de que a família busca novo domicílio.

As ideias são igualmente mais explicitadas.  A famosa teoria
estética de Stephen, por exemplo, aparece mais esmiuçada em
\textit{Stephen Herói}.  No \textit{Retrato}, a teoria é dogmaticamente
exposta a Lynch, e Stephen parece estar tão convicto de suas ideias que
pouco lhe importa o consenso em torno delas.  Um tanto altivo,
ele se posiciona acima de qualquer discordância.  Em \textit{Stephen
Herói}, no entanto, a teoria é algo sumamente importante e, em vez de
apresentá"-la informalmente a um amigo, ele a postula na forma de um
ensaio cuidadosamente pensado e redigido.  O trabalho é apresentado
diante de uma Associação de Debates, onde as ideias suscitam ferrenha
discussão, e Stephen “[é] submetido ao fogo de seis ou sete opositores
hostis” (\pageref{refintro}).

E a explicitação de ideias é deveras conspícua.  No capítulo \textsc{xxv}, na
discussão sobre os atributos do belo segundo Tomás de Aquino, Joyce
deixa clara sua importante definição de epifania, conceito crucial para
a construção e o entendimento dos quinze contos reunidos sob o título
\textit{Dublinenses} (1914): 

\begin{quote}
Para [Stephen] epifania significava uma
súbita manifestação espiritual, fosse na vulgaridade de uma fala ou de
um gesto, ou na memória da própria mente (\pageref{refintro2}). 
\end{quote}
 
E mais, invocando
Stephen, Joyce ressalta a responsabilidade do literato diante da
epifania, esse momento aparentemente trivial, mas que na verdade
torna"-se decisivo na vida de tantos personagens: 

\begin{quote}
Ele acreditava que cabia ao homem de letras registrar essas epifanias com grande cuidado,
percebendo que nelas se encerram os momentos mais delicados e furtivos
(\pageref{refintro3}).
\end{quote}

E os personagens, conforme caracterizados na primeira versão do
\textit{Retrato}, aparecem mais destacados, por exemplo Mrs.~Dedalus,
Emma Clery, Maurice, os amigos e, sobretudo, o próprio Stephen.  A mãe
de Stephen, que em \textit{Stephen Herói} promove grande tensão entre o
amor do filho e a descrença deste pela fé católica, no \textit{Retrato}
não é objeto de dramatização, mas apenas de uma reflexão do
protagonista durante uma de suas longas caminhadas ao lado do amigo
Cranly.  Além disso, em \textit{Stephen Herói}, a relação entre Stephen
e a mãe suscita uma cena vibrante, escatológica, na ocasião da morte de
Isabel, irmã de Stephen, quando, nas palavras de Harry Levin, “a
devoção ortodoxa da mãe se curva aos mistérios do corpo”.  Quanto		
a Maurice, irmão de Stephen, figura paralela a Stanislaus, conforme
vimos, irmão de Joyce, cuja figura é bastante presente e marcante em
\textit{Stephen Herói}, é depurada ao ponto de ser suprimida no
\textit{Retrato.} 

No \textit{Retrato}, os amigos --- Cranly, Madden, Lynch, McCann e
outros --- são apresentados como seres que, por assim dizer, habitam a
mente de Stephen.  Não são verdadeiramente construídos.  Em
\textit{Stephen Herói}, no entanto, os amigos são claramente
identificados.  Joyce nos descreve a aparência e as ideias dos rapazes,
que possuem realidades independentes, a exemplo dos personagens de
\textit{Dublinenses.}  Aqui, ao contrário do que se constata no
\textit{Retrato}, os amigos não são meros ouvintes das ideias
de um Stephen sabichão.

O detalhamento na construção dos personagens fica também patente no
caso da jovem pela qual Stephen se sente fisicamente atraído.  No	
\textit{Retrato}, a heroína tem existência rarefeita, aparecendo em
alusões sutis e apenas nas iniciais E--- C---.  Em
\textit{Stephen Herói}, contudo, ela é Emma Clery, jovem exuberante em
cuja companhia Stephen assiste a uma aula de língua irlandesa e cujo
relacionamento atinge um clímax inesperado, quando ele a vê no parque
e, às pressas, deixa a aula particular de literatura italiana, aborda a
jovem e propõe, explicitamente, uma noite de amor seguida de um simples
adeus de manhã.

Contudo, as diferenças mais notáveis entre a primeira e a última
versão do \textit{Retrato} dizem respeito à caracterização do próprio
Stephen.  O fragmento descreve cerca de dois anos da vida de Stephen,
iniciando, conforme já foi dito, logo após o ingresso do jovem na
universidade e concluindo no momento da emancipação de Stephen diante
de tudo o que a universidade representa.  Temos aqui, nas palavras de
Spencer:

\begin{quote}
um retrato vívido e coerente do ‘jovem’ chamado Stephen
Dedalus, que, em aparência, atitudes e ideias é, evidentemente, James
Joyce.  
\end{quote}
 
E o ânimo autobiográfico verificável no manuscrito é tão		
contundente que, no capítulo \textsc{xxiv}, ao menos em dois trechos de um mesmo
parágrafo, o narrador “escorrega” da terceira para a primeira pessoa. 
Contudo, se, de um lado, em \textit{Stephen Herói} esse protagonista é
emocionalmente menos amadurecido do que no \textit{Retrato} (e.g., as
citadas perseguição e proposta à Emma), de outro, ele é também menos
arrogante e presunçoso.  


\subsection*{Transição para o «Retrato»}

Em 8 de setembro de 1907, Joyce informa Stanislaus que, assim que
concluir a redação de “Os mortos”, último conto de \textit{Dublinenses},\footnote{ Cf. James Joyce. 
\textit{Dublinenses}. Tradução e apresentação de José Roberto O'Shea. São Paulo: Hedra, 2012.} 
procederá a reescrever inteiramente
\textit{Stephen Herói.}  Stanislaus registra em seu diário que Joyce
lhe dissera que omitiria os capítulos iniciais e começaria com Stephen,
a quem chamaria de Daly, frequentando a universidade, e que, na versão
revista, o livro teria cinco longos capítulos.  Seguramente, ao enxugar
o extenso arcabouço episódico de \textit{Stephen Herói} em cinco
capítulos, Joyce chega à estrutura de \textit{Um retrato do artista
quando jovem}, configurada por um conjunto de cenas que se irradiam no
passado e no presente.\footnote{ Cf. Ellmann, op.cit., 274 e 307.}

Percebe"-se que, ao reescrever a primeira versão, Joyce busca não
apenas parcimônia no registro de detalhes, mas manter a atenção centralizada
tanto quanto possível na consciência do herói.  O resultado é que o
\textit{Retrato}, por comparação, é menos dramático, mas é também mais
intenso, mais concentrado, mais focado, do que a versão anterior. 
Segundo Harry Levin, num argumento com o qual concordo plenamente, ao
reescrever o livro, Joyce transpõe “a ação, da esfera social, para a
esfera psicológica”, ou seja, temos em \textit{Stephen Herói} mais	
realismo social e menos realismo psicológico.  Depois de deixar a
Irlanda, e relembrando os “conflitos com a ortodoxia” na relativa
serenidade do exílio, Joyce parece concluir que os verdadeiros embates
teriam ocorrido na mente de Stephen.


\subsection*{Sentimento de Joyce em relação ao livro}

Novamente, segundo a secretária de Joyce, o manuscrito era considerado
pelo autor uma obra “colegial”, composta quando este tinha dezenove ou
vinte anos.\footnote{ Spencer, 8.}  Já em fins de 1908, Joyce declara a Stanislaus		
que o livro jamais será publicado: 

\begin{quote}
O que escrevo com as intenções mais
lúgubres provavelmente seria acusado de pornográfico na Inglaterra.\footnote{			
Citado em Ellmann, 274.} 
\end{quote}

Joyce pensa em mudar o título de
\textit{Stephen Herói} para \textit{Um retrato do artista}, ou
\textit{Capítulos da vida de um jovem}, talvez, por achar que o
primeiro título pudesse sugerir uma visão mais sardônica do herói do
que ele pretendia.\footnote{ Ibid., 200.}					

Seja lá como for, o leitor decerto perceberá que, mais do que mera base
para a versão final --- o célebre \textit{Um retrato do artista quando
jovem} --- \textit{Stephen Herói}, que vem a ser o \textit{Ur"-Retrato},
propicia aos admiradores de James Joyce uma oportunidade única de
perceber detalhes singulares e cruciais do retrato de Stephen Dedalus		
e do contexto que o cerca, detalhes que são, por assim dizer,
expurgados pelo próprio Joyce na versão final do “seu” retrato.  É fato
que temos, em \textit{Stephen Herói}, uma das descrições mais ricamente
detalhadas do desenvolvimento de uma mente humana.


\subsection*{Sobre a edição}

O texto"-base utilizado para esta tradução foi \textit{Stephen Hero}, A New
Edition, A New Directions Paperbook (New York, 1963).  Essa nova edição traz o
texto conforme fixado por Theodore Spencer, a partir do manuscrito que consta
do acervo da biblioteca da Universidade de Harvard, e inclui novas páginas do
manuscrito joyceano, que constam dos acervos das bibliotecas das universidades
de Yale e Cornell, material editado por John J.~Slocum e Herbert Cahoon.  O
manuscrito, conforme deixado por Joyce, apresenta dois tipos de correções.
Primeiramente, algumas palavras são suprimidas, outras substituídas.  Nesta tradução, 
para não interrompermos o fluxo de leitura, essas correções aparecem indicadas em notas 
com numeração arábica, reunidas no fim do livro, à página \pageref{notas"-arabicas}.  
O segundo tipo de correção reflete prática adotada pelo próprio Joyce, ao marcar, sublinhar ou riscar o
manuscrito, com lápis de cera vermelho ou azul.  Presume"-se que o escritor
desaprovasse esses trechos e desejasse alterá"-los, ou mesmo suprimi"-los.
Essas passagens foram coligidas, nesta edição, em uma seção ao fim do livro, à página \pageref{passagens} 
com indicação da página onde aparecem, seguida do trecho marcado por Joyce. Com esse 
procedimento, pretendemos garantir o fluxo de leitura sem abrir mão de oferecer aos 
leitores, leigos ou não, interessados na crítica genética do manuscrito, um valioso 
aparato de estudo. Dito isso, resta"-nos lembrar o leitor de que a tradução procurou preservar
características marcantes da pontuação joyceana, a saber, a parcimônia no
emprego da vírgula, o uso do travessão (além da marcação de diálogos) para fins
de ênfase e o uso idiossincrático dos dois"-pontos, por vezes, repetidos dentro
de uma mesma oração.


\section{Sobre o gênero}

Este \textit{Stephen Herói}, assim como \textit{Um retrato do artista
quando jovem}, pode ser classificado no gênero do romance de formação, tradução
de \textit{Bildungsroman}, principale tradição narrativa da língua alemã.
Seu surgimento remonta ao século \textsc{xviii}, momento em que a razão iluminista
rompia com a mentalidade fatalista da Igreja católica.
Alguns autores desse período histórico são Goethe, Voltaire, Rousseau, Locke, Diderot e Montesquieu que, apesar de suas diferenças ideológicas e estilísticas, tinham em comum
uma visão mais cética, empírica e materialista, valorizando a razão em detrimento do obscurantismo da Igreja e dos desmandos dos reis absolutistas.

Nascendo nesse contexto de intensas transformações sociais, políticas, econômicas e religiosas, e de renovação da produção cultural, o romance de formação, invertendo a importância conferida à transcendência na clássica metafísica cristã, foca"-se no homem, coloca"-o como centro autônomo de seu próprio mundo. Não por acaso, é também o momento de consolidação da burguesia, que acompanha um movimento de individualização do sujeito e crença no progresso. Na análise do escritor e crítico literário Caio Gagliardi, a juventude é o símbolo do romance de formação por excelência: representa o anseio pela modernidade, personifica a instabilidade da sociedade capitalista industrial e do próprio indivíduo, assim como de seu meio social, com o dinanismo das classes e dos valores.

Segundo Gagliardi, ``a característica predominante do
\textit{Bildungsroman} é o tema de que
trata: o desenvolvimento interior de um protagonista criança ou jovem
mediante suas dificuldades de interação com os demais e o meio em que vive''.\footnote{\textsc{gagliardi}, Caio. ``Introdução''. In: \textsc{pompeia}, Raul. \textit{O Ateneu}. São Paulo: Hedra, 2020, p.\,27.} Segundo o crítico, essa característica se evidencia ao se constatar que antes do século \textsc{xix} nunca houve, na literatura, um enfoque tão grande no protagonista em formação.

O enfoque no desdobramento interior de um personagem, no entanto, não faz do romance de formação um gênero subjetivo ou introspecto, alerta Gagliardi. 
Para o escritor, o \textit{Bildungsroman} está atrelado à realidade histórica, que tem um papel fundamental na transformação do indivíduo. ``Se o autoconhecimento define"-se
pela interação do indivíduo com a sociedade, o caráter subjetivo do
texto não apaga sua outra dimensão, objetiva. Realidade íntima e
histórica devem manter"-se em contato. Daí uma narrativa difícil de
definir: que nem é idealista, [\ldots] nem realista [\ldots]. Por outro ângulo, trata"-se de uma narrativa que se aproxima da \textit{autobiografia} (na focalização tanto externa 
quanto interna do protagonista, isto é, no desenrolar das experiências vividas e incorporadas)''.\footnote{Ibidem, p.\,31.}.

Na conjunção desses aspectos, subjetivos, históricos e autobiográficos, Gagliardi localiza o texto de James Joyce:

\begin{quote}
Em termos estilísticos, \textit{O retrato do
artista quando jovem} (1916), de James Joyce, é uma
narrativa nitidamente moderna, isto é, autoconsciente de seus
procedimentos de escrita, e revela a evolução do
\textit{Bildungsroman}. Também considerado o
romance autobiográfico de Joyce, narra episódios traumáticos referentes
à sua infância, tal como o castigo de palmatória, que o jovem
protagonista sofre de um bedel inábil, desconfiado de que o menino (que
havia perdido os óculos) não fazia sua tarefa em sala por mera
preguiça. O ensino jesuíta é o responsável pela impressionante cultura
clássica, pelo conhecimento de línguas, mas também pelos fantasmas de
Joyce. No romance, Stephen Dedalus é um autor"-menino que procura
libertar"-se da educação recebida e dos dilemas da pátria para
encontrar seu próprio caminho como
escritor.
\end{quote}

\begin{bibliohedra}

\tit{ELLMANN}, Richard.  \textit{James Joyce.}  Oxford: Oxford University
Press, 1959.

\tit{Fargnoli}, A.~Nicholas \& \textsc{gillespie}, Michael Patrick. 
\textit{Critical Companion to James Joyce: A Literary Reference to His Life and Work}.
New York: Facts on File, 2006.

\tit{LEVIN}, Harry.  \textit{James Joyce.}  Revised, augmented edition.  New
York: James Laughlin, 1960.

\tit{SPENCER}, Theodore.  Introduction.  \textit{Stephen Hero: James Joyce}. 
A New Edition, incorporating the Additional Manuscript Pages in the Yale
University Library and the Cornell University Library, eds. John J.
Slocum and Herbert Cahoon.  New Directions Series.  New York: James
Laughlin, 1963.

\end{bibliohedra}