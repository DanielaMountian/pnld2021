
\chapter[1 -- A história da porta]{1\break A história da porta}

Mr. Utterson, o advogado, tinha uma fisionomia austera, nunca suavizada
por um sorriso; falava de maneira fria, lacônica e pouco à vontade; era
retraído em suas emoções; em suma, um homem magro, alto, seco, sombrio
e ainda assim merecedor da afeição alheia.  Nas ocasiões sociais, e
quando o vinho era do seu agrado, uma luz essencialmente humana emanava
do seu olhar; algo que nunca encontrava vazão em sua conversa, mas se
expressava não apenas nos símbolos silenciosos de um rosto após um
jantar satisfatório, mas, de maneira mais vigorosa e eloquente, nos
atos concretos de sua existência.  Era um homem severo consigo mesmo,
que bebia gin quando estava a sós em casa, para mortificar sua
preferência pelos vinhos de boa safra; e embora apreciasse o teatro,
não cruzava as portas de uma sala de espetáculos há vinte anos.  Era um
homem tolerante com os demais; às vezes punha-se a cismar, quase com
inveja, sobre as fortes pressões espirituais que impeliam um homem a
praticar um mau passo; e, de um modo geral, tinha uma tendência maior
para ajudar do que para reprovar seus semelhantes.

“Pratico a heresia de Caim”, costumava dizer, à sua maneira antiquada;
“deixo que meu irmão vá para o inferno do modo que ele achar melhor”. 
Graças a essa atitude, cabia-lhe com frequência o destino de ser a
última amizade respeitável ou a última boa influência na vida de homens
em processo de decadência moral.  E quando algum deles se aproximava de
seu escritório, sua expressão e sua atitude não sofriam nenhuma
alteração perceptível.

Não há dúvida de que isto não custava muito a Mr. Utterson, uma vez que
ele aparentava ser um homem tão reservado quanto lhe era possível, e
suas manifestações de amizade pareciam estar baseadas na tolerância
natural de seu caráter.  É típico dos indivíduos íntegros aceitar o
círculo de amizades que as circunstâncias lhe oferecem; e assim
procedia o advogado.  Seus amigos mais próximos eram ligados a ele por
laços consanguíneos ou por uma longa convivência; sua afeição, como a
hera, brotava como um fruto natural do tempo, sem requerer dele nenhuma
predisposição especial para tanto.  Disso advinha, sem dúvida, a
amizade que o unia ao seu parente distante, Mr. Richard Enfield,
cavalheiro bem conhecido nos círculos sociais.  Muitos não conseguiam
perceber que tipo de identificação ou de assunto em comum esses dois
homens tão diferentes pudessem ter.  Os amigos que os encontravam às
vezes durante suas caminhadas dominicais reportavam que os dois
mantinham-se em silêncio, tinham uma aparência entediada e saudavam com
evidente alívio o aparecimento de um conhecido.  Apesar disto, os dois
davam grande importância a essas excursões e as tinham como um momento
privilegiado de sua semana, chegando a abrir mão, por elas, de
atividades mais prazerosas ou mesmo de compromissos profissionais.

Num desses passeios ocorreu por acaso que cruzassem uma rua secundária
num dos bairros mais agitados de Londres.  A rua era pequena e poderia
ser considerada tranquila, ainda que nos dias de semana o comércio por
ali fosse intenso.  Os comerciantes pareciam estar atravessando um bom
momento, e competiam esperançosos entre si, reinvestindo seus lucros
para dar a suas lojas uma aparência sedutora, de sorte que as fachadas
que se alinhavam ao longo das calçadas tinham todas um ar convidativo,
como fileiras de vendedoras sorridentes.  Mesmo aos domingos, quando os
seus encantos eram menos visíveis e a via pública ficava quase vazia, a
rua ainda reluzia em contraste com os quarteirões mal cuidados da
vizinhança, como uma fogueira flamejando em meio à floresta; suas
venezianas recém-pintadas, seus reluzentes ornamentos de latão e sua
aparência geral de limpeza atraíam de modo agradável os olhos dos
transeuntes.

Duas casas depois de uma esquina, no lado esquerdo de quem seguia na
direção leste, a série de fachadas era interrompida por um recuo que
dava acesso a um pátio, ao fundo do qual uma construção sinistra erguia
sua fachada na direção da rua.  Tinha dois andares de altura, não tinha
janelas, e exibia apenas uma porta no térreo, encimada por uma parede
descolorida na parte superior; tudo nela dava indícios de uma
prolongada falta de cuidados e de manutenção.  A porta, que não
dispunha de sineta ou de aldraba, estava descascada e cheia de nódoas. 
Vagabundos costumavam se recolher naquele espaço e riscar fósforos nas
paredes; crianças vendiam miudezas nos degraus; colegiais tinham
descascado a canivete os ornamentos; e durante o período de toda uma
geração ninguém aparecera para afugentar esses visitantes casuais ou
para consertar os danos.

Mr. Enfield e o advogado vinham caminhando pela calçada oposta quando,
ao passar diante daquela entrada, o primeiro ergueu a bengala e apontou
naquela direção.

-- Já reparou naquela porta? -- perguntou ele, e quando seu companheiro
respondeu afirmativamente, prosseguiu: -- Na minha memória ela está
relacionada a um episódio bastante estranho.

-- De fato? -- perguntou Mr. Utterson, com uma leve mudança de voz. -- E
o que foi? 

-- Bem, aconteceu assim -- disse Mr. Enfield. -- Eu estava voltando para
casa, depois de um compromisso num fim-de-mundo qualquer.  Eram três
horas de uma madrugada escura de inverno, e cruzei um trecho da cidade
em que não conseguia literalmente ver nada além dos lampiões da rua. 
Ruas e mais ruas se sucediam, e o mundo inteiro parecia estar
dormindo\ldots{} Ruas e mais ruas, todas iluminadas como se esperassem uma
procissão, e todas vazias como uma igreja, até que mergulhei naquele
estado de espírito em que um homem apura o ouvido cada vez mais,
ansioso para avistar um policial.  E de repente avistei dois vultos. 
Um deles era um homem de estatura baixa, que caminhava pesadamente para
oeste, com o passo apressado, e o outro uma menina de seus oito ou dez
anos que corria o mais depressa que podia, descendo uma rua lateral. 
Pois muito bem, meu amigo: os dois convergiram na direção um do outro,
naturalmente, e se esbarraram bem na esquina.  Então ocorreu a parte
chocante da história, porque o homem pisoteou calmamente o corpo da
garota, que havia caído no chão, e afastou-se deixando-a ali na
calçada, aos gritos.  Dito assim não parece grande coisa, mas foi
terrível de se ver.  Ele não parecia um homem; lembrou-me um Juggernaut.\footnote{
Carruagem usada em desfiles religiosos hindus, que às vezes atropelava,
sem se deter, a multidão de fiéis. [\textsc{n.t.}]} Dei uns gritos, corri na sua direção e o agarrei
pela gola do casaco, arrastando-o de volta ao local onde um grupo de
pessoas já havia se aglomerado em torno da menina, que continuava
gritando.  O homem manteve-se perfeitamente calmo e não ofereceu
resistência, mas me lançou um olhar tão maligno que me fez ficar
coberto de suor.  As pessoas que estavam em volta eram parentes da garota, e
logo em seguida apareceu ali um médico, que ela mesma havia sido
encarregada antes de ir chamar com urgência.  Bem, a criança não estava
gravemente ferida, apesar de aterrorizada, de acordo com o doutor, e
tudo indicava que o caso se encerraria ali mesmo.  Mas havia um aspecto
curioso.  Eu tinha sentido uma vívida repugnância por aquele homem,
assim que pus os olhos nele.  O mesmo acontecera com os parentes da
menina, o que é muito natural.  Mas o que me chamou a atenção foi a
reação do médico.  Ele era um tipo comum de farmacêutico, sem nada de
peculiar quanto à idade ou condição social, um sujeito com forte
sotaque de Edimburgo e tão emotivo quanto uma gaita de foles.  Pois
bem, ele reagiu do mesmo modo que nós todos; sempre que encarava o meu
prisioneiro eu o via ficar pálido, com desejos de matá-lo.  Eu sabia o
que ele estava pensando, e ele percebia o mesmo quanto a mim; e como
matar o homem estava fora de cogitação, fizemos o melhor que podíamos. 
Garantimos que iríamos criar um escândalo a respeito daquele fato até
fazer com que o nome dele fosse objeto de repulsa em toda Londres,
afastando de sua convivência qualquer amigo que porventura tivesse.  E
durante todo aquele tempo em que o ameaçávamos da forma mais veemente
tínhamos que manter as mulheres afastadas dele, porque estavam mais
selvagens do que harpias.  Nunca vi um círculo de rostos mais tomados
pelo ódio, e ali estava o homem bem no centro, ostentando um ar frio e
desdenhoso; eu podia ver que ele estava com medo também, mas a verdade
é que enfrentava a situação como se fosse o demônio em pessoa.  “Se os
senhores preferem capitalizar em cima deste acidente”, disse ele, “nada
posso fazer.  A um cavalheiro tudo que pode interessar é evitar uma
cena.  Digam que soma pode servir de indenização.”  Bem, conseguimos
arrancar-lhe cem libras para a família da garota.  Era bastante
evidente que ele preferiria safar-se ileso, mas havia uma tal ameaça em
nossas fisionomias que por fim ele aceitou o trato.  Restava-nos apenas
receber o dinheiro; e onde acha que ele nos conduziu, senão àquela
porta?!  Puxou do bolso uma chave, entrou, e retornou em seguida com
dez libras em ouro e um cheque do restante a ser sacado pelo portador
no banco Coutt’s, assinado por um nome que não posso revelar, embora
seja este um dos detalhes mais importantes da minha história; em todo
caso, um nome bastante conhecido, e que volta e meia é mencionado na
imprensa.  A soma era respeitável, mas a assinatura, se genuína,
poderia garantir bem mais do que aquilo.  Tomei a liberdade de dizer
àquele indivíduo que o documento me parecia de autenticidade duvidosa,
porque, afinal de contas, na vida real um homem não entra por uma porta
dos fundos às quatro da madrugada e sai dali com um cheque de quase cem
libras assinado por outro homem.  Mas o sujeito permaneceu
imperturbável, e irônico.  “Pode ficar tranquilo,” disse ele, “porque
ficarei em sua companhia até que o banco abra, e eu mesmo descontarei o
cheque”.  E assim nos pusemos a caminho, eu, o doutor, o estranho e o
pai da menina, viemos para os meus aposentos e passamos ali o resto da
noite; e no dia seguinte, depois que tomamos o café da manhã, fomos
juntos até o banco.  Eu próprio apresentei o cheque, avisando que tinha
todos os motivos para julgar que se tratasse de uma falsificação.  Mas
nada disso.  O cheque era genuíno.

-- Ora, ora -- disse Mr. Utterson. 

-- Vejo que está pensando o mesmo que eu -- disse Mr. Enfield. -- Sim, é
uma história escabrosa.  Porque o indivíduo em questão não seria boa
companhia para ninguém, um tipo verdadeiramente abominável; e a pessoa
que emitiu o cheque era alguém da nata de nossa sociedade, um homem
famoso, e (o que torna a situação ainda pior) um cidadão cujas boas
ações são notórias.  Suponho que se tratasse de alguma forma de
chantagem; um homem honesto sendo forçado a pagar, contra a vontade,
por alguma ação irrefletida que praticou no passado.  Passei a pensar
naquela porta desde então como a Casa da Chantagem.  Embora mesmo isso
esteja longe de explicar tudo que se passou, como irá concordar -- 
concluiu ele, e em seguida entregou-se a uma silenciosa reflexão.

Foi arrancado dos seus pensamentos pela pergunta súbita de Mr. Utterson:

-- E o senhor não sabe se o signatário do cheque mora aí?

-- Seria o mais provável, não é mesmo? -- disse Mr. Enfield. -- Mas já vi
seu endereço; ele mora diante de uma praça. 

-- E o senhor nunca fez nenhuma investigação com relação à casa onde fica
essa porta?

-- Não, senhor.  Minha discrição foi mais forte -- respondeu ele. --
Tenho princípios que vão de encontro a esse tipo de bisbilhotice; é
algo que se aproxima excessivamente ao espírito do Juízo Final. 
Fazemos uma pergunta, e é o mesmo que revirar uma pedra; ficamos no
topo da colina e a pedra começa a rolar, chocando-se com outras; e mais
cedo ou mais tarde algum sujeito despreocupado, a última pessoa na
terra em quem se poderia pensar, acaba sendo atingido na cabeça ao
cruzar o seu jardim e a família precisa mudar de nome.  Não,
cavalheiro, eu me governo por esta regra: quanto mais uma coisa se
parece com Queer Street, menos perguntas eu faço a seu respeito.\footnote{
Queer Street: expressão londrina para uma situação de inadimplência
financeira ou de falência. [\textsc{n.t.}]}

-- Uma boa regra, essa -- disse o advogado. 

-- Mas investiguei o local por conta própria -- continuou Mr. Enfield. --
Não se assemelha em nada a uma residência.  Não há outra entrada, e
ninguém mais cruza aquela porta que vimos, a não ser, de tempos em
tempos, o cavalheiro que foi personagem da minha pequena aventura.  Há
três janelas que dão para o pátio no andar superior, mas nenhuma no
térreo; as janelas estão sempre fechadas mas têm as vidraças limpas. 
Há também uma chaminé que em geral está produzindo fumaça, portanto
alguém deve viver ali.  E ainda assim não há como ter certeza, porque
os prédios naquele trecho estão tão amontoados uns sobre os outros que
não há como dizer onde terminam ou começam.

A dupla continuou caminhando em silêncio, e então Mr. Utterson disse:

-- Enfield, a regra que mencionou é uma regra notável.

-- Sim, acho que é -- disse Mr. Enfield.

-- Ainda assim -- continuou o advogado -- há um detalhe que me interessa:
o nome do homem que pisoteou a criança.

-- Bem, -- disse Mr. Enfield -- não vejo nada de mau em dizê-lo.  Era um
homem chamado Hyde.

-- Hmmm -- disse Mr. Utterson. -- Que aparência tem ele?

-- Não é fácil descrevê-lo.  Há algo de errado com sua aparência;
alguma coisa incômoda, alguma coisa profundamente detestável.  Nunca vi
um homem que me desagradasse tanto, e no entanto não sei dizer o
porquê.  Ele deve sofrer algum tipo de deformação; a impressão que nos
dá é de algo disforme, embora eu não consiga dizer especificamente em
que aspecto.  É um homem de aparência extraordinária, mas de fato eu
não posso apontar nada que seja fora do comum.  Não, cavalheiro,
descrevê-lo está além de minha capacidade.  E não é por esquecimento,
porque afirmo que neste instante consigo vê-lo com toda clareza.

Mr. Utterson continuou a caminhar em silêncio, visivelmente mergulhado
em profundas reflexões.

-- Tem certeza de que ele usou uma chave? -- perguntou.

-- Meu caro senhor\ldots{} -- começou a dizer Enfield, tomado pela surpresa.

-- Sim, eu sei -- disse Utterson. -- Sei que parece estranho.  O fato é
que, se não lhe pergunto o nome do outro cavalheiro envolvido nesse
episódio, é porque já sei de quem se trata.  Sabe, Richard, sua
história veio ao encontro de algo que eu já sabia.  Se tiver se
equivocado em algum ponto, é melhor corrigir-se agora.

-- Acho que deveria ter me prevenido -- disse o outro, com ar soturno. --
Mas eu fui pedantemente preciso, como você diria.  O homem tinha, sim,
uma chave; e o que é mais, ainda a possui.  Não faz uma semana que o
avistei usando-a novamente. 

Mr. Utterson soltou um suspiro profundo mas não disse uma palavra; e o
homem mais jovem prosseguiu.

-- Eis uma boa lição para que guardemos segredo -- disse ele. -- Estou		%Eis uma boa lição para não revelarmos segredos.
envergonhado da minha tagarelice.  Vamos fazer um trato de nunca mais
tocarmos neste assunto.

-- De todo o coração -- concordou o advogado. -- Aperte minha mão,
Richard, e estamos combinados. 


\chapter[2 -- Em busca de Mr. Hyde]{2\break Em busca de Mr. Hyde}

Naquela mesma noite, Mr. Utterson retornou a sua casa de solteirão num
estado de espírito taciturno, e sentou-se para jantar sem demonstrar
contentamento.  Era seu costume, aos domingos, ao terminar a refeição,
sentar-se junto à lareira, com um volume de severos textos religiosos
sobre a mesa de leitura, até que o relógio da igreja vizinha desse
meia-noite, quando então ele se recolhia, sóbrio e satisfeito.  Naquela
noite, no entanto, assim que a mesa foi retirada ele empunhou uma vela
e dirigiu-se ao escritório.  Ali abriu o cofre, retirou de um dos seus
escaninhos mais protegidos um envelope em cuja face exterior estava
escrito “Testamento do Dr. Jekyll” e sentou-se de cenho franzido para
examinar seu conteúdo.  O documento era redigido de próprio punho,
porque Mr. Utterson, mesmo tendo-o aceito sob sua guarda, recusara-se a
participar de sua elaboração; e não apenas estatuía que, em caso do
falecimento de Henry Jekyll, \textsc{m.d.}, \textsc{d.c.l.}, \textsc{l.l.d.}, \textsc{f.r.s.} etc.,\footnote{
Doutor em Medicina, Doutor em Direito Civil, Doutor em Advocacia e
Membro da Sociedade Real.  No século \textsc{xix}, os dois primeiros títulos eram
obtidos na Universidade e os demais durante a carreira profissional. [\textsc{n.t.}]}
todos os seus bens deveriam passar às mãos do seu
“amigo e benfeitor Edward Hyde”, mas também que “no caso do seu
desaparecimento ou de sua ausência inexplicável por um período superior
a três meses”, o dito Edward Hyde deveria apossar-se de suas
propriedades sem demora, livre de qualquer compromisso ou obrigação a
não ser o pagamento de pequenas somas aos membros de sua criadagem. 
Este documento tinha sido por muitos anos uma pedra no sapato do
advogado.  Seu teor o ofendia tanto como advogado quanto como cidadão
amante dos aspectos sadios e ordeiros da vida, alguém para quem
qualquer extravagância era ofensiva.

Até aquele momento, sua indignação tinha sido alimentada pela sua
absoluta ignorância sobre quem pudesse ser o tal Mr. Hyde; e agora,
numa reviravolta súbita, isto lhe era revelado.  Tudo aquilo já tinha
um mau aspecto quando “Hyde” era apenas um nome do qual ele nada mais
sabia.  O que pensar agora, quando tal nome começava a ser revestido
dos atributos mais detestáveis?  Por entre a névoa insubstancial que
até aquele momento tinha desfocado sua visão, começou a se delinear a
súbita e nítida presença de um inimigo. 

“Pensei que era uma loucura”, disse ele consigo, voltando a guardar no
cofre o detestável documento, “mas agora temo que seja uma desgraça”.

Com isso, ele soprou a vela, vestiu um sobretudo e saiu na direção de
Cavendish Square, aquela cidadela da medicina, onde seu amigo, o grande
Dr. Lanyon, tinha sua casa e o consultório onde atendia seus numerosos
pacientes.  “Se alguém souber, tem que ser Lanyon”, pensou ele.

O empertigado mordomo o conhecia, e deu-lhe as boas-vindas; sem demora
Mr. Utterson foi conduzido diretamente à sala de jantar onde o Dr.
Lanyon estava sentado a sós, diante do seu vinho. Era um cavalheiro
jovial, saudável, elegante, de rosto avermelhado, com uma franja do
cabelo prematuramente branca, e modos expansivos e resolutos.  Ao
avistar Mr. Utterson ele se pôs de pé e foi apertar ambas as suas mãos.
 Aquela jovialidade, que lhe era espontânea, parecia algo teatral a um
observador externo; mas exprimia sentimentos genuínos.  Aqueles dois
eram velhos amigos, que tinham estudado juntos no colégio e na
universidade, e que se tratavam com o máximo respeito, além de, o que
nem sempre decorre da condição anterior, terem o maior prazer na
companhia um do outro. 

Depois de alguns minutos de conversa superficial, o advogado tocou no
assunto que lhe desassossegava a mente.

-- Suponho, Lanyon, -- disse ele -- que eu e você somos os amigos mais
velhos de Henry Jekyll, não é assim? 

-- Gostaria que fôssemos amigos mais jovens, -- disse o Dr. Lanyon com
uma risada -- mas é assim, de fato.  Por quê?  Tenho-o visto muito pouco
ultimamente.

-- É verdade? -- disse Utterson. -- Achei que vocês dois eram unidos por
muitos interesses comuns.

-- E realmente o fomos -- foi a resposta. -- Mas já se vão cerca de dez
anos que Henry Jekyll tornou-se esquisito demais para mim.  Começou a
ter atitudes erradas, ideias erradas; e embora eu continue a me
interessar por ele, em honra dos bons e velhos tempos, como se diz,
tenho-o visto muito pouco.  As suas baboseiras científicas -- prosseguiu
o doutor, com o rosto avermelhando-se ainda mais -- são algo capaz de
separar Damon e Pítias.\footnote{ Damon e Pítias são os exemplos de amigos
leais na mitologia grega. [\textsc{n.t.}]}

Esta leve explosão de temperamento acabou por trazer um pouco de alívio
a Mr. Utterson. “Eles apenas divergiram em algum aspecto científico”,
pensou ele; e sendo um homem em quem a ciência não despertava paixões
(exceto quanto à exatidão do linguajar jurídico) ele concluiu: “Não é
nada mais sério do que isso”.  Deu ao amigo alguns segundos para
recuperar a calma, e então fez a pergunta que era a razão de sua
visita.

-- Chegou a conhecer um protegido dele, um tal de Hyde?

-- Hyde? -- repetiu Lanyon. -- Não.  Nunca ouvi falar nele.  Desde o meu
tempo.

Isto foi toda a informação que o advogado trouxe de volta consigo até o
grande leito, mergulhado na escuridão, em que se virou e revirou de um
lado para o outro até que a manhã começou a raiar.  Foi uma noite de
pouco repouso para sua mente inquieta, tateando nas trevas e acossada
por indagações.

O sino tocou seis horas na igreja, tão convenientemente próxima da casa
de Mr. Utterson, e ele ainda não tinha chegado a uma conclusão.  De
início o problema tinha lhe inquietado apenas o lado intelectual; mas
agora sua imaginação também se deixara arrebatar, ou melhor,
escravizar; e enquanto ele se revolvia no leito na escuridão da noite e
do quarto de cortinas cerradas, a história de Mr. Enfield se repetia em
sua mente como um desfile de imagens luminosas.  Ele percebia o brilho
dos lampiões de uma grande cidade; depois a figura de um homem
caminhando apressado; em seguida uma criança correndo para chamar um
médico; e então os dois convergiam um sobre o outro, e aquele
Juggernaut humano atropelava e pisoteava a criança, sem ligar para os
seus gritos.  Ou então imaginava estar vendo o interior de uma casa
luxuosa, onde seu amigo jazia adormecido, sonhando, e sorrindo durante
o sonho; e então a porta do quarto era aberta, as cortinas se abriam, o
homem adormecido era despertado, e ali surgia a criatura que detinha
sobre ele algum poder; e ele era forçado a erguer-se e fazer o que lhe
era ordenado.  O vulto que surgia nestas duas fases assombrou a mente
do advogado durante a noite inteira; e se em algum momento ele chegava
a cochilar, era para ver aquele vulto deslizando de modo sorrateiro por
entre as ruas adormecidas, ou caminhando mais e mais depressa, até se
tornar um movimento indistinto, através dos labirintos mais amplos da
cidade iluminada pelos lampiões, e em cada esquina esmagar sob os pés
uma criança e abandoná-la aos gritos.  E ainda assim o vulto não tinha
um rosto reconhecível; mesmo nos sonhos aquele ser não tinha rosto
algum, ou tinha um rosto que o iludia, desfazendo-se diante do seu
olhar; e tanto foi assim que no espírito do advogado brotou uma
curiosidade estranhamente forte, quase sem controle, de avistar as
verdadeiras feições do Mr. Hyde real.  Se pudesse vê-lo pelo menos uma
vez, talvez o mistério fosse atenuado e quem sabe até se dissipasse por
completo, como ocorria com muitos mistérios quando submetidos a um
rigoroso escrutínio.  Ele veria então um motivo para a estranha
preferência ou servidão (dessem-lhe o nome que bem entendessem) de seu
amigo, e até mesmo para aquela surpreendente cláusula em seu
testamento.  Pelo menos seria um rosto digno de se ver: o rosto de um
homem destituído de misericórdia, um rosto cuja mera aparição era capaz
de despertar sentimentos de um ódio duradouro num espírito tão pouco
impressionável quanto o de Enfield.

Daquele momento em diante, Mr. Utterson começou a vigiar aquela porta na
ruazinha cheia de lojas.  De manhã, antes de ir para o escritório; ao
meio-dia, quando o movimento no local parecia mais intenso e o tempo
mais escasso; e à noite, sob o olhar da enevoada lua urbana, sob todos
os tipos de luz e em todas as horas de solidão ou de grande movimento,
o advogado podia ser visto no posto de observação que tinha escolhido.

“Se ele é Mr. Hyde, -- pensava ele -- “eu serei Mr. Seek”.\footnote{ Trocadilho
com o verbo “to hide” (esconder) e “to seek” (procurar), que formam o nome
inglês da popular brincadeira de esconde-esconde, “hide and seek”. [\textsc{n.t.}]}

Por fim sua paciência foi recompensada.  Era uma noite fresca e
agradável, com um pouco de geada no ar; as ruas estavam limpas como um
salão de baile; os lampiões, sem nenhum vento para agitá-los,
desenhavam um padrão regular de luzes e sombras.  Por volta das dez
horas, quando as lojas já tinham fechado, a rua estava com uma
aparência deserta e, a despeito do murmúrio confuso de toda a Londres à
sua volta, ficara silenciosa.  Os menores ruídos podiam ser escutados a
distância; os barulhos domésticos de algumas casas podiam ser
percebidos de ambos os lados da calçada;  e a aproximação de qualquer
transeunte era precedida de longe pelo som dos seus passos. Mr.
Utterson estava no seu posto há alguns minutos, quando teve consciência
de passos estranhamente leves que se aproximavam.  No decorrer daquelas
patrulhas noturnas ele tinha se acostumado ao modo peculiar como os
passos de uma pessoa, ainda a grande distância, se destacavam de súbito
por entre o vasto murmúrio e o vozerio de uma grande cidade.  E ainda
assim nunca sua atenção tinha sido atraída de maneira tão brusca e
definida; foi com uma profunda e supersticiosa premonição de sucesso
que ele recuou para as sombras do pequeno pátio.

Os passos se aproximaram, rápidos, e o ruído aumentou quando o pedestre
desembocou na rua.  O advogado, olhando às escondidas, logo percebeu o
tipo do homem que estava a ponto de encarar.  Era de estatura pequena,
bem vestido, e sua aparência, mesmo a certa distância, produziu uma
impressão de aversão no homem que o observava.  Ele se encaminhou
direto para a porta, atravessando a rua como quem não quer perder
tempo; e ao se aproximar enfiou a mão no bolso e tirou dele uma chave,
com o gesto de alguém que chega à porta de casa.  

Mr. Utterson deu um passo à frente e tocou no seu ombro:

-- Mr. Hyde, suponho?

Mr. Hyde encolheu-se, com um arquejo brusco de surpresa; mas seu susto
foi momentâneo e embora não encarasse de frente o advogado respondeu
com toda a calma:

-- Este é o meu nome.  O que deseja?

-- Vejo que vai entrar em casa -- disse o advogado. -- Sou um velho amigo
do Dr. Jekyll, sou Mr. Utterson, de Gaunt Street.  Talvez já tenha
ouvido meu nome.  Já que nos encontramos aqui, pensei que pudesse me
deixar entrar.

-- Não vai encontrar o Dr. Jekyll.  Ele não está em casa -- replicou Mr.
Hyde, assoprando a chave, com um gesto distraído. E de repente, mas
ainda de cabeça baixa, perguntou: -- De onde me conhece? 

-- Pode fazer-me um favor, de sua parte? 

-- Com prazer -- disse o outro.  

-- Posso ver o seu rosto? -- pediu o advogado.

Mr. Hyde pareceu hesitar, e então, como que cedendo a um pensamento
súbito, ergueu o rosto para o outro numa atitude de desafio; e os dois
se encararam fixamente por alguns segundos.

-- Agora sou capaz de reconhecê-lo -- disse Mr. Utterson. -- Pode vir a
ser útil. 

-- Sim -- retrucou Mr. Hyde. -- Foi bom que tenhamos nos encontrado, e, a
propósito, preciso dar-lhe meu endereço. -- E estendeu um cartão com o
endereço de uma rua no Soho.

“Bom Deus!”, pensou Mr. Utterson. “Será possível que ele também
estivesse a pensar no testamento?”  Mas guardou seus pensamentos para
si, e deu apenas um grunhido de agradecimento ao guardar o papel.

-- E agora, -- disse o outro -- como me conhece?

-- Por descrição -- foi a resposta.

-- Descrição feita por quem?

-- Temos alguns amigos em comum -- disse Mr. Utterson.

-- Amigos em comum -- ecoou Mr. Hyde, com voz um tanto rouca. -- E quem
são eles?

-- Jekyll, por exemplo -- disse o advogado.

-- Ele nunca lhe disse nada! -- exclamou Mr. Hyde, num acesso de fúria.
-- Não imaginei que o senhor fosse capaz de mentir.

-- Calma -- disse Mr. Utterson. -- Não está usando uma linguagem
adequada.

O outro soltou um misto de rosnado com uma gargalhada selvagem, e no
instante seguinte, com extraordinária agilidade, destrancou a porta e
sumiu no interior da casa.

O advogado deteve-se ali por mais algum tempo depois da entrada de Mr.
Hyde, e sua atitude era a imagem viva do desconforto.  Depois saiu
caminhando devagar pela rua, fazendo pausas constantes e levando a mão à
testa como um homem cheio de perplexidade mental.  O problema com que
se debatia enquanto caminhava era de uma natureza que raras vezes se
deixa solucionar.  Mr. Hyde era pálido e com aparência de anão; dava
uma impressão de deformidade sem que houvesse nele nenhuma má-formação
visível, tinha um sorriso desagradável, tinha sabido se impor ao
advogado com uma mistura ameaçadora de timidez e ousadia, e falava com
uma voz enrouquecida, sussurrante e meio alquebrada; todos estes pontos
se somavam a seu desfavor, mas nem mesmo todos eles juntos podiam
explicar a sensação de repulsa, nojo e medo experimentada por Mr.
Utterson.  “Deve haver alguma outra coisa”, pensava o perplexo
cavalheiro, “Existe algo mais, e gostaria de dar um nome a isso.  Deus
me perdoe, o indivíduo mal parecia humano!  Tinha algo de troglodita,
será?  Ou será como na antiga história do Dr. Fell?\footnote{
Alusão aos versos feitos por Tom Brown, aluno de Oxford, para o reitor
da universidade, John Fell (1625-1686), que o ameaçava de expulsão:
“Não gosto do sr., dr. Fell / e a razão não sei dizer. / Mas uma coisa
eu sei, e sei muito bem: / não gosto do sr., Dr. Fell”. [\textsc{n.t.}]} 
Ou ainda a simples irradiação de uma alma maligna, que transpira
através do barro que a hospeda, e o transfigura?  Penso que seja esta
última hipótese, porque, oh, meu pobre amigo Henry Jekyll, se alguma
vez eu enxerguei a assinatura de Satã sobre um rosto, foi o desse seu
novo amigo”.

Dando a volta à esquina daquela rua havia uma praça, rodeada de
residências belas e antigas, a maior parte delas já em processo de
decadência, repartidas em apartamentos e quartos para todos os tipos de
condições de inquilinos: gravadores de mapas, arquitetos, advogados
obscuros e agentes de empresas pouco recomendáveis.  Uma casa, contudo,
a segunda a partir da esquina, ainda demonstrava ser uma residência
particular, e foi à sua porta, que conservava uma aparência de
prosperidade e conforto, embora mergulhada agora na escuridão, a não
ser pela luz emanada através dos painéis de vidro que a encimavam, que
Mr. Utterson se deteve e bateu.  Um mordomo idoso e impecavelmente
vestido veio atender.

-- O Dr. Jekyll está em casa, Poole? -- perguntou o advogado.

-- Vou verificar, Mr. Utterson -- disse ele, introduzindo o visitante
num saguão espaçoso, confortável, de teto baixo, com piso de lajes,
aquecido (à maneira das casas de campo) por uma lareira descoberta, e
mobiliada com caros móveis de carvalho.

-- Quer esperar aqui ao pé do fogo, senhor? -- perguntou o criado. -- Ou
prefere levar uma luz até a sala de jantar?

-- Aqui está bem, obrigado -- disse o advogado, caminhando para perto da
lareira e apoiando-se no aparador.

A sala em que ficou a sós era a menina dos olhos de seu amigo, o médico;
e o próprio Utterson costumava referir-se a ela como o aposento mais
agradável de Londres.  Mas nessa noite havia algo que lhe gelava o
sangue; o rosto de Hyde se instalava pesadamente em sua memória; ele
sentia (o que era raro) uma espécie de náusea e de repugnância pela
vida; e no estado de espírito soturno em que se encontrava pensava ver
uma ameaça no modo como a luz das chamas bruxuleava e se refletia sobre
os armários polidos, bem como na sombra que se projetava sobre o teto. 
Envergonhou-se do alívio que sentiu quando Poole retornou com a
mensagem de que o dr. Jekyll não se encontrava em casa.

-- Vi Mr. Hyde entrar pela porta do antigo quarto de dissecação, Poole
-- disse ele. -- Isto é correto, quando o dr. Jekyll não se encontra em
casa?

-- Perfeitamente, Mr. Utterson -- disse o criado. -- Mr. Hyde possui uma
chave.

-- Seu patrão parece depositar uma grande confiança naquele jovem,
Poole -- disse o advogado, em tom casual.

-- Sim, senhor, sem dúvida -- disse Poole. -- Todos nós temos instruções
para obedecer às suas ordens.

-- Será que eu já encontrei Mr. Hyde antes? -- perguntou Utterson.

-- Oh, acho que não, senhor.  Ele nunca ceia aqui -- retrucou o mordomo.
-- Na verdade ele é visto muito pouco neste lado da casa; costuma entrar
e sair pela porta do laboratório.

-- Bem, boa noite, Poole.

-- Boa noite, Mr. Utterson.

E o advogado voltou para casa com o coração ainda mais pesado.

“Pobre Henry Jekyll”, pensou ele.  “Alguma coisa me diz que ele está
numa situação perigosa!  Ele era indisciplinado, quando jovem; há muito
tempo, sem dúvida; mas as leis de Deus não têm um estatuto de
limitação”.  

“Ah, deve ser isso: o fantasma dos erros do passado, o câncer de alguma
vergonha oculta, uma punição que se aproxima, \textit{pede claudo},\footnote{
“Pé manco”. Referência a um verso latino de Horácio, em suas Odes, onde
ele diz que “raras vezes a vingança abandona um homem perverso só por
ter um pé manco, mesmo que ele tenha uma grande dianteira sobre ela”. [\textsc{n.t.}]}
anos depois que a memória se encarregou de esquecer e o
amor-próprio de justificar o pecado cometido”.  E o advogado,
amedrontado com aquela ideia, pôs-se a rememorar seu próprio passado,
vasculhando todos os recantos da memória, receoso de que alguma velha
iniquidade saltasse à luz de repente, como um boneco de molas.  Seu
passado era razoavelmente isento de manchas; poucos homens poderiam
consultar o registro da própria vida com menos apreensão; e ainda assim
ele se mortificava ao pensar nos muitos erros que tinha cometido, e em
seguida se erguia, cheio de uma gratidão austera e respeitosa, ao
pensar nos muitos que tinha estado a ponto de cometer mas soubera
evitar.  E ao voltar seus pensamentos ao primeiro assunto que os
ocupara, permitiu-se experimentar uma centelha de esperança.

“Este senhor Hyde, se for bem investigado”, pensou, “deve revelar alguns
segredos sobre si mesmo; segredos tenebrosos, a julgar pela sua
aparência; segredos junto aos quais os piores defeitos do pobre Jekyll
reluziriam como a luz do sol.  As coisas não podem continuar como
estão. Sinto um calafrio ao imaginar essa criatura deslizando, furtiva,
como um ladrão, junto à cama de Harry; pobre Harry, que terrível
despertar! E o perigo de toda a situação: porque se esse Hyde
desconfiar da existência do testamento pode ficar impaciente para
entrar de posse da herança.  Ah, devo fazer tudo que está ao meu
alcance\ldots{} caso Jekyll me permita”.  E completou: “Se ao menos Jekyll
me permitisse!”.  Porque nesse instante ele via com os olhos da mente,
claro como uma imagem projetada, as estranhas cláusulas do testamento.


\chapter[3 -- O Dr. Jekyll estava bem à vontade]{3\break O Dr. Jekyll estava bem à vontade}

Cerca de duas semanas depois, por um golpe de sorte, o Dr. Jekyll
realizou um dos seus agradáveis jantares para cinco ou seis amigos de
longa data, todos eles homens inteligentes e de boa reputação, e todos
conhecedores dos bons vinhos; e Mr. Utterson arranjou as coisas de tal
modo que ficou para trás depois que todos os outros haviam se
despedido.  Isso não era um fato inusitado, mas algo que já sucedera
dezenas de vezes.  Quando as pessoas se afeiçoavam a Utterson, era de
verdade.  Os donos da casa gostavam de reter consigo o taciturno
advogado quando os convidados de temperamento alegre e de verbo
fácil pousavam o pé no umbral da porta de saída; gostavam de voltar a
sentar por mais algum tempo junto a alguém que fazia companhia de modo
discreto, reacostumando-se à solidão, serenando suas mentes junto
àquele homem de um silêncio tão rico, depois de horas de incessante
jovialidade.  O Dr. Jekyll não era exceção a essa regra, e agora,
sentado diante do fogo -- um homem corpulento, bem proporcionado, de
rosto liso aos cinquenta anos, com uma leve tintura de malícia talvez,
mas ostentando todos os sinais da competência e da bondade -- era
possível ver na sua expressão o quanto ele nutria, para com Mr.
Utterson, uma afeição calorosa e sincera.

-- Ando querendo ter uma conversa com você, Jekyll -- começou o
visitante. -- Lembra-se daquele seu testamento? 

Um observador atento poderia ter pensado que o assunto era
inconveniente; mas o doutor saiu-se de maneira descontraída.

-- Meu pobre Utterson, -- disse ele -- você é um homem sem sorte em ter um
cliente como eu.  Nunca vi um homem tão incomodado quanto você ficou em
relação ao meu testamento; a não ser que fosse aquele pedante
encadernado, Lanyon, diante do que ele chama de minhas heresias
científicas.  Oh, claro, sei que ele é um bom sujeito -- não precisa
franzir a testa --, um excelente sujeito, e eu gostaria de desfrutar mais
vezes de sua companhia; mas é um pedante encadernado, assim mesmo,\footnote{
Um trocadilho intraduzível: “hide-bound pedant” significa “um pedante em
encadernação de couro”, ou seja, excessivamente formal e cheio de pose.
A palavra “hide”, couro, remete a Mr. Hyde, e sugere a imagem de uma
aparência conservadora ocultando algo menos recomendável dentro de si. [\textsc{n.t.}]}
um pedante desinformado e boquirroto.  Nenhum homem
me desapontou tanto quanto Lanyon.

-- Você sabe que eu nunca concordei com aquilo -- retomou Utterson,
teimosamente, não dando atenção àquele novo assunto. 

-- O meu testamento? Sim, claro que sei -- disse o doutor, com certa
aspereza. -- Você já me disse.

-- Bem, preciso dizê-lo novamente -- continuou o advogado. -- Andei
recebendo algumas informações sobre o jovem Hyde.

O rosto grande e simpático do Dr. Jekyll empalideceu até os lábios, e
uma sombra escura desceu sobre seus olhos.

-- Não quero ouvir mais nada -- disse ele. -- Creio que tínhamos
concordado em não tocar nesse assunto.

-- O que ouvi foi abominável -- disse Utterson.

-- Não faz diferença. Você não entende a minha posição -- retrucou o
doutor, de modo um tanto incoerente. -- Estou numa situação difícil,
Utterson; uma posição muito estranha, muito estranha mesmo.  É um
daqueles casos em que falar a respeito de nada adianta.

-- Jekyll, -- disse Utterson -- você me conhece.  Sou um homem digno de
confiança.  Abra o seu coração e confie em mim; e não tenho dúvida de
que poderei ajudá-lo a sair dessa situação.

-- Meu bom Utterson, -- disse o doutor -- esse é um belo gesto de sua
parte, um gesto muito digno, e não tenho palavras para agradecer-lhe. 
Acredito totalmente na sua palavra; confiaria em você mais do que em
qualquer outro ser humano vivo, e até mais do que em mim mesmo, se me
fosse dada a escolha; mas nada disso é o que você imagina; não é uma
coisa tão grave assim; e, só para que seu bom coração possa repousar
sossegado, vou dizer-lhe uma coisa a mais: no momento que eu quiser,
posso me ver livre de Mr. Hyde.  Estendo-lhe a minha mão ao lhe afirmar
isto, e direi apenas mais uma coisa, Utterson, algo que sei que você
saberá ouvir com sensatez: este é um assunto pessoal, e peço-lhe que
deixe as coisas como estão.

Utterson refletiu um pouco, contemplando as chamas da lareira.

-- Não tenho dúvidas de que você está certo -- disse ele, pondo-se de
pé.

-- Bem, mas já que tocamos no assunto, e eu espero que seja esta a
última vez, -- continuou o doutor -- há um último detalhe que eu gostaria
que você entendesse bem.  Eu tenho de fato um grande interesse no pobre
Hyde.  Sei que você o encontrou pessoalmente; ele próprio me disse; e
receio que ele tenha se comportado com rudeza.  Mas eu de fato tenho um
grande interesse naquele jovem, bastante grande; e se eu me for daqui,
Utterson, gostaria de ter sua promessa de que você lhe dará apoio e
defenderá os seus direitos.  Acho que você o faria, se soubesse de
tudo; e receber sua promessa tiraria um peso da minha consciência.

-- Não posso mentir dizendo que espero gostar dele um dia -- disse o
advogado.

-- Não peço isso -- insistiu o doutor, pousando a mão no ombro do outro.
-- Peço apenas que aja com justiça; peço apenas que o ajude em meu nome,
quando eu não estiver mais aqui.

Utterson soltou um suspiro que não conseguia mais conter.

-- Bem, -- disse ele -- eu prometo.


\chapter[4 -- O assassinato de Sir Danvers Carew]{4\break O assassinato de Sir Danvers Carew}

Um ano depois destes acontecimentos, no mês de outubro de 18\ldots{}, a
cidade de Londres foi abalada por um crime de rara ferocidade, tornado
ainda mais notável pela alta posição social da vítima.  Os detalhes
eram escassos e chocantes.  Uma criada que vivia sozinha, numa casa não
muito distante do rio, tinha se recolhido ao seu quarto, no andar
superior, por volta das onze horas da noite.  Embora o nevoeiro tivesse
envolvido a cidade durante a madrugada, no começo da noite o céu
estivera limpo, e o beco para onde dava o quarto da criada estava
brilhantemente iluminado pela lua cheia.  Parece que a moça tinha
inclinações românticas, porque ela se sentou sobre um baú encostado
junto à janela, e caiu em devaneio.  Nunca antes (era o que ela
repetia, aos prantos, sempre que narrava o acontecido) tinha se sentido
mais em paz com a humanidade ou pensado no mundo com um espírito mais
generoso.  E enquanto estava ali sentada ela percebeu um belo e idoso
cavalheiro de cabelos brancos caminhando pelo beco, e, indo ao seu
encontro, outro cavalheiro de baixa estatura, ao qual de início ela não
deu muita atenção.  Quando os dois estavam a ponto de se cruzar, logo
abaixo da janela da criada, o homem mais idoso fez um cumprimento e
abordou o outro de maneira polida e respeitosa.  O que disse não
parecia ser de grande importância, e, pelo modo como apontou, poderia
estar apenas perguntando uma direção qualquer; mas o luar brilhou sobre
seu rosto enquanto falava, e a moça o observou com prazer, pois ele
exprimia uma certa inocência e uma bondade à moda antiga, misturadas a
uma certa nobreza, como alguém que tem boas razões para estar em paz
consigo mesmo.  Em seguida os olhos dela se dirigiram para o outro
cavalheiro, e ela ficou surpresa ao reconhecer nele um certo Mr. Hyde
que certa vez visitara seu patrão e pelo qual ela passara a nutrir
aversão.  Ele empunhava uma pesada bengala, com a qual brincava
distraidamente; mas não deu uma palavra de resposta e pareceu escutar o
outro com impaciência mal reprimida.  E então, de súbito, explodiu num
violento acesso de fúria, batendo com o pé no chão, erguendo a bengala
e avançando sobre o outro (assim descreveu a criada) como um homem
enlouquecido.  O cavalheiro idoso recuou um passo, com uma expressão de
surpresa e de quem está ofendido; e nesse instante Mr. Hyde perdeu
totalmente o controle e o agrediu com a bengala, derrubando-o.  No
momento seguinte, com a fúria de um gorila, estava pisoteando o homem
caído e cobrindo-o com uma saraivada de golpes tão fortes que se podia
ouvir o ruído dos ossos partidos, enquanto o corpo do homem se estorcia em
convulsões sobre o pavimento.  Diante de uma visão e de sons tão
horríveis, a criada desmaiou.

Eram duas horas quando ela voltou a si e chamou a polícia.  O assassino
já tinha desaparecido há muito tempo; mas ali continuava sua vítima, o
corpo caído no chão do beco, incrivelmente desfigurado.  O bastão com
que o crime havia sido cometido, mesmo sendo de uma madeira rara e
muito resistente, havia se partido ao meio devido à fúria daquele
ataque cruel e insensato; uma das metades tinha rolado pelo chão até
cair na sarjeta, e a outra, sem dúvida, havia sido levada pelo
criminoso.  Uma bolsa e um relógio de ouro foram encontrados junto à
vítima, mas nenhum cartão de visita ou outro documento, exceto um
envelope fechado e selado, que ele provavelmente estava levando à caixa
de correio, e que tinha escritos na frente o nome e o endereço de Mr.
Utterson.

Este envelope chegou às mãos do advogado na manhã seguinte, antes mesmo
que ele se erguesse da cama; e mal ele o tinha recebido e se inteirado
das circunstâncias, cerrou os lábios com firmeza, dizendo ao portador:

-- Não direi nada enquanto não tiver visto o corpo; isso pode ser muito
sério.  Espere enquanto me visto.

E com a mesma gravidade de expressão ele tomou às pressas o seu café da
manhã e rumou para a delegacia de polícia, para onde o corpo tinha sido		
removido.  Assim que entrou no aposento, ele fez um sinal afirmativo.

-- Sim, -- disse -- eu o reconheço.  Lamento dizer que este é Sir
Danvers Carew. 

-- Bom Deus, senhor, -- exclamou o oficial -- será possível?!  -- E no
mesmo instante seus olhos brilharam com ambição profissional. -- Isto
vai causar muito rumor, e talvez o senhor possa nos ajudar a capturar o
sujeito.

Mr. Utterson já tinha recuado ao ouvir o nome de Hyde; mas quando o
pedaço de bengala lhe foi apresentado suas últimas dúvidas se esvaíram;
mesmo quebrada e estilhaçada como estava, ele reconheceu de imediato
uma bengala que ele mesmo presenteara a Henry Jekyll muitos anos atrás.

-- Esse Mr. Hyde é uma pessoa de baixa estatura? -- perguntou ele.

-- Pronunciadamente baixa; e com uma aparência pronunciadamente
desagradável, de acordo com o testemunho da criada -- disse o oficial.

Mr.  Utterson refletiu um pouco e depois, erguendo a cabeça, disse: 

-- Se quiser me acompanhar em meu cabriolé, acho que posso levá-lo à
casa onde ele mora.

Eram cerca de nove da manhã, durante o primeiro nevoeiro daquela
estação.  Um grande dossel cor de chocolate parecia estar baixando do
céu, mas o vento arremetia continuamente de encontro a esses vapores em
torvelinho; e enquanto o cabriolé avançava devagar de rua em rua Mr.
Utterson contemplou uma quantidade maravilhosa de nuances e tonalidades
de lusco-fusco; aqui, a rua era tão escura quanto ao anoitecer; ali,
havia por toda parte um resplendor de um marrom profundo e esmaecido,
como se fosse a luz de algum estranho incêndio; e adiante, por um breve
momento, o nevoeiro se rasgava e um jato esmaecido de sol se esgueirava
por entre as massas em turbilhão.  O bairro decrépito do Soho,
vislumbrado por entre essas mudanças de luz, com suas ruas lamacentas,
suas prostitutas mal vestidas, seus lampiões, que não tinham sido
extintos desde a noite anterior ou já tinham sido reavivados para fazer
frente àquela nova ofensiva da escuridão, tudo aquilo parecia, aos
olhos do advogado, um quarteirão de uma cidade vista em pesadelo.  Além
do mais, os pensamentos que se agitavam em sua mente eram da natureza
mais lúgubre possível, e quando ele relanceava os olhos para o seu
companheiro de trajeto experimentava um pouco daquele terror da lei e
dos seus oficiais que de tempos em tempos podem acometer mesmo o mais
honesto dos homens. 

Quando o cabriolé parou diante do endereço indicado, o nevoeiro tinha se
erguido um pouco, revelando uma rua miserável, um botequim de péssima
aparência, um restaurante francês numa casa pouco elevada, uma loja que
vendia no varejo revistas baratas e saladas de dois pence, crianças
andrajosas encolhidas sob os portais, e muitas mulheres de diferentes
nacionalidades que cruzavam a rua, de chave em punho, para tomar a
primeira bebida do dia; logo em seguida o nevoeiro voltou a envolver a
área com sua luminosidade cor de âmbar, isolando-os daquele ambiente
miserável.  Ali ficava a residência do protegido de Henry Jekyll, o
homem que era herdeiro de um quarto de milhão de libras esterlinas.

A porta foi aberta por uma mulher idosa, de rosto amarelo como o marfim,
e cabelos cor de prata.  Tinha um rosto maldoso suavizado pela
hipocrisia, mas seus modos eram impecáveis.  Sim, disse ela, era ali
que morava Mr. Hyde, mas ele não estava em casa; tinha chegado bastante
tarde na noite anterior, mas voltara a sair em menos de uma hora; não,
não havia nada de estranho naquilo; seus hábitos eram muito
irregulares, e se ausentava com frequência; por exemplo, antes da noite
anterior ela não o tinha visto por mais de dois meses.

-- Muito bem.  Neste caso, precisamos examinar os aposentos dele --
disse o advogado, e quando a mulher começou a dizer que aquilo era
impossível, completou: -- Será melhor dizer-lhe quem é este cavalheiro. 
É o inspetor Newcomen, da Scotland Yard.

Um brilho de alegria maligna apareceu no rosto da mulher.

-- Ah! -- exclamou ela. -- Ele está com problemas!  O que foi que fez?

Mr. Utterson e o inspetor se entreolharam.

-- Ele não parece ser um sujeito muito popular -- disse este último; e
para a mulher: -- E agora, minha boa senhora, permita que eu e o meu
companheiro façamos um pequeno exame no quarto dele.

Na casa, onde o único outro morador era aquela velha, Mr. Hyde tinha
alugado apenas dois quartos, mas ambos eram mobiliados com luxo e bom
gosto.  Havia um armário cheio de garrafas de vinho; a baixela era de
prata, o serviço de mesa elegante; havia um belo quadro pendurado numa
parede, um presente (Mr. Utterson imaginou) de Henry Jekyll, que era um
grande conhecedor de arte; e os tapetes eram de boa qualidade e de
cores agradáveis. Naquele momento, no entanto, os quartos davam mostras
de terem sido saqueados recentemente, e às pressas; havia roupas
jogadas no chão, com os bolsos revirados; gavetas com trancas estavam
escancaradas; e diante da lareira havia uma pilha de cinzas escuras,
como se uma certa quantidade de papéis tivessem sido queimados.  Do
meio das brasas o inspetor conseguiu extrair um pedaço da borda de um
talão de cheques, de cor verde, que tinha resistido à ação do fogo;
além disso, a outra metade da bengala partida foi encontrada atrás da
porta; e o inspetor proclamou sua satisfação ao ver como todas estas
peças se encaixavam.  Uma visita ao banco comprovou que a conta do
suspeito tinha um crédito de vários milhares de libras, o que fechou o
círculo.

-- Pode ter certeza, cavalheiro -- disse o inspetor a Mr. Utterson. --
Ele está nas nossas mãos.  Deve ter perdido a cabeça, ou nunca teria
deixado para trás a bengala, ou, mais do que tudo, queimado o talão de
cheques.  Ora, todo homem precisa de dinheiro para viver.  Não temos
outra coisa a fazer senão esperar que vá ao banco, e distribuir
panfletos com sua descrição.

Isto contudo, não teve os resultados que eram esperados, porque havia
muito poucas pessoas que conhecessem Mr. Hyde; mesmo o patrão da criada
que denunciara o crime o vira apenas duas vezes; nenhum traço foi
encontrado de sua família; ele nunca fora fotografado; e as poucas
pessoas capazes de descrevê-lo o faziam de modo contraditório, como
tantas vezes ocorre com testemunhas visuais. Todas concordavam em
apenas um ponto: a sensação inquietante, de uma deformação indefinível,
que o suspeito provocava em quem pusesse os olhos nele. 


\chapter[5 -- O incidente da carta]{5\break O incidente da carta}

Foi apenas no final da tarde que Mr. Utterson conseguiu finalmente bater
à porta da casa do Dr. Jekyll, onde foi imediatamente recebido por
Poole e conduzido, através da cozinha, até um pátio interno que em
outra época havia sido um jardim, e depois à construção que ficava nos
fundos, e que tanto era denominada de “laboratório” quanto de “quarto
de dissecação”.  O doutor Jekyll havia comprado aquela casa aos
herdeiros de um cirurgião famoso; e como suas próprias pesquisas se
inclinavam mais na direção da Química do que da Anatomia, ele tinha
mudado a função daquele bloco, transformando-o em laboratório.  Era a
primeira vez que o advogado tinha acesso àquela parte da residência do
seu amigo; e ele observou com curiosidade  aquela estrutura sombria,
manchada de fuligem, e sem janelas.  Tomado por uma sensação sinistra,
ele entrou para um pequeno anfiteatro interno, onde tempos atrás
estudantes ansiosos por conhecimento tinham se sentado, e que agora
estava vazio e silencioso.  As mesas estavam cheias de aparelhos
químicos, o chão coberto por caixotes e pela palha usada para proteger
as encomendas, enquanto a luz se escoava fracamente por uma claraboia
embaçada. Na extremidade oposta, um lance de escadas subia até uma
porta coberta de tecido vermelho, através da qual Mr. Utterson
finamente teve acesso ao escritório particular do doutor.  Era uma sala
espaçosa, com armários de madeira para guardar frascos e outros
recipientes de vidro, um espelho de corpo inteiro que girava na
moldura, e uma mesa de trabalho.  Três janelas dando para o pátio da
rua de trás, fechadas por barras de ferro, eram a única abertura para o
exterior.  O fogo ardia na lareira, que tinha uma lâmpada acesa sobre a
borda da chaminé, porque mesmo no interior da casa o nevoeiro se
infiltrava, espesso; e ali, junto ao calor, estava sentado o Dr.
Jekyll, com a aparência de um homem mortalmente enfermo.  Não se ergueu
para receber o visitante, mas estendeu-lhe uma mão fria e deu-lhe as
boas-vindas com uma voz diferente do normal.

-- E então, -- disse Mr. Utterson, assim que Poole se retirou -- leu os
jornais de hoje?

O doutor estremeceu.

-- Os jornaleiros estavam gritando as manchetes na praça -- disse. --
Ouvi tudo da minha sala de jantar.

-- Serei breve -- disse o advogado. -- Carew era meu cliente, mas você
também o é, e preciso saber o que estou fazendo. Suponho que você não
cometeu a loucura de esconder aqui aquele indivíduo.

-- Utterson, juro-lhe por Deus! -- exclamou o doutor. -- Juro por Deus
que nunca mais voltarei a pôr os olhos nele.  Dou-lhe minha palavra de
honra de que tudo acabou neste mundo entre mim e ele.  Chegou ao fim da
linha.  E a verdade é que ele não quer minha ajuda; você não o conhece
tanto quanto eu; ele está seguro, bastante seguro; anote o que estou
lhe dizendo, ninguém jamais voltará a ouvir falar dele. 

O advogado o escutou com ar taciturno; a atitude febril do seu amigo não
lhe agradava nem um pouco.

-- Você parece muito seguro disso -- comentou ele. -- E, para o seu bem,
espero que tenha razão.  Porque se este caso for aos tribunais, seu
nome forçosamente será citado.

-- Tenho certeza quanto a ele -- replicou Jekyll. -- Tenho motivos para
saber isto com certeza, motivos que não posso compartilhar com ninguém.
 Mas há uma coisa sobre a qual preciso de seus conselhos. Eu\ldots{} eu
recebi uma carta, e estou sem saber se devo mostrá-la à polícia. 
Preferiria deixá-la em suas mãos, Utterson; você poderia avaliá-la com
grande ponderação, tenho certeza; tão grande é minha confiança em você.

-- Suponho que seu receio é de que a carta leve à descoberta de Hyde? --
perguntou o advogado.

-- Não, -- disse o outro -- não posso dizer que me preocupo com o que
pode acontecer com Hyde.  Minha relação com ele acabou.  Estava
preocupado comigo mesmo, porque este assunto detestável está me
deixando numa condição muito exposta.

Utterson ficou ruminando aquilo por algum tempo; estava surpreso com o
comportamento egoísta do seu amigo, e ao mesmo tempo aliviado ao vê-lo
agir assim.

A carta estava escrita numa caligrafia estranha, empertigada, e estava
assinada “Edward Hyde”; e comunicava, com brevidade, que o benfeitor do
signatário, Dr. Jekyll, a quem ele mal fora capaz de pagar as mil
generosidades que recebera, não precisava se preocupar com a sua
segurança, uma vez que ele tinha meios de fuga dignos de confiança.  O
advogado apreciou bastante esta carta; ela dava uma diferente coloração
à amizade entre aqueles dois; e ele se censurou por algumas das
suspeitas que chegara a alimentar no passado.

-- Tem o envelope? -- perguntou.

-- Eu o queimei -- replicou Jekyll -- antes  mesmo de perceber a que se
referia a carta.  Mas não tinha carimbos do correio.  Deve ter sido
trazida pessoalmente.

-- Devo guardar isto e manter silêncio? -- perguntou o advogado.

-- Confio inteiramente na sua decisão -- foi a resposta. -- Perdi a
confiança em mim mesmo.

-- Bem, vou pensar a respeito -- disse o advogado. -- Agora, só mais uma
coisa: foi Hyde quem ditou os termos do testamento, referindo-se à
possibilidade de que você morresse?

O doutor pareceu a ponto de desfalecer, mas cerrou os lábios com força e
com um sinal de cabeça respondeu que sim.

-- Eu sabia -- disse Utterson. -- Ele planejava matá-lo.  Você teve sorte
em escapar. 

-- Tive muito mais do que isso -- respondeu o doutor, solenemente. --
Tive uma lição\ldots{}  Ah, meu Deus, Utterson, a lição que recebi!  --  E
cobriu o rosto com as mãos.

Ao sair, o advogado parou para trocar algumas palavras com Poole.

-- A propósito, -- disse ele -- trouxeram uma carta para aqui, hoje; que
aparência tinha o portador?

Mas Poole foi taxativo: nada tinha chegado a casa com exceção da remessa
normal do correio, “e mesmo esta continha apenas circulares”, completou
ele.

Esta informação fez com que o visitante se despedisse dali com seus
temores reavivados.  Parecia certo que a carta tinha sido entregue
através da porta do laboratório; talvez tivesse sido escrita no próprio
escritório do médico; e se isso fosse verdade deveria ser interpretada
de outro modo, e tratada com mais cuidado.  Os jornaleiros, enquanto
ele cruzava a rua, gritavam pelas calçadas: “Edição especial!  O
assassinato chocante de um membro do Parlamento!”  Esta era a oração
fúnebre dedicada a um amigo e cliente seu, e Mr. Utterson não pôde
evitar um certo receio de ver o nome de outro amigo arrastado no
torvelinho daquele escândalo.  Era no mínimo uma decisão delicada que
lhe cabia tomar; e, mesmo com a autossuficiência que lhe era habitual,
ele começou a sentir a necessidade de aconselhamento.  Talvez não de
forma direta, mas através de outros meios.

Algum tempo depois ele estava sentado diante de sua própria lareira, na
companhia de Mr. Guest, seu chefe de escriturários, e entre os dois, a
uma distância do fogo calculada com a máxima precisão, uma garrafa de
um vinho especial que por muito tempo jazera longe do sol, nas
profundezas de sua adega.  O nevoeiro continuava a flutuar sobre a
cidade, onde os lampiões cintilavam como carbúnculos; e através do ar
abafado por essas nuvens baixas, a procissão da vida urbana continuava
a fluir pelas grandes artérias com o som de uma incessante ventania. 
Mas aquele aposento era alegrado pela luz do fogo; na garrafa, os
ácidos do vinho tinham se atenuado, a cor púrpura imperial se suavizara
com o tempo, assim como as cores se tornam mais vívidas através de
janelas com vitrais; o brilho cálido das tardes de outono, nos vinhedos
das colinas, estava pronto para ser libertado e para dispersar o
nevoeiro londrino.  Sem o perceber, o advogado foi relaxando pouco a
pouco.  Não havia homem de quem ele ocultasse menos segredos do que Mr.
Guest; e ele mesmo não sabia se guardava para si tantos quanto
pretendia.  Guest visitava o doutor Jekyll de vez em quando, a serviço;
conhecia bem Poole, o mordomo; dificilmente teria deixado de ouvir
algum comentário sobre a familiaridade com que Mr. Hyde se movimentava
naquela casa; e era capaz de tirar suas próprias conclusões; que mal
faria se visse aquela carta que talvez fosse capaz de esclarecer todo o
mistério? Principalmente sendo Guest um grande estudioso e crítico da
arte da grafologia, não consideraria essa consulta um fato natural, e
não acederia de bom grado?  Além do mais, o escriturário era um homem
ponderado; dificilmente leria um documento tão estranho sem emitir
algum tipo de comentário; e talvez Mr. Utterson pudesse, a partir desse
comentário, traçar sua futura linha de ação.

-- Uma coisa muito triste isso que aconteceu com Danvers -- disse Mr.
Utterson.

-- Sim, senhor, muito triste de fato.  Despertou uma grande comoção
entre o público -- replicou Guest. -- O criminoso, é claro, devia estar
louco.

-- Gostaria de saber sua opinião sobre este assunto -- disse Utterson. --
Tenho aqui comigo um documento de próprio punho desse indivíduo; isto
fica entre nós,  porque ainda não tenho ideia do que vou fazer com ele;
e trata-se na melhor das hipóteses de um assunto escabroso.  Aqui está:
algo dentro de sua especialidade, o autógrafo de um assassino.

Os olhos de Guest brilharam, e ele sentou-se mais aprumado na poltrona,
passando imediatamente a examinar o papel com a máxima atenção.

-- Não, senhor -- disse ele. -- Não é louco.  Mas trata-se de uma
caligrafia estranha.

-- E por tudo que ouvimos dizer, escrita por um homem estranho.

Neste instante o criado entrou com uma mensagem.

-- É a letra do Dr. Jekyll, senhor? -- perguntou o escriturário. - Achei
que conhecia essa caligrafia.  Algo pessoal, Mr. Utterson?

-- Apenas um convite para jantar.  Por que?  Quer examiná-lo?

-- Só por um instante.  Obrigado, senhor.

Ele colocou os dois papéis lado a lado e observou cuidadosamente o
conteúdo de ambos.

-- Obrigado, senhor -- disse por fim, devolvendo-os a Mr. Utterson. -- Um
documento autógrafo muito interessante.

Houve uma pausa durante a qual Mr. Utterson lutou para conter a
curiosidade, mas acabou perguntando de repente:

-- Por que pediu para compará-los, Guest?

-- Bem, senhor, -- disse o outro -- há uma semelhança muito singular.  As
duas caligrafias são idênticas em muitos pontos, têm apenas uma
inclinação diferente.

-- Muito esquisito -- disse Utterson.

-- Muito esquisito mesmo -- concordou Guest.

-- Seria melhor não mencionarmos esta observação para ninguém -- disse o
advogado.

-- De fato, senhor, seria bem melhor.  Compreendo.

Mas assim que Mr. Utterson se viu a sós naquela noite, foi logo trancar
a nota no cofre, onde ela ficou dali em diante.

“O quê!”, pensou ele.  “Henry Jekyll falsificando um documento para
proteger um assassino!”  E o sangue gelou nas suas veias. 


\chapter[6 -- O estranho incidente com o Dr. Lanyon]{6\break O estranho incidente com o Dr. Lanyon}

O tempo passou.  Milhares de libras esterlinas foram oferecidas em
recompensa, porque a morte de Sir Danvers foi encarada como uma afronta
à ordem pública, mas Mr. Hyde desapareceu da vista da polícia como se
nunca tivesse existido.  Muitos fatos foram descobertos sobre seu
passado, sem dúvida, e todos eles eram comprometedores: histórias sobre
seus atos de crueldade, ao mesmo tempo insensíveis e brutais; sua vida
viciosa, suas estranhas companhias, o ódio que o acompanhara ao longo
de toda sua carreira; mas não se ouviu um sussurro sequer a respeito de
seu paradeiro atual.  A partir do momento em que deixara seus aposentos
no Soho, na madrugada do crime, ele havia simplesmente evaporado; e
à medida que o tempo foi passando, Mr. Utterson foi perdendo aquela
sensação de alarme e ficou em paz consigo mesmo.  A morte de Sir
Danvers fora, no seu modo de ver, mais do que compensada pelo
desaparecimento de Mr. Hyde.  Agora que a má influência não se fazia
mais presente, uma nova vida parecia começar para o Dr. Jekyll. Ele
emergiu de sua vida reclusa, reatou relações com os amigos, voltou a
ser o anfitrião e o companheiro jovial que todos conheciam, e, ainda
que ele sempre tivesse sido conhecido pelas suas numerosas atitudes
beneficentes, agora ele não era menos notório por sua vida religiosa. 
Sempre ocupado, vivendo ao ar livre, praticando o bem.  Seu rosto tinha
uma expressão mais franca e mais satisfeita, como que traduzindo uma
consciência íntima de estar sendo útil; e por mais de dois meses o
doutor teve uma vida pacífica.

No dia 8 de janeiro, Utterson havia ceado na casa do doutor, com
um pequeno grupo de amigos; Lanyon estivera presente; e os olhos do
doutor se moviam de um para o outro amigo como nos velhos tempos em que
os três eram companheiros inseparáveis.  No dia 12, e logo em seguida
no dia 14, o doutor encontrou fechada a porta do amigo.  “O doutor
estava recolhido, sem poder receber ninguém”, declarou Poole.  No dia
15 Mr. Utterson tentou novamente, e mais uma vez foi barrado; e tendo
se acostumado nos dois últimos meses a encontrar o amigo quase
diariamente, sentiu outra vez o peso da solidão em seu espírito.  Na
quinta noite ele convidou Guest para jantar em sua casa; e na sexta
dirigiu-se à casa de Lanyon.

Ali, pelo menos, conseguiu ser recebido; mas ao entrar ficou chocado com
a aparência do médico.  Era como se uma sentença de morte estivesse
gravada no seu rosto.  O homem outrora rosado tinha se tornado lívido,
e de rosto flácido; estava visivelmente mais calvo e mais envelhecido;
e no entanto não foram esses sinais de uma rápida decadência física que
atraíram a atenção do advogado, mas uma expressão no olhar e nas
atitudes do amigo que pareciam revelar um terror profundamente
incrustado na sua alma.  Era improvável que o médico tivesse razões
para temer pela própria vida; e no entanto foi esta a impressão que
teve Mr. Utterson.  “Sim,” pensou ele, “ele é médico, deve conhecer o
estado em que se encontra e deve saber que está com os dias contados; e
isso é mais do que é capaz de suportar”.  E no entanto, quando o
advogado mencionou essa mudança de aparência, foi com um ar de firmeza
que Lanyon afirmou saber que era um homem condenado.

-- Sofri um choque -- disse ele -- do qual nunca irei me recuperar.  É
uma questão de semanas.  Bem, tive uma vida agradável; sim, cavalheiro,
acostumei-me a gostar da vida que tinha.  Às vezes penso que se
conhecêssemos tudo quanto existe nos sentiríamos mais contentes na hora
de partir.

-- Jekyll também está doente -- comentou Utterson. -- Você o viu estes
dias?

Mas o rosto de Lanyon sofreu uma mudança, e ele ergueu uma mão trêmula.

-- Não quero mais ver o doutor Jekyll nem ouvir falar no seu nome --
disse ele com veemência, numa voz vacilante. -- Encerrei minhas relações
com aquele indivíduo, e peço-lhe que não faça nenhuma alusão a alguém
que considero como se estivesse morto.

-- Ora, ora -- disse Mr. Utterson.  E, depois de uma longa pausa: --
Posso fazer algo a respeito?  Somos três velhos amigos, Lanyon; e
talvez não vivamos o bastante para fazer outras amizades assim.

-- Nada pode ser feito -- disse Lanyon. -- Pergunte a ele próprio.

-- Ele não me recebeu estes dias -- disse o advogado.

-- Isso não me surpreende -- foi a resposta. -- Algum dia, Utterson,
depois que eu estiver morto, você talvez seja capaz de ver os dois lados desta
questão.  Não posso dizer-lhe mais nada.  Enquanto isso, se quiser
sentar-se e conversar comigo sobre outra coisa, pelo amor de Deus,
fique; mas se não conseguir evitar esse maldito assunto, então, em nome
de Deus, vá embora, porque isso é algo que não posso suportar.

Assim que chegou em casa, Utterson sentou-se e redigiu uma carta para
Jekyll, queixando-se por não ter sido admitido em sua casa, e
perguntando os motivos de seu rompimento com Lanyon; e no
dia seguinte chegou-lhe uma longa resposta, que chegava a empregar
expressões patéticas, e às vezes um tom velado e misterioso, para
explicar que seu afastamento de Lanyon era irremediável.  “Não culpo
nosso velho amigo,” escreveu Jekyll, “mas concordo com a opinião dele
de que não devemos voltar a conviver.  Pretendo de agora em diante
levar uma vida completamente reclusa; peço que não se surpreenda, e que
não duvide da minha amizade sincera, caso venha a encontrar com
frequência a minha porta fechada para você.  Espero que seja paciente
comigo e me permita seguir no meu caminho, por mais obscuro que lhe
pareça.  Eu atraí sobre mim uma punição e um perigo cuja natureza não
posso revelar.  Se por um lado sou o maior dos pecadores, por outro sou
também o maior dos sofredores.  Nunca imaginei que este mundo tivesse
espaço para sofrimentos e terrores tão desmoralizantes.  E você não
pode fazer outra coisa, Utterson, para aliviar o meu fardo, a não ser
respeitar o meu silêncio”.

Utterson ficou perplexo; a influência negativa de Hyde tinha sido
removida, o médico tinha retornado aos seus antigos afazeres e amigos;
uma semana atrás estava sorridente, como se tudo na vida lhe
assegurasse uma velhice alegre e honrada; e agora, de um momento para
outro, amizade, paz de espírito e todas as coisas que lhe davam razão à
vida pareciam ter sido destruídas.  Uma mudança tão radical e súbita
parecia indício de loucura; mas à vista das atitudes e das palavras de
Lanyon, devia haver alguma razão oculta que a justificava.

Uma semana depois o dr. Lanyon caiu de cama, e estava morto em menos de
quinze dias.  Na noite que se seguiu ao funeral, durante o qual se
sentira fortemente deprimido, Utterson trancou-se em seu
escritório, e ali, à luz melancólica de uma vela, retirou e abriu
diante de si um envelope endereçado com a caligrafia e selado com o
timbre de seu falecido amigo.  

\begin{quote}
“\textsc{pessoal}: às mãos, \textsc{exclusivamente}, de G. J. Utterson, e para ser
destruído, sem ler, no caso de seu falecimento preceder o meu”.
\end{quote}

Assim o envelope estava endereçado, de maneira enfática, e o advogado
temia pelo que pudesse descobrir lá dentro.  “Sepultei hoje uma grande
amizade,” pensou ele; “será que este documento irá me custar outra?”. 
E em seguida, censurando este receio como uma forma de deslealdade, 
partiu o lacre.  No interior havia outro envelope, igualmente selado,
no qual estava escrito: “Não deve ser aberto até a morte ou
desaparecimento do dr. Henry Jekyll”.  Utterson não pôde acreditar no
que via.  Sim, falava-se em desaparecimento; também aqui, a exemplo do
insólito testamento que ele há muito tempo devolvera ao seu autor,
também aqui a ideia de desaparecimento e o nome de Henry Jekyll estavam
claramente associados.  Mas no testamento essa ideia brotara de uma
sinistra sugestão do tal Hyde; estava expressa ali com um propósito
bastante claro e horrível.  Escrito pela mão de Lanyon, que significado
poderia ter?  Uma grande curiosidade lhe veio ao espírito, a de não dar
ouvidos à proibição e mergulhar de vez no fundo daquele mistério; mas a
honra profissional e o compromisso com seu falecido amigo eram
muito mais fortes; e o envelope foi dormir intacto no
fundo do seu cofre.

Uma coisa é combater a curiosidade, outra é derrotá-la; e é duvidoso
que, a partir daquele dia, Utterson voltasse a desejar a proximidade de
seu velho amigo com a mesma ansiedade de antes.  Pensava nele com
simpatia; mas seus pensamentos eram inquietos e amedrontados. 
Procurou-o, sem dúvida, algumas vezes; mas é possível que tenha se
sentido aliviado quando lhe foi negado acesso, e talvez, em
seu coração, ele preferisse ficar conversando com Poole na porta de
entrada, cercado pelo ar e pelos sons da cidade, do que ser admitido no
interior daquela casa de voluntária servidão, e sentar-se para
conversar com o seu dono inescrutável e recluso.  Quanto a Poole, não
tinha boas notícias para trazer-lhe.  Ao que parece o médico, mais do
que nunca, estava confinado no gabinete que ficava na parte superior do
laboratório, chegando mesmo a dormir lá às vezes; estava abatido,
silencioso, não lia mais; era como se algo estivesse incrustado em
sua mente.  Utterson acabou se acostumando de tal forma a esses
relatórios, que nunca variavam, que suas visitas foram se tornando cada
vez mais espaçadas. 


\chapter[7 -- O incidente da janela]{7\break O incidente da janela}

Aconteceu certo domingo, quando Mr. Utterson estava fazendo seu passeio
habitual com Mr. Enfield, que o seu trajeto se encaminhasse mais uma
vez para aquela rua transversal; e ao passar diante da porta, pararam
para \mbox{observá-la}.

-- Bem, -- disse Enfield -- essa história, pelo menos, já se encerrou. 
Nunca mais veremos de novo Mr. Hyde. 

-- Espero que não -- disse Mr. Utterson. -- Já lhe contei que
encontrei-me com ele uma vez, e que mal pude controlar minha repulsa?

-- É impossível que uma coisa aconteça sem que experimentemos a outra --
retornou Enfield. -- E a propósito, você deve ter me achado muito
estúpido por não saber que isto é uma entrada dos fundos para a casa de
Jekyll.  Foi em parte por culpa sua que acabei descobrindo.

-- Então você descobriu, não é? -- disse Utterson. -- Neste caso, podemos
ir até o pátio e olhar as janelas.  Para falar a verdade, estou
inquieto a respeito do pobre Jekyll; e mesmo estando aqui fora creio
que a presença de um amigo pode fazer-lhe algum bem.

O pátio estava frio e um tanto úmido, e já mergulhado numa penumbra
prematura, embora o céu ainda brilhasse com as cores do crepúsculo.  A
janela do meio estava semiaberta, e sentado junto a ela, com uma
expressão de infinita tristeza, como um prisioneiro sem esperanças,
Utterson avistou o Dr. Jekyll.  

-- Ora!  Jekyll! -- exclamou ele.  -- Espero que esteja se sentindo
melhor.

-- Estou mal, Utterson -- replicou o médico com expressão sombria. --
Muito, muito mal.  Não vou durar muito tempo, graças a Deus.

-- Você passa um tempo excessivo dentro de casa -- disse o advogado. --
Devia sair um pouco mais e fazer o sangue circular, como fazemos eu e
Enfield. A propósito, este é o meu primo, Mr. Enfield\ldots{}  O Dr.
Jekyll\ldots{} Então, venha!  Pegue seu chapéu e vamos dar uma volta. 

-- Você é muito bondoso -- suspirou o outro. -- Gostaria muito de
fazê-lo; mas não, não, isto é impossível; não me atrevo.  Mas sem
dúvida, Utterson, estou muito feliz em vê-lo, isso é um enorme prazer. 
Gostaria de convidar você e Mr. Enfield para entrarem um pouco, mas o
local não está em ordem. 

-- Ora, -- disse o advogado, com bom humor -- então a melhor coisa a
fazermos é ficar aqui embaixo e conversar com você onde se encontra.

-- É justamente o que eu estava a ponto de sugerir -- disse o médico com
um sorriso.  Porém, mal tinha pronunciado estas palavras, quando o
sorriso foi varrido do seu rosto e substituído por uma expressão de um
terror e de um desespero tão abjetos que gelou o sangue dos dois
cavalheiros que o observavam de baixo.  Durou apenas um instante,
porque logo a janela foi fechada com violência; mas aquele vislumbre
fora suficiente, e eles se viraram e abandonaram o pátio sem dizer uma
palavra.  Também em silêncio atravessaram a rua, e não foi senão quando
tinham chegado a uma avenida próxima, onde, apesar de ser domingo, via-se
um certo movimento, que Mr. Utterson por fim virou-se e olhou para seu
companheiro.  Os dois estavam igualmente pálidos; e a resposta nos
olhos de ambos era a mesma expressão de horror.

-- Deus nos perdoe, Deus nos perdoe -- disse Mr. Utterson.

Mas Mr. Enfield apenas assentiu vigorosamente com a cabeça, e eles
voltaram a caminhar em silêncio.


\chapter[8 -- A última noite]{8\break A última noite}

Mr. Utterson estava sentado junto à lareira certa noite, após o jantar,
quando foi surpreendido pela visita de Poole.

-- Céus, Poole, o que o traz aqui? -- exclamou ele; e ao observar melhor
a atitude do outro: -- Que aflição é essa? O doutor está doente?

-- Mr. Utterson, -- disse o homem -- há algo de errado.

-- Sente-se, e aqui está uma taça de vinho -- disse o advogado. -- Agora,
descanse, e depois diga-me com clareza o que deseja.

-- O sr. conhece os hábitos do doutor -- replicou Poole -- e como ele
gosta de se trancar.  Bem, ele está mais uma vez trancado naquele
escritório; e não gosto nada disso, senhor\ldots{}  Preferiria morrer.  Mr.
Utterson, estou com medo.

-- Ora, meu bom homem, --  disse o doutor -- seja mais explícito.  Está
com medo do quê?

-- Estou com medo há uma semana -- replicou Poole, evitando a pergunta.
-- E não aguento mais.

A aparência do mordomo confirmava amplamente suas palavras; seus modos
tinham mudado para pior; e exceto pelo momento em que se referiu ao seu
terror, não olhou o advogado nos olhos uma vez sequer.  Mesmo agora,
estava sentado, apoiando sobre o joelho a taça de vinho que não
provara, e com os olhos fitos num canto do aposento.

-- Não aguento mais -- repetiu.

-- Ora, -- disse o advogado -- vejo que deve ter suas razões, Poole;
vejo que existe aqui alguma coisa que não está nos eixos.  Tente me
dizer do que se trata.

-- Acho que aconteceu um crime -- disse Poole, com voz rouca.

-- Um crime! -- exclamou Mr. Utterson, já assaltado pelo medo e irritado
por esse mesmo motivo. -- Que tipo de crime?  O que quer dizer?

-- Não sei o que dizer-lhe, senhor -- foi a resposta. -- Não quer me
acompanhar e ver com seus próprios olhos?

A única resposta de Mr. Utterson foi levantar-se e ir em busca do chapéu
e do sobretudo; mas ele percebeu com espanto a expressão de imenso
alívio que surgiu no rosto do mordomo, e notou também que o vinho
permanecia intocado quando ele pousou a taça na mesa para acompanhá-lo.

Era uma noite tumultuosa e fria de março, típica da estação, com uma lua
pálida que parecia estar deitada de costas pela força do vento, e uma
camada baixa de nuvens flutuantes de textura de cambraia.  O vento
tornava difícil a conversação, e parecia coagular o sangue no rosto. 
Parecia também ter afugentado das ruas os transeuntes, porque Mr.
Utterson não se lembrava de ter visto aquela parte de Londres tão
deserta.  Ele preferiria que não fosse assim; nunca em sua vida tinha
experimentado de forma tão intensa o desejo de ver e tocar seus
semelhantes; porque, mesmo que se esforçasse, crescia cada vez mais
em sua mente a esmagadora premonição de uma calamidade.  Quando
chegaram à praça, o vento fazia girar turbilhões de poeira, e as
árvores esguias do jardim fustigavam com seus galhos as grades do
parque.  Poole, que durante todo o trajeto se mantivera um ou dois
passos à frente, deteve-se na calçada e, a despeito do frio cortante,
tirou o chapéu e enxugou a testa com um lenço vermelho.  Mas apesar de
toda a pressa com que tinham caminhado, o suor que estava enxugando não
era do esforço físico, mas de uma angústia que o sufocava; seu rosto
estava lívido e sua voz, quando falou, estava áspera e vacilante.

-- Bem, senhor, -- disse ele -- aqui estamos, e queira Deus que não haja
nada errado.

-- Amém, Poole -- disse o advogado.

O mordomo bateu à bota de maneira discreta; ouviu-se o ruído da corrente
sendo retirada, e pela fresta que se entreabriu em seguida uma voz
perguntou de dentro:

-- É você, Poole? 

-- Sim, está tudo bem -- disse Poole. -- Abra a porta.

Quando entraram, o saguão estava brilhantemente iluminado; a lareira
ardia em toda sua capacidade; e ali naquele aposento toda criadagem da
casa, homens e mulheres, se amontoava como um rebanho de ovelhas.  Ao
avistar Mr. Utterson, a governanta começou a soluçar histericamente, e
a cozinheira correu como se fosse abraçar-se com ele, gritando:

-- Abençoado seja Deus!  É Mr. Utterson!

-- O que é isso, estão todos aqui? -- disse o advogado, num tom
ranzinza. -- Isso é muito irregular, muito inconveniente. Seu patrão
não vai gostar nem um pouco.

-- Estão com medo -- disse Poole.

Seguiu-se um silêncio total; ninguém protestou, apenas a governanta
começou a chorar cada vez mais alto.

-- Controle-se -- disse-lhe Poole, e o tom de ferocidade na sua voz
mostrava que ele próprio estava a ponto de ter uma crise de nervos. 
Quando a mulher começou a lamentar-se, os criados, em sobressalto,
foram se encaminhando para a porta interna, com uma expressão de
aterrorizada expectativa. 

-- E agora -- disse o mordomo, dirigindo-se ao copeiro -- traga-me uma
vela e vamos esclarecer logo este assunto.

Pedindo a Mr. Utterson que o acompanhasse, ele seguiu na frente rumo aos
aposentos no fundo da casa.

-- Muito bem, senhor, -- disse ele -- procure não fazer barulho.  Quero
que escute, e que ele não ouça sua presença.  E preste atenção, senhor,
se por acaso ele lhe pedir para entrar, não atenda.

Os nervos de Mr. Utterson, diante deste pedido inesperado, produziram um
estremecimento que quase o desequilibrou; mas ele recompôs sua coragem
e seguiu o mordomo até o laboratório, através do anfiteatro cirúrgico,
por entre as pilhas de caixas e de frascos de vidro, até o pé da
escada.  Ali, Poole lhe fez um sinal para que se detivesse e ficasse à
escuta enquanto ele próprio, pousando a vela e fazendo um visível
esforço de vontade, subiu os degraus e bateu, com mão insegura, no
tecido vermelho que forrava a porta do escritório.

-- Mr. Utterson está aqui, senhor, e pede para vê-lo -- disse ele
elevando a voz, e enquanto o fazia repetiu energicamente o gesto para
que o advogado apurasse os ouvidos.

Uma voz respondeu de dentro, num tom lamentoso:

-- Diga-lhe que não posso ver ninguém.

-- Obrigado, senhor -- disse Poole, e havia como que uma nota de triunfo
em sua voz; pegando novamente a vela, ele conduziu Mr. Utterson de
volta através do pátio interno, até a enorme cozinha, onde o fogo
estava apagado e insetos minúsculos se arrastavam pelo piso. 

-- Senhor, -- disse Poole olhando Mr. Utterson nos olhos -- aquela era a
voz do meu patrão?

-- Parece muito mudada -- replicou o advogado, muito pálido, mas
encarando o outro sem titubear.

-- Mudada? Oh, claro, acho que sim -- disse o mordomo. -- Será que eu,
que trabalho há vinte anos na casa de um homem, poderia me enganar
quanto à voz dele?  Não, senhor; meu patrão desapareceu. Alguém deu
sumiço nele há cerca de uma semana, quando o ouvimos gritar chamando
pelo nome de Deus; e \textit{quem} está lá dentro em seu lugar, e
\textit{por que }fica lá dentro, é algo que ofende aos céus, Mr.
Utterson!

-- Esta é uma história estranha, Poole, uma história das mais medonhas,
amigo -- disse Mr. Utterson, mordendo a junta do dedo. -- Suponhamos que
tudo aconteceu como você imagina; suponhamos que o Dr. Jekyll tenha
sido, digamos, assassinado; por que motivo o assassino teria
permanecido lá dentro? Não, a história não se sustenta; não parece nada
razoável.

-- Bem, Mr. Utterson, o senhor é um homem difícil de satisfazer, mas
farei o possível -- disse Poole. -- Durante toda a semana passada, o
senhor deve saber, ele, ou quem quer que esteja trancado naquele
aposento, tem chorado dia e noite implorando por um certo tipo de
remédio, e não consegue produzi-lo da maneira que deseja. Às vezes ele
tinha o hábito, estou me referindo ao meu patrão, de escrever seus
pedidos numa folha de papel e jogá-la na escada. Foi este o único
contato que tivemos com ele durante esta semana; nada mais senão
recados escritos num papel e uma porta trancada.  As refeições têm de
ser deixadas na porta para que ele as leve para dentro quando não há
ninguém olhando.  Pois bem, senhor, todos os dias, e às vezes duas ou
três vezes durante um só dia têm aparecido esses recados, cheios de
reclamações, e lá vou eu voando para todos os farmacêuticos da cidade. 
Cada vez que trago uma substância surge outro papel dizendo-me para
devolvê-la, porque não é pura, e outro pedido endereçado a outra
farmácia.  Ele precisa demais dessa droga, senhor, seja ela o que for.

-- Guardou algum desses papéis? -- perguntou Mr. Utterson.

Poole remexeu no bolso e entregou-lhe uma nota amarfanhada, que o
advogado, inclinando-se mais para perto da vela, examinou detidamente. 
O texto dizia:

“O Dr. Jekyll apresenta seus cumprimentos aos Srs. Maw.  Ele garante que
a última amostra enviada é impura e não pode ser utilizada para o seu
propósito.  No ano de 18\ldots{}, o Dr. J. adquiriu uma grande quantidade
dessa substância a V.Sas.  Agora, ele lhes pede encarecidamente que
procurem com o máximo cuidado, e, no caso de ainda lhes restar alguma
quantidade com o mesmo nível de pureza, que a enviem imediatamente.  O
preço é uma questão irrelevante.  A importância desta droga para o Dr.
J. é fundamental.”

Até este ponto o bilhete exibia uma relativa compostura, mas então, com
um rabisco desordenado da pena, o homem que a escrevera demonstrava a
perda do controle emocional, e completava: “Pelo amor de Deus,
consigam-me um pouco daquela substância anterior!”

-- É um bilhete muito estranho -- disse Mr. Utterson.  E de repente: --
Como essa nota voltou às suas mãos?

-- O homem em Maw ficou muito irritado, senhor, e a jogou de volta para
mim como se fosse algo sujo.

-- Esta é sem dúvida a letra do doutor, não acha? -- disse o advogado.

-- Achei-a parecida -- disse o criado com mau humor; e então, noutro tom
de voz: -- Mas que diferença faz a caligrafia?  Eu o vi!

-- Viu-o? -- repetiu Mr. Utterson. -- E então?

-- Isso mesmo! -- disse Poole. -- Aconteceu assim.  Eu entrei de repente
no anfiteatro, vindo do pátio interno.  Acho que ele tinha saído para
procurar a substância, ou por qualquer outra razão; porque a porta do
escritório estava aberta, e ele estava no lado oposto, remexendo nas
caixas. Ergueu os olhos quando me viu entrar, deu uma espécie de grito,
e subiu as escadas correndo, trancando-se no escritório.  Vi-o apenas
durante um minuto, mas meu cabelo arrepiou-se todo.  Senhor, se aquele
era o meu patrão, por que motivo tinha uma máscara cobrindo o rosto? Se
era meu patrão, por que guinchou como um rato, e fugiu de mim?  Fui seu
criado durante muito tempo.  E agora\ldots{} -- O homem calou-se e passou a
mão sobre o rosto.

-- Estas circunstâncias são todas muito estranhas, -- disse Mr. Utterson
-- mas acho que começo a enxergar um pouco de luz.  Seu patrão, Poole,
está claramente sendo vítima de uma dessas enfermidades que ao mesmo
tempo deformam e fazem sofrer o doente; daí, pelo que posso supor, a
alteração na sua voz; daí a máscara e sua insistência em evitar os
amigos; daí seu desespero em encontrar essa droga, na qual sua pobre
alma deposita suas últimas esperanças de recuperação.  Queira Deus que
ele não se engane!  Esta é minha explicação; algo muito triste, Poole,
e difícil de acreditar; mas algo simples e natural, que faz sentido, e
que nos afasta de temores mais exagerados.

-- Senhor, -- disse o mordomo, cuja palidez se acentuava ao contrastar
com manchas escuras em seu rosto -- aquela coisa não era o meu patrão,
e esta é toda a verdade.  Meu patrão -- e olhando em volta ele começou a
sussurrar -- é um homem alto, de boa compleição, e aquele era quase um
anão. 

Utterson tentou protestar, mas ele prosseguiu:

-- Senhor, acha que não conheço meu patrão depois de vinte anos?  Acha
que não sei onde sua cabeça alcança junto à porta do escritório, onde o
vi todas as manhãs de minha vida?  Não, senhor, aquela coisa sob a
máscara nunca foi o Dr. Jekyll.  Deus sabe do que se trata, mas o Dr.
Jekyll é que não é; e acredito de coração que ocorreu um crime aqui.

-- Poole, -- replicou o advogado -- se você diz isso, tenho a obrigação de
verificar.  Por mais que eu deseje poupar sofrimento ao seu patrão, por
mais que eu esteja perplexo diante desse bilhete, que parece provar que
ele está vivo, considero que minha obrigação é arrombar aquela porta.

-- Ah, Mr. Utterson, assim é que se fala! -- exclamou o mordomo. 

-- E agora vem a segunda questão -- prosseguiu Utterson. -- Quem vai
fazê-lo?

-- Ora, senhor, nós dois -- foi a destemida resposta.

-- Disse muito bem, -- tornou o advogado -- e aconteça o que acontecer, é
meu dever garantir que nenhuma culpa recaia sobre você.

-- No anfiteatro há um machado -- continuou Poole -- e o senhor pode
utilizar um dos atiçadores que há na cozinha.

O advogado foi buscar aquele rude mas maciço instrumento, e o sopesou
nas mãos.

-- Sabe, Poole, -- disse ele, encarando o outro -- que eu e você estamos
a ponto de nos colocarmos numa posição muito arriscada?

-- É possível, senhor, sem nenhuma dúvida -- retornou o mordomo.

-- Então o melhor é falarmos com franqueza -- disse o outro. -- Nós dois
temos na mente algo mais do que dissemos em voz alta; vamos ser
totalmente francos.  O indivíduo mascarado que você viu\ldots{} chegou a
reconhecer quem era ele?

-- Bem, senhor, aconteceu tão rápido, e a criatura estava tão
encurvada, que não me atrevo a jurar -- foi a resposta. -- Mas se o senhor
está me perguntando se era Mr. Hyde, bem, sim, acho que era ele!  Veja
bem, era do mesmo tamanho, e tinha também um passo rápido, leve; além
do mais, que outra pessoa poderia ter entrado pela porta do
laboratório?  Lembra-se decerto, senhor, que na época daquele
assassinato ele ainda tinha consigo uma chave?  Mas isso não é tudo. 
Não sei se já encontrou esse Mr. Hyde alguma vez, Mr. Utterson.

-- Sim -- disse o advogado. -- Conversei com ele, uma vez.

-- Então deve saber tanto quanto o resto de nós que existe algo de
esquisito naquele cavalheiro, algo que incomoda um ser humano\ldots{}  Não
sei exatamente como me expressar, senhor, a não ser assim: que ele nos
faz sentir um gelo bem na medula.

-- Confesso que senti algo como o que você descreve -- disse Mr.
Utterson.

-- Pois assim é, senhor -- disse Poole. -- Bem, quando aquela criatura
mascarada saltou como um macaco por entre os aparelhos de química e
sumiu dentro do escritório, foi como se minha espinha tivesse virado
gelo.  Oh, sei que isso não constitui uma prova, Mr. Utterson; tenho
leitura bastante para ter consciência disso; mas um homem tem seu modo
de sentir as coisas, e dou-lhe minha palavra, sobre a Bíblia, de que
aquele era Mr. Hyde!

-- Ai, ai -- disse o advogado. -- Meus receios me inclinam a pensar da
mesma forma. Temo que o mal tenha criado a ligação entre aqueles dois,
e dessa ligação somente o mal poderá advir.  Sim, creio no que me diz;
creio que o pobre Harry foi assassinado; e creio que seu assassino,
sabe Deus com que propósito, ainda está escondido no escritório da
vítima.  Bem, vingança é o nosso nome a partir de agora!  Chame
Bradshaw.

O lacaio atendeu prontamente, muito pálido e nervoso.

-- Recomponha-se, Bradshaw -- disse o advogado. -- Este suspense, eu sei,
está exigindo muito de vocês todos; mas a nossa intenção é justamente
acabar com ele.  Eu e Poole vamos forçar a entrada no escritório.  Se
tudo correr bem, tenho condições de assumir toda a responsabilidade por
isso.  Enquanto isso, para que nada nos pegue desprevenidos, ou algum
criminoso tente nos escapar, quero que você e outro dos criados rodeiem
o quarteirão, armados de bastões, e fiquem de vigia junto à porta
externa do laboratório.  Dou-lhes dez minutos para que estejam lá a
postos.

Quando Bradshaw saiu, o advogado olhou o relógio.

-- E agora, Poole, é a nossa vez.

Colocando o atiçador de ferro embaixo do braço, ele partiu na direção do
pátio interno.  Nuvens encobriam a lua, e a noite estava escura.  O
vento, que naquele recesso interior dos prédios batia apenas em sopros
ocasionais, fazia tremular a chama da vela enquanto eles caminhavam,
até que entraram no anfiteatro, onde se sentaram em silêncio e à
espera.  Sentiam à sua volta o murmúrio profundo de Londres, mas ali,
perto de onde estavam, o silêncio era interrompido apenas pelo ruído de
passos caminhando de um lado para o outro, por trás da porta do
escritório.

-- Caminha assim o dia inteiro, senhor; -- sussurrou Poole -- ah, e
durante a maior parte da noite.  Só se interrompe quando chega alguma
encomenda do farmacêutico.  Ah, só uma consciência má torna-se assim
tão inimiga do repouso! Ah, senhor, com certeza é o sangue derramado
que motiva cada um desses passos!  Mas preste atenção, chegue mais
perto\ldots{} escute bem, Mr. Utterson, e diga-me\ldots{} este é o passo do
doutor?

Os passos eram leves e peculiares, com um certo ritmo, mesmo com o homem
caminhando devagar; sem dúvida era diferente do passo pesado de Henry
Jekyll, que fazia estalar o assoalho.  Utterson suspirou.

-- Há mais alguma coisa? -- perguntou.

Poole assentiu com a cabeça.

-- Sim, uma vez -- disse. -- Uma vez eu o ouvi chorar.

-- Chorar?  Como é possível? -- disse o advogado, experimentando um novo
calafrio de horror.

-- Chorar como uma mulher, ou como uma alma penada -- disse o mordomo. --
Afastei-me daqui com esse peso no coração, e quase chorei também.

Os dez minutos se esgotaram.  Poole foi buscar o machado, que estava
embaixo de um monte de palha junto às embalagens; a vela foi colocada
sobre a mesa mais próxima, para iluminar a arremetida dos dois; e eles
se aproximaram, prendendo a respiração, daquela porta por trás da qual
os passos incansáveis continuavam a ir e voltar, ir e voltar, no
silêncio da noite.

-- Jekyll! -- gritou Utterson, com voz possante. -- Preciso vê-lo e exijo
que me receba. -- Fez uma pausa, mas não houve resposta. -- Estou lhe
avisando com lealdade.  Estamos cheios de suspeitas e preciso conversar
com você de qualquer maneira -- prosseguiu ele; -- se não for por bem,
será por mal; se não for com o seu consentimento, será pela força
bruta!

-- Utterson, -- disse uma voz lá de dentro -- pelo amor de Deus, tenha
misericórdia!

-- Ah!  Essa não é a voz de Jekyll, é a voz de Hyde! -- gritou Utterson.
-- Vamos botar a porta abaixo, Poole!

Poole alçou o machado por cima do ombro e desferiu um golpe que abalou a
casa inteira, fazendo a porta forrada de vermelho sacudir-se nos
trincos e nas dobradiças.  Um grito gutural, de mero terror animalesco,
soou dentro do escritório.  O machado voltou a se erguer, e mais uma
vez a porta rachou e os caixilhos estremeceram; foram quatro golpes
poderosos, mas a madeira era maciça e as junturas eram de excelente
qualidade; não foi senão no quinto golpe que a fechadura se despedaçou
e a porta se abateu, arrebentada, tombando sobre o tapete do lado de
dentro. 

Os arrombadores, atônitos diante do barulho que tinham feito e do
silêncio que se seguiu, recuaram, olhando para dentro do aposento. 
Diante deles via-se o escritório à luz tranquila das lâmpadas, com um
bom fogo aceso e faiscando na lareira, a chaleira produzindo uma fita
delgada de vapor, uma ou duas gavetas abertas, papéis cuidadosamente
arrumados sobre a escrivaninha, e, perto do fogo, uma bandeja com os
apetrechos para um chá; alguém poderia dizer que era o mais tranquilo
dos aposentos e, se não fosse pelos armários repletos de instrumentos
científicos, o mais comum dos aposentos que havia naquela noite em
Londres.

E bem no meio dele jazia o corpo de um homem, todo contorcido, e ainda
percorrido por espasmos.  Os dois se aproximaram na ponta dos pés,
viraram-no para cima e se depararam com o rosto de Edward Hyde.  O
corpo vestia roupas grandes demais para ele, roupas de um tamanho mais
adequado ao médico; os músculos de seu rosto ainda se moviam num
arremedo de vida, mas a vida já se fora; e pelo vidro estilhaçado que
havia em sua mão e o forte cheiro de amêndoas que pairava no ar,
Utterson percebeu que estava contemplando o corpo de um homem que
matara a si mesmo.

-- Chegamos muito tarde -- disse ele, com voz soturna, -- tanto para
salvar quanto para punir.  Hyde escapou de nós por seus próprios meios;
tudo que nos resta a fazer é encontrar o corpo do seu patrão.

A maior parte da casa era ocupada pelo anfiteatro, que preenchia quase
todo o andar térreo e era iluminado pelo alto; e pelo escritório, que
formava um andar superior a um canto dele, e dava para o pátio externo
da rua dos fundos.  Havia um corredor ligando o anfiteatro à porta
daquele pátio; e o escritório era ligado a ele por uma diferente
escada.  Havia também um certo número de armários e um porão bastante
amplo. Todos estes espaços foram examinados minuciosamente.  Para os
armários bastava um olhar, pois estavam vazios, e todos, pela poeira
que caía das portas ao serem abertos, trancados há muito tempo.  O
porão, sem dúvida, estava todo atravancado com pedaços de móveis, a
maioria deles datando do tempo do cirurgião, proprietário da casa antes
de passá-la para o Dr. Jekyll; mas assim que abriram a porta eles
perceberam a inutilidade de procurar ali, pela presença de espessas
teias de aranha que durante anos tinham sido tecidas na parte interna
daquela passagem.  Em nenhum outro lugar encontraram traços de Henry
Jekyll, morto ou vivo.

Poole bateu forte com os pés nas lajes do corredor.

-- Deve estar enterrado aqui -- disse ele, examinando os sons produzidos
por seus pés.

-- Ou pode ter fugido -- disse Utterson, virando-se para examinar a porta
que dava para a rua de trás.  Estava trancada; e, caída sobre as lajes,
eles encontraram a chave, já com manchas de ferrugem.

-- Não tem a aparência de ter sido usada -- observou o advogado.

-- Usada! -- ecoou Poole. -- Não vê que está quebrada, senhor?  Como se
um homem tivesse pisado nela com força.

-- Sim, -- disse Utterson -- e as fraturas, também, estão enferrujadas. --
Os dois homens se entreolharam, com uma expressão de medo. 

-- Isso está além da minha compreensão, Poole -- disse o advogado. --
Vamos voltar ao escritório.

Subiram a escada em silêncio, e, ainda dando uma olhada de vez em quando
na direção do cadáver, passaram a examinar de modo mais detalhado o
aposento.  Numa mesa, havia muitos traços de experiências químicas,
várias pequenas quantidades de um sal branco colocadas em diferentes
frascos de vidro, como se fossem preparativos para uma experiência que
o infeliz cientista não chegara a concluir.

-- É a mesma droga que ele estava sempre encomendando -- disse Poole; e
no momento em que falava a chaleira começou a ferver e a transbordar.

Isso conduziu os dois para perto do fogo, onde uma poltrona havia
sido colocada cuidadosamente, tendo ao lado o serviço de chá devidamente
posto, com o açúcar já colocado na xícara.  Havia uma estante cheia de
livros; um deles estava aberto ao lado da bandeja, e Utterson ficou
perplexo ao descobrir que se tratava de uma obra religiosa pela qual
Jekyll tinha manifestado mais de uma vez sua estima; o livro estava
coberto de anotações, com a letra do médico, com surpreendentes
blasfêmias.

Em seguida, no curso do seu exame, os dois se aproximaram do grande
espelho, que contemplaram com involuntário horror.  Mas o espelho, que
era montado sobre gonzos de modo a girar verticalmente sobre si
próprio, estava apontado para o teto, mostrando nada mais do que o
brilho rosado das chamas bruxuleando no teto, as mil cintilações
criadas pelo fogo ao longo dos armários envidraçados, e os seus
próprios rostos, pálidos e temerosos, debruçando-se para olhar.

-- Este espelho deve ter visto algumas coisas estranhas, senhor --
sussurrou Poole.

-- E com certeza nenhuma mais estranha do que ele próprio -- respondeu o
advogado no mesmo tom. -- Porque vejamos, por que motivo Jekyll\ldots{} -- ele
se interrompeu com um sobressalto ao dizer esta palavra, mas logo se
recompôs desta fraqueza -- \ldots{}para que Jekyll precisaria dele aqui?   %conferir espaço entre travessão e reticências

-- Bem observado -- disse Poole.

Voltaram-se então para a escrivaninha.  Por entre os papéis
cuidadosamente arrumados sobre ela, destacava-se um grande envelope,
que ostentava, na caligrafia do doutor, o nome de Mr. Utterson.  O
advogado o abriu, e algumas folhas de papel caíram no chão.  A primeira
era um testamento, nos mesmos termos excêntricos daquele que Utterson
devolvera ao doutor seis meses antes; exprimia a vontade do signatário
sobre o destino dos seus bens em caso de morte ou desaparecimento; mas
no lugar do nome de Edward Hyde, o advogado leu, para seu indescritível
pasmo, o nome de Gabriel John Utterson.  Ele olhou para Poole, olhou de
novo para o papel, e por fim para o cadáver do delinquente estendido
sobre o tapete.

-- Minha cabeça está girando -- disse. -- Hyde ficou trancado neste
aposento vários dias; não tinha nenhum motivo para simpatizar comigo;
deve ter ficado furioso quando foi deserdado, e ainda assim não
destruiu este documento. 

Ele pegou em seguida o próximo documento; era um curto bilhete com a
letra do doutor, e tendo na primeira linha a data daquele mesmo dia.

-- Poole! -- exclamou o advogado. -- Ele ainda estava vivo, e aqui neste
local, hoje!  Alguém não poderia fazer sumir o seu corpo num tempo tão
curto; ele deve estar vivo, deve ter fugido!  E sendo assim, por que
fugiu?  E como?  E neste caso, ousaremos considerar isto um suicídio? 
Ah, temos que ter muito cuidado.  Temo que ainda acabemos envolvendo
seu patrão em alguma desgraça sem tamanho.

-- Por que não lê logo, senhor? -- perguntou Poole.

-- Porque estou com medo -- disse o advogado solenemente. -- Queira Deus
que não tenha motivos!  -- E com isto ele trouxe o papel mais para
perto dos olhos e leu:

\begin{quote}
\textsc{meu caro utterson}. Quando estes documentos caírem em suas mãos, eu terei
desaparecido, em circunstâncias que não me é dado antever; mas meu
instinto e todos os outros aspectos de minha infeliz condição me dizem
que meu fim é certo e não deve demorar.  Prossiga; leia primeiro a
narrativa feita por Lanyon, o qual me preveniu de que iria colocá-la em
suas mãos.  E se ainda fizer questão de saber mais, leia a confissão do

{\raggedleft Seu indigno e infeliz amigo\\
\textsc{Henry Jekyll} \par}
\end{quote}

-- Há um terceiro documento? -- perguntou Utterson. 

-- Aqui, senhor -- disse Poole, entregando-lhe um maço de documentos
fechado e selado em vários lugares.

O advogado o guardou no bolso, dizendo:

-- Prefiro não falar nada a respeito destes papéis.  Se o seu patrão
está morto ou se fugiu, podemos pelo menos preservar sua reputação. 
Bem, já passam das dez horas; preciso ir para casa e ler estes
documentos com calma; mas voltarei antes da meia-noite, quando
chamaremos a polícia.

Saíram e fecharam a porta do anfiteatro atrás de si; e Utterson,
deixando os criados mais uma vez aglomerados no saguão, retornou ao seu
próprio escritório para ler as duas narrativas que poderiam agora
solucionar aquele mistério.


\chapter[9 -- A narrativa do Dr. Lanyon]{9\break A narrativa do Dr. Lanyon}

No dia 9 de janeiro, ou seja, quatro dias atrás, recebi pelo correio
vespertino um envelope registrado, endereçado com a letra do meu velho
amigo e colega de estudos, Henry Jekyll.  Fiquei bastante surpreso,
porque não tínhamos o hábito de nos correspondermos; eu o encontrara,
jantara com ele, na verdade, na noite anterior; e não podia imaginar
nada em nossas relações que justificasse a chegada de uma comunicação
assim tão formal.  O conteúdo da missiva aumentou minha perplexidade,
porque eis o que dizia ela:

10 de dezembro, 18\ldots{}\footnote{ Existe aqui uma aparente incoerência no original de Stevenson.  Sabemos
que no dia 8 de janeiro Utterson e Lanyon cearam na casa do Dr. Jekyll,
onde tudo parecia normal.  Somente a partir do dia 12 Jekyll recusou-se
a recebê-lo, e na sexta noite (portanto no dia 17) Utterson visitou
Lanyon, que a esta altura estava devastado pela revelação do segredo
de Jekyll.  A carta deste, portanto, deveria ser datada de 9
de janeiro, conforme o próprio Lanyon deixa claro na primeira linha, e
não 10 de dezembro. [\textsc{n.t.}]}

\begin{quote}
Caro Lanyon:

Você é um dos meus amigos mais antigos, e, embora tenhamos divergido de
vez em quando em assuntos científicos, não posso lembrar, pelo menos do
meu ponto de vista, nenhuma quebra na nossa afeição. Nunca houve um dia
em que, se você me dissesse “Jekyll, minha vida, minha honra, minha
razão estão dependendo de uma atitude sua”, eu não tivesse sacrificado
minha mão esquerda para ajudá-lo.  Lanyon: agora, minha vida, minha
honra e minha razão estão dependendo de sua misericórdia; se você me
faltar esta noite, estou perdido.  Você pode imaginar, após esta
introdução, que irei lhe pedir um favor que implica em desonra de sua
parte.  Julgue por você mesmo.

Peço-lhe que adie todos os compromissos que tiver para esta noite, mesmo
que tenha sido convocado à cabeceira de um imperador!  Pegue um
cabriolé, a menos que sua charrete já esteja pronta diante da porta; e,
com esta carta em seu poder para o caso de precisar consultá-la, venha
direto para minha casa.  Poole, meu mordomo, já recebeu minhas
instruções; você o encontrará à sua espera, tendo consigo um
serralheiro.  Vocês deverão arrombar a porta do meu escritório; você
deverá entrar nele, sozinho; deve abrir o armário envidraçado marcado
com a letra \textsc{e}, do lado esquerdo, quebrando o cadeado, se for o caso, e
retirar dali, com todo o seu conteúdo, do jeito que se encontra, a
quarta gaveta a contar de cima para baixo ou (o que dá no mesmo) a
terceira de baixo para cima.  Tenho um receio mórbido de me equivocar
nestas instruções, mas, mesmo que eu me equivoque, você saberá que é a
gaveta certa pelo que ela contém: alguns preparados em pó, um frasco e
um caderno de notas.  Peço-lhe que retire essa gaveta e a leve de volta
consigo para sua casa em Cavendish Square, exatamente como ela se
encontra.

Esta é a primeira parte de sua tarefa; agora vamos à segunda.  Você deve
estar de volta a sua casa (caso entre em ação imediatamente após o
recebimento desta) bem antes da meia-noite; mas eu lhe darei esta
margem de segurança, não apenas por conta de algum desses obstáculos
que não podem ser evitados nem previstos, mas porque para a conclusão
desta missão é preferível um horário em que seus criados já tenham se
recolhido.  À meia-noite, então, peço-lhe que esteja a sós na sala que
lhe serve de consultório, para permitir ali a entrada de um homem que
irá se apresentar como meu enviado, e passar às mãos dele a gaveta que
você trouxe do meu escritório.  Feito isso, você terá cumprido seu
papel e merece toda a minha gratidão.  Cinco minutos depois, caso você
precise de uma explicação, poderá compreender por que motivo essas
medidas são de importância vital; e que negligenciar qualquer uma
delas, por mais fantásticas que pareçam, implicará em colocar sobre sua
consciência o peso da minha morte ou do naufrágio da minha razão.

Embora confie que você não irá tratar com leviandade este meu apelo, meu
coração afunda e minha mão treme ao pensar na simples possibilidade de
que isso aconteça.  Pense em mim, nesta hora, num lugar estranho,
debatendo-me no negror de um desespero que nenhuma imaginação é capaz
de exagerar, e ainda assim cônscio de que, se você atender meu pedido
com exatidão, meus problemas desaparecerão como uma história que chega
ao seu fim.  Ajude-me, meu caro Lanyon, e salve

{\raggedleft seu amigo\\
\textsc{h.j.} \par}

\textsc{ps} -- Já havia selado esta carta quando uma ideia encheu novamente de
terror a minha alma.  É possível que o correio se atrase por algum
motivo, e que esta carta não chegue às suas mãos senão amanhã de manhã.
 Neste caso, caro amigo, execute a tarefa na hora que lhe for mais
conveniente no transcurso do dia; e, mais uma vez, fique à espera do
meu mensageiro por volta da meia-noite.  Talvez já seja então tarde
demais; e se essa noite transcorrer sem que nada aconteça, você saberá
que nunca mais irá pôr os olhos em Henry Jekyll.
\end{quote}

Ao concluir a leitura desta carta, tive certeza de que meu colega se
encontrava insano; mas enquanto isto não pudesse ser provado sem
possibilidade de dúvida, senti-me obrigado a proceder como ele me
pedia.  Quanto menos eu soubesse a respeito dessa confusão, menos
estaria em condições de avaliar sua importância; e um pedido feito
naqueles termos não podia ser posto de lado sem acarretar com isso uma
grave responsabilidade.  Assim, ergui-me da mesa, chamei um cabriolé e
fui direto para a casa de Jekyll.  O mordomo estava à minha espera;
tinha recebido pelo correio, como eu, uma carta registrada contendo
instruções, e imediatamente mandara chamar um serralheiro e um
carpinteiro.  Os dois chegaram enquanto conversávamos, e nos
encaminhamos todos para o velho anfiteatro cirúrgico do Dr. Denman, que
é (como você sem dúvida sabe) o acesso preferencial ao escritório de
Jekyll. 

A porta era muito resistente, o cadeado era dos mais fortes; o
carpinteiro nos avisou que iria ter muito trabalho e causar grandes
avarias, se tivéssemos mesmo que entrar ali pela força; e o serralheiro
estava quase desesperado.  Mas este último era um sujeito habilidoso, e
depois de duas horas de esforço a porta foi aberta.  O armário com a
letra \textsc{e} estava destrancado; retirei a gaveta, preenchi seu conteúdo com
palha, envolvi-a num pano, bem amarrado, e trouxe-a comigo para
Cavendish Square.

Uma vez aqui, passei a examinar o que a gaveta continha.  Os preparados
em pó estavam acondicionados corretamente, mas não com os cuidados
característicos de um farmacêutico profissional, e isto para mim
deixava claro que tinham sido produzidos pelo próprio Jekyll.  Quando
abri um dos pacotinhos, tudo que vi foi um sal cristalino de cor
branca.  O frasco, para o qual logo dirigi minha atenção, estava cheio
pela metade de um líquido vermelho, cor de sangue, de cheiro pungente e
que me pareceu conter fósforo, além de um tipo de éter volátil.  Quanto
aos seus outros ingredientes não faço ideia do que fossem.  O livro era
uma espécie comum de caderneta escolar em branco, para anotações, e
continha apenas uma série de datas.  Estas cobriam um período de muitos
anos, mas observei que a lista se interrompia, de maneira abrupta, em
torno de um ano atrás.  Aqui e ali havia uma curta observação junto a
alguma data, em geral não mais do que uma palavra, “duplo”, ocorrendo
talvez seis vezes num total de várias centenas de registros; e, apenas
uma vez, bem no começo da lista, seguido por vários pontos de
exclamação, o comentário “fracasso total!!!”.  Tudo isso, embora
aguçasse minha curiosidade, pouco me esclareceu.  Ali
estava um frasco com uma substância qualquer, e os registros de uma
série de experiências que não tinham conduzido (como tantas outras
experiências de Jekyll) a qualquer resultado prático.  Como poderia a
presença daqueles itens em minha casa afetar a honra, a sanidade mental
e a própria vida de meu errático amigo?  Se o tal mensageiro podia ir
até um local, por que não podia ir ao outro?  E mesmo admitindo a
existência de um obstáculo qualquer, por que esse cavalheiro tinha que
ser recebido por mim sob completo segredo?  Quanto mais eu refletia
mais me convencia de que se tratava de um caso de doença mental; e
embora eu desse ordens aos meus criados para que se recolhessem,
coloquei algumas balas num velho revólver, para que em caso de
necessidade eu tivesse como me defender.

Mal havia soado a meia-noite em Londres quando a aldraba soou de leve à
minha porta.  Fui abrir pessoalmente, e me deparei com um homem
baixinho, encolhido junto às pilastras do pórtico.

-- Foi o Dr. Jekyll quem o enviou? -- perguntei.

Ele respondeu que sim com um gesto contraído, e, quando o convidei a
entrar, não me atendeu sem antes olhar por sobre o ombro para a praça,
que estava envolta na escuridão.  Havia um policial não muito longe,
avançando com a lanterna acesa, e tive a impressão de que ao avistá-lo
o meu visitante teve um sobressalto e apressou-se a entrar.

Estes detalhes chamaram minha atenção, confesso, de maneira
desagradável; e quando o acompanhei até o meu consultório, que estava
com todas as luzes acesas, mantive minha mão na arma, dentro do bolso. 
Lá dentro, pelo menos, tive a possibilidade de examiná-lo melhor. 
Nunca o vira antes, disso tive certeza.  Era um homem pequeno, como já
falei; tive um outro choque ao observar a expressão do seu rosto, que
tinha uma notável combinação de grande atividade muscular e uma
aparente debilidade de constituição; e, por último, mas não o menos
importante, percebi o estranho desconforto íntimo que me produzia a sua
proximidade.  Esse desconforto assemelhava-se um pouco a uma rigidez
muscular, e era acompanhado por uma desaceleração do pulso.  Naquela
hora, eu o atribuí a algum tipo de aversão instintiva e pessoal, e me
admirei apenas diante da intensidade dos sintomas; mas desde então tive
motivos para crer que as causas estão localizadas em algo mais
profundo na natureza humana, e estão ligadas a um aspecto mais nobre do
que a mera noção de ódio.

Essa pessoa (que até então, desde o momento em que chegara, tinha me
despertado o que só posso descrever como uma curiosidade onda de
antipatia) vestia-se de uma maneira que teria provocado hilaridade, se
fosse em uma pessoa comum; ou seja, suas roupas, embora fossem de um
tecido caro, eram grandes demais para ele sob todos os aspectos -- as
calças eram muito frouxas e tinham a barra enrolada para cima, para não
arrastar no chão; a cintura do casaco caía-lhe abaixo dos quadris, e o
colarinho se espalhava sobre os ombros.  É estranho dizê-lo, mas aquele
arranjo ridículo estava longe de me provocar risadas.  Na verdade, uma
vez que existia algo de anormal e desabonador na própria essência da
criatura que agora me encarava, algo que me invadia, me assustava e me
provocava revolta, aquela disparidade imprevista parecia se encaixar
nessa impressão e reforçá-la; e assim somava-se, ao meu interesse
quanto à natureza e ao caráter daquele indivíduo, uma curiosidade
quanto à sua origem, sua vida, suas posses e sua situação social.

Estas impressões, embora precisem ser tão longamente descritas,
impuseram-se ao meu espírito numa questão de segundos.  Meu visitante
estava, sem dúvida, tomado de uma excitação ameaçadora. 

-- Conseguiu? -- exclamou ele. -- Conseguiu? 

Tal era sua impaciência que ele chegou a agarrar meu braço e sacudi-lo. 
Afastei-o de mim, enquanto sentia como resultado do seu toque uma
pontada gélida se espalhar pelo meu sangue.

-- Venha, cavalheiro -- falei. -- Esquece-se de que ainda não tive o
prazer de nos apresentarmos.  Por favor, sente-se.

Dei-lhe o exemplo acomodando-me na minha cadeira, procurando seguir meu
comportamento habitual ao receber um paciente, tanto quanto me era
possível visto o adiantado da hora, a natureza de minhas preocupações e
o horror que despertava o visitante.

-- Peço desculpas, Dr. Lanyon -- respondeu ele, com bastante cortesia. --
O que diz é mais do que razoável, e minha impaciência acabou tomando a
frente da minha polidez.  Venho aqui a pedido do seu colega, Dr. Henry
Jekyll, para tratar de um assunto de certa importância; e pelo quem
compreendi\ldots{} -- Ele fez uma pausa, levou a mão à garganta, e pude
perceber que, a despeito do controle que mantinha sobre seus gestos,
estava a ponto de ter um ataque histérico.\footnote{ A medicina
dos tempos vitorianos via a histeria como uma doença nervosa
que tinha início na garganta. [\textsc{n.t.}]} --
Pelo que compreendi, há uma gaveta\ldots{}

A essa altura apiedei-me da angústia do meu visitante, e cedi talvez à
minha própria curiosidade.

-- Aqui está ela, senhor -- falei, apontando a gaveta, que estava no
chão, por trás de uma mesa, e ainda coberta pelo lençol.

Ele deu um salto naquela direção, mas logo se deteve, com a mão pousada
sobre o coração; ouvi seus dentes rangendo com o movimento convulsivo
de suas mandíbulas, e seu rosto estava tão lívido que cheguei a temer
pela sua vida e pela sua razão.

-- Controle-se -- pedi.

Ele me endereçou um sorriso tenebroso, e com um gesto brusco que
indicava seu desespero, arrancou o lençol.  Ao ver o conteúdo da
gaveta, soltou um soluço de tamanho alívio que me quedei petrificado. 
No instante seguinte, com uma voz que parecia prestes a fugir-lhe ao
controle, perguntou:

-- Tem um tubo de ensaio graduado?

Ergui-me da cadeira com algum esforço e fui buscar o que ele me pedira.

Ele me agradeceu com um aceno e um sorriso, derramou no tubo uma pequena
quantidade do líquido vermelho, e despejou ali um dos preparados em pó.
 À medida que os cristais se dissolviam, o líquido foi tomando uma cor
mais clara, enquanto se elevava dele uma audível efervescência e um
pouco de vapor.  De repente essa ebulição cessou por completo, enquanto
assumia uma tonalidade de púrpura escuro, que foi aos poucos mudando
para um verde aquoso.  Meu visitante, que observava essas metamorfoses
com olho atento, sorriu, pousou o tubo sobre a mesa, e então virou-se
para me encarar com um olhar inquisitivo.

-- E agora -- disse ele -- vamos para o desfecho.  O senhor será sensato?
 Terá juízo?  Aceitará que eu tome este frasco em minhas mãos e vá
embora de sua casa sem mais delongas?  Ou será que é dominado pela
avidez da curiosidade?  Pense bem antes de responder, porque farei o
que me disser.  Depois de tomar sua decisão, poderá continuar tal como
era antes: nem mais rico nem mais sábio, a não ser que a sensação de um
serviço prestado a um homem em desespero mortal possa ser considerada
uma forma de enriquecimento da alma.  Ou, se assim escolher, poderá
abrir para si próprio uma nova fronteira do conhecimento e novas
avenidas que o conduzirão à fama e ao poder, aqui mesmo, neste
aposento, neste instante; porque seus olhos contemplarão um prodígio
capaz de abalar a descrença do próprio Satã.

-- Cavalheiro! -- exclamei, aparentando uma calma que estava longe de
sentir. -- O senhor fala por enigmas, e não deve se admirar de que eu o
escute sem excesso de credulidade.  Mas já avancei demais ao longo de
uma situação tão inexplicável para deter-me antes de ver o fim.

-- Então muito bem -- disse o meu visitante. -- Lanyon, lembre-se do seu
juramento: o que vai acontecer agora está protegido pelo segredo da
nossa profissão.  E agora, você, que sempre foi apegado a uma visão do
tipo mais estreito e materialista, você que negava as virtudes da
medicina transcendental, você que sempre zombou dos que lhe eram
superiores\ldots{} contemple!

Ele pôs o frasco nos lábios e sorveu seu conteúdo com um único gole. 
Ouviu-se um grito; o homem oscilou, cambaleou, e agarrou-se à mesa para
manter-se de pé, olhando-me com olhos esbugalhados, e a boca aberta,
arquejante; e enquanto eu o observava julguei perceber uma mudança --
ele pareceu inchar -- seu rosto escureceu e suas feições pareceram
derreter-se, alterar-se, e no momento seguinte eu tinha ficado de pé e
dado um pulo para trás, de encontro à parede, com os braços erguidos
para me proteger daquele prodígio, e minha mente engolfada pelo terror.

-- Oh meu Deus! -- gritei, e outra vez, e mais outra; porque ali, diante
dos meus olhos, pálido e trêmulo, quase desmaiado, e tateando diante de
si com mãos incertas, como um homem arrancado à morte, ali estava
Henry Jekyll!  

O que ele me contou durante a hora seguinte eu não me atrevo a passar
para o papel.  Vi o que vi, ouvi o que ouvi, e isto envenenou minha
alma; e ainda assim, agora, depois que aquela visão se esvaiu dos meus
olhos, pergunto a mim mesmo se acredito no que vi, e não consigo
responder.  Minha vida está abalada até as mais fundas raízes; não
consigo mais dormir; a todas as horas do dia ou da noite sou
acompanhado pelo terror mais mortal; sinto que estou com os dias
contados, e que não tardarei a morrer; e ainda assim morrerei sem
acreditar.  Quanto às torpezas morais que aquele homem, mesmo com
lágrimas de penitência, me revelou, não posso sequer pensar nelas sem
um estremecimento de horror.  Direi apenas uma coisa, Utterson, e esta
(se você conseguir acreditar em mim) será mais do que suficiente.  A
criatura que deslizou para dentro de minha casa naquela noite era, pela
confissão do próprio Jekyll, conhecida pelo nome de Hyde e estava sendo
caçada nos quatro cantos deste país pelo assassinato de Carew.

\textsc{hastie lanyon}.


\chapter[10 -- A confissão completa do Dr. Jekyll]{10\break A confissão completa do Dr. Jekyll}

Nasci no ano de 18\ldots{}, numa família de grande fortuna, dotado de
talentos consideráveis, com uma tendência natural para o trabalho,
afeiçoado ao respeito dos meus concidadãos mais sábios e de melhor
caráter, e deste modo, como é fácil supor, com todas as garantias de um
futuro honrado e brilhante.  E no entanto o meu defeito mais grave era
uma certa impaciência para desfrutar os prazeres da vida, algo que
trouxe a felicidade a muitos, mas que me foi difícil conciliar com
minha firme vontade de caminhar de cabeça erguida e de apresentar ao
mundo uma imagem mais respeitável que a da maioria dos homens.  O que
resultou disto foi que passei a dissimular esses meus prazeres; e
quando cheguei à idade da razão e comecei a olhar em torno e avaliar
meu progresso e minha posição no mundo, já estava há muito comprometido
com essa minha profunda duplicidade íntima.  Muitos homens poderiam até
vangloriar-se dessas irregularidades das quais eu me sentia culpado,
mas, em função dos altos parâmetros que eu tinha estabelecido para mim
mesmo, tive que encará-los e escondê-los com uma sensação quase mórbida
de vergonha.  Foi, portanto, a natureza exigente das minhas aspirações,
mais do que qualquer degradação específica decorrente dos meus
defeitos, que me fez ser aquilo em que me tornei, e, criando uma
divisão ainda mais profunda do que na maioria dos homens, afastou de
mim a parte sã e a parte doentia que dividem e formam a natureza dual
do ser humano.  No meu caso, eu era levado a meditar profunda e
reiteradamente sobre a dura lei da vida que está na raiz de todas as
religiões e que é uma das fontes mais intensas de desassossego.  Embora
mantendo essa dupla face, eu não era de modo algum um hipócrita; ambos
os lados do meu ser eram extremamente sinceros; eu não era mais eu
mesmo quando abandonava o autocontrole e me entregava à depravação do
que quando trabalhava, à luz do dia, para aumentar o conhecimento
humano ou para aliviar a dor e o sofrimento alheios.  E quis a sorte
que o rumo dos meus estudos científicos, dirigidos para tudo que  é
místico e transcendental, acabasse lançando uma poderosa luz nesta
minha consciência sobre a eterna guerra entre os elementos que me
compõem.  A cada dia, e de ambos os lados da minha inteligência, o lado
moral e o lado intelectual, eu me aproximava mais dessa verdade, cuja
descoberta parcial me conduziu à presente catástrofe: a descoberta de
que o homem na verdade não é um, mas dois.  Digo dois, porque o
presente estado  dos meus conhecimentos não vai além desse ponto. 
Outros me seguirão, outros irão me ultrapassar nesse caminho; e eu
arrisco a suposição de que o homem acabará sendo reconhecido como uma
assembleia de inquilinos múltiplos, incongruentes e autônomos. Eu, de
minha parte, pela própria natureza da minha vida, avancei
inexoravelmente em uma direção, e apenas uma.  Foi pelo lado moral, e
na minha própria pessoa, que aprendi a reconhecer a profunda e
primitiva dualidade do homem; percebi que, das duas naturezas que se
digladiavam no campo da minha consciência, mesmo que eu pudesse me
identificar com qualquer uma delas, era apenas porque eu era
radicalmente as duas; e desde muito cedo, antes mesmo de que o curso
das minhas descobertas científicas tivesse começado a sugerir a
possibilidade nítida de um tal milagre, eu já aprendera a meditar com
deleite, como num devaneio prazeroso, na possibilidade de separação
desses elementos.  Se cada um deles (eu dizia a mim mesmo) pudesse
ficar alojado numa diferente identidade, a vida ficaria livre de tudo
que nos é insuportável; o injusto iria numa direção, libertado das
aspirações e dos remorsos de seu gêmeo mais íntegro; e o justo poderia
trilhar à vontade e com segurança o seu caminho ascendente, praticando
as boas ações que lhe dão prazer, agora sem se ver exposto à desgraça e
à penitência pelas ações desse mal que lhe é estranho.  A maldição que
caiu sobre a humanidade foi que esses feixes de incongruências
acabassem sendo amarrados uns aos outros, de tal modo que no ventre
torturado de nossa consciência esses gêmeos opostos estejam em perpétua
luta.  Como, então, foi possível separá-los?

Eu estava neste ponto de minhas reflexões quando, como já falei, uma luz
lateral começou a brilhar sobre o meu objeto de estudos, na mesa do
laboratório.  Comecei a perceber, mais profundamente do que alguém
jamais afirmou fazê-lo, a trêmula imaterialidade, a transitoriedade de
névoa deste corpo aparentemente tão sólido que nos serve de vestimenta.
 Descobri certos agentes químicos capazes de abalar e arrancar das
raízes esta nossa roupagem de carne e osso, tal como uma ventania
arrebata uma tenda.  Por dois bons motivos não descerei a muitos
detalhes científicos da minha confissão.  Primeiro, porque aprendi às
minhas próprias custas que a desgraça e o fardo da existência estão
pousados para sempre nos nossos ombros, e quando tentamos nos ver
livres deles o seu peso volta a nos oprimir com uma pressão inaudita e
ainda mais terrível.  Segundo, porque, como minha narrativa tornará
bastante claro, ai de mim! -- minhas descobertas ficaram incompletas. 
Basta-me dizer, portanto, que não apenas percebi que meu corpo material
não passava da simples aura e efulgência de certas forças que
constituem meu espírito, como também consegui sintetizar uma droga
através da qual essas forças perdiam sua supremacia, e uma segunda
forma, ou aparência física, era capaz de surgir, um outro corpo que não
me era menos natural, por ser ele a expressão fiel dos elementos mais
inferiores de minha alma.

Hesitei muito antes de submeter minha teoria ao teste da prática.  Sabia
que estava correndo um perigo mortal; porque qualquer droga capaz de
controlar e abalar de forma tão violenta a própria fortaleza da nossa
identidade, poderia, por alguma mínima diferença, produzir uma
overdose, ou, por qualquer inadequação no momento da experiência,
destruir por completo o tabernáculo imaterial que era meu propósito
alterar.  Eu já havia há muito tempo preparado a minha tintura; comprei
de uma vez só, a uma empresa de ingredientes químicos por atacado, uma
grande quantidade de um sal específico que eu sabia, pelas minhas
experiências, ser o último ingrediente necessário; e, na madrugada de
uma noite para sempre maldita, misturei estes elementos, vi-os
fervilhar e fumaçar juntos no frasco, e, quando a ebulição amainou, com
um estranho impulso de coragem bebi toda a poção.

Seguiram-se as dores mais excruciantes, um rangido nos ossos, uma náusea
mortal, e na minha alma um horror que não pode ser excedido, seja no
instante do nascimento ou no da morte.  Aos poucos essas agonias foram
se atenuando, e voltei a mim, como quem desperta de uma grave doença. 
Havia algo de estranho nas minhas sensações, algo indescritivelmente
novo e, por sua própria novidade, incrivelmente prazeroso.  Eu me
sentia mais jovem, mais leve, mais feliz em meu próprio corpo; e
experimentava uma inquietação inebriante, uma corrente desordenada de
imagens sensuais, canalizadas numa torrente poderosa em minha
imaginação, junto a uma dissolução da noção de dever, uma liberdade
desconhecida, mas não inocente, de todo o meu espírito.  Percebi de
imediato, com o primeiro sopro daquela nova vida, que agora era mais
perverso, dez vezes mais perverso, vendido como um escravo à minha
própria maldade; e este pensamento me arrebatou e me embriagou como se
fosse vinho.  Estendi as mãos, exultante com o frescor daquelas novas
sensações; e com esse gesto percebi de súbito que tinha diminuído em
estatura.

Não havia espelho no aposento, naquela época; este que agora está diante
de mim enquanto escrevo foi trazido para cá bem depois, com o propósito
de acompanhar estas transformações.  A noite, porém, já avançava
madrugada adentro, e a madrugada estava pronta para ceder a vez ao dia;
os criados da minha casa estavam profundamente mergulhados no sono; e,
arrebatado de esperanças e triunfo, arrisquei-me a ir, na minha nova
forma, até o meu quarto de dormir.  Cruzei o pátio, sob o olhar das
constelações, e pensei, com espanto, que eu era a primeira criatura
desta espécie que elas chegaram a contemplar em sua vigília insone;
entrei furtivamente pelos corredores, um estranho em minha própria
casa, e chegando ao quarto vi pela primeira vez a aparência de Edward
Hyde.

Devo falar aqui de um ponto de vista apenas teórico, afirmando não o que
de fato sei, mas o que suponho ser mais provável.  O lado mau da minha
natureza, para o qual eu acabava de transferir o poder de produzir a
própria imagem, era menos robusto e menos desenvolvido do que o lado
bom, do qual eu acabara de retirar o poder. Do mesmo modo, no
transcorrer de minha vida, que tinha sido, apesar de tudo, em noventa
por cento uma vida de esforço, virtude e autocontrole, ele tinha sido
muito menos exercitado e muito menos desgastado.  Daí, creio eu, o fato
de que Edward Hyde era bastante menor, mais leve e mais jovem do que
Henry Jekyll.  E assim como o bem reluzia na fisionomia de um, o mal
estava escrito de modo claro e inequívoco no rosto do outro.  O mal,
além disso (que eu ainda creio ser o lado letal de todo ser humano),
imprimira sobre aquele corpo uma aura de deformação e de decadência.  E
no entanto quando eu contemplava aquele feio ídolo no espelho, não
experimentava nenhum tipo de repugnância, e sim um impulso de
boas-vindas.  Porque aquele, também, era eu mesmo.  Parecia-me natural
e humano.  Aos meus olhos encarnava uma imagem mais vívida do meu
espírito, parecia-me mais precisa e mais única do que a aparência
imperfeita e dividida que eu até então me acostumara a considerar
minha.  E neste aspecto eu estava certo, sem dúvida.  Vim a perceber
que quando eu assumia a aparência de Hyde, ninguém era capaz de se
aproximar de mim sem experimentar uma visível repulsa física.  Isto,
imagino, se devia ao fato de que todos os seres humanos que conhecemos
são um misto do bem e do mal: e Edward Hyde era o único nas fileiras da
humanidade a ser feito do mal em estado puro.

Demorei-me alguns momentos diante do espelho; a segunda e conclusiva
experiência ainda tinha que ser tentada; eu ainda não podia saber se
perdera minha identidade para além de qualquer esperança de redenção, e
se deveria fugir, antes da alvorada, daquela casa que não era mais
minha; correndo de volta ao escritório, preparei e bebi mais um frasco
da poção, sofri mais uma vez as dores terríveis da dissolução, e voltei
a mim novamente com o caráter, a estatura e o rosto do Dr. Jekyll.

Naquela noite cheguei à encruzilhada fatal.  Se tivesse empreendido a
minha descoberta com espírito mais nobre, se tivesse me arriscado
naquela experiência quando sob a influência de aspirações generosas ou
piedosas, tudo poderia ter sido diferente, e, daquelas agonias tão
intensas quanto as da morte e do nascimento, eu poderia ter emergido
como um anjo, ao invés de um demônio.  A ação da droga não
discriminava; não era em si diabólica nem divina; ela apenas arrombava
as portas da prisão da minha vontade; e como os cativos de Philippi,		%Cativos de Philippi?
aquele que estava mais pronto foi o primeiro a fugir.\footnote{ Alusão a um episódio da
Bíblia em Atos, 16:26. [\textsc{n.t.}]} Naquele momento minha virtude cochilava; minha maldade, mantida
desperta pela minha ambição, estava alerta e pronta para aproveitar a
ocasião; e a criatura que foi projetada foi Edward Hyde.  Daí que,
embora eu tivesse agora duas personalidades, bem como duas aparências,
uma delas era totalmente maligna, e a outra era ainda o velho Henry
Jekyll, aquele misto incongruente que eu já perdera as esperanças de
mudar e aperfeiçoar.  O movimento ocorrido, portanto, foi totalmente
para o pior.

Mesmo naquela época eu não tinha vencido por completo minha aversão à
austeridade de uma vida dedicada ao estudo.  Sentia-me muitas vezes
ansioso para me divertir; e como os meus prazeres eram prazeres
inconfessáveis, para dizer o mínimo, e eu era não apenas um homem muito
conhecido e considerado, mas aproximando-me da idade madura, essa
incoerência em minha vida era-me especialmente incômoda.  Foi por este
ângulo que meu novo poder exerceu sua tentação até me transformar em
seu escravo.  Eu precisava apenas beber um frasco, abandonar o corpo do
famoso professor e assumir, como se fosse uma espessa capa, a aparência
de Edward Hyde.  Sorri àquela ideia; naquele momento me pareceu
engraçada; e comecei meus preparativos com o maior cuidado.  Aluguei e
mobiliei aquela casa no Soho, até onde a polícia seguiu o rastro de Hyde; e
coloquei como governanta uma mulher que eu sabia ser discreta e
inescrupulosa.  Por outro lado, anunciei aos meus criados que um tal
Mr. Hyde (que descrevi) devia ter livre acesso e toda liberdade em
minha casa situada em frente à praça, e, para evitar qualquer
mal-entendido, por algumas vezes fiz este meu segundo personagem
circular ali até se tornar uma presença familiar.  Em seguida, redigi
aquele testamento contra o qual você ergueu tantas objeções, de modo
que, se algo me acontecesse quando na pessoa do Dr. Jekyll, eu poderia
assumir a forma de Edward Hyde e não sofrer nenhum prejuízo financeiro.
 Com minha posição (imaginei) assegurada desta forma, comecei a me
aproveitar das estranhas imunidades que ela me oferecia.

Já houve homens que alugaram bandidos para cometer crimes em seu nome,
enquanto sua própria pessoa e reputação permaneciam bem abrigadas.  Fui
o primeiro a fazer isso para desfrutar prazeres.  Fui o primeiro a
poder me apresentar diante da opinião pública com uma aparência de
simpática respeitabilidade, e um instante depois, como um estudante,
despir esse uniforme e mergulhar de cabeça no mar da libertinagem.  Mas
para mim, no meu disfarce impenetrável, a segurança era completa. 
Pense nisso: eu nem sequer existia!  Precisava apenas voltar para casa
pela porta do laboratório, misturar e beber em alguns segundos a poção
cujos ingredientes estavam sempre prontos; e, não importa o que tivesse
feito, Edward Hyde desaparecia como a mancha deixada pelo hálito no
vidro de um espelho; e ali estava, calmamente sentado em casa, afilando
o pavio da lâmpada em seu escritório, um homem capaz de dar risadas
diante de qualquer suspeita, o dr. Henry Jekyll.

Os prazeres que eu me apressava a procurar quando sob meu disfarce eram,
como já falei, inconfessáveis; não encontro termo mais forte para
descrevê-los.  Mas nas mãos de Edward Hyde eles logo passaram a ser
monstruosos.  Quando eu regressava dessas expedições, mergulhava muitas
vezes numa espécie de fascínio diante dos meus vícios desfrutados por
procuração. Este  espírito sobrenatural que eu invocara do interior de
minha própria alma, e deixara à solta no mundo para ir em busca de seus
prazeres, era um ser inerentemente maligno e vil; todos os seus
pensamentos e atos eram voltados para si mesmo; bebia o prazer com
avidez bestial ao contemplar as torturas infligidas a outrem;
implacável como uma estátua de pedra.  Henry Jekyll erguia-se às vezes,
revoltado contra os atos de Edward Hyde; mas uma tal situação não podia
ser governada pelas leis habituais, e ele se permitia relaxar o aperto
sobre a própria consciência.  Afinal era Hyde, e somente Hyde, que era
o culpado.  Jekyll permanecia o mesmo; ao voltar a si encontrava suas
boas qualidades aparentemente intactas; às vezes até se apressava,
quando isto era possível, a desfazer o mal causado por Hyde.  E assim
conseguia fazer com que sua consciência adormecesse.

Não pretendo descrever em detalhes as infâmias das quais fui cúmplice
(porque mesmo agora é-me difícil admitir que as cometi eu mesmo); quero
apenas indicar os avisos e os sucessivos passos com que meu castigo foi
se aproximando aos poucos.  Tive um acidente que mencionarei apenas de
passagem, pois não trouxe outras consequências.  Um ato de crueldade
contra uma criança atraiu sobre mim a ira de um transeunte, um homem
que reconheci, dias atrás, na pessoa de um parente seu; um médico e a
família da garota juntaram-se a ele; a certa altura cheguei a temer por
minha vida; finalmente, a fim de acalmar sua justa indignação, Edward
Hyde foi forçado a vir com eles até a porta e pagar-lhes com um cheque
em nome de Henry Jekyll.  Esse perigo foi eliminado em seguida, através
da abertura de outra conta bancária em nome do próprio Edward Hyde, e
quando, inclinando para trás minha própria caligrafia, dotei meu duplo
de uma assinatura, achei que tinha me colocado fora do alcance da mão
do destino.

Cerca de dois meses antes do assassinato de Sir Danvers, eu tinha saído
para uma das minhas aventuras, tinha voltado para casa tarde da noite,
e acordei no dia seguinte na cama com uma sensação um tanto estranha. 
Em vão olhei ao meu redor; em vão vi a mobília sóbria e as amplas
proporções do quarto da minha própria casa em frente à praça; em vão
reconheci a estampa das cortinas do leito, sua moldura de mogno; alguma
coisa insistia em me dizer que eu não estava de fato ali, que não tinha
despertado onde parecia estar, mas no pequeno quarto do Soho onde me
acostumara a dormir quando no corpo de Edward Hyde.  Sorri para mim
mesmo, e, cedendo ao meu hábito de análise psicológica, comecei a
examinar preguiçosamente os elementos daquela ilusão, chegando mesmo,
enquanto o fazia, a mergulhar num agradável cochilo matinal.  Assim
estava quando, num dos momentos em que me vi mais desperto, pousei o
olhar sobre a minha mão.  Ora, a mão de Henry Jekyll (como você notou
muitas vezes) era de tamanho e formato adequados à minha profissão: uma
mão grande, firme, branca e de boa aparência. Mas a mão que eu via
agora, com toda clareza, na luz amarelada de uma manhã londrina,
semicerrada sobre as cobertas, era fina, coberta de veias, com juntas
salientes, a pele com uma palidez doentia e sombreada por tufos de
pelos.  Era a mão de Edward Hyde.

Devo ter ficado olhando-a por meio minuto, mergulhado, como estava, na
estupidez do espanto, até que o terror despertou de vez em meu peito
tão súbito e atordoante quanto um estrondar de címbalos; e, saltando da
cama, corri para diante do espelho.  Diante da imagem que surgiu
a meus olhos, meu sangue transformou-se em alguma outra coisa, rala e
gélida.  Sim, eu tinha ido para a cama como Henry Jekyll, e tinha
acordado como Edward Hyde.  Como se explica isso?, foi o que me
perguntei; e depois, com outro sobressalto de terror: Como isto pode
ser remediado?  Já estávamos no meio da manhã; os criados já estavam de
pé; todas as minhas drogas estavam em meu escritório -- um longo trajeto
que implicava em descer duas escadas, cruzar a passagem dos fundos,
atravessar o pátio interno e entrar no anfiteatro anatômico; tudo muito
distante de onde eu me encontrava agora, paralisado pelo terror.  Eu
poderia sem dúvida encobrir meu rosto, mas de que isto me adiantaria,
se não podia esconder a alteração na minha estatura?  E então, com uma
poderosa e doce sensação de alívio, lembrei-me de que os criados já
estavam acostumados às idas e vindas do meu segundo Eu.  Vesti-me
rapidamente, tanto quanto pude, em roupas conformes à minha altura;
atravessei a casa, onde Bradshaw me avistou e recuou ao ver Mr. Hyde
àquela hora e em trajes tão improvisados; e dez minutos depois o Dr.
Jekyll tinha reassumido sua própria aparência e sentava à mesa, com o
rosto sombrio, fingindo tomar o café da manhã.

Muito pouco era o meu apetite.  Este incidente inexplicável, essa
reversão de minhas experiências anteriores, me parecia, como o dedo
babilônico que escrevia na parede,\footnote{ Alusão a um episódio da
Bíblia, em Daniel 5:5 e 5:23. [\textsc{n.t.}]} estar gravando
as letras da minha sentença, e comecei a refletir, com mais seriedade
do que nunca, nos problemas  e nas possibilidades da minha dupla
existência.  Aquela parte de mim mesmo que eu tinha o poder de projetar
externamente tinha nos últimos tempos sido muito exercitada e
fortalecida; parecia-me ultimamente que o corpo de Edward Hyde tinha
crescido em estatura, como se (quando eu assumia sua forma) eu sentisse
nele uma energia acima do normal; e passei então a entrever o risco de
que, caso isto se prolongasse, o equilíbrio da minha natureza fosse
permanentemente comprometido, meu poder de metamorfose voluntária
ficasse ameaçado, e a pessoa de Edward Hyde viesse a predominar.  O
poder da droga não tinha se manifestado sempre com a mesma intensidade.
 Uma vez, bem cedo na história das minhas experiências, tinha me
falhado por completo; desde então eu fora forçado, mais de uma vez, a
duplicar, e uma vez, com enorme risco de vida, a triplicar sua
quantidade; e essas incertezas tinham projetado a única sombra que
pairava sobre minha satisfação.  Agora, no entanto, e à luz do
incidente daquela manhã, eu me sentia propenso a admitir que, enquanto
no começo minha maior dificuldade fora modificar o corpo de Jekyll,
essa dificuldade estava se invertendo.  Tudo àquela altura parecia
indicar que eu estava perdendo o contato com o meu Eu original e de bom
caráter, e sendo gradualmente incorporado pela minha segunda e pior
versão.

Senti que teria agora de escolher entre elas.  Minhas duas naturezas
tinham uma memória em comum, mas todas as outras faculdades eram
compartilhadas entre elas de forma desigual.  Jekyll (que era uma
personalidade mista) ora demonstrava sensibilidade e apreensão, ora um
prazer ávido em se projetar nos prazeres e nas aventuras de Hyde; mas
Hyde era indiferente a Jekyll, ou talvez só pensasse nele como um
bandoleiro das montanhas pensa na caverna que lhe serve de esconderijo.
 Jekyll tinha mais do que as preocupações de um pai; Hyde tinha menos
do que a indiferença de um filho.  Optar por Jekyll seria morrer para
sempre para aqueles apetites nos quais eu me deleitava em segredo e que
ultimamente tinham se tornado indispensáveis.  Optar por Hyde seria
morrer para mil interesses e aspirações, e me tornar, de um só golpe e
para sempre, alguém desprezado e sem amigos.  A comparação pode parecer
desigual; mas havia ainda outra consideração a ser pesada na balança;
porque enquanto Jekyll iria sofrer lucidamente a dor da abstinência,
Hyde nem sequer teria consciência do quanto perdera.  Por mais
estranhas que fossem as circunstâncias em que eu me via, os termos
daquele dilema eram tão antigos e tão corriqueiros quanto o ser humano;
são os mesmos argumentos e receios que decidem a sorte de todo pecador
que cai em tentação; e sucedeu comigo, como sucede com a maioria dos
meus semelhantes, que acabei por escolher a minha metade melhor,
somente para depois ver-me sem forças para sustentá-la. 

Sim, preferi continuar a ser o médico maduro e insatisfeito, cercado de
amigos e alimentando esperanças honestas; e dei um resoluto adeus à
minha liberdade, à minha relativa juventude, ao passo ágil, aos
impulsos vigorosos e aos prazeres secretos, a tudo que eu desfrutara
quando sob o disfarce de Hyde.  Fiz esta escolha, talvez, com algumas
ressalvas inconscientes, porque mantive a casa no Soho e não destruí as
roupas de Edward Hyde, que continuaram à minha disposição no
escritório.  Durante dois meses, no entanto, fui fiel à minha
determinação; durante dois meses levei uma vida de austeridade como
jamais o fizera, e desfrutei das recompensas de uma consciência em paz.
 Mas o tempo foi obliterando aos poucos a urgência do meu alarme; os
elogios da minha consciência tornaram-se matéria de rotina; comecei a
ser torturado por impulsos e ânsias, como se Hyde estivesse se
debatendo para voltar à liberdade; e por fim, num momento de fraqueza
moral, voltei a misturar e a ingerir a poção transformadora.

Imagino que, quando um ébrio medita consigo mesmo sobre o seu vício,
talvez uma vez entre quinhentas ele tenha consciência dos perigos a que
se expõe através de sua brutal insensibilidade física; do mesmo modo,
eu também, depois de tomada minha decisão, não me dei conta da completa
insensibilidade moral e da insensata propensão à crueldade que eram as
características principais de Edward Hyde.  E foram justamente estas
que trouxeram minha punição.  Meu demônio tinha sido encarcerado por
muito tempo, e emergiu rugindo.  Eu tinha consciência, no momento mesmo
em que bebia a poção, de estar experimentando uma inclinação mais
irrefreada e mais furiosa para a prática do mal.  Deve ter sido isto,
suponho, que despertou em minha alma aquela tempestade de impaciência
com que escutei as civilidades da minha infeliz vítima; declaro, pelo
menos, diante de Deus, que nenhum homem moralmente são podia ser
culpado de um tal crime por motivo tão fútil; e que agi com um espírito
tão pouco razoável quanto o de uma criança que despedaça um brinquedo. 
Mas eu me despira voluntariamente de todos aqueles instintos de
equilíbrio com que mesmo os piores dentre nós continuam a caminhar com
firmeza mesmo cercados de tentações; e no meu caso, ser tentado, ainda
que da forma mais leve, era sinônimo de cair.

Bastou um instante para que aquele espírito infernal despertasse em mim,
cheio de fúria.  Com um arrebatamento de júbilo, maltratei o corpo
indefeso da minha vítima, deliciando-me com cada pancada; e não foi
senão quando o cansaço começou a me dominar que, de repente, no auge do
meu delírio, senti no coração um aperto gelado de terror.  Foi como se
uma névoa se dissipasse; vi que minha vida havia sido comprometida por
aquele ato; fugi do local em que praticara aqueles excessos, ao mesmo
tempo jubiloso e trêmulo de medo, com minha sede de maldade satisfeita
e estimulada, e meu amor pela vida encerrado num caixão onde o último
prego acabava de ser batido.  Corri para a casa do Soho e, para maior
segurança, destruí meus papéis; depois parti pelas ruas ainda
iluminadas pelos lampiões, com a mente dividida e em êxtase,
deliciando-me com o meu crime, devaneando sobre outros que cometeria no
futuro, e ao mesmo tempo apressando o passo e olhando por sobre o
ombro, temeroso de que a vingança viesse ao meu encalço.  Hyde
cantarolava uma canção quando preparou a poção transformadora e, quando
a bebeu, brindou ao homem que acabara de matar.  As dores da
transformação mal tinham acabado de flagelar seu corpo quando Henry
Jekyll, com o rosto banhado em lágrimas de gratidão e remorso, caía de
joelhos e erguia as mãos para Deus.  O seu véu de autocomplacência
tinha sido rasgado de cima abaixo.  Vi a minha vida por inteiro:
acompanhei-a desde os dias da infância, quando eu caminhara segurando a
mão do meu pai, e através de toda a abnegação e sacrifícios de minha
vida profissional, até chegar, vezes seguidas, com o mesmo sentimento
de irrealidade, aos horrores malditos daquela noite.  Quase cheguei a
gritar, mas procurei, com lágrimas e preces, conter aquela torrente
horrível de imagens e sons com que minha memória me martirizava; e
ainda assim, entre uma e outra súplica, minha alma tinha de contemplar
a face tenebrosa da minha iniquidade. Quando a dor aguda do remorso
começou a amainar, foi substituída por uma sensação de alegria.  O
problema da minha conduta estava resolvido.  De agora em diante Hyde
não era mais possível; quisesse ou não, eu estava agora confinado à
parte mais nobre da minha existência, e, ah! que júbilo isto me
causava!  Com que pressurosa humildade passei a assumir as restrições
da minha vida diária!  Com que renúncia sincera tranquei a porta por
onde tantas vezes tinha entrado e saído, e parti aquela chave sob o
salto da minha bota!

No dia seguinte, os jornais revelaram que o crime fora presenciado por
uma testemunha, que a culpa de Hyde era conhecida por todos, e que a
vítima era um homem público altamente considerado.  Não tinha sido
apenas um crime, mas uma trágica imprudência.  Acho que fiquei contente
em saber disto; alegrei-me de ter os meus melhores impulsos a salvo,
protegidos pelo medo do cadafalso.  Jekyll era agora minha cidadela de
refúgio, porque se Hyde entremostrasse seu rosto por um só instante as
mãos de todos os homens estavam prontas para agarrá-lo e fazer-lhe
justiça. 

Tomei a decisão de fazer com que minha conduta dali em diante servisse
para me redimir do meu passado; e posso afirmar honestamente que minha
resolução deu bons frutos.  Você tem conhecimento do quanto, nos
últimos meses do ano passado, eu me dediquei a trabalhos para aliviar o
sofrimento alheio; sabe que fiz muito por meus semelhantes, e que os
meus dias se passaram de maneira tranquila e quase poderia dizer feliz.
Também não posso dizer que me cansei dessa existência inocente e
altruísta; penso, ao invés disso, que dia a dia me afeiçoava mais a
ela; mas eu ainda sofria a maldição da minha duplicidade de propósitos;
e, quando o gume do meu espírito de penitência foi se embotando, meu
lado bestial, que eu alimentara por tanto tempo, e agora mantinha
acorrentado, começou a rosnar sua impaciência.  Não que eu pensasse,
nem por sonhos, em ressuscitar Hyde; essa mera ideia teria me
deixado em pânico; não, era em minha própria personalidade que eu me
sentia agora impelido a violar minha consciência, e foi como um pecador
comum, às escondidas, que finalmente cedi ao assédio da tentação. 

Todas as coisas chegam um dia ao seu fim; a mais espaçosa das medidas
acaba por ser preenchida, cedo ou tarde; e essa breve concessão à minha
própria maldade acabou por destruir o equilíbrio de minha alma.  No
entanto, isto não me deixou alarmado; a queda pareceu-me natural, como
se fosse uma volta aos velhos tempos que antecederam a minha
descoberta.  Era um dia claro e agradável de janeiro, com o chão ainda
úmido nos pontos onde a neve se derretera, mas o céu límpido sobre
nossas cabeças; o Regent’s Park estava cheio dos gorjeios do inverno e
já perfumado pelos aromas da primavera.  Sentei-me ao sol, num banco do
parque, e o animal que havia em mim deleitava-se lambendo os bocados
mais suculentos da memória; meu lado espiritual estava sonolento,
prometendo a si mesmo uma penitência subsequente, mas sem muito ânimo
para iniciá-la.  Afinal de contas, pensei, eu não era muito diferente
dos meus vizinhos; e então sorri, comparando-me aos outros homens,
comparando minha boa-vontade cheia de energia à crueldade preguiçosa de
sua indiferença.  E no momento exato em que fui tomado por esse
pensamento cheio de vanglória, um mal-estar se apoderou de mim, uma
horrível náusea acompanhada de violentos tremores.  Isto se dissipou
daí a algum tempo, deixando-me quase desmaiado; e quando retornei
desse desmaio, comecei a perceber uma mudança em minhas emoções e meus
pensamentos, uma certa ousadia, um desdém pelo perigo, uma dissolução
das amarras da moral.  Abaixei os olhos; minhas roupas pendiam frouxas
em torno dos membros diminuídos; a mão pousada em meu joelho estava
coberta de pelos e de veias salientes.  Eu tinha voltado a ser Edward
Hyde.  Um momento antes eu estava ao abrigo de todo o respeito da
humanidade, era rico, era amado; uma mesa estava sendo posta em casa, à
minha espera; e agora era a presa caçada por todos os homens,
perseguido, sem casa, um assassino notório, cujo destino era a forca.

Minha razão vacilou, mas não me fugiu de todo.  Observei mais de uma vez
que nessa minha segunda personalidade minhas faculdades pareciam se
aguçar ao maior grau e meu espírito se tornava mais adaptável; e assim
se deu que, numa circunstância em que Jekyll talvez tivesse sucumbido,
Hyde soube se erguer à altura da situação.  Minhas drogas estavam
guardadas num armário em meu escritório; como me seria possível
alcançá-las?  Este era o problema que eu, apertando a cabeça entre as
mãos, dediquei-me a resolver.  A porta do laboratório estava trancada. 
Se eu tentasse entrar pela porta da frente, meus próprios criados me
entregariam ao carrasco.  Percebi que precisava usar um intermediário,
e pensei em Lanyon.  Como poderia entrar em contato com ele?  Como
persuadi-lo?  Mesmo que conseguisse não ser capturado em plena rua,
como poderia chegar à presença dele?  E como poderia eu, na pele de um
visitante desconhecido e repugnante, convencer o doutor a violar o
gabinete de trabalho de seu colega, o dr. Jekyll?  Então me veio à
mente que, da minha personalidade original, pelo menos um traço eu
conservava: podia escrever com a minha própria caligrafia; e uma vez
que esta ideia cintilou no meu espírito o meu percurso daí em diante
tornou-se claro do princípio ao fim. 

Deste modo, compus minhas roupas o melhor que pude, e, chamando um
cabriolé que passava, dirigi-me para um hotel em Portland Street, cujo
nome me veio à memória.  Diante da minha aparência (que era bastante
cômica, por mais trágico que fosse o destino que essas vestes
revelavam) o cocheiro não escondeu sua zombaria.  Rangi os dentes para
ele num acesso de fúria demoníaca, e o sorriso se esvaiu do seu rosto,
para sua sorte, e para minha sorte ainda maior, porque um instante a
mais e eu o teria posto abaixo do seu assento.  Chegando ao hotel,
encarei o recepcionista com ar tão ameaçador que o fiz estremecer; os
empregados não trocaram sequer um olhar na minha presença, mas
cumpriram obsequiosamente todas as minhas ordens: levaram-me a um
quarto privado, e trouxeram-me material de escrita.  Hyde numa situação
de perigo era uma criatura nova para mim; trêmulo com uma ira mal
controlada, tenso a ponto de cometer um crime, ansioso para infligir
dor em alguém.  E no entanto a criatura tinha muita astúcia; soube
controlar sua fúria com um grande esforço da vontade; redigiu duas
cartas, uma para Lanyon e outra para Poole; e, para ter certeza de que
tinham sido enviadas, exigiu que fossem registradas.  Depois, passou o
resto do dia sentado a sós no quarto, ao lado do fogo, roendo as unhas;
 ali ceou, sentado a sós com seus terrores, vendo o criado encolher-se
de medo ao seu olhar; e dali, quando caiu a noite, partiu,
refugiando-se no interior de um cabriolé fechado, no qual ficou
percorrendo a esmo as ruas da cidade.  Digo-o assim: ele, porque não
posso dizer “eu”.  Aquele filho do Inferno nada tinha em si de humano;
nada havia de vivo nele senão o medo e o ódio.  E quando por fim,
percebendo que o cocheiro começava a achá-lo suspeito, ele dispensou o
carro e arriscou-se a andar a pé, com aquelas roupas mal ajustadas
tornando-o alvo dos olhares dos transeuntes noturnos, essas duas
paixões primitivas arderam dentro dele como uma tempestade.  Caminhou
depressa, acossado por terrores, balbuciando consigo mesmo,
desviando-se sorrateiro pelas vielas menos frequentadas, contando os
minutos que ainda o separavam da meia-noite.  A certa altura uma mulher
lhe dirigiu a palavra, oferecendo-lhe à venda, acho, uma caixa de
fósforos. Ele a esbofeteou, fazendo-a fugir.

Quando voltei a ser eu mesmo na casa de Lanyon, o horror que contemplei
no rosto do meu velho amigo talvez tenha me afetado.  Não tenho
certeza; para mim era como uma gota d'água no oceano diante do horror
que aquelas horas me trazem à lembrança.  Uma mudança tinha ocorrido em
mim.  Já não era mais o medo da forca; o que me atormentava era o medo
de voltar a ser Hyde.  Ouvi as censuras de Lanyon como que num sonho; e
foi como num sonho que regressei a minha própria casa e fui para a
cama.  Dormi, após o esgotamento produzido por aquele dia, com um sono
profundo e tenso, que nem os pesadelos que me assaltaram foram capazes
de interromper.  Acordei, pela manhã, abalado, enfraquecido, mas
aliviado.  Ainda me causava ódio e medo o pensamento de ter aquele ser
bruto adormecido dentro de mim, e certamente eu não esquecera os
perigos espantosos por que passara na véspera; mas estava de novo em
minha própria casa, e próximo às minhas drogas; e a gratidão pelo meu
salvamento resplandecia com tal intensidade em minha alma que quase
chegava a rivalizar com o brilho da esperança.

Eu estava passeando descuidadamente pelo pátio interno, após o café da
manhã, quando fui acometido de novo por aquelas sensações
indescritíveis que eram o prenúncio da minha metamorfose; e mal tive
tempo de me abrigar no escritório, antes de me ver mais uma vez
invadido e dominado pelas paixões de Hyde.  Tomei nessa ocasião uma
dose dupla para voltar a ser eu mesmo, e, -- ai de mim! -- seis horas
depois, quando estava tristemente sentado à beira do fogo, as dores
voltaram, e tive que tomar a droga outra vez.  Em resumo: daquele dia
em diante foi apenas com um esforço digno de um ginasta, e somente com
a aplicação imediata da droga, que pude retornar à aparência física de
Jekyll.  A qualquer hora do dia ou da noite eu podia ser tomado por
aqueles espasmos premonitórios; e, mais do que tudo, se eu me deixasse
adormecer ou pelo menos cochilar alguns instantes em minha poltrona,
era sempre na pessoa de Hyde que viria a despertar.  Sob a tensão dessa
desgraça que se acometia sobre mim, e fatigado pela insônia voluntária
a que me condenei, e mais ainda do que eu jamais imaginara ser possível
a um homem suportar, eu me tornei, mesmo quando em minha própria
aparência, um indivíduo esvaziado e enfraquecido pela febre, debilitado
tanto no corpo quanto na mente, e dominado por uma única ideia: o
horror do meu outro Eu.  Mas quando adormecia, ou quando a droga
deixava de produzir efeito, eu me via, quase sem transição (porque as
dores da transformação diminuíam dia a dia), possuído por um delírio de
imagens aterrorizantes em que minha alma fervia em ódios irracionais, e
eu me encontrava num corpo que não me parecia forte o bastante para
conter as energias furiosas da vida.  As forças de Hyde pareciam ter
aumentado proporcionalmente à debilitação de Jekyll.  E certamente o
ódio que os separava era o mesmo, de parte a parte.  Com Jekyll, era
uma questão de instinto de sobrevivência.  Ele já vira por inteiro a
deformidade daquela criatura que com ele compartilhava alguns dos
fenômenos da consciência, e que deveria acompanhá-lo até a morte; e,
além desses aspectos que compartilhavam, e que por si sós constituíam a
parte mais dolorosa de sua aflição, ele pensava em Hyde, apesar de toda
a energia vital que o animava, como algo não apenas infernal mas
inorgânico.  Este era o aspecto mais chocante: que o lodo do fundo de
um poço pudesse emitir gritos e palavras; que a poeira amorfa pudesse
agir e pecar; que aquilo que não possuía vida nem forma pudesse usurpar
as funções da vida.  E, também, que esse horror insubordinado estivesse
preso a ele de modo mais íntimo que uma esposa, mais próximo do que um
olho; que estivesse encarcerado em sua própria carne, onde ele podia
ouvi-lo a murmurar em sua luta incessante para vir à luz; que em cada
momento de fraqueza, e que na tranquilidade do sono, esse ser
prevalecesse sobre sua vontade, e ocupasse o seu lugar entre os vivos. 
O ódio de Hyde por Jekyll era de outra natureza. O terror do cadafalso
o levava continuamente a cometer o suicídio temporário, voltando à sua
condição subordinada de ser uma parte ao invés de uma pessoa; mas ele
detestava essa necessidade, detestava o abatimento de que Jekyll era
vítima, e se ressentia do desagrado com que era visto.  Surgiam daí as
peças que ele me pregava, rabiscando com minha própria caligrafia
blasfêmias nas páginas dos meus livros, queimando cartas, destruindo o
retrato do meu pai; e, sem dúvida, se não fosse pelo seu próprio medo
da morte, teria há muito tempo arruinado a si mesmo desde que pudesse
me arrastar nessa ruína. Mas seu apego à vida era prodigioso; e digo
mais: eu, que adoeço e fico gelado à sua simples lembrança, quando
recordo a abjeção e a intensidade da nossa ligação, e quando penso no
quanto ele teme o meu poder de destruí-lo através do suicídio, consigo
encontrar no meu coração forças para apiedar-me dele.

É inútil prolongar esta descrição, e não me resta muito tempo para isto;
nenhum ser humano sofreu jamais tormentos semelhantes, e isto me basta;
e mesmo para estes sofrimentos, o hábito me trouxe, não o alívio, mas
uma certa insensibilidade da alma, uma certa aquiescência ao desespero;
e meu castigo poderia ter se prolongado durante anos, não fosse pela
derradeira calamidade que se abateu sobre mim, e que finalmente me
apartou do meu próprio rosto e da minha própria natureza.  

Meu suprimento do sal necessário para a poção, que nunca tinha sido
renovado desde a data da primeira experiência, começou a escassear. 
Encomendei uma nova entrega e fiz o preparado; houve a efervescência do
líquido, e deu-se a primeira mudança de cor, mas não a segunda; bebi-a,
e não teve efeito algum.  Você pode confirmar com Poole que o fiz
vasculhar Londres inteira; foi em vão; e agora estou persuadido de que
era a minha primeira amostra que era impura, e que foi esta impureza
desconhecida que resultou na eficácia da fórmula.

Uma semana se passou, e agora estou encerrando este depoimento sob a
influência do que me restou daquele pó.  Esta, portanto, é a última
vez, salvo ocorra algum milagre, que Henry Jekyll pode ter seus
próprios pensamentos ou contemplar o próprio rosto (tão mudado, agora!)
no espelho.  Não devo me prolongar muito; porque se minha narrativa até
agora escapou à destruição foi por uma combinação de enorme prudência e
de boa sorte.  Se as primeiras agonias da transformação me acometerem
enquanto ainda escrevo, Hyde reduzirá este documento a farrapos; mas se
decorrer algum tempo depois que eu o colocar em lugar seguro, seu
imenso egocentrismo e sua atenção aos problemas mais imediatos poderão
salvar estas linhas do seu rancor simiesco.  E a verdade é que a
desgraça que se avizinha de nós já o deixou mudado e abatido.  Daqui a
meia hora, quando eu tiver me transformado pela última e definitiva vez
nessa odiosa pessoa, sei que estarei sentado na minha poltrona,
trêmulo, chorando, ou continuarei, com os ouvidos apurados e à escuta,
a andar de um lado para o outro neste aposento, meu derradeiro refúgio
na Terra, atento para qualquer som ameaçador.  Caberá a Hyde morrer na
forca? Ou ele terá coragem de libertar a si mesmo, no derradeiro
instante?  Somente Deus sabe; eu não me importo mais.  Esta é a hora
verdadeira da minha morte, e o que acontecerá depois concerne a outro
homem, não a mim. Aqui, portanto, no momento em que pouso a pena e
começo a selar minha confissão, a vida do desventurado Henry Jekyll
chega ao seu fim.  

