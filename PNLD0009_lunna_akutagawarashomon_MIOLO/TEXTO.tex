
\chapter{Rashômon}

\noindent\textsc{Era num entardecer.} Um servo de baixa condição esperava, sob o
Rashômon,\footnote{ \textit{Rashômon}: nome do Portal que, na era Heian
(794--1192), se situava na entrada principal da milenar Capital, atual
cidade de Quioto, hoje equivalente à região em que se encontra a
Estação Central.} que a chuva passasse.

Sob o amplo portal, além daquele homem, não havia mais ninguém.
Somente um grilo, que permanecia pousado na enorme coluna circular
com áreas descascadas no laqueado alaranjado. Uma vez que o Rashômon se
situava na Avenida Suzaku, era de esperar que houvesse mais duas ou
três pessoas, com seus chapéus femininos cônicos ou masculinos
alongados, abrigando"-se da chuva. Entretanto, além daquele homem não
havia mais ninguém.

Isso porque, nos últimos dois ou três anos, Quioto sofrera seguidas
calamidades: terremotos, redemoinhos, incêndios e fome. Por essas
razões, era enorme a desolação no centro da Capital. Rezam as antigas
crônicas que naquele tempo estátuas de Buda e objetos de culto budista
eram destruídos empilhando"-se na beira da estrada a madeira ainda
laqueada ou folheada a ouro e prata para ser vendida como lenha. Se até
o centro da Capital se encontrava naquelas condições, da conservação do
Rashômon, então, nem sequer se cogitava. Assim, tirando partido do
abandono em que o Portal se encontrava, raposas e texugos começaram a
se abrigar ali. E também ladrões. Até que, afinal, passado um tempo,
virou hábito abandonar, no Rashômon, cadáveres não reclamados. Por
isso, quando a luz do dia não podia mais ser vista, era tamanho o pavor
que ninguém mais ousava se aproximar.

Corvos começaram então a se juntar em bandos, vindos sabe"-se lá de
onde. Durante o dia, inumeráveis, eles descreviam círculos e grasnavam
ao redor da alta cumeeira. No crepúsculo, quando o sol se avermelhava
sobre o Portal, facilmente podiam ser divisados, como grãos de gergelim
dispersos no ar. Vinham, obviamente, alimentar"-se da carne dos mortos
abandonados na galeria\ldots{} Se bem que, naquele dia, não se avistasse
nenhum deles, talvez devido ao adiantado da hora. Mas podiam"-se notar
seus excrementos pontilhados de branco sobre os degraus de pedra quase
em ruínas, em cujas fendas o capim crescia. Acocorado no último dos
sete degraus, sob o tecido surrado de sua vestimenta azul"-escura, o
servo olhava a chuva, distraído, sentindo"-se incomodado com a enorme
espinha que lhe aparecera na face direita.

Escreveu o autor anteriormente: ``Um servo de baixa condição esperava a
chuva passar''. Entretanto, mesmo que a chuva passasse, o servo não
teria, na verdade, nada a fazer. Normalmente, é claro, deveria retornar
à casa de seu senhor. Acontece que fora dispensado havia quatro ou
cinco dias. Como também se escreveu antes, a cidade de Quioto, por essa
época, se encontrava em acentuado estado de decadência. E o fato de ter
sido dispensado pelo senhor, a quem servira durante longos anos, não
passava de uma pequena consequência daquela decadência geral. Seria,
portanto, mais adequado dizer ``Um servo de baixa condição, preso pela
chuva, estava desnorteado, sem saber para onde ir'' do que ``Um servo de
baixa condição esperava a chuva passar''. Além do mais, o tempo chuvoso
contribuía sensivelmente para a disposição de espírito daquele homem da
era Heian. A chuva que começara a cair depois das quatro horas da tarde
parecia não mais parar. Assim, havia algum tempo, o servo escutava, com
ar ausente, o barulho da chuva que caía na Avenida Suzaku ruminando
pensamentos desconexos, procurando resolver, antes de mais nada, a
questão de sua sobrevivência --- questão que ele sabia ser insolúvel.

A chuva que envolvia o Portal trazia a massa do som até das gotas mais
longínquas. A escuridão aos poucos fazia baixar o céu; quem levantasse
os olhos veria o telhado do Rashômon, que se projetava em diagonal,
sustentando nuvens pesadas e sombrias.

Quando se tenta resolver uma questão insolúvel, não há tempo para
escolher os meios. Se demorasse muito na escolha, o servo certamente
terminaria morrendo de fome ao pé de um muro de barro ou à beira de uma
estrada. E certamente seria trazido até o Portal e abandonado como um
cão. ``Se não escolher\ldots{}'' Seu pensamento, depois de muitos rodeios,
finalmente empacou nesse ponto. Entretanto, esse ``se'' continua sendo,
afinal de contas, o mesmo ``se''. Mesmo admitindo não haver escolha de
meios, ele não tinha coragem suficiente para aceitar de forma positiva
a resposta inevitável à questão: ``A única saída é tornar"-me ladrão''.

Depois de um forte espirro, o servo se ergueu preguiçosamente. Em
Quioto, onde as tardes são frias, a temperatura já baixara a ponto de
fazê"-lo ansiar por um braseiro. Na escuridão, o vento soprava
implacável por entre as colunas do Portal. Até o grilo pousado na
coluna laqueada de alaranjado já havia desaparecido.

Encolhendo"-se todo e erguendo a gola da vestimenta azul"-escura que
envergava sobre a roupa amarela, correu os olhos em volta do Portal.
Procurava um lugar onde pudesse passar a noite tranquilamente, longe de
olhares estranhos e sob a proteção do vento e da chuva. Então, por
sorte, descobriu uma escada larga, também laqueada de alaranjado, que
conduzia a uma galeria sobre o Rashômon. Lá em cima, o máximo que ele
poderia encontrar seriam cadáveres. O servo, assim, cuidando para que a
espada presa à sua cintura não se soltasse da bainha, pousou no
primeiro degrau o pé calçado de sandália de palha.

Subiu então, daí a alguns minutos, a meia altura da ampla escada que
conduzia à galeria do Rashômon. Um homem, o corpo encolhido como um
gato, sustando a respiração, espreitava o que se passava ali em cima. A
luz que vinha da galeria tocava levemente sua face direita. Era uma
face com uma espinha vermelha e purulenta em meio a uma barba rala. O
servo, desde o início, tinha a certeza de que ali no alto só haveria
cadáveres. Todavia, depois de subir dois ou três degraus, pareceu"-lhe
notar uma sombra que se movimentava. Logo isso se confirmou, pois uma
claridade turva e amarelada se refletia, oscilante, nos vãos do teto
cobertos de teias de aranha. Não podia tratar"-se apenas de uma pessoa
comum que, numa noite de chuva como aquela, portasse um luzeiro no
interior de uma galeria como aquela do Rashômon.

Abafando seus passos como uma lagartixa, o servo finalmente atingiu o
último degrau da difícil escada. E então, com o corpo mais retesado
possível, alongando o pescoço o mais que podia, ele perscrutou, 
transfigurado de medo, o interior da galeria.

De fato, conforme ouvira dizer, alguns cadáveres achavam"-se jogados,
desordenadamente, no seu interior. Mas, sendo o campo de luz mais
limitado do que supunha, não conseguia precisar quantos. Ele somente
podia distinguir, sob a fraca luminosidade, alguns corpos nus e outros
ainda vestidos. Entre eles, parecia haver tanto homens quanto mulheres.
E todos aqueles cadáveres jaziam sobre o assoalho, como bonecos de
barro, as bocas abertas, os braços estirados, fazendo até duvidar que
um dia tivessem sido humanos. Além do mais, à luz das chamas que
iluminavam as partes salientes, como ombros e bustos, as outras partes
pareciam ainda mais escuras. Os corpos conservavam"-se mudos, para
sempre calados.

O servo tapou instintivamente o nariz ao perceber o odor pútrido. Mas já
no instante seguinte se esquecia de cobri"-lo. Uma emoção mais forte
anulou por completo seu olfato.

Pois só então seus olhos distinguiram um ser humano, agachado em meio
aos cadáveres. Era uma velha de aparência simiesca, os cabelos brancos,
magra, baixa, vestida de ocre. Tendo na mão direita uma tocha de pinho,
observava, detidamente, o rosto de um dos cadáveres. Pelos cabelos
compridos, supunha"-se que fosse um cadáver de mulher.

Tomado de sessenta por cento de terror e quarenta de curiosidade, o
servo, por alguns instantes, até se esqueceu de respirar. Arrepiou"-se
e, para empregar a expressão de um antigo cronista, sentiu que ``até os
pelos do corpo haviam ficado mais espessos''. Nisso, a velha prendeu a tocha de 
pinho numa fresta do assoalho e, erguendo com as duas mãos o pescoço do
cadáver que até então examinava, começou a arrancar um a um os longos
fios de cabelo, exatamente como uma macaca catando piolhos do filhote.
Os cabelos pareciam soltar"-se facilmente ao movimento de suas mãos.

À medida em que os fios iam sendo arrancados, o terror que assaltara o
servo foi desaparecendo aos poucos. E, ao mesmo tempo, foi crescendo,
pouco a pouco, um forte ódio contra aquela velha. Não, não seria exato
dizer ``contra a velha''. Na verdade, o que a cada minuto se tornava mais
forte era uma repulsa contra todos os males. Se naquele instante alguém
lhe propusesse, outra vez, o dilema que antes o atormentara --- morrer de
fome ou tornar"-se ladrão ---, não hesitaria mais em escolher a morte pela
fome. Pois seu ódio ao mal começava a se inflamar mais e mais, como a
tocha fincada pela velha no assoalho.

O servo não compreendia por que a velha arrancava os cabelos dos
cadáveres. Por conseguinte, não tinha condições de julgar segundo a
razão a moralidade daquele ato. Entretanto, para ele, o simples fato de
arrancar cabelos de cadáveres, numa noite de chuva como aquela, num
lugar como aquele, já constituía um mal imperdoável. Obviamente, o
servo já nem recordava que, havia poucos minutos, tencionava tornar"-se
ladrão.

Nesse instante, num movimento brusco, o servo pulou para dentro da
galeria. E, com a mão na espada, aproximou"-se da velha a passos largos.
O autor nem precisa dizer o susto que ela levou.

Ao ver o servo, ela pulou, como uma pedra lançada por uma catapulta.

--- Ei! Aonde vai? --- vociferou o servo, barrando o caminho da velha, que
procurava fugir, afobada, tropeçando entre os cadáveres.

Mas, mesmo barrada, ela o empurrou, tentando escapar. Ele, por sua vez,
para impedi"-la de fugir, também a empurrou. Por um momento os dois se
engalfinharam, mudos, em meio aos cadáveres. Mas o resultado era
previsível. O servo, torcendo"-lhe o braço, terminou por derrubá"-la.
Quais pés de galinha, seus braços eram somente pele e osso.

--- O que estava fazendo? Diga! Senão\ldots{}

O servo atirou"-a ao chão e, desembainhando a espada, apontou a lâmina de
aço branca bem no meio de seus olhos. Entretanto, a velha se conservava
calada. Com as mãos trêmulas, a respiração ofegante e os olhos
esbugalhados --- a ponto de lhe saltarem os globos oculares para fora das
órbitas ---, obstinava"-se em permanecer calada. Vendo"-a assim, só então o
servo percebeu claramente que aquela vida se encontrava totalmente em
suas mãos, e tal consciência acabou por arrefecer o ódio que até então
lhe inflamava o peito. Sentiu a satisfação e a confiança de quem
executa um trabalho bem"-sucedido. Assim, olhando a velha de cima,
abrandou a voz.

--- Não me tome por agente da polícia. Sou apenas um viajante que, por
acaso, passava por esse Portal. Por isso, não vou prendê"-la nem
incomodá"-la. Basta que me conte o que estava fazendo na galeria numa
hora dessas.

Nisso, a velha arregalou ainda mais os olhos e fixou"-os no servo.
Encarava"-o com um olhar penetrante, as pálpebras vermelhas como as de
aves de rapina. E a seguir, como se estivesse mastigando, moveu uns
lábios que quase se confundiam com o nariz devido ao número de rugas.
Em seu pescoço descarnado notava"-se um pontiagudo pomo"-de"-adão que se
agitava. Foi naquele instante que uma voz grasnada, como a de um corvo,
se fez ouvir num arquejo:

--- Estou arrancando estes cabelos, sabe?\ldots{} Estes cabelos\ldots{} pensando em
fazer perucas\ldots{}

O servo ficou desapontado com a resposta, inesperadamente banal. E, com
o desapontamento, sentiu retornar ao seu íntimo o ódio anterior, mas
dessa vez acrescido de frio desprezo. A mudança de ânimo foi notada
pela velha, que, ainda segurando os cabelos compridos que arrancara do
cadáver, gaguejou, como se coaxasse baixinho:

--- Pois é\ldots{} Arrancar cabelos dos cadáveres pode ser errado. Mas todos os
mortos que estão aqui, sem exceção, bem o merecem. Essa mulher, por
exemplo, de quem arranquei os cabelos, costumava vender cobra seca por
peixe seco nas guaritas dos vigias do Palácio. Ela cortava as cobras em
pedaços de meio palmo e as secava. Se não tivesse morrido na epidemia,
certamente ainda estaria fazendo a mesma coisa. E note que os guardas
achavam os peixes muitos saborosos e sempre compravam dela. Para mim, o
que ela fazia não era ruim. Não tinha outro jeito, senão morreria de
fome. Não acho tampouco que eu esteja agindo errado. Eu também morreria
de fome, não tenho escolha. Por conseguinte, essa mulher, que sabia
muito bem disso, sem duvida há de me perdoar.

Foi aproximadamente isso o que a velha disse. O servo ouviu com
indiferença a história da velha, conservando a mão esquerda no punho da
espada já embainhada. Enquanto ouvia, sua mão direita apalpava a grande
espinha vermelha e purulenta que o incomodava. E, aos poucos, lhe
brotava certa coragem que, antes, quando estava debaixo do Portal, lhe
fizera falta. Era uma coragem que crescia numa direção oposta àquela do
momento em que agarrara a velha, ao subir à galeria. O servo não
hesitava mais entre morrer de fome ou tornar"-se ladrão. Nesse momento,
morrer de fome nem passava por sua cabeça; era uma alternativa que lhe
fugira por completo à consciência.

--- É isso mesmo! --- disse o servo em tom de escárnio ao ouvir o fim do
relato da velha. Adiantando"-se um passo, subitamente afastou a mão
direita da espinha, agarrou a mulher pela gola e vociferou: --- Se é
assim, não me leve a mal se eu roubá"-la. Se eu não fizer isso, também o
meu corpo irá morrer de fome.

Rapidamente, tirou"-lhe as roupas. Depois, chutou com violência aquela
velha que se agarrava a seus pés e a derrubou sobre os cadáveres.
Estava apenas a cinco passos da saída. Carregando a roupa de cor ocre
sob o braço, precipitou"-se escada abaixo rumo a uma noite profunda.

A velha, como que morta por alguns instantes, ergueu o corpo nu somente
um tempo depois por entre os cadáveres. Numa voz quase um murmúrio,
quase um gemido, ela, guiando"-se pela claridade do fogo que ainda ardia
no pinho, arrastou"-se até a escada. E então, a cabeça pendida para
frente, os cabelos brancos e ralos suspensos, espiou para baixo do
Portal. Lá fora, apenas a escuridão das cavernas a envolver a noite.

O paradeiro do servo ninguém jamais soube.

\begin{flushright}
\textit{Setembro de 1915}
\end{flushright}

\chapter{Dentro do bosque}

\section*{depoimento de um lenhador interrogado pelo alto comissário de polícia}

Sim, Senhor Comissário, é verdade.

Quem encontrou o cadáver fui eu mesmo. Nesta manhã, como de costume, fui
cortar cedro na montanha do outro lado. Nisso, encontrei aquele cadáver
dentro do bosque, no sopé da montanha --- onde foi exatamente que o
encontrei? A cerca de quinhentos metros da estrada de Yamashina. Num
lugar ermo, onde cedros finos se misturam aos bambus.

O cadáver estava deitado de costas, vestia um quimono de seda azul e
trazia um chapéu pregueado à moda da Capital. Via"-se um só golpe de
espada, mas, como era muito profundo e estava bem no meio do peito, as
folhas secas de bambu ao redor do cadáver pareciam tingidas de
vermelho.

 Não, Senhor Comissário, não corria mais sangue. Pareceu"-me que a ferida
havia secado. Lembro"-me bem de que havia uma mosca que lambia o sangue,
e que nem deu mostras de perceber meus passos.

Pergunta"-me o senhor se não vi uma espada ou outra coisa qualquer? Não,
senhor, não havia nada. Só um pedaço de corda jogado ao pé do cedro.
Depois\ldots{} Ah, ia"-me esquecendo! Além da corda, havia um pente. Foi tudo
o que encontrei à volta do corpo. Mas, como as plantas e as folhas de
bambu caídas ao redor do cadáver estavam muito pisadas, não há dúvida
de que o homem, antes de ser assassinado, resistiu bravamente. 

Como? Se eu não vi nenhum cavalo? É um lugar inacessível a cavalos. Há
uma mata densa separando o local do caminho por onde eles passam.

\section*{depoimento de um monge budista peregrino interrogado pelo alto
comissário de polícia}

Tenho certeza de que ontem vi este homem cujo cadáver os senhores
encontraram hoje. Ontem, por volta do meio"-dia, creio eu. Foi a meio
caminho entre Sekiyama e Yamashina. Ele vinha a pé no rumo de Sekiyama,
acompanhando uma mulher a cavalo. Não podia ver o rosto dela, pois seu
chapéu era provido de um longo véu. Tudo o que pude divisar foi a cor
de suas vestes: púrpura sobre azul. 

Quanto ao cavalo, parecia ser um alazão de crina aparada.

Qual a altura do animal? Teria cerca de um metro e trinta centímetros?
Como sou monge, não saberia dizer.

E o homem? Sim, além da espada, também portava arco e flechas. Ainda
agora me lembro muito bem de ter visto cerca de vinte flechas em sua
aljava laqueada de preto.

Nem em sonhos imaginei o destino que o esperava. Realmente, a vida
humana é mesmo frágil como o orvalho da manhã e breve como um clarão de
luz\ldots{} 

Pois é, nem encontro palavras para expressar o quanto o lastimo\ldots{}

\section*{depoimento do policial interrogado pelo alto comissário de polícia}

O homem que eu prendi? Não há dúvida de que é o conhecido ladrão
Tajômaru. Quando o prendi, na ponte de pedra de Awataguchi, acho que
tinha caído do cavalo, pois estava gemendo de dor.

Que horas eram? Foi logo no começo da noite. Dias atrás, quando tentei
prendê"-lo, mas não consegui, ele vestia a mesma roupa azul"-escura e
trazia a mesma espada ornada de detalhes metálicos. Como o senhor agora
bem pode ver, também portava arco e flechas.

É mesmo? Aquele homem também possuía arco e flechas antes de ser morto?
Então não há dúvidas de que o assassino é Tajômaru. Arco revestido de
couro, aljava laqueada de preto, dezessete flechas com penas de
falcão\ldots{} tudo, então, deve pertencer àquele homem!

Sim, como diz o senhor, o cavalo também é um alazão com a crina aparada.
O ladrão deve ter sido derrubado pelo animal por castigo divino. O
cavalo pastava pouco adiante da ponte, a rédea comprida arrastando no
chão. Esse tal de Tajômaru, de todos os ladrões que rondam a Capital, é
o que mais persegue mulheres. No outono passado, na montanha que fica
atrás do templo Toribe, foi encontrada uma dama da corte, morta, que
possivelmente fora rezar pela cura de alguém, juntamente com uma jovem
servente. Suspeita"-se que tenha sido esse indivíduo.

Se for esse bandido aí quem matou aquele homem, vá se saber também o que
fez com a mulher que montava o alazão…

Por favor, Senhor Comissário, não é da minha alçada, mas peço"-lhe que
seja investigada essa questão.

\section*{depoimento de uma velha interrogada pelo alto comissário de polícia}

Sim, senhor. Aquele é o cadáver do homem com quem casei minha filha. Ele
não era da Capital. Era um samurai do governo da província de Wakasa.
Chamava"-se Kanazawano Takehiro e tinha vinte e seis anos de idade.

Não, senhor. Como era muito gentil, jamais provocaria a ira de alguém.

Minha filha? Ela se chama Masago, tem dezenove anos. Sua personalidade é
tão forte como a de qualquer homem; no entanto, até agora sempre foi
fiel a Takehiro. Seu rosto é pequeno e oval, tem uma pele amorenada e
uma pinta no canto do olho esquerdo.

Takehiro partiu ontem para Wakasa em companhia de minha filha. Mas que
infelicidade! Quem poderia imaginar uma coisa dessas? O que teria
acontecido à minha filha? Quanto a meu genro, até posso me conformar;
no entanto, só de pensar nela, fico doente.

Suplico"-lhe, é o único desejo desta velha: descubra o paradeiro da minha
filha, nem que para isso seja preciso revirar montanhas e matas. Custe
o que custar, encontre"-a! Esse ladrão, como é mesmo que se chama?
Tajômaru\ldots{} Como o odeio! Não somente o meu genro, mas também a minha
filha\ldots{}

(Lágrimas sufocam suas ultimas palavras.)

\section*{confissão de tajômaru}

Sim, fui eu quem matou aquele homem. Mas a mulher, não.

Então, onde ela está? Isso, nem eu sei.

Ei, esperem! Nenhuma tortura pode me fazer dizer o que não sei! Além do
mais, nessas condições, não pretendo esconder"-lhes nenhum segredo à
toa. Ontem, pouco depois do meio"-dia, deparei"-me com o casal. Naquele
momento, com o sopro do vento, o véu se ergueu e pude ver, por breves
segundos, o rosto da mulher. Por alguns segundos --- foi um vislumbre,
apenas isso. Pode ter sido por causa da brevidade da visão, mas o rosto
dela  apareceu perante mim como se fosse um Boddhisatva mulher. Foi
naquele instante que decidi possuí"-la, mesmo que tivesse de matar"-lhe o
marido.

Bah, matar um homem não é lá grande coisa, como vocês pensam. De
qualquer forma, para tomar uma mulher, sempre é preciso matar o homem.
A diferença é que, quando eu mato, uso a espada que trago à cintura,
mas vocês, não. Vocês não se utilizam da espada, matam apenas com o seu
poder, matam com o seu ouro. Às vezes matam somente com palavras, a
pretexto de o fazerem para o próprio bem deles. É verdade que não corre
sangue, que os homens continuam vivendo, mas, mesmo assim, vocês os
mataram. Se pensarmos na gravidade dos crimes, não saberia dizer quem
de nós, vocês ou eu, seria o pior. (Sorriso irônico.) Mas, se eu
pudesse tomar a mulher sem matar o marido, tanto melhor. Aliás, meu
estado de espírito, naquela hora, era o de possuir a mulher e, se
possível, não matar o homem. Entretanto, fazer uma coisa dessas na
estrada de Yamashina era realmente impossível. Por isso armei um plano
para fazer o casal acompanhar"-me montanha adentro.

Não foi nada difícil. Fazendo"-me seu companheiro de viagem, contei"-lhes
que havia túmulos antigos na montanha do outro lado e que, ao explorar
aquelas sepulturas, tinha encontrado espelhos de metal e espadas em
grande quantidade. Disse"-lhes ainda que os havia escondido,
enterrando"-os dentro do bosque, à sombra da montanha, e que, se
houvesse interessados, faria um bom preço. O homem, pouco a pouco, foi
sendo atraído pela minha conversa. E depois\ldots{} --- a cobiça é uma coisa
terrível, não acham? --- e depois, em menos de meia hora, aquele casal já
conduzia o cavalo rumo à montanha, junto comigo. Chegando em frente ao
bosque, disse"-lhes que o tesouro estava enterrado lá dentro e os
convidei a verificá"-lo. O homem, cego pela cobiça, nem titubeou. Mas a
mulher preferiu esperar, sem descer do cavalo. Não sem razão, já que
aquele bosque era muito fechado. E, para dizer a verdade, as coisas
caminhavam como eu queria; penetramos no bosque, deixando a mulher
sozinha. Por um trecho, só havia bambus no bosque. Cerca de cinquenta
metros adiante, porém, havia uma clareira entre os cedros\ldots{} Não
haveria lugar melhor que aquele para executar meu plano. Abrindo
caminho pela mata, preguei"-lhe a mentira --- bastante plausível --- de que
o tesouro estava enterrado sob os cedros. Mal lhe disse isso e o homem
se lançou em direção aos troncos finos dos cedros, que dali se
enxergava. Os bambus rareavam, alguns cedros já se enfileiravam --- e foi
justo nesse local que, bruscamente, eu o derrubei e dominei. Como o
homem portava uma espada, poderia ser muito perigoso, mas, apanhado de
surpresa, não teve como resistir. Num segundo, estava amarrado ao pé de
um cedro.

A corda? Sendo ladrão, sempre trago uma à cintura, pois sabe"-se lá
quando terei de escalar algum muro. Afora encher sua boca de folhas
secas de bambu para impedi"-lo de gritar, não tive nenhum trabalho.
Terminada a primeira parte, fui ter com a mulher e lhe disse para vir
comigo ver o marido, que passava mal. Nem preciso lhes dizer do sucesso
do meu plano. Com chapéu na mão, a mulher foi penetrando no interior do
bosque, comigo a conduzi"-la pela mão. Mas, ao chegar ao local onde o
homem estava amarrado ao pé de cedro --- a mulher, mal percebeu a cena,
fez reluzir num átimo um punhal que havia retirado de sua roupa, sem
que eu o notasse. Nunca antes havia encontrado uma mulher de
temperamento tão violento. Bem, mesmo me esquivando rapidamente, era
difícil evitar os golpes ante uma investida tão feroz. Porém, como sou
o famoso Tajômaru, finalmente derrubei o seu punhal sem precisar sequer
desembainhar a espada. Por mais decidida que fosse, desarmada, ela nada
poderia fazer. Assim, finalmente consegui possuir a mulher sem tirar a
vida do homem.

Sem tirar a vida do homem --- é isso mesmo. Eu não tinha mesmo intenção de
matá"-lo. Acontece que, quando eu já ia fugindo do bosque, deixando
atrás a mulher em prantos, de repente ela agarrou"-me o braço,
desesperada. Com gritos entrecortados de soluços, ela dizia: ``Morra
você ou o meu marido, morra um dos dois; expor a própria desonra a dois
homens é pior do que a morte!'' E dizia ainda, ofegante, que se uniria
àquele que sobrevivesse. Foi nesse momento que me tomou  um violento
desejo de matar o homem. (Comoção lúgubre.)

Ouvindo"-me falar assim, sem dúvida devo lhes parecer mais cruel do que
vocês. Mas isso é porque vocês não viram o rosto daquela mulher.
Principalmente porque não viram o ardor que brilhava em seus olhos
naquele instante. Quando olhei para aqueles olhos, quis tê"-la como
esposa, mesmo que tivesse de ser fulminado por um raio. Esposá"-la --- era
tudo o que eu queria naquele momento. Não era por nenhum desejo vil e
licencioso, como podem vocês acreditar. Se tudo o que eu sentisse fosse
um desejo físico, certamente me contentaria em dar"-lhe um pontapé e
fugir. E minha espada não se teria manchado com o sangue do homem. Mas,
no momento em que fixei o olhar naquele rosto, tomei a decisão de não
partir dali sem antes matar o seu marido.

Entretanto, mesmo que tivesse de matá"-lo, não queria fazê"-lo de forma injusta. 
Desamarrei"-lhe a corda e então lhe disse para lutarmos de igual para igual. 
(A corda que estava caída ao pé do cedro era aquela que eu tinha jogado e esquecido ali.) 
Com a expressão alterada, o homem desembainhou sua grossa espada e, sem dizer uma palavra, 
avançou em minha direção, cheio de rancor.

Bem, não há necessidade de lhes contar o fim da luta. Minha espada lhe
atravessou o peito no vigésimo terceiro golpe. No vigésimo terceiro
golpe! Não se esqueçam disso\ldots{} Porque essa façanha ainda hoje me
impressiona. Foi o único adversário em toda a minha vida a resistir a
mais de vinte golpes. (Sorriso satisfeito.)

Assim que o homem tombou, voltei"-me para a mulher, ainda segurando a
espada ensanguentada. 

Nisso, o que tinha acontecido? Não é que ela tinha desaparecido? 
Andei por entre os cedros para ver por onde fugira. Mas
não encontrei nenhum vestígio dela sobre as folhas secas de bambu. Mesmo
aguçando o ouvido, só pude distinguir os últimos gemidos do homem que
agonizava. Pode ser que, enquanto trocávamos golpes de espada, ela
tenha fugido pelo bosque para pedir socorro. Se tivesse sido assim,
minha vida é que estaria em perigo, e então, apoderando"-me da espada,
do arco e das flechas, logo voltei à estrada que percorria antes. Ali,
o cavalo da mulher ainda pastava calmamente.

O que aconteceu depois não tem nenhuma importância no caso. O único
detalhe é que, antes de entrar na Capital, desfiz"-me da espada.

Minha confissão termina aqui. Já que, cedo ou tarde, terei a cabeça
cortada e exposta nos galhos das árvores, então me condenem à pena
máxima! (Atitude desafiadora.)

\section*{confissão da mulher, que se abrigou no templo kiyomizu}

Esse homem de quimono curto azul"-escuro, após haver"-me violentado,
riu"-se com sarcasmo, enquanto observava meu marido que estava amarrado.
Como o meu marido deve ter se sentido humilhado! Mas, quanto mais se
debatia, mais a corda que o amarrava lhe penetrava dolorosamente a
carne.

Instintivamente, corri, cambaleando, em sua direção. Ou melhor, tentei
correr. Mas o homem, num rápido gesto, me derrubou com um chute. Foi
naquele exato instante que percebi nos olhos de meu marido um brilho
muito estranho. Realmente estranho\ldots{} Ainda agora, quando me lembro
daquele olhar, tremo de pavor. Não podendo emitir um único som, meu
marido transmitiu somente naquele breve olhar todos os seus
sentimentos. Mas o que então relampejou não era nem ira nem tristeza\ldots{} 
--- não é que foi um gélido brilho de desprezo? Atingida mais pela
expressão daqueles olhos do que pela brutalidade do pontapé que aquele
homem me deu, gritei alguma coisa, sem querer, e desmaiei.

Algum tempo se passou até que recuperei os sentidos, mas nessa hora o
homem de quimono azul"-escuro havia desaparecido. Vi somente meu marido
amarrado no tronco de cedro. Levantando"-me com dificuldade em meio às
folhas de bambu, fixei"-lhe os olhos no rosto. Mas seu olhar continuava
exatamente o mesmo. No fundo daquele desprezo gélido, havia também
ódio. Vergonha? Tristeza? Raiva? Nem sei como exprimir o sentimento que
passou por minha alma naquele momento. Ergui"-me quase sem forças e
dirigi"-me a meu marido:

--- Meu marido, não posso mais viver com você depois de tudo o que
aconteceu. Estou decidida a me matar. Mas\ldots{} por favor, morra junto
também. Você testemunhou a minha vergonha. Não posso permitir que
continue vivendo após a minha morte.					%\EP[-1]

Isso foi tudo o que consegui dizer. E, no entanto, ele continuava a me
olhar com repulsa. Com o coração partido de dor, passei a procurar sua
espada. Entretanto, não encontrei no bosque nem espada nem arco e
flechas; o assaltante devia ter levado tudo. Ainda bem que, pelo menos,
pude encontrar o punhal, caído no chão. Levantando"-o sobre a cabeça,
disse uma vez mais a meu marido:

--- Então, deixe"-me tomar agora a sua vida. Eu o acompanharei
imediatamente.

Quando ele ouviu essas palavras, mexeu os lábios com dificuldade. Como
sua boca estava cheia de folhas, não podia ouvir a sua voz. Mas, num
olhar, entendi o que ele queria dizer. Ainda com a mesma expressão de
desprezo, balbuciou apenas uma palavra: ``Mate"-me!''. Como em meio a um
sonho, cravei"-lhe fundo o punhal no peito, que atravessou o quimono de
caça de seda azul clara.

Devo então ter perdido novamente os sentidos. Quando voltei a mim, meu
marido, ainda amarrado, estava morto havia muito. Através da mistura de
galhos de bambu e cedros, o sol poente deixava vagar um raio de luz
sobre o seu rosto lívido. Sufocando os soluços, desamarrei a corda do cadáver.

Depois\ldots{} O que aconteceu? Quanto a isso, já não tenho mais forças para
relatar. Enfim, faltou"-me coragem para me matar. Feri"-me na garganta
com o punhal, joguei"-me no lago ao pé da montanha, tentei vários meios,
mas, uma vez que ainda estou viva, não vejo de que me orgulhar.
(Sorriso melancólico.)

Mesmo o misericordioso Boddhisatva Kannon\footnote{ No budismo japonês,
o Boddhisatva da infinita compaixão e misericórdia, que equivale a
Avalokitesvara e que, não raro, é representado com formas femininas.}
deve ter me abandonado, tão covarde que sou!

Mas eu, que matei meu próprio marido, que fui violentada, o que devo
fazer? O que posso eu\ldots{} posso\ldots{} (Soluços repentinos e violentos.)

\section*{narrativa do morto, transmitida por uma médium}

``Após violentar minha esposa, o assaltante, sentando"-se ali mesmo,
pôs"-se a confortá"-la de várias formas. Naturalmente, eu não podia
falar. Além disso, meu corpo estava amarrado ao pé de cedro.
Entretanto, lancei"-lhe várias vezes uns sinais com os olhos. `Não
acredite nas palavras dele. Tudo o que ele disser será mentira.' Era
isso o que eu lhe queria transmitir. Mas minha esposa, sentada em
desalento sobre as folhas secas de bambu, tinha os olhos fixados nos
joelhos. E não é que ela parecia estar absorvendo as palavras do
ladrão? Eu me contorcia de ciúmes. Mas o ladrão continuava a conversa,
com muita habilidade, passando de um argumento a outro. Chegou até a
fazer essa proposta atrevida: `Quando a mulher tem o corpo desonrado,
nem que seja uma única vez, as relações com o marido nunca mais podem
ser as mesmas. Em vez de continuar com seu marido, que tal ser minha
esposa? Toda a minha ousadia nasceu do amor que você me inspirou'. Ao
ouvir aquelas palavras, minha esposa ergueu a cabeça, extasiada. Nunca
vi minha mulher tão bela como naquele instante! Mas o que minha linda
esposa respondeu ao ladrão, diante de mim, ainda amarrado? Mesmo
vagando no limbo, toda vez que me lembrava de suas palavras, me
inflamava de ódio. Minha mulher respondeu"-lhe claramente:

--- Então, leve"-me para onde você for. (Longo silêncio.)

Esse não foi o único mal que ela cometeu. Se tivesse sido apenas isso,
eu não estaria sofrendo tanto nesta escuridão. Quando, conduzida pela
mão do ladrão, como num sonho, ia saindo do bosque, ela de repente
empalideceu e apontou para mim, ainda amarrado ao pé do cedro.

--- Mate este homem! Se ele continuar vivo, não poderei viver com você!

Minha esposa, como se tivesse enlouquecido, gritou várias vezes:

--- Mate este homem!

Tais palavras, como um turbilhão, ainda agora ameaçam fazer"-me despencar
no abismo sem fundo da escuridão. Será que alguma vez palavras tão
abomináveis já saíram da boca de algum ser humano? Será que alguma vez
palavras assim malditas já chegaram a ouvidos humanos? Será que alguma
vez\ldots{} (Riso súbito de escárnio.) Ao ouvir essas palavras, até mesmo o
ladrão empalideceu.

--- Mate este homem! --- assim gritando, ela lhe agarrava o braço.

O ladrão, com os olhos fixos em minha esposa, não respondia sim nem não.
No instante seguinte, derrubada por um violento pontapé, ela já estava
caída entre as folhas de bambu. (Novo riso de escárnio.) O ladrão,
cruzando calmamente os braços, voltou"-se para mim:

--- O que você quer que eu faça com ela? Mato"-a ou deixo"-a ir?\ldots{} Basta
responder movendo a cabeça. Mato"-a?

Bastariam essas palavras para que eu perdoasse o assaltante. (Outra vez,
longo silêncio.) Fiquei hesitante por um tempo, e, enquanto isso, minha
esposa gritou e saiu a correr para as profundezas do bosque. O ladrão
foi em sua direção, mas não conseguiu agarrar"-lhe sequer a manga. Como
num sonho, eu observava a cena. Depois da fuga de minha esposa, o
ladrão apanhou minha espada, arco e flechas e cortou um ponto apenas da
corda que me amarrava. Lembro"-me ainda de seu murmúrio ao sair do
bosque e desaparecer:

--- Agora, vou é tratar da minha pele\ldots{}

Depois, tudo foi silêncio\ldots{} Não, ainda se ouvia o choro de alguém.
Livrando"-me da corda, apurei o ouvido. Mas, não, era eu mesmo que
estava a chorar\ldots{} (Pela terceira vez, um longo silêncio.)

Levantei o corpo, exausto, com dificuldade. À minha frente, brilhava o
punhal que minha esposa deixara cair. Tomando"-o nas mãos, cravei"-o de
um só golpe no peito. Subiu"-me à garganta um jorro de sangue acre. Não
sentia, entretanto, dor alguma. Quando meu corpo esfriou, o silêncio em
volta se tornou mais profundo. Ah, que silêncio! Nem um único pássaro
se ouvia no céu daquele bosque à sombra das montanhas. Por entre os
bambus e cedros, havia somente um solitário raio de sol que ainda
vagava. Aquele raio ia tornando"-se cada vez mais tênue\ldots{} Nem enxergava
mais os bambus e os cedros. Senti"-me tomado por um profundo silêncio.

Nesse momento, ouvi passos furtivos de alguém se aproximando. Tentei
ver quem era. Mas a escuridão já me envolvia. Alguém --- esse alguém, com
uma mão invisível, retirou cuidadosamente o punhal do meu peito. Com
isso, mais uma vez o sangue aflorou à minha boca. Depois disso,
mergulhei na escuridão eterna do limbo\ldots{}

\begin{flushright}
\textit{Dezembro de 1921}\\  
\end{flushright}


\chapter[Memorando «Ryôsai Ogata»]{Memorando «Ryôsai Ogata»}
\hedramarkboth{Memorando «Ryôsai Ogata»}{}

\noindent\textsc{Tendo em vista} que, ultimamente, em nossa aldeia, um bando de adeptos da
seita \textit{kirishitan}\footnote{ \textit{Kirishitan}: transcrição
fonética de ``cristão'' para o japonês da época da
introdução do cristianismo.} tem praticado ritos blasfemos com os quais 
confundem o povo, foi"-me solicitado pelas autoridades um
relato minucioso de tudo o que vi e ouvi, o que faço em nome da verdade.

Passo então, modestamente, a relatar o seguinte acontecimento. No sétimo
dia do terceiro mês lunar do presente ano, uma mulher de nome Shino,
viúva de Yosaku, lavrador de nossa aldeia, veio a minha casa solicitar
encarecidamente meus préstimos para atender a sua filha Sato, de nove
anos de idade, acometida por grave doença.

A acima referida Shino, terceira filha do lavrador Sôbei, casou"-se há
dez anos com Yosaku, que faleceu pouco tempo depois de nascer sua filha
Sato; sem ter contraído novas núpcias, ela tem ganho a vida tecendo
panos e fazendo alguns pequenos serviços. No entanto, não se sabe por
que equívoco, depois da morte do marido converteu"-se numa fervorosa
seguidora da seita \textit{kirishitan}, passando a visitar assiduamente
um \textit{bateren} [padre] da aldeia
vizinha, chamado Rodrigues. Com o tempo, chegaram até nossa aldeia os
rumores de que se tornara amante do referido padre, sendo por isso
duramente criticada. Assim, a começar por seu pai Sôbei, até seus
irmãos e irmãs, todos tentaram de tudo para dissuadi"-la; no entanto,
dizendo que somente \textit{Deusu Nyôrai}\footnote{ \textit{Deusu 
Nyôrai}: \textit{Deusu} é transcrição para o japonês de ``Deus''; na
introdução do cristianismo no Japão, utilizavam"-se termos do budismo,
como \textit{Nyôrai} (um dos dez nomes de Buda, equivalente ao buda
Amida), para facilitar sua difusão. A partir daqui, as palavras 
correspondentes às transcrições japonesas serão indicadas entre colchetes.} 
era digno de veneração,
ela rejeitou todos os conselhos que lhe deram. Dia e noite, juntamente
com sua filha Sato, ela não fazia senão adorar um pequeno objeto em
forma de crucifixo denominado \textit{kurusu} [cruz], 
chegando a negligenciar as visitas ao túmulo de Yosaku, seu marido.

Diante de tais circunstâncias, dei"-lhe a entender que, apesar de seus
insistentes pedidos, não poderia atendê"-los. Na primeira vez, ela se
retirou em lágrimas para sua casa, mas voltou no dia seguinte dizendo:
``Por favor, socorra minha filha, eu lhe serei eternamente agradecida!''. 
Ela não se conformou com minhas recusas e, prostrando"-se aos prantos à
soleira da porta, pôs"-se a censurar"-me: ``Sempre achei que o dever dos
médicos fosse cuidar dos doentes. No entanto, o senhor abandona minha
filha que está gravemente enferma! Realmente, não posso entender!''. De
minha parte, respondi: ``A senhora tem toda a razão, mas eu tenho meus
motivos para me recusar a examiná"-la. Sua conduta não tem sido
aceitável e sei de fonte segura que a senhora nos tem caluniado com
frequência, afirmando que não só eu, mas todas as pessoas desta aldeia,
que adoramos nossos deuses e o Buda, estão possuídas pelo demônio. Não
vejo, portanto, por que eu, que estou tomado pelo demônio, deva curar a
moléstia da filha de alguém como a senhora, que se considera pura e no
caminho certo. A senhora deveria fazer esse pedido ao Senhor Deus a
quem cultua todos os dias; se a senhora deseja tanto os meus socorros,
eu lhe apelo que abjure a fé dos \textit{kirishitan}. Caso não aceite
tais condições, por mais que me repita que a medicina é a arte de
cuidar do próximo, recuso"-me terminantemente a atender a seu pedido,
pois também receio me expor aos castigos dos deuses e de Buda.'' Shino
pareceu não mais encontrar palavras para insistir e se retirou, consternada.

No dia seguinte, nono dia do terceiro mês lunar, chovia muito, desde a madrugada,
o que momentaneamente fizera cessar o movimento na aldeia, quando, por
volta das seis horas, Shino voltou a aparecer, sem guarda"-chuva,
molhada até os ossos, para reiterar insistentemente o seu pedido, ao
qual respondi: ``Pessoa de bem que sou, não tenho duas palavras.
Portanto, acredito ser importante que a senhora ou renuncie ao seu
Senhor Deus, ou sacrifique a vida de sua filha.'' Ao ouvir minhas
palavras, Shino desta vez parecia ter enlouquecido: de joelhos,
abaixando a cabeça várias vezes e com as mãos postas no chão, em
súplica, rogou, caindo em prantos: ``O senhor tem absolutamente toda a
razão. Mas, segundo os ensinamentos cristãos, uma vez que reneguemos
nossa fé, nossos corpos e almas passarão a arder encarnações e
encarnações, por séculos e séculos. Por favor, tenha piedade de mim! Eu
lhe suplico: não me imponha esta condição!''. Embora fosse adepta de uma
seita perversa, isso não parecia ter mudado em nada seu coração de mãe.
Senti um pouco de compaixão por ela, mas não podia permitir que
sentimentos pessoais interferissem no interesse geral e lhe disse que,
por mais razões que me apresentasse, se não renegasse sua fé, não
poderia socorrer sua filha. Shino, numa expressão indescritível, por
instantes fixou seu olhar em meu rosto. De repente, as lágrimas
escorreram em profusão e, com as mãos postas no chão, a meus pés, ela
se pôs a murmurar numa voz débil que lembrava o zunido do pernilongo.
Mas, devido ao barulho da forte chuva que começava a cair, não pude
escutá"-la bem. Após fazê"-la repetir duas ou três vezes o que dissera,
compreendi finalmente que, sem ter mais alternativa, ela decidira
renegar a fé. Mas suas palavras não podiam me assegurar sua decisão, de
modo que lhe solicitei uma prova concreta. Ela retirou em silêncio a
cruz anteriormente referida de sob as vestes, colocou"-a sobre a madeira
da soleira e pisou"-a calmamente por três vezes. Nessa hora, ela não se
mostrou particularmente perturbada, suas lágrimas pareciam já haver
secado, mas seus olhos, que contemplavam a cruz sob seus pés, ardiam
como os de uma doente acometida de forte febre, o que impressionou
muito a todos os meus serviçais.

Tendo sido, pois, atendida minha condição, fiz meu serviçal trazer a
caixa de medicamentos e, debaixo de forte chuva, acompanhei Shino até
sua casa. Num quarto minúsculo, Sato dormia sozinha, com a cabeceira
voltada para o sul. Ela parecia estar fora de si, em consequência de
uma febre alta: com suas frágeis mãos de criança, repetidas vezes
traçava no ar sinais em forma de cruz, sorrindo de alegria a cada vez
que murmurava, como que num outro mundo, a palavra
\textit{harureya} [aleluia]. Chorando, Shino me
explicou à cabeceira da criança que \textit{harureya} era uma palavra
proferida pelos cristãos ao louvar seu Senhor. Imediatamente, tomei o
pulso da doente; tratava"-se, sem dúvida, de uma febre maligna; já era
tarde demais e, provavelmente, ela não passaria daquele dia. Nada mais
podia fazer, e contei a verdade a Shino, que se pôs novamente, como uma
louca, a dizer: ``Se reneguei minha fé, foi só porque queria salvar a
vida de minha filha. Se ela morrer, meu sacrifício terá sido em vão.
Por favor, tenha piedade do meu sofrimento por ter voltado as costas ao
Senhor Deus e salve a vida de minha filha, custe o que custar!''. 
Prostrando"-se não só diante de mim, mas também de meu serviçal, ela
continuava suplicando, porém não havia mais nenhum recurso humano que a
salvasse. Aconselhando"-a insistentemente a não se deixar tomar pelo
desespero, deixei"-lhe três envelopes de folhas de infusão e,
aproveitando um momento em que a chuva amainara, já me preparava para
me retirar, quando Shino se agarrou às mangas de minha veste. Ela movia
os lábios tentando dizer"-me alguma coisa, mas, antes que pudesse
proferir uma só palavra, foi empalidecendo e caiu sem sentidos. Tomado
de grande susto, corri para socorrê"-la com a ajuda de meu serviçal. Ela
recobrou os sentidos, mas não tinha forças para se manter de pé. ``Que
desgraça! Devido à minha leviandade, não só perco minha filha como
também o Senhor Deus!'', dizia ela, chorando copiosamente. Tentei
consolá"-la de diversas maneiras, mas parecia não me escutar. Uma vez
que não havia mais nada que eu pudesse fazer, dado o estado da criança,
voltei às pressas para casa com meu serviçal.

No entanto, naquele mesmo dia, depois das duas horas da tarde, quando
fui à casa de Yazaemon Tsukagoshi, o chefe da aldeia, para examinar sua
mãe, fiquei sabendo que a filha de Shino havia morrido e que ela, por
sua vez, acabara enlouquecendo de tanta tristeza. Segundo consta, Sato
veio a falecer uma hora depois da minha visita e, por volta das dez
horas da manhã, Shino, já com manifestações de demência, agarrando"-se
ao corpo inerte da filha, proferia rezas numa língua bárbara. Cumpre
assinalar que o senhor Yazaemon foi testemunha de tudo o que ocorreu,
assim como os senhores Kaemon, Tôgo, Jihei, todos moradores desta
aldeia, que, por estarem presentes naquele momento, podem confirmar o
fato com exatidão.

No dia seguinte, décimo dia do terceiro mês lunar, caía uma chuva fina
desde cedo, mas, a partir das oito horas, após reboarem as trovoadas de
primavera, o tempo pareceu se abrir. Foi quando o senhor Kinjûrô Yanase, 
lavrador e também soldado, enviou"-me um cavalo, solicitando
minha presença para uma consulta. Parti imediatamente. Quando passava
diante da casa de Shino, uma multidão à sua frente vociferava:
``\textit{Bateren! Kirishitan!}''. A agitação era tanta que não pude
fazer avançar o cavalo, de modo que, do alto da sela, espreitei o
interior da casa. A porta estava totalmente aberta e, em seu interior,
um estrangeiro de cabelos vermelhos e mais três japoneses, todos
vestidos de preto como os monges, elevavam em suas mãos aquelas
referidas cruzes e uma espécie de incensório, entoando em coro:
``\textit{Harureya! Harureya!}''. E não era tudo. Aos pés do referido
estrangeiro, Shino, os cabelos emaranhados, agarrada ao corpo da filha,
estava prostrada, como que inconsciente. O que, no entanto, mais me
saltou à vista foi que Sato, os dois braços firmemente em volta do
pescoço da mãe, ora balbuciava seu nome, ora entoava aleluias, numa voz
cândida. É verdade que a distância me impedia de distinguir os detalhes
com clareza, mas o rosto de Sato parecia estar cheio de viço. De vez em
quando tirava os braços do pescoço da mãe, como que tentando agarrar a
fumaça que se elevava daquela espécie de incensório. Desci, então, do
cavalo e interroguei os aldeões sobre a ressurreição de Sato, ao que me
responderam, atemorizados, que, naquela manhã, Rodrigues, o
\textit{bateren} de cabelos vermelhos antes referido, chegara da aldeia
vizinha acompanhado dos \textit{iruman} [irmãos]; 
após ouvir a confissão de Shino, todos juntos recitaram
suas preces, alguns queimando um incenso de aroma exótico, outros
espargindo água"-benta, até que a insanidade de Shino foi se acalmando e
Sato veio a ressuscitar. É verdade que, desde os tempos remotos, já
houve ressurreições, mas, em sua maior parte, referem"-se a casos de
envenenamento por álcool ou de contaminação por miasmas. Nunca tive
conhecimento de uma história como a de Sato, em que a vida reanimasse
um morto acometido por doença maligna. Portanto, só esse fato já nos
revela a bruxaria desta seita \textit{kirishitan}. Somam"-se a isso os
estrondos da trovoada de primavera, prova da ira dos céus, no momento
em que os \textit{bateren} se dirigiam a esta aldeia.

Cumpre ainda notar que, nesse mesmo dia, Shino e sua filha Sato foram
levadas à aldeia vizinha pelo \textit{bateren} Rodrigues e que sua casa
foi queimada por ordem do monge"-chefe Nikkan, do templo Jigen. Tendo em
vista que esses fatos já foram relatados a Vossa Senhoria pelo chefe da
aldeia, o senhor Yazaemon Tsukagoshi, limito"-me a relatar os fatos que
testemunhei. Se houver qualquer omissão involuntária neste meu
relatório, farei chegar a seu conhecimento uma nota complementar.

Sem mais para o momento,

Ano do Macaco, vigésimo dia do terceiro mês lunar.

Província de Iyo, distrito de Uwa, aldeia de\ldots{}

Ryôsai Ogata, médico.

\begin{flushright}
\textit{Dezembro de 1916}\\  
\end{flushright}


\chapter{Ogin}

\noindent\textsc{Terá sido} na era Genna? Ou na era Kan'ei?\footnote{ Trata"-se de divisões
históricas do Japão: era Genna (1615--1624); era Kan'ei (1624--1644).}
De qualquer forma, foi há muito, muito tempo.

Naquela época, os seguidores da Santa Doutrina do Senhor Deus, uma vez
descobertos, ainda eram queimados vivos ou crucificados. Mas tudo nos
leva a crer que, quanto mais violentas eram as perseguições aos fiéis
no país, mais miraculosa era a proteção do Santo Deus, que tudo podia
realizar. As aldeias de Nagasaki às vezes eram visitadas por anjos e
santos sob os raios do sol poente. De fato, dizem que uma vez até São
João Batista fez sua aparição no moinho de Miguel Yahei, um fiel de
Urakami. E o diabo, por seu lado, para impedir a devoção dos fiéis,
também aparecia naquelas vilas, tomando a figura antes nunca vista de
um negro, ou de plantas e flores de terras estrangeiras, ou de carroças
cobertas de bambu trançado. Dizem também que os ratos que atormentaram
Miguel Yahei no calabouço, onde os dias e as noites se confundiam, eram, na verdade,
encarnações do diabo. Yahei foi queimado vivo no outono do ano oito da
era Genna, juntamente com mais onze fiéis. Teria sido na era Genna? Ou
na era Kan'ei? De qualquer forma, foi há muito, muito tempo.

Também nessa mesma Urakami, na aldeia de Yamazato, vivia uma jovem
donzela chamada Ogin. Seus pais  vagaram a esmo desde Ôsaka até atingir
a longínqua Nagasaki. Mas, antes que pudessem se estabelecer de maneira
adequada, ambos partiram deste mundo, deixando Ogin órfã. Naturalmente,
forasteiros que eram, não podiam conhecer os santos ensinamentos do
Senhor Deus. Era no budismo que eles acreditavam. Zen, Hokke ou Jôdô,
pouco importa a corrente --- eles tinham fé nos ensinamentos de Buda.
Dizia um jesuíta francês que Shakyamuni, 
ardiloso por natureza, percorrera várias cidades da China e propagara
os ensinamentos do Caminho do Buda Amida. Depois, veio também ao Japão
para ensinar o mesmo caminho. De acordo com os ensinamentos por ele
difundidos, a alma humana metamorfoseia"-se em ave, animal ou planta,
conforme a gravidade de seus pecados. Não somente isso, dizia ainda o
jesuíta que Shakyamuni, ao nascer, matara sua mãe. ``O exagero dos ensinamentos 
de Shakyamuni é tão evidente quanto é clara a sua infâmia'' (Jean Crasset). 
Mas, como já foi dito antes, os pais de Ogin não tinham conhecimento dessas
verdades. Mesmo depois de exalarem seu último suspiro, continuavam
ainda acreditando nos ensinamentos de Shakyamuni. No cemitério
desolado, à sombra dos pinheiros, ignorando que acabariam caindo no
\textit{inheruno} [Inferno], sonhavam como o evanescente Paraíso.

Mas, felizmente, Ogin não fora influenciada pela ignorância dos pais. O
caridoso João Magoshichi, um camponês da aldeia de Yamazato, há muito
havia aspergido a água"-benta do \textit{bapuchizumo} [batismo] na testa
da jovem, conferindo"-lhe o nome de Maria. Ogin não acreditava em
histórias como a de que Shakyamuni, por ocasião de seu nascimento,
teria proclamado com autoridade, enquanto apontava o céu e a terra: ``No
céu como na terra, sou o único digno de se honrar''. Acreditava, sim,
na concepção imaculada da ``Santa Maria, Mãe de Misericórdia, Vida,
Doçura, Esperança nossa!''. Acreditava na ressurreição de Jesus ao
terceiro dia, que ``foi crucificado, morto e sepultado em túmulo de
pedra, que desceu à mansão dos mortos'' nas profundezas da terra.
Acreditava que ao soar das trombetas do Juízo Final ``o Senhor descerá
dos céus em toda a Sua Glória e Poder, reunindo às almas os corpos que
haviam se transformado em pó, elevando os bons à felicidade dos Céus e
fazendo descer os maus ao Inferno juntamente com o Diabo''. Acreditava
principalmente no Santo Sacramento, no qual ``o pão e o vinho por obra
do Divino Espírito Santo se transubstanciam no sangue e na carne de
Nosso Senhor Jesus Cristo''. O coração de Ogin não era como o dos pais,
um deserto assolado por ventos escaldantes. Era um fértil trigal,
matizado por singelas rosas silvestres. Depois da perda dos pais, Ogin
foi adotada por João Magoshichi. Joana Osumi, sua esposa, tinha, também
ela, bom coração. Ogin passava dias felizes a seu lado, cuidando do
gado e colhendo trigo. Mas, mesmo levando tal vida, desde que os
aldeões não o percebessem, não negligenciava suas orações e jejuns.
Frequentemente, à sombra da figueira junto ao poço, com os olhos
voltados para a enorme lua crescente, absorvia"-se numa profunda prece.
A oração dessa jovem de cabelos escorridos era, assim, muito simples:
``Santa Mãe de Misericórdia, rendemos"-vos Graça. A Vós bradamos, os
degredados filhos de Eva. Dirigi"-nos, nós Vos suplicamos, Vossa
misericórdia nesse vale de lágrimas. Amém!''

Ocorreu então que, num certo ano, na noite de \textit{Natara} [Natal], 
o Diabo adentrou repentinamente a casa de
Magoshichi, juntamente com algumas autoridades da aldeia. Na grande
lareira da casa, a lenha da vigília de Natal ardia em chamas. E para
aquela noite havia sido colocada, na parede enegrecida pela fuligem,
uma cruz para o culto. Se fossem à cocheira nos fundos da casa,
encontrariam água na manjedoura para as primeiras abluções de Jesus. As
autoridades trocaram sinais com a cabeça e ataram o casal com cordas.
Ogin também foi amarrada. Mas nenhum deles demonstrava ter consciência
de qualquer maldade. Estavam prontos a receber qualquer suplício pela
salvação de suas almas. O Senhor, certamente, conceder"-lhes"-ia a
proteção divina. Além do mais, o fato de terem sido presos na noite de
Natal não seria uma prova da imensa graça de Deus? Eles estavam, todos
os três, convictos, como se tivessem acertado tudo entre eles. Depois
de amarrá"-los, as autoridades levaram"-nos à residência do governador
local. Durante o percurso, mesmo sendo açoitados pelo vento noturno, os
três não cessavam de recitar as preces de Natal.

--- Oh, Senhor, que nascestes em Belém, onde estareis agora? Vosso nome
seja louvado!

Vendo"-os serem presos, o Diabo se regozijou, batendo palmas. Mas parece
ter ficado muito irritado pela atitude deles, de corajosa resignação.
Quando se viu só, o Diabo cuspiu com um desgosto cheio de desdém,
transformando"-se imediatamente num grande pilão de pedra e, rolando
ruidosamente pelo chão, desapareceu nas trevas.

Os três cristãos, João Magoshichi, Joana Osumi e Maria Ogin, além de
serem jogados num calabouço, sofreram torturas sem conta para que
abjurassem os santos ensinamentos do Senhor. Mas os suplícios da água e
do fogo não foram suficientes para abalar sua convicção. Mesmo que sua
pele e carne se dilacerassem, apenas um sopro mais de paciência e
atingiriam as portas do \textit{Haraiso} [Paraíso]. Pensando na
misericórdia infinita de Deus, até aquele calabouço escuro não diferia
do esplendor do Paraíso. Além do mais, veneráveis santos e anjos
frequentemente vinham confortá"-los enquanto se encontravam entre o
sonho e a vigília. Ogin parecia ter sido especialmente tocada por essa
felicidade. Ela viu São João Batista dizendo"-lhe que comesse os
gafanhotos que ele trazia, em abundância, nas palmas grandes de suas
mãos. E viu também o Arcanjo São Gabriel, com suas asas brancas
fechadas, a lhe oferecer água numa linda taça dourada.

É obvio que o governador local não conhecia os ensinamentos do Senhor,
mas ignorava também os de Shakyamuni, de modo que não entendia a causa
da obstinação dos três. Às vezes até se perguntava se não seriam, todos
eles, loucos. Mas, quando constatava que não eram loucos, passava a
vê"-los como grandes serpentes, ou como unicórnios, enfim, animais que
nada tinham a ver com a espécie humana. Deixar vivos esses animais não
somente era uma transgressão à lei, mas também uma ameaça à segurança
local. Por isso, após mantê"-los presos um mês no calabouço, resolveu
condená"-los à morte pelo fogo\ldots{} Na realidade, esse governador local,
como a maioria das pessoas, pouco se indagou se eles constituíam mesmo
uma ameaça à segurança local. Isso porque, em primeiro lugar, havia a
lei e, em segundo, a moral do povo. Assim, não viu nenhum inconveniente
em não se esforçar por esclarecer melhor a questão.

Os três fiéis, com João Magoshichi à frente, não demonstravam medo algum
nem quando se dirigiam ao campo de suplícios, situado fora da aldeia. A
execução teve lugar num terreno vazio, pedregoso, bem ao lado do
cemitério. Ao chegarem, ouviram suas sentenças uma a uma e depois
foram amarrados a um largo pilar anguloso. Os três foram colocados no
centro do campo, com Joana Osumi à direita, João Magoshichi no meio e
Maria Ogin à esquerda. Osumi, devido à tortura que sofrera por dias
seguidos, parecia ter envelhecido de um dia para o outro. Magoshichi
tinha as faces exangues em meio à barba crescida. Quanto a Ogin,
comparada aos dois, não mostrava nenhuma mudança acentuada. Mas todos
os três, pisando firmemente a lenha empilhada, tinham o semblante tranquilo.

Ao redor do local, já uma multidão de curiosos se apinhava. No céu,
acima de todas aquelas pessoas, cinco ou seis pinheiros do cemitério
estendiam seus galhos como um pálio de seda.

Quando terminaram os preparativos, um funcionário aproximou"-se
cerimoniosamente dos três e lhes perguntou se abjuravam, ou não, os
santos ensinamentos do Senhor, dizendo"-lhes que meditassem bem uma vez
mais, que lhes dariam ainda algum tempo e que, se dissessem que sim,
suas cordas imediatamente lhes seriam retiradas. Mas eles nada
responderam. Tinham os olhos fitos num ponto do céu longínquo e em seus
lábios até se esboçava um sorriso. O povo, e principalmente as
autoridades, em momento algum se manteve em tão profundo silêncio
quanto naqueles poucos instantes. Inúmeros olhos se fixaram, sem
piscar, sobre aqueles três rostos. Mas não era a piedade que lhes
sustinha a respiração. É que os curiosos, em sua maioria, esperavam
impacientemente a hora de se atear o fogo. E os funcionários, por seu
lado, muito entediados pela demora da execução, haviam perdido até o
ânimo para conversar.

De repente, os ouvidos dos presentes forma surpreendidos por uma frase
que, inesperadamente, soou alta e clara:

--- Eu declaro abjurar os santos ensinamentos do Senhor!

Era a voz de Ogin. Num momento, a agitação tomou conta da multidão. Mas
no instante seguinte ao alvoroço, o silêncio retornou, ao se fazer
ouvir a voz debilitada de Magoshichi, que, voltando"-se para Ogin, disse
com tristeza:

--- Ogin! Foste tentada pelo Diabo? Mais um pouco de paciência e poderás
adorar a santa face do Senhor!

Antes que ele terminasse de falar, Osumi também se voltou para a filha
adotiva e, à distância, disse"-lhe, suplicante:

--- Ogin! Ogin! Foste possuída pelo diabo! Reza, por favor, reza!

Mas Ogin não lhes respondeu. Mantinha os olhos fixos nos pinheiros do
cemitério, que estendiam seus galhos como um pálio de seda acima da
multidão. E, logo depois, um dos funcionários ordenou que lhe
retirassem a corda.

Vendo isso, João Magoshichi cerrou os olhos como que resignado.

--- Senhor Deus, que tudo podeis, seja feita a Vossa vontade! --- murmurou ele.

Livre finalmente das cordas, Ogin se manteve de pé por uns instantes,
atordoada. Mas, ao ver seus pais adotivos, caiu de joelhos diante deles
e verteu lágrimas sem nada dizer. Magoshichi mantinha os olhos
cerrados. Osumi, com o rosto virado para o outro lado, nem fez menção
de olhar para ela.

--- Papai, mamãe! Perdoai"-me --- disse Ogin, enfim. --- Abjurei os
ensinamentos do Senhor, pois percebi, de repente, os galhos daqueles
pinheiros que se estendem como um pálio de seda, ali adiante. Meus pais
de nascimento, que ali repousam à sombra dos pinheiros do cemitério,
não conheceram os santos ensinamentos do Senhor e certamente agora
devem estar no Inferno. Como poderia eu me justificar diante deles, se
entrasse sozinha pela portas do Paraíso? Quero, sim, segui"-los até as
profundezas do Inferno. Papai! Mamãe! Ide ambos, eu vos suplico, para
junto de Nosso Senhor Jesus Cristo e da Santa Virgem Maria. Mas eu, uma
vez que abandonei a Santa Doutrina, já não posso continuar vivendo\ldots{}

Ogin, após proferir essas palavras num ritmo entrecortado, mergulhou em
soluços. Desta vez, foi Joana Osumi quem se pôs a derramar lágrimas
silenciosas sobre a lenha abaixo de seus pés. Perder"-se em
recriminações inúteis no momento em que estavam prestes a entrar no
Paraíso não era, naturalmente, um ato digno de um fiel de Cristo.
Então, João Magoshichi, voltando"-se amargamente para a esposa,
repreendeu"-a com voz estridente:

--- Também foste possuída pelo Diabo? Renega os ensinamentos do Senhor, se
assim quiseres. Mas eu, mesmo só, morrerei nas chamas.

--- Oh, não! Eu te acompanharei. Mas é que\ldots{} é que\ldots{} --- e, engolindo as
lágrimas, continuou quase aos gritos --- \ldots{} não é porque eu queira ir ao
Paraíso, e só\ldots{} porque eu quero acompanhar"-te.

Magoshichi se manteve calado por um longo tempo. Seu rosto ora
empalidecia, ora se tingia da cor do sangue. Ao mesmo tempo, gotas de
suor começavam a se formar. Naquele momento, ele contemplava sua alma,
com os olhos do coração. Contemplava sua alma, que era disputada pelo
Anjo e pelo Diabo. Se Ogin, a seus pés, não houvesse naquele instante
erguido para ele o rosto molhado de lágrimas\ldots{} Mas ela havia erguido
seu rosto. Além disso, seus olhos inundados de lágrimas abrigavam um
brilho misterioso e haviam"-no fitado longamente. O que fulgurava no
fundo de seus olhos não era somente a alma inocente de uma donzela, era
também a alma de toda a espécie humana, os ``degredados filhos de Eva''.

--- Papai! Vamos juntos para o Inferno! Mamãe também, eu também, também 
meus pais que ali descansam, deixemos"-nos tomar pelo Diabo!

Magoshichi finalmente se rendeu.

Esta história, dentre várias outras acerca dos mártires de nosso país,
foi transmitida à posteridade como um dos fracassos mais vergonhosos.
Dizem que os curiosos, mesmo aqueles que ignoravam o que fosse aquele
``Deus'' --- homens e mulheres, jovens e velhos ---, odiaram os três quando 
abjuraram sua fé. Talvez se tenham ressentido por não terem podido
assistir ao suplício pelo qual tanto esperavam. Conta"-se também que o
Diabo, tomado nessa hora por imenso regozijo, transformou"-se em um
grande livro e sobrevoou o campo de execução durante toda a noite. Mas
o autor se pergunta com ceticismo se o Diabo deveria ter se alegrado
tão excessivamente com uma vitória tão duvidosa.

\begin{flushright}
\textit{Agosto de 1922}\\  
\end{flushright}

\chapter{O mártir}

\epigraph{Ainda que vivamos trezentos anos e nos saciemos de todos os prazeres,
que será tudo, se não um sonho efêmero, em comparação com este júbilo
eterno?}{\textit{Guia do Pecador}\\ (Fragmento de uma tradução feita na era
Keichô)\footnotemark}
\footnotetext{ Era Keichô: período histórico japonês, de 1596 a 1615.}

\epigraph{Aqueles que entrarem no caminho do Bem conhecerão a indizível doçura
da Santa Doutrina.}{\textit{Imitatione Christi}\\ (Fragmento de uma tradução feita na
era Keichô)}

\sectionitem

\noindent\textsc{Havia naquela} época, numa \textit{Ekereshiya} [igreja] chamada Santa
Lúcia, em Nagasaki, no Japão, um rapaz nativo daquele país que tinha
por nome Lorenzo. Haviam"-no encontrado na noite da festa da Natividade,
prostrado de fome e de fadiga diante do portão da \textit{Ekereshiya}.
Os fiéis que ali iam rezar o socorreram e o padre, tomado de piedade,
resolveu adotá"-lo na igreja. Mas, coisa curiosa, cada vez que lhe
perguntavam sobre sua origem, nunca esclarecia a verdade e, rindo com
naturalidade, dizia que sua terra natal era \textit{Haraiso} [Paraíso]
e que o nome de seu pai era \textit{Deusu}. Mas, como um
\textit{kontatsu} [rosário] de contas azuis enrolado em torno de seu
punho indicava que ele não havia nascido de pais \textit{zencho} [gentios, pagãos], 
o padre superior e todos os outros \textit{Irumano} [Irmãos de fé] 
concluíram que não havia razão alguma para suspeitar
dele e o ajudaram com muito boa disposição. Apesar de sua pouca idade,
o ardor de sua devoção impressionava os \textit{Superioresu} [Padres
superiores]. Todos diziam que ele sem dúvida era a encarnação de um
anjo e, embora ignorassem onde nascera e quem eram seus pais, tinham
por Lorenzo uma afeição sem limites. Além disso, seu rosto puro como
uma pérola, sua voz delicada, quase feminina, devem ter inspirado ainda
mais caridade entre os fiéis. Entre eles encontrava"-se o também nativo
Irmão Simeão, que o tratava como se fosse seu irmão mais velho,
podendo"-se sempre vê"-los de mãos dadas como dois amigos, dentro ou fora
da igreja. Simeão vinha de uma família importante de samurais que
estivera a serviço de um daimiô importante. Sendo de estatura notável e
tendo por natureza uma força extraordinária, não foram somente uma ou
duas vezes que ele protegeu os Padres dos apedrejamentos dos infames
\textit{zencho}. Quando era visto com Lorenzo, tinha"-se a impressão de
uma águia selvagem estava ligada pela amizade a uma pombinha. Ou de uma
videira selvagem em flor que se enrolava em torno dos ciprestes do
Monte Líbano. Mais de três anos rapidamente se passaram e, afinal,
Lorenzo estava pronto para a cerimônia de iniciação ao mundo dos
adultos.\footnote{ \textit{Genpuku}: cerimônia de maioridade realizada
para os meninos entre os 12 e 15 anos de idade. } No entanto, naquela
época, espalhou"-se um boato duvidoso segundo o qual a filha de um
comerciante de guarda"-chuvas de um bairro não muito distante de Santa
Lúcia se encontrava com Lorenzo. O velho comerciante, que também
cultuava os ensinamentos do Senhor \textit{Deusu}, frequentava a igreja
com a filha. Os olhos da moça, mesmo entre as preces, não se desviavam
de Lorenzo, que, incensório à mão, servia no altar. Sempre que ia à
igreja, penteava"-se cuidadosamente, nunca deixando de lançar olhares em
sua direção. Aquela atitude não poderia passar despercebida aos outros
fiéis: um dizia que a jovem havia pisado, de passagem, o pé de Lorenzo;
outro jurava mesmo haver visto os dois jovens trocarem cartas de amor.
Então, parece que o padre pensou não poder deixar aquela situação como
estava. Um dia, tendo convocado Lorenzo, perguntou"-lhe docemente,
mordendo a ponta da barba branca:

--- Ouvi falar de uns rumores sobre você e a filha do comerciante de
guarda"-chuvas. Isso não é verdade, é? E então?

A essas palavras, Lorenzo, meneando tristemente a cabeça, não fez mais
que repetir numa voz chorosa:

--- Não há motivos para o senhor acreditar numa coisa dessas, meu pai.

O padre, de fato, ouvindo"-o tão afirmativo, teve de se render à ideia de
que, dadas sua idade e sua constante devoção, Lorenzo não poderia estar
mentindo. A dúvida foi assim provisoriamente dissipada do espírito do
padre. Mas entre os fiéis de Santa Lúcia os rumores não se extinguiram
tão facilmente. Mesmo Simeão, para quem Lorenzo era como um irmão,
estava extremamente desconfiado. No início tinha vergonha de inquiri"-lo
abertamente sobre tais rumores indecentes, não ousando sequer olhar
Lorenzo no rosto. Um dia, no entanto, no jardim de trás da igreja de
Santa Lúcia, ele encontrou uma carta de amor da moça endereçada a
Lorenzo e, aproveitando"-se de estarem a sós num aposento, sacudiu"-lhe a
carta sob o nariz e interrogou"-o, alternando ameaças e persuasão. No
entanto, Lorenzo, o rubor manchando seu belo rosto, contentou"-se em
responder:

--- Sem dúvida essa jovem está enamorada de mim. Mas, quanto a mim, o que
fiz foi receber suas cartas, nada mais. Não lhe dei nem mesmo a
oportunidade de falar comigo.

Mas Simeão, sensível à reprovação pública, prosseguiu seu severo
interrogatório. Lorenzo, olhando fixamente o Irmão com um ar desolado,
disse"-lhe como se o repreendesse:

--- Então, quer dizer que eu seria capaz de mentir até a você?

E, virando"-se com a rapidez de uma andorinha, deixou bruscamente o
quarto. A essas palavras, Simeão, envergonhado por ter ido longe demais
em sua suspeita, estava para sair também ele, desalentado, quando
alguém entrou subitamente no aposento. Era Lorenzo. Lançou"-se no
pescoço de Simeão, abraçando"-o, e sussurrou arquejante:

--- A culpa foi minha! Perdoe"-me!

E, antes que Simeão tivesse tempo de lhe dizer qualquer coisa, talvez
para esconder o rosto molhado de lágrimas, Lorenzo se afastou
bruscamente, quase atropelando"-o, e fugiu inopinadamente na direção de
onde viera. Dizem que Simeão se perguntou então, sem poder tirar
nenhuma conclusão, se por aquelas palavras ``A culpa foi minha'' o rapaz
se arrependia de sua ligação secreta com a jovem ou se simplesmente
expressava o arrependimento de lhe haver respondido com certa
brusquidão.

Pouco tempo depois, espalhou"-se o boato de que a filha do comerciante de
guarda"-chuvas estava grávida. A filha declarou a seu pai que esperava
um filho de Lorenzo de Santa Lúcia. Assim, o velho comerciante de
guarda"-chuvas, ardendo de raiva, não tardou a clamar por justiça ao
padre. Acuado pelos acontecimentos, Lorenzo não encontrava nenhuma
justificativa em seu favor. No curso do mesmo dia, após uma reunião com
o padre e os irmãos, decidiu"-se que Lorenzo seria excomungado. Isso
feito, ele deveria perder também a proteção do padre e, evidentemente,
se acharia sem recursos. Com efeito, guardar tal pecador na comunidade
de Santa Lúcia poderia comprometer a \textit{Guroriya} [Glória] de
Nosso Senhor e, assim, mesmo aqueles que lhe queriam bem, contendo as
lágrimas, consentiram em excomungar Lorenzo.

O mais infeliz de todos era Simeão, que o considerava seu irmão. Mais do
que tristeza pelo fato de Lorenzo ter sido excomungado, o que Simeão
sentia era cólera por ter sido enganado por ele. Assim, esmurrou
violentamente o belo rosto de Lorenzo, no momento em que o frágil
rapaz, sem forças, deixava sozinho a \textit{Ekereshiya} de Santa
Lúcia, enfrentando o vento do início de inverno. Diz"-se que Lorenzo
caiu por terra com a força do golpe brutal, mas que se levantou em
seguida e que, erguendo aos céus os olhos marejados de lágrimas, disse
numa voz trêmula:

--- Senhor, perdoai"-o! Simeão não sabe o que faz!

A essas palavras, Simeão perdeu o alento. Por uns instantes, permaneceu
de pé à porta, dando murros no vazio, mas os outros irmãos intervieram
acalmando"-o e, cruzando então os braços, o rosto arisco e sombrio como
um céu que anuncia tempestade, ficou a olhar fixamente, como se a
quisesse devorar, a figura de Lorenzo, que, esmorecido, atravessava o
portão de Santa Lúcia. Segundo os fiéis que assistiram à cena, o astro
Sol, tremulando ao vento frio, declinava no horizonte a oeste de
Nagasaki, justamente do lado para onde se dirigia a cabeça solitária de
Lorenzo; dizem que sua silhueta delicada se sobrelevava dentro da flama
crepuscular que abrasava o céu. Depois disso, Lorenzo, completamente
diferente da época em que portava o incensório diante do altar de Santa
Lúcia, transformou"-se num indigente miserável, vegetando num arrabalde
distante da cidade, em um dos barracos destinados aos
\textit{hinin}.\footnote{ \textit{Hinin}: ``não"-pessoas'', não pertenciam
a nenhuma classe social no período Edo (1603--1868); viviam da
mendicância, da execução e do enterro dos condenados e do mundo das
diversões vulgares.} Além do mais, para a comunidade dos
\textit{zencho}, ele era desprezado como os \textit{etori},\footnote{ \textit{Etori} ou 
\textit{eta}: juntamente com os \textit{hinin}, não
eram considerados classe social no período Edo (1603--1868);
dedicavam"-se principalmente ao abate de bois e cavalos e ao tratamento
de couros.} pois cultuava os santos ensinamentos do Senhor. Cheguei
até a saber que, quando por essa época ele caminhava pela cidade, era
objeto de escárnio entre as crianças desapiedadas; às vezes ele era até
ameaçado com golpes de espadas e de varas, ou lhe atiravam pedras e
telhas quebradas. E diziam também que, uma vez, acometido por uma febre
assustadora que grassava na cidade de Nagasaki, sofreu convulsões e
ficou prostrado à beira do caminho por sete dias e sete noites. Mas a
misericórdia infinita de \textit{Deusu} salvou a vida de Lorenzo mais
de uma vez: quando, por exemplo, ele não recebia esmolas, arroz nem
dinheiro, \textit{Deusu}, então, o provia de alimentação diária,
ofertando"-lhe frutas das árvores das montanhas, peixes e conchas do
mar. Dizem também que Lorenzo, de sua parte, sem se esquecer do tempo
em que estivera em Santa Lúcia, rezava de manhã e à noite, conservando
enrolado em torno do punho o rosário, que guardava ainda sua cor fresca
e azulada. Imaginem! Isso não é tudo: tarde da noite, quando cessava
todo ruído humano e tudo mergulhava no silêncio, aquele jovem saía
secretamente do arrabalde, deixando seu barraco de \textit{hinin}, e,
andando sob a claridade da lua, dirigia"-se a Santa Lúcia, que fora seu
lar por tanto tempo, a fim de implorar a proteção do Senhor 
\textit{Zesu Kirishito} [Jesus Cristo].

Mas, nessa época, os fiéis que frequentavam a Igreja já o tratavam com
frieza; ninguém, nem mesmo o padre, tinha piedade dele. E com toda a
razão. Convencidos de sua conduta amoral desde sua excomunhão, como
poderiam imaginar que ele fosse um praticante tão fervoroso a ponto de
ir rezar todas as noites, sozinho, na igreja? Mesmo considerando que
nada poderia fazer contra aquele desígnio insondável de \textit{Deusu},
que infelicidade Lorenzo devia suportar!

Agora, voltemos à filha do comerciante de guarda"-chuvas. Pouco após a
excomunhão de Lorenzo, ela deu à luz uma menina prematura. Até mesmo
seu velho pai, apesar da intransigência de caráter, não pôde deixar de
se enternecer à vista da primeira netinha. Juntamente com a filha,
tratava cuidadosamente do bebê, chegando mesmo a carregar a criança em
seus braços, trazendo"-lhe às vezes uma boneca para alegrá"-la.
Compreendem"-se bem as reações daquele velho pai. Porém, mais inesperadas
foram as do Irmão Simeão. Aquele homem forte, que subjugaria até o
\textit{Diyabo} [Diabo], aproveitava qualquer ocasião, desde que a
criança nascera, para visitar o velho comerciante de guarda"-chuvas.
Tomando o pequeno ser em seus braços grosseiros, as lágrimas lhe
escorriam sobre o rosto perturbado; ele certamente se recordava da
silhueta delicada de Lorenzo, por ele amado como um irmão. Mas a jovem
mãe, como que ressentida e lamentosa de que Lorenzo não lhe tivesse
dado qualquer sinal de vida desde sua partida de Santa Lúcia, não
parecia apreciar as visitas de Simeão. Assim como um provérbio nosso
que diz ``Ninguém se livra da barreira do tempo'', vejam que, entre isso
e aquilo, mais de um ano se passou num piscar de olhos. Foi então que
aconteceu uma catástrofe totalmente imprevista: um grande incêndio
destruiu, em uma noite, metade da cidade de Nagasaki. O horror da visão
daquelas horas foi algo tão arrepiante que até nos levou a crer que os
sons das trombetas do Juízo Final estivessem retumbando, a dilacerar o
céu abrasado. Naquela ocasião, a casa do velho comerciante de
guarda"-chuvas, como se encontrasse, infelizmente, na direção do vento,
foi envolvida pelo fogo num instante, mas foi só após todos os membros
da família fugir em pânico que perceberam que o bebê não se encontrava
entre eles. Na hora da fuga, haviam"-se, certamente, esquecido que o
bebê estava dormindo num dos quartos. O velho então vociferou, batendo
os pés no chão. A mãe se teria lançado às chamas para resgatar sua
criança, se não a tivessem impedido. Nesse meio tempo, o vento foi se
tornando cada vez mais forte, as labaredas das chamas se lançavam tão
alto que pareciam queimar até as estrelas do firmamento. Os habitantes
da cidade que tinham vindo ajudar a dominar o fogo se agitavam em vão e
nada mais podiam fazem além de tentar acalmar a mãe, quase
enlouquecida. Foi bem naquele instante que, abrindo caminho na
multidão, um homem se aproximou correndo. Era o Irmão Simeão. Homem
forte, como que acostumado a desafiar o perigo nos campos de batalha,
Simeão se lançou corajosamente nas chamas assim que percebeu o que se
passava. No entanto, talvez tenha sido a intensidade das chamas que o
tivesse feito recuar. Depois de tentar afrontar duas ou três vezes as
nuvens de fumaça, deu meia"-volta e recuou de vez. Dirigiu"-se ao velho
comerciante e à sua filha e lhes disse:

--- Este fogo é também um efeito da Providência de \textit{Deusu}. É
preciso que nos resignemos.

Nesse instante, ouviu"-se uma voz não se sabe de quem, vinda do lado do
velho comerciante, a clamar em alto e bom tom:

--- Senhor! Ajudai"-me!

Simeão, que pareceu reconhecer aquela voz, virou a cabeça e, olhando
fixamente para aquele que assim gritara, oh, assombro!, era\ldots{} Lorenzo,
o inconfundível Lorenzo! O rosto emaciado e puro refletia o brilho
vermelho do fogo e seus cabelos negros, agitados pelo vento, se
alongavam cheios sobre os ombros, mas se reconhecia num só olhar o seu
perfil gracioso e melancólico. Diante da multidão que se aglomerava,
Lorenzo, com sua aparência de mendigo, contemplava fixamente a casa em
chamas. Foi uma impressão que não durou mais que um instante e, quando
um vento terrível soprou avivando ainda mais o fogo, a figura de
Lorenzo já desaparecera em meio às pilastras, às paredes e às vigas,
todas em chamas. Simeão, com o suor escorrendo em todo seu corpo,
desenhou um sinal da cruz em direção ao alto do céu e gritou, ele
também: ``Senhor, ajudai"-nos!''.

Nesse instante, sem que nem ele mesmo soubesse o porquê, irrompeu no
fundo de seus olhos, como deveria contar mais tarde, a silhueta
melancólica e graciosa de Lorenzo destacando"-se no portão de Santa
Lúcia, banhado pelo sol crepuscular que tremulava ante o vento de
inverno. Enquanto isso, os fiéis ali presentes, embora surpreendidos
pela corajosa atitude de Lorenzo, não conseguiam esquecer seu passado
pecaminoso. Imediatamente, várias críticas atravessaram, carregadas
pelo vento, a multidão em bulício. Muitas vozes se juntavam no
maldizer: ``Não há dúvida, a afeição paternal é o mais forte dos
sentimentos. Vejam Lorenzo, que não aparecia mais entre nós porque
tinha vergonha de seus pecados, agora enfrenta o fogo para salvar a filha!''.

Desde que vira Lorenzo, o velho comerciante também parecia concordar com
a multidão e, sozinho, talvez para esconder a estranha emoção que o
agitava, ora se levantava, ora se agachava nervosamente, proferindo
apenas absurdos. Mas a filha, nesse meio tempo, transfigurada e
ajoelhada sobre o solo, enterrava o rosto entre as mãos e parecia não
mover o corpo, tão concentrada estava na prece. Choviam fagulhas
naquele céu. A fumaça em movimento célere açoitava"-lhe o rosto. Mas,
com a cabeça abaixada e em silêncio, ela estava imersa na prece,
esquecida do mundo e de si mesma.

Nesse ínterim, a multidão aglomerada de novo diante do fogo reverberou
em uníssono e, cabelos emaranhados, Lorenzo já reaparecia, em meio às
labaredas que se desprendiam, como se descesse do céu, o bebê entre
os braços. Justo naquele instante, uma viga pareceu ter se rompido de
repente pela metade. Com um barulho estrondoso, levantou"-se em direção
ao céu um enorme feixe de chamas e de fumaça e imediatamente a silhueta
de Lorenzo desapareceu, restando apenas uma coluna de fogo semelhante a
um enorme coral a se abrasar no alto do céu.

Penalizados por tamanha falta de sorte, todos os fiéis, a começar por
Simeão, e até o velho comerciante, foram tomados por uma sensação de
vertigem. Perturbou"-se especialmente a jovem mãe, que, dizem, lançou um
grito lancinante, pulou num salto que deixou entrever suas pernas e
logo se atirou sobre o solo, como se tivesse sido atingida por um raio.
Mas quando se viu --- ninguém sabe em que momento isso se deu --- o bebê
arrancado à morte iminente nos braços da jovem mãe prosternada, ah!,
qual não foi a emoção! Ah! A sabedoria e o poder de \textit{Deusu}
infinitamente misericordioso! Nenhuma palavra poderia lhe render
homenagem! O bebê, que Lorenzo havia atirado com todas as suas forças
no momento em que a viga em chamas caía sobre ele, havia rolado sem o
menor ferimento bem aos pés da jovem mãe.

Enquanto sua filha, atirando"-se à terra, sufocava as lágrimas de
alegria, o velho, os braços voltados para o céu, louvava solenemente o
coração misericordioso do Senhor --- ou, talvez devêssemos dizer, se
preparava para louvá"-Lo, quando Simeão se lançou no meio daquele
turbilhão de fogo com a intenção de salvar Lorenzo; e, então, a voz do
velho se elevou de novo, numa prece aflita e pungente ao céu noturno.
Mas sua voz já não era a única. Todos os fiéis que estavam em volta do
velho e de sua filha se juntaram a eles em lágrimas, rezando numa só
voz: ``Senhor! Salvai"-os!''. E o Filho da \textit{Biruzen Maria} [Virgem
Maria], Nosso Senhor \textit{Zesu Kirishito}, que chama a Si as dores e
as tristezas de todos os homens, escutou aquela prece. Vejam! Eis
Lorenzo, que, cruelmente ferido pelo fogo, já está a salvo das chamas e
da fumaça carregado nos braços de Simeão! Mas a catástrofe daquela
noite ainda não terminara. Foi quando se apressavam a transportar
Lorenzo agonizante para diante da porta da \textit{Ekereshiya}, que o
fogo não atingira; foi então que a filha do comerciante de
guarda"-chuvas, com os olhos molhados de lágrimas, apertando sua criança
ao seio, ajoelhou"-se aos pés do padre que naquele preciso momento tinha
vindo à porta e, em presença de todas as pessoas ali aglomeradas, fez
sua \textit{kohisan} [confissão], para surpresa geral.

--- Esta criança não é filha de Lorenzo. A verdade é que ela nasceu das
relações secretas que eu tive com o filho do vizinho, um
\textit{zencho}!

O tremor de sua voz tensa e decidida, o brilho de seus olhos molhados,
nada permitia fazer duvidar, nem por um instante, da veracidade daquela
confissão. Era bem natural que os fiéis que ali se acotovelavam
perdessem a voz, emocionados, num angustiado silêncio, esquecendo"-se
até das chamas que abrasavam o céu.

A jovem, enxugando as lágrimas, continuou:

--- Naquela época, eu estava enamorada de Lorenzo. Mas, pela devoção de
sua fé, ele sempre me tratava com indiferença. Então, isso provocou
ressentimento em meu coração e, afirmando falsamente que ele era o pai
da minha filha, quis que ele soubesse do sofrimento profundo que sua
atitude suscitara em mim. Mas a nobre generosidade de Lorenzo, longe de
fazê"-lo odiar a vileza do meu pecado, o fez esta noite enfrentar chamas
tão terríveis quanto as do \textit{Inheruno} [Inferno] para salvar
minha filha. Sua piedade, sua atitude, tudo me faz ver nelas a nova
vinda de Nosso Senhor \textit{Zesu Kirishito}. Considerando a extrema
gravidade de meus pecados, não vou poder reclamar quando meu corpo for
esquartejado pelas garras do \textit{Dyabo}. 

Sem nem mesmo acabar sua confissão, jogou"-se por terra e caiu em
prantos. Foi justamente naquele instante que, das bocas dos fiéis apinhados em
volta deles, ondas de gritos se elevaram: ``\textit{Maruchiri!
Maruchiri!}'' [Mártir! Mártir!]. Admiravelmente, com o coração cheio de
piedade para com a pecadora, Lorenzo rebaixou"-se até a condição de
mendigo, seguindo os passos de Nosso Senhor \textit{Zesu Kirishito}.
Nem o padre, que ele respeitava como seu próprio pai, nem o Irmão
Simeão, que ele considerava seu irmão, ninguém conhecia seu coração.
Que seria ele, senão um \textit{maruchiri}? 

Quanto a Lorenzo, ele se contentou em aquiescer duas ou três vezes
enquanto ouvia a \textit{kohisan} da jovem mãe. Seus cabelos estavam
queimados, a pele calcinada, os membros inertes, os lábios já não
demonstravam condições de proferir o que fosse. Com o coração partido
após ouvir a \textit{kohisan} da jovem, o velho e Simeão se desfaziam
em cuidados, acocorados em volta de Lorenzo. Mas sua respiração, cada
vez mais entrecortada, anunciava que o fim estava próximo. Somente seu
olhar puro como as estrelas ainda permanecia o mesmo, voltado em
direção ao longínquo céu.

Depois de ouvir a \textit{kohisan} da jovem, a barba branca agitada pelo
vento noturno em fúria, o padre declarou solenemente, tendo às suas
costas o portão de Santa Lúcia:

--- Felizes os que se arrependem! Que ser humano ousaria infligir uma
punição a esses bem"-aventurados? Daqui em diante, minha filha, deverás
observar bem os mandamentos do Senhor para poderes esperar em paz o dia
do Juízo Final! Os votos que Lorenzo tomou a si de seguir a conduta de
Nosso Senhor \textit{Zesu Kirishito} dão mostra de uma virtude sem 
igual entre os fiéis deste país. Sobretudo, não nos esqueçamos de que,
sendo ainda muito jovem\ldots{}

Mas o que será que teria acontecido? O padre, que assim se pronunciava,
calou"-se de repente, como se vislumbrasse a luz do \textit{Haraiso},
contemplando Lorenzo estendido a seus pés. Que respeito não exprimia
então a atitude do padre! O tremor de suas mãos anunciava alguma coisa
de extraordinário. Oh! Em suas faces ressecadas, as lágrimas escorriam
sem parar!

Olhe bem, Simeão! Olhe bem, velho comerciante de guarda"-chuvas! No peito
do garoto de uma beleza maravilhosa, que se encontrava deitado em
silêncio sob o portão de Santa Lúcia, iluminado pelos reflexos das
chamas, mais vermelhas ainda do que o sangue de Nosso Senhor
\textit{Zesu Kirishito}, apareciam, através dos buracos de sua roupa
queimada, dois seios imaculados semelhantes a duas grandes pérolas. 
Seu rosto cruelmente queimado expressava uma doçura inata agora
impossível de dissimular. Ah! Lorenzo era uma mulher! Lorenzo era uma
mulher! Vejam vocês, que se enfileiram como um muro, de costas para as
chamas! Lorenzo, expulso de Santa Lúcia por haver pecado contra a
luxúria, era uma mulher deste país e seus olhos eram tão encantadores 
quanto os da filha do comerciante! 

Momento de sublime terror! Dizem que se tinha mesmo a impressão de se
ouvir a sagrada voz do Senhor a se propagar de um céu mais longínquo
que a esfera estrelada. Assim, todos os fiéis reunidos diante de Santa
Lúcia, qual espigas de trigo ondulando ao vento, abaixaram a cabeça e
se ajoelharam ao redor de Lorenzo, sem que se soubessem quem teria sido  
o primeiro. Tudo o que se ouvia eram os rugidos do enorme incêndio que
abrasava o alto céu. Não, havia também alguém que soluçava. Seria a
filha do comerciante de guarda"-chuvas? Ou talvez Simeão, que se havia
considerado seu irmão? Logo, os braços estendidos sobre Lorenzo, o
padre entoou uma prece solene e triste que ecoou no silêncio
melancólico. No instante mesmo em que findou a prece, aquela moça
nativa do país que fora chamada de ``Lorenzo'', havendo entrevisto a
\textit{Guroriya} do \textit{Haraiso} para além da noite ainda escura,
expirou docemente, um sorriso sereno pairando sobre os lábios.

Ouvi dizer que isso é tudo quanto se pôde saber da vida daquela mulher.
Mas haveria necessidade de mais dados? O que é mais precioso no mundo
condensa"-se no interior de um instante insubstituível de emoção. Dizem
que a vida verdadeiramente digna de ser vivida é aquela que lança uma
onda no céu das paixões e dos desejos mundanos, que são como um mar de
noite escuro, captando assim o clarão da lua ainda invisível nas
espumas das ondas. Assim, pois, os que conhecem o fim último de Lorenzo
não conhecem também toda a sua vida?
\sectionitem
Um dos livros de minha biblioteca, publicado em Nagasaki pela Companhia
de Jesus, tem por título \textit{Regenda aurea}. Trata"-se, a meu ver,
da \textit{Legenda áurea}, mas seu conteúdo não coincide
necessariamente com o que comumente se conhece na Europa como ``Lendas
áureas''. O livro registra ao mesmo tempo pregações e atos de apóstolos
e santos daquelas terras e feitos de bravura e devoção dos cristãos de 
nosso país. Desejava"-se assim contribuir para a evangelização do Japão.

Compõe"-se de dois tomos, escritos em \textit{hiragana}\footnote{ \textit{Hiragana}: 
designação de um dos sistemas de transcrição
fonética da língua japonesa.} mesclados com ideogramas de estilo
cursivo, impressos em papel Mino. A impressão falha não revela se é ou
não tipográfica. Sobre o frontispício do tomo \textsc{i}, o título está impresso
horizontalmente em latim e, logo abaixo, impressas verticalmente em
caracteres chineses, há duas linhas, nas quais se pode ler a data:
``Gravado no ano de 1596 do nascimento de Nosso Senhor, primeira dezena
do terceiro mês lunar, do segundo ano de Keichô''. De cada lado, há uma
imagem de um anjo tocando uma trombeta. Embora de técnica bastante
pueril, não lhe falta certo sabor. O frontispício do tomo \textsc{ii} está
composto da mesma maneira, diferindo somente na data: ``Gravado na
segunda dezena do quinto mês lunar\ldots{}''.

Os dois tomos comportam, cada um, sessenta páginas aproximadamente. As
``Lendas áureas'' aqui publicadas totalizam oito capítulos no tomo \textsc{i} e
dez no tomo \textsc{ii}. Em cada um dos tomos, encontramos um prefácio de autor
anônimo e um índice em japonês e também em latim. Nos prefácios notamos
uma linguagem não muito elegante, muitas vezes ocorrendo expressões que
parecem construir uma tradução literal de textos europeus, fatos que
nos fazem pensar se não teriam sido redigidos pelas mãos de algum padre
ocidental.

``O mártir'', que apresentei acima, está baseado no capítulo \textsc{ii} do tomo
\textsc{ii}. Trata"-se certamente de um relato fiel de um acontecimento que
ocorreu numa igreja cristã de Nagasaki. Entretanto, a ``Crônica do porto
de Nagasaki'', assim como outros documentos, guarda silêncio sobre o
grande incêndio a que se aludiu no texto, não se podendo, portanto,
determinar sua data exata.

Em relação a ``O mártir'', visto que seria publicado, concedi"-me a
liberdade de acrescentar alguns floreios retóricos. Se o estilo claro e
elegante do texto original não tiver sido muito deturpado, sentir"-me"-ei
lisonjeado.

\begin{flushright}
\textit{Agosto de 1918}\\  
\end{flushright}


\chapter{Terra morta}

\epigraph{Convocando Jôsô e Kyorai, disse o mestre: ``Ontem à noite, insone, tomado por uma súbita
inspiração, pedi a Donshû que a pusesse por escrito. Apreciem, por favor, este poema:
\smallskip
\hspace{1.2em}Doente em viagem,\\
\hspace{1.2em}Peregrinam sonhos meus\\
\hspace{1.2em}Por uma terra morta''.}{Diário de Hanaya\footnotemark}\footnotetext{ \textit{Diário de Hanaya} é a coletânea de impressões, diálogos
e cartas referentes às viagens, ao adoecimento e à morte de Bashô, escritas por
seus discípulos. \textit{Jôsô Naitô} (1662--1704): tal como certos nomes
citados na obra (Kyorai, Kikaku, Shikô), é um dos dez principais discípulos de
Bashô (\textit{Bamon jittetsu}), mestre de haicai. \textit{Kyorai}: trata"-se
de Kyorai Mukai (1651--1704), organizador da coletânea \textit{Sarumino} de
Bashô. \textit{Donshû}: haicaísta de Ôsaka.}

\noindent\textsc{Era pela tarde} do dia doze do décimo mês lunar do ano sete da era
Genroku.\footnote{ Era Genroku: divisão histórica do Japão, de 1688 a 1704. 
A data aqui referida é novembro de 1694.} O céu avermelhado do amanhecer
convidava os comerciantes de Ôsaka que acabavam de se levantar a tornar
os olhos para além dos telhados mais distantes, a fim de averiguar se
choveria como na véspera. Felizmente, a chuva nem sequer chegou a
esfumaçar as copas cheias dos salgueiros e logo se firmou uma tarde de
inverno serena e clara, embora um pouco nublada. Naquele dia, mesmo as
águas dos rios, que corriam preguiçosamente por entre as casas de
comércio enfileiradas nas margens, estavam pálidas, sem seu brilho
costumeiro, e, talvez fosse impressão, os restos de cebolinha que nele
flutuavam não apresentavam sua fria cor verde habitual. Além disso, os
transeuntes que passavam pelas margens do rio, uns com a cabeça
encapuzada, outros calçados de \textit{tabi}\footnote{ \textit{Tabi}:
calçado usado junto com trajes japoneses, com uma divisão entre o dedão
e os outros dedos.} de couro, caminhavam, todos, com um ar ausente,
como se tivessem se esquecido do vento frio do norte que soprava mundo
afora. As cores dos cortinados das lojas, o vaivém dos carros, o som de
\textit{shamisen} longínquo do teatro de bonecos --- tudo resguardava o
silêncio daquela tarde calma e clara de inverno, na qual nem sequer se
movia a poeira acumulada sobre os adornos dos balaústres das pontes da cidade\ldots{}

Naquela hora, na sala do fundo da casa de Nizaemon Hanaya, situada no
bairro de Midômae Minami Kyûtarô, Bashôan Matsuo Tôsei,\footnote{ Bashô Matsuo 
(1644--1694): haicaísta que renovou a poesia, tratando tópicos do
cotidiano com lirismo e enaltecimento, tendo escrito também diários de
viagens.} a quem então se venerava como o grande mestre de haicai,
estava para exalar seu último suspiro, encerrando em silêncio uma vida
de cinquenta e um anos, ``como se o fogo coberto pelas
cinzas lentamente se fosse esfriando'', assistido pelos
discípulos que haviam acorrido dos quatro cantos do país. Seriam quatro
e meia? Ou já se aproximavam das cinco horas? Dentro da sala imensa, da
qual haviam sido retiradas as paredes divisórias corrediças, o incenso
aceso na cabeceira fazia subir um fio de fumaça que lançava sombras em
algumas partes do papel novo da porta de correr, provocando uma aguda
sensação de frio. A porta separava a sala do jardim, onde reinava o
inverno. Ao redor de Bashô, que estava serenamente deitado com o
travesseiro voltado para a referida porta corrediça, encontrava"-se,
mais próximo, o médico Mokusetsu, que, arqueando as sobrancelhas num ar
de preocupação, observava sua pulsação espaçada, com a mão deposta sob
o acolchoado. Encolhido bem atrás do médico, murmurando incessantemente
o nome de Buda, via"-se o velho criado Jirobê, que recentemente
acompanhara o amo em sua última viagem, da região de Iga até Ôsaka.
Depois, ao lado de Mokusetsu, ninguém deixaria de reconhecer o
corpulento Kikaku Shinshi, que, inflando generosamente o peito sob a
veste de seda, atentamente espreitava o estado do mestre, ao lado de
Kyorai, que, com seu ar altivo, tinha os ombros elevados bem vestidos
numa estampa miúda. Em seguida, atrás de Kikaku, portando no pulso um
terço budista de figueira, postava"-se Jôsô, ereto, com seu aspecto de
monge. E ao seu lado estava Otsushû, que não cessava de fungar o nariz,
talvez porque já não conseguisse suportar a tristeza que o dominava.
Quanto à figura de monge de estatura baixa, que tentava disfarçar o
estado já desgastado de sua veste e observava o ambiente com
reprovação, o queixo levantado, tratava"-se de Inen; ele estava sentado
em frente de Mokusetsu e ao lado de Shikô,\footnote{ Um dos dez
principais discípulos de Bashô, trata"-se de Shikô Kagami (1665--1731).}
que tinha a pele amorenada e aparentava teimosia. Além desses, não
havia mais que alguns discípulos respeitando absoluto silêncio,
parecendo nem respirar, dispostos à direita e à esquerda do leito do
mestre, a quem prestavam a derradeira homenagem. Mas o único que,
dentre todos, encolhido num canto da sala e prostrado sobre o tatame,
deixava escapar dolorosos gemidos, parecia ser Seishû. Contudo,
absorvido pelo frio silêncio do interior da sala, ele não chegava a
emitir gemidos que perturbassem o imperceptível perfume de incenso à
cabeceira do doente.

Um instante após haver pronunciado suas últimas palavras de maneira
ambígua, a voz enrouquecida pela tosse e pelo pigarro, Bashô, com os
olhos entreabertos, parecia ter entrado em estado de coma. Seu rosto,
ligeiramente marcado pelas manchas de varíola, era de uma extrema
magreza, sobressaindo"-lhe apenas os ossos das maçãs do rosto; havia
muito que a cor de vida se retirara de seus lábios cercados de rugas.
Mas o que mais confrangia a vista era a expressão de seus olhos --- uma
luz vaga errava naqueles olhos que em vão fitavam algum ponto ao longe,
como se aspirassem avistar o céu frio e sem limites que se estendia
além dos telhados.

\begin{verse}
Doente em viagem,\\
Peregrinam sonhos meus\\
Por uma terra morta.
\end{verse}

Pode ser que naquele instante, diante de seu olhar insondável,
oscilasse, como que num sonho, a visão das sombras do crepúsculo de uma
infinita terra morta, onde não brilharia sequer um raio de luar,
exatamente como ele próprio vislumbrara em seu poema de despedida, três
ou quatro dias antes.

--- Água!

Assim dizendo, Mokusetsu virou"-se para Jirobê, sentado atrás dele em
silêncio. O velho criado havia deixado preparados uma tigela de água e
um palito com uma pluma na ponta. Pousando os dois objetos,
timidamente, à cabeceira do amo, acelerou o movimento de seus lábios,
como se a ideia subitamente lhe ocorresse, começando então a recitar
com fervor o nome de Buda. Dentro do coração simples de Jirobê, homem
criado nas montanhas, devia estar enraizada a sólida fé na crença de
que, se todos renascerão na Terra Pura, Bashô, tanto quanto qualquer
outra pessoa, precisaria recorrer à benevolência de Buda.

Por outro lado, durante a fração de segundo em que pedia a água,
Mokusetsu foi tomado pela mesma dúvida de sempre: será que ele, como
médico, tinha realmente tentado tudo o que estava ao seu alcance? Mas
logo se recuperou, voltou"-se para Kikaku, sentado ao seu lado, e, em
silêncio, fez"-lhe um breve aceno com a cabeça. Foi naquele instante que
uma sensação tensa, a de que o último momento havia chegado, atingiu o
espírito de todos aqueles que rodeavam o leito de Bashô. Mas não se
pode negar que, ao mesmo tempo, certa sensação de distensão --- ou
melhor, um estado de espírito semelhante ao alívio --- tenha passado por
eles: o que haveria de chegar havia finalmente chegado. Porém, aquele
estado de espírito, semelhante ao alívio, era de uma natureza tão sutil
que ninguém parecia querer aceitar conscientemente sua existência, de
modo que até Kikaku, o mais realista de todos os presentes, quando o
leu, num relance, também em Mokusetsu, com quem casualmente cruzara o
olhar, mal conseguiu ocultar o embaraço. Desviou apressadamente o olhar
para o lado e, tomando da pluma como se nada houvesse acontecido,
dirigiu a palavra a Kyorai:

--- Com sua licença, permita"-me ser o primeiro. --- E, molhando a pluma com
a água da tigela, aproximou os joelhos grossos, lançando um olhar
furtivo ao rosto do mestre em seu derradeiro momento. Para dizer a
verdade, mesmo antes desse instante 
chegar, ele já imaginara quanto seria dolorosa sua despedida final de
Bashô. Mas seu estado de espírito, ao umedecer"-lhe os lábios em penhor
de uma última homenagem, contrariando totalmente suas previsões um
tanto quanto teatrais, foi de completa frieza. Além disso, Kikaku não
havia imaginado que o aspecto lúgubre do mestre moribundo, literalmente
pele e ossos, fosse provocar"-lhe uma repugnância assim tão profunda, a
ponto de ele não conseguir se conter e ter de virar o rosto. Não, dizer
simplesmente que era ``profunda'' ainda é pouco. O que sentia era a mais
insuportável repugnância, que, qual um veneno invisível a olho nu,
trazia"-lhe até mesmo reações fisiológicas. Será que ele vertia sobre o
corpo enfermo de seu mestre, que o acaso lhe apresentava naquele
momento, sua aversão a todas as formas de feiura? Ou, então, será que,
para o amante da ``vida'' que ele era, a realidade da ``morte'' constituía
uma ameaça à natureza, que mais que tudo deveria ser amaldiçoada? O que
quer que fosse, Kikaku, sentindo uma indizível repulsa pelo rosto de
Bashô à beira da morte, mal acabando de passar, sem nenhuma tristeza, a
pluma embebida de água em seus lábios finos e violáceos, retirou"-se,
contraindo o rosto. Embora uma espécie de sentimento de culpa --- é
verdade que apenas por um breve segundo --- lhe tivesse aflorado no
momento em que se retirava, a repulsa que acabara de sentir parecia ser
demasiado forte para dar lugar a considerações morais.

Depois de Kikaku, quem retirou a pluma foi Kyorai, que parecia haver
perdido toda a tranquilidade de espírito desde que Mokusetsu lhe fizera
o sinal com a cabeça. Reputado por sua extrema modéstia, cumprimentou
discretamente os presentes e deslizou para a cabeceira de Bashô. Ao
contemplar, no entanto, o rosto devastado pela enfermidade do velho
mestre de haicai, não pôde deixar de sentir, mesmo a contragosto, uma
estranha mistura de satisfação e arrependimento. Exatamente como a
sombra e o sol que estão ligados pelo carma indissolúvel, essa mistura
de satisfação e arrependimento havia atormentado aquele homem tímido
nos últimos quatro ou cinco dias, sem lhe dar um momento sequer de
folga. Em outras palavras, assim que soube que Bashô se encontrava em
estado grave, tomou imediatamente o barco em Fushimi e, desde o
instante em que bateu à porta de Hanaya, apesar do avançado da hora,
não negligenciou os cuidados ao mestre um único dia. Além disso,
solicitando a Shidô que se encarregasse de contratar ajudantes,
enviando alguém ao santuário para orar ao deus Sumiyoshi por seu
restabelecimento, consultando"-se com Nizaemon Hanaya para a aquisição
de alguns objetos indispensáveis, ele foi, sozinho, as rodas que movem 
a carruagem, assumindo a responsabilidade de toda e
qualquer coisa. E é claro que foi ele mesmo quem tomou a iniciativa de
cada tarefa, que depois se revelou necessária, sem ter a mínima
intenção de provocar nos outros qualquer sentimento de dívida para com
ele. No entanto, a consciência de sua completa dedicação na assistência
ao mestre fez germinar no âmago de sua alma um profundo contentamento
de si. Enquanto essa satisfação ainda permanecia inconsciente,
envolvendo suas atividades numa atmosfera agradável, ele não parecia
ter sentido nenhum embaraço em seu comportamento cotidiano. Se não
fosse assim, quando, numa noite de vigília, conversava com Shikô acerca
das histórias deste  mundo flutuante, sob a luz da lamparina,
certamente não se ateria a apregoar tão longamente o princípio do
\textit{kôdô},\footnote{ \textit{Kôdô}: dentro da doutrina
confucionista, o princípio da piedade filial.} declarando
devotar"-se ao mestre como um filho se dedica aos pais. Mas, naquele
instante em que ele assim se exaltava, percebeu no rosto do sarcástico
Shikô o fugaz clarão de um sorriso malicioso e teve consciência de que
alguma coisa se desregulara subitamente dentro da harmonia de seu
espírito. Foi então que descobriu que a origem de tal desarranjo se
encontrava em sua satisfação, que então percebeu pela primeira vez, e
na reprovação que ele próprio sentia em relação a ela.

Velando o mestre gravemente adoentado, que provavelmente não passaria
daquele dia, ele estaria mesmo preocupado com seu estado? Qual! Estava
mesmo é contemplando com olhos preguiçosos de satisfação o desempenho
das tarefas às quais ele mesmo se propusera. Isso certamente lhe
provocou um sentimento de remorso, homem íntegro que era. A partir daí,
começou naturalmente a sentir certo refreamento para fazer qualquer
coisa, devido ao antagonismo de seus sentimentos. E, quando percebia
nos olhos de Shikô a luz de um sorriso, mesmo fortuito, a consciência
da satisfação que sentira se acentuava indizivelmente e muitas vezes o
fazia sentir"-se ainda mais miserável pela sua baixeza.

E essa situação se prolongou por alguns dias, até aquele momento em que,
à cabeceira do mestre, ele o homenageava oferecendo"-lhe a última água:
era digno de pena, mas até compreensível, ver aquele homem, que aliava
uma grande probidade moral a uma inesperada fragilidade nervosa, perder
totalmente o controle de si ante essa contradição interior. Kyorai, ao
pegar a pluma, sentiu o corpo se enrijecer estranhamente e foi
assaltado por uma excitação tão singular que a ponta branca embebida de
água que roçava os lábios de Bashô não parava de tremer. Mas, por
sorte, gotas de lágrimas se formaram enquanto isso, já quase
transbordando de seus cílios, de modo que os discípulos que o
observavam --- entre eles provavelmente até o incisivo Shikô --- sem dúvida
devem ter interpretado sua excitação como consequência da aflição que
sentia.

Levantando os ombros bem vestidos numa estampa miúda, Kyorai retornou
timidamente a seu lugar e entregou a pluma às mãos de Jôsô, sentado
imediatamente atrás. A atitude daquele homem devotado que, com os olhos
baixos em sinal de respeito, umedecia calmamente os lábios do mestre
enquanto revolvia em sua boca uma prece inaudível, sem a menor dúvida
se revestia, aos olhos dos circunstantes, de uma imponente solenidade.
Mas, naquele instante solene, um riso sinistro se fez ouvir
repentinamente, vindo de um dos cantos da sala. Não, naquele momento o
que tiveram foi apenas a impressão de ter ouvido algo. Assemelhava"-se a
uma gargalhada eclodindo do mais profundo das entranhas, que, embora
obstruída pela garganta e pelos lábios, parecia jorrar aos borbotões
pelas narinas, sob um impulso irreprimível. Mas, desnecessário dizer,
numa circunstância como aquela, ninguém poderia cometer o desrespeito
de rir. Tratava"-se, na verdade, dos soluços de Seishû, até então
sufocados e reprimidos, que transbordavam rasgando"-lhe o peito. Seus
soluços eram a expressão extrema da dor mais profunda. Dentre os
discípulos ali reunidos, vários foram os que se lembraram do famoso
poema do mestre:\footnote{ Trata"-se de um haicai da obra \textit{Okuno
hosomichi} (Sendas de Oku) composto em 1690, em homenagem a um poeta 
que falecera um ano antes da esperada visita de Bashô.}

\begin{verse}
Agita"-te, túmulo!\\
Minha voz que se lamenta\\
É vento de outono.
\end{verse}

Mas, apesar das lágrimas que também o sufocavam, Otsushû não pôde deixar
de sentir certo mal"-estar ante o exagero presente naqueles soluços
exaltados ou, se a expressão não for apropriada, diante da falta de
autocontrole que eles traíam. Mas esse mal"-estar certamente devia ser
de uma natureza apenas intelectual. Embora sua mente dissesse não, seu
coração fora atingido pelos gritos lancinantes de Seishû e, sem que o
percebesse, seus olhos se marejaram de lágrimas. No entanto, isso não
alterava o mal"-estar que sentia diante dos soluços de Seishû, pois
tampouco se sentia orgulhoso de suas próprias lágrimas. No entanto,
elas continuavam a lhe subir aos olhos sem parar --- e então Otsushû, as
mãos pousadas sobre os joelhos, acabou deixando escapar alguns gemidos
sufocados. Mas Otsushû não era o único que, naquele momento, não
conseguia conter o pranto aflito. Entre os numerosos discípulos
discretamente sentados aos pés do leito de Bashô, começaram quase que a
um só tempo a elevar"-se sons de choro e de soluços entrecortados,
fazendo vibrar a quietude do ar frio e transparente da sala.

Em meio a essas vozes de angústia e aflição, Jôsô, com o terço de contas
de figueira no pulso, retornou a seu lugar, ainda mantendo a calma
anterior. Shikô, que estava sentado em frente de Kikaku e Kyorai,
dirigiu"-se então à cabeceira. Shikô, também conhecido pelo nome de
Tôkabô, famoso pela ironia, não parecia ter uma sensibilidade tão
aguçada a ponto de, influenciado pelos sentimentos dos circundantes,
chegar a verter lágrimas desnecessárias. Com a expressão sempre zombeteira 
no rosto amorenado e, ainda mais, mostrando
curiosamente a insolência de praxe, passou a água nos lábios do mestre
sem demonstrar o mínimo cuidado. Mas era inegável que até ele sentia,
naquela circunstância, certa emoção.
 
\begin{verse}
Pensar em meus ossos\\
Expostos no campo ao vento\\
Que penetra o corpo.
\end{verse}

Quatro ou cinco dias antes, o mestre agradecera"-lhes repetidas vezes, dizendo:

--- Eu, que pensava que morreria tendo por leito as folhas secas e por
travesseiro a terra, poder realizar, sobre esse lindo acolchoado, os
votos tão caros de uma morte serena constitui para mim a maior das
felicidades.

Mas, em verdade, fosse em meio a um campo seco ou fosse na sala da casa
de Hanaya, não haveria grande diferença. Na realidade, mesmo ele, que
naquele momento lhe umedecia a boca, preocupara"-se até três ou quatro
dias antes por seu mestre ainda não ter composto os versos de
despedida. E, na véspera, esteve planejando organizar uma compilação
dos \textit{hokku}\footnote{ \textit{Hokku}: historicamente, nome
atribuído ao conjunto dos três versos iniciais, compostos por 5, 7 e 5
sílabas respectivamente, do poema encadeado humorístico chamado
\textit{rengano haikai}; ao adquirir existência independente, foi
chamado de haicai e, posteriormente, na era Meiji, de \textit{haiku}.}
de Bashô, após seu desaparecimento. E enfim, naquele dia, até poucos
instantes antes, contemplava com olhos atentos o mestre que lentamente
se aproximava da morte, como se algo naquele processo lhe despertasse
especial interesse. Se fôssemos mais longe, poderíamos até pensar
ironicamente que por trás de seu olhar perscrutador se ocultava uma
passagem do \textit{Diário da morte do mestre}, que posteriormente
deveria ser escrito por ele mesmo. Se assim fosse, enquanto assistia
aos últimos momentos de Bashô, não estaria ele pensando em outra coisa
senão na reputação de sua escola comparada às outras, nas vantagens e
nas inconveniências dos discípulos ou então no cálculo do benefício
pessoal --- em suma, em nada que tivesse relação direta com o mestre
agonizante. Portanto, poder"-se"-ia afirmar que Bashô, conforme havia
pressentido com certa frequência em seus poemas, fora abandonado e
exposto em meio à infinita terra seca da vida humana. Eles, os
discípulos, longe de lamentar a morte do mestre, lamentavam, sim, a si
mesmos, que o perderiam; longe de prantear a morte de seu guia
completamente desamparado em meio a uma terra morta, eles choravam,
sim, a si próprios, que perderiam o guia ao entardecer. Mas, mesmo
acusando moralmente essa atitude, o que fazer, se a ingratidão é
própria dos humanos? Mergulhado nesse tipo de reflexão pessimista, que
ele considerava sua qualidade superior, Shikô terminou de umedecer
os lábios do mestre, descansou a pluma na tigela d'água e, após encarar
com um olhar de desprezo os discípulos à sua volta, que se afogavam num
mar de lágrimas, retornou sem pressa a seu lugar. Entre eles, Kyorai,
por exemplo, que era um homem de gênio bom, havia sido marcado desde o
começo pela frieza de Shikô, o que só lhe acentuou a insegurança;
Kikaku, por outro lado, mostrava uma estranha expressão de incômodo,
certamente provocada pela sua irritação ante aquela mania de Tôkabô,
cujo caráter e atitude demonstravam desprezo em toda e qualquer
circunstância. Quando, depois de Shikô, Inen fez roçar no tatame a
barra do seu hábito negro, enquanto lentamente se arrastava até o
mestre, era manifesto que o fim definitivo de Bashô já estava iminente.
Sua face estava ainda mais pálida e, às vezes, como que por
esquecimento, a respiração cessava por entre os lábios que seus
discípulos molhavam. De vez em quando, porém, como que por efeito de
uma lembrança súbita, a garganta se contraía num grande espasmo e fazia
passar de novo um débil sopro. Além disso, umas duas ou três vezes até
se pôde ouvir o silvo de um pigarro no fundo de sua garganta. Sua
respiração parecia se tornar gradualmente mais fraca. No momento em que
Inen estava prestes a tocar com a ponta branca da pluma os lábios do
agonizante, foi brutalmente assaltado por certo pavor que nada tinha a
ver com a tristeza da separação pela morte. Era o pavor quase
irracional de que, depois do mestre, o próximo a morrer talvez viesse a
ser ele próprio. Por irracional que fosse, uma vez assaltado por aquele
pavor, não havia como opor"-lhe resistência, por mais que quisesse. Por
natureza, ele era dessas pessoas que, à simples menção da palavra
\textit{morte}, amedrontava"-se de modo doentio. Desde há muito, mesmo
quando fazia sua peregrinação de ordenamento, só de pensar em sua morte
sentia um temor sinistro que fazia o suor escorrer"-lhe pelo corpo todo.
Assim, quando ouvia falar da morte de outra pessoa, de algum modo ele
se sentia aliviado, pensando: ``Ah! Que bom que não fui eu quem
morreu!''. Ao mesmo tempo, havia vezes em que sentia, ao contrário, uma
grande indiferença, pensando como seria se ele próprio estivesse
morrendo. A morte de Bashô não constituiu exceção à regra. No início,
enquanto a morte do mestre ainda não estava tão iminente --- enquanto o
sol das tardes claras de inverno batia no papel das portas corrediças e 
o aroma puro dos narcisos presenteados por Sonojo envolvia os
discípulos que se reuniam à cabeceira do mestre, compondo poemas para
consolá"-lo ---, esses dois estados de espírito, tão diferentes quanto luz
e sombras, se alternavam nele conforme o momento. Mas, à medida que a
morte do mestre se aproximava --- desde o inesquecível dia das primeiras
pancadas de chuva em que Mokusetsu inclinou a cabeça num gesto de
preocupação, vendo o estado do mestre, que nem conseguia mais comer as
peras que tanto apreciava ---, desde aquele dia, sua tranquilidade foi
pouco a pouco sendo substituída pela ansiedade e, no final, aquela
ansiedade chegou até a deitar sobre seu espírito a fria sombra de um
terrível mau agouro --- o de que o próximo a morrer poderia ser ele. Foi
por isso que, assaltado pelo pavor durante todo o tempo em que, sentado
à cabeceira, meticulosamente umedecia os lábios do mestre, não pôde
olhar de frente aquele rosto quase à beira da morte. Não, parece que
uma vez ele até tentou fitá"-lo nos olhos, mas exatamente naquele
momento ouviu"-se o som surdo de um pigarro que obstruía a garganta de
Bashô e, com isso, toda a coragem de Inen ruiu a meio caminho. ``O
próximo a morrer depois do mestre talvez seja você!'', zumbia sem
cessar, no fundo de seu ouvido, esse pressentimento. Inen, mesmo após
retornar a seu lugar, encolhendo o corpo já mirrado, olhava somente
para cima, para não ver o rosto de ninguém, o que tornava ainda mais
amargo o seu rosto azedo.

Em seguida, Otsushû, Seishû, Shidô, Mokusetsu e os outros discípulos que
rodeavam o leito do enfermo umedeceram, um após o outro, os lábios do
mestre. Mas a respiração de Bashô, enquanto isso, se fazia mais tênue a
cada sopro, cada vez mais espaçada. Mesmo sua garganta já não se movia
mais naquele momento. O rosto emaciado e semelhante à cera, com as
pequenas marcas de varíola, as cores das pupilas de brilho opaco, que
fitavam o espaço ao longe, e a barba branca como prata que lhe crescia
no queixo\ldots{} Enregelado numa frieza insensível, tudo nele parecia
absorto na contemplação da Terra Pura para onde logo se encaminharia.
Foi então que Jôsô, sentado silencioso e cabisbaixo atrás de Kyorai, o
devotado Jôsô, monge do monastério zen, começou a sentir, à medida que
a respiração de Bashô se debilitava, uma infinita tristeza e uma
sensação de paz igualmente infinita a lhe fluírem docemente na alma.
Quanto à tristeza, não é preciso explicá"-la. Mas a sensação de paz era
de uma estranha alegria, exatamente como se a luz fria do amanhecer se
expandisse paulatinamente por entre as trevas. E aquela paz, que
eliminava a cada segundo todos os pensamentos mundanos, no final
transformou até suas próprias lágrimas em pura tristeza, sem dores que
lhe atormentassem o coração. Estaria ele se rejubilando pela alma do
mestre, que transcendia a quimérica dualidade da vida e da morte,
retornando à Terra Pura do Nirvana Eterno? Não, nem mesmo ele poderia
afirmar que fossem esses os seus motivos. Então\ldots{} Ah, quem se
prestaria à tolice de enganar"-se a si próprio, hesitando
indefinidamente em vão? A serenidade que sentia Jôsô se devia à alegria
ante a emancipação de seu espírito livre, que por tanto tempo estivera
subjugado pelos grilhões da influência da personalidade de Bashô e se
preparava para, finalmente, com suas próprias forças, movimentar braços
e pernas. Em meio a uma alegria triste e arrebatadora, ele passava as
contas do rosário de figueira entre os dedos como se seus olhos não
divisassem mais os discípulos que soluçavam ao redor; esboçando nos
lábios um ínfimo sorriso, cumprimentou respeitosamente o mestre à hora
suprema da morte.

Foi assim que Bashô"-an Matsuo Tôsei, o grande mestre do haicai,
inigualável no passado e no presente, se extinguiu \textit{subitamente}, 
tendo à sua volta discípulos esmagados por uma ``angústia sem limites''.

\begin{flushright}
\textit{Setembro de 1918}\\  
\end{flushright}


\chapter[Devoção à literatura popular]{Devoção à literatura popular\footnoteInSection{ \textit{Gesaku zanmai}
é o título original deste conto. \textit{Gesaku}: em contraposição à literatura clássica, é
a literatura popular \textit{zokubungaku}, que floresceu no período Edo
(1603--1868), composta principalmente de narrativas seriadas. Compõem
esse gênero as brochuras xilografadas denominadas \textit{sharebon},
\textit{kokkeibon}, \textit{ninjôbon}, \textit{yomihon},
\textit{kibyôshi} e outros. \textit{Zanmai} (\textit{sanmai}): termo do
budismo que provém do sânscrito \textit{samâdhi}: concentração;
absorção; estado perfeito de concentração espiritual.}}

\sectionitem
\noindent\textsc{Era uma manhã} do nono mês lunar do ano dois da 
era Tenpô.\footnote{ Ano dois da era Tenpô: 1831 da era cristã.} Como de costume, no banho
público Matsunoyu (``Vapores de Pinheiro''), que se situava no bairro
Dôbôchô, em Kanda, havia muita gente desde cedo. O cenário que Sanba Shikitei   
descrevera alguns anos antes no \textit{kokkeibon}\footnote{ \textit{Kokkeibon}: 
tipo de \textit{gesaku} (literatura popular do período Edo) de teor cômico; Sanba Shikitei (1776--1822) é
seu escritor mais representativo.} como sendo ``o banho deste
mundo flutuante que mistura os deuses xintoístas, os ensinamentos de
Buda, o amor e o efêmero'' não sofrera qualquer mudança. Um velho com um
penteado caseiro feito por sua própria mulher cantarolava baladas,
imerso na água; um, penteado de samurai ao estilo Honda, torcia sua
toalha à saída do banho; um, penteado \textit{ôichô}\footnote{ Literalmente, 
``ginkgo biloba grande'', penteado masculino, comum entre 
os samurais, cuja ponta do tufo de rabo de cavalo engomado, preso ao alto da cabeça raspada, 
desenha um triângulo, tal qual a folha que o nomeia.} com a
parte frontal da cabeça raspada, tinha lavadas suas costas tatuadas; outro, 
com um viril penteado Yoshibê, lavava somente o rosto; uma cabeça raspada de
monge sentada em frente à cuba cobria"-se de água; cabelos presos como
asas de libélulas, crianças brincavam entretidas com peixes vermelhos
de cerâmica em baldes de bambu: sobre o pavimento estreito, pessoas das
mais diferentes condições, todas fazendo reluzir seus corpos molhados,
moviam"-se no ar e se impregnavam do vapor denso que se levantava e da
luz do sol matinal a penetrar pela janela. E era enorme a animação.
Antes de mais nada, havia os sons do correr da água e do movimentar dos
baldes. Depois, havia os sons das conversas e das canções. Por último,
havia o som das baquetas batendo no balcão, chamando pelos serviços.
Portanto, a entrada do banho a vapor estava num tumulto em tudo 
semelhante a um verdadeiro campo de batalha. E ali vinham os vendedores, que
abriam o cortinado da entrada. E vinham os pedintes. Também havia
clientes que entravam e saíam. Em meio a essa confusão\ldots{}

Recolhido a um canto, em meio a essa algazarra, um velho sexagenário
lavava"-se calmamente. Ele parecia contar pouco mais de sessenta anos.
Além de ter os cabelos das têmporas amarelados e mal"-cuidados, seus
olhos pareciam enfermos. Mas, embora magro, sua ossatura era firme, até
mesmo robusta, e nas pernas e braços enrugados ainda conservava um
vigor que resistia à velhice. O mesmo se pode dizer do rosto, que
mostrava uma vigorosa energia animal, quase ameaçadora, que não diferia
em nada dos seus tempos de maturidade, na região onde despontava a mandíbula, 
e ao redor da boca, ligeiramente grande.

Ao terminar de esfregar cuidadosamente a parte superior do corpo, o
velho, sem sequer se enxaguar com a água quente do balde, começou a
lavar a parte inferior. Mas, por mais que se esfregasse com uma bucha
de seda preta, de sua pele ressecada e estriada de pequenas rugas não
se desprendia nenhuma crosta de sujeira. Talvez isso lhe tenha
provocado uma melancolia outonal. Após lavar somente um pé, o velho,
como se de súbito perdesse as forças, parou de movimentar a mão com a
qual se esfregava e abaixou os olhos para a água turva do balde, onde
se espelhava nitidamente o céu além da janela. Lá, os caquis vermelhos
pendiam sobre uma parte do telhado, ligando os galhos poucos e esparsos. 

Nesse momento, a sombra da ``morte'' projetou"-se no coração do velho. Mas 
essa ``morte'' agora nada tinha de ameaçadora, comparada com a que o
intimidara no passado. Pode"-se até dizer que era uma consciência 
tranquila de nirvana, calma mas desejada, como o céu refletido na água
do balde. ``Se pudesse, em meio a essa `morte', livrar"-me de todos os 
desejos mundanos --- se pudesse dormir sem preocupações, sem sonhos, como
uma criança inocente, que felicidade não seria! Estou cansado, não só 
da vida. Estou cansado do sofrimento de décadas de criação
incessante\ldots{}''

Resignado, o velho ergueu os olhos. Ao redor, inúmeras pessoas, todas
nuas, movimentavam"-se vertiginosamente em meio ao vapor; às baladas de   
monges em romaria juntaram"-se outras, como canções \textit{meriyasu} do
teatro \textit{kabuki} e canções de amor \textit{yoshikono}. Ali não se percebia
nenhum vestígio da imagem da eternidade que tocara seu coração.

--- Oh, mestre, encontrarmo"-nos neste lugar! Nem em sonho pensei
encontrar o mestre Kyokutei no banho matinal!

O velho se assustou com a voz repentina que assim o interpelava. A seu
lado, um homem de penteado \textit{hosoichô}\footnote{ Literalmente, ``ginkgo biloba fina'', 
penteado masculino do período Edo, cujo tufo de rabo de cavalo 
engomado, preso ao alto da cabeça raspada, tinha uma forma estreita de folha de ginkgo biloba.} 
de comerciante, de estatura mediana, a aparência saudável, sorria bem"-humorado, com a
toalhinha molhada num ombro, em frente ao balde. Ele parecia ter saído
do banho para tomar a última ducha de água limpa.

--- Você, como sempre, de bom humor!

Sakichi Takizawa, também chamado Bakin, assim respondeu, um pouco
ironicamente, enquanto sorria.

\sectionitem
--- Obrigado, mestre, mas não estou tão bem assim. A propósito, mestre, o
seu \textit{Hakkenden},\footnote{ (\textit{Nansô Satomi}) \textit{Hakkenden},
``História dos Oito Cachorros (das Terras do Sul)'', refere oito
virtuosos samurais cujos nomes contêm o ideograma para ``cachorros'';
obra seriada de Bakin Kyokutei, também conhecido como Bakin Takizawa 
(1767--1848), muito popular no
período, escrito sob influência da literatura chinesa então
recentemente importada, entre os anos de 1814 a 1832.} de suspense em
suspense, fica cada vez mais interessante, é uma obra"-prima!

Enquanto punha no balde a toalha que tinha no ombro, o penteado
\textit{hosoichô} começou a discursar um tom acima.  

--- A maldosa viúva Funamushi se disfarça de tocadora cega de
\textit{shamisen} para se vingar e tenta matar Kobungo. Ela é presa e
torturada, até que Sôsuke a salva. Essa passagem é realmente genial. E
tudo isso vai ser decisivo para o reencontro inesperado de Sôsuke e
Kobungo, não é? Não leve a mal a minha impertinência. Eu, Heikichi Ômiya, 
não passo de um comerciante de miudezas, mas em matéria de
\textit{yomihon}\footnote{ \textit{Yomihon}: tipo de \textit{gesaku} 
(literatura popular do período Edo), principalmente para leitura,
contendo poucas ilustrações, de enredo complexo e conteúdo moralizante.
A obra \textit{Hakkenden}, de Bakin, pertence a essa denominação.} me
considero um entendedor. E, no entanto, à sua obra \textit{Hakkenden},
eu não tenho nenhuma crítica a fazer! Realmente, o senhor merece os
meus parabéns.

Bakin, sem nada dizer, começou novamente a lavar os pés. Ele,
naturalmente, sempre havia nutrido uma afeição comedida para com os
leitores que admiravam sua obra. Mas essa afeição pelos leitores nunca
influíra na sua avaliação deles como pessoas. Era uma consequência mais
do que natural, dada a elevada inteligência de Bakin. Mas,
curiosamente, o inverso, a influência do julgamento das pessoas sobre
sua afeição, também pouco ocorria. Portanto, acontecia que ele
sentisse, simultaneamente, desprezo e afeição por uma mesma pessoa.
Assim sucedia em relação a Heikichi Ômiya, que era um de seus assíduos leitores.

--- Imagine, escrever uma obra desse porte deve ter sido um esforço
descomunal! Hoje o senhor é, sem dúvida, o Rakanchû\footnote{ Rakanchû:
nome de um escritor chinês da dinastia Ming. Todos os nomes chineses
deste texto são transcritos de acordo com a adaptação fonética da
língua japonesa.} do Japão. Mas desculpe"-me se estou falando demais.

Heikichi de novo elevou a voz e riu alto. Talvez chocado por essa voz,
um homem estrábico de penteado \textit{koichô},\footnote{ Literalmente, 
``ginkgo biloba pequeno'', penteado masculino dos citadinos 
do período Edo, cujo tufo de rabo de cavalo engomado, preso ao alto da cabeça raspada, 
tinha a ponta afinada e pendia da nuca.} de baixa estatura,
moreno, que se banhava ao lado, virou"-se, passando os olhos de Heikichi
a Bakin, e, num esgar crítico, escarrou sobre o pavimento.

--- E você, continua aficionado por haicai?

Bakin mudou habilmente de assunto. Mas não porque se importasse com a
expressão do estrábico. Os olhos de Bakin estavam,
\textit{felizmente}, tão fracos que distinguiam já muito pouco.

--- Ah, eu fico lisonjeado com a sua pergunta. Não passo de um diletante.
Hoje um sarau, amanhã outro, assim vou me enfronhando sem a mínima
cerimônia, mas os haicais não me vêm com facilidade. 
A propósito, mestre, o que o senhor pensa
sobre haicai e \textit{tanka}? Qual desses poemas prefere?

--- Bem, tratando"-se de poesia, sou de uma total incompetência. A verdade
é que pratiquei por um tempo, mas\ldots{}

--- Ah, o senhor não está falando sério.

--- É verdade, não combina com meu temperamento. Ainda hoje sou um
ignorante no assunto.

Assim dizendo, acentuou especialmente a frase ``não combina com meu
temperamento''. Ele não se achava incapaz de compor haicai ou
\textit{tanka}. Em consequência, tinha a certeza de que seu
conhecimento nessas áreas não era superficial. Mas sentia, desde muito
tempo, desprezo por esse tipo de arte. Tanto haicai quanto
\textit{tanka} tinham uma estrutura minúscula demais para que ele
pudesse se expressar por completo.

Por isso, por engenhoso que fosse, o que era expresso num haicai ou num
\textit{tanka}, seja lirismo, seja descrição, não tinha a capacidade de
preencher mais que umas poucas linhas de qualquer de suas obras. Essas
formas não passavam, para ele, de artes menores.

\sectionitem
Ocultava"-se, por detrás da ênfase dada a ``não combina com meu
temperamento'', um desprezo dessa ordem. Mas, infelizmente, esse sentido
parece ter escapado por inteiro a Heikichi Ômiya.

--- Ah, então é mesmo assim? E eu que pensava que um grande mestre como o 
senhor dominasse todas as artes com a maior facilidade. Bem que se diz:
``Do céu não se recebem duas dádivas!''

Heikichi disse isso num tom um tanto cerimonioso, enquanto esfregava o
corpo vigorosamente com a toalha, fazendo a pele se avermelhar. Mas,
para o orgulhoso Bakin, o mais insuportável de tudo era que sua modéstia
fosse tomada ao pé da letra. Ante isso, ele jogou a toalhinha e a bucha
no chão e, levantando"-se a meio corpo, respondeu vangloriando"-se, 
numa expressão amarga:

--- Se bem que eu me igualaria a esses mestres de haicai e de
\textit{tanka} de hoje em dia, se assim o quisesse.

Mas, ao dizer isso, subitamente sentiu vergonha por seu orgulho
infantil. Mesmo quando Heikichi havia elogiado seu \textit{Hakkenden}
nos termos mais entusiastas, ele não ficara especialmente lisonjeado.
Por conseguinte, o fato de ser considerado naquele momento incapaz de
compor haicai e \textit{tanka} não lhe deveria ser um motivo de
insatisfação. Depois de assim se autoanalisar, jogou às pressas água
quente sobre os ombros, como se quisesse esconder seu rubor interior.

--- Não é mesmo? Se não fosse assim, seria impossível escrever tal
obra"-prima. Ah! Então fui mesmo perspicaz em julgar que um mestre
como o senhor certamente também comporia poemas! Oh, desculpe"-me pela gabolice.

Heikichi riu de novo, em alto e bom tom. O homem estrábico já não se
encontrava mais lá. O escarro também já havia sido levado pela água com
a qual Bakin se lavara. Mas é evidente que Bakin estava mais embaraçado
do que antes.

--- Oh, estou falando demais. Vou tomar um banho.

Um pouco envergonhado e sentindo"-se um tanto irritado consigo mesmo, ele
se levantou lentamente, enquanto proferia essas palavras, para se
afastar daquele admirador incondicional. Mas o fato de o autor ter se
vangloriado despertou naquele leitor como que um sentimento de orgulho.
Lançou então ao mestre as seguintes palavras:

--- Então, mestre, qualquer dia desses será que o senhor poderia compor
para mim um haicai ou um \textit{tanka}? Será que o senhor não poderia?
Não se esqueça, por favor. Eu também já me vou. O senhor deve estar
muito atarefado, mas, se passar perto de minha casa, por favor,
visite"-me. Eu também irei visitá"-lo.

Lavando mais uma vez a toalhinha e seguindo com o olhar a silhueta de
Bakin, que se dirigia à entrada do banho, pôs"-se a pensar como contaria
à esposa seu encontro com o mestre Kyokutei, quando chegasse à sua casa.

\sectionitem
Dentro do banho público estava escuro como o entardecer. E o vapor
estava mais denso do que a neblina. Bakin, com sua vista fraca, abrindo
caminho tropegamente entre as pessoas que ali se encontravam, procurou,
tateando, um lugar num canto e nele mergulhou seu corpo todo enrugado.

A temperatura da água lhe pareceu bem alta. Sentindo o calor da água a
lhe penetrar nas pontas dos dedos, respirou profundamente e passeou o
olhar lentamente a seu redor. Deveria haver sete ou oito cabeças
mergulhadas naquela penumbra. À volta daquelas pessoas que ora
conversavam, ora cantavam, o brilho turvo da porta de entrada se
refletia na superfície plana da água, misturando"-se à gordura humana,
que preguiçosamente se movia em ondas. A isso se somava, atingindo as
narinas, o cheiro enjoativo dos banhos públicos.

A imaginação de Bakin tendia sempre ao romantismo. ``Em meio ao vapor
daquele banho público, ele viu, sem o menor esforço, formar"-se uma cena
que gostaria de descrever num romance. Sobre o barco, havia uma pesada
cobertura de lona. Lá fora, no mar, parecia começar a ventar ao cair do
sol. Podia"-se ouvir, como se sacudissem um tonel de óleo, o som
opressivo das vagas batendo no casco. O que movia a lona, juntamente
com esse som, parecia ser a batida de asas de morcego. Um dos marujos,
como que incomodado com isso, espiava para fora do navio. Sobre o mar
coberto de neblina, uma lua crescente e vermelha pairava no céu,
lúgubre. Nisso\ldots{}''

O seu devaneio foi bruscamente interrompido naquele ponto. É que seus
ouvidos captaram a voz de alguém que, dentro do mesmo banho, fazia a
crítica dos \textit{yomihon} de sua autoria. A voz, o tom, tudo
revelava a intenção de fazê"-lo ouvir o que se dizia. Bakin preparava"-se
para sair do banho, mas decidiu ficar e ouvir atentamente a crítica.

--- Ele, que se gaba de ser o grande mestre Kyokutei ou ``Chefe da Casa
Literária'', tudo o que ele, Bakin, escreve não passa de puro plágio.
Afinal, o \textit{Hakkenden} não é uma mera imitação do
\textit{Suikoden}? Mas isso ainda passa, já que afinal de contas o
original vem da China. Assim, só por ele ter lido o livro, já é um
grande feito. Aliás, como não passa de uma ordinária cópia do escritor
Kyôden,\footnote{ Kyôden Santô (1761--1816), pintor e escritor de vários 
gêneros de \textit{gesaku}, nasceu na cidade de Edo numa típica família de comerciantes 
e alcançou grande popularidade com seu estilo cômico, cheio de trocadilhos 
linguísticos e jocosos sobre os hábitos dos habitantes do ``mundo alegre'' 
das áreas de prazeres.} nem vale mesmo a pena se irritar com coisa tão ridícula.

Bakin voltou os olhos debilitados em direção ao homem que assim o
insultava. Devido ao vapor, ele não podia ver bem, mas parecia"-lhe ser
aquele indivíduo estrábico, de penteado \textit{koichô}, que pouco
antes estava ao seu lado. Aquele homem, certamente irritado pelos
elogios que Heikichi fizera ao \textit{Hakkenden}, devia estar
descarregando sua raiva com o propósito de irritá"-lo.

--- Em primeiro lugar, Bakin utiliza um pincel muito superficial. No
fundo, nada tem a dizer. Se tiver, não deve passar de explicações dos
quatro livros e cinco cânones do Confucionismo do mesmo nível de
qualquer professorzinho de primário. É por isso que ele não conhece
nada da atualidade. E para provar, tudo o que escreve são histórias
referentes ao passado. Como não consegue escrever histórias atuais de
amores como o de Osome e Hisamatsu, trata esses temas em estilo
clássico, como na sua obra \textit{Histórias de amor de Osome e
Hisamatsu: sete flores de outono}. Tomando emprestadas as palavras do
emérito Bakin, ``exemplos desse gênero abundam''.

O sentimento de ódio não aflora, mesmo quando se quer, se um dos
oponentes tem consciência de sua superioridade. Assim, Bakin não chegou
a odiar o outro, mesmo tendo se irritado com suas palavras. Em vez
disso, o que sentiu foi uma vontade de expressar seu desprezo. No
entanto, sua sabedoria, talvez devido à idade, impediu"-o de realizar
esse desejo.

--- Nesse aspecto, escritores como Ikku\footnote{ Ikku Jippensha 
(1765--1831), de origem samurai, torna"-se citadino e escreve nos
gêneros \textit{sharebon, kokkeibon}, entre outros, tendo produzido mais de
360 obras seriadas. Seu trabalho mais conhecido é \textit{Tôkaidô hizakurige} 
(Peregrinação a pé pela estrada Tôkaidô), cujo primeiro
volume foi lançado em 1802, tendo sido interpretado por xilogravuras de 
Hiroshige Andô.} e Sanba são admiráveis. Em suas obras, pode"-se
apreender a verdadeira natureza do homem. Certamente não foram
elaboradas com ligeireza de recursos nem com conhecimentos mal
digeridos. É aí que se encontra a grande diferença entre eles e esse
Bakin, que também se autodenomina ``Ermitão Saryûken''.

Pela sua experiência, ouvir difamações não era apenas desagradável, mas
também muito arriscado. Ou seja, não é que ele se desencorajasse ao
acatar a difamação, mas, como reação contra ela, ele acabava
introduzindo elementos opostos na sua obra literária posterior. E, como
resultado desse tipo de motivação impura, frequentemente corria o risco
de produzir uma arte deformada. Excetuando"-se os autores que têm como
único objetivo o de agradar ao público, é curioso notar que justamente
aqueles que têm um mínimo de personalidade correm, muitas vezes, o
risco de se deixar levar por esse perigo. Por isso Bakin, até aquela
idade, prudentemente decidira não ler críticas maledicentes de suas
obras. Mas, embora assim pensasse, não deixava de sentir, por outro
lado, certa tentação de lê"-las. Se ele ficara à escuta dos insultos
daquele \textit{koichô} no banho público, em parte fora por ceder a
essa tentação.

Tomando consciência disso, ele reprovou a sua tolice em continuar,
ociosamente, imerso na água. Deixando de ouvir a voz aguda do
\textit{koichô}, saiu, transpondo com passadas vigorosas a soleira do
banho. Lá fora, por entre o vapor, podia"-se ver o céu azul, e naquele
céu azul, caquis banhavam"-se calidamente ao sol. Sentado em frente à
cuba, Bakin tranquilamente tomou a última ducha.

--- De qualquer forma, Bakin é um impostor. E salve o Rakanchû do Japão!

Dentro do banho, talvez pensando que Bakin ainda se encontrasse lá, o
homem estrábico continuava, como antes, lançando sua filípica acerba.
Pode ser que ele, devido a seus olhos estrábicos, não tenha visto a
figura que saía, transpondo a soleira do banho.

\sectionitem
Mas, quando saía do banho público, o desalento já tomava conta de Bakin.
Isso queria dizer que a crítica venenosa do estrábico certamente
atingira seu objetivo. Caminhando pela cidade de Edo, sob o céu claro
de outono, ele examinou meticulosamente, passando pelo crivo de seu
senso crítico, as palavras malévolas que ouvira no banho, uma por uma.
E  conseguiu dessa forma convencer"-se, afinal, de que aqueles 
argumentos idiotas nem sequer mereciam ser considerados sob qualquer
ponto de vista. No entanto, seu estado de espírito, uma vez perturbado,
parecia não recuperar facilmente a calma anterior.

Ergueu o olhar descontente e contemplou as lojas de ambos os lados da
rua. Nas lojas, alheias ao seu estado de espírito, as pessoas
concentravam"-se em suas ocupações cotidianas. 
Assim, um cortinado alaranjado da tabacaria trazia seu nome: 
``Folhas Célebres das Regiões''; numa tabuleta amarela, com formato de pente, 
estava escrito: ``Excelência em Pentes''; lia"-se ``Palanquin'' nas lanternas 
penduradas frente a uma loja; numa bandeira com bastões grafava"-se: ``Adivinhações'', 
tudo formava uma fila sem sentido que passava desordenadamente por seus olhos.

``Por que será que me aflijo tanto com as críticas malévolas daqueles a
quem desprezo?''

Bakin continuava a pensar.

``O que me desgosta é, em primeiro lugar, o fato de aquele estrábico ter
tido más intenções para comigo. Já o simples fato de ser alvo de más
intenções me é desagradável, não importa por que motivos; então, não há
nada que eu possa fazer.''

Pensando daquela forma, sentiu"-se embaraçado ante sua vulnerabilidade.
De fato, poucos homens eram tão suscetíveis à hostilidade alheia como
ele. Naturalmente, já percebera havia muito tempo a verdade: aquelas
duas atitudes, que do ponto de vista da ação pareciam totalmente
contraditórias, provinham na realidade de uma mesma causa, de uma mesma 
reação nervosa.

``Mas ainda há outra coisa que me desagradou. É que eu fui colocado no
mesmo nível que aquele homem estrábico. Nunca apreciei esse tipo de
tratamento. É por isso que não participo de disputas e jogos.''

Quando sua análise chegou àquele ponto e ia avançar mais um passo,
repentinamente seu estado de espírito mudou. Isso se tornou
perceptível, pois seus lábios, até então firmemente cerrados,
descontraíram"-se subitamente.

``E, por último, certamente me desagradou também o fato de que aquele
estrábico foi quem me colocou nessa posição. Se ele fosse um homem de
nível um pouco mais elevado, sem dúvida eu teria reagido para repelir
esse sentimento de desagrado. De qualquer forma, não há nada que eu
possa fazer com relação àquele estrábico.''

Com um sorriso forçado, voltou"-se para o alto céu. Daquele céu, os
gritos sonoros do milhafre caíam como chuva que chove junto com raios
do sol. Percebeu que o desalento que até então o tomara aos poucos ia
desaparecendo.

``Mas, por mais que o estrábico me critique, o único efeito é me
desagradar. Por mais que o milhafre cante, ele não consegue impedir o
curso do sol. É certo que o meu \textit{Hakkenden} um dia terá sua
conclusão. E nesse dia o Japão passará a ter dentro de sua história uma
obra"-prima literária incomparável.''

Velando a autoconfiança recém recobrada, virou"-se no rumo de casa,
enveredando calmamente por uma ruela.

\sectionitem
Ao chegar à casa, Bakin reconheceu, sobre as pedras da entrada em
penumbra, um par de sandálias com tiras finas trançadas. Vendo"-as, logo
lhe veio à mente o rosto inexpressivo da visita. E também pensou
amargamente no aborrecimento de ter de perder seu tempo.

--- Mais uma manhã perdida!

Assim pensava enquanto subia o degrau, quando a empregada Sugi veio às
pressas recebê"-lo e, ainda com as mãos no assoalho, disse, levantando
seu rosto para ele:

--- O senhor da Editora Izumiya está à sua espera na sala de visitas.

Assentindo com a cabeça, ele passou a toalhinha molhada para as mãos de
Sugi. Mas não tinha a menor vontade de se dirigir logo à sala de
estudo.

--- Onde está Ohyaku?

--- A senhora foi para o templo.

--- Omichi também?

--- Sim. Juntamente com o filho.

--- E o meu filho?

--- Ele foi à casa do senhor Yamamoto.

Toda a família estava ausente. Ele experimentou uma sensação semelhante
à decepção. Sem ter alternativa, abriu a porta corrediça da sala de
estudo que ficava ao lado da entrada.

Lá dentro, no meio da sala, um homem algo afetado, de rosto branco e 
brilhante, com uma piteira fina de prata na boca, encontrava"-se sentado
em posição ereta. Na sala de estudo, além de um biombo com uma
inscrição colada e um par de rolos pintados com motivos de bordos
avermelhados e crisântemos amarelos pendurados no nicho, não havia
qualquer outro adorno. Ao longo da parede, mais de cinquenta estantes
alinhavam, num plano sóbrio, a cor envelhecida de sua madeira de
paulônia. Podia"-se supor ter passado um inverno desde a troca do papel
das portas corrediças. Sobre o branco quadriculado dos papéis colados,
uma grande sombra de bananeira rasgada, iluminada ao sol de outono,
projetava"-se em movimentos diagonais. Por conseguinte, o traje
demasiadamente luxuoso do visitante não se harmonizava em nada com o meio.

--- Oh, mestre, prazer em revê"-lo.

Ao deslizar da porta corrediça, a visita abaixou a cabeça com
reverência, falando com naturalidade. Era o editor Ichibei Izumiya, que 
publicara o \textit{Kinpeibai}, obra que, após o \textit{Hakkenden},
era a mais popular das que escrevera.

--- Desculpe tê"-lo feito esperar. É que hoje, excepcionalmente, fui ao
banho público.

Fazendo involuntariamente uma ligeira careta, Bakin sentou"-se, como
sempre, formalmente.

--- Verdade, mestre? No banho matinal?

O tom da voz que Ichibei emitiu era de admiração.

Existiam poucas pessoas que, como ele, admiravam, tão prontamente, mesmo
os acontecimentos mais insignificantes. Ou melhor, era mesmo raro
encontrar pessoas que fingissem tanta admiração. Tirando uma lenta
baforada, Bakin foi, como sempre, conduzindo o assunto ao que interessava.
Não lhe agradavam em nada os modos lisonjeiros de Izumiya.

--- E, então, o que o traz hoje aqui?

--- Bem, será que o senhor poderia me ceder mais um manuscrito? ---
respondeu Ichibei numa voz suave como a das mulheres, girando a piteira
com a ponta dos dedos. Aquele homem tinha um caráter estranho. Isto é,
sua conduta externa não correspondia, na maior parte das vezes, à sua
vontade interna. Longe de corresponder, manifestava"-se sempre num
sentido completamente oposto. Portanto, quando tomava uma decisão, o
tom de voz que emitia era tanto mais suave quanto mais firme sua
intenção.

Ao ouvir essa voz, novamente Bakin fez uma careta involuntária.

--- Mais um manuscrito? Isso me é impossível.

--- Mas como? O senhor tem algum problema?

--- Mais que um problema! Como este ano estou me dedicando a escrever 
\textit{yomihon}, não vejo possibilidades de trabalhar com os
\textit{gôkan}.\footnote{ \textit{Gôkan}: subclasse de
\textit{kusazôshi} (tipo de \textit{gesaku}). Caracteriza"-se
principalmente pelo grande número de ilustrações e por ser seriado em
vários tomos, cada um com pelo menos cinco livros.}

--- Realmente, parece que o senhor está muito ocupado!

Assim dizendo, Ichibei bateu com a piteira no cinzeiro e, como se isso
fosse um sinal, pôs"-se de repente a falar sobre o ladrão Nezumikozô
Jirôdaifu, numa expressão completamente esquecida do assunto anterior.

\sectionitem
Nezumikozô Jirôdaifu era o célebre ladrão que fora preso no começo do
quinto mês daquele ano, tendo sido decapitado em meados de agosto. Por
ter o costume de penetrar furtivamente nas residências de grandes
senhores, para depois distribuir o dinheiro roubado entre os pobres, na
época era conhecido pelo estranho apelido de ``ladrão"-de"-alma"-caridosa''
e era muito aplaudido por toda parte.

--- É surpreendente, mestre, que ele tenha assaltado setenta e seis
mansões, roubando três mil, cento e oitenta e três \textit{ryô} e dois
\textit{bu}!\footnote{ \textit{Ryô}, \textit{bu}: unidades monetárias do
período Edo (1603--1868).} Embora seja um ladrão, não se trata de
um ladrão qualquer.

Sem querer, Bakin se tomou de curiosidade. Por trás das histórias que
Ichibei contava, escondia"-se sempre a pretensão de fornecer temas para
os escritores. Essa pretensão, é verdade, nunca deixava de irritar
Bakin. Mas, mesmo irritado, sentia"-se movido pela curiosidade. Talvez
porque ele, um artista com bastante talento, caísse facilmente naquele
tipo particular de tentação.

--- Hum, isso é realmente extraordinário. Já tinha ouvido várias
histórias, mas não imaginava que chegasse a tal ponto.

--- Enfim, ele deve ser então o maior dos ladrões. Ouvi dizer que
trabalhava na escolta de Arao, o Senhor de Tajima, o que parece tê"-lo
familiarizado com os interiores dessas mansões. Segundo os que o viram
exposto como prisioneiro, tratava"-se de um homem corpulento e simpático
e, na ocasião, vestia crepe azul"-marinho de Echigo por sobre o quimono
de seda branca. Não se parece em tudo com um desses personagens de suas
obras?

Respondendo evasivamente, Bakin tirou mais uma baforada. Mas Ichibei não
era homem de se desencorajar apenas por conta de respostas evasivas.

--- O que acha disso? Não poderia escrever mais um episódio de
\textit{Kinpeibai}, incluindo esse Jirôdaifu? Sei que o senhor
evidentemente está ocupadíssimo. Mas, por favor, faça um esforço e
acate esse meu pedido.

Deixando o tema do ladrão, ele rapidamente voltou ao pedido do
manuscrito. Mas Bakin, acostumado a seus métodos usuais, manteve"-se
irredutível. Aliás, seu mau humor se acentuou ainda mais. Isso porque
ele se achou ridículo por ter sido levado a sentir certa curiosidade,
mesmo que momentânea, pelas artimanhas de Ichibei. Fumando da piteira
com enfado, ele finalmente começou a argumentar da seguinte forma:

--- Antes de mais nada, se eu for forçado a escrever, não conseguirei
fazer nada decente. É desnecessário dizer que isso influirá na venda e,
dessa forma, não seria interessante nem para o senhor. Pensando assim,
não seria melhor para ambos aceitar a minha posição?

--- Pode ser, mas gostaria que o senhor tentasse mesmo assim. Que lhe
parece?

Assim dizendo, com o olhar Ichibei ``acariciou'' (foi com essa palavra que
Bakin adjetivou o olhar de Izumiya) o rosto de Bakin. E soltou em 
baforadas a fumaça do tabaco pelas narinas.

--- Impossível escrever. Mesmo se quisesse, não poderia, pois não tenho
tempo.

--- Ah, isso me causará um enorme transtorno!

E, tendo isso posto, passou bruscamente a falar de outros escritores contemporâneos, 
ainda com a piteira fina de prata entre os lábios delgados\ldots{}

\sectionitem
--- Parece que vai sair um novo livro de Tanehiko.\footnote{ Tanehiko Ryûtei 
(1783--1842), escritor de \textit{gesaku}, poema do tipo
``louco'' (\textit{kyôka}) e teatro, cuja obra mais famosa é \textit{Nise
Murasaki Inaka Genji }(O Genji das províncias da falsa Murasaki), de
1829, ligeiramente influenciada por Bakin e ilustrada por Kunisada Utagawa.} 
Deve ser, como sempre, uma obra sentimental, mas
principalmente de uma beleza elegante. Tenho a impressão de que somente
ele seria capaz de escrever o que escreve.

Não se sabe por que, mas Ichibei tinha o costume de citar todos os
escritores sem utilizar nenhum pronome de tratamento.\footnote{ Na
língua japonesa, os pronomes de tratamento são quase sempre
indispensáveis. A atitude de Izumiya revela, portanto, um descaso muito
grande de sua parte em relação aos escritores que edita. \par } Sempre
que Bakin o ouvia se referir a eles, imaginava que, em sua ausência,
ele também seria chamado simplesmente de ``Bakin''. Que necessidade teria
ele de fazer o favor de produzir um manuscrito para um homem frívolo
como aquele, que considerava os escritores seus operários, chegando
mesmo a se referir a eles sem nenhum pronome de tratamento? Não eram
raros os momentos de irritação quando assim pensava e se exasperava.
Também nesse dia, ao ouvir o nome de Tanehiko, sua expressão já amarga
se acentuou ainda mais. Mas Ichibei parecia pouco se importar com isso.

--- E sabe? Também estamos pensando em publicar Shunsui.\footnote{ Shunsui Tamenaga 
(1790--1843), escritor de \textit{ninjôbon}, cuja obra"-prima é 
(\textit{Shunshoku}) \textit{Umegoyomi} [O calendário de ameixeira (tingido de
amor erótico)], de 1832--33.} Sei que o senhor não o aprecia, mas
parece que, de fato, ele agrada muito à plebe.

--- Ah, é verdade?

Da memória de Bakin surge, exageradamente aviltado, o rosto de Shunsui,
que entrevira em alguma ocasião. ``Eu não sou um escritor. Sou um
empregado que escreve histórias de amor para oferecê"-las a uma
clientela, de acordo com seu gosto.'' Bakin sabia havia muito que
Shunsui assim se definia. Portanto, era óbvio que desprezasse do fundo
de sua alma aquele escritor que nem parecia escritor. Mas,
independentemente disso, ao ouvir Ichibei chamá"-lo sem o tratamento de
respeito, não conseguia controlar a mesma sensação de desagrado.

--- No final das contas, tratando"-se de histórias de amor, ele é um
especialista. E é famoso pela rapidez com que escreve.

Assim dizendo, Ichibei olhou de relance para Bakin e logo depois pousou
os olhos na piteira de prata que mantinha entre os lábios. Nessa
expressão fugaz havia algo de terrivelmente vulgar. Pelo menos, era
assim que Bakin o percebia.

--- Para produzir tal quantidade, dizem que seu pincel desliza sem parar e 
não se afasta do papel até completar uns dois ou três capítulos. Por
falar nisso, mestre, o senhor também escreve rápido?

A essa pergunta, Bakin sentiu ao mesmo tempo desagrado e ameaça.
Naturalmente, para ele, que era muito orgulhoso, não era nada agradável
ser comparado com Shunsui ou Tanehiko na rapidez do pincel. Além do
mais, ele escrevia devagar. Havia até ocasiões em que desanimava,
achando que isso talvez refletisse sua própria incompetência. Mas, por
outro lado, havia vezes em que valorizava sua lentidão, fazendo dela um
parâmetro que media seu bom senso artístico. Só que, apesar dessas
reflexões, jamais pensou em delegar ao vulgo o julgamento de sua
lentidão. Assim, passeando o olhar pelos bordos vermelhos e crisântemos
amarelos do nicho da sala, vomitou essas palavras:

--- Depende da hora e da ocasião. Às vezes sou rápido, às vezes lento. 

--- Ah, sim, da hora e da ocasião. É claro!

Ichibei admirou"-se pela terceira vez. Mas era óbvio que não passava de
uma admiração pura e simples. Depois disso, ele logo atacou novamente:

--- É, mas\ldots{} Será que não poderia aceitar o pedido do manuscrito a que eu
me referi? Também Shunsui\ldots{}

--- O senhor Shunsui e eu somos diferentes.

Quando se irritava, Bakin tinha o hábito de torcer o lábio inferior para
o lado esquerdo. Naquele momento, repuxou o lábio violentamente para a
esquerda.

--- Bem, o senhor terá de me desculpar\ldots{} Sugi, ó Sugi, você já deixou
os sapatos do senhor Izumiya arrumados?

\sectionitem
Desembaraçando"-se de Ichibei Izumiya, agora sozinho, Bakin recostou"-se
na coluna da varanda e, contemplando a vista do pequeno jardim,
esforçou"-se para controlar a todo custo a irritação que ainda sentia.

No jardim banhado pelos raios de sol, o outono de cores quentes reinava
em alguns metros cúbicos, sobre a bananeira de folhas rasgadas, a
paulônia verde já quase sem folhas, o verde dos ciprestes e bambus. O
lótus do pequeno tanque de água da frente já tinha menos flores, mas os
jasmins"-do"-imperador plantados do outro lado da cerca continuavam
exalando seu doce perfume. Do alto do céu longínquo, de vez em quando
tombavam naquela direção os gritos do milhafre, semelhantes ao trinado
da flauta.

Comparada àquela natureza, naquele momento ele sentiu ainda mais a 
vulgaridade do mundo. A infelicidade de um homem que vive num mundo
vulgar encontra"-se justamente no fato de que, perturbado pela
vulgaridade, ele próprio também seja obrigado a tomar atitudes
vulgares. Ele, na verdade, havia expulsado Ichibei Izumiya. Expulsar
alguém não é, naturalmente, um ato nada nobre. Mas, levado pela
vulgaridade do outro, ele sentira"-se obrigado a praticar um ato vulgar.
E assim o fez. Tê"-lo feito significava simplesmente que se tornara tão
ignóbil quanto Ichibei. Isto é, o fato é que ele se corrompera.

Essa reflexão o fez lembrar"-se de um acontecimento análogo que lhe  
ocorrera bem recentemente. Na primavera do ano anterior, um homem de
nome Masabei Nagashima, que morava em Kamishinden, distrito de Kuchiki,
da província de Sagami, enviara"-lhe uma carta pedindo"-lhe que o tomasse
como discípulo. De acordo com tal carta, aquele homem, desde que
perdera a audição, aos vinte e um anos, até aquele dia, quando contava
vinte e quatro, havia"-se dedicado a escrever somente \textit{yomihon},
com a determinação de se tornar conhecido no mundo. É desnecessário
dizer que era leitor assíduo do \textit{Hakkenden} e do
\textit{Juntôki}. Porém, morar no interior lhe era um obstáculo ao
trabalho. ``Por isso, será que o senhor não poderia me aceitar como um
dependente na sua casa? Além disso, tenho manuscritos de seis volumes
de \textit{yomihon}. Gostaria de publicá"-los por uma editora
apropriada, depois que o senhor os revisasse.'' O conteúdo da carta era
mais ou menos esse. Aos olhos de Bakin, as demandas todas daquele homem
eram, naturalmente, puramente egoístas. Mas sua surdez inspirou"-lhe
certa compaixão, pois ele próprio tinha problemas de visão. A resposta
de Bakin, de que infelizmente lhe era impossível atender ao pedido, foi
a mais respeitosa, contrariando o seu hábito. No entanto, na carta
seguinte que recebera, havia somente palavras de acusação do começo ao
fim e nada mais.

``Eu lhe fiz o favor de ler pacientemente o seu \textit{Hakkenden}, o seu
\textit{Juntôki}, obras longuíssimas e de péssima qualidade, mas o
senhor não se dignou sequer a passar os olhos nos poucos seis volumes do meu \textit{yomihon}. 
Isso não é uma prova da vulgaridade de seu caráter?''. Começando com essas
palavras, a carta terminava simplesmente com a acusação de que o fato
de um veterano não aceitar em sua casa um iniciante era um atestado da
sua avareza. Furioso, Bakin logo se pôs a escrever a resposta. Na
carta, declarava que a maior ofensa de sua vida era que suas obras
fossem lidas por um sujeito tão falso como ele. Depois do episódio, não
teve mais notícias dele. Será que ainda estaria escrevendo
\textit{yomihon}? E será que ainda estaria sonhando em ser lido, um 
dia, por todas as pessoas do Japão?\ldots{}

Imerso na recordação, Bakin não pôde deixar de sentir ao mesmo tempo um
desgosto tanto em relação a Masabei Nagashima quanto a si próprio. Mas
o sol dissolvia, candidamente, o aroma das flores de lótus. Tanto a
bananeira quanto os pinheiros, silenciosos, não moviam uma folha
sequer. Os gritos do milhafre também soavam alegres como antes. ``Ah,
esta natureza e\ldots{} pessoas como aquelas\ldots{}'' Ele continuou recostado à
coluna da varanda, como se estivesse sonhando, até que, dez minutos
mais tarde, a empregada Sugi veio avisar que o almoço estava pronto.

\sectionitem
Após haver terminado seu almoço solitário, retirou"-se finalmente à sala
de estudo e, para acalmar seu estado de espírito inquieto e cheio de
desagrado, abriu a obra \textit{Suikoden}, que ele não relia há muito.
Na passagem aberta ao acaso, o personagem Rinchû Hyôshitô contemplava
um celeiro em chamas, do alto de um templo de montanha, numa noite de
neve e tempestade. Naquele cenário dramático, ele conseguiu recuperar
sua emoção habitual. Mas, depois de certo tempo, começou a sentir, sem
saber por que, uma estranha insegurança.

A família, que fora ao templo, ainda não regressara. O interior da casa
estava imerso em silêncio. Escondendo a expressão sombria de seu rosto
diante do \textit{Suikoden}, Bakin dava baforadas no tabaco sem sentir
nenhum prazer. E, no meio da fumaça, flutuou uma dúvida que há muito o
acometia.

Era uma dúvida que sempre atormentava Bakin, em relação a suas atitudes
confucionista e artística. Fazia muito tempo que não punha em dúvida a
autoridade da doutrina do ``Caminho dos Sábios Monarcas''. As suas obras,
como ele mesmo declarava, constituíam genuínas expressões artísticas do
``Caminho dos Sábios Monarcas''. Portanto, sob esse aspecto não existia
nenhuma contradição. Mas entre o valor que esse caminho atribuía à arte
e o que seus sentimentos lhe procuravam conferir havia uma discrepância
maior do que se supunha. Por conseguinte, da mesma forma que seu lado
confucionista afirmava certo valor à arte, seu lado artístico
naturalmente afirmava outro. É certo que, para vencer essa contradição,
ele não deixava de recorrer a algumas ideias conciliadoras e banais. De
fato, houve vezes em que, perante o público, ele tentou esconder sua
atitude ambígua em relação à arte por meio dessa teoria dúbia de harmonização.

Mas, por mais que enganasse o público, não enganava a si próprio. Mesmo
negando o valor dos escritos \textit{gesaku}, chamando"-os de
``instrumentos de edificação'', quando se deparava com a emoção artística
que sempre se expandia em seu interior, imediatamente começava a se
sentir inseguro --- na verdade era esse o motivo pelo qual,
ocasionalmente, sentia um efeito inesperado numa certa passagem do
\textit{Suikoden}. Em relação a essa questão, as ideias de Bakin eram
covardes, e ele tentou desviar seu pensamento para longe, para a
família, enquanto fumava da piteira em silêncio. Mas o
\textit{Suikoden} se encontrava à sua frente. Não podia afastar com
tanta facilidade a insegurança que emanava do livro. Foi naquele
momento que Noboru Watanabe, cujo nome artístico era Kazan,\footnote{ Kazan Watanabe 
(1793--1841), pintor do estilo \textit{nanga} (do sul da
China) que estudou também confucionismo e autores holandeses. Tendo
escrito livro criticando a política de exclusão do governo, foi banido
para sua terra natal, onde se suicidou.} chegou para visitá"-lo, coisa
que não fazia há muito tempo. Vestido de traje formal com
\textit{haori} e \textit{hakama}, carregava um pacote de tecido violeta
e parecia ter vindo para devolver os livros que havia tomado emprestado.

Alegrando"-se, Bakin foi especialmente receber o amigo na entrada da casa.

--- Vim aqui hoje para lhe devolver os livros que me emprestou. Além
disso, tenho algo a mostrar"-lhe.

Kazan assim disse, ao ser conduzido à sala de estudo. Com efeito,
trazia, além do pacote de pano, um rolo de seda de pintura, envolto em
papel.

--- Se dispuser de tempo, gostaria que o visse.

--- Oh, imediatamente, sim, deixe"-me apreciá"-lo.

Esboçando um sorriso deliberado, como se disfarçasse certa excitação,
Kazan expôs o rolo de seda para fora do papel. A pintura retratava
árvores desfolhadas e tristes, espalhadas umas ao longe e outras bem
perto, em meio às quais dois homens de pé conversavam animadamente,
esfregando as mãos. As folhas amarelas caídas no chão do bosque, os
corvos em bandos entre os galhos das árvores --- por mais que
contemplasse qualquer ponto da pintura, não havia uma única parte onde
não pairasse a fria atmosfera de outono.

Os olhos de Bakin, ao pousar nos dois monges Kanzan e Jittoku pintados
em leve aguada, começaram a brilhar, ganhando cada vez mais uma
expressão de ternura.

--- Como sempre, é realmente uma obra"-prima. Faz"-me lembrar aqueles versos
do poeta chinês Ô Makitsu:

\begin{verse}
Ao soar do gongo,\\
Descem dos ninhos os corvos,\\
Ao pisar o bosque deserto,\\
Soam as folhas secas.
\end{verse}

\sectionitem
--- Terminei essa pintura ontem e, como ela me agradou muito, pensei em
oferecê"-la ao senhor, se assim o desejar, logicamente.

Com ar de satisfação, Kazan assim proferiu, acariciando o queixo azulado
da barba bem"-feita.

--- Quer dizer, das pinturas que fiz até agora, esta é a que mais me
agrada. As imagens que eu sempre idealizo nunca consigo realizar.

--- Ah, eu lhe agradeço muito! Mas fico embaraçado de ficar somente
recebendo presentes\ldots{}

Contemplando a pintura, Bakin agradeceu num murmúrio. Isso porque
naquele momento alguma questão de seu trabalho inacabado repentinamente
se iluminou em seu espírito. Mas Kazan, por sua vez, pareceu absorto em
suas pinturas.

--- Todas as vezes que vejo as pinturas dos antigos, sempre me pergunto
como é que conseguiram pintar daquela forma. Mesmo árvores, mesmo
pedras, mesmo personagens acabam se transformando realmente em árvores,
pedras, personagens e, mais ainda, seus espíritos vivem livremente
dentro das coisas que pintaram. Isso é realmente admirável. Eu, então,
comparando"-me a eles, não passo de uma criança.

Como se estivesse a sentir inveja, Bakin olhou para Kazan, que pensava
somente na sua própria pintura, e fez uma brincadeira, contrariando seu
hábito:

--- Mas parece que os antigos diziam que a nova geração deve ser temida.

--- Ah, a nova geração é mesmo temerária. Portanto, nós, prensados entre
os antigos e os novos, avançamos empurrados de um e de outro lado, sem
poder nos movermos. Isso não se passa somente conosco. Também era assim
com os antigos e provavelmente também o será com os novos.

--- Certamente, se não avançarmos, seremos logo destruídos. Por isso,
parece ser essencial procurar uma forma de avançar, ainda que seja um
passo só.

--- Sim, isso é mesmo essencial.

O anfitrião e o visitante, emocionados por suas próprias palavras,
calaram"-se por instantes. E ambos voltaram seus ouvidos ao silêncio de
um dia calmo de outono.

--- E como está o \textit{Hakkenden}, avançando sempre? --- Kazan iniciou
outro assunto, após um tempo.

--- Não, estou sofrendo porque não avanço. Parece que eu também não me
comparo aos antigos.

--- Mas uma pessoa como o senhor, mestre, não deveria dizer isso. Fico
embaraçado.

--- Embaraçado fico eu. Mas não há outra solução senão prosseguir até onde
puder. Assim pensando, ultimamente decidi"-me a travar uma batalha de
vida ou morte com o \textit{Hakkenden}.

Dizendo isso, Bakin riu amargamente, como se sentisse vergonha de si
mesmo, e continuou:

--- Mesmo considerando que o que estou escrevendo não passa de
\textit{gesaku}, não me é tão simples assim, sabe? 

--- Ah, acontece o mesmo com as minhas pinturas. Mas, uma vez começadas,
também penso em chegar até onde puder.

--- Então, será que morreremos ambos na batalha?

Os dois riram em voz alta. Mas em seu riso pairava certa solitude que
somente eles compreendiam. O anfitrião e o visitante sentiram, ao mesmo
tempo, uma espécie de forte emoção nessa solitude. 

--- Mas até que invejo vocês, pintores. É tão bom não sofrer a censura das
autoridades políticas.

Dessa vez, foi Bakin quem mudou de assunto.

\sectionitem
--- É, realmente não a sofremos, mas\ldots{} No que o senhor escreve não deve
haver esse tipo de preocupação.

--- Muito pelo contrário.

Como um exemplo da mesquinharia dos censores de livros, Bakin citou a
ordem que recebera de reescrever um capítulo em que tratava da
corrupção dos funcionários do governo. E acrescentou essa crítica:

--- Não é engraçado como os censores traem"-se a si mesmos quanto mais
acusam os outros? Como eles próprios são corruptos, não gostam que
escrevamos sobre a corrupção e nos mandam reescrever. E mais, como
facilmente são tomados por sentimentos indecentes, consideram obsceno
qualquer livro que trate do amor entre um homem e uma mulher. E assim
se colocam numa posição ridícula, porque acham que sua moralidade é
mais elevada do que a dos escritores. Em outras palavras, são como
macacos que rangem os dentes à sua própria imagem refletida no espelho.
Isso porque se irritam com sua própria vulgaridade.

Vendo Bakin se empenhando tanto nas comparações, Kazan não pôde conter o
riso.

--- Certamente o senhor tem razão. Mas, mesmo tendo de reescrever, isso
não lhe é motivo de desonra. Seja o que for que os censores digam, 
quando se trata de uma grande obra, ela continuará sempre sendo uma
grande obra.

--- Mesmo assim, é frequente eles passarem um pouco dos limites. Sim, de
fato, uma vez me mandaram suprimir cinco ou seis linhas de uma passagem
sobre o envio de víveres aos prisioneiros.

Assim dizendo, o próprio Bakin se juntou a Kazan no riso.

--- Mas daqui a cinquenta ou cem anos os censores terão desaparecido,
porém certamente o \textit{Hakkenden} permanecerá.

--- Permanecendo ou não o \textit{Hakkenden}, tenho a impressão de que,
quanto aos censores, eles sempre existirão.

--- Será mesmo? Eu não consigo pensar deste modo.

--- Não, ainda que os censores desapareçam, pessoas como eles haverão de
existir em qualquer época. Se acharmos que só no passado se queimavam
livros e se enterravam vivos os estudiosos confucionistas, estaremos
completamente enganados.

--- Mestre, ultimamente o senhor tem pronunciado apenas palavras
desencorajadoras.

--- Eu não estou sendo desencorajador. O que me desanima é essa sociedade
onde pululam tais censores.

--- Então, de sua parte, o senhor deve prosseguir trabalhando.

--- É, com efeito, parece"-me que não há outra maneira.

--- E, assim, pereceremos juntos na batalha?

Desta vez, nenhum dos dois riu. Não somente isso, mas também, por alguns
instantes, Bakin olhou para Kazan com uma expressão dura. Pois em suas
palavras, que pareciam levianas, havia uma singular perspicácia.

--- Mas os jovens devem pensar, em primeiro lugar, na sobrevivência.
Morrer em batalha, isso é possível a qualquer hora --- disse Bakin,
passados alguns instantes. Talvez porque ele conhecesse as opiniões
políticas de Kazan, sentiu"-se tomado nesse momento por uma repentina
inquietação. Mas, apenas sorrindo, Kazan nem tentou responder à sua
pergunta.

\sectionitem
Depois que Kazan se retirou, Bakin sentou"-se à mesa, como de hábito, com
o intuito de continuar o manuscrito do \textit{Hakkenden}, aproveitando
a emoção que ainda sentia. Era seu costume, desde há muito, reler o que
produzira no dia anterior, antes de continuar a escrever. Então, também
nesse dia, leu com vagar e atenção as folhas do manuscrito, cheias de
correções em vermelho entre as linhas cerradas.

Mas, por que seria? Aquilo que estava escrito não sintonizava com seu
espírito. Ocultas dissonâncias entre as palavras rompiam a harmonia do
conjunto. No início, ele interpretou isso como efeito do seu
nervosismo.

--- O problema, hoje, é meu estado de espírito. Pois o que está escrito,
escrevi da melhor forma que pude.

Assim pensando, retomou uma vez mais a leitura. Mas a tonalidade
dissonante continuava inalterada. Ele começou a ser tomado pelo pavor,
apesar de sua idade avançada.

--- Como estará a outra parte?

Passou os olhos pela parte que escrevera antes. Também ali via frases 
cruas e inúteis, espalhadas em desordem. Leu a parte anterior àquela. E
depois leu a parte anterior a essa.

Mas, à medida que as lia, a composição imatura e a redação caótica iam
se desenrolando frente a seus olhos. Havia descrições de cenas que não
sugeriam imagem alguma. Havia exclamações que não continham qualquer
entusiasmo. E havia, ainda, uma retórica que não seguia nenhuma lógica.
Todos aqueles manuscritos que lhe custaram muitos dias de trabalho, de
seu ponto de vista atual, não passavam de uma inútil tagarelice. Ele
sentiu de súbito uma dor lancinante a lhe penetrar a alma.

--- Não há outra solução senão reescrever desde o começo! --- gritando assim
com toda a alma, afastou para o lado os manuscritos, como que enojado,
e deitou"-se pesadamente, apoiado num dos cotovelos. Mas, talvez ainda
preocupado, não desviava os olhos da mesa. Sobre aquela mesa, ele havia
escrito o \textit{Yumiharizuki} (Lua crescente), o \textit{Nankano
yume} (Sonhos de Nanka) e agora trabalhava no \textit{Hakkenden}. Todos
os objetos que estavam sobre a mesa --- a pedra de tinta de
Tankei,\footnote{ Tankei: nome da província em que se situa Cantão.} o
peso de papel com formato de \textit{sonri},\footnote{ \textit{Sonri}: 
imagem de um dragão sem chifres, todo
encolhido.} a garrafinha de bronze para água modelada em forma de
sapo, o pára"-vento de porcelana azul ultramar ornada de leão chinês e
peônias e o porta"-pincéis grosso de bambu com orquídeas cinzeladas ---,
todos esses objetos de escritório desde há tempos estavam
familiarizados com a dor que neles acompanhava a criação. Ao olhá"-los,
Bakin não podia reprimir uma inquietação abominável, como se o fracasso
daquele momento projetasse uma sombra escura sobre as obras de sua vida
inteira e como se sua própria competência estivesse dramaticamente
sendo posta em questão.

--- Até há pouco, eu tinha a intenção de escrever uma grande obra"-prima,
sem igual no país. Mas pode ser que esta seja, quem sabe, uma pretensão
bem banal.

Essa inquietação lhe provocou um sentimento quase insuportável de
solitude e desolação. Ele não era de esquecer jamais a modéstia\textit{
}que devia ter diante dos gênios da China e de seu país, a quem
respeitava. Mas, por outro lado, acontecia de se mostrar orgulhoso e
até mesmo insolente em relação aos escritores insignificantes de seu
tempo. Naquele momento, poderia reconhecer facilmente que também ele,
afinal de contas, era tão pouco competente quanto os outros ou que,
como eles, não passava de um desprezível ``porco de Ryôtô''?\footnote{ ``Porco de Ryôtô'': 
porco de Liao"-Toung. História narrada na China, onde
um porco de cabeça branca, tido como raro, foi levado como oferenda ao
Imperador; ao chegar lá, perceberam que muitos porcos ali também tinham
a cabeça branca.} Além disso, o seu poderoso ``ego'' fervilhava demais,
de emoção, para se refugiar dentro da ``iluminação'' e da ``resignação''.

Ainda deitado diante da mesinha, continuava lutando em silêncio contra a
força opressiva do desespero enquanto contemplava seus manuscritos
fracassados, com o olhar de um capitão que vê seu navio afundar. Se,
naquele momento, a porta corrediça de trás não se abrisse bruscamente
e, juntamente com a voz que dizia ``Vovô, cheguei!'', pequenos braços
roliços não abraçassem o seu pescoço, ele certamente ficaria imerso por
um tempo indefinido naquela atmosfera deprimente. Mas o neto Tarô,
assim que abriu a porta, pulou súbito e ligeiro sobre o colo de Bakin,
com a audácia e a espontaneidade características das crianças.

--- Vovô, cheguei.

--- Oh, você voltou cedo.

Com essas palavras, o rosto rugoso do autor de \textit{Hakkenden}
irradiou"-se de alegria, fazendo dele outra pessoa.

\sectionitem
Da sala de jantar, podiam"-se ouvir as vozes animadas da esposa Ohyaku, que
era aguda, e a da nora Omichi, que parecia tímida. A voz masculina
grave que às vezes se misturava parecia indicar que o filho Sôhaku
também já voltara. Sentado a cavalo sobre os joelhos do avô, como se
escutasse atentamente aquelas vozes, Tarô, numa expressão
intencionalmente séria, contemplava o teto. Suas faces, que haviam
estado expostas ao ar frio de fora, estavam vermelhas; as aletas de seu
pequeno nariz fremiam cada vez que respirava.

--- Sabe, vovô, para o senhor\ldots{}

Assim, subitamente, Tarô começou a falar, vestido de um pequeno
\textit{montsuki}\footnote{ \textit{Montsuki}: traje formal
com o brasão da família.} de cor acastanhada. Acompanhando seus
esforços para pensar e para conter o riso, covinhas ora apareciam, ora
desapareciam de suas faces. Seu trejeito convidou Bakin a um riso
espontâneo.

--- \ldots{} é preciso todos os dias\ldots{}

--- Sim, todos os dias\ldots{}?

--- \ldots{} estudar bastante.

Bakin irrompeu num riso. Mas, em meio ao riso, logo prosseguiu:

--- E que mais?

--- E também\ldots{} Espere um pouco\ldots{} O senhor não deve ficar nervoso.

--- É mesmo? Isso é tudo?

--- Não, ainda tem mais.

Virando para trás sua cabeça penteada em um pequeno coque, Tarô se pôs a
rir, também ele. Ao vê"-lo rir, fechando os olhinhos, mostrando os
dentes brancos e formando covinhas, quem poderia imaginar que também
ele, ao crescer, um dia teria a feição miserável dos adultos? Assim
pensou Bakin, imerso na plenitude da felicidade. E isso o fez sentir"-se
ainda mais alegre.

--- Ainda tem mais alguma coisa?

--- Ainda. Muitas coisas.

--- Que tipo de coisas?

--- É que\ldots{} Sabe? Como o senhor vai ficar ainda mais importante\ldots{}

--- Vou ficar importante?

--- Portanto, deve ser mais paciente.

--- Mas eu sou paciente. --- Bakin, sem pensar, falou num tom sério.

--- Que tenha muito mais paciência.

--- Quem é que falou isso?

--- Foi\ldots{}

Com um ar travesso, Tarô espiou"-lhe o rosto. E riu.

--- Vovô, adivinhe quem foi?

--- Bem\ldots{} Como vocês estiveram hoje no templo, deve ter sido o monge.

--- Errou!

Tarô negou, meneando firmemente a cabeça, e, levantando um pouco os
quadris do colo de Bakin, disse, estendendo o queixo para frente:

--- Foi\ldots{}

--- Hum\ldots{}

--- Foi a deusa da misericórdia Kannon, de Asakusa, quem falou isso.

Assim dizendo, a criança, com uma voz que parecia ecoar pela casa toda,
riu alegremente, saltando às pressas de seu colo como se receasse ser
agarrado pelo avô. Satisfeito de lhe ter pregado uma peça, fugiu para a
sala de jantar, batendo palmas com as mãozinhas.

Foi nesse momento que fulgurou na alma de Bakin algo de solene. Em seus
lábios aflorou um sorriso de felicidade. Ao mesmo tempo, seus olhos
foram se enchendo de lágrimas. Não era o caso de perguntar se aquela
brincadeira fora inventada pelo próprio Tarô ou se a mãe é quem o
instruíra. O milagre foi que, naquele momento, ouvira aquelas palavras
pela boca de seu neto.

--- Foi a deusa Kannon quem disse isso? ``Estude bastante. Não se irrite. E
seja mais paciente''?

O velho artista de sessenta e tantos anos aquiesceu como uma criança,
rindo em meio às lágrimas.

\sectionitem
Foi naquela mesma noite.

Bakin retomou o manuscrito do \textit{Hakkenden}, sob a luz fugitiva da
lamparina redonda. Quando trabalhava na escrita, nem a família podia
entrar em seu escritório. No interior da quietude da sala, o crepitar
do pavio que absorvia o óleo e o canto dos grilos ressaltavam em vão a
solidão da longa noite.

Ao pegar em seu pincel, um ponto luminoso quase imperceptível pulsava
dentro de sua cabeça. Mas, à medida que o pincel avançava dez linhas,
vinte linhas, o ponto luminoso gradativamente foi crescendo. Bakin,
que, graças à sua experiência, compreendia o que se passava, foi
conduzindo o pincel, com cuidados redobrados. O advento da inspiração
não difere em nada do fogo. Se não se souber mantê"-lo, mesmo já aceso,
logo acaba por se apagar de novo\ldots{}

--- Não se apresse. E reflita o mais profundamente possível.

Sussurrava isso a si próprio muitas vezes, retendo seu pincel propenso a
correr. Mas, dentro de sua cabeça, aquele brilho de estrelas
fragmentadas de antes já fluía mais rápido do que um rio. E sua força
não cessava de aumentar, acabando por arrastá"-lo imperiosamente.

O canto dos grilos já não chegava mais a seus ouvidos. Seus olhos não
sofriam nem um pouco com a luz fraca da lamparina redonda. O pincel
ganhou vida própria e começou a deslizar num ímpeto sobre o papel. Como
se lutasse contra os homens e os deuses, ele continuava a escrever com
fúria.

Exatamente como a Via Láctea que corre o céu, a correnteza no interior
de sua cabeça transbordava, incessante, não se sabe vinda de onde.
Receando aquela força terrível, ele se inquietou com a possibilidade de
seu corpo físico não suportar tal jorro. E, segurando firmemente o
pincel, apelou a si mesmo vezes sem conta:

--- Continue escrevendo até o limite de suas forças. O que você está
escrevendo neste momento pode ser que, se não for agora, nunca mais
possa ser escrito.

Mas a corrente, semelhante a uma neblina luminosa, não diminuía nem um
pouco sua velocidade. Pelo contrário, vinha"-o assaltando,
transbordante, inundando tudo, na vertigem da voragem. Ele se tornou,
afinal, presa sua. E, esquecendo"-se de tudo, fez galopar o pincel com o
ímpeto de uma tempestade, em direção a essa correnteza.

O que se refletia naquele momento em seu olhar soberano não era
interesse, nem amor, nem ódio. Muito menos era um espírito suscetível à
opinião pública, que já se apagara havia muito do fundo de seus olhos.
O que existia era, somente, uma incompreensível alegria. Ou, então, uma
emoção trágica, que o levava até o êxtase. Sem conhecer aquela emoção,
como poderiam alcançar o espírito da devoção à literatura popular? Como
poderiam compreender a alma digna de um escritor da literatura popular?
Era ali, sim, que a ``Vida'' resplandecia, magnífica, aos olhos do autor,
exatamente como um minério virgem, lavadas de todas as suas
impurezas\ldots{}

Enquanto isso, Ohyaku e sua nora Omichi, uma frente à outra, continuavam
a costura em volta da lamparina da sala de jantar. Tarô já devia estar
dormindo. Um pouco além, Sôhaku, que parecia doente, ocupava"-se havia
algum tempo em preparar seus comprimidos.

Pouco depois, Ohyaku, enquanto passava a agulha no cabelo para
lubrificá"-la, resmungou:

--- Papai ainda não vai dormir?

--- Deve estar completamente absorto no que está escrevendo, como sempre 
--- respondeu Omichi, sem desviar os olhos da agulha.

--- Que obstinado! Ainda se desse muito dinheiro\ldots{}

Assim dizendo, Ohyaku olhou para o filho e para a nora. 

Sôhaku não respondeu, fingindo não ter escutado. Calada, Omichi também
continuava trabalhando a agulha. Também ali, como no escritório,
invariavelmente, os grilos cantavam o outono até a exaustão.

\begin{flushright}
\textit{Novembro de 1917}\\  
\end{flushright}


\chapter{O baile}

\sectionitem

\noindent\textsc{Foi na noite} de três de novembro do décimo nono ano da era Meiji (1886).
Tendo então dezessete anos de idade, Akiko, a jovem senhorita de uma
família eminente, subia a escadaria do Rokumeikan, onde naquela noite
se oferecia um baile, em companhia de seu pai, um senhor calvo. De cada
lado da ampla escadaria inundada pela luz brilhante das lâmpadas a gás,
enormes crisântemos, que quase pareciam artificiais, compunham um
arranjo em três camadas. As flores misturavam em profusão tufos de
pétalas carmesins na parte posterior, amarelo"-ouro no centro e branco
puríssimo no primeiro nível. E, do alto da escadaria, diante do salão
de baile onde terminavam os arranjos de flores, os sons festivos da
orquestra desaguavam incessantemente, como os suspiros de uma
felicidade demasiado grande, de uma incontida felicidade. Akiko já se
adiantara no estudo do francês e da dança. Mas aquela noite, pela
primeira vez na vida, iria participar de um baile oficial. Daí por que,
dentro da carruagem, respondesse apenas por meio de evasivas a seu pai
quando, de tempos em tempos, ele lhe dirigia a palavra. Tal era o
nervosismo que se enraizava em sua alma, que se expressava em vaga
inquietude, mas também em deliciosa ansiedade. Quando a carruagem
finalmente se deteve em frente ao Rokumeikan, ela nem lembrava mais
quantas vezes havia lançado o olhar nervoso em direção às luzes
escassas da cidade de Tóquio, que iam desaparecendo pela janela.
Entretanto, ao entrar no Rokumeikan, ela se confrontou com um incidente
que a fez esquecer"-se de toda a inquietação. Isso se deu quando ela e o
pai já estavam no meio da escadaria e alcançavam um alto funcionário
chinês que subia alguns passos à sua frente. Naquele momento, quando
ele desviou o corpo forte e obeso para deixá"-los passar, lançou sobre
Akiko um olhar perplexo. Seu singelo vestido de baile rosa"-claro, a
fita azul"-clara elegantemente atada em torno do pescoço e, em seus
cabelos escuros, uma rosa de suave perfume --- de fato, a silhueta de
Akiko naquela noite devia desconcertar os olhos do alto funcionário
chinês de rabicho comprido, pois ela encarnava a beleza das moças de um
Japão que se abria para o Ocidente. Quase ao mesmo tempo, um jovem
japonês de fraque, que descia a escadaria com um passo precipitado,
cruzou com eles no meio do caminho e, virando"-se levemente numa ação
reflexa, derramou seu olhar também perplexo sobre a silhueta de Akiko.
Depois, como se uma ideia lhe ocorresse subitamente ao espírito,
verificou sua gravata branca e retomou seu ritmo apressado, descendo
por entre os crisântemos até o saguão de entrada. Quando pai e filha
terminaram de subir a escadaria, na entrada do salão de baile do
primeiro andar, depararam com o anfitrião --- um conde com magníficas
suíças grisalhas e que portava no peito inúmeras condecorações --- 
recebendo os convidados com grande pompa, em companhia de sua esposa,
mais velha que ele, que se esmerava num vestido à moda Luís \textsc{xv}. Akiko
não deixou de perceber a ingênua admiração que por um momento tingiu o
rosto astuto do conde quando ele notou sua silhueta. Sorrindo
alegremente, seu afável pai a apresentou com breves palavras ao conde e
à condessa. Akiko se achava dividida entre o pudor e o orgulho, mas
assim mesmo pôde perceber nos traços altivos da condessa como que uma
ponta de vulgaridade. Também o salão de baile estava esplendidamente
decorado com uma profusão de crisântemos. As rendas, as flores e os
leques de marfim das damas, que aguardavam os cavalheiros que viriam
tirá"-las para dançar, se moviam em vagas silenciosas em meio ao aroma
fresco dos perfumes. Afastando"-se logo de seu pai, Akiko se juntou a um
grupo dessas damas. Eram moças aparentemente da mesma idade, todas
elegantes em seus vestidos de baile, rosa ou azul"-claro. Recebendo"-a em
seu grupo, elas se puseram a gorjear como passarinhos, elogiando em
uníssono a beleza ofuscante que ela transmitia naquela noite. Mal Akiko
se juntara a suas colegas, já um desconhecido --- um oficial da Marinha
Francesa --- se aproximou dela com bastante calma. Os braços ao longo do
corpo, ele polidamente se inclinou à maneira japonesa. Akiko sentiu um
rubor imperceptível subir"-lhe à face. No entanto, aquele cumprimento
fora eloquente o bastante. Ela se voltou então para a moça do lado, de
vestido de baile azul"-claro, para pedir que lhe segurasse o leque. Mas
qual não foi sua surpresa ao ouvir o oficial da Marinha lhe dizer com
voz clara, um fino sorriso nos lábios, num japonês colorido por um
estranho sotaque:

--- A senhorita poderia conceder"-me a honra desta dança?

Um instante depois, ela valsava nos braços do oficial da Marinha
Francesa aos acordes do \textit{Danúbio azul. }Seu par tinha o rosto
bronzeado, os olhos e o nariz bem desenhados, e portava um farto
bigode. A moça era pequena demais para pousar a mão coberta por uma
longa luva no ombro esquerdo do uniforme militar. Mas, experiente, ele
a guiava com firmeza, fazendo"-a virar"-se ligeiramente em meio à
multidão. De tempos em tempos, o oficial murmurava em seu ouvido
cumprimentos em francês.

Sempre respondendo a essas palavras amáveis com um tímido sorriso, a
jovem lançava olhares ao redor do salão de baile. Sob o cortinado de
crepe roxo estampado com o brasão imperial, e sob a bandeira da China,
na qual se retorciam dragões azuis de longas garras, os crisântemos
faziam luzir, entre as vagas de dançarinos, ora um prateado leve, ora
um dourado sombrio. E a multidão, enlevada pelas ondas melódicas,
emitidas pela esfuziante orquestra alemã, que jorravam como champanhe,
não se furtava um único instante ao turbilhão vertiginoso da valsa.
Quando os olhos de Akiko cruzavam os de uma amiga que também dançava,
as duas trocavam um alegre sinal de cabeça em meio à agitação febril
que tomava conta do salão. Mas, já no momento seguinte, outra moça
dançando irrompia bruscamente em seu lugar, como uma enorme mariposa
enlouquecida.

Akiko tinha plena consciência de que o oficial da Marinha Francesa não	%\EP[-1]
perdia, por um segundo sequer, o menor de seus gestos. Aquele gesto
constituía um sinal evidente do interesse que seu gracioso modo de
dançar despertava naquele estrangeiro, que ainda não se familiarizara
com o Japão. ``Esta linda senhorita, será que também mora como uma
boneca, numa casa de papel e bambu? Será que come arroz pegando os
grãos com palitos finos de metal, numa tigela do tamanho da palma da
mão, decorada de flores azuis?'' Em seus olhos, que lhe sorriam
gentilmente, parecia"-lhe aflorar seguidas vezes esse tipo de pergunta.
Ela se divertia mas também se sentia lisonjeada. Assim, cada vez que
seu parceiro dirigia para o chão o olhar curioso, seus delicados
sapatos de baile cor"-de"-rosa se punham a deslizar sobre o assoalho liso
ainda com mais leveza. Mas, sem dúvida adivinhando a fadiga daquela
jovem, graciosa como um gatinho, o oficial se inclinou para ela,
solícito, e perguntou:

--- A senhorita deseja ainda dançar?

--- \textit{Non, merci} --- arfando, Akiko então recusou com toda a
franqueza. Sem interromper o ritmo da dança, ele a acompanhou com
elegância até os buquês de crisântemos arranjados ao longo da parede,
atravessando vagas de rendas e flores. Depois de um último passo de
valsa, ele graciosamente a fez sentar"-se numa cadeira e, inflando uma
vez o peito sob o uniforme, de novo se inclinou diante dela, segundo a
cortesia japonesa.

Depois de dançar polcas e mazurcas, Akiko desceu a escadaria ao braço do
oficial da Marinha Francesa, entre os arranjos em três camadas de
crisântemos brancos, amarelos e carmesins, em direção à ampla sala do
pavimento térreo. Lá, no contínuo vaivém dos fraques e dos ombros		%\EP[-1]
brancos, algumas mesas cobertas de prataria e cristais suportavam
brancas montanhas de carne e cogumelos, outras concentravam torres de
sanduíches e sorvetes, outras ainda sustinham pirâmides de romãs e
figos. Uma das paredes do salão que não fora decorada com crisântemos
estava adornada com uma grade dourada na qual se entrelaçavam parreiras
verdes artificiais que pareciam de verdade. E, entre as folhas de
videira, copiosos cachos de uva, semelhantes a colmeias, pendiam
rubramente. Akiko deparou com a cabeça calva de seu pai, que fumava um
charuto diante da grade dourada, em companhia de um cavalheiro de mesma
idade. Ele lhe dirigiu um ligeiro aceno de satisfação, mas logo,
virando"-se para seu interlocutor, voltou a tirar baforadas de seu
charuto, sem mais se importar com ela.

O oficial da Marinha Francesa e Akiko se aproximaram de uma mesa e se
serviram de uma taça de sorvete. A jovem sentia de vez em quando o
olhar de seu cavalheiro pousar"-lhe sobre as mãos, os cabelos, o pescoço
envolto pela fita azul. Para ela, nada havia de desagradável naqueles
gestos. Em alguns momentos, entretanto, uma desconfiança bem feminina a
assaltava. Por isso, para insinuar aquela desconfiança, quando passaram
duas jovens, que pareciam alemãs, vestidas de veludo negro realçado por
camélias vermelhas, teve a ideia de mostrar sua admiração:

--- Como são lindas as europeias!

Ao ouvir aquelas palavras, o oficial da Marinha se opôs com inesperada
seriedade, meneando a cabeça:

--- As damas japonesas também são lindas. E a senhorita, em especial\ldots{}

--- Quem me dera!

--- Não, não estou querendo somente lisonjeá"-la. A senhorita poderia
perfeitamente participar de um baile parisiense. Todos ficariam
embevecidos! Pois a senhorita se assemelha a uma princesinha dos quadros de Watteau!

Akiko não conhecia Watteau. Por isso, a magnífica visão do passado, a
imagem de rosas lânguidas e fontes na penumbra das florestas, evocada
nas palavras do oficial francês, perdeu"-se num instante, sem deixar
vestígios. Mas, enquanto movia sua colher, Akiko, que tinha a
sensibilidade bem aguçada, não perdeu a ocasião de atacar o único tema
de conversação ainda disponível:

--- Eu também adoraria participar de um baile parisiense!

--- Não vale a pena. Os bailes parisienses são exatamente como este.

Enquanto falava, o oficial da Marinha deixou vagar o olhar sobre a
multidão e os crisântemos que os cercavam; com um sorriso irônico a lhe
assomar ao fundo das pupilas, largou subitamente a taça, acrescentando,
como que para si mesmo:

--- Não somente os de Paris. Os bailes são sempre iguais, em todos os
lugares.

Uma hora mais tarde, Akiko se achava, sempre de braço dado ao oficial da 
Marinha Francesa, no meio de uma multidão de japoneses e estrangeiros
que haviam ido ao terraço contíguo ao salão de baile contemplar a noite
estrelada.

Do outro lado da balaustrada, as coníferas se comprimiam no parque
espaçoso, entrelaçando silenciosamente seus galhos e deixando entrever
em suas copas pequenos pontos da luz das lanternas vermelhas. Além
disso, na profundeza do ar gelado, um aroma de musgos e de folhas secas
que subia do jardim fazia pairar uma leve atmosfera melancólica de
outono. Mas logo atrás deles, no salão de baile, as vagas de rendas e
flores continuavam seu tumulto incessante sob os cortinados de crepe
roxo decorados de crisântemos de dezesseis pétalas. E também o remoinho
da orquestra, com sua tonalidade aguda, açoitava sem parar o mar
humano. Também no terraço, vozes e risos alegres faziam fremir o ar
noturno. E mais, quando um magnífico fogo de artifício foi lançado no
céu escuro das coníferas, um som surdo reverberou em uníssono na
multidão. Em meio àquelas pessoas todas, Akiko tagarelava despreocupada
com as amigas que encontrava. Mas não demorou a notar que o oficial da
Marinha Francesa, que ainda lhe dava o braço, contemplava em silêncio o
céu estrelado sobre o jardim. Pareceu à moça que talvez ele sentisse
certa nostalgia de sua pátria. Dirigindo"-lhe um olhar furtivo, ela
perguntou, com certa meiguice:

--- O senhor está pensando em seu país, não é?

O oficial da Marinha se voltou lentamente para ela com os olhos sempre
sorridentes. Depois, em vez de responder ``non'', sacudiu a cabeça como
um garoto.

--- Mas o senhor estava pensando em algo, não?

--- Tente adivinhar no quê.

Naquele instante, um som agitado como o do vento começou a vibrar por 
entre as pessoas reunidas no terraço. Interrompendo a conversação,
Akiko e o oficial ergueram o olhar para a noite que pesava sobre as
coníferas do parque. Os fogos de artifício vermelhos e azuis já estavam
a ponto de se apagar, salpicando suas luzes excêntricas na escuridão.
Akiko não sabia por que, mas aqueles fogos de artifício eram tão belos
que chegavam a entristecê"-la.



--- Eu pensava nos fogos de artifício, nos fogos de artifício que se
assemelham tanto à nossa vida --- disse, finalmente, o oficial da Marinha
Francesa, num tom algo professoral, inclinando com muita doçura os
olhos sobre o rosto de Akiko.

\sectionitem

Foi no outono do sétimo ano da era Taishô (1918). Quando se dirigia à
sua casa de veraneio em Kamakura, Akiko por acaso encontrou no trem um
jovem escritor seu conhecido. O rapaz colocou no bagageiro um buquê de
crisântemos destinados a uma amiga. Akiko --- agora a idosa senhora H.~---
contou"-lhe então que os crisântemos sempre lhe evocavam certa noite e
lhe narrou em pormenores o baile do Rokumeikan. 

O moço não pôde deixar de se impressionar com essa história contada 
pela própria pessoa que a vivera. Terminada a narrativa, o jovem 
perguntou à idosa senhora H., como que casualmente:

--- A senhora não saberia o nome desse oficial da Marinha Francesa?

Ao que a senhora H.~respondeu inesperadamente:

--- Mas claro. Ele se chamava Julien Viaud.\footnote{ Julien Viaud
(1850--1923), oficial da Marinha Francesa que escreveu vários romances
sob o nome de Pierre Loti, entre eles \textit{Madame Crisântemo} (na
década de 1890), nos quais as impressões de viagem e o exotismo dos
países distantes ocupavam lugar de destaque.}

--- Oh! Trata"-se então de Loti, Pierre Loti, o autor de \textit{Madame
Crisântemo}! 

%\vspace{1em}\pagebreak

O jovem sentiu uma agradável excitação. Mas, fitando nele
um olhar de completo estranhamento, a idosa senhora H.~não cessava de
murmurar entre dentes:

--- Não, não, ele não se chamava Loti. Já lhe disse que seu nome era
Viaud, Julien Viaud!

\begin{flushright}
\textit{Dezembro de 1919}\\  
\end{flushright}


\chapter[Passagens do caderno de notas de Yasukichi]{Passagens do caderno de notas de Yasukichi}

\section*{au!}

\noindent\textsc{Certo fim} de tarde de inverno, no primeiro andar de um restaurante um
tanto sujo, Yasukichi mastigava um pão tostado que recendia a óleo.
Diante de sua mesa, via uma parede branca com várias rachaduras. Nela,
haviam colado uma folha estreita e comprida de papel onde estava
escrito em diagonal: ``Temos também sanduíches \textit{hot} (quentes)''.
(Tomando ``hot'' por uma palavra japonesa e interpretando: ``Que alívio!
Sanduíches quentes!'', um de seus colegas ficou seriamente perplexo.) À
sua esquerda havia uma escada que conduzia ao térreo e, logo à direita,
uma janela de vidro. De tempos em tempos lançava um olhar vago pela
janela, enquanto mastigava o pão torrado. Lá fora, do outro lado da
rua, numa loja de roupas usadas com telhado de zinco, estavam
penduradas roupas azuis para trabalhadores, bem como casacos de cor
cáqui. Naquela noite, o grupo de inglês deveria reunir"-se na escola a
partir das seis e meia. Ele havia se comprometido a comparecer,
mas, como não residia na cidade, viu"-se obrigado, muito a
contragosto, a ficar esperando num lugar daqueles, terminadas as aulas,
até as seis e meia. Creio que entre os poemas de Aika Toki --- 
desculpem"-me se me engano --- lê"-se um que diz:

\begin{verse}
Vindo de tão longe,\\
Devo ainda engolir este bife de merda.\\
Mulher, mulher, que falta tu me fazes!
\end{verse}

Toda vez que ele ia àquele lugar, pensava naquele poema. É verdade que a
mulher pela qual ele deveria suspirar ainda não existia. Mas só de
olhar as lojas de roupas usadas, de mastigar seu pão recendendo a óleo
e de ver ``sanduíches quentes'', aquelas palavras ``Mulher,
mulher, que falta tu me fazes!'' lhe vinham espontaneamente aos lábios.
Nesse meio"-tempo, Yasukichi reparou, bem atrás dele, em dois jovens
oficiais da Marinha que tomavam cerveja. Reconheceu um deles, o oficial
contador da escola na qual também ele ensinava. Não tendo muito contato
com os oficiais, não sabia sequer o nome daquele. Aliás, não era
somente o nome, pois nem mesmo sabia se ele ensinava no primeiro ou no
segundo nível. Tudo o que sabia era que, para receber seu salário no
fim de cada mês, tinha de passar por aquele homem. O outro lhe era
completamente desconhecido. Cada vez que pediam mais uma cerveja,
lançavam palavras como ``psiu!'' e ``ei!''. Mesmo assim, a servente não se
ofendia e, subindo e descendo as escadas com diligência, carregava os
copos com ambas as mãos. E, no entanto, ela não vinha à mesa de
Yasukichi para lhe servir a xícara de chá que pedira. Os dois oficiais
conversavam ruidosamente enquanto bebiam cerveja. Yasukichi não
prestava, naturalmente, nenhuma atenção particular à sua conversa. Mas
sua atenção foi subitamente despertada, ao ouvir um deles ordenar:
``Diga au!''. Ele não gostava de cães. Sentia muito prazer em pensar que,
entre os escritores que não gostavam de cães, se contavam Goethe e
Strindberg. Por isso, quando aquelas palavras ressoaram em seus
ouvidos, imaginou que enorme cão europeu poderia ser criado em
semelhante lugar. Ao mesmo tempo, teve a horrível sensação de que a
besta estava lá, vagando às suas costas.

Ele se virou com cuidado. Mas, felizmente, nada se via por ali que se
assemelhasse a um cão. Havia apenas o oficial responsável, que, olhando
pela janela, sorria ironicamente. Yasukichi supôs que o animal talvez
se encontrasse sob a janela. Sentiu, no entanto, uma sensação estranha.
Naquele instante, o oficial responsável repetiu:

--- Diga au! Vamos, diga au!

Com uma ligeira torção do corpo, Yasukichi espiou do outro lado, sob a
janela. A primeira coisa que viu foi um luminoso ainda apagado de uma
loja qualquer, que fazia a propaganda de alguma"-coisa"-Masamune. Depois,
um toldo enrolado. Depois, pedaços de couro para tamancos postos a
secar e esquecidos sobre os barris de cerveja vazios que serviam para
recolher a água da chuva. Depois, as poças d'água das calçadas.
Depois\ldots{} --- enfim, pouco importa, mas em parte alguma ele viu a sombra
de um cão. A silhueta de um mendigo de doze ou treze anos que parecia
sentir frio, de pé, os olhos levantados para a janela do primeiro andar
era vagamente visível.

--- Diga au! Não vai dizer au? --- lançou de novo o oficial responsável.
Aquelas palavras pareciam deter algum poder sobre o espírito do
mendigo. Sem despregar os olhos da janela, avançou um ou dois passos em
sua direção, como um sonâmbulo. Yasukichi entendeu então o cruel jogo
do maligno oficial contador. Jogo cruel? Ou talvez não o fosse. Talvez
não passasse de uma experiência. Seria, quem sabe, apenas uma
experiência com o objetivo de determinar até que ponto um ser esfomeado
poderia sacrificar sua dignidade. De seu ponto de vista, não valia a
pena tornar a aferir essa questão. Esaú havia renunciado a seu direito
de primogenitura por alguns pedaços de carne assada; ele mesmo,
Yasukichi, havia se tornado professor por algumas fatias de pão. Esses
fatos eram mais do que suficientes. Mas exemplos tão fúteis
provavelmente não satisfariam a curiosidade científica dos grandes
especialistas da psicologia experimental. Se assim fosse, seria como
havia ensinado naquele dia mesmo a seus alunos: \textit{De gustibus 
non est disputandum}. Cada um tinha o direito de gostar ou
não do que fosse. Se alguém quisesse fazer uma experiência, ora, que a
fizesse!\ldots{} Ruminando aqueles pensamentos, Yasukichi observava o
mendigo de pé sob a janela.

O oficial contador aguardou em silêncio alguns instantes. O mendigo se
pôs a olhar inquieto a seu redor. Mesmo que não visse inconveniente em
imitar um cão, evidentemente receava que alguém o visse. Mas, antes que
seus olhos se tranquilizassem, o oficial contador colocou o rosto
vermelho janela afora, brandindo alguma coisa na mão.

--- Diga au! Se você disser au, veja o que lhe dou.

O rosto do mendigo pareceu se iluminar por um instante, de cobiça. Os
mendigos às vezes inspiravam em Yasukichi um interesse romântico. Mas
nem uma só vez ele sentira algo semelhante a piedade ou comiseração.
Acreditava que, se alguém dissesse sentir tais coisas, esse alguém não
passaria de um tolo, ou de um mentiroso. No entanto, ao ver naquela
hora aquele pequeno mendigo com a cabeça ligeiramente inclinada para
trás e os olhos brilhando, sentiu um pouquinho de piedade. Certo, mas
aquilo não passava realmente da medida de ``um pouquinho''. Longe de se
achar piedoso propriamente falando, ele apreciava certo efeito à
Rembrandt na figura daquele rapazola.

--- Não vai dizer? Ei, é só dizer au!

O mendigo fez uma espécie de careta.

--- Au!

Sua voz era quase imperceptível.

--- Mais forte!

--- Au! Au!

O mendigo se decidiu a ladrar duas vezes. Uma laranja foi então atirada
para fora da janela --- nem há necessidade de escrever o que se passou a
seguir. O mendigo evidentemente se jogou sobre a laranja e o oficial
contador evidentemente caiu numa gargalhada. Por volta de uma semana
mais tarde, no dia de pagamento, Yasukichi foi ao serviço de
contabilidade para receber seu salário. O ar atarefado, o mesmo oficial
contador ora abria um registro, ora consultava um documento. Ao ver
Yasukichi, perguntou"-lhe, secamente:

--- É por seus vencimentos, não?

--- Isso mesmo --- respondeu ele brevemente.

Mas, talvez por estar muito atarefado, o oficial responsável não se
apressava o mínimo a lhe pagar o ordenado. Não somente isso, mas por
fim, virando"-lhe as costas de seu uniforme, ficou indefinidamente
inclinado sobre seu ábaco.

--- Senhor oficial contador?

Após esperar algum tempo, Yasukichi chamou"-o com um tom de súplica. O
oficial lhe lançou um olhar por trás dos ombros. ``Sim, um instantinho'';
essas palavras se liam já claramente sobre seus lábios. Mas, sem lhe
deixar tempo de responder, Yasukichi encadeou uma resposta que já havia
longamente meditado:

--- Senhor oficial contador, quem sabe gostaria que eu dissesse au? Não é,
senhor oficial contador?

A se crer em Yasukichi, sua voz, naquele instante, estava mais doce que
a de um anjo.

\section*{dois ocidentais}

Havia dois ocidentais naquela escola que tinham vindo ensinar
conversação e redação em inglês. Um deles era um inglês de nome
Townsend; o outro, um americano chamado Starlet. O senhor Townsend era
um velhinho gentil e calvo, que falava o japonês com perfeição. De modo
geral, os professores ocidentais não paravam de discorrer ora sobre
Shakespeare, ora sobre Goethe, por fútil que fosse o assunto. Mas,
felizmente, o senhor Townsend não fingia compreender uma mínima palavra
de literatura. Um dia, quando falavam de Wordsworth, ele havia
declarado: ``Não compreendo estritamente nada de poesia. Pergunto"-me
sempre o que é que acham de tão interessante em Wordsworth!''.

Como residissem na mesma cidade à beira"-mar, Yasukichi e o senhor
Townsend se encontravam duas vezes por dia no mesmo trem. O trajeto
durava cerca de trinta minutos. Um cachimbo de Glasgow entre os dentes,
ambos passavam o tempo no trem discutindo ora tabaco, ora a escola, ora
fantasmas. Porque, se Hamlet deixava o senhor Townsend perfeitamente
frio, em compensação, o fantasma de seu pai interessava o teósofo que
ele era. Mas, quando abordavam o tema das \textit{ocult sciences} ---
fosse magia, fosse alquimia ---, ele nunca deixava de dizer, balançando a
testa e o cachimbo com um mesmo ar triste: ``As portas do Mistério não
são assim tão difíceis de abrir como pensam as pessoas comuns. Seu
caráter assustador consiste em que elas não tornam a se fechar
facilmente. É melhor não tocar nesse tipo de coisas!''.

O outro ocidental, o senhor Starlet, era um homem bem mais moço, que se
preocupava muito com sua elegância. No inverno, podíamos, por exemplo,
vê"-lo trajando um sobretudo verde"-escuro, com um cachecol vermelho em
volta do pescoço. Contrariamente ao senhor Townsend, parecia que às
vezes dava uma espiada nas últimas publicações. Por ocasião de um
seminário de inglês organizado na escola, havia feito uma conferência
magistral sobre o tema \textit{Os escritores americanos dos últimos anos.} 
A crer"-se em sua exposição, os escritores americanos dos
últimos anos seriam representados por Robert Louis Stevenson e O.
Henry! O senhor Starlet não vivia na mesma cidade, mas, como sua casa
também fosse servida pela mesma linha da estrada de ferro, acontecia às
vezes de fazerem, juntos, o trajeto de trem. Yasukichi quase não se
lembrava das conversas que mantinha com ele. A única de que recordava
havia se passado num dia em que eles esperavam pelo trem, instalados
diante do aquecedor da sala de espera. Yasukichi havia começado a
falar, reprimindo um bocejo, do tédio inerente à profissão de
professor. Nisso, o senhor Starlet, um homem másculo e belo que usava
óculos sem armação, disse, com uma expressão um tanto estranha no
rosto:

--- Ensinar não é uma profissão. Creio que seria mais justo dizer que é
uma vocação. \textit{You know, Socrates and Plato are two great
teachers\ldots{}}\footnote{ ``Como você sabe, Sócrates e Platão são dois
grandes professores\ldots{}'', em inglês, no original.} 
Yasukichi não via inconveniente algum no fato de Robert Louis Stevenson
ser ianque. Mas ouvir dizer que Sócrates e Platão eram professores!\ldots{}\
A partir desse dia, Yasukichi limitou"-se a manifestar ao senhor Starlet
uma polida amizade.

\section*{pausa de meio"-dia (um devaneio)}

Yasukichi saiu do restaurante do primeiro andar. Depois do almoço, a
maior parte dos professores civis se transferia para a sala de fumar
vizinha. Naquele dia, ele preferiu descer a escada que levava ao
jardim. Nisso, chocou"-se com um oficial subalterno que subia os degraus
galgando"-os três a três, como um gafanhoto. Quando se deparou com
Yasukichi, bateu prontamente uma continência, levando uma mão à testa,
numa saudação rígida. E rapidamente desapareceu no alto da escada.
Enquanto retribuía com uma ligeira inclinação de cabeça que caiu no
vazio, Yasukichi continuou despreocupadamente a descer a escada.

No meio das paulônias e nogueiras, as magnólias estavam em flor. Por não
se sabe que obscura razão, o pé de magnólias não voltava suas flores
enfim desabrochadas para o sul ensolarado. Mas os pés de pepinos, tão
parecidos com os de magnólias, viravam suas flores para aquela direção.
Acendendo um cigarro, Yasukichi abençoou a originalidade das magnólias.
Uma pastorinha desceu dançando, como se fosse uma pedra a tombar do
céu. Aquela avezinha já não lhe era uma estranha. Ela abanava sua
pequena cauda, como um sinal de que o guiaria.

--- Por aqui! Por aqui! Não é por aí, não! Por aqui! Por aqui!

Guiado pela pastorinha, foi seguindo pelas trilhas cobertas de
cascalhos. Mas o que lhe teria passado na cabeça? O pássaro subiu
ondulando de novo para o céu. Em compensação, um mecânico de alto talhe
veio caminhando em sua direção. Yasukichi teve a impressão de já ter
visto seu rosto em algum lugar. Após a continência de lei, o soldado o
ultrapassou com um passo rápido. Continuando a fumar o cigarro, ele se
perguntava quem seria aquele homem. Dois passos, três, cinco\ldots{} No
décimo passo, Yasukichi descobriu. Mas aquele homem era Paul Gauguin!
Ou pelo menos sua reencarnação. Estava certo de que, em vez de uma pá,
ele deveria ter na mão um pincel. E, no fim de tudo, um amigo louco lhe
daria um tiro de revólver pelas costas. Era uma lástima, mas o que fazer?

De trilha em trilha, Yasukichi finalmente chegou à praça fronteira ao
portão de entrada. Ali, dois canhões --- troféus de guerra --- estavam
alinhados em meio a pinheiros e bambus anões. Aproximando a orelha de
um dos canos, ouviu um som como que de respiração. Bem podia ser que os
canhões também bocejassem. Ele se sentou à sombra dos canhões. Depois
acendeu um segundo cigarro. Uma lagartixa brilhava sobre o cascalho da
praça onde os carros trafegavam. Se um ser humano perdesse uma perna,
ela nunca se refaria. Quanto à lagartixa, não, bem lhe poderíamos
cortar a cauda: ela logo geraria uma nova em seu lugar. Com o cigarro
na boca, Yasukichi pensou que sem dúvida as lagartixas são mais
lamarckianas do que o próprio Lamarck. Mas, passado um tempo a
contemplá"-la, viu a lagartixa acabar subitamente se transformando numa
mancha de óleo sobre o cascalho. Yasukichi se levantou, afinal.
Atravessando o jardim em sentido inverso, ao longo dos prédios pintados
da escola, saiu na quadra de esportes em frente ao mar. Na quadra de
tênis de terra vermelha, alguns professores militares disputavam uma
partida com animação. Sobre a quadra, um ruído seco e regular claqueava
sem parar. Ao mesmo tempo, linhas retas e brancas eram jogadas à
direita e à esquerda da rede. Não, não eram bolas que voavam. Era a
espuma de garrafas de champanhe invisíveis a olho nu. Era o champanhe
que deuses em camisa branca bebiam, deliciados. Exprimindo a maior
veneração pelos deuses, Yasukichi se dirigiu então ao jardim situado atrás.

Naquele jardim havia um grande número de roseiras. No entanto, ainda não
se via nenhuma flor. Enquanto Yasukichi perambulava por ele, descobriu
uma lagarta sobre o galho de uma roseira que crescia na direção da
trilha. Mal a viu, encontrou mais uma, que se arrastava sobre uma folha
vizinha. As lagartas pareciam estar falando dele, ou de outra coisa,
trocando sinais de assentimento mútuo. Yasukichi decidiu permanecer lá
silenciosamente, para escutar o que elas diziam.

A primeira lagarta:

--- Quando será que este professor vai se transformar em borboleta? 
Desde a época do meu tetra"-tetravô que ele somente rasteja por cima da terra.

A segunda lagarta:

--- Pode ser que os seres humanos não se transformem jamais em borboletas.

A primeira lagarta:

--- Não, transformar parece que eles se transformam. Veja a prova, eis
justamente um que voa lá embaixo.

A segunda lagarta:

--- Mas, sim, é verdade, é um que voa. Não é por nada, mas ele é realmente
horroroso! É, francamente! Dá para ver que os humanos são desprovidos
de qualquer consciência estética!

Usando a mão como viseira, Yasukichi elevou os olhos no rumo do avião
que passava por sobre sua cabeça.

Naquele momento veio a seu encontro, com um ar algo divertido, um
demônio, transformado em um de seus colegas. O demônio, que outrora
lecionava alquimia, atualmente ensinava química aplicada a seus alunos.
Interpelou Yasukichi com um sorriso zombeteiro:

--- E, então, esta noite você não me faz companhia?

No sorriso do demônio, Yasukichi podia perfeitamente sentir duas linhas
de \textit{Fausto}: ``Todas as teorias são de cor cinza como as ruínas;
somente é verde a árvore da vida que produz a espécie dourada''. Depois
de deixar o demônio, ele penetrou no interior do prédio da escola.
Todas as salas de aula estavam desertas. Um lance de olhos sobre o
quadro"-negro de uma delas lhe permitiu descobrir somente uma figura
geométrica que alguém havia se esquecido de apagar. Quando a figura
geométrica sentiu que a tinham visto, prontamente pensou que seria
apagada. Num átimo, enquanto se encolhia e se estendia, disse:

--- Precisarão de mim na próxima aula!

Yasukichi subiu a escada que antes havia descido e entrou na sala
reservada aos professores de língua e matemática. Na sala não havia
ninguém, afora o calvo senhor Townsend. No entanto, para afugentar seu
tédio, o idoso professor, assoviando sem parar, ensaiava sozinho alguns
passos de dança. Yasukichi se dirigiu à pia para lavar as mãos sem
poder reprimir um ligeiro sorriso. Nessa hora, ao olhar para o espelho,
ficou atônito: o senhor Townsend se metamorfoseara num adolescente belo
e gracioso e ele, Yasukichi, num velho de cabelos brancos, com as
costas arqueadas.\ \\

\section*{a vergonha}

Yasukichi nunca vinha dar sua aula sem a ter previamente preparado. Para
dizer a verdade, isso não acontecia em função do salário que recebia ou
devido a qualquer obrigação moral que o impedisse de cometer gestos
irresponsáveis. Mas, em razão do caráter particular da escola, os
livros didáticos estavam cheios de termos técnicos da Marinha. Se ele
não estudasse antes os significados das palavras, correria o risco de
praticar as traduções as mais extravagantes. Por exemplo, a expressão
``\textit{cat's paw}'', que se poderia pensar tratar"-se de uma ``pata de
gato'', era uma espécie de vento: uma brisa leve!

Um dia, ele estava ensinando aos alunos do segundo ano um pequeno texto
cujo título não recordava mais e que tratava justamente de navegação.
Era tão mal redigido que até assustava. O vento fazia bramir as velas,
as ondas invadiam as escotilhas, mas não aparecia no texto nenhum sinal
das palavras ``ondas'' ou ``vento''. Enquanto os fazia ler e traduzir, ele
próprio começou a se entediar. Principalmente naquelas horas, tomava"-o
o desejo de discutir com os alunos problemas filosóficos ou temas da
atualidade. Por definição, os professores sempre têm vontade de ensinar
outras coisas além das matérias do programa escolar. Ética, gostos
pessoais, concepções de vida\ldots{} Pouco importa o quê. Numa palavra, mais
que livros didáticos ou quadros"-negros, o que os professores aspiram
ensinar são coisas que lhes digam respeito intimamente. Mas é uma pena
que os alunos não tenham outro desejo que o de aprender apenas o que os
livros didáticos contêm. Aliás, não é bem que apenas lhes falte esse
desejo; na verdade, o que sentem é um santo horror de aprender aquilo
que está fora dos manuais. Como Yasukichi bem o sabia, o único recurso
que lhe restava naquela ocasião era o de mandá"-los ler e traduzir,
enquanto aguardava, entediado.

No entanto, mesmo quando ele não se entediava, achava fastidioso ter de
prestar atenção à tradução dos alunos e lhes fazer retificações sutis.
Embora a aula durasse uma hora, interrompeu o exercício passados uns
trinta minutos e passou ele mesmo a ler e traduzir frase por frase. A
navegação tal como a descreviam no manual era"-lhe mortalmente tediosa,
como sempre. Assim, não havia dúvida de que seu modo de dar aula também
deveria ser mortalmente tedioso! Como um veleiro atravessando uma área
de calmaria, ele avançava arduamente, ora misturando os tempos verbais,
ora se enganando nos pronomes relativos.

Assim fazendo, de repente percebeu que a parte que havia preparado
terminava três ou quatro linhas mais adiante. Para além daquele limite
se estendia um mar bravio, onde todo o cuidado era pouco, semeado das
inomináveis rochas que eram os termos técnicos de navegação. Olhou para
o relógio com o canto dos olhos. Ainda restavam uns bons vinte minutos
até que o sino tocasse para o intervalo. Ele traduziu o mais
cuidadosamente que pôde as quatro ou cinco linhas que preparara. Mas,
até terminar o trecho, o ponteiro do relógio não avançaria mais que
três minutos.

Sentiu"-se em pânico. Naquela situação, o único recurso seria o de
responder às questões dos alunos. E, se lhe restasse ainda um pouco de
tempo, ele bem poderia terminar a aula um pouquinho mais cedo.
Repousando o manual, apressou"-se a perguntar.

--- Alguma pergunta?

Mas, de súbito, sentiu"-se vermelho como um pimentão. Nem ele mesmo podia
explicar por quê. Ele, que em princípio achava que enganar os alunos
nada tinha de mais, somente naquela hora se pôs vermelho como um
pimentão. Os alunos, que obviamente não desconfiavam de nada,
continuavam olhando para o seu rosto, sem piscar. Ele olhou mais uma
vez para o relógio. Depois\ldots{} Pegou o livro rapidamente e começou a
lê"-lo a torto e a direito.

Pode ser que a navegação descrita nos livros didáticos, mesmo depois do
ocorrido, continuasse tediosa. Mas o seu modo de dar aula!\ldots{} Yasukichi
não tinha dúvida alguma: era muito mais heroico do que um marinheiro
lutando contra um tufão.

\section*{um guarda corajoso}

Não se lembrava muito bem se tinha sido no fim do outono ou no começo do
inverno. Mas, de qualquer forma, fora numa época do ano na qual vestia
um sobretudo para ir à escola. Havia acabado de sentar"-se à mesa do
almoço quando seu vizinho, um jovem professor militar, começou a lhe
contar um curioso incidente ocorrido pouco tempo antes.

Havia alguns dias, dois ou três ladrões, que tinham roubado ferro no
meio da noite, amarraram sua embarcação atrás da escola. O guarda
noturno que os descobriu tentou prendê"-los sozinho. Todavia, ao fim de
uma luta violenta, foi jogado ao mar sem mais rodeios. Mal conseguiu,
feito um rato molhado, arrastar"-se até a praia. Evidentemente, o barco
dos ladrões já estava longe, perdido nas trevas do alto"-mar.

--- O guarda se chama Ôura. Veja que maus bocados ele passou!

Enfiando um pedaço de pão na boca, o oficial ria com dificuldade.

Yasukichi conhecia Ôura. Certo número de guardas se revezava nas
guaritas situadas do lado da entrada principal. Cada professor que
passasse, fosse civil ou militar, deveria fazer a saudação
regulamentar. Já que Yasukichi não gostava de bater nem de receber a
continência, havia se habituado a apressar os passos diante das
guaritas, a fim de evitar aos guardas qualquer tentativa de saudá"-lo.
Mas o guarda Ôura era o único que não se deixava enganar facilmente.
Sentado na primeira guarita, não desgrudava os olhos das adjacências da
entrada, cobrindo assim o raio de uma boa dezena de metros. E por isso,
ao avistar a silhueta de Yasukichi, e antes mesmo que este chegasse
diante da guarita, já estava ele na postura militar. Assim sendo, só
restou a Yasukichi achar que aquele era seu \textit{karma}. No fim, até
acabou por se resignar. Não apenas se resignou mas, nos últimos dias,
mal percebia Ôura, era ele mesmo quem saudava primeiro, tirando o
chapéu bem alto, como um coelho espreitado por uma cobra cascavel. Era
aquele o homem, a se crer na história, que havia sido jogado ao mar
pelos ladrões. Embora tomado ligeiramente de piedade, Yasukichi não
pôde deixar de rir.

Passados cinco ou seis dias, Yasukichi deparou com Ôura por acaso na
sala de espera da estação. Ao vê"-lo, Ôura, sem se importar com o lugar
em que se encontravam, empinou o peito com rigor e, como sempre,
saudou"-o com uma austera continência militar. Yasukichi teve a nítida
impressão de enxergar o portão de entrada da guarita atrás dele.

Depois de alguns minutos de silêncio, Yasukichi lhe dirigiu a palavra:

--- No outro dia, você\ldots{}

--- É, sim, os ladrões me escaparam por entre os dedos!

--- Deve ter passado um mau momento, não?

--- Mas, por sorte, saí sem ferimentos.

Sorrindo com dificuldade, Ôura prosseguia com uma ponta de autoironia
na voz:

--- Para dizer a verdade, se eu tivesse querido prendê"-los, bem que teria
pego pelo menos um. Mas, se assim fizesse, não teria mais nada para
contar.

--- Como assim, mais nada para contar?

--- É que eu não ganharia nada, nem uma recompensa. Porque, dentro do
regulamento dos guardas, nada se diz sobre esse tema.

--- Mesmo se for morto em serviço?

--- Sim, mesmo nesse caso.

Yasukichi olhou rapidamente para Ôura. A crer em suas palavras, não era
que ele tivesse arriscado a vida, como um herói. A verdade é que
deixara fugir os ladrões que deveria prender por um interesse egoísta
numa recompensa! No entanto\ldots{} Tirando um cigarro, Yasukichi demonstrou
aprovação, com o máximo de jovialidade que pôde exprimir:

--- Sim, então não há nenhum sentido. Se for simplesmente uma questão de
afrontar o perigo, não se ganha nada jamais.

Ôura murmurou um ``Hmm'' ou qualquer coisa do gênero. E, no entanto, tinha
um ar curiosamente abatido.

--- Se ao menos alguém se lembrasse de uma recompensa\ldots{} --- acrescentou
Yasukichi com um tom um pouco melancólico. --- Se ao menos se lembrassem
de uma recompensa, será que todos se arriscariam para enfrentar o
perigo? Também isso é matéria para dúvida, não é?\ldots{} Se ao menos se
lembrassem de uma recompensa? E como! Já foi dito que nós nos
arriscamos para afrontar o perigo!

Ôura terminou por se fechar em seu silêncio. Mas, quando Yasukichi pôs
um cigarro na boca, o guarda tirou um fósforo de sua própria caixa e o
aproximou do professor. Levando seu cigarro à chama de um vermelho vivo
ondulante, Yasukichi, para não se trair, reprimiu um sorriso que
involuntariamente se havia formado no canto de seus lábios.

--- Obrigado.

--- De nada.

Juntamente com essas palavras casuais, Ôura recolocou a caixa de
fósforos em seu bolso. Mas Yasukichi se convenceu de haver finalmente
descoberto, naquela hora, o segredo do valoroso guarda. Aquele fósforo,
não fora somente para Yasukichi que ele havia acendido: havia sido em
verdade para os deuses que, em meio à escuridão, divisavam o caminho
justo dos guerreiros.

\begin{flushright}
\textit{Abril de 1923}\\  
\end{flushright}

\chapter{A vida de um idiota}


\begin{flushright}
A Masao Kume\\
\end{flushright}

\noindent\textit{Deixo"-lhe toda a liberdade de publicar ou não este manuscrito, assim
como o direito de escolher o momento e a maneira de fazê"-lo. Você
conhece, penso eu, a maior parte dos personagens que aparecem neste
texto. Mas, no caso de sua publicação, desejaria que nenhum índice de
nomes fosse acrescentado. Vivo, no momento, a mais infeliz das
felicidades. Mas, por estranho que possa parecer, não me arrependo de
nada. Lamento somente aqueles que tiveram o mau marido, o mau filho, o
mau pai que eu fui. Sendo assim, adeus. Neste manuscrito, não creio que
tenha, ao menos conscientemente, tentado defender minha posição pessoal.}

\textit{Uma última coisa: confio este manuscrito a você, em particular, pois
creio que certamente é quem melhor me conhece. Tente --- se puder ---
rir"-se de minha idiotice visível neste manuscrito (isso se você
desnudar minha máscara de homem urbano).}\\

\noindent20 de junho de 1927\hspace{\stretch{1}}Ryûnosuke Akutagawa\\  

\newcommand{\es}{\ \ \ \ \ }

\section*{\textit{1}\es época}
Era no primeiro andar de uma livraria. Ele, aos vinte anos, montado numa
escada de estilo europeu apoiada contra a prateleira, procurava livros
novos. Maupassant, Baudelaire, Strindberg, Ibsen, Shaw, Tolstói\ldots{}\
Enquanto isso, o fim do dia se aproximava. Mas ele continuava a ler,
avidamente, os títulos inscritos nas lombadas dos livros. Mais do que
livros, enfileirava"-se ali o próprio fim do século. Nietzsche,
Verlaine, os irmãos Goncourt, Dostoiévski, Hauptmann, Flaubert\ldots{}\
Lutando contra a penumbra, ia enumerando seus nomes. Mas os livros
começaram, um após o outro, a se fundir na sombra letárgica.
Finalmente, abandonou sua perseverança e começou a descer a escada de
estilo europeu. Nesse momento, uma lâmpada nua se acendeu, de repente,
bem sobre sua cabeça. Detendo"-se na escada, ele observou de cima os
vendedores e os clientes que se moviam entre os livros. Estavam
estranhamente pequenos. E ainda mais: pareciam tão miseráveis\ldots{}

``A vida humana não vale nem mesmo um verso de Baudelaire.''

Do alto da escada, durante algum tempo, ele deixou seu olhar percorrer
aqueles seres\ldots{}

\section*{\textit{2}\es a mãe}

Todos os loucos estavam igualmente vestidos de cinza"-escuro. A ampla
sala, por causa disso, parecia ainda mais deprimente. Sentado ao órgão,
um dos loucos tocava com fervor um hino cristão. Ao mesmo tempo, bem no
meio da sala, outro dançava ou, mais exatamente, se agitava.

Ele observava o espetáculo em companhia de um médico de aspecto
saudável. Sua própria mãe, dez anos atrás, não diferia em nada daqueles
seres. Em nada\ldots{} --- em seu mau cheiro, ele reconhecia perfeitamente o
de sua mãe.

--- Bem, vamos!

Precedendo"-o no corredor, o médico se dirigiu a uma outra sala. Lá, num
canto, havia alguns cérebros mergulhados em enormes garrafas
arredondadas, de vidro, cheias de álcool. Sobre um dos cérebros, ele
percebeu uma substância esbranquiçada. Era bem parecido com uma clara
de ovo escorrida. Enquanto conversava com o médico, de pé, pensou ainda
uma vez em sua mãe.

--- Mas, sabe? Este cérebro era de um engenheiro da firma de instalações
elétricas. Ele pensava que era um grande dínamo, de cor preta, brilhante.

Evitando os olhos do médico, contemplava a vista além da janela. Mas lá
nada havia além de um muro de tijolos coberto por cacos de vidro.
Naquele muro, cresciam musgos ralos que deixavam áreas de um vago
brilho esbranquiçado.

\section*{\textit{3}\es a casa}

Ele ocupava um quarto no primeiro andar de uma casa no subúrbio. Por
causa da instabilidade do solo, era um pavimento estranhamente
inclinado. Era lá que muitas vezes sua tia brigava com ele e, não raro,
seus pais adotivos também intervinham. No entanto, a pessoa a quem mais
amava era sua tia. Permanecera solteira a vida toda e era, já na época
em que ele tinha vinte anos, uma senhora de quase sessenta. No primeiro
andar dessa casa de subúrbio, várias vezes ele se perguntara se os que
se amam deveriam necessariamente se atormentar uns aos outros.
Sentindo, enquanto isso, na inclinação do primeiro andar, qualquer
coisa de sinistro\ldots{}

\section*{\textit{4}\es tóquio}

O rio Sumida estava carregado de nuvens pesadas. Ele contemplava as
cerejeiras de Mukôjima pela janela de um pequeno barco a vapor. As
cerejeiras em flor eram, a seus olhos, tão deprimentes quanto uma
fileira de trapos. Mas naquelas árvores --- naquelas cerejeiras de
Mukôjima que lá estavam desde a era Edo ---, ele reconhecia sua própria
imagem.

\section*{\textit{5}\es o ego}

Sentado à mesa de um café em companhia de um de seus colegas mais
velhos,\footnote{ Refere"-se a Jun'ichirô Tanizaki (1886--1965).} ele
tirava sem parar baforadas de seu cigarro. Pouco abria a boca. No
entanto, ouvia atentamente as palavras do colega.

--- Hoje passei metade do dia andando de carro.

--- Você tinha algum assunto a tratar?

O queixo apoiado na mão, o colega lhe respondeu com a maior
espontaneidade:

--- Não, foi apenas porque tive vontade.

Aquelas palavras o libertaram em direção a um mundo que ele desconhecia
--- o reino do Ego próximo dos deuses. Sentiu uma espécie de dor. Ao
mesmo tempo, no entanto, sentiu também alegria. Aquele café era
minúsculo. Contudo, sob o quadro do deus Pã, uma seringueira plantada
num vaso vermelho deixava pender preguiçosamente suas folhas polpudas.

\section*{\textit{6}\es a doença}

Exposto ao vento que soprava do mar sem parar, ele abriu um volumoso
dicionário de inglês, seguindo as palavras com o dedo.

\textsc{talaria}: calçado com tiras, sandália.

\textsc{tale}: história.

\textsc{talipot}: palmeira oriunda das Índias Orientais. O tronco atinge uma
altura de cinquenta a cem pés. Folhas utilizadas em guarda"-chuva, leque, 
chapéu etc. Floresce uma vez a cada setenta anos (\ldots{})

Em sua imaginação aflorou claramente o desenho das flores dessa
palmeira. Nesse momento, sentiu uma coceira na garganta até então não
experimentada e, involuntariamente, expeliu um escarro sobre o
dicionário. Um escarro?\ldots{} Não, não era um escarro. Ele pensa na
brevidade da vida e imagina uma vez mais as flores daquela palmeira. As
flores da palmeira que, lá longe, do outro lado do mar, se elevavam às
maiores alturas.

\section*{\textit{7}\es a pintura}

Ele, de repente --- aquilo aconteceu realmente de repente. Ele estava
parado em frente a uma livraria, vendo um livro sobre a obra de Van
Gogh, quando, de repente, compreendeu o que era a pintura.
Naturalmente, aquele livro das obras de Van Gogh continha somente
reproduções fotográficas. Mas, mesmo através delas, ele sentiu a
natureza emergir com todo o esplendor. A paixão por aquelas pinturas
lhe abriu um novo horizonte. Começou a prestar a mais firme atenção à
sinuosidade dos galhos das árvores e às formas arredondadas das faces
femininas.

No entardecer de um dia chuvoso de outono, passava sob uma ponte de
ferro de um subúrbio qualquer. Sob a ribanceira do outro lado da ponte,
havia uma carroça parada. Enquanto passava por aquele lugar, atingiu"-o
de súbito a sensação de que alguém antes dele já percorrera aquele
caminho. Alguém? Já não havia mais necessidade de se perguntar quem. Do
interior de sua alma de vinte e três anos, um holandês com uma orelha
cortada, um longo cachimbo entre os dentes, derramava sobre aquela
paisagem deprimente um olhar penetrante\ldots{}

\section*{\textit{8}\es as faíscas}

Ele ia, molhado pela chuva, pisando a estrada asfaltada. Era uma chuva
bastante forte. Sentiu no ar impregnado de respingos o cheiro de seu
casaco emborrachado.

Nisso, diante de seus olhos, um cabo elétrico começou a soltar faíscas
violetas. Uma estranha excitação o assaltou. Dentro do bolso do
sobretudo se escondia o manuscrito que deveria publicar na revista de
seu pequeno círculo literário. Prosseguindo a marcha sob a chuva,
virou"-se e ergueu os olhos mais uma vez para o cabo elétrico. O cabo
elétrico continuava soltando faíscas fulminantes. Por mais que
interrogasse a vida, nada encontrava que desejasse em particular. Mas
apenas aquelas faíscas violetas --- aquelas faíscas incontroláveis que se
fundiam no ar ---, ele queria agarrá"-las com as mãos, mesmo que tivesse
de pagar com a vida.

\section*{\textit{9}\es os cadáveres}

Todos os cadáveres tinham uma etiqueta pendurada, amarrada no dedão do
pé com um arame. Na etiqueta estavam registrados nome, idade e outros
dados diversos. Inclinado para frente e manejando o bisturi com
destreza, seu amigo começou a esfolar a pele do rosto de um dos
cadáveres. Sob a pele se percebia uma bonita camada de gordura amarela.

Ele observava o cadáver com atenção. Sem dúvida, isso se fazia
necessário para que terminasse um conto --- um conto ambientado na era
Heian.\footnote{ Akutagawa fez esta visita ao amigo médico para escrever
o conto \textit{Rashômon}, que dá título a esta coletânea.} Mas o mau cheiro
do cadáver, semelhante ao do abricó podre, não lhe era nada agradável.
Com as sobrancelhas arqueadas, seu amigo continuava tranquilamente a
manejar o bisturi.

--- Ultimamente, estamos com falta de cadáveres, sabe? --- dizia o colega.

Nesse momento, veio"-lhe uma pronta resposta: ``Se me faltassem cadáveres,
eu mataria alguém --- sem a mínima má intenção, claro!''. Obviamente, no
entanto, ele guardou a resposta somente para si.

\section*{\textit{10}\es o mestre}

Lia um livro do mestre à sombra de um grande carvalho. Sob a luz do sol 
de outono, nenhuma folha sequer se movia. Em algum lugar, no espaço
longínquo, uma balança, com um prato de vidro pendurado, mantinha um
equilíbrio exato. Tal era a imagem que ele via enquanto lia o livro do
mestre\ldots{}\footnote{ Trata"-se de Sôseki Natsume.}

\section*{\textit{11}\es o amanhecer}

O dia ia clareando pouco a pouco. Num certo momento, deu com os olhos
num vasto mercado, numa esquina da cidade. As pessoas que se
aglomeravam, no mercado, os carros, tudo começou a ser tingido por uma
tonalidade rósea.

Acendeu um cigarro e calmamente foi se dirigindo para o mercado. Naquele
momento, um cachorro preto e magro bruscamente se pôs a latir em sua
direção. Mas ele não se assustou. Pelo contrário: sentia amor até mesmo
pelo cão.

Bem no meio do mercado, um plátano estendia os galhos para todas as
direções. Parando ao pé da árvore, ele ergueu os olhos, por entre os
galhos, para o alto céu. Lá, bem sobre sua cabeça, brilhava uma
estrela.

Isso aconteceu quando ele tinha vinte e cinco anos --- três meses depois
de conhecer o mestre.

\section*{\textit{12}\es o porto militar}

O interior do submarino estava mergulhado na penumbra. Agachado dentro
de uma máquina coberta por todos os lados, ele espiava através de um
periscópio. O que se refletia naquele periscópio era uma paisagem
ensolarada de um porto militar.

--- Lá longe se pode ver o navio Kongô, não é? --- diz"-lhe um oficial da
Marinha.

Enquanto ele observava a imagem minúscula do navio de guerra sobre o
espelho quadrado, sem saber por que se lembrou de repente da salsa --- a
salsa que mantém seu perfume delicado mesmo sobre um
\textit{beef"-steak} de trinta tostões a fatia.

\section*{\textit{13}\es a morte do mestre}

Imerso no vento que havia sucedido à chuva, ele caminhava pela
plataforma de uma estação de trem recém"-inaugurada. O céu ainda estava
escuro. Do outro lado da plataforma, enquanto levantavam suas enxadas
em cadência, três ou quatro operários da estrada de ferro cantavam em
voz alta uma melodia qualquer. O vento que viera depois da chuva
dilacerava a canção dos operários e a alma dele. Esquecendo"-se até de
acender o cigarro, ele sentia um sofrimento próximo da alegria. No
bolso de seu casaco, o telegrama: ``Mestre estado crítico''\ldots{}

Em sua direção, vindo das sombras da montanha coberta de pinheiros, o
trem das seis horas da manhã para Tóquio começou a se aproximar
sinuoso, coberto por uma ondulante linha de fumaça.

\section*{\textit{14}\es o casamento}

No dia seguinte ao de seu casamento, ele repreendeu levemente a esposa:

--- Será um problema se você já começar a gastar demais.

Mas, para dizer a verdade, fora sua tia quem lhe mandara fazer aquela
advertência. Sua esposa pediu desculpas, a ele evidentemente, mas
também à tia. Tendo, à sua frente, o vaso de narcisos amarelos que ela
havia comprado\ldots{}

\section*{\textit{15}\es eles}

Eles levavam uma vida pacífica, à sombra das largas folhas de bananeira
--- pois a casa onde moravam se localizava numa cidade à beira"-mar, a uma
boa hora de trem de Tóquio.

\section*{\textit{16}\es o travesseiro}

Ele lia um livro de Anatole France, tendo como travesseiro o ceticismo,
que exalava um perfume de folhas de rosa. Mas não se dera conta de que
um centauro se havia infiltrado, furtivamente, no travesseiro.

\section*{\textit{17}\es a borboleta}

Uma borboleta volteava no vento impregnado por um cheiro de ervas
aquáticas. Durante apenas um ínfimo segundo, ele sentiu o roçar de suas
asas sobre os lábios ressecados. Mas o pó das asas que assim fora
espalhado sobre seus lábios continuou a brilhar, mesmo muitos anos
depois.

\section*{\textit{18}\es a lua}

Ele a encontrou por acaso no meio da escadaria de um hotel. Mesmo assim,
em pleno dia, seu rosto lhe pareceu banhado pela claridade da lua.
Enquanto a seguia com o olhar (eles nunca haviam se encontrado antes),
foi tomado por uma melancolia que nunca sentira até aquele momento\ldots{}

\section*{\textit{19}\es as asas artificiais}

De Anatole France, ele foi passando aos filósofos do século \textsc{xviii}. Mas
não se aproximou de Rousseau. Talvez porque uma parte sua --- seu lado
impulsivo --- se assemelhasse a Rousseau; pode ser que tenha sido por
causa disso. Aproximou"-se, sim, do filósofo de \textit{Cândido}, a quem
se assemelhava por outro de seus lados --- sua parte friamente racional.

Já aos vinte e nove anos, para ele a vida não possuía mais encanto
algum. Mas, pelo menos, Voltaire o provia de asas artificiais.
Abrindo"-as, subiu ao céu, leve e flutuante. Ao mesmo tempo, as alegrias
e as tristezas da vida, banhadas pelo brilho da razão, foram se
afundando sob seus olhos. Deixando cair sorrisos e antífrases sobre as
cidades miseráveis, foi subindo direto para o sol, num percurso sem
nenhum obstáculo. Como se nem se lembrasse daquele grego de outrora
que, por ter exatamente as mesmas asas artificiais queimadas pelo sol,
no fim despencou no mar e morreu\ldots{}

\section*{\textit{20}\es as algemas}

Foi decidido que ele e sua mulher viveriam numa mesma casa com os pais
adotivos dele. Isso porque ele começaria a trabalhar num jornal. Ele
havia acreditado naquele contrato redigido sobre uma folha de papel
amarelo. Mas, mais tarde, compreendeu que o contrato dizia que o jornal
não assumia nenhuma obrigação, apenas ele as assumia.

\section*{\textit{21}\es a filha de uma louca}

Os dois riquixás iam correndo pela estrada deserta em meio aos campos,
sob um céu nublado. Podia"-se adivinhar, só pelo sopro da brisa
marítima, que o caminho levava ao mar. Dentro do segundo riquixá,
estranhando sua falta de interesse naquele encontro, ele se perguntava
o que, então, o teria conduzido àquele local.

Definitivamente, não era amor. Se não fosse amor\ldots{} A fim de evitar
artificiosamente a resposta, não pôde deixar de considerar: ``De
qualquer forma, estamos em condições de igualdade''.

Quem se encontrava no riquixá da frente era a filha de uma louca. Além
do mais, tinha uma irmã mais nova que se suicidara por causa de ciúmes.
``Não, não há mais solução.''

Ele, em relação àquela filha de uma louca\ldots{} Aquela mulher que vivia
somente em função de seu instinto animal chegou a lhe provocar até
certo ódio.

Os dois riquixás, nesse ínterim, passaram ao longo de um cemitério que
cheirava a mar. Por trás da cerca de galhos cobertos de conchas de
ostra, havia várias lápides enegrecidas. Ele contemplava o mar que
cintilava docemente para além das lápides de pedra quando, subitamente,
foi tomado de desprezo pelo marido da filha da louca --- um marido que
não conseguia cativar seu coração\ldots{}

\section*{\textit{22}\es um pintor}

Era uma simples ilustração numa revista. Mas o galo pintado em
\textit{sumie} refletia uma personalidade extraordinária. Ele se
informou sobre o artista junto a um amigo.

Cerca de uma semana mais tarde, o pintor o visitou. Aquele foi um dos
acontecimentos mais marcantes de sua vida. Descobriu naquele pintor uma
poesia ignorada por todos. Bem mais: descobriu sua própria alma, que
até então ele mesmo desconhecia.

No entardecer de um dia frio de outono, o milho da China o fez
lembrar"-se do pintor. Com a armadura das folhas agrestes, o alto pé de
milho deixava entrever, sobre o montículo de terra, suas finas raízes
semelhantes a fibras nervosas. Por outro lado, aquela imagem, tão
vulnerável, em nada diferia de seu próprio autoretrato. Mas tal
descoberta não fez mais que deprimi"-lo.

--- Já é tarde demais. Mas, em último caso\ldots{}

\section*{\textit{23}\es ela}

Escurecia em frente a uma praça. Ele a atravessava, com o corpo um pouco
febril. Vários prédios altos faziam refletir as luzes acesas de suas
janelas num céu claro ligeiramente prateado.

Parou à beira da calçada, decidindo esperá"-la ali. Uns cinco minutos
mais tarde, ela já avançava em sua direção, com ar de fadiga. Mas,
assim que o viu, pôs"-se a sorrir, dizendo:

--- Ah, como estou cansada!

Ombro contra ombro, os dois foram caminhando pela praça à meia"-luz. Para
eles, era a primeira vez. Para ficar com ela, ele se sentia capaz de
abandonar tudo.

Depois, no carro, ela lhe disse, fixando"-lhe o olhar:

 --- Você não se arrepende?

Ele respondeu com um tom firme:

--- Não me arrependo de nada.

Ela apertou sua mão.

--- Eu também não.

Naquele momento também, seu rosto parecia banhado pela claridade da lua.

\section*{\textit{24}\es o nascimento}

De pé junto à porta corrediça, ele observava uma parteira de avental
branco lavando o bebê. Cada vez que o sabão penetrava em seus olhos, o
bebê fazia caretas enternecedoras. Com efeito, chorava a altos brados.
Sentindo o cheiro do bebê, que parecia um ratinho, ele não conseguiu
deixar de pensar no seu íntimo: ``Para que será que ele veio ao mundo? A
este mundo cheio de sofrimentos\ldots{} Por que padecer o destino de ter um
pai como eu?''.

Era, no entanto, o primeiro filho que sua mulher punha no mundo.

\section*{\textit{25}\es strindberg}

Ele observava da soleira do quarto alguns chineses sujos que jogavam
\textit{mahjong} sob a luz do luar que iluminava as flores da
romãzeira. Depois, voltando a seu quarto, começou a ler sob a lâmpada
baixa a \textit{Confissão de um imbecil}, de Strindberg. Mas nem havia
lido duas páginas quando um sorriso amargo lhe aflorou aos lábios:
também Strindberg, nas cartas destinadas à condessa sua amante,
escrevia mentiras semelhantes às dele\ldots{}

\section*{\textit{26}\es as eras remotas}

Os budas, as divindades, os cavalos e as flores de lótus de cores
esmaecidas quase o oprimiam. Os olhos voltados para o alto, ele se
esquecia de tudo. Até de sua própria sorte por haver se libertado da
filha da louca\ldots{}

\section*{\textit{27}\es a disciplina espartana}

Caminhava com um amigo por umas ruelas. Do outro lado, um riquixá,
coberto com uma capota, vinha aproximando"-se. Para sua surpresa,
percebeu lá dentro a mulher da noite anterior. Mesmo em pleno dia, seu
rosto parecia banhado pela claridade da lua.  Naturalmente, na frente
de seu amigo eles não se cumprimentaram.

--- Que bela mulher! --- exclamou o amigo.

Ao que ele respondeu sem hesitar, os olhos fixos à frente, sobre a
montanha tingida de primavera:

--- Sim, de fato, é bela.

\section*{\textit{28}\es o assassino}

A estrada do campo fazia pairar um odor de estrume de vaca no ar
ensolarado. Enquanto enxugava o suor, ele ia escalando o caminho
ligeiramente escarpado das margens da estrada. O trigo maduro exalava
um perfume embriagador.

--- Mate"-o, mate"-o!\ldots{} --- Seus lábios repetiam essas palavras
inconscientemente. Matar quem? Para ele, isso estava muito claro.
Pensava naquele homem de cabelos curtos, que era realmente vil.

Nesse ínterim, para além dos trigos dourados, tornava"-se visível a
cúpula redonda de uma igreja católica romana\ldots{}

\section*{\textit{29}\es a forma}

Era um frasco de ferro para saquê. Aquele frasco cinzelado de finas
nervuras lhe havia, sem que percebesse, ensinado a beleza da ``forma''.

\section*{\textit{30}\es a chuva}

Sobre o grande leito, eles conversavam sobre vários assuntos. Chovia
para além das janelas do quarto. Parecia que as flores de crino iam
acabar apodrecendo dentro daquela chuva. O rosto dela parecia, como
sempre, banhado pela claridade da lua. Mas ele não podia dizer que
conversar com ela não o entediasse. Deitado de bruços, acendeu
calmamente um cigarro e lembrou"-se de que já fazia sete anos que vivia
com ela. Ele se perguntou:

--- Será que amo esta mulher?

Mesmo para ele, tão habituado a se autoanalisar, a resposta foi
inesperada:

--- Sim, eu ainda a amo.

\section*{\textit{31}\es o grande terremoto\footnoteInSection{ Trata"-se do Grande Terremoto
de Kantô, ocorrido em 1º de setembro de 1923, testemunhado também por Kawabata. 
A casa de Akutagawa não sofreu nada.}}

Aquilo se parecia, de alguma forma, com o odor que se desprendia de
abricós demasiado maduros. Enquanto caminhava pelos escombros dos
incêndios, sentiu vagamente aquele cheiro e achou que o odor dos
cadáveres putrefeitos pelo calor tórrido não era, afinal de contas,
assim tão desagradável. Mas, parando e olhando o lago pesado de tantos
cadáveres, descobriu pelos sentidos que a palavra ``execrável'' não era
absolutamente nenhum exagero. Foi, sobretudo, o cadáver de uma criança
de doze, treze anos que mais o comoveu. Contemplando aquele cadáver,
sentiu como que certa inveja. Uma frase lhe vem à memória: ``Aqueles que
os deuses amam morrem jovens''. Sua irmã mais velha e seu meio"-irmão
tiveram as casas queimadas. No entanto, o marido de sua irmã, condenado
por crime de perjúrio, se encontrava em suspensão condicional da pena\ldots{}

--- Que morram todos, que importa?

Parado em meio aos escombros, era"-lhe impossível, no íntimo, pensar de
outra forma.

\section*{\textit{32}\es a briga}

Ele teve uma boa briga física com seu meio"-irmão. A verdade é que, por
causa dele, o irmão mais novo se tornara muito vulnerável a pressões.
Ao mesmo tempo, no entanto, ele também havia perdido a liberdade por
causa do irmão. Toda a família continuamente dizia a seu jovem irmão:

--- Siga o exemplo de seu irmão mais velho.

Mas, para ele, aquilo era exatamente o mesmo que lhe amarrarem os pés e
as mãos. Os dois corpos engalfinhados, eles terminaram rolando até a
varanda. No jardim da varanda --- ele ainda se lembrava --- havia um pé de
murta, que, sob um céu pesado de chuva, erguia miríades de flores num
brilho avermelhado.

\section*{\textit{33}\es o herói}

Um dia, por uma das janelas da casa de Voltaire, ele contemplava as
montanhas altas. Sobre os picos cobertos de gelo não se via sequer a
sombra de uma águia. Mas via"-se um russo de baixa estatura que escalava
obstinadamente as trilhas da montanha. Depois que a noite envolveu
também a casa de Voltaire, ele escrevia sob a claridade da lâmpada o
seguinte poema ``político'' --- enquanto pensava na silhueta daquele
russo\footnote{ Poema composto a Lênin, em decassílabos, forma 
inexistente no Japão clássico.} que escalava as trilhas da montanha\ldots{}

\begin{verse}
Tu, que mais que ninguém observou os dez mandamentos,\\
És tu, que mais que ninguém violou os dez mandamentos.\\
Tu, que mais que ninguém amou o povo,\\
És tu, que mais que ninguém desprezou o povo.\\
Tu, que mais que ninguém ardeu no fogo do ideal,\\
És tu, que melhor que ninguém conhece a realidade.\\
Tu és este trem elétrico de perfume campestre\\
Que nosso Oriente engendrou.
\end{verse}

\section*{\textit{34}\es as cores}

Aos trinta anos, ele notou seu amor por certo terreno baldio. Ali, no
chão coberto apenas de musgos, tudo o que havia eram alguns pedaços de
tijolos e telhas. Mas, aos seus olhos, a cena era uma verdadeira
paisagem de Cézanne. Ele se lembrou, de súbito, da paixão que o animava
sete ou oito anos antes. Mas descobriu, ao mesmo tempo, que ele, sete
ou oito anos atrás, não sabia o que era a cor.

\section*{\textit{35}\es o polichinelo}

Queria viver intensamente, de maneira a não se arrepender de nada,
qualquer que fosse a hora de sua morte. Mas, como sempre, continuava a
levar uma vida bem modesta, em consideração a seus pais adotivos e à
sua tia. Isso construiu em sua vida uma cisão em duas partes: uma de
sombra, outra de luz. Ele havia visto um dia um polichinelo, numa loja
de roupas ocidentais, e se perguntara até que ponto ele próprio não
seria um boneco. Mas seu inconsciente --- ou, em outros termos, seu
segundo eu --- já havia expressado muito antes esse espírito em um de
seus contos.

\section*{\textit{36}\es a lassidão}

Ele caminhava, em companhia de alguns estudantes universitários, pelos
campos de eulálias.

--- Todos vocês ainda devem ter muita sede de vida, não é?

--- Oh, é claro\ldots{} Mas o senhor também\ldots{}

--- Justamente, eu não! A sede de escrever, essa sim, eu ainda tenho.

Era a mais pura verdade. Há algum tempo, ele havia perdido realmente
todo o interesse pela vida.

--- Mas a sede de escrever não é também a sede de viver?

Ele não respondeu nada. O campo de eulálias deixava entrever nitidamente
um vulcão ativo por sobre as espigas vermelhas. Ele sentiu quase que
inveja daquele vulcão. No entanto, nem mesmo ele sabia por quê\ldots{}

\section*{\textit{37}\es a mulher do norte\footnoteInSection{ Em geral, refere"-se a ``mulher
como norte'' como sendo originária de Etchû, Echizen ou Echigo, mas no
caso, supõe"-se ser ela de Karuizawa, rica estância ao norte de Tóquio.}}

Ele se confrontou com uma mulher que poderia, pela inteligência, se
medir a ele. Mas compondo poemas líricos --- como ``A mulher do Norte'' ---
escapou por pouco àquele perigo. E isso lhe provocou uma sensação
dolorosa, como se derrubassem a camada de neve brilhante congelada no
tronco de uma árvore.

\begin{verse}
Levado pelo vento o chapéu de palha\\
Sobre o caminho um dia tombará\\
Por que cuidar de meu nome?\\
Somente o teu importa!
\end{verse}

\section*{\textit{38}\es a vingança}

A cena se passou num balcão de um hotel, em meio às árvores cobertas de
renovos. Lá, ele fazia uns desenhos para distrair um garoto --- o filho
único da filha de uma louca, com a qual ele rompera relações sete anos
atrás.

A filha da louca, acendendo um cigarro, observava os dois se entretendo.
O coração pesado e torturado, ele continuou a desenhar trens e aviões.
O garoto, felizmente, não era seu filho. No entanto, ser chamado de
``titio'' lhe era intolerável.

Depois que o garoto saiu, a filha da louca, continuando a fumar,
dirige"-lhe a palavra, em tom sedutor:

--- Você não acha o menino parecido com você?

--- Nem um pouco! Em primeiro lugar\ldots{}

--- A ``puericultura pré"-natal'', isso bem que existe, não acha?

Ele desviou os olhos sem nada responder. Mas, no íntimo, sentiu uma
vontade cruel de estrangular uma mulher como aquela\ldots{}

\section*{\textit{39}\es o espelho}

Conversava com um amigo, no canto de um café. Comendo maçãs assadas, o
amigo falava sobre o frio que estava fazendo havia já algum tempo. Ele
percebeu de súbito como que uma contradição no meio de toda aquela
tagarelice.

--- Você ainda está solteiro, não é?

--- Não realmente. Eu me caso já no próximo mês.

Ele acabou por se calar involuntariamente. O espelho pendurado na parede
do café multiplicava o reflexo de sua imagem. Friamente, como uma
ameaça\ldots{}

\section*{\textit{40}\es o diálogo}

--- Por que você ataca o sistema social atual?

--- Porque estou vendo os males engendrados pelo capitalismo.

--- Os males? Eu pensava que você não discernisse o Bem do Mal. Bem, e a
vida?

Ele dialogava assim com um anjo. Com um anjo, veja só, de cartola e
que não tinha do que se envergonhar diante de ninguém\ldots{}

\section*{\textit{41}\es a doença}

Ele começou a sofrer de insônia. E, além disso, sua resistência física
também começou a se debilitar. Cada um dos médicos diagnosticou duas ou
três doenças: hipercloridria, atonia gástrica, pleurisia seca,
neurastenia, conjuntivite crônica, fadiga cerebral\ldots{}

No entanto, ele próprio sabia muito bem quais eram as raízes de seu mal:
tudo vinha da vergonha que sentia de si mesmo e do medo dos outros; os
outros\ldots{} --- daquela sociedade que ele desprezava!

Numa tarde coberta de pesada neve, um cigarro aceso à boca, no canto de
um café, ele escutava atentamente a música que emanava de um gramofone,
do outro lado da sala. Era uma música que penetrava estranhamente seu
estado de espírito. Esperou a música terminar e, caminhando até o
gramofone, verificou a etiqueta colada sobre o disco: \textit{Magic flute} --- Mozart.

Ele compreendeu num átimo. Mozart --- que havia violado os dez mandamentos
--- certamente havia sofrido muito. Mas será que tanto quanto ele\ldots{} A
cabeça abaixada, retornou em silencio à sua mesa.

\section*{\textit{42}\es o riso dos deuses}

Com trinta e cinco anos de idade, ele ia caminhando pela floresta de
pinheiros banhada pelo sol de primavera. Recordava as palavras que
escrevera dois ou três anos antes: ``Para sua infelicidade, os deuses
não podem, como nós, se suicidar\ldots{}''.

\section*{\textit{43}\es a noite}

A noite se aproximava uma vez mais. O mar em fúria lançava ao alto,
incessante, seus respingos dentro da penumbra. Ele celebrava, sob
aquele céu, um segundo casamento com sua mulher. Para ele, era uma
alegria. Mas era também um sofrimento. Seus três filhos contemplavam
com eles os relâmpagos que abriam sulcos no alto"-mar. Com uma criança
no colo, sua mulher parecia estar contendo as lágrimas.

--- Bem lá embaixo, não seria um navio?

--- Sim.

--- Um navio com um mastro partido ao meio.

\section*{\textit{44}\es a morte}

Aproveitando"-se de estar sozinho no quarto, ele tentou se enforcar,
pendurando uma faixa de quimono na grade da janela. Contudo, depois de
passar o pescoço por dentro do laço, ele de repente se pôs a temer a
morte. Mas não porque temesse o sofrimento do instante em que ela
ocorreria. Na segunda tentativa, pegou seu relógio do bolso e decidiu
medir, a título experimental, o tempo que levaria até morrer. Então,
depois de um breve instante de sofrimento, tudo começou a se apagar.
Uma vez superado aquele ponto, certamente acabaria deslizando para a
morte. Consultou os ponteiros do relógio e descobriu que começara a
sentir a dor após um minuto e uns vinte segundos. Por trás da grade da
janela estava totalmente escuro. No entanto, naquela escuridão ouvia"-se
também o canto selvagem dos galos.

\section*{\textit{45}\es divan}

O \textit{Divan} voltou a insuflar forças novas em sua alma. Tratava"-se
do ``Goethe oriental'', que ele ainda não conhecia. Vendo Goethe
serenamente em pé no Nirvana que não discrimina nem Bem nem Mal, sentiu
uma inveja próxima ao desespero. A seus olhos, o poeta Goethe era muito
mais grandioso que o poeta Jesus Cristo. No coração daquele poeta não
resplandeciam somente as rosas da Acrópole e do Gólgota, mas também as
da Arábia. Se tivesse contado com forças para seguir os passos daquele
poeta\ldots{} Terminou de ler o \textit{Divan} e, uma vez apaziguado o
excesso de emoção, não conseguiu evitar um profundo desprezo por si
próprio, que fora um eunuco na vida cotidiana.

\section*{\textit{46}\es as mentiras}

O suicídio do marido da irmã o aniquilou por completo. Por causa disso,
ele teve de cuidar da família dela. Seu futuro era --- pelo menos a seus
olhos --- tão sombrio quanto um dia ao crepúsculo. Sua falência
espiritual não lhe inspirava mais que um frio sorriso de escárnio
(conhecia, sem exceção, todos os seus vícios e todos os seus pontos
fracos), e ele continuou invariavelmente a ler todos os tipos de
livros. Mas mesmo \textit{As Confissões} de Rousseau estavam cheias de
mentiras heroicas. Sem falar de \textit{Vida Nova}!\ldots{}\footnote{ Romance
longo em forma autobiográfica de Tôson Shimazaki, publicado de forma
seriada no jornal \textit{Asahi Shinbun} entre 1918 e 1919 e cujo
protagonista sofre grave dilema ao impor relação amorosa indevida.} 
Ele jamais havia visto um hipócrita mais velhaco do que o herói daquela
obra. Mas pelo menos François Villon abria brechas de poesia em seu
coração. Ele havia, em alguns de seus poemas, descoberto um ``belo
macho''. A figura de Villon esperando ser enforcado lhe aparecia até em
sonhos. Várias vezes ele tentou, como Villon, descer ao mais baixo da
vida humana. Mas nem suas condições de vida nem sua energia física lhe
concederam tal liberdade. Foi se debilitando mais e mais. Exatamente
como a árvore que Swift vira outrora --- uma árvore que começava a
ressecar pelo alto\ldots{}

\section*{\textit{47}\es o jogo de fogo}

Ela estava com o rosto radiante. Era exatamente como se o brilho do sol
da manhã cintilasse sobre uma fina camada de gelo. Ele lhe tinha muita
afeição. Entretanto, não sentia amor. Além disso, jamais tocara sequer
com o dedo o corpo daquela mulher.

--- Então, quer dizer que você quer morrer?

--- Sim. Não\ldots{} Enfim, não é que eu queira morrer; é que estou enjoado de
viver.

Depois daquele diálogo, eles fizeram o pacto de morrer juntos.

--- Um suicídio platônico, não é?

--- Duplo suicídio platônico!

Ele não pôde deixar de estranhar sua calma.

\section*{\textit{48}\es a morte}

Ele não morreu com ela. Mas sentia alguma satisfação indefinida por não
haver jamais tocado seu corpo. Ela conversava às vezes com ele, como se
nada houvesse acontecido. Além do mais, ela lhe entregou um frasco com
cianureto que possuía, dizendo"-lhe:

--- Tendo isso, nós dois nos sentiremos mais seguros.

Aquele gesto com certeza deve ter fortalecido sua alma. Sozinho, sentado
numa cadeira de junco, contemplando as folhas novas das castanheiras,
por vezes não conseguia deixar de pensar na paz que a morte lhe
proporcionaria.

\section*{\textit{49}\es o cisne empalhado}

Juntando as últimas forças que lhe restavam, tentou escrever sua
autobiografia. Mas não foi tão simples como havia imaginado: era ainda
orgulhoso demais, cético demais e muito interesseiro. Não conseguia
deixar de desprezar a si próprio. No entanto, por outro lado, não
conseguia deixar também de pensar que ``todos são iguais, uma vez
desmascarados''. \textit{Poesia e Verdade}: tal título, em sua opinião,
deveria ser o de qualquer autobiografia.\footnote{ É o título da
autobiografia de Goethe.} E, ainda mais, ele compreendia claramente
que uma obra artística seguramente não mudaria ninguém. Aqueles que
suas obras atingissem só poderiam ser pessoas que se assemelhassem a
ele e tivessem vivido uma vida semelhante à sua. Tal estado de espírito
também o influenciou. Por isso, decidiu escrever de modo breve e
simples a sua ``Poesia e verdade''.

Quando terminou de escrever \textit{A vida de um idiota},
descobriu por acaso, numa loja de antiguidades, um cisne empalhado. O
pássaro estava de pé, a cabeça levantada, mas tinha suas asas
amareladas comidas por insetos. Pensando em toda a sua vida, ele sentiu
aflorarem lágrimas, a que se misturava um riso de escárnio. A loucura
ou o suicídio, era só o que via à sua frente. Enquanto caminhava
solitário pelas ruas onde tombava a noite, resolveu esperar pelo
destino que, lenta mas decididamente, viria destruí"-lo.

\section*{\textit{50}\es o prisioneiro}

Um de seus amigos sucumbiu à loucura. Sempre tinha sentido certa
simpatia por ele. Isso porque compreendia profundamente sua solidão ---
toda a solidão que aquele amigo ocultava sob uma máscara jovial. Depois
de ter enlouquecido, visitou"-o duas ou três vezes.

--- Você e eu estamos ambos possuídos por um demônio, o famoso demônio do
Fim do Século --- dizia"-lhe o amigo, abaixando a voz.

Mas, dois ou três dias mais tarde, a caminho de uma estação termal,
disseram que ele já estava comendo até rosas.

Depois da hospitalização do amigo, recordou"-se do busto do autor do
\textit{Inspetor geral}, de que seu amigo gostava. Lembrando que Gogol
também morrera louco, não pôde deixar de sentir uma força desconhecida
que os subjugava, a todos.

Já se sentia às raias do esgotamento quando, de repente, ouviu de novo o
riso dos deuses, ao ler as últimas palavras escritas por Radiguet. Era
a seguinte frase: ``Os soldados de Deus virão me prender!''. Tentou lutar
contra sua superstição e seu sentimentalismo. No entanto, estava
fisicamente impossibilitado de lutar contra o que fosse. Não tinha
dúvidas de que, realmente, o ``demônio do Fim do Século'' o estava
torturando. Sentiu inveja dos homens da Idade Média, que faziam de Deus
a sua força. Entretanto, crer em Deus\ldots{} --- a crença no amor de Deus lhe
era definitivamente impossível. Daquele Deus em quem até mesmo Cocteau
acreditava!

\section*{\textit{51}\es a derrota}

Até a mão que sustinha a caneta começou a tremer. Até mesmo a saliva
começou a lhe escorrer pela boca. Sua mente esteve lúcida apenas por um
curto tempo, quando acordou após a ingestão do Veronal 0,8. Os momentos
de lucidez duraram meia hora ou, no máximo, uma hora. Imerso na
penumbra, vivia uma vida inerte. De certo modo, usando como muleta uma
espada fina cuja lâmina havia perdido o fio\ldots{}

\begin{flushright}
\textit{Junho de 1927}\footnote{ Como se vê, esta não é uma obra de ficção.
Akutagawa se matou com uma dose excessiva de Veronal, um mês
depois, em 24 de julho de 1927. Refere"-se neste texto à sua mulher,
Fumiko Tsukamoto, que desposou em 1918, quando ela tinha 18, e ele,
26 anos; a seus três filhos, dos quais o primeiro, Hiroshi,
nasceu em 1920; a seu meio"-irmão; a seu amigo que enlouquecera, o escritor
Kôji Uno, que sofreu uma grande depressão entre 1927 e 1933, embora
somente viesse a morrer em 1961 (Nota de Renato Janine, da 1ª edição).
Os tradutores acham importante esta nota anterior, pois todos os textos
referentes à época contemporânea de Akutagawa, os referidos
``Yasukichimono'' devem ser compreendidos pelo viés de seu forte caráter
autobiográfico, em oposição às narrativas dos outros grupos, mais
reconhecíveis como  ``obra de ficção''. A ambiguidade entre ``vida'' e
``obra'' encontra"-se latente.}\\  
\end{flushright}
