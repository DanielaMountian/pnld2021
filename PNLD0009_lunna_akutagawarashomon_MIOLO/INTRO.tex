\chapter{A estética contida de Akutagawa}

\begin{flushright}
\textsc{madalena h.\,cordaro}
\end{flushright}\bigskip

\section{Sobre o autor}

\noindent{}Nomeado  Ryûnosuke por ter nascido em 1892, na hora, dia, mês e ano do
Dragão (\textit{tatsu} ou \textit{ryû}), Akutagawa cresceu na ``cidade
baixa'', no bairro de Kyôbashi, em Tóquio, como filho mais velho
(tinha já duas irmãs), mas, quando tinha nove meses, sua mãe Fuku
enlouqueceu, e seu pai, Tokizô Niihira, enviou"-o para viver e ser
educado na casa do tio paterno, em Odawara"-chô (atual Kuroda"-ku
Ryôgoku), dele adotando o sobrenome, juntamente com uma tia solteirona,
Fuki. O pai (tio) adotivo é praticante de pintura \textit{nanga} 
(pintura preto"-e-branco, com tinta \textit{sumi}, à moda do 
sul da China), poesia \textit{haikai}, jardinagem \textit{bonsai}; a
mãe adotiva, Tomo, é sobrinha de um grande conhecedor dos caminhos dos
prazeres hedonistas de fins do período Edo e cultivado nos caminhos das
artes. A loucura de sua mãe natural, entretanto, marca o imaginário do
escritor --- na época acreditava"-se poder ser hereditária a loucura ---,
embora críticos hoje relevem o estado da mãe em prol de um caráter
delicado e nervoso  face a um marido violento e à morte da filha mais
velha e, ainda, ao forçado abandono do bebê --- fossem outros os tempos,
talvez recebesse um eficiente tratamento para depressão pós"-parto.
Outro motivo para ter sido enviado para longe de seus pais naturais é
que Ryûnosuke nascera quando o pai tinha 43 anos de idade e a mãe, 33,
idade considerada de ``grande perigo'' para se ter filhos, segundo o credo 
ainda hoje vigente no Japão, e que resultava no sistema de adoção chamado \textit{suteji} ou
\textit{sutego}, ``criança abandonada''. Crescido em meio a uma família
tradicional de antigos e então empobrecidos oficiais menores do sistema
de xogunato, os tios retinham ainda muita preocupação com as aparências de
uma quimera de posição social em um mundo convulsionado por mudanças.
Akutagawa começou a escrever obras literárias aos dez anos de idade e na
adolescência já lia poesia chinesa, ficção japonesa moderna, traduções
de Henrik Ibsen e Anatole France. Aos 12 anos de idade, em 1904,
irrompeu a Guerra Russo"-Japonesa.

Aluno de destaque, ingressou rapidamente no Terceiro Ginásio de Tóquio e 
alguns de seus colegas de classe nos anos de 1910 tornaram"-se
também escritores: Kan Kikuchi, Masao Kume, Yûzô Yamamoto e Bunmei Tsuchiya. 
Graduando"-se no curso secundário em segundo lugar, depois de
ter devorado autores como Baudelaire, Gogol e Strindberg, aliados aos
autores nacionais de teatro \textit{jôruri}, romances e poemas
\textit{haiku}, além de começar a se interessar pelo cristianismo,
Akutagawa foi admitido na Universidade Imperial de Tóquio, onde se
especializou em literatura inglesa.

Foi ali que, na revista \textit{Shinshichô}, publicou traduções de
Anatole France (1844--1924) e William Butler Yeats (1865--1923).
Mas foi em 1915, durante seu último ano na universidade, que publicou
\textit{Rashômon}, na revista \textit{Teikoku Bungaku}
(Literatura Imperial), obra que na época não foi particularmente
destacada. Neste mesmo ano, Akutagawa começou a frequentar os 
\textit{Mokuyôkai} (Encontros das Quintas"-Feiras) promovidos pelo
veterano  Sôseki Natsume (1867--1916) em sua casa, por intermédio de um
amigo da universidade, passando a se considerar discípulo do grande
escritor, mentor intelectual que o estimulou, particularmente a partir
da publicação do conto \textit{Hana} (O nariz, 1916). Iniciou"-se nesse
mesmo ano como professor de inglês, 12 horas semanais, no Colégio de
Engenharia Naval de Yokosuka, morando sobre uma lavanderia na cidade de
Kamakura, província de Kanagawa. Casou"-se em 1918, com Fumiko
Tsukamoto, tendo rompido com Yayoi Yoshida dois anos antes, obedecendo
à oposição de sua família, e começou a trabalhar simultaneamente no
jornal \textit{Ôsaka Mainichi Shinbun}, com o qual firmaria contrato de
exclusividade a partir do ano seguinte, iniciando uma nova vida na
cidade de Kamakura, com a esposa, a tia Fumi e até, luxo inaudito para
sua posição, uma empregada. Foi através do jornal que conheceu os
únicos países estrangeiros em que esteve, China e Coreia, em 1921, como
enviado especial --- diferentemente do mestre Sôseki, que vivera dois
anos na Inglaterra como enviado do governo, o contato de Akutagawa com
a reverenciada Europa ocorre através da leitura e de livros estampados.
Um ano depois, sua saúde e seus nervos começam a se deteriorar, e o
fantasma da loucura de sua mãe, a persegui"-lo.

Akutagawa escreveu certo número de poemas \textit{haiku} e 
poesia moderna (com métrica ocidental, rima ocasional, tema político,
formatos fixos), enquanto se dedicava a escritos curtos em prosa, sua
grande contribuição à literatura japonesa, até que sofreu outro colapso
nervoso em 1926, sendo que, um ano depois, teve ainda de assumir
dívidas herdadas pela morte de seu cunhado, julgado culpado em processo
de perjúrio por ter incendiado sua casa a fim de receber certo valor
assegurado. O peso de tal responsabilidade, o fantasma da loucura e,
dizem, a tensão de um debate literário com Jun'ichirô Tanizaki, podem
tê"-lo levado ao suicídio aos 35 anos, em 24 de julho de 1927. Assim
como o suicídio ritual do general Nogi pela morte do imperador em 1912
tem sido considerado o fim da tradicionalista e modernizadora era Meiji
e o início do moderno e repressor período Taishô, o suicídio de
Akutagawa tem sido tomado como símbolo de seu fim.

O suicídio do general Nogi dividiu os intelectuais: os mais idosos, de
formação tradicional como Ôgai Mori (1862--1922) ou  Sôseki Natsume
(1867--1916), embora longe de assumirem os valores do xogunato e do
confucionismo do passado, encontravam"-se distantes dos mais jovens.
Romancistas característicos do período Meiji, suas problemáticas ainda
se embatiam com padrões recém"-importados principalmente de teor
naturalista e correntes filosóficas das mais variadas, notáveis nos
conflitos de suas personagens, muitas delas tipicamente hedonistas ou
confucionistas do período Edo. Sôseki, por exemplo, em seu último
romance intitulado \textit{Kokoro} (Coração, 1916), cria uma
passagem em que, inspirando"-se no general Nogi, faz do suicídio do
protagonista, um ato paralelo de \textit{junshi}, ou seja, de ``morte
acompanhando o senhor'' (no caso do general, em louvor do imperador Meiji).

Também contemporâneos de Akutagawa, já os escritores do grupo Shirakaba
descendiam dos que haviam se inserido sem grandes problemas em uma nova
burocracia e no capitalismo. A este grupo predominantemente humanista
se integram escritores como Saneatsu Mushanokôji (1885--1976), Naoya Shiga 
(1883--1971) ou Takeo Arishima (1878--1923). Houve também posições
políticas revolucionárias, de teor anarquista ou marxista, em
escritores como Sakae Ôsugi  (1885--1923), Karoku Miyaji (1884--1958) ou
Sukeo Miyajima  (1886--1951). A influência das vanguardas europeias,
especialmente do dadaísmo e do surrealismo, também se fez sentir em
vários escritores --- especialmente em Yasunari Kawabata, cuja lírica se
imiscui de imagens quiméricas ---, embora muitos deles terminassem em algum tipo
de militância política que acabaria por ser reprimida pela nação.

Akutagawa é considerado parte do grupo de intelectuais e estetas
contrários ao naturalismo (que engendrou grande número de obras no
Japão), ao humanismo de cunho social de Shirakaba (seus membros em
geral provinham de elite econômica) e à literatura proletária (que se
manteve bem restrita quanto à forma). Jun'ichirô Tanizaki  (1886--1965),
Haruo Satô  (1892--1964) e Mantarô Kubota  (1889--1963) foram companheiros
literários mais próximos, conquanto dissonantes em questões estéticas. O
rótulo mais frequentemente atribuído a Akutagawa é o de ter sido 
um ``intelectual esteta'' e, acrescentemos, tocado por um vago e transoceânico 
``demônio do fim"-de"-século'' (\textsc{xix}), tipicamente europeu, 
que provocou não poucos suicídios também entre seus escritores.

\section{Sobre a obra}
Vista retrospectivamente, a obra de Ryûnosuke Akutagawa  pode ser
dividida, \textit{grosso modo}, em narrativas intimamente relacionadas
à sua própria vivência, ainda que sublimadas e estetizadas, e
narrativas inspiradas na história literária e em personagens históricos
e seus embates éticos e estéticos.

A feiura do egoísmo humano e o valor da arte enquanto redentora da
miséria da vida cotidiana são considerados temas recorrentes em seus
escritos, encontrando"-se disseminados em uma série de obras de gêneros
distintos. Incluem"-se neste segundo grupo, de narrativas históricas:
\textit{ôchômono}  (``coisas da nobreza''), escritos centrados 
não apenas no universo da elite nobre como também no do povo
comum do período Heian (794--1192); \textit{kirishitanmono} 
(``coisas cristãs'') são narrativas que discutem dramas vividos pelos 
convertidos ao catolicismo pela Companhia de Jesus no Japão, especialmente
em Nagasaki; \textit{Edomono}  (``coisas de Edo'') inclui
narrativas que têm como pano de fundo a cultura do período Edo
(1603--1868); e \textit{kaikamono} (``coisas do iluminismo''), 
os que apresentam personagens do início do período Meiji em
interação com as recém"-importadas culturas e seus representantes --- nem
se faz necessário dizer que o admirado mestre Sôseki Natsume trabalhou
à exaustão as mesmas questões deste último grupo em romances de longa
extensão e profunda discussão filosófica.

As obras que se centram na cultura da corte da Capital Heian (atual
Quioto) foram em grande número e em vários tons de humor e crítica,
entre as quais se encontra o conto \textit{Jigokuhen} 
(Biombo dos infernos, 1918), ausente nesta coletânea, que trata de um
talentoso pintor oficial que coloca a arte acima de sua própria filha
na representação dos infernos budistas. Também neste grupo estão
\textit{Rashômon} (1915) e \textit{Yabuno naka} (Dentro do bosque, 1922), 
que foram baseados em \textit{Konjaku monogatarishû} 
(Coletânea de narrativas de ontem e de hoje, século \textsc{xii}), e são, sem
dúvida, seus contos mais conhecidos, devido em muito ao prêmio recebido
por Akira Kurosawa, pelo filme \textit{Rashômon} no Festival
Internacional de Cinema de Veneza de 1951. Embora nomeie o filme, o
primeiro conto é pouco utilizado no filme, servindo, entretanto, como
poderoso espaço simbólico da ação, sendo que a discussão fundamental
encontra"-se nos vários depoimentos em primeira pessoa de envolvidos num
crime ocorrido ``dentro do bosque''. Alguns críticos associam a obra como
metonímia de julgamentos de crimes de guerra ocorridos após a Primeira
Guerra Mundial, no qual os depoimentos dos acusados todos, analisados
individualmente, parecem indicar sempre sua própria inocência. Ora, um
dos recursos máximos da tradição literária japonesa é \textit{honkadori} 
(``tirar de um poema original''), alusão ou reinterpretação de trechos,
versos, trama de autor respeitado. Akutagawa, quando retira de uma
coletânea compilada oito séculos antes pequenas cenas ou relatos
sucintos, estes são deliberadamente transformados em obra sua, sendo 
utilizadas como motor para sua discussão contemporânea acerca da ética;
e Kurosawa, ao retomar e adaptar dois contos seus, também se apropria
das reflexões desenvolvidas, mas, criando novos personagens, reitera
sua confiança final no ser humano, sentido ausente nos originais.

Encontram"-se representadas nesta coletânea três narrativas do grupo de
temas cristãos: \textit{Ogata Ryôsai oboegaki} (Memorando ``Ryôsai Ogata'', 
1917), \textit{Ogin} (Ogin, 1923) e
\textit{Hokyôninno shi}  (O mártir, 1918).  Em
1534, Ignácio de Loyola e mais sete missionários formam a Companhia de
Jesus, que correu o mundo em prol da catequização cristã, tendo
atingido as terras da ilha de Kyûshû, onde se passam os três contos de
Akutagawa. Aos leitores do Brasil e de Portugal, é de especial
interesse este contato cultural, que deixou marcas, ainda que poucas,
até na língua japonesa: \textit{kirishitan} para ``cristão'',
\textit{bateren} para ``padre'', \textit{kurusu} para ``cruz'',
\textit{harureya} para ``aleluia'', \textit{iruman} para ``irmão'',
\textit{inheruno} para ``inferno'', \textit{bapuchizumo} para ``batismo''
e muitas outras de cunho mais geral (\textit{kasutera} para ``bolo pão
de ló'', \textit{tenpura} para ``tempero, ou seja, fritura'',
\textit{zubon} para ``pantalonas do tipo gibão''). Entretanto, para além
dos empréstimos linguísticos, são dignos de nota os embates filosóficos 
e a tensão entre o sistema de valores corrente, de fundo essencialmente
confucionista e medieval, contra uma nova religião centrada no
estrangeiro, na distante Roma. Pior que isso: em vez de um imperador e um
xógum na terra próxima, um papa longínquo e um Deus único no interior
de seus próprios corações. Proibido em todo o território japonês em
1587, o catolicismo permanecerá uma corrente subterrânea até 1889, 
quando será permitido pelos novos dirigentes políticos e se tornará
motivo de grande reflexão filosófica por parte também de muitos escritores.

Dois dos contos aqui traduzidos pertencem ao grupo das ``coisas de Edo''
(atual Tóquio): \textit{Gesaku zanmai} (Devoção à literatura popular, 1917), 
que trata de um tema muito caro a Akutagawa: a escrita
enquanto arte e conhecimento em conflito com sua veiculação enquanto
entretenimento para citadinos da Cidade Baixa de Edo, ávidos de
participarem de uma cultura ``alta'', representada pelo escritor mais
prolífico do momento, Bakin Takizawa  (1767--1848), de vertente
notadamente chinesa e confucionista; e \textit{Karenoshô}  (Terra morta, 1918), 
que focaliza o momento da morte do amado poeta Bashô Matsuo (1644--94), 
cercado por seus discípulos, qual cena de morte de
Buda, de vertente notadamente japonesa e budista. Em ambos os contos, 
Akutagawa mostra grandezas e mesquinharias que subjazem às apreciações
estéticas de mestres e discípulos, e também de leitores, tanto no caso
de prosa de modelos importados quanto de poesia autóctone.

O período Meiji (1868--1912) aparece como tema no grupo que trata das
ideias da grande abertura ao Ocidente, quando se acreditava que
perseguir um ``iluminismo'' e uma crença em um progresso nacionalista e
tecnológico colocaria o Japão entre as grandes potências mundiais:
lê"-se em \textit{Butôkai} (O baile, 1912) o encontro entre
o escritor Pierre Loti, então oficial da Marinha Francesa, e Akiko,
``jovem senhorita de uma família eminente'', num baile encenado em
Rokumeikan, palácio governamental de estilo arquitetônico
minuciosamente ocidental que caracteriza a inserção de robustas parede:
de pedra em oposição à clássica leveza da construção nipônica, no qual
se dança o \textit{Danúbio Azul} em vestuário de gala à Luís \textsc{xv}.

Como se nota pelas datas de sua composição, os contos não foram escritos
segundo uma agenda determinada, cronologicamente, como parece sugerir a
ordem em que se encontram nesta coletânea, organizada de modo mais
didático. Assim, passando"-se pelos períodos históricos japoneses Heian,
Kamakura"-Muromachi, Edo e Meiji, chegamos às épocas vividas por
Akutagawa: Meiji (1868--1912) e Taishô (1912--1926). Atente"-se, na
leitura, o extremo cuidado tomado pelo autor não somente com
anacronismos históricos (sua pesquisa é minuciosa em termos de imagem,
vestuário, objetos de cena, cargos, fatos verídicos), mas 
principalmente, para desespero dos tradutores, com a utilização de um
vernáculo adequado a cada gênero e época. São ressuscitados detalhes
arquitetônicos, cargos e peças de vestuários antigos e expressões
linguísticas que revelam um pesquisador atento que, embora contrário ao
realismo enquanto escola literária, dela absorve certo ``senso de realidade''.

De fato, poder"-se"-ia agora tratar do primeiro grupo de obras, mais um 
gênero temático de escritos relativos ao seu tempo: \textit{gendaimono} 
(``coisas contemporâneas''), no qual se inseririam
\textit{Yasukichino techôkara} (Passagens do caderno de notas de Yasukichi, 
1923) e \textit{Aru ahôno isshô} (A vida de um idiota, publicado em 1927), 
estando ausente desta coletânea o conto também póstumo e notável \textit{Haguruma} 
(Engrenagens, 1927), nos quais podemos seguir as crises de
forte enxaqueca do protagonista, seu alter"-ego. Muitas vezes referidos como
relatos autobiográficos, os escritos deste grupo surpreendem não poucos
leitores ocidentais em sua porção indissociável de escritor e autor,
segundo o gênero controverso do ``romance do eu'', que o próprio
escritor abominava pela demonstração de exibicionismo de seus exegetas.
 É, contudo, chocante que o suicídio revelado na obra literária se
tenha transformado em verdade, num processo de amálgama de arte e vida
raras vezes atingido no Ocidente, com cálculo e determinação, a ponto
de alguns afirmarem que \textit{A vida de um idiota} não é obra de ficção e,
sim, testamento literário. Sobretudo, o cultivo de uma argúcia
abrangente fez com que Akutagawa se debruçasse sobre várias épocas da
literatura japonesa, a fim de fazer reviver, para aqueles seus leitores
de um Japão ``moderno'', a tradição sempre permanente e a ser cultivada,
num enorme esforço de abertura de pensamento e emoção para interpretar
seu momento presente.

Compreendido, como se disse, por alguns leitores como um ``testamento
literário'', por outros como ``carta suicida'', \textit{A vida de um idiota} 
escapa de qualquer classificação: dividido em 51 trechos, compõe"-se de
pequenas joias, instantâneos oblíquos e intensos de sua situação de
vida, na qual abundam seus dramas pessoais --- a mãe louca, a modesta casa
a sustentar, a tia solteirona a controlar o casal, a esposa silenciosa
mas amada, os pais adotivos a manter, a amante inconstante, a amante
louca, a amante de duplo suicídio irrealizado, o egocêntrico escritor
Tanizaki, os filhos desprotegidos --- e literários ou artísticos: o
companheiro Masao Kume, a excelência literária de Sôseki Natsume, o
egocentrismo de Jun'ichirô Tanizaki, o ceticismo de Anatole France, o
racionalismo de Voltaire, a loucura colorida de Van Gogh, a pintura
geométrica de Cézanne, as confissões de Strindberg e Rousseau, a música
de Mozart, a poesia de Goethe, as últimas palavras de Radiguet,
permeados em cada linha por desespero sutil e crescente.

Outra forma de classificação dos escritos contemporâneos seria
\textit{Yasukichimono}, nome do protagonista, alter"-ego do escritor. 
Obras como \textit{Daidôji Shinsukeno hansei} (A primeira metade da vida de 
Shinsuke Daidôji, 1925) e \textit{Tenkibo} (Registro do passado, 1926) também
compartilham de introspecção pessoal, especialmente de uma sensação de
desespero que, entretanto, também se reflete em obras como
\textit{Kappa}  (O espírito das águas, 1927) e no ensaio 
\textit{Bungeitekina, amarini bungeitekina} (Literário, por demais literário, 1927). 

Teoricamente, poder"-se"-ia afirmar, portanto, que os escritos de
Akutagawa são disseminações de seu conflito profundo entre uma vida de
sucesso material e uma tendência profundamente melancólica e
sedenta de uma linguagem moral. Com efeito, mesmo nas narrativas curtas
de cunho cômico, sempre se encontra subjacente um conceito moral; mesmo
as narrativas para crianças (\textit{dôwa}), grupo ainda pouco
avaliado, com suas mensagens positivas contra o egoísmo essencial do
ser humano, podem ser enfocadas tendo como referência o período da corte, ou dos
samurais, ou de céus fantásticos, com seus santos e ogros.

\textit{Kappa} (1927), por exemplo, é protagonizado por 
um ser folclórico. Um louco narra sua experiência no
país dos Kappa, caracterizado como uma sociedade de homens, onde há
muitos costumes e hábitos curiosos: a justiça e a moral dos homens são
piadas para eles. No momento do parto, a vontade do embrião é
respeitada por todos; a diversão cotidiana é maltratarem"-se pais e
filhos, maridos e esposas e irmãos; no amor, é a fêmea quem corre atrás
do macho; o capitalismo alcança um alto progresso, a ponto de controlar
a política; todos sabem que a política é uma mentira; na religião, nem
o ancião do grande templo acredita em Deus. Embora o ser folclórico
pudesse ser compreendido como ``fantástico'', é"-o tanto quanto Pierre
Loti e suas mocinhas japonesas vestidas com saias"-balão sob requintados
\textit{chandeliers}.

Conforme aponta Seiichi Yoshida, historiador da literatura
japonesa, responsável por uma preciosa edição anotada da obra de Akutagawa, a
principal formação deste inclui não só os clássicos chineses e
japoneses mas também os ocidentais, tendo escrito não somente obras
literárias no sentido pleno de ``ficção'', como também crítica e
comentários filosóficos, não necessariamente em compartimentos estanques.

Kan Kikuchi, amigo de longa data, estabeleceu o Prêmio Akutagawa em
1935 com o intuito de lhe preservar a memória, o que de fato ocorreu,
pois até hoje é o mais cobiçado pelos novos escritores, que têm
asseguradas suas publicações --- efeito do poder do dragão?

\section{Sobre o gênero}

Ryūnosuke Akutagawa é considerado até hoje o ``pai do conto japonês'', pela sensibilidade e habilidade que tinha em compor histórias suscintas mas ricas em detalhes e questionamentos sobre a natureza humana. Apesar de usar largamente da técnica literária japonesa denominada \textit{honkadori} --- alusão ou reinterpretação de trechos de textos clássicos, como já mencionado ---, seus contos são profundamente modernos, aproximando"-se de outros representantes máximos do conto moderno como Edgar Allan Poe, Maupassant, Truman Capote, Jorge Luís Borges, Tolstói, Hemingway, Tchekhov e kafka.

O autor japonês pode ser considerado, ao lado dos nomes acima citados, um mestre do conto ao abordar, em um texto muito conciso, ``uma realidade infinitamente mais vasta que a do seu mero argumento'', influindo ``em nós com uma força que nos faria suspeitar da modéstia do seu conteúdo aparente, da brevidade do seu texto''.\footnote{\textsc{CORTÁZAR}, Julio. \textit{Valise de cronópio}. São Paulo: Editora Perspectiva, 2008, p.\,155.} Em uma narrativa como ``Dentro do bosque'', por exemplo, Akutagawa se vale de elementos de uma das crônicas do clássico \textit{Konjaru Monogatari} para abordar uma questão central à compreensão do homem: a composição multifacetada de suas verdades.

Apesar da aparente banalidade do argumento --- o assassinato de um samurai ---, que poderia muito bem se desenvolver em uma narrativa policial, o autor desenvolve o tema em profundidade, em contraposição à aparente concisão narrativa. Realiza plenamente, assim, o que Cortázar define como o gênero do conto:

\begin{quote}
Um escritor argentino, muito amigo do boxe, dizia"-me que nesse combate que se travra entre um texto apaixonante e o leitor, o romance ganha sempre por pontos, enquanto que o conto deve ganhar por \textit{knock"-out}. É verdade, na medida em que o romance acumula progressivamente seus efeitos no leitor, enquanto que um bom conto é incisivo, mordente, sem trégua desde as primeiras frases. Não se entenda isto demasiado literalmente, porque o bom contista é um boxeador muito astuto, e muitos dos seus golpes iniciais podem parecer pouco eficazes quando, na realidade, estão minando já as resistências mais sólidas do adversário.
Tomem os senhores qualquer grande conto que seja de sua preferência, e analisem a primeira página. Surpreender"-me"-ia se encontrassem elementos gratuitos, meramente decorativos. O contista sabe que não pode proceder acumulativamente, que não tem o tempo por aliado; seu único recurso é trabalhar em profundidade, verticalmente, seja para cima ou para baixo do espaço literário.\footnote{Ibid., p.\,152.}
\end{quote}

\subsection{Nota sobre a tradução}

Todas as notas de rodapé são dos tradutores.		%\EP[1]
Para a transcrição dos termos japoneses, adotou"-se o sistema de
romanização Hepburn, por ser o mais comumente utilizado no Japão, na
Europa e nos Estados Unidos. Eis algumas diferenças em relação à
leitura do português:

--- \textbf{â, ê, î, ô, û} são vogais longas. O acento circunflexo sobre as vogais
indica que sua pronúncia é alongada. ``Rashômon'' lê"-se ``Ra"-xo"-o-mo"-n''.

--- \textbf{ya, yo, yu} são encontros vocálicos mais breves que ia, io, iu.

--- \textbf{ha, he, hi, ho} --- o ``h'' dessas sílabas é sempre aspirado.

--- \textbf{cha, che, chi, cho, chu} --- leem"-se, respectivamente, tcha, tche, tchi,
tcho, tchu.

--- \textbf{ge, gi} --- leem"-se gue, gui. \textit{Ogin} lê"-se ``Oguin''.

--- \textbf{ra, re, ri, ro, ru} --- trata"-se de ``r'' brando, mesmo em início
de vocábulo. O ``ra'' de ``Rashômon'' pronuncia"-se como o ``ra'' de ``cera''.

--- \textbf{sa, se, so, su} intervocálicos leem"-se ça, ce, ço, çu. 
\textit{Gesaku} lê"-se ``guessaku''.

--- \textbf{n} --- trata"-se de sílaba e não de nasalização de sons anteriores ou
posteriores. ``Rashômon'' lê"-se ``Ra"-xo"-o-mo"-n''.

 --- \textbf{consoantes dobradas} correspondem a ligeira suspensão da sílaba. \textit{Kappa} lê"-se ``ka [suspensão breve] pa''.

