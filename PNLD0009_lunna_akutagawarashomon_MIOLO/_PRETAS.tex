


\textbf{Ryûnosuke Akutagawa} (Tóquio, 1892--\textit{id.} 1927) é o grande expoente 
do moderno conto japonês. Nasceu no bairro Kyôbashi, na ``cidade
baixa'', filho de um pai extremamente rígido e de uma mãe louca, sob a égide de 
``filho do Dragão''. Adotado pelos tios maternos, mais cultos,
deixou de utilizar o sobrenome do pai, Niihara. Ainda criança, entrou
em contato com traduções de Ibsen e Anatole France. Na primeira
juventude, traduziu Yeats, e especializou"-se em Literatura Inglesa na
Universidade Imperial de Tóquio, período em que se tornou discípulo do
escritor japonês Sôseki Natsume (1867--1916) e passou a escrever os
primeiros de seus cento e cinquenta textos curtos em prosa. Aos 26
anos, casou"-se com Fumiko Tsukamoto, com quem teve três filhos.
Na década de 1920, sua obra passa a revelar fortes traços
autobiográficos: a loucura, o suicídio, a ética cristã, os antigos 
costumes japoneses e a modernização do período Meiji (1868--1912),
num profundo conflito em busca de uma solução moral definitiva. 
Suicidou"-se aos 35 anos com uma dose de Veronal. 
        
\textbf{Rashômon e outros contos} reúne dez contos de diversos períodos da
breve existência do autor: \textit{Rashômon} (1915) e \textit{Dentro do bosque} (1922) 
retratam a cultura de Heian (atual Quioto). Em \textit{Memorando ``Ryôsai Ogata''} (1917), 
\textit{Ogin} (1923) e \textit{O mártir} (1918), a temática cristã é o fio condutor. 
\textit{Devoção à literatura popular} (1917) e \textit{Terra morta} (1918) têm como pano de fundo 
a cultura de Edo (atual Tóquio). A abertura do Japão para o Ocidente no período 
Meiji compõe o enredo de \textit{O baile} (1912). Por fim, dois contos de caráter autobiográfico, 
do final da vida de Akutagawa: \textit{Passagens do caderno de notas de Yasukichi} (1923) 
e \textit{A vida de um idiota} (1927). Esta nova edição, com texto revisto pelas tradutoras, 
conta ainda com nova introdução e acréscimo de notas.      
\pagebreak 

\textbf{Madalena (Natsuko) Hashimoto Cordaro} é docente de Literatura Japonesa na 
\textsc{fflch"-usp} desde 1990 e tem formação em Artes Plásticas (\textsc{eca"-usp} 
e Washington University), Letras (Português"-Espanhol"-Japonês) e Filosofia (doutorado em Estética). 
Dedica"-se à pesquisa de arte e literatura japonesas do período Edo (1603--1867), 
à tradução de autores japoneses e à produção de obras visuais.
 
\textbf{Junko Ota} é docente e pesquisadora de língua japonesa na \textsc{fflch"-usp} 
desde 1988. Bacharel em Letras (Japonês e Português) e doutora em Linguística pela \textsc{usp}, é 
também mestre em Letras"-Japonês pela Osaka University (Japão). Tem atuado no ensino e pesquisa 
de língua japonesa, além de dedicar parte de seu tempo a traduções do japonês para o português.   	


