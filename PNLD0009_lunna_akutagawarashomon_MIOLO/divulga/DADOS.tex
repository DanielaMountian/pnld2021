\titulo{rashômon e outros contos}
\autor{Akutagawa}
\organizador{Organização e tradução}{Madalena Hashimoto Cordaro e Junko Ota}
\isbn{978-85-7715-094-6}
\preco{20}
\pag{210}
\release{\textsc{O livro traz dez contos} escritos por Ryunosuke Akutagawa,‭ ‬considerado o principal representante do moderno conto japonês.‭ ‬Esta nova edição,‭ ‬cujo texto foi revisto pelas tradutoras Madalena Hashimoto Cordaro e Junko Ota,‭ ‬reúne histórias de diferentes períodos da curta vida do autor,‭ ‬que cometeu suicídio aos‭ ‬35‭ ‬anos de idade,‭ ‬e durante a qual escreveu mais de‭ ‬150‭ ‬contos,‭ ‬além de poemas.‭ ‬Em‭ \textit{‬Rashômon e outros‭ ‬contos},‭ ‬Akutagawa aborda desde aspectos da cultura ancestral japonesa até a fase contemporânea de seu tempo,‭ ‬a da abertura de seu país à cultura ocidental,‭ ‬incluindo a temática cristã e a crença de que progresso tecnológico alçaria o Japão ao grupo das grandes potências mundiais.‭ ‬Dois de seus contos escritos entre‭ ‬1923‭ ‬e‭ ‬1927,‭ ` `‬Passagem do caderno de notas de Yasukichi'' e ``A vida de um idiota'',‭ ‬são tidos como autobiográficos.‭ ‬O conto ``Rashômon'' serviu de inspiração para o filme homônimo e uma das obras-primas do famoso cineasta japonês Akira Kurosawa,‭ ‬premiado em‭ ‬1950‭ ‬no Festival de Veneza.‭  

\textbf{Ryunosuke Akutagawa} nasceu em março de‭ ‬1892,‭ ‬em Tóquio,‭ ‬filho de um comerciante que o entregou à adoção para um tio,‭ ‬constatado o enlouquecimento da mãe.‭ ‬Desde cedo o escritor teve contato com traduções de Ibsen e Anatole France.‭ ‬Ainda na adolescência traduziu Yates e formou-se em literatura inglesa pela Universidade Imperial de Tóquio.‭ ‬Nesse período tornou-se discípulo de Sôseki Matsuname,‭ ‬um dos grandes nomes da literatura de seu país e seu grande incentivador.‭ ‬Vítima de freqüentes alucinações,‭ ‬decidiu‭ ‬se suicidar em julho de‭ ‬1927,‭ ‬aos‭ ‬35‭ ‬anos.‭ ‬Em carta escrita antes do gesto extremo,‭ ‬Akutagawa confessou ser a morte a única forma de encontrar a paz.‭  

As tradutoras e organizadoras da obra,‭ ‬Madalena‭ (‬Natsuko‭) ‬Hashimoto Cordaro e Junko Ota são docentes da Universidade de São Paulo‭ (‬USP‭)‬.}


‎  


