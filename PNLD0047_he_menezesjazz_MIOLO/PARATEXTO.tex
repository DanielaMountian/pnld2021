\chapter{A pesquisa folclórica de Mário de Andrade}

\section{Sobre o autor}

Mário Raul Morais de Andrade teve a sua aurora no dia 9 de outubro de
1893, na cidade de São Paulo, no número 320 da rua que carrega
simbolicamente a projeção e permanência de sua produção intelectual e
artística. Desde a infância, demonstrou talento para a música,
destacando"-se como exímio pianista, levando"-o a ser matriculado no
Conservatório Dramático e Musical de São Paulo ao completar 18 anos.
Como autodidata, se dedicou também ao estudo da literatura e outras
artes, mas foi por meio da poesia que se lançou como escritor, atividade
a qual se dedicou intensamente até o fim da vida.

Seu primeiro livro foi \emph{Há uma gota de sangue em cada poema}
(1917), publicado sob o pseudônimo de Mário Sobral. Nos textos de
estreia, ainda estão presentes as influências das tendências estéticas
da virada do século \textsc{xix} ao \textsc{xx}, como o simbolismo e o parnasianismo,
embora estejam esboçadas as questões que permeariam toda a sua obra
posterior. Poucos anos depois, Mário de Andrade se engajou no
modernismo, movimento que, em sua fase inicial, se oporia radicalmente a
essas estéticas anteriores, influenciado pelas vanguardas artísticas
europeias.

Em 1922, Mário publicou o livro de poemas \emph{Pauliceia desvairada},
sua obra"-manifesto, no mesmo ano em que trabalhava na organização de um
dos eventos mais importantes da vida intelectual e cultural brasileira
no século \textsc{xx}, a Semana de Arte Moderna, ocorrida entre os dias 11 e 18
de fevereiro de 1922, no Theatro Municipal de São Paulo, e que contou
ainda com a participação de artistas como Oswald de Andrade, Anita
Malfatti e Heitor Villa"-Lobos. Com o ensaio \emph{A escrava que não é
Isaura} (1925), Mário lançou mais um manifesto, mas, desta vez, por meio
de uma reflexão mais séria sobre a poesia moderna no Brasil.

No mesmo decênio, Mário de Andrade começou a se embrenhar pelos caminhos
da prosa de ficção. Como contista, publicou o livro \emph{Primeiro
andar} (1926), mas foram seus romances \emph{Amar, verbo intransitivo}
(1927) e \emph{Macunaíma, o herói sem nenhum caráter} (1928) que o
destacaram como prosador. Também se dedicou à crítica de artes
plásticas, de música e de literatura, além de ter se dedicado a
importantes estudos sobre a cultura popular nacional.

Como missivista, Mário de Andrade foi um dos mais prolíficos
intelectuais de seu tempo. A correspondência do escritor paulista é
volumosa e diversa tanto em sua dimensão numérica quanto no que diz
respeito aos temas e interlocutores envolvidos no diálogo epistolar, de
escritores a pintores, de folcloristas a políticos. As cartas de Mário
não são apenas os bastidores da intimidade dos seus correspondentes, mas
também espaço de memória da formação e transformação da vida cultural,
intelectual e política do Brasil no século \textsc{xx}.

Em 22 de fevereiro de 1945, um infarto no miocárdio abreviou sua vida.
No entanto, a importância de sua obra, já reconhecida por seus
contemporâneos, se ampliou ainda mais com o passar dos anos, sendo hoje
objeto de estudos de variadas perspectivas críticas e suscitando um
interesse cada vez maior dos leitores.

\section{Sobre a obra}

A proposta central desse livro é recuperar o autor modernista Mário de
Andrade a partir de sua dimensão de pesquisador etnográfico. O trabalho
de Menezes resgata dois textos escritos por Mário ``Dansa de Santa
Cruz''(1933) e ``Moçambique''(1935) estudos feitos nos anos 1930 sobre
as tradições culturais/musicais do interior do país. Recuperando e
reeditando estes sem interferir na construção gramatical adotada pelo
autor é notável a forma como a linguagem particular desse material é
explorada ao longo da obra. Outro ponto a destacar acerca da linguagem
do livro é a sua própria constituição enquanto um disco-livro. O
material em áudio reúne a fonografia experimental coletada por Mário em
suas viagens e também conta com gravações novas, produzidas por Menezes
e um conjunto de músicos. O texto escrito se complementa com o material
fonográfico, pois um álbum de músicas é lançado juntamente com a
publicação da obra, sendo possível enxergar essa interface musical como
uma parte indissociável da totalidade da publicação.

O livro pode ser dividido de duas maneiras, respeitando a ordem proposta
pelo organizador. Na primeira parte, apresenta dois trabalhos publicados
por Mário de Andrade, estudos feitos a partir de viagens que buscavam
registrar manifestações culturais em festejos populares de duas
localidades no interior de São Paulo(Vilarejo de Carapicuíba e Cidade de
Santa Isabel). A segunda parte contém três ensaios elaborados por
Menezes e outros autores, trabalhos que além de dialogar com os dois
textos de Mário também exploram outras faces do autor. Os ensaios
comentam a sua trajetória na vanguarda modernista, inserindo-o em seu
contexto histórico e discutindo de forma crítica a sua contribuição no
campo cultural brasileiro do início do século \textsc{xx}. Aprofundaremos em
seguida essas duas grandes dimensões que a obra nos apresenta, Mário
enquanto um agente da narrativa e Mário enquanto objeto da narrativa.

\subsection{Dansa da Santa Cruz}

O primeiro capítulo do livro se constitui no relato do festejo popular e
religioso da ``Dansa'' da Santa Cruz, assistido e registrado por Mário
de Andrade no vilarejo de Carapicuíba (atual município de Cotia-SP) em 3
de Maio de 1935, dia do início do evento. Mário destaca minuciosamente
os diversos aspectos que rondam esta manifestação popular, se preocupa
em evidenciar as particularidades rítmico-melódicas, coreográficas,
espaciais e histórico-sociológicas que ao seu ver organizam o fenômeno
observado. Ele se atém às etapas das danças e das canções, nos espaços
que ela ocorre e qual o sentido próprio que essa ``dansa'' assume no
vilarejo.

O evento religioso popular do dia de santa cruz, sob os olhos
etnográficos de Mário, ganha uma ampla dimensão. Uma delas é o aspecto
musical do evento. A ``dansa'' é acompanhada por um cortejo de
cantadores e instrumentistas. Notável a sensibilidade de captar os
aspectos rítmico-melódicos particulares presentes nas canções. As
músicas são puxadas pelos cantadores ``tradicionais'', mais experientes,
que segundo o texto guardam estes versos desde tempos imemoriais.

Mário insere no corpo de seu texto as partituras das melodias,
ilustrações de instrumentos e os versos(em quadras) que são cantados ao
longo dos dias. Neste ponto é interessante notar a riqueza do conteúdo
daquilo que era cantado. O autor salienta a presença das quadras de
caráter ``profano'' e cômico em meio aos versos religiosos.

Os improvisos nas melodias e as imperfeições vocais chamam a atenção de
Mário de Andrade. Ao seu ver, o cantar sem melodia fixa é um aspecto
``\emph{de cantar muito nosso, especialmente da parte central do
país}''\footnote{MENEZES,2020, p.13.}, aqui podemos notar um claro
exemplo do destaque etnográfico. Também podemos perceber esse tipo de
ênfase nos momentos em que o autor percebe a herança indígena na dança
``\emph{Mas todos estão delirando de\ldots{} prazer paulista, sêco, e se
diria tristonho. É a base de índio"}\footnote{MENEZES, 2020, p.23.} .
Salienta que essa tradição indígena se manteve viva e conservada por
conta do isolamento e atraso do vilarejo.

O texto deve ser lido enquanto uma fonte histórica, necessitando de
certo cuidado em acatar os apontamentos de Mário enquanto verdades
plenamente científicas. O próprio autor reconhece as limitações de seu
relato em determinados momentos. Contudo, não se diminui a importância
do registro enquanto documento rico em análises da cultura popular
brasileira.

\subsection{Moçambique}

A dança-dramática ``Moçambique'' é mais uma manifestação popular
registrada por Mário, desta vez na cidade de Santa Izabel em junho de
1933. O autor mantém a sua abordagem etnográfica e detalhista. Pontua,
logo de início, o seu entendimento da dança como algo fortemente marcado
pelas influências africanas. Faz comparações desta com outros bailados
como os Congos, Caiapós e até mesmo enxergando similitudes com o
Maracatu pernambucano. É interessante perceber que, apesar de enxergar o
Moçambique pela forte influência da matriz negra, o autor faz questão de
evidenciar que essa tradição já tinha sido assimilada por outros grupos,
como é possível notar em sua fala se referindo aos integrantes do
cortejo ``na maioria eram totalmente brancos, caipiras
legítimos''\footnote{MENEZES, 2020, p.26.}. Outro elemento sincrético é
o fato dessa manifestação ocorrer durante as celebrações da festa
católica do Divino Espírito Santo.

O texto se organiza em algumas seções, onde Mário explora os personagens
do cortejo, suas indumentárias, seus instrumentos e as coreografias. A
seção coreográfica é a mais extensa, pois sete danças diferentes são
identificadas durante a festa. Com isso, podemos supor que este foi um
aspecto que chamou muita atenção do autor. Cada uma das descrições é
acompanhada de versos e partituras referentes às melodias executadas.
Mário de Andrade chega até mesmo a criar uma notação gráfica para tentar
ser mais didático em seu relato das coreografias.

\subsection{Mário de Andrade, Moçambique e a Santa Cruz}

O ensaio de Enrique Menezes busca discutir e aprofundar a questão
cultural presente nos relatos de Mário. Menezes entende a necessidade de
atualizar certas discussões, partindo de uma investigação que dialoga
com diversos autores. Em seu texto a temática dos estudos culturais
brasileiros passa por uma chave interpretativa histórica, partindo para
uma discussão que explora o contexto histórico no qual Mário de Andrade
escreveu sua obra e de que forma a intelectualidade enxergava a cultura
no Brasil. Também é apresentado um debate musicológico em conexão com os
domínios culturais da linguagem, da literatura, das artes dramáticas e
visuais, aprofundando as manifestações registradas pelo autor
modernista.

A discussão inicia trazendo à tona o paradigma das ``três raças''.
Durante o século XIX e início do XX muitos intelectuais buscavam
entender a fisionomia da cultura brasileira a partir das influências das
três matrizes culturais que compunham a sociedade brasileira: a
negra/africana, branca/europeia e vermelha/ameríndia. Menezes resgata a
visão de teóricos que se inspiraram nesse mito fundamental como Olavo
Bilac, Câmara Cascudo, Von Martius e Cornélio Pires. O autor entende que
Mário era um homem do seu tempo e que de certa forma era influenciado
por essas concepções.

A intelectualidade do início do século passado enxergava os espaços
interioranos como verdadeiros repositórios da cultura brasileira ainda
intacta, diferente dos cenários urbanos que estariam impregnados pelas
influências estrangeiras. Menezes vai explorar criticamente a visão dos
intelectuais acerca dos locais que guardam a essência da brasilidade,
espaços de certa forma idealizados e estagnados no tempo. Contudo, o
autor reconhece as boas intenções desses agentes que procuravam
registrar e preservar as manifestações culturais que corriam risco de se
perderem no processo de modernização brasileira daquele momento.

Uma grande parcela do texto é direcionada para a história dos estudos
musicais brasileiros, evidenciando a contribuição de Mário nessas
produções. Menezes coloca o autor modernista como um pioneiro na
pesquisa da influência africana na música brasileira, destacando-o como
contraponto às elaborações racistas e eugenistas da época.

Enrique Menezes busca expor as variadas dimensões das pesquisas de Mário
de Andrade acerca das danças paulistas. Um desses aspectos é a relação
orgânica existente entre o domínio musical e o da dança, do movimento, a
expressão da música pelo prisma dos corpos. Mário não separava a música
da sua dimensão festiva e gestual, contrapondo a visão erudita europeia
que menosprezava os elementos rítmicos, mais atrelados ao aspecto
musical dançante.

São resgatadas as matrizes africanas para destacar a íntima relação
entre dança e música nas tradições populares brasileiras. Menezes faz um
estudo minucioso da carga semântica, cultural e histórica dos
instrumentos musicais e dos nomes das danças. Ele parte para um estudo
integrado da história africana a partir dessa abordagem musical,
discutindo a transformação e ressignificação dos instrumentos originais
africanos na realidade dos povos escravizados trazidos para o Brasil.

O autor também enfatiza a presença do ritmo cruzado(polifônico) na
tradição musical africana e brasileira, explorando o sentido social
dessa sonoridade, como podemos notar no trecho: ``\emph{A estrutura
musical construída com componentes cruzados, na qual a percepção do
tempo e do espaço é multifacetada, convida o ouvinte a participar
ativamente da música através da possibilidade de perceber as diferentes
métricas que formam o todo'' (MENEZES, 2020, p.82).} Menezes explora a
forma como a música expressa as visões de mundo dessas culturas.

\subsection{Paulicéia desordenada -- Modernismo e poder}

Este quarto capítulo lança um olhar para a particularidade do cenário
paulista das décadas de 1920 e 1930, buscando relacionar as mudanças
políticas e sociais do período com o nascente movimento modernista. O
texto faz uma análise das relações existentes entre os artistas da
vanguarda paulista com os agentes políticos da época, salientando as
contribuições de Mário de Andrade nas políticas culturais que começaram
a surgir naquele momento.

O texto inicia destacando a singularidade do intenso e desordenado
crescimento demográfico da cidade de São Paulo no início do século XX.
Um cenário de convergência caótica de diversas populações caipiras,
ex-escravizadas e imigrantes que se deslocavam para a paulicéia. Os
autores pontuam a contradição do processo, que colocava em cena uma
massa empobrecida provinda de diversas partes com uma elite governante
oligárquica que não pensava em projetos urbanos integradores. A
influência dessas questões nas artes também é um ponto notável do
capítulo. As artes aqui são enxergadas a partir de duas esferas, uma
popular, manifesta nas expressões coletivas do cotidiano das classes
populares, e a arte individualizada de um grupo seleto de artistas e
intelectuais.

O movimento modernista surge como expressão dessa esfera artística
seleta. Era o momento no qual a elite intelectual paulistana começava a
olhar para a situação brasileira, buscando ``encontrar o povo''. São
destacadas as colaborações da elite econômica nessa perspectiva que
enxerga os saberes do povo como elementos essenciais na formação de uma
identidade nacional. Os autores pontuam a singularidade do mecenato que
patrocinava a ``aventura'' modernista, elencando figuras importantes da
classe dominante paulista como Freitas Valle e Olívia Guedes. As
residências desses personagens foram o reduto protetor da nascente arte
de vanguarda. Pires e Menezes salientam que essa característica é algo
particular do Brasil, pois, na Europa, as vanguardas eram desprezadas
pela elite tradicional. A contradição brasileira se dá nessa aliança
entre uma classe dominante conservadora bancando um projeto inovador.

A análise prossegue se aprofundando na cena político-institucional,
elencando os principais acontecimentos do período como a criação do
Partido Democrático(visto como um partido de oposição à oligarquia
cafeicultora) e a Revolução Constitucionalista de 1932. Eventos estes
que marcaram o momento de transformação e renovação das décadas de 1920
e 1930. Um período em que o Estado de São Paulo passou a criar diversas
instituições públicas científicas e de pesquisa com a intenção projetar
a sua influência cultural no cenário nacional, após as derrotas
políticas de 1932.

É interessante a forma como o texto recupera a presença de Mário de
Andrade nesse meio, destacando a trajetória do autor no movimento
modernista e as suas ideias para as políticas culturais. Mário e a sua
sensibilidade de enxergar as potencialidades das ideias modernistas no
plano político e social se torna, em 1935, o primeiro diretor do
Departamento de Cultura do município de São Paulo. Para o escritor, o
Estado devia assumir uma posição de fomento às artes e às políticas
culturais de forma ampla. Ele forjava uma nova concepção de cultura e de
políticas públicas para essas áreas, se alinhando a uma perspectiva de
expansão do acesso à cultura para as classes populares, que
historicamente foram excluídas de ações desse tipo.

\subsection{São Paulo fonografado}

O último capítulo da obra explora a direção de Mário de Andrade no
Departamento de Cultura. Os autores se debruçam nas ações inovadoras do
departamento, seu impacto e legado. Destacam a importância dos projetos
fonográficos da Discoteca Pública Municipal para o registro e
preservação das manifestações culturais. Segundo o texto, as novas
políticas públicas nesse campo, alinhadas com os novos recursos
tecnológicos, impactaram diretamente o estudo da cultura popular
brasileira.

Binazzi e Menezes dão enfoque a essa importante instituição criada pelo
Departamento de Cultura em 1936. Apresentam as justificativas de Mário
para a criação da Discoteca Pública, salientando o papel do poder
público no fomento e na execução de projetos dedicados à cultura musical
brasileira. Mário de Andrade sentia uma pobreza na discografia nacional,
tanto erudita quanto popular, pontuando que as gravadoras do período não
se interessavam por esses objetos, pois seriam pouco rentáveis. O selo
fonográfico da Discoteca era visto como alternativa, ao oferecer
serviços de gravações etnográficas, mas também de artigos musicais dos
concertistas paulistas. Havia a intenção de criar uma ``biblioteca
sonora'' que viabilizasse o estudo dessas brasilidades, justamente em
uma conjuntura que se buscava modelar uma identidade nacional.

Para tornar mais claro o entendimento, os autores, nos apresentam de que
forma estavam organizadas as divisões da Discoteca. Os projetos eram
trabalhados por meio de três frentes: a Série Música Erudita, a Série
Arquivo da Palavra e a Série Folclore. A seção folclórica é apontada
como a mais rica em registros, com 33 horas de artigos gravados. É
justamente desta coleção que Enrique Menezes recupera o material
fonográfico presente em algumas faixas do projeto Jazz Rural. O capítulo
se aprofunda na explicação das singularidades das músicas originais que
compõem o disco-livro.

Mais um aspecto de destaque do texto é justamente Mário enquanto
entusiasta das novas tecnologias fonográficas. As gravações, segundo o
autor, ultrapassam as possibilidades da escrita musical, pois teriam a
capacidade de captar os timbres e as polifonias das manifestações. O
capítulo recupera escritos de Mário de Andrade sobre este ponto
\emph{``Basta verificar que exclusivamente por meio da fonografia, é que
podemos obter a coisa legítima''}\footnote{MENEZES, 2020, p.121}. Uma
sensibilidade que era tendência no cenário internacional. Projetos
semelhantes estavam sendo executados por instituições governamentais em
países como Alemanha, Áustria, França, Itália e Romênia. O autor
modernista é colocado enquanto um agente que buscava colocar o Brasil na
esteira desses estudos etnomusicológicos.

\section{Sobre o gênero}

O gênero do relato, em linhas fundamentais, define"-se como a narrativa em que um sujeito, inscrito em um determinado tempo histórico, debruça"-se sobre fatos, descrições e interpretações desse momento histórico no qual vive. Para o historiador francês Paul Veyne, o relato histórico segue uma forma similar à forma tradicional de escrever história, seguindo um \textit{continuum} espaço"-temporal.

Apesar dessa relativa unidade, o relato, considerado como uma forma de fazer história,
é parcial e subjetivo, pois não consegue apreender a globalidade dos acontecimentos, apenas
aquilo que está ao alcance do narrador e, mesmo isso, não de uma forma pura, mas filtrado pela sua subjetividade e pelos objetivos de seu relato.
Para Veyne, estaríamos assim quase próximos do romance:

\begin{quote}
A história é uma narrativa de eventos: todo o resto resulta disso. Já que é, de fato, uma narrativa, ela não faz reviver esses eventos, assim como tampouco o faz o romance; o vivido, tal como ressai das mãos do historiador, não é o dos atores; é uma narração,
o que permite evitar alguns falsos problemas. Como o romance, a
história seleciona, simplifica, organiza, faz com que um século
caiba numa página.\footnote{\textsc{veyne}, Paul. \textit{Como se escreve a história}. Brasília: Editora Universidade de Brasília, 1999, p.\,18.}
\end{quote}

Seguindo nessa linha de pensamento, podemos observar, com o historiador francês Marc Bloch, que o relato é apenas um ``vestígio'' da história, um pequeno pedaço do factual que, pela pena de um narrador, pôde"-se cristalizar no tempo e ser transmitido a gerações posteriores, sendo apenas uma das infinitas possibilidades de apreensão e compreensão de determinados fenômenos:

\begin{quote}
Quer se trate das ossadas
emparedadas nas muralhas da Síria, de uma palavra cuja forma ou emprego revele um
costume, de um relato escrito pela testemunha de uma cena antiga [ou recente], o que
entendemos efetivamente por documentos senão um ``vestígio'' quer dizer, a marca,
perceptível aos sentidos, deixada por um fenômeno em si mesmo impossível de captar?\footnote{\textsc{bloch}, Marc. \textit{Apologia da história}. Rio de Janeiro: Zahar, 2002, p.\,73.}
\end{quote}

Ao procurar e registrar as manifestações típicas da cultura popular brasileira, antevendo seu desaparecimento diante da modernização do país, Mário de Andrade produziu um relato único, que retratou as múltiplas faces do Brasil e de sua formação sociocultural, sem deixar de destacar seus problemas.
É na cultura popular, em suas festas, danças e cantos, que o escritor vai encontrar a chave para compreender a nossa identidade.

Ao valorizar o ritmo da oralidade na linguagem escrita e a cultura
popular, os escritos de Mário de Andrade dialogam com a produção dos
principais escritores do século \textsc{xx}, como Jorge Amado e Guimarães Rosa.

Como analisa o crítico Anatol Rosenfeld, a busca de Mário de Andrade por essa oralidade na escrita era, igualmente, uma busca pela sua própria identidade através da procura de uma identidade nacional. Para isso, o escritor foge da petrificação da língua, a forma fixa e estratificada no espírito coletivo que é apenas aparência, disfarce dessa identidade que o escritor intenta explorar pela linguagem.

Em seus \textit{Contos novos}, por exemplo, pode"-se observar, na interpretação de Rosenfeld, uma unidade temática que perpassa todas as narrativas do volume: a cisão da subjetividade, o homem disfarçado, desdobrado entre o ser e a aparência.
Para recriar essa separação contraditória do homem, Mário utiliza"-se largamente da recriação da própria língua:

\begin{quote}
O próprio abrasileiramento da língua é parte dessa reconstrução, na medida em que representa linguisticamente a busca do autêntico; mas na medida em que é uma estilização cuidadosamente elaborada, partilha também os fingimentos, tornando"-se a máscara do genuíno. A intenção da sinceridade implica sempre a ``segunda intenção''.\footnote{\textsc{rosenfeld}, Anatol. ``Mário e o cabotinismo''. In: \textit{Texto/Contexto I}. São Paulo: Perspectiva, 1996, p.\,194.}
\end{quote}

É o diálogo entre essas duas intenções, ou duas sinceridades, que Rosenfeld chama de o ``cabotinismo'' de Mário de Andrade: uma obra que, de um lado, quer transmitir a ``paisagem profunda'' do autor, os motivos que impelem o artista à criação, e, de outro, trabalha essa verdade primeira no nível artesanal e da elaboração de suas possibilidades de comunicação. É a partir desse espírito coletivo, exterior,
que o escritor molda sua interioridade. Para Mário de Andrade, no entanto, essas ``máscaras'' com que se enxergava a realidade interior não a afetava, ao contrário, fazia parte da totalidade da subjetividade e da sinceridade poética.

Na dimensão dessa sinceridade total, nas palavras de Rosenfeld, dessa virtude nietzschiana da verdade subjetiva, é que ``se subentende o seu empenho heroico por uma sintética língua falada"-escrita'', capaz de abraçar amorosamente todas as regiões do Brasil''.\footnote{Ibid., p. 192.}

Aqui vão duas pequenas amostras desse empreendimento titânico que Mário de Andrade impôs a si: o registro único e rico de festas tradicionais da população caipira de São Paulo.