\textbf{Jazz Rural}: disco"-livro, ou fonocaderneta, com textos de Mário de Andrade e gravações musicais de campo comandadas por ele, tudo feito na década de 1930 no interior de São Paulo. Desses textos e músicas paulistas deriva a reflexão contemporânea proposta pelo grupo Jazz Rural, com textos críticos e composições experimentais inspiradas na pesquisa musical de Mário em São Paulo.
        
\textbf{Enrique Menezes}: flautista e violeiro, anda sem muito pudor pelo meio dos chorões, da música experimental, de folias de reis e entre os selvagens acadêmicos. Depois, a salvo, tenta articular tudo em reflexão, usando indistintamente letras e notas musicais. Doutor e Mestre em musicologia pela Universidade de São Paulo, graduado em composição pela \versal{ECA/USP} com pós"-doutorado em etnomusicologia pela Unicamp.

