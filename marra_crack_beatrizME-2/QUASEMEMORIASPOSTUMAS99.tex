\chapterspecial{Quase memórias póstumas}{}{}
 

Ele tem cinquenta anos, rugas salientes, rosto chupado, dentes
amarelecidos bastante gastos e uma marcante palidez cutânea.

Trabalhou desde muito jovem como funcionário público, merecendo o
respeito habitual do ofício, até enjoar da rotina do trabalho e sair em
busca de novidades. Como se retornasse a uma infância simbólica da qual,
na verdade, nunca se livrara de fato.

Foi quando tudo começou a desandar em sua vida.

Há muitos anos ele faz tratamento para o que se considera por aí
dependência química, e não tem constrangimento em assumir o que chama de
``vício'' no crack.

Quando pergunto a respeito da fissura pelo crack, ele me diz que se
desligou disso há um bom tempo. Na verdade, ele usa a pedra como se a
mesma fosse um reflexo automático mantido por um correspondente hábito
adquirido de longa data.

O ato de fumar a pedra é tão constante que ele nem tem mais tempo para a
fissura. Sua mente está tão íntima da pedra que não se ocupa com as
tentações da pedra, as quais já não se tornam necessárias.

Ou então, mesmo que ele já tenha tido seus momentos de fissura, ela
acomodou"-se tanto que o seu cérebro não consegue mais produzir aquela
vontade explícita de manter a mesma voracidade de antes.

Ele logo acrescenta, no entanto, estar meio confuso em suas afirmativas.
E~apura dúvidas e contradições ao dizer que lhe resta, enfim, uma falsa
vontade de enganar a própria vontade.

— O senhor me compreende?

— Ainda não.

O homem procura resumir para mim, mas sem obter sucesso em sua
explicação. Ele diz que tudo isso seriam ``brisas'' estranhas da
``noia'', ou etapas bastante avançadas do ``vício maldito'' que o vem
apertando cada vez mais num círculo de fogo da banalidade do uso e da
repetição.

Eu digo que o caso dele me parece ser delicado, se não grave. Se ele não
tomar cuidados, vários remédios poderão ser necessários. Ou talvez uma
internação.

O homem dá de ombros e diz um tanto faz como tanto fez, joga um pouco
mais de conversa fora, despede"-se e pede para marcar um retorno.

\asterisc{}

Hoje, na última entrevista, ele chega animado. De repente, olha
atentamente para um livro meu deixado sobre a mesa. Abre o livro e
admira"-se de ser aquela uma edição bilíngue português/inglês de \emph{Memórias
de um Sargento de Milícias}, de Manuel Antônio de Almeida.

Ele faz observações interessantes a respeito da obra. Comenta que é a
história de um rapaz agitado, meio moleque, chamado Leonardo que sempre
foge das garras do chefe de polícia, o Major Vidigal, num jogo de gato e
rato.

Eu confirmo que a história está correta e me ponho a recordar o enredo.
Digo a ele que esta é uma obra testemunho fecundo da vida brasileira, e
da vida carioca e fluminense, no início do século \versal{XIX}.

Ele então revela que leu bem estas Memórias, e apreciou muito. Revela
que sempre gostou de confissões alheias, ainda mais porque se considera
um rato de bibliotecas e sebos, onde garimpa testemunhos que, vindos de
outras criaturas, remetam à sua alma sedenta de novidades.

Ele faz a seguir, de passagem, breves comentários a respeito de obras
feito \emph{O Cortiço}, de Aluísio de Azevedo, e de outras obras de peso, e vai
embora falando um pouco sobre a importância da literatura e do português
correto.

Admite que toda uma vida na ``noia'' não lhe impediu flertar com o
idioma, mas ultimamente ele fica cada vez mais ausente de si mesmo, e
cada vez mais preso a buscar nas memórias alheias uma contraparte ---
espelhada ou estilhaçada --- de suas próprias memórias.

Eu lhe pergunto qual teria sido a obra que mais fundo tocou sua alma. A~resposta demora poucos segundos de reflexão.

Ele me olha bem nos olhos e cita enfaticamente \emph{Memórias Póstumas de Brás
Cubas} porque, como muita gente sabe --- e o senhor deve saber bem --- é
uma história bastante conhecida de um morto se recordando de sua
existência enquanto marca sua presença vivíssima entre os vivos.

— Esta é a magia da história.

— Sim, eu concordo.

Pois este homem agora, que foi por tanto tempo funcionário público e
nunca inventou seu próprio remédio, nem um emplastro ou mero placebo,
resume o teor de sua ``viagem'' embriagada naquilo que ele tem como
``maldição da pedra''.

Eis aí uma estranha ``viagem'' que, como ele detalha com ar melancólico,
tem produzido nele uma espécie de dupla personalidade, a vivenciar
alternadamente um apagamento e um despertar no mundo.

— O senhor agora me compreende?

Fico um pouco em dúvida e respondo:

— Pode ser que sim.

Ele vai embora carregando talvez uma nova versão de sua pessoa.

Esta versão é como se, em inúmeros momentos, ele estivesse escrevendo,
sem papel ou outro registro, suas evanescentes memórias enquanto vive se
escondendo em becos obscuros na companhia de uma ``noia'' medíocre
fissurada apenas à pobreza da pedra.

Mas, para ele, esta pedra incandescente e repetitiva é uma ampulheta por
onde escorre uma areia da vida, e por onde também escorrem suas
lembranças.

E ele segue destilando suas diversas memórias que parecem planar no ar,
etéreas e sem peso, desaparecendo confusas na embriaguez desprovida,
agora, da antiga fissura juvenil, fissura cada vez mais distante porém
presente e sorrateira.

Memórias que são evocadas pela fumaça tóxica do crack.

Memórias duvidosas constantemente recriadas.

Memórias quase ausentes pela fuga constante.

Póstumas, porém, e, ao mesmo tempo, em vida que se vai ou que já foi.

\begin{center}\asterisc{}\end{center}
%\begingroup\small

\emph{Eu conheço este homem já faz alguns anos. Ele vem ao meu serviço
em aparições intermitentes, entre recaídas verdadeiras e falsas, ou,
enfim, no meio desta velha história das vicissitudes perigosas e
suspeitas da droga e, no caso, da droga crack.}

\emph{Este homem, como tantos outros, é um sobrevivente de guerra de um
longo tempo de abuso. Acredito que ele tenha uns vinte anos de crack, ou
até mais.}

\emph{No entanto, continua vivíssimo da silva e razoavelmente
preservado, a despeito do fato de que a dependência do crack lhe tenha
desarranjado bastante a vida familiar e social.}

\emph{Como tantos outros, ele me desperta um interesse médico ou
acadêmico para investigar como o abuso prolongado de uma droga pesada
venha a ser possível como um abuso estável, ou até ``social''. Na
verdade, é bem possível sim, e nada incomum, aliás, como em tantos outros
casos que atendi.}

\emph{Eis que surge então uma questão muito importante dizendo respeito
às estratégias sensatas de abordar as dependências das chamadas ``drogas
psicoativas'', principalmente as drogas ilícitas e mormente o crack.}

\emph{Ainda mais porque são muito comuns os sensacionalismos na mídia.
Isso pode distorcer uma avaliação precisa a partir de amostras que nos
são continuamente escancaradas, porém relevadas e tidas como pouco
significativas ou mesmo como raridades.}

\emph{Para início de conversa, caro leitor, uma droga que nem o crack
não é tão somente a cocaína empedrada para ser fumada. Não é tão somente
uma substância ativa que pode ser isolada. É~um conjunto que abrange
principalmente uma relação tríplice, ou triádica, droga"-indivíduo"-meio.
Outra maneira de olhar para isso é: a droga é somente droga com seu
invólucro e todo o entorno social, todo o ritual, todo o contexto.}

\emph{Existem usuários e usuários, os mais diversos, de maneira a diluir
o típico e o estereótipo do ``noia'' que, mesmo assim, existe. Mas, lá no
fundo, é a velha história de cada um cada qual, como eu já ouvi num
rap.}

\emph{No meio deste baita problema coletivo e geral, e diante das
especificidades individuais, permanece a velha questão profunda e
caprichosa: afinal de contas --- pergunta"-se --- o que move as pessoas
adictas a um ``prazer'' compulsivo que acaba deixando de ser prazer?}

\emph{Quanto mais complicados se tornam esses caminhos ou descaminhos do
prazer, mais essas drogas são mantidas pelo desejo ou pelo vulgo
``vício'', ou pelo desejo infrene por vezes terrível que acaba virando o
que se tem como fissura.}

\emph{A~palavra ``fissura'' é bem complexa. É~um signo muito maior do que
sua denotação habitual. E~dentre suas conotações há uma fissura que
existe no silêncio, há outra fissura que se mantém na ausência da
fissura convencional, e há até uma fissura na inconsciência do adicto
que mal sabe estar seguindo um ritual invisível.}

\emph{Eu ainda insisto em dizer a você que me lê: a realidade do crack e
de outras drogas pesadas não é bem essa que está na voz pequena e em
modelos aterrorizantes e apocalípticos veiculados pela mídia.}

\emph{E este homem é mais um exemplo de uma dificuldade basal de
compreensão e estudo de caso e de diagnóstico. Mesmo o conhecendo há
bastante tempo, eu só tinha avançado no máximo até lhe dar o velho
rótulo de dependente crônico de crack.}

\emph{De repente eu comecei a conhecê"-lo um pouco melhor por ter evitado
me prender apenas ao olhar clínico, por ter ido atrás de certos
meandros, de certas observações que ele me fez. E~confesso: o que fez
essa mediação --- tanto para mim quanto para ele --- foi a literatura.}

\emph{Então seu caso original acabou me parecendo uma nobre exceção
quando ficou patente para mim uma sofisticação dele, qual seja a sua
eleição de obras comentadas da literatura brasileira.}

\emph{Foi quando certo dia ele me falou de várias obras de peso e
escolheu o} Memórias Póstumas de Brás Cubas\emph{. Aquilo foi uma revelação,
mas não revelação de qualquer diagnóstico, de qualquer condição
psíquica.}

\emph{Aquilo foi uma revelação sutil de como uma fissura primária se
desvanece em prol de um conformismo mórbido com a mesmice ou com a
chatice de não ver nada demais naquela fumaça besta e, no entanto, vê"-la
se repetir a cada dia sem nenhuma graça.}

\emph{Chega"-se, pois, numa espécie de paradoxo: trata"-se de uma fissura
tão sutil que parece tudo, menos fissura. O~hábito vira uma monotonia que
disfarça a busca pela droga numa rotina insossa. Ocorre um automatismo
que suprime o desejo, ou o mascara.}

\emph{As memórias (quase póstumas) então preenchem um vácuo existencial.
O~indivíduo adicto começa a lembrar de si mesmo como um passado
totalizante e aterrador, sem presente e sem futuro. O~flerte dele com a
morte sofistica"-se diante da própria literatura, a exemplo da escolha da
obra prima de Machado de Assis.}

\emph{A~dependência se torna particularmente cruel na medida em que a
fumaça tóxica se repete de maneira estupidamente monótona em um homem
razoavelmente preservado e, acima de tudo, muito sofrido. Um homem
gasto, mas se equilibrando e sobrevivendo. Um homem não tão doente
fisicamente, porém aos trancos e barrancos quando se vê no espelho da
alma.}

\emph{No seu caso, não havia apenas uma vulgar intoxicação crônica, e
sim outro nível de transtorno inclusive mental porque o mesmo transtorno
estava assentado não na loucura do usuário na busca ensandecida pela
droga, porém no conformismo do usuário que já contempla um longo caminho
do desencanto e, no entanto, não consegue se desviar da droga.}

\emph{Foi quando eu comecei a perceber que era justamente isso que fazia
o interesse dele pela literatura e, particularmente, era o que fazia seu
gosto peculiar pelo} Memórias Póstumas\emph{. Esse gosto peculiar deu"-se
justamente por meio de uma empatia que ele passou a sentir pelo
arquétipo do morto glorioso que vive; esse gosto seguiu por meio do
fascínio que ele passou a sentir pelo indivíduo extinto, porém indivíduo
singular vindo ao mundo chamado sensível e pairando nas dimensões
mágicas do que a arte alcança dizer, mas onde a linguagem do concreto
nada diz.}

\emph{Eis uma presença situada além do factual e seguindo no desafio do
paradoxo. No caminho da ambivalência do dito pelo não dito.}

\emph{Eis uma presença polissêmica da droga subvertendo uma realidade
íntima, através de uma lógica enviesada e de uma desordem do mundo, como
se tudo isso fosse um ardil psicótico de uma semiótica perversa da
droga. Droga que é menos química e mais relacional, droga a produzir
significantes novos que sempre remetem, no entanto, à mesmice da
repetição.}

\emph{Enfim, de repente eu, ainda perdido e me encontrando, e tentando
compreender o longo caminho dele na adicção pesada ao crack, consegui
obter uma precária síntese, até de maneira racional, dessas quase
memórias póstumas. Que eu digo ``quase póstumas'' por deferência à
lógica temporal situada entre a vida e a morte e para me ater ao valor
simbólico do que se considera como morte em sentido amplo.}

\emph{Afinal de contas, toda adicção pesada e verdadeira é um flerte com
a morte. Além de ser uma compulsão à repetição, como bem observou
Freud.}

\emph{Confesso que foi assim que intuí a gravidade do ``quadro'' deste
homem para além de certos diagnósticos comuns.}

\emph{Mas não porque eu temesse sua morte física por overdose de
crack/cocaína. Nada disso. Ele já havia me provado estar bem adaptado à
droga. Eu temia seu apagamento existencial, através da depressão ou da
melancolia, quando suas memórias quase póstumas se esgotassem e quando,
na vida real, não houvesse mais lugar para memórias.}

\emph{Este homem então acabou virando um enigma de si mesmo, um enigma
cuja resolução se deu parcialmente através da arte; através de um
desafio que se revelou pela recorrência ao cânone universal; um desafio
vindo, porém, do bruxo do Cosme Velho, o velho Machado; um desafio
também vindo um pouco de Aluísio de Azevedo e, de quebra, do} Memórias
de um Sargento de Milícias.

\emph{Creio que tudo isso tenha sido a chave para a compreensão da
psicodinâmica desse homem preso à necessidade de fazer memória, de se
eternizar como memória no meio da crueza do anonimato da periferia,
perdido entre ``noias'' medíocres que não lhe seriam jamais plateia, e
estando ele sozinho com seus caprichos literários.}

\emph{Ah, caro leitor, eu percebi o quanto um diagnóstico clínico ou
psiquiátrico pode ser supérfluo, redundante ou dispensável diante das
especificidades existenciais agônicas de um dependente de crack, cuja
gravidade está justamente no distanciamento ao prazer banal da droga e
não na aproximação a ela por meio da fissura convencional.}

\emph{Quanto mais esse usuário despreza e se entedia com aquilo que é
objeto de prazer, mais aquilo se aproxima dele de maneira aparentemente
incoerente. Quer dizer, aproxima"-se num caminho perverso que costuma ser
um beco sem saída, como já é bem sabido.}

\emph{Mas é justamente aí onde a arte pode acenar como solução, porque o
paradoxo da existência de um ser humano póstumo ainda em vida é uma
questão artística.}

\emph{Digo mais: o tratamento de uma pessoa assim pode ser uma viagem de
Virgílio com Dante no Inferno. Pode ser um simulacro empático do que
seria morte com alguém que, vivo, coloca"-se como morto, antecipa"-se ou
resignifica"-se como morto. Isso seria até uma antinomia e, como tal, é
arte.}

\emph{A~arte se abre para a ``loucura'' e vira arte terapêutica.}

\emph{Trata"-se também de uma questão que traz à tona uma outra questão
muito importante para mim: a de que o crack não é tão perigoso pelo
risco de vida devido à intoxicação e sim pelo risco da sua dependência
virar um monstro invisível cheio de tentáculos da ilusão.}

\emph{E estou me referindo ao crack transacionado em seu aspecto vulgar
de mercadoria marginal e desprezada, porém mercadoria poderosa a exercer
um fascínio tão grande em certos adictos marginalizados, de maneira que
eles deixam de perceber o poder do fascínio, embora insistam nele.}

\emph{No entanto, este homem algo ilustrado parece indicar alguma saída.
Sua ``doença'' fala disso, seu sofrimento é eloquente. E~se ele segue na
embriaguez de uma droga não etílica, ele procede como um autômato do que
ele insiste ser o seu ``vício'', querendo rescrever à sua maneira as
suas quase póstumas memórias.}

\emph{Ele o faz enquanto escritor de sua ficção tão real quanto ela é
real e dura como uma pedra de cocaína desaparecida em minutos.}

\emph{Minutos que seriam antítese da memória da qual se espera que
jamais chegue a um fim. Daí brota nele um certo fascínio (e também brota
um terror) por algum infinito desconhecido, por uma viagem sem volta,
por alguma eternidade misteriosa, como está no famoso monólogo de
Hamlet.}

\emph{Eternidade que é mais do que ter algum éter"-na"-mente.}
%\endgroup