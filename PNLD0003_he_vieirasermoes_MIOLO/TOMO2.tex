\documentclass[11pt]{hedrabook}
\usepackage[brazilian]{babel}
\usepackage{ucs}
\usepackage[utf8x]{inputenc}
\usepackage[center,cam,a4]{hedracrop}
\usepackage{hedrabolsolayout,hedraextra}
\usepackage[chapterdot]{hedratoc}
\usepackage[protrusion=true,expansion]{microtype}
\usepackage{comment,lipsum,footmisc,pdfpages}

\geometry{bottom=1cm,%textheight=148mm,%
		top=10mm,bottom=10mm,
		paperwidth=143mm,%
		paperheight=222mm}

\begin{document}
\SVN $Id: TEMPLATE.tex 6339 2010-04-28 17:52:17Z bruno $


\selectlanguage{brazilian}
\title{Sermões, vol.~i} 
\author{Antonio Vieira} 
\begin{blackpages}
	\maketitle
	\begin{techpage}{5cm}
		\vspace{-1cm}
		\putline{Copyright}{Hedra \the\year}
		\putline{Tradução$^\copyright$}{   } 
%		\putline{Título original}{}
%		\putline{Edição consultada}{\emph{ }, }
%		\putline{Primeira edição}
%		\putline{Indicação}{}
%		\putline{Agradecimento}
		\putline{Corpo editorial}{
			Adriano Scatolin,\\
			Alexandre B.~de Souza,\\
			Bruno Costa,
			Caio Gagliardi,\\
			Fábio Mantegari,
			Iuri Pereira,\\
			Jorge Sallum,
			Oliver Tolle,\\
 		   Ricardo Musse,
			Ricardo Valle
		}
		\ \\

		\putline{Dados}{\fichacatalografica{H331}% Código de autor
				% Autor (data nascimento--morte)
				{Heine, Heinrich (1797–1856)}% 
				% Dados bibliográficos
				{O Rabi de Bacherach. / Heinrich Heine. 
				Tradução e organização de Marcus
				Vinicius  Mazzari. – São Paulo: Hedra, 2009. (Estudos Libertários).}%
				% Isbn
				{978-85-7715-131-8}%
				% Catalogação. Ex. 1.~Blabla.(Não esquecer ~)
				{1.~Literatura Alemã. 
				2.~Romance Histórico. 3.~Religião. 4.~Judaismo. 5.~Ódio Racial. 
				6.~Antissemitismo. 
				\textsc{i}.~Título. 
				\textsc{ii}.~Série. \textsc{iii}.~Mazzari, Marcus Vinicius, Tradutor.}%
				%CDU
				{830}
				%CDD 
				{833.7} %
			       }
		\direitos
		\dadoseditora
		\depositolegal
	\end{techpage}
	\begin{frontispiciopage}{4cm}{}	
	\putline{\hspace{12ex}Organização}{\textsc{Alcir Pécora}}
	\end{frontispiciopage}

\end{blackpages}
\setcounter{tocdepth}{0}     % amplitude da presença das partes no índice
\setcounter{secnumdepth}{-2} % amplitude da numeração das partes


% Favor não alterar o segundo parâmetro (\baselineskip). 
% Para acertar entrelinhas, usar comando \linespread
%\fontsize{10.5pt}{\baselineskip}\selectfont
\baselineskip=13.2pt   % Parâmetro válido apenas para corpo 11, para outros, acrescer 20% do valor do corpo%%

\paginabranca
\includepdf[pages=3-284]{TOMO2A.pdf}
\includepdf[pages=1-299]{TOMO2B.pdf}

\SVN $Id: FINAIS.tex 7286 2010-08-24 13:20:34Z bruno $



\pagestyle{empty}
\ifodd\thepage\paginabranca\else\clearpage\fi
\textsc{títulos publicados}
\begin{enumerate}
\setlength\itemsep{-1.1mm}
%Se o corpo for 11 ou maior, colocar \scripsize
{
%\def\normalsize{\fontsize{7}{7}\selectfont}
\tiny
\item \textit{Iracema}, Alencar
\item \textit{Don Juan}, Molière
\item \textit{Contos indianos}, Mallarmé
\item \textit{Auto da barca do Inferno}, Gil Vicente
\item \textit{Poemas completos de Alberto Caeiro}, Pessoa
\item \textit{Triunfos}, Petrarca
\item \textit{A cidade e as serras}, Eça
\item \textit{O retrato de Dorian Gray}, Wilde
\item \textit{A história trágica do Doutor Fausto}, Marlowe
\item \textit{Os sofrimentos do jovem Werther}, Goethe
\item \textit{Dos novos sistemas na arte}, Maliévitch
\item \textit{Mensagem}, Pessoa
\item \textit{Metamorfoses}, Ovídio
\item \textit{Micromegas e outros contos}, Voltaire
\item \textit{O sobrinho de Rameau}, Diderot
\item \textit{Carta sobre a tolerância}, Locke
\item \textit{Discursos ímpios}, Sade
\item \textit{O príncipe}, Maquiavel
\item \textit{Dao De Jing}, Laozi
\item \textit{O fim do ciúme e outros contos}, Proust
\item \textit{Pequenos poemas em prosa}, Baudelaire
\item \textit{Fé e saber}, Hegel
\item \textit{Joana d'Arc}, Michelet
\item \textit{Livro dos mandamentos: 248 preceitos positivos}, Maimônides
\item \mbox{\textit{O indivíduo, a sociedade e o Estado, e outros ensaios}, 
		Emma Goldman}
\item \textit{Eu acuso!}, Zola | \textit{O processo do capitão Dreyfus}, Rui Barbosa
\item \textit{Apologia de Galileu}, Campanella 
\item \textit{Sobre verdade e mentira}, Nietzsche
\item \textit{O princípio anarquista e outros ensaios}, Kropotkin
\item \textit{Os sovietes traídos pelos bolcheviques}, Rocker
\item \textit{Poemas}, Byron
\item \textit{Sonetos}, Shakespeare
\item \textit{A vida é sonho}, Calderón
\item \textit{Escritos revolucionários}, Malatesta
\item \textit{Sagas}, Strindberg
\item \textit{O mundo ou tratado da luz}, Descartes
\item \textit{O Ateneu}, Raul Pompeia
\item \textit{Fábula de Polifemo e Galateia e outros poemas}, Góngora
\item \textit{A vênus das peles}, Sacher{}-Masoch
\item \textit{Escritos sobre arte}, Baudelaire
\item \textit{Cântico dos cânticos}, [Salomão]
\item \textit{Americanismo e fordismo}, Gramsci
\item \textit{O princípio do Estado e outros ensaios}, Bakunin
\item \textit{O gato preto e outros contos}, Poe
\item \textit{História da província Santa Cruz}, Gandavo
\item \textit{Balada dos enforcados e outros poemas}, Villon
\item \textit{Sátiras, fábulas, aforismos e profecias}, Da Vinci
\item \textit{O cego e outros contos}, D.H.~Lawrence
\item \textit{Rashômon e outros contos}, Akutagawa
\item \textit{História da anarquia (vol.~1)}, Max Nettlau
\item \textit{Imitação de Cristo}, Tomás de Kempis
\item \textit{O casamento do Céu e do Inferno}, Blake
\item \textit{Cartas a favor da escravidão}, Alencar
\item \textit{Utopia Brasil}, Darcy Ribeiro
\item \textit{Flossie, a Vênus de quinze anos}, [Swinburne]
\item \textit{Teleny, ou o reverso da medalha}, [Wilde et al.]
\item \textit{A filosofia na era trágica dos gregos}, Nietzsche
\item \textit{No coração das trevas}, Conrad
\item \textit{Viagem sentimental}, Sterne
\item \textit{Arcana C\oe lestia} e \textit{Apocalipsis revelata}, Swedenborg
\item \textit{Saga dos Volsungos}, Anônimo do séc.~\textsc{xiii}
\item \textit{Um anarquista e outros contos}, Conrad
\item \textit{A monadologia e outros textos}, Leibniz
\item \textit{Cultura estética e liberdade}, Schiller
\item \textit{A pele do lobo e outras peças}, Artur Azevedo
\item \textit{Poesia basca: das origens à Guerra Civil} 
\item \textit{Poesia catalã: das origens à Guerra Civil} 
\item \textit{Poesia espanhola: das origens à Guerra Civil} 
\item \textit{Poesia galega: das origens à Guerra Civil} 
\item \textit{O chamado de Cthulhu e outros contos}, H.P.~Lovecraft 
\item \textit{O pequeno Zacarias, chamado Cinábrio}, E.T.A.~Hoffmann
\item \textit{Tratados da terra e gente do Brasil}, Fernão Cardim 
\item \textit{Entre camponeses}, Malatesta
\item \textit{O Rabi de Bacherach}, Heine
\item \textit{Bom Crioulo}, Adolfo Caminha
\item \textit{Um gato indiscreto e outros contos}, Saki
\item \textit{Viagem em volta do meu quarto}, Xavier de Maistre 
\item \textit{Hawthorne e seus musgos}, Melville
\item \textit{A metamorfose}, Kafka
\item \textit{Ode ao Vento Oeste e outros poemas}, Shelley
\item \textit{Oração aos moços}, Rui Barbosa
\item \textit{Feitiço de amor e outros contos}, Ludwig Tieck
\item \textit{O corno de si próprio e outros contos}, Sade
\item \textit{Investigação sobre o entendimento humano}, Hume
\item \textit{Sobre os sonhos e outros diálogos}, Borges | Osvaldo Ferrari
\item \textit{Sobre a filosofia e outros diálogos}, Borges | Osvaldo Ferrari
\item \textit{Sobre a amizade e outros diálogos}, Borges | Osvaldo Ferrari
\item \textit{A voz dos botequins e outros poemas}, Verlaine 
\item \textit{Gente de Hemsö}, Strindberg 
\item \textit{Senhorita Júlia e outras peças}, Strindberg 
\item \textit{Correspondência}, Goethe | Schiller
\item \textit{Índice das coisas notáveis}, Vieira
\item \textit{Tratado descritivo do Brasil em 1587}, Gabriel Soares de Sousa
\item \textit{Poemas da cabana montanhesa}, Saigy\=o
\item \textit{Autobiografia de uma pulga}, [Stanislas de Rhodes]
\item \textit{A volta do parafuso}, Henry James
\item \textit{Ode sobre a melancolia e outros poemas}, Keats 
\item \textit{Teatro de êxtase}, Pessoa
\item \textit{Carmilla -- A vampira de Karnstein}, Sheridan Le Fanu
\item \textit{Introdução ao pensamento político de Maquiavel}, Fichte
\item \textit{Inferno}, Strindberg
\vfill
}%
\end{enumerate}


\pagebreak

\begin{blackpages}
\begin{techpage}{42mm}
		\putline{Edição}{Bruno Costa}
		\putline{Coedição}{Iuri Pereira e Jorge Sallum}
		\putline{Capa e projeto gráfico}{Júlio Dui e Renan Costa Lima}
%		\putline{Imagem de capa}{}
		\putline{Programação em LaTeX}{Marcelo Freitas}
%		\putline{Revisão}{}
		\putline{Assistência editorial}{Bruno Oliveira}
%		\putline{Preparação}{}          
		\putline{Colofão}{Adverte-se aos curiosos que se
			imprimiu esta obra em nossas oficinas em \today, em papel 
			\mbox{off-set} 90~g/m$^2$,
			composta em tipologia Minion Pro, 
			em \textsc{gnu}/Linux (Gentoo, Sabayon e Ubuntu), 
			com os softwares livres 
			\LaTeX, De\TeX, \textsc{vim}, Evince, Pdftk, 
			Aspell, \textsc{svn} e \textsc{trac}.}

\end{techpage}
\end{blackpages}


\ifdefined\printcheck\printcheck\fi

\end{document}
