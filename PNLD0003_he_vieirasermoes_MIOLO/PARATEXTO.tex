\chapter{Vida e obra do Padre Antônio Vieira}

\section{Sobre o autor}

\noindent{}Nascido em 1608 na cidade de Lisboa, Antônio Vieira era filho de
Cristóvão Vieira Ravasco e Maria de Azevedo. Quando tinha seis anos, sua
família transferiu"-se para Salvador, onde Vieira estudou no colégio da
Companhia de Jesus, na qual ingressou como noviço em 1623.

Aos dezessete anos, tornou"-se redator das cartas"-ânuas e aos dezoito já
era professor de Retórica em Olinda. Entre 1641 e 1652 permanece na
Europa em missões diplomáticas na Inglaterra, França e Holanda. Em 1653
torna"-se Superior das missões do Maranhão e Grão"-Pará.

Em 1661, embarca para Portugal, onde estava sendo processado pela
Inquisição. Em 1665, está em Coimbra, proibido de retornar ao Maranhão.
Será preso em outubro. O primeiro volume de seus \emph{Sermões} foi
publicado em 1679. Retornou para a Bahia em 1681. Aos 89 anos, em 1697,
faleceu o grande jesuíta.

A essa altura, tinha preparado para publicação doze tomos dos sermões.
Embora a parte mais conhecida de sua obra sejam os mais de duzentos
sermões, Antônio Vieira escreveu também uma abundante correspondência,
somando mais setecentas cartas. Também deixou escritos proféticos,
\emph{A história do futuro} e \emph{A chave dos profetas}.

\subsection{Carta"-Ânua de 1626}

As cartas"-ânuas eram longas cartas"-relatório que jesuítas escreviam
anualmente informando os progressos do trabalho catequético. A primeira
carta"-ânua escrita por Vieira foi no ano de 1626. A carta tem cerca de
quarenta páginas e dá notícias dos acontecimentos dos anos de 1624 e
1625. É o primeiro escrito preservado do famoso pregador. Vale citar a
descrição geral da situação dos missionários:

\begin{quote}
Sustenta esta Província do Brasil, pouco mais ou menos, 120 padres da
Companhia: 90 sacerdotes, dos quais 31 são professos de quatro votos, de
três solenes, 2, coadjutores espirituais formados, 20; 62 estudantes;
coadjutores 50, e destes, 30 formados. Estes todos divididos em três
colégios, seis casas, e treze aldeias anexas às mesmas casas e colégios.
No colégio {[}da{]} Bahia residem comummente, 80; no de Pernambuco, 40;
35 no do Rio de Janeiro; na Residência do Espírito Santo, 12; na de
Santos, 5; na de S. Paulo, 7; na Casa dos Ilhéus, 4; em Porto Seguro, 4;
e 4 no Maranhão. Todos eles se ocupam em procurar de alcançar a salvação
e perfeição própria e das almas, que é o fim da nossa
Companhia.\footnote{\textsc{vieira}, António. \textit{Cartas}. Org. J. Lúcio de Azevedo.
  Lisboa: Imprensa Nacional"-Casa da Moeda, 1997, p.\,4.}
\end{quote}

Após esta síntese da distribuição dos missionários, Vieira irá detalhar
a situação de cada um destes lugares. As cartas de Vieira e de outros
jesuítas, escritas sistematicamente, representam um diagrama verbal da
formação das sociedades coloniais.

\section{Sobre a obra}

Aqui estão reunidos dez sermões dos mais representativos do padre Antonio Vieira.
São textos que continuam mostrando uma das facetas mais impressionantes dos
séculos \textsc{xvi} e \textsc{xvii}, que foi a aventura de missionários enviados a essas
distantes e desconhecidas terras para difundir a religião católica.

O estilo de Vieira é muito elaborado e florido, isto é, é um autor que
usa a língua portuguesa de maneira exemplar, explorando suas riquíssimas
belezas expressivas. Seu texto é muito ornamentado, marcado por
enumerações, interrogações, citações, marcado por uma fraseado musical e
persuasivo. Talvez por isso, um poeta da envergadura de Fernando Pessoa chegou a declarar que Antonio Vieira era o melhor escritor da língua portuguesa.

\subsection{resumo dos sermões}

\paragraph{Sermão da Sexagésima}

O sermão volta"-se para sua própria composição e examina os 3 ``concursos'' essenciais que
há nele (Graça, pregador e ouvinte), para saber qual deles pode ser causa da falta de eficácia dos sermões contemporâneos na reforma dos cristãos. Admitida que a falta apenas pode
ser do pregador, examina as suas 5 ``circunstâncias'' (pessoa, estilo, ciência, matéria e voz) e admite em todas a existência de faltas graves, embora nenhuma delas possa ser tomada
como causa principal do fracasso do sermão. Este deve"-se sobretudo ao ``falso testemunho''
do pregador que, embora utilizando palavras de Deus, não as toma em seu sentido original,
mas distorce"-as segundo seus interesses eo propósito de agradar ao auditório, em vez de
desenganá"-lo e reformar os seus costumes como é sua obrigação.

\paragraph{Sermão de quarta"-feira de cinza}

O sermão é conduzido da seguinte forma:
\begin{enumerate}
\item Mostra que o verdadeiro ser do homem apenas se determina em função de seus limites
no passado (o nada) e no futuro (a morte);

\item Que a própria morte tem limites: a ressurreição, a eternidade;

\item Que a eternidade pode cumprir"-se alternativamente de duas formas: o ser do Paraíso e
o não ser do Inferno;

\item Que um e outro apenas se definem em função dos atos da vida.
\end{enumerate}

Assim, o sermão propõe que, se é possível dizer que a vida é não ser (como a morte) face ao
ser da eternidade, a consideração de que este apenas se define em função das escolhas
feitas durante a vida obriga a entendê"-la de modo mais complexo: pois a vida não é
simplesmente não ser ou falsidade, quando tem consequências decisivas para o ser. Ou
seja, o ser da eternidade define"-se como efeito da existência, ainda que ela mesma não
tenha senão um ser precário.

\paragraph{Sermão de Santo Antonio}

Os pregadores são o sal da terra, e, portanto, os que devem conservá"-la na fé cristã. O estado atual de corrupção do mundo deve"-se a falhas no pregador (que não prega a verdadeira
doutrina, nem age de acordo com ela, mas segundo seus interesses próprios) e no ouvinte
(que igualmente não deseja a verdadeira doutrina, e apenas age por interesses pessoais).
Simula dirigir"-se então aos peixes e não aos homens (à imagem do anedotário de S.\,Antônio).
Tomando o que os peixes têm de bom, propõe para o pregador as funções ``rêmora'' (resistência
à paixão do ouvinte), ``torpedo'' (que o faz tremer pela autoridade), ``quatro"-olhos'' (visão de recompensas e castigos) e ``sardinha'' (sustento dos pobres). Por outro lado, combate os defeitos de ouvintes, à imagem de tipos defeituosos entre os peixes: ``roncadores'' (que ameaçam,
sendo fracos), ``pegadores'' (parasitas e bajuladores que dependem do poder alheio);
``voadores'' (ambiciosos que se enganam com sua condição); e, enfim, os que são ``polvo''
(padres oportunistas, sem caráter) --- todos apropriadores dos bens alheios.

\paragraph{Sermão do Espírito Santo}

Para o ensino de bárbaros (uma vez que não há nações menos importantes que outras para
a pregação cristã), há mais necessidade de amor do que de sabedoria. Em particular, é
necessário amor para converter o índio do Brasil, pela condição bruta da gente e pela
dificuldade de aprendizado das línguas. O esforço de domínio das línguas para propagação
da fé é desdobramento do amor de Deus e da missão da Igreja. Os modernos sucessores dos
apóstolos têm, no Novo Mundo, ``fogo de línguas'', que aprendem não por milagre, mas por
trabalho. O maior serviço de Deus que pode ser feito pelos apóstolos não é de contemplá"-lo
no Paraíso, mas o de enfrentar as ``cegueiras da terra'' de modo a produzir a saúde espiritual dos cristãos.
Assim, o sermão inicia"-se pela valorização do trabalho dos missionários, por meio da
amplificação das dificuldades de entendimento dos ouvintes e das línguas indígenas, e
avança até promover a extensão da tarefa da sua conversão ao conjunto dos moradores.
Eclesiásticos e leigos estão todos sob o ministério divino e a autoridade da Igreja e fugir à
responsabilidade da conversão equivale à recusa de Deus e ao castigo no Juízo.

\paragraph{Sermão da Epifania}

O sermão propõe que o nascimento de Cristo renova"-se com a nova cristandade resultante
da conversão dos gentios pelos portugueses, causa final das conquistas portuguesas e da
glória do Reino. Os missionários renovam a presença de Cristo na história e são condição
da manutenção legítima da conquista da América, o que é reconhecido pelos próprios
gentios, mas paradoxalmente ignorado pelos portugueses do Brasil, principal obstáculo à
ação dos jesuítas. O sermão também defende a inalienabilidade dos poderes temporal e
espiritual sobre os gentios, pois apenas se justifica a servidão que se padece por Deus,
qualquer outra sendo contrária à razão e à justiça. Por fim, reclama medidas práticas
imediatas da Corte lisboeta, entre elas: qualificação dos governadores; congregações
eclesiásticas com prelados instruídos; garantia de cumprimento das leis dos índios na colônia.

\paragraph{Sermão \textsc{xiv} do Rosário}

O trabalho no engenho é comparado ao de Cristo no calvário, o que, por um lado, amplifica
a injustiça e rudeza do tratamento dispensado aos escravos negros, e, por outro, justifica"-o
enquanto forma de conhecer a religião, desconhecida de seus pais, única forma de salvá"-los
espiritualmente. Ou seja, o escravo apenas aproveita de seu estado de privação da liberdade
e sofrimento físico pela devoção, e o senhor do engenho apenas tem justificativa no que
cativa e obriga o negro pelo que lhe ensina da religião que não tinha. Na ordem dos mistérios
do Rosário, estão em primeiro lugar os ``gozosos'' (efetuados analogamente pelos senhores),
depois estão os ``dolorosos'' (efetivados pelos negros) e finalmente os ``gloriosos''. Os ``gostos'' desta vida costumam ter como consequência as dores na eterna, e vice"-versa.

\paragraph{Sermão da Visitação de Nossa Senhora a Santa Isabel}

O Sermão identifica a raiz dos ``males'' do Brasil com a constante transferência de seus bens para a Metrópole. Propõe ao Vice"-Rei que, agindo contrariamente ao que tem sido praticado pelos governantes, aplique na Colônia a riqueza produzida nela.

\paragraph{Sermão dos Bons Anos}

O sermão postula que a posse do trono português por D.\,João \textsc{iv} estava já profetizada, e assinala o início de um novo tempo de bonança. Em termos mais específicos, adota uma posição ``neosebastianista'', na qual o Rei Encoberto das trovas populares não é D.\,Sebastião, dado efetivamente como morto, mas o novo rei brigantino: o rei ``esperado'' identifica"-se assim com o rei atualmente empossado. O sermão também justifica três medidas tomadas no primeiro ano do governo de D.\,João \textsc{iv}: a restrição de gastos, a premiação de antigos
vassalos e a convivência com os inimigos.

\paragraph{Sermão de São José}

D.\,João \textsc{iv} é o verdadeiro ``Encoberto'' das profecias do Bandarra, sob as saudades de D.\,Sebastião. O ``encobrir'' é propriedade de S.\,José, pois escondeu a paternidade divina de Cristo, ao tornar"-se Esposo da Virgem. Ou seja, o sermão reinterpreta o mito popular do ``Encoberto'' por meio da figura de S.\,José, terminando por transferi"-lo a D.\,João \textsc{iv}.

\paragraph{Lágrimas de Heráclito}

Em relação ao mundo terreno, são mais próprias as lágrimas de Heráclito do que o riso de
Demócrito. Em primeiro lugar, porque o riso de Demócrito, uma vez que é constante e não
se dá pela novidade, é irônico e tem o mesmo valor das lágrimas pelas misérias do mundo;
em segundo, porque, a ser verdadeiramente riso, sua constância produz ridículo, enquanto
a constância das lágrimas demonstra sua seriedade e ``concilia'' a benevolência do auditório.

\subsection{Análise de um trecho do «Sermão de Santo Antônio»}

\begin{quote}
Dos animais terrestres o cão é tão doméstico, o cavalo tão sujeito, o
boi tão serviçal, o bugio\footnote{Macaco.} tão amigo, ou tão
lisonjeiro, e até os leões e os tigres com arte e benefícios se amansam.
Dos animais do ar, afora aquelas aves que se criam e vivem conosco, o
papagaio nos fala, o rouxinol nos canta, o açor\footnote{Ave de rapina
  que pode ser treinada para caçar animais pequenos.} nos ajuda e nos
recreia; e até as grandes aves de rapina, encolhendo as unhas,
reconhecem a mão de quem recebem o sustento. Os peixes pelo contrário lá
se vivem nos seus mares e rios, lá se mergulham nos seus pegos,\footnote{Regiões
  profundas dos mares.} lá se escondem nas suas grutas, e não há nenhum
tão grande, que se fie do homem, nem tão pequeno, que não fuja dele. Os
Autores comumente condenam esta condição de peixes, e a deitam\footnote{Neste
  caso, ``deitar'' significa atribuir.} à pouca docilidade, ou demasiada
bruteza; mas eu sou de mui diferente opinião. Não condeno, antes louvo
muito aos peixes este seu retiro, e me parece que se não fora natureza,
era grande prudência. Peixes! Quanto mais longe dos homens tanto melhor:
trato e familiaridade com eles, Deus vos livre. Se os animais da terra e
do ar querem ser seus familiares, façam"-no muito embora, que com suas
pensões o fazem. Cante"-lhes aos homens o rouxinol, mas na sua gaiola:
diga"-lhes ditos o papagaio, mas na sua cadeia: vá com eles à caça o
açor, mas nas suas piozes:\footnote{Correias que se prendem à parte
  inferior da perna dos gaviões.} faça"-lhes bufonarias\footnote{Gracejos,
  piadas.} o bugio, mas no seu cepo:\footnote{Tronco ao qual se prende o
  macaco de estimação.} contente"-se o cão de lhe roer um osso, mas
levado onde não quer pela trela:\footnote{Coleira.} preze"-se o boi de
lhe chamarem formoso ou fidalgo, mas com o jugo\footnote{Peça de madeira
  que prende o boi à carroça que ele puxa.} sobre a cerviz,\footnote{Parte
  de trás do pescoço.} puxando pelo arado e pelo carro: glorie"-se o
cavalo de mastigar freios dourados, mas debaixo da vara e da espora: e
se os tigres e os leões lhe comem a ração da carne que não caçaram no
bosque, sejam presos e encerrados com grades de ferro. E entretanto vós,
peixes, longe dos homens, e fora dessas cortesanias,\footnote{Conjunto de
  formas de comportamento que caracterizam a vida social nas Cortes
  europeias, marcadas pela afetação e artificialismo.} vivereis só
convosco, sim, mas como peixe na água.\footnote{\textsc{vieira}, Antonio.
  \emph{Sermões}, vol. I. Org. Alcir Pécora. São Paulo: Hedra, 2001, p.
  321.}
\end{quote}

No ``Sermão de Santo Antônio'', pregado no Maranhão em 1654, o padre
Vieira afirma que apesar de fazer muita pregação, são poucos e poucas os
que se convertem. As pessoas estão surdas aos conselhos e escolhem agir
mal. Desiludido por ver o pouco fruto da palavra de Deus, o padre, nesse
sermão, diz que vai pregar aos peixes, animais considerados insensíveis.
Para isso, constrói um elogio do comportamento desses animais.

No trecho reproduzido acima, o jesuíta compara os animais segundo os
reinos a que pertencem: animais terrestres, animais aéreos e animais
aquáticos. Em seguida, o autor dá exemplos de animais desses três reinos
que se deixam amansar pelas pessoas em troca de um benefício material.
Da terra cita o cão, o cavalo, o boi, o macaco e o leão. Do ar, o
papagaio, o rouxinol, o açor e outra aves de rapina, como o gavião. Já
os peixes, em contraponto, não querem manter relação nenhuma com
pessoas, mostrando"-se ``reservados, desconfiados e prudentes''.

Depois de considerar que os Autores da Antiguidade atribuem a
insensibilidade dos peixes à bruteza, Vieira considera que talvez seja
por prudência. Após, o padre retira a sentença moral ou o ensinamento
que percorre o sermão: ``Quanto mais longe dos homens tanto melhor:
trato e familiaridade com eles, Deus vos livre''.

Em seguida, Vieira desenvolve cada um dos exemplos dos animais que se
deixam domesticar, associando a recompensa e o sacrifício da submissão:
``contente"-se o cão de lhe roer um osso, mas levado onde não quer pela
trela''.

Observe"-se que todo o trecho é ``inventado''. A fábula de Vieira é um
argumento narrativo e um ornamento do discurso. Podemos supor que se
estivéssemos na igreja ouvindo esse sermão, nos sentiríamos
constrangidos e envergonhados pela comparação. Principalmente por não
serem os peixes seres morais, mas seres que se orientam apenas pelo
instinto.

Esse constrangimento moral é chamado de contrição. É o remorso e o
arrependimento por agir mal. A dor de consciência leva o fiel e a fiel a
se corrigirem e não voltar a cometer o mesmo pecado.


\section{Sobre o gênero}

Os sermões se encaixam no gênero da prosa doutrinal, caracterizada pela oralidade, pois é pensada para ser falada de memória diante
de uma plateia católica na igreja durante a missa. O objetivo do sermão
é relembrar aos fiéis o caminho da salvação de suas almas. É um gênero
de extensão média, contendo em geral entre vinte e quarenta páginas.

Os sermões de Vieira são sempre divididos em muitas partes, que indicam
os pontos em que seu pensamento vai se desenvolvendo. Quando publicados
como livro, os sermões passaram a trazer algumas marcas fixas, como
título, local e data em que foi pregado, às vezes, uma referência à
ocasião e uma epígrafe retirada da \emph{Bíblia}, que antecipa o tema. O
sermão tem muitos intertextos retirados da \emph{Bíblia}, dos santos
católicos e de outros autores antigos e modernos.

A produção dos sermões de Vieira está inserida no contexto histórico 
do século \textsc{xvi}, marcado por uma série de mudanças tão intensas e
impactantes que fez com que esse período ficasse conhecido como
Renascimento. Isso porque, nos centros urbanos da Europa, havia uma
sensação de que um mundo diferente estava nascendo, ou melhor
renascendo. Esse mundo novo seria marcado pelo deslocamento da essência
da existência humana da esfera coletiva e religiosa para a esfera
particular e política.

Em razão dessa mudança, o movimento literário do período também ficou
conhecido como Humanismo. Essa renovação do pensamento que se apresentou
no século \textsc{xvi} se alimentou muito fortemente nos autores gregos, como
Aristóteles, e romanos, como Cícero. Esses autores passaram a ser
conhecidos como ``clássicos''. Por isso, a renovação artística que
ocorre no século \textsc{xvi}, e que pode ser sintetizada pelo grande artista
Leonardo da Vinci, também ficaria conhecida como Classicismo.

Foi um século de renovação técnica e científica. O desenvolvimento da
astronomia resultou no aperfeiçoamento dos mapas celestes. O astrolábio
foi trazido por árabes e judeus para a Península Ibérica e seria
adaptado para a orientação marítima. As técnicas de navegação e os
próprios tipos de navio também se adaptam aos novos mares e os
portugueses ganham a corrida pela exploração marítima por meio da
caravela.

Entre as revoluções técnicas, uma das mais relevantes seria a imprensa
de tipos móveis, de Gutemberg, que faz com que a publicação de livros se
expanda muito e vá ao encontro de um público novo que está se
estabelecendo, os universitários.

Todos esses elementos contribuiriam para um dos maiores acontecimentos
da história da Europa: a descoberta da América, em 1492, e oito anos
depois, a descoberta do Brasil.

Nas primeiras cinco décadas após a descoberta do Brasil, houve tráfego
de pessoas entre a Europa e a costa da colônia lusitana, mas a Coroa
Portuguesa ainda não havia enviado colonos. A primeira expedição de
colonizadores chegaria a Salvador em 1549. Nela, viriam os seis
primeiros missionários jesuítas, chefiados pelo Padre Manoel da Nóbrega.

Os jesuítas instalaram os primeiros colégios na costa do Estado do
Brasil. Vale lembrar que ainda não havia a ideia de Brasil, como
entendemos hoje, por isso falamos em Estado do Brasil, para marcar a
condição de dependência da Monarquia Portuguesa. A Costa do Brasil fazia
parte de Portugal. Inicialmente instalados em Salvador, os jesuítas se
dividiram e se dirigiram para o Recife, o Espírito Santo, São Vicente e,
mais tarde, São Paulo de Piratininga e o novo estado do Maranhão e
Grão"-Pará, onde o padre Antônio Vieira seria missionário.

Os padres vinham para fazer a catequese, isto é, a pregação e o ensino
da religião católica. O objetivo era a conversão dos povos originários,
que eram, na visão católica, pagãos.

Por sua vez, a motivação dos colonizadores era a busca e exploração das
riquezas naturais da nova terra. Essa dupla orientação colonizatória
seria assunto de muitos sermões de Vieira, porque muitas vezes os padres
entraram em conflitos diretos com os colonizadores, que não estavam
muito interessados em salvar almas.