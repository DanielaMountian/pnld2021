\chapter{Vida e obra}

\section{Sobre o autor}

Nascido em 1608 na cidade de Lisboa, Antônio Vieira era filho de
Cristóvão Vieira Ravasco e Maria de Azevedo. Quando tinha seis anos, sua
família transferiu-se para Salvador, onde Vieira estudou no colégio da
Companhia de Jesus, na qual ingressou como noviço em 1623.

Aos dezessete anos, tornou-se redator das cartas-ânuas e aos dezoito já
era professor de Retórica em Olinda. Entre 1641 e 1652 permanece na
Europa em missões diplomáticas na Inglaterra, França e Holanda. Em 1653
torna-se Superior das missões do Maranhão e Grão-Pará.

Em 1661, embarca para Portugal, onde estava sendo processado pela
Inquisição. Em 1665, está em Coimbra, proibido de retornar ao Maranhão.
Será preso em outubro. O primeiro volume de seus \emph{Sermões} foi
publicado em 1679. Retornou para a Bahia em 1681. Aos 89 anos, em 1697,
faleceu o grande jesuíta.

A essa altura, tinha preparado para publicação doze tomos dos sermões.
Embora a parte mais conhecida de sua obra sejam os mais de duzentos
sermões, Antônio Vieira escreveu também uma abundante correspondência,
somando mais setecentas cartas. Também deixou escritos proféticos,
\emph{A história do futuro} e \emph{A chave dos profetas}.

\subsection{Carta-Ânua de 1626}

As cartas-ânuas eram longas cartas-relatório que jesuítas escreviam
anualmente informando os progressos do trabalho catequético. A primeira
carta-ânua escrita por Vieira foi no ano de 1626. A carta tem cerca de
quarenta páginas e dá notícias dos acontecimentos dos anos de 1624 e
1625. É o primeiro escrito preservado do famoso pregador. Vale citar a
descrição geral da situação dos missionários:

\begin{quote}
Sustenta esta Província do Brasil, pouco mais ou menos, 120 padres da
Companhia: 90 sacerdotes, dos quais 31 são professos de quatro votos, de
três solenes, 2, coadjutores espirituais formados, 20; 62 estudantes;
coadjutores 50, e destes, 30 formados. Estes todos divididos em três
colégios, seis casas, e treze aldeias anexas às mesmas casas e colégios.
No colégio {[}da{]} Bahia residem comummente, 80; no de Pernambuco, 40;
35 no do Rio de Janeiro; na Residência do Espírito Santo, 12; na de
Santos, 5; na de S. Paulo, 7; na Casa dos Ilhéus, 4; em Porto Seguro, 4;
e 4 no Maranhão. Todos eles se ocupam em procurar de alcançar a salvação
e perfeição própria e das almas, que é o fim da nossa
Companhia.\footnote{VIEIRA, ANTÓNIO. Cartas. Org. J. Lúcio de Azevedo.
  Lisboa: Imprensa Nacional-Casa da Moeda, 1997, p. 4.}
\end{quote}

Após esta síntese da distribuição dos missionários, Vieira irá detalhar
a situação de cada um destes lugares. As cartas de Vieira e de outros
jesuítas, escritas sistematicamente, representam um diagrama verbal da
formação das sociedades coloniais.

\section{Sobre a obra}

Aqui estão reunidos dez sermões dos mais representativos do padre Antonio vieira.
São textos que continuam mostrando uma das facetas mais impressionantes dos
séculos XVI e XVII, que foi a aventura de missionários enviados a essas
distantes e desconhecidas terras para difundir a religião católica.

O estilo de Vieira é muito elaborado e florido, isto é, é um autor que
usa a língua portuguesa de maneira exemplar, explorando suas riquíssimas
belezas expressivas. Seu texto é muito ornamentado, marcado por
enumerações, interrogações, citações, marcado por uma fraseado musical e
persuasivo. Talvez por isso, um poeta da envergadura de Fernando Pessoa chegou a declarar que Antonio Vieira era o melhor escritor da língua portuguesa.

\subsection{Análise de um trecho do sermão de Santo Antônio}

\begin{quote}
Dos animais terrestres o cão é tão doméstico, o cavalo tão sujeito, o
boi tão serviçal, o bugio\footnote{Macaco.} tão amigo, ou tão
lisonjeiro, e até os leões e os tigres com arte e benefícios se amansam.
Dos animais do ar, afora aquelas aves que se criam e vivem conosco, o
papagaio nos fala, o rouxinol nos canta, o açor\footnote{Ave de rapina
  que pode ser treinada para caçar animais pequenos.} nos ajuda e nos
recreia; e até as grandes aves de rapina, encolhendo as unhas,
reconhecem a mão de quem recebem o sustento. Os peixes pelo contrário lá
se vivem nos seus mares e rios, lá se mergulham nos seus pegos\footnote{Regiões
  profundas dos mares.}, lá se escondem nas suas grutas, e não há nenhum
tão grande, que se fie do homem, nem tão pequeno, que não fuja dele. Os
Autores comumente condenam esta condição de peixes, e a deitam\footnote{Neste
  caso, ``deitar'' significa atribuir.} à pouca docilidade, ou demasiada
bruteza; mas eu sou de mui diferente opinião. Não condeno, antes louvo
muito aos peixes este seu retiro, e me parece que se não fora natureza,
era grande prudência. Peixes! Quanto mais longe dos homens tanto melhor:
trato e familiaridade com eles, Deus vos livre. Se os animais da terra e
do ar querem ser seus familiares, façam-no muito embora, que com suas
pensões o fazem. Cante-lhes aos homens o rouxinol, mas na sua gaiola:
diga-lhes ditos o papagaio, mas na sua cadeia: vá com eles à caça o
açor, mas nas suas piozes\footnote{Correias que se prendem à parte
  inferior da perna dos gaviões.}: faça-lhes bufonarias\footnote{Gracejos,
  piadas.} o bugio, mas no seu cepo\footnote{Tronco ao qual se prende o
  macaco de estimação.}: contente-se o cão de lhe roer um osso, mas
levado onde não quer pela trela\footnote{Coleira.}: preze-se o boi de
lhe chamarem formoso ou fidalgo, mas com o jugo\footnote{Peça de madeira
  que prende o boi à carroça que ele puxa.} sobre a cerviz\footnote{Parte
  de trás do pescoço.}, puxando pelo arado e pelo carro: glorie-se o
cavalo de mastigar freios dourados, mas debaixo da vara e da espora: e
se os tigres e os leões lhe comem a ração da carne que não caçaram no
bosque, sejam presos e encerrados com grades de ferro. E entretanto vós,
peixes, longe dos homens, e fora dessas cortesanias\footnote{Conjunto de
  formas de comportamento que caracterizam a vida social nas Cortes
  europeias, marcadas pela afetação e artificialismo.}, vivereis só
convosco, sim, mas como peixe na água.\footnote{VIEIRA, Antonio.
  \emph{Sermões}, vol. I. Org. Alcir Pécora. São Paulo: Hedra, 2001, p.
  321.}
\end{quote}

No ``Sermão de Santo Antônio'', pregado no Maranhão em 1654, o padre
Vieira afirma que apesar de fazer muita pregação, são poucos e poucas os
que se convertem. As pessoas estão surdas aos conselhos e escolhem agir
mal. Desiludido por ver o pouco fruto da palavra de Deus, o padre, nesse
sermão, diz que vai pregar aos peixes, animais considerados insensíveis.
Para isso, constrói um elogio do comportamento desses animais.

No trecho reproduzido acima, o jesuíta compara os animais segundo os
reinos a que pertencem: animais terrestres, animais aéreos e animais
aquáticos. Em seguida, o autor dá exemplos de animais desses três reinos
que se deixam amansar pelas pessoas em troca de um benefício material.
Da terra cita o cão, o cavalo, o boi, o macaco e o leão. Do ar, o
papagaio, o rouxinol, o açor e outra aves de rapina, como o gavião. Já
os peixes, em contraponto, não querem manter relação nenhuma com
pessoas, mostrando-se ``reservados, desconfiados e prudentes''.

Depois de considerar que os Autores da Antiguidade atribuem a
insensibilidade dos peixes à bruteza, Vieira considera que talvez seja
por prudência. Após, o padre retira a sentença moral ou o ensinamento
que percorre o sermão: ``Quanto mais longe dos homens tanto melhor:
trato e familiaridade com eles, Deus vos livre''.

Em seguida, Vieira desenvolve cada um dos exemplos dos animais que se
deixam domesticar, associando a recompensa e o sacrifício da submissão:
``contente-se o cão de lhe roer um osso, mas levado onde não quer pela
trela''.

Observe-se que todo o trecho é ``inventado''. A fábula de Vieira é um
argumento narrativo e um ornamento do discurso. Podemos supor que se
estivéssemos na igreja ouvindo esse sermão, nos sentiríamos
constrangidos e envergonhados pela comparação. Principalmente por não
serem os peixes seres morais, mas seres que se orientam apenas pelo
instinto.

Esse constrangimento moral é chamado de contrição. É o remorso e o
arrependimento por agir mal. A dor de consciência leva o fiel e a fiel a
se corrigirem e não voltar a cometer o mesmo pecado.


\section{Sobre o gênero}

Os sermões são um gênero oral, pensado para ser falado de memória diante
de uma plateia católica na igreja durante a missa. O objetivo do sermão
é relembrar aos fiéis o caminho da salvação de suas almas. É um gênero
de extensão média, contendo em geral entre vinte e quarenta páginas.

Os sermões de Vieira são sempre divididos em muitas partes, que indicam
os pontos em que seu pensamento vai se desenvolvendo. Quando publicados
como livro, os sermões passaram a trazer algumas marcas fixas, como
título, local e data em que foi pregado, às vezes, uma referência à
ocasião e uma epígrafe retirada da \emph{Bíblia}, que antecipa o tema. O
sermão tem muitos intertextos retirados da \emph{Bíblia}, dos santos
católicos e de outros autores antigos e modernos.

A produção dos sermões de Vieira está inserida no contexto histórico 
do século XVI, marcado por uma série de mudanças tão intensas e
impactantes que fez com que esse período ficasse conhecido como
Renascimento. Isso porque, nos centros urbanos da Europa, havia uma
sensação de que um mundo diferente estava nascendo, ou melhor
renascendo. Esse mundo novo seria marcado pelo deslocamento da essência
da existência humana da esfera coletiva e religiosa para a esfera
particular e política.

Em razão dessa mudança, o movimento literário do período também ficou
conhecido como Humanismo. Essa renovação do pensamento que se apresentou
no século XVI se alimentou muito fortemente nos autores gregos, como
Aristóteles, e romanos, como Cícero. Esses autores passaram a ser
conhecidos como ``clássicos''. Por isso, a renovação artística que
ocorre no século XVI, e que pode ser sintetizada pelo grande artista
Leonardo da Vinci, também ficaria conhecida como Classicismo.

Foi um século de renovação técnica e científica. O desenvolvimento da
astronomia resultou no aperfeiçoamento dos mapas celestes. O astrolábio
foi trazido por árabes e judeus para a Península Ibérica e seria
adaptado para a orientação marítima. As técnicas de navegação e os
próprios tipos de navio também se adaptam aos novos mares e os
portugueses ganham a corrida pela exploração marítima por meio da
caravela.

Entre as revoluções técnicas, uma das mais relevantes seria a imprensa
de tipos móveis, de Gutemberg, que faz com que a publicação de livros se
expanda muito e vá ao encontro de um público novo que está se
estabelecendo, os universitários.

Todos esses elementos contribuiriam para um dos maiores acontecimentos
da história da Europa: a descoberta da América, em 1492, e oito anos
depois, a descoberta do Brasil.

Nas primeiras cinco décadas após a descoberta do Brasil, houve tráfego
de pessoas entre a Europa e a costa da colônia lusitana, mas a Coroa
Portuguesa ainda não havia enviado colonos. A primeira expedição de
colonizadores chegaria a Salvador em 1549. Nela, viriam os seis
primeiros missionários jesuítas, chefiados pelo Padre Manoel da Nóbrega.

Os jesuítas instalaram os primeiros colégios na costa do Estado do
Brasil. Vale lembrar que ainda não havia a ideia de Brasil, como
entendemos hoje, por isso falamos em Estado do Brasil, para marcar a
condição de dependência da Monarquia Portuguesa. A Costa do Brasil fazia
parte de Portugal. Inicialmente instalados em Salvador, os jesuítas se
dividiram e se dirigiram para o Recife, o Espírito Santo, São Vicente e,
mais tarde, São Paulo de Piratininga e o novo estado do Maranhão e
Grão-Pará, onde o padre Antônio Vieira seria missionário.

Os padres vinham para fazer a catequese, isto é, a pregação e o ensino
da religião católica. O objetivo era a conversão dos povos originários,
que eram, na visão católica, pagãos.

Por sua vez, a motivação dos colonizadores era a busca e exploração das
riquezas naturais da nova terra. Essa dupla orientação colonizatória
seria assunto de muitos sermões de Vieira, porque muitas vezes os padres
entraram em conflitos diretos com os colonizadores, que não estavam
muito interessados em salvar almas.