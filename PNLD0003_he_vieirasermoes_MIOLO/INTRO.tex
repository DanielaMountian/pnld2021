\chapter{Apresentação}

\begin{flushright}
\textsc{iuri pereira}
\end{flushright}\bigskip

\noindent{}Os sermões presentes nesta edição foram escolhidos de acordo com os
seguintes critérios:

\begin{enumerate}
\item centralidade na tradição crítica: por esse critério, qualquer edição
de sermões de Vieira pareceria incompleta sem a presença de sermões como
o \emph{Sermão da sexagésima}, o \emph{Sermão de Santo Antonio} (aos
peixes), de 1654, e o \emph{Sermão da quarta"-feira de cinza}.

\item representatividade cronológica: por esse critério, foram escolhidos
sermões pregados no início da atividade pastoral de Vieira, como o
\emph{Sermão} \textsc{xiv} da série dedicada ao rosário mariano, de 1633, e
discursos do final de sua vida, como o \emph{Lágrimas de Heráclito}, de
1674.

\item relevância temática: por esse critério, selecionamos sermões como o
do \emph{Espírito Santo}, que traz a famosa alegoria que opõe dois tipos
de pessoas, umas semelhantes a estátuas de mármore, e outras a estátuas
de murta, e o \textsc{xiv} do rosário, por trazer uma justificativa teológica
para a escravização de pessoas negras trazidas de diversos lugares da
África, em termos providencialistas.

\item desenvoltura crítico"-argumentativa, excelência da imaginação poética
e da expressão: por esse critério escolhemos sermões como o \emph{Sermão
de Santo Antonio} (aos peixes), de 1654, que é comparável às grandes
realizações poética do Século de Ouro, de Gôngora a Cervantes, no que se
refere à exploração dos recursos expressivos da alegoria, e o da
\emph{Sexagésima}, que desempenha argumentos críticos, poéticos,
retóricos e teológicos que podem exemplificar os mais ricos e refinados
sistemas argumentativos da filosofia do período, de Galileu a Gracián.

\item limitação numérica: o número de sermões reunidos aqui atende à
expectativa de um leitor ou de uma leitora iniciante, que queira fazer
seu primeiro contato com o pensamento de Vieira, e de um leitor ou
leitora que deseje aprofundar"-se na leitura da obra do jesuíta.
\end{enumerate}

\section*{breves notas aos sermões}

\paragraph{\textsc{i}. \emph{Sermão da sexagésima}}

O \emph{Sermão da Sexagésima}, pregado na Capela Real, em Lisboa, em
1655, é dos mais célebres por ser um sermão que tematiza a própria
atividade do pregador, analisando as formas pelas quais as palavras do
sermão realizam a comunhão. Esse sermão foi escolhido por Vieira para
abrir a edição que ele organizou e até hoje é o primeiro sermão quase
obrigatório, pois, como o autor justifica, ``para que possais ir
desenganados com o Sermão, tratarei nele uma matéria de grande peso e
importância. Servirá como de prólogo aos Sermões que vos hei de pregar,
e aos mais que ouvirdes esta Quaresma.''

Vieira introduz uma diferença entre pregar sem deslocar"-se e pregar nos
confins do mundo: Índia, China e Japão. Índia, aqui, inclui a costa do
Estado do Brasil, que é capitania da Coroa Portuguesa no ultramar. No
dia do acerto de contas final, os primeiros seriam julgados pelas
palavras que disseram, os outros seriam julgados pelas palavras e pelos
passos que tiveram de dar para proferi"-las.

Aqui aparece um recurso expressivo que podemos chamar de agudo, adjetivo
que designa, no período de Vieira, um pensamento rebuscado, visionário
das relações obscuras entre as coisas e as palavras, que é a exploração
da homofonia entre as palavras ``passo'' e ``paço''. Enquanto o passo é
atributo dos padres missionários, que vão para longe resgatar as almas
condenadas, o paço, ou seja, o palácio ou a corte da monarquia, será
atributo dos padres que escolhem ficar na Metrópole.

O prêmio espiritual de quem se dedica às missões é maior, porque estes
escolhem o caminho mais árduo para realizar o ideal cristão de conversão
universal. Quem fica na Metrópole escolhe o caminho fácil e confortável,
escolhe ficar onde pode viver em pleno conforto, habitando boas
residências, entre pessoas conhecidas que falam o mesmo idioma. Mas
escolhe também um ambiente de corrupção moral, de vaidade, de
dissimulação, de sensualidade e de luxúria, aceitando conviver com esse
rol de tentações constantes.

Vieira desenvolve uma alegoria agrícola muito presente em discursos
religiosos. A semeadura é a pregação, a semente é a palavra e a
colheita, ou messe, é o resultado da catequese, ou seja, a conversão da
audiência. O sermão investiga as razões da assimetria entre a semeadura
e a colheita e propõe que a semente encontra obstáculos na natureza:
espinhos, pedras, pássaros e pessoas: ``todas as criaturas do mundo se
armaram contra esta sementeira''. Exemplificando o pensamento categórico
do jesuíta, cada um desses obstáculos inviabiliza o crescimento da
semente por meio de diferentes ações: os espinhos afogam"-nas, as pedras
secam"-nas, os pássaros as comem e as pessoas as pisam. Aqui está uma
parte da força persuasiva do pensamento de Vieira: as construções
simétricas, que tem em si mesmas valor de verdade ao desenvolver"-se no
discurso como uma partitura ou um mosaico ou um silogismo, revelando a
racionalidade da criação, com sua conclusão simples, ``como queríamos
demonstrar''.

Desenvolvendo a alegoria, Vieira transfere as qualidades dos objetos da
natureza para as pessoas, afirmando que existem homens homens, homens
brutos, homens tronco e homens pedra e faz o retrato das dificuldades
dos missionários do Maranhão:

\begin{quote}
Tudo isto padeceram os semeadores Evangélicos da Missão do Maranhão de
doze anos a esta parte. Houve Missionários afogados, porque uns se
afogaram na boca do grande Rio das Amazonas: houve Missionários comidos,
porque a outros comeram os bárbaros na Ilha dos Aruãs: houve
Missionários mirrados, porque tais tornaram os da jornada dos Tocantins,
mirrados da fome e da doença, onde tal houve, que andando vinte e dois
dias perdido nas brenhas, matou somente a sede com o orvalho que lambia
das folhas.
\end{quote}

\paragraph{\textsc{ii}. \emph{Sermão da quarta"-feira de cinza}}

O \emph{Sermão da quarta"-feira de cinza}, foi pregado em Roma, na Igreja
de Santo Antonio dos Portugueses, em 1672. Nele, Vieira explica um ponto
do livro do \emph{Gênesis} bíblico: ``tu és pó, e em pó te
converterás''. Por que a \emph{Bíblia} diz que somos pó, se somos seres
humanos?:

\begin{quote}
A Igreja diz"-me, e supõe que sou homem: logo não sou pó. O homem é uma
substância vivente, sensitiva, racional. O pó vive? Não. Pois como é pó
o vivente? O pó sente? Não. Pois como é pó o sensitivo? O pó entende e
discorre? Não. Pois como é pó o racional? Enfim, se me concedem que sou
homem: \emph{Memento homo}, como me pregam que sou pó: \emph{Quia pulvis
es}?
\end{quote}

Vieira parte de um argumento que remete à essência: nada é aquilo que
parece ser, mas aquilo que já foi e aquilo que vai se tornar: ``se foi
terra, e há de ser terra, é terra; se foi nada, e há de ser nada, é
nada, porque tudo o que vive neste mundo é o que foi e o que há de ser.
Só Deus é o que é.''

Como se explica o argumento de que já fomos pó? Recorrendo à narrativa
do \emph{Gênesis}, que explica que o primeiro homem, Adão, foi criado do
barro, barro é terra e terra é pó, logo, antes de ser humano, Adão era
pó. O próximo movimento associa esse argumento moral à vida humana,
reativando a doutrina do desengano, que nos lembra de nossa alma
imortal, essência eterna recoberta pelo corpo físico:

\begin{quote}
Pó levantado, lembra"-te outra vez que hás de ser pó caído, e que tudo
há de cair e ser pó contigo. Estátua de Nabuco: ouro, prata, bronze,
ferro, lustre, riqueza, fama, poder, lembra"-te que tudo há de cair de um
golpe, e que então se verá o que agora não queremos ver: que tudo é pó,
e pó de terra.
\end{quote}

``Lembre"-se de que vai morrer'' é um dos lugares"-comuns morais muito
usados no século \textsc{xvii} ibérico. A mortalidade e a imortalidade são
condições que nunca podem ser esquecidas pelo fiel católico. A condição
mortal de nosso corpo, dependente do tempo, nos remete à condição
imortal de nossa alma, na eternidade.

\paragraph{\textsc{iii}. \emph{Sermão de Santo Antonio} (aos peixes)}

O \emph{Sermão de Santo Antonio} (aos peixes) foi pregado no Maranhão no
ano de 1654. Um dos mais admirados sermões de Vieira, desenvolve"-se
inteiramente por meio de uma alegoria que equipara pessoas e peixes.
Parte do pressuposto de que a terra está cheia de corrupção e glosa uma
citação da \emph{Bíblia}, segundo a qual Jesus afirmara que os
pregadores são o sal da terra. O sal é usado na conservação de alimentos
para evitar que apodreçam, e o uso é transferido metaforicamente ao
trabalho dos missionários. A palavra de Deus difundida pelos padres
seria o sal que evitaria a corrupção das pessoas, metaforizadas como a
terra ou o solo onde a semente da palavra pode frutificar quando
encontra as condições ideias. Para Vieira, ou o sal não produz o efeito
esperado ou a terra rejeita esses efeitos.

Como em muitos outros, neste sermão o tema é moral e se debruça sobre a
análise das causas pelas quais a palavra de Deus tão abundantemente
administrada produz tão poucos frutos:

\begin{quote}
Muitas vezes vos tenho pregado nesta igreja, e noutras, de manhã e de
tarde, de dia e de noite, sempre com doutrina muito clara, muito sólida,
muito verdadeira, e a que mais necessária e importante é a esta terra
para emenda e reforma dos vícios que a corrompem. O fruto que tenho
colhido desta doutrina, e se a terra tem tomado o sal, ou se tem tomado
dele, vós o sabeis e eu por vós o sinto.
\end{quote}

Explorando um tema correlato, o habitat dos peixes, o mar, elemento tão
importante na vida portuguesa dos séculos \textsc{xvi} e \textsc{xvii}, Vieira se vale de
séries metafóricas ou alegóricas que exploram esse velho lugar"-comum que
associa o mar e a vida:

\begin{quote}
O leme da natureza humana é o alvedrio, o piloto é a razão: mas quão
poucas vezes obedecem à razão os ímpetos precipitados do alvedrio? Neste
leme, porém, tão desobediente e rebelde, mostrou a língua de Antônio
quanta força tinha, como rêmora, para domar a fúria das paixões
humanas.
\end{quote}

Cabe ressaltar aqui o extraordinário trecho em que o jesuíta aproveita
uma experiência pessoal direta ocorrida no Pará, que serve como
argumento ``natural'' em seu sermão. A partir desse argumento, há um
torneio extremamente inventivo: um peixe, a quem o sermão é dirigido,
ensina uma lição de moral ao jesuíta, sensível e racional, como em uma
demonstração da validade de todo o sermão, que afirma que os peixes,
seres brutos e insensíveis, mostram mais prudência do que as pessoas:

\begin{quote}
Quero acabar este discurso dos louvores e virtudes dos peixes com um,
que não sei se foi ouvinte de Santo Antônio e aprendeu dele a pregar. A
verdade é que me pregou a mim, e se eu fora outro, também me convertera.
Navegando de aqui para o Pará (que é bem não fiquem de fora os peixes da
nossa costa), vi correr pela tona da água de quando em quando, a saltos,
um cardume de peixinhos que não conhecia; e como me dissessem que os
Portugueses lhe chamavam quatro"-olhos, quis averiguar ocularmente a
razão deste nome, e achei que verdadeiramente têm quatro olhos, em tudo
cabais e perfeitos. Dá graças a Deus, lhe disse, e louva a liberalidade
de sua divina providência para contigo; pois às águias, que são os
linces do ar, deu somente dois olhos, e aos linces, que são as águias da
terra, também dois; e a ti, peixezinho, quatro. Mais me admirei ainda,
considerando nesta maravilha a circunstância do lugar. Tantos
instrumentos de vista a um bichinho do mar, nas praias daquelas mesmas
terras vastíssimas, onde permite Deus que estejam vivendo em cegueira
tantos milhares de gentes há tantos séculos! Oh quão altas e
incompreensíveis são as razões de Deus, e quão profundo o abismo de seus
juízos!
\end{quote}

Na conclusão do sermão, Vieira desdobra a alegoria, mostrando para o
público que estava falando da corrupção humana, figurando as pessoas e
seus vícios como peixes e suas características:

\begin{quote}
Olhai, peixes, lá do mar para a terra. Não, não: não é isso o que vos
digo. Vós virais os olhos para os matos e para o sertão? Para cá, para
cá; para a cidade é que haveis de olhar. Cuidais que só os Tapuias se
comem uns aos outros? Muito maior açougue é o de cá, muito mais se comem
os Brancos. Vedes vós todo aquele bulir, vedes todo aquele andar, vedes
aquele concorrer às praças e cruzar as ruas; vedes aquele subir e descer
as calçadas, vedes aquele entrar e sair sem quietação nem sossego? Pois
tudo aquilo é andarem buscando os homens como hão de comer e como se hão
de comer.
\end{quote}

O uso do tema da antropofagia e a atribuição da antropofagia aos brancos
em relação uns aos outros é muito ousado, porque se trata da categoria
mais desumanizadora da descrição dos povos da costa da América
Portuguesa e mais ainda dizer que o verdadeiro açougue em que as pessoas
são devoradas é a sociedade organizada branca e cristã e não as matas
habitadas pelo gentio selvagem. Vieira é guia moral de sua audiência,
por isso seu sermão desengana ao desnudar o vício e a surdez, a cobiça e
a avareza, a gula bestial e a ambição desenfreada com um simples ditado
popular: ``À vista deste exemplo, peixes, tomai todos na memória esta
sentença: Quem quer mais do que lhe convém, perde o que quer e o que
tem.''

\paragraph{\textsc{iv}. \emph{Sermão do Espírito Santo}}

O \emph{Sermão do Espírito Santo} foi pregado no Maranhão em 1657. No
\emph{Sermão da sexagésima}, Vieira pretende que o público saia
desenganado do pregador, mostrando que o sermão não é feito para seu
autor mostrar sua capacidade como letrado, mas para que o público possa,
através do sermão, desenganar"-se de si mesmo e corrigir sua conduta.

O sermão foi pregado na véspera da partida de uma expedição de resgate
de indígenas:

\begin{quote}
Primeiramente nesta missão do Rio das Amazonas, que amanhã parte (e
que Deus seja servido levar e trazer tão carregada de despojos do céu,
como esperamos, e com tanto remédio para a terra, como se deseja) que
português vai de escolta, que não vá fazendo ofício de apóstolo? Não só
são apóstolos os missionários, senão também os soldados e capitães,
porque todos vão buscar gentio e trazê"-los ao lume da fé e ao grêmio da
Igreja.
\end{quote}

No \emph{Sermão do Espírito Santo}, Vieira dá outra razão para o público
não se deixar iludir pela beleza da expressão verbal, que é só
instrumento, e se fixar no sentido pastoral do discurso, que é o
resultado almejado: ``O Espírito Santo vos ensinará o que eu vos tenho
dito, porque o pregador, ainda que seja Cristo, diz: o que ensina é o
Espírito Santo.'' O pregador não pode se vangloriar pelo efeito de seu
sermão porque em suas palavras atua o Espírito Santo instilando a luz da
Graça de sua presença no sacramento da comunhão.

A Graça é uma atividade divina pela qual Deus envia benefícios, muitas
vezes figurados em forma de fachos de luz, que criam uma disposição
interna na alma e na consciência católicas para a percepção da presença
de Deus na experiência temporal. O Espírito Santo ilumina com sua Graça
o interior da pessoa, enquanto as palavras ressoam em seu exterior
sensível, operando uma colaboração que instala a virtude católica:

\begin{quote}
Por que vos parece que apareceu o Espírito Santo hoje sobre os
apóstolos, não só em línguas, mas em línguas de fogo? Porque as línguas
falam, o fogo alumia. Para converter almas, não bastam só palavras: são
necessárias palavras e luz. Se quando o pregador fala por fora, o
Espírito Santo alumia por dentro, se quando as nossas vozes vão aos
ouvidos, os raios da sua luz entram ao coração, logo se converte o
mundo.
\end{quote}

O tema das virtudes do pregador também é tratado nesse sermão, em termos
das qualidades necessárias aos missionários, que precisam de sabedoria e
amor, mas em quantidades diferentes a depender de sua audiência. Aqui,
Vieira propõe um germe de pedagogia católica:

\begin{quote}
Porque para ensinar homens infiéis e bárbaros, ainda que é muito
necessária a sabedoria, é muito mais necessário o amor. Para ensinar,
sempre é necessário amar e saber, porque quem não ama não quer, e quem
não sabe não pode; mas esta necessidade de sabedoria e amor não é sempre
com a mesma igualdade. Para ensinar nações fiéis e políticas, é
necessário maior sabedoria que amor; para ensinar nações bárbaras e
incultas, é necessário maior amor que sabedoria.
\end{quote}

Ocorre também a recorrência de um tema presente no \emph{Sermão da}
\emph{Sexagésima} e no \emph{Sermão de Santo Antonio} (aos peixes):
deve"-se pregar universalmente, a todas as criaturas, até mesmo a
selvagens canibais: ``Se é necessário amor para ser pastor de ovelhas
que comem no prado e bebem no rio, que amor será necessário para ser
pastor de ovelhas, que talvez comem os pastores e lhes bebem o sangue?''

Pode"-se ler nesse sermão uma justificativa bíblica para o trabalho dos
missionários, quando Vieira remete à passagem da vida de Jesus em que
ele teria enviado seus apóstolos para missões de conversão pelo mundo.
Também é relevante a referência a São Tomé, que aparece nas cartas dos
jesuítas desde que Manoel da Nóbrega iniciou essa correspondência em
1549:

\begin{quote}
Repreendeu Cristo aos discípulos da incredulidade e dureza de coração,
com que não tinham dado crédito aos que o viram ressuscitado, e sobre
esta repreensão os mandou que fossem pregar por todo o mundo. A S. Pedro
coube"-lhe Roma e Itália; a S. João, a Ásia Menor; a São Tiago, Espanha;
a S. Mateus, Etiópia; a S. Simão, Mesopotâmia; a S. Judas Tadeu, o
Egito; aos outros, outras províncias, e finalmente a Santo Tomé esta
parte da América em que estamos, a que vulgar e indignamente chamaram
Brasil. Agora pergunto eu: e por que nesta repartição coube o Brasil a
Santo Tomé e não a outro apóstolo? Ouvi a razão.
\end{quote}

Nesse sermão, Vieira vai formular a célebre alegoria da estátua de
mármore e da estátua de murta. Essa alegoria é uma atualização da também
famosa metáfora da página em branco, com a qual os primeiros jesuítas
caracterizaram, equivocadamente, o indígena da América Portuguesa, que
seria como uma página em branco pronta para receber a escrita católica.
Mais tarde, os padres descobrem que nessa página era fácil de escrever,
mas que o texto não se fixava: ``Outros gentios são incrédulos até crer;
os brasis, ainda depois de crer, são incrédulos. Em outros gentios a
incredulidade é incredulidade, e a fé é fé; nos brasis a mesma fé ou é,
ou parece incredulidade.''

Em nome de sua popularidade, reproduzimos abaixo o parágrafo da famosa
alegoria:

\begin{quote}
Os que andastes pelo mundo, e entrastes em casas de prazer de
príncipes, veríeis naqueles quadros e naquelas ruas dos jardins dois
gêneros de estátuas muito diferentes, umas de mármore, outras de murta.
A estátua de mármore custa muito a fazer, pela dureza e resistência da
matéria; mas, depois de feita uma vez, não é necessário que lhe ponham
mais a mão: sempre conserva e sustenta a mesma figura; a estátua de
murta é mais fácil de formar, pela facilidade com que se dobram os
ramos, mas é necessário andar sempre reformando e trabalhando nela, para
que se conserve. Se deixa o jardineiro de assistir, em quatro dias sai
um ramo que lhe atravessa os olhos, sai outro que lhe descompõe as
orelhas, saem dois que de cinco dedos lhe fazem sete, e o que pouco
antes era homem, já é uma confusão verde de murtas. Eis aqui a diferença
que há entre umas nações e outras na doutrina da fé. Há umas nações
naturalmente duras, tenazes e constantes, as quais dificultosamente
recebem a fé e deixam os erros de seus antepassados; resistem com as
armas, duvidam com o entendimento, repugnam com a vontade, cerram"-se,
teimam, argumentam, replicam, dão grande trabalho até se renderem; mas,
uma vez rendidos, uma vez que receberam a fé, ficam nela firmes e
constantes, como estátuas de mármore: não é necessário trabalhar mais
com elas. Há outras nações, pelo contrário (e estas são as do Brasil),
que recebem tudo o que lhes ensinam, com grande docilidade e facilidade,
sem argumentar, sem replicar, sem duvidar, sem resistir; mas são
estátuas de murta que, em levantando a mão e a tesoura o jardineiro,
logo perdem a nova figura, e tornam à bruteza antiga e natural, e a ser
mato como dantes eram. É necessário que assista sempre a estas estátuas
o mestre delas: uma vez, que lhes corte o que vicejam os olhos, para que
creiam o que não veem; outra vez, que lhes cerceie o que vicejam as
orelhas, para que não deem ouvidos às fábulas de seus antepassados;
outra vez, que lhes decepe o que vicejam as mãos e os pés, para que se
abstenham das ações e costumes bárbaros da gentilidade. E só desta
maneira, trabalhando sempre contra a natureza do tronco e humor das
raízes, se pode conservar nestas plantas rudes a forma não natural, e
compostura dos ramos.
\end{quote}

Outro grande obstáculo da conversão dos povos da América Portuguesa é a
diversidade dos idiomas, que são incontáveis. O tema do aprendizado das
línguas estrangeiras como barreira à conversão de povos distantes é
muito recorrente nas cartas dos jesuítas, por isso é importante
ressaltar as reflexões de Vieira a esse propósito:

\begin{quote}
Manda Portugal missionários à China, império vastíssimo, dividido em
quinze províncias, capaz cada uma de muitos reinos; mas a língua, ainda
que desconhecida, é também uma {[}\ldots{}{]}. Manda Portugal missionários ao
Mogor, à Pérsia, ao Preste João, impérios grandes, poderosos, dilatados,
e dos maiores do mundo; mas cada um de uma só língua.
\end{quote}

Enquanto na costa do Estado do Brasil, as línguas são tantas, que o mito
da Torre de Babel, com a multiplicação das línguas, não se compara em
variedade:

\begin{quote}
E vem"-lhe curto também o nome de Babel, porque na Torre de Babel, como
diz S. Jerônimo, houve somente setenta e duas línguas, e as que se falam
no Rio das Almazonas são tantas e tão diversas, que se lhes não sabe o
nome nem o número. As conhecidas até o ano de {[}1{]}639, no
descobrimento do Rio de Quito, eram cento e cinquenta.
\end{quote}

Além dessa dificuldade, as línguas locais são tão estranhas e difíceis,
que aprendê"-las é tarefa quase milagrosa, como o próprio Vieira pode
testemunhar:

\begin{quote}
Por vezes me aconteceu estar com o ouvido aplicado à boca do bárbaro,
e ainda do intérprete, sem poder distinguir as sílabas, nem perceber as
vogais ou consoantes de que se formavam, equivocando"-se a mesma letra
com duas e três semelhantes, ou compondo"-se (o que é mais certo) com
mistura de todas elas: umas tão delgadas e sutis, outras tão duras e
escabrosas, outras tão interiores e escuras, e mais afogadas na garganta
que pronunciadas na língua; outras tão curtas e subidas, outras tão
estendidas e multiplicadas, que não percebem os ouvidos mais que a
confusão, sendo certo, em todo rigor, que as tais línguas não se ouvem,
pois se não ouve delas mais que o sonido, e não palavras desarticuladas
e humanas {[}\ldots{}{]}. Que será aprender o nheengaiba, o juruuna, o
tapajó, o teremembé, o mamaiana, que só os nomes parece que fazem
horror?
\end{quote}

As mulheres são muito raramente referidas nos escritos dos séculos \textsc{xvi} e
\textsc{xvii} sobre a colonização da América, por isso uma passagem notável nesse
sermão é aquela em que Vieira ressalta o papel das mulheres na
catequese, chamando"-as de apóstolas:

\begin{quote}
E se eu agora dissesse que nesta conquista, assim como os homens fazem
ofício de apóstolos na campanha, assim o podem fazer as mulheres em suas
casas? Diria o que já disseram grandes autores: eles na campanha
trazendo almas para a Igreja, fazem ofício de apóstolos; e elas em suas
casas, doutrinando seus escravos e escravas, fazem ofícios de apóstolas.
Não é o nome nem a gramática minha; é do doutíssimo Salmeirão, o qual
chamou às Marias: \emph{Apostolorum apostolas}: Apóstolas dos
apóstolos.
\end{quote}

Outra passagem muito interessante é aquela em que Vieira, sempre
utilizando referências bíblicas, fala da catequese em termos de morte e
devoração. Isso porque, novamente, o tema da antropofagia é muito
sensível e desumanizador, mesmo usado como metáfora. Ainda mais, Vieira
parece fazer referências à antropofagia ritual praticada por alguns
povos, para os quais o ritual representa a assimilação de uma qualidade
estranha, que, paradoxalmente, reforça os vínculos internos de cada povo
ou tribo. Finalmente, Vieira evidencia a extrema violência da
implantação dos hábitos católicos entre os povos originários, cujas
identidades culturais devem ser extirpadas:

\begin{quote}
{[}\ldots{}{]} o modo de converter feras em homens, é matando"-as e
comendo"-as, e não há coisa mais parecida ao ensinar e doutrinar que o
matar e o comer. Para uma fera se converter em homem há de deixar de ser
o que era e começar a ser o que não era, e tudo isto se faz matando"-a e
comendo"-a: matando"-a, deixa de ser o que era, porque, morta, já não é
fera; comendo"-a, começa a ser o que não era, porque, comida, já é homem.
E porque Deus queria que S. Pedro convertesse em homens, e homens fiéis,
todas aquelas feras que lhe mostrava, por isso a voz do céu lhe dizia
que as matasse e as comesse.
\end{quote}

\paragraph{\textsc{v}. \emph{Sermão da epifania}}

O \emph{Sermão da epifania} foi pregado na Capela Real, em Lisboa,
em1662. Seu tema e ``o mistério próprio deste dia é a vocação da
gentilidade à fé''. Na ocasião em que foi proferido, estavam os jesuítas
de volta à Metrópole depois de serem expulsos da colônia por tentarem,
segundo o texto, defender a liberdade dos índios. Neste sentido, o
sermão pode ser visto como parte de um processo judicial: ``Assim
permitiu a divina Providência que eu em tal forma, e as pessoas
reverendas de meus companheiros, viéssemos remetidos aos olhos desta
corte, para que ela visse e não duvidasse de crer o que doutro modo
pareceria incrível.''

Vieira retoma a narrativa do nascimento de Jesus, que recebeu a visita
dos três reis magos. Os intérpretes da \emph{Bíblia} afirmaram que os
três reis representariam as três partes do mundo conhecido: África, Ásia
e Europa. Então, o jesuíta se pergunta por que a América não estaria
nessa representação das partes do mundo. Argumenta que o cristianismo
teria duas vocações, a primeira no tempo de Cristo, foi uma vocação para
a conversão do Oriente, e a segunda nos tempos de Vieira, foi uma
vocação para que o rei de Portugal convertesse todo o Ocidente:

\begin{quote}
{[}\ldots{}{]} assim aconteceu no descobrimento do Mundo Novo. Desapareceu
a terra antiga, porque a terra dali por diante já não era a que tinha
sido, senão outra muito maior, muito mais estendida e dilatada em novas
costas, em novos cabos, em novas ilhas, em novas regiões, em novas
gentes, em novos animais, em novas plantas. Da mesma maneira o céu
também começou a ser outro. Outros astros, outras figuras celestes,
outras alturas, outras declinações, outros aspectos, outras influências,
outras luzes, outras sombras, e tantas outras coisas todas outras.
\end{quote}

Nesse sermão, Vieira, que foi perseguido por seus próprios compatriotas
cristãos, afirma que entende a razão da ruptura protestante, e que ela
mostrava como é injusto que uns se achem melhores do que outros por
serem mais brancos. O jesuíta critica desta forma os colonos que, por se
entenderem como brancos, se julgam melhores do que os índios, julgados
como negros:

\begin{quote}
Já considerei algumas vezes por que permitiu a divina Providência, ou
ordenou a divina Justiça, que aquelas terras e outras vizinhas fossem
dominadas dos hereges do Norte. E a razão me parece que é porque nós
somos tão pretos em respeito deles, como os índios em respeito de nós e
era justo que, pois fizemos tais leis, por ela se executasse em nós o
castigo. Como se dissera Deus: já que vós fazeis cativos a estes, porque
sois mais brancos que eles, eu vos farei cativos de outros, que sejam
também mais brancos que vós. A grande sem razão desta injustiça declarou
Salomão em nome alheio com uma demonstração muito natural. Introduz a
etiopisa, mulher de Moisés, que era preta, falando com as senhoras de
Jerusalém, que eram brancas, e por isso a desprezavam, e diz assim:
\emph{Filiae Jerusalem, nolite considerare quod fusca sum, quia
decoloravit me sol}: Se me desestimais porque sois brancas, e eu preta,
não considereis a cor, considerai a causa: considerai que a causa desta
cor é o sol, e logo vereis quão inconsideradamente julgais. As nações,
umas são mais brancas, outras mais pretas, porque umas estão mais
vizinhas, outras mais remotas do sol. E pode haver a maior
inconsideração do entendimento, nem maior erro do juízo entre homens,
que cuidar eu que hei de ser vosso senhor, porque nasci mais longe do
sol, e que vós haveis de ser meu escravo, porque nascestes mais perto?
\end{quote}

Arrematando sua contundente crítica à ação dos colonos que sequestram e
escravizam indígenas, Vieira mais uma vez se refere ao canibalismo, mas
o canibalismo representado pela cobiça dos colonos brancos, que são
verdadeiros comedores de índios e índias:

\begin{quote}
Vede que razão esta para se ouvir com ouvidos católicos, e para se
articular e apresentar diante de um tribunal ou rei cristão! Não nos
podemos sustentar doutra sorte, senão com a carne e sangue dos
miseráveis índios! Então eles são os que comem gente? Nós, nós somos os
que os imos comer a eles. Esta era a fome insaciável dos maus criados de
Jó.
\end{quote}

\paragraph{\textsc{vi}. \emph{Sermão \textsc{xiv}} da série Maria, rosa mística}

O \emph{Sermão \textsc{xiv}} do Rosário de Maria, pregado em 1633, em uma
irmandade de escravizados de um engenho, é o primeiro sermão conhecido
de Vieira e o próprio autor se declara como jovem noviço: ``Mas que fará
cercado das mesmas obrigações {[}de pregar o rosário de Maria{]}, tantas
e tão grandes, quem não só falto de semelhante espírito, mas novo, ou
noviço, no exercício e na arte, é esta a primeira vez que subido
indignamente a tão sagrado lugar, há de falar dele em público?''

O sermão foi pregado no dia da festa de São João Evangelista, mas também
é dedicado à Virgem Maria e aos escravizados devotos do rosário:

\begin{quote}
{[}\ldots{}{]} esta mesma será a matéria do Sermão, dividido também em três
partes. Na primeira veremos com novo nascimento nascido de Maria a
Jesus; na segunda com outro novo nascimento nascido de Maria a S. João;
e na terceira, também com novo nascimento nascidos de Maria aos Pretos
seus devotos.
\end{quote}

O rosário é uma sequência de 150 orações da Ave"-Maria que são calculadas
por meio de um terço ou rosário de contas. O rosário se baseia na
meditação sobre os mistérios, que são de três tipos: gozosos, dolorosos
e gloriosos. No \emph{Sermão \textsc{xiv}}, Vieira faz uma interpretação
providencialista do sofrimento de negros e negras sequestrados e
escravizados nos engenhos da América Portuguesa. Hoje, nós tendemos a
compreender a interpretação religiosa como uma forma de atenuar a
consciência de culpa dos opressores de populações africanas e indígenas
escravizadas e desumanizadas:

\begin{quote}
De maneira que vós os Pretos, que tão humilde figura fazeis no mundo,
e na estimação dos homens, por vosso próprio nome, e por vossa própria
nação, estais escritos e matriculados nos livros de Deus, e nas Sagradas
Escrituras: e não com menos título, nem com menos foro, que de Filhos da
Mãe do mesmo Deus: \emph{Et Populus Aethiopum hii fuerunt illic}.
\end{quote}

Por mais repulsivo que nos pareça, precisamos nos esforçar por
compreender a visão de Vieira como parte de um mundo no qual a
superioridade técnica e moral da cultura europeia sobre a cultura dos
povos ameríndios, superioridade autoatribuída, era razão suficiente para
que uns povos exercessem o império sobre outros, a pretexto de
retirá"-los das trevas do paganismo e trazê"-los à luz da Igreja Católica.
No século \textsc{xxi}, o mesmo argumento e a mesma presunção de superioridade
são usados para justificar guerras motivadas pela simples cobiça. Nesse
sentido, o trecho a seguir patenteia a ideologia religiosa que coloca a
vida da alma como valor muito acima da vida temporal e corporal,
justificando os sofrimentos presentes do corpo em termos de prêmios a
ser recebidos pela eternidade após a morte:

\begin{quote}
Oh! se a gente preta tirada das brenhas da sua Etiópia, e passada ao
Brasil, conhecera bem quanto deve a Deus, e a sua Santíssima Mãe por
este que pode parecer desterro, cativeiro e desgraça, e não é senão
milagre, e grande milagre! Dizei"-me: vossos pais, que nasceram nas
trevas da gentilidade, e nela vivem e acabam a vida sem lume da Fé, nem
conhecimento de Deus, aonde vão depois da morte? Todos, como já credes e
confessais, vão ao inferno, e lá estão ardendo e arderão por toda a
eternidade.
\end{quote}

Para os homens e mulheres escravizados nos engenhos da América
Portuguesa, o inferno começou bem antes da morte e dispensou a
colaboração do Diabo, porque foram as pessoas que escravizaram umas às
outras.

\paragraph{\textsc{vii}. \emph{Sermão da visitação de Nossa Senhora}}

O \emph{Sermão da visitação de Nossa Senhora} foi pregado na Bahia, em
1638, por ocasião de uma vitória obtida contra os holandeses que
investiram na tomada de conquistas portuguesas, se estabeleceram em
Pernambuco e chegaram em 1640 até a Ilha de São Luís, no Maranhão:
``Muitos dias há que esta nossa cidade festeja a ilustre vitória com que
Deus lhe fez mercê de se defender tão gloriosamente do poder do inimigo
comum, com que se viu sitiada.''

O assunto desse sermão é notadamente político, por causa dessa
circunstância grave, de confronto e resistência ao inimigo europeu. Após
a vitória, Vieira dirigindo"-se à cidade da Bahia, apela para que a
cidade pense nos pobres e miseráveis, nas viúvas e órfãos: ``E isto
mesmo é o que eu digo à Bahia, não só enquanto composta da parte
política e civil, senão também da militar: que a primeira parte dos
despojos da nossa vitória seja dos pobres enfermos e feridos deste
hospital, e dos que a mesma guerra, pela morte dos pais, ou maridos, fez
órfãos e viúvas''.

O assunto faz com que o sermão ganhe tonalidades épicas, ao descrever a
cidade inteira em vigília e união defendendo"-se do assédio militar:

\begin{quote}
Lembremo"-nos agora de nós. Quem visse interiormente a Bahia naqueles
quarenta dias e quarenta noites em que esteve sitiada, mais a julgaria,
na contínua oração, por uma Tebaida de anacoretas que por um povo e
comunidade civil, divertida em tantos outros ofícios e exercícios. Nos
conventos religiosos, nas igrejas públicas, nas casas e famílias
particulares, todos oravam. Os pais, os filhos, e quantos podiam menear
as armas, assistiam com Josué na campanha; e as mães, as filhas, e todo
o outro sexo ou idade imbele, orando continuamente pelas vidas daqueles
que por instantes temiam lhes entrassem pelas portas ou mal feridos ou
mortos. O estrondo das batarias inimigas e nossas, espertando com a
evidência e temor do perigo os ânimos, não lhes permitia quietação nem
sossego; e então a Bahia, como propriamente Bahia de Todos os Santos,
invocando a intercessão e auxílio de todos, não por intervalos, como
Moisés, mas perpetuamente e sem cessar, batia as muralhas do céu.
\end{quote}

\paragraph{\textsc{viii}. \emph{Sermão dos bons anos}}

O \emph{Sermão dos bons} anos foi pregado em Lisboa, em 1641. Assim como
o \emph{Sermão à visitação de Nossa Senhora}, é um sermão de argumento
político, que celebra a Restauração da Monarquia Portuguesa ocorrida em
1640:

\begin{quote}
Será pois a matéria e empresa do sermão esta: Felicidades de Portugal,
juízo dos anos que vem. Digo dos anos, e não do ano, porque quem tem
obrigação de dar bons"-anos, não satisfaz com um só, senão com muitos.
Funda"-me o pensamento o mesmo Evangelho, que parece o desfavorecia;
porque toda a matéria e sentido dele é um prognóstico de felicidades
futuras.
\end{quote}

Pode"-se perceber pela citação acima, que este sermão também toma para si
uma destinação profética, pela qual se diagnosticaria as felicidades
futuras do reino.

Vieira expressa o longo constrangimento de Portugal, que foi anexado
pela Coroa de Castela entre 1580 e 1640:

\begin{quote}
E se oito dias de esperar pela redenção, e ainda três dias, é tanto
tempo; quanto seria, ou quanto pareceria, não três dias, nem oito dias,
não três anos, nem oito anos, senão sessenta anos inteiros nos quais
Portugal esteve esperando sua redenção, debaixo de um cativeiro tão duro
e tão injusto!
\end{quote}

Em seguida, o jesuíta explica as razões de o domínio estrangeiro ter
sido tão extenso em termos de humildade e prudência:

\begin{quote}
Se Portugal se levantara enquanto Castela estava vitoriosa, ou, quando
menos, enquanto estava pacífica, segundo o miserável estado em que nos
tinham posto, era a empresa mui arriscada, eram os dias críticos e
perigosos; mas como a Providência Divina cuidava tão particularmente de
nosso bem, por isso ordenou que se dilatasse nossa restauração tanto
tempo, e que se esperasse a ocasião oportuna do ano de quarenta
{[}1640{]}, em que Castela estava tão embaraçada com inimigos, tão
apertada com guerras de dentro e de fora; para que, na diversão de suas
impossibilidades, se lograsse mais segura a nossa resolução. Dilatou"-se
o remédio, mas segurou"-se o perigo.
\end{quote}

A revelação profética de Vieira exige que se compreenda um modo de
entender a História diferente do nosso. Para Vieira, passado e futuro
são partes reversíveis de uma realidade espiritual que não está sujeita
ao tempo, a eternidade:

\begin{quote}
Acabou"-se o Evangelho, e eu tenho acabado o sermão. Mas vejo que me
estão caluniando e arguindo, porque não provei o que prometi. Prometi
fazer neste sermão um juízo dos anos que vem, e eu não fiz mais que
referir os sucessos dos anos passados. Mostrei a razão das profecias, as
dilações da esperança e oportunidade do tempo, o acerto dos decretos, a
propriedade e merecimento do nome, e tudo isto é história do que foi, e
não prognóstico do que há de ser. Ora, ainda que o não pareça, eu me
tenho desempenhado do que prometi, e todo este discurso foi um
prognóstico certo e um juízo infalível dos anos que vem. Tudo o que
disse, ou foram profecias cumpridas, ou benefícios manifestos da mão de
Deus: e em profecias e benefícios começados, o mesmo é referir o
passado, que prognosticar e segurar o futuro.
\end{quote}

O Tempo histórico reversível permite que Vieira leia na História do
passado a profecia do presente, que confirma e relança outra profecia
para outro futuro, que será por sua vez reversível a este presente, que,
no futuro, será passado. A crítica chama este procedimento de
interpretação figural. Para a Igreja Católica, a \emph{Bíblia} é um
livro revelado que contém a História verdadeira, que é sagrada. Essa
História verdadeira é recoberta pela História cronológica, que é a
interpretação humana da sucessão dos fatos sociais:

\begin{quote}
{[}\ldots{}{]} como se começaram a cumprir as profecias em sua restauração,
assim as levará Deus por diante e lhes dará o cumprimento gloriosíssimo
que elas prometam. Até agora era necessária pia afeição para dar fé às
nossas profecias mas já hoje basta o discurso e boa razão, por que os
efeitos presentes das passadas são nova profecia dos futuros.
\end{quote}

A última passagem que destaco, é uma fortíssima exortação à resistência
nacionalista, ao ímpeto conquistador, comparável à famosa passagem em
que o personagem Gaunt faz um elogio à Inglaterra na peça \emph{Henrique
\textsc{ii}}, de Shakespeare. Também destaco a referência à ideia do Orbe
Católico, ideal político de origem europeia de um mundo inteiro
católico:

\begin{quote}
Grande ânimo, valentes soldados, grande confiança, valorosos
Portugueses, que assim como vencestes felizmente estes inimigos, assim
haveis de vencer todos os demais; que, como são vitórias dadas por Deus,
este pouco sangue que derramastes em fé de seu poderoso braço, é
prognóstico certíssimo do muito que haveis que derramar vencedores; não
digo sangue de católicos, que espero em Deus que se hão de desapaixonar
muito cedo nossos competidores e que em vosso valor e em seu desengano
hão de estudar a verdade de nossa justiça; mas sangue de hereges na
Europa, sangue de mouros na África, sangue de gentios na Ásia e na
América, vencendo e sujeitando todas as partes do Mundo a um só império,
para todas em uma coroa as meterem gloriosamente debaixo dos pés do
sucessor de S. Pedro.
\end{quote}

\paragraph{\textsc{ix}. \emph{Sermão de S. José}}

O \emph{Sermão de S. José}, pregado na cidade de Lisboa em 1642, é um
sermão celebrativo dedicado ao aniversário do rei Dom João \textsc{iv}, apenas
dois anos após a Restauração da Monarquia. O pregador começa com uma
questão que, como costuma acontecer, será rapidamente respondida por
meio de uma consulta à \emph{Bíblia}: qual seria o dia mais feliz, o do
nascimento ou o da morte? A resposta encontrada é que o dia mais feliz é
o da morte. A partir dessa abertura, Vieira vai conciliar os dois dias
que se unem na mesma celebração: o dia da morte do patriarca São José,
pai de Jesus, e do nascimento do rei Dom João \textsc{iv}.

Após dar exemplos narrativos retirados da \emph{Bíblia} sobre o
nascimento misterioso de Jesus, Vieira interpretará o rei Dom João \textsc{iv}
como alguém destinado, como um novo messias, a conduzir Portugal para a
glória. Nesse sermão, Vieira identifica Dom João \textsc{iv} ao rei Encoberto,
mito messiânico muito difundido, que restauraria a soberania de
Portugal:

\begin{quote}
Sendo, pois, estes dois reis nascidos ambos reis, ambos redentores e
ambos encobertos, o primeiro, como diz a profecia de Isaías: \emph{Vere
tu es Deus absconditus, Deus Israel Salvator}: O segundo, prometido pela
profecia e tradição de Santo Isidoro a Espanha, não com outro nome ou
antonomásia, senão a do Encoberto.
\end{quote}

\paragraph{\textsc{x}. \emph{Lágrimas de Heráclito}}

\emph{Lágrimas de Heráclito}, discurso pronunciado em Roma diante da
rainha da Suécia e de séquito de religiosos e cortesãos em 1674, não é
um sermão, mas um exercício acadêmico, no sentido em que o pensavam e
praticavam as sociedades de corte da Europa seiscentista. Trata"-se de um
divertimento intelectual cortesão, de uma controvérsia fictícia,
inventada para que dois letrados engenhosos exibissem seus recursos de
erudição e argumentação defendendo ou atacando um tema filosófico ou
mitológico. O tema clássico escolhido opõe dois personagens que reagem
de maneiras opostas ao desconcerto do mundo, Heráclito só chora e
Demócrito só ri. Antonio Vieira e o padre Jerônimo Cataneo foram os
convidados a disputar qual dos dois tinha mais razão.

Vale citar a descrição do mundo corrompido feita por Vieira, que se
referia ao mundo antigo testemunhado por Heráclito, mas, ao mesmo tempo,
descrevia a corrupção do mundo do próprio Vieira, no fim do século \textsc{xvii},
e que serve para descrever ainda a desgraça social que testemunhamos no
século \textsc{xxi}:

\begin{quote}
Que é este mundo, senão um mapa universal de misérias, de trabalhos,
de perigos, de desgraças, de mortes? E à vista de um teatro imenso, tão
trágico, tão funesto, tão lamentável, aonde cada reino, cada cidade e
cada casa continuamente mudam a cena, aonde cada sol que nasce é um
cometa, cada dia que passa um estrago, cada hora e cada instante mil
infortúnios, que homem haverá (se acaso é homem) que não chore? Se não
chora, mostra que não é racional; e se ri, mostra que também são
risíveis as feras.
\end{quote}

Em um mundo como aquele, ninguém poderia rir, ainda mais alguém com fama
de sábio como Demócrito. Então, por que Demócrito ri? Responde Vieira
que Demócrito não ri, porque quem sempre ri, não ri nunca:

\begin{quote}
Que Demócrito não risse eu o provo: Demócrito ria sempre; logo nunca
ria. A consequência parece difícil, e é evidente. O riso, como dizem
todos os filósofos, nasce da novidade e da admiração, e cessando a
novidade, ou a admiração, cessa também o riso; e como Demócrito se ria
dos ordinários desconcertos do mundo, e o que é ordinário, e se vê
sempre, não pode causar admiração nem novidade, segue"-se que nunca ria
rindo sempre, pois não havia matéria que lhe motivasse o riso.
\end{quote}
