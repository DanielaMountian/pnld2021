

\footnote{Lc 2[:21] [E, \textit{depois que se completaram os oito dias para ser circuncidado o menino, foi-lhe posto o nome de Jesus, como lhe tinha chamado o anjo, antes que fosse concebido no ventre}.]}

\footnote{Sílio.}

\footnote{S.\,Cirilo.}

\footnote{Mt 16:2 [Ele, porém, respondendo, disse-lhes: Vós, quando vai chegando a noite, dizeis: \textit{Haverá tempo sereno, porque o céu está vermelho.}]}

\footnote{Jo 13:19 [\textit{Desde agora vo-lo digo, antes que suceda, para que, quando suceder, creiais que sou eu.}]}

\footnote{Gn 21:6 [E Sara disse: \textit{Deus me deu riso}, e todo aquele que ouvir rirá juntamente comigo.]}

\footnote{At 12:9 [E ele, saindo, seguia-o, e não sabia que era realidade o que se fazia por intervenção do anjo, mas \textit{julgava ver uma visão}.]}

\footnote{At 12:4 [E tendo-o mandado prender, meteu-o no cárcere, dando-o a guardar a \textit{quatro piquetes de quatro soldados cada um}, tendo intenção de o apresentar ao povo depois da Páscoa.]}

\footnote{S1 125:1 [Cântico das subidas. \textit{Quando o Senhor fez voltar os cativos de Sião, nós ficamos como que a sonhar.}]}

\footnote{Lc 24:21 [\textit{Ora, nós esperávamos que ele fosse o que havia de resgatar Israel; e agora, depois de tudo isto, é já hoje o terceiro dia} depois que estas coisas sucederam.]}

\footnote{[\textit{PL}: Vol. 35: \textit{Opera Omnia Augustini Hipponensis}. In Joannis Evangelium Tractatus \textsc{cxxiv}. Augustinus H ipponensis: S.\,Aurelii Augustini Hipponensis Episcopi in Joannis Evangelium Tractatus \textsc{cxxiv}. (C) Tractatus \textsc{xxxi}.]}


\footnote{S.\,Frei Gil.}

\footnote{Rm 4:18 [Ele \textit{creu na esperança}, contra a esperança, de que seria pai de muitas gentes, segundo o que lhe foi dito: Assim será a tua descendência.]}

\footnote{Jo 20:13 [E eles disseram-lhe: Mulher, por que choras? Respondeu-lhes: Porque \textit{levaram o meu Senhor}, e não sei onde o puseram.]}

\footnote{Jo 20:15 [Disse-lhe Jesus: Mulher, por que choras? A quem procuras? Ela, julgando que era o hortelão, disse-lhe: Senhor, \textit{se tu o tiraste, dize-me} onde o puseste; e \textit{eu o levarei}.]}

\footnote{[\textit{PL}: Vol. 76: \textit{Opera Omnia Gregorii}. Homiliae in Evangelia. Gregorius I: Sancti Gregorii Magni Romani Pontificis \textisc{xl} Homiliarum in Evangelia Libri Duo. (C,S) Liber Secundus. Homilia \textsc{xxv}.)}

\footnote{17 Lv 12:3 [E, \textit{no oitavo dia, será o menino circuncidado}.]}

\footnote{Lc 1:32 [Este será grande, e será chamado filho do Altíssimo, e o \textit{Senhor Deus lhe dará o trono de seu pai, Davi; e reinará} eternamente na casa de Jacó.]}

\footnote{Mt 2:14 [E ele, levantando-se, tomou de noite o menino e sua mãe, e \textit{retirou-se para o Egito.}]}

\footnote{Mt 2:2 [Dizendo: \textit{Onde está o rei dos judeus, que acaba de nascer?} Porque nós vimos a sua estrela no Oriente, e viemos adorá-lo.]}

\footnote{Mt 2:18 [\textit{Uma voz se ouviu em Ramá, grandes prantos e lamentações: Raquel chorando seus filhos}, sem admitir consolação, porque já não existem.]}

\footnote{Mt 2:13 [Tendo eles partido, eis que um anjo do Senhor apareceu em sonhos a José e lhe disse: Levanta-te, toma o menino e sua mãe, e \textit{foge para o Egito}, e fica lá até que eu te avise; porque Herodes vai procurar o menino para o matar.]}

\footnote{[\textit{PL}. Vol. 52: Sanctus Petrus Chrysologus. \textit{Sermones S. Petri Chrysologi}. Petrus Chrysologus: S. Petri Chrysologi Ravennatis Archiepiscopi Sermones. (C,G)* Sermo CL. De fuga Christi in Aegyptum.]}

\footnote{Mc 15:33 [E Simão respondeu-lhe nestes termos: \textit{Nós não temos usurpado o país de ninguém, nem retemos os bens doutrem, mas temos somente recuperado a herança de nossos pais, que de algum tempo a esta parte estava injustamente possuída pelos nossos inimigos.}]}

\footnote{[\textit{PL}: Vol.\,183: \textit{Sancti Bernardi Abbatis Clarae-Vallensis Operum Tomus Tertius}, Compleetens Sermones
de Tempore et de Sanetis, ae de Diversis. Saneti Bernardi Abbatis C1arae-Vallensis Sermones de
Tempore. Bernardus Claraevallensis: in Cireumcisione Domini. Sermo \textsc{i}. De leetione
evangeliea,llPostquam eonsummati sunt dies oeto, ut cireumcideretur puer, voeatum est nomen
ejus Jesus.]}

\footnote{Santo Epifânio.}

\footnote{Tertuliano.}

\footnote{Lc 2 :11 [\textit{Nasceu-vos} na cidade de Davi \textit{um Salvador, que é o Cristo Senhor}.]}

\footnote{Jo 19:17[19] [E Pilatos escreveu também um título, e o pôs sobre a cruz. E estava escrito: \textit{Jesus Nazareno, Rei dos Judeus.}]}

\footnote{Tertuliano.}

\footnote{Mt 2:15 [E lá esteve até a morte de Herodes, \texit{cumprindo-se deste modo o que tinha sido dito pelo Senhor por meio do profeta}, que disse: \textit{Do Egito chamei o meu filho}.]}

\footnote{Is 19:1 [Oráculo contra o Egito. \textit{Eis que o Senhor subirá sobre uma nuvem leve, e entrará no Egito}, e os ídolos do Egito se comoverão diante da sua face, e o coração do Egito se mirrará em seu peito.]}
 
\footnote{Lc 1:31 [31--33] [\textit{Eis que conceberás no teu ventre, e darás à luz um filho, e pôr-lhe-ás o nome de Jesus. Este será grande, e será chamado Filho do Altíssimo, e o Senhor Deus lhe dará o trono de seu pai, Davi; e reinará eternamente na casa de Jacó; e o seu reino não terá fim.}]}

\footnote{Lc 1:45 [E bem-aventurada tu, que creste, \textit{porque se hão de cumprir as coisas que da parte do Senhor te foram ditas.}]}

\footnote{Mt 1:21 [E dará à luz um filho ao qual \textit{porás o nome de Jesus, porque ele salvará o seu povo dos seus pecados.}]}

\footnote{Lc 1:66 [E todos os que as ouviram \textit{as ponderavam no seu coração, dizendo: Quem julgas que virá ser este menino?} Porque a mão do Senhor era com ele.]}

\footnote{Lc 1:66 [E todos ós que as ouviram as ponderavam no seu coração, dizendo: \textit{Quem julgas que virá ser este menino? Porque a mão do Senhor era com ele.}]}

\footnote{Js 10:25 [Disse-lhes de novo: Não temais nem vos acobardeis, \textit{tende ânimo, e sede fortes; porque assim fará o Senhor a todos os vossos inimigos, contra quem pelejais.}]}

\footnote{[\textit{PL}: Vol. 183: \textit{Sancti Bernardi Abbatis Clarae-Vallensis Operum Tomus Tertius, Compleetens Sermones de Tempore et de Sanctis, ac de Diversis}. Sancti Bernardi Abbatis Clarae-Vallensis Sermones de Tempore. Bernardus Claraevallensis: in Circumcisione Domini. Sermo \textsc{ii}. De variis Christi nominibus.]}
