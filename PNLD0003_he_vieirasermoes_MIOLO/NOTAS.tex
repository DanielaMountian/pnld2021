
\footnote{Mt 1:18 [Ora, o nascimento de Jesus Cristo foi assim: \textit{Estando Maria, sua mãe, desposada com José}, antes de se ajuntarem, achou-se ter concebido do Espírito Santo.]}

\footnote{Ecl 7:2 [7:1] [\textit{Melhor é} a boa fama do que o melhor unguento, \textit{e o dia da morte do que o dia do nascimento} de alguém.]}

\footnote{Jo 19:26 [Ora, Jesus, vendo ali sua mãe e que o discípulo a quem ele amava estava presente, disse à sua mãe: \textit{Mulher, eis aí o teu filho.}]}

\footnote{Jo 19:27 [\textit{Depois, disse ao discípulo: Eis aí tua mãe}. Edesde aquela hora o discípulo a recebeu em sua casa.]}

\footnote{[\textit{PL}: Vol.\,14: \textit{Opera Omnia S. Petri Damiani}. Operum S. Petri Damiani in Editione Cajetani Tomus Secundus Complectens Sermones et Sanctorum Historias. Petrus Damianus: B. Petri Damiani Sanctae Romanae Ecclesiae Cardinalis, Episcopi Ostiensis, Ordinis S. Benedicti, Sermones Ordine Mensium Servato (\textsc{c,g})* Sermo \textsc{lxiv}. De Sancto Joanne Apostolo et Evangelista.]}

\footnote{Bocarra in Anacephaleosim Reg. lusit.}

\footnote{Ex 1:6--7 [\textit{Sendo, pois, José falecido}, e todos os seus irmãos, e toda aquela geração, \textit{{os filhos de Israel frutificaram, e aumentaram muito, e multiplicaram-se, e foram fortalecidos grandemente; de maneira que a terra se encheu deles.}]}

\footnote{Gn 42 :13 [E eles disseram: Nós, teus servos, \textit{somos doze irmãos}, filhos de um varão da terra de Canaã; \textit{{e eis que o mais novo está com nosso pai, hoje; mas um já não existe}.]}

\footnote{Sl 18:3 [\textit{Um dia transmite esta mensagem ao outro dia}, e uma noite comunica-a a outra noite.]}

\footnote{Mt 1:20 [E, projetando ele isso, eis que, em sonho, lhe apareceu um anjo do Senhor, dizendo: José, filho de Davi, \textit{não temas receber a Maria, tua mulher}, porque o que nela está gerado é do Espírito Santo.]}

\footnote{[\textit{PL}: Vol. 16: \textit{Sancti Ambrosii Opera Omnia}. De lapsu Virginis Consecratae. Ambrosius Mediolanensis: Sancti Ambrosii Mediolanensis Episcopi de lapsu Virginis Consecratae liber Unus. (\textsc{c,g,s})* Caput \textsc{viii}.]}

\footnote{Is 7:14 [Portanto, o mesmo Senhor vos dará um sinal: \textit{eis que uma virgem conceberá, e dará à luz um filho}, e será o seu nome Emanuel.]}

\footnote{[\textit{PL}: Vol. 167 Col. 1540A: \textit{Opera Omnia Ruperti Tuitiensis}. Rupertus Tuitiensis: R.D.D. Ruperti Abbatis Tuitiensis de Trinitate et Operibus Ejus libri XLII. R.D.D. Ruperti Abbatis Tuitiensis De Trinitate et Operibus Ejus libri XLII. In VaI. Quatuor Evangelistarum Commentariorum liber Unus. Caput VI. Quod in eadem generatione tres isti Abraham, David, et joseph, magis insignes sint, et quod ad istos per incrementa ternaria promissio Christi facta sit. Caput VI.]}

\footnote{Is 45:15 [\textit{Verdadeiramente, tu és o Deus que te ocultas, o Deus de Israel, o Salvador.}]}

\footnote{Lc 1:38 [Lc 1:35] [E, respondendo o anjo, disse-lhe: O Espírito Santo descerá sobre tí, e \textit{a virtude do altíssimo te cobrirá com a sua sombra}. E, por isso mesmo, o santo, que há de nascer de ti, será chamado Filho de Deus.]}

\footnote{Mt 2:2 [E perguntaram: \textit{Onde está aquele que é nascido rei dos judeus?} Porque vimos a sua estrel no Oriente e viemos a adorá-lo.]}

\footnote{Lc 24:16 [\textit{Mas os olhos deles estavam como que fechados, para que o não conhecessem}.]}

\footnote{Is 61 :1-3 [O espírito do Senhor repousou sobre mim, porque o Senhor me ungiu; ele me enviou para evangelizar os mansos, \textit{para curar os contritos de coração, e pregar a redenção aos cativos}, e a liberdade aos encarcerados; para publicar o ano da reconciliação do Senhor, e o dia da vingança do nosso Deus, \textit{para consolar todos os que choram}; para conceder \textit{e dar} aos de Sião, que choram, \textit{uma coroa em vez de cinza}, óleo de gozo em vez de pranto, um vestido de glória em troca de seu espírito de aflição; e os que habitarem nela serão chamados fortes na justiça, plantas do Senhor para lhe darem glória.]}

\footnote{4Rs 11:2 [Mas Jeoseba, filha do rei Jeorão, irmã de Acazias, tomou a Joás, filho de Acazias, e o furtou dentre os filhos do rei, aos quais matavam, e o pôs, a ele e à sua ama, na recâmara, e \textit{o escondeu} de Atalia, \textit{e assim não o mataram}.]}

\footnote{Dt 33:16 [E com o mais excelente da terra, e com a sua plenitude, e \textit{com a benevolência daquele que habitava na sarça, a bênção venha sobre a cabeça de José} e sobre o alto da cabeça do que foi separado de seus irmãos.]}

\footnote{Ex 3:7-8 [E disse o Senhor: \textit{Tenho visto atentamente a aflição do meu povo}, que está no Egito, e tenho ouvido o seu clamor por causa dos seus exatores, porque conheci as suas dores. \textit{Portanto, desci para livrá-lo} da mão dos egípcios e para fazê-lo subir daquela terra a uma terra boa e larga, a uma terra que mana leite e mel; ao lugar do cananeu, e do heteu, e do amorreu, e do ferezeu, e do heveu, e do jebuseu.]}

\footnote{Gn 41:36 [41 :37] [\textit{Agradou o conselho a Faraó} e a todos os seus ministros.]}

 
\footnote{Gn 41:39 [Depois, disse Faraó a José: Pois que Deus te fez saber tudo isto, \textit{ninguém há tão inteligente e sábio como tu.}]}

\footnote{Gn 47:25 [E disseram: \textit{A vida nos tens dado; achemos graça aos olhos de meu senhor e seremos servos de Faraó.}]}

\footnote{}

\footnote{}

\footnote{}

\footnote{}

\footnote{}

\footnote{}

\footnote{}

\footnote{}

\footnote{}

\footnote{}

\footnote{}

\footnote{}

\footnote{}

\footnote{}

\footnote{}

 
\footnote{}

\footnote{}

\footnote{}

\footnote{}

\footnote{}

\footnote{}

\footnote{}

\footnote{}

\footnote{}

\footnote{}

\footnote{}

\footnote{}

\footnote{}

\footnote{}

\footnote{}

\footnote{}

\footnote{}



\footnote{}

\footnote{}

\footnote{}

\footnote{}

\footnote{}

\footnote{}

\footnote{}

\footnote{}

\footnote{}

\footnote{}

\footnote{}

\footnote{}

\footnote{}

\footnote{}

\footnote{}

\footnote{}

\footnote{}

\footnote{}

\footnote{}

\footnote{}

\footnote{}

\footnote{}

\footnote{}

\footnote{}

\footnote{}

\footnote{}

\footnote{}

\footnote{}

\footnote{}

\footnote{}

\footnote{}

\footnote{}

\footnote{}

\footnote{}








